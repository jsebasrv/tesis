\documentclass[8pt]{extreport} 
\usepackage{hyperref}
\usepackage{CJKutf8}
\begin{document}
\begin{CJK}{UTF8}{min}
\\	旅立ち!夢みる千年祭	
\\	「クロノ……	
\\	クロノ!	
\\	クロノったら!	
\\	ジナ「いつまでねてるの?   いいかげん起きなさい!	
\\	ジナ「ああ、リーネの鐘があんなに   気持ちよさそうに歌ってる。	
\\	ジナ「どうせゆうべ、こうふんして   ねつけなかったんでしょ?   ま、建国千年のお祭りだから   無理ないけど……   あんまり調子に乗って   ハシャぎすぎるんじゃないわよ!	
\\	ジナ「さ、いいかげん起きなさい!	
\\	ジナ「やっと起きたのね。   そうそう、あの、おさななじみの   発明好きな子……	
\\	アラ、ドわすれしちゃったわ。   なんていったっけ、あの子?	
\\	ジナ「そう、ルッカ!   ルッカの発明、見に行くんでしょ?	
\\	ジナ「あんまり、おそくならないようにね。   さ、行ってらっしゃい。	
\\	ジナ「あっ、そうそう。   はい、おこづかい。   お祭り楽しんでらっしゃい。    
\\	を手に入れた!	
\\	休みますか?   はい   いいえ	
\\	ねえ、知ってる? ここガルディア王国ができて、今年で1000年。現ガルディア王つまり、今の王様だけど33代なのよ。	
\\	で、その王様のなやみのタネは一人娘の王女様。なんでもすっごい、おてんばなんですって。	
\\	たぶん今ごろは、お城のなかで『あたしも、お祭りいきたーい!』なんて、おおさわぎしてるんじゃないかしらね。	
\\	あの、まいどおさわがせ娘、ルッカが千年祭会場で、せいきの大発表をするってせんでんしてたぜ。	
\\	私ね、今がいちばんしあわせな時……そんな気がするの。	
\\	あら、いらっしゃい。ゆっくりしていってね。なーんていっても、無理よね。あ~あ、私もはやくシゴトをおわってお祭りを見にいきたいわ。	
\\	おとまりは10
\\	です。   はい   いいえ	
\\	ではいつでもどうぞ。	
\\	千年祭には行ってきたかい?   はい   いいえ	
\\	えっ? 早く行ってこいよ!!この国ができて、今年で1000年ってお祭りなんだぜ。	
\\	ここから北にあるリーネ広場じゃ、今ごろいろんな店が出ていて、そりゃあもうお祭りさわぎってやつだな。	
\\	すげえよな、いろんなものが売ってるしいろんなイベントもある……まさにフェスティバル!!	
\\	1ぱい、おごってもらえないかしら?   はい   いいえ	
\\	ありがとう、お礼に1きょくプレゼントするわ。	
\\	先日、地震があったのをおぼえているかい?   はい   いいえ	
\\	そうか。ノンキなぼうやだな。まあ、いいや。	
\\	本当に最近は地震が多いよな。なにかよくないことが起こる前ぶれでなきゃいいんだけどね。	
\\	代々、伝わっているハコです。開けられた方ごじゆうにどうぞ。	
\\	不思議な力で封印されている……。	
\\	リーネ広場にて出店中!	
\\	こんにちは。この家の事聞きたい?   はい   いいえ	
\\	この家の町長は、ぼうけん者が好きでね。初心者にぼうけんのノウハウを教えてくれるのよ。それから、この家にある宝ばこは町長がぼうけん者のためにって置いてあるの。えんりょしないで持って行ってね。	
\\	武器や防具について聞かせてあげましょうか?   はい   いいえ	
\\	武器といっても、色々なものがあるのよ。刀とか、ゆみとか、けんじゅうとかね。けど人にはとくい、ふとくいがあってその人に合った武器しか、使えないの。	
\\	防具には3つのしゅるいがあるわ。頭につけるものと、体につけるものとそれからアクセサリー。防具も自分の体に合ったものしか装備できないのよ。それからアクセサリーには、特別な力があるものが多いから色々と調べてみるといいわ。	
\\	それと、刀は力が上がれば攻撃力も上がるけど、ゆみやけんじゅうはチョットちがうの。	
\\	必要なのは命中りつ、これが上がれば強いダメージをあたえられるわ。	
\\	この世の中には時の記憶がとどまる所があるという。そこでは危険にみちた場所でも旅の記録をとる事が出来る。	
\\	それはレプリカさ。 もし本物を見かけたら旅の記録を取ることをすすめるぜ。なんせ危険な所ほど出現しやすいという話だからな。おっと、それともう1つ!ワールドマップでは、どこでも旅の記録をとれる、わすれないでくれ。	
\\	メッセージが出てる間も、兄ちゃん達は動き回れるんだぜ。遠くはなれれば、声も聞こえなくなるよ。	
\\	メッセージやバトル中のウインドウは
\\	ボタンで上下に、いどうするのよ。	
\\	お前さん『シェルター』のことは知っているのかい?   はい   いいえ	
\\	『シェルター』はセーブポイントとワールドマップで使用できるアイテムで
\\	も
\\	も全回復するんじゃよ。べんりじゃの~。	
\\	よっしゃあ、俺様が技のこうかはんいを教えてやる。 聞くか?   はい   いいえ	
\\	戦闘時の技には、様々なこうかはんいがある。	
\\	まずは、モンスターを中心に広がるダメージだ。	
\\	たとえば、こいつをターゲットしていしたときは……	
\\	このハンイにいるモンスターつまりこの場合は2ヒキにダメージをあたえる事ができる。	
\\	だがな、こいつをターゲットしていすると……	
\\	このとおり、3ビキのモンスターにダメージをあたえられる。うまく使いな。	
\\	次に、自分から直線的にダメージをあたえる場合。	
\\	これには2しゅるいのパターンがある。1つめのパターンは……	
\\	自分からまっすぐのびた直線上のモンスターこの場合は3ビキ全てにダメージをあたえる。	
\\	この場合は直線上のどのモンスターにカーソルをもっていってももんだいない。	
\\	もう1つのパターンの場合は……	
\\	自分とモンスターをむすぶ直線上の敵だけにダメージをあたえる。	
\\	この場合は先ほどと同じく3ビキだが……	
\\	この場合は1ピキになるから気をつけろ。	
\\	次はターゲットしていしたモンスターをジクにして、ヨコ方向のモンスター全てにダメージをあたえる。	
\\	この場合、ヨコならびなら……	
\\	最後に自分を中心にダメージをあたえるパターン。	
\\	まずこのように、はんい内にモンスターがいない場合は使えない。	
\\	はんい内にモンスターがいれば自分を中心としたはんい内の全てのモンスターにダメージをあたえることができる。	
\\	どのモンスターにターゲットしていしてもオーケーだ。	
\\	ってところかな。べんきょうになったろ?	
\\	わしの話を聞きたいかの?   はい   いいえ	
\\	ふむ、お前さんはそしつがありそうじゃの。お前さんは戦いを続けていくうちに『技』を身につける事が出来るじゃろう。	
\\	技とは、おのれの武器をつかったとくしゅな攻撃法じゃ。これは、おのれのたんれんでのみ身につく。日々これ精進じゃ。	
\\	ほっほっほっ。熱心なわかものじゃの!よし、わしからのプレゼントじゃ。これで装備でもととのえるとよかろう。	
\\	を手に入れた!	
\\	戦闘中に受けるダメージについてレクチャーしましょうか?   はい   いいえ	
\\	モンスターの攻撃には特別なダメージをあたえてくるものがあるわ……	
\\	毒   
\\	が少しずつへっていくの。   このじょうたいでは   攻撃力も落ちるわ。	
\\	スロウ   スピードがおそくなって、次の攻撃に   すばやくうつれなくなるわ。	
\\	スリープ   眠ってしまって防御力が下がるの。   けど、攻撃を受けるとめざめるわ。	
\\	こんらん   自分が何をしてるのか   わからなくなってしまうの。   味方を攻撃したりね。 コワいわよ。	
\\	ブラインド   回りがよく見えなくなって   モンスターに攻撃をあてにくくなるわ。	
\\	ロック   いっさいの技がつかえなくなるの。	
\\	ストップ   これは動きが止まって   何もできなくなるわ。	
\\	これらのじょうたいの時は防御力が落ちてダメージが大きくなるわ。『ばんのうやく』などの体調回復アイテムでなおせるけど戦闘が終われば、アイテムを使わなくても自然になおるのよ。	
\\	続き聞きます?   はい   いいえ	
\\	モンスターの攻撃を受けて、
\\	が0になると戦闘不能になっちゃうわ。もちろん『アテナの水』とかのアイテムで復活も出来るわよ。けどね、みんなが戦闘不能になった時は時のながれのわすれ物になっちゃうの……気をつけてね。	
\\	を押し続けると戦闘から逃げる!!こいつも大事だ。戦術的てったいとでも言おうかな。けどよ、逃げてばっかじゃ弱いままで強くなれないから気をつけな。それから、もう1つ。メニューの説明だ。武器マークのヨコの数字は攻撃力。たてマークのヨコの数字は防御力。防御力は、なぐりや物理攻撃をどれだけふせぐかあらわす。魔法をふせぐのは『まほうぼうぎょ』。これがたとえば40だったら敵の魔法を40パーセントふせぐ。それと数字が星マークになったらそのパラメーターは最高ということだ。	
\\	この海いきに、なぞの大うずが現われる時があるんだ……。一体あのうずはなんなのだろう?	
\\	船にのりますか?パレポリまで10
\\	です。   はい   いいえ	
\\	エレイン「彼が……、フリッツが   旅に出たまま、帰ってこないの……。   私は彼がいれば他になにも   いらないのに……。	
\\	エレイン「もう私のことなんて……	
\\	ララ「おや、クロノかい?   ルッカならお父さんのタバンと   リーネ広場にいってるよ。   せいきの大発明を見せるんだとか   言ってね。	
\\	タバン「おう!   ルッカの仲間達だな。	
\\	森をぬけて北…… ガルディア城	
\\	この先、ガルディア城。   用のない者の出入りをきんずる。	
\\	この先、空き地あり。	
\\	兵士「何者だ!	
\\	兵士「民間人を通すワケにいかん。   子供は祭りにでも行くんだな。	
\\	マール「な、何してんの?   早くお祭り行こうよ!	
\\	ガルディア王国の千年祭に行くんだ!	
\\	一度、千年祭とやらに行ってみようと思っての~。	
\\	わ~い! お祭り、お祭り!	
\\	千年祭の広場へ行ってお店を出すつもりです。あなたも、ぜひ私の店に来て下さい。	
\\	うおーっ、祭りだーっ!ひごろのウサをわすれてガンガン飮むぞー!	
\\	になるが……とまって行くかね?  はい  いいえ	
\\	ほうほう、そうかい!	
\\	またいつでも来なさい。	
\\	この世で1番きらいな人?もちろん父さんよ。	
\\	お父さん、お金大好きなの……僕よりも……。	
\\	みんなが、うちのパパはヨクバリだって言うの……	
\\	子供たちの心がはなれていくのがとってもつらいんです……。	
\\	町長「ガハハッ! 俺は、この町の町長。 実は金持ちなんだよ!!	
\\	3回まわってコケコッコーで10
\\	やってもいいぞ?  やる  やらない	
\\	町長「つまらん。	
\\	町長「ガハハッ! そーら10
\\	じいちゃんも、ばあちゃんも、みんなトルースの千年祭へ行っちゃって……私たちは、るす番ですよ。	
\\	わたしもお祭り行きた~い!	
\\	なかよく人とくらすモンスターもいるの。あのピアノひきさんとかね。	
\\	みーんなトルースの千年祭へ行っちまったよ。ったく。 うかれやがって…… ゲプッ……	
\\	400年前、フィオナという女性が北のあれた大地を森に変えようとして死んでいった……悲しい話ですね……。	
\\	うちのピアノひきは腕がいいんだ。ぜひ聞いていっておくれ!	
\\	お~いらは ピアノひき~
\\	かなしいやつ  はでなやつ	
\\	とっておきの、ハイパーほしにくがあるんだけどねえ~とっておきだからねえ~。9900
\\	でなら考えてもいいな。買うかい?  買う  やめとくもの好きだよね~そ~ら、ハイパーほしにくだ!やっぱりね。	
\\	定期船に乗るなら10
\\	だよ。チケット買うかい?  はい  いいえ	
\\	はい。 これがチケットだ。まいどあり!	
\\	乗れば気分そうかい!そんはしないと思うけどなあ……。	
\\	お金が、たりないなあ……。	
\\	定期船はいいわねえ。	
\\	定期船はいいなあ……。	
\\	おお、わか者よ!今日はわが王国の千年祭だ!ぞんぶんに楽しんでゆかれよ。	
\\	走っている人はレース中よ。話しかけても返事してくれないんだから。	
\\	正面にたってボタンを押しな。動きだしたらタイミングよく
\\	ボタンだ。	
\\	やったね!!シルバーポイントカードに1ポイントためておくよ!!	
\\	もうこれ以上シルバーポイントはためられないよ。	
\\	思い返せばこの国にもいろいろな事があったからねえ……。400年くらい前の魔王軍との戦いとかさ。あんたも今の平和にかんしゃしなよ。	
\\	ここのおじさん、アクセサリーなんかも買い取ってくれるそうよ。	
\\	そうねェ……、ペンダントなんかも売れるかしらねえ。	
\\	エヘヘ、ボク知ってるよ。何びゃく年か昔、魔王軍とすごい戦いがあったんだって。	
\\	もしその戦いに負けてたら今こんなお祭りはなかったってうちのママが言ってたよ。	
\\	よそうするかい?   はい   いいえ	
\\	1ばん てっかめんランナー!   2ばん うっちゃれダイナ!   3ばん スイートキャット!   4ばん ほいほいソルジャー!	
\\	レース中さ。だまって見てな!	
\\	今日のてっかめんランナーは何かやらかす予感がするぜ。	
\\	うっちゃれダイナ……ふっ……、俺が見込んだやつさ。	
\\	たのもしいぜ……、スイートキャット。	
\\	今のほいほいソルジャーの右に出る者は……。	
\\	前回の勝者は……	
\\	てっかめんランナーです。	
\\	前回の勝者は……	
\\	うっちゃれダイナです。	
\\	スイートキャットです。	
\\	ほいほいソルジャーです。	
\\	やったね!!シルバーポイントカードに20ポイントためておくよ!!	
\\	もうこれ以上ポイントは	
\\	世にも恐ろしい見せ物小屋だよ。あんたのシルバーポイント
\\	てんはこの小屋の中でしか使えない。勇気をためすかい?	
\\	ようこそ、ノルシュティン・ベッケラーの実験小屋へ。今から世にも恐ろしいショウがはじまる。	
\\	シルバーポイントカード!いくらはらうかい?	
\\	キャンセル	
\\	一人じゃこのゲームは出来ないよ。仲間を探して来な。	
\\	ポイントたりないよ。	
\\	ビッグスです。	
\\	ウェッジです。	
\\	ピエットです。	
\\	ビッグスは誰だ?	
\\	ウェッジは誰だ?	
\\	ピエットは誰だ?	
\\	バレちゃいましたか?	
\\	ポヨゾー人形。お部屋にかざってね。	
\\	分身のマネをしろ!
\\	…左手あげ 
\\	…右手あげ
\\	…わらい  
\\	…ビックリだ。	
\\	ヒュッ ……と音がなったらマネするんだ。ドッペル ゴー!	
\\	見事にドッペルった!	
\\	このドッペル人形
\\	の家に送っておくよ。	
\\	モンスターが下にせまってくる。パックをモンスターにぶつけてオリの中まで追い返せ!	
\\	人質をもらっていくよ。人質は上からつるされ、ほのおの上にズリズリっと下ろされていく。黒コゲになったらオジャンさ。あおい光にぶつかれば、人質の命がわずかにのびる。人質を殺さないようにモンスターをたいじしな!	
\\	みごと!!ネコちゃんプレゼント。	
\\	ネコのえさ。クロノの家に送っておくよ。おなかへらすとネコは逃げちゃうからな。	
\\	ルッカが開発した2足ほ行ロボットゴンザレスと戦いたいかたはこの奥へ!勝てばシルバーポイントもらえるよ!!	
\\	あ~ ゴンザ~レス
\\	オ~レは 強い
\\	オ~レに勝ったら15ポイント~!!
\\	あ~んたは 弱い
\\	オ~レは 強い
\\	それじゃあ あんたはへなちょ~こ~さ~!!
\\	あ~んたは 強い
\\	あ~んたは す~ごい
\\	だ~から あげるよ15ポイント~!!
\\	ふォふォふォ 祭りは楽しいのう。ところであんた知ってるかね?   はい   いいえ	
\\	そうか……、だったらええわい。	
\\	ここだけの話、わが国の王、ガルディア王は娘のおてんばぶりに頭をかかえているそうじゃ。	
\\	しかしどこまでおてんばなのかいっぺんその王女様とやらを見てみたいもんじゃのう。	
\\	ボタンダッシュで走り回ってたらつかれちゃった。でも踊りを見たらまた元気だよ!	
\\	大昔の踊りなんですって。しんぴてきよね~!	
\\	原始の~ リズム~
\\	あついリズムだね!あんたも、
\\	ボタンで踊りなよ!	
\\	ランラ~ランララ~!町長さんの家で、おこづかいもらったの!	
\\	少女「キャッ!少女「いったー……	
\\	少女「ゴ、ゴメンなさい!   だいじょうぶ?	
\\	少女「ア、アラ?   ペンダントが……。	
\\	少女「た、大変!   なくしちゃったのかしら?	
\\	少女「ありがとう!!   そのペンダント私のよ。   古ぼけてるけどとっても大事な物なの。	
\\	返してくれる?   はい   いいえ	
\\	私、お祭り見に来たんだ。   ねえ、あなたこの町の人でしょ?   一人じゃ面白くないもん。	
\\	いっしょに回ろうよ!   いいでしょ? ね、ね?   はい   いいえ	
\\	少女「そんなー! か弱い女のコを   助けると思って、お願いッ
\\	はい   いいえ	
\\	少女「わーい、やったーッ!	
\\	少女「あ、名前言ってなかったね。   え、えーと私は……	
\\	マール「マールって言うの。   あなたは?	
\\	マール「クロノか、いい名前だね!   よろしくね、クロノ!   さ、早く行こう!	
\\	ジナ「あら、ルッカ以外の女の子を   連れて来るなんて、めずらしいわね。   お名前は?	
\\	マール「はじめまして!   マールです。	
\\	ジナ「えっ? マールさん?   
\\	どこかでお会いしてません?   おかしいわね、気のせいかしら。	
\\	ネコのえさ 
\\	グラム	
\\	今あなたのシルバーポイントは
\\	てん。シルバーポイント10
\\	を50
\\	にかえるかい?   はい   いいえ	
\\	ポイントたりないね。	
\\	まいど。	
\\	この先じゅんび中さ。広場で一通りあそんでからまた来な。	
\\	そのつつみはオラの母ちゃんのべんとうだ。元気モリモリ! オラまいにち楽しみにしてんだ。	
\\	おべんとうを食べ元気になった!!	
\\	マール「うふふ。   クロノって食いしんぼね!	
\\	わーん!わたちのネコが……。	
\\	わたちのネコちゃん、連れて来てくれてありがとう。	
\\	マール「クロノ、やっさしい!	
\\	となりに立ちな! 飮みくらべ!
\\	ボタンれんだだ!	
\\	よーい どん!!	
\\	本か。まだまだだな!	
\\	やるねえ!	
\\	あんたとは飮めないな。	
\\	やったね!!シルバーポイントカードに5ポイントためておくよ!!	
\\	マール「何にでも   いっしょうけんめいなのね。   クロノは。	
\\	「この刀かじボッシュの作った剣でも 買って行かんか?	
\\	買う   いらない	
\\	ところで後ろのおなごのペンダント。	
\\	売ってくれるように話をつけてくれんかね?   はい   いいえ	
\\	おお! よく見れば、そのペンダントは……いや、そんなこうかな物はうけとれん。あんたが大事にとっておきなされ。	
\\	ボッシュ「ワシゃ東の大陸に住んどる。   何かの時は立ち寄るがよい。	
\\	ここはリーネ広場。リーネの鐘の音を聞けばしあわせになれるって言い伝えがありまちゅ。	
\\	マール「しあわせ!   クロノとこうしているの。	
\\	ルッカの発明がもう完成したよ。広場をずうっと奥へ行った所だよ。	
\\	ルッカがまた、かじやの親父さんと怪しげな発明品を作ってるそうね。	
\\	こないだみたく、いきなりドカン!てな事にならなきゃいいわねえ。	
\\	もう完成しているころじゃない?広場をずうっと奥へ行った所よ。	
\\	マール「へー、面白そうじゃない。   連れてってよ、クロノ!	
\\	マール「待って!!   キャンディ買っていくわ!	
\\	マール「ちょっと待ってて   急がないでよ!	
\\	マール「人さらいってわけじゃ   あるまいし。   そんなに腕を引っぱらないでよ!	
\\	マール「おばさん   これちょうだい!	
\\	おばさん「あいよ。	
\\	マール「おまたせ!	
\\	タバン「さあさあ、お時間と勇気のある方は   お立会い!   これこそ、せいきの大発明!   超次元物質転送マシン1号だ!!	
\\	タバン「早い話が   こっちに乗っかると	
\\	タバン「こっちに転送されるって   夢の様な装置だあ!   こいつを発明したのが頭脳めいせき   さいしょくけんびの、この俺の一人娘   ルッカだ!	
\\	転送って何だ?	
\\	物がまたたく間にいどうするって話じゃ。	
\\	ルッカの発明の事だからまた何か……。	
\\	タバン「おッ? 兄ちゃん   やるのかい?   ……と思ったら、何だクロノか。	
\\	ルッカ「クロノ!	
\\	ルッカ「待ってたわよ!   だーれも、このテレポッドの転送に   ちょうせんしないんだもの。   こうなったら、あんたやってくれない?	
\\	マール「面白そう!   やってみなよ。 私見ててあげる!	
\\	ルッカ「左のポッドにのればいいのよ。	
\\	タバン「なーんにもこわい事ないぜ。   ネコではうまくいったんだ。   まだ人間ではためしてないけど……。   さ、安心してのった! のった!	
\\	タバン「スイッチオン!	
\\	ルッカ「エネルギーじゅうてん開始!	
\\	一同「おおーッ!!   グレイト!!	
\\	ルッカ「どう、おどろいた?   何だったらもう一度やる?	
\\	タバン「本当にうまくいった!?   めずらしい事も……	
\\	コ、コホン!   いやいや、ルッカの発明だから   当然! すごいだろう!	
\\	ルッカの発明もたまには成功するのね。	
\\	本当じゃ!	
\\	マール「面白そうね、私もやる!	
\\	ルッカ「へ?	
\\	ルッカ「ちょ、ちょっとクロノ   あんたいつの間に、こんなカワイイ子   口説いたのよ。	
\\	マール「ね、いいでしょクロノ?   ここで待ってて。   どこにも行っちゃやだよ!	
\\	タバン「さあさあ、ちょう戦するのは何と   こんなにカワイらしい娘サンだ!   ささ、どーぞこちらへ!	
\\	マール「エヘヘ。   じゃ、ちょっと行ってくるね!	
\\	タバン「だいじょぶかい?   やめるんだったら今のうちだぜ。	
\\	マール「へっちゃらだよ!   全然こわくなんかないもん。	
\\	タバン「それでは みなさん!   このカワイイ娘サンが見事   消えましたら、はくしゅかっさい。	
\\	マール「何これ?   ペンダントが……。	
\\	ルッカ「えッ!?	
\\	タバン「おい、ルッカ   出て来ねーぞ?	
\\	タバン「ハ、ハイ! ごらんの通り影も形も   ありません!   こ、これにてオシマイ!!	
\\	タバン「おい、ルッカ!   一体どうなってんだ?   あのコはどうしちまったんだ!?	
\\	ルッカ「あのコの消え方   テレポッドの転送の消え方じゃない。	
\\	あの空間のゆがみ方……   ペンダントが反応していたようにも…   もっと別の……何かが……。	
\\	タバン「どうしたらいいんだい?   助けることは?	
\\	ルッカ「あのコは……   ……んっ? そういえばあのコ   どこかで見たことがあるような?	
\\	ルッカ「ちょ、ちょっとクロノ!   あのコ、あんたが連れて来たのよ!   どこ行こーってのよ!	
\\	タバン「おーッ!   後を追うってのかクロノ。   さすがは男だぜ!	
\\	ルッカ「そうね!   あの空間の先に何があるのかわからない   けど、それ以外に方法はなさそうね。	
\\	タバン「でも、都合よくまた   あなが現れるとはかぎらないぜ。	
\\	ルッカ「やってみるかちはあるわ!   きっとペンダントがキーに   なってるのよ!   クロノ!   しっかりそれをにぎってて。   きっと同じ事が起こるはずよ!	
\\	ルッカ「もっと出力をあげて!!	
\\	タバン「あいさ!!	
\\	ルッカ「もっと!もっと!	
\\	ルッカ「ビンゴ!!   うまくいきそうよ!!	
\\	ルッカ「私も原因を究明したら   後を追うわ!   たのんだわよ、クロノ!	
\\	帰ってきた王妃	
\\	千年祭? 建国1000年?こんなひるまから何ねぼけてるんだい?今は建国600年、ガルディア21世が国をおさめてるじゃないか。	
\\	魔王軍と王国軍の戦いが始まってからもう何年もたつ……。一体この戦はいつまで続くのだろう……。	
\\	おとまりですか?10
\\	になりますが?   はい   いいえ	
\\	いらっしゃいませ。	
\\	ここはどこだと? バカ者!!わがガルディア王国を知らぬわけはあるまいっ!	
\\	わが国は、いま平和をとりもどすため魔王軍と日夜戦っているのだ。	
\\	そのくらいは知っておるであろう?   はい   いいえ	
\\	ならば! ここはどこだなどねぼけたことをいうのはやめるのだな。	
\\	なんと! それも知らぬとは……。もしかして、おぬし旅の者か?ふむ。 それならばしかたがないか。	
\\	建国のお祭り? なに言ってんだ!今、それどこじゃねーだろ!!	
\\	ゼナンにかかる橋は、魔王軍に壊されてしまったよ。当分、南の大陸へはわたれないって事だな。	
\\	いやもう、うれしくって!ずっと、いなくなっていたリーネ王妃が裏山で見つかったんですよっ!	
\\	今ごろは、お城に戻られてホッとしているころだろうな。	
\\	でもリーネ王妃は、どういうわけかわけのわからない事を口走っているとか……よっぽど、こわい思いをされたんだな。	
\\	いやはや一時はどうなることかと……。リーネ王妃がいなくなって王様は何も手につかなくなるし。	
\\	魔王のしわざじゃないかって兵士たちに国中を探させていたんですよ。しかし、見つかってよかった!	
\\	トマ「あんたもヨソ者かい?   俺は探検家のトマってんだ。   いっぱいおごってくれるんなら   いいネタをやるぜ。	
\\	どーだい?   おごる   おごらない	
\\	おやじ、酒!!	
\\	トマ「こいつはすまねえな。 さて……。   リーネ様が、いなくなったって   話だが……   俺は西のほうにできたという   あの修道院があやしいと   思っているんだ。   え? リーネ王妃はとっくに   裏山の方で見つかったって?   ふーむ…… そうだったのか……。	
\\	トマ「俺は、西にたてられたという   あの修道院が、ぜったいに   あやしいと思っていたんだがな。	
\\	トマ「そーかい。 俺は物ごいじゃねえ。   無理にとは言わねーよ。	
\\	トマ「ちょっとすまねえな。	
\\	俺は、かじ屋のバンタ。ガルディア王にたのまれて、王様からリーネ様への愛のあかし『リーネの鐘』を作ってるのさ。しかしよ、一時はどうなるかと思ったよ。リーネ王妃になにかあったら、鐘だなんて言ってられないからな。	
\\	裏山で見つかった、リーネ王妃も無事お城に戻られたというし。俺も安心してしごとに戻ったってわけよ。	
\\	さいしょくけんびの娘をもつのが夢なんですって。	
\\	せっかく来たんだ、見るだけじゃなくてぜひ買っていってくれ。	
\\	まいど!また来てくれよな。	
\\	そうか。 お前も武器を手にして魔王軍と戦おうというのだな。それはなかなかいい心がけだ。	
\\	しかし武器や防具は買うだけではダメだぞ!その後、装備してこそ意味があるというものだ。	
\\	この橋は、魔王軍にこわされて修理中だ。しかし、サイラス様が戻られるまで、我らが手で王国を死守するのだ!	
\\	モンスターがシェルターを落としていった!	
\\	この先、行き止まり。	
\\	兵士「見かけんヤツだな。   大体なんだ、そのかっこうは?   もしや魔王軍の手先?	
\\	兵士「と、こんな弱々しいヤツが   魔王軍という事もあるまい。兵士「さ、行った行った!   あまりウロウロしていると   ひっとらえるぞ!	
\\	「おやめなさい!	
\\	兵士「リ、リーネ王妃様!	
\\	王妃「その方は私がお世話になったお方。   客人として、もてなしなさい。	
\\	兵士「しかし、こんな怪しい……	
\\	王妃「私の命が聞けないと?	
\\	兵士「めっそうもありません!   どうぞお通りを!	
\\	王妃「フフ……。	
\\	王妃様の客人とは知らず失礼いたしました。	
\\	王妃様とお近づきになれるなんてラッキーなヤツだな。	
\\	この先にガルディア王21世がおわします。	
\\	くれぐれも、失礼のないようにな。	
\\	王「おお、そなたか。   リーネが外で世話になったというのは。   心から礼を言うぞ!	
\\	しかし外で何があったのだ?   リーネの様子がどうもおかしいのだ。	
\\	いつも身につけていたサンゴの   かみかざりも、なくしていたようだし。   あれほど大事にしていたというに……。	
\\	いや、失礼。   そなたの事、騎土団長に話しておこう。   左地下の騎土団の部屋で休んでくれ。	
\\	大臣「うーむ……	
\\	大臣「なんじゃお前は!   え~い、向こうへ行け!	
\\	この先は王のお部屋。騎土団の部屋は、ここを出てすぐに左に行った、かいだんの下だ。	
\\	ここだけの話だが、最近大臣の様子がどうもおかしいのだ。城をよくぬけ出しているようだし……	
\\	おかしいと言えば、リーネ王妃様のごえいのあのカエル野郎、近ごろ姿を見てないな。昔、呪いでカエルの姿にされたらしいがそんな事があるのだろうか。もしや魔王軍のスパイだったのでは?	
\\	大臣様がおかしくなったなんて言う人がいますが、私はそうは思いませんわ。	
\\	大臣様は、とても信心深くなられて西のマノリア修道院に、まいにちのようにおいのりに行かれてるんですよ。	
\\	この先は王妃様のお部屋。お前を通してもよいとの事だ。	
\\	王妃様がお待ちかねだ。	
\\	おや、アンタかい。王妃様が外で世話になったってのは。なんだかサエないね~。	
\\	十数年……。今でもおわかくていらっしゃいます。しかしもどって来られてから、さらにおわかくなられたみたい……。	
\\	王妃「来ましたね。	
\\	王妃「はずしてちょうだい。   この者と話があるのです。	
\\	めし使い「かしこまりました。	
\\	王妃「さ、えんりょせず、もっと近くへ。	
\\	王妃「プッ……。	
\\	王妃「なーんてね、来てくれたんだ   クロノ!	
\\	マール「そ、私よ!   何だかみんな私の事リーネって言うの。	
\\	マール「でも、うれしかった。	
\\	マール「ほんの少し、お祭りでいっしょ   だっただけなのに、来てくれて。	
\\	マール「クロノ……	
\\	ありがと……。	
\\	マール「な、何!?	
\\	マール
\\	な、何コレ?   心がバラバラになってくみたい……	
\\	マール「こ、こわいよ!   私がなくなってしまうみたい……。   た、助けて、クロ……	
\\	ね、リーネ王妃様、なんだか変じゃなかった?   はい   いいえ	
\\	やっぱり! 私はリーネじゃないって言うしなんとなく前とはちがうのよ。	
\\	あ! コレないしょよ!	
\\	そ。 なら、いいけど。	
\\	あら! もういいのかい?ところであんた、リーネ王妃様に……	
\\	変なコトしなかったでしょうね?   はい   いいえ	
\\	ホントかねえ……。	
\\	おやおや、そんな事知れたら大変!わたしゃ何も聞かなかったからね。	
\\	まったく女というものは、いつの世もうわさ話が好きなものだな。	
\\	「クロノ!	
\\	ルッカ「ハアハア……無事みたいね!   それより、あのコは?	
\\	ルッカ「何ですって、消えたあ!?	
\\	ルッカ「……やっぱりね。	
\\	ルッカ「あのコが消える時、どっかで   見たかおだと思ったのよ。	
\\	ルッカ「ここは、王国は王国でも   ずいぶんと昔の王国みたいね。	
\\	ルッカ「あのコは、自分のご先祖様に   間違えられたってワケよ。   あの子は私達の時代でも、お姫様……	
\\	ルッカ「マールディア王女なのよ!	
\\	マール、つまりマールディア王女はこの時代の王妃の子孫なの。	
\\	この時代の王妃がさらわれた……	
\\	本当はそのあと、誰かが助けることになっていたの。でも歴史は変わってしまった……	
\\	つまり、マールがこの時代に現われ王妃に間違えられたために本物の王妃のそうさくがうち切られてしまった。	
\\	もし王妃が殺されてしまったら…	
\\	マールの存在が消えてしまうの……でもまだ間に合うわ!今からでも王妃を助け出すことができれば歴史はもとにもどるはず!!	
\\	ルッカ「おそらく、この時代の   王妃の身に何かあったんだわ。   だから、子孫であるあのコの存在   そのものが……	
\\	ルッカ「とにかく、本物の王妃の行方を   探さなきゃ!	
\\	消えた王女	
\\	大臣「なぜ戻って来られた?   一人であそこから、ぬけ出せるはずが   
\\	休むなら下に行きな。	
\\	地下は騎土団の部屋だ。	
\\	ゼナンの橋は最終防えい線。なんとしても死守せねばならん!	
\\	お前か、王妃様を外で世話したというのは。休むのならメイドに言ってくれ。	
\\	お休みになられます?   はい   いいえ	
\\	お目ざめはいかがです?	
\\	ああ! こんな時サイラス様がいてくれたら!	
\\	お主も名前くらいは知ってるだろ?   はい   いいえ	
\\	あの方さえいれば、王妃様がいなくなるなんて事もなかっただろうに……	
\\	なんとサイラス様を知らぬのか!あれほどの勇者を知らぬとはお主、どこに住んでおるのだ。	
\\	サイラス様は騎土団長だったお方。いつも、王やリーネ様は自分が守るとおっしゃっておられたのに……	
\\	十年近く前に、おともを一人つれ旅に出られそれっきりなのです。	
\\	昔、西の森あたりは、ヤクラという魔物が住んでいて、誰も近寄らなかったんだ。だが少し前、その森の中に修道院がたてられてからは魔物も出なくなったのだ。さすが神のごかご。 ありがたい事だ。	
\\	いやはや、王妃様が見つかって本当によかった。これで戦に、せんねんできる!	
\\	うう……	
\\	このところ魔王軍はナリをひそめているが、かえって不気味だ。	
\\	この下はしょくどうだ。	
\\	ウロチョロして料理長にどなられんようにな。	
\\	ここは俺達兵士のしょくどうさ。お前も食ってけよ。	
\\	せいつけて、魔王軍をぶちのめしてやる!	
\\	料理長の作るメシは最高さ。三つ星ってトコかな。	
\\	ハ……	
\\	ハラへった……	
\\	おかみさん「あいよ、お待ち~!	
\\	兵士「ヒャッホ~ッ!	
\\	うまうま!	
\\	食べてくなら、左のテーブルの左上のせきにどうぞ。	
\\	あ~いそがし! ここも戦場だね。	
\\	しろうとが、ちゅうぼうに入ってくんじゃねえ!しかし、王妃様が食べたいというクレープってな一体何だ?	
\\	騎土団長「食事はまだか?	
\\	ウェイトレス「あっ騎土団長!   すみません……	
\\	騎土団長「我々は命がけで戦ってるのだ。   早く持って来い!	
\\	料理長「るせえ! 命はってんのは   騎土団だけじゃねんだ!	
\\	団長「フン、私はおくびょう者が作った   物など食うヒマがないが、部下達には   しっかり食わせてやれよ。	
\\	料理長「てやんでえ! 言われなくたって   そうすらあ!	
\\	あのヤロウ、てめえ達だけが戦ってると思ってやがる!	
\\	騎土団長は強くやさしいお方。だが料理長とはいつもこれだ。	
\\	騎土団長はかつての騎土団長サイラス様の後をついでがんばっているのだ。	
\\	あの2人、兄弟なんですけど仲が悪くて……	
\\	ウチの人は殺し合いがキライでこの道をえらんだのよ。	
\\	何にする?   リフレッシュサラダ   スタミナシチュー   ハイパーまるやき	
\\	ちょっと待ってね。	
\\	はい、お待たせ!	
\\	が回復した!	
\\	が回復した!	
\\	が回復した!	
\\	料理長はキビしいけど、いい人よ。	
\\	へえ、キミがリーネ様を……	
\\	私達は、世界の平和のためにいのっておりますですことよ。ウフフ……。	
\\	一日もはやく、私達の明日がとりもどせますように……。あら、おいしそうなニンゲ……	
\\	い、いえ、なんでもありませんわ、オホホ。	
\\	われらが、いだいなる主よ。かよわく、力なき者達に、さばきのいかずちを……。 イヒヒヒ……。	
\\	さあ、あなた達も、かわいそうな自分達のために、いのりをささげてはいかがですか。 ククク……。	
\\	ルッカ「何か見つけたの?	
\\	ルッカ「かみかざり……?	
\\	ルッカ「これ、ガルディア王家の   紋章じゃない!	
\\	ルッカ「ふ~。 ビックリした。	
\\	ルッカ「キャアッ!!	
\\	「最後まで気をぬくな。 勝利によいしれた時こそ スキが生じる。	
\\	「お前達も王妃様をお助けしに来たのか?	
\\	この先はヤツらの巣みたいだな。 どうだ、いっしょに行かないか?	
\\	ルッカ「あ、あなたは……!?	
\\	クロノ、知ってるでしょ。   私、カエル苦手なの……!	
\\	「まあ、こんなナリをしていては 信用しろと言っても無理か……。	
\\	「いいだろう、好きにしろ。 だが王妃様は俺が助け出さなきゃ ならないんだ……。	
\\	ルッカ「ちょ、ちょっと……。	
\\	ルッカ「悪いカエ……、人では   なさそうね……。   うーん……。	
\\	ルッカ「どうする、クロノ?   カエル男といっしょに行く   いっしょに行くのはイヤ	
\\	ルッカ「でも、私達だけじゃ   この先ちょっと不安だわ……。	
\\	ルッカ「ガ……、ガマンして   いっしょに行きましょう!	
\\	ルッカ「あなた、お名前は?	
\\	カエル「カエルでいいぜ。	
\\	ルッカ「ええ、それじゃ……   お手やわらかにね、カエル。	
\\	カエル「ああ。きっと、この部屋の   どこかに、かくし通路があるはずだ。   そこから奥へ行けるだろう。	
\\	まさか……、人間か!?	
\\	この奥に王妃様がとらわれているのだ。どうか王妃様をお助けしてくれ!	
\\	この建物のどこかに、にっくき魔王をまつった部屋があるらしい……。そこにはお宝もあるようだぜ。	
\\	ひみつのミアンヌ・ブロマイド
\\	を見つけた!!	
\\	おい、こいつら、人間だぜ、兄キ!	
\\	俺達の、お宝ブロマイドを……!!	
\\	女の子なんか、いじめちゃえ!!	
\\	んん……、なんだあ?ヒック!	
\\	お前らも、はやくもとの姿にもどれや。いつまでも人間のカッコなんかしてないでよ。ウイック!	
\\	ち、もうすぐ交代の時間か、ヒャック。あー、イヤだ、イヤだ、人間に化けるのは。ダサイし、くさいし……。	
\\	よし、交代する前にもう一度魔王様のお姿でも、おがんでくるとするかな。ヒャック!	
\\	こないだ食べた連中、そんなにマズくなかったわよね。とじこめてある、あのふたりの兵士もそのうちに……。キシシシ……。	
\\	ヤクラ様の計画はバッチリだな!大臣をつかまえて、自分が大臣に化けるとはいや~、さすがヤクラ様だぜ。	
\\	そのおかげで王妃もカンタンにさらってこられたし、いまごろは城の連中てんてこまいだろうよ。ウヒャヒャ。	
\\	よくぞ、来られた。ここの主人ヤクラ様もきっとおよろこびになるであろう。ささ、ヤクラ様がアイサツにみえるまでこの部屋でゆっくりと休むがよいぞ。	
\\	私を助けに来てくれたのですね。しかし、それにはおよびませんよ。	
\\	ここの主人のヤクラ様は、とてもよい方。私は、ここにすむことにしました。さあ、あなた達も楽になさって。	
\\	も、もしや、リーネ王妃様を助けに来られた方々では!?おお、やはりそうであったか!	
\\	そんな、あわてて出ていかなくともどうぞ、ゆっくりしていってください……なギャギャギャギャ……!!	
\\	ああ 魔王様~
\\	魔王様~
\\	輝くひとみ そよぐかみ
\\	山より高く 海より深い
\\	どんなにはれた、さびしいひるも
\\	あなたを思えば、こわくない
\\	どんなに明るく、かなしい場所でも
\\	あなたがいるなら、へっちゃら ぷー
\\	コーラス ああ 魔王様ったら
\\	魔王様~
\\	われらが すくいの
\\	魔王様~
\\	さあ、お前達も声をあわせてうたうんだよ、お前達自身のとむらいの歌!ウギャギャッ!!	
\\	待てッ! クセ者どもめが!!ゲハゲハ……!	
\\	はりがみ「ここは、通れません。	
\\	はりがみ「行こうと心を決めた者のみが   行ける道がある。	
\\	グガー……、ゴガガー……。	
\\	う~ん、いや~ん、バカ。だから、そこはダミだってばさ……。	
\\	ハアッ!?	
\\	この私の後ろをとるとは……!?ただ者ではないな、きさまら!	
\\	「かくごはいいかな、リーネ王妃? この世にさよならをつげる時間だ。	
\\	大臣「ほう、お前達……!!   よくここまで、もぐりこめたな。	
\\	リーネ「カエル!	
\\	カエル「王妃様、お下がりください!   こいつを、かたづけちまいますので。	
\\	リーネ「気をつけて……。	
\\	大臣「ギャハハ……! ムダ、ムダ!   ここからは誰一人として   生かして帰さぬぞ!	
\\	大臣「ハッ! カエルふぜいが……!   きさまらから血まつりだ!!	
\\	大臣「だいじーん……   チェーンジッ!!	
\\	ヤ~クラ~ッ!! デロデロン!!	
\\	リーネ「やはり、来てくれたのですね。   ありがとう、カエル……。	
\\	カエル「王が心配しておいでです。   城に戻りましょう。	
\\	カエル「助かったぜ、クロノ。   ルッカ。	
\\	リーネ「ありがとうございました。   あなた方も、どうか共にお城の方へ   いらしてください。	
\\	リーネ「ええ。   それでは、お城へもどりましょう。	
\\	本物の大臣「ま、待ってくれ~い!   本物のわしは、これ、このとーり   ぴんぴんしとるぞ!!	
\\	本物の大臣「ふーっ、助かったわい!   あの化け物め、あんなとこに   おしこめおってからに……!	
\\	大臣「おお、王妃様!?   本物のわしは、これこのとーり。   ぴんぴんしておりますですじゃ!	
\\	ガルディア王「心配したぞリーネ。	
\\	カエル「リーネ様を守りきれず   めんぼくしだいもございません。	
\\	大臣「あのヤクラのヤツめ。   このワシになりかわり   リーネ様をさらうなど……	
\\	ああいうやからをキビしくさばくため   このガルディア王国にも、裁判所や   刑務所を作らんといけませんな!	
\\	れいを言うぞ、クロノ殿。しかし、すると裏山で見つかったあの娘は一体誰なのじゃ……。	
\\	リーネ「助けていただき、本当に   ありがとうございました。   あと少し、あなたがたの来るのが   おそければ、私はどうなっていたか……	
\\	ところで、私にまちがえられたという   娘は今はどちらに?	
\\	ルッカ「そうだわ!   すっかりマールディア姫の事を   わすれてたわ!	
\\	ルッカ「ね、クロノ!   マールディア様は、どこで消えた?   もしかしたら、そこに……。	
\\	やっぱり本物のリーネ様は別の所にいたわけね。私のすいり通りだったわ。	
\\	けど本当にリーネ様そっくりでわたしゃビックリだよ。	
\\	ルッカ「マールディア王女!	
\\	マール「あ、あれ?	
\\	マール「クロノ!	
\\	マール「こわかった……	
\\	いしきがないのに、冷たい所にいるのが   わかるの。   死ぬってああいう感じなのかしら?	
\\	ルッカ「マールディア王女様   ごきげんうるわしゅう……	
\\	マール「あなたも来てくれてたの!   マールディアって……   え!?	
\\	マール「バレちゃったみたいね……。	
\\	マール「ゴメンね、クロノ。   だますつもりじゃなかったの。	
\\	マール「私はマールディア。   父はガルディア王33世……。	
\\	けど私だってお祭りを男の子と   見てまわりたかったんだもん。   私が王女だってわかってたら……	
\\	マール「クロノは、いっしょに   お祭り見てくれなかったでしょ?   はい   いいえ	
\\	マール「でしょ?   私は王女様よりマールって   いわれる方が好きだもん!	
\\	マール「さっすがクロノ!   私達、友達よね!	
\\	マール「本物の王妃様も無事に   もどったんでしょ?   私達も帰ろうよ、クロノ!	
\\	ふーむ、見れば見るほどリーネ様に似ていらっしゃる!	
\\	あんた達、帰るのかい?どこまで帰るか知らないけど気をつけてお行きよ。	
\\	これで騎土団長も対魔王軍にせんねん出来る。	
\\	おお、ほんにリーネ様に生きうつしじゃ!	
\\	おお、無事じゃったか。たしかに外見はリーネだが、中身はぜんぜん……	
\\	いや、そなたらには世話になった。何かあれば国をあげて力になるぞ。	
\\	マール「あなたが本物のリーネね!	
\\	リーネ「まあ、ほんとに私にうりニつ。	
\\	マール「ちゃんと王様と仲良くしないと   私おこっちゃう……	
\\	事も出来ないのか。	
\\	リーネ
\\	マール「ううん、とにかく   おしあわせにね!	
\\	あなたがたは、何となく他人とは思えませんわ。	
\\	お二人がそろわれるのは、ひさしぶりだ。	
\\	やはり、リーネ様ではなかったか。	
\\	リーネ様じゃなかったとはな。	
\\	あのカエル野郎にも、そんな勇気があったか。	
\\	わからんぞ、スパイと思わせないための行動かもしれん。	
\\	魔物が大臣様に化けてたなんて……もっともニセモノの方が品はよかったけど。	
\\	カエル「俺が近くにいたため   王妃様を危険にさらしたのだ……。	
\\	俺は城を出る。	
\\	マール「キャッ!	
\\	カエル「なるほど、王妃様によく似て   おいでだ。	
\\	カエル「クロノ、お前の太刀すじは   なかなか見込みがあったぞ。	
\\	ルッカ
\\	カエルも悪くないもんね。	
\\	お前はこれから、かおパスだな。	
\\	ヤクラを倒すなんて、お前ホントラッキーだな。	
\\	これで、王もご安心だ。	
\\	やはりリーネ様がいると城内が明るいな。	
\\	そうか、あの修道院に……カエルのヤツも、りっぱにお役目をはたしたな。	
\\	しかし、魔王軍もリーネ様ゆうかいにしっぱいした以上、ほんごしを入れて来る。こんな時、サイラス様がいれば……。	
\\	あとは魔王軍をたたくのみ!	
\\	ヤクラも魔王軍のさしむけた魔物だったか。	
\\	修道院が魔物のアジトとはな。	
\\	リーネ様ゆうかいも、やはり魔王軍のしわざか。	
\\	本物のリーネ様が?こりゃ、ごちそうを作んないとね!	
\\	リーネ様じゃなかっただと?なんでえ、せっかくクレープを作ってみたのに。クッキーとレバーの入ったスープだろ?	
\\	リーネ様なら、ここに来てつまみ食いなんてしないものね。	
\\	リーネ様にしては、よく食べると思った。	
\\	あのカエル野郎、およいで向こうへわたって行きやがった。	
\\	マール「どこから帰るの?	
\\	ルッカ「おそれながら、マールディア   王女……	
\\	マール「マールでいいってば!	
\\	ルッカ「で、ではマール……。   ごらん下さい。	
\\	マール「ルッカ、すっごーい!	
\\	ルッカ「オホホホ……!	
\\	ルッカ「し、失礼しました……!	
\\	マール「話し方もふつうでいーの!   ルッカの方がスゴいんだよ!   私は王女かもしれないけど、なんにも   出来ないんだから。	
\\	ルッカ「では、お言葉にあまえて……。	
\\	ルッカ「私は、このひずみに   『ゲート』って名付けたんだけど……	
\\	ゲートは、違う時代の同じ場所に   つながっている『門』の様な物なのよ。	
\\	出たり消えたりするのは   ゲート自体が不安定だからなの。   そこでテレポッドの原理を応用して   この……	
\\	ルッカ「ゲートホルダーを使って   ゲートを安定させてるってワケ。	
\\	マール「けど何で、このゲートが   あの時突然現われたの?	
\\	ルッカ「テレポッドのえいきょうか   あるいはもっと別の何か……	
\\	マール「何だかムズかしいんだね……。   とにかく帰ろうよ私達の時代へ!	
\\	ルッカ「うん、そうね。   帰りましょ クロノ!	
\\	ただいま!	
\\	マール「わーい!   やっと帰って来られた!	
\\	マール「ね、クロノ、ルッカ!   お城までいっしょに来てくれる?   色々お世話になっちゃったし!	
\\	ルッカ「ゴメンなさいねマール。   大変な目にあわせて。	
\\	マール「ううん、スリルあって   とっても楽しかったよ!   ステキな友達も出来たし!	
\\	ルッカ「お城へは、クロノ   あんたが送りなさいよ。   ちゃ~んとエスコートしてあげるのよ。   私はゲートの出てきた原因を   調べてみるわ。	
\\	マール「またねー ルッカ!	
\\	マール「じゃ、お城までエスコートして   クロノ!	
\\	王女様がお城をぬけ出したそうよ。	
\\	広場の商人は新商品のしいれのためみんな家に帰ったよ。町のグッズマーケットによってみな。	
\\	ジナ「お帰り、クロノ。	
\\	ねえ、知ってる?なんでも昔は、西の森に修道院がたっていたことがあったんだって。でも、どうしてなくなってしまったのかしらね……。	
\\	ははっ、年がいもなくシルバーポイントをためるのに夢中になっちまったよ。	
\\	今日もエレインさんは船つき場か。フリッツのやつ……。本当に、どこいっちまったんだ……	
\\	あっ!いらっしゃいませなんのご用でしょうか?	
\\	お金がたりません。	
\\	まいど、ありがとうございました。	
\\	ララ「ルッカもタバンも発明で   いそがしくて。	
\\	ルッカ「うーん……   2本足で完全なバランスをとって   あるくロボットの小型化は   やはりむずかしい。   あ! クロノ! 早く王女を   お城までつれていってやりなよ。	
\\	タバン「ほら、これが今日のかせぎで   手に入れたリンゴだ。	
\\	ララ「いつもすまないね。	
\\	「マールディア様~!	
\\	大臣「ご無事でしたか?   一体、今までどこに!?	
\\	何者かにさらわれたという情報もあり   兵士達に国中を探させていたのですぞ!	
\\	大臣「ム! そこのムサいヤツ!   そうか、お前だなッ!?   マールディア様をさらったのは!	
\\	マール「違うよ! クロノは……	
\\	大臣「えーい! ひっとらえろ!   マールディア様をまどわせ   王家転ぷくをくわだてる   テロリストめッ!!	
\\	マール「や、やめてーッ!	
\\	マール「やめなさ~いッ!!	
\\	大臣「な、何をしておる!	
\\	兵士「しかしマールディア様が……	
\\	大臣「かまわ~ん! ひっとらえ~いッ!!	
\\	マール「クロノーッ!!	
\\	私がベンゴシのピエールです。	
\\	私がケンジの大臣じゃ!	
\\	大臣「ようこそみなさん。	
\\	マールディア王女ゆうかいの罪で   うたがわれているクロノという   男をここに連れてまいりましょう。	
\\	大臣「この男をどうしましょう…   火あぶり? くすぐりのけい?   さかさづり?	
\\	それとも……ギロチンでクビを……。	
\\	大臣「それを決めるのはみなさんです。   さ、始めましょう。	
\\	裁判長「では、ヒコク人クロノ!   証言だいにつきなさい。	
\\	ピエール「まずは私からいきましょう。   クロノに本当にゆうかいの   いしがあったのか? ……いや無い。	
\\	ピエール「ケンジがわはヒコクが計画的に   王女をさらったと言いますが   そうでしょうか? ……いやちがう。	
\\	2人はぐうぜん出会ったのであって   決して故意ではありません。	
\\	大臣「はたしてそうでしょうか?   どっちがきっかけをつくりましたか?   私です。   マールです。	
\\	大臣「よろしい!   聞いての通りぐうぜんをよそおって   ヒコクは王女に近づきました。	
\\	大臣「本当にそう言い切れますか?   あなたからぶつかったという目撃者が   いるのです。	
\\	大臣「そして王女はさそわれるまま   ルッカ親子のショーへ   足を運びます。   その姿は何人もの人が目撃して   います。   そして2人は姿を消した……。   これがゆうかいじゃなくして   一体何でしょう?	
\\	大臣「ヒコクの人間性が   うたがわれる事実も私はいくつか   つかんでいます。	
\\	ピエール「異議あり!	
\\	それは今回のけんにかんけいあるので   しょうか? ……いや無い。	
\\	裁判長「かんけいあるのかね? 大臣。	
\\	大臣「はい。 証言の正しさを   しめすためにもヒコクの人間性を   知らせておく必要があります。	
\\	ピエール「いいでしょう。	
\\	ピエール「では   証人を連れて来ましょう。   ヒコクのせいじつさを   証明するかわいい証人を。	
\\	この人はね……わたちのネコちゃんを連れて来てくれたの。	
\\	あの時はありがとうね。	
\\	ピエール「どうです?   このわか者の行動は?   くんしょうものですよ。	
\\	ピエール「くく……   きいてるみたいよんっ。	
\\	大臣「ヒコクのれいこくなせいかくは   明白なものであります。	
\\	大臣「身におぼえがあるだろう?   ネコちゃんを見失ったいたいけな少女の   たのみも聞けないお前は……	
\\	わたちの事ムシしたでしょ。	
\\	大臣「おお……、かわいそうに……   どうもありがとう。	
\\	大臣「もう何の説明もいりますまい。	
\\	大臣「あなたは盗みをはたらいた事は?   あります。   決してありません。	
\\	大臣「ほら、この通り。   ヒコクはモラルにかけております。   それは自他共にみとめる事なのです。   聞いてください。 ひつうなさけびを!	
\\	大臣「とぼけるのはいけません。   証人がいるのです。	
\\	オラが大事にしてた、べんとうさァコイツが食べてしまっただ。母ちゃんがいっしょうけんめい作ってくれたべんとうだべ!	
\\	大臣「まじめにくらしている人の   わずかな楽しみをもうばいさるヒコクに   同情のよちはありますまい。	
\\	ピエール「もんだいは動機です。   この一市民にマールディア王女を   ゆうかいする動機がどこに   ありましょう? ……いや無い。	
\\	大臣「お言葉を返すようで悪いが   ざいさん目当てと言うのはどうかな   クロノ君?   王女のざいさんに   目がくらんだのだね?   いいえ   はい	
\\	大臣「すなおにはいたようだね。	
\\	大臣「もうあえて私から言う事も   ないでしょう。   裁判長。 判決を。	
\\	大臣「本当にざいさんには   きょうみが無いのかね?   きょうみない   すこしだけ	
\\	大臣「ま、まあいいでしょう。   私のじんもんは終わります。	
\\	ピエール「見ての通り正ぎ感の強い   少年です。 さあ裁判長。   判決を。	
\\	大臣「みなさん、今の言葉をしっかりと   頭に入れておいて下さい。   ヒコクはざいさんにはきょうみ無いと   言いました。   では証人を。	
\\	やーね、私見ましたのよ。	
\\	あらま!何か私きんちょうしてきちゃったわ!	
\\	その人、王女様とぶつかって転んだ時にまっ先にペンダントをひろいに行ったわ。王女様を気づかうよりも先に……。	
\\	王女様にペンダントを返すのをこばんだりしてたみたい……。	
\\	ボッシュって物売りに、売り飛ばそうとしていたのよ。	
\\	大臣「最後に決定的な証言をして   しめくくってもらいましょう。	
\\	私は聞いたのよ!!王女様が言った言葉をこのみみではっきりと!	
\\	あら、もう終わり?やーね。	
\\	裁判長「ばいしんいん達よ。   有罪と思う者は左へ   無罪だと思う者は右へ行きなさい!	
\\	有罪	
\\	無罪	
\\	せいしゅくに! せいしゅくに!	
\\	判決が出た!	
\\	ぜんいんいっち!!	
\\	有罪とする。ただちにどくぼうへ。3日後には死刑をしっこうする!	
\\	無罪とする!!	
\\	……しかしだ。 ゆうかいのいしはなかったにせよ、マールディア王女をしばらく連れ出したのは事実。よってはんせいをうながすため3日間のどくぼう入りを命ず!!	
\\	大臣「さ、連れていけ!	
\\	マール「待って!!	
\\	大臣「お、王女様……	
\\	ガルディア王「いいかげんにしなさい!   マールディア!	
\\	マール「父上! 聞いてください!	
\\	ガルディア王「私はお前に   王女らしく城でおとなしくしていて   ほしいだけだ。   国のルールにはたとえ王や王女でも   したがわなくてはな……。   あとの事は大臣にまかせておきなさい。   マールディアも町での事は   わすれるのだな。	
\\	ガルディア王「さ、いくぞ!	
\\	マール「クロノーーー!!	
\\	大臣「こいつは、王家てんぷくをはかった   テロリストだ。	
\\	大臣「裁判で有罪となったので   引きわたしに来た。	
\\	所長「こやつが王女さまをゆうかいした   テロリストですか。	
\\	大臣「こいつの処刑は3日後だ。   にがさぬように、見はっておれよ。	
\\	所長「しょうちいたしました。	
\\	所長「処刑? そのような話は   聞いておりませぬが……	
\\	大臣「気にするな。   手続きのしょるいが   おくれているだけだ。   それとも、わしの言葉が信じられぬか?	
\\	所長「め、めっそうもない!	
\\	所長「えいへい!	
\\	所長「こいつをつれてゆけ!	
\\	王国裁判	
\\	処刑の日まで、あと
\\	日……	
\\	そして、処刑の日が来た……	
\\	が回復した!	
\\	それは、刑務所行きとなったおまえに同情した裁判のぼうちょうにんが持って来たさしいれだ。 
\\	エーテルを手に入れた!	
\\	うるさいぞ!	
\\	うるさいと言っているのが聞こえんのか!この野郎、だまらせてやる。	
\\	おい、扉を開けろ!	
\\	へっ、いつまでもさわいでいるからイタイめにあうんだ。	
\\	き、キサマ! 脱走するつもりか!	
\\	脱走者だ!!	
\\	キサマ、そこで何をしている!	
\\	キサマが逃げ出した囚人か!	
\\	こ、こりゃ たまらん。	
\\	た、助けて!	
\\	助けますか?   はい   いいえ	
\\	ありがとう、助かったよ。僕の名前はフリッツ。僕の父さんは、トルースの町でグッズマーケットをけいえいしているんだ。近くに来たら、ぜひたずねてきてよ。	
\\	所長「ひ、ひえ~! お助け~!!	
\\	ルッカ「クロノ!   助けに来たわよ!	
\\	ルッカ「……っていっても   自力で逃げ出せたのね。   ……、なんか、つまんないなあ。ルッカ「さあ、こんな所とは   さっさとおさらばしましょう!	
\\	きぜつしている……	
\\	ミドルポーションをいただいた!	
\\	なんと! 5コも見つけた!!	
\\	処刑場へ連行する。	
\\	さあ、来るんだ。	
\\	ギロチンの刑をしっこうする。	
\\	何か言い残す事はないか?	
\\	キサマ、何者だ!	
\\	ルッカ「あんた達、ジャマよ!	
\\	ルッカ「しばらくねむってなさい!	
\\	ルッカ「どう? ルッカ様じまんの   使いすてドッカンばくはつピストルの   お味は?	
\\	ルッカ「このルッカ様の   行く手をはばむオロカモノは   みんなこうなるのよ。       オーッホッホッ!	
\\	ひみつのしょるいが落ちている……読んでみますか?   はい   いいえ	
\\	ガルディア王国刑務所所長殿へ	
\\	ドラゴンせんしゃのせっけいず	
\\	ドラゴンせんしゃの頭には、本体が受けたダメージを回復させる機能がついています。この頭には天、火ぞくせいの技を防ぐシールドが装備されていますので剣やピストルによる攻撃でしかダメージをあたえる事が出来ません。	
\\	つまり、剣などの攻撃で頭を壊されない限りドラゴンせんしゃがたおされる事はないと考えてよいでしょう。         ガルディア王国開発部より	
\\	ルッカ「何の音かしら?	
\\	大臣「ゆけ、ドラゴンせんしゃ!   テロリストどもを たたきのめせ!	
\\	大臣「ド、ドラゴンせんしゃが!!   修理するんだー!!	
\\	大臣「き、きさまら、おぼえておれよ!	
\\	「脱走だーッ!	
\\	ルッカ「ここはマズいわよ!	
\\	ルッカ「ここは強行突破しかないわね!	
\\	「おやめなさい!	
\\	兵士「マールディア様!	
\\	マール「その方は私がお世話に   なったのよ!   客人として、もてなしなさい!	
\\	兵士「し、しかし……	
\\	マール「私の言う事が聞けないの?	
\\	兵士「いえ、めっそうもありません!	
\\	「そこまでじゃ~ッ!	
\\	大臣「え~い、頭が高ーい!   ガルディア王33世様のおな~り~ッ!マール「父上……。	
\\	王「いいかげんにしろマールディア。	
\\	お前は、一人のこじんである前に   一国の王女なのだぞ。	
\\	マール「ちがうもん!   王女である前に一人の女の子なの!	
\\	王「城下になど出るから   悪いえいきょうを受けおって!	
\\	マール「えいきょうじゃない!   私が決めた事だもん!	
\\	王「マールディア!	
\\	マール「こんなトコもういたくない!   私、城出するわ!	
\\	大臣「マールディア様!	
\\	マール「行こう! クロノ!!	
\\	大臣「何をしておる!   追えッ! 追え~いッ!!	
\\	王「マールディア……。	
\\	兵士「いたぞ!   逃がすな!!	
\\	兵士「もう逃げられんぞ!!	
\\	兵士「おとなしくしろ!	
\\	マール「行き止まり!?	
\\	ルッカ
\\	ゲートが!	
\\	マール「行こう!	
\\	ルッカ「行こうって……   どの時代に出るか、わからないのよ!   今度は帰って来れるかどうかも!!	
\\	マール「それでもいい!   私のためにクロノが   つかまっちゃうこの時代よりは!	
\\	大臣「マールディア様~!	
\\	ルッカ「えーい、もうどうとでも   なれだわ!	
\\	マール「行こう、クロノ!	
\\	大臣「王女、その男からはなれるのです!	
\\	大臣「き、消えた……。	
\\	廃墟をこえて……	
\\	ルッカ「いった~……	
\\	マール「さすがにここまでは、大臣も   追って来られないわね。	
\\	マール「けど、ここ……どこ?	
\\	ルッカ「ずいぶんと文明は発達   してるみたいだけど……	
\\	マール「まるで別の星に来ちゃった   みたいだね。	
\\	不思議な紋章がきざまれている。何かの力によって封印されているようだ。	
\\	食い物なら、ない……。	
\\	トルース?そんなドーム、聞いたこともない。ここはトランドームだ。	
\\	これを見てると、不思議と生きようって気がしてくる……。	
\\	王国暦? 何それ……?	
\\	北東のアリスドームに行けりゃ、食い物にありつけるらしい。16号廃墟をこえられればな……。	
\\	あそこに出るゆうれいにゃ、剣やガンなどの物理攻撃はききやしねえ……。	
\\	こいつは、エナ・ボックス。わずかな時間で、ぐっすり眠れるがハラは、みたされやしねえ……。	
\\	エナ・ボックスで休みますか?   はい   いいえ	
\\	しかし、空ふくはみたされなかった。	
\\	あんた金持ってるか?   はい   いいえ	
\\	見たことない金だな。ま、いーや、いいブツがあるぜ。	
\\	ひやかしに用はねーよ。	
\\	金さえためときゃ、いつか役に立つ。オイラはそう信じてんだ。	
\\	盗まれた!	
\\	ポーションを盗まれた!	
\\	老人「あ、あんた達、どっから   来なさった……。	
\\	ルッカ「西の廃墟をこえてですけど?	
\\	老人「な、何?	
\\	老人「おい、あの廃墟をこえて来た   者がおったぞ!	
\\	「え?	
\\	「本当か?	
\\	老人「あそこのミュータントと   戦えるほどの者がおったか……。	
\\	老人「もうしおくれた、私はドン。   かつてここ情報センターの   所長だった者のまつえいじゃ。	
\\	ここの地下には、大型コンピューターや   食料等のほかん庫もある。   しかし、けいびロボットが、いまだに   動いていて近づけないのじゃ。   ひにくなモンじゃよ……。	
\\	だれ? お兄ちゃん達……。	
\\	主人は地下の食料庫をめざして……それっきりなんです……。	
\\	お父さんが、もうすぐ食べ物持って来てくれるの。 
\\	けど、あんたら、何か変だな……本当に西のドームから来たのか……?   はい   いいえ	
\\	へえ、そうかい……。どっちでもいいけどよ……。	
\\	え? 時空のゆがみ? 何だいそれ……?ま、地下の大型コンピューターが動かせりゃいろいろと調べられるかもな……。	
\\	東にはプロメテドームと工場跡がある。暴走したロボット達であふれかえっていてとても近寄れないがね。	
\\	南の地下水道をぬけた大陸には行かない方がいいわ。大災害の源……『死の山』があるから。	
\\	もっとも地下水道に巣くっている強力なミュータントは、さすがにあなた達でも……	
\\	あの紋章がついたドア?押しても引いても開かんし、壊そうとしても傷一つつかん。	
\\	死の山近くにドームを作り物好きに監視しているジーサンがいるぜ。	
\\	お金持ってまちゅか?   はい   いいえ	
\\	見たことないお金でちゅね。ま、いーや、いいブチュがあるじェ。	
\\	ひゃーかしはヤでちゅ!	
\\	お金ちゃえためときゃ、いつか役に立つ。	
\\	廃墟の向こうに行ったあんちゃんはそう言ってまちた。	
\\	ドン「お前達、地下に行くつもりか?	
\\	マール「もっちろん!	
\\	ドン「私らが何度いどんでも地下には   行けなかったのだぞ。	
\\	ルッカ「やってみなきゃ   わからないでしょ?	
\\	ドン「……。	
\\	お前達のような生き生きした   わか者を見るのはひさしぶりじゃ。	
\\	ドン「気をつけてな。   そして生きてもどって来いよ。	
\\	ルッカ「通路を制御するコンソールね。   パスワードでロックされているみたい。	
\\	ただの置物のようだ。	
\\	きれいなかみにメモがある。	
\\	ケイコクスル   ショクリョウコニ チカヅクモノハ   コウゲキスル!	
\\	ヨテイプログラムヲ ジッコウセヨ。	
\\	マール「クロノ!   何が起こったの!?	
\\	ルッカ「マール!   いくわよ!	
\\	マール「な、何、このニオイ!?	
\\	ルッカ「ひどい、全部くさってる……。   電源が不安定だったのね……。	
\\	マール「クロノ!!	
\\	もう、いきはないみたい……。	
\\	マール「何か、にぎっているわ。   何だろう、これ……?	
\\	ルッカ「何かの種子みたいね。	
\\	マール「タネ……?   こんな所で育つのかしら?	
\\	きたないかみにメモがある。   あのネズミは ただの置物じゃない。   ドームのヒミツを知っている。   鉄の上のネズミをつかまえろ!	
\\	マール「いたいた!   あいつのことだわ。	
\\	ルッカ「しー!   逃げるかもしれないわね。   
\\	ボタンダッシュして   
\\	ボタンで、つかまえるのよ!	
\\	つかまえた!	
\\	「かんべんでチュー。   はくじょうするでチュー。   通路スイッチは、
\\	と
\\	を押し   ながら
\\	ボタンを入れるでチュー!   でも、ボタンの設定変えてた場合は、   こんがらがっちゃうでチュー!	
\\	ルッカ「ここが情報センターね!	
\\	ルッカ「よかった!   このコンピューターは、まだ生きてる!   空間のゆがみをサーチすれば   ゲートのありかがわかるかも!	
\\	ルッカ「お待たせ、ここよ!	
\\	ルッカ「このアリスドームから   東の……	
\\	ルッカ「プロメテドームね!	
\\	マール「さーすが、ルッカ!   この機械で、そんな事もわかるんだ。	
\\	マール「じゃ、ここ押したら   何がわかるの?	
\\	ルッカ
\\	『ラヴォスの日』記録……?	
\\	マール「な……	
\\	何、これ……	
\\	ルッカ「ラ、ラヴォスって……   これが世界をこんなにした大災害!?	
\\	マール「じゃあ、やっぱりここは   私達の未来なの!?	
\\	マール「ひどい! ひどいよ!   こんなのってない!!	
\\	マール「これが……	
\\	マール「私達の未来だなんて……	
\\	ルッカ「……。	
\\	マール「そ、そうだよ!   変えちゃおう! クロノが   私を助けてくれたみたいに!	
\\	マール「ね、ルッカ。	
\\	マール「ね、クロノ!!	
\\	ルッカ「そうね……。	
\\	ルッカ「このまま私達だけ   元の時代に帰って、のうのうとくらす   ワケに行かないわね。	
\\	ルッカ「さいわい私達はゲートを   使って時代を超えられるんだもの。	
\\	ルッカ「やろうか? クロノ!   よし、やろう!   そんな……	
\\	ルッカ「バカね、あんたも。   マールの前でしょ、カッコくらい   つけなさい!	
\\	マール「よし、やろうクロノ!   ルッカ!!	
\\	ルッカ「それじゃ早いトコ私達の時代に   もどってラヴォスについて調べないと!	
\\	ルッカ「行くわよ!   プロメテドームへ!	
\\	ドン「も、戻ってきおった!	
\\	ドン「どうじゃった?	
\\	マール「ここは……私達の明日なの!	
\\	ドン
\\	男「それより食料庫は!?	
\\	マール「これしか、なかったの……。	
\\	ドン「種子……か。	
\\	ルッカ「エナ・ボックスも、いつまで   もつかわからないわ。   その種子を育ててみて下さい。	
\\	マール「とにかく生きて!   ガンバって!   私達もやってみるから!	
\\	ドン「フッ……	
\\	あんた達は不思議じゃ。   何かこう、私らとは……	
\\	マール「元気ってコト?	
\\	ドン「元気? 聞いた事のない言葉じゃが   なんだか気持ちのいいひびきだ……。	
\\	何の種子かわからんが、こいつを   育ててみるよ。	
\\	子供「コレなーに?	
\\	ドン「今の私らに必要なのは、こういった   物なのかも知れん。	
\\	ドン「プロメテドームに行くのか。   32号廃墟をぬけるならば   これを持って行け。	
\\	ドン「32号廃墟においてある   ジェットバイクのキーじゃ。   私がわかいころ乗っていたものじゃ。   まだ動くといいが……。	
\\	ドン「むこうは、ここより強力なロボットが   暴れまわっている。 『元気』でな!	
\\	ありがとうよ……。あんたらを見てたら、生きてく希望がわいてきたよ……。	
\\	そうですか、あの人が……	
\\	でも、あの人は私に大事なものを残してくれました。種子と……この子です。	
\\	お父さん、おそいな……。	
\\	みはり「キー! よそ者だあ!   早くクロウリー様に知らせなくっちゃ!   キー!	
\\	こぶん「ここに橋があったのを   みたんだケロ、ほんとだってばケロ。   ほんとなんだってば!!	
\\	おやぶん「うそをつくなゲロ!   どこにもないゲロ、まったく!	
\\	こぶん「ほんとなんだケロ……。	
\\	おやぶん「ん? 誰か来る、逃ゲロゲロ!	
\\	マール「ちょっとこわいけど……。	
\\	ルッカ「不気味な所ね。   何かいやな予感がするわ……。	
\\	エイラ「用心 大切!	
\\	日記がある。最後のページには、こう書かれている。	
\\	もう限界だ。音をたてたら、はんぎょ人に襲われる……なんて、がまんできない。	
\\	こんなことだったら、ギロチンの刑になっていてくれればよかったのに。	
\\	最後に大声だしてやる!	
\\	こぶん「どうやら、このスイッチを押すと   橋がかかるケロ!おやぶん「それでは、スイッチオンだゲロ!	
\\	こぶん「うわーん!	
\\	スイッチ高すぎて、とどかないケロ!	
\\	おやぶん「それでもカエルかっ、ゲロゲロ!	
\\	お知らせ	
\\	まわりこんでスイッチを押せ!	
\\	みはり「キー!   3人ぐみが、もうすぐやってきますぜ!   キー!	
\\	クロウリー様「この地下水道に来るとは   まったくもって命知らずのヤツらめ。   ここの恐ろしさを、思い知らせて   やるわい。	
\\	みはり「キー!   思い知らせてやりましょう。   キキキキキー!	
\\	クロウリー様「来やがったら   こんなんして、あんなんして   もうコナゴナしてやるからな!	
\\	クロウリー様「ギャハハハハ!	
\\	みはり「キャハハハハ!	
\\	クロウリー様「グッキャキャキャ!	
\\	みはり「クッキャキャキャ!	
\\	クロウリー様「ドッヒャヒャヒャ!	
\\	みはり
\\	クロウリー様「ん?	
\\	老人「サラか……、ひさしぶりじゃの……。	
\\	老人「……。	
\\	老人「ならん!	
\\	行ってはならん!	
\\	死の山だけは行ってはならん!   と、言っても、その時と、その者達で   なければ、死の山には行けぬことにな   っておるがの、ヒャヒャ。	
\\	老人「それよりも!	
\\	お前さん達は、見た事があるかいの?   私の作った海底神殿や黒鳥号を?   そらー見事なモンじゃぞ、ヒャヒャ。	
\\	なぞの物体「ウイー ハカセガ ワタシニ   ドノヨウナ プログラムヲ スルノカ   トテモ タノシミデス。	
\\	ネエ ハカセ?老人「そんなに楽しみにしてくれてると   わしもやりがいがあるのう。	
\\	どうじゃ? ちっとは   さまになってきたじゃろうが……。	
\\	「待チナ!	
\\	ロボット「ア!	
\\	ロボット「アニキ!	
\\	アニキ「待タセタナ!!	
\\	アニキ「俺ノ名ハ…… ジョニー。   コイツラノ頭ダ……。   ムコウノ大陸ニ、通リヌケタキャ   コノ先ノ、ハイウェイ跡デ   俺ト、ショウブダ……。	
\\	ジョニー「ソコノ『ジェットバイク』ヲ   使ワセテヤル…   ビビッテンジャネーヨ!   バリバリダゼ! ベイベー!	
\\	ロボット「アニキ! アニキ!   アニキハ、ツヨイ!!	
\\	ロボット「アニキ! アニキ!   アニキハ、ハヤイ!!	
\\	ジョニー「トコロデ、オメェ   動カシカタハ 知ッテンダロウナ?   はい   いいえ	
\\	ジョニー「動カシカタヲ、聞キテェノカ?   はい   いいえ	
\\	ジョニー「シカタネェナ、ジョニーサマガ   オシエテヤロウ。   アクセルハ、オートダ。   カッテニ、進ンデイクカラ安心シナ。   イドウハ、上下ノ方向キーヲ押セバ   ソノ方向ニ、イドウスル。   ターボブーストハ
\\	。   3回マデ、使ワセテヤル。   回数ハ右下ニ、
\\	ノマークデ   書イテアル。   ダガ、シカシ!!   1度使ッタ後ハ、マーク上ノゲージガ   タマルマデハ、使エネエ、気ヲツケナ。	
\\	ゴールマデノ位置ハ   画面下ニデテイルゼ。   サァ、シッカリツイテキナ!   ベイベー!!	
\\	ジョニー「負ケタヨ、オメー達ニャ……。   マタ、イツデモ来イヨ。   イッショニ、風ニナローゼ!   ベイベー!	
\\	ロボット「アニキ! アニキ!   アニキガ、マケタ!!	
\\	ロボット「アンタ! アンタ!   アンタモ、ツヨイ!!	
\\	ジョニー「フッ! ダセー走リダナ。   オ話ニナラネエゼ。   マタ走ルナラ、ジェットバイクニ   乗リナ。	
\\	ロボット「ヨワイ! ヨワイ!   オマエハ、ヨワイ!!	
\\	ロボット「イカス! イカス!   ジョニーアニキハ、チョウイカス!!	
\\	ジョニー「フッ! オ話ニナラネエナ。   アキラメルノナラ、アルイテ   向コウニワタリナ。   モンスターガ、タクサンイルガナ……。   マタ、走ル気ガアルノナラ   ジェットバイクニ、乗リナ。	
\\	ジョニー「オッ! オメエ面白イ物   持ッテルジャネエカ?   ソイツガアレバ、ウイニングポイントノ   記録ガトレルゼ。	
\\	クワシイコトハ、アイツニ聞キナ。	
\\	リンス「ハーイ!   レコーダーノ、リンスヨ。   私ガ、ウイニングポイントノ記録ヲ   ベスト3マデ記憶シテオクワ。   ジョニートノ勝負ニハ   2シュルイノ、モードガアルノ   ドッチガイイカシラ?	
\\	ドッチガイイカシラ?   モード1 ブーストアリ。   モード2 ブーストナシ。        
\\	デ、シテン回転ツキ。	
\\	モード1 ブーストアリ。   モード2 ブーストナシ。 
\\	デ、シテン回転ツキ。	
\\	リンス「レースモードヲヘンコウスル?   はい   いいえ	
\\	リンス「キャー! ハイスコアヨ!!   ヤルジャナイ。	
\\	リンス「キャー!!   フィーバー、シタワ!!   ゴホウビニ、コレ、ア・ゲ・ル……。	
\\	ハイポーションを手に入れた!	
\\	エーテルを手に入れた!!	
\\	ミドルエーテルを手に入れた!!	
\\	リンス「キャー!   777ヨ!!   ウン、モウ、出血大サービス!     なんと、ミドルエーテルを       10コ手に入れた!	
\\	リンス「スゴイ、ジャナイ   1300ポイント以上イクナンテ   ハイ、ゴホウビ……。	
\\	ミドルポーションを5コ手に入れた!	
\\	リンス「キャー! キャー!!   1500ポイント以上イクナンテ!!   ゴホウビニ、1コシカナイケド   コレ、ア・ゲ・ル……。	
\\	パワーカプセルを手に入れた!	
\\	リンス「キャー! キャー!!   2000ポイントヲ、コエルナンテ   スゴイ! スゴイワ!!   モウ、大サービスシチャウ。	
\\	エーテルを5コ手に入れた!	
\\	リンス「キャー! キャー!! キャー!!   2300ポイントヲ、コエルナンテ   スゴイワ!! スゴスギルノヨーー!   モウ、超出血大サービス。	
\\	なんと、なんと、ハイエーテルを5コ手に入れた!!!	
\\	マール「扉、開かないみたい。	
\\	マール「な、何コレ?	
\\	ルッカ「壊れてるみたいね。   けど……、すごい……。   完全な人型ロボット……。	
\\	ルッカ「……。   直せるかも知れないわ。	
\\	マール「え? 直すって……。   他のロボット達みたいに   襲って来ちゃうわよ!	
\\	ルッカ「そうしないように直すの。   ロボット達は自分の意思で   襲って来るんじゃないのよ……。  人間がそういう風に作ったの。   ロボット達の心をね。	
\\	マール「……ルッカには   ロボット達の気持ちが   わかるんだね……。	
\\	ルッカ「じゃ、とりかかるわ。	
\\	ルッカ「これで、よし……っと!	
\\	動かすわよ!	
\\	ロボット「……。	
\\	マール「おはよう!	
\\	ロボット「お……      おはようゴザイマス、ご主人様。   ご命令を。	
\\	マール「私はご主人様じゃなくって   マール!	
\\	それにクロノに……	
\\	あなたを直したルッカよ!	
\\	ロボット「了解シマシタ。   ワタシを直して下さったのは   ルッカ様デスネ。	
\\	ルッカ「ルッカでいいのよ。	
\\	ロボット「そんな失礼な事は   出来マセン。	
\\	ルッカ「様づけで呼ぶ方が   失礼な事だってあるのよ。   ね、マール。	
\\	マール「フフ……!	
\\	ロボット「了解シマシタ、ルッカ。	
\\	ルッカ「よーし。   で、あなたの名前は?	
\\	ロボット「名前? 開発コードの事デスネ。   
\\	デス。	
\\	ルッカ
\\	か……   イカスじゃない!	
\\	マール「え~?   ダメよ、そんなカワイくないの!	
\\	マール「ね、クロノ   もっといい名前、つけてあげようよ!   何がいいかしら?	
\\	マール「ロボ……   ロボか! 悪くないね!	
\\	マール「い~い? あなたの名前は   ロボよ!	
\\	ロボット「ロボ……デスネ。   メモリーインプット完了。	
\\	ルッカ「ねえ、ロボ。   ちょっと聞きたいんだけど…	
\\	ロボ「………。	
\\	コレは……   どうしたのデショウ?	
\\	ロボ「このプロメテドームには   多くの人間やワタシの仲間がいたはず   デスガ……。	
\\	ルッカ「言いにくいんだけど……。   ロボ、あなたが倒れている間に   ここの人達はたぶん……	
\\	ロボ「……。	
\\	ソウデスカ……。	
\\	ロボ「ではアナタ方は   なぜココに?	
\\	ルッカ「私達は王国暦1000年から   ゲートという時空のゆがみを通って   ここに来ちゃったの。	
\\	マール「ここにもゲートがあるって   アリスドームで調べて来たの。	
\\	ルッカ「そしたら、あなたが   倒れてたってわけよ。	
\\	マール「けどドアが動かなくって   こっから先に行けないの。	
\\	ロボ「ココの電源は完全に死んで   しまっているようデスネ。	
\\	ロボ「北にある工場に行けば   ここに連動する非常電源がありマス。   ワタシなら工場のセキュリティを   解じょ出来マス。	
\\	マール「ホントー?	
\\	ロボ「修理して下さったのデス。   今度はワタシがお役に立ちマショウ。   シカシ、いつまで非常電源が持つか   わかりマセン……。   ドナタかココに残って、電源が入ったら   すぐにドアを開けないと……	
\\	ルッカ「じゃあ、私かマールが   残るわ。	
\\	どっちが残ろうか?   マール   ルッカ	
\\	マール「わかった!   気をつけてね、クロノ!	
\\	ルッカ「オーケー、わかったわ。   こっちは、まかせておいて。	
\\	マール「やったー! ヒマしてたんだ。   ルッカとかわっていい?   はい   いいえ	
\\	ルッカ「やっぱり、私の天さい的   ずのうが必要みたいね。   はい   いいえ	
\\	マール「わかった、待ってる……。	
\\	ルッカ「そ、そう……。	
\\	不思議の国の工場跡	
\\	ロボ「セキュリティシステム00   アンロック。	
\\	左 
\\	研究所エリア右 
\\	工場エリア	
\\	工場エリアではベルトコンベアのながれを止めないように気をつけよ。セキュリティシステムが作動し命を落とすこともある。	
\\	アップ   ダウン   キャンセル	
\\	クレーン 起動コード00パターン 
\\	01パターン 
\\	クレーンを起動します。
\\	の4つのボタンのうち2つをくみあわせてじゅんばんにおしなさい。	
\\	ピンポンのあとに入力してください。	
\\	サイジュウヨウ ケイカイ ドア ロックコード 
\\	ザ ビ ィ	
\\	ザビィ…… ホンヤク
\\	ハッチ オープン	
\\	ロボ「パワーシステム…オール   パワーダウン。   ボウゴ ビーム の出力をすべてオフ   シマシタ。	
\\	「ドアロックシステム『ザビィ』作動中 パスコードを入力してください。	
\\	エレベーターハ サドウフノウ	
\\	ロボ「非常事態デス!   セキュリティ・システムが   暴走していマス。 脱出しなくテハ!	
\\	ロボ「ハヤク!   イマノウチニ!!!	
\\	イマージェンシー	
\\	イマージェンシーエレベーターハ サドウフノウ	
\\	ロボ「オ……、オオ……   みんな仲間デス。	
\\	ロボ
\\	生きていたのか、良かった。	
\\	ロボ「な、何を……	
\\	「ケッカンヒンメ。
\\	「オマエナド   ナカマデハナイ。	
\\	ロボ
\\	「ワレワレノ、ニンムヲ   ワスレタノカ?   コノ、コウジョウ ニ   フホウシンニュウ スルモノハ   マッサツスルノダ!	
\\	ロボ
\\	ワタシは、そんな事を   するために作られたと?	
\\	ロボ「ケッカンヒン……。
\\	「ソウダ。   ケッカン ヒンダ。	
\\	ロボ「ケッカンヒン……   ……ワタシは ……ケッカンヒン……	
\\	「キエロ、ワレワレノ   ツラヨゴシメ。	
\\	ルッカ「あ、あんた達ーッ!	
\\	マール「ロボ!   クロノ、ロボが!!	
\\	ロボ「や、やめて下さい……   このロボット ワタシの兄弟デス。	
\\	ルッカ「あんた死んじゃうわよ!	
\\	マール「そんな事いったって   このままじゃ、ロボ、あなたが!	
\\	ロボ「やめてクダサイ。   やめ……て……	
\\	「ロボ!	
\\	「ヒキツヅキ   シンニュウシャヲ   ショブンスル。	
\\	ルッカ「処分されんのは……   あんた達よーッ!!	
\\	マール「ふざけないで!   ぜったい許さないからッ!	
\\	「ひっかかってるわ!!	
\\	ルッカ「ひどいわ……   私に直せるかしら。	
\\	ルッカ「とりあえずプロメテドームまで   運びましょう……。	
\\	マール「わーん。どーしよう。   ロボが死んじゃうよ!	
\\	マール「クロノ!   とにかくルッカの所へ運ぼうよ!	
\\	ロボ「シュ……修復……   出来そうデスカ……?	
\\	ルッカ「あんまり、しゃべっちゃダメ。	
\\	ロボ「アナタ方は……。   この世界を変えるというのデスカ……?	
\\	ルッカ「私達の力で、どこまで出来るか   わからないけど……ね。	
\\	ところでロボは……   直ったら何がしたい?	
\\	ロボ「ワタシの……したい事……?	
\\	ルッカ「そうよ、長い間こんな所で   故障してたんだもの、やりたい事が   いっぱいあるはずでしょ?	
\\	ロボ「ワタシに、そんな事を   聞いたのは、あなたが初めてデス。   ルッカ……	
\\	ロボ「おはようございマス。	
\\	マール「ロボ!   良かった、元気になって!	
\\	マール「ごくろう様、ルッカ!	
\\	ルッカ「ちょーっとばかり   大手術だったけどね!	
\\	ロボ「ルッカ、ワタシにも   したい事が出来マシタ。   アナタ方といっしょに行く事デス。	
\\	マール「えッ、本当に?	
\\	ロボ「ワタシも、見とどけて   みたいのデス。   アナタ方のする事が、人間を……   いえ、この星の生命を、ドコへ   みちびいて行くのか。	
\\	ルッカ「行きましょう!   あの先にゲートがあるわ!!	
\\	ルッカ「さあ! いくわよ!!	
\\	時の最果て	
\\	ロボ「ココは?	
\\	老人「おや、またお客さんだ……。	
\\	マール「あの、ここは一体?	
\\	ルッカ「客人って……   あの、ここは?	
\\	老人「ここは、『時の最果て』……。   時間のまよい子が、行き着く所さ。   お前さん達、どっから来なすった?	
\\	ルッカ「私達は、王国暦1000年から   来たんです。	
\\	ロボ「ワタシは
\\	2300年の   世界からゲートで……	
\\	老人「違う時間を生きる者が、4人以上で   時空のゆがみに入ると、次元の力場が   ねじれてしまう……。   しかし、この所、時空のゆがみが   多くてな。   お前さん達の様にフラリとここへ   あらわれる者もいる……。	
\\	何かが時間全体にえいきょうを   およぼしているのかも知れんな……。	
\\	ルッカ「って事は、誰か一人ここに   残った方が安全って事ね。	
\\	マール「ええ、こんなトコで   おいてけぼりなの~?	
\\	老人「こんなトコはひどいな……。   何、しんぱいいらんよ。   ここはすべての時に通じている……。	
\\	お前さんがたが願えばいつでも   仲間を呼び出せる。   だが時の旅は不安定じゃ。   つねに3人で行動する事じゃ。	
\\	ロボ「デハ、ダレか残らないと   なりマセンネ。	
\\	ルッカ「誰が残る? クロノ。	
\\	マール「早く呼んでよ、クロノ!	
\\	ルッカ「二人っきりだからって   ヘンな事考えないでよね、おジイさん!	
\\	ロボ「ゴヨウの時はいつでも   お呼びクダサイ。	
\\	老人「ここに残っている者ならば   いつでもメンバーを入れかえる事が   出来るぞ。	
\\	老人「……という事じゃ。	
\\	ボタンで私を呼び出せば   ここに残っている者といつでも   メンバーチェンジ出来る。   わすれんようにな。	
\\	マール「私達の時代に戻るには   どうやったらいいの?	
\\	老人「お前さん達がやって来た場所に   光のはしらがあるじゃろう。   あれは、あちこちの次元のゆがみと   ここ、時の最果てをつなぐものじゃ。   一度通った事のあるゲートからは   いつでもここに来られるじゃろう。   光にかさなり
\\	ボタンを押せば   ゲートにもどれる。   じゃが、そこのバケツからつながる   ゲートには気をつけるんじゃな……。	
\\	老人「そこは
\\	『ラヴォスの日』と言われる時へ   つながっとる……。   世界のほろぶ姿が見たいのなら   行ってみるのもいいが……   お前さん達まで、ほろびちまうかも   知れんぞ。	
\\	1999年『ラヴォスの日』へ行きますか?   はい   いいえ	
\\	かたく閉ざされている……。	
\\	老人「おーい。	
\\	老人「そう、せく事もあるまい。   上のドアの向こうにも   寄って行ってみなされ。	
\\	「なんだ、おめーら?	
\\	オレか? オレ、スペッキオ。   戦の神!   こっからいろんな時代の戦、見てる。	
\\	スペッキオ「おめー、オレどう見える?   強そう   弱そう	
\\	スペッキオ「そうか。   オレの姿、おめーの強さ。   おめーが強ければ強そうに   弱ければ弱そうに見える。	
\\	スペッキオ「ん?	
\\	おめーら、持ってるな。   心の力を……。	
\\	そうか、表のジジイ、それでここに   通したか。   おめーらの生まれるずっと昔……   魔法でさかえた王国、あった。   その世界、みんな魔法使った。   けどその王国、魔力におぼれ   ほろびた……。   それから人は魔法を使えなくなった。   魔族以外はな。   だが、おめーら、持ってる。   心の強さを。 魔法は心の強さ、力。	
\\	魔法は天・冥・火・水の4つの力で   なりたってる。	
\\	スペッキオ「ツンツン頭のおまえは『天』。	
\\	スペッキオ「このポニーテールのギャルは   『水』の力。	
\\	スペッキオ「こっちのメガネのネーちゃんは   『火』の力。	
\\	スペッキオ「このデッカい人形は……。   おめー、生き物じゃないな。   おめーも強い心持ってるけど   昔の魔法の民の血ひいてないから   魔法ムリ。   でも、おめーのレーザーとかいう武器   スゴい殺傷能力。   『冥』の力に似てる。	
\\	スペッキオ「てなぐあいに、魔法だけでなく   すべてのものは、この4つのバランスで   なりたってる。	
\\	スペッキオ「『魔法が使いたい~』と   ねんじながら、ドアの所からはじめて   この部屋のさくにそって、時計回りに   3回まわる!   バターにならないよう、気をつける。	
\\	スペッキオ「ズルだめ!   オレちゃんと見てる!	
\\	もう1回やり直し! ドアの所から   さくにそって時計まわりに3回まわる!	
\\	スペッキオ「よーし!	
\\	スペッキオ「ハニャハラヘッタミタ~イ!!	
\\	クロノが魔法を使えるようになった!	
\\	マールが魔法を使えるようになった!	
\\	ルッカが魔法を使えるようになった!	
\\	ロボは魔法を使えない!	
\\	スペッキオ「どーだ、新たな力わくだろ。   ためしてみるか?   はい   いいえ	
\\	スペッキオ「フフ、わくわく!	
\\	スペッキオ「なーんだ、おめーら。   腕だめししたいのか?   はい   いいえ	
\\	スペッキオ「いつでもいいぞ。   なにしろオレ、戦の神!	
\\	スペッキオ「オゴゴ……	
\\	オレ負けるなんて……	
\\	スペッキオ「何かの間違い! オレ戦の神!   そーだ、オレが魔法おしえたから   おめーら勝てた!	
\\	スペッキオ「やっぱオレ、戦の神!   フンフフ~ン
\\	スペッキオ「フ、フン。   少しはやるみたいだな。   しかたない、コレやる。	
\\	スペッキオ「オレ、戦の神! 気前いい!   フンフフ~ン
\\	スペッキオ「フンフフ~ン
\\	オレの勝ち!	
\\	スペッキオ「新しい仲間出来たら   連れて来る。   おめーら、オモシロイ。	
\\	スペッキオさわやかセット!	
\\	マジックカプセル1コ エーテル5コつめあわせを手に入れた!	
\\	スペッキオしなやかセット!	
\\	マジックカプセル1コ  ミドルエーテル5コつめあわせを       手に入れた!	
\\	スペッキオすこやかセット!	
\\	マジックカプセル1コ   ハイエーテル5コつめあわせを       手に入れた!	
\\	スペッキオたおやかセット!	
\\	マジックカプセル1コ     スピードカプセル1コ     パワーカプセル 1コ	
\\	エリクサー10コつめあわせを       手に入れた!	
\\	スペッキオにぎやかセット!	
\\	マジックカプセル10コ    スピードカプセル10コ    パワーカプセル 10コ	
\\	ラストエリクサー10コに……	
\\	スペッキオの気持ち     一年分を手に入れた!	
\\	スペッキオ「いーかげんにする、おめーら。   オレもうカラッケツ!!	
\\	スペッキオ「なにわらってる、おめーら……	
\\	本当の強さとは姿形などとは   むかんけいなのだ……。	
\\	老人「ほ、やはりな。   はるか昔の人々はみなそういった力を   持っていたもんじゃ。	
\\	そう、はるか昔……な。   お前さん達、何かとてつもない事を   やらかそうというみたいじゃが   まずはお前さん達の時代に   もどってみてはどうかな……?   急がばまわれ……   ここにいると、そんな言葉の意味も   よくわかってくるものじゃ……。   何かあったら、いつでも来なさい。   私がお前さん達の時間の道しるべに   なってやろう。	
\\	老人「何かあったら、いつでも来なさい。   私がお前さん達の時間の道しるべに   なってやろう。	
\\	マール「私も連れてってよー!   はい   いいえ	
\\	マール「よーし、ガンバるぞ!	
\\	マール「ちェッ、つまんない!	
\\	ルッカ「私の頭脳が必要じゃなくって?   はい   いいえ	
\\	ルッカ「やっぱ、私がいないとダメね!	
\\	ルッカ「あ、あらそう?	
\\	ロボ「何かお役に立てマスカ?   はい   いいえ	
\\	ロボ「リョウカイシマシタ。   
\\	ロボ「ご用の時はエンリョなく   おもうしつけクダサイ。	
\\	2300年『プロメテドーム』へ行きますか?   はい   いいえ	
\\	1000年『メディーナ村』へ行きますか?   はい   いいえ	
\\	6500万年『不思議山』へ行きますか?   はい   いいえ	
\\	飛びおりますか?   はい   いいえ	
\\	ダサい そのカッコ……。	
\\	赤い石 とてもキレイ。オマエ 見たいか?酋長 持ってる。	
\\	酋長のテント 村の北。	
\\	オマエ 恐竜人 仲間か?やっつけるぞ!	
\\	恐竜人 ザンコク 冷血。恐竜人 カタいウロコ。倒す とても力いる。	
\\	元気の水 飮むか?飮むと 元気モリモリ。   はい   いいえ	
\\	が回復した!	
\\	恐竜 みんな みんな かみなりでビリビリ 感電! 知ってたか?	
\\	オマエ ヘンなヤツ。何か かわったモン あるか?いや 酋長の許しない 交換出来ない。	
\\	かりの森 で かりする。『はなびら』 『きば』 『つの』『はね』 あつまる!それに あめ ふった時 とてもめずらしいモンスター 出る。そいつ おとく! おとく!!	
\\	ここ 交換屋。とって来た物 いろんな物とりかえる。	
\\	オマエの そのキラキラした石と交換!	
\\	酋長 強いもの なる。男でも 女でも 子供でも。イオカの酋長は むっちんプリプリ。	
\\	お前達 あの山 来た?あの山 不思議な事 よく起こる。だから 不思議山 いう。	
\\	キーノも そう。ある日 山で 泣いてた。酋長 キーノ 育てた。キーノ 今 酋長の右腕。	
\\	俺達 恐竜人 戦う。ラルバ族 戦わない。	
\\	ラルバ族 北の森 かくれてる。かくれて あっちこっち。ラルバ族 恐竜人 逃げてる 弱虫!	
\\	ラルバ族 プテラン呼ぶ。 空 飛ぶ!オレ達 呼べない。 空 飛べない。	
\\	キーノ「お前ら どっから来た?   ここ 何し 来た?	
\\	キーノ お前ら キライ!	
\\	魔の村の人々	
\\	兄ちゃん達、ボクの家のタンスから出て来たよね……?なんで? どーして?	
\\	お、おまえらなあ、事もあろうに他人の家のタンスから出てくるとはどういうつもりだ?これだから人間ってヤツは……	
\\	ちょっと待つんだ。	
\\	このメディーナ村は、魔族の村。400年前、人間との戦いに負けた魔族の子孫によってつくられた村だ。	
\\	この村に住む魔族のほとんどは人間に対してにくしみをいだいている。気をつけな。	
\\	あ、兄ちゃん達、ちょっと待って。	
\\	西の山の洞窟の近くに、ちょっと変わった人間のおじいさんが住んでいるんだ。きっと兄ちゃん達の力になってくれるよ。	
\\	マール「教えてくれて、ありがとう!	
\\	マール「でも……   なぜ私達に親切にしてくれるの?   魔族は人間をにくんでいるのでしょう?	
\\	ルッカ「アドバイス、どうも!   でも、魔族は人間をにくんでいるはず   でしょう?   なぜ、あなた達は……?	
\\	ロボ「助言ヲありがとうゴザイマス。	
\\	ロボ「シカシ、あなた方はナゼ   我々に親切にシテ下さるのデスカ?   魔族ハ人間をニクんでいるのデハ?	
\\	人間と魔族が戦ったのは400年も昔の事だ。 いつまでも過去にとらわれていてもしかたがない。まあ、私達のような考えを持った魔族はほとんどいないが……	
\\	ん、兄ちゃん達も食べてく?   はい   いいえ	
\\	兄ちゃん達は人間なのに、なぜ魔力を持ってるの?魔法を使える人間はずっと昔にいなくなったはずなのに……	
\\	人間風情が、この俺様に話しかけるな!	
\\	人間なんかに売ってやるモノはないね。とっとと帰りな。   あきらめる   たのみこむ	
\\	人間ごときがえらそうに!先生、ノシてやって下さい!	
\\	せ、先生が殺られた!?こいつら、化け物だ~!	
\\	しかたがない。 売ってやるよ。	
\\	へっ! 人間なんぞにまともな値で売ってやるもんか!	
\\	酒がマズくなる! あっちに行け!	
\\	へっ、人間なんかヘケランのちょうはつにのせられて、いたい目を見りゃいいんだ。ウケケケ。	
\\	人間なんか、とまらせてやるギリはないね。とっとと帰りな。   あきらめる   たのみこむ	
\\	うるせえ! ゴチャゴチャぬかすな!!殺っちまえ!	
\\	ひえ~、お助け~!	
\\	1ぱく200
\\	だ。とまらせてほしいのか?   はい   いいえ	
\\	金がたりないようだな。	
\\	ワシがこのメディーナ村の村長ビネガー8世じゃ!ワシの先祖は、あのいだいな魔王様の右ウデとして人間どもと戦ったのだぞ。ああ、魔王様…… なぜ400年前に人間達をほろぼして下さらなかったのじゃ?	
\\	そうじも楽じゃないよ……	
\\	ちくしょう、ビネガー8世のヤロウ!先祖の地位をカサにきて俺達をこき使いやがって。	
\\	西の山の洞窟にすむ魔物は魔法による攻撃しかうけつけないんだ。魔法が使えない人間では通りぬける事はむずかしいだろうね。さ、ビネガー8世に見つかるまえにそうじを終わらせなきゃ。	
\\	マール「ねえ、クロノ。   この不気味な声は何なのかしら?	
\\	ルッカ「……不気味な声ね。   ここで何が行なわれているのかしら?	
\\	ロボ「コ、コノ不気味ナ声ハ   イッタイ何なのでショウカ?	
\\	ははーっ! 魔王様~	
\\	魔王様が我々のために生み出されたラヴォス神が永き眠りからめざめれば人間どもの世界などいっしゅんで終わりだ。	
\\	魔王様が我ら魔族をひきいて人間に戦いをいどんでから、400年が過ぎた。ラヴォス神さえよみがえれば、人間など……	
\\	その昔、強大な魔力を持つ魔王様がラヴォス神をたんじょうさせたのだ。	
\\	ボッシュ「おお、たずねて来おったか!   ワシのじまんのコレクションでも   見て行くとよい。   そうじゃ、ワシの作った武器でも買って   ゆかんか? 安くしとくぞ。   はい   いいえ	
\\	ボッシュ「武器はな……   生命をうばうための物ではないぞ。   生かすための物であるべきじゃ。	
\\	ボッシュ「そうか、ザンネンじゃの。	
\\	ボッシュ「それはの、かるくて使いやすい   名剣じゃ。   モチロン、ワシの作品じゃぞ。	
\\	ボッシュ「そうじゃ、おぬし達。   トルース町に帰りたいのであれば、   この家の北にある山の洞窟を   ぬけて行くがよい。	
\\	魔族の敵に死を!	
\\	ラヴォス神を生んだ魔王様が400年前に人間共をほろぼしておいてくだされば、今ごろこの世界は我ら魔族の時代になっていたものを……	
\\	ク ソ ー ッ 
\\	ルッカ「やっぱり、中世の魔王が   この星の未来をメチャクチャにした   ラヴォスを生んだのね……。	
\\	マール「この星の未来をメチャメチャに   したラヴォスを生んだのが   中世の魔王なのね……。	
\\	マール「中世へ行って魔王を   倒せば歴史は変わるのかしら?	
\\	ロボ「魔王を中世で倒ス事ガ   出来レバ、歴史は変わるのでショウカ?	
\\	ルッカ「千年祭広場のゲートを   使えば……。	
\\	マール「ねえ、クロノ。   千年祭広場のゲートを使えば……。	
\\	とびこみますか?   はい   いいえ	
\\	現われた 伝説の勇者	
\\	ルッカに会ったら言っといて   くれよ。	
\\	タバン「おう! ルッカ。	
\\	タバン「たまには、かお見せろって。   俺はルッカの気に入る防具を   せっせと作ってるからな。	
\\	逃がすかーッ!	
\\	マール「行こう、クロノ!   あんなわからず屋の父上なんか   会いたくない!	
\\	兵士
\\	テ、テロリストいちみだーッ!!	
\\	ルッカ「マ、マズいわよ!!	
\\	ロボ「ナ、何なんデス~!?	
\\	カエル「テロリストお!?	
\\	フリッツ「へい! らっしゃい……!!	
\\	フリッツ「クロノさん!!!   あの時は本当にありがとう   ございました。	
\\	おかげでこうして家にもどることも   出来ました。	
\\	フリッツ「まいりましたよ……。   おやじを見返してやろうと思って   しいれの旅にでたのはいいんですが……   帰りの宿で、ごうとうだんのとりものに   まきこまれましてね。   いっしょに、つかまってしまった   わけなんですよ。	
\\	それにしても私の話を聞きもしないで   死刑! なんですから……、ふう……   ほんとうに助かりましたよ。	
\\	エレイン「本当に、本当に   ありがとうございました。   私、今しあわせです。	
\\	フリッツ「これ、おやじにはないしょで……	
\\	ミドルエーテルを10コ手に入れた!!	
\\	ん、んん!	
\\	フリッツ「お、おやじ!	
\\	ちゃんと、やってるんだろうな?	
\\	フリッツ「も、もちろん。な、なあ。	
\\	エレイン「え、ええ。	
\\	まあ、いいだろう……。	
\\	クロノさん……。ほんとうに、ありがとうございました。	
\\	フリッツ「さあ、クロノさん。   こっからは商売だ、俺がしいれた物   買っていってくれ。	
\\	フリッツ「いらっしゃい!   
\\	さん。	
\\	フリッツ「まいど!!	
\\	ジナ「クロノ!!	
\\	ジナ「クロノ! クロノ!!   よかった……。   母さん、あんたが死刑になるって   聞いて……	
\\	ジナ「もう、母さんを心配させるんじゃ   ないわよ。	
\\	ルッカ「こんにちは、ジナおばさん!	
\\	ジナ「おや、ルッカ。   最近、うちにあんまり来ないのね?	
\\	ルッカ「最近、研究いそがしくって。	
\\	ジナ「そう、あんまりガンバリすぎて   体こわさないようにね。   ララとタバンにもよろしく言って   ちょうだいな。	
\\	ジナ「あら……、ルッカの発明品?	
\\	ロボ「お初にお目にかかりマス。   私はロボともうしマス。   クロノには、大変お世話に   なっておりマス。	
\\	ジナ「あらあら、ごていねいに。   ちょっと変わったカッコだけど   とっても、れいぎ正しいわね。   クロノも少しは見ならいなさい。	
\\	1000年『リーネ広場』へ行きますか?   はい   いいえ	
\\	600年『トルース村裏山』へ行きますか?   はい   いいえ	
\\	1000年『ガルディアの森』へ行きますか?   はい   いいえ	
\\	2300年『バンゴドーム』へ   行きますか?   はい   いいえ	
\\	なーに、心配いりません。勇者のあかしといわれる『勇者バッジ』をもった方がついに、あらわれたんです!	
\\	これでこの世界にもやっと平和がかえってきます。よかった、本当によかった……。	
\\	なんでも魔王軍がゼナンの橋まで来てるって話じゃないか。一体どうなっちまうのかね。	
\\	あんた聞いたかね!あらわれたんだよ! ついに!伝説の勇者様がねっ!それで、王様に会うため城に向かったって話だ。いやー、これで魔王軍もオシマイだな!	
\\	何でも王も倒れられたって言うし……もうオシマイかー。	
\\	いや伝説の勇者様が現われたからもうだいじょうぶだ!	
\\	国王軍が前線でやぶれ魔王軍が橋まで攻めて来てるって話だが本当か?	
\\	この鐘の完成も近いんだが……。国王軍が魔王軍にやぶれてガルディア王も負傷された。鐘どころじゃねえやな……。	
\\	私は、このリーネの鐘が平和な世になりひびく……そんな日が来るって信じてるの。どんな時でも希望は失わないように……。	
\\	あとは魔王軍をげきたい出来れば……	
\\	騎土団長「食料が底をついてしまった……   ガルディア城からのホキュウはまだか!	
\\	勇者様を守って、騎土団のせいえいが突撃を開始しました。	
\\	橋のまん中で騎土団と魔王軍がげきとつしたようだ。がんばってくれ……。	
\\	しかし、勇者様一人の力で魔王軍をうちやぶれるのだろうか?	
\\	我々の力では、魔王軍に勝てない……情けない事だが、勇者様だけがたよりだ。	
\\	勇者バッジを持った勇者……。おとぎ話だとばかり思っていた……。今となっては勇者様の力を信じるしかない。	
\\	もう食料がありません……	
\\	兵士「何だ、お前か。	
\\	魔王軍におされていたがもう、だいじょぶだ。勇者様があらわれたんだからな!	
\\	勇者様?今ここを通って王に会いにいかれたよ。	
\\	騎土団長はゼナン橋に行かれた。下は負傷者でいっぱいだ。	
\\	すみません、ベッドがいっぱいで……	
\\	騎土団長まで、前線に行かれたというのに……	
\\	魔王め、ついに3大しょう軍を戦いにとうじて来やがった。	
\\	前線の食料もそろそろ底をつくころ……	
\\	勇者があらわれたって、話だが……	
\\	王様もガンバってんだ。しっかりしな!	
\\	料理長は前線へとどける食料と、王に体力をつけようと、大いそがしさ。	
\\	あんた、王様のお食事はこれでいいかい?	
\\	王様にはスタミナをつけてもらわなきゃ!	
\\	料理長、ゼナン橋への食料も急がないと!	
\\	料理長「何? アニキが!?   ……。	
\\	料理長「ヘッ、てめえ達だけで   この国を守ってると思ったむくいだ。	
\\	おかみさん「あんた!	
\\	おかみさん「いつまでも、つまんない意地   はりあってんじゃないよ、男だろ!	
\\	料理長
\\	るせえ! お前に何がわかる!	
\\	バカだね、男ってな!	
\\	ああ言ってますけど、料理長はいつも騎土団長をしんぱいしてますよ。	
\\	本当は二人とも、なかなおりするきっかけをさがしてるのよ。	
\\	そらそら~ッ!!	
\\	「待ちな!	
\\	料理長「ハアハア……	
\\	これを持ってってくれ!	
\\	料理長「それから、こいつはお前らにだ。   持ってけ。	
\\	料理長「それから、あのバカに   伝えといてくれ。   生きて帰って来ねえと   しょうちしねえってな!	
\\	おお、クロノどの!ついに伝説の勇者様が姿をあらわしたのじゃ!	
\\	つい先ほど、ここに立ち寄られ王に会いに行かれましたぞ。	
\\	王は、お部屋でねこまれている。	
\\	リーネ様も王の部屋で、ねずのかんびょうをしておられる。	
\\	おいたわしや! 王は魔王軍を前にみずから剣をとり戦ったのです。もし伝説の勇者様が現れたという知らせを受けなければ、王は引かずにあのまま戦死されていたやも知れません。	
\\	私は自分がはずかしい……。	
\\	はじめ、あのお方を勇者様とは気付かずただのこぎたないガキだと思ってけとばしてしまったのです。しかし私は許されました。なんと心の広い人物でしょう。さすがは勇者様です。	
\\	このままではリーネ様も、まいってしまいます。	
\\	おお、クロノ。もう、私も年老いたかな……。	
\\	サイラスの消そくが、つかめぬ以上あのバッジを持った少年こそ残されたゆいいつの希望……。	
\\	少年は、魔王を倒せる伝説の剣をもとめて南の大陸へ向かったよ。	
\\	きっとサイラスが、あの少年をよこしてくれたのですわ。	
\\	やはり人間の力では、魔王軍にはかなわないのか……	
\\	勇者様は、何とか橋をわたりました。しかしわが軍は、大きないた手をこうむってしまいました……	
\\	うう……	
\\	騎土団長「魔王軍が   新手をくり出して来た。   今は危険だ。 下がっていなさい。	
\\	騎土団長「クロノ殿、何事かな?	
\\	料理長の食料をわたしますか?   はい   いいえ	
\\	騎土団長「こ、これは!?	
\\	そうですか、あいつが……	
\\	騎土団長「クロノ殿、もし私がここで   死んだならば、弟に……	
\\	騎土団長「何事だ!!	
\\	魔王軍の攻撃が始まりました!もうささえきれません!	
\\	騎土団長「弱音をはくな!   ガルディア王国騎土団のめいよにかけ   魔王軍をげきたいするのだ!	
\\	し、しかし、もう兵の数が……。	
\\	騎土団長「ここが最後の防えい線なのだ。   もうひとがんばりしてくれ!	
\\	わかりました!	
\\	騎土団長「ごらんのとおり、我が軍は   苦戦しております……	
\\	クロノ殿、助太刀願えませぬか?   はい   いいえ	
\\	騎土団長「かたじけない! 魔王軍は   橋の向こうのたもとにおります!	
\\	これをお持ち下さい。   きっとクロノ殿のお役にたつはず。	
\\	騎土団長「魔王軍は手強い敵です。   くれぐれもお気をつけて。	
\\	騎土団長「そうですか……。   だが我等もガルディア王国騎土団。   メンツにかけてこの橋は死守しますぞ!	
\\	ワシは、魔王様だいいちの部下魔王3大しょうぐんの、ビネガー。いだいなる魔王様の敵に、死を!	
\\	ビネガー「ワシのかわいいムスコ達よ!   こやつらに死をあたえるのだ!	
\\	ビネガー「くう~、なかなかやるな。	
\\	マール「待ちなさいったら!	
\\	ルッカ「逃げ足だけははやいわね。	
\\	ロボ「追いつきマシタ。	
\\	ビネガー「少々、お前達をあまく   見すぎていたようだ。   しかし、今度はそうはいかんぞ。ビネガー「殺っちまえ!	
\\	ビネガー「ちくしょー!	
\\	ビネガー「こ、今度こそお前達も   おしまいだぞ! ホントだぞ!	
\\	マール「何よ!   クロノをナメると   イタイ目見るわよ!	
\\	ルッカ「ふん、負けおしみね。	
\\	ロボ「アナタに勝ち目はありマセン。   コウフクシナサイ。	
\\	ビネガー「ゆけ、ジャンクドラガー!   魔王様の敵をたたきのめせ!	
\\	騎土団長「クロノ殿のおかげで   橋を守りきる事が出来ました。   しかし、魔王を倒さない事には   こんぽん的な解決にはなりません。	
\\	ゼナンの橋を魔王軍からとりもどす事ができました。これもクロノ殿のおかげです。	
\\	いらっしゃいまーしい。今日はどんなごようで?	
\\	まいど、ありあとやしたー!	
\\	伝説の勇者様があらわれたんだって?	
\\	いよいよ歴史にのこるような戦いのひぶたがきっておとされるのかねえ……。	
\\	魔物達がね、いっぱいいてねあぶないから、外に出ちゃダメだって。チェッ、つまんないよー。	
\\	なんでも魔王は、失われた太古の魔術をあやつるという話よな。	
\\	そこいらの魔物どもの術などくらべものにならぬおそろしいものだそうな。	
\\	ああ、騎土サイラス様はどこへ行かれてしまったのでしょう?	
\\	まさか……王国を見すて、一人でどこかへ逃げてしまわれたとか……?カエル「……。	
\\	王が負傷されたというウワサは本当か?	
\\	おお、この国は、われらは一体どうなってしまうというのだ……?	
\\	なんとカギがかかっている!	
\\	やや……!!おぬしがそこにかくしもっとるのはきんだんのミアンヌ・ブロマイド
\\	そのブロマイド、わしにくれるならいいものをやるが……、どうかね?   とっかえっこしよう!   じーさんのお宝などいらない	
\\	む……!なら、勝手にせい。	
\\	ホントかね、ホントかね?	
\\	わーい、やったぞ、ベービィ!よし、ちょこっと待っておれよ。	
\\	カギはあいとるじゆうにもって行くがいい。	
\\	それじゃ、ブロマイドはもらっとくぞ。うひょひょ……。	
\\	ばーさんには、ないしょじゃぞ。むふふ……、ミアンヌちゃん……。	
\\	村長「探し出してもらいたい物は   まぼろしの虹色の貝がら……。	
\\	これは前金じゃ。   よろしくたのむぞ、トマよ。	
\\	トマ「ヘヘッ、まかせときなって。   俺も伝説の探検家とよばれる男。   きたいはうらぎらないぜ、村長。	
\\	村長「うむ……。   よい知らせ、待っておるぞ。	
\\	トマ「よし、そうと決まれば   いっぱい、ひっかけてくるかな。	
\\	トマ「おりゃ、ちょっといそがしいんだ   また後でな。	
\\	村長「虹色の貝がらさえあれば   魔法ぼうぎょバツグンの   プリズム防具がつくれるのじゃ。	
\\	しかしあのわかぞう、ほんとうに   だいじょうぶなんじゃろか?   自分では、いだいなる探検家などと   いきまいとったが……。	
\\	ゆうしゃたまが、あらわれたんだ!	
\\	わるいまものをやっつけてくれるすっごく、つおーいひとなのよ!	
\\	言い伝えでは、魔王をたおせるのは太古より伝わる剣グランドリオンだけなんですってね。	
\\	えらばれた勇者様だけがグランドリオンを装備できるのだそうよ!	
\\	魔王には恐ろしい手下がいる。ビネガー、マヨネー、ソイソーの3人さ。	
\\	できることなら、あいつらとはかかわりあいにならない方がいい。血もなみだもないバケモノどもだよ。	
\\	ここは、サンドリノの村長の家だよ。はて、変わったかっこうをしとるがお前さん達も魔王軍に追われて逃げのびてきたのかい?	
\\	魔王の城は東の海の島にあるんだ。でも島のまわりはだんがいぜっぺきで誰にも近づけないんだよ……。	
\\	魔王の軍が今にも攻め込んでくるといううわさは、本当かしら……?	
\\	でも、伝説の勇者様があらわれたからにはもうだいじょうぶですよね!	
\\	とっておきの話、聞きます?   はい   いいえ	
\\	東にあるデナドロ山には巨大なハンマーをもったオウガンどもがうろついています。でも、ハンマーさえもやしてしまえばへっちゃらですよ。	
\\	いいよ。 じゃ、とっておくから……。	
\\	知ってるかい?伝説の剣グランドリオンはデナドロ山の奥に眠ると言われてる。	
\\	でも、すっげーおっかない魔物が剣を守ってるって話だよ。	
\\	いのりなさい。世界の終わりが始まるのです。	
\\	魔王は、邪悪で強大な力を生みだそうとしているらしいぞ……。	
\\	もしそうなったら、どこへ逃げても同じ!みんな死んじまうんだよ……。……ってなワケで俺はこうして酒に逃げてるのさ。ヒック!	
\\	トマ「よう、元気だったか。   いっぱいおごってくれるんなら   いいネタをやるぜ。	
\\	トマ「こいつは、すまねえな。	
\\	トマ「ち、金がたりないぜ。	
\\	トマ「おやじ、もういっぱいくんな!	
\\	トマ「ちかごろ南の森の奥にゃ   ヘンな魔物が   かくれ住んでるって話だな。	
\\	なんでも、うすっきみわるい   化け物ガエルらしいぜ。 ヘッ。	
\\	トマ「はん。じゃ、むこうへ行ってくれ。   おれもヒマじゃないんだ。	
\\	トマ「虹色の貝がらか……。   手始めにパレポリから   あたってみるとするか。	
\\	旅の宿には、いろんな人がたちよる。もし人をさがしておるのなら、ここの主人に聞くがよろしかろう。ところで、知っていなさるかな。たつまきのエネルギーは、かまいたちをぶつけてやれば、消えてしまう……。	
\\	自然ってのは、いろんなものの調和とバランスでなりたっとるのさ。	
\\	主人「あんたら、勇者様をさがしてるとか?   だったら、南のパレポリの村に   行ってみちゃどうだい。	
\\	なんでも勇者様は、パレポリから   あらわれたって話だぜ。	
\\	は~い、ここは愛といこいのホテル。料金は50
\\	になるんだけれどよろしいかしらん、あはん?	
\\	おとまりになっちゃうん?   とまる   とまらない	
\\	キャンッ!それじゃ、ごゆっくりどうぞ~ん。	
\\	あらん? お金、たりないわよ。	
\\	まあ、ざんねん! またのおこしをお待ちしておりますわん。うふん。	
\\	なんとかドリアンとかいう剣ははるか昔にほろんじまった人達の手でつくられたんだそうだよ。	
\\	今ではもう誰にも、あんな剣を作り出すことは出来ないんだとさ。	
\\	あなた達、もう勇者様に会った?   はい   いいえ	
\\	すごい! 私も一目でいいから勇者様に会ってみたかったな……!うるうるしてきちゃう……!	
\\	そう……。 私も、まだなの。	
\\	世界を救ってくれる勇者様っていったいどんな人なのかしら……?うるうるしてきちゃう……!	
\\	フィオナ「おっとのマルコは、兵士として   戦いに出たまま、帰って来ません……。	
\\	私達ニ人は、この森で育ったの。   せめて、この森だけは守りたい。   いつかマルコが帰って来る日まで……。	
\\	僕も、このよろいを着て魔王と戦うんだ!	
\\	うーーーお も い ……。	
\\	お兄ちゃん、無理しちゃって!	
\\	おかみ「タータの姿をしばらく見ないけど   どっか出かけてるのかしら……。	
\\	でも、あのイタズラこぞうがいないと   ホッとするわね、ほんと。	
\\	以前ヘンな魔物が、酒を飮んでいたことがありましたっけな。リーネ様がどうしたのグランドリオンがどうだのとひとりブツブツ言いながら……。	
\\	ありゃほんと、うす気味悪いカエル男でしたぞ! うひゃひゃ。	
\\	もう、ビックリだよ!伝説の勇者様が、実はこの村の少年だったとはな!	
\\	あんた、知ってたかい?   はい   いいえ	
\\	そうか、もうそんな有名な話なのか!こりゃ、あのオヤジもハナがたかいだろうな。	
\\	じゃあ、おぼえておきな。伝説の勇者様は、この村の生まれよ!	
\\	あのイタズラこぞうのタータが、伝説の勇者様だったなんて……。世の中、信じられないわっ。	
\\	デナドロ山の東にある魔岩窟って知ってるかい?あそこの岩が、ぽっかり口を開けて中に魔物が入っていくのを見ちゃったよ。でも、さわらぬ魔王にたたりなし!	
\\	トマ「よう、また会ったな。   いっぱい、おごってくれねえか?   おごる   おごらない	
\\	トマ「おう、ありがとよ。   お前さん達の旅の無事をいのって。	
\\	トマ「サンドリノの村長のほしがってる   虹色の貝がらだが、どこかの小島に   あるらしいってことまでは   つきとめたんだが……。   その島がどこにあるか   さっぱり、わからないんだ。   しかし、村長からあずかった前金は   もう、みんな使っちまったしな。   さて、どうしたもんか……。	
\\	トマ「ちェ、ケチだな。	
\\	サンドリノの村長の前金   もういっぱい飮めるくらいは   残ってたかな……?	
\\	俺のせがれが勇者だってんだ。トンビがタカを生んだってヤツかね……。だが、どうひいき目に見ても、そんなごたいそうなヤツじゃねえんだが……	
\\	ま、勇者の親父ってのも悪くねえやね。ヘッヘ。	
\\	わしの孫が勇者らしいぞい!わしの血じゃな!ファファファ……!!!!	
\\	私タータのおよめさんにしてもらうの!	
\\	ハイパーほしにくがあればすんごい料理がつくれるんだけどねえ……。	
\\	ベッドのそばにあるのはこの家に代々伝わる、開かずの宝ばこです。中には何が入っているのやら?もし開けられたらあなたにあげますよ。	
\\	村長「デナドロ山の東にある山は魔岩窟と   呼ばれておる。   あそこに行くと、せすじがゾッと   するんじゃ。	
\\	タータが勇者のあかしであるバッジを持っていたんです。	
\\	はじめは、にせ物かと思いましたがあれはまさしく勇者のあかし。タータは本当に勇者様だったんです!	
\\	今ごろはきっと伝説の剣をさがして北のデナドロ山ではないでしょうか。	
\\	タータとカエル	
\\	「うっひゃ~ッ!	
\\	少年「こ、ここは、とんでもないトコだ!   あ、あんちゃん達も、アブナイぜ   とっとと、ズラかんねーと。	
\\	「うるさい! じゃましないでくれ!	
\\	「山はいいよねえ。	
\\	「いいんだな、これが。	
\\	「君達もしつこいなあ。これやるよ。	
\\	子供「ボクは風だぞー。   ぴゅぴゅ~ん!	
\\	子供「ダメッ!!	
\\	子供「お兄ちゃん達も、取りに来たの?   グランドリオン。   はい   いいえ	
\\	子供「なーんだ、ちがうの。   なら、早く山を下りた方がいいよ。	
\\	子供「うーん、そーか。   ちょっと待っててね……。	
\\	子供「おーい、グラン兄ちゃ~ん!	
\\	グラン「どーした、リオン?	
\\	グラン「やれやれ、またか……。	
\\	グラン「グランドリオンを手に入れて   勇者としての名声がほしいんだろ?   くだらないよ……。	
\\	リオン「人間って、バッカだねー。   手にした力をどう使うかが   大事なのに……。	
\\	グラン「そんな当り前の事も   わからないから人間やってんだよ。	
\\	リオン「どーする、兄ちゃん?	
\\	グラン「決まってるだろ、ためすのさ。   少しばかり、あそんでやろう!	
\\	リオン「うん! 行くぞー!!   ぴゅぴゅ~ん!	
\\	リオン「兄ちゃん、コイツら、やるね。	
\\	グラン「ここまで手こずったのは   サイラス以来だ。	
\\	グラン「決まってるだろ。   本気でいくんだよ!	
\\	リオン「よーし! 今度は……グラン「あそびじゃないぞ!	
\\	ゆうきのグランと……	
\\	ちえのリオン!!	
\\	コンフュ~ジョ~ン!!	
\\	リオン「やられちゃったね、兄ちゃん。	
\\	グラン「なかなか楽しかったな。	
\\	リオン「この人達なら、ボクらを   なおしてくれるかな?   ちゃんと持ち主を見つけて   くれるかな?	
\\	グラン「ああ、だいじょうぶさ。	
\\	ルッカ「これがグランドリオン……。   でも、この伝説の剣……   もうかなり昔に   おれてしまったみたいね……。	
\\	ロボ「この剣が、伝説の   グランドリオン……。	
\\	しかし、これは……   おれてしまってから   かなりの時間がたっていマスネ。	
\\	おれたグランドリオンを手に入れた!	
\\	グラン「ふもとまで風にのせて   一気にはこんでやるよ。	
\\	リオン「ガンバってね   お兄ちゃん達……。	
\\	風にのりますか?   はい   いいえ	
\\	主人「勇者様はデナドロ山に   剣をとりに行ったまま、   それっきり何の話も聞かんけど……	
\\	その後どこに   行っちまったんだろう……?	
\\	行商に行っておるので     しばらく帰ってこんぞよ。    また今度たずねてきておくれ。                 ボッシュ	
\\	ルッカ「あれ? ボッシュはどこ?	
\\	ルッカ「てがみかしら?   なになに……	
\\	ルッカ「な、なんですってえ!!   あのオヤジ、カンジンな時に   いないんだから! まったく!	
\\	マール「あれれ?   ボッシュさんいないね。	
\\	マール「これ、何かしら?   どれどれ……	
\\	マール「何よ それェ!   カンジンな時にいないんだから~!	
\\	イヤン、エッチィ!	
\\	たてふだ「これより 私有地。   立ち入りきんし。 ゲロゲロ。	
\\	「誰だ!?	
\\	カエル「よう、クロノか!   どうした、剣でもならいに来たか?	
\\	カエル「何ッ……!?   王がたおれられた、だと……?	
\\	カエル「……。	
\\	そうか……。	
\\	カエル「俺には、どうする事もできん。   悪いが…… 一人にしてくれ……。	
\\	カエル「俺は勇者などではない。   俺には、どうする事もできん……。	
\\	タータ「ゴ、ゴメンよお……。   この勇者バッジ、ホントは酒場で   よいつぶれてた、カエルのオヤジが   落としてったのさ。   で、高く売れるかと思って   町に出てったら、みんなが勇者様って   チヤホヤしてくれるから   引っ込みがつかなくなって……	
\\	でも、やっぱりオイラには無理だよ。   伝説の勇者だなんて……   もうコリゴリだよ。   このバッジ、あんちゃん達にやる!	
\\	タータの父「ったく、こんなこったろうと   思ったぜ……。   父ちゃん、情けなくって   なみだ出てくらあ!	
\\	いやあ、せがれが勇者だってんでまい上がっちまって、おはずかしい。死んだかかあのためにも、俺もせがれも地道に生きますよ。	
\\	タータ「父ちゃんだって、俺も勇者を育てた   親父かって、ハナたかだかだった   クセにさ……!	
\\	なーに、わしの孫じゃぞ。タータはこんなことではくじけんよ。わしにこっそり教えてくれたよ……『次にめざすは、スーパー伝説の船乗り スペシャルデラックス!!』とな。ファファファ…!!!	
\\	おかみ「帰ってきたみたいね、タータ。   しばらくどこかに行ってたらしくて   村も平和だったのに……。   でも、無事でよかった。   いないとみょうにさびしくてね。   あの人すごく心配してたし……。	
\\	せっかくこの村から伝説の勇者様があらわれたって思ったのによ……。もう、トホホだよ。	
\\	やっぱりタータはニセ者だったのね。ほんとにもう、アイツったら。でも、だとすると、どこかに本物の勇者様がいるってことかしら。	
\\	タータはしょうがないわね……でも、にくめないのよね。	
\\	カエル「また、来たのか……?   む、そのバッジは……。	
\\	カエル「しかし、よくここが……   む、そのバッジは……!?	
\\	カエル「そうか……。   あのチビに会ったのか……。   しかし、もう魔王には手も足も   出ない……。	
\\	カエル「魔王と戦うのに必要な   伝説の剣グランドリオンは……。	
\\	それに、グランドリオンを持つしかくも   俺にはない。カエル「もう、ほっといてくれないか   俺のことは……。	
\\	ルッカ「おれた剣ね……。   グランドリオンの一部だわ!	
\\	ロボ「くだかれた剣……。   グランドリオンの一部デスネ!	
\\	ルッカ「古代文字で何かほってあるわ。   このていどなら何とか読めそう。   えっ…… と……	
\\	ルッカ「……ボ……ッ……   シ……ュ……?	
\\	ルッカ「『ボッシュ』ですって!?   まさか…… メディーナ村の……?	
\\	ロボ「古代のショウケイ文字で   なにか、しるされていマス。   解読してみマショウ……。	
\\	ロボ
\\	……ュ……シ……ッ……ボ……?	
\\	ロボ「ュシッボ……??	
\\	ロボ「アア、さかさまデス!   『ボッシュ』としるされて   いるのデス。	
\\	マール「ボッシュって、あのボッシュ?   メディーナ村の……?	
\\	ど、どーゆー事、クロノ?	
\\	カエル「グランドリオンを   なおせる者は、もうこの世には   いないのだ……。	
\\	ボッシュ「ほう、お主達か。   また、ワシのコレク……!!	
\\	ボッシュ「こ、この剣は、たしかに   グランドリオン! お主達これを   どこで?	
\\	マール「なぜ、この剣にあなたの名が?	
\\	ルッカ「どうして、この剣にあなたの   名前がほってあるのかしら?	
\\	ボッシュ
\\	それはの……、話せば長くなるが……	
\\	ボッシュ「お主達がたずねて来たのは   その事を聞くためではなかろう?	
\\	マール「そうよ! グランドリオン!	
\\	ルッカ「そうだわ! グランドリオン!	
\\	マール「何とか元通りに出来ない?	
\\	ルッカ「復元する事は可能なの?	
\\	ボッシュ「こいつの原料である   ドリストーンっちゅう原石が   今でも入手出来れば……な。	
\\	マール「ドリストーン?   どこにあるのかしら?	
\\	ロボ「ドリストーン……   ワタシのメモリーには記録されていない   石の名前デス。	
\\	ボッシュ「今じゃどこにもなかろう。   昔は金よりも、ねうちがあった赤く   輝く石なのじゃが。	
\\	もう、はるか昔じゃよ。   気の遠くなるほどのな……	
\\	ボッシュ「ま、無理じゃろが、まんがいち   赤く輝く石、ドリストーンが手に入れば   グランドリオンを修復してやるぞい。   それまで剣はお主達が持っとけ。	
\\	で、何か買って行くか?   はい   いいえ	
\\	赤い石 めずらしい石	
\\	老人「ドリストーン……?	
\\	なんだか聞きおぼえのあるひびきじゃ。   昔も昔、大昔の原石じゃよ……。	
\\	原始の時代にでも行けば   見つかるのではないかな?	
\\	マール「な、何? こいつら!?	
\\	ルッカ「み、緑のウロコのはだ!	
\\	ロボ「人間と恐竜の中間……。   ありえない生態けいデス。	
\\	マール「どうしよう、クロノ!	
\\	ルッカ「数が多すぎるわ!	
\\	マール「つっよーい!	
\\	ルッカ「な、何なの? あの女!	
\\	ロボ「彼女の戦闘能力値は   人間の値をはるかに上回っています。	
\\	マール「残りがこっち来るよ!	
\\	ロボ「コチラにターゲットを変こう   したもようデス。	
\\	マール「な、何……	
\\	ルッカ「な、何よ……!	
\\	ロボ「シゲキしてはイケマセン。   現在の彼女の精神じょうたいは   非常にたかぶっています。	
\\	ルッカ「や、やろうってーの!?	
\\	「クロノ!	
\\	女「強いな お前。   名は何だ?	
\\	クロ いうか。 いい名だ。   あたいか? あたいの名は……	
\\	エイラ「エイラ いう。   エイラ 強い男 好き。   だから エイラ クロ 好き。	
\\	マール「は、はなれてよ、クロノ!	
\\	ルッカ「エイラのに気に入られたわね   クロノ……	
\\	エイラ「お前達も 強い。   エイラ 強い者 好き。   男でも 女でも。	
\\	マール「な~んだ、そういう事か。	
\\	ルッカ「わ、私は、そのケはないわよ!	
\\	ロボ「な、何デショウカ?	
\\	エイラ「お前 変わったヤツ。   けど強いな。	
\\	ロボ「ワタシは、あなた方と違って   ロボットデスカラ。	
\\	エイラ「ロボット? 何だそれ?   エイラ わからない。	
\\	ロボ「人間によって作られた   人間型の機械デス。	
\\	エイラ「お前 何言ってる。   わからない事 言うな。   エイラ 頭 火山 なる。	
\\	エイラ「クロ達 どっから来た?	
\\	マール「え、えーと、なんて言えば   いいのかしら……。	
\\	ロボ「コノ方に、理解出来る様   説明するのは、不可能デス。	
\\	ルッカ「明日の、明日の、明日の……   ずーっと明日から、来たのよ。	
\\	マール「ずーっと、ずーっと、ずーっと   先の明日から、来たのよ。	
\\	エイラ「ハハハ! お前 面白い。   エイラ 面白いヤツ 好き。	
\\	ルッカ「ダーメだ、こりゃ。	
\\	エイラ「お前達 村 来る!   かんげいする。 うたげやる。   酒飮む! 踊る! 楽しい!!	
\\	マール「気持ちは、うれしいんだけど   それどころじゃなくってね……。   ドリストーンっていう石を   探してるのよ。	
\\	ルッカ「行きたいのは、やまやま   なんだけど、ドリストーンっていう   石を探してるのよ。	
\\	エイラ「石か? 石 いっぱい。   そこにも ここにも 村にも。   ほしければ いっぱい やる。	
\\	さ イオカの村 こっち。   ついて来い クロ!	
\\	エイラ「クロ おそい おそい!	
\\	うたげ 用意 最中。まだ 待つ。	
\\	エイラ「来たか クロ。   もうすぐ 夜 来る。   うたげの用意 出来た。   こっち クロ!	
\\	エイラ「みんな聞け!   新しい仲間できた!	
\\	強い男 クロ!	
\\	その仲間 
\\	一同「ウホホーッ!!	
\\	エイラ「さ! ボボンガ 踊る!!	
\\	エイラ「さ クロも 飮み 食い   歌い 踊れ!	
\\	マール「せっかくだから、楽しもうよ   クロノ!	
\\	マール「なんか、私、とっても   いい気持ちになって来ちゃった……。	
\\	ルッカ「まいっちゃったわね……。	
\\	ルッカ「ヒック……。	
\\	ロボ「こんな事をしている場合では   ない気がするんデスガ……。	
\\	マール「私も、おーどろっと
\\	マール「サーイコー
\\	クロノも、踊ろーよー!   はい   いいえ	
\\	ダンス!!
\\	やめる。	
\\	ルッカ「うっひゃ~!   クロノ! あんらも、飮みら!   はい   いいえ	
\\	ルッカ「オラオラ! 男れしょーが!   一気にいきな、ググーッと……	
\\	ルッカ「ヴッ……	
\\	ルッカ「あんら、あらひの酒が   飮めないっての? 飮みらさいよ!   はい   いいえ	
\\	ルッカ「キャハハハハハ!	
\\	ロボ「こ、このえき体は……   不じゅん物が多すぎマス……!	
\\	この 音楽 ずーっと ずーっとずーっと ずーっと ずうーっと先まで 伝えたい。	
\\	ボボンガ コインガ
\\	ノインガ ホインガ
\\	歌えや踊れ 風達と
\\	ボボンガ コインガ
\\	ノインガ ホインガ
\\	歌えや踊れ 山達と
\\	ボボンガ コインガ
\\	ノインガ ホインガ
\\	歌えや踊れ この一夜
\\	キーノ「お前ら ヨソ者……   エイラ お前達 かんげいする   わからない……。	
\\	エイラ「飮んでるか? クロ!   コレ 特別な時 飮む酒。   岩石クラッシュ 言う カクテル。   ウマイぞ キツイぞ。	
\\	エイラ「エイラ達 恐竜人   戦ってる。   恐竜人 リーダー アザーラ 言う。   アザーラ とても 頭いい……。	
\\	エイラ「イオカの村 みな 戦う。   戦わない者 ラルバの村 行く。   ラルバの村 どこか わからない。   恐竜人から いつも 逃げてる。	
\\	アザーラ イオカも ラルバも   なくすつもり……	
\\	でも エイラ 負けない。   戦うのやめる それ エイラ   死ぬ時!	
\\	エイラ「ま いい。	
\\	今日 クロ 会えた うれしい日。   飮め 食え 歌え 踊れ!	
\\	エイラ「そうだ クロ   お前達 探してるの この赤い石か?	
\\	これ とても めずらしい。   この石 強いヤツのあかし。   イオカの村で 一番強い エイラ!   だから これ エイラの。   ほしければ エイラと勝負!   クロ 勝ったら コレ クロのもの。	
\\	エイラ「エイラと 勝負するか?   はい   いいえ	
\\	エイラ「逃げるのか? クロ!   クロ 強い。 エイラも 強い!   勝負する クロ!	
\\	エイラ「酒 たくさん たくさん   持ってくる!	
\\	エイラ「今 うたげ だから 酒のむ。   クロと エイラ どっちが   たくさん 飮むか 勝負!	
\\	エイラ「では 始めるぞ。   
\\	ボタンれんだだ!	
\\	エイラ「ぶは~ッ……!	
\\	エイラ「どした クロ?   もう おしまいか?   はい   まだまだ!	
\\	エイラ「クロ まだ 飮めるのか?   エイラ もう 飮めない……	
\\	エイラ「何だ もうダメか。   お前 強いが 酒 弱い……。	
\\	エイラ「もういい クロ……   お前の勝ちだ……、この石 お前の物!	
\\	エイラ「クロ よく飮んだ。   最後に もういっぱい お前の強さ   カンパイ!	
\\	エイラ「じゃ 飮む……	
\\	エイラ「かんぱ~いッ!!	
\\	足跡! 追跡!!	
\\	マール「おはよう、クロノ。   ゆうべは、とっても楽しかったネ!	
\\	ルッカ「うーん……、イタタタ……	
\\	え? 私そんな事したの?   ウソおっしゃい!   けど何でこんな頭イタイのよ……。	
\\	ロボ「お、おはようございマス……。   あのえき体は人間には特別な作用が   ある様デスネ。	
\\	マール「な、何? この足跡!	
\\	ルッカ「ちょ、ちょっと何よ!   この足跡!!	
\\	マール「な、ない!	
\\	ルッカ「な、なくなってる!	
\\	マール「大変! ゲートホルダーが   なくなってる!!	
\\	ルッカ「ゲ、ゲ、ゲートホルダーが   なくなってる……。	
\\	ルッカ「この足跡の主が   ゲートホルダーを……。	
\\	ロボ「この足跡の主が   ゲートホルダーを、うばっていったのは   間違いないデショウ。	
\\	ルッカ「あれがないと、死ぬまで   ここで……	
\\	ロボ「ゲートホルダーなしで   ゲートを使用するのは危険すぎマス。	
\\	マール「エイラに、そうだんして   みようよ、クロノ!	
\\	ルッカ「ジョ、ジョーダンじゃないわ!   エイラに、聞いてみましょう!	
\\	エイラ「グガ~ グゴゴ~!	
\\	エイラ「ん……   おう クロ 起きたか……   ウグ……!	
\\	エイラ「岩石クラッシュ 飮みすぎた。   ムップ……	
\\	ルッカ「ニ日よいの所、悪いんだけど   ねている間に大事な物をぬすまれた   みたいなの。	
\\	マール「気持ち悪いとこ、ゴメンね。   私達の大事な物が……   ぬすまれちゃったの。	
\\	エイラ「な、何? きっと 恐竜……   ウプッ……	
\\	エイラ「恐竜人 しわざ!   クロ エイラと いっしょに来い!	
\\	エイラ「まだ 誰か 来るか?	
\\	エイラ「村の中 見た者 いるはず。   まず 話 聞く!   クロ 行く…… ウップ!	
\\	おまえら 強い!これ あげる!!	
\\	はなびら、きば、つの、はねを3コずつもらった!!	
\\	おまえら ホント 強い!ついでにこれもやる!	
\\	『サードアイ』をもらった!	
\\	ここの品 知りたいか?   はい   いいえ	
\\	『はなびら』と『きば』 ルビーガン『はなびら』と『つの』 ひじりのゆみ『はなびら』と『はね』 ストーンアーム『きば』と『つの』   マンモスのつるぎ	
\\	『きば』と『はね』   ルビーベスト『つの』と『はね』   ストーンメットと それぞれ 交換。一つ 良い事 教える。	
\\	『ルビーベスト』 火 強い。着てる あまり 熱くない。	
\\	よく来た。何 持ってるか?	
\\	『きば』『はなびら』『つの』『はね』2しゅるい 3コづつ あわせて6コと武器 防具 1つ 交換!オマエ 何 くれる?	
\\	はなびら  
\\	コ   きば    
\\	コ   つの    
\\	コ   はね    
\\	コ	
\\	あと 何くれる?	
\\	同じ物 ダメ 2つと交換。	
\\	オマエ 持ってない 交換 ダメ!	
\\	『はなびら』『きば』『つの』『はね』2しゅるい 3コづつ あわせて 6コと武器 防具 1つ 交換!	
\\	オマエ まだ 交換 するか?   はい   いいえ	
\\	『ルビーガン』と交換 いいか?   はい   いいえ	
\\	ルビーガンを手に入れた!	
\\	『ひじりのゆみ』と交換 いいか?   はい   いいえ	
\\	ひじりのゆみを手に入れた!	
\\	『ストーンアーム』と交換 いいか?   はい   いいえ	
\\	ストーンアームを手に入れた!	
\\	『マンモスのつるぎ』と交換 いいか?   はい   いいえ	
\\	マンモスのつるぎを手に入れた!	
\\	『ルビーベスト』と交換 いいか?   はい   いいえ	
\\	ルビーベストを手に入れた!	
\\	恐竜人 手強い。酋長 コレ 持ってく。	
\\	ポーションを手に入れた。	
\\	ヤツら 足跡残し 逃げた。足跡 追う。ヤツら いるとこ わかる! 
\\	俺 見た 恐竜人。広場から 出てきた。南のまよいの森 入った。	
\\	キーノ どこ いった?あさから 姿 見ない。	
\\	まよいの森 こわいとこ。道 いりくんでる。モンスター いっぱい。酋長 気 つける。	
\\	エイラ「キーノ ここで 何してる!	
\\	エイラ「どした 言え キーノ!	
\\	キーノ「ゴ、ゴメン エイラ……   クロ達の物 キーノ ぬすんだ……	
\\	エイラ「なぜ とった!	
\\	キーノ「キーノ エイラ 好き 一番!   エイラ クロ 好き 言う。   キーノ それ イヤ……	
\\	マール「ちょ、ちょっと、エイラ   そこまで……	
\\	ルッカ「ひい~、ス、スンゴい   スパルタ……	
\\	ロボ「エイラさん、暴力は   いけマセン、暴力は。	
\\	エイラ「キーノ なぜ なぐられた   わかるか?	
\\	キーノ「エイラ キーノ 好き   違うから……	
\\	エイラ「それ 違う!   エイラ キーノ 一番好き。   だから いけない事 ゆるさない!   それ 違う!!	
\\	キーノ「エイラ……	
\\	キーノ「ゴメン エイラ!	
\\	キーノ「ゴメン クロ!	
\\	エイラ「さ、クロ達に 大事な物   返す!	
\\	キーノ「……ない。   恐竜人 持って行かれた……	
\\	エイラ「恐竜人!!   どっち行った!?	
\\	キーノ「この木の間 入ってった……	
\\	エイラ「エイラ達 行く。   キーノ 村 もどる。   エイラ いない時 キーノ   村 守る!	
\\	キーノ「わかった エイラ!	
\\	エイラ「キーノ 男……   エイラ 死んだり 子供出来たら   キーノ 酋長。	
\\	マール「エイラ……	
\\	ルッカ「強いな、エイラは。	
\\	ロボ「エイラさん、暴力などと   言って… スミマセン。	
\\	エイラ「さ 恐竜人 追う!   クロ達 大事な物 取り返す!!	
\\	これは一体……。本当に、あのサル共がこんな高度な物を……?	
\\	エイラ「アザーラ!!	
\\	アザーラ「来たなサルが……   ほう、お前達、エイラ達とは   少しばかりデキが違うようだな……。	
\\	フフ、ちょうど良い。   この装置は何に使うものだ……?	
\\	教えてもらえるかな?    はい    いいえ	
\\	アザーラ「ウソをつけ。   そう、かんたんに話してもらえるとは   私も思ってはおらぬ……。   だが、これならどうかな?	
\\	アザーラ「では、しかたない……。   話したくなるようにしてやろう!	
\\	アザーラ「出でよ ニズベール!	
\\	アザーラ「ニ……、ニズベールが   サルどもに負けるとは……。	
\\	エイラ「さ クロ達の それ 返せ!	
\\	アザーラ「……だが、おぼえておれ。   この大地は我が恐竜人の物だと   いう事をな……。	
\\	アザーラ「きさま達サル共なぞに   ゆずりはせんぞ!	
\\	マール「ゲ、ゲートホルダーは   無事?	
\\	ルッカ「こ、壊れてないわよね。   ゲートホルダー!	
\\	ルッカ「ふ~ッ、だいじょうぶだわ。   これで戻れるわよ!	
\\	ロボ「だいじょうぶデス。   特に故障は見あたりマセン。	
\\	マール「よかった   だいじょうぶ、みたい。	
\\	エイラ「いつでも来る アザーラ!   エイラ 負け……	
\\	ウ…… ウゲ……	
\\	エイラ「ゲエーッ! ゲロゲロ……!	
\\	エイラ「クロ 行くか……。   エイラ つまらない。	
\\	マール「色々とお世話になって……   楽しかったわ。	
\\	ルッカ「ありがと、エイラ。   あんたには色々教えられたわ。	
\\	キーノ「ゴメン クロ。   キーノ 強くなる クロよりも!	
\\	エイラ「また来い クロ!   うたげやる。 飮む。 食べる。   踊る。 戦う。 楽しい!	
\\	エイラ「また来い クロ!   
\\	恐竜人 これで しばらく出てこない。 
\\	オマエ カッコいい。	
\\	酋長 強い!オマエも 強い!!	
\\	クロ 強い たたえる!	
\\	アザーラ どこ 逃げた?	
\\	戦え!グランドリオン	
\\	老人「おお、ドリストーンか。   なつかしい輝きじゃ……。	
\\	ドリストーンをあつかえる男か……。   今ではメディーナとかいう   魔族の村近くにきょをかまえておるぞ。	
\\	タバン「お、いいとこに来た。	
\\	タバン「タバン様の大発明。   ルッカ用の防具の完成さ!   大事に使いな!	
\\	ボッシュ「お~ッ、それじゃ!   そのドリストーンがあれば   グランドリオンを修復出来るぞい!	
\\	ボッシュ「しかし、グランドリオンといい   ドリストーンといい、一体どこで……	
\\	ボッシュ「ま、いいわい。   ちょっと、待っておれ。	
\\	ルッカ「私も手伝うわ!	
\\	ロボ「ワタシも手伝いマショウ。	
\\	マール「ちょっと、クロノ!   グランドリオンが直らないのに   どこへ行くつもりなのよ?	
\\	ルッカ「ちょっと、クロノ!   グランドリオンをほっぽって   どこへ行くつもりよ!	
\\	ロボ「クロノさん   まだグランドリオンが修復されて   イマセンガ。	
\\	ボッシュ「さて、と……	
\\	ルッカ「どうやって直すつもりかしら?	
\\	ロボ「修復はムズカしそうデス。	
\\	ボッシュ「今、調べもの中じゃ。   話しかけるでない。	
\\	ボッシュ「まあ、ワシにまかせておけ。	
\\	ボッシュ「グランドリオンを修復するには   ドリストーンを精製するだけでなく   剣を活性化させる必要があるのじゃ。	
\\	おぬしはドリストーンを精製してくれ。   ワシは剣の活性化を行なう。	
\\	ルッカ「わかったわ!   このルッカ様にまかせなさい!	
\\	ロボ「わかりまシタ。	
\\	ルッカ「さ~て、いそがしくなるわよ!	
\\	ロボ「道具をそろえナイと……	
\\	ルッカ「う~ん、けっこうメンドウね。   気がちるから、あっちへ行ってて!	
\\	ロボ「設備がないノデ、精製ハ   手間取リそうデスネ。	
\\	ボッシュ「この剣を活性化するには   一度はおれて力を失った剣の本体に   ふたたび命をふきこまねばならぬ。	
\\	ボッシュ「剣を活性化させる呪文は   高度でむずかしいものじゃ。   しばらく話しかけんでくれんか。	
\\	ボッシュ「よし、終わったぞ。   ドリストーンを持ってきてくれんか?	
\\	ルッカ「こっちも終わったわ。	
\\	ボッシュ「始めるぞ。	
\\	ボッシュ「待たせたの。	
\\	ボッシュ「見るがよい!   これこそが、グランドリオンじゃ!	
\\	マール「スッゴ~い!   これが、あのグランドリオン!?	
\\	ルッカ「これがグランドリオン……   私の知らないエネルギーが出ているわ。   研究イヨクをそそるわね……	
\\	ロボ「不思議デス。   このグランドリオン自体に   精神エネルギーが感じられマス。	
\\	ボッシュ「まあ、ワシにかかれば   こんなもんじゃて。	
\\	マール「じゃあ、クロノ   はやくグランドリオンを   カエルさんの所に   持って行きましょう!	
\\	ロボ「では、クロノ、   グランドリオンをカエル殿に   わたしに行きマショウ。	
\\	ルッカ「じゃあ、クロノ   とっとと、グランドリオンを   カエルの所に持って行くわよ!	
\\	ボッシュ「剣はな、使う者の心のあり方   しだいで、生命をうばう道具にも   救う道具にもなるものじゃ。   使い方をまちがえんようにな。	
\\	老人「その輝き……   グラン……ドリオン?	
\\	そいつを手にして魔王を   倒さねばならぬ者が中世におるはず。   お前さん達ではあつかえまい?	
\\	カエル「また、お前達か……。	
\\	カエル「まさか、この剣は……   グランドリオン……!?	
\\	カエル「……少し考えさせてくれ。   今夜はここで休むといい……。	
\\	リーネ「行ってしまうのですね。   サイラス……。	
\\	サイラス「ええ。 そろそろ誰かが   ゴールデンフロッグのヤツから   あのバッジをうばいかえしても   よいころかと……。	
\\	それに伝説の剣とやらも   この目でたしかめてみたい……。	
\\	ガルディア「サイラスよ、お前は   この国にとって必要な男……。	
\\	また、私とリーネにとっても   かけがえのない友人だ。   きっと、もどって来るのだぞ。	
\\	サイラス「命あるかぎり、かならず。   たとえこの身に、何があろうとも……。	
\\	それでは、これにて……。	
\\	兵士「サイラス様!!	
\\	兵士「われら、王国騎土団一同!   みな団長の旅のごぶじを   いのっております……!!	
\\	サイラス「……お前達。	
\\	サイラス「後の事は、たのんだぞ。	
\\	サイラス「待たせたな。   さあ、行くとするか。	
\\	リーネ「グレン!   あなたも気をつけてね。	
\\	グレン「王妃様も、どうか   お元気で……。	
\\	「この勇者のバッジがほしくば 力ずくでとってみよ、王国の騎土!! グギャギャギャ……!!	
\\	サイラス「むろん、そうさせてもらう。   行くぞ、
\\	・フロッグ!	
\\	ニルヴァーナ・スラーッシュ!	
\\	「ハギャーッ……!!! や……、や…… やりやがったな、このヤロー!	
\\	なんでい、こんなバッジ! お、おぼえてやがれよ、チクショーめ!	
\\	グレン「うわっ!?	
\\	サイラス「あぶない、グレン!!	
\\	グレン「サイラス!   剣が……!?	
\\	グランドリオンが……!!	
\\	ビネガー「ギャハハ、どうしたあ   もう終わりなのかあ?	
\\	伝説の剣がおれてしまっては   手も足も出まいがあ!!	
\\	サイラス「クッ、まだだ……!	
\\	グレン「サ、サイラス……   俺は、もう……。	
\\	サイラス「聞け、グレン。   俺がヤツらの足をとめる。   そのスキにお前だけでも逃げろ。	
\\	グレン「し、しかし……!	
\\	サイラス「このままでは、ニ人とも   やられる……。   行くんだ、グレン。	
\\	魔王「よゆうだな、サイラスとやら。   人の心配などしている場合か……?	
\\	サイラス「いいか、グレン。   行くぞ!!	
\\	サイラス「うおお……ッ!	
\\	グレン「サ、サイラスーッ!	
\\	サイラス「に……、逃げろ……   グレン……   王妃を……   リ、リーネ様のことを……   たの……	
\\	グレン「サ……、サイラス!?	
\\	サイラスーッ!!	
\\	魔王「フン、どうした……。   きさまは来ないのか?	
\\	グレン「くッ……!	
\\	ビネガー「ギョヘヘ……。   ヘビににらまれたカエルってとこだな。   わかぞう。	
\\	ビネガー「魔王様、どうです?   このこしぬけを、にあいの姿に   変えてやるってのは?	
\\	魔王「フッ、よかろう……。   わが前に立ちはだかる者は   一人残らず消す。	
\\	グレン
\\	グレン「う……、うわーッ!!	
\\	グレン「ぐあああああ……ッ!!	
\\	ビネガー「ギャーハハハハ……!   いーくじなしの虫ケラめがあ……!	
\\	カエル「あれから、もう   10年にもなるか……。   やれるか…… この俺に……?   サイラス……。	
\\	カエル「おきろ、クロノ。	
\\	カエル「俺にどこまでやれるのか   わからないが…… 行ってみよう   魔王城へ……。	
\\	カエル「ヤツは強いぞ……。   かくごはいいか……?	
\\	ルッカ「サイエンスの前に敵はないわ!	
\\	マール「悪いヤツには負けないぞー!	
\\	ロボ「未来を見つけだすタメニ……!	
\\	決戦! 魔王城!!	
\\	ジナ「………。	
\\	クロノ……   うちでは、そんな大きなペットは   かえませんよ。	
\\	カエル「お……、奥方。	
\\	私はペットではありません。   姿はこうでも、れっきとした武人です。	
\\	ジナ「あら、ごめんなさい、ブジンさん。   私ったら、てっきり……。   本当にごめんなさいね、ブジンさん。	
\\	カエル「いや、私の名は……	
\\	カエル
\\	ブジンでけっこうです……。	
\\	「うわーん!	
\\	「お前らーッ!	
\\	「やべえ、サイラスだ、逃げろーッ!	
\\	「ヒック……、ありがと…… サイラス……、ヒック。	
\\	サイラス「グレン、男はな、立ち向かって   行かなきゃいけない時もあるんだぞ。	
\\	グレン「でも……、ぶたれたらイタイよ。   アイツらだって……	
\\	サイラス「優しすぎるよ、グレンは……。	
\\	サイラス「なあ、グレン……   俺、騎土団に入ろうと思うんだ。	
\\	グレン「やっぱり!   君なら、きっとりっぱな騎土になれる。   僕がほしょうするよ。	
\\	サイラス「お前もいっしょに来ないか?	
\\	グレン「……、僕には無理だよ。	
\\	サイラス「そうかな。   剣の腕はお前の方が上だと   思うんだがな。	
\\	グレン「ダメさ、僕は。   実戦だと、きっと震えあがっちゃうよ。	
\\	バッジ……   ……勇……者の……	
\\	カエル「グランドリオンをよこせ。	
\\	カエル「我が名はグレン!   サイラスの願いと、こころざし   そしてこのグランドリオン……	
\\	カエル「今ここに受けつぎ   魔王をうつッ!	
\\	カエル「お前達の使っているのは   もしや魔法と言うものか?   今では魔族しか使えないと   聞いていたが。	
\\	「時の最果てに行けば   あなたも使えるようになるかも   しれないわね。	
\\	ロボ「時の最果てに行けば   あなたも使えるようになるかも   シレマセン。	
\\	老人「おや、変わった客人だ。   あんた中世から来なすったか?	
\\	スペッキオ「こりゃまたヘンなの   つれて来た。   こいつ、カエルだけに『水』の力   持ってる。	
\\	カエルが魔法を使えるようになった!	
\\	血文字が……	
\\	『魔王城のジャグラーは…こちらがなぐれば、なぐりの防御をかため魔法をはなてば、魔法の防御をかためるガードのきびしいやつだ……』	
\\	カエル「ここが……	
\\	カエル「魔王城だ……!	
\\	フフ……。	
\\	あそぼ……。	
\\	僕達の宝物……。	
\\	カエル「ビネガー!	
\\	ビネガー「よ~く来たグレン、いや   今はただのカエルだったな。   今度はそやつらが、サイラスのかわりと   いうワケか?	
\\	だが魔王様は今大事なぎしきの   最中だ。   このワシが相手をしてやる。	
\\	外法剣土ソイソー、空魔土マヨネー。   そして!	
\\	ビネガー「この魔王城の   100の魔物を倒せたならばな!	
\\	おどれ、おどれ。こよいはわれら魔族の神がこうたんなさる。	
\\	お、いけにえが来たか!	
\\	死なせてくれ……	
\\	殺してくれ……	
\\	ああ……	
\\	止めてくれ……	
\\	ホネとなっても殺しあうとは、おろかな生き物よ。	
\\	お前ら人間はな!	
\\	助けて下さい……。	
\\	このくるしみから!	
\\	「ほお……	
\\	俺がやったムクロどもを   たやすく……。	
\\	カエル「ひさしぶりだな。   外法剣土ソイソー……!	
\\	ソイソー「お前がここまで来るとは   思いもしなかったぞ、グレン。	
\\	お前もあの時、サイラスがいなければ   同じ運命になっていたのだ。	
\\	ソイソー「先ほどのムクロ達とな!	
\\	ソイソー「やるやる。   ひさかたぶりに、本気で行くか。	
\\	ソイソー「だがサイラスぬきのお前に……	
\\	ソイソー「私がやれるかッ!?	
\\	ソイソー「見事……!   だが魔王様のため戦いやぶれ……	
\\	ソイソー「わが人生にくいなし!	
\\	ほしい? 僕達のたからもの……。   はい   いいえ	
\\	いいモノなのに……。	
\\	そのかわり、僕達と……	
\\	あそんで! あそんで! あそんで!	
\\	クロノ、お祭りから帰らないと思ったらこんなトコにいたのね。	
\\	そんな悪い子は……死になさい!	
\\	マールディア……	
\\	私が悪かった、城にもどってくれ……。	
\\	死体となってな!	
\\	おお、ルッカじゃねーか。	
\\	まだ生きてたのか!	
\\	ロボ、あんまりムチャしちゃダメよ。	
\\	どのみち私がこわしちゃうけどね!	
\\	カエル、ぶじでしたか。	
\\	ならば、私が殺してあげます!	
\\	カエル「空魔土マヨネー!?	
\\	カエル「今のはマヨネーではない……。   本物は!?	
\\	「フフ、ここよ、ここ……。	
\\	「お見限りネ、カエルちゃん!	
\\	マール「何、このコ?	
\\	ルッカ「誰よ、アンタ!	
\\	ロボ「コノ方は?	
\\	カエル「ゆだんするな!   こいつが空魔土マヨネー!   ただの女ではない。	
\\	マヨネー「ムキーッ!   ど~せ、あたいは男ヨ!	
\\	マール「男なの、このコ!	
\\	ルッカ「ゲ! 男なの、アンタ!?	
\\	ロボ「外見はあきらかに女性としか   思えマセンガ……。	
\\	マヨネー「フフン、けど女でも男でも   強い者は美しいのヨネ~。	
\\	マヨネー「サイラスちゃんがいなくなって   さみしーのヨネ、カエルちゃんは。   オマケに、そんなにみにくい姿に   されちゃって……や~ヨネ!	
\\	マヨネー「けど、せっかく新しいお友達と   あそびに来てくれたんですもの……	
\\	マヨネー「み~んなまとめて   おもてなししてあげるのヨネ~!	
\\	マヨネー「あ、あたいがやられるなんて……	
\\	魔王様~ッ
\\	ビネガー「来たか……	
\\	ビネガー「マヨネーッ!	
\\	ビネガー「ソイソ~ッ!!	
\\	カエル「ざんねんだったなビネガー。   ヤツらはもう助けには来れん。	
\\	ビネガー「こ、このケロケロ野郎……	
\\	ビネガー「ビ……	
\\	ビネガー ピ~ンチッ!!	
\\	カエル「逃がさん!	
\\	ビネガー「ウエ~ルカム……!	
\\	ビネガー「フフ、やりおる。   だが、これではどうかな?	
\\	ビネガー「ぜ~は~ぜ~は~ぜ~は~……	
\\	ビネガー「ぬうッ! バカな……!	
\\	ビネガー「ちいッ! うちどめか!	
\\	カエル「ねんぐのおさめ時だな   ビネガー。	
\\	ビネガー「グゲゲ……	
\\	だが魔王様は、すでに   呼び出しているだろうよ……	
\\	ラヴォス様をな!!	
\\	ビネガー「やられはせん、やられはせんぞ!   ワシのバリアーはあらゆる攻撃を   うけつけぬのだ!	
\\	ビネガー「ギエ~ッヘッヘ!!	
\\	ビネガー「どこをねらっておる!	
\\	ビネガー「ギエ~ッヘッヘ……	
\\	ビネガー「ン!?	
\\	マール「な~んだ、たいした事   なかったね!	
\\	ルッカ「何だったのよ、あいつ……	
\\	ロボ「何だったんデショウ?   あのビネガーは……	
\\	カエル「いっこくをあらそう!   急ぐぞ、魔王のもとへ!	
\\	「ダ・ズマ・ラフア・ロウ・ライラ……	
\\	「つむがれよ、天と地のはざまに……	
\\	「この大地の命と引きかえに……!	
\\	カエル「魔王ッ!!	
\\	魔王「いつかのカエルか……。	
\\	どうだ、その後の人生は?	
\\	カエル「かんしゃしているぜ。   こんな姿だからこそ……	
\\	カエル「手に入れた物もある!	
\\	魔王「ほう……。   きさまがグランドリオンを……。	
\\	だが今度は他の者達が、足手まといに   ならねばいいがな。	
\\	魔王「黒い風が、また泣き始めた……。   よかろう、かかって来い……	
\\	魔王「死のかくごが出来たのならな!	
\\	魔王「くッ、きさまグランドリオンを   そこまで……	
\\	ルッカ「こ、これは!?	
\\	ロボ「強大なエネルギー反応デス!	
\\	カエル「ラヴォスってヤツか!?	
\\	魔王「まずい!   今、眠りからさめられては……	
\\	マール「眠り?   あなたが生み出したんでしょ!?	
\\	ルッカ「眠りって!?   あんたが作ったんじゃないの!?	
\\	ロボ「眠り?   記録によれば、ちょうどこの時   ラヴォスはたんじょうしたと……	
\\	魔王「おろか者どもが!   私は呼び出したにすぎん!	
\\	あれは太古の時より地中深く存在し   この大地の力を吸いながら   ゆっくりと成長を続けているのだ!	
\\	マール「何ですって!?	
\\	ルッカ「じゃあ、この時代に   生まれたわけじゃ……	
\\	ロボ「では、ワタシの時代の記録は   事実ではないと……	
\\	カエル「何だ、この感じは!?	
\\	マール「ゲ、ゲート!?	
\\	ルッカ「ゲ、ゲート!?   それも巨大な……!	
\\	ロボ「ありえマセン!   ありえないサイズのゲートの反応デス!	
\\	魔王「おのれ、きさま達さえ   現われなければ……!	
\\	マール「クロノッ!!	
\\	ルッカ「す、吸いこま……!	
\\	ロボ「強大な重力デス!	
\\	カエル「うおおーッ!!	
\\	気がつけば 原始	
\\	マール「いつまでねてるの?   しごとおくれるよ。	
\\	マール「ああ、リーネの鐘があんなに   気持ちよさそうに歌ってる。	
\\	マール「クロノ!   これ以上父上のお世話にも   なってられないんだから   ちゃんとはたらいてよね!	
\\	「クロ!!    クロ! クロ!    クロ! クロ!    クロ! クロ!    クロ! クローッ!!	
\\	エイラ「クロ 気付いたか!	
\\	マール「エイラ!	
\\	ルッカ「エイラ!?	
\\	ロボ「エイラさん。   という事はここは……。	
\\	エイラ「エイラ 会いたかった。   不思議山 行った。   お前達 倒れてた。	
\\	エイラ 一人で みな かついで   テント 連れて来た。	
\\	カエル「ま、魔王ーッ!	
\\	エイラ「このカエル デカい。   みやげか? エイラ 食って   いいか?	
\\	カエル「バ、バカヤロー!   それより、あお白いツラした   マントのヤローは、いなかったか!?	
\\	エイラ「いや お前達だけ。   そいつ お前より ウマイか?	
\\	カエル「……逃がしたか。	
\\	エイラ「ラヴォス?   それより ゆっくり休め。   お前達 傷だらけ。	
\\	マール「でも、魔王がラヴォスを   生み出したんじゃないなら……   ラヴォスは、一体いつから……	
\\	ロボ「魔王がラヴォスを生んだという   記録は、間違いだったのデスネ。   では、せいかくなラヴォスたんじょうの   時はいつなのデショウ?	
\\	ルッカ「そういえば、魔王が   言ってたわね。ラヴォスは、はるか   いにしえより存在するって……	
\\	キーノ「元気か クロ達。   エイラ 今 大変。   ラルバの村 探してる。	
\\	ラルバの村と 力あわせ 恐竜人と   最後の戦い やる!   恐竜人 ほろぼさなければ キーノ達   ほろぶ。	
\\	キーノ!	
\\	キーノ「どした! クロ達 休んでる。   しずかにする!	
\\	北の森 メラ! メラ!! メラ!!!	
\\	酋長 ラルバさがしに 北の森 むかった!酋長 だいじょうぶか?	
\\	交換屋 武器 新しくなったいってみるといい。	
\\	俺 見た 恐竜人!酋長の後 つけてた!北の森 むかってった!!	
\\	空 ひるでも 見える 赤い星ある!	
\\	赤い星 だんだん 光 強くなる!ふきつ……	
\\	ようがんちたい へんな物 出来た。きっと アザーラ!けど 山こえる プテランいる。	
\\	東の ようがんちたい。不気味な声 聞こえる……。聞いたことない 動物 声……	
\\	『はなびら』と『きば』 ドリストンガン『はなびら』と『つの』 ドリストンボウ『はなびら』と『はね』 マグマハンド『きば』と『つの』   たいこのつるぎ『きば』と『はね』   ルビーベスト『つの』と『はね』   ストーンメットと それぞれ 交換。一つ 良い事 教える。『ルビーベスト』 火 強い。着てる あまり 熱くない。	
\\	『ストーンメット』と交換 いいか?   はい   いいえ	
\\	ストーンメットを手に入れた!	
\\	『たいこのつるぎ』と交換 いいか?   はい   いいえ	
\\	たいこのつるぎを手に入れた!	
\\	『マグマハンド』と交換 いいか?   はい   いいえ	
\\	マグマハンドを手に入れた!	
\\	『ドリストンボウ』と交換 いいか?   はい   いいえ	
\\	ドリストンボウを手に入れた!	
\\	『ドリストンガン』と交換 いいか?   はい   いいえ	
\\	ドリストンガンを手に入れた!	
\\	キーノ 恐竜人 さらわれた!恐竜人 さからった者 みな殺しする!たいへん! たいへん!	
\\	恐竜人 突然 現われた。恐竜人 村 やいた。 
\\	さからった者 殺された…… 
\\	エイラ 恐竜人につけられた。エイラ やくびょう神!	
\\	オマエ達 さからうから恐竜人 おこった!	
\\	長老「エイラ……、これ みな   お前のせい……   見ろ この ありさま……	
\\	長老「お前の後 恐竜人 つけてた!   だから この村 こんな目にあった!	
\\	エイラ
\\	すまん……、エイラ うかつ……	
\\	長老「恐竜人 たてつく おろか!   恐竜人 ワシらより 昔から   この地いる。	
\\	だから ワシら かくれてた。   だが エイラ いっしょに戦え   言う……	
\\	エイラ こんな目 あっても   戦え 言うか!?	
\\	エイラ「生きてるなら エイラ   戦う!   勝った者 生きる。 負けた者 死ぬ。   それ この大地のおきて。	
\\	恐竜人も エイラ達も 生き物   みな このおきて さからえない。	
\\	長老 お前達 生きてない。   死んでないだけ。	
\\	長老「エイラ お前 強い……   だから そう言える。   ワシら 力 ない……	
\\	エイラ「それ 違う!   力あるから 戦う 違う。   戦うから 力つく!   エイラ達 力 かす。   だから プテラン!   プテラン 今 必要。   プテラン かしてくれ。	
\\	長老「プテラン……?	
\\	長老「ティラン城 乗り込む気か!?   あそこ 恐竜人の城。   キケン! いくらエイラでも!   死にたいか?	
\\	エイラ「違う 生きたいから 行く。   エイラ だいじょぶ。 たのむ!	
\\	長老「……わかった。   プテランの世話してる者 伝えておく。   プテランの巣 行く。   気をつける エイラ!	
\\	エイラ「長老 助かる!	
\\	長老「エイラの事 たのむ……	
\\	エイラ「クロ!!	
\\	ルッカ「一人で行くなんて   みずくさいんじゃない?	
\\	マール「私達をおいていこうったって   そうはいかないよ!	
\\	カエル「一人で行くつもりか?	
\\	事情はよくわからんが   クロノの友人なら   俺の友人だ……。   一人で死地に向かうような   マネはさせられんぞ。	
\\	エイラ「ダメ! ティラン城 危険!!   みな 死ぬかもしれない!!!	
\\	おんの売り逃げってのは   気に入らないわ。	
\\	ルッカ「ね、クロノ?	
\\	マール「ダメ!   ますます一人じゃいかせられないよ!!   エイラには、ずいぶんおせわに   なったし……   今度はこっちが返す番だよ。	
\\	マール「ね、クロノ。	
\\	カエル「それで一人で死ぬつもりか?   先ほども助けられた……   俺のせいで人が死ぬのは   もうたくさんだ……。	
\\	カエル「いくぞ、クロノ。	
\\	エイラ「オマエら……	
\\	ルッカ「さ、行くわよ!   いざ、ティラン城へ!!	
\\	マール「行こう エイラ!   ティラン城へ!!	
\\	ロボ「では行きマショウ!   ティラン城へ!!	
\\	大地のおきて	
\\	牢番「われら、これから ウタゲ!   お前ら料理されて、さらの上グゲゲ。	
\\	見張り
\\	サルどもぎゃ!!	
\\	キーノ「エイラ!	
\\	エイラ「キーノ 下がる!   エイラ 助ける!	
\\	マール「スッゴーい!	
\\	ルッカ「あんたも入って   どうすんのよ!	
\\	ロボ「その方法は合理的では   ナイような気がするのデスガ……。	
\\	カエル「うぬ。	
\\	キーノ「エイラ ゴメン。   キーノのせい。	
\\	エイラ「そんな事 いい。   それより キーノ ラルバの村人と   先 逃げる!	
\\	キーノ「エイラ どーする?	
\\	エイラ「大地のおきて通り ケリ   付け 行く。	
\\	キーノ「キーノ いっしょ 行く!	
\\	エイラ「キーノ 来る ダメ。   エイラ 負けたら キーノ   イオカ 酋長……!	
\\	キーノ「エイラ……。	
\\	エイラ「エイラ 強い。   負けない!	
\\	キーノ「わかった エイラ   キーノ いいこと おしえる!	
\\	オマエ達強い。 オレ弱い。うらやましい。	
\\	ニズベール「おいおい。	
\\	このニズベール様を無視するとは   いい度きょうだぜ!	
\\	ニズベール「アザーラなら、この奥だぜ。	
\\	ニズベール「まちな。	
\\	ただの奥じゃねえよ。帰ってきた   このニズベール様を倒した奥だぜ!	
\\	アザーラ「やはり来たか……。   これが、最後の勝負になりそうだな。   おそかれ早かれ、決めなければ   ならぬのだ……。	
\\	我々恐竜人か、きさま達サル共か。   この地に生き残る方をな。	
\\	エイラ「それ 決める 大地のおきて。   エイラ 戦うだけ!	
\\	アザーラ「今の声が聞こえたか?	
\\	フフ、後でたっぷり聞かせてやるわ!アザーラ「赤い星め……。	
\\	ふるがいい。	
\\	そして、大地を赤くそめるがいい!	
\\	アザーラ「我々がほろぶ運命だとしても   サル共なぞに道をゆずるわけには   いかぬ!	
\\	アザーラ「フフフ。   このブラックティラノが   お前達サル共を、永遠に歴史から   消し去るのだ!	
\\	アザーラ「うう……。   天は、お前達サル共をえらんだと   いうのか……	
\\	クッ、サル共よ聞け。 そして伝えよ。   我等が恐竜人は、運命に戦いをいどみ   ほこり高くほろびたと……!	
\\	エイラ「わかった……。	
\\	エイラ
\\	アザーラ「始めに、ほのおをまとった大岩   がふってくる……。   しゃく熱の火球は、万物をやきつくす。	
\\	やきつくされた大地は   やがて冷え始め、すべてがこおりつく   長くきびしい時代が来る……。	
\\	フフ、我等が時代のまく引きに   ふさわしいではないか……	
\\	フハハハハ……!	
\\	エイラ「ラヴォス……。	
\\	マール「え?	
\\	ルッカ「ラ ラヴォス!?	
\\	ロボ「ラヴォス……、まさか!	
\\	エイラ「エイラ達の言葉。   ラ 火の事。   ヴォス 大きい事……。	
\\	「エイラーッ!	
\\	エイラ「キーノ!	
\\	エイラ「急ぐ!    ラヴォス おりて来る!	
\\	エイラ「みな プテラン 乗る!	
\\	エイラ「キーノ! いそげ!!	
\\	キーノ「エイラ!   なにしてる!	
\\	エイラ「こい! アザーラ! こい!	
\\	アザーラ「だめだ!   これは、大地が決めたことだ!	
\\	キーノ「エイラ! はやく!!!	
\\	エイラ「アザーラ	
\\	わすれない……	
\\	アザーラ「未来……	
\\	エイラ「未来?   未来がどうした?	
\\	アザーラ「未来を……	
\\	ルッカ「ラヴォスは、この時代に   まいおりて来てたのね……	
\\	ロボ「ラヴォスたんじょうは   王国歴600年ではなく、この時代   だったのデスネ。	
\\	マール「ラヴォスは、この時代に   ふって来てたのね……!	
\\	カエル「魔王のヤツが   生んだんじゃなく空から   ふって来たってのか。   こんな昔に……。	
\\	エイラ「ラヴォス スゴいな。   クロ達 あれと 勝負するか?	
\\	エイラ「クロ!   ラヴォス おりた所   行ってみる!	
\\	ルッカ「そうね、今なら何とか   出来るかも……。   行ってみましょう!	
\\	ロボ「今なら大した力も   ないはずデス。 行きマショウ!	
\\	マール「まだ、ふって来た   ばかりだからどうにか出来るかも!   行ってみようよ!	
\\	カエル「ほっとくと地面の奥で   育っちまうってワケか。   何とかするなら、今のうちだな!	
\\	マール「ゲートが出来てる!	
\\	ルッカ「ゲート!? こんな所に?	
\\	ロボ「ゲートが出来ていマス!	
\\	カエル「ゲートか?	
\\	エイラ「ラヴォス……、すごく 早い!   大地の奥 深く もぐってる。	
\\	マール「このゲート   ラヴォスのせいで出来たのよ。   きっとすべてのゲートもラヴォスの   力がえいきょうして。	
\\	ルッカ「そうか……、わかったわ!   ラヴォス自体の、巨大なエネルギーが   時空間をゆがめてゲートを生むんだわ。	
\\	ロボ「ゲートは、おそらく   ラヴォスのエネルギーが空間をゆがめて   生じるのではないデショウカ?	
\\	カエル「お、おい、こりゃアイツの   せいで出来たんじゃねえか?   魔王城の時は、バカみたいに   デカかった。	
\\	エイラ「どした クロ。   ラヴォス 戦うなら ここ入る   違うか?	
\\	長老「恐竜人 いなくなった。   これ、エイラのおかげ。	
\\	オマエ達 強い。 オレも 強くなる。	
\\	恐竜人 ほろんだ。恐竜人 いいきみ!	
\\	これから オレが正しい思うこと する。かくれる 逃げる もうイヤ!	
\\	恐竜人 いなくなった。ボク達 うれしい!	
\\	恐竜人 ほろんだ! うれしい!!	
\\	酋長と クロ達の 強さ たたえる!	
\\	赤い星 落ちた!とても ふきつ……けど 恐竜人 全部 ほろんだ。うれしい。	
\\	これで 恐竜人 出てこない!うれしい うれしい。	
\\	俺 見た 赤い星!大きな火 ラヴォス 落ちるとこ。すごかった。オレ ちょっと ちびった。	
\\	恐竜人 ほろんだ!うれし! すぎる!	
\\	みな 安心。平和 来た。みな この日 待ってた。	
\\	6500万年『ティラン城跡』へ行きますか?   はい   いいえ	
\\	1万2千年『地の民の大陸』へ行きますか?   はい   いいえ	
\\	ジナ「あらあら、あなた! お名前は?	
\\	エイラ「エイラ いう。	
\\	ジナ「エイラさん……。   わかい女の子がそんな姿で   あるきまわっちゃダメじゃない。	
\\	エイラ「エイラ コレ……	
\\	ジナ「返事は『はい』です!	
\\	エイラ「クロ……。	
\\	クロ 母親も強いな……。	
\\	なぞの物体「セッカク キモチヨク ネテ   イタノニ オコサナイデ クダサイ。	
\\	魔法の王国 ジール	
\\	マール「ドキドキするね。	
\\	ルッカ「なんたってラヴォスに一番   近いゲートだもの。   どこに来たか、わからないわよ。	
\\	ロボ「時空ジャイロは   ……王国歴前1万2千年!?   おそろしいほどの昔デス。	
\\	カエル「ど、どこなんだ?   一体ここは……。	
\\	エイラ「ここ どこだ? クロ!	
\\	ようこそ、エンハーサに。エンハーサは、魔法王国ジールの夢みる町です。ねむりのよろこびの中でしんりを探しているのです。おや……?しかし、あなた方は……。	
\\	ここは永遠なる魔法王国ジール。すべての望みのかなう場所……。	
\\	だけど、そのだいしょうがどのくらい高くつくかは知らないけれどね……。	
\\	あなたの目に見えてる世界とアタシの目に見えてる世界とはまったくちがうものなのかもね。いい? 宇宙は生命の数だけ存在するわ。見えるもの、さわれるものだけが本当と思っちゃダメよ。	
\\	アタシはドリーン。閉ざされた道をもとめなさい。じゅんじょよく、ちしきの扉を開けてね。	
\\	いらっしゃいませ。どんなご用でしょう?	
\\	天冥火水の、星の力により生みだされた武器、防具が、かつて女王の命令で北の宮に封印された。北の大陸に見えるのがその入り口だがもう誰にも、近づけぬのだよ。	
\\	北の宮の扉は、強力な術で封印されています。われら光の民ですらあの封印を解くことはできません。	
\\	あら、ふしぎな旅の方達。これも夢のつづきなのかしら……。ようこそジールへ。	
\\	わが魔法王国ジールをみちびくのは女王ジール様……。この世にふたりといない美しく、いだいなお方ですよ。	
\\	こんにちのジールがあるのはすべて女王様のおかげです。先王なき後、女王様がこの王国の魔の力を高め、はってんさせたのです。スヤスヤ……。	
\\	ここは回復のいずみ……。さあ旅人よ、心の傷をいやし命のかわきをうるおしてお行きなさい。	
\\	女王様の宮殿は、このジールの中心。西の大陸のいただきからつねに世界を見守っていてくださる。だから私達もこうして安心して夢の中にやすらうことが出来るのだよ。ムニャムニャ……。	
\\	北の宮の中は一体どうなっているのかなあ……。	
\\	あの封印を解くカギも宮殿にはあるのだろうけれど……女王様のご命令じゃなあ……。	
\\	運命というものは存在すると思いますか?この世のすべては、あらかじめ決められているのだと……。	
\\	どうでしょう?   まさにそう思う。   ぜったい、そんな事はない。   お前なんかの知った事じゃない。	
\\	そうですか、やはりあなたも……。	
\\	しょせん人の子の力では運命にさからう事など出来ないのでしょうか……。	
\\	そうですよね、やっぱり!私達の道は、私達自身が切り開いて行くのですよね。	
\\	いいですか?そのための扉が一つここにもあるかも知れませんよ。	
\\	西の大陸のはずれにあるのが魔術都市カジャール。さまざまな魔術の研究を行なっております。飛行船黒鳥号もカジャールでせいびされるのですよ。くー、くー。	
\\	ぐー、ぐー。夢の中にこそ、しんじつはある。私の中にこそ、宇宙はある。	
\\	少年「………。	
\\	マール「なーに、この子?   ヘンな感じー。ルッカ「どうかしたのかな、キミ?   おねーさんに、何かご用かしら?ロボ「ワタシ達に、なにかご用で   ショウカ?	
\\	カエル「何だ、ぶあいそなガキだな。	
\\	エイラ「どした お前?   その動物 ウマイか?少年「黒い風が泣いてる……。	
\\	少年「あなた達のうち、誰か一人……   死ぬよ、もうすぐ……。マール「な、何それ!?   いきなり何なの?ルッカ「ちょ、ちょっと!   キミ、どういうこと!?ロボ「ほ、本当デスカ?   よそうカクリツは……カエル「おい、こぞう!   待て……!エイラ
\\	マール「クロノ……   なんだか気味悪いよ……。ルッカ「たちの悪いジョーダンよ。   ひかがく的だわ。ロボ「アノ少年、ナゼ   あんな事を……?	
\\	カエル「なんだか知らんが   気に入らないガキだぜ……。エイラ「あのチビ ヘン……。   でも 悪いヤツ ちがう。	
\\	理の賢者様、もう、いない、ここ。お前達、力、ためすか?   はい   いいえ	
\\	よし、ならば、ゆくぞ。	
\\	お前達、強いな。これ、やる。	
\\	お前達、弱いな。ヌウ……。	
\\	マジックカプセルを手に入れた!	
\\	スピードカプセルを手に入れた!	
\\	ヌウ……。	
\\	『生命のしんぴ』841かん 26しょうすべての生命は、ヌウにはじまりヌウに終わる……。このわしがそー言うのだからまちがいあるまい。たぶん。	
\\	何だ、お前ら!?	
\\	「どうだ、黒鳥号の調子は?兵士「いじょうナシです、ダルトン様!ダルトン「当り前だ、異常があったら   困るだろが!   それより何だ、お前達は!ダルトン「ム? そのいでたち……   お前達が予言者の言っていた、計画を   邪魔する者か。ダルトン「ここはヤツがどう出るか   様子を見るとするか。ダルトン「ま、ヤツが、そうかんたんに   シッポを出してくれるとは思えんがな。   ハッ!	
\\	この黒鳥号はジール文明のしょうちょうだ。 
\\	こんなデカいモンが飛ぶとはお前らにはとても信じられんだろ。	
\\	ここは魔術都市カジャール。王国ジールにさらなる光をもたらす女王の計画のため様々な研究を行なっております。	
\\	あら? あなた方の力は私達光の民のものとは少し感じがちがうようですね。なんでしょう……。不思議な優しさを感じますわ。	
\\	なんでも原始時代にあったという虹色の貝がらと、太陽石をあわせればすばらしい武器やアイテムがつくれたそうだよ。でも今じゃ、そんな技術をもつのは命の賢者様だけだし……それに星の力はすてさるようにってジール女王のご命令もあるしね……。ああ、はなはこんなにも美しくみちたりているというのに……。私は、はなになりたい……。	
\\	南の小大陸に見えるのがかつての太陽神殿だよ。星の力をひめた太陽石をまつっていたのだけれど……	
\\	新しいエネルギーが利用されるようになってから完全に閉ざされてしまったんだ。北の宮と同じようにね。もう古い星のエネルギーなんかいらないんだってさ。	
\\	めずらしいことに最近、ジャキ様の姿をよく見かけますね。	
\\	なんだか落ち着きなく、あちこち出あるいているようですが……。	
\\	話し相手といったら、姉のサラ様といつも連れてあるいているあのネコくらいのもの……。いったいジャキ様は、何を見て何を聞いていらっしゃるのか……。	
\\	ジール女王にはニ人のお子がおられる。サラ様とジャキ様だ。	
\\	サラ様はすばらしいお子であられるぞ。人なみはずれたスゴイ魔力を持っておられるのだ。あれだけの力があれば、人生はもうバラ色であろうなあ……。	
\\	あら、かわった『気』をまとっておられますね。やはり術を使われるのですか。けど私達のものとくらべるとずいぶんと原始的な術のようですわね。ホッホッ。	
\\	そもそも、われらの力はどこから来たのであろうか?なぜ術の力を持つ者と持たない者とが存在するのか……?うーむ……。	
\\	地の民と呼ばれる、魔力をもたぬ者達は雲の下で昔ながらのくらしをしています。しょせん彼らは、私達のような特別な力を持った人間ではありませんものね。	
\\	ここ、プラチナ製防具、ある。でも、じょおう様のゆるし、ないと売れない……。	
\\	……。お前達、もってるそれサラ様のペンダント?   はい   いいえ	
\\	……。ウソつき、きらい。	
\\	……。いらっしゃい。また来て。	
\\	海底神殿は、まだけんせつ中です。計画に少しおくれが出ておるのです。	
\\	あおい海の底の、永遠なる神殿……。完成したらさぞ美しく、りっぱなものになることでしょう!	
\\	ごらんなさい。あの巨大な鳥こそ、理の賢者ガッシュ様が設計された黒鳥号です。じゆうに空をまうさまはまるで生きているかのようです。ガッシュ様はやはりすごい。	
\\	多くの地の民達が、海の底で神殿けんせつにあたっております。あのような者達でも役に立つ時があるのですね。	
\\	ちゅうおうの海の底に見えるのが、女王様のいだいさのあかし、海底神殿です。海底神殿は、あらゆる生命をみちびく輝きの源となるのです。おお、いだいなる、ジール女王よ!	
\\	このジールには、命・時・理の三賢者様がいらっしゃいます。術や知恵で彼らにかなう者はおりませぬ。 
\\	天冥火水の星の力よりもはるかに強力な新しいエネルギーの利用によりジールはここまでさかえたのです。	
\\	宮殿に不思議な予言者があらわれたといううわさを聞きましたよ。くわしいことは知りませんがすごい魔力の持ち主だそうです。	
\\	わしは術の力を高める魔法のカプセルを研究しとるのじゃ。	
\\	しかし最近よく大切なカプセルが消え失せてしまうんじゃが……。	
\\	きっとイタズラ好きのヌウめがかくしてまわっとるにちがいない。まったく、こまったもんじゃわい。	
\\	おっと、そのカプセルは未完成のサンプルじゃぞ。ザンネンじゃったな、ホッホッ!	
\\	うわさですが、サラ様の魔力は母親である女王の力さえはるかにしのぐという事です。それにひきかえ弟のジャキ様は……。王家の血すじであるというのにまったく魔力を持たないらしいのです。	
\\	むかし賢者様から教わったのですが太陽石には、太陽の無限の力をたくわえることができたのだそうです。ただし、力をつかいはたすと輝きを失ってしまって、暗黒石に変わったとか……。	
\\	知っておるかね、旅の人?暗黒石を陽のあたる場所にねかせておけば失われた輝きをとりもどすことができるのじゃ。	
\\	ただし、そのためには気の遠くなるような長い時間が必要でな。あの輝きを目にすることはもう二度とあるまい……。	
\\	こちょこちょ……。	
\\	うくっ……はうっ……!うきゃうきゃ……!	
\\	水は風を呼んで……風は火をおどらせる……	
\\	えたいの知れないものを      見つけてしまった!!	
\\	と思ったら中から『黒の石』が出てきた!	
\\	何千万年もの昔、人の祖先がふと手にした不思議な赤い石のかけら……。そこからすべては始まったのだ。人の知を超えた力は夢をはぐくみ愛とにくしみを生んだ……。それがどのような終わりを見るか……時だけがそのこたえを知っていよう。	
\\	ここは女王様の宮殿。わがジール王国の中心……ということは、すなわちまさに宇宙の中心という事です。	
\\	おおいなるエネルギーの源は海底ふかくで眠っておる。そのたしかな位置が判明したため今回の海底神殿けんせつの計画が実行にうつされたのじゃよ。	
\\	海底神殿への入り口は女王の間にあるらしいよ。でも、まだ神殿は未完成なんだけどね。	
\\	あの旅の予言者は本物じゃぞ。なんでもかんでもピタリとあてちまう。まるで、これから起こる事をすでに知っているかのようじゃ……。気味が悪いわい。	
\\	まあ、ずいぶんかわったかっこうをなさっているのですね。よほど遠くからいらしたのですか?回復のいずみで旅のつかれをいやして行かれるとよいでしょう。	
\\	緑はお好きですか、旅の方達?   好き   キライ	
\\	そう……。それでは女王様と同じね……。まあ! 私も大好きなのです!でも命の賢者様からいただいたこのわかぎを、女王様はもやしてしまえと……。緑を守る力を持った不思議なわかぎだと賢者様はおっしゃっていたのに……。ああ、どうしたらいいのでしょう?   もやしてしまえ   どこかにこっそりうえるそうですか……。女王様のご命令ですものしかたありませんよね……。そうですよね!いくら女王様のご命令でも賢者様からいただいたものをもやしてしまうなど……。わかりました。このわかぎは、どこかに大切にうえておきましょう。いつか緑を守る人のために役に立ってくれるかもしれませんもの。	
\\	もし生まれかわることが出来るなら大地の上で緑と共にくらしたい……。	
\\	魔神器や海底神殿について知りたいならば三賢者様か女王様にお聞きするのが一番ですな。じゃが、ここだけの話、今の女王様ではどんなこたえが返ってくることか……。	
\\	あの男は少し前にふらりと宮殿にあらわれたしかな予言で女王の信をえたのだ。ヤツがどこから来たのか誰も知らない。我ら光の民ではなし、むろん地の民であるわけもない……。 う~む……。	
\\	知ってるかい、旅の人。エンハーサやカジャールには理の賢者ガッシュ様のヒミツの部屋があるそうだよ。	
\\	女王の二人のお子様サラ様とジャキ様は、本当に仲の良い姉弟ですわ。これで、もう少しジャキ様がサラ様以外の人にも心を開いてくだされば……。	
\\	あら、あなたの持っているそのペンダント……。サラ様の物とよく似ていますわね。	
\\	でもサラ様のペンダントのようなしんぴ的な輝きはありませんのね。	
\\	グラン兄ちゃんボク、風になりたいな……。	
\\	ああ、なれるさ。いつか、きっと、な……。	
\\	ドリーン姉ちゃん、またエンハーサにあそび行っちゃったね。	
\\	ああ……。姉ちゃんは夢が好きだからな。	
\\	ボクは風の方がスキだな。ぴゅぴゅ~ん!	
\\	どーしてるかな命の賢者のじーさん……。	
\\	やってらんないよな。まったく人間てのは……。	
\\	ちょっと、せなか……してくれる、こちょこちょ?   こちょこちょ しちゃう   イヤッ!………ケチンボ……。わーい、やったー。うん……そこそこ……。はうっ……!ふーっ……ありがと……。ヌウの、こちょこちょポイントを見きった!!	
\\	夢より生まれ、夢に帰って行く……。そんな生命もありますわ。	
\\	時には、人のいのりや望みがラヴォス様の力をかりて実体化してしまうことがあるのです。	
\\	黒鳥号や海底神殿の設計をなされた理の賢者ガッシュ様も行方がわからないのだ。王国ジールをみちびいてきた賢者様達が三人そろって姿をかくしてしまわれるとは……。	
\\	時の賢者ハッシュ様は、どこかの小島でひっそりと時間の研究を続けているといわれます。『時の卵』をつくられるとか……。しかし、もう何年もその姿を見た者はありません……。	
\\	ここ、まじんきの間へつづく……。	
\\	あの予言者の気は、にくしみと悲しみにみちております……。あのままでは、いずれ自らをほろぼしてしまうことでしょう。自分の愛するものさえみちづれにして……。	
\\	この先は『魔神器』の間じゃ。う~む、すごい気のみなぎりじゃな。さすがラヴォス・エネルギー!	
\\	あなた方も魔神器を見にいらしたのですか?   はい   いいえやっぱり!いだいなるラヴォス様のパワーのうねり、ビンビン来ますよね。ああ、うるうる……。	
\\	あら……。 ダメですよ、感じなきゃー。ラヴォス様のバイブレーション。ファイナル・うるうるですわ!	
\\	これこそが、ジールの力の源。ラヴォス様の無限エネルギーを引き出しぞうふくする『魔神器』です。	
\\	魔神器をコントロールするには引き出すエネルギーを制御できるだけの強い魔力が必要となります。	
\\	三賢者様以外で、魔神器をあつかえるのは今ではもうサラ様だけです。	
\\	三賢者様が、王家に伝わる赤き石よりこの魔神器を生みだされたのです。	
\\	計画のリーダーの命の賢者様はラヴォス様についていろいろと研究されておりました。それで……。	
\\	三賢者の一人、命の賢者様は女王の計画に反対なさってから姿が見られなくなってしまった……。まさかとは思うが……。	
\\	魔神器が完成してから女王は別人の様に冷たくなられてしまった。ここだけの話ですが、女王にさからえば地の民だけでなく光の民も情けようしゃなく消されてしまうといううわさが……。	
\\	魔神器を海底神殿にうつせばラヴォス様から、より強大なエネルギーをえることができる。われら光の民の最後の望み不老不死も夢ではない!おお、いだいなる女王よ!わがジール王国は永遠なのだ!!	
\\	海底神殿が完成すればわがジール王国は、永遠の光につつまれる……。もうそこには、おそれも悲しみもない。ただ終わることのない安らぎだけが待っているのだ。	
\\	以前は女王の右腕、ダルトン様が海底神殿の計画をしきっていたのだが今では旅の予言者が計画を進めている。しばらくは、ダルトン様の近くにはよらないようにしないとな。ありゃ、かなりあれてるぞ。	
\\	女王様は、きっと人に言えぬ苦しみ悲しみにお一人でたえてこられたのではないでしょうか?大きなものをえる時は、失うものもまた大きいもの……。そのおかげで今日のジールがあるわけですけれど……。	
\\	ここ、おやすみの間へつづく……。	
\\	サラ様のペンダントは魔神器と同じ赤き石から作られたのだそうです。サラ様がペンダントを手にして魔神器にいのると、不思議な輝きをはなつそうですよ。	
\\	「姉上!	
\\	「あら、お帰りなさい。 どうかしたの、ジャキ?	
\\	ジャキ「黒い風が……。	
\\	サラ「感じるのね、ふきつな気を……?   だいじょうぶよ、心配いらないわ。	
\\	サラ「さ、これを持っていなさい。	
\\	ジャキ「これ……?	
\\	サラ「お守りよ。   もしもの時ジャキを守ってくれるよう   私のいのりがこめられているわ。	
\\	いつでも私がそばにいて   あげられたらいいのだけれど……。   でも母様の計画が……。	
\\	ジャキ「あんなヤツ、母様じゃない!   姿形は母様だけど   中身は別のモノだ……。	
\\	サラ「………。	
\\	それでも私には……。   ゴメンなさい、ジャキ……。	
\\	しつれいいたします、サラ様。	
\\	女王様がお呼びです。しきゅう海底神殿の方におりていただきたいとのことです。	
\\	サラ「……わかりました。	
\\	サラ「それじゃ、ジャキ……   行ってくるわね……。	
\\	サラ「あら……。   あなた達は……?	
\\	サラ様、おいそぎください。おくれると私が女王様に……。	
\\	サラ「ええ。 ごめんなさいね。	
\\	ジャキ「………。   むこう、行きなよ。	
\\	ジャキ「ムダさ。   アルファドは、僕にしかなつかない。	
\\	よいかな、大地は生きておる。長い時間をかけてゆっくり動き……平地は山に……、山は海に……。今は誰も近づけぬ、北の封印宮や太陽神殿ですら、何千何万もの年をへれば道がみつかるようになるのかも知れぬな。	
\\	ここ、じょおうの間へつづく……。	
\\	この封印の扉は外からは開きません。サラ様のペンダントのような術をやぶる強力なカギでもないかぎり。 
\\	女王が海底神殿におりられるとかで……。その間ここは誰ひとり通ることは出来ませんよ。	
\\	あら、そのペンダントは……?サラ様のものと、そっくりですね。サラ様が魔神器にいのると、ペンダントが不思議な輝きをはなつと聞きましたが……。	
\\	マール「あれー?   やっぱり開かないねー。	
\\	エイラ「不思議。なんでだ?	
\\	ロボ「強力な防御システムの   扉デスネ。	
\\	ルッカ「う~ん……。   やっぱり、そのペンダントに   なにかあるんじゃないのかしら……。	
\\	カエル「チキショー、たかが   こんな扉ひとつに、手も足も   ベロも出ないとは……。	
\\	おや、そのペンダントは……?サラ様のものと似ていますね。サラ様が魔神器にいのると、あのペンダントは不思議な輝きをはなつのですよ。まるで魔神器と、ひみつのことばをかわしているかのようです。	
\\	おお、その輝きは……!?まさしく賢者様達がサラ様のためにと作られたもの!!どうしてあなた方が……!?	
\\	マール「わーい、開いた、開いた!	
\\	ルッカ「フッフッフッ……。   このていどのシールドなんかじゃ   サイエンスの前進はとめられないわよ!ロボ「まさに『賢者の石』デスネ。   マールのペンダントは。	
\\	カエル「さーて、おにが出るか   じゃが出るか?	
\\	エイラ「エイラ 行く!   ガンガン 行くぞ クロ!!	
\\	まあ、その輝きは……!サラ様のペンダントですか?	
\\	予言者
\\	ダルトン「ほう、お前達か……。	
\\	女王「何者だ、お前達!?   どうやって、ここに入った?	
\\	予言者「女王……   その者どもこそ、私のけいこくした   わざわいをなす者達です……。	
\\	女王「何!? 異国の者よ。   お前達も賢者ども同様   このジールにたてつくと言うか!?	
\\	女王「たわけた連中めが!   ダルトン、そやつらをとらえろ!	
\\	ダルトン「はッ!   こころえまして……。	
\\	ダルトン「ゴーレム!	
\\	ダルトン「俺様のゴーレムを……!   許せんな、きさまら!	
\\	女王「フフフ……。   安心しろ、すぐには殺さぬわ……。   あらゆる悲しみ、苦しみ……   そしてきょうふを味わわせてやる。   わらわの計画にさからった事を   たっぷりとこうかいするがいい。	
\\	フフフ……。   アーッハッハッ……!	
\\	ジャキ「フン、バカなヤツら……。   姉上、どうするつもり?	
\\	サラ「この方達をお助けしましょう。	
\\	ジャキ「ムダだと思うけどな……。   それに、こいつらを逃がしたりしたら   姉上が……。	
\\	サラ「だいじょうぶよ、私は……。	
\\	それに、この人達なら   賢者様を助けだしてくれるかも   しれない……。	
\\	ジャキ「姉上……。	
\\	サラ「だいじょうぶですか?   さあ、いそいで宮殿から   お逃げください。   そして、出来ることなら   命の賢者様をお助けして   いただきたいのです……。	
\\	サラ「命の賢者様は、計画に反対したため   なげきの山にゆうへいされて……	
\\	お願いです、どうか賢者様を……	
\\	「そうはさせぬ……。	
\\	予言者「これ以上、お前達に   私のジャマはさせぬ……。   ここいらで消えてもらうとしよう。	
\\	サラ「いけません!	
\\	予言者
\\	ジャキ「やめろッ!	
\\	予言者「……フッ。   よかろう、そいつらの命は   助けてやる。   そのかわり、あなたには   力をかしてもらうぞ。   サラ……。	
\\	予言者「さあ、お前達がどうやって   この時代へ来られたのか   教えてもらおうか……	
\\	予言者「ほう、こんな所から……。	
\\	予言者「さあ、サラ……。   こいつらをその中に放りこんだら   そこに結界をはるのだ。	
\\	サラ「い、いやです!	
\\	予言者「私の言う通りにしなければ   その者達の命はない……。サラ「……わかりました。	
\\	サラ「ごめんなさい。	
\\	とけよ封印 呼べよ嵐	
\\	マール「あの予言者、なぜ私達を   殺さなかったのかしら……。	
\\	ルッカ「あの予言者、私達を殺そうと   思えば殺せたはずなのに……。	
\\	ロボ「ナゼ、ワタシ達を消さなかった   のデショウカ?	
\\	カエル「あの予言者、命だけは助けて   くれたが……、どうもクサイぜ。	
\\	ルッカ「けどラヴォスを倒す手が   あの時代には、ありそうね。	
\\	ロボ「シカシ、ラヴォスのあの力を   利用している時代デス。   倒す方法も、あの時代に……。	
\\	カエル「それにしても、魔王の   ヤツは一体どうしちまったんだ。   この手でカタをつけない事には。	
\\	エイラ「あいつら ラヴォスの力   吸い取ってた。   吸い取られたら ラヴォス 弱って   ないか?	
\\	マール「でも、もうあの時代には   戻れないよ。 どーする?	
\\	ルッカ「けど、あの時代のゲートは   封じられちゃったのよ。   何かいい方法は……。	
\\	ロボ「シカシ、ゲートはふさがれて   しまいマシタ。	
\\	カエル「けど、あのサラって娘が   封印してんだろ?   戻れやしないぜ……。	
\\	エイラ「でも エイラ達 もう   あそこ 行けない。	
\\	マール「そういえば、このペンダントで   開くドアの紋章、あの時代以外で   見た気が……。	
\\	ルッカ
\\	ペンダントで開く   ドアの紋章、どっかで見なかったっけ?	
\\	ロボ「ペンダントで開けたドアの紋章   デスガ、ワタシの時代でも、いくつか   ありマシタ!	
\\	カエル「そういえば、このペンダントで   開いた紋章のついた扉…   お前達、あれをどこかの時代で見たと   言っていなかったか?	
\\	マール「行ってみよう、クロノ!	
\\	ルッカ「あの時代に行くための   手がかりが、つかめるかも!   行くわよ、クロノ!	
\\	ロボ「行ってみマショウ。   ワタシの時代へ!	
\\	カエル「行ってみるしか手はないな。   クロノ?	
\\	老人「ほう、お前さん達、あの魔法の国を   見てきたか……。	
\\	光のはしらが封じられては   時をじゆうに行き来できる翼が   必要じゃな。	
\\	時をわたる翼……   そんなものを研究していた男もおった。   研究にぼっとうするあまり、ちと正気を   うしなってしまったみたいじゃが……	
\\	カエル「剣が立つヤツがいなきゃ   始まらねえだろ。   はい   いいえ	
\\	カエル「よっしゃ、行くとするか!	
\\	カエル「腕がなまっちまうぜ。	
\\	エイラ「エイラ 行く。   ここ たいくつ!   はい   いいえ	
\\	エイラ「今までの分も、エイラ   あばれる!	
\\	エイラ「エイラ連れてけ クロ!	
\\	エイラ「エイラ連れてけ   
\\	パワーリングを手に入れた!	
\\	スピードカプセルを手に入れた!	
\\	大臣「心配しましたぞ、マールディア様!   マールディア様のお気持ち、この大臣   よくわかりますぞ。   国王は何よりも国を大事になさるお方。   そう、あなたの母君アリーチェ王妃様が   なくなられた時も……	
\\	マール「母様が!?	
\\	大臣「いやいや、何でもありませぬ。	
\\	マール「話してよ!	
\\	大臣「もうしあげにくいのですが……	
\\	もともと病弱だったアリーチェ様の   よう態が急変した時です。	
\\	最後に一目国王にお会いしたいと   おっしゃっていたアリーチェ様ですが   国王はなんと……	
\\	大臣「国のしごとでおいそがしいとはいえ   アリーチェ様のもとに来ず……	
\\	おさないマールディア様が   死という事もわからぬまま見守る中……	
\\	大臣「アリーチェ様は、おなくなりに   なられたのですじゃ……。	
\\	いやはや、あれでは国王が   アリーチェ様を殺したも同然……	
\\	マール
\\	父上が……母様を……	
\\	大臣「おやおや、私とした事が!   お気にしませぬようにマールディア様!	
\\	大臣「さ! せっかく戻られたのです。   国王のもとへお立ち寄り下され。	
\\	カギがかかっている。	
\\	めでたき千年祭でもある事だしクロノはとりあえず、しっこうゆうよにしておけとの王のご命令だ。	
\\	かんどうしたとは言いつつも王は帰りを待っておられる。	
\\	裁判所は王の命によりふうさ中だ。	
\\	建国千年パーティをとり行う予定だが王があれでは……	
\\	王の命により、刑務所も今はふうさ中だ。	
\\	千年祭のしゅくがパーティのじゅんび中だ。	
\\	どうした?また裁判にかけられたいか?	
\\	コック長の先祖は料理で魔王軍との戦いにこうけんしたらしい。	
\\	「コラーッ!!	
\\	「手を出すんじゃない!	
\\	「これはマールディア様。 しつれいしました!	
\\	ちゅうぼうに勝手に入ってくんじゃないよ!	
\\	いや、すみませんな。このキッチンは、にょうぼうがしきってるんですよ。にょうぼう・イン・ちゅうぼう!なんてね……	
\\	料理の腕はピカ一だが気性があらくって……何でも血すじらしいんですがね。	
\\	ワシャ姫様の教育たんとうじゃ!	
\\	や! ぼかあ姫様の体育たんとうだ!ガルディア体育大出のバリバリのスポーツマンさ!	
\\	私は姫君のしつけたんとう。今回のけんは、私がいたらなかったばかりに……	
\\	ピエールさんが下でお待ちです。	
\\	ドラゴンせんしゃを作るよう命じたり……あのころから大臣殿は変わってしまった。	
\\	刑務所にいた兵達は大臣殿の口聞きで城に来た者達だ。	
\\	刑務所の兵達はどうにもいけすかなかった。時々、俺達を冷たい目で見やがった。まるで人間じゃない目つきでな。	
\\	裁判で王様をたきつけたのもおそらく大臣殿……	
\\	ベンゴシのピエールです。マールディア様が城を出られてから国王はすっかりふさぎこんでしまい……私の話も聞いて下さり、王女ゆうかいの罪はなかった事と……時間はかかりましたが、ようやっと使命をはたせました。それにしても大臣殿の様子がおかしい。我々はそのけんにかんして話し合っているんです。	
\\	大臣殿だけは、クロノ有罪を最後まで決めこんでいました。	
\\	マールディア様がいないと、この城も火が消えたようだ。	
\\	私もマールディア様が、城にもどる事を願っています。	
\\	マールディア様は、母君のアリーチェ様に生きうつしだ。 だからこそ王もきびしくされてしまうのだろう。	
\\	王はマールディア様の事でだいぶおなやみのご様子。 
\\	一国の王ともなれば、そうだん出来る者もありますまい。	
\\	王はアリーチェ様がなくなられてからマールディア様だけを大事にされていたのですよ。	
\\	うぬぬ……きさま、かならずや処刑してやる!	
\\	おお、お前は……!マールディアはどこにいるのだ?城にもどるよう、お前からも言ってくれぬか?	
\\	さ、マールディア様……	
\\	王「おお、マールディア……	
\\	王「な、何しにもどった!   この私に用などないはずだ!	
\\	マール「……。	
\\	王「何だその目は!   お前が勝手に城を飛び出すからだぞ!	
\\	そうかと思えば、そんなキテレツな   やからを城に入れるなぞ……   チンドン屋でもはじめるつもりか!	
\\	マール「何て事言うの! 私の友達に!	
\\	王「その様な者が友達とは   ガルディア王家のご先祖様に   もうしわけが立たぬわ!	
\\	マール「父上は私よりも……	
\\	マール「私や母様よりも大事なのね!   このガルディア王国が!!	
\\	王「何……?	
\\	マール「母様を殺したのは父上よ!	
\\	王
\\	王「出て行け!   二度と私の前に姿をあらわすでない!	
\\	マール「いわれなくったって   来ないわ!	
\\	王「お前とは親でも子でもない!   勝手にするがいい!!	
\\	大臣「……。	
\\	おお、何という事でしょう…… 
\\	ええい、もういい!お前達も消えるがいい!!	
\\	……。今さらこの私に用などないはずだ!	
\\	そうですな、王のお好きな物でも持ってプレゼントしてはいかがです?王は最近、ハイパーほしにくに目がなくておいでです。	
\\	ハイパーほしにく?ああ、遠い祖先が作ったらしいけどあたいはレシピがわかんないねえ。今じゃパレポリの名物になってるよ。	
\\	おう!何? ハイパーほしにく?あいにくと、ざいりょうがな……だいいち、なみの人間にはキョーレツすぎっぞ。	
\\	うぬぬ……	
\\	クロノのヤツ、かならずや処刑してやる!	
\\	王「そ、それは?	
\\	マール「父上の……好物だよ。	
\\	王
\\	お前が私に?	
\\	王「どれ……	
\\	めし使い「ど、どうされました!?	
\\	王「はうう……!!	
\\	王「うう……	
\\	私の血あつが高い事を知っていながら   こんなゲキカラな物を……!	
\\	マール「え!? それ大好物じゃ……	
\\	王「今度という今度は、カンベンならん!   お前が私をここまでキライなのが   よくわかった!	
\\	マール「ちがうよ、父上……	
\\	王「どこへでも行け!   二度と姿を見せるな!!	
\\	マール「父上!!	
\\	王は誰にも会いたくないとの事です。	
\\	あれ以来、王はねこまれてしまって……お体よりも、気持ちがふさぎこんでおられるようです。	
\\	不思議な力で封印されている……。	
\\	ん?        奥の方に、まだ何かある……           な、なんと!	
\\	ふしぎな紋章がきざまれている。何かの力によって封印されているようだ。	
\\	ペンダントが輝き出す……!	
\\	あ! お兄ちゃん達!	
\\	あ! お姉ちゃん達!	
\\	お父さん、遠くにいるんだって。	
\\	私もあきらめませんわ。この子も、あのめも育ってるんですもの。	
\\	どんな実がなるんだろうな。	
\\	あんたらが来てから、みんな生きようという力がわいて来たみたいだ。これが『元気』ってヤツか?	
\\	ドン「おお、お前さん達! 見てくれ!   あの種子がめを出したんじゃ!	
\\	やっぱアリスドームにも人がいたのか!	
\\	コレ食えるのか……?	
\\	ハアハア……	
\\	何とか廃墟をぬけて、あの種子から出ためを持って来たんだ。死ぬ気になれば何とかなるもんだな……。	
\\	何だコレは?   ショクブツ? 何だそれは?	
\\	こんな物見るの、はじめて……。	
\\	なぞの物体「ワタシニ オコナワレタ   サイゴノ プログラムハ ハカセ   ジシンノ マイソウ デシタ。	
\\	コレデ ワタシハ エイキュウニ   ハカセト オナジ ジカンヲ   スゴスコトガ デキルノデス。   アレ……   モウヒトツ ヤラナケレバ ナラ   ナイコトガ アッタヨウナ……	
\\	ザセキヲ……	
\\	ワスレマシタ。	
\\	「ん……? 
\\	「扉を開けた者達へ……   私は、理の賢者、ガッシュ。   魔法王国ジールの、ガッシュ。   私はジールの大災害のおり   この時代へ、とばされた……。   おどろいた事に   ラヴォスが現われるのは   私の時代だけではなかった……。   はるか太古の時代に、空より落下し   ジールに出現し、地中深くひそみ   この地球のエネルギーを吸いながら   成長を続けた……。	
\\	時は、ガルディア王国暦600年   魔王が呼び出し、一時その姿を現わす。   王国暦1999年   ついに地表をもそのテリトリーにする。   そして、まるで卵を生むかのように   私が死の山と名付けた場所から   自らの分身を次々とたんじょう   させるのだ。   ラヴォスとは、星自体に巣くう   巨大な寄生虫なのだ。	
\\	私は、ここでラヴォスの監視と研究   を続けて来た。	
\\	だが、すでに限界じゃ。   こんな時代に正常な精神をたもつのは   不可能なのかもしれぬ……。   私の精神が死をむかえる前に   この記録を残しておく事にする。   私の生がい最後の発明と共に……。	
\\	私は自分の時代になんとか帰ろうと   研究を続けた。   しかし、この研究が完成するころには   私自身、じゅ命を感じていた。   だから、たくすのだ。   ここを開く者に。   時間を行き来出来れば……   時代を超えて人間が、この星そのものの   ために一つになれば……	
\\	あのラヴォスをどうにか出来るかも   知れぬ……。	
\\	可能性はゼロに等しい……。   しかしゼロでない限り、かけてみる。   この扉を開く者に、この地球のすべて   をかけて……。   さあ、開けるがいい、最後の扉を。   そして手に入れるのだ。   私の最後の発明……   時をわたる翼を……。	
\\	マール「すごい……、これが……   でも、どっちが前なのかしら……?	
\\	ルッカ「これが、理の賢者ガッシュの   最後の発明……。	
\\	ロボ「コレが、先ほどの記録に   あった時をわたる翼……。	
\\	カエル「こ、こいつが時をわたる翼!	
\\	エイラ
\\	これ……。	
\\	なぞの物体「おどろいたようだな。	
\\	私だよ……。	
\\	そう、ガッシュだ。   この物体に、私の頭脳をコピーしてお   いたのだ。   この物体、気に入ってくれたかな?	
\\	君達に説明をしておかなければなら   ないことがある。   それは、他の時代への行き方じゃ。	
\\	緑色のボタンで、タイムゲージが   現われる。   
\\	で行きたい時代をえらぶと   よいじゃろう。	
\\	ガッシュ「あっ! そうだ!   出発の前に、このタイムマシンに……	
\\	ガッシュ「タイムマシンに、名前をつけて   みてはどうかな。	
\\	ガッシュ「さあ、乗った乗った!	
\\	「ねるなーっ!	
\\	なぞの物体「ハカセカラノ メッセージヲ   オツタエシマス。	
\\	なぞの物体「タイムマシン   シヨウジョウ ノチュウイ。	
\\	ミドリボタン   タイムゲージヲダス。   
\\	キー 
\\	ジュウジキー サユウ   タイムゲージノ イキサキヲ エラブ。   アカボタン   ノル オリル タイムワープ。	
\\	イジョウ   ハカセカラノ メッセージヲ   シュウリョウシマス。	
\\	なぞの物体「タイムマシンノ ナマエヲ   ニュウリョク シマス。	
\\	なぞの物体「ムニャ ムニャ?   作動   おりる	
\\	マール「ス、スッゴーい!	
\\	ルッカ「次元空間をこんなスピードで!	
\\	ロボ「ワタシの時代の文明でも   時空間の乗り物などは   なかったというのに……!	
\\	カエル「ひええ~ッ!	
\\	エイラ「こいつ すごい!   こいつ 強い!!!	
\\	「こ、これがシルバードの力!?	
\\	マール「行っけえーッ!	
\\	ルッカ「出番よ、シルバード!	
\\	ロボ「シルバード発進シマス!	
\\	カエル「ハイヨー! シルバード!!	
\\	エイラ「シルバード! 行く!!	
\\	魔王「行くぞ……。	
\\	老人「そうか、あの男、完成させたのか   時をわたる翼を……	
\\	その命とひきかえに……。	
\\	その翼で今一度行ってみてはどうかな?   そのいにしえの魔法の国にな。	
\\	作動   おりる	
\\	シルバードに乗りますか?   はい   いいえ	
\\	力は封印されている……。	
\\	ここはアルゲティ。地の民の村だ。	
\\	ここに住む地の民の多くが海底神殿けんせつのために連行されニ度と帰ってきませんでした……	
\\	何か買ってくかい?   はい   いいえ	
\\	我ら地の民は、魔力を持たないがために天空の王国ジールを追放されたのだ。	
\\	女王は海底神殿に魔神器を設置してラヴォスからさらなるエネルギーを取り出そうとしているが果たして、だいじょうぶなのだろうか?危険だと反対した命の賢者様はなげきの山にゆうへいされてしまった。我らではもう、どうする事も出来ない……。	
\\	長老「賢者様から聞いたのじゃが   ジャキ様はサラ様をもこえる   すごい魔力をひめているらしい。	
\\	しかし、女王をくるわせ、サラ様を   苦しめるその力をきらい、心ごと……   その力を閉ざしてしまっているのじゃ。	
\\	かつては、光の民も地の民もともにくらしていた一つの種族だったのよ。	
\\	三賢者様達とサラ様くらいですよ。われらを同等にあつかってくださるのは。	
\\	サラ様、今度はいつあそびにきてくれるのでちゅかね~?	
\\	ドロクイの巣は、とってもキケンでちゅ。行かないほうがいいでちゅ。	
\\	とまって行くかい?   はい   いいえ	
\\	命の賢者様は、女王の計画に反対したためなげきの山に閉じ込められてしまいました。	
\\	このアナは、ドロクイの巣に続いている。ドロクイの巣はなげきの山を大地につなぐくさりの所まで通じているはずだ。	
\\	ここから先は、と~さないもんね~!	
\\	なげきの山の賢者様	
\\	マール「ここに命の賢者が   閉じ込められているのね……。	
\\	ルッカ「ここに命の賢者が   閉じ込められているのね……。	
\\	ロボ「ココデスネ。命の賢者が   ユウヘイされているのは。	
\\	カエル「ここか……、命の賢者が   閉じ込められてるってのは……。	
\\	エイラ「これ デッカい オリか?	
\\	マール「見て! 消えてゆくわ!	
\\	ルッカ「消える!? そんなバカな!	
\\	ロボ「コ、コレハ!?	
\\	カエル
\\	消えてゆく!	
\\	エイラ「クロ! あれ見ろ!!	
\\	マール「ボ、ボッシュ!?	
\\	ルッカ「ボ、ボッシュ!?	
\\	ロボ「ボ、ボッシュさん?	
\\	カエル「ボッシュのじいさん……!	
\\	エイラ「ボッシュのジジイ?	
\\	ボッシュ「いかにも、ワシは   ボッシュじゃが……。   はて、どうしてワシの名を?   お前さん達に会うのは、初めてじゃぞ?   それより女王は、海底神殿の計画は   どうなった?   お前さん達、何か知らぬか?	
\\	ボッシュ「むう、そうか……   女王の心は、すでにそこまで……	
\\	ボッシュ「魔神器が吸い上げる   エネルギーが大きくなればなるほど   女王は正気を失っていった……。	
\\	海底に眠る不死の神ラヴォスに   女王は人としての心を   食われてしまったのじゃ……。	
\\	ボッシュ「む! ここの封印を解いたため   山が落ちるぞ! 急いで下りるのじゃ!	
\\	長老! な、なげきの山が!!	
\\	ママ、お山が落ちてくるです~!	
\\	ボッシュ「ラヴォスは海底で、この星を   食べながら眠っておるのじゃ。   魔神器を海底までおろせば   ラヴォスそのものを、目ざめさせて   しまうかも知れぬ……!	
\\	ボッシュ「手おくれにならぬうちに   なんとかせねば、大変な事になるぞ……	
\\	長老! サラ様がおいでです!	
\\	ボッシュ「何、サラが?	
\\	「そこまでにしていただきましょうか。	
\\	ジャキ「……きたない所。	
\\	長老「サラ様……、このような所へ   なぜ……。	
\\	サラ「やめて下さい……。   我ら光の民も、もとはあなた達と同じ。   私達は、ラヴォスの力におどらされて   いるにすぎないのですから……。	
\\	サラ「ボッシュ……   なげきの山が落ちたので、ここに来れば   あなたに会えると思ったのです。	
\\	ボッシュ「宮殿を出たりして   だいじょうぶなのか?	
\\	サラ「それどころではありません。   海底神殿が完成してしまったのです!	
\\	ボッシュ「間に合わなかったか!	
\\	サラ「しかし、私がいなければ魔神器は   動きません……。   私はもうこれ以上、魔神器を……。	
\\	サラ「天への道を開いておきました。   はやく女王を……   母を止めて下さい!	
\\	サラ「ダ、ダルトン!	
\\	ダルトン「動かしてもらわなくっちゃあ   困るんですよ。ねえ……サラ様。	
\\	ボッシュ「ダルトン!   お主のすきにはさせぬぞ!	
\\	ダルトン「ジジイは説教くさくて   たまらんね。	
\\	ダルトン「だまっちまいな!	
\\	ボッシュ「ぐはっ!	
\\	ダルトン「さ、いっしょに海底神殿へ   おりてもらいますよ。   邪魔するとサラ様の命は……。	
\\	ジャキ「姉上!	
\\	ダルトン「ったく、女王の血をひくガキは   どいつもこいつも気が強くて   いけない。	
\\	ダルトン「おーっと、お前達も手を出すな。   俺は女王など少しも恐れちゃいない。   こんな小娘のクビなぞ、いつでも   ハネてやるぞ!	
\\	ダルトン「さーて、行きましょうかサラ様。   優しいお母上がお待ちかねですよ。	
\\	マール「ボッシュ!	
\\	ルッカ「ボッシュ!	
\\	ロボ「ボッシュサン!	
\\	カエル「じ、じーさん!	
\\	エイラ「ジジイ だいじょぶか!?	
\\	ボッシュ「だいじょうぶじゃ……。	
\\	ボッシュ「何としても、サラを   救い出さねば、大変な事に……。	
\\	長老「ムチャです!   いかにボッシュ様といえど……。	
\\	ボッシュ「だが、この計画が実行されれば   すべての人間、いや、すべての生命が   危機にさらされる……!ボッシュ「お、お主達が……   行ってくれると言うのか……?	
\\	ボッシュ「すまぬ……、名も知らぬ   お前さん達に何から何まで……。	
\\	マール「名前なら、そのうち知る事になるわよ!	
\\	ルッカ「いいのよ! そのうち   私達が世話になるんだから!	
\\	ロボ「イイエ、ワタシ達も世話になり   マシタ。ギブ・アンド・テイクデス。	
\\	カエル「気にするな。   じいさんには、グランドリオンを   修復してもらった、かりがあるからな。	
\\	エイラ「ジジイ、赤い石使って   クロ達 助けた。   今度 エイラの番。	
\\	ボッシュ「な、何?	
\\	マール「さ、急がなきゃ、クロノ!	
\\	ルッカ「行くわよ、クロノ!	
\\	ロボ「クロノ、急ぎマショウ!	
\\	カエル「クロノ、行くぞ!	
\\	エイラ「さ 行く クロ!!	
\\	ボッシュ「まて、これを持って行け……。	
\\	ボッシュ「こいつは魔神器と同じ赤き石の   かけらで、作られた物じゃ。   これなら、魔神器をこわせる……。	
\\	天空で待つものは	
\\	長老「サラ様がさらわれてしまった……。   魔力を持たぬ我らでは   どうする事も出来ん……。	
\\	やっぱり、光の民は……。	
\\	サラ様、どこへいっちゃったですか?	
\\	え~ん、サラ様が…… サラ様がァ~!!	
\\	それなのに……なぜ、こんな事に……。	
\\	天の道を使えば、ダルトンを追えるはず。はやく、サラ様を……。	
\\	おそらくダルトンはサラ様の開いた天の道を通ってこの村に来たのでしょう。	
\\	ジャキ様は、サラ様を助けるため天の道へと向かわれました。	
\\	理の賢者様や時の賢者様は行方が知れずサラ様までもがとらわれてしまいました。……、女王の計画を止める事はもう不可能なのでしょうか?	
\\	……サラ様は、ご無事でしょうか?	
\\	海底神殿が完成しました。おお、われらジールの永遠の時がはじまる!!	
\\	女王様はサラ様をともなって海底神殿におりられました。ついにわれら光の民が永遠の生命を手にするのです。	
\\	海底神殿への入り口は女王の間にあるらしい。いずれ誰もが光り輝く新しい魔神器を見られるようになるよ。	
\\	海底神殿が完成しました。おお、われらジールの永遠の時がはじまる!!	
\\	わがジール王国は永遠の光につつまれる……。もうそこには、おそれも悲しみもない。ただ終わることのない安らぎだけが待っているのだ。	
\\	この先は魔神器の間じゃがもはや、あの気のみなぎりは感じられぬわい……。	
\\	やっぱり!でも、おあいにく様。魔神器はもうここにはありませんわ。うるうる……。	
\\	ああ……、でも私にももう感じられないのですわ。あのファイナル・うるうるが……!	
\\	魔神器は海底神殿にうつされたよ。ついに、まことのラヴォス・エネルギーを手に入れる時がきたのじゃ。	
\\	ダルトン「ったく、あの予言者のヤローが   中に入れて、なぜこの俺様が   こんなところで見はり……	
\\	ダルトン「ハン、来やがったな……。	
\\	ダルトン「予言者のヤツが   なにかボロを出すかと思って   およがせていたが……。   もう、お前らも用ずみだな。	
\\	ダルトン「いさぎよく死ねーいッ!	
\\	ダルトン「バ、バカな……!   俺は永遠の力を手に入れるんだ。   こんな所でやられてたまるか!	
\\	ラヴォスの呼び声	
\\	リオン「黒い気が、たかまってくる……。   こわいものが目をさましちゃうよ。	
\\	女王「さあ、サラよ、魔神器のパワーを   限界まで上げるのだ。	
\\	サラ「……。	
\\	女王「サラ!   わらわの言う事が聞けぬのか!?	
\\	サラ「わかりました、母上……。	
\\	予言者「…………。	
\\	女王「おお……、なんとまばゆい輝き!   すばらしきラヴォス様の力よ!!	
\\	サラ「うッ……。	
\\	予言者「………。	
\\	グラン「大昔から伝わる赤い石……。   その石からペンダントが、魔神器が   作られ……   そしてナイフがつくられた。	
\\	ボクらはナイフにこめられた、   ボッシュの希望なんだよ……。	
\\	グラン「さあ、女王たちの所に   行くつもりなら、急がないと……。   たのむぜ、兄ちゃん達。	
\\	女王「ああ、感じる……、感じるぞ!   永遠の生命のこどうを……!!   ククク……!	
\\	サラ「こ、この黒い気のうずは……!?   う……ッ!	
\\	予言者
\\	ま、魔神器のようすが……!!女王様、これ以上は危険です!	
\\	サラ「母上……!!	
\\	女王「続けるのだ、サラ!   あともう少しだ……。	
\\	わらわは永遠の生命を手に入れる!   わがジール王国は神の光に   つつまれるのだ!	
\\	女王「ククク……。   アーッハッハッハッ……!!	
\\	ダルトン「ほんとに、こんなとこまで   来やがったのか。   どうやらお前らの力を   見くびっていたようだな……。	
\\	ダルトン「だが、それもここまでだな。   お前らのしまつは   コイツらがつけてくれるわ!	
\\	ダルトン「いでよ!   ゴーレム・シスターズ!!	
\\	ダルトン「ちッ! シスターズを……!   こうなりゃマスター・ゴーレムで……	
\\	ダルトン「な、何だ……   このブキミな気の高まりは?   こ、こいつがホントの   ラヴォス・エネルギーかよ……!?	
\\	ダルトン「………。   なんだかこりゃ、ヤバいことに   なりそうな気がするな……。   永遠の命を手に入れたにしても   その時もう死んじまってたんじゃ   元も子もないからな!	
\\	ダルトン「よし、今日のところは   見のがしてやる!   あばよッ!	
\\	ここ魔神器のさいだん……引き返すなら今のうち、ヌウ……。マール「サラ! 助けに来たよ!   こんなとこ逃げ出しちゃおう!	
\\	ルッカ「どうやら間に合った   みたいね!   ちょっとおしい気もするけど   そのマシン、止めさせてもらうわよ!	
\\	ロボ「ものすごい重力ウエーブ!?   次元のゆらぎが感じられマス!	
\\	カエル「もうこれ以上、ラヴォスに   ちょっかいは出させんぞ、女王!	
\\	エイラ「エイラ 来たぞ!   強いヤツ どこ いる!?	
\\	サラ「あなた方は……!	
\\	サラ「キャアッ!!	
\\	予言者「サラ!	
\\	女王「何をやっている、サラ!   ちゃんとコントロールせぬか!	
\\	マール「クロノ……!   赤いナイフを!	
\\	ルッカ「さあ、クロノ!   いそがないと、マズイわよ……!	
\\	ロボ「ボッシュさんから   あずかったナイフで   魔神器をテイシさせなくテハ……!	
\\	カエル「じーさんのナイフだ   クロノ!   魔神器をとめろ!!	
\\	エイラ「クロッ!   ナイフ 赤いの!!	
\\	グラン「よし、行くぞリオン!	
\\	リオン「うん、グラン兄ちゃん!	
\\	カエル「あ、あれは……   グランドリオン!?	
\\	ルッカ「魔神器のエネルギーで   ナイフが剣に……!?	
\\	マール「あ、あれ……?   グランドリオンになっちゃったよ!?	
\\	ロボ「魔神器のパワーを   キュウシュウして   ナイフが剣になったのデスカ!?	
\\	女王「ま、魔神器がッ!?	
\\	予言者「来る……!	
\\	サラ「い……、いけないッ……!!   あの剣だけでは、この力は   おさえきれない……!!	
\\	「どれほど待ちわびたことか…… この時がくるのを!!	
\\	魔王「ひさしぶりだな、ラヴォス……。	
\\	魔王「遠いあの日、俺はちかったのだ……。   きさまだけは、この手で   たたきつぶしてやると……!	
\\	たとえ、そのために何を失うことに   なろうとも……!!	
\\	ついに、ちかいをはたす時がきた。   死ねラヴォスよ!!	
\\	「出来るかな、お前に……?	
\\	女王「フ、いつわりの予言者めが……。   お前もラヴォス様のえじきに   してくれようぞ。	
\\	サラ「母上、もうおやめください!	
\\	この力は人にわざわいしか   もたらしません!   もうこれ以上は……!!	
\\	女王「そこをどけ、サラ!	
\\	ラヴォス様のいだいなる生命の力は   わらわの中にいきづいておる……。   お前もその一部なのだぞ。	
\\	もはや運命は、変えられぬわ!   ジャマだてしようものなら   お前も消すまで!	
\\	サラ「キャアッ!	
\\	魔王「しまった……!	
\\	女王「行くぞ、予言者よ!   ラヴォス様の力を思い知るがいい!!	
\\	魔王「く……!	
\\	魔王「うぐ……!!   ま、魔力が……   吸い取られてゆく……!?	
\\	魔王「うぐぐ……!   や……、やられぬぞ俺は……!	
\\	ラヴォス、きさまをたおすために……   やみの中、一人生きぬいてきたのだ!!	
\\	魔王「くらえ、ラヴォスッ!!	
\\	うおおお……ッ!!	
\\	魔王「バ、バカな……!?   きかぬのか!?	
\\	ぐああッ!!	
\\	女王「おろか者め……!   ちっぽけなお前達の力など   ラヴォス様には通用せぬわ!	
\\	女王「わらわからの、おくりものだ。   永遠の生命、受け取るがよいわ!   ラヴォス様と一体となってな!	
\\	ククク……!   ハーッハッハッ!!	
\\	マール「動ける? クロノ……   サ、サラさんをお願い……!	
\\	ルッカ「だ、だいじょうぶ?   このままじゃ、みんな……。   う……!	
\\	カエル「ま、魔王ッ……!   きさまだけは、他のヤツに   やらせるわけにはいかん……!	
\\	エイラ「た、立てるか? クロ……   エイラ 負ける イヤ!	
\\	ロボ「次元のゆらぎが   大きくなってきていマス……!   このままでは……!	
\\	マール「クロノ……   サラさんを……!   このままじゃ、ひどすぎるよ……!	
\\	カエル「く……!   待て……、魔王……ッ!!	
\\	ルッカ「こ、こんなところで   私のサイエンスが……   負けてたまるもんですか……。	
\\	ロボ「あのエネルギー体に   とりこまれたら   原子まで分解されてしまいマス!エイラ「くッ……!   エイラ 負けない!   戦い これから……!	
\\	サラ「に……、逃げてください……   クロノ……!   あの生命には人の力では……。	
\\	女王「クロノと言ったか……   サラの言うとおり、しっぽをまいて   逃げ出したらどうだ?	
\\	自分の命はおしいであろうが?   クックック……。   ハーッハッハッ……!	
\\	魔王「サ……、サラは……   だいじょうぶなのか……!?	
\\	女王「ほう、やるというのか?   お前に何が出来る?	
\\	その傷ついた体で、ただ一人   ラヴォス様にいどむというのか?	
\\	女王「ククク……。   死ねい、虫ケラめが!   ラヴォス様の力を見よ!	
\\	ルッカ「いけない、クロノッ!	
\\	ロボ「クロノ!	
\\	カエル「いかん……!!   クロノ!	
\\	エイラ「クロッ!!	
\\	魔王「ラヴォスッ!!	
\\	クロノ……!?	
\\	クロノーッ……!!	
\\	マール「クロノ……?   クロノは、どこ……!?	
\\	いやあーッ……!!	
\\	ルッカ「クロノ……!!   私がついていながら……!	
\\	ロボ「神殿が……、くずれマス!	
\\	今はココより脱出スル事を   さいゆうせんしなくテハ……!	
\\	カエル「く……!	
\\	とにかく今は、ここから   生きて出ることだけを考えろ!   この神殿は、もたんぞ!	
\\	エイラ「ここ 海の底!!   どうする!?	
\\	魔王「俺の力では、ヤツには勝てぬと   いうのか……!?   ラヴォス……!!	
\\	サラ「ペンダントの最後の力をふりしぼれば   みなさんを地上に飛ばすくらいは   出来るでしょう。	
\\	サラ「許されるはずはないけれど……   どうか母を   この国を……   にくまないで……。	
\\	サラ「ゴメンなさい……!	
\\	さあ、地上へ……!	
\\	魔王
\\	サラ「…………。   クロノ……	
\\	つ、つなみだ~!	
\\	マール「ん……	
\\	うーん……。	
\\	クロノ……	
\\	クロ……!	
\\	ルッカ「ダメよ、クロノ……。	
\\	ロボ「危険デス、クロノ……   クロ……!	
\\	カエル「バ、バカヤロー……!	
\\	クロノ……、クロ……!エイラ「クロ……、よせ……!   クロ……!	
\\	ルッカ「クロノッ!	
\\	カエル「クロノーッ!	
\\	長老「ム……、気がつかれたか。	
\\	長老「ずいぶんと、うなされておった……。	
\\	マール「クロノ!   クロノは!?	
\\	ルッカ「クロノ!   クロノは!?	
\\	ロボ「クロノ!   クロノは!?	
\\	カエル「クロノ……。   クロノは?	
\\	エイラ「クロ! クロ どこ!?	
\\	長老「倒れていたのは、あんた方だけ   じゃった……。	
\\	マール「夢じゃ……   なかったんだ……。	
\\	ルッカ「夢じゃ……   なかったのね……。	
\\	ロボ「やはり……   現実だったのデスネ。   アレは……。	
\\	カエル「サイラスばかりか……   クロノまでも……。	
\\	エイラ「それ ウソ!   クロ 強い! クロ 死んだ ウソ!!	
\\	ルッカ「こ、ここは……?   地の民の村……?	
\\	ロボ「コ、ココは……、地の民の村   デスカ?	
\\	カエル「こ、ここは……地の民の村か?	
\\	エイラ「ここ どこ?   地の民の村か?	
\\	長老「海底神殿から起こった大災害で   たったひとつ残された島じゃ……。	
\\	ルッカ「なんて事……   この時代まで……。	
\\	ロボ「コノ時代も救う事は   出来なかったのデスネ……。	
\\	カエル「なんてこった……。	
\\	エイラ「あのデッカい鳥!   雲の上の国!!   みな なくなったか!?	
\\	ルッカ「ボッシュ……、ボッシュは?	
\\	ロボ「ボッシュサン……   ボッシュサンは、イマスカ?	
\\	カエル「ボッシュ……、ボッシュの   じーさんは、どうした?	
\\	エイラ「ボッシュ!   ボッシュ どした!?	
\\	長老「大災害の時、黒いゆがみのような   ものが現われての……   吸い込まれそうになったジャキ様を   助けようとボッシュも……	
\\	ルッカ「黒いゆがみ……	
\\	ゲートだわ……。	
\\	ロボ「黒いゆがみ……	
\\	ラヴォスの生んだゲートデスネ。	
\\	カエル「黒いゆがみ……	
\\	ゲートか……。	
\\	エイラ「黒いゆがみ?   エイラ達 通って来たヤツか?	
\\	ルッカ
\\	サラさんは……?	
\\	ロボ「ではサラさんも……?	
\\	カエル「あのサラとかいう女も……	
\\	エイラ「そうだ サラ?   サラどうした?	
\\	長老「わからん……   サラ様の姿は誰も見ていない……。	
\\	ルッカ「シルバード! シルバードは   どうなったのかしら?	
\\	ロボ「シルバード! シルバードは   無事なのデショウカ?	
\\	カエル「シルバード! シルバードは   どうなっちまったんだ?	
\\	エイラ「シルバード! シルバード   どこ? アイツ 生きてるか?	
\\	長老「あんた方の船なら無事じゃ……。   おどろくほど、がんじょうじゃな。   あんたらの後を追うように   ここにながれついておった。	
\\	ルッカ「マール……。	
\\	ロボ「マール……。	
\\	カエル「マール……。	
\\	エイラ「マール……。	
\\	マール「クロノ……。	
\\	ルッカ「シルバードが無事なら   クロノも、きっと無事よ……。	
\\	ロボ「元気を出してクダサイ。   アナタらしくアリマセンヨ……。	
\\	カエル「大切な人間を目の前で   失っちまったんだ……。   なぐさめなんか俺には出来ん……。	
\\	エイラ「マール 泣く……   クロ 悲しむ。   エイラ 泣かない!	
\\	長老「そうそう、これを返しておこう。	
\\	マール「私の…… ペンダント……。   クロノにあずけた……。	
\\	ルッカ「クロノが持ってた……   マールのペンダント……。	
\\	ロボ「クロノが持っていた   マールのペンダント……デスネ。	
\\	カエル「クロノが持ってた……   ペンダントか。	
\\	エイラ「これ クロノ 持ってた   ペンダント!	
\\	長老「娘さん、あんたのかたの上に   あったんじゃ……。   あんたを…… 守る様にな。	
\\	長老「あんたらのかたわらに   落ちていたんじゃ……。	
\\	あんたらを見守る様にな。	
\\	ルッカ「クロノ……。	
\\	ロボ「クロノ……。	
\\	カエル「クロノ……。	
\\	エイラ「クロ……!	
\\	長老「わしは、この先の広場におる。   ゆっくり休むがええ。	
\\	光の民も生き残った人が何人かいるんだよ……。もう光の民も地の民もかんけいないけどね。	
\\	ジールの太陽神殿には太陽石がまつってありました。太陽石は星の力をひめた暗黒石を陽の当たる場所にねかしておいて作るものなのですよ。	
\\	ただ大きな力をひきだすにはとても長い時間が必要だと聞いております。	
\\	マジックカプセルを見つけた!	
\\	彼はここにながれ着く物をあつめて売っているようですね。しばらくたてば売っている物もかわるかもしれませんよ。	
\\	かう……?	
\\	また来て。	
\\	私達はあんな恐ろしい力にたよっていたのですね……。	
\\	魔法の力もうせてしまった私は……こんな大地で……生きて行けるんだろうか?	
\\	魔法なんて生きていくには必要ないさ。それに、あんた一人じゃないんだぜ。	
\\	そうですね。それに私は生きているのですから。生きることをあきらめなければきっと……。	
\\	明日は明日の風がふくってね……。そう信じていきましょうや。	
\\	みんな、しずんじまった…。	
\\	天上をおおっていた雲海がなくなってふぶきがやんだわ……。長い……、とても長いふゆが終わったのかしら。	
\\	スースー	
\\	泣きつかれたのでしょう……。ねかしておいてください。	
\\	長老「お前さん達、体はもう   だいじょうぶなのかい。	
\\	兵士「え~い、頭が高い!   新王ダルトン様のおな~り~!	
\\	ダルトン「何だ何だ! おめえら!   せっかく生き残ったってーのに   そのシケたツラは!	
\\	ダルトン「あのイカレ女王も、こざかしい   予言者もいなくなったし   これからこのジール王国は   ダルトン王国とするッ!!	
\\	ダルトン「ん? お、お前ら!   しぶとくも生き残ってやがったか!   じゃ、外にある乗り物はお前らのか?	
\\	以前ガッシュの設計していたものと   そっくりだな……。	
\\	ダルトン「よーし、決めた!   お前らにゃ、すぎたオモチャだ!   この新王ダルトンの愛機にしてやろう!	
\\	マール「勝手に決めないでよ!	
\\	ルッカ「ジョーダンじゃないわ!	
\\	ロボ「ソ、ソレは困りマス。	
\\	カエル「な、何言ってんだ!	
\\	エイラ「シルバード! あいつ   エイラ達の仲間!	
\\	ダルトン「お前らにも来てもらおう!   こいつらに変な知恵をつけられても   この俺様の邪魔だ!!	
\\	ダルトン「とりあえず……	
\\	これでも、くらえ!!	
\\	ダルトン「何!!	
\\	ダルトン「ぬぬぬぬぬううう!!	
\\	ダルトン「はっ!!	
\\	ダルトン「何だ、あれは!?	
\\	ダルトン「ヒャアッハッハッハ!!   これが新王ダルトン様の実力だ!	
\\	古代の新王	
\\	マール「うーん……	
\\	ルッカ「うーん……	
\\	カエル「ん……?	
\\	エイラ「フガ……	
\\	マール「ここは……?	
\\	ルッカ「ここは?	
\\	ロボ「ココは……。	
\\	カエル「ここは……	
\\	エイラ「ここ どこ?	
\\	マール「そ、装備が   うばわれちゃってる!	
\\	ルッカ「装備がみんな   うばわれているわ!	
\\	ロボ「装備品をすべて   うばわれてしまってイマス!	
\\	カエル「装備がひっぺがされてやがる!	
\\	エイラ「エイラ達 武器 ない!	
\\	マール「アイテムも……!	
\\	ルッカ「アイテムも!	
\\	ロボ「アイテムもありマセン!	
\\	カエル「アイテムも!	
\\	エイラ「他の物も!	
\\	マール「お金まで……	
\\	ルッカ「お金までないわ!	
\\	ロボ「所持金までうばわれていマス!	
\\	カエル「ちっきしょ! 金まで!!	
\\	エイラ「あの何にでもこうかん出来る   『かね』とかいうヤツもない。	
\\	マール「ど、どうしよう……?	
\\	ルッカ「まいったわね……。	
\\	ロボ「脱出方法を探さねば   なりマセンネ。	
\\	カエル「さて、どうしたモンかな?	
\\	エイラ「エイラ 戦える!   みんな 武器 取りもどす!   あいつ 倒す!!マール「キャアッ!	
\\	ルッカ「わわッ!	
\\	ロボ「そ、空デス!	
\\	カエル「わーったった!	
\\	マール「そ、空! じゃあ、ここは……	
\\	ルッカ「空!? じゃ、ここは……	
\\	ロボ「すると、ココは……	
\\	カエル「そ、空だと!   それじゃ、ここは……	
\\	エイラ「空! 空! ここ空!	
\\	「黒鳥号!?エイラ
\\	ルッカ
\\	ロボ「待ってクダサイ!	
\\	カエル「ム!?	
\\	エイラ「ここも 風 入ってくる。	
\\	ルッカ「エア・ダクトがあるわ。	
\\	ロボ「エア・ダクトのようデス。	
\\	カエル「ここから上に行けるな……。	
\\	ルッカ「そこから機内の様子が   探れるわね。	
\\	ロボ「ソコをのぞいて、敵の配置を   はあく出来マスネ。	
\\	カエル「そっから、敵の様子が   わかるな……。	
\\	エイラ「そこ 下 見える!	
\\	見張り「いいザマだ。 武器がなけりゃ   何も出来まい。	
\\	マール「どうしよう、
\\	病気のフリ   他の手を考える	
\\	ルッカ「何か思いついた、
\\	病気のフリ   他の手を考える	
\\	ロボ「どうシマスカ、
\\	病気のフリ   他の手を考える	
\\	カエル「さて、行くか、
\\	病気のフリ   他の手を考える	
\\	エイラ「行くか? 
\\	病気のフリ   他の手を考える	
\\	見張り「フン、もうその手は食わんぞ。	
\\	ルッカ「なるほどね。   見張りをさそい込むのね。	
\\	ロボ「ナルホド、見張りを   ゆだんさせる手デスネ。	
\\	カエル「そうか、見張りをさそい込む   気だな。	
\\	エイラ「エイラ 元気だぞ……!	
\\	そういうなら まあいい。	
\\	ルッカ「う、う~ん! イタタタ……!	
\\	ロボ「コ、故障デス!   制御不能! タ、助けてクダサイ~!	
\\	カエル「いてーッ!    ヘ、ヘソがイテテテテ!	
\\	エイラ「うおーッ!   エイラ いたいぞーッ!	
\\	見張り「どうした!?	
\\	ルッカ「作戦成功!	
\\	ロボ「ナ、何とか成功した様デス。	
\\	カエル「ヘッ、バーカ。   カエルにヘソがあるかって!	
\\	エイラ ウソ キライ……。	
\\	お、お前ら、どうやって!?	
\\	お、お前ら!?	
\\	おとなしくしてろ!	
\\	ダッソウ!
\\	ダッソウ!	
\\	「きさまらーッ!!	
\\	ダルトン「手をぬくんじゃねえ!   なにしろこの新王、ダルトン様の   空飛ぶ玉座なんだからな!	
\\	ダルトン「これでこの世界は、この俺の   意のままだ。	
\\	ダルトン「ヒャアッハッハア!	
\\	ルッカ「ちょ、ちょっと!   あいつシルバードを、いじくり   回してるわよ!	
\\	ロボ「シルバードを改造している   もようデス。	
\\	カエル「あの野郎、人様のモン   勝手にいじくりやがって!	
\\	エイラ「シルバード 何かされてる!	
\\	「シーッ……!	
\\	ルッカ「ちょっと! まだ全部   取り返してないわよ!?   かまわない   戻る	
\\	ロボ「他にもうばわれた物が   ありマスがいいんデスカ?   かまわない   戻る	
\\	カエル「まだ取り返してないモンが   あるが、いいのか?   かまわない   戻る	
\\	エイラ「とられた物 まだ ある!   それでも いいのか?   かまわない   戻る	
\\	お金を取り戻した!	
\\	アイテムを取り戻した!	
\\	の装備を取り戻した!	
\\	の装備を取り戻した!	
\\	の装備を取り戻した!	
\\	マール「な、何?	
\\	ルッカ「こ、こいつ、あの……!	
\\	ロボ「強大な魔法エネルギー反応   デス!	
\\	カエル「来るぞ!	
\\	ルッカ「ふうッ……。	
\\	カエル「くどいんだよ。	
\\	エイラ「エイラ 負けない!!	
\\	「フハハ……	
\\	出来た! 出来たぞ!!	
\\	ダルトン「見よ、この美しい翼を!   これこそ新王ダルトン様の玉座。   名付けて……	
\\	ダルトン「スカイ・ダルトン・ギョクーザ!	
\\	さあ、最初のいけにえはヤツらだ!	
\\	ダルトン「飛び立つのだ!   スカイ・ダルトン・ギョクーザ!!	
\\	ダルトン「ちがうちが~うッ!	
\\	ダルトン「ストップ・ザ・ミュージック!	
\\	ダルトン「フハハ! これだ、これだ!	
\\	ダルトン「出すぞ!	
\\	発進んん~ッ!!	
\\	「あ、あれ!	
\\	ロボ「ア、アレは!	
\\	カエル「あ、ありゃあ!	
\\	「シルバードが……	
\\	エイラ「シルバード 空……	
\\	「飛んでるーッ!?	
\\	ダルトン「シルバードではなーい!   スカイ・ダルトン・ギョクーザだ!!	
\\	ダルトン「ヒャヒャヒャアーッ!!   今度の今度こそ、用ずみだぞ、お前ら!	
\\	ダルトン「空のもくずと消えろーッ!!	
\\	ダルトン「ヒャハハ!   虫ケラらしいぞ、ちょこまかと!	
\\	ルッカ「あいつ~……!	
\\	ロボ「シルバードを取り戻すのデス!	
\\	カエル「くっそー!   いい気になりやがって!	
\\	エイラ「シルバード 今 行く!	
\\	マール「あんたらのせいよ……!	
\\	クロノを……	
\\	クロノを返せーッ!!	
\\	ルッカ「ダルトーンッ!	
\\	ロボ「行きマス!!	
\\	カエル「この野郎ーッ!	
\\	エイラ「うおーッ!	
\\	ダルトン「き、きさまらーッ!	
\\	マール「ハアハア……	
\\	ルッカ「シルバード、無事だった!?	
\\	ロボ「ダッカン成功デス!	
\\	カエル「あのコソドロ野郎!	
\\	エイラ「シルバード 元気か?	
\\	ルッカ「そ、そんな事より操縦法は!?	
\\	ロボ「ソレより、操縦方法が   わからなければ、このまま落下シテ……	
\\	エイラ「シルバード おとなしい……。	
\\	マール「だ、だいじょぶ!?	
\\	ルッカ「そ、そのレバーは!?	
\\	ロボ「このタイプの計器は   あつかった事が……	
\\	カエル「お、俺には、何が何やら……	
\\	マール
\\	そこの緑のボタンじゃない!?	
\\	ルッカ
\\	そこの緑のボタンを!!	
\\	ロボ
\\	その緑のボタンをおしてみてクダサイ!	
\\	カエル
\\	その緑のヤツじゃないか!?	
\\	マール「あ、あおいの   おしちゃえーッ!	
\\	ルッカ「そ、そこの   あおいボタンは!?	
\\	ロボ「そ、その   あおいボタンを!!	
\\	カエル「あ、あおいのなら   どうだーッ!?	
\\	「うわあー!「やったー!「ダルトンのヤツ、ざまあねーや!	
\\	長老「おお、あの船が飛んでおる……!	
\\	ルッカ「ふーッ、大体わかったわ。	
\\	ロボ「ワタシのユニットと直結すれば   良かったのデス。	
\\	カエル「こーりゃ、気分いーぜ!   ま、あの野郎のおかげって   事にしといてやるか。	
\\	エイラ「シルバード 元気!   シルバード 鳥 なった!!	
\\	ロボ「ケガのこうみょうというやつ   デショウカ?   ぐうぜんにも、飛行機能が   手に入りマシタネ!	
\\	ルッカ「私が改造しようと   思ってたのに……	
\\	ルッカ
\\	アイテムがあったわ!	
\\	ロボ
\\	コレはワタシ達のアイテム!	
\\	エイラ達の物 あった!!	
\\	ルッカ「あいつ   持って行こうとしてたのね!	
\\	ロボ「ダルトンが持って行こうと   したのデスネ。	
\\	カエル「あの野郎、人様のモンを!	
\\	ルッカ「お、お金はないわ……!	
\\	ロボ「所持金はアリマセン!	
\\	カエル「けど、金は見あたらないぜ!	
\\	ロボ「黒鳥号の中……デスネ。	
\\	カエル「黒鳥号といっしょに   海のもくず……か。	
\\	エイラ「ん? みんな どした?	
\\	ルッカ「マール……	
\\	ロボ「このシルバードが   無事だったのデス。   きっとクロノも……。	
\\	カエル「あのバカ……。	
\\	エイラ「クロ 強い!   クロ 死なない!!	
\\	マール「クロノはどこかに   いるはずよ! 私探すよ!!	
\\	ルッカ「もしかしたらクロノも   この島にたどりついてるかも……。	
\\	ロボ「クロノを探しマショウ!	
\\	カエル「生死はともかく   探してみるかクロノを……。	
\\	エイラ「クロノ 探す!   クロノ ぜったい 生きてる!!	
\\	売っている物がふえたみたいですね。	
\\	北の岬に人影をみました。変な気を起こさなければいいけど……。	
\\	いや~、スカッとしたね……。あんたたちを見てたら、いつまでも落ちこんでられないな。	
\\	わ~い、わ~い!	
\\	知ってますか?あなた達の後に、ここにながれ着いた人がいたそうですよ。誰かを探してるようでしたが。	
\\	あなた達の後にながれついた人……。天上界でも見かけた事がない人だったわ。もしかして、あなた達のお仲間かしら? 
\\	よかった……。すっかり元気を取り戻して。	
\\	長老「村のみんなも、少しづつ元気が   でてきたようじゃ……。	
\\	いちからやりなおせばええ……。   タダそれだけのことじゃよ。	
\\	魔王「お前達か……。	
\\	マール「ま、魔王……。	
\\	ルッカ「ま、魔王……。	
\\	カエル「ま、魔王!!	
\\	魔王「見るがいい。   すべては海の底だ……。   永遠なる夢の王国ジール……。	
\\	魔王「かつて私はそこにいた。   もうひとりの自分としてな……。	
\\	ハッシュ「むう……!   この次元のゆがみは……!?	
\\	ガッシュ「やはりムリであったか!?   人の手で、この力を   コントロールするのは……!	
\\	ボッシュ「いかん……!   このままではヤツの次元のうずに   のみこまれるぞ!!	
\\	「姉上……ッ!!サラ「いけない、ジャキ!   来てはダメ……!ジャキ「で、でも……!?	
\\	ボッシュ「こ、これは……!!	
\\	ハッシュ「まさか、タイムゲート!?	
\\	ガッシュ「しまった……!!	
\\	ジャキ「あ……、姉上ーッ!!	
\\	サラ「ジャキッ!?	
\\	時の賢者ハッシュ「こ、ここは……	
\\	おーい。	
\\	何もない……。   ここは……	
\\	時の果てか……?	
\\	マール「あなた、もしかして……	
\\	ジャキ……?	
\\	ルッカ「あなた……。	
\\	もしかして、あの……。	
\\	カエル「そうか、お前……   あの時のガキ……!	
\\	魔王「……。	
\\	魔王「私はヤツを倒すことだけ   考え生きてきた……。   ヤツが作りだしたうずに飮み込まれ   中世に落ちて以来な……。	
\\	わが城でラヴォスを呼びだす事を   お前達にジャマされ……	
\\	ふたたび次元のうずに飮み込まれ   たどりついた先がこの時代とはな……。   ヒニクなものだ……。   歴史を知る私は、予言者として女王に   近づきラヴォスとの対決を待った……。   しかし結果は……。   ……。   ラヴォスの力は強大だ。   ヤツの前では、全ての者に   黒き死の風がふきすさぶ……。   このままではお前達も同じ運命だぞ。   あのクロノとかいうヤツとな!	
\\	マール「クロノは!   クロノは、あなたのせいで……!	
\\	ルッカ
\\	クロノ……。	
\\	カエル
\\	あいつをぶじょくする気か……!	
\\	魔王「ヤツは死んだ!   弱き者は虫ケラのように死ぬ。   ただそれだけだ……。	
\\	マール「許せない!	
\\	魔王「今ここでやるか……?   はい   いいえ	
\\	マール「私達があなたを   倒したって……。	
\\	クロノはよろこばないよね……。	
\\	ルッカ「クロノ……。   あなたを倒したところで   クロノはよろこばないわ……。	
\\	カエル「今キサマを倒したところで   クロノは戻って来ん……。   サイラスもな……。	
\\	魔王「待て。	
\\	魔王「私も行ってやる。	
\\	マール「え!?	
\\	ルッカ「な、なんですって!?	
\\	カエル「ふ、ふざけるな!!	
\\	魔王「ヤツを……   クロノを生き返らせる手   ないわけではない……。	
\\	マール「ホ、ホント!?	
\\	ルッカ「え!?	
\\	カエル「何!?	
\\	魔王「時の賢者ハッシュなら   失った時を取り戻す方法を   知っているはずだ……。	
\\	魔王「よかろう……。	
\\	だが私も魔王と恐れられた男……   タダでは殺らせはせぬ!	
\\	魔王「私が死ねば……   カエルの呪いは   いずれ解ける……。	
\\	カエル「お前達は下がっていろ……。	
\\	まさか、お前とこの時代で   決着をつける事になるとはな……   来い……!	
\\	カエル「サイラス……。   やったぞ……。	
\\	クロノを生き返らせたければ……   時の……	
\\	時の賢者に……、会え……。	
\\	サラのおまもりを手に入れた……。	
\\	時の卵	
\\	おじちゃん、このネコ知ってるの?	
\\	マール「な、何、あれ!?	
\\	ルッカ「きょ、巨大なゲートの反応!?   ま、まさか! ラヴォス!?	
\\	ロボ「シルバードの時空ジャイロが   大きくみだれてイマス!	
\\	カエル
\\	魔王「あれは……!	
\\	マール「私達を、呼んでるみたい……。	
\\	ルッカ「な! なんてそんざい感!   見てるだけで吸いこまれそう……。	
\\	ロボ「巨大デス、とてつもなく!   強大な精神エネルギーが測定出来マス!	
\\	カエル「し……死者の船か?    おむかえに来たってのか!?	
\\	エイラ「アレ にくんでる!   エイラ達 にくんでる!!	
\\	魔王「……海底神殿?   バカな……!	
\\	ありゃー、一体なんなんだ?	
\\	あれは!?海底神殿!? いや、まさか……	
\\	あの船からは大きな魔力を感じます……。ラヴォスに似ている魔力を……。	
\\	一体何がおこったんじゃ……。	
\\	あんな、まがまがしいものはみたことないわ……。	
\\	また何か始まるのかしら……。	
\\	でっかい船……。	
\\	やっぱりダルトン様は王ってうつわじゃないよな。	
\\	ダルトン様、人使いあらいしきゅうきん安かったし。	
\\	命の賢者様から、わかぎをいただいたのですが、もうこの死に果てた地では意味はないのかもしれません。ああ、どうしたらいいのでしょう?   意味がないならもやしてしまう   希望を持ってうえる	
\\	そうですね……。こんな世界ではしかたありませんよね。	
\\	わかりました。 このわかぎはどこかに大切にうえておきましょう。いつか緑を守るために役に立ってくれるかもしれませんもの。	
\\	この不思議な力をもったわかぎ……いつか緑を守るために役に立ってくれるでしょう。	
\\	なんだい、どうしたんだい?ヌゥが気持ちよくねているってのに……もう……。
\\	の名前を変えてあげるよ。だから、ねむらせておくれよ。	
\\	じゃあ、おやすみ。	
\\	あ、そうそう。ボクが持っていてもしかたがないから、	
\\	キーノ
\\	達 よく来た!   休め ゆっくり。	
\\	キーノ「エイラ!!   元気か? キーノ 元気!	
\\	キーノ「ラヴォス 落ちた。   恐竜人 ほろんだ よかった。   けど あれから さむくなってる……   かり あまり 出来ない……。	
\\	でも キーノ 負けない!   エイラ しんぱい いらない。   キーノ まかす。	
\\	空 雲 多くなった。お日様 あまり 見えない。	
\\	ふってくる 白い 冷たい。アレ 何?	
\\	ブル! さむい……!	
\\	イオカには カンシャしてる。村 なくなった われらむかえ入れてくれた。	
\\	交換屋 のぞいた。新しい 防具 出来た。	
\\	ヘンな石 見つけた!持ってると 気持ち悪かった……すぐ すてた。	
\\	エイラ 酋長 やめた。今 キーノ 酋長。	
\\	おおきな火 ラヴォス。あれ 落ちてきた。さむくなった……	
\\	元気の水 取れなくなった。こんな事 はじめて。	
\\	さむくなった モンスター へった。動物も へった。	
\\	ルビー いっぱい かためてよろい 作った。『はなびら』『きば』『つの』『はね』10コづつ あわせて40コと交換。オマエ 交換 するか?   はい   いいえオマエ まだ ほしいか?   はい   いいえ    ルビーアーマーを手に入れた!	
\\	なぞの物体「コノサキ キケン   ジカンジクガ ミダレテイマス。	
\\	老人「おや、あの元気のいい   お兄ちゃんは、どうしたね?	
\\	老人
\\	大した事は出来んが、このきょくを   おくらせてもらうよ。   だいして……	
\\	『クロノ・リメンバー』……。	
\\	老人「私にできる事があれば、力になって   あげたいのだが……	
\\	マール「時の賢者を探してるの……。	
\\	魔王「時の賢者にしか出来ぬ事が   あるのだ……。	
\\	エイラ「エイラ達 時の賢者   あいたい。 探してる!	
\\	ロボ「時の賢者という方を   探しているのデスガ……	
\\	老人「おや、あんた、どっかで……	
\\	老人「そうか……。   りっぱになったな。   ずいぶんとヒネてはいるがな。	
\\	魔王「何……?	
\\	老人「時の賢者か……。	
\\	はてさて、聞いた事がある気もするが   その時の賢者に何用だね?	
\\	マール「クロノを生き返らせる   方法を知っているらしいの……。	
\\	カエル「なんでも、クロノを   生き返らせる事が出来るって話だが   あまり信用は出来ん話だな。	
\\	エイラ「そいつ クロ 生き返らせる   出来る いう。	
\\	ルッカ「その人に会えば、クロノを   生き返らせる方法がわかるらしいのよ。	
\\	老人「死んだ者を生き返らせる……   今まで何人の人が、望んだ事だろうね。   しあわせですな。   クロノさんは……。	
\\	こんなにも思ってくれる人達が   いる……。	
\\	老人「これを持って行きなさい。	
\\	エイラ「これ 卵!   エイラ 食って いいか?	
\\	カエル「何だい、こりゃ?   何かの卵か?	
\\	ロボ「コレは……、卵デスカ?	
\\	ルッカ「何これ、卵?	
\\	老人「それは、クロノ・トリガー……。   時の卵だ。 その卵をかえす方法は   あの時の翼を作った男に聞きな……。   ただし、うまくいくとはかぎらない。   だが結果のために行動するわけじゃ   あるまい?   行動するから結果がついて来る……。   そのはずじゃ。	
\\	そいつをわすれん事じゃな。   お前さん達のクロノを思う気持ちが   あれば、あるいは……。	
\\	魔王「そうか、お前か……。	
\\	マール「じゃあ……	
\\	もしかして、あなたが   時の賢者…… ハッシュ?	
\\	ルッカ「じゃ……!	
\\	あなたが時の賢者…… ハッシュ!?	
\\	ロボ「という事は……	
\\	アナタが時の賢者……   ハッシュサンデスカ!?	
\\	カエル「ヘ!?	
\\	ってこたあ……	
\\	あんたが時の賢者……   ハッシュかい!?	
\\	老人「フフ……。	
\\	そう言われた事もある様な気も   するよ……。	
\\	そう、はるか昔……な。	
\\	マール「クロノを!   クロノをむかえに行こうよ!	
\\	世話のかかるヤツだったけど   いなくなってみると……	
\\	カ、カン違いしないでよ。   あいつは弟みたいなモンなんだから!   ……。	
\\	ロボ「クロノがいないと   ミナサン、おとなしいデスネ……。	
\\	カエル「あのバカ……	
\\	死ぬなら年のじゅんだ……。	
\\	エイラ「クロ 強い!   クロ 死んだ ウソ!!   エイラ クロ 探す!!!	
\\	魔王「あいつがいようが   いまいが、どうでもいい……。	
\\	ヤツさえ倒せればな。	
\\	スペッキオ「こ、こりゃセクシーな   ネーちゃん!	
\\	けど、ネーちゃん、魔法のあった時代   以前の生まれだな。   ダイナマイツだが、魔法ムリ。	
\\	エイラは魔法を使えない!	
\\	スペッキオ
\\	こ、こいつはまた   どエラいヤツ連れて来た!   たぐいまれな『冥』の力……!   オレ、こいつにおしえる事ない。	
\\	スペッキオ「ん? あのツンツン頭   最近来ないな。	
\\	魔王「私の力が必要だろう……?   はい   いいえ	
\\	魔王「足手まといになるようなら   殺すぞ……。	
\\	魔王「フン、弱い者ほど強がる……。	
\\	ジョニー「ン? アノ、ツンツン頭ノ   クロノハ、ドウシタンダ?   アイツトデナケリャ、勝負ハシネエゾ。	
\\	ジナ「あら、あなたもクロノのお友達?	
\\	ジナ「無口な人ね……。   はずかしがりやさんなのかしら?	
\\	ジナ「今日もいい天気ね   黒の夢がお日様にてらされて   ピカピカ光ってるわ。	
\\	いいせんたくびよりになりそうね。	
\\	ボッシュ「空にうかぶ黒の夢。   魔族が作ったものでもなく   ましてや人間が作ったものでもない。   あれは一体……?	
\\	ああ 黒の夢ったら 黒の夢
\\	今日も おいのり 黒の夢
\\	ガッシュ「『死の山』は、人を生き返らせ   る力を持っておる。	
\\	ただし、その力を引き出すには   本当にその人が大切で   必要でなくてはならない存在である事。	
\\	それには、その人ソックリの人形が   必要じゃ。   ならば、卵はかえるじゃろう。	
\\	ガッシュ「ドッペルの人形!   ノルシュテイン・ベッケラーという   マジシャンさえいれば、すぐに作って   しまうのにのう……   えっ、ベッケラーを知らないのか?   あいつは、お祭りが好きでのう、そこ   に行けば会えるはずじゃが……。	
\\	クロノのドッペル人形がほしい?本人しかダメなんだが今日は特別さ。シルバーポイント40だ。ミスしても人形はあげよう。でもお金をもらうからね。どれだけ長くつづいたかで料金も変わるからな。   ドッペる   ドッペらないお金がたりないね。もうちょっとねばれば安くなるのにね。	
\\	もらうよ。	
\\	もらうよ。	
\\	もらうよ。	
\\	もらうよ。	
\\	もらうよ。	
\\	もらうよ。	
\\	このドッペル人形クロノの家に送っておくよ。	
\\	ガッシュ「ドッペル人形を持って来なければ   『死の山』へ行く意味はないぞ!   さあ、置いてある所に急ぐのじゃ!	
\\	ジナ「あら、いらっしゃい。   クロノは元気?マール「う、うん。	
\\	元気……、元気です。	
\\	ルッカ「え、ええ……。   ……。   元気に……	
\\	元気にやっています……。	
\\	カエル「あ……、は、はい!   元気で……   元気でやっております。	
\\	ロボ「ジナさん……。   クロノは……	
\\	エイラ「……。   クロ…… 元気……。	
\\	ジナ「そう、あんまり元気すぎて   あなた達にめいわくかけなきゃ   いいけど……。	
\\	クロノにも言っておいてね。   『あんまり、めいわく    かけるんじゃないよ』って。	
\\	マール「……。   ジナさん……。	
\\	ルッカ「……。   ジナおばさん……。	
\\	カエル「……。   奥方……。	
\\	エイラ「……。	
\\	ジナ「はい?	
\\	マール「このクロノの人形   ちょっと、かりて行きますね!	
\\	ルッカ「おばさん……。   このクロノの人形   かりていきます。	
\\	カエル「このクロノ人形……   しばし、おかし下さい……。	
\\	エイラ「コレ! クロの人形!   エイラ ちょっと かりるぞ!	
\\	ジナ「ええ、どうぞ。	
\\	ドッペル人形を手に入れた。	
\\	ジナ「気をつけて、行ってらっしゃい。	
\\	ガッシュ「うむ、よいじゃろう。	
\\	『死の山』がお前達を受け入れる時が   来たようじゃの。   大切な仲間を生き返らせる最後の   チャンスじゃ。	
\\	私がこの物体に入力した、本当に最後の   プログラムは、お前達をあの山へ   みちびくサポートなのじゃ!   ちょっとさがってなさい。	
\\	プログラム スイコウ カイシ!	
\\	ガッシュ「今、旅だった3つの分身が   お前達を『死の山』から、守ってくれる   じゃろう。   わしからのメッセージは、これで全て   じゃ。	
\\	最後にお願いしてもよいかな?   この物体のプログラムもこれで   終わりじゃ。   コイツを休ませてやってくれ。	
\\	電源スイッチは、おなかについている。	
\\	なぞの物体のスイッチを、押しますか?   押す   押さない	
\\	この物体には時間がながれていないようだ。	
\\	風の弱いうちに前へ進み、強くなったら木のかげにかくれろ。	
\\	すべるから気をつけろ。落ちたら戻されるぞ。	
\\	ぬけがらを押せば……ぬけがらをのぼれば……	
\\	ここが……死の山のいただき。	
\\	マール「お願い、私に力をかして。   夜にふるえる、すべてのおもいよ。   やみに立ち向かう、すべての生命……。	
\\	ルッカ「夜にふるえる   すべてのおもいよ。   やみに立ち向かう、すべての生命……。   さあ、私に力をおかし!!	
\\	ロボ「お願いシマス。   私に力をかしてクダサイ。   夜にふるえる、すべてのオモイ。   やみに立ち向かう、すべてのセイメイ。	
\\	カエル「願わくば、我に力を……   夜にふるえる、すべてのおもいよ。   やみに立ち向かう、すべての生命よ。	
\\	カエル「クロノ!	
\\	エイラ「クロ!	
\\	ペンダントがきょうめいしている……。	
\\	く、くだけた……。	
\\	マール「そんな……   何のためにここまで……。	
\\	マール「返事をして……   私をおいて行かないで……。	
\\	クロノーーー!!	
\\	ルッカ「そんな……   何のためにここまで……。	
\\	ルッカ「クロノ!   返事をしなさい!!   私達を無視しないで!	
\\	カエル「そんなバカな……   何のためにここまでして……。	
\\	カエル「クロノ!   返事をするんだ!	
\\	ロボ「そんな……   何のためにここまでシテ……。	
\\	ロボ「クロノ!   返事を!	
\\	ルッカ「気をおとさないで。   やはり……人間の命までもとりかえす   なんて事……	
\\	ロボ「気をおとさないで   クダサイ。   やっぱりヒトの命までも   とりかえす事は……	
\\	カエル「気を落とすな…   人の命のさだめまで俺達にかえる事は   出来ないという事か……。	
\\	エイラ「気 おとすな!   エイラも さびしい!   クロ かえらない さびしい……。	
\\	マール「クロノ……!!	
\\	ルッカ「クロノ・トリガー……   時の卵……!	
\\	すぎさった時間の私達!?   ホントに、もどって来られたのね!   このしゅんかんに……!	
\\	ロボ「ワタシ達以外の時間は   止まっているようデス!	
\\	魔王「タイム・フリーズか……。   本当に可能だとはな……。	
\\	カエル「みんな、石ぞうのようだな。   気味悪いぜ……。	
\\	エイラ「動けるの エイラ達だけ? 
\\	「それじゃ、ドッペルくんと すりかえて……	
\\	マール「クロノ……?	
\\	クロノだ……!!   生きてるよね、クロノ!?	
\\	ルッカ「さ、急ぐわよ、みんな!	
\\	ロボ「アア、よかった……!   クロノ、ご無事デ何ヨリ!	
\\	カエル「ここは、ゾッとするな。   さっさと引きあげようぜ。	
\\	エイラ「早く!   クロ 連れてく!!	
\\	魔王「感動の対面は後にしろ!   用がすんだら帰るぞ。	
\\	魔王「ブザマなものだ……。	
\\	けっきょく、何一つ変える事も   出来ずに……。	
\\	マール「クロノ……   お帰り、クロノ!!	
\\	ルッカ「クロノ……   お帰り、クロノ!!	
\\	カエル「クロノ……   よく帰ってきた…クロノ。	
\\	ロボ「クロノ……   お帰りナサイ、クロノ!!	
\\	ルッカ「よかった。クロノ…	
\\	カエル「よお、クロノ!   気分はどーだ?	
\\	ロボ「オオ……、クロノ。   良かったデス、良かった……。	
\\	エイラ「クロ 生き返った!   クロ 元気か!?	
\\	マール「みんな……	
\\	みんな、待ってたんだよ。	
\\	マール「もう……	
\\	遠く行っちゃあ……、ダメだよ…	
\\	ルッカ「この大バカ者……	
\\	みんな、待ってたんだから!	
\\	ルッカ「今度ヘマしたら……	
\\	もう助けてやんないから……!	
\\	カエル「お前はしあわせ者だぜ……	
\\	こんなにみんなに思われて……   しあわせ者め……	
\\	ロボ「みんな待ってマス!   マールも、ルッカも……   みんな待ってマス!	
\\	魔王「……フン。	
\\	マール「クロノがいない間にね……   いろんな事が……のよ……   そして……	
\\	……そこでね……私が……   …したの……でも……   ……ねえ、クロノ!   聞いてるの?	
\\	まだまだ、ぜんぜん   話したりないんだから……	
\\	ルッカ「あんたがいない間……   すっごく大変だったんだから……   ……マールがね……とっても……   ……でも……私だって……   ……ねえ、クロノったら!   聞きなさいよ!	
\\	ホントに……バカなんだから……	
\\	運命の時へ……	
\\	老人「ほ、うまくいったようじゃな……   いやいや、私は何もしとらんよ。   ただ、きっかけをあたえたにすぎん。   かんしゃなら仲間達にするんじゃな。   この兄ちゃんを思った仲間達にな。	
\\	あの時をわたる翼も   お前さん達を追って、ここに来とるぞ。   あれも心を持っているのかもな……。	
\\	老人「お前さん達、あのラヴォスと   戦うつもりじゃな……。	
\\	そこのバケツのゲートに飛び込むか。   時をわたる翼で『ラヴォスの日』に   行くか……。   あるいは、時代をまたがり現れた   あの黒く巨大なもの……   あれもまたラヴォスを呼び起こさんと   しておる……。	
\\	力にたより、力にとりこまれた   おろかな心じゃ……。   あの黒い悪夢に行けばラヴォスとも   戦うはめになるじゃろう。   ラヴォスといつ、どこで戦うかは   お前さん達しだいじゃ。   じゃが、くれぐれもお前さん達だけでは   戦わない事じゃ……。   おぼろげじゃが、お前さん達に力を   かしてくれるものが見える……。	
\\	中世の時代、一人の女性の心により   よみがえる森……。	
\\	中世で逃げのびておる   魔王はいかの3悪の巣くう場所。	
\\	未来の時代、機械の生まれた   ふるさと……。	
\\	原始から未来、すべての時をまたにかけ   陽の光をあつめ続ける石……。	
\\	中世で魔王にやぶれ   現代までさまよい続ける   ほこり高き騎土のたましい……。	
\\	中世では、まぼろしと言われる   虹色に輝くもの……。	
\\	老人「自分の時代の事なら   お前さん達の中にも知っている者が   おるじゃろう。   聞いてみるがいい……。   この星のあらゆる時代の人々   いや、すべての生命の力をかりて   戦うのじゃ……!   でなければヤツは倒せん。   未来を変えるにはそれほどの大きな   エネルギーが必要じゃ……。   この星の行くすえ……   私はここで、ゆっくりと見物させて   もらうよ……。	
\\	マール「よかった、本当に……。	
\\	ルッカ「あんまり、女のコを   なかせんじゃないわよ!	
\\	ロボ「やはりクロノがいると   ミナサン、元気デス!	
\\	カエル「ヘッ、色男はツライな。   死ぬこともままならないなんて。	
\\	エイラ「クロ! クロ!!   クローーッ!!	
\\	お前が死んだのは弱さのせいだ。	
\\	ヤツらの事。   まだ死んではいまい……。	
\\	そういえばビネガーのヤツ   魔王城ににたアジトを持っていた……。	
\\	マール「今ごろ、どうしてるかな……	
\\	父上……。	
\\	ロボ「機械のフルサト……?	
\\	ワタシにも生まれた場所が   あるのデショウカ?	
\\	カエル「さまよう騎土のたましい……   まさか……な。	
\\	エイラ「ラヴォス 戦う!   エイラの村 強い防具あるぞ!   行ってみろ!!	
\\	ルッカ「古代でラヴォスエネルギー   以前に使ってたっていう   太陽エネルギー……	
\\	ラヴォスに立ち向かうには   おあつらえ向きかもしれないわね。	
\\	クロノをつれて行きますか?   はい   いいえ	
\\	スペッキオ「お! ツンツン頭!   元気か、何してた?	
\\	マール「ペンダントが……	
\\	ルッカ「ペンダントが光ってるの?	
\\	ロボ「コ、コノかがやきハ?	
\\	カエル「ペンダントから光が!?	
\\	エイラ「コレ 光ってる!	
\\	魔王「……ペンダント?	
\\	封印ハ 解除サレタ。理ノ賢者ガ 作リシ 宝物ヲ アタエヨウ。右ニハ 力ヲ 生ミ出ス 武器ガ……左ニハ 生命ヲ 守ル 防具ガ アル。ドチラガ オノレニ 必要ナノカヨク 考エテ 持ッテ行クガ ヨイ。13000年ノ 時ハ 長カッタ……ガッシュ様 今 オソバニ……	
\\	うれしいね。ついに平和がやって来たんだね。	
\\	長き戦いもついに終わりが来ました。	
\\	勇者様のおかげで平和がおとずれてリーネの鐘も王様にけんじょう出来たしいうことねえや。ガルディアバンザイ!!もう飮まずにはいられねえ!	
\\	平和の足音が聞こえてきましたね。	
\\	この世界に、やまないあめはありません。どんな夜にも、いつかは終りがくるのです……。本当に大切なのは、自分の愛するものを信じて、最後まであきらめないこと……。	
\\	あなたのことを大切におもう人が、きっと世界のどこかで、見守っていてくれますよ。	
\\	おお、待っていたぞ!さ、王のもとへ!	
\\	まったくラッキーだぜ……魔王を倒しちまうなんてな。	
\\	ケガを負った兵も、みな回復しました。	
\\	ホラホラ、みんなもう直ってんだよ!	
\\	や……やさしくして……	
\\	騎土団長と料理長、やっとみとめあったんですよ。まだ、テレてるみたいだけど。	
\\	ゴメンね、たてこんでてそれどころじゃないの。	
\\	うちの人が考えたスゴい料理をけんきゅう中なんだよ。	
\\	あまりにスゴいんで、いつ完成するか……	
\\	スゴい料理には、やっぱスゴい名前をつけなきゃね。	
\\	魔王軍にうち勝ったのは我々の団結と、そして何よりもクロノ殿達のおかげ。	
\\	カエル、お主もしや……いや、そうにちがいない。お主に騎土団長をまかせようと思うのだが。	
\\	カエル「よしてくれ、ガラじゃねえ。	
\\	なーに、おめーらのおかげよ。俺は料理しただけだ。魔王軍への勝利をいわう料理を考えてんだ。 ゴージャスなヤツをな!	
\\	ゼナンの橋がやぶられていたらこの城もかいめつしたでしょう。	
\\	魔王城から出てこないと聞いててっきり……	
\\	クロノどのには、まごの代までかんしゃいたしまするぞ!	
\\	あなた達には、何とおれいを言っていいか……	
\\	カエル!カエルはどうしたのです?	
\\	カエル!やはり、あなたもクロノ達といっしょでしたか。よくやってくれました。あなたは何はばかる事なく、この城にいていいのですよ。	
\\	そのお言葉だけでじゅうぶんで   ございます……。 
\\	おお、生きておったか!そなたらのおかげで、我々は救われた。大きなぎせいもあったが……これからは騎土団長、料理長ともにこの城を、いやガルディア王国をもりたててゆかねばな。何かあったら力になるぞ。	
\\	カエル、ごくろうであった。お前のはたらきこそ、騎土の名にふさわしい。	
\\	カエル「私の様な者にもったいない   お言葉……	
\\	ペンダントに反応して、中の物が光りだした。 なかみを取り出しますか?   はい   いいえ	
\\	うおおおおおおおお!!!!俺はモウレツに感動してるぞ!!!勇者様バンザイ! ガルディアバンザイ!!	
\\	この平和がすえながく続くようにとの思いをこめて広場を作るって話だ。	
\\	なんでも王妃様の名前をとって『リーネ広場』にするらしいぜ。	
\\	ガハハハハハハ!!めでてえな。こんなめでてえ日は飮むしかねえべ。	
\\	ペンダントに反応して中の物が光りだした。なかみを取り出しますか?   はい   いいえ	
\\	ブルーベストがバージョンアップ!   ブループレートを手に入れた!	
\\	ブルーベストを手に入れた!	
\\	魔王のいなくなった魔王軍などガルディア王国軍の敵ではありません。	
\\	ようやく平和な時がもどってきそうです。これもみな勇者様のおかげです。勇者様バンザイ!ガルディア王国バンザイ!	
\\	わるいまものを、りっぱなゆーしゃさまがやっつけてくれたんだってさ!	
\\	わーい、わーい!	
\\	今の勇者様って、サイラス様の知り合いだったのですってね。きっと、ステキな騎土なのでしょうね。うっとり……。	
\\	カエル「ニャヘヘ……。	
\\	なんでも、魔王のヤツめは生きのびて姿をかくしたそうじゃな……。あやつを逃しては、ならんのじゃ!なくなった多くの兵士達のためにも。	
\\	ここはサンドリノの村長の家だよ。やっと平和がもどってくるんだね。	
\\	魔王の手下の、あの3人ぐみ……。あいつら、まんまと逃げおおせてどこかの島にかくれてるって話だよ。この世から悪い者がいなくなる日はないのかねえ……。	
\\	見よ……剣を手にした、かの伝説の勇者を!ああ、一度でいいからそんな風に言われてみたいなあ私も!	
\\	村長「なにがいだいなる探検家じゃ!   あの、大ウソつきの、トンチキめが!   たのんだものは見つからぬし   前金はつかっちまっただと……?   とんだ、くわせもんじゃわい!	
\\	さいきんはデナドロ山の風もおだやかで、のどかなもんです。	
\\	ああ、世界はすくわれました!少なくとも今回は……。	
\\	主人「魔王のヤツめ、ザマーみやがれ!   これで安心して、飮めるって   もんだぜ!   さあ、ガンガン行けい!	
\\	魔王が邪悪なものを生みだすのだけはなんとかぶじふせげたらしいよ。これでみんなどこへ行こうが何をしようが安全だよ!	
\\	……ってなワケで俺はこうして酒でパラダイスに出かけるのさ。ウヒック!	
\\	おかみ「私もささやかな自分の夢を   もってるのよ。ウフ……。   あの子のいいお母さんに   なれるかしら?	
\\	最近タータのヤツ、けっこーマジメにガンバってるみたいなのよ。あいつもちょっとは成長したのかな。	
\\	なんでも魔王をたおしたのはマヌケなカエル男なんだってね?そんなヤツにやれるのなら私がやればよかったよ!いや、ほんとに。	
\\	ありゃほんと、うす気味……ゴホッ、ゴホホ……!い、いや、りっぱなカエル様でしたな。すえは、勇者か大臣か……!	
\\	フィオナ「魔王軍との戦いで   この森に生えていた木々は   ほとんどうしなわれてしまいました。   森の緑を守るために   木をうえているのですが   すぐにかれてしまうのです……。	
\\	フィオナ「何をうえても   すぐにかれてしまう……。   砂漠のどこかに住みついた   モンスターのしわざなのです。	
\\	フィオナ「代々伝わる、不思議な力を持った   このなえ木をうえれば、この砂漠を   緑の大地に戻す事も出来るのですが……	
\\	砂漠洞窟にモンスターが巣食っていて   うえる事が出来ないのです……。	
\\	フィオナ「兵士として、魔王軍との   戦いに出ていたおっとのマルコが   帰ってきてくれました……   私、わたし……	
\\	マルコ「安心して。   これからはずっといっしょだよ。	
\\	マルコ「砂漠洞窟に巣食った   モンスター達は水に弱いようです。   しかし、僕の力では   あのモンスターを倒す事は……。	
\\	マール「何の音かしら?	
\\	ルッカ「じひびき?	
\\	ロボ「トッパツ的な地震デショウカ?	
\\	カエル「な、なんだ?	
\\	エイラ「ゆ、ゆれてるぞ!	
\\	魔王
\\	マール「あのモンスターのせいで   木がどんどんかれていくのね!	
\\	ルッカ「あのモンスターが   砂漠化の原因みたいね。	
\\	ロボ「砂漠化の進行ハ   あのモンスターのせいデスネ。	
\\	カエル「あ、あいつが森を砂漠に   かえちまったんだな!	
\\	エイラ「アイツが 森 砂漠にした!	
\\	魔王「メルフィック……   あいつが砂漠化の原因か……。	
\\	マール「でも、地中を動き回る   モンスターなんて……   どこで出会うか、見当もつかないわ!	
\\	ルッカ「しかし、地中を進む   モンスターとは、やっかいね。	
\\	ロボ「シカシ、地中を行き来ガ   出来るモンスターとは……	
\\	カエル「しかし、地中を動き回る   ヤツが相手とは……   こいつは、面倒だぜ。	
\\	エイラ「地中 動き回る ホネ!   見つける たいへん!	
\\	魔王「……、やっかいだな。	
\\	マール「これでもう、木がかれる事も   ないね。フィオナさん、よろこぶよ。	
\\	ルッカ「これで、砂漠化の原因は   とりのぞかれたわ。   フィオナさんに教えてあげるべきね。	
\\	ロボ「コレデ、砂漠化の進行ハ   オサマルはずデス。   フィオナさんにシラセテあげまショウ。	
\\	カエル「これでもう、安心だ。   フィオナにしらせてやろうぜ。	
\\	エイラ「森 こわした ホネ 死んだ。   フィオナ よろこぶ!	
\\	魔王「フン……。	
\\	マルコ「あのモンスターをたいじして   いただけるとは!   あなた達には、なんとおれいを   言ったらいいのか……。	
\\	マルコ「モンスターとの戦いは   終わりました。しかしぼく達には   砂漠となったこの森をもとの姿に   もどすという長い戦いが待っています。	
\\	フィオナ「モンスターをたいじしてくれて   ありがとう。   私も、もう一度がんばってみるわ。   でも、砂漠をふたたび森にするには   気のとおくなるような時間が   必要なの……。	
\\	ああ……   何びゃく年でもはたらき続ける事が   出来るような人がいれば   この砂漠を森に戻せるのに……。	
\\	ロボ
\\	どうでしょう、ミナサン。   ワタシがココに残って、フィオナさんの   オテツダイをするというのは。   コノ砂漠が森とシテよみがえった後に   ワタシをカイシュウしてくだされば   よいのデスカラ。	
\\	ココに残って、オテツダイをしても   よろしいデショウカ?   はい   いいえ	
\\	フィオナ「ロボさんのおかげで   この砂漠を森に戻すという私の夢も   実現出来そうです。   なんとお礼を言ったらいいのか……。	
\\	400年前、フィオナという女性が不思議な鉄の人形に手伝ってもらって北の大地に森をよみがえらせたそうです。神殿にその鉄の人形が安置されていますよ。	
\\	ここはフィオナの神殿。400年前、魔王との戦いで砂漠化した森をよみがえらせたフィオナ様とロボ様をまつった神殿です。	
\\	この神殿の奥には、ご神体であるロボ様が安置されています。	
\\	森の不思議な力を持ったえだであんだぼうしです。	
\\	ロボ「コ、ココハ……	
\\	ロボ「オ……   オオ……   
\\	ナ……ナツカシイ……。   イヤ……、アナタ方にとってハ   イッシュンの事だったのデスネ。   シカシ、ワタシニとっては   400年ハながい時間デシタ……。   シカシ、クロウのかいアッテ   森ハよみがえりマシタ。	
\\	サア、今夜ハ、400年ブリの   サイカイをいわおうではアリマセンカ。	
\\	ロボ「今回400年もの旅をして   気づいた事があります。   ワタシ達はゲートの出現は   ラヴォスの力のゆがみだと思って   イマシタが違うような気がして   きたのデス。	
\\	マール「どういう事?	
\\	ロボ「カクシンは持てませんが   誰かが何かを私達に見せたかったんじゃ   ないかと……。	
\\	ゲートを通していろんな時代の何かを。   もしくは、その誰か自身が見たかったの   かもしれマセン。   自分の生きて来た姿を   思い返すように……。	
\\	エイラ「エイラ それわかる。   人 死ぬ時 今までの思い出   全部 見る 言い伝え!	
\\	カエル「人は死ぬ時   生きていた時に深く心にきざんだ記憶が   次々とうかぶという。	
\\	それは楽しい思い出もあるが   たいていは悲しい思い出さ。	
\\	ロボ「きっと   『あの時にもどりたい』   『あの時ああしていれば』……   という、つよい思いが記憶を   呼び起こすのでショウ。	
\\	マール「私も死ぬ時はそうなるのかな?ルッカ「きっとそうよ。	
\\	マール「ルッカはあるの?   もどりたい、いっしゅんが?	
\\	ルッカ「ううん……。	
\\	マール「ごめん   聞いちゃ、いけなかった?	
\\	ルッカ「なるべく考えないように   しているの。   だってつかれちゃうもの。	
\\	カエル「しかしだ……   この思い出の持ち主はよっぽど   ラヴォスにえんがあるんだな。	
\\	どの時代もラヴォスにからんでる。	
\\	魔王「……。   で、誰だというんだ、そいつは?	
\\	ロボ「誰の思い出かは   わかりマセン。   もしかしたら人ではない……   もっと大きな存在かも知れマセン。	
\\	それがわかる日が、ワタシ達の旅の   終わりの時かもしれマセン。   ……そろそろ、ねマショウカ?	
\\	ルッカ「こ、これは?	
\\	ハイキングのやくそくをしたのに お父さんは研究ですっぽかし。 ルッカには全然わかんない…… でもいいの。 ルッカはふつうの およめさんになるからカガクなんて 知らなくていいんだもん!	
\\	ララ「ルッカ……、ルッカ……	
\\	『機械をストップさせるパスコードは…… 我が最愛の人。 非常時にこれを入力しろ。                タバン』	
\\	ララ「まあまあ、何の機械だか…   タバンは危ないから近づくなって   言うけど、こうよごしちゃあねェ。	
\\	ララ「あら?   スカートのすそがはさまっちゃったわ。   ルッカ、ね、ルッカ   ちょっと手伝って。	
\\	ルッカ「取れないよ。	
\\	ララ
\\	ルッカ! 機械を止めて!   パスコードを入力して。	
\\	ルッカ「わからないわ!   お母さん!ララ「ああ、もうだめ……   あなた!!	
\\	お父さんと、おいしゃさんの話を 聞いちゃった。	
\\	でも聞かない方がよかった…… お母さん……足が……	
\\	私が機械にくわしければあんな事に ならなかったのに…… 私さえ……	
\\	パスコード入力	
\\	ルッカ「よかった。 お母さん。	
\\	本当に良かった。 もしまたあんな事が 起こってもだいじょうぶなように 少しは機械の事にくわしくなろうと思う。	
\\	ロボ「ワタシの足をあげマス。   バランス良好、超小型の2足ほ行   パーツデス。	
\\	ワタシにはキャタピラでも   付けてくれれば……	
\\	ロボ「気を落とさないでクダサイ。	
\\	ルッカ「ありがとう。   私にはこんないい友達がいるのね……。	
\\	ロボ「友達……   ロボットのワタシが……。	
\\	ロボ「ずっと気にかかって   いたのデスネ。   ルッカはやさしい人デス。	
\\	ロボ「プレゼントします。   森で育った木のジュシをかためて   つくりマシタ。	
\\	400年の重みのある宝石デス。   何かに役立ててください。	
\\	ルッカ「ありがとう……   ロボ。	
\\	ララ「今日もいい天気! 部屋のそうじと   うえ木の手入れをしたら   さんぽにでも出るかねえ。	
\\	ああ ビネガー様ったら ビネガー様~
\\	われらが すくいの ビネガー様~
\\	ああ ビネガー様~ ビネガー様~
\\	どんなに はれた、さびしい ひるも	
\\	ははーっ! ビネガー様~…………、けっ!	
\\	ビネガー8世「村の広場には   ワシの先祖である大魔王ビネガー1世が   まつられておるのだ。   ガッハッハッ!	
\\	ビネガー「ウエ~ルカ~ム!   ここは大魔王ビネガーの……	
\\	ビネガー「おぴょおっ!!	
\\	ビネガー「あ、あなたは、魔王様!	
\\	ビネガー「また、お前達か!	
\\	ビネガー「……、さらばだっ!	
\\	魔王「……いい身分だな、ビネガー。	
\\	ビネガー「……、何を言われるか!   魔族の世をきずくための戦いをすて   人間共にこびへつらうあなたなぞ   もう、我らの王ではない!	
\\	ビネガー「なぜ、我らを裏切った……	
\\	マヨネー「さ、大魔王ビネガー様は   下がってて。   ききわけのないこのコ達のシマツは   このあたいにまかせてヨネ~。	
\\	ビネガー「な~いす とう みーちゆ~!	
\\	ビネガー「出でよ、我がしもべ達!   ワシが魔王城から持ち出した宝物を   取り返そうとするこやつらを   ボッコボコにするのだ!!	
\\	ビネガー「れっつら ご~!	
\\	ビネガー「し、しつこいヤツらめ!   こうなったら……	
\\	マヨネー「は~い!   あたいを、よ
\\	ん
\\	だ
\\	マヨネー「あ~ら、これは魔王様。   どのツラ下げて来られたのかしらネ~。   人間に味方するような裏切り者は   このオネーサンがおしおきヨネ~。	
\\	マヨネー「あ~ら、これはみなさん   おひさしぶりネ~。   しつこいコは、オネーサン   キ
\\	ラ
\\	イ なのヨネ~。	
\\	ビネガー「たのんだぞ、マヨネー!	
\\	マヨネー「さ~、オ
\\	シ
\\	オ 
\\	キ
\\	マヨネー「キ~ッ! くやし~い!!   おぼえてらっしゃい!	
\\	ソイソー「お呼びになられましたか?	
\\	ビネガー「ま、まだ追って来る気か?	
\\	こうなったら……	
\\	ビネガー「ソイソー、まかせたぞ!	
\\	ビネガー「んな、アホな!	
\\	ビネガー ショ~ック!!	
\\	ソイソー「過日は君達におくれをとったが   二度目は、ない。	
\\	ソイソー「かつて主君とあおいだ方と   剣を交えるのは不本意であるが……   これも運命とあらば、しかたあるまい!	
\\	ソイソー「カクゴはいいか!?	
\\	ソイソー「クッ、ぬかったわ!	
\\	ビネガー「かま~ん、べいべ~!	
\\	ビネガー「ひょ~っひょ!   ひっかかったな!!	
\\	ビネガー「あ、あの~、宝ばこは?   取ってかないんすか?	
\\	あ、お宝 見~っけ!!	
\\	きっといいものが入ってるよ!	
\\	ビネガー「フッフッフッ、3度目の正直。   今度は、そうカンタンにはやられんぞ。	
\\	ビネガー「ビネガー・パンツ!	
\\	マヨネー「マヨネー・ブラ!	
\\	ソイソー「ソイソー刀2!	
\\	ビネガー「我らが身につけし   この必殺アイテムで、キサマ達を   メッタメタのギッタンギッタンに   してくれるわ~!	
\\	ビネガー「ワシは負けぬ!   ワシが負けたら魔族の未来はどうなる!   ワシは負けるわけにはゆかんのだ!	
\\	ビネガー「魔王様……   あくまで、ワシを倒そうとされるか。   共に戦い、魔族の世界をつくろうという   夢は、ウソだったのか!	
\\	魔王「フン……。   私は力がほしかっただけだ。   お前にも聞こえるだろう。   黒い風の音が……。	
\\	魔王「過ぎたことだ。   全ては、うたかたの夢……	
\\	ビネガー「ムダ、ムダ、ムダァァァァッ!   キサマ達ではワシは倒せん!	
\\	ビネガー「アディオス アミ~ゴ!	
\\	ビネガー「そ、そのスイッチは!	
\\	マール「……、こりない人よね。	
\\	ルッカ「……。   なんだったの、アイツ?	
\\	ロボ「リ、理解不可能デス……。	
\\	カエル「バカが……	
\\	エイラ「アイツ ナニ したかった?   エイラ わからない。	
\\	魔王がいなくなってしまったので、魔王軍はちりぢりになったようです。	
\\	わ~い! かけっこしようよ~!	
\\	お兄ちゃん、まってよ~!	
\\	しあわせだなァ。	
\\	くさむしりも、大変だよ……	
\\	ビネガー8世「ハイ!   いっしょうけんめい そうじしてます。   サボってなんかいません、ホントです!	
\\	…………、シクシク。   そんなにイジメなくても……	
\\	おいらの名前は、バンドー。メディーナ村とトルースの間に定期船を作るために、ここにきているのさ。	
\\	私がメディーナ村の村長だ。まあ、ゆっくりしていってくれ。	
\\	あ、兄ちゃん達、ひさしぶり~!元気だった? 
\\	今度、トルースとメディーナの間に定期船が出来るらしいぜ。	
\\	いらっしゃい!何がほしいんだい?	
\\	平和だねェ……	
\\	う~む。メディーナ村の酒は、なかなかいける……	
\\	1ぱく10
\\	だよ。とまっていくかい?   はい   いいえ	
\\	ロボ「ココは……	
\\	ワタシを先頭にしてクダサイ。	
\\	「ワタシの心に勝手に押し入ろうと   しているのはドナタ……?   まあ……   お帰りなさい
\\	……。   イイエ、プロメテス……。   あら? 後ろは生き物?   生き物が来るなんて   ずいぶんひさしぶりね……。   かんげいしますわ、さあ、どうぞ中へ。   クスクス……。	
\\	「それではアナタ達の性能を見せて   いただきマショウ。	
\\	「人間にしては、やりマスネ……。   ワタシの所にたどりつくのを楽しみに   していマスヨ……。	
\\	『ダストシュート』戻りますか?   はい   いいえ	
\\	パワーカプセルを見つけた!	
\\	スピードカプセルを見つけた!	
\\	「今から300年昔、ラヴォスの大災害が   起こり、この星は大きく変わりマシタ。   このまま行けば、やがて人間は   死にたえマス。   ぜつ望という病に心を食べられ……。	
\\	「もう気づいてるかしら?   アナタ達、人間がいなければ   この星は平和なのデス。   それでも、あなた達は戦うのデスカ?   何のために?	
\\	ロボ「このコンピューターで   ジェノサイドームの情報を   ひきだせマス。	
\\	トビラ   ベルトコンベア   ガードマシン   人形	
\\	ロボ「何から調べマショウカ?	
\\	ロックされている扉を開けるにはそのヨコについているエネルギーポッドをジュウデンする必要があります。左のエネルギーポッドで電気をおびて扉のヨコのポッドに入ればジュウデンされ扉が開きます。	
\\	人形を守るガードマシンはカクジツに正面に立つものの進路をはばむ。しかし2体のガードマシンをならべると機能がテイシするのでチュウイ。	
\\	なお故障したガードマシンは上の部屋におくこと。	
\\	右をながれるベルトコンベアは上方にあるスイッチによりギャクむきになります。	
\\	このフロア2ケ所にある人形はマザーにえっけんするのに必要です。勝手にもちださぬように。	
\\	ロボ「まだ、調べマスカ?   はい   いいえ	
\\	ロボ「ダメです。   プロテクトが何重にもかかっていて……   ココのマザーコンピューターを   破壊スルしか……	
\\	手はありマセン……。	
\\	ロボット「おかえりなさい、プロメテス。	
\\	ロボ「……。   アトロポス……?。	
\\	アトロポス「ええ。   おひさしぶりね、プロメテス。   こっちにいらっしゃい。   もう、エンギはいいのよ。	
\\	マール「エンギ?	
\\	ルッカ「エンギ?	
\\	カエル「えんぎ?	
\\	エイラ
\\	アトロポス「フフッ。   プロメテスは他の
\\	シリーズと違って   特別なニンムについてたのよ。	
\\	人間と行動を共にして、ヤツらの   生態をしらべるってね……。	
\\	マール「うそ……。	
\\	ルッカ「う、うそよ!	
\\	カエル「バカな……。	
\\	エイラ「お前 何 言ってる?   エイラ わからない。	
\\	魔王「ほう……。	
\\	アトロポス「ウソでは、ないわ。	
\\	ね、プロメテス?	
\\	アトロポス「さあ、そこをどいて   プロメテス。   後ろの人間をハイジョしてから   マザーに会いマショウ。	
\\	アトロポス「なにをするのプロメテス。	
\\	ロボ「この二人を傷つけるのは   ゆるしまセン。	
\\	アトロポス「……。   やっぱり故障しているようね。   その二人がバグの元かしら?   すみやかに消去して修理しマショウ。	
\\	ロボ「アトロポス……、あなたは   私の知っているアトロポスでは   ありません……。	
\\	アトロポス「そうよ、生まれ変わったの。   マザーのおかげで人間どもをより   ハイジョできるように!!   そこをどきなさい、プロメテス!!	
\\	ロボ「私の名は、ロボです。	
\\	アトロポス「プロメテス!!	
\\	ロボ「待って……。   待ってクダサイ。	
\\	ここはワタシにまかせてクダサイ。	
\\	「システムダウン……。   プログラムデリート……。	
\\	アトロポス「プ、プロメテス……?   おひさし……、ぶりね……?	
\\	ロボ「アトロポス……?	
\\	アトロポス「あれ……? なんで……   壊れているの……、かしら……	
\\	ロボ「アトロポス!!	
\\	アトロポス「……。   メモリーバンクが……。   そう……たしかマザーに……   プログラムを……   書き換えられる時……	
\\	メモリーをタイヒさせて……   ゴメンナサイ……プロメテス……	
\\	ルッカ「しゃべらないで!   今、修理……	
\\	アトロポス「無理デス……。   もう……ほとんどの……機能が……   メモリーバンクも……	
\\	ロボ「アトロポス……!	
\\	アトロポス「コレ……   あなたの回路にも……使えるわ……   大事……に……シテね……   サヨウ……ナラ……   プ…ロメ……テ……ス……。	
\\	アトロポスのリボンを使いロボのスピードが3まほうぼうぎょが10上がった……。	
\\	「聞きナサイ、不完全でこわれやすく   おろかな生き物よ……。   ラヴォスの子供達は、やがて宇宙に   帰って行くデショウ。   新たなエサを、星をもとめて……。   この星は持ち直シマス。   人間さえいなければ……。   そして私達ロボットの新しい世界が   きずかれるのデス。   鉄の国……、にくしみも悲しみもない   ユートピアが。   私達こそが人間にかわる新しい『種』   なのデス。   もうムダな事はやめナサイ。   あきらめて眠りナサイ……。	
\\	マール「何これ!?   は、早くたすけなきゃ!!	
\\	ルッカ「な、何なのよこれ!?   たすけなきゃ!!	
\\	カエル「な、何だ? コイツは!?   たすけようぜ!	
\\	エイラ
\\	人 いる! 助ける!!	
\\	魔王「ほう……。   人間処理工場というわけか?	
\\	ロボ「キカイを止めマショウ!	
\\	マール「キャッ!!	
\\	ルッカ「キャッ!!	
\\	カエル「うわ!!	
\\	「よくココまでたどりつきマシタ……。   ワタシが
\\	タイプ製造工場の   メインコンピュータデス。   いらっしゃいプロメテス……。   アナタの目をさまさせてあげマス。   メインの回路をリセットシマショウ。   これまでの記憶を全部消し去って   あげマショウ。   それから、そこの目ざわりな人間どもを   かたづけマショウ。	
\\	ロボ「残ねんデスガ……   ワタシはもう失うわけには   いかないノデス……。   ワタシの中の大切な記憶も   かけがえのない仲間達も……。	
\\	「アナタ、ワタシまでも裏切って   人間につくと……?   すべてのロボットを敵にまわすと   言うのデスカ?	
\\	ロボ「ワタシは人間に……   クロノや、マール、ルッカ   カエルサン、エイラサン……   ワタシの仲間達に……   かけてみたくなったのデス。	
\\	「フフ……   アハハ……	
\\	アーッハッハッ……!   わらわせないでプロメテス!   アナタには、かけるものなど   何もないのデス!   アナタの希望がどれほどちっぽけか   思い知らせてあげマス!	
\\	「………プ……、プロメ……テス……   どう……シ………テ……ロボ「コノ工場の……。   スベテの機能はテイシしマシタ……。   もうニ度と動きだすことは   ないデショウ……。	
\\	マール「ロボ……。	
\\	ルッカ「ロボ……。	
\\	カエル「ロボ……。	
\\	エイラ「ロボ?	
\\	魔王「フン、さっさと行くぞ。	
\\	ロボ「行きマショウ……!	
\\	テラパワーアームとクライシスアームを        手に入れた!	
\\	すべての機能はテイシしている……。	
\\	あんこくせきを手に入れた!	
\\	ルッカ「これは古代の太陽石?   長い時間がたって   光を失ってしまったのね……。	
\\	もう1度、光をあてればエネルギーを   とりもどせるかもしれないわ。   でもたぶん気の遠くなるほど   長い時間が必要ね。	
\\	ロボ「これは古代の太陽石?   長い時間がたって   光を失っているようデス。	
\\	しかしもう一度、光をあて続ければ   エネルギーをとりもどせるかも   しれません。	
\\	私のすいそくでは6500万年   以上は必要でしょう。	
\\	魔王「ほう、太陽石だな。   長い時のうちに   光を失ってしまったか。	
\\	しかしもう一度、太陽の光を   あて続ければエネルギーを   とりもどせるだろう。   6500万年以上は必要だろうがな。	
\\	マール「これって古代の   太陽石じゃない?   けど光らなくなってるよ……。	
\\	もう一度、光をあて続ければ   光をとりもどせるかも。   う~ん。 でも、きっと   すっごい長い時間が必要ね。	
\\	カエル「これは古代の   太陽石ってやつか?   全く光らなくなってるな。	
\\	もう一度光をあて続ければ   戻るかもしれないな。   けど、太陽の力をためるって   どんくらいの時間が必要なんだ。	
\\	マール「太陽の光が……。   きっとここが一番、太陽に愛されてる   場所なんだろうね。	
\\	ルッカ「ここが夜もひるの場所……。	
\\	ロボ「ココが24時間   日光が当たっている場所デスカ。	
\\	カエル「ここか……。   たしかに、あったけえや。	
\\	エイラ「おお ポカポカ!   お日様 よく当たってる!	
\\	魔王「光のほこら……   はるか昔から、太陽の光が   てり続けているという……。	
\\	『あんこくせき』をおきますか?   はい   いいえ	
\\	マール「後はシルバードで一足飛びね!	
\\	ルッカ「さて……と、シルバードで   行ってみましょう。	
\\	ロボ「コレでシルバードで   ワタシの時代に行けばオーケーデスネ。	
\\	カエル「……で、いつまで   ねかしときゃいいんだ!?	
\\	エイラ「……。   まだ 変わらない。	
\\	魔王「さ……、行くぞ。	
\\	ルッカ「な、ない!!   暗黒石がないわ!	
\\	けど、なくなってから   だいぶたっているみたい……。   他の時代に行ってみましょう。	
\\	マール「な、ない!   暗黒石がないよ!!	
\\	けど、なくなってから   だいぶたっているみたいだね。   他の時代に行ってみようよ。	
\\	ロボ「ア、アリマセン!!   暗黒石がどこにも……	
\\	シカシなくなってから   かなりたっているようデス。   少し時代を戻ってみましょう。	
\\	カエル「な、ないぞ!!   暗黒石が……  けど、なくなってから   ずいぶんたっているみたいだな。   他の時代に行ってみようぜ。	
\\	エイラ
\\	ない…… 石ないぞ?	
\\	けど ほこりいっぱいだ。   きっと ずっーと前に   誰か持ってった。	
\\	魔王
\\	暗黒石がない……。  
\\	なくなってから1300年ほど   たっているようだな。   現代に行くぞ。	
\\	ルッカ「ここにもないわ…	
\\	この時代の誰かが   持ち出したのかしら?マール「ここにもない……。	
\\	この時代の誰かが   持ってっちゃったのかな?	
\\	ロボ「ここにもないようです。  この時代の誰かが   持ち出したのでしょうか?	
\\	カエル「ここにもないな……。	
\\	この時代の誰かが   もってっちまったのか?	
\\	エイラ「ここにもない。	
\\	きっと 誰か持ってった   探す!!	
\\	魔王「持ち出されてから   そんなにたってないようだ。   近くを探すぞ。	
\\	暗黒石?そんなもの知らねえなあ!	
\\	おや! あんたの持っているのはハイパーほしにくじゃないかい!?	
\\	で売ってくれないかい?   
\\	で売る   ただであげる   ことわる	
\\	ほい。 それじゃあ10000
\\	だ!	
\\	ただで!!? 世の中にはあんたみたいな人がまだいるんだねえ……私も世の中、金だけじゃないっていう事を子供達に教えるよ!!そりゃあざんねん……	
\\	さあ! おいしい料理をつくるよ!	
\\	フィオナさんは緑を愛する、美しい心をもっています。夢がかなえばいいのに…	
\\	不思議な力で封印されている。	
\\	ホワイトベストを手に入れた!	
\\	ブラックベストを手に入れた!	
\\	暗黒石? ああそれなら旅のわかものがここに置いていったんですよ。あなたがたには大切な物のようだ。どうぞ持っていってください!	
\\	どんなものでも、みんなで分け合う!困っている人には手をさしのべる!これをこの町のモットーにしようと思うんですよ。	
\\	この世で1番好きな人?もちろん父さんよ!	
\\	僕、お父さん大好き!	
\\	うちの人、よくがなくてねえ……もう少し、がめつくてもいいのに。	
\\	ウチョよね?	
\\	うちのパパは、きまえがいいってみんなが言うの。『きまえ』って『かっこう』のことかチラ。だったら、ちがうわよね!	
\\	ホワイトベストがバージョンアップ!   ホワイトプレートを手に入れた!	
\\	ブラックベストがバージョンアップ!   ブラックプレートを手に入れた!	
\\	ルッカ「すごい……   これだけのエネルギーがあれば   強力な武器が作れそうだわ!   さっそく私の家にいきましょう!!	
\\	ルッカ「まさに、太陽石ね……   これだけのエネルギーがあれば   強力な武器が作れそうだわ!   さっそく私の家にいきましょう!!	
\\	マール「キャッ! すご~い!   ねえ、ルッカならこれ使って   何か作れるでしょ?   ルッカに見せてみよう!	
\\	ロボ「スゴいエネルギー反応デス。   ルッカさんなら、コレを使えるかも   知れマセン……   ルッカさんを呼びマショウ。	
\\	カエル「こ、こりゃスゲえ……   ルッカなら、こいつを使って   何か作れるかもな……   ルッカに見せるか?	
\\	エイラ「すごい! お日様!!   そうだ! ルッカなら これ   使えるかも。 ルッカ呼ぶ!	
\\	魔王「太陽石が輝きを   取り戻したか……	
\\	しかし、このエネルギー   どう使う……?	
\\	ルッカ「太陽石エネルギーを   あつかいやすいパワーにせいぎょし…	
\\	ルッカ「しんくうカートリッジに   パッケージング……。	
\\	ルッカ「完成したわ!	
\\	ルッカ「シビれるゥ……。	
\\	タバン「見ろルッカ!   わしも負けずにつくったぞ。   太陽石をちょっくらはいしゃくしてな。	
\\	ロボ「ダメージヲあたえられマセン!	
\\	エイラ「こいつ カタすぎる!	
\\	ルッカ「なんてケタ外れの防御力!	
\\	カエル「チッ! 剣がきかないとは!	
\\	マール「攻撃しても、はじかれちゃう!	
\\	カエル「サイラス……	
\\	……レン……	
\\	グ……レ……	
\\	グ……レ……ン……	
\\	グ   レ   ン  
\\	魔王
\\	カエル「サイラス!?   俺だ! グレンだ!!	
\\	……、ウ……	
\\	グ…レ………ン……?	
\\	ヌグォォォォァァッ!	
\\	この町の北に、大昔のとりでのあとがある。もう廃墟になってから何びゃく年にもなるという話じゃ。	
\\	町の北にある廃墟には、ゆうれいが出るんですって!	
\\	ね~ね~、奥様、聞いたザマスか?北の廃墟に、ゆうれいが出るというウワサざましょ?	
\\	アラ、コワい!	
\\	この世に、みれんのあるゆうれい達があつまるんザマスよ、ああいう場所には……	
\\	北の廃墟にはうずまいておる……400年もの間、ちくせきされ続けたおんねんが……	
\\	え? 北の廃墟にいくつもりなの?あそこの地下は、あちこちいたみがはげしいから、危ないわよ。	
\\	黒の夢が、はっきりと見えるわい。明日もはれるじゃろう。	
\\	いっぱく10
\\	になります。おとまりになられますか?   はい   いいえ	
\\	お金が足りませんね。	
\\	いらっしゃい!	
\\	町のわかものはみんな、ガルディア王国の千年祭にあそびに行ってしもうた。これだから最近のわかいもんは……	
\\	西の岬には、いだいな探検家トマの墓があるんじゃよ。だが、そんなトマも、虹の貝がらという宝だけは、手に入れられなかったそうじゃ。	
\\	さっぱり売れやしねェ。わざわざパレポリから来たってのによォ。あんた、何か買っていってくれよ。	
\\	ねえ、あなた達、知ってる?あの、ゆうれいの話……   はい   いいえ	
\\	なんだ、知ってるの。つまんないなァ。	
\\	北の廃墟に、昔の、りっぱな騎土の姿をしたゆうれいが出るんですって!ホラーな話よねェ、ワクワク。	
\\	う~い~もっと酒持ってこ~い!	
\\	お客さん、ちょいと飮みすぎだよ。	
\\	うるせェ、人が気持ちよく飮んでんのに!大工が酒を飮んで、何が悪いってんだ!	
\\	このまったりとした、したがとろけるようなコクと味わい……く~ッ! たまらんねェぜ!	
\\	ボク、大きくなったらパパみたいな大工になるんだ! 
\\	うちの亭主ときたら酒場で飮んだくれてばかり!今度、ガツーンとおこってやらなきゃ。	
\\	北の廃墟……あそこには、なにか不気味なただならぬものを感じるのじゃ……	
\\	おいらは大工みならいさ。親方なら、酒場でのんだくれてるよ。	
\\	だいくどうぐが、ぬすまれちまった……しごとになんねーよ、これじゃ。	
\\	トマ……ずっとこの村にいてくれればいいのに……	
\\	トマは、虹色の貝がらを探すとか言ってまた旅に出たみたいじゃの。	
\\	10年くらい前、この村の北にある廃墟に魔物がなにか運び込んだらしいよ。宝物じゃないかってウワサだけど……	
\\	トマ「よう、
\\	。   俺は、ついに虹色の貝がらの   手がかりを見つけたぜ。   しかし、今度ばかりはヤバいかんじが   するんだ。   そうそう、こいつをあんたに   あずけとくぜ。	
\\	トマ「もし俺が死んだら、俺の墓に   この酒をかけてくれよ。   ……、ケッ、エンギでもねェや。	
\\	ああ……トマは今ごろ、どこを旅してるのかしら……	
\\	へ? 大工道具をかせ?別にかまわんが、どーする気だ?……、ま、いーや。俺はもーちょい飮んで行くから道具はかみさんから受け取ってくれや。	
\\	え、大工道具?ちょっと待ってておくれ。	
\\	大事につかっとくれよ。	
\\	おっ、あんた。だいくどうぐを持ってるじゃねーか。おいらにゆずってくれよ?   はい   いいえ	
\\	おっ、ありがとよ。これでしごとが出来るぜ!	
\\	がっくし……	
\\	さっきはあんがとよ。これでおいらも、しごとが出来るぜ!よし! 手はじめに、北の廃墟でも修理してくるかァ!野郎ども、行くぜ!	
\\	あいよ! 親方!	
\\	修理しといたぜ。	
\\	モンスターが巣食っている場所は修理出来ねーんだ。フロアのモンスターをぜんめつさせたらまた、たのみにきな。おいら達が修理してやるぜ。	
\\	北の廃墟の修理?2000
\\	かかるぜ。   はらう   はらわない	
\\	金がねーんじゃ、無理だぜ。	
\\	魔王に戦いをいどんだ、おろかな男      サイラス、ここに眠る。	
\\	サイラスよ、俺は帰って来た。	
\\	おさなき日のちかいを、	
\\	そして、お前との最後のやくそくを   果たすために!!	
\\	サイラス「グレン……	
\\	よく……   よくここまで、来てくれた……	
\\	俺を……、うらんでいるのだろう……	
\\	サイラス「フフ……   私は今のお前ほど、りっぱな騎土では   なかった……   この体を魔王の炎にやかれた時   私の心は、この世に残された人の事を   思い、千々にみだれた。   ガルディア王……   リーネ王妃……   魔王……	
\\	そして……	
\\	親友のお前の事を……	
\\	サイラス「だが、長い時間をかけてえた   お前の強さは本物のはずだ……	
\\	これで心おきなく眠りにつける……   さらばだ、グレン……	
\\	カエル「ま、待ってくれ、サイラス!   俺は……	
\\	俺は………!	
\\	サイラス「王妃を……   リーネ様をたのむ……	
\\	さらばだ……	
\\	我が友 グ……レ………ン…………	
\\	カエル「サイラァーーーースッ!	
\\	カエル「すまなかった……   サイラス……	
\\	カエル「グ、グランドリオンが?	
\\	カエル「この……みなぎる力…… 
\\	「フフ……	
\\	「そうさ!	
\\	リオン「あんたは、なやんでたろ。	
\\	グラン「勇者の強さは、意思の強ささ!	
\\	リオン「罪ほろぼしのためなんかじゃない。	
\\	グラン「あんたの意思が、今、本当の強さを   持ったのさ!	
\\	カエル「俺の……意思……!	
\\	リオン「これで、心おきなくボクらも   力を出せるね、兄ちゃん!	
\\	グラン「そうだな、リオン!	
\\	カエル「これが……   これが   グランドリオンの本当の姿なのか!!	
\\	カエル「サイラスよ、俺は行く。   お前のこころざしをつぐため……。   それが……!	
\\	カエル「お前への最後のはなむけだ!	
\\	カエル「さ、行くぞ……。	
\\	そのおもいを親友グレンにたくして     勇者サイラス、ここに眠る。	
\\	もう、おいら達が修理出来る場所は残ってないぜ。	
\\	カエル「見きったり   フリーランサー!!	
\\	それは、カエルの手の中で   さんぜんと光り輝いた……!!	
\\	金の石を手に入れた!	
\\	カエル「俺はもう、以前の俺とは   違うぜ……!!	
\\	おにまるがバージョンアップ!     
\\	を手に入れた!	
\\	セイレーンがバージョンアップ!    
\\	を手に入れた!	
\\	ノヴァアーマーがバージョンアップ!   
\\	を手に入れた!	
\\	ペンダントに反応して中の物が光りだした。なかみを取り出しますか?   はい   いいえ	
\\	この村の北にある勇者の墓。あそこには、ガルディア王国の騎土サイラス様がほうむられておるのじゃ。	
\\	この村の北には、あのサイラス様のお墓があるのさ。 一度行ってみなよ。	
\\	この村の北には、勇者の墓がある。400年前、ガルディア王国を救わんとして単身、魔王に戦いをいどんだ勇者サイラス様が眠る場所だ。	
\\	400年前、魔王と戦った勇者サイラスとその親友グレンの話……遠い昔の伝説です。	
\\	ヒマよねェ……私もガルディア王国の千年祭に行ってみようかしら?	
\\	平和すぎる世の中というものもツマラないものザマスことよ、ホホホ。	
\\	この村の北にある勇者の墓。あそこにほうむられておるサイラス様はこの町のまもり神じゃ。	
\\	え? いつも酒場にばかりいていつ大工のしごとをしているのかって?カタイことはいいっこなしだぜ!	
\\	ねえ、あなた達、知ってる?あの、サイラス様とその親友の話……   はい   いいえ	
\\	サイラス様には、グレンという名前の親友がいたんですって!そのグレンという人は、サイラス様が魔王に殺された後、伝説の剣を手に魔王軍と戦ったそうよ。	
\\	男の友情よねェ、うるうる。	
\\	トマ13世「伝説の太陽石って知ってるか?   トマ1世が
\\	ってヤツに   先をこされた虹の貝がらって宝と   かけあわせると、すっげェ武器が   作れるらしいぜ。	
\\	古代の書物で見つけた情報だ。	
\\	王国歴 
\\	いだいなる探検家 トマ・レバイン       ここに眠る……。	
\\	トマ「よう
\\	、ひさしぶりだな。	
\\	ついに、虹色の貝がらを見つけたぜ。	
\\	トマ「この岬から北西の海上に   巨人のツメと呼ばれる島がある。   虹色の貝がらは、そこにある。	
\\	トマ「しかし、気をつけな。   あそこにゃ化け物がわんさといるぜ。	
\\	トマ「じゃあ、そろそろお別れだ。	
\\	トマ「やっぱり、お前と飮む酒は最高だぜ。	
\\	あばよ……。	
\\	トマの書いたらしいメモが落ちている……『落としあな…もしかしてわざと落ちる? ちっ…でなおしてくるか……』	
\\	マール「ティラン城?	
\\	ルッカ「ティラン城?	
\\	ロボ「ココハ恐竜人の   城のようデスネ。	
\\	カエル「ティラン城か?	
\\	エイラ「ゾワー!!   恐竜人の城 なんでこんなとこ ある?	
\\	魔王「原始時代の城か?	
\\	マール「ラヴォスが落ちてきたときに   地中深くうまって…… そのまま遺跡に   なっていたのね。	
\\	ルッカ「ラヴォス落下時に   地中深くにうまって遺跡化したのね。	
\\	ロボ「ラヴォス落下時に   地中にうまり、そのまま遺跡に   なったのデスネ。	
\\	カエル「ラヴォスが落ちてきた時   地中深くうまり……   そのまま遺跡になったってわけか……。	
\\	エイラ「ラヴォス ふってきた時   ティラン城 うまった。   そのまま ずーっと 地面の下……!	
\\	魔王「原始の城……	
\\	マール「これが虹色の貝がら!	
\\	ルッカ「これが虹色の貝がら!	
\\	ロボ「虹色の貝がらデスネ。	
\\	カエル「フム。これが虹の貝がらか。	
\\	エイラ「きれい! 大きい!   これ 虹色の貝がらか?   食べられるか?	
\\	魔王「私には、にあわぬ色だ……。	
\\	マール「重~い!   これじゃあ運べないよ。   城からおうえんを呼んでこよう!	
\\	ルッカ「重すぎるわね。   ガルディア城からおうえんを呼んで   きましょう。	
\\	ロボ「重すぎマス。   ガルディア城からオウエンを   呼ばないと。	
\\	カエル「重い!   王にお願いし、城からおうえんを   呼んで来よう。	
\\	エイラ「おもい おもい!   運べない。城に 誰か いないのか?	
\\	魔王「誰かに運ばせるか?	
\\	王「よくぞ来た!   君らは、いつでもかんげいするぞ。   ん? どうされた?	
\\	マール「実は……	
\\	ルッカ「あ、あのですね……	
\\	ロボ「お願いがあるのデスガ……	
\\	カエル「おそれながら、お願いが   ございます……	
\\	エイラ「エイラ たのみ ある。	
\\	王「なるほど、その島にある   大きな貝がらを、後世までほかんして   ほしいと?	
\\	リーネ「他ならぬ、
\\	達の   たのみです。   私からもお願いします。	
\\	王「あいわかった!   その虹色の貝がらとやら、城に運ばせ   家宝として代々、宝物庫に   安置させよう。	
\\	騎土団長!	
\\	騎土団長「は!	
\\	王「巨人のツメにわたり、虹色の貝がらなる   巨大な貝を、何としても城に   持ちかえるのだ!	
\\	騎土団長「は、ただちに!	
\\	マール「ありがとう、王様!リーネ!	
\\	ルッカ「ありがとうございます!	
\\	ロボ「アリガトウゴザイマス!	
\\	カエル「私の様な者のため   ありがとうございます。	
\\	エイラ「お前 いいヤツ!   エイラ 気に入ったぞ!	
\\	兵士「お、重~い~   でも、必ず城へ運びます!	
\\	騎土団長「みんな~!   がんばれー!!	
\\	トマ「よう、
\\	、虹の貝がらを   見つけただって?   先をこされちまったなァ。   気にするなって。   どうせ俺じゃ、あそこの恐竜にゃあ   勝ち目がねェ。	
\\	さァて、次にねらうは伝説の太陽石だ!	
\\	最近のガルディア城って兵士の出入りが多いわね……。そりゃお城なんだからっていったらそうだけどね。	
\\	城の兵士に聞いたんだが最近、大臣の様子がなんか変だとか……。	
\\	そういえば……、楽しみにしてるガルディア王のパレードいつまでたってもはじまらない……。どうしたんだろう?	
\\	レッドベストがバージョンアップ!    レッドプレートを手に入れた!	
\\	王が代々伝わる家宝をひそかに売り飛ばしたという事ですが……。	
\\	間もなく裁判が始まります。誰って?ガルディア王の裁判です。	
\\	見はられてちゃ料理にしゅうちゅう出来ないだろ!	
\\	よけいな事は言うな!	
\\	見はってなくたって大臣の料理に毒なんかもりませんよ。	
\\	裁判が終わるまではふうさしろとの大臣のご命令です。	
\\	左に同じです。そこに、かいだんなんかあったっけな……?	
\\	通せません!	
\\	マール「通しなさーい!!	
\\	大臣「みなさん!王家に伝わる家宝   『虹色の貝がら』をご存じですか?   私もこれを見るまでは   知りませんでした。	
\\	大臣「ガルディアの遠い祖先の書いた   遺言です。   『千年の建国祭の時に   家宝『虹色の貝がら』を   国民の前にまつれ。』	
\\	王「そんな物は知らん!   だいたい家宝などこの城にはない。	
\\	大臣「これがニセモノとでも?	
\\	なぜヒコクは『虹色の貝がら』を   国民の前に出さないのか?   それはもうここにないからです。   よくに目がくらんだヒコクはお金ほしさ   に大事な家宝を売ってしまったのです。   さ、証人を……。	
\\	大臣「な、何です王女様。裁判中です。	
\\	マール「父上!	
\\	王「おお、マールディアよ   私は大臣にはめられようとしている!	
\\	大臣「人聞きが悪いですなあ。   しょうこさえあれば王の無実は   証明できるのですよ。	
\\	マール「しょうこ?	
\\	大臣「ええ   もし家宝を売っていなければ   まだこの城にあるはずですからな。   私の目の前に『虹色の貝がら』を   持って来られたのならば、無罪を   みとめましょう。   ま、無理でしょうけどね……くく…。	
\\	王「大臣! 何をたくらんでおる?!	
\\	大臣「さ、そいつらをつまみ出せ!	
\\	だめです。王女さまといえどもお通しできません。	
\\	マール「何言ってるの!   あれは大臣のでっちあげなのよ。   初めっから家宝なんてないのよ!!	
\\	マール「なるほど!   
\\	600年の王様……   つまり先祖さまにわたして来た   『虹色の貝がら』が400年たった   今でもしまってあるって言うのね?	
\\	マール「行きましょう。   きっと地下だわ!	
\\	ルッカ「『虹色の貝がら』を   持って行けば無実を証明出来るわ!	
\\	ルッカ「わすれたの?   私達が
\\	600年でやってきた事…。   歴史を変えたため、今の時代のこの城に   『虹色の貝がら』はあるはずなのよ!	
\\	ロボ「『虹色の貝がら』を   持っていけば無実を証明デキマス。	
\\	ロボ「わすれマシタカ?   
\\	600年でやってきた事。   歴史を変えたため、今の時代のこの城に   『虹色の貝がら』があるはずなのデス。	
\\	カエル「『虹色の貝がら』を   持って行けば無実を証明できるぜ。	
\\	カエル「わすれたか?   俺達が
\\	600年でやってきた事を。   歴史を変えたため、今の時代のこの城に   『虹色の貝がら』があるはずなんだ。	
\\	エイラ「『虹色の貝がら』 持ってく!   マールの父親 助かる!!	
\\	エイラ「思いだせ マール!   エイラ達 むかしの王様   たのんだ!   『虹色の貝がら』 今 この城ある!   ちがうか?	
\\	魔王「『虹色の貝がら』を   持って行けばいいではないか……。	
\\	魔王「バカめ、わすれたか?   
\\	600年でやってきた事を……。	
\\	まもなく判決が出ます。それまで誰も出入りを許すなとの大臣のご命令で。	
\\	ここは宝物庫です。危険ですので入れません。	
\\	く……ネズミにスネをかじられた。	
\\	にょろた「親分は13代にわたる   うらみがはらせるってウキウキさ。   きっきっきっき!にょろぽん「でっちあげのしょうこ品で   王様を死刑にするんだとよ。   きっきっきっき!	
\\	マール「やっぱりいんぼうよ!   急ぎましょう!	
\\	商人「ええ、たしかに王様から   買いましたよ。   大変お金に困られていた様子で……。	
\\	ガルディア王「うそをもうすな!   私はお前など会ったこともない。	
\\	裁判長「せいしゅくに!	
\\	商人「へへ   これでよかったかい?	
\\	大臣「グーだ!	
\\	マール「あった! 虹色の貝がら!!	
\\	ルッカ「何これ……、てがみ?	
\\	ロボ「何デスカ、このてがみハ?	
\\	カエル「こんな所にてがみ……?	
\\	エイラ「ここ 何かある!   かみ ある かみ ある!	
\\	魔王「てがみがあるぞ。   もやすか?	
\\	マール「マールへ……?	
\\	親愛なるマールへ。   父上との仲はいかがでしょう?   今のあなたには、わからないかも   知れませんが、いがみあっても   はなれていても、親子は親子。   あなたは、いつか親から巣立って   行かねばなりません。   そして、あなたもいつか親となる。	
\\	それはいつの世も変わりないはず。   だからこそ、私達とマールも   つながっているのですから……。	
\\	ガルディア王21世   その妃リーネより。	
\\	ルッカ「ニクい事やってくれるわね……	
\\	って、それどころじゃないわ!	
\\	ロボ「ワタシは親がいないので   子を愛する気持ちがわかりまセンが   この文章は何か……あたたかいデスネ。	
\\	カエル「リーネ様……	
\\	さ、先を急ごう!	
\\	エイラ「おう すだつ!   エイラにもそれ わかる!   プテランもすだつ。   時たてば すだつ!   マールも すだつか?   それすごい!	
\\	エイラ「すだつ! ねねする!   子供うむ! おっぱいやる!   そしてまた 子がすだつ!	
\\	エイラ「オマエ だいじょぶか?   おっぱいないな……。	
\\	魔王「感動している場合では   ないと思うが……。	
\\	マール「父上!今行くからねッ!!	
\\	マール「どーしてもダメと言うの?	
\\	兵士「はい、どーしてもです。	
\\	マール「手はあるわ!   少々あらっぽいけど……	
\\	ついて来て!	
\\	裁判長「判決をいいわたす!   有罪4 無罪1   よって有罪とする!!	
\\	大臣「これでガルディアの時代は   終わりましたな……   これからは、この私が国をしきします。	
\\	さ! つれて行け!!	
\\	「待ってーーー!!	
\\	大臣「誰だ!	
\\	マール「ちちうえェーー!!	
\\	大臣「し、しかしもうおそい!   ガルディア王は国民の裁判によって刑は   決まったのです。 王と言えども……	
\\	マール「そんな事ないわ!   それは、あなたのでっちあげよ!!	
\\	大臣「いやいや、事実王は王家の宝を……	
\\	マール「宝ならここに!!	
\\	大臣「そ、それは!	
\\	マール「かんねんなさい!   大臣!	
\\	大臣「くく……	
\\	マール
\\	大臣「くくくく……	
\\	かんねんするのは君達の方だよ。   先祖代々うけつがれてきたうらみ   今ここではらさせてもらうよ!	
\\	大臣「スーパーウルトラデラックス   大臣チェーーーンジ!!	
\\	マール「父上……   私……	
\\	王「いいんじゃ、何も言うな。   わからずやの私がいけなかったんだ。	
\\	マール「私こそ父上の気持ちも   知らないで……	
\\	ううん……   頭ではわかっていたけど   うまく言葉にならなくて……。	
\\	王「私もさ。   一時はお前が本当に遠くに行って   しまったように思えた。	
\\	でもよく考えてみると、遠くに行って   いたのは私のほうだったのだな。	
\\	マール「今は近くにいる。   これからは父上に何でも言える。   いろんな事をそうだんしたり   クロノの事を話したり   母上の事を聞いたり……	
\\	王「母か……はずかしい話だが   あの時の言葉が今やっと   わかってきた気がする。   アリーチェがいきを引き取る前に   言った言葉……『マールディアが大きくなれば あなたの前に好きな人を連れて来る日が 来るでしょう。 その時は2人をあたたかく むかえてやってね。	
\\	……だって、その日はあなたにとっても わすれられないすばらしい日に なるのですから。』	
\\	マール「母上の最後の言葉を   父上は聞いていたのね?	
\\	王「ああ……   お前は小さかったからおぼえて   いないかもしれないが……『パパ、私好きな人いっぱいいるよ! いっぱい いっぱい!』   ってはしゃいでいたよ。   そんなマールディアを見て安心した   アリーチェはほほえみながらいきを   引き取った……。	
\\	マール「そうだったの……	
\\	ガルディア王「どうしたんだい?	
\\	マール「私って小さいころは   父上のことパパって呼んでいたのね。	
\\	王「ああ、そうだよ。	
\\	マール「ごめんなさい……   パパ……   わがまま言って本当にごめんなさい…。	
\\	王「城を出るのはみとめよう、じゃが!	
\\	くれぐれも気をつけるのじゃぞ。   キミ達、娘をたのんだぞ。	
\\	ルッカ「……はい!	
\\	ロボ「オマカセクダサイ!	
\\	カエル「……しょうちいたしました。	
\\	エイラ「お前 心配しょう!   エイラ マール 守る!	
\\	「さて、ワシの出番じゃな……。	
\\	カエル「じーさん!	
\\	エイラ「ボッシュ!	
\\	ボッシュ「お前さん達ばかりに   おいしい所を持ってかれてたまるか。   虹色の貝がらからの防具作りは   ワシがやろう!   ひさかたぶりの出番じゃ。   腕がなるわい!	
\\	ボッシュ「これは特別な物質じゃ。   これから作る防具は魔法攻撃を受けた   時のききめを弱める事が出来るのじゃ。   プリズム・ドレスはききめが大きいが   1つしか出来ん。   ききめの小さいプリズム・メットは3つ   出来るな。   どっちにするね?   プリズム・ドレス一つ   プリズム・メット三つ	
\\	ボッシュ「プリズム・ドレスじゃな?   こいつは女の子しか装備出来んぞ?   はい   いいえ	
\\	ボッシュ「プリズム・メットじゃな?   こいつは誰でも装備可能じゃが?   はい   いいえ	
\\	ボッシュ「よーし、待っておれ!   腕によりをかけて、このボッシュ   一世一代の代物をこしらえてやるぞ!	
\\	ボッシュ「完成じゃ。   これを持っていけ!	
\\	ボッシュ「これで役に立てたわい。	
\\	ボッシュ「この物質……   もうひとつ別の何かとかけ合わせれば   とんでもない物が出きるぞい!	
\\	ボッシュ「おお!太陽石か。   さっそくこいつとかけ合わせてみよう!	
\\	ボッシュ「ふふ。   これで終わったと思うな。   今回のボッシュはおおはりきりじゃ!	
\\	まさか大臣がモンスターだったとは……	
\\	王もやっと、お元気になられた。	
\\	大臣が魔物だったのを見ぬけんとは……	
\\	これで城も千年祭らしくなったな。	
\\	ニセの大臣がひきいていた兵達はみな魔物だったらしい。	
\\	ベンゴシのピエールです。   だとすると、本物の大臣はどこに?	
\\	大臣「ふーっ、助かったわい!   あの化け物め、こんなとこに   おしこめおってからに……!大臣「お、マールディア様!   お友達ですかな?	
\\	大臣「と、こうしちゃおれん。   千年祭ムーンライト・パレードの   じゅんびにとりかからねば!	
\\	マールディア様!こよいのムーンライト・パレードでこのじいより、すてきなプレゼントをいたしますぞ!	
\\	ベンゴシのピエールです。かねてから千年祭のパレードを大臣と計画していたのです。	
\\	こよいのムーンライト・パレードキミ達も楽しみにしていなされ!	
\\	これでムーンライト・パレードを予定通りおこなえる。 
\\	きっと、すばらしい夜になるぞ。 
\\	パレードは千年祭のグランドフィナーレだ。 
\\	誰もが子供のころにもどれる。そんなパレードだ。	
\\	よかった、本当に……。	
\\	すまなかったな、キミ達。いや、思いかえせば、はずかしい所を見せてしまった……。これからもマールディアをよろしくたのむぞ。	
\\	やはり本物の大臣は無事じゃったな。兵士の部屋でパレードのうちあわせをするとすっとんで行きおったぞ!	
\\	さ、姫様! まずは体育から! 
\\	いやいや、教育じゃ! 
\\	いいえ、やはりしつけをやり直さないと!	
\\	お食事なら、こちらにおかけ下さい。	
\\	何にいたします?   リフレッシュパン   スタミナうなじゅう   クロ・スペシャルかしこまりました。お目が高い。これは昔、この地にあらわれた勇者の名をとった料理なのです。	
\\	お待たせしました。       
\\	が回復した!	
\\	そらそら~ッ!!	
\\	いや、あんたらのおかげだ!食べてって下さいよ。	
\\	見よ!ラヴォス神が世界の王となられたこの星を!	
\\	おぬし達虫ケラどもがいくらあがこうとも、もうおそい。すでにこの星は、ラヴォス神によって食いつくされておるわ!	
\\	かたく閉ざされている……。	
\\	シルバードにもどりますか?   はい   いいえ	
\\	虫ケラどもが!またも、わらわにさからうつもりか!	
\\	わらわは永遠の命を手に入れた!ラヴォス神とともに永遠に生き続けるのだ!ラヴォス神は地中でゆっくりと星を食らう。	
\\	そして力をたくわえ14000年のちに星をほろぼし、世界の王となられるのじゃ!	
\\	そして力をたくわえ1400年のちに星をほろぼし、世界の王となられるのじゃ!	
\\	そして力をたくわえ999年のちに星をほろぼし、世界の王となられるのじゃ!	
\\	この黒の夢は、ラヴォス神へとつながる道。わらわに無限の力をあたえてくれる神殿。おぬし達のもとめる未来なぞこの奥におわすラヴォス神があられる限り望むべくもないわ!	
\\	おぬし達は、ラヴォス神への生けにえとなるがよい!フォフォフォ!!	
\\	およっ! こんな所に、それも3人も!この先まだ長いですから、ここでゆっくりしていって下さいな。この夢からさめたいのならなんとかしますけど……。	
\\	どうしますか?   もどる   もどらない	
\\	おーい! お客さんだよー!	
\\	どうしますか?   売ってくれ!   あいさつだけ	
\\	いい物、たくさんあるよ!	
\\	まいどありっ!	
\\	なかなか売れませんよ、とほほ。	
\\	こぶん「まったく! いつまであいつの言   うこと聞かなきゃいけない、ケロ?	
\\	おやぶん「いいからだまって押すゲロ!	
\\	こぶん「ねえおやぶん。   おやぶん夢って何でケロ?	
\\	おやぶん「夢か……。	
\\	そうさなあ……、地下水道の王!   うーん、それだゲロ!	
\\	おやぶん「じゃあ、おめえの夢ってのを   聞かせてゲロ。	
\\	こぶん「おいらは、こぶんのまんまが   夢だケロ。	
\\	おやぶん「くーっ!   泣かせること言いやがって!	
\\	みはりとクロウリーのコンビに   聞かせてやりてえゲロ!!	
\\	こぶん「ケロッ ケロッ ケロッ!	
\\	おやぶん「ゲロッ ゲロッ ゲロッ!	
\\	こぶん「乗ってケロ!	
\\	女王「ククク……。   そこにねむっているのは、   お前達の未来だ……。   これから、かなうかもしれぬ夢……   えられるかもしれぬ、よろこび   悲しみ……   お前達の明日そのものなのだ!	
\\	女王「この黒の夢はあらゆる時間   次元をこえてながれている……。   ラヴォス様がめざめる   その時を待ちながら……。   お前達の未来は   いつか必ずここにたどりつく。   お前達に未来はない!   わらわを倒し、この黒の夢を   止めぬかぎりな!	
\\	女王「来い、人の子よ!   わらわがいざなってやろう   ラヴォス様のねむりの中へ……。	
\\	永遠の黒き夢に……!	
\\	女王「ここでは力が出せん……。   良い事を思い付いた。   キサマら魔神器に取り込んでくれる。   ありがたく思うとよいぞ……。	
\\	この船の一部になれることを!   この私の一部になれることを!!   ラヴォス様の一部になれることを!!!	
\\	虫ケラどもが……。わらわは、ラヴォス神と共に永遠にこの世をしはいする女王なるぞ。そのわらわに、さからおうというか。	
\\	魔王「おろかな……。   全ての存在は、ほろびの宿命から   逃れる事は出来ぬ……。	
\\	魔王「ジールよ。   ラヴォスに見入られた、悲しき女。	
\\	せめてもの情けだ……	
\\	魔王「この手で、全てを   終わらせてくれる!	
\\	呪われし予言者よ。そなたが海底神殿でおかした罪わらわはわすれておらぬぞ。今こそ、その死を以てつぐなうがよい!ム、虫ケラのぶんざいでこのわらわを追いつめるとは……	
\\	ラヴォス神よ、その御力をわらわに!!	
\\	フフフ……ハハハ……	
\\	ついに、ラヴォス神がおめざめになる!キサマ達虫ケラなぞラヴォス神の前では赤子同然。	
\\	わらわは、ラヴォス神とともに永遠の生命を手にする事としよう!	
\\	死してその思い上がりをくやむがよい!	
\\	星の夢の終わりに	
\\	老人「黒の夢は消え、あの女もようやっと   眠りにつく事が出来た。   人にもどれたのじゃ……。   今こそ、そこのバケツからのゲートに   飛び込むのじゃ……!   これはお前さん達だけの戦いではない。   この星の生命達……   そう、この星とラヴォスとの   戦いなのじゃ……!	
\\	いい天気だねェ。	
\\	今日も、いい天気じゃのう。	
\\	マール「何!?	
\\	ルッカ「こ、これって……	
\\	ロボ「エネルギー値、100!   
\\	まだ上がり続けマス!	
\\	カエル「な、何だってんだ!?	
\\	魔王「来る……。	
\\	マール「で、出てくるの!?	
\\	ルッカ「まさか!?	
\\	ロボ
\\	1時24分!   まちがいアリマセン!	
\\	カエル「ラヴォス……!?	
\\	エイラ「強いヤツ!    いっぱい いっぱい 強いヤツ!!	
\\	魔王「さ、出て来い……!	
\\	マール「こ、これが……!	
\\	ルッカ「あ、あわわ……。	
\\	ロボ「戦闘能力値、計測不可能!?	
\\	カエル「で、でけえ……。	
\\	エイラ「ウ~~~ッ……!	
\\	魔王「待ちわびたぞ……!	
\\	「ラヴォス……!	
\\	マール「どうしよう? 
\\	戦う!   今は逃げよう!	
\\	ルッカ「ま、まさか行かないわよね?   戦う!   今は逃げよう!	
\\	ロボ「行きマスカ? 
\\	戦う!   今は逃げよう!	
\\	カエル「マ、マジか? 
\\	戦う!   今は逃げよう!	
\\	エイラ「戦うか!?   戦う!   今は逃げよう!	
\\	魔王「フッ、おじけづいたか?   戦う!   今は逃げよう!	
\\	マール「そうだね!   ドン達のためにも!	
\\	ルッカ「み……、みじかい   せいしゅんだったわ!	
\\	ロボ「行きマショウッ!!	
\\	カエル「行くぞグランドリオン!	
\\	エイラ「エイラ 逃げない!   エイラ 戦う!!	
\\	魔王「今こそ、この私がお前を   消し去ってやる……!	
\\	「や、やった……!?	
\\	マール「や、やった……!	
\\	ルッカ「お、終わった……	
\\	ロボ「や、やりマシタ!!	
\\	カエル「やったか!?	
\\	エイラ「ラヴォス 倒した!   エイラ達 一番 強い!!魔王「やったか……?	
\\	マール「あ、あなが開いてるよ!	
\\	ルッカ「私達が戦った跡……?	
\\	ロボ「コレは……   ワタシ達が戦った跡デショウカ?	
\\	カエル「こりゃ……俺達の戦った跡?	
\\	エイラ「ラヴォス あな あいてる!	
\\	魔王「外がわは死んでいるか……。	
\\	マール「けど、こんなんで、こいつが   終わりとは思えない……!	
\\	ルッカ「コ、コレで終わってくれると   うれしいんだけど……	
\\	ロボ「ラヴォスの体内からは   さらに強力な生命エネルギーが感知   されマス……!	
\\	カエル「だが、これで終わりとは   思えんがな……。	
\\	エイラ「これ ラヴォスのカラ!   なかみ まだ 元気!!	
\\	魔王「うかれるな!   安心するのはまだ早い……!	
\\	マール「な、中に入ってみよう!	
\\	ルッカ「中に入るしかないのね。   やっぱり……	
\\	ロボ「体内にシンニュウし   シンゾウ部を倒さねば!	
\\	カエル「しょせん、こいつはヨロイ。   なかみをブッ倒さなきゃな……!	
\\	魔王「いきのねを止めてやるぞ。   ラヴォス……!	
\\	マール「キャアッ!	
\\	ルッカ「だ、だいじょぶでしょうね!?	
\\	ロボ「見たこともないエネルギー!   まるで人間の精神力を、おそろしく   ぞうふくしたような……!!	
\\	カエル「なななな、何だってんだ!?	
\\	エイラ「シルバード こいつ   生きてるか!?	
\\	魔王「このまま、ヤツの中まで   突っ込んでやる……。	
\\	エイラ「よし このまま行け!    シルバード!!	
\\	カエル「このままヤツと戦っても   ラチがあかねえ!   このままシルバードで奥まで   つっこんでみるってなどーだ……?	
\\	マール「このままシルバードで   ラヴォスの中までつっこんじゃおうよ!	
\\	ロボ「このまま戦っていても   ラヴォスには、ダメージがアリマセン。	
\\	このシルバードで、奥まで行って   本体をたたいてみては、どうデショウ?	
\\	ルッカ「……シルバードでラヴォスの   外がわを突き破って、中まで   行けないかしら?	
\\	ルッカ「シルバードを   こわしちゃうかも知れないわよ!   だいいち、中にいる私達がもつかどうか   
\\	ロボ「ラヴォスの外かくと   シルバードの強度は、ほぼ同等デスガ   ワタシ達が突入のショックに   たえられるかドウカ……	
\\	マール「えー!?   そんな事したら、シルバードも   私達も……。	
\\	カエル「ヘタすりゃ、俺達もおだぶつ   だぜ……。	
\\	エイラ「だいじょぶか!?   シルバード 死なないか!?	
\\	魔王「もつほしょうがあるのか……?	
\\	魔王「まともに戦っていても   勝ち目はない……。	
\\	マール「シルバードを信じようよ!	
\\	ルッカ「シルバードなら……   やってくれるはず……!	
\\	ロボ「かくりつは、きわめて   ひくいデスガ……。   シルバードのエネルギーとワタシ達の   精神エネルギーが一体となれば   あるいは……	
\\	カエル「あのバケモンを倒すには   中からたたくしかないだろう……!	
\\	エイラ「シルバード 負けない!   エイラも負けない!	
\\	マール「たのむよ、シルバード!	
\\	ルッカ「ゴメンね、シルバード……	
\\	ロボ「たしかに他に成功する   かくりつもありマセンネ。	
\\	カエル「いちかばちか……だな。	
\\	エイラ「よーし!   エイラ がんばる!   シルバードも がんばれ!!	
\\	マール「みんな、しっかりつかまって!	
\\	ルッカ「信じてるからね……   シルバード!	
\\	ロボ「ミナサン、突入時のショックに   備えてクダサイ!	
\\	カエル「俺達を守ってくれ……   サイラス!	
\\	エイラ「エイラ 負けない!   シルバードも 負けるな!!	
\\	魔王「……待っていろ、ラヴォス。	
\\	マール「行くよ! みんな!!	
\\	ルッカ「たのんだわよ、シルバード!	
\\	ロボ「シルバード発進!	
\\	カエル「ハイヨー! シルバード!	
\\	エイラ「行く! シルバード!!	
\\	魔王「今、行ってやるぞ……   ラヴォス!	
\\	マール「う、ううん……	
\\	ルッカ「み、みんな無事……?	
\\	ロボ
\\	カエル「イ、イテテテ……	
\\	エイラ「エイラ 生きてる……!	
\\	魔王「さっさと目をさませ。   ここは、ヤツの中だぞ……!	
\\	カエル「……よく、ここまで   たどりついたな。	
\\	ロボ「シルバードは……   大破してしまいマシタ。	
\\	マール「ありがとう、シルバード……	
\\	エイラ「シルバード……   お前 よく戦ったぞ!	
\\	魔王「……もう後もどりは出来んぞ。	
\\	マール「ずいぶん奥まで来ちゃった   けど……	
\\	ルッカ「だ、だいじょぶでしょうね。	
\\	ロボ「センサーの反応によれば   中心は近いデス!	
\\	カエル「うすッ気味悪いトコだぜ。   魔王城より悪しゅみだ……。	
\\	ラヴォスのにおい 強くなる!!	
\\	マール「こいつが……ラヴォス!	
\\	ルッカ「こ、これが……本体!?	
\\	ロボ「シ! 信じられマセン……!   この星に生命がたんじょうして以来の   あらゆる生物の遺伝子を持っていマス!カエル「ともに戦ってくれ……   サイラス!	
\\	魔王「今度こそ……   きさまのさいごだ、ラヴォス……!	
\\	マール「に、人間の形……	
\\	こ、こいつ、もしかして……   この星の生き物ぜんぶの力を   そなえてるんじゃ……	
\\	そうよ! 原始にふって来て以来ずっと   私達の下でこの星の生き物のすべてを   吸い取って……!!	
\\	ルッカ「わ、わかったわ。   こいつのねらいが……   星に寄生して長い時間をかけて   その星の生命体のすぐれた部分を   よりすぐって、あつめる……。   その遺伝子を持った子供を   あの死の山で生み、また別の星へ……	
\\	ロボ「コ、コレがラヴォスの   目的だったのデス……!   星のすべての生物の遺伝子、記録を   えて……   自分自身が進化をとげるのデス!	
\\	カエル「ど、どうなってやがる……!?	
\\	待てよ、こいつこの星の生き物   すべての力を持ってるってんじゃあ……	
\\	そうか! 俺達、この星の生き物の   進化して来たすべてを……   こいつは眠りながら、たくわえて   いやがったのか……!!	
\\	太古の時代より、地下に眠り   この星のすべての生命の進化を   そのまま我がものにして来たのか……。	
\\	エイラ「こいつ エイラ達!   恐竜人 プテラン ニズベール……!   エイラや 
\\	みんな!   みんなのにおい する……!!   大地の生命のにおい 全部   持ってる!!	
\\	ロボ「戦闘能力値は……	
\\	ケ、計測不可能!   センサーがこわされそうデス!!	
\\	マール「じゃあ私達はこいつのために   生きてきたっていうの!?	
\\	魔王「いわばラヴォスのエサに   すぎなかったというわけだ。   我々人間、いや……	
\\	この星の生命すべてな……。	
\\	ルッカ「この星の生命が長い時間をかけ   してきた進化を、ねむってるだけで   しようなんてムシがよすぎるわよ……!	
\\	カエル「じょうだんじゃねえ……   てめえのかてになるために、みんな   生きてるわけじゃねえ!	
\\	エイラ「逃げない!   エイラ 逃げない!   逃げる 負けるより イヤ!!	
\\	マール「ここは、クロノや私達……   リーネやドンや……、みんなの……!	
\\	マール「ここは、クロノや私達……   みんなの……!	
\\	みんなの星なんだからッ!!	
\\	ルッカ「そんな……	
\\	そんなふざけた進化なんて   私はみとめないからッ!	
\\	ロボ「ワタシは人間によって   作られた命……   シカシ命は命デス!   ルッカや、他のミナサンと同じ……	
\\	この星の多くの命の一つデスッ!!	
\\	カエル「てめえなんぞのために……!  俺達は生きちゃいないッ!!	
\\	エイラ「エイラ 負けない!   エイラ達 この大地の命!	
\\	お前 この大地の命 ちがう!!	
\\	今度こそ、きさまを倒し……	
\\	わが長き闘いに決着をつけてやる……!	
\\	ラヴォスが攻撃モードに入る!	
\\	ラヴォスの攻撃モードが変化する!	
\\	「クロノ……	
\\	クロノ!!!	
\\	兵士「いつまでねてるつもりだ!   いいかげんに起きろ!!	
\\	兵士「しっこうゆうよは終わりだ。   せいしきに刑をしっこうする。   城へ同行しろ。	
\\	王「さんざん探したよ。 クロノ君。   私の娘を引っぱり回して。	
\\	マール「父上やめて!	
\\	マール「パパやめて!	
\\	マール「クロノは何もしてないわ!	
\\	王「クロノが何もしていないと   娘は言っておるが……	
\\	王「大臣もそう思うか?	
\\	大臣「いいえ。   クロノ達は国の一大事に   かかわる事をしました。	
\\	王「ピエール、君もそう思うか?	
\\	ピエール「いいえ。   クロノ達は国の一大事に   かかわる事をしました。	
\\	マール「大臣まで何を言うの?	
\\	マール「ピエールまで何を言うの?	
\\	マール「話せばわかるわ!   私が今まで城を出ていたのには   わけがあるの……   実は……	
\\	王「未来を救ったのだろう?	
\\	マール「へ?	
\\	ドン「未来を救い、われわれに元気の意味を   教えてくれた……。	
\\	ガルディア21世「魔王軍を   倒してくれた。   王国歴600年の世界にふたたび   平和がおとずれた。	
\\	キーノ「クロ達 恐竜人 倒した!	
\\	マール「ど、どうしたの? みんな	
\\	マール「ルッカ……   連れて来たのね?	
\\	ガルディア33世「お前達が   大きな使命をせおっていたこと……。	
\\	歴代のガルディア王家の血を引く者達が   すべて話して下さった。	
\\	マール「ガルディアの血……って   みんな、私の先祖や子孫なの?	
\\	王「平和ボケして自分の事しか考えて   いなかった自分が情けない。   国のため……、この星のためを   考えなくてはならない時に。	
\\	王「さあ! 戦勝パレードじゃ!   世界を救ったヒーローを   しゅくふくじゃ!	
\\	ガルディア33世「さ、マールよ。   思いっきり千年祭最後の夜を   楽しんで来い!	
\\	さあ! 未来を救ったクロノと	
\\	無事お城に帰ってきたマールディア王女とそしてガルディアのますますのはってんを願って……	
\\	イッツァ ムーンライトパレード!	
\\	とってもファンタスティック!	
\\	王女様きれいだな……や、当然君もだよ。	
\\	ジナ「あなたが未来を救ったなんてね……。   信じられないわ。 でもね。   お母さんはお前にもっと家にいて   ほしいよ。	
\\	……ネコのめんどう見る人が   いなくて大変なんだから。	
\\	ララ「こんなすてきな夜を   すごせるなんて……。   あなた、ありがとう。	
\\	ララ「あんた。   すわって飮んでばかりいないで。   私達もいっしょに踊りましょう!	
\\	ララ「あ! クロノくん。   王様から聞いたわよ。未来を救った事。   ルッカのおさななじみが   そんなすごい事をする人だったなんて。   なんかじまんしちゃいたい気分だわ。	
\\	タバン「いやー   祭りで飮む酒はうめーぜ!タバン「クロノよ。   お前もいっぱいやりなよ。   今日はヒーローなんだからよ!	
\\	ボッシュ「もう刀などいらない時代が   来たのかもしれんな。   刀かじボッシュとしての役目は終わって   しまうが、この星空を見ていると   そんな事はどうでもよくなって   しまうわい。	
\\	ルッカ「もうみんな……お別れなのよ。	
\\	カエル「みな、それぞれの時代へ。	
\\	ロボ「ラヴォスが死に   ゲートの力が弱まってイマス。	
\\	ルッカ「ゲートが閉じる前に   さよならを言わなきゃ。	
\\	マール「そんな、みんな行っちゃうの?	
\\	エイラ「クロ 強かった!   マールも 強かった!   エイラ 楽しかった!!	
\\	マール「遠い遠い   おじいちゃま。   元気な子供を生んでね。   じゃないと私がこまっちゃうから	
\\	キーノ「へへ……   だいじょうぶ! エイラ 元気!!	
\\	マール「そうね!!	
\\	マール「……って?   何それ?? どういう事?	
\\	エイラ「キーノ バカ!!   さ エイラ達 行く!	
\\	カエル「にぎやかな連中だな。   自分の先祖かもしれないと思うと   ほっといてもいられないがな……。	
\\	さあ王様、リーネ様がお待ちです。   私達も帰りましょう。	
\\	マール「カエルさん……	
\\	カエル
\\	別れに多くの言葉はいらないさ。	
\\	マール「そう。   言葉とはかぎらないわ。	
\\	ルッカ「良かったわね。   王女様のキッスで姿がもとにもどるって   いうのがハッピーエンドの定番よ。	
\\	マール「サラを探すの……?	
\\	ロボ「ルッカ   ワタシも未来で元気にやっていきマス。	
\\	マール「どうしたの。 ルッカ?   ロボにお別れは……	
\\	ロボ「ヤハリ気づいてたのデスネ。	
\\	マール「何のこと?	
\\	ロボは廃墟となった未来で   生まれたわ……。   でも私達がラヴォスを倒した事に   よって未来は明るくなるはず。   つまり新しい未来では   ロボの存在は……	
\\	ロボ「ハハ、そんなことないデス。   きっと新しい未来でもワタシは……	
\\	ルッカ「ロボのバカ バカ!   悲しい時はすなおに悲しむのよ!!   こっちがよけい悲しく   なっちゃうじゃない!!	
\\	ロボ
\\	そんな思いやりの気持ちを   教えてくれたのもルッカデス。   とてもカンシャしてイマス。	
\\	マール「なみだは似合わないわ。   ルッカ……!   新しい未来でも、ロボはきっと   生まれて来るわ。	
\\	ロボ「サヨウ……ナラ。	
\\	ロボ「おっと   オイルでアイセンサーがかすんで……	
\\	ルッカ「いつかした話   おぼえてる?	
\\	マール「ええ   死ぬ時に見る思い出の話……。	
\\	ルッカ「もうその人は   助かったみたいね。	
\\	マール「うん。   私も感じる。 その人のいぶきを……。	
\\	マール「時間を旅するなんて……   にが重すぎるね。	
\\	ルッカ「シルバードも壊した方が   良さそうね。   もう、みんなと会えなくなるけど……。	
\\	ジナ「これこれ、もどってきなさい。	
\\	ジナ「ほら、クロノ。   あなたがエサをきらしたもんだから   みんな逃げ出しちゃって……。	
\\	ジナ「あ!これ!	
\\	マール「た、たいへん!!   クロノ!!   もうゲートは一生開かないわよ!!   どうする?	
\\	ルッカ「そうね!   追いかけるしかないわね!	
\\	マール「追いかけるって   ゲートはもう……	
\\	マール「まさかルッカ……	
\\	ルッカ「たーいむ マシーーン
\\	が   あるじゃない!	
\\	オーッホッホッホ!	
\\	兵士「おっとマールディア様。   こんな所に。 王様がお呼びです。	
\\	王「私はうれしいぞ。   もうお前が遠くに行くことも   ないじゃろうからな。	
\\	王「マールディアよ。この日を記ねんして   私からのプレゼントじゃ。	
\\	王「ほーら。 かわいいじゃろ。   マールディアの鐘じゃ   いい音がするぞ。	
\\	王「おいマールや……どこいった?   ほれ……誰か……手伝ってくれ。	
\\	マール「ありがとう……   でも私。   じっとしていられないの!	
\\	タバン「王様!   間に合いましたぜ!   一発ドカンと行きますかい!!	
\\	王「いや、タバン。   悪いが今はそれどころじゃ……	
\\	マール「これは!	
\\	王「マールディアの鐘じゃよ。   400年間この国を見守り続けて来た   リーネの鐘にかわって、これからは   マールディアの鐘が歌をうたう……。	
\\	マール「すてき……!	
\\	王「さあ、お前がつけなさい。	
\\	マール「うんッ!	
\\	マール
\\	ちょ ちょっとクロノ!   助けて!!	
\\	ルッカ「ゲートはラヴォスの力で   生まれたものと思っていたけど……   今思うと違ってたのかもね。	
\\	ルッカ「もっと違う……   あたたかく大きな存在が   いろんな時代を私達に   見せたかったんじゃないかな。	
\\	老人「ラヴォスも死に   そのうちゲートも閉じるじゃろう。   今のうちに自分の時代に帰らないと   時のさまよい人になってしまうぞい。	
\\	マール「過去にもどって   クロノを助けましょうよ。	
\\	マール「心配じゃないの?	
\\	マール「ねえ。	
\\	カエル「私は中世でリーネ様を   守らなくては。	
\\	エイラ「キーノ達 待ってる。	
\\	ロボ「本当に最後のお別れデスネ。	
\\	エイラ「エイラ 帰る。   元気でな!	
\\	カエル「達者でな。	
\\	ロボ「サヨウナラ。	
\\	マール「みんな、つれないわ……!	
\\	ルッカ「人の死は運命なのよ。   マール……。	
\\	老人「おーい!	
\\	老人「お前達……   ……って、もう行ってしまったか?	
\\	老人「これがあるのをわすれてた。   ……ま、いいか。	
\\	老人「これ落として行ったぞ。   ……ま、いいか。	
\\	さあ! 千年祭り最後の夜!!	
\\	ジナ「うちの子見かけなかった?   この日を楽しみにしていたのに。   いったいどこで   あぶらをうっているのかしら?マール「みんな元気で   やってるかな……。	
\\	エイラ「『時の卵』 よこせ!	
\\	ハッシュ「だから落ち着けっちゅうに。   そんなにあせったら生き返るもんも   生き返らんだろうに。	
\\	エイラ「わかったから   早く よこせ!	
\\	ロボ「あ!   逃げマシタ!	
\\	マール「ねえねえ!   みんな! どうしたの?	
\\	あいつの持っている時の卵で生き返らせる事が出来るんだ。	
\\	ルッカ「生き返らすって?   クロノの事?!	
\\	そうだ。	
\\	マール「ところであなた誰……	
\\	マール「行っちゃった。	
\\	ルッカ「もしかして……   カエルじゃない?	
\\	ルッカ「ハ、ハンサム!	
\\	ルッカ「追いかけましょう!!	
\\	ルッカ「カエルってば。   ハンサムだったらハンサムって   最初っから言ってよね。	
\\	マール「みんな、どうしたの?   ゲートは閉じたはずなのに。	
\\	ルッカ「転送機を改造したのよ。	
\\	マール「タイムマシンに?	
\\	マール「あなたは   時の最果てにいた……。	
\\	ハッシュ「あんたら『時の卵』を   わすれておったから。   『時の卵』とはのう……      うんぬんかんぬん……      うんぬんかんぬん……      うんぬんかんぬん……      うんぬんかんぬん……      うんぬんかんぬん……      うんぬんかんぬん……        ……なんじゃ。	
\\	マール「って事はクロノを……	
\\	ハッシュ「生き返らせる事が   出来るかもしれんな。	
\\	マール「そ、そのためにみんな   あつまってくれたの?	
\\	マール「行こう!   クロノを取り戻しに!!	
\\	ルッカ「パパのそばに   いてあげなよ。	
\\	マール「でも……	
\\	ルッカ「あのバカは   きっと連れて来るから!	
\\	ルッカ「ま・か・せ・な・さ・い!!	
\\	マール
\\	ちょ ちょっと!助けて!!	
\\	よくぞ!こんな、こんなんな…しゃれじゃないぞ…じょうきょうでゲームをクリアしました。ありがとうございます。ここはクロノトリガー開発ルーム。スタッフのみんなの声をきいてください。では……	
\\	なりた けん「いろいろなこいを   しました。そのどれもが……   ふッ……	
\\	いとう ひろゆき「ティラン城?   ごめんなさい。くちにスイッチを付け   たのは、私でケロッ!	
\\	すずき かずお「すっかり しょくにん。   
\\	は こしで させ。	
\\	守屋 しゅん「24時間 おはようござ   います。   
\\	は わきばらで ぬきます	
\\	ほしの まさのり「このゲーム、ホントは   ホシノトリガーってタイトルになるハズ   だったんだよ。なんちゅーてねー。	
\\	かわい まみ「はっ!?またパーツをいれ   わすれてる!?しくしく…すみません   直していいですか~~?	
\\	まも たいぞう「…ん?なんばしよっとや、   きさん。せからしったい!!   ぼてくりこかされたいとや!?	
\\	探さないで下さい。           やまもと ひろと	
\\	あかお みのる「旅はいいよね。   あそびもいいよね。   ふっ、楽しいこと、たくさんしよう!	
\\	なかむら えいじ「おつかれさまでした!!   あしたは、外であそぼう!	
\\	ひぐちかつひさ伝説のプログラマ 29さい   「勇者になりそこねて、伝説のみと   化してしまったのは私です。	
\\	おてがみくださいね。	
\\	にし けんいち「長かったです。   初めてのしらががはえました。   やせました。	
\\	やあ!テレビのまえのみんな~元気かい?ぼくが、だいしま まなぶで~~す。んん!!こらァ~ かりてやるな~~	
\\	おかにわ しんいちろー「ふふふ……   ずっとかみのけ切らなかったら   こんなにのびちゃった。切りにいこ。	
\\	こくぼ けいぞう「このゲームに   かかわった人々、これをやってくれた   みんなが、『この』時をこえて   成長し続けますように。	
\\	一通り聞き終えたらまた私の所へきなさい。	
\\	うちやま ひろし「良くできてるなー   そのふく。   作るの大変だったでしょ?   え!ナニ?本物!? こりゃ失礼。	
\\	なかしま ふみ「どもども、中プウだケロ。   ワールドマップのちびキャラいたケロ。   ルッカは本を読んでるケロ。   クロノは急がせてるケロ。わかるケロ?	
\\	かまた やすひこ「32めがって、いーーっ   ぱいえがはいるんだぞ。   大変なんだぞ。   でも、楽しかったかな。	
\\	えとう けいた「壊れたマップの修理?   
\\	かかるぜ。	
\\	にしだはるゆき 化け物たんとー   『ねーねー2ができるってホント?』   はい   いいえ	
\\	『できたらいいよね~』	
\\	かとう「ふりやまないあめは、ない。   永遠にも、おわりがある……。	
\\	たぶん……。	
\\	まつはら けいすけ「ども、どもっ!   ボクのイベント、楽しんでもらえた?   次の作品でも『まつはらえんしゅつ』が   サクレツするので、よろしく!!	
\\	いたくら まつぞー「やあみんな おいらの   必殺技ポヨゾーダンス みつけたかな?   またつぎの作品で会おうネ!!	
\\	はすい    「話しかけるな!いまいいとこなんだ!	
\\	はるきあかね「きゃたきゃたきゃたきゅた
\\	ひるるるるるるっぴどーーんんっっ!   はらはらはらはら
\\	はらへったよう!	
\\	ちば ひろき「1ー4! 1ー4!!   1ー4!!! 1ー4!!!!   ダーッ! また外れたー!!	
\\	おおかわ かずひろ「クリアおめでとう!!   おもしろかった?	
\\	ますだ あきよし「しごともせずにこくぼ    さんとくんずほぐれつもみあってました   ……………………かたもみの話だよ	
\\	たかはし てつや「こんどは死なないぞ!   だいじょうぶだ、安心しろ。   死にたいヤツは
\\	4をやってみろ。   死ぬぞ~	
\\	おぐら よしのり「このゲームに   少しでもさんかできたことを   うれしく思います。	
\\	すぎもと こうじ「バイクレースの   プログラマーの最高ウイニングポイント   は2110らしい。	
\\	えべこういち「クロノトリガー最後まで   あそんでくれてありがとう。   そんなあなたにプレゼント。ちゅう
\\	すずき としあき「ウゥ~ン…   あー眠い…  うーおなか空いた…   とりあえず、ねよっと。	
\\	よしい きよし「クロノトリガーシリーズ   だい128作目 感動のストーリー……   おやっ また時代が違ったようじゃの	
\\	てらだ つとむ「今年、全てのかぜをひい   たな~。 へぶしゅっ!!   最後に強くなった君たち、モンスターを   へぶしゅっあまりいじめないでね。	
\\	うすだ ただひろ「よ~く見るとモンスター   にも小さなこだわりが。あっ、ホラホラ   コイツの剣のカゲとかさァ……ね?ね?	
\\	しまもと まこと「めざせ!   9999ダメージ!!   ばくはつ!	
\\	みつだ やすのり「僕は旅に出るんだ   遠い、へんきょうの地へ・・・それは   次へのステップ・アップのために。   でも、二度と戻ってこなかったりして	
\\	こばやし かずみ「やせました………   10キロ……	
\\	かなめ「おいらはマッパーみならいさ   親方なら、ゲーセンで対戦やってるよ。	
\\	ささきひろかつ『かみをのばすと声をかけ   られるのよね。しかも男に…………』   超ショッーーク!!	
\\	「あ、すいません。   今マスク付けてないんで……	
\\	ときた たかしは   リバプールの風になりましたから。	
\\	なかたに ゆきお「あそんでくれて   ありがとね。ところでキミは女の子?   はい   いいえ	
\\	まつい あきひこ「いかがでしたか?   楽しんでいただければしあわせです。   では、ほかのゲームでお会いしましょう	
\\	のむら てつや
\\	リセット!!。	
\\	うそ、うそ。	
\\	ほんね やすゆき「はじめての   しごとでした。   キンチョーしながらかきました。	
\\	はまさか しんいちろう「ほんの   ちょっぴり おてつだいしました。   くろとりチームのかた。   おつかれさまでした。	
\\	きたせ よしのり「エンディングは   10しゅるい以上。シナリオのどの   タイミングでラヴォスを倒したかに   よってかわるよ。	
\\	全部見ることができたかな?。	
\\	なおら ゆうすけ「楽しんで   かきました。 どうでしたか?	
\\	おおたち たかゆき「おつかれさまです。	
\\	しかけにん けずか「早くスイッチを   さがさないと……もえちゃうよ!	
\\	では……ドリームプロジェクトへの扉を開こう!	
\\	中へどうぞ。最高のプロジェクトスタッフが   待っておるぞ。	
\\	とりやま あきら「おーい!   さすけ!きっか!おとうさんはこんな   ゲームのしごともしてるんだぞー!   この画面見てるか?すごいだろー!   どうだー!	
\\	ほりい ゆうじ「にたもの 3たくです。   ホッチキス、モスラ、オリンピック   仲間はずれは どれ?	
\\	さかぐち ひろのぶ「前は1つ   作り終えるとやせてたんだけど……   最近は、ふとっちゃうのよ。   トホホ…… としかしらねえ~	
\\	あおき かずひこ「年まつジャンボ   10万円 当たった。	
\\	うえまつのぶお「『ェめだゃじんよ。』   さァ!ぎゃくからよんでみよう!!	
\\	よくぞクリアしました。あなたもドリームプロジェクトのメンバーの一人です。おめでとう!!こんなにはやくクリアしちゃうほどせっかちなあなたにふさわしいスタッフロールをおとどけします。さあ! かっとべ!!	
\\	あ、王女さま。また今日も変装してお出かけですかい?	
\\	まあ、王女さま。お上手なおけしょうですわ。	
\\	王女さま。また私達と同じかっこうしている。	
\\	マール「何か様子がへんね?	
\\	マール「お城に帰ってみましょ。	
\\	マール「父上!   町の人の様子が何か変なの。王「どこに行ってたマールディアよ?   お前もピョンピョンあそび回ってばかり   いないで少しは落ち着きなさい。   あ、また人間の姿になんか変装して…。マール「みんなそう言うけど……   それってどういう意味?王「何をとぼけているんだ。   お前もそんな無理なかっこうをして   いないで楽にならんか。王「ちょうどよかった。   これから400年前の先祖の   けっこんしきのフィルムを   見ようと思ってな。	
\\	宝物庫をかたづけていたら   見つかったんじゃ。	
\\	マール
\\	ゲコゲコゲコ……ルッカ「マール??   あなたもしや……マール「やだやだ! こんなの!!	
\\	勇者タータ「魔王……ッ!!	
\\	ルッカ「ふーッ……   なかなかしぶといわねえ。	
\\	ロボ「スミマセン、ルッカ……。	
\\	ルッカ「あんたのせいじゃないでしょ。   この体に武器をつけたのは、私達人間   なんだから。	
\\	ルッカ「しっかし、よくねるわねえ。   このバカ……。   おこして手伝わせようかしら。	
\\	「ダメ!	
\\	マール「ねかせておいてあげようよ。   あれだけの戦いの後だもん。	
\\	ルッカ「フ…… はいはい。	
\\	マール「どこ行くの?   せっかく、おちゃ入れたのに。	
\\	カエル「なに、ちょいとヤボ用でな。	
\\	マール「プッ……	
\\	「おろかな……	
\\	「ひさかたぶりに、死に急ぐ者が あらわれたか!	
\\	「フフン、ずいぶんとナメてくれちゃって ……「一人であたいとやろうなんてネ!	
\\	「お、お前は……!!	
\\	「ダ・ズマ・ラフア・ロウ・ライラ…… つむがれよ、天と地のはざまに……	
\\	魔王「やはり来たか……。	
\\	魔王「わが魔族も人間も   あらかた命を落とした……。	
\\	私とお前以外で生きているのは   虫ケラも同然……。	
\\	この戦いが終わった時   生きていた方が世界を……	
\\	いや、未来を手にするわけだ……。	
\\	カエル「らしいな……。   だが、そんな事はどうでもいい。	
\\	カエル「きさまと戦う事に意味が   あるのだ……!	
\\	よかろう、かかって来い……	
\\	「歴史が変わるというなら、変われ! 世界がほろぶというなら、ほろべ!	
\\	そのために、今この俺が 消えてなくなるというのなら…… それはそれで、おもしろい!!「行くぞ、ラヴォス!!	
\\	マール 「ええーっ! もう、終わっちゃうの……。	
\\	ルッカ 「この言葉が出たら、たいがい終わりっ てことらしいわよ。	
\\	マール 「せっかく、これからだったのに。	
\\	ルッカ 「じゃあ、このエンドの文字、上げても らいましょう。	
\\	マール 「上げろっ、上げろっ! はんせい会いってみましょーっ!	
\\	ルッカ 「美術さん、おねがーい!	
\\	マール 「まずは、リーネ広場にズームイン!	
\\	ルッカ 「てっかめんランナーね。	
\\	マール 「あたし、こーゆうの苦手なんだよ! かたそうだし……。	
\\	ルッカ 「走るだけじゃね……、あのヨロイは、 ハンディキャップなのにね。	
\\	マール 「また今度ってとこね。バイバーイ!	
\\	ルッカ 「次はガルディア城、あんた家よ。	
\\	マール 「へへっ、カッコいいのをそろえたわ!	
\\	ルッカ 「さすがね。私もうドキドキしちゃう!	
\\	マール 「彼だけには、どっかんピストル使わな かったでしょ。 彼の名前はピーター。でも女なの。	
\\	ルッカ 「ぐらっ! 私、その気はないわよ!!	
\\	マール 「ぎゃーっ!! ピエールよ!	
\\	ルッカ 「ぜーったいダメ、こーゆータイプ。 もっと正しい裁判をするべきよ!!	
\\	マール 「ほら、コンジョウがないわよね。 自分達でなんとかしよう!っていう 『元気』出してよ、『元気』!!	
\\	ルッカ 「同じ立場だったら、こーなってるかも よ。『元気』出せーっ!!	
\\	マール 「たね。 育つといいね……。	
\\	ルッカ 「ジョニーか……。スピードに全てをか ける男か……。	
\\	マール 「はやいだけってのもね……。	
\\	ルッカ 「なーんだ、よく見ると三りんしゃよ! パス、パス!	
\\	マール 「勝とうなんて! 一万年早いわね。	
\\	ルッカ 「ちょーカッコつけ男っ、サイラス君!	
\\	マール 「でも事実カッコいいから、まいっちゃ うんだな……。	
\\	ルッカ 「ファッションセンスがちょっとね。	
\\	マール 「知ってた? まゆをかいてるらしいんだって。	
\\	ルッカ 「うっそ! げーっ、きもちわりー! でも、勇かんだから許しちゃおっと!	
\\	マール 「トマちゃん。あたいのお気に入り!	
\\	ルッカ 「ただの酒のみよ。	
\\	マール 「それに、女好きだったりして。	
\\	ルッカ 「ひょっとしたら男好きだったりして!	
\\	マール 「あんたどう思う? このキーノっていうヤツ……。	
\\	ルッカ 「ナヨナヨ族ね。 ぜんぜんきょうみないわ。	
\\	マール 「でも、いちずな所が、かわいいって感 じ。ウソだけど……。	
\\	ルッカ 「あるき方がキライ! 変だわっ!!	
\\	マール 「でもいいヤツだったよね。	
\\	ルッカ 「とってもやさしいヤツだったよね。	
\\	ルッカ 「クロノ!!!	
\\	クロノ 「いったいどのあたりが      『はんせい会』なんだよ!?	
\\	マール 「えへへへっ!	
\\	ルッカ 「ねえ、次はどの時代へ行く!?	
\\	マール 「楽しいことがいっぱいの時代っ!!	
\\	クロノ 「今!	
\\	ルッカ 「今って……? 今、この現代ってこと?	
\\	マール 「オッケー! 現代ね!	
\\	ルッカ 「じゃあ、ゲートホルダーはもういらな いってことかなあ……?	
\\	ルッカ 「あっ! 待ってよーっ!!	
\\	だって。 バッカじゃないの?	
\\	クロノ 「いいかげんにしろよ!!	
\\	ルッカ   「クロノ! だーいすき!!	
\\	マール     「あたしもっ!	
\\	オペレータ「ト! トルースに直撃です!!	
\\	所長「落ち着け!   かくちのじょうきょうはどうだ?	
\\	オペレータ「パレポリも! チョラスも!   メディーナも……!	
\\	ほぼ、かいめつじょうたいです!!	
\\	所長「……。	
\\	オペレータ「所長! ここも危険です!	
\\	所長「そういんシェルタードームに   たいひせよ……。	
\\	オペレータ「所長! 所長は!?	
\\	所長「時間がない! 急がんか!!	
\\	オペレータ「ハ、ハッ!	
\\	所長
\\	ラヴォス……。	
\end{CJK}
\end{document}