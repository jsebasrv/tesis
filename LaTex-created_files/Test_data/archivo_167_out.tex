\documentclass[8pt]{extreport} 
\usepackage{hyperref}
\usepackage{CJKutf8}
\begin{document}
\begin{CJK}{UTF8}{min}
\\	寂しげ	=寂しそう 
\\	いー>
\\	げ 
\\	なー>
\\	げ 
\\	たいー>
\\	たげ 良いー>良さげ ないー>なさげ 
\\	げー>自信ありげ 
\\	病気がち	
\\	ことが多い 
\\	がち 
\\	がちー>遅れがち ほにゃららする回数が多い 毎日じゃない、回数が多い 今はその状態があるかどうか、関わらない 
\\	忘れっぽい	=よく忘れるようになる 
\\	っぽい 子供っぽい 男っぽい 女っぽい 油っぽい 水っぽい 黒っぽい 白っぽい 
\\	っぽいー>怒りっぽい すぐこういう様子になる 今そういう様子であるともでないとも言えるけど、何かしたらすぐそういう風になる 
\\	疲れ気味	=ちょっと疲れてる 
\\	気味 
\\	気味ー>太り気味 すこしほにゃららしている状態 前も今もの状態 
\\	っぽい(見える), 
\\	気味
\\	ちょっとだけでも) 風邪気味で、熱っぽいんだ
\\	帰れるものなら	=帰れないが、もしも帰れるなら 
\\	ものなら/もんならー>やれるもんなら、やってみろ
\\	暑いものだから	=暑いので 
\\	いー>
\\	ものだから 
\\	なー>
\\	なものだから 
\\	ものだから 
\\	なものだから /もんだから
\\	知らなかったんだもの	=知らなかったから 
\\	いー>
\\	もの 
\\	なー>
\\	なんだもの 
\\	なんだもの 
\\	ものー>疲れるんだもん 『女性や子供がよく使う』
\\	持っているものの	
\\	いー>
\\	ものの/
\\	いとはいうものの 
\\	なー>
\\	なものの 
\\	であるものの/
\\	とはいうものの 
\\	もののー>車の免許は持っているものの、ほとんど運転したことがない。
\\	車はもとより自転車も	
\\	はもちろん~も」の硬い表現 
\\	1はもとより
\\	2も 
\\	3のはもとより
\\	ても
\\	見た目はともかく味は	=見た目はどうか、わからないが 
\\	1はともかく(として)
\\	2は/が あなたはともかく、私は 
\\	旅行はまだしも	=旅行は 
\\	は/ならまだしも 私の英語力は、旅行はまだしも、留学なんて無理です。
\\	仕事の話は抜きにして	
\\	(は) 抜きにして 
\\	(を) 抜きで 抜きに 抜きの
\\	例)仕事の話は抜きにして楽しく飲みましょう(=仕事の話はやめて) 
\\	(を) 抜きにして(は)     抜きで(は)     抜きに(は) 
\\	れない 
\\	例)田中抜きではパーティーは始められない
\\	心配でたまらない	=我慢できないぐらいとても心配だ 
\\	いー>
\\	くてたまらない 
\\	なー>
\\	でたまらない 
\\	たいー>
\\	たくてたまらない
\\	暇でしょうがない	=とても暇だ 
\\	いー>
\\	くて仕方がない 
\\	なー>
\\	でしようがない 
\\	るー>
\\	てしょうがない
\\	は感情、感覚、困った状態を表すもの。 
\\	気になってしょうがない  お腹が空いてしょうがない
\\	うるさくてかなわない	
\\	硬い =うるさくて困る 
\\	いー>
\\	くてかなわない 
\\	なー>
\\	でかなわない *感情には使わない
\\	残念でならない	硬い =とても残念だ(自然になる気持ちの場合に使う) 
\\	いー>
\\	くてならない 
\\	なー>
\\	でならない 
\\	るー>
\\	てならない
\\	食べないことはないが、あまり好きじゃない	
\\	いー>
\\	くないことはない 
\\	なー>
\\	じゃないことはない 
\\	ないー>
\\	ないことはない 
\\	られないー>
\\	られないことはない
\\	覚えられないこともない	
\\	/覚えられるかもしれない/覚えられるように思う
\\	くない 
\\	じゃない 
\\	ない 
\\	られない 
\\	こともない
\\	言わないではいられない	
\\	ないでは いられない
\\	飲まずにはいられない	「ないではいられない」の硬い表現 
\\	ずには いられない せず
\\	帰らねばならない	「帰らなければならない」の硬い 
\\	1ねば ならない 
\\	ならぬ せねば
\\	忘れてはならない	「忘れてはいけない」の硬い表現 禁止を表す。
\\	待っていられない	=待ち続けられない 
\\	て(い)られない 
\\	てられない 
\\	てらんない (話)
\\	遊んでばかりはいられない	
\\	てばかりは  いられない 
\\	てばかりも  いられない
\\	努力のかいがあって	=努力の効果があって
\\	る/
\\	た  かいがある 
\\	の    かい(が)あって
\\	手術のかいもなく	=手術の効果がなく
\\	た かいがない 
\\	の かい(も)なく~
\\	やりがい	=やる価値 
\\	2がい 生きがいを感じる 苦労のしがいがある/ない 作りがいがある/ない
\\	借金してまで/借金までして	
\\	てまで(も) 
\\	まで
\\	て そんなことまで、という驚きを表す。
\\	読みかける	=読み終わっていない 
\\	かける    かけの
\\	かけだ    かけた
\\	読み切る	=全部読んだ 
\\	切る    切れる    切れない=完了しない 全部使って、残っていない様子
\\	ありえる/ありうる (得る)	=可能性がある、
\\	ことができる 
\\	2える/うる 
\\	2えない(=あるはずがない)
\\	やり抜く	=頑張って最後までやる 
\\	2ぬく 悩み抜く 考え抜く 耐え抜く 守り抜く 戦い抜く 生き抜く 最後までその状態を続けること ダメ!!使い抜く
\\	使い切る
\\	忘れないうちに	=覚えている間に、
\\	ない     
\\	うちに 
\\	ている    
\\	い      
\\	な      
\\	の     
\\	終わるか終わらないかのうちに	=終わると同時に、
\\	るか 
\\	ないかの うちに
\\	日本にいる限り/私の調べた限り(では)	
\\	=日本にいる間, 
\\	る/
\\	ない     
\\	限り   
\\	い/
\\	くない    
\\	限りは   
\\	な/
\\	である   
\\	である       
\\	、「〜の範囲内で判断すると」の意味   
\\	る/
\\	た 
\\	限り(では) 私の知る限り 彼から聞いた限り 昨日見た限り   
\\	ている  
\\	70歳以上の方に限り	
\\	に   限り      限って       限らず〜も 
\\	女性に限らず男性も化粧をする(=女性だけでなく男性も) 
\\	<数量/時間> 限り 
\\	本日限り/一回限り
\\	これさえあれば	=これがあるだけで, 
\\	さえ  
\\	ば      
\\	ければ 
\\	なら 
\\	なら 
\\	2さえ すれば
\\	かわいいからこそ	=本当に可愛いから、
\\	普通 からこそ 
\\	ば     こそ 
\\	ければ   こそ 
\\	であれば  こそ 
\\	であれば  こそ 理由を強調する
\\	信頼してこそ	
\\	て初めて 
\\	上がるばかりだ 
\\	ただ祈るばかりだ	
\\	=上がり続けている, 
\\	るばかり 
\\	=祈るだけだ, 
\\	(ただ) 
\\	る ばかりだ      のみだ  (硬)
\\	子供にしたら	=子供には, 
\\	に  したら     すれば     してみたら     してみれば
\\	本当だとしたら	=本当なら, 
\\	(普) と したら           すれば
\\	行くとしても	=行くかどうかわからないが、行く場合でも, 
\\	としても 
\\	としたって (話)
\\	監督にしても	=監督の場合でも, 
\\	にしても 
\\	にしたって (話)
\\	社会参加を目的として	=社会参加が目的で, 
\\	を 
\\	と して
\\	する
\\	した
\\	家族と共に   「とも」   硬い	=家族と一緒に 
\\	と  共に 
\\	ると
\\	増加に伴って   「ともなって」   硬い	
\\	を伴う 
\\	=増加と一緒に 
\\	に   伴い、 
\\	るのに ともなって 
\\	たのに 伴う
\\	年を取るにつれて	=年を取ると、だんだんに だんだん変化することを表す言葉と一緒に使う。 「につれて」
\\	「に従って」
\\	に  つれて 
\\	るに つれ
\\	北へ行くに従って 
\\	予定表に従い   「したがい」	
\\	=北へ行くにつれて、
\\	=予定表の通りに 「に従って」
\\	に  従って 
\\	るに 従い
\\	言った通り	=言ったこととは同じでトオリ 
\\	る   通り 
\\	た   通りに (=~と同じように) 
\\	の   通りだ      通りの
\\	通り (ドオリ)     通りに     通りだ     通りの
\\	言われるままに	=言われる通りに 
\\	る    まま(に) 
\\	られるー
\\	られるが(まま)」は他の人の意志の通りという意味。 足の向くままに歩く=
\\	驚いたことに	=驚いたことですが, 
\\	い  ことに 
\\	な   
\\	た
\\	緊張のあまり	=緊張しすぎて, 
\\	の   あまり 
\\	な 
\\	る 
\\	嬉しさのあまり    悲しみのあまり    怒りのあまり    驚きのあまり    急ぐあまり    恐れるあまり
\\	寒いわけだ	=寒いのは当然だ; 
\\	な   わけだ 
\\	ている/
\\	ていた  わけだ 
\\	られる/
\\	させる  わけだ
\\	欲しくないわけではない	
\\	る わけではない 
\\	な   わけでもない 
\\	ている/
\\	ていた わけではない 
\\	られる/
\\	させる わけでもない
\\	するわけがない	=するはずがない/絶対にしない 
\\	る  わけがない 
\\	な    わけはない 
\\	ている/
\\	ていた わけはない 
\\	られる/
\\	させる わけがない
\\	休むわけにはいかない	=休むことはできない, 
\\	る 
\\	ない 
\\	ている わけにはいかない 
\\	させる   わけにもいかない
\\	開けたとたん	=開けると同時に、
\\	た とたん    とたんに 
\\	た。そのとたん… 
\\	騒いだあげく	=騒いで大変だったが最後には、
\\	た 
\\	の  あげく    あげくに    あげくの
\\	た。そのあげく〜
\\	悩んだ末	=悩んだ結果, 
\\	スエ 
\\	た 
\\	の  末    末に    末の
\\	来たかと思ったら	=来てすぐに 
\\	たかと 
\\	たと  思ったら      思うと
\\	お忙しいところ//思っていたところ	
\\	//=今ちょうど思っていました 
\\	い  ところ 
\\	の  ところに 
\\	た  ところへ 
\\	ている  ところを 
\\	ていた  ところだ
\\	検査したところ	=検査してみたら、
\\	た ところ
\\	仕事どころではない	=仕事をする状態ではなかった 
\\	どころ 
\\	るどころ  ではない〜 
\\	ているどころ  じゃなく〜 それどころではない
\\	夏休みどころか	
\\	どころか=
\\	=夏休みなんてとんでもない 
\\	どころか
\\	も…ない)」=
\\	はもちろん
\\	もない 
\\	る 
\\	な 
\\	い  どころか
\\	間違いだらけ	
\\	だらけ =間違いがいっぱい ゴミだらけ 傷だらけ 泥だらけ シワだらけ 血だらけ 
\\	否定的な意味
\\	行ったきり	
\\	た  きり
\\	ない       っきり
\\	ない(話)       きりだ       っきりだ(話) これ(っ)きり =これからずっと それ(っ)きり あれ(っ)きり =行ったままずっと 
\\	きり     っきり(話) =他の
\\	とはしないで、それだけしている 一人(っ)きり=一人だけ 二人(っ)きり=二人だけ これ(っ)きり=これだけ、今回だけ
\\	立ちっぱなし	=ずっと立っていた 
\\	2っぱなし テレビをつけっぱなしで寝てしまった。 =つけたままで 否定的な意味
\\	予想に反して	=予想とは反対に 
\\	に  反して        反し     反する
\\	便利な反面	=便利だが, 
\\	である   反面 
\\	な    半面 
\\	である 
\\	い 
\\	る
\\	水に強い一方	
\\	である  一方(で) 
\\	な 
\\	である 
\\	い 
\\	る 
\\	予算がないと言われているが、一方では予算が無駄に使われている。 
\\	進む一方だ	=どんどん
\\	進む 
\\	る一方だ 
\\	るばかりだ)   変化を表す動詞につく。
\\	できる上に	
\\	できるし、それに。。。 
\\	る 
\\	い 
\\	な/である 
\\	の/である  上(に)
\\	考えた上で	=考えてから 
\\	た 
\\	の  上で     上の     上での     上
\\	選ばれた上は	=「以上」「からには」、選ばれたのだから 硬い 
\\	上は 決意を表す言葉やアドバイス、禁止などが続く。
\\	天気図の上では	=天気図で見ると、
\\	の 上では/上でも 
\\	上(•じょう)/上は/上も
\\	初心者向け	=初心者のため、
\\	向け
\\	だ 
\\	向き に      の
\\	向け=
\\	を対象とした/
\\	向き=
\\	天気次第で	=天気によって、
\\	次第 だ     で     では
\\	戻り次第	=戻ったらすぐに 
\\	次第 
\\	過去のことには使えない! 
\\	伺った次第です	=伺ったわけです、
\\	る 
\\	た 
\\	ている  次第です。
\\	意見に応えて	こたえて=意見を聞いて/意見通り 
\\	に 応えて    応え    応える
\\	目上の人に対して	=に向かって/と反対に 
\\	に     対して 「文」のに  対し これに    対しては それに    対する
\\	法律により/地震による被害	硬い 
\\	//地震が原因 
\\	に より    よる
\\	「によって」の硬い表現 手段•根拠•原因•場合を表す。
\\	事件に関わって	=に関係して//に影響する 
\\	に 関わって    関わり      関わる。    関わる
\\	「命」「誇り」など重要なものに影響するという意味でも使う。
\\	汚職事件	オショク 
\\	ダイヤ	
\\	乱れる	
\\	何らか
\\	の
\\	と知りながら	
\\	//知っているのに 
\\	ながら 
\\	ながらも=…けれど 
\\	い 
\\	忙しいと言いつつ	1。=忙しいというが 
\\	2つつ(も)
\\	「ながら(も)」の硬い表現 2。=忙しいながら 
\\	2つつ
\\	「~ながら(=同時に)」の意味
\\	進歩しつつある	=だんだん進歩している、
\\	2つつある
\\	知らないくせして	話 =知らないくせに 
\\	る  くせして 
\\	い 
\\	な 
\\	の 「くせに」の会話的表現 非難
\\	の気持ち
\\	事業	ジギョウ 
\\	社会事業 
\\	事業を始める 
\\	すべきではない	
\\	るべき だ すべき  ではない 
\\	であるべき だった・ではなかった 
\\	くあるべき 
\\	続けざるを得ない	ざるをえない =続けるしかない, 
\\	1ざるをえない しないー
\\	せざるをえない
\\	行われることになっている	=行われることが予定されている、決められている。 
\\	る|ない 
\\	という  ことになっている
\\	言い間違いに過ぎない   硬い	=言い間違いだけだ 
\\	る
\\	い
\\	である
\\	にすぎない
\\	単に 
\\	単なる	
\\	利用にあたり  硬い	=利用する時に, 
\\	に  あたって、あたり 
\\	るに あたっては     あたっての
\\	資料に沿って	=資料の順番や内容に合わせて 
\\	に 沿って    そい    沿った
\\	開店に先立ち  硬い	さきだち 開店する前に、
\\	先立って 
\\	る 先立ち    先立つ
\\	広い範囲にわたって	
\\	期間 
\\	範囲) 
\\	に わたって    渡り    渡る
\\	研究開発に努める	
\\	方針	
\\	ホウシン
\\	試写会	シシャカイ 
\\	覚えられっこない   話	=絶対覚えられない 
\\	2っこない
\\	試験に受かる	
\\	言いかねない   (ー)	=言うかもしれない 
\\	2かねない 
\\	たら、
\\	ならと一緒に使うことが多い。 悪い結果について使う。
\\	わかりかねる  硬い	=わかりません 
\\	2かねる 改まった表現, 
\\	信じ難い  硬	ガタイ =信じられない 
\\	2がたい
\\	長時間	チョウジカン 
\\	要求する	
\\	乗客同士	
\\	富士山が見えることから	=見えるという理由で 
\\	る  ことから 
\\	い 
\\	な 
\\	である
\\	いつも遅刻する彼のことだから	=彼はいつも遅刻するから、きっと、
\\	のことだから
\\	休むことなく	
\\	ることなく
\\	やってみないことにはわからない	
\\	1、やってみなければ 
\\	1ないことには
\\	2ない
\\	実物	
\\	ジツブツ
\\	嫌われて当然だ	=普通は嫌われる、
\\	て  当然だ 
\\	くて  当然の
\\	で  当たり前だ     当たり前の
\\	怒るのももっともだ	=普通は怒る、
\\	るの も もっともだ 
\\	いの は もっともな
\\	なの  もっとも=当たり前|当然
\\	新品も同然だ	=新品と同じようだ、
\\	る も 同然だ 
\\	(も) 同然の
\\	ない も
\\	あるだけましだ	=あるからまだいい、
\\	な だけましだ
\\	よそ	
\\	うろうろ	
\\	株価	
\\	カブカ
\\	暴落する	ボウラク 
\\	株価
\\	が暴落する
\\	株券	
\\	一流ホテルだけあって	=一流ホテルらしく、なるほどその価値があると納得できる 
\\	る  だけあって 
\\	い  だけに 
\\	な  だけのことはある 
\\	成績がいいばかりかスポーツも	=成績がいいだけでなく、その上、
\\	ばかりか/ばかりでなく
\\	も』 
\\	る ばかりか 
\\	い ばかりでなく 
\\	な 
\\	あの飛行機に乗ったばかりに	
\\	乗ったために 
\\	る ばかりに 
\\	い 
\\	な 
\\	である 悪いことが起きた理由の強調。
\\	日本のみならず外国でも	=日本だけでなく更に、
\\	る  のみならず 
\\	い  のみか 
\\	である 
\\	だけでなく、ばかりでなく の硬い表現。
\\	はまる	
\\	に 
\\	たまたま	
\\	飲もうではないか	=飲みましょう。 硬い 
\\	よう ではないか。     じゃないか。 強い誘い、強い意志。 男性的。
\\	言いようがない	=いう方法がない 
\\	ようがない    ようもない
\\	泣いているかのようだ	
\\	泣いているみたいだ 
\\	る  かのようだ。 
\\	い  かのように〜 
\\	である 
\\	である 
\\	た 
\\	ている
\\	葬式	ソウシキ 
\\	行けそうにない	=行ける可能性は低い 
\\	そうにない 
\\	そうもない 来(き)そうにない
\\	申し込みに際して	硬い =申し込む時 
\\	に 際して 
\\	るに 際し     際しては     際しての
\\	=にあたり
\\	計画に基づいて	=計画の通りに、
\\	に 基づいて    基づき    基づく
\\	基づく
\\	基づいた
\\	必要に応じて	=必要に合わせて、
\\	に応じて   応じ   応じた
\\	青空の下で	のモトで =青空の下(した)で、
\\	の 下で    下    下に
\\	全身ずぶ濡れ	
\\	ヌレ
\\	人間工学	コウガク 
\\	日本生きて以来	=日本に来てからずっと、
\\	て  以来 それ あれ
\\	試験を受ける以上	=受けるのだから, 
\\	=上は 
\\	る    以上 
\\	である  以上は 当然という気持ち。
\\	約束したからには	=約束したのだから 
\\	るからには 
\\	た =以上、上は 当然という気持ち。
\\	来日の折には  硬い	オリ =来日した時には 
\\	る 
\\	た  折(に) 
\\	の  折(に)は
\\	客に立場から言うと	=客として意見を言えば、客の方から見て言えば 
\\	から 言うと    言えば 
\\	人から言うと 
\\	の考えから、その点から。。。
\\	症状からすると	
\\	からすると    すれば    いって
\\	服装からして	=まず、服装だけ見ても, 
\\	からして
\\	外国人から見ると	
\\	から 見ると     見れば     見て     見ても 人から見ると 
\\	よほど
\\	よっぽど	
\\	2。よっぽどー
\\	よっぽど国に帰ろうかと思った
\\	好きだからといって	
\\	。。。ない 
\\	からといって、からって
\\	手続きしてからでないと	
\\	てからで ないと〜できない       なければ〜できない
\\	2007年から2009年にかけて	=2007年から2009年までの間 
\\	1から
\\	2にかけて 
\\	2=時/場所
\\	足の速さにかけては	
\\	足が速いという点では 
\\	に 
\\	の
\\	い→
\\	さに かけては 
\\	の
\\	さに   かけても
\\	中止だとか	=だと聞いた 
\\	とか 「そうだ」より「はっきりわからないが」という意味を含む
\\	二度と行くまい  硬い	=行かないつもりだ 
\\	一段るまい・
\\	2まい 
\\	五段るまい するまい、すまい 来るまい、こまい 例 決して。。まい   何も。。まい
\\	わかるまい 硬い	=わからないだろう、
\\	一段るまい・
\\	2まい 
\\	五段るまい するまい、すまい 来るまい、こまい 夢ではあるまいか 
\\	夢ではないだろうか
\\	話そうか話すまいか 硬い	=話すか話さないか(迷ってる。。) 
\\	ようか
\\	るまいか  
\\	一段るまい・
\\	2まい  
\\	五段るまい  するまい、すまい  来るまい、こまい)
\\	入梅 梅雨入り	にゅうばい つゆいり 
\\	負けるに決まっいてる	=きっと勝つ 
\\	に 決まっている 
\\	に   違いない    
\\	に   相違ない(硬い)
\\	勝つとは限らない	=勝つとは決まっていない 
\\	とは限らない
\\	祈る(より)ほかない 硬い	
\\	る   ほかない 
\\	るより ほかはない 
\\	るしかない
\\	努力の結果にほかならない 硬い	
\\	に ほかならない
\\	中国をはじめアジアの国々が	=まず中国、そして 
\\	を はじめ    はじめとして    はじめとする
\\	憲法改正をめぐって	=憲法改正に関して、
\\	を めぐって    めぐる
\\	京都において  硬い	=京都で 
\\	に おいて      おいては    おいても    おける
\\	=での
\\	が状況の時は「ついて」の意味
\\	主張	
\\	現地にて	=で 
\\	にて お知らせや通信文アドでよく使われる 例:京都にて 芽木より ←旅先からのハガキの最後などでよく使う
\\	人目も構わず	
\\	(人目=
\\	も 
\\	るのも かまわず
\\	雨にもかかわらず	=雨なのに, 
\\	にも 
\\	にも  かかわらず
\\	来る来ないにかかわらず	
\\	、来る場合もこない場合も 
\\	る
\\	ない に かかわらず 
\\	い
\\	くない に 関わりなく 
\\	に あるなしに関わらず 好き嫌いにかかわらず 上手下手に関わらず 性別に関わらず
\\	年齢を問わず	
\\	を問わず 男女、性別、〜の有無、国の内外、季節、昼夜 性別は問いません。
\\	本やらノートやら	=本やノートや色々・
\\	し
\\	し 
\\	やら
\\	やら」 
\\	やら 
\\	3やら
\\	見るにつけ聞くにつけ	=写真を見ても、歌を聴いても。。。 
\\	につけ
\\	につけ)」 
\\	るに 
\\	いに つけ 何かにつけ=何かあるたびに
\\	行くにしろ行かないにしろ	=行く場合も、行かない場合でも、
\\	にしろ
\\	にしろ)」 
\\	にせよ
\\	にせよ)」 
\\	る  
\\	い  にせよ 
\\	に にしろ 
\\	に
\\	勉強もできればスポーツもできる	=勉強もできるし、スポーツもできる 
\\	1も
\\	ば
\\	2も
\\	る 
\\	1も
\\	ければ
\\	2も
\\	い 
\\	1も
\\	なら
\\	2も
\\	だ 
\\	1も
\\	なら
\\	2も
\\	だ
\\	薬は苦いものだ・行ってみたいものだ	
\\	=苦いのが普通だ 
\\	る・
\\	ない  ものだ 
\\	い・
\\	くない もんだ(話) 
\\	な・
\\	じゃない 一般的傾向 
\\	=行ってみたいなあ 
\\	たい・
\\	たくない ものだ 
\\	てほしい・
\\	て欲しくない もんだ(話) 希望
\\	するものではない	=すベきではない、
\\	る ものではない    もんじゃない(話) アドバイス
\\	無理というものだ	=普通に考えれば無理だ 
\\	る というものだ 
\\	というもんだ(話) 
\\	行くものか	=絶対行かない, 
\\	る ものか 
\\	い もんか 
\\	な ものですか 
\\	な もんですか 強い否定 決して〜ものか 絶対〜ものか 何も〜ものか
\\	くたくた	
\\	東京を中心に	=特に東京、そして、
\\	を 中心に    中心にして    中心にした
\\	中心として    中心とした
\\	感謝の気持ちを込めて	=気持ちを入れて、と一緒に 
\\	を込めて 思い、恨み、感謝、祈り、願い。。。 心を込める→心がこもる→心のこもった手紙
\\	友人を通じて	=友人を通して、友人によって 
\\	を通じて (ユウジテ)  を通して (トオシテ) 
\\	が時期なら、「〜中」
\\	地図を頼りに	
\\	を頼りに
\\	頼りとして   頼りにして
\\	杖	ツエ 
\\	行事	ギョウジ 
\\	倒れたりする恐れがある 硬い	=心配がある 
\\	る 
\\	の 恐れがある
\\	辛いものがある 硬い	=辛く感じられる、
\\	ように感じられる 
\\	る 
\\	い 
\\	な ものがある
\\	多ければいいというものでもない 硬い	=多ければいいとは言えない、
\\	ば 
\\	ければ 
\\	いという 
\\	なら 
\\	(だ)という ものでもない 
\\	なら 
\\	(だ)という ものではない
\\	どうにかならないものか。  硬い	
\\	どうにかなるだろうか、どうにかしたい (どうにか) (なんとか) 
\\	ないもの(だろう)か (もう少し) 
\\	れないもの(だろう)か 例:上手にできないものか、直せないものか、できないものか
\\	引きつける	
\\	事実をもとに	=事実に基づいて、
\\	を もとに   もとにして
\\	調整中につき 硬い	
\\	のため、
\\	なので 
\\	につき 書面
\\	や改まった場面でよく使われる
\\	入学をきっかけに	=大学入学の機会に、のためその時から、
\\	を   きっかけに 
\\	るのを きっかけとして 
\\	たのを きっかけにして      契機に(けいき)硬い      契機として   硬い      契機にして   硬い
\\	受験の際に  硬い	=の時に 
\\	の   際に 
\\	る・た 際 「時」「〜にあたり」の硬い表現
\\	出産	
\\	の最中	
\\	高かった。それなのにすぐ壊れた。	=高かったのに・ところが 
\\	<事実>。それなのに
\\	<予想外の結果>
\\	大雨だ。それでも出かける。	=大雨だが・だけど、けれど、しかし 
\\	<状況>。それでも
\\	に反する意見・結果>
\\	渋滞	ジュウタイ 
\\	「渋滞だとか。」「それなら電車で行こう。」	=渋滞しているなら・じゃあ、それじゃあ話 
\\	。」「それなら(ば)
\\	に基づく判断・意見>
\\	働きすぎた。それで病気になった。	1。=働きすぎたために・だから、そのために 
\\	<理由>。それで
\\	<結果> 2。「今日面接を受けたんだ。」「で、どうだった?」 =それからどうした? 
\\	。」「それで・で話
\\	の結果を尋ねる>」 相手の話を聞き出すときに使う。
\\	聞き出す	
\\	それがまだなんです。	
\\	=そう思うでしょうが... 
\\	。」「それが
\\	と反する内容>」 予想外の結果を言うときに使う。 
\\	タバコをやめるを誓った。それが、たった三日でまだ吸ってしまった。 =やめると誓ったのに・ところが・それなのに 
\\	<予定・予想>。それが
\\	と異なる結果>
\\	渋滞するらしい。そこで早く出発したい。	=そのことから判断して・そのことに関係して・そう言うわけで ー
\\	<事実の説明>。そこで
\\	<解決方法・提案・依頼> 話を展開するときに使い
\\	展開	
\\	そういえば田中くん、元気?	=今思い出したけれど 
\\	。そういえば
\\	<思い出した関連する話題>
\\	設計士さん	
\\	それはそうと試験はいつだっけ?	=話は変わりますが、
\\	。それはそうと
\\	と関係ない話> 話題を換えるときに使う
\\	母の兄、すなわちおじさん 硬い	=別の言い方では、
\\	つまりと似てるが、結論を言うときすなわちは使えない。
\\	ファックスあるいはメールで 硬い	=ファックスかメールで、「または」の硬い表現 2つのどちらかを選ぶ
\\	貧しい。だが幸せだ。 硬い	=貧しいけれども、しかしの硬い表現
\\	挑戦	チョウセン 
\\	怒ってる。だって約束を破ったから。 話	=怒ってるのは、約束を破ったからだ。 なぜなら、なぜかと言うと、どうしてかと言うとの会話的表現。 だって〜だもの だって〜だもん
\\	まだ来ない。ということは欠席だ。	
\\	、つまり 
\\	と言うことは
\\	から判断した内容>
\\	外出できない。というのは父の具合が悪いんです。	=家を出られない理由は、なぜかというと、どうしてかというと 
\\	というのは
\\	<理由>
\\	彼は真面目だ。したがって信頼されている。 硬い	
\\	、だから、その結果 
\\	<理由>。
\\	<結果>
\\	3割引。ただしこの棚の商品を除く。 硬い	前の文に条件や例外を付け加えるときに使う
\\	シナはいい。ただ値段が高すぎる。	前の文に問題点や上限を付け加えるときに使う
\\	飲食禁止。もっとも水は構いません。	=ただし 前の文に例外や条件を付け加えて修正するときに使う。
\\	以上です。なお詳細は…	話を付け加える時に使う。前の文と関係のないことでもいい。
\\	終わります。さて来週は…	=話は換わって 新しい話題に換えるときに使う。
\\	1。薬を塗った。すると痛みが治まった。 2。「外出していました。」「すると家に誰もいなかったんですね。」	1。=薬の塗って、その後すぐに   
\\	<きっかけ>。すると
\\	<起こったこと> 2。=そこから判断すると・ということは   
\\	。すると
\\	から判断した内容>
\\	蛾	ガ 
\\	大敗する	タイハイ 
\\	大敗した。要するに力の差があった。	=つまり・まとめて言うと, 
\\	前の文の内容を簡単に説明すると。。。
\\	美人だ。しかも性格もいい。	=その上・さらに, 
\\	高いし、まずい。(/し、)おまけにサービスも。。。話	=それに加えて、「その上・しかも」 「しかも」より話し言葉的。
\\	一般に…。ちなみにうちにも…。	=ついでに言えば、
\\	前の文の説明に軽い情報を付け加えるときに使う。
\\	義理の父	
\\	明くる	あくる 
\\	あくる日 
\\	日本流自己流 我流	
\\	りゅう=
\\	のやり方 
\\	がりゅう
\\	しんしん	
\\	土砂降り	どしゃぶり 
\\	土砂降りの雨
\\	俗に	ぞく 
\\	労働	
\\	きつい・汚い・危険) 
\\	親になってこそ	=親になってから初めて 
\\	てこそ=
\\	て初めて 過去のことには使えないー
\\	苦労がわかる。
\\	苦労こそあれ   硬い	=苦労はあるけれど、
\\	こそ   あれ 
\\	でこそ あるが
\\	感謝こそすれ 硬い	=感謝はしているけれど、絶対に…ない、
\\	2こそすれ 
\\	こそすれ 後文が「絶対〜ではない」と言うことの強調
\\	見た目こそ悪いが…	=悪いのは見た目だけで 
\\	こそ 〜が・けれど… 強調
\\	恨む	ウラム 
\\	人を
\end{CJK}
\end{document}