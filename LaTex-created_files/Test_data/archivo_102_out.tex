\documentclass[8pt]{extreport} 
\usepackage{hyperref}
\usepackage{CJKutf8}
\begin{document}
\begin{CJK}{UTF8}{min}
\\	社会進出、 政界へ進出する		進出(しんしゅつ)
\\	しんしゅつ【進出】名詞 
\\	その会社は海外進出をねらっている⦅海外へ事業を広げようとしている⦆ 
\\	進出する動詞 
\\	決勝に進出する 
\\	海外市場に進出する 
\\	政界へ進出する⦅入る⦆ 
\\	その会社は電子産業に進出した⦅手を広げた⦆ 
\\	せいかい【政界】名詞 【政治の世界】
\\	【政治生活】
\\	【政治活動】
\\	(!単複両扱い) 
\\	政界に入る 
\\	政界を退く 
\\	政界の形容詞 
\\	政界のドン 
\\	経済成長		経済(けいざい) 成長(せいちょう) 
\\	差別(する)		差別(さべつ) 
\\	少子化		少子化(しょうしか) 
\\	少子化対策を講じる 
\\	急速な少子化 
\\	高齢化や少子化に対処する 
\\	少子化社会 
\\	高齢化社会		高齢 
\\	高齢化(こうれいか)
\\	専業主婦		専業(せんぎょう)
\\	主婦(しゅふ)
\\	共働き		共働き(ともばたらき) 
\\	(ふたん)家事の負担		家事(かじ)
\\	負担(ふたん)
\\	ふたん【負担】名詞 〖過重な義務責任〗⦅やや書⦆
\\	〖責任罪などの重圧〗
\\	〖責任〗
\\	納税者の税負担 
\\	負担をかける 
\\	彼の負担を軽く[重く; 軽減]する 
\\	来る日も来る日も病人の世話をするのは相当な負担だ 
\\	彼の負担が重すぎる 
\\	その仕事をそんなに負担に感じなくてもいい 
\\	負担する動詞 〖支払う〗
\\	〖義務として引き受ける〗
\\	だれが君の旅行費用を負担するのですか 
\\	会社がその事故の損害(賠償金)を負担した 
\\	子育て		子育て(こそだて) 
\\	遠足		遠足(えんそく) 
\\	とおい【遠い】 
\\	【距離】 
\\	えんいん【遠因】 【遠い原因】
\\	【間接的な原因】
\\	【底流にある原因】
\\	とおえん【遠縁】 
\\	彼女は私の遠縁に当たる 
\\	―そく【―足】 
\\	新しい靴1足 
\\	(!前の方が普通) 
\\	ストッキング5足 
\\	そくせき【足跡】 〖足あと〗
\\	(!いずれも複数形で
\\	には「業績」などの意の比喩的用法はない (⇨足跡(あしあと))) 
\\	〖業績〗
\\	【貢献】
\\	彼の足跡を振り返る 
\\	ダーウィンは生物学史に偉大な足跡を残した 
\\	彼は植村直己の最後の足跡をたどるためにマッキンリーに向けて出発した 
\\	足跡をしるす 
\\	芭蕉は東北各地に足跡をしるした⦅訪れた⦆ 
\\	給料		給料(きゅうりょう) 
\\	きゅうりょう【給料】 
\\	使い分け
\\	給与給料賃金を表す最も一般的な語
\\	固定給制の月給年俸などを表す
\\	肉体労働等に対する日給週給を表す
\\	また,広義で各種の労働の対価としての賃金を表す
\\	給料の 
\\	給料のいい仕事 
\\	給料の前借りを頼んだ 
\\	(⇨前借り) 
\\	いやな仕事だがこれも給料のうちだと思ってがまんする 
\\	(!最後に 
\\	を入れると「他の仕事もあるがこれも仕事の一部である」という意味になり,「給料のうち」というニュアンスが出る) 
\\	給料が[は] 
\\	給料が安いとぼやく 
\\	あの会社は給料がいい 
\\	今年給料が1万円上がった 
\\	⦅米⦆ 
\\	⦅英⦆] 
\\	(⇨昇給) 
\\	我々の給料は安い⦅薄給だ⦆ 
\\	看護師の給料はもっと高くあるべきだ 
\\	給料はどのくらいお望みですか 
\\	会話 
\\	「給料はいくらですか」「月20万円です」 
\\	(⇨月給) 
\\	給料を 
\\	給料を払う 
\\	給料を家に入れる ⦅話⦆
\\	彼は金曜日に給料をもらう 
\\	(⇨給料日) 
\\	私は何とかやっていける程度の給料をもらっています 
\\	給料で 
\\	この給料で私たちは十分な暮らしをしている 
\\	この給料では生計を立てるのに不十分です⦅やっていけない⦆ 
\\	給料日 
\\	(!通例無冠詞) (!米英では週給制が一般的で通例金曜日
\\	⦅米⦆ では通例給料支払い小切手 
\\	で支払われる) 給料袋 
\\	[⦅英⦆ 
\\	給料明細(書) 
\\	〜(の)くせに		
\\	残業(する)		残業する(ざんぎょうする) 
\\	のこす【残す】 
\\	ざんえい【残映】 【夕映え,夕焼け】 
\\	ざんがい【残骸】 
\\	【乗り物建物の】
\\	⦅集合的⦆ 
\\	【廃墟】(はいきょ)
\\	飛行機の残骸 
\\	ざんがく【残額】 【差引残高】
\\	【残った金額】
\\	〖借金の未払い分〗
\\	ざんぎょう【残業】名詞 
\\	サービス残業 
\\	残業する動詞 
\\	【会社で遅くまで働く】
\\	2時間残業する 
\\	彼は毎月相当量の残業をする 
\\	残業時間 
\\	残業手当 
\\	ざんだか【残高】 〖収支の差額〗
\\	銀行預金残高 
\\	繰越残高 
\\	残高照会 
\\	残高表 【貸借対照表】
\\	ざんきん【残金】 〖残った金〗
\\	とはいわない) 
\\	〖差引残高〗
\\	残金を支払う 
\\	[(残り) 
\\	セカセカ		
\\	せかせか (副)スル 
\\	動作態度が忙しそうで落ち着きのないさま。「―した話し方」「―(と)歩く」 
\\	することが小さいさま。こせこせ。「―と置かうよりも一度にどうと置いて使はふと思ふ」〈狂言鼻取相撲〉 せかせか 
\\	せかせかした 【落ち着きのない】
\\	【忙しい】
\\	せかせかする⦅忙しくする⦆ 
\\	せかせかした人 
\\	せかせかした生活をする 
\\	彼はいつもせかせかと⦅急いで⦆部屋に入ってくる 
\\	お客を迎えるときは母はよく台所でせかせか立ち働く 
\\	彼は、いつもせかせかしています。 
\\	いつもせかせかと歩いている 
\\	せかせかしている東京人 
\\	地域		地域(ちいき) 
\\	つうねん通念		通念(つうねん) 
\\	社会通念 
\\	つうじる【通じる】 
\\	【道路交通機関などが】 【道路などが】
\\	【乗り物が運行する】
\\	【ドア部屋などが】
\\	【鉄道などが敷設される】
\\	【開通する】
\\	【電話電流などが】 【電話で連絡をつける】
\\	(!前の方が口語的) 
\\	【電力を供給する】
\\	【了解される】 【理解される】
\\	[⦅書⦆ 
\\	【自分の意思を人に分からせる】
\\	【意思を通じ合わせる】
\\	【考えなどを理解させる】
\\	【言葉などが理解できる】
\\	【精通する】 【熟知している】
\\	(⇨精通する)【見聞知識が広い】⦅やや書⦆
\\	【熟達している】⦅やや書⦆
\\	【その他の表現】 
\\	これは教員全体に通じる⦅共通する⦆問題だ 
\\	手抜き		手抜き(てぬき) 
\\	てぬき【手抜き】名詞 
\\	手抜き工事をする 
\\	手抜きする動詞 【仕事をぞんざいにする】
\\	【経費時間手間を切り詰める】⦅話⦆
\\	ぬき【抜き】 
\\	【…を除くこと】 
\\	…抜きで 
\\	朝食抜きで 
\\	そんな難しい話は抜きにして⦅脇に置いて⦆ 
\\	カフェイン抜きのコーヒーでもけっこういける 
\\	ともいう) 
\\	【…人連続で負かすこと】 
\\	このトーナメントでは6人抜きすれば優勝です 
\\	熱を出す		熱(ねつ) 出す(だす) 
\\	ムリ(な)		
\\	やくわり役割		役割(やくわり) 
\\	あせり		
\\	あせり【焦り】 【じりじりする気持ち】
\\	【急ぐこと】
\\	1点を取ろうとするフランスに焦りが見える 
\\	何をするにせよ焦りは禁物だ 
\\	あせり3【焦り・〈焦燥〉り】 あせること。気がいらだつこと。「敵に―の色が見えてきた」「相手の―を誘う」 あせる【焦る】 〖急ぐ〗
\\	(!状態をいうときは 
\\	〖待ち切れない〗⦅書⦆
\\	焦らない 
\\	【冷静を保つ】
\\	成功を焦る⦅熱望する⦆ 
\\	焦らずゆっくりやりなさい 
\\	彼は試験が近いので焦っていた⦅いらいらしていた⦆ 
\\	圧倒的(な)		圧倒的(あっとうてき) 
\\	あっとうてき―たう―【圧倒的】 (形動) 比べものにならないほど,他より優勢であるさま。「―な勝利をおさめた」(優勢 ゆうせい 
\\	あっとう【圧倒】 圧倒的(な)形容詞 
\\	【強烈な】
\\	圧倒的(な)多数で勝つ 
\\	圧倒的な勝利 (⇨圧勝) 
\\	チャンピオンの圧倒的(な)強さには敬服するばかりだ 
\\	圧倒的に副詞 
\\	彼は圧倒的に強い 
\\	出席者は若い女性が圧倒的に多かった⦅大部分若い女性だった⦆ 
\\	圧倒する動詞 〖断然優位な力[数]で〗
\\	〖より強い力で〗
\\	私たちは数において敵を圧倒した 
\\	【敵より断然優位に立った】
\\	圧倒的多数で勝利する 
\\	圧倒的多数で可決される 
\\	(法案などが) 
\\	圧倒的な貧富(ひんぷ)の差 
\\	圧倒的な軍事力(ぐんじりょく) 
\\	圧倒的な軍事力の優位を維持する 
\\	圧倒的な軍事力を持ち続ける 
\\	圧倒的な軍事力を持つ 
\\	圧倒的な軍事力を有する 
\\	あっしょう【圧勝】名詞 
\\	【選挙での】
\\	圧勝の試合 
\\	圧勝する動詞 
\\	彼はその試合で圧勝した 
\\	⦅やや話⦆
\\	【相手をこてんぱんに負かした】
\\	ジャイアンツはタイガースに10対1で圧勝した 
\\	共和党が圧勝した 
\\	あっぱく【圧迫】名詞 【圧力】
\\	【重圧】
\\	【抑圧】
\\	彼女は胸に圧迫を感じた 【だれかに押されたりして】
\\	だと「彼女は胸に不快な圧迫感があった」の意) 圧迫する動詞 【しいたげる】
\\	【抑圧する】
\\	暴君は人民を圧迫した 
\\	政府は言論の自由を圧迫した 
\\	インフレで家計が圧迫された⦅家計を悪くした⦆ 
\\	たおす【倒す】 
\\	【転倒させる】 
\\	【投げ倒す】
\\	【当たって倒す】
\\	【切り倒す】
\\	【取り壊す】
\\	【破壊する】
\\	【引っくり返す】
\\	(倒れる
\\	彼を投げ倒す 
\\	相手を殴り倒す 
\\	古家を倒す⦅壊す⦆ 
\\	座席を倒す 
\\	彼をアッパーカットで倒した 
\\	電気スタンドを倒さない⦅当たってひっくり返さない⦆ように気をつけなさい 
\\	【負かす】 
\\	(!後の方が堅い語) 
\\	【転覆させる】
\\	【殺す】
\\	そのチームを倒す 
\\	政府を倒す 
\\	暴君を倒す 【失脚させる】
\\	【殺す】
\\	[(意図的に) 
\\	(暗殺する) 
\\	独裁者を倒せ 【…なんかいらない】
\\	は「…を倒せ」の意の慣用表現) とうかい【倒壊】名詞 
\\	倒壊寸前である 
\\	倒壊する動詞 【つぶれる】
\\	【倒れる】
\\	【破壊される】
\\	地震でそのビルが倒壊した 
\\	倒壊家屋 
\\	とうりつ【倒立】名詞 
\\	【頭と両手をつけてする倒立】
\\	倒立する動詞 
\\	とうかく【倒閣】名詞 
\\	倒閣する動詞 
\\	とうさく【倒錯】 ⦅やや書⦆
\\	倒錯した愛情 
\\	性的倒錯者 
\\	とうさん【倒産】名詞 
\\	【破綻】
\\	倒産寸前に追い込まれる 
\\	中小企業の倒産が相次いだ 
\\	は不可)
\\	その国の経済は安定している		安定(あんてい) 
\\	あんてい【安定】名詞 
\\	【安定させること】
\\	【足場地位などの】
\\	〖均衡〗
\\	~(の)安定 
\\	経済[社会]の安定 
\\	物価の安定 
\\	ドルの安定 
\\	安定を 
\\	片足で立って安定を保つ[失う] 
\\	政治の安定を維持する 
\\	情緒の安定を欠く 
\\	安定した形容詞 
\\	安定した仕事[収入] 
\\	安定したはしご 
\\	安定した⦅釣り合いの取れた⦆心 
\\	安定する動詞 
\\	その国の経済は安定している 
\\	患者の容態は安定した[している] 
\\	この2年間物価は安定している 
\\	そのうち天候も安定するでしょう 
\\	安定させる動詞 
\\	物価を安定させる 
\\	テーブルの脚を安定させる 
\\	安定感 
\\	安定成長 
\\	安定多数 
\\	安定賃金 
\\	安定通貨 
\\	いきいきと		
\\	いきいき【生き生き】 【元気な】
\\	【新鮮な】
\\	【鮮やかな】
\\	彼女は生き生きとした表情をしている 
\\	雨でアジサイの葉が生き生きしている 
\\	その本の人物描写は生き生きとして真に迫っている 
\\	彼は子供にお話をしてやっているとき実に生き生きとしている⦅彼の最高の状態にある⦆ 
\\	この町が生き生きしてくる⦅活気づく⦆のは暗くなってからです 
\\	意識(する)		意識(いしき) 
\\	いしき【意識】名詞 
\\	(!事実真実状態などを内心で自覚すること) 
\\	(!見たり, 聞いたり, 感じたりなどして五感で気づくこと) 
\\	(!五官による体の感覚. この意味では複数形) 
\\	意識不明の 
\\	~(の)意識 
\\	民族[政治; 階級]意識 
\\	強い共同体[対抗; 危機]意識 
\\	被害者意識をつのらせる 
\\	彼には罪の意識がない 
\\	は道徳上宗教上の罪) 
\\	彼を動乱のイラクに駆り立てたのはジャーナリストとしてのプロ意識であった 
\\	彼は私に激しい競争意識を抱いているようだ 
\\	意識が 
\\	その患者は最後まで意識があった 
\\	その国では女性の間で人権意識が高まってきた 
\\	意識を 
\\	意識を失う 
\\	⦅やや話⦆ 
\\	意識を回復する 
\\	環境問題に対する意識を高める 
\\	彼女は数分間意識を失って横になっていた 
\\	意識する動詞 
\\	自分の短所を意識する 
\\	彼女はその男が自分をうっとり見つめていることを意識していた 
\\	順調にいっているときは自分ではそれを意識しない人もいる 
\\	生き方		
\\	至る		至る(いたる) 
\\	いたる【至る・到る】 (動ラ五[四]) 
\\	その場所に行き着く。到達する。「この道は京都を経て大阪に―る」「ここから頂上に―るまでの間には岩場が二か所もある」 
\\	その時期時刻になる。「会議は紛糾し,深夜に―っても結論が出ない」「四月から八月に―る五か月間」「先月家を出たまま,今に―るまで連絡がない」 
\\	その段階状態になる。「大事に―らぬうちに火事を消し止める」「事ここに―ってはもう手の打ちようがない」 
\\	極端な例であることを示す。(「…から…にいたるまで」の形で)両端のものを挙げて,すべてのものの意を表す。「社長から新人社員に―るまで全社こぞって」「挨拶の仕方から箸の上げ下ろしに―るまで口うるさく注意する」(「…にいたっては」の形で)前に示したものよりももっと極端な例を示す。「その日は遅刻する人が多く,
\\	君に―っては一時間も遅れて来た」 
\\	注意が十分に行き渡る。行き届く。「爰(ここ)は―つた茶屋ぢやぞや」〈浮世草子風流曲三味線2〉 →いたらぬ 
\\	到来する。「好機―れり」「悲喜こもごも―る」「万感こもごも―る」 
\\	程度が高くなる。最高の段階になる。「これは徳―りたる翁(おきな)どもにて候」〈大鏡昔物語〉高い地位に達する。「京上りして,大納言に―る」〈宇治拾遺物語1〉 
\\	それのもたらした結果である。「若気の―る所にや,かぶりの板におしあてて」〈狂言文蔵〉〔「いたす」に対する自動詞〕 いたる【至る】 
\\	【通じる】 
\\	山頂に至る道 
\\	この道は大津を経て京都に至る 
\\	【ある結果になる】 
\\	【結局…になる】
\\	に続く名詞動名詞は通例好ましくない事態を示す) 
\\	【…するようになる】
\\	〖引き起こす〗【原因となる】
\\	【結果としてもたらす】
\\	〖結論などに達する〗
\\	その実験でがんの治療法を発見するに至った 
\\	その国境紛争は戦争には至らなかった 
\\	[(発展する) 
\\	どうして君はそんなばかなことをするに至ったのか 
\\	(…する気にさせる) 
\\	を用いるのが最も口語的) 
\\	その問題はまだ解決に至っていない 
\\	決定には至らなかった 
\\	【及ぶ】 
\\	社長から用務員に至るまで 
\\	今に至るまで 
\\	【今日まで】
\\	一段落(する)		一段落(いちだんらく) 
\\	異文化		異文化(いぶんか) 
\\	運良く		運良く(うんよく) 
\\	影響(する)		影響する(えいきょう) 
\\	追いかける		追いかける(おいかける) 
\\	お互い		お互い(おたがい) 
\\	海外		
\\	学生運動		学生運動(がくせいうんどう) 
\\	価値		価値(かち) 
\\	活躍(する)		活躍(かつやく) 
\\	がむしゃらに		
\\	関心		関心(かんしん) 
\\	感動する		感動する(かんどうする) 
\\	〜期		―き【―期】 〖時代〗
\\	〖期間〗
\\	【期限つきの】
\\	〖会期〗
\\	〖段階〗
\\	氷河期 
\\	市長を3期勤める 
\\	彼は公職を2期勤めた 
\\	この法案は今期は間に合わない 
\\	きかん【期間】 
\\	【契約任期などの】
\\	契約期間 
\\	休暇期間中に 
\\	短期間で英語をマスターする 
\\	そこでの私たちの滞在期間は1週間です 
\\	彼は長期間入院していた 
\\	(⦅米⦆ 
\\	この時計の保証期間は1年です 
\\	きげん【期限】 【指定期間】
\\	【契約などの】
\\	【期間】
\\	【締め切りの時】
\\	(⇨締め切り) 
\\	~期限,期限~ 
\\	期限付きの 
\\	この切符の有効期限は2日だ 
\\	[⦅書⦆ 
\\	(!「有効期限二日の切符」は 
\\	納税期限まであと10日しかない 
\\	期限が[は] 
\\	契約の期限が切れた 
\\	[⦅やや書⦆ 
\\	手形の期限が5月19日に来る 
\\	会話 
\\	「その請求書の支払い期限はいつですか」「5月10日です」 
\\	期限に[を] 
\\	期限に間に合う[合わない] 
\\	期限を延ばす[決める] 
\\	⦅やや話⦆ 
\\	図書館から借りた本は返却期限を3日間過ぎている 
\\	(!「返却期限切れの本」は 
\\	きたい【期待】名詞 
\\	(!しばしば複数形で) 
\\	〖希望〗
\\	〖見込み〗
\\	〖予期〗
\\	期待~ 
\\	その映画は期待はずれだった 
\\	彼は親の期待どおりには勉強しなかった 
\\	節内は通例過去完了形だが, 口語ではしばしば過去形を用いる) 
\\	試験の結果は期待以上だった 
\\	(!前の方は堅い言い方) 
\\	期待が[は] 
\\	彼の生存はあまり期待が持てない[期待薄だ] 
\\	[⦅米⦆ 
\\	私の期待ははずれた 
\\	そのことを知ると息子に対する期待が高まった 
\\	期待に 
\\	彼は我々の期待にこたえてくれた 
\\	私の期待に反して彼は失敗した 
\\	私たちは期待にわくわくした[胸を躍らせた] 
\\	新しい世界に第一歩を踏み出したとき,彼女は期待に胸をふくらませていた 
\\	期待を 
\\	両親は私に期待をかけすぎる 
\\	[⦅書⦆ 
\\	は不可) 
\\	信じてくれ, 今度は君の期待を裏切る⦅がっかりさせる⦆ようなことはしないよ 
\\	会話 
\\	「例のパーティーではスーパーマンの扮装をするつもりなんだ. 君はどうするの」「乞うご期待」 
\\	期待する動詞 
\\	【今に起こると】
\\	〖希望する〗
\\	〖楽しみにする〗
\\	(!しばしば進行形で) 
\\	〖当てにする〗
\\	使い分け 
\\	根拠に基づく確かな予想を表す
\\	よいことへの期待だけでなく,悪いことの予測も含意する
\\	現実的な根拠に基づく可能性のある希望を表す. 
\\	差し迫った事柄を予期し,心構えをすることを表す
\\	歓迎不安の両方の文脈で用いる
\\	あなたが手伝ってくれると期待しています 
\\	(!口語では通例 
\\	を省略する. 
\\	の方が相手に強いる気持ちが強い) 
\\	節を従える場合より期待の度合いが強い) 
\\	彼が勝つものと期待された 
\\	⦅書⦆ 
\\	その本は期待していたよりずっといい 
\\	は主節が現在形であっても 
\\	にはならない) (⇨名①) 
\\	彼は昇給を期待して新車を買った 
\\	その歌手はアンコールを3回受けることを期待していた 
\\	あなたからの便りを期待しています 
\\	期待感 
\\	きじつ【期日】 【定められた日】
\\	【約束の日】
\\	【締切日】
\\	(⇨期限) 
\\	期日を定める 
\\	期日に間に合わす 
\\	約束の期日までに仕事を仕上げた 
\\	きまつ【期末】 
\\	期末に 
\\	期末のレポート 
\\	[⦅英⦆ 
\\	期末試験 
\\	[⦅英⦆ 
\\	⦅まれ⦆ 
\\	【大学の最終試験】
\\	(!通例複数形で) 
\\	危機感		危機感(ききかん) 
\\	企業		企業(きぎょう) 
\\	犠牲		犠牲(ぎせい) 
\\	ぎせい【犠牲】 
\\	【被害者】
\\	犠牲に 
\\	スポーツのために勉学を犠牲にする 
\\	健康を犠牲にして仕事を成し遂げる 
\\	彼女は子供のために自分を犠牲にした 
\\	多くの人が阪神大震災の犠牲になった 
\\	は「他人のために自分の命を犠牲する」で不可) 
\\	彼はその病気の犠牲になった 
\\	犠牲を 
\\	多大の犠牲を払って得た勝利 
\\	親は子供の教育のためには多くの犠牲を払うものだ 
\\	我々はどんな犠牲を払っても行方不明の少女を捜さねばならない 
\\	彼は多大の犠牲を払ってやっと成功した 
\\	私たちはその若者の払った多くの犠牲に報いなければならない 
\\	犠牲者 (⇨犠牲者) 犠牲的精神 
\\	きっかけ		
\\	基本		基本(きほん) 
\\	逆に		逆に(ぎゃくに) 
\\	(きゃっかんてき) 客観的(な)		客観的(きゃっかんてき) 
\\	協力的(な)		協力的(きょうりょくてき) 
\\	切り捨てる		切り捨てる(きりすてる) 
\\	偶然		偶然(ぐうぜん) 
\\	具体的(な)		具体的(ぐたいてき) 
\\	口癖		口癖(くちぐせ) 
\\	景気		景気(けいき) 
\\	経験する		経験(けいけん) 
\\	経済		経済(けいざい) 
\\	経歴		経歴(けいれき) 
\\	研究する		研究する(けんきゅうする) 
\\	こうえん講演する		講演する(こうえんする) 
\\	航空会社		航空 
\\	航空会社(こうくうがいしゃ) 
\\	その地域の経済は深刻な景気後退状態にあります。		"後退する(こうたいする) 
\\	こうたい【後退】名詞 【退却】
\\	【撤退】
\\	(⇨退却, ⇨撤退)〖景気の〗
\\	後退する動詞 
\\	【軍隊などが】
\\	【景気が】
\\	後退させる 【車を】
\\	これまでの政府の姿勢(しせい)から後退する 
\\	景気後退(けいきこうたい) 
\\	その国は、今年度の1月から6月までの期間、自律的景気後退の局面に入った。 
\\	その地域の経済は深刻な景気後退状態にあります。 
\\	それは国が景気後退から抜(ぬ)け出しているかもしれないという印である。 
\\	しりぞく【退く】 
\\	【後ろへ下がる】 
\\	【1歩】
\\	【兵が】
\\	【不利になって】
\\	彼女は蛇を見て退いた 
\\	軍隊は前線から退いた⦅撤退した⦆ 
\\	【職などを】 【通例定年で】
\\	【辞職する】
\\	(!後の方が口語的) 
\\	公の生活から退く 
\\	教授の職を退く 
\\	代打を送られて退く 
\\	どける【退ける】 【他の場所に移す】
\\	【現在地から移す】
\\	【邪魔にならないようにする】
\\	庭からその石をどける⦅取り除く⦆ 
\\	雪をどける 
\\	は「(好ましくないものを)排除する」の意) 
\\	そのオートバイをどけてくれ 
\\	たいい【退位】名詞 
\\	退位する動詞 
\\	王は王子に位を譲るため退位した 
\\	たいかい【退会】名詞 
\\	退会届を出す 
\\	退会する動詞 
\\	医師会を退会する 
\\	""みたいな...
\\	高度成長		高度(こうど)
\\	成長(せいちょう) 
\\	国粋的		国粋 
\\	国粋的(こくすいてき) 
\\	志す		志す(こころざす) 
\\	こころざす【志す】 〖意図する〗
\\	〖目ざす〗
\\	〖大望を抱く〗
\\	〖決心する〗
\\	〖したいことを心に決める〗
\\	若いころ私は政治家を志した 
\\	腰掛け就職		"腰掛け(こしかけ) 就職(しゅうしょく) 
\\	こなす		
\\	子持ち		子持ち(こもち) 
\\	最大		最大(さいだい) 
\\	賛成(する)		賛成(さんせい) 
\\	しかく資格		資格(しかく) 
\\	自信		自信(じしん) 
\\	実現(する)		実現(じつげん) 
\\	失敗(する)		失敗する(しっぱいする) 
\\	志望(する)		志望(しぼう) 
\\	就職(する)		就職(しゅうしょく) 
\\	終了(する)		終了(しゅうりょう) 
\\	しゅうりょう―れう【終了】 (名)スル 物事が終わりになること。終わりにすること。 ↔開始。「作業を―する」 しゅうりょう【終了】名詞 
\\	【閉じること】
\\	終了する動詞 
\\	〖会などが[を]〗
\\	〖仕事などが[を]〗
\\	(⇨終わる, ⇨終える) 
\\	試合は終了しました 
\\	本日の作業は終了しました 
\\	「終了」は完成しなくても時間切れで終わった場合も使えます。 「完了」はやるべきことが完全に終わった状況で利用します。
\\	出版社		出版社(しゅっぱんしゃ) 
\\	順序		順序(じゅんじょ) 
\\	じゅんじょ【順序】 
\\	何が先で,何があとに来るかという,物事の相互(そうご
\\	の関係。「―よく乗車する」「―を立てる」 
\\	物事を行う段取り。手順。「―をふむ」 →順番(補説欄) じゅんじょ【順序】 
\\	【決まった並び方】 
\\	【連続して起こる順序】
\\	順序立てて⦅秩序立てて⦆説明する 
\\	事件の起こった順序をたどる 
\\	この名簿(めいぼ 
\\	は順序が逆だ[狂っている] 
\\	彼の本はいつも順序正しく並んでいる 
\\	【手順】 
\\	順序を誤る(あやまる) 
\\	順序を踏む 
\\	式は順序どおりに進んだ 
\\	(順調に) 
\\	順序不同 
\\	(!米英ではこの種の断わり書きをつける習慣はない) じゅん【順】 【順序】
\\	【順番】
\\	(⇨順番) じょ【序】 【始め】
\\	【序文】
\\	【序論】
\\	順調(な)		順調(じゅんちょう) 
\\	消極的(な)		消極的(しょうきょくてき) 
\\	生じる		生じる(しょうじる) 
\\	しょうじる【生じる】 〖起こる〗【偶然に】
\\	【起こるべくして】
\\	(⇨起こる)〖問題などが発生する〗
\\	〖…の結果として生じる〗
\\	〖引き起こす〗
\\	〖生み出す〗
\\	事故は彼の不注意から生じた 
\\	戦後日本には大きな変化が生じた 
\\	彼の話は疑惑[誤解]を生じた 
\\	どうして貿易不均衝が生じたのですか 
\\	【原因は何ですか】
\\	ふちゅうい【不注意】名詞 
\\	〖怠慢〗
\\	〖無思慮〗
\\	はえる【生える】 〖成育する〗
\\	〖芽を出す〗
\\	【急に】
\\	〖歯が〗
\\	(歯
\\	雑草はどこにでも生える 
\\	うちの子に最初の歯が生えた 
\\	雑草が家の庭一面に生えている 
\\	はえぎわ【生え際】 
\\	彼は生え際が少し後退している 
\\	はえぬき【生え抜き】 生え抜きの形容詞 
\\	生え抜きの⦅最初から外交を職業としている⦆外交官 
\\	生え抜きの⦅生(き)っ粋の⦆江戸っ子 ⦅書⦆
\\	いかす【生かす】 
\\	【生きたままにしておく】 
\\	【許して】
\\	できるだけ長い間この魚を生かしておいてください 
\\	あの殺人犯は生かしておけない 
\\	【命を助けてやれない】
\\	政治家を生かすも殺すもマスコミ次第 
\\	【活用する】 【使用する】
\\	【利用する】
\\	【効果的に】⦅書⦆
\\	自分の経験を十分生かす 
\\	経験を仕事に生かす 
\\	核エネルギーを平和目的に生かして使う 
\\	英語を生かせる仕事を探しています 
\\	君はこの機会をうまく生かしなさい 
\\	[⦅書⦆ 
\\	いきた【生きた】 〖生きている〗
\\	(!限定的に) 
\\	(!叙述的に) 
\\	(!限定的にも叙述的にも用い, 前の2語より意味範囲が広い) 
\\	生きたネズミで実験する 
\\	生きた英語を学ぶ 
\\	[(実際使われている) 
\\	(日常の) 
\\	彼は生きたまま埋められた 
\\	生きた心地がしなかった 
\\	はこの動作表現) 【命が縮むほどおびえた】
\\	いきぬく【生き抜く】 【生き延びる】
\\	【切り抜ける】
\\	【生き残る】
\\	二つの戦争を生き抜く 
\\	いきもの【生き物】 
\\	⦅集合的⦆
\\	【植物を除く】
\\	(⇨生物) 
\\	密林の生き物を殺す 
\\	政治は生き物,一寸先は闇だ[分からない] 
\\	一寸先は闇(やみ) 未来のことは全く予測することができないことをいう。
\\	将来性		将来性(しょうらいせい) 
\\	しょうらいせいしやう―【将来性】 将来,成長発展するであろうという見込み。「―のある人物」「―で株を購入する」 将来性のない仕事にはつきたくないからまた学校へ行くつもりだ。 
\\	自分の仕事の将来性を高める 
\\	自由国家としての将来性 
\\	自立(する)		自立(じりつ) 
\\	進学する		進学する(しんがくする) 
\\	しんけん真剣(な)		"真剣(しんけん) 
\\	進出(する)		進出(しんしゅつ) 
\\	新鮮(な)		新鮮(しんせん) 
\\	進路		進路(しんろ) 
\\	しんろ【進路】 
\\	進んで行く道。ゆくて。 ↔退路 
\\	人が将来進む方向。「卒業後の―を決定する」〔同音語の「針路」は船舶航空機などで羅針盤が示す方向のことであるが,それに対して「進路」は進んでいく道のこと,特に人の将来進む方向のことをいう〕 しんろ【進路】 
\\	【将来の】
\\	進路を変更する 
\\	進路指導をする 
\\	台風の進路⦅通り道⦆に当たる 
\\	卒業後の進路は決まりましたか 
\\	(!後の方は堅い言い方) 進路を変える時間はあると思っている 
\\	その任務に誰を選ぶかは、社の今後10年の進路を決めることになるだろう。 
\\	先生は彼をその会社に強く推薦した		推薦(すいせん) 
\\	すいせん【推薦】 (名)スル 人や物を,すぐれていると認めて他人にすすめること。適当なものとして紹介すること。推挙。「議長に―する」「参考書を―する」「―状」 すいせん【推薦】名詞 
\\	強い[消極的な]推薦 
\\	彼女の推薦状を書く 
\\	彼女はその大学に推薦入学した 
\\	推薦する動詞 
\\	〘物
\\	人; 
\\	〖候補者に指名する〗
\\	彼女を通訳に[その職に]推薦する 
\\	彼を議長候補に推薦する 
\\	先生は彼をその会社に強く推薦した 
\\	気軽に読める本を推薦してくださいませんか 
\\	の文型は今は不可) 推薦者 
\\	【指名者】
\\	推薦図書 
\\	勧める		勧める(すすめる) 
\\	世代		世代(せだい) 
\\	せだい【世代】 
\\	約30年間をさす
\\	集合的に1世代の人々をさす
\\	その場合単複両扱い) 
\\	2世代前 
\\	母の世代の人々 
\\	その料理法は母から娘へ何世代にもわたって受け継がれてきた 
\\	(!「母から娘へ」をとると次のようになる: 
\\	私たちはテレビ世代である 
\\	若い世代の人々は適応力がある 
\\	相撲界では世代交代が行われている 
\\	世代間のギャップ 
\\	積極的(な)		積極的(せっきょくてき) 
\\	せっしょく接触(する)		接触(せっしょく) 
\\	絶対的(な)		絶対的(ぜったいてき) 
\\	戦後		戦後(せんご) 
\\	専攻(する)		専攻(せんこう) 
\\	選択肢		選択肢(せんたくし) 
\\	せんたくし43【選択肢】 質問に対する回答として用意されているいくつかの項目。 せんたく【選択】名詞 〖二つ以上から選ぶこと〗
\\	〖三つ以上から最適なものの選抜〗
\\	選択肢 
\\	専念(する)		専念(せんねん) 
\\	せんねん【専念】 専念する動詞 〖努力注意を集中する〗
\\	〖ささげる〗
\\	〖一心に従事する〗
\\	〖熱中する〗
\\	庭造りに専念する⦅精を出す⦆ 
\\	休日は読書に専念する 
\\	彼は1日中その仕事に専念した 
\\	彼女は子育てに専念するために仕事をやめた 
\\	彼女はその小説の執筆に専念している 
\\	ともいえる)
\\	それなりに		
\\	損をする		損をする(そんをする) 
\\	株(かぶ)で損をする 
\\	己の最大の欠点を知って、損をすることはない。 
\\	〔ことわざ〕 売り上げで何百万ドルも損をする 
\\	そん【損】名詞 
\\	【損失】 
\\	(⇨損失) ①損は 
\\	よい物を買えば損はない⦅割に合う⦆ 
\\	②損に 
\\	結局損になる 
\\	自分の損になることをする 
\\	③損を 
\\	私は損を承知でそれを売った 
\\	この取り引きでは損さえしなければ⦅収支がとんとんならば⦆よしとせねばなるまい 
\\	【不利】 
\\	(⇨不利) 
\\	彼は英語を話せないので非常に損をしている 
\\	長男は損だと思う 
\\	【むだ】 
\\	やるだけ時間の損だ 
\\	損な形容詞 〖もうからない〗
\\	〖不利な〗
\\	【好ましくない】
\\	損な商売[取り引き] 
\\	彼は損な立場にいる 
\\	損(を)する動詞 
\\	彼は商売で大損[100万円損]をした 
\\	(失った) 
\\	親切にして損をすることはなかろう 
\\	私は20ドル損をしてそれを売った 
\\	損して得取れ ⦅ことわざ⦆
\\	第一線		第一線(だいいっせん) 
\\	体制		体制(たいせい) 
\\	対等(な)		対等(たいとう) 
\\	たいとう【対等】名詞 
\\	〖対等の人〗
\\	対等の形容詞 〖能力体力地位などが匹敵する〗
\\	〖互角の〗
\\	対等の立場[条件]で 
\\	彼らを対等の人間として扱う 
\\	彼女は彼と能力の点で対等だ 
\\	対等に副詞 
\\	彼を対等に扱う 
\\	君は彼と対等には戦えないだろう
\\	彼は強すぎる 
\\	他人		他人(たにん) 
\\	ためる		
\\	多様性		多様性(たようせい) 
\\	単身		単身(たんしん) 
\\	つきあい		
\\	つきあい【付き合い】 
\\	【交際】 ⦅やや書⦆
\\	(!仕事友情愛情のいずれの関係にも用いる) 
\\	【友達の】
\\	【仲間の】
\\	【知人の】⦅やや書⦆
\\	〖人間関係〗
\\	①~つきあい 
\\	長い間の付き合い 
\\	わずかなつきあいで結婚する 
\\	付き合いが狭い[広い] 
\\	太郎とは10年来のつきあいです 
\\	[⦅やや書⦆ 
\\	と複数形に注意) 
\\	我々は会ったら話をする程度のつきあいです 
\\	(!後の方が口語的) ②つきあいが[は] 
\\	(人)つきあいがよい[悪い] 【社交的である[ない]】
\\	【(複数の人に)うまくとけ込む[とけ込まない]】
\\	彼はつきあいが広い 
\\	[(多くの知人がある) 
\\	当方は暴力団とのつきあいはない 
\\	【暴力団員に知人[友人]はいない】
\\	彼女とのつきあいは長くは続かなかった 
\\	田舎はつきあいが難しい 
\\	[(わずらわしい) 
\\	(複雑だ) 
\\	も可だが「腹立たしい」の意味合が強くここでは不適) 【田舎の人たちはつきあいにくい】
\\	(⇨付き合う) ③つきあいを 
\\	つきあいをやめる (付き合う
\\	我が家は近所の人と親しいつきあいをしています 
\\	君は人とのつきあいをもっとよくすべきです 
\\	⦅話⦆ 
\\	(外向的) 
\\	④つきあいで 
\\	つきあいで彼女と酒を飲みに行った 
\\	(!「つきあいで飲む人」は 
\\	【いっしょにいること】 
\\	彼とのつきあいはうんざりだ
\\	人の悪口ばかり言うんだもの 
\\	(演説で) みなさん,最後までおつきあいください 
\\	同級生		同級生(どうきゅうせい) 
\\	どうきゅうせい―きふ―【同級生】 同じ学級の生徒。クラスメート。 どうきゅう【同級】 
\\	和子と私は同級です 【同じクラスにいる】
\\	【私の同級生だ】
\\	会話 
\\	「どのように知り合ったの?」「高校で同級生でした」「それ以来ずっと付き合っていたわけ?」「そうよ
\\	ずっと仲良しだったの」 
\\	では学校が同じであっても学年が同じとは限らない) 
\\	どうきゅう―きふ【同級】 
\\	同じ学級。「―会」 
\\	同じ等級。同じ階級。「―の品」
\\	当時		当時(とうじ) 
\\	独身		独身(どくしん) 
\\	ともなう伴う		伴う(ともなう) 
\\	眺める		眺める(ながめる) 
\\	ながめる【眺める】 (動マ下一)《文マ下二ながむ》 
\\	遠くまたは広く見渡す。「沖を―める」「窓から―める」 
\\	じっと見つめる。「相手の顔をしげしげと―める」 
\\	傍観する。「しばらく様子を―めていよう」 
\\	物思いに沈んでぼんやり見る。また,ぼんやり見ながら物思いにふける。「かぎりなく遠くも来にける哉と思ひわびて,―めをるに」〈古今和歌集羇旅詞〉 〖見る〗
\\	〖凝視する〗【集中して】
\\	【じろじろと】
\\	〖景色などを見渡す〗
\\	(⇨見る, ⇨見晴らす) 
\\	星を眺める 
\\	[(じっと) 
\\	彼の顔をじっと眺める 
\\	望遠鏡で星を眺める 
\\	[(観測する) 
\\	湖の眺められる部屋 
\\	彼女はぼんやりと窓の外を眺めていた 
\\	[⦅米⦆ 
\\	私は彼女を冷ややかに眺めた 
\\	は「ある感情を抱いて眺める」の意) 
\\	丘から海の美しい景色が眺められる 
\\	(⇨見晴らす) ぼんやりと雲を眺めていた. 
\\	無関心な感じで~を眺める 
\\	熱いコーヒーをすすりながらテレビ画面をぼんやり眺める 
\\	年齢		年齢(ねんれい) 
\\	乗り越える		乗り越える(のりこえる) 
\\	漠然とした		漠然とした(ばくぜんとした) 
\\	激しい		激しい(はげしい) 
\\	発見(する)		発見(はっけん) 
\\	バブル期		バブル期(バブルき) 
\\	バブル経済		経済(けいざい) 
\\	ばりばり		
\\	反映(する)		反映(はんえい) 
\\	はんえい【反映】 (名)スル 
\\	光や色が反射してうつること。「木々の緑が湖面に―する」 
\\	色や光が互いにうつり合って,美しくはえること。 
\\	影響が他に及び,ある形をとってあらわれること。反影。「世相を―した事件」〔明治時代につくられた語〕 はんえい【反映】名詞 
\\	反映する動詞 
\\	この記事は世論を反映している 
\\	【受身形で】
\\	サッカーは試合ぶりに国民性が反映しやすいスポーツと言われる 
\\	日本社会を反映したもの メディアで報じられていることで本当の人物像を反映しているものはめったにない。 
\\	比較的		比較的(ひかくてき) 
\\	ひかく【比較】名詞 
\\	彼の作品と彼女の作品では比較⦅比べ物⦆にならない 
\\	(匹敵しない) 
\\	(!いずれも彼女の作品の方が断然優れていることを暗示) 比較的副詞 
\\	この店は比較的安い 
\\	比較する動詞 
\\	は特に受身で好まれる) 
\\	〖対比対照する〗
\\	(⇨比べる) 
\\	その二つを比較する 
\\	都会生活と田舎の生活を比較する 
\\	私は何事にせよ彼と比較されるのは好きではない 
\\	トムと比較するとボブの方が客観的だ 
\\	は通例省略される. 
\\	は不可) 
\\	比較級 〘文法〙
\\	比較研究 
\\	比較言語学 
\\	比較広告 
\\	[⦅話⦆ 
\\	比較宗教学 
\\	比較文化 
\\	比較文学 
\\	かくさ【較差】 【変動幅】
\\	年間の気温較差 
\\	ひれい【比例】名詞 〖釣り合い〗
\\	〖比率〗
\\	(⇨比率) 
\\	正[反]比例 
\\	比例する動詞 
\\	収穫高は7月の雨量の多寡(たか)に比例する 
\\	彼の出費は収入に比例していない 
\\	売り上げが伸びるのに比例して利益が増えた 
\\	【応じて】
\\	ひゆ【比喩】名詞 〖言葉のあや〗
\\	(!直喩 
\\	隠喩 
\\	などを総称した言い方) 
\\	〖寓話(ぐうわ)〗
\\	比喩的な形容詞 
\\	その語は比喩的な意味で用いられている 
\\	英詩では「ばら」がよく「愛」の比喩的表現として使われる. 
\\	比喩的に副詞 
\\	必要に応じて		必要(ひつよう) 応じて(おうじて) 
\\	否定的(な)		否定的(ひていてき) 
\\	一人暮らし		
\\	評価(する)		評価(ひょうか) 
\\	ひょうが氷河		氷河(ひょうが) ひょうが【氷河】 
\\	氷河時代 
\\	北極(ほっきょく)氷河 
\\	ほっきょく【北極】名詞 
\\	南極) 
\\	北極回り(の空路)でパリに行く 
\\	北極の形容詞 
\\	北極海 
\\	北極圏 
\\	北極光(ほっきょくこう) 
\\	北極探検(隊) 
\\	北極地方 
\\	北極点 
\\	就職氷河期 
\\	政治的氷河期に入る 
\\	表面的(な)		表面的(ひょうめんてき) 
\\	表面的ではない深いレベルで(人)と気持ちを通じ合わせる 
\\	不安(な)		不安(ふあん) 
\\	ふくせんこう副専攻		副専攻(ふくせんこう) 
\\	含める		含める(ふくめる) 
\\	復帰(する)		復帰(ふっき) 
\\	復興 (する)		復興(ふっこう) 
\\	ふっこうふく―【復興】 (名)スル 一度衰えたものが,再び盛んになること。また,盛んにすること。「戦災都市を―する」「文芸―」 おとろえる【衰える】 〖弱くなる〗
\\	〖あらし風などが〗
\\	では「完全になくなった状態」をさす) 
\\	⦅書⦆ 
\\	⦅書⦆ 
\\	〖力価値などが〗⦅やや書⦆
\\	〖健康などが〗
\\	(弱る
\\	ふたたび【再び】 〖二度〗
\\	〖もう一度〗
\\	(もう 
\\	〖二度目に〗
\\	さかん【盛ん】 盛んな形容詞 〖繁盛している〗
\\	が最も一般的) 
\\	〖成功した〗
\\	〖活発な〗【積極的な, 活動的な】
\\	【精力的な】
\\	【強健で活発な】
\\	【元気のよい】
\\	〖熱烈な〗
\\	〖人気のある〗
\\	盛んになる 
\\	ふっこう【復興】名詞 〖復旧〗
\\	【回復】
\\	【再建】
\\	〖古い習慣などの復活〗
\\	経済復興 
\\	文芸復興 【ヨーロッパ史上の】
\\	戦争で破壊された国の復興を支援する 
\\	復興する動詞 
\\	復興させる 
\\	町が復興した 
\\	赴任(する)		赴任(ふにん) 
\\	ふにん【赴任】 (名)スル 任地へ赴くこと。「東京に―する」 任地(にんち)へ赴く(おもむく)こと。 
\\	ハリスって家族と一緒に日本に来たのかな?それとも単身赴任? 
\\	たんしんふにん【単身赴任】 遠方へ転勤する際,家族を残して,本人だけが任地へ赴くこと。 〈親項目〉 単身 今回ロンドンに赴任した際は、ヨーロッパのカスタマーサポートを設立してもらいます。 
\\	海外赴任に就いている 
\\	海外赴任のビジネスマン 
\\	海外赴任を引き受ける 
\\	おもむく【赴く】 〖行く〗
\\	〖出発する〗
\\	〖訪問する〗
\\	新しい任地に赴く 
\\	[(出発する) 
\\	感情の赴くにまかせる 
\\	私たちは美しい田園を足のおもむくままに散策した 
\\	にん【任】 〖職〗
\\	〖地位〗
\\	〖任務〗
\\	〖責任〗
\\	任にある 
\\	任を果たす 
\\	[⦅書⦆ 
\\	その任にあらず 
\\	社長は私の任ではない 
\\	まかせる【任せる】 
\\	【ゆだねる】 
\\	【信用して大事なものを委託する】
\\	の方が普通) 
\\	財産管理を彼に任せる 
\\	一切を運に任せる 
\\	その問題の解決は弁護士に任せた 
\\	子供たちの世話を彼女に任せられるだろうか 
\\	【信用して】
\\	承知した. 一切任せておいてくれ 
\\	会話 
\\	「パーティーにはだれを呼ぼうか」「一切君に任せるよ」 
\\	【放任する】 【…させておく】
\\	【放っておく】
\\	自然の成り行きに任せる 
\\	激情に身を任せる⦅身をゆだねる⦆ 
\\	家のことは妻に任せている 
\\	彼は畑を荒れるに任せた 
\\	【存分に使う】 
\\	力に任せて⦅全力で⦆彼を殴る 
\\	暇に任せて読書する 
\\	彼は金に任せて⦅費用はお構いなしに⦆家を建てた 
\\	にんき【任期】 
\\	4年の任期を務める 
\\	大統領として(2回目)の任期中に 
\\	任期終了前に 
\\	大統領の任期は4年である 
\\	にんむ【任務】 〖義務〗
\\	(!しばしば複数形で) 
\\	〖課せられた仕事〗
\\	(!特につらくて困難な仕事) 
\\	〖目的のために派遣されてする仕事〗
\\	~の任務 
\\	国会議員の任務 
\\	秘書の任務の一つは手紙をタイプすることだ 
\\	諸君の任務は山の遭難者を救出することだ 
\\	任務を[に] 
\\	任務を果たす 
\\	(!後の方が堅い語) 
\\	任務に就く 
\\	特別な任務を帯びて米国へ派遣される 
\\	首相は彼にインフレを抑える⦅と戦う⦆という任務を与えた 
\\	(上役に)任務を無事完了いたしました 
\\	の略) にんめい【任命】名詞 【任用】
\\	【指名】
\\	任命する動詞 
\\	【公式に】
\\	首相は彼を自分の後任に任命した 
\\	(!通例 
\\	は省略する) 
\\	彼は校長に任命された 
\\	(!補語に1人だけの身分官職を表す名詞がくる場合は通例無冠詞) 
\\	振り返る		振り返る(ふりかえる) 
\\	触れ合う		触れ合う(ふれあう) 
\\	家事を分担する男性が増えている		分担(ぶんたん) 
\\	ぶんたん0【分担】 (名)スル 仕事責任費用などを分けて受け持つこと。分けて負担すること。「組み立て作業を―する」「―金」 ぶんたん【分担】名詞 〖費用仕事などの負担〗
\\	〖計画的に割り当てた仕事〗
\\	〖任意または偶発的な割り当て〗⦅やや書⦆
\\	分担が決まった仕事 
\\	自分の分担の仕事をする 
\\	分担する動詞 〖分け合う〗
\\	〖分割する〗
\\	〖割り当てる〗
\\	責任を分担する 
\\	[⦅やや書⦆ 
\\	費用(ひよう)を分担する 
\\	その仕事を我々4人で分担しよう 
\\	になう【担う】 
\\	【引き受ける】 
\\	重要な役割を担う 
\\	(!後の方は口語的) 
\\	日本は国際的な責任を担うべきだ 
\\	日本の将来は君たちが担っている 
\\	彼は私たちの期待を担っている 
\\	【背負う】 
\\	かつぎあげる【担ぎ上げる】 
\\	肩に担ぎ上げる 
\\	重いいすを2階へ担ぎ上げる 
\\	彼女を生徒会の会長に担ぎ上げる⦅押し立てる⦆ 
\\	たんか【担架】 
\\	【台車付きの】⦅米⦆
\\	担架で運ぶ 
\\	たんとう【担当】 
\\	君の担当の医者[先生]は誰ですか 
\\	(!言い方に注意) 担当する動詞 
\\	彼にそのプロジェクトを担当させる 
\\	英語を担当する⦅教える⦆ 
\\	政権を担当している⦅権力の座にある⦆ 
\\	私は50人の生徒を担当している 
\\	[⦅書⦆ 
\\	(⇨担任) 
\\	だれがその実験[部門](の責任)を担当するのですか 
\\	その販売員はこの地域を担当している 
\\	担当者 【管理責任者】
\\	たんにん【担任】名詞 (⇨担当) 
\\	担任の教師 
\\	私たちの学級担任は島田先生です 
\\	[⦅米⦆ 
\\	⦅英⦆ 
\\	【担任している】
\\	[⦅書⦆ 
\\	相田先生が彼の担任です⦅相田先生に担任してもらっている⦆ 
\\	(!後の方では 
\\	に 
\\	をつけることに注意) 担任する動詞 
\\	[(している) 
\\	彼が担任している生徒 
\\	平均的(な)		平均的 (へいきんてき) 
\\	平均的なアメリカ人男性 
\\	平均的な家庭の暮らしは、昔とはずいぶん変わっている。 
\\	(家庭 かてい) 平均的な成人は1週間に_時間ぐらいテレビを見ている。
\\	(成人 せいじん) 
\\	均: ならす【均す】 〖平らにする〗
\\	〖平均する〗
\\	地面をローラーでならす 
\\	きんとう【均等】 
\\	(⇨平等) 
\\	教育の機会均等 
\\	と複数形も可) きんぶん【均分】名詞 
\\	均分する動詞 
\\	均分相続 
\\	きんいつ【均一】名詞 
\\	均一の[な]形容詞 ⦅やや書⦆
\\	【料金価格などが】
\\	【均等の】
\\	均一値段 
\\	均一料金 
\\	【電車などの】
\\	大きさと色が均一な花 
\\	そのおもちゃは100円均一です⦅すべて100円です⦆ 
\\	ベビーブーム		
\\	変更(する)		変更(へんこう) 
\\	ほうかい崩壊(する)		崩壊(ほうかい) 
\\	崩れる 壊れる ほうかい―くわい【崩壊・崩潰】 (名)スル 
\\	くずれること。こわれてしまうこと。「堤防が―する」 
\\	〘物〙 不安定な素粒子が自発的に分裂して別種の素粒子に変化すること。また,不安定な原子核が放射線を放出して,別種の原子核に変化する現象。放出する放射線の種類によって,α崩壊,β崩壊,γ崩壊がある。原子核崩壊。壊変。 ほうかい【崩壊】名詞 
\\	【破滅】
\\	【没落】
\\	家庭の崩壊 【一家の没落】
\\	【離婚などによる】
\\	(!その結果生じる「崩壊家庭」は 
\\	崩壊する動詞 【つぶれる, つぶす】
\\	【ばらばらになる[する]】
\\	【破滅する[させる]】
\\	その建物は崩壊しかかっていた 
\\	【崩壊寸前だった】
\\	彼が死んだら家庭は崩壊するであろう 
\\	放送局		放送局(ほうそうきょく) 
\\	補助する(ほじょする)		補助(ほじょ) 
\\	迷う		迷う(まよう) 
\\	右肩上がり		"右肩上がり(みぎかたあがり) 
\\	みぎかたあがり【右肩上がり】 →右上がりに同じ。特に,経済的な分野の成長率などが年を追うごとに上昇してゆくことの形容。〔数値の推移を示すグラフが,右側へゆくにつれ,上がることから〕 みぎかたあがり【右肩上がり】 
\\	右肩上がりの成長をする⦅着実に成長する⦆ 
\\	彼は右肩上がりに書く 
\\	向上する、上向く、急激に増加する、などの表現で適宜置き換えるのでしょうね…。 とりあえず例文3つ、
\\	に入れたのが当該箇所です(勿論本来 
\\	不要) 
\\	みぢか(な)身近(な)		身近(みぢか) 
\\	道筋		道筋(みちすじ) 
\\	模索(する)		模索(もさく) 
\\	優先(する)		優先(ゆうせん) 
\\	優先すること 
\\	〈文〉 
\\	(重要度や緊急度により) 
\\	ゆうせんいう―【優先】 (名)スル 他のものより先に扱うこと。他をさしおいて行うこと。「公益を私益に―させる」「―道路」 扱う(あつかう)
\\	ゆうせん【優先】 【順序時間などの】
\\	(!「優先すべきこと」の意では 
\\	【順序重要性などの】
\\	【選択などの】
\\	最優先事項 
\\	この契約では君より私の方に優先権がある 
\\	それは最優先の議題である 
\\	会員は優先的に切符が手に入る 
\\	幹線道路を走っているからこちらに(走行の)優先権がある 
\\	優先する動詞 
\\	憲法は他の一切の法律に優先する 
\\	日本は産業の発展を優先させてきた 
\\	彼は何よりも仕事を優先させた 
\\	優先順位 
\\	優先席 【老人などの】
\\	両立(する)		"両立(りょうりつ) 
\\	りょうりつ【両立】 両立する動詞 【一致する】
\\	【矛盾しない】
\\	その二つの考えは両立する[しない] 
\\	彼女はどうにか仕事と家庭を両立させた 
\\	と
\\	を完全に両立させる 
\\	「あなたなら仕事と子育てを両立できますよ」「でも私につとまるでしょうか」 
\\	どちらの仕事も両立させる 
\\	割合		割合(わりあい) 
\\	この写真はぶれている		ぶれている
\\	優れる		すぐれる【優れる勝れる】動詞 〖秀でる〗
\\	〖勝る〗
\\	⦅書⦆ 
\\	人よりすぐれる 
\\	彼は英語[スポーツ]にすぐれている 
\\	は学科能力, 
\\	は運動技術などに用いられることが多い) 【得意である】
\\	とすれば「学科」がすぐれていることをいう) 
\\	この辞書はあれよりすぐれている 
\\	すぐれない動詞 〖気分体の調子が〗
\\	〖健康が〗
\\	⦅話⦆ 
\\	〖顔色が〗
\\	⦅主に米⦆ 
\\	⦅主に英⦆ 
\\	今日は気分がすぐれない 
\\	のように進行形で用いると特に一時的状態を強調) 
\\	どうしたんだい
\\	今日は顔色がすぐれないね 
\\	すぐれた形容詞 
\\	【傑出した】
\\	すぐれた作品 
\\	すぐれた点 
\\	【利点】
\\	非常にすぐれた仕事をする 
\\	は次の形容詞より弱く発音することに注意
\\	の代わりに 
\\	は用いない) 
\\	物理ですぐれた成績をとる 
\\	これは今まで見た中で最もすぐれた映画だ 
\\	すぐれる【優れる・勝れる】 (動ラ下一)《文ラ下二すぐる》 
\\	他のものよりも内容程度技量などが上である。まさる。「―れた脚力」「人に―れた色彩感覚」「理解力に―れる」「―れて時めかし給ふことならびなかりける程に」〈源氏物語須磨〉 
\\	(多く打ち消しの語を伴う)よい状態である。「健康が―れない」「顔色が―れない」「天気が―れない」
\\	あなたの国では、家の仲と家の外で、男の人と女の人の役割が違いますか。		役割(やくわり) やくわり【役割】 〖仕事の一部〗
\\	〖役目〗
\\	〖仕事〗
\\	〖機能〗
\\	(⇨役, ⇨役目) 
\\	当委員会の役割は新社屋を設計することだ 
\\	このソファはベッドの役割もする⦅としても役に立つ⦆ 
\\	[(用いられる) 
\\	大気は温室のような役割をする⦅働きをする⦆ 
\\	ライオンは群れで狩りをし,役割分担をした⦅連携した⦆攻撃方法をとる 
\\	(⇨手分けする)
\\	男女差別の問題について、それぞれご自分なりの意見を述べてください。		
\\	なりに
\\	1なりの
\\	男女(だんじょ) のべる【述べる】 〖言い表す〗
\\	〖話す〗
\\	〖陳述する〗
\\	より堅い語) 
\\	〖記述する〗
\\	〖言及する〗
\\	先に述べたとおり ⦅書⦆
\\	自分の考えを述べる 
\\	は正式にはっきり述べること) 
\\	(!言葉だけでなく身振りでも, また考えだけでなく感情を表してもよい) 
\\	礼を述べる 
\\	(!「神に(食前などに)感謝を捧げる」意にも用いる) 
\\	真実を述べる 
\\	(話す
\\	事故の様子を詳しく述べる 
\\	彼は自分がそれをしたと述べた 
\\	は形式ばった言い方)
\\	いろいろと苦労してこそ、人間として成長する。		
\\	〜てこそ〜 
\\	〜てはじめて 
\\	くろう【苦労】名詞 
\\	【困り事】 
\\	〖困難〗
\\	〖苦難〗
\\	(!いずれも具体的には 
\\	仕事の苦労話をする 
\\	彼は苦労人だ 【多くの苦難を経験した】
\\	【人生経験が豊富だ】
\\	君には苦労が足りない 【人生経験が十分でない】
\\	【まだ学ぶことがたくさんある】
\\	【心配】 
\\	(心配⇨類語) (!いずれの語も「心配事」の意では 
\\	苦労が多い 
\\	彼は苦労性だ 【心配しすぎる】
\\	【物事を深刻に考える】
\\	苦労が絶えない 
\\	彼は両親に大変苦労をかけている 
\\	【心配の種だ】
\\	【面倒,手数】 
\\	ご苦労さまでした 〘米英人的発想〙【ご苦労を感謝します】
\\	〘日本人的発想〙【ご苦労をかけてすみません】
\\	【努力】 
\\	(!しばしば複数形で) 
\\	【苦闘】
\\	〖骨折り〗
\\	長年の苦労のすえ, その画家はついに真価を認められた 
\\	苦労する動詞 ―こそ 
\\	【意味を強める】 
\\	これこそ⦅これはまさに⦆私が捜していたペンです 
\\	自分自身の子供をもってこそはじめて親の愛情が分かる 
\\	私こそ一言お礼を申さねばなりません 
\\	を強めた強調構文. 
\\	日常的には 
\\	や, 簡単に「こちらこそ」
\\	のようにいう) 
\\	冬に雪があってこそ北海道だ⦅冬に雪のない北海道なんて考えられない⦆ 
\\	(⦅主に米⦆ 
\\	今度こそきっと全力を尽くそう 
\\	と強く読む) 会話 
\\	「座りなさい,友子
\\	見えないわ」「あなたこそ立ちなさいよ」 
\\	(!このように強調する場合, 命令文でも 
\\	を置く) 
\\	「口いっぱいほおばったままでしゃべらないで」「じゃあ,あんたこそやめてよ」 
\\	を強く読む) 談話 
\\	親こそ子供に食卓での行儀を教えるべきだ 
\\	(…すべきだ)は主張を表すディスコースマーカー) 
\\	【一応の肯定を表す】 【…だがしかし】
\\	【…であるけれども】
\\	彼は私をほめこそすれ,決して笑わなかった 
\\	〖難儀する〗
\\	事人; 
\\	の方が口語的) 
\\	〖つらい目にあう〗
\\	事; 
\\	【苦しむ】
\\	〖苦難を経験する〗
\\	(⇨名①)〖骨折る〗
\\	金で苦労する 
\\	(!しばしば複数形で) ⦅話⦆
\\	英語ではずいぶん苦労しました 
\\	彼の家を見つけるのに少しも[大して]苦労しなかった 
\\	(!⦅話⦆ では通例 
\\	を省略する. また 
\\	は不可) 
\\	彼はその仕事を終えるのに非常に苦労した 
\\	彼は苦労して⦅働いて⦆学校を出た 
\\	人に頼らず、自分一人でやってこそ、本当の市からがつくものだ。		頼らず(たよらず) たよる【頼る】 〖依存する〗【物質的に】
\\	【援助安らぎなどを求めて】
\\	〖当てにする〗【信頼して】
\\	は過去の経験に基づく信頼を暗示) 
\\	【計算して】
\\	〖最後の手段として頼る〗
\\	解説 
\\	と 
\\	一般に 
\\	より 
\\	の方が堅い語だが, リズムの関係で 
\\	が好まれることがある. 
\\	彼女は金銭面で息子に頼っている 
\\	では 
\\	が普通) 
\\	彼には頼る身寄りが1人もいなかった 
\\	(!文尾では 
\\	が好まれる) 
\\	彼は気がめいるといつも酒に頼る 
\\	彼は親戚を頼って福岡に来た 
\\	忙しくて料理できない場合はレトルト食品に頼ることが多い 
\\	つく【着く】 
\\	【到着する】 
\\	【列車が駅プラットホームに入ってくる】
\\	【人乗り物が到着予定である】
\\	(⇨到着する) 使い分け 
\\	「到着する」 の意の最も一般的な語. 前置詞には,通例狭い場所や単なる到達地点は 
\\	広い場所や滞在する場合は 
\\	場所の表面が意識されると 
\\	が用いられる. 
\\	より口語的で, しばしば努力して着くことを含意する. 
\\	やや堅い語で,通例何らかの努力をして到達することを含意する: 人類が月に着いた 
\\	①(場所)に[から]着く 
\\	家に着く 
\\	は「家に」という副詞. したがって 
\\	や 
\\	は不可) 
\\	東京から着く 
\\	1時間ほど歩いたら小さな村に着いた 
\\	⦅やや書⦆
\\	さあ,上野駅に着きましたよ 
\\	会話 
\\	「彼らはこの前の金曜日にシドニーを発ったんだよ」「じゃああしたにならなければここに着かないでしょうね」 
\\	②(人物)が[は]着く 
\\	スミスさんはいつ着きますか 
\\	は目的語が必要) 
\\	この列車は何時にパリに着きますか 
\\	(!列車の発着など確実な未来の予定には現在形を用いる) 
\\	あなたの手紙は今日着きましたよ 
\\	【受け取った】
\\	あら,もう飛行機着いたの? すぐ迎えに行くわ 
\\	は「すぐそちらに」の意) 会話 
\\	「(私たちは)間に合うように着くかしら」「時間前に着くだろうよ」 
\\	【届いて触れる】 【届く】
\\	【触れる】
\\	天井が低いので私の頭は天井につきそうだった 
\\	【身を置く】 【座る】
\\	食卓につく[ついている] 
\\	正餐(せいさん)の卓につく 
\\	席について⦅着席して⦆ください 
\\	(!後の方ほど形式ばった言い方になる. 
\\	ともいい 
\\	などより丁寧な言い方)
\\	6行では収まらない場合は余白に書いてもかまいません。6 行では足りず、余白にもメッセージが書かれていると、エネルギーや情熱が相手に伝わるのではないかと思います。		よはく【余白】 
\\	〖欄外〗
\\	余白を残す[埋める] 
\\	ページの余白に書き込みをする 
\\	おさまる【収まる】 
\\	【きちんと中に入る】 
\\	その服は全部この箱に収まるでしょう 
\\	この記事は20行では収まらない⦅20行で書く[に減らす]ことができない⦆ 
\\	【解決がつく】 
\\	ストはまだ収まっていない 
\\	ぎょう【行】1 【文章文字の】
\\	〘略 
\\	複数
\\	〖詩の〗
\\	30ページの10行目に
\\	(⇨ページ)
\\	1行おきに書く
\\	3ページの下から[上から]3行目から読む
\\	行をかえる⦅改める⦆
\\	[(段落) 
\\	行間(⇨行間)
\\	10ページの3行目を見よ
\\	【宗教の】【苦行】
\\	【勤行】
\\	行をする
\\	じょうねつ【情熱】名詞 〖理性を圧倒するほど激しい〗
\\	【燃える】
\\	【変わらぬ】
\\	(⇨熱情)〖熱狂的な〗
\\	政治に情熱を持っている 
\\	教育に情熱を燃やす 
\\	今でもその仕事に情熱をかけられるかどうか自問してみるべきかもしれないね 
\\	情熱的な形容詞 
\\	情熱的な人 
\\	情熱的に副詞 
\\	情熱的に語る 
\\	営業に最適な手書きハガキのフォーマットハガキフォーマットこんにちは! 町工場のためのトップ営業コンサルタントの奥村政治です。6月9日に手書きハガキの効用についてお話ししました。今回、私が使っているハガキのフォーマットについてご紹介します。 写真のハガキですが、私が使っているフォーマットになります。開業当初から、試行錯誤を繰り返した結果、このような形になりました。 文章を書くのが苦手な方にとって、ハガキのスペースに文字をびっしり埋めることは大変な作業になります。このハガキフォーマットの良い面は、6行分の枠内で御礼やご挨拶を書けばいいことです。かなり楽になると思います。6行では収まらない場合は余白に書いてもかまいません。6行では足りず、余白にもメッセージが書かれていると、エネルギーや情熱が相手に伝わるのではないかと思います。 私の場合、名刺交換後、送付することを心掛けています。そして、ほぼ習慣化されています。メールが全盛の今、時代遅れかもしれませんが、人の温もりが伝わるハガキを送ってほしいものです。
\\	女子学生の多くが、将来への不安を抱きつつ自分の生き方を模索していました。		抱き(いだき) 
\\	ながら 
\\	つつ 女子(じょし)
\\	いだく【抱く】 〖心の中に〗
\\	【悪意などを】
\\	【望みなどを】
\\	〖両腕に〗
\\	(⇨抱(だ)く). 
\\	彼に悪意を抱く 
\\	【恨みを】
\\	密かな望みを抱く 
\\	関心を抱く 
\\	山に抱かれた静かな町 
\\	あなたは将来に何か不安を抱いていますか 
\\	もさく【模索】 模索する動詞 
\\	(⇨暗中模索) もさく0【模索・摸索】 (名)スル 手さぐりでさがすこと。あれこれとさがしもとめること。「暗中―」「最善の道を―する」
\\	偶然耳にした講演がきっかけで、日本語教育を専攻しました。		
\\	がきっかけで 
\\	偶然(ぐうぜん) 講演(こうせん) 専攻(せんこう) こうえん【講演】名詞 〖講義〗
\\	(⇨講義)〖演説〗
\\	(⇨演説)〖講話〗
\\	講演する動詞 
\\	講演会 
\\	講演者 
\\	ぐうぜん【偶然】名詞 
\\	〖偶然の出来事〗
\\	〖偶然の一致〗
\\	彼のアメリカ発見はまったく[ほんの]偶然だった 
\\	(!強調構文を用いて 
\\	のようにもいえる (⇨副)) 
\\	2人が同じバスに乗り合わせたのは単なる偶然(の一致)だった 
\\	ともいえる) 
\\	会話 
\\	「今夜そのコンサートへ行くんだ」「私もよ. 偶然ね」 
\\	[⦅話⦆ 
\\	偶然の形容詞 
\\	(!以上3語のうち叙述的にも用いられるのは 
\\	のみ) 
\\	偶然の死 
\\	偶然の出会い 
\\	彼らの出会いは偶然だった 
\\	偶然(に)副詞 
\\	私は京都で偶然彼女に出会った 
\\	[⦅書⦆ 
\\	⦅話⦆
\\	彼は通りで偶然財布を見つけた 
\\	(!偶然性を含み持つ動詞は 
\\	とともには用いないので 
\\	は不可) せんこう【専攻】 〖専門〗
\\	専攻は文学です 【専門家として】
\\	(!⦅主に米⦆ では大学での専攻には 
\\	や 
\\	を, 大学院での専攻には 
\\	を区別して用いる. また, ⦅英やや書⦆ では大学での専攻に 
\\	も用いる) 専攻科目 
\\	⦅主に米⦆【大学の学部学生の】
\\	(!副専攻科目 
\\	これまでいっしょうけんめい勉強してきた。受験まで1週間、最後まで頑張るしかない
\\	〜る しかない 
\\	じゅけん【受験】名詞 
\\	昨年
\\	大学の受験に合格した 
\\	(⇨入学試験) 
\\	彼は受験勉強をしている 
\\	[(準備をしている) 
\\	(詰め込んでいる) 
\\	彼は受験資格がない 
\\	何人の受験生が合格しましたか 
\\	[(志願者) 
\\	受験する動詞 
\\	[⦅英⦆ 
\\	(!米英とも 
\\	が普通) 
\\	【出願する】
\\	(!受身可) 
\\	来年は
\\	大学を受験するつもりです 
\\	受験科目 
\\	受験産業 
\\	受験地獄 
\\	(!日本語をなぞった直訳なので通例引用符をつける) 受験番号 
\\	受験票 【身分証】
\\	【許可証】
\\	受験料 
\\	自分で車をなおそうとしたけど、だめだった。 諦めるしかない。		こしょう【故障】名詞 〖機械からだなどの〗
\\	〖機械車などの突然の〗
\\	〖機能停止〗
\\	故障車 
\\	[⦅やや書⦆ 
\\	故障者リストに載る[から復帰する] 〘野球〙
\\	エンジンの故障 
\\	故障の原因を突きとめる 
\\	燃料系統に故障がある 
\\	バスの故障で学校に遅れた 
\\	その投手は60日故障者リストから戻された 
\\	故障 ⦅掲示⦆
\\	故障する動詞 
\\	(!電話など主に公共性の高い機械機器などについて用いる) 
\\	〖車機械などが〗
\\	(!状態を表すときは 
\\	エアコンが故障した 
\\	彼がどうやってあなたに知らせられたっていうのよ
\\	あなたの電話は故障してたのよ 
\\	エレベーターが突然故障して階と階の間で止まってしまった 
\\	この時計は故障している⦅壊れている⦆ 
\\	(!比較的小さいものには 
\\	より 
\\	を用いるのが普通) 【動かない】
\\	(!「ちゃんと動いていない」なら 
\\	【どこかおかしい】
\\	家事も子育てもパーフェクトにできていたつもりでした。しかし、あるとき、
\\	〜た つもりだ 
\\	だった 
\\	〜た つもりだ」は話す人 
\\	書く人がしたと思っていたことや信じていたことが、実はそうではなかったときの意外な気持ちを表します
\\	育て(そだて)
\\	育児		いくじ【育児】 
\\	育児に追われる[専念する; 悩む] 
\\	育児休暇 
\\	[〘婉曲的〙 
\\	育児室 
\\	育児書 
\\	【説明的に】
\\	育児ノイローゼ 
\\	[(産後の) 
\\	一生		
\\	いっしょう【一生】 〖生涯〗
\\	(複数 
\\	〖存命期間〗
\\	彼は一生独身で通した 
\\	は不可) 
\\	その辞書を作るのには一生かかるだろう 
\\	君たちそんなことをしたら一生⦅残りの生涯⦆悔やむことになるよ 
\\	一生は 
\\	彼の一生は不幸の連続だった 
\\	一生の 
\\	一生の⦅生涯続く⦆友 
\\	一生の仕事 
\\	【一生をかけた大仕事】
\\	一生のお願いです
\\	パリに留学させてください 
\\	一生に 
\\	そんなことは一生に一度しか起こらない 
\\	【一生にまたとないチャンスだ】
\\	一生を 
\\	幸せな一生を送る 
\\	その町で一生を過ごす 
\\	(⇨生涯) 
\\	教育に一生を捧げる 
\\	彼はかけ事で一生を棒に振った 
\\	[(経歴) 
\\	彼はその町で一生を終えた 
\\	産む		
\\	うむ【生む産む】 
\\	【出産する】 【人動物が】
\\	最も一般的な語) 
\\	⦅書⦆ 
\\	*(⇨生まれる)【分娩(ぶんべん)する】⦅書⦆
\\	【動物が】
\\	【魚カエルなどが】
\\	【卵を】
\\	産みたての卵 
\\	彼女は先月女の子を産んだ 
\\	⦅書⦆ 
\\	⦅書⦆ 
\\	は日本語のように能動形で用いることはまれ (⇨[次例])) 
\\	私が洋子を産んだとき⦅洋子が産まれたとき⦆,主人は仕事でニューヨークにいました 
\\	あなたの犬は何匹子を産みましたか 
\\	ウサギはたくさん子供を産む 
\\	うちのめんどりは毎日卵を産む 
\\	【生ずる】 【産出する】
\\	【引き起こす】
\\	【主に悪いことを】
\\	【至る】
\\	11パーセントの利子を生む 
\\	彼は日本が生んだ最大の科学者である 
\\	不衛生は病気を生む 
\\	こういう行動は疑惑を生む 
\\	[(呼び起こす) 
\\	金が金を生む ⦅ことわざ⦆
\\	解消する		
\\	かいしょう【解消】名詞 
\\	【解除】
\\	彼らの婚姻関係の解消 
\\	[(破局) 
\\	ストレス解消にスポーツをする 
\\	解消する動詞 【約束などを取り消す】
\\	【関係などを急に絶つ】
\\	契約を解消する 
\\	婚約を解消⦅破棄⦆する 
\\	提携を解消する 
\\	疑いがすっかり解消した 
\\	とかす【溶かす解かす】 〖固体を熱で〗
\\	〖固体を液体の中で〗
\\	〖凍った物を〗
\\	〖金属などを〗
\\	【液化する】⦅書⦆
\\	(⇨溶ける) 
\\	砂糖を紅茶に溶かす 
\\	湯をかけて凍結した水道管を解かす 
\\	ときほぐす【解きほぐす】 
\\	【からまったものを】 【糸などを】
\\	【謎を】
\\	糸のもつれを解きほぐす 
\\	謎を一つ一つ解きほぐす 
\\	【凝り固まったものを】 【筋肉のこりを】
\\	【心の緊張などを】
\\	【不信感などを】
\\	彼からの1通の手紙で彼に対する不信感は解きほぐされた 
\\	かいけつ【解決】名詞 【問題などの】
\\	【紛争などの】
\\	⦅書⦆
\\	はどちらの場合にも用いる) 
\\	その問題の二つの解決策 
\\	労使紛争に円満な解決をつける 
\\	エネルギー問題の早急な解決が望まれる 
\\	その紛争の解決には相当時間がかかるだろう 
\\	それは短絡的な解決法だ 
\\	は「(苦境ジレンマなどの)解決法」の意) 解決する動詞 〖解く〗
\\	【苦労して】
\\	〖決着をつける〗
\\	⦅書⦆
\\	金では[何もしないで心配していても]問題は解決しませんよ 
\\	警察は指紋を見つけてその事件を解決した 
\\	愚痴をこぼしても何も解決しはしない 
\\	は「何の役にも立たない」の意) 
\\	彼らは(その件を)双方話し合って500万円で解決した 
\\	(法廷に持ち込まないで) 
\\	両国間の紛争は武力によって[平和的に]解決された 
\\	格差		かくさ【格差】 〖隔たり〗
\\	〖違い〗
\\	〖不釣り合い〗⦅書⦆
\\	賃金格差 
\\	⦅英⦆
\\	格差社会 
\\	格差のない社会 
\\	合計する		
\\	ごうけい【合計】名詞 
\\	使い分け 
\\	数量を単純に加算した合計の意で,最も一般的な語. 
\\	すべての結果を入れた合計をさし,しばしばそれが大きいことを暗示する
\\	とほぼ同じ意味だが,簿記などで用いる堅い語
\\	7と3の合計は10だ 
\\	被害者の合計は100人になった 
\\	その惨事で合計30人が死亡した 
\\	彼女は請求書を手にとって合計(額)を見た
\\	大変な額だった 
\\	会話 
\\	「おいくらになりますか」「合計90ドルになります,お客様」 
\\	と現在形でいうよりやわらかい言い方) 
\\	「合計でいくらになりますか」「2万円です」 
\\	合計する動詞 
\\	彼は数字を合計して平均を出した 
\\	の代わりに 
\\	も可) 
\\	合計してその数字を教えてください 
\\	国際		国際(こくさい) 
\\	雇用(する)		
\\	こよう【雇用】名詞 
\\	完全[終身]雇用制 
\\	彼の仕事は労働者の雇用と解雇だった 
\\	大型の公共事業は雇用の創出に役立つかもしれない 
\\	不況の長期化で⦅長期化している状況下では⦆雇用(事情)は厳しい 
\\	雇用する動詞 
\\	⦅米⦆ 
\\	【一時的に】⦅英⦆
\\	(雇う⇨類語) 雇用機会均等法 
\\	雇用期間 
\\	雇用契約 
\\	雇用者[主] 
\\	(!被雇用者 
\\	雇用条件 
\\	雇用対策 
\\	雇用調整 
\\	雇用(こよう)保険(ほけん) 
\\	作成する		
\\	さくせい【作成】 作成する動詞 【文書などを】
\\	【書類請求書などを正式に】
\\	【草稿などを】
\\	(⇨作る) 
\\	遺言書を作成する 
\\	リストを作成する 
\\	請求書を作成する 
\\	支援(する)		
\\	しえん【支援】名詞 
\\	(⇨援助) 支援する動詞 
\\	平和運動を支援する 
\\	支援者 
\\	指標		
\\	しひょう【指標】 ⦅書⦆
\\	(複数 
\\	【経済の】
\\	【政策などの】
\\	経済指標 
\\	その数字は今後の土地の相場の指標となるだろう 
\\	し【指】 
\\	【親指】
\\	第一指 
\\	さす【指す】 
\\	【指し示す】 
\\	では対象そのものを, 
\\	では対象の位置する方向を表す) 
\\	⦅書⦆
\\	〖指名する〗
\\	地図上の場所を指でさす 
\\	授業中にさされる 
\\	人を指さすのは失礼だ 
\\	その矢印は東を[彼の家の方を]さしている 
\\	より漠然とした方向は 
\\	で示す) 
\\	気圧計は1,000をさしている 
\\	【意味する】 
\\	【言及する】
\\	「彼」とはだれをさしているのですか 
\\	ひょうご【標語】 〖うたい文句〗【スローガン】
\\	【キャッチフレーズ】
\\	〖座右銘〗
\\	(複数 
\\	「安全運転」という交通安全週間の標語 
\\	ひょうしき【標識】 
\\	〖航空航路の〗
\\	交通[道路]標識 
\\	「止まれ」の標識 
\\	標識に従って進む 
\\	ひょうじ【標示】 
\\	社会保障		
\\	ほしょう【保障】名詞 〖安全保障〗
\\	社会保障 
\\	安全保障理事会 【国連の】
\\	日米安全保障条約 
\\	保障する動詞 
\\	【確保する】
\\	平和を保障する 
\\	権利を保障する 
\\	基本的人権は憲法で保障されている 
\\	就業(する)		
\\	しゅうぎょう【就業】 就業する動詞 【仕事を始める】
\\	就業している 
\\	就業規則 
\\	就業時間 
\\	就業人口 
\\	重視(する)		
\\	じゅうし【重視】 重視する動詞 〖強調する〗
\\	〖重要だと考える〗
\\	〖真剣に受けとめる〗
\\	(⇨重要) 
\\	その問題についての彼の意見を重視する 
\\	従事(する)		
\\	じゅうじ【従事】 従事する動詞 〖仕事研究などに〗
\\	〖職業に〗【従う】
\\	【追求する】
\\	彼は20年間農業に従事してきた 
\\	彼は医業に従事した 
\\	充実(する)		
\\	じゅうじつ【充実】名詞 
\\	充実感 
\\	国防の充実 
\\	充実した形容詞 【十分な】
\\	【完全な】
\\	【内容のある】
\\	【実り多い】
\\	充実した生活を送る 
\\	彼は気力が充実している 
\\	出生率		出生率(しゅっしょうりつ) 
\\	しゅっせい【出生】 
\\	出生率の増加[低下] 
\\	出生地 
\\	の方が口語的) 【特に有名人の】
\\	出生届 
\\	生涯		
\\	しょうがい【生涯】 〖人生〗
\\	(複数 
\\	〖職業的経歴〗
\\	〖存命期間〗
\\	(⇨一生) 
\\	生涯を終える 
\\	生涯の伴侶(はんりょ) 
\\	政治家としての彼の生涯 
\\	生涯に五つの小説を書く 
\\	生涯独身で通す 
\\	彼は生涯の大半をウィーンで過ごした 
\\	生涯ご恩は忘れません 
\\	の方が決意の度合いが強い) 
\\	彼らの友情は生涯続いた 
\\	(!「生涯の友情」は 
\\	(⇨終生) 生涯教育 
\\	[(生涯続ける) 
\\	⦅米⦆ 
\\	職業		
\\	しょくぎょう【職業】 
\\	使い分け 
\\	職業をさす最も一般的な語で, 正式な文書などにも使うやや堅い語. 
\\	医師弁護士教師など知的な専門知識や訓練を必要とする職業
\\	技術や熟練を要求する職業
\\	利害を離れ社会に貢献する職業
\\	一生たずさわる,または経歴になるほど長くたずさわる職業
\\	賃金をもらってする仕事勤め口
\\	職業や勤務だけでなく,労働作業任務など幅広い意味を持つ一般的な語
\\	商業や経済,生産と関わる仕事
\\	複合語で用いられることが多い
\\	職業~ 
\\	職業的自立 
\\	職業上の 
\\	職業別の 
\\	職業柄室内にいることが多い 
\\	職業は 
\\	彼の職業は何ですか 
\\	【何をしている】
\\	を用いる言い方は目上の人には避ける. 
\\	も職業を問うが, 相手に向かって 
\\	と尋ねるのは失礼なので避ける. 
\\	は 通例名前を聞く表現だが, 職業に用いることもある. 
\\	は一時的な職業であることを暗示する. 
\\	答え方はいずれも 
\\	などでよい) 
\\	彼女の職業はプログラマーです 
\\	(!無冠詞に注意. 職種に応じて 
\\	も用いる: 
\\	職業に 
\\	教師を職業に選ぶ 
\\	著述業を職業にする 
\\	医学関係の職業に就く 
\\	彼はどんな職業に向いていますか 
\\	職業を 
\\	職業を変える 
\\	【次々と】
\\	職業を探す 
\\	⦅書⦆
\\	職業安定所 
\\	職業意識 
\\	職業教育 
\\	職業訓練 
\\	職業(訓練)学校 
\\	職業指導 
\\	職業紹介所 ⦅米⦆
\\	⦅英⦆
\\	(就職⇨第4文例) 職業適性検査 
\\	職業病 
\\	職業婦人 
\\	は通例「官公庁に勤める男性」の意) 
\\	職業別労働組合 
\\	職業(しょくぎょう)倫理(りんり) 
\\	すいい推移(する)		
\\	すいい【推移】名詞 〖変化〗
\\	〖進展〗
\\	〖移行〗
\\	時代の推移とともに 
\\	事態の推移⦅進展⦆を見守る 
\\	推移する動詞 
\\	徐々に推移する 
\\	ほぼ現状のまま推移する 
\\	すうち数値		
\\	すうち【数値】 〘数学〙
\\	数値(すうち)目標(もくひょう) 
\\	世帯		
\\	せたい【世帯】 【同居人を含めた家族】
\\	【家族】
\\	2人世帯 
\\	2世帯用住宅 
\\	[⦅米⦆ 
\\	世帯数 
\\	世帯主 
\\	かてい【家庭】名詞 
\\	〖所帯〗
\\	使い分け 
\\	家族とともに暮らす場としての家をさす
\\	暖かい家庭的雰囲気を含意する. 
\\	社会単位としての家族をさす
\\	使用人や下宿人なども含む,所帯の全員をさす
\\	~家庭 
\\	楽しい家庭 
\\	裕福な家庭 
\\	崩壊家庭 
\\	母[父]子家庭 
\\	上流[中流]家庭 
\\	あそこ[お隣り]は新婚家庭だ 
\\	幸福な家庭はみな同じように似ているが,不幸な家庭は不幸なさまもそれぞれ違うものだ 
\\	(!「幸福な家庭」は 
\\	と複数形で, 「不幸な家庭」は 
\\	と個々を強調するために単数形になっていることに注意) 
\\	家庭(の)~ 
\\	家庭(の)事情 
\\	家庭の事情で 
\\	家庭の幸福[不和] 
\\	談話 
\\	家庭ゴミの量に関しては,変化はない 
\\	(…に関しては)は関連を表すディスコースマーカー) 
\\	家庭に[を] 
\\	家庭に入る⦅結婚する[している]⦆ 
\\	(所帯⇨第1文例) 
\\	家庭を破壊する 
\\	彼は政治家の家庭に生まれた 
\\	どうか幸せなご家庭を築いてください 
\\	私たちは宮崎で家庭を持った 
\\	妻は家庭を切り盛りするのがうまい 
\\	仕事のために家庭を犠牲にした 
\\	家庭で 
\\	家庭で礼儀作法を教える 
\\	貧しい[厳格な]家庭で育つ 
\\	家庭的な形容詞 ⦅米⦆
\\	⦅英⦆ 
\\	【我が家のような】
\\	【家庭を愛する】
\\	家庭的な雰囲気 
\\	家庭的な男性 
\\	[(しばしばおどけて) 
\\	は ⦅英⦆ では「所帯の切り盛りの上手な」の意) 
\\	家庭的な女 
\\	は ⦅米⦆ 「器量の悪い人」の意) 
\\	この旅館はこの前泊まったときはもっと家庭的だった 
\\	家庭科 
\\	(!単数扱い) 
\\	家庭環境 
\\	家庭菜園 
\\	[⦅主に英⦆ 
\\	家庭裁判所 
\\	家庭生活 
\\	家庭争議 
\\	家庭内暴力 
\\	(!しばしば 
\\	という) 家庭訪問 
\\	家庭訪問をする 
\\	(!米国には 
\\	がいる) 家庭用品 【主に台所用品】
\\	【備品】
\\	そうすう総数		
\\	そうすう【総数】 
\\	出席者の総数は500人だった 
\\	[(達した) 
\\	測定(する)		
\\	そくてい【測定】名詞 
\\	時間の測定 
\\	測定する動詞 
\\	(⇨測る)
\\	対象		
\\	たいしょう【対象】 
\\	【行為思考感情などの】 
\\	〖非難などの〗
\\	研究の対象 
\\	[(題目) 
\\	輸入(ゆにゅう)制限(せいげん)の対象となっている食品(しょくひん) 
\\	彼のふるまいはしばしば非難の対象となった 
\\	【目標】 
\\	子供を対象とした本 
\\	この歴史書は高校生を対象に書かれている 
\\	調査(する)		
\\	ちょうさ【調査】名詞 
\\	【質問などによる】
\\	【学術的な】
\\	(研究⇨類語)【統計測量などによる】
\\	使い分け 
\\	調査検査の意を表す一般的な語
\\	公的機関などによる詳しい事実原因の調査
\\	市場調査 
\\	実地調査 
\\	身辺調査 【身元調査】
\\	(⇨身元) 
\\	実態調査 (⇨実態) 
\\	調査方法 
\\	その事故[伝染病]の原因調査 
\\	調査中の事柄 
\\	何人の人がその製品を使用したか調査が行われた 
\\	調査する動詞 
\\	(⇨研究する) 
\\	その事実を調査する 
\\	その殺人事件を調査する 
\\	ヘリコプターからその土地を調査する 
\\	警察はその火事の原因を調査している 
\\	さらに事件を調査した結果,彼はわいろを受け取ったことが明らかになった 
\\	調査委員会 
\\	(⇨委員会) 調査員[官] 
\\	調査書 【学業成績の】
\\	調査報告書 
\\	調査用紙 【アンケート用紙】
\\	賃金		ちんぎん【賃金】 【労賃】
\\	(⇨給料) 
\\	~賃金 
\\	最低[基本]賃金 
\\	高[低]賃金で働く 
\\	賃金~ 
\\	男女の賃金格差は大きい 
\\	(!⦅主に英⦆ では同一会社内での同種の仕事における労働者間の「賃金格差」は通例複数形で 
\\	という) 
\\	賃金交渉で組合は1万円の賃上げを獲得した 
\\	この産業の賃金水準は高い 
\\	賃金が[は] 
\\	今年は賃金がずいぶん上がった 
\\	我々の賃金が下がった 
\\	賃金は消費者物価にスライドすることになっている 
\\	賃金カット 
\\	賃金体系 
\\	賃金凍結 
\\	賃金労働者 
\\	⦅主に米⦆
\\	(⇨サラリーマン)
\\	特殊		
\\	とくしゅ【特殊】 特殊な形容詞 〖特別な, 普通でない〗
\\	〖特定の〗
\\	〖風変わりで特有の〗
\\	〖比類のない〗
\\	(⇨特別) 
\\	特殊訓練を受けた警官 
\\	単語の一般的な意味と特殊な意味 
\\	真実をゆがめてしまう特殊⦅特異⦆な才能を持っている 
\\	特殊な経験をする 
\\	特殊学級 
\\	【説明的に】
\\	特殊教育 
\\	特殊鋼 
\\	特殊講義 
\\	特殊撮影 (⇨特撮) 特殊法人 【政府の】
\\	内閣府		
\\	ないかく【内閣】 
\\	(!しばしば 
\\	個々の閣僚を考えるときは複数扱い) 
\\	単独[連立]内閣 
\\	小泉内閣 
\\	米⦆ 
\\	英⦆]. (!単複両扱い) 
\\	新内閣を組織する 
\\	内閣改造 
\\	(!「内閣を改造する」は 
\\	内閣官房長官 
\\	内閣総辞職 
\\	(⇨総辞職) 内閣総理大臣 
\\	【フランスイタリア中国などの】
\\	【ドイツなどの】
\\	(⇨首相) 内閣府 
\\	比較(する)		比較(ひかく) 
\\	比較する動詞 
\\	は特に受身で好まれる) 
\\	〖対比対照する〗
\\	(⇨比べる) 
\\	その二つを比較する 
\\	都会生活と田舎の生活を比較する 
\\	私は何事にせよ彼と比較されるのは好きではない 
\\	トムと比較するとボブの方が客観的だ 
\\	は通例省略される. 
\\	は不可) 
\\	分類(する)		
\\	ぶんるい【分類】名詞 
\\	植物の分類 
\\	分類する動詞 
\\	【グループに分ける】
\\	【仕分けする】
\\	本を著者別[アルファベット順]に分類する 
\\	それらは題目によって三つの種類に分類することができる 
\\	その本は伝記として分類されていますかそれとも小説としてですか 
\\	この本はノンフィクションに分類される 
\\	分類表 
\\	平均(する)		平均(へいきん) 
\\	保育(する)		
\\	ほいく【保育】 保育する動詞 
\\	⦅書⦆
\\	(!通例受身で) 
\\	保育園[所] 
\\	【2--5歳児の教育機関】
\\	(!⦅米⦆ では通例2--3時間預かる) 
\\	子供を保育園に預ける 
\\	保育器 
\\	保育士 
\\	目的		目的(もくてき) 
\\	労働(する)		
\\	ろうどう【労働】名詞 
\\	(仕事⇨類語) 
\\	肉体労働 
\\	筋肉労働 
\\	頭脳労働 
\\	は 
\\	のように1語でもつづる) 
\\	強制労働 
\\	重労働 
\\	日雇い労働 
\\	単純労働 
\\	時間外労働 
\\	週5日労働 ⦅米⦆
\\	⦅英⦆
\\	1日8時間労働⦅勤務⦆をする 
\\	労働する動詞 
\\	(⇨働く) 労働委員会 
\\	労働運動 
\\	労働歌 
\\	ともいう) 労働関係調整法 
\\	労働基準監督署 
\\	労働基準局 
\\	労働基準法 
\\	労働基本権 
\\	労働協約[契約] 
\\	労働権 
\\	労働災害 
\\	労働災害補償 
\\	労働三権 
\\	(!団結権争議権団体交渉権のこと) 労働時間 
\\	彼の1日の労働時間は平均8時間です 
\\	【彼は1日平均8時間働く】
\\	労働集約的[型]産業 
\\	労働条件 
\\	労働条件の悪化 
\\	労働人口 
\\	労働生産性 
\\	労働争議 
\\	【スト】
\\	労働党 【英国の】
\\	労働法 
\\	労働問題 
\\	労働力		労働力(ろうどうりょく) 
\end{CJK}
\end{document}