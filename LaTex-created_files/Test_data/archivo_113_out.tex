\documentclass[8pt]{extreport} 
\usepackage{hyperref}
\usepackage{CJKutf8}
\begin{document}
\begin{CJK}{UTF8}{min}
\\	奥	ナンプラはどこですか。 。。。その引き出しの上から二段目の奥にあります	奥[おく]					
\\	寂しい	田舎にいた時は、映画館もないし、レストランもないし、田舎の生活は嫌だと思った。 でも最近 疲れた時や 寂しい時、よく 田舎の青い空や 緑の山を思い出します	寂[さび]しい			
\\	辞書	昨日、教室に 辞書を忘れてちゃったんだ。どう したらいい 。。。松浦さんに聞いて もらいます	辞書[じしょ]			
\\	雑誌	これは 女の人が 読む雑誌です	雑誌[ざっし]			
\\	新聞	今インタネットが普通だし、毎朝 新聞を読む人がだんだん 少なくなります。	新聞[しんぶん]					
\\	手帳	今朝 どこかで 大切な手帳が落としてしまいました。 。。。そうですか。大変ですね。	手帳[てちょう]			
\\	名刺	それは部長の名刺ですか。 。。。うん、そうです	名刺[めいし]					
\\	鉛筆	ここに鉛筆で カタカナで名前を書いてください。 。。。わかりました。	鉛筆[えんぴつ]			
\\	ボールペン	ここに ボールペンで カタカナで名前を書いてください。 。。。わかりました。	ボールペン			
\\	シャープペンシル	そこに シャープペンシルで カタカナで名前を書いてください。 。。。わかりました。	シャープペンシル			
\\	机	会議室には  明日 会議もありますが、机やいすをそのままにして置いてください。	机[つくえ]			
\\	チョコレート	このケーキはチョコレートの味です	チョコレート			
\\	違います	この傘はヒエップのですか。 。。。いいえ、違います。 あ、私のです。ありがとうございます	違[ちが]います			
\\	これから お世話になります	
\\	です。 こんにちは。ヒエップです。これから お世話になります。どうぞ よろしくお願いします。 。。。こちらこそ よろしくお願いします。	これから お 世話[せわ]になります					
\\	郵便局	10区の郵便局は何時から 何時まですか。 。。。朝の8時から 昼の5時までです	郵便[ゆうびん] 局[きょく]			
\\	図書館	工科大学の図書館は 新しい本を借りることができますか。 。。。いいえ、できません。そこでは 新しい本しか 読めません。 はい、わかりました。どうも	図書館[としょかん]			
\\	い→え;
\\	られ;
\\	でき-こられ)
\\	銀行	
\\	の銀行では 金額が どのぐらい 下ろせますか。 。。。500万円下ろせます。 はい、わかりました	銀行[ぎんこう]			
\\	金額	
\\	の銀行では 金額が どのぐらい 出せますか。 。。。500万円出せます。 はい、わかりました	金額[きんがく]			
\\	出します	
\\	の銀行では 金額が どのぐらい 出せますか。 。。。500万円出せます。 はい、わかりました すみません、紙がないとき、どうしますか。 。。。あの箱から だします	出[だ]します			
\\	一昨日	もう 英語の履歴書を出しましたか。 。。。もう 一昨日 出しました	一昨日[おととい]			
\\	昨日	昨日 家に遅れて帰りました。 。。。パーティーがあったんですか	昨日[きのう]			
\\	明後日		明後日[みょうごにち]					
\\	普通	日本の電車は 普通や急行や特急 がりますね。 。。。交通が便利です	普通[ふつう]			
\\	急行	日本の電車は 普通や急行や特急 がりますね。 。。。交通が便利です	急行[きゅうこう]			
\\	卵	まず どんぶりに めん、卵、炒めた野菜を入れます。 次にどんぶりに お湯を入れて、ふたをして、3分間待ちます。 3分だったら、ふたを開けて、細かく切ったねぎを載せて、できあります	卵[たまご]			
\\	野菜	まず どんぶりに めん、卵、炒めた野菜を入れます。 次にどんぶりに お湯を入れて、ふたをして、3分間待ちます。 3分だったら、ふたを開けて、細かく切ったねぎをのせて、できあります	野菜[やさい]			
\\	紅茶	毎朝 起きたら、紅茶を飲んでいます	紅茶[こうちゃ]			
\\	牛乳	この店は牛乳が 安いし、おいしいし、いつもここで牛乳を買って飲んでいます	牛乳[ぎゅうにゅう]			
\\	~し、~し、~。 
\\	写真	美術館の中で写真を撮れますか 。。。自由に 撮ってください。	写真[しゃしん]			
\\	庭	家の庭が広くて、明るいですので。木を植えられます。	庭[にわ]			
\\	宿題	毎日 宿題がたくさん あっても、私はまじめでしています	宿題[しゅくだい]			
\\	胡瓜	私の庭に 胡瓜が植えてあります	胡瓜[きゅうり]			
\\	トマト		トマト					
\\	豆	豆は栄養がたくさんある食べ物です	豆[まめ]					
\\	葱	3分だったら ふたを開けて、細かく切った葱を載せて、それから 食べましょう	葱[ねぎ]					
\\	じゃがいも	じゃがいもから作ったスナックはおいしいですが、健康によくない	じゃがいも						
\\	たまねぎ	材料(一人分) 鳥肉(50グラム)、卵(1個)、たまねぎ(4分の1
\\	個) 調味料: (しょうゆ、砂糖、酒)、御飯	たまねぎ						
\\	人参	にんじんは 
\\	ビタミンが多いので、子どもを食べさせた方がいいです	人参[にんじん]						
\\	バナナ		バナナ						
\\	りんご	大きいりんごは小さいりんごより 酸っぱいです。	りんご						
\\	みかん	
\\	の果物の中で ミカン がおいしいです	みかん						
\\	柿	
\\	のかきがいちばん おいしいです	柿[かき]						
\\	梨	子どものとき 梨を食べるのがすきかったです。今 食べなくなりました	梨[なし]						
\\	すいか	昔、
\\	には ベトナム人は すいかを食べられます。現在 1年に いつも すいかが 食べられます	すいか						
\\	いちご	今 ベトナムには 中国のイチゴがたくさなるので、注意してください	いちご						
\\	桃	中国の果物の中で 桃が一番 好きです	桃[もも]					
\\	牛肉	牛肉は鳥肉より 高いです	牛肉[ぎゅうにく]						
\\	鳥肉	牛肉は鳥肉より 高いです	鳥[とり] 肉[にく]						
\\	豚肉	牛肉は豚肉より健康にいいです	豚肉[ぶたにく]					
\\	ソーセージ	私は洋食の中でソーセージが大好きです	ソーセージ						
\\	鯖	東京
\\	屋で 食事するとき。ぜひ 鯖を食べようと思っています	鯖[さば]			
\\	鮭	鮭の鮨が一番おいしいです	鮭[さけ]			
\\	鮪	暇なときは 海へ 鮪の釣りに行っています	鮪[まぐろ]			
\\	海老	
\\	の海老屋は とても高いな	海老[えび]			
\\	蟹	蟹はカルシウムが たくさんあるので、よく 食べた方がいいです	蟹[かに]			
\\	烏賊	烏賊なら 焼くと、おいしいです	烏賊[いか]			
\\	蛸	蛸焼はおいしいですよ	蛸[たこ]			
\\	箸	箸で食事をしていることは日本人の習慣です	箸[はし]					
\\	スプーン	西洋人はよく スプーンやナイフやフォークで 食事をしています	スプーン						
\\	ナイフ	西洋人はよく スプーンやナイフやフォークで 食事をしています	ナイフ						
\\	フォーク	西洋人はよく スプーンやナイフやフォークで 食事をしています	フォーク						
\\	フォーク	西洋人はよく スプーンやナイフやフォークで 食事をしています	フォーク						
\\	紙	昔 紙は手紙を書くのに使いませんが。材木は使います	紙[かみ]					
\\	手紙	昔 紙は手紙を書くのに使いませんが。材木は使います	手紙					
\\	材木	昔 紙は手紙を書くのに使いませんが。材木は使います	材木[ざいもく]			
\\	いらっしゃい。どうぞ お上がりください	いらっしゃい。どうぞ お上がりください 。。。失礼します。	いらっしゃい。どうぞ お 上[あ]がりください			
\\	失礼します	いらっしゃい。どうぞ お上がりください 。。。失礼します。 部長の部屋に入るとき、「失礼します」と いいます	失礼[しつれい]します			
\\	る、ない・い
\\	い・な
\\	な・
\\	の
\\	とき、~。
\\	人1は人2に
\\	と 
\\	いいます。
\\	賑やか	東京は 賑やかです 賑やかな声がしますね。 。。。ええ、パーティーでも して いるようですね。	賑[にぎ]やか				
\\	い、な・
\\	の
\\	ようです。
\\	どうも
\\	声・音・におい・味が します。
\\	静か	どうして 田舎に住みたいんですか。 。。。田舎は 静かだし、元気だし、それに 暇なとき、釣りもできます	静[しず]か			
\\	田舎	どうして 田舎に住みたいんですか。 。。。田舎は 静かだし、元気だし、それに 暇なとき、釣りもできます	田舎[いなか]					
\\	暇	どうして 田舎に住みたいんですか。 。。。田舎は 静かだし、元気だし、それに 暇なとき、釣りもできます	暇[ひま]					
\\	釣りをします	どうして 田舎に住みたいんですか。 。。。田舎は 静かだし、元気だし、それに 暇なとき、釣りもできます	釣[つ]り			
\\	素敵	いろいろな国の料理を食べながら、日本人と友達になりましょう。 踊るし、すてきなプレゼントもあるし。皆さん、ぜひ 参加してください	素敵[すてき]			
\\	冷たい	今日の天気はとても 冷たいです	冷[つめ]たい			
\\	難しい	入試験は難しい問題がたくさんあるし、合格しにくいです	難[むずか]しい			
\\	易しい	宿題は易しいです。簡単ですね	易[やさ]しい			
\\	優しい	
\\	さんは どんな 人ですか。 。。。
\\	さんは  優しくて、 上手なエンジニアですね。そうですね。	優[やさ]しい			
\\	低い	妻は背が低くても、可愛いです。	低[ひく]い			
\\	可愛い	妻は背が低くても、可愛いです。	可愛[かわい]い			
\\	桜	春になると、桜が咲きます。	桜[さくら]			
\\	寮	工科大学の寮はとても便利だ	寮[りょう]			
\\	白い		白[しろ]い			
\\	黒い	友達の結婚式に 何を着て行ったらいいですか。 。。。日本には  男の人は 黒か紺のスーツを 着て、ネクタイをして 行ったらいいです。 分かりました、どうも ありがとう ございます。	黒[くろ]い					
\\	赤い	カリナさんのかばんは どれですか。 。。。赤くて、小さいかばんです	赤[あか]い			
\\	て、
\\	て、~)・(い
\\	、い
\\	・(な
\\	な
\\	どの
\\	どれですか 
\\	。どの人ですか 
\\	青い	青いのですか。あれは 古い銅だから 緑青が かかってて青いんです。そして 250ドルですよ	青[あお]い					
\\	紫	私たちが 外に出るまで、私は紫と白のピジャマを着ていたことに 気が付かなかった。	紫[むらさき]			
\\	黄色い	黄色いバナナは一番おいしいです	黄色[きいろ]い			
\\	オレンジ	柿は 熟れるにつれ、オレンジ色に変わった	オレンジ						
\\	グレー	グレーサッツが好きじゃない。	グレー			
\\	塩辛い	塩辛い漬物は 雨水で 作った御飯と 食べるのが最高ですね	塩辛[しおから]い			
\\	酸っぱい	美には 青々とした葉があるかもしれないが、その果実は 酸っぱい。	酸[す]っぱい			
\\	嫌い	私は一人で旅行すること嫌いです	嫌[きら]い			
\\	野球	よく テレビを 見ますか。 。。。そうですね、野球の試合が あるとき、見ます	野球[やきゅう]					
\\	クラシック	どんな音楽だすきですか。 。。。クラシックが好きです	クラシック						
\\	細かいお金	ニャムさん 細かいお金がありますか。 。。。いいえ、ありません。	細[こま]かいお 金[かね]					
\\	約束	恋人に会う約束がありますが。参加できません。 。。。ちょっと 今度 お願いします	約束[やくそく]			
\\	ご主人	診んさん。ご主人はどんな人ですか。 。。。そうですか。夫はハンサムで、料理が上手ですよ。 幸せですね	ご 主人[しゅじん]					
\\	奥さん	
\\	さん、奥さんの田舎はどこですか 。。。
\\	県ですよ。 えぇ、私たちと同じですか。いいですね	奥[おく]さん					
\\	よく	この問題は本当に複雑です。よく わかりません。	よく			よくわかります 
\\	よくになります
\\	よく ばいくで 来ます
\\	だいたい	
\\	さん、韓国語が話せますか。 。。。だいたい 話せます	だいたい			
\\	少し	フランス語がわかりますか。 。。。少し わかります。	少[すこ]し					
\\	全然	今日の授業が全然わかりません。 。。。はい、そうです。難しいです	少[すこ]し			
\\	いっしょに いかがですか	ミラーさん、民謡のコンサート、いっしょに いかがですか. 。。。いいですね、いつですか。 来週の金曜日です。 。。。金曜日ですか。ちょっと 金曜日の晩は  ちょっと。。。 だめですか 。。。ええ、残念ですが、友達と約束がありますから、。。。 そうですか。 。。。ええ、また今度お願いします	いっしょに いかがですか			
\\	民謡	ミラーさん、民謡のコンサート、いっしょに いかがですか. 。。。いいですね、いつですか。 来週の金曜日です。 。。。金曜日ですか。ちょっと 金曜日の晩は  ちょっと。。。 だめですか 。。。ええ、残念ですが、友達と約束がありますから、。。。 そうですか。 。。。ええ、また今度お願いします	民謡[みんよう]			
\\	また今度お願いします	ミラーさん、民謡のコンサート、いっしょに いかがですか. 。。。いいですね、いつですか。 来週の金曜日です。 。。。金曜日ですか。ちょっと 金曜日の晩は  ちょっと。。。 だめですか 。。。ええ、残念ですが、友達と約束がありますから、。。。 そうですか。 。。。ええ、また今度お願いします	また 今度[こんど]お 願[ねが]いします			
\\	ポップス	ポップスのチケットをもらいました。一緒に行きませんか。 。。。いつですか 今週の日曜日です。 。。。日曜日ですか。残念ですが、残業をする用事がありますから。 そうですか	ポップス						
\\	ロック	私はロックが大好きです。ロックを聞く時はストレスがありません。 。。。そうですか、いいですね	ロック						
\\	ドキュメンタリ	ドキュメンタリは子どもにいいです。 。。。SF映画から たくさん物を勉強できます	ドキュメンタリ			
\\	アクション	私は映画を見ると、いつもアクション映画を見ています。 。。。なせですか。 おもしろくて、劇的です	アクション			
\\	バレーボール	どんなスポーツが好きですか 。。。バレーボールです。せが高くしたいですから、毎朝、バレーボールをしています	バレーボール			
\\	ポーリング	週末 何をしていますか。 。。。たいてい つまと いっしょに ポーリングをしています	ポーリング						
\\	スキー	日本へ来た時、ぜひ スキーやスケートをしようと思っています	スキー						
\\	スケート	日本へ来た時、ぜひ スキーやスケートをしようと思っています	スケート						
\\	卓球	大学生の時、毎日 卓球を していました	卓球[たっきゅう]			
\\	水泳	子どものとき、水泳が好きです、1週間3回に ブールで泳ぎに行きました。でも、今 時間がありませんから、泳ぎに行けないです	水泳[すいえい]			
\\	ポスト	あのう、近くポストがありますか 。。。ええ、あそこに 銀行と美術館の間に郵便局がりますが、郵便局の前にあります わかりました、ありがとうございます	ポスト			
\\	間	あのう、近くポストがありますか 。。。ええ、あそこに 銀行と美術館の間に郵便局がりますが、郵便局の前にあります わかりました、ありがとうございます	間[あいだ]			
\\	二段目	あのう、こめがありますか。 。。。はい、あちらに ベトナムの料理のコーナーがあります。 
\\	の米は上から3段目	二 段[だん] 目[め]			
\\	チリソース	あのう、チリソースがありますか。 。。。はい、あちらに ベトナムの料理のコーナーがあります。 チリソースは上から3段目	チリソース						
\\	風呂場	私のタオルがどこに置いてありますか。 。。。風呂場に置きました	風呂[ふろ] 場[じょう]			
\\	居間	母はどこにいますか。 。。。居間で妹の先生と話している	居間[いま]			
\\	切手	80円の切手や50円の切手をお願いします。 。。。何枚ですか。 80円の切手を10枚と 50えんのを 5枚です。全部でいくらですか 。。。あの、1050円です	切手[きって]			
\\	葉書	明日 
\\	さんに 葉書を送ろうと思っています。 あのお寺を見学できますか。 。。。ええ、葉書で 申し込めば、できます	葉書[はがき]			
\\	い→え
\\	ば;
\\	れば;
\\	すれば-くれば)― 
\\	い
\\	い 
\\	ければ 
\\	な 
\\	な
\\	なら 
\\	なら
\\	速達	料理は 速達で 大阪へ送ってもいいですか。 。。。はい、いいです	速達[そくたつ]			
\\	書留	大切な資料ですから。書留で田中さんに送ってください。 。。。はい、わかりました	書留[かきとめ]			
\\	航空便	これ オーストラリアまで お願いします。 。。。はい。船便ですか、航空便ですか. 航空便はいくらですか。 。。。7600円です 船便は? 。。。3450円です どのくらい かかりますか 。。。航空便で 7日、船便で 二か月 ぐらいです じゃ、船便で お願いします	航空便			
\\	船便	これ オーストラリアまで お願いします。 。。。はい。船便ですか、航空便ですか. 航空便はいくらですか。 。。。7600円です 船便は? 。。。3450円です どのくらい かかりますか 。。。航空便で 7日、船便で 二か月 ぐらいです じゃ、船便で お願いします	船便[ふなびん]			
\\	ただいま。	妻、ただいま。 。。。お帰りなさい	ただいま。			
\\	お帰りなさい	妻、ただいま。 。。。お帰りなさい	お 帰[かえ]りなさい			
\\	速い	空港まで バスと電車と どちらが 速いですか 。。。電車の方が速いです	速[はや]い			
\\	と
\\	と どちらが 
\\	ですか。。。。
\\	の方が
\\	です。
\\	生け花	日本の芸術の中で何が一番好きですか。 。。。生け花が 一番好きです	生け花[いけばな]			
\\	涼しい	京都は 東京より 涼しいですか。 。。。ええ、ずっと 涼しいです	涼[すず]しい	
\\	は 
\\	より 
\\	です。
\\	ずっと	京都は 東京より 涼しいですか。 。。。ええ、ずっと 涼しいです	ずっと			
\\	は 
\\	より 
\\	です。
\\	お祭り	祇園祭は 京都の祭りで いちばん 有名ですからね。 。。。そうですか。写真をたくさん 撮りました。これです。 わあ、多い人ですね 。。。ええ、ちょっと 疲れました	お 祭[まつ]り			
\\	甘い	タイのみかんと りんごと どちらが 甘いですか 。。 タイのみかんの方があまいです	甘[あま]い			
\\	と
\\	と どちらが 
\\	ですか。。。。
\\	の方が
\\	です。
\\	季節	日本の季節の中で 何が一番好きですか 。。。春や秋が いちばん好きです。夏は嫌いです	季節[きせつ]			
\\	と
\\	と どちらが 
\\	ですか。。。。
\\	の方が
\\	です。
\\	重い	重い掃除機と 軽いのよりと どちらが 安いです 。。。もちろん 重い掃除機は安いです	重[おも]い				
\\	と
\\	と どちらが 
\\	ですか。
\\	は 
\\	より 
\\	です。
\\	軽い	重い掃除機と 軽いのよりと どちらが 安いです 。。。もちろん 重い掃除機は安いです	軽[かる]い				
\\	と
\\	と どちらが 
\\	ですか。
\\	は 
\\	より 
\\	です。
\\	少ない	その先生の授業は生徒がだんだん 少なくなります。	少[すく]ない			
\\	辛い	タイの料理は どうでしたか。 。。。とても辛いです	辛[から]い				どうでしたか。
\\	曇り	明日の天気は 曇りでしょう。 。。。本当ですか。どうしてですか。	曇[くも]り			
\\	雪	享年の冬は雪があるでしょうか。 。。。たぶん あるでしょう。	雪[ゆき]		雪	
\\	美術	日本へ何の勉強に来ましたか。 。。。ええ。美術の勉強に来ました	美術[びじゅつ]					
\\	ご注文は?	ご注文は? 。。。私はてんぷら定食。 。。。私は牛丼。 てんぷら定食と 牛丼ですね。少々お待ちください	ご 注文[ちゅうもん]は?			
\\	定食	ご注文は? 。。。私はてんぷら定食。 。。。私は牛丼。 てんぷら定食と 牛丼ですね。少々お待ちください	定食[ていしょく]					
\\	少々お待ちください	ご注文は? 。。。私はてんぷら定食。 。。。私は牛丼。 てんぷら定食と 牛丼ですね。少々お待ちください	少々[しょうしょう]お 待[ま]ちください			
\\	牛どん	ご注文は? 。。。私はてんぷら定食。 。。。私は牛どん。 てんぷら定食と 牛丼ですね。少々お待ちください	牛[うし]どん			
\\	別々に お願いします	1680円です 。。。すみません・別々に お願いします はい、てんぷら定食は980円です。牛どんは 700円です	別々[べつべつ]に お 願[ねが]いします			
\\	お土産	京都のお土産です。どうぞ 。。。ええ、ありがとうございます	お 土産[みやげ]			
\\	遊びます	どこへ 遊びに行きますか。 。。。友達の家へ遊びに行きます	遊[あそ]びます			
\\	食事します	どこへ 食事に行きますか。 。。。インドの料理のレストランへ 料理に行きます	食事[しょくじ]します			
\\	迎えます	何時に 子どもを迎えに行きますか。 。。。2時ごろです	迎[むか]えます			
\\	広い	すみません、「おはようテレビ」ですが。今何が 欲しいですか 。。。広いうちがほしいです。今のうちは 狭いですから。 そうですか.ありがとう ござます	広[ひろ]い			
\\	が欲しいです。(何が一番欲しいですか)	
\\	狭い	すみません、「おはようテレビ」ですが。今何が 欲しいですか 。。。広いうちがほしいです。今のうちは 狭いですから。 そうですか.ありがとう ござます	狭[せま]い				
\\	が欲しいです。(何が一番欲しいですか)	
\\	結婚します	すみません、「おはようテレビ」ですが。今何が 欲しいですか 。。お金が欲しいです。来年結婚しますから。 そうですか.ありがとう ござます	結婚[けっこん]します					
\\	のどが乾きました	のどが乾きましたね。 。。。ええ、何を飲みたいですね。 あそこで 何か 飲みませんか。 。。。ええ、そう しましょう。	のどが 乾[かわ]きました			
\\	そう しましょう。	のどが乾きましたね。 。。。ええ、何を飲みたいですね。 あそこで 何か 飲みませんか。 。。。ええ、そう しましょう。	そう しましょう。			
\\	お腹がすきましたね	お腹がすきましたね。 。。。ええ、何か食べたいですね。 あそこ喫茶店を入って、 何か 食べませんか。 。。。ええ、そう しましょう。	お 腹[なか]がすきましたね			
\\	お腹がいっぱいです	今 お腹がいっぱいです。なんでも 食べられない。 。。。そうですか。一人で食べに行かないと。さびしいな	お 腹[なか]がいっぱいです			
\\	喫茶店を出ます	喫茶店を出たら、どこか 行きませんか。 。。。ええと、ああ映画を見に行きましょう そう しましょう	喫茶店[きっさてん]を 出[で]ます			
\\	住所	すみませんが、お住所を教えていただけませんか 。。。はい、
\\	区 
\\	氏」です	住所[じゅうしょ]			
\\	右へ曲がります	すみません、あの信号を右へ曲がってください。 。。。右ですね ええ	曲[ま]がります			
\\	まっすぐ	まっすぐですか。 。。。ええ、まっすぐ 行って ください。 。。。あの 花屋の前で 止めてください。 はい。1800円です 。。。これでお願いします 3200円のお釣りです。ありがとうございます。	まっすぐ			
\\	これでお願いします	まっすぐですか。 。。。ええ、まっすぐ 行って ください。 。。。あの 花屋の前で 止めてください。 はい。1800円です 。。。これでお願いします 3200円のお釣りです。ありがとうございます。	これでお 願[ねが]いします			
\\	お釣りです	まっすぐですか。 。。。ええ、まっすぐ 行って ください。 。。。あの 花屋の前で 止めてください。 はい。1800円です 。。。これでお願いします 3200円のお釣りです。ありがとうございます。	お 釣[つ]りです			
\\	塩	塩を取りましょうか。 。。。いいえ、けっこうです	塩[しお]					
\\	ゆっくり	今 頭ががんがんしますよ 。。。ええ、 ゆっくり 休んだほうがいいです	ゆっくり			
\\	呼びます	タクシーを呼びましょうか。 ええ、2台呼んでください。 今朝 部長に 呼ばれました。 。。。何か あったんですか 出張のレポートの 書き方に ついて 注意されました	呼[よ]びます			
\\	ます
\\	ます
\\	ましょうか。
\\	すみません、お願いします- 
\\	いいえ、けっこうです 
\\	人1は 
\\	2に 
\\	3を 
\\	人1 
\\	い→かれます;
\\	られます;
\\	されます-こられます)
\\	砂糖	すみません、 。。。はい ちょっと 砂糖を取ってください。 はい、どうぞ コーヒーが 砂糖を入れて 飲みます	砂糖[さとう]				
\\	て・Vないで 
\\	砂糖	すみません、 。。。はい ちょっと 砂糖を取ってください。 はい、どうぞ	砂糖[さとう]					
\\	あれ?	さあ、行きましょう。あれ?ミラーさんが いませんね。 。。。あとらで 写真を 撮っています すみませんが、呼んでください	あれ?			
\\	ています。
\\	浴びます	朝、 ジョギングをして、シャワーを浴びて、大学に 通います	浴[あ]びます			
\\	て、
\\	て、~)・(い
\\	、い
\\	・(な
\\	な
\\	ジョギング	朝、 ジョギングをして、シャワーを浴びて、大学に 通います	ジョギング			
\\	て、
\\	て、~)・(い
\\	、い
\\	・(な
\\	な
\\	明るい	この部屋は 広くて、明るいです このつまみを 右へ回すと、でんきが 明るく なります	明[あか]るい			
\\	て、
\\	て、~)・(い
\\	、い
\\	・(な
\\	な
\\	ると、~。
\\	い
\\	い
\\	く・
\\	な
\\	な
\\	に・
\\	に 
\\	なります。
\\	乗ります	大学まで どうやって 行きますか。 。。。京都駅から 16番のバスに 乗って、大学前で降ります	乗[の]ります			
\\	(電車に)	
\\	て、
\\	て、~)・(い
\\	、い
\\	・(な
\\	な
\\	降ります	大学まで どうやって 行きますか。 。。。京都駅から 16番のバスに 乗って、大学前で降ります	降[お]ります			
\\	(電車を降ります)	
\\	て、
\\	て、~)・(い
\\	、い
\\	・(な
\\	な
\\	髪	マリアさんは どの人ですか 。。。あの 髪が長い人です	髪[かみ]			
\\	どの
\\	若い	香川さんはどんな人ですか 。。。香川さんは 若くて、頭がいいです	若[わか]い			
\\	て、
\\	て、~)・(い
\\	、い
\\	・(な
\\	な
\\	頭がいい	香川さんはどんな人ですか 。。。香川さんは 若くて、頭がいいです	頭[あたま]がいい			
\\	て、
\\	て、~)・(い
\\	、い
\\	・(な
\\	な
\\	乗り換えます	京都駅から 
\\	に 乗って、大阪で 地下鉄に乗り換えます 。。。はい、わかりました	乗り換[のりか]えます			
\\	て、
\\	て、~)・(い
\\	、い
\\	・(な
\\	な
\\	下ろします	お金を下ろして、買い物に行きます 荷物を 下ろしたいんですが。ちょっと 車を 止めさせて いただけませんか。 。。。いいですよ	下[お]ろします			
\\	てから、
\\	私は人に
\\	てもらいます。
\\	人
\\	ていただけませんか。
\\	い→かせます;
\\	させます;
\\	されます-こさせます)
\\	ボタンを押します	お金を入れて、ボタンを押してください 。。。はい、わかりました。どうも	押[お]します				
\\	てから、
\\	一度	もう レポートを 出しましたか。 。。。いいえ、まだです。 もう 一度読んでから、出します。	一 度[ど]				もう~た。
\\	もう~た。- 
\\	いいえ、まだです 
\\	暗い	私の部屋は 狭くて、暗いです	暗[くら]い			
\\	て、
\\	て、~)・(い
\\	、い
\\	・(な
\\	な
\\	サービス	大阪には ホテルが どうでしたか。 。。。静かで、サービスがよかったです	サービス			
\\	て、
\\	て、~)・(い
\\	、い
\\	・(な
\\	な
\\	お寺	昨日はどこか 行きましたか。 。。。ええ、京都へ行きました そうですか。京都で何をしましたか 。。。友達に会って、食事して、それから お寺を見ました。	お 寺[てら]			
\\	て、
\\	て、~)・(い
\\	、い
\\	・(な
\\	な
\\	出ます	日本語が上手です。 。。。どのくらい 勉強しましたか。 1年ぐらいです。大学を出て、始めました。 。。。そうですか。すごいですね いいえ、まだまだです	出[で]ます			
\\	て、
\\	て、~)・(い
\\	、い
\\	・(な
\\	な
\\	入ります	日本語が上手です。 。。。どのくらい 勉強しましたか。 1年ぐらいです。会社に入って、始めました。 。。。そうですか。すごいですね いいえ、まだまだです	入[はい]ります					
\\	大丈夫	もう 12時ですよ。一人で 大丈夫ですか 。。。ええ 心配しないでください。タクシで 帰りますから	出[で][で]ます			
\\	ないで ください。
\\	心配します	もう 12時ですよ。一人で 大丈夫ですか 。。。ええ 心配しないでください。タクシで 帰りますから 家族が 心配しないように、毎週連絡しています 。。。はい、わかりました	心配[しんぱい]します			
\\	ないで ください。
\\	払います	子どもも お金を 払わなければなりませんか。 。。。いいえ、払わなくても いいです	払[はら]います				なければなりません・ないといけません
\\	ないと)。
\\	なくても いいです。
\\	明日 ここに 来なくてもいいです)	
\\	残業します	もしもし、夫 もう 家へ帰りましたか。 。。。いいえ、まだです。残業しなければなりませんから	残業[ざんぎょう]します			
\\	もう~たか。
\\	もし~た。
\\	いいえ、まだです。 
\\	なければなりません・ないといけません
\\	ないと)。
\\	出張します	今晩 飲みに 行きませんか。 。。。すみません、明日から ホンコンへ 出張しないと。ですから、 早く帰ります 部長は ミラーさんを アメリカへ 出張させました	出張[しゅっちょう]します			
\\	なければなりません・ないといけません
\\	ないと)。
\\	は人を
\\	は人に
\\	を
\\	は人に
\\	を
\\	い→かせます;
\\	させます;
\\	されます-こさせます)
\\	吸います	禁煙ですから。タバコを吸わないでください 。。。はい、 わかりました、すみません	吸[す]います			
\\	ないで ください。
\\	なくします	大切な資料ですから。なくさないでください 。。。はい、わかりました	なくします			
\\	ないで ください。
\\	覚えます	毎日 いくつ 漢字を 覚えなければなりませんか。 。。。六つ 覚えなければなりません。	覚[おぼ]えます			
\\	なければなりません・ないといけません
\\	ないと)。
\\	出かけます	用事があるので、出かけなければなりません。 ミラーさんが 来たら、出かけましょう	出[で]かけます			
\\	なければなりません・ないといけません
\\	ないと)。
\\	たら、~。
\\	(~てから、~)(~た・
\\	の
\\	あとで、)		
\\	脱ぎます	ここで 靴を脱がなければなりませんか 。。。はい、靴を履くのは 入ってはいけません	脱[ぬ]ぎます			
\\	なければなりません・ないといけません
\\	ないと)。
\\	3日	はい、終わりましたよ。2、3日 お風呂に 入らないでください。 。。。はい、わかりました。 ありがとう ございます	
\\	日[にち]			
\\	ないで ください。
\\	どう しましたか。	どう しましたか。 。。。昨日から のどが痛くて、熱も 少しあります 風ですね。2、3日 ゆっくり 休んで ください。それから 今晩は お風呂に入らないでください。 。。。はい、わかりました じゃ、お大事に	どう しましたか。			
\\	のどが痛い	どう しましたか。 。。。昨日から のどが痛くて、熱も 少しあります 風ですね。2、3日 ゆっくり 休んで ください。それから 今晩は お風呂に入らないでください。 。。。はい、わかりました じゃ、お大事に	のどが 痛[いた]い					
\\	熱があります	どう しましたか。 。。。昨日から のどが痛くて、熱も 少しあります 風ですね。2、3日 ゆっくり 休んで ください。それから 今晩は お風呂に入らないでください。 。。。はい、わかりました じゃ、お大事に	のどが 痛[いた]い					
\\	お大事に	どう しましたか。 。。。昨日から のどが痛くて、熱も 少しあります 風ですね。2、3日 ゆっくり 休んで ください。それから 今晩は お風呂に入らないでください。 。。。はい、わかりました じゃ、お大事に	お 大事[だいじ]に			
\\	頭が痛い	どう しましたか。 。。。昨日から 頭が痛います。頭が がんがんする ゆっくり休んで方がいい	頭[あたま]が 痛[いた]い			
\\	お腹が痛い	どう しましたか。 。。。昨日から お腹が痛いですが。何でも 食べたいです ゆっくり休んで方がいい	お 腹[なか]が 痛[いた]い			
\\	がんがんします	どう しましたか。 。。。昨日から 頭が痛います。頭が がんがんする ゆっくり休んで方がいい	がんがんします			
\\	歯が痛い	どう しましたか。 。。。昨日から 歯が痛くて、熱が続けているんですが。 ゆっくり休んで方がいいですよ。また お風呂に入らないでください	歯[は]が 痛[いた]			
\\	て、
\\	て、~)・(い
\\	、い
\\	・(な
\\	な
\\	ないで ください。
\\	鼻水が出ます	どう しましたか。 。。。昨日から せきが出て、鼻水が出るんですが。 すぐ 薬を飲んだほうがいいです。 。。。薬を飲んでも、治さなかったです 病院へ行ったほうがいいです	せきが 出[で]ます			
\\	て、
\\	て、~)・(い
\\	、い
\\	・(な
\\	な
\\	ない。た ほうがいいです。
\\	て・いTTくて・なTTで・
\\	で
\\	も、~。
\\	いくら~も、 ~。
\\	血が出ます	どう しましたか。 。。。今朝は 手に けがをしたので、血が少し出ているのが続けているんですが。 病院へ行ったほうがいいです	血[ち]が 出[で]ます			
\\	て、
\\	て、~)・(い
\\	、い
\\	・(な
\\	な
\\	ない。た ほうがいいです。
\\	吐き気がします	どう しましたか。 。。。今朝は 
\\	屋のパンを食べた後で、ずっと 吐き気がしてきます。 もう 薬を飲みましたか。 。。。いいえ、まだ 飲んでいません。 すぐ 飲んでください。	吐き気[はきけ]がします			
\\	て、
\\	て、~)・(い
\\	、い
\\	・(な
\\	な
\\	もう~たか。
\\	もし~た。
\\	いいえ、まだです。 
\\	て ください。 
\\	て いださいませんか。
\\	ていただけませんか 
\\	寒気がします	どう しましたか。 。。。いつも 夜になると、いつも 寒気がしてきますよ。 なぜですか。 。。。原因をしりません。 すぐ 病院に 行ったほうがいいですよ	寒気[さむけ]がします			
\\	て、
\\	て、~)・(い
\\	、い
\\	・(な
\\	な
\\	ない。た ほうがいいです。
\\	めまいがします	今朝 妻は 朝ごはんを作っているとき、急に めまいがしました。また よく めまいしてきていす。先生 妻に原因を調べてくれますか。 。。。はい、大丈夫です。ちょっと 診てみてあげます はい、ありがとう ございます	めまいがします			
\\	人1は 人2(私
\\	に 
\\	て)くれます・くれます・くださいます。
\\	人1
\\	人2. (くれます:
\\	(くれます:
\\	-(くださいます:
\\	人1(私)は 人2に 
\\	て)やります・あげます・さしあげます。
\\	人1
\\	人2
\\	(やります:
\\	(あげます:
\\	-(さしあげます:
\\	下痢をします	大変だよ。今朝から お腹の調子が悪いですけど。 ずっと 下痢をします。まだ 薬を飲んでも、治さないです。 。。。大変ですね。病院に行ったほうがいいですよ。連れて行きましょう。 すみません、お願いします	下痢[げり]をします			
\\	て・いTTくて・なTTで・
\\	で
\\	も、~。
\\	いくら~も、 ~。
\\	ない。た ほうがいいです。
\\	便秘をします	2週間ぐらい ずっと 便秘をするのが続けてきますから。大変ですよ。 。。。今まで たくさん薬を飲んでも、よく ならないです。 ああ、この葉を飲んでみてください。子どもから 便秘をしたとき、たいてい 飲んでいます	便秘[べんぴ]をします			
\\	食欲がない	昨日から 食欲がなかったです。どうして 食べないか、知りません。 。。。いま とても 疲れます。	食欲[しょくよく]がない			
\\	肩がこります	母は年を取るので、よく 肩がこってきます。大変ですよ。 。。。お母さんは 毎日 体操をしたら、健康にいいです	肩[かた]がこります			
\\	体がだるい	もう 10時ですよ。大変ですね。今朝から ずっと 働いているので、体がだるいですが。 。。。なかなか 帰りましょう ちょっと メールを送っていまいますから。お先に どうぞ 。。。お大事に	体[からだ]がだるい			
\\	かゆい
\\	蚊に手を刺されたので、とてもかゆいですよ。	かゆい
\\	い→かれます;
\\	られます;
\\	されます-こられます) 人1は 
\\	2に 
\\	3を 
\\	人1 
\\	捻挫	今日の試合で足の捻挫があったので、今度の試合に出られません。 。。。残念ですよ。	捻挫[ねんざ]			
\\	足	今日の試合で足の捻挫があったので、今度の試合に出られません。 。。。残念ですよ。	足[あし]			
\\	骨折	今日は バイクを運転するとき、事故がありましたので、足が骨折します。 。。。治すのは 1年ぐらい かかります。	骨折[こっせつ]			
\\	耳	今 耳の調子が悪いですけど。何でも 聞こえません。ちょっと すみません。 。。。はい、すぐ 病院に行ったほうがいいですよ。 なになに、聞こえません。 。。。書きましょう。どうぞ	耳[みみ]			
\\	顎	顎はちょっと。。。美容院に行ったほうがいいですよ。 。。。わかった。まあ いか。。。	顎[あご]			
\\	くび	皆さん、長いくびがある主人公がいる昔話を知っていますか。 。。。いいえ、しりません	くび			
\\	肩	母は年を取るので、よく 肩がこってきます。大変ですよ。 。。。お母さんは 毎日 体操をしたら、健康にいいです	肩[かた]			
\\	背中	今背中が強いになるように、毎日 運動しています	背中[せなか]			
\\	腰	妻の腰が細くてしたいですけど。どうしたらいいですか。 。。。これから ダイエットをしながら 運動します	腰[こし]			
\\	腕	今 腕の調子が悪いですけど、重い物を持てません。 。。。はい、わかった	腕[うで]			
\\	ひじ		ひじ			
\\	爪	今爪を切っていますね。	爪[つめ]			
\\	尻	きれいな尻ですね。どうやって きれいになりますか	尻[しり]			
\\	弾きます	先生 ギターを弾くことができますか。 。。。はい、簡単な曲を 弾くことができます いつから ギターを弾きましたか 。。。大学1年生 からです。 すごいですね。 。。。いいえ、まだまだです	弾[はじ]きます			
\\	ること) 
\\	ができます。
\\	い→え;
\\	られ;
\\	でき-こられ)	
\\	歌い	リサン どんな歌を歌いたいですか。 。。。ポップスを歌いたいます	歌[うた]い			
\\	趣味	
\\	さんの趣味は何ですか。 。。。私の趣味はサッカーを見ることや、バイクで旅行に行くことです	趣味[しゅみ]			
\\	私の(趣味・夢)は 
\\	ること)です。
\\	自転車に乗ります	妹は 自転車に乗ることができませんので、毎日私は 学校まで 連れて行ってやります。 。。。大変だよ	自転車[じてんしゃ]に 乗[の]ります			
\\	ること) 
\\	ができます。
\\	い→え;
\\	られ;
\\	でき-こられ) 人1(私)は 人2に 
\\	て)やります・あげます・さしあげます。
\\	人1
\\	人2
\\	(やります:
\\	(あげます:
\\	-(さしあげます:
\\	運転します	眠いとき、ガムをかみながら 運転します。 。。。そうですか。私は 車を止めて、しばらく寝ます	運転[うんてん]します			
\\	現金	カードで 払うことができますか。 。。。すみませんが、現金でお願いします	現金[げんきん]			
\\	ること) 
\\	ができます。
\\	い→え;
\\	られ;
\\	でき-こられ)	
\\	予約します	旅行の前に、ホテルを予約したほうがいいです 。。。はい、わかりました。どうも	予約[よやく]します				
\\	る・
\\	の・
\\	まえに、
\\	換えます	ここで お金を換えることができますか。 。。。ドルと 円は 換えられますが、ドンは帰られません はい、わかりました	換[か]えます			
\\	ること) 
\\	ができます。
\\	い→え;
\\	られ;
\\	でき-こられ)
\\	馬に乗ります	ニャムさん 馬に乗ったことがありますか。 。。。あ、ダラットへ旅行に行ったとき、馬に乗ったことがあります。 どうでしたか 。。。とても 難しいですが、おもしろいです	馬[うま]に 乗[の]ります			
\\	たことがあります。
\\	もう~た
\\	日記	毎晩寝る前に、何をしていますか。 。。。ええ、いつも 寝る前に日記を書いています 毎日 日記を 書くように しています	日記[にっき]				
\\	る・
\\	ない
\\	ように します
\\	・しています
\\	る・
\\	ない
\\	ように してください。
\\	お祈りをします	毎晩寝る前に、何をしていますか。 。。。ええ、いつも 寝る前にお祈りをしています	お 祈[いの]りをします			
\\	跳びます	子どもを見たとおりに、跳んでください。	跳[と]びます			
\\	た・
\\	の
\\	とおりに、V2
\\	登ります	皆さん、公園の中で きを登ってはいけません。 。。。わかりました。ありがとう ござます	登[のぼ]ります			
\\	ては いけません(いけない)。
\\	もぐります	沖縄の海は本当にきれいです。ぜひ 夏休みは沖縄へもぐりに行こうと思っています	もぐります			
\\	ようと 思っている。
\\	飛び込みます	ブールに飛び込みることがとても難しいです。何回も選手を見たとおりに、飛び込みできません。 。。。できるだけ 頑張れ	飛び込[とびこ]みます			
\\	た・
\\	の
\\	とおりに、V2
\\	逆立ちします	私は合気道を練習したことがありますから、逆立ちすることができます 。。。すごいですね いいえ、まだまだです	逆立[さかだ]ちします			
\\	たことがあります。
\\	もう~た
\\	ること) 
\\	ができます。
\\	い→え;
\\	られ;
\\	でき-こられ)	
\\	這います	このビルでは 火事に 走るな。その場合は 這えよ。 。。。わかりました	這[は]います			
\\	る、ない、た・
\\	い、な・
\\	の
\\	場合は、~。
\\	い→え;
\\	ろ;
\\	しろ-こい)
\\	蹴ります	
\\	がほうしで、私が蹴ったとおりに、蹴ってください	蹴[け]ります			
\\	た・
\\	の
\\	とおりに、V2
\\	振ります	警察が振っている時、車を止めなければなりません	振[ふ]ります			
\\	なければなりません・ないといけません
\\	ないと)。
\\	持ち上げます	今背中の調子が悪いので、この荷物を持ち上げできません。 。。。はい、すぐ休んだほうがいいですよ	持ち上[もちあ]げます			
\\	叩きます	妻 つかれたようですね。背中に叩きましょうか。 。。。すみません。お願いします	叩[たた]きます			
\\	い、な・
\\	の
\\	ようです。
\\	どうも
\\	ます
\\	ます
\\	ましょうか。
\\	すみません、お願いします- 
\\	いいえ、けっこうです 
\\	曲げます	日に三回 これのように 足を曲げてください。 。。。はい、先生 ありがとう ございます	曲[ま]げます			
\\	振り向きます	振り向かないでください	振り向[ふりむ]きます			
\\	山に登ります	東京へ 山に登りに行った事が ありますか。 。。。はい、一度 あります。2年前に、友達と 行きました	山[やま]に 登[のぼ]ります			
\\	たことがあります。
\\	もう~た
\\	調子がいい	体の調子はどうですか。 。。。あかげさまで よくなります。 いいですね。	調子[ちょうし]がいい			
\\	い
\\	い
\\	く・
\\	な
\\	な
\\	に・
\\	に 
\\	なります。
\\	おかげさまで	体の調子はどうですか。 。。。あかげさまで よくなります。 いいですね。	あかげさまで			
\\	ダイエット	マリアさん あまり 食べませんね、 。。。ええ、 昨日から ダイエットをして います。 そうですか。私も ダイエットを したことが あります。 。。。どんな ダイエットですか 毎日 りんごだけ 食べたり、水をたくさん 飲んだりしました。でも 無理なダイエットは 体に  よくないですね。	ダイエット			
\\	無理な	マリアさん あまり 食べませんね、 。。。ええ、実は 昨日から ダイエットをして います。 そうですか。私も ダイエットを したことが あります。 。。。どんな ダイエットですか 毎日 りんごだけ 食べたり、水をたくさん 飲んだりしました。でも 無理なダイエットは 体に  よくないですね。	無理[むり]な			
\\	上ります	東京スカイツリーに上ったことがありますか。 はい、先週 友達と 上りました。 東京スカイツリーへ 行った事がありますか 。。。はい、この間 行ったばかりです	上[のぼ]ります			
\\	たことがあります。
\\	もう~た
\\	た ばかりです。
\\	泊まります	
\\	のホテルに 泊まったことがありますか。 はい、2年前に 忘年会がありました。同僚と 泊まります。 友達の家に 泊まるときは 必ず  連絡するように してください	泊[と]まります			
\\	掃除します	週末は何をしていますか。 。。。私は たいてい 掃除したり、洗濯した しています	掃除[そうじ]します			
\\	たり、
\\	たり します・しました・している。
\\	洗濯します	週末は何をしていますか。 。。。私は たいてい 掃除したり、洗濯した しています	洗濯[せんたく]します			
\\	たり、
\\	たり します・しました・している。
\\	強い	体が強くなるように、毎日ゾギンルをしています	強[つよ]い			
\\	弱い	この本は弱いあなたが勉強するのに役に立ちます 。。。そうですね。	弱[よわ]い			
\\	ビザ	ベトナム人は日本の旅行がビザが要ります。 。。。はい、要りますね	ビザ			
\\	要ります	ベトナム人は日本の旅行がビザが要ります。 。。。はい、要りますね	要[い]ります			
\\	たぶん	ミラーさんはどこですか。 ・・・たぶん もう 帰ったと 思います	たぶん			
\\	と 思います。
\\	でしょう、かもしれません)		
\\	物価	日本に ついて どう思いますか。 ・・・物価が高いと思います	物価[ぶっか]			
\\	と 思います。
\\	でしょう、かもしれません)	
\\	お祈り	食事の前に お祈りを しますか。 ・・・いいえ、しませんが、「いただきます」といます。	お 祈[いの]り			
\\	人1は人2に
\\	と 
\\	いいます。
\\	月	かぐや姫は「月へ帰らなければなりません」と いました。 そして、 月へ 帰りました。終わり。 。。。終わり?お母さん、私も月へ行きたい。	月[つき]			
\\	人1は人2に
\\	と 
\\	いいます。
\\	無駄	会議で 何か 意見を 言いましたか。 。。。はい、無駄なコピーが 多いと 言いました。	無駄[むだ]			
\\	で。
\\	会議で~) 人1は人2に
\\	と 
\\	いいます。
\\	お祭りがあります	7月に 京都で お祭りが あるでしょう? 。。。ええ、あります	お 祭[まつ]りがあります			
\\	・い
\\	・(な
\\	だ
\\	でしょう。
\\	でしょう 
\\	ほんとう	私は彼の話は 本当だと思います	ほんとう			
\\	と 思います。
\\	でしょう、かもしれません)	
\\	勝ちます	明日は 「ヨーネン」と「アキックス」どちらが勝ちますか。 。。。きっと 「ヨーネン」が勝つと思います	勝[か]ちます			
\\	と 思います。
\\	でしょう、かもしれません)	
\\	きっと	明日は 「ヨーネン」と「アキックス」どちらが勝ちますか。 。。。きっと 「ヨーネン」が勝つと思います	きっと			
\\	交通	日本の交通に ついて どう 思いますか。 ・・・交通が便利だと思います	交通[こうつう]				
\\	と 思います。
\\	でしょう、かもしれません)	
\\	地球	ガリレオは 「地球はうごきます」と言いました ガガーリンは 「地峡は 青かったです」と 言いました	地球[ちきゅう]			
\\	人1は人2に
\\	と 
\\	いいます。
\\	試合	昨日 サッカーの試合があったでしょう?	試合[しあい]			
\\	・い
\\	・(な
\\	だ
\\	でしょう。
\\	でしょう 
\\	そんなに	その カメラは 高かったでしょう。 。。。いいえ、そんなに 高くなかったです。	そんなに			
\\	・い
\\	・(な
\\	だ
\\	でしょう。
\\	でしょう 
\\	すごい	ここは 冬雪が すごいでしょう? ・・・ええ、本当にすごいです	すごい			
\\	・い
\\	・(な
\\	だ
\\	でしょう。
\\	でしょう 
\\	気をつけます	インターネットに ついて どう 思いますか。 ・・・そうですね。便利ですが。使い方に 気をつけなければならないと 思います ・・・ワットさんは どう思いますか 私も そう 思います	気[き]をつけます			
\\	と 思います。
\\	でしょう、かもしれません)		
\\	放送	今 放送が ありましたね。何と言いましたか 。。。今日は9時半から 中に入ることが できると言いました そうですか。ありがとう	放送[ほうそう]			
\\	人1は人2に
\\	と 
\\	いいます。
\\	久しぶりですね	あ、サントスさん、久しぶりですね。 。。。あ、松本さん、お元気ですか ええ、ちょっと ビールでも飲みませんか 。。。いいですね	久[ひさ]しぶりですね			
\\	でも 
\\	ませんか。
\\	ちょっと ビールでも 飲みませんか)	
\\	撮ります	これは 万里の長城で 撮った写真です 。。。ええ。とても きれいですね	撮[と]ります			
\\	シャツを着ます	あの着物を着ている人は だれですか 。。。ああ、山田先生です	シャツを 着[き]ます			
\\	和室	では、こちらは いかがですか。駅から 10分で、家賃は83000円です 。。。ダイニングキッチンと 和室ですね。	和室[わしつ]			
\\	押入れ	すみません、ここは何ですか。 。。。押入れです。布団を 入れるところですよ。 そうですか、今日 見ることができますか 。。。ええ、今から 行きましょうか ええ、お願いします	和室[わしつ]			
\\	布団	すみません、ここは何ですか。 。。。押入れです。布団を 入れるところですよ。 そうですか、今日 見ることができますか 。。。ええ、今から 行きましょうか ええ、お願いします	布団[ふとん]			
\\	眼鏡をかけます	私は あの 眼鏡をかけている人を 知っています	眼鏡[めがね]をかけます			
\\	帽子をかぶります	カリナさんが かぶっている 帽子は デザインがおもしろくて、きれいです	帽子[ぼうし]をかぶります						
\\	コートを着ます	ああ、奥さんが着ているコートはきれいです。課長が奥さんに誕生日のプレゼントをあげますか。 。。。はい、そうですよ。	コートを 着[き]ます			
\\	靴をはきます	すみません、小林さんが はいている 靴は どんな靴ですか。 。。。
\\	の運動靴です	靴[くつ]をはきます			
\\	運動靴	すみません、小林さんが はいている 靴は どんな靴ですか。 。。。
\\	の運動靴です	運動[うんどう] 靴[くつ]			
\\	生まれます	私が生まれた所は
\\	県です	生[う]まれます			
\\	お弁当	私が 毎朝 お弁当を買っているコンビニは 駅の前にあります	お 弁当[べんとう]			
\\	捨てます	要りない物を 捨ててください。 。。。はい、わかりました	捨[す]てます			
\\	ヒューモア	ヒューモアが ある人が好きです	ヒューモア			
\\	シャツ	パーティーで着る服が 要りますが。日本で パーティーに 何がいいですか 。。。Tシャツや ジーンズが いいです	
\\	シャツ[てぃーしゃつ]			
\\	ジーンズ	パーティーで着る服が 要りますが。日本で パーティーに 何がいいですか 。。。Tシャツや ジーンズが いいです	ジーンズ			
\\	セーター	あの人は だれですか 。。。どの人ですか 赤いセーターを着ている人です 。。。ああ、高山先生です。	セーター			
\\	ホワイトシャツ	日本会社の面接には 何を 着たらいいですか。 。。。ああ、スーツを着たほうがいいです スーツ?何ですか 。。。ああ、こは ホワイトシャツと ズボンと ネクタイと ベストと ベルトを 一緒に 着ることです 。。。そうですか。わかりました。ありがとうございます	ホワイトシャツ			
\\	ズボン	日本会社の面接には 何を 着たらいいですか。 。。。ああ、スーツを着たほうがいいです スーツ?何ですか 。。。ああ、こは ホワイトシャツと ズボンと ネクタイと ベストと ベルトを 一緒に 着ることです 。。。そうですか。わかりました。ありがとうございます	ホワイトシャツ			
\\	ネクタイ	日本会社の面接には 何を 着たらいいですか。 。。。ああ、スーツを着たほうがいいです スーツ?何ですか 。。。ああ、こは ホワイトシャツと ズボンと ネクタイと ベストと ベルトを 一緒に 着ることです 。。。そうですか。わかりました。ありがとうございます	ネクタイ			
\\	ベスト	日本会社の面接には 何を 着たらいいですか。 。。。ああ、スーツを着たほうがいいです スーツ?何ですか 。。。ああ、こは ホワイトシャツと ズボンと ネクタイと ベストと ベルトを 一緒に 着ることです 。。。そうですか。わかりました。ありがとうございます	ベスト			
\\	ワンピース	
\\	さん、このワンピースがきれいなんですが。どこか 買ったんですか。 。。。おくさんにあげて、かいたいでしょう?ABCスーパーで買ったんです はい、どうも この ワンピースを 着てみても いいですか。 。。。はい、いいです	ワンピース			
\\	て みます。
\\	上着	今寒いですが。上着を買いたいんですが。どこか いい店がありますか。 。。。上着なら、
\\	屋が 一番いいです	上着[うわぎ]			
\\	スカート	妻の誕生日に スカートを買ってあげたいんですが。でも、いい店が知りません。 。。。教えていただけませんか	スカート			
\\	ブラウス	ブラウスは 事務所に役に立ちますか。 。。。日本人は よく ブラウスを着て 働きに行っていますよ。 ええ、そうですか、	ブラウス			
\\	下着	昨日は 買い物をしまったんです。何を買ったんですか。 。。。ああ、下着や靴下を買ったんです	下着[したぎ]			
\\	靴下	昨日は 買い物をしまったんです。何を買ったんですか。 。。。ああ、下着や靴下を買ったんです	靴下[くつした]			
\\	レインコート	妻、今から 雨がふりそうですよ。レインコートを持って行ってください。 。。。ああ、どうも、	レインコート			
\\	ます 
\\	ます そうです。
\\	今にも もうすぐ これから)		
\\	お湯	お湯を出ません。 。。。そこを 押すと、出ますよ	お 湯[ゆ]			
\\	ると、~。
\\	市役所	すみません、市役所は どこですか 。。。この道を まっすぐ 行くと、左に あります。古い建物です	市役所[しやくしょ]			
\\	ると、~。
\\	渡ります	道を渡るとき、車に気をつけます 。。。はい、ありがとう ございます 昔は 船で 渡りましたが、今は 橋があるので、歩いて渡れます	渡[わた]ります			
\\	(をわかります)	
\\	る、ない・い
\\	い・な
\\	な・
\\	の
\\	とき、~。
\\	聞きます	使い方がわからないとき、私に 聞いてください 。。。はい、ありがとう	聞[き]きます			
\\	に聞きます)	
\\	る、ない・い
\\	い・な
\\	な・
\\	の
\\	とき、~。
\\	受付	会社へ来た時、受付で 社長に 会いました。 。。。そうですか。社長は何を話しましたか。 いいえ、。。。。。	受付[うけつけ]			
\\	る・た
\\	とき、~。
\\	る 
\\	た
\\	回します	このつまみは 右へ回すと、音が大きくなります。 。。。わかりました	回[まわ]します			
\\	ると、~。
\\	つまみ	このつまみは 右へ回すと、音が大きくなります。 。。。わかりました	つまみ			
\\	ると、~。
\\	行ってきます	出かけるとき、「行ってきます」と 言います	行[い]ってきます			
\\	る・た
\\	とき、~。
\\	る 
\\	た
\\	人1は人2に
\\	と 
\\	いいます。
\\	風をひきます	風を引いたとき。どう しますか。 。。。薬を飲んで、寝ます	風[かぜ]をひきます			
\\	る、ない・い
\\	い・な
\\	な・
\\	の
\\	とき、~。
\\	顔	眠いとき、顔を洗います。 。。。そうですか。私は しばらく寝ます	顔[かお]			
\\	る、ない・い
\\	い・な
\\	な・
\\	の
\\	とき、~。
\\	切符	このボタンを押すと、切符が出ます 。。。はい、ありがとう	切符[きっぷ]			
\\	ると、~。
\\	引きます	これを引くと、いすが 動きます	引[ひ]きます			
\\	ると、~。
\\	触ります	これに 触ると、水が出ます 子どもが 触らないように、はさみを しまっておきます 。。。はい、わかりました	触[さわ]ります			
\\	に触ります)	
\\	ると、~。
\\	1る・
\\	1ない
\\	ように、
\\	交差点	銀行はどこですか。 。。。あの 交差点を 右へ 曲がると、左に あります。 はい、ありがとう ございます	交差点[こうさてん]			
\\	ると、~。
\\	角	市役所はどこですか。 。。。あの角を 右へ曲がると、左にあります はい、わかりました	角[かど]			
\\	ると、~。
\\	信号	美術館はどこですか。 。。。あの信号を左へ曲がると、右にあります はい、ありがとうござます	信号[しんごう]			
\\	ると、~。
\\	会社に入ります	すみません、 。。。なんですか  友達が 会社に入るとき、どんなものを あげますか。 。。。そうですね。ネクタイや かばんなです そうですね。	会社[かいしゃ]に 入[はい]ります				
\\	る、ない・い
\\	い・な
\\	な・
\\	の
\\	とき、~。
\\	変えます	すみません。 。。。紙のサイズを 変えたいとき、どう しますか。 このボタンを 押します わかりました。どうも	変[か]えます			
\\	る、ない・い
\\	い・な
\\	な・
\\	の
\\	とき、~。
\\	~目	ちょっと すみません。近くに郵便局が ありますか。 。。。郵便局ですか。あそこに 信号がありますね。 ええ。 。。。あそこを わたって、一つ目の角を右へ曲がると右に ありますよ どうも	
\\	目[め]			
\\	連れて行きます	私は 山田さんに 大阪城へ 連れて行って もらいました。 。。。いいですね。大阪城は どうでしたか 。。。。。。。	連[つ]れて 行[い]きます			
\\	人1(私)は 人2に 
\\	て)もらいます・もらいます・いただきます。
\\	人1
\\	人2. (もらいます:
\\	(もらいます:
\\	-(いただきます:
\\	紹介します	私は タワポンさんに 田中さんを 紹介してもらいます 。。。はい、いいですね。田中さんはどんな人ですか。 田中さんは よくて、親切な人です	紹介[しょうかい]します			
\\	人1(私)は 人2に 
\\	て)もらいます・もらいます・いただきます。
\\	人1
\\	人2. (もらいます:
\\	(もらいます:
\\	-(いただきます:
\\	案内します	私は 木村さんに 京都を 案内してもらいます 。。。はい、いいですね。京都は どうでしたか	案内[あんない]します			
\\	人1(私)は 人2に 
\\	て)もらいます・もらいます・いただきます。
\\	人1
\\	人2. (もらいます:
\\	(もらいます:
\\	-(いただきます:
\\	連れて来ます	去年 富士山に息子をつれてきてやります	連[つ]れて 来[き]ます			
\\	人1(私)は 人2に 
\\	て)やります・あげます・さしあげます。
\\	人1
\\	人2
\\	(やります:
\\	(あげます:
\\	-(さしあげます:
\\	説明します	ホームステイは どうでしたか。 。。。楽しかったです。お母さんはすき焼きを 作って くれました すき焼きですか。 。。。お父さんは 町について 説明して くれました そうですか。よかったですね。 私は 部長に 説明して いただけきました。 
\\	説明[せつめい]します			
\\	人1は 人2(私
\\	に 
\\	て)くれます・くれます・くださいます。
\\	人1
\\	人2. (くれます:
\\	(くれます:
\\	-(くださいます:
\\	コーヒーをいれます	毎朝 私は妻に コーヒーを入れてもらいます 。。。しあわせですね。	コーヒーをいれます			
\\	人1(私)は 人2に 
\\	て)もらいます・もらいます・いただきます。
\\	人1
\\	人2. (もらいます:
\\	(もらいます:
\\	-(いただきます:
\\	引越しします	ワンさん、日曜日 引っ越しですね。手伝いに 行きましょうか。 。。。ありがとう ございます。じゃ すみませんが、九時ごろ お願いします だれが 引っ越しを手伝ってくれますか 。。。会社の人は 手伝ってくれます	引越[ひっこ]しします				人1は 人2(私
\\	に 
\\	て)くれます・くれます・くださいます。
\\	人1
\\	人2. (くれます:
\\	(くれます:
\\	-(くださいます:
\\	名前	もう 子どもの名前を 考えましたか。 。。。ええ、男の子だったら、「ひかる」です 。。。女の子だったら、「おや」です	名前[なまえ]				
\\	た
\\	ら,~。
\\	もし
\\	~たら、~	
\\	意味	先生、この漢字の意味が わかりません。 。。。辞書を 見ましたか。 ええ、何回も 見ても、わかりません	意味[いみ]			
\\	暑い	今 暑いですが、エアコンを つけてもいいですか。 。。。いいえ、暑くても、つけてはいけません。体に よくないと 思います	暑[あつ]い				
\\	て・いTTくて・なTTで・
\\	で
\\	も、~。
\\	いくら~も、 ~。
\\	と 思います。
\\	でしょう、かもしれません)	
\\	もちろん	東京へ行っても、大阪のことを わすれないでください。 。。。もちろん。皆さん、暇があったら、ぜひ 東京へ 遊びに来てください。 ミラーさんも 大阪へ 来たら、電話をください。一緒に飲みましょう 。。。ええ、ぜひ、本当に いろいろ お世話に なりました 頑張ってください。 お体に 気をつけて。 はい、皆さんも どうぞ お元気で	もちろん			
\\	た
\\	ら,~。
\\	もし
\\	~たら、~		
\\	いろいろお世話になりました	東京へ行っても、大阪のことを わすれないでください。 。。。もちろん。皆さん、暇があったら、ぜひ 東京へ 遊びに来てください。 ミラーさんも 大阪へ 来たら、電話をください。一緒に飲みましょう 。。。ええ、ぜひ、本当に いろいろ お世話に なりました 頑張ってください。 お体に 気をつけて。 はい、皆さんも どうぞ お元気で	いろいろお 世話[せわ]になりました			
\\	どうぞお元気で	東京へ行っても、大阪のことを わすれないでください。 。。。もちろん。皆さん、暇があったら、ぜひ 東京へ 遊びに来てください。 ミラーさんも 大阪へ 来たら、電話をください。一緒に飲みましょう 。。。ええ、ぜひ、本当に いろいろ お世話に なりました 頑張ってください。 お体に 気をつけて。 はい、皆さんも どうぞ お元気で	お 元気[げんき]で			
\\	三つ目	そちらまで、どうやって 行きますか。 。。。。本田駅から 12番のバスに 乗って 図書館前で 降りてください。三つ目です 三つ目ですね。 。。。ええ、降りると、前に 公園があります。図書館は 公園の中の白い建物です	三[みっ]つ 目[め]			
\end{CJK}
\end{document}