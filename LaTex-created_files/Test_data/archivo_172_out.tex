\documentclass[8pt]{extreport} 
\usepackage{hyperref}
\usepackage{CJKutf8}
\begin{document}
\begin{CJK}{UTF8}{min}
\\	アアンと	ああんと	
\\	赤あかちゃんが______泣なく。	
\\	アハハ	あはは	
\\	太郎たろう: このテレビ面白おもしろいよ~ 佐助さすけ: あっ、ほんとだ!______	
\\	アングリ	あんぐり	
\\	サーカスの驚おどろきのパフォーマンスに口くちを______と開あける。	
\\	アップアップ	あっぷあっぷ	
\\	彼かれは海うみでおぼれて______する。	
\\	アタフタ	あたふた	
\\	学校がっこうに遅刻ちこくしそうになり______した。	
\\	アツアツ	あつあつ	
\\	あの子こは何年なんねんも前まえから今いまの彼かれと______だ。	
\\	バッチリ	ばっちり	
\\	お母かあさん: テストが帰かえってきたんでしょう?見みせなさい。 息子むすこ: はい、どうぞ。 お母かあさん: すごい!満点まんてんじゃない? 息子むすこ: 
\\	よ!	
\\	バタバタ	ばたばた	
\\	旅行りょこうの前まえはいつも______する。	
\\	ビリビリ	びりびり	
\\	個人情報こじんじょうほうが書かいてあるから______に破やぶって手紙てがみを捨すてた。	
\\	ブスッと	ぶすっと	
\\	お姉ねえちゃんにプリンを食たべられて______している。	
\\	ブヨブヨ	ぶよぶよ	
\\	一ヶ月いっかげつ前まえに買かったリンゴが______している。	
\\	ビュウビュウ	びゅうびゅう	
\\	台風たいふうが過すぎ去さった後あとなのに風かぜがまだ______吹ふいている。	
\\	チビチビ	ちびちび	
\\	仕事しごとから帰かえると______お酒さけを飲のむのが日課にっかだ。	
\\	チカチカ	ちかちか	
\\	バイクを運転うんてんするときは______とライトが点滅てんめつする車くるまに気きをつけよう。	
\\	チクリと	ちくりと	
\\	義理ぎりのお母かあさんは普段ふだんとても優やさしいけれど、ときどき______痛いたいことをいう。	
\\	チラリと	ちらりと	
\\	頭あたまをぶつけた子供こどもがお母かあさんに甘あまえたくて______顔かおを見みる。	
\\	チョキチョキ	ちょきちょき	
\\	妹いもうとの誕生日たんじょうびに折いのりり紙がみを______と切きって家いえの飾かざりつけを付ついた。	
\\	チョコチョコ	ちょこちょこ	
\\	恵子けいこ: 近状きんじょうで有名ゆうめいな風船ふうせんおじさんって最近さいきん見みた? 知子ともこ: うん、______見みるよ。	
\\	チョロチョロ	ちょろちょろ	
\\	蛇へびが______する。	
\\	チュンチュン	ちゅんちゅん	
\\	朝あさはスズメの______と鳴なく声こえで目めが覚さめる。	
\\	チュウチュウ	ちゅうちゅう	
\\	屋根裏やねうらからネズミの______鳴なく声こえが聞きこえた。	
\\	ダブダブ	だぶだぶ	
\\	2年前ねんまえの自分じぶんの服ふくを着きるとサイズがちょっと______だ。	
\\	ドキドキ	どきどき	
\\	憧あこがれの先輩せんぱいが隣となりの席せきに座すわったので______してた。	
\\	ドッサリ	どっさり	
\\	田中たなか: 今夜こんや飲のみに行いかない? 上田うえだ: いや、今夜こんやは仕事しごとが______残のこってるから無理むりだ。	
\\	フックラ	ふっくら	
\\	朝食ちょうしょくのホットケーキが久ひさしぶりに______焼やけた。	
\\	フンワリ	ふんわり	
\\	犬いぬの毛けづくろいを久ひさしぶりにしたので毛けが______と仕上がしあがった。	
\\	フラフラ	ふらふら	
\\	久美子くみこ: お宅たくの息子むすこさんは最近さいきんどう? 亜季あき: うちの息子むすこは最近さいきん______してばっかりで働はたらかないの。	
\\	フワフワ	ふわふわ	
\\	太郎たろう: その______した物ものはな~に? 幸子さちこ: これは綿菓子わたがしっていう甘あまいお菓子かしだよ。 太郎たろう: うわー、雲くもみたい!	
\\	ガッチリ	がっちり	
\\	試合しあいが始はじまる前まえにキャプテン同士どうしが______と握手あくしゅを交かわした。	
\\	ガラガラ	がらがら	
\\	僕ぼくの人生設計じんせいせっけいが______と音おとを立たてて崩くずれた。	
\\	ガサガサ	がさがさ	
\\	ボールが草くさむらに飛とんでいたので皆みなで______と探さがし回まわった。	
\\	ガツガツ	がつがつ	
\\	彼かれは普段ふだん内気うちきな性格せいかくだが、女おんなの女ひとのことになると非常ひじょうに______している。	
\\	ギクッと	ぎくっと	
\\	家いえの中なかにだれもいないと思おもって冷蔵庫れいぞうこのプリンをこそっと食たべようとしたら後うしろに母ははがいたので______した。	
\\	ギラギラ	ぎらぎら	
\\	と太陽たいようが輝かがやいている。	
\\	ギザギザ	ぎざぎざ	
\\	この紙かみは端はしっこが______しているから取とり扱あついには注意ちゅういしてください。	
\\	ゴクゴク	ごくごく	
\\	炎天下えんてんかで外そとで仕事しごとをしていたのでのどが渇かわいて______とビールを飲かのんだ。	
\\	ゴロゴロ	ごろごろ	
\\	さや: 顔かお色いろは悪わるいけどどうしたの? ひろし: さっきからお腹なかが______するんだ。	
\\	ゴシゴシ	ごしごし	
\\	買かって欲ほしいものがあるのでお風呂ふろで父親ちちおやの背中せなかを______と洗あらった。	
\\	ゴソゴソ	ごそごそ	
\\	正弘まさひろ: 何なにか外そとから______聞きこえない? 智里ちさと: 聞きこえる!外そとに何なにかいる!絶対ぜったいにテントから外そとに出でちゃだめだよ。	
\\	ゴッソリ	ごっそり	
\\	由梨ゆり: 家いえに泥棒どろぼうが入はいった! 安藤あんど: 何なにか取とられたの? 由梨ゆり: 私わたしの下着したぎが______と盗ぬすまれている!	
\\	グルグル	ぐるぐる	
\\	中華ちゅうか料理りょうり屋やさんのテーブルは______回まわる。	
\\	グシャグシャ	ぐしゃぐしゃ	
\\	いくら頑張がんばってもいいポエムが書かけないので紙かみを______にした。	
\\	グッスリ	ぐっすり	
\\	和也かずや: 顔かおがすっきりしてるね。 健太郎けんたろう: 昨日きのう久ひさしぶりに______に寝ねれたの。	
\\	ハラハラ	はらはら	
\\	偉えらい人ひとが集あつまるパーティで自分じぶんが失礼しつれいなことしないかどうか______する。	
\\	ヘナヘナ	へなへな	
\\	植物しょくぶつに水みずをあげていなかったので______と萎しおれてしまった。	
\\	ヘトヘト	へとへと	
\\	さすがに42.195キロ走はしると______になってなにも食たべれない。	
\\	ヒラヒラ	ひらひら	
\\	夏なつになるとひまわり畑ばたけにちょうちょうが______とたくさんやってくる。	
\\	ヒソヒソ	ひそひそ	
\\	近所きんじょの奥様おくさまたちは集あつめて______とうわさ話はなしが大好だいすき。	
\\	ホッコリ	ほっこり	
\\	せっかく温泉街おんせんがいに来きたんだから温泉おんせんに入はいって______しましょうよ。	
\\	ホッと	ほっと	
\\	車くるまのローンが払はらい終おわって______した。	
\\	ヒョロヒョロ	ひょろひょろ	
\\	亜季あき: あの人ひとがオリンピック選手せんしゅなの? 瑞穂みずほ: あんなに______の体からだで3000メートルを走はしるなんてすごい!	
\\	イジイジ	いじいじ	
\\	流行はやりのドラマの話はなしについていけずクラスメートから仲間なかまはずれにされて______する。	
\\	イキイキ	いきいき	
\\	部長ぶちょうは仕事中しごとちゅうはとてもやる気きがないがゴルフのことになるととても______とする。	
\\	イライラ	いらいら	
\\	手てについた納豆なっとうがネバネバしておまけにくさいので______する。	
\\	ジャアジャア	じゃあじゃあ	
\\	買かったばかりの大好だいすきなお花はなに______と水みずをやる。	
\\	ジンジン	じんじん	
\\	昨日きのうこけたところがまだまだ______と痛いたむ。	
\\	ジロジロ	じろじろ	
\\	サングラスとマスクを付つけていると知しらない人ひとから______と見みられる。	
\\	チラチラ	ちらちら	
\\	テスト中ちゅうに隣となりの席の人ひとが______とカンニングしてくる。	
\\	ジッと	じっと	
\\	お母かあさん: ご飯はんのときは______していなさい。 娘むすめ: だって納豆なっとが嫌きいなんだもん。	
\\	カチャカチャ	かちゃかちゃ	
\\	上司じょうしは仕事しごとができるが______とパソコンの操作そうさがうるさい。	
\\	カチカチ	かちかち	
\\	この液体えきたいにバナナをつけると一瞬いっしゅんで______に凍こおってしまいます。	
\\	カラッポ	からっぽ	
\\	嫁よめにお財布さいふを貸かしたら一瞬いっしゅんで______になって返かえってきた。	
\\	カラリ	からり	
\\	昨日きのうまでの雨あめがうそみたいに今日きょうは______と晴はれている。	
\\	カラット	からっと	
\\	お母かあさんのから揚あげは______ジューシーでとても美味おいしい。	
\\	カリカリ	かりかり	
\\	幸子さちこ: 何なにかカリカリ音おとがするよ。 紀子: 昨日きのう上うえでリスがどんぐりを______食たべてる音おとだよ。	
\\	カサカサ	かさかさ	
\\	日焼ひやけしすぎたせいで肌はだが______です。	
\\	カッと	かっと	
\\	居眠いねむりしている友達ともだちにいたずらしようとしたら突然とつぜん目を______開あけたのでびっくりした。	
\\	キッチリ	きっちり	
\\	ちちはお金かねのお札さつを綺麗きれいに______とそろえて財布さいふにしまっている。	
\\	キチンと	きちんと	
\\	お母かあさん: 遊あそんだ後うしろはおもちゃを______元もとあった場所ばしょになおしなさい。 息子むすこ: はーい、きちんとしま~す。	
\\	キラキラ	きらきら	
\\	田舎いなかに帰かえると夜空よぞらに星ほしが______と瞬かがやいているのが見みえる。	
\\	コナゴナ	こなごな	
\\	隕石いんせきが大気圏たいきけんで______ に燃もえ尽つきた。	
\\	コンコン	こんこん	
\\	登山中とさんちゅうに______と水みずが湧わきき出てている泉いずみがあったのでのどを潤うろおした。	
\\	コロコロ	ころころ	
\\	山頂さんちょうでおにぎりを食たべていたら落おとしてしまって______ところがり落おちてしまった。	
\\	コソコソ	こそこそ	
\\	先生せんせい: そこで何なにを______している? 学生がくせい: 先生せんせいの誕生日たんじょうびプレゼントを用意よういしていたんです。 先生せんせい: こそこそしてよし!	
\\	コッソリ	こっそり	
\\	社員しゃいん
\\	今日きょうは上司じょうしが一日中いちにちじゅういないぞ! 社員しゃいん
\\	今日きょうはこっそり携帯けいたいをいじらなくてもいいぞ!	
\\	コワゴワ	こわごわ	
\\	娘むすめはとても慎重しんちょうなのでチワワを触さわるときも______と触こわごわさわる。	
\\	クチャクチャ	くちゃくちゃ	
\\	私わたしの友達ともだちはとても美人びじんなのに______と音おとを立たてて食たべる。	
\\	クッキリ	くっきり	
\\	長ながく続つづいた雨あめがやんで遠とおくの富士山ふじさんが______と見みえた。	
\\	クネクネ	くねくね	
\\	美穂みほ: さっきから______してどうしたの? 飛鳥あすか: トイレに行いきたいのを我慢がまんしてるんだ!	
\\	クラクラ	くらくら	
\\	お酒さけを飲のみすぎで寝ねても覚さめても頭あたまが______する。	
\\	クルッと	くるっと	
\\	道みちを間違まちがえたので車くるまを______Uターンさせた。	
\\	クシャクシャ	くしゃくしゃ	
\\	テストがあまりにも悪わるい点数てんすうだったので親おやに見みせる前まえに______丸まるめてゴミ箱ばこに捨すてた。	
\\	クタクタ	くたくた	
\\	最近さいきんできた大おおきいデパートを一日中いちにちじゅう歩あるいたので足あしが______だ。	
\\	キョロキョロ	きょろきょろ	
\\	初はじめての都会とかいなのでついつい周まわりを______見みてしまう。	
\\	キョトン	きょとん	
\\	くじ引きびでハワイ旅行りょこうが当あたったので一瞬いっしゅん______としてしまった。	
\end{CJK}
\end{document}