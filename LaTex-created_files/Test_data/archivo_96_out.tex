\documentclass[8pt]{extreport} 
\usepackage{hyperref}
\usepackage{CJKutf8}
\begin{document}
\begin{CJK}{UTF8}{min}
\\	おはよう	
\\	おはよう		
\\	おはようございます	
\\	おはようございます		
\\	こんにちは	
\\	こんにちは		
\\	こんばんは	
\\	こんばんは		
\\	さようなら	
\\	さようなら		
\\	おやすみなさい	
\\	おやすみなさい		
\\	ありがとう	
\\	ありがとう		
\\	ありがとうございます	
\\	ありがとうございます		
\\	すみません	
\\	すみません		
\\	いいえ	
\\	いいえ		
\\	いってきます	
\\	いってきます		
\\	いってらっしゃい	
\\	いってらっしゃい		
\\	ただいま	
\\	ただいま		
\\	お帰りなさい	
\\	お 帰[かえ]りなさい		
\\	頂きます	
\\	頂[いただ]きます		
\\	ごちそうさまでした	
\\	ごちそうさまでした		
\\	初めまして	
\\	初[はじ]めまして		
\\	よろしくお願いします	
\\	よろしくお 願[ねが]いします		
\\	すみません。今何時ですか。	今: 
\\	時: 
\\	すみません。 今[いま] 何[なん] 時[じ]ですか。	
\\	12時半です。	半: 
\\	時半[じはん]です。	
\\	あの、留学生ですか。	留学生: 
\\	あの、 留学生[りゅうがくせい]ですか。		
\\	ええ、アリゾナ大学の学生です。	ええ: 
\\	大学: 
\\	ええ、アリゾナ 大学[だいがく]の 学生[がくせい]です。	
\\	そうですか。専攻は何ですか。	そうですか: 
\\	そうですか。 専攻[せんこう]は 何[なん]ですか。	
\\	日本語です。今二年生です。	日本語: 
\\	日本語[にほんご]です。 今[いま]二 年生[ねんせい]です。	
\\	私は学生です。	"は: 
\\	学生: 
\\	です: 
\\	私[わたし]は 学生[がくせい]です。	
\\	山下さんは先生です。	山下: 
\\	山下[やました]さんは 先生[せんせい]です。	
\\	メアリーさんはアメリカ人です。	
\\	メアリーさんはアメリカ 人[じん]です。	
\\	私の専攻は英語です。	"の: 
\\	専攻: 
\\	私[わたし]の 専攻[せんこう]は 英語[えいご]です。	
\\	メアリーさんは何歳ですか。	何: 
\\	か: 
\\	メアリーさんは 何[なん] 歳[さい]ですか。	
\\	あなたは何年生ですか。	年生: 
\\	あなたは 何[なん] 年生[ねんせい]ですか。	
\\	あなたの電話番号は何ですか。	
\\	あなたの 電話[でんわ] 番号[ばんごう]は 何[なん]ですか。	
\\	武さんの電話番号。	武: 
\\	武[たけし]さんの 電話[でんわ] 番号[ばんごう]。	
\\	武さんのお母さんは高校の先生です。	
\\	武[たけし]さんのお 母[かあ]さんは 高校[こうこう]の 先生[せんせい]です。	
\\	一時	
\\	一時[いちじ]		
\\	二時	
\\	二 時[にじ]		
\\	三時	
\\	三時[さんじ]		
\\	四時	
\\	四時[よじ]		
\\	五時	
\\	五時[ごじ]		
\\	六時	
\\	六時[ろくじ]		
\\	七時	
\\	七時[しちじ]		
\\	八時	
\\	八時[はちじ]		
\\	九時	
\\	九時[くじ]		
\\	十時	
\\	十時[じゅうじ]		
\\	十一時	
\\	十一時[じゅういちじ]		
\\	十二時	
\\	十二時[じゅうにじ]		
\\	1分	
\\	1分[いっぷん]		
\\	2分	
\\	2分[にふん]		
\\	3分	
\\	3分[さんぷん]		
\\	4分	
\\	4分[よんぷん]		
\\	5分	
\\	5分[ごふん]		
\\	6分	
\\	6分[ろっぷん]		
\\	7分	
\\	7分[ななふん]		
\\	8分	
\\	8分[はっぷん]		
\\	九分	
\\	九分[きゅうふん]		
\\	10分	
\\	10分[じゅっぷん]		
\\	1歳	
\\	1歳[いっさい]		
\\	2歳	
\\	2歳[にさい]		
\\	3歳	
\\	3歳[さんさい]		
\\	4歳	
\\	4歳[よんさい]		
\\	5歳	
\\	5歳[ごさい]		
\\	6歳	
\\	6歳[ろくさい]		
\\	7歳	
\\	7歳[ななさい]		
\\	8歳	
\\	8歳[はっさい]		
\\	9歳	
\\	9歳[きゅうさい]		
\\	10歳	
\\	10歳[じゅっさい]		
\\	11歳	
\\	11歳[じゅういっさい]		
\\	20歳	
\\	20歳[はたち]		
\\	すみません。これはいくらですか。	これ: 
\\	いくら: 
\\	すみません。これはいくらですか。		
\\	それは3000円です。	それ: 
\\	円: 
\\	それは 
\\	円[えん]です。		
\\	高いですね。じゃあ、あの時計はいくらですか。	高い: 
\\	じゃあ: 
\\	あの: 
\\	時計: 
\\	高[たか]いですね。じゃあ、あの 時計[とけい]はいくらですか。		
\\	あれは3500円です。	あれ: 
\\	あれは 
\\	円[えん]です。		
\\	そうですか。あれも高いですね。	"も: 
\\	そうですか。あれも 高[たか]いですね。		
\\	これは1800円ですよ。	よ: 
\\	これは 
\\	円[えん]ですよ。		
\\	じゃあ、その時計を下さい。	その: 
\\	下さい: 
\\	じゃあ、その 時計[とけい]を 下[くだ]さい。		
\\	これは誰の財布ですか。	誰: 
\\	財布: 
\\	これは 誰[だれ]の 財布[さいふ]ですか。		
\\	いらっしゃいませ。メニューをどうぞ。	いらっしゃいませ: 
\\	いらっしゃいませ。メニューをどうぞ。		
\\	どうも。これは何ですか。	どうも: 
\\	どうも。これは 何[なん]ですか。		
\\	どれですか。ああ、とんかつです。	どれ: 
\\	ああ: 
\\	とんかつ: 
\\	どれですか。ああ、とんかつです。		
\\	とんかつ?魚ですか。	魚: 
\\	とんかつ? 魚[さかな]ですか。		
\\	いいえ、魚じゃないです。肉です。美味しいですよ。	いいえ: 
\\	肉: 
\\	美味しい: 
\\	いいえ、 魚[さかな]じゃないです。 肉[にく]です。 美味[おい]しいですよ。		
\\	じゃあ、これをお願いします。	お願いします: 
\\	じゃあ、これをお 願[ねが]いします。		
\\	すみません。トイレはどこですか。	どこ: 
\\	すみません。トイレはどこですか。		
\\	二百	
\\	百: 
\\	二 百[ひゃく]		
\\	三百	
\\	三 百[びゃく]		
\\	六百	
\\	六[ろく] 百[ぴゃく]		
\\	八百	
\\	八 百[ぴゃく]		
\\	二千	
\\	千: 
\\	二 千[せん]		
\\	三千	
\\	三 千[ぜん]		
\\	八千	
\\	八 千[せん]		
\\	一万	
\\	一 万[まん]		
\\	二万	
\\	二 万[まん]		
\\	分かりましたか。	
\\	分[わ]かりましたか。		
\\	分かりました。	
\\	分[わ]かりました。		
\\	分かりません。	
\\	分[わ]かりません。		
\\	ゆっくり言って下さい。	ゆっくり: 
\\	言って (言う): 
\\	ゆっくり 言[い]って 下[くだ]さい。		
\\	もう一度言って下さい。	もう一度: 
\\	もう 一度[いちど] 言[い]って 下[くだ]さい。		
\\	ちょっと待って下さい。	ちょっと: 
\\	待って (待つ): 
\\	ちょっと 待[ま]って 下[くだ]さい。		
\\	聞いて下さい。	聞いて (聞く): 
\\	聞[き]いて 下[くだ]さい。		
\\	メアリーさん、週末は大抵何をしますか。	週末: 
\\	大抵: 
\\	します (する): 
\\	メアリーさん、 週末[しゅうまつ]は 大抵[たいてい] 何[なに]をしますか。		
\\	そうですね。大抵家で勉強します。	そうですね: 
\\	勉強 [勉強する]:
\\	そうですね。 大抵[たいてい] 家[うち]で 勉強[べんきょう]します。		
\\	そうですか。じゃあ、土曜日に映画を観ませんか。	"土曜日: 
\\	ませんか: 
\\	そうですか。じゃあ、 土曜日[どようび]に 映画[えいが]を 観[み]ませんか。		
\\	じゃあ、日曜日はどうですか。	日曜日: 
\\	どうですか: 
\\	じゃあ、 日曜日[にちようび]はどうですか。		
\\	良いですね。	良い: 
\\	良[い]いですね。		
\\	おはよう。早いですね。	早い: 
\\	おはよう。 早[はや]いですね。		
\\	ええ、今日は京都に行きます。京都で映画を観ます。	今日: 
\\	京都: 
\\	行き (行く): 
\\	ええ、 今日[きょう]は 京都[きょうと]に 行[い]きます。 京都[きょうと]で 映画[えいが]を 観[み]ます。		
\\	良いですね。何時頃帰りますか。	頃: 
\\	帰り (帰る): 
\\	良[い]いですね。 何[なん] 時[じ]頃[ごろ] 帰[かえ]りますか。		
\\	晩御飯は?	晩御飯: 
\\	は: 
\\	は 
\\	晩[ばん] 御飯[ごはん]は?		
\\	食べません。	食べ (食べる): 
\\	食[た]べません。		
\\	あそこにデパートがありますね。	デパート: 
\\	あそこにデパートがありますね。		
\\	お帰りなさい。映画はどうでしたか。	どうでしたか: 
\\	お 帰[かえ]りなさい。 映画[えいが]はどうでしたか。		
\\	観ませんでした。武さんは来ませんでした。	来 [来る]: 
\\	観[み]ませんでした。 武[たけし]さんは 来[き]ませんでした。		
\\	えっ、どうしてですか。	どうして: 
\\	えっ、どうしてですか。		
\\	分かりません。だから、一人で本屋とお寺に行きました。	だから: 
\\	一人: 
\\	本屋: 
\\	お寺: 
\\	分[わ]かりません。だから、一 人[ひとり]で 本屋[ほんや]とお 寺[てら]に 行[い]きました。		
\\	人が沢山いましたか。	人: 
\\	沢山: 
\\	い [いる]: 
\\	が: 
\\	人[ひと]が 沢山[たくさん]いましたか。		
\\	はい。お寺で写真を沢山撮りました。	写真: 
\\	撮り[撮る]: 
\\	はい。お 寺[てら]で 写真[しゃしん]を 沢山[たくさん] 撮[と]りました。		
\\	はい、お土産です。	お土産: 
\\	はい、お 土産[みやげ]です。		
\\	あっ、武さん。昨日来ませんでしたね。	昨日: 
\\	あっ、 武[たけし]さん。 昨日[きのう] 来[き]ませんでしたね。		
\\	行きましたよ。モスバーガーの前で一時間待ちました。	
\\	行[い]きましたよ。モスバーガーの 前[まえ]で一 時間[じかん] 待[ま]ちました。	
\\	モスバーガーじゃないですよ。マクドナルドですよ。	じゃない: 
\\	モスバーガーじゃないですよ。マクドナルドですよ。		
\\	買い物	
\\	買い物[かいもの]		
\\	椅子	
\\	椅子[いす]		
\\	犬	
\\	犬[いぬ]		
\\	子供	
\\	子供[こども]		
\\	ご飯	
\\	ご 飯[はん]		
\\	机	
\\	机[つくえ]		
\\	手紙	
\\	手紙[てがみ]		
\\	猫	
\\	猫[ねこ]		
\\	公園	
\\	公園[こうえん]		
\\	バス停	
\\	バス停[ばすてい]		
\\	病院	
\\	病院[びょういん]		
\\	町	
\\	町[まち]		
\\	一時間	
\\	一 時間[じかん]		
\\	先週	
\\	先週[せんしゅう]		
\\	月曜日	
\\	月曜日[げつようび]		
\\	火曜日	
\\	火曜日[かようび]		
\\	水曜日	
\\	水曜日[すいようび]		
\\	木曜日	
\\	木曜日[もくようび]		
\\	金曜日	
\\	金曜日[きんようび]		
\\	会う	
\\	会[あ]う		
\\	買う	
\\	買[か]う		
\\	書く	
\\	書[か]く		
\\	位	
\\	位[ぐらい]		
\\	右	
\\	右[みぎ]		
\\	左	
\\	左[ひだり]		
\\	後ろ	
\\	後[うし]ろ		
\\	中	
\\	中[なか]		
\\	上	
\\	上[うえ]		
\\	下	
\\	下[した]		
\\	近く	
\\	近[ちか]く		
\\	隣	
\\	隣[となり]		
\\	間	
\\	間[あいだ]		
\\	あそこにマクドナルドがあります。	が: 
\\	あります: 
\\	あそこにマクドナルドがあります。	
\\	僕はテレビがありません。	
\\	僕[ぼく]はテレビがありません。	
\\	時間がありますか。	
\\	時間[じかん]がありますか。	
\\	火曜日にテストがあります。	
\\	火曜日[かようび]にテストがあります。	
\\	明日は日本語のクラスがありません。	
\\	明日[あした]は 日本語[にほんご]のクラスがありません。	
\\	あそこに留学生がいます。	います: 
\\	あります 
\\	あそこに 留学生[りゅうがくせい]がいます。	
\\	日本人の友達がいます。	
\\	日本人[にほんじん]の 友達[ともだち]がいます。	
\\	銀行は図書館の隣です。	
\\	銀行[ぎんこう]は 図書館[としょかん]の 隣[となり]です。	
\\	傘はテーブルの下です。	
\\	傘[かさ]はテーブルの 下[した]です。	
\\	レストランはデパートと病院の間です。	
\\	レストランはデパートと 病院[びょういん]の 間[あいだ]です。	
\\	私はモスバーガーの前でメアリーさんを待ちました。	
\\	私[わたし]はモスバーガーの 前[まえ]でメアリーさんを 待[ま]ちました。	
\\	山下先生は桜大学の学生でした。	でした: 
\\	です。 桜: 
\\	山下[やました] 先生[せんせい]は 桜[さくら] 大学[だいがく]の 学生[がくせい]でした。	
\\	あれは日本の映画じゃなかったです。	じゃなかったです:
\\	じゃないです。	あれは 日本[にっぽん]の 映画[えいが]じゃなかったです。	
\\	メアリーさんは九時頃家に帰りました。	ました: 
\\	ます。	メアリーさんは九 時[じ] 頃[ごろ] 家[うち]に 帰[かえ]りました。	
\\	私は昨日日本語を勉強しませんでした。	ませんでした: 
\\	ません。	私[わたし]は 昨日[きのう] 日本語[にほんご]を 勉強[べんきょう]しませんでした。	
\\	私は昨日京都に行きました。	
\\	私[わたし]は 昨日[きのう] 京都[きょうと]に 行[い]きました。	
\\	山下先生も昨日京都に行きました。	
\\	山下[やました] 先生[せんせい]も 昨日[きのう] 京都[きょうと]に 行[い]きました。	
\\	メアリーさんは靴を買いました。	
\\	メアリーさんは 靴[くつ]を 買[か]いました。	
\\	メアリーさんは鞄も買いました。	
\\	メアリーさんは 鞄[かばん]も 買[か]いました。	
\\	私は先週京都に行きました。	
\\	私[わたし]は 先週[せんしゅう] 京都[きょうと]に 行[い]きました。	
\\	大阪にも行きました。	大阪: 
\\	大阪[おおさか]にも 行[い]きました。	
\\	ロバートさんは土曜日にパーティーに行きました。	
\\	ロバートさんは 土曜日[どようび]にパーティーに 行[い]きました。	
\\	日曜日にもパーティーに行きました。	
\\	日曜日[にちようび]にもパーティーに 行[い]きました。	
\\	メアリーさんはそこで武さんを一時間待ちました。	
\\	メアリーさんはそこで 武[たけし]さんを一 時間[じかん] 待[ま]ちました。	
\\	私は昨日日本語を三時間位勉強しました。	
\\	私[わたし]は 昨日[きのう] 日本語[にほんご]を三 時間[じかん] 位[ぐらい] 勉強[べんきょう]しました。	
\\	昨日七時間半寝ました。	
\\	昨日[きのう]七 時間[じかん] 半[はん] 寝[ね]ました。	
\\	日本語と英語を話します。	"と: 
\\	と 
\\	と 
\\	日本語[にほんご]と 英語[えいご]を 話[はな]します。	
\\	京都と大阪に行きました。	
\\	京都[きょうと]と 大阪[おおさか]に 行[い]きました。	
\\	メアリーさんはスーさんと韓国に行きます。	
\\	メアリーさんはスーさんと 韓国[かんこく]に 行[い]きます。	
\\	今週	
\\	今週[こんしゅう]		
\\	来週	
\\	来週[らいしゅう]		
\\	先月	
\\	先月[せんげつ]		
\\	今月	
\\	今月[こんげつ]		
\\	来月	
\\	来月[らいげつ]		
\\	去年	
\\	去年[きょねん]		
\\	今年	
\\	今年[ことし]		
\\	来年	
\\	来年[らいねん]		
\\	1日	
\\	1日[ついたち]		
\\	2日	
\\	日[ふつか]		
\\	3日	
\\	日[みっか]		
\\	4日	
\\	日[よっか]		
\\	5日	
\\	日[いつか]		
\\	6日	
\\	日[むいか]		
\\	7日	
\\	日[なのか]		
\\	8日	
\\	日[ようか]		
\\	9日	
\\	日[ここのか]		
\\	10日	
\\	日[とおか]		
\\	11日	
\\	日[じゅういちにち]		
\\	12日	
\\	日[じゅうににち]		
\\	13日	
\\	日[じゅうさんにち]		
\\	14日	
\\	日[じゅうよっか]		
\\	19日	
\\	日[じゅうくにち]		
\\	20日	
\\	日[はつか]		
\\	23日	
\\	日[にじゅうさんにち]		
\\	24日	
\\	日[にじゅうよっか]		
\\	1月	
\\	1月[いちがつ]		
\\	2月	
\\	2月[にがつ]		
\\	5月	
\\	5月[ごがつ]		
\\	9月	
\\	9月[くがつ]		
\\	11月	
\\	11月[じゅういちがつ]		
\\	良い天気ですね。	天気: 
\\	良[い]い 天気[てんき]ですね。		
\\	そうですね。でも、ちょっと暑いですね。	暑い: 
\\	そうですね。でも、ちょっと 暑[あつ]いですね。		
\\	ええ。わあ、綺麗な海。	綺麗: 
\\	海: 
\\	ええ。わあ、 綺麗[きれい]な 海[うみ]。		
\\	泳ぎましょう。	泳ぎ: 
\\	泳[およ]ぎましょう。		
\\	ロバートさんはどんなスポーツが好きですか。	好き: 
\\	どんな: 
\\	ロバートさんはどんなスポーツが 好[す]きですか。		
\\	サーフィンが好きです。明日一緒にやりましょうか。	サーフィン: 
\\	一緒に: 
\\	やり [やる]: 
\\	ましょうか: 
\\	サーフィンが 好[す]きです。 明日[あした] 一緒[いっしょ]にやりましょうか。		
\\	でも、難しくないですか。	難しい: 
\\	でも、 難[むずか]しくないですか。		
\\	大丈夫ですよ。	大丈夫: 
\\	大丈夫[だいじょうぶ]ですよ。		
\\	すみません。葉書は、イギリス迄いくらですか。	葉書: 
\\	迄: 
\\	すみません。 葉書[はがき]は、イギリス 迄[まで]いくらですか。		
\\	じゃあ、70円切手を2枚お願いします。	切手: 
\\	枚: 
\\	じゃあ、 
\\	円[えん] 切手[きって]を 
\\	枚[まい]お 願[ねが]いします。		
\\	ロバートさん、葉書、ありがとう。	
\\	ロバートさん、 葉書[はがき]、ありがとう。		
\\	ええ。沖縄の海は凄く綺麗でしたよ。	沖縄: 
\\	凄く: 
\\	ええ。 沖縄[おきなわ]の 海[うみ]は 凄[すご]く 綺麗[きれい]でしたよ。		
\\	良かったですね。僕も海が大好きです。	僕: 
\\	大好き:
\\	良[よ]かったですね。 僕[ぼく]も 海[うみ]が 大好[だいす]きです。		
\\	いいえ、あまり高くなかったです。	
\\	いいえ、あまり 高[たか]くなかったです。	
\\	宿題	
\\	宿題[しゅくだい]		
\\	食べ物	
\\	食べ物[たべもの]		
\\	誕生日	
\\	誕生[たんじょう] 日[び]		
\\	テスト	
\\	テスト		
\\	飲み物	
\\	飲み物[のみもの]		
\\	バス	
\\	バス		
\\	部屋	
\\	部屋[へや]		
\\	休み	
\\	休[やす]み		
\\	新しい	
\\	新[あたら]しい		
\\	熱い	
\\	熱[あつ]い		
\\	忙しい	
\\	忙[いそが]しい		
\\	大きい	
\\	大[おお]きい		
\\	面白い	
\\	面白[おもしろ]い		
\\	格好いい	
\\	格好[かっこう]いい		
\\	怖い	
\\	怖[こわ]い		
\\	寒い	
\\	寒[さむ]い		
\\	小さい	
\\	小[ちい]さい		
\\	つまらない	
\\	つまらない		
\\	古い	
\\	古[ふる]い		
\\	優しい	
\\	優[やさ]しい		
\\	安い	
\\	安[やす]い		
\\	嫌いな	
\\	嫌[きら]いな		
\\	元気な	
\\	元気[げんき]な		
\\	静かな	
\\	静[しず]かな		
\\	大嫌いな	
\\	大嫌[だいきら]いな		
\\	大好きな	
\\	大好[だいす]きな		
\\	賑やかな	
\\	賑[にぎ]やかな		
\\	暇な	
\\	暇[ひま]な		
\\	乗る	
\\	乗[の]る		
\\	出かける	
\\	出[で]かける		
\\	とても	
\\	とても		
\\	昨日、面白い映画を観ました。	
\\	昨日[きのう]、 面白[おもしろ]い 映画[えいが]を 観[み]ました。	
\\	山下先生は怖い先生です。	
\\	山下[やました] 先生[せんせい]は 怖[こわ]い 先生[せんせい]です。	
\\	京都で綺麗な写真を撮りました。	
\\	京都[きょうと]で 綺麗[きれい]な 写真[しゃしん]を 撮[と]りました。	
\\	山下先生は元気な先生です。	
\\	山下[やました] 先生[せんせい]は 元気[げんき]な 先生[せんせい]です。	
\\	沖縄の海はとても綺麗でした。	
\\	沖縄[おきなわ]の 海[うみ]はとても 綺麗[きれい]でした。	
\\	この部屋はちょっと暑いです。	
\\	この 部屋[へや]はちょっと 暑[あつ]いです。	
\\	ロバートさんは日本語のクラスが好きです。	
\\	ロバートさんは 日本語[にほんご]のクラスが 好[す]きです。	
\\	山下先生は魚が嫌いです。	
\\	山下[やました] 先生[せんせい]は 魚[さかな]が 嫌[きら]いです。	
\\	武さんはコーヒーが大好きです。	
\\	武[たけし]さんはコーヒーが 大好[だいす]きです。	
\\	ライアンさんは納豆が大嫌いです。	
\\	ライアンさんは 納豆[なっとう]が 大嫌[だいきら]いです。	
\\	好きでも嫌いでもないです。	
\\	好[す]きでも 嫌[きら]いでもないです。	
\\	これは私の好きな本です。	
\\	これは 私[わたし]の 好[す]きな 本[ほん]です。	
\\	一緒に図書館で勉強しましょう。	
\\	一緒[いっしょ]に 図書館[としょかん]で 勉強[べんきょう]しましょう。	
\\	喫茶店でコーヒーを飲みましょうか。	
\\	喫茶店[きっさてん]でコーヒーを 飲[の]みましょうか。	
\\	りーさんは切手を3枚買いました。	
\\	りーさんは 切手[きって]を3 枚[まい] 買[か]いました。	
\\	小包	
\\	小包[こづつみ]		
\\	航空便	
\\	航空[こうくう] 便[びん]		
\\	船便	
\\	船便[ふなびん]		
\\	速達	
\\	速達[そくたつ]		
\\	書留	
\\	書留[かきとめ]		
\\	保険	
\\	保険[ほけん]		
\\	ロバートさん、次のページを読んで下さい。	"次: 
\\	ページ: 
\\	下さい: 
\\	ロバートさん、 次[つぎ]のページを 読[よ]んで 下[くだ]さい。		
\\	ロバートさん、起きて下さい。	
\\	ロバートさん、 起[お]きて 下[くだ]さい。	
\\	先生、教科書を忘れました。	教科書: 
\\	忘れ [忘れる]: 
\\	先生[せんせい]、 教科書[きょうかしょ]を 忘[わす]れました。		
\\	教科書を持ってきて下さいね。	持ってきて [持ってくる]: 
\\	教科書[きょうかしょ]を 持[も]ってきて 下[くだ]さいね。		
\\	ロバートさん、今日は大変でしたね。	大変: 
\\	ロバートさん、 今日[きょう]は 大変[たいへん]でしたね。		
\\	ええ。後でスーさんのノートを借りても良いですか。	"後で: 
\\	借りて [借りる]: 
\\	も良いです: 
\\	か 
\\	ええ。 後[あと]でスーさんのノートを 借[か]りても 良[い]いですか。		
\\	ありがとう。すぐ返します。	すぐ: 
\\	返し [返す]: 
\\	ありがとう。すぐ 返[かえ]します。		
\\	ロバートさん、明日テストがありますよ。	
\\	ロバートさん、 明日[あした]テストがありますよ。	
\\	えっ。本当ですか。	本当: 
\\	えっ。 本当[ほんとう]ですか。		
\\	ええ。ロバートさん、金曜日に休みましたからね。	休み: 
\\	から: 
\\	から 
\\	ええ。ロバートさん、 金曜日[きんようび]に 休[やす]みましたからね。		
\\	じゃあ、今日は家に帰って、勉強します。	
\\	じゃあ、 今日[きょう]は 家[いえ]に 帰[かえ]って、 勉強[べんきょう]します。	
\\	あの、すみません。このバスは市民病院へ行きますか。	市民病院: 
\\	あの、すみません。このバスは 市民[しみん] 病院[びょういん]へ 行[い]きますか。		
\\	ええ、行きますよ。あの、どうぞ座って下さい。	座って [座る]: 
\\	ええ、 行[い]きますよ。あの、どうぞ 座[すわ]って 下[くだ]さい。		
\\	いいえ、結構です。すぐ降りますから。	"結構: 
\\	降り [降りる]: 
\\	いいえ、 結構[けっこう]です。すぐ 降[お]りますから。		
\\	そうですか。じゃあ、荷物を持ちましょうか。	"ましょうか: 
\\	そうですか。じゃあ、 荷物[にもつ]を 持[も]ちましょうか。		
\\	お金	
\\	お 金[かね]		
\\	お風呂	
\\	お 風呂[ふろ]		
\\	漢字	
\\	漢字[かんじ]		
\\	教科書	
\\	教科書[きょうかしょ]		
\\	市民病院	
\\	市民[しみん] 病院[びょういん]		
\\	シャワー	
\\	シャワー		
\\	次	
\\	次[つぎ]		
\\	電気	
\\	電気[でんき]		
\\	電車	
\\	電車[でんしゃ]		
\\	荷物	
\\	荷物[にもつ]		
\\	パソコン	
\\	パソコン		
\\	ページ	
\\	ページ		
\\	窓	
\\	窓[まど]		
\\	夜	
\\	夜[よる]		
\\	大変な	
\\	大変[たいへん]な		
\\	遊ぶ	
\\	遊[あそ]ぶ		
\\	急ぐ	
\\	急[いそ]ぐ		
\\	お風呂に入る	
\\	お 風呂[ふろ]に 入[はい]る		
\\	返す	
\\	返[かえ]す		
\\	消す	
\\	消[け]す		
\\	死ぬ	
\\	死[し]ぬ		
\\	座る	
\\	座[すわ]る		
\\	立つ	
\\	立[た]つ		
\\	たばこを吸う	
\\	たばこを 吸[す]う		
\\	使う	
\\	使[つか]う		
\\	手伝う	
\\	手伝[てつだ]う		
\\	入る	
\\	入[はい]る		
\\	持つ	
\\	持[も]つ		
\\	休む	
\\	休[やす]む		
\\	開ける	
\\	開[あ]ける		
\\	教える	
\\	教[おし]える		
\\	降りる	
\\	降[お]りる		
\\	借りる	
\\	借[か]りる		
\\	閉める	
\\	閉[し]める		
\\	シャワーを浴びる	
\\	シャワーを 浴[あ]びる		
\\	つける	
\\	つける		
\\	電話をかける	
\\	電話[でんわ]をかける		
\\	忘れる	
\\	忘[わす]れる		
\\	連れてくる	
\\	連[つ]れてくる		
\\	持ってくる	
\\	持[も]ってくる		
\\	後で	
\\	後[あと]で		
\\	遅く	
\\	遅[おそ]く		
\\	から	
\\	から		
\\	結構です	
\\	結構[けっこう]です		
\\	すぐ	
\\	すぐ		
\\	本当ですか	
\\	本当[ほんとう]ですか		
\\	ゆっくり	
\\	ゆっくり		
\\	教科書を読んで下さい。	
\\	教科書[きょうかしょ]を 読[よ]んで 下[くだ]さい。	
\\	すみません。ちょっと教えて下さい。	
\\	すみません。ちょっと 教[おし]えて 下[くだ]さい。	
\\	教科書を見ても良いですか。	
\\	教科書[きょうかしょ]を 見[み]ても 良[い]いですか。	
\\	ここで写真を撮ってはいけません。	"いけません: 
\\	ここで 写真[しゃしん]を 撮[と]ってはいけません。		
\\	ノートを借りて、コピーします。	
\\	(借りる 
\\	コピー) 
\\	て 
\\	(借りて).	ノートを 借[か]りて、コピーします。	
\\	今日は、6時に起きて、勉強しました。	
\\	今日[きょう]は、6 時[じ]に 起[お]きて、 勉強[べんきょう]しました。	
\\	食堂に行って、昼ご飯を食べましょう。	食堂: 
\\	食堂[しょくどう]に 行[い]って、 昼[ひる]ご 飯[はん]を 食[た]べましょう。	
\\	バスに乗って、会社に行きます。	"会社: 
\\	て 
\\	(乗る) 
\\	バスに 乗[の]って、 会社[かいしゃ]に 行[い]きます。		
\\	教科書を忘れて、すみません。	
\\	教科書[きょうかしょ]を 忘[わす]れて、すみません。	
\\	私は今晩勉強します。明日テストがありますから。	
\\	私[わたし]は 今晩[こんばん] 勉強[べんきょう]します。 明日[あした]テストがありますから。	
\\	バスに乗りましょう。タクシーは高いですから。	
\\	バスに 乗[の]りましょう。タクシーは 高[たか]いですから。	
\\	マクドナルドはあのデパートの前ですよ。	前: 
\\	マクドナルドはあのデパートの 前[まえ]ですよ。	
\\	それから、50円切手を1枚下さい。	それから: 
\\	それから、 
\\	円[えん] 切手[きって]を 
\\	枚[まい] 下[くだ]さい。		
\\	旅行は楽しかったですか。	旅行: 
\\	楽しい: 
\\	旅行[りょこう]は 楽[たの]しかったですか。		
\\	飛行機の切符は高かったですか。	飛行機: 
\\	切符: 
\\	飛行機[ひこうき]の 切符[きっぷ]は 高[たか]かったですか。		
\\	武さんのデートはどうでしたか。	
\\	武[たけし]さんのデートはどうでしたか。	
\\	クラスで寝てはいけませんよ。	
\\	クラスで 寝[ね]てはいけませんよ。	
\\	毎日使いますから。	使い: 
\\	毎日[まいにち] 使[つか]いますから。		
\\	デパートにも行きました。	
\\	デパートにも 行[い]きました。	
\\	でも、時々映画を観ます。	でも: 
\\	時々: 
\\	映画: 
\\	でも、 時々[ときどき] 映画[えいが]を 観[み]ます。		
\\	英語	
\\	英語[えいご]		
\\	高校	
\\	高校[こうこう]		
\\	午後	
\\	午後[ごご]		
\\	午前	
\\	午前[ごぜん]		
\\	歳	
\\	歳[さい]		
\\	さん	
\\	さん		
\\	人	
\\	人[じん]		
\\	先生	
\\	先生[せんせい]		
\\	そうです	
\\	そうです		
\\	電話	
\\	電話[でんわ]		
\\	友達	
\\	友達[ともだち]		
\\	名前	
\\	名前[なまえ]		
\\	はい	
\\	はい		
\\	番号	
\\	番号[ばんごう]		
\\	私	
\\	私[わたし]		
\\	アメリカ	
\\	アメリカ		
\\	イギリス	
\\	イギリス		
\\	オーストラリア	
\\	オーストラリア		
\\	スウェーデン	
\\	スウェーデン		
\\	韓国	
\\	韓国[かんこく]		
\\	中国	
\\	中国[ちゅうごく]		
\\	科学	
\\	科学[かがく]		
\\	アジア研究	
\\	アジア 研究[けんきゅう]		
\\	経済	
\\	経済[けいざい]		
\\	国際関係	
\\	国際[こくさい] 関係[かんけい]		
\\	コンピューター	
\\	コンピューター		
\\	人類学	
\\	人類[じんるい] 学[がく]		
\\	政治	
\\	政治[せいじ]		
\\	ビジネス	
\\	ビジネス		
\\	文学	
\\	文学[ぶんがく]		
\\	歴史	
\\	歴史[れきし]		
\\	仕事	
\\	仕事[しごと]		
\\	医者	
\\	医者[いしゃ]		
\\	会社員	
\\	会社[かいしゃ] 員[いん]		
\\	高校生	
\\	高校生[こうこうせい]		
\\	主婦	
\\	主婦[しゅふ]		
\\	大学院生	
\\	大学院生[だいがくいんせい]		
\\	弁護士	
\\	弁護士[べんごし]		
\\	お母さん	
\\	お 母[かあ]さん		
\\	お父さん	
\\	お 父[とう]さん		
\\	お姉さん	
\\	お 姉[ねえ]さん		
\\	お兄さん	
\\	お 兄[にい]さん		
\\	妹	
\\	妹[いもうと]		
\\	弟	
\\	弟[おとうと]		
\\	大学生	
\\	大学生[だいがくせい]		
\\	この	
\\	この		
\\	どの	
\\	どの		
\\	ここ	
\\	ここ		
\\	そこ	
\\	そこ		
\\	野菜	
\\	野菜[やさい]		
\\	鉛筆	
\\	鉛筆[えんぴつ]		
\\	傘	
\\	傘[かさ]		
\\	鞄	
\\	鞄[かばん]		
\\	靴	
\\	靴[くつ]		
\\	ジーンズ	
\\	ジーンズ		
\\	辞書	
\\	辞書[じしょ]		
\\	自転車	
\\	自転車[じてんしゃ]		
\\	新聞	
\\	新聞[しんぶん]		
\\	シャツ	
\\	シャツ		
\\	ノート	
\\	ノート		
\\	ペン	
\\	ペン		
\\	帽子	
\\	帽子[ぼうし]		
\\	本	
\\	本[ほん]		
\\	喫茶店	
\\	喫茶店[きっさてん]		
\\	銀行	
\\	銀行[ぎんこう]		
\\	図書館	
\\	図書館[としょかん]		
\\	郵便局	
\\	郵便[ゆうびん] 局[きょく]		
\\	すみません。郵便局はどこですか。	
\\	すみません。 郵便[ゆうびん] 局[きょく]はどこですか。	
\\	これは誰の鞄ですか。	
\\	これは 誰[だれ]の 鞄[かばん]ですか。	
\\	それはスーさんの鞄です。	
\\	それはスーさんの 鞄[かばん]です。	
\\	武さんは日本人です。	
\\	武[たけし]さんは 日本人[にほんじん]です。	
\\	美智子さんも日本人です。	美智子:
\\	美智子[みちこ]さんも 日本人[にほんじん]です。	
\\	山田さんは学生じゃないです。	
\\	です 
\\	じゃないです。	山田[やまだ]さんは 学生[がくせい]じゃないです。	
\\	山田さんは学生じゃありません。	じゃないです 
\\	ないです 
\\	ありません。	山田[やまだ]さんは 学生[がくせい]じゃありません。	
\\	山田さんは学生ではありません。	では 
\\	じゃあ。	山田[やまだ]さんは 学生[がくせい]ではありません。	
\\	武さんの専攻は文学ですね。	
\\	武[たけし]さんの 専攻[せんこう]は 文学[ぶんがく]ですね。	
\\	これは肉じゃないですね。	
\\	これは 肉[にく]じゃないですね。	
\\	音楽	
\\	音楽[おんがく]		
\\	雑誌	
\\	雑誌[ざっし]		
\\	スポーツ	
\\	スポーツ		
\\	デート	
\\	デート		
\\	テニス	
\\	テニス		
\\	テレビ	
\\	テレビ		
\\	アイスクリーム	
\\	アイスクリーム		
\\	朝ご飯	
\\	朝[あさ]ご 飯[はん]		
\\	お酒	
\\	お 酒[さけ]		
\\	お茶	
\\	お 茶[ちゃ]		
\\	コーヒー	
\\	コーヒー		
\\	ハンバーガー	
\\	ハンバーガー		
\\	昼ご飯	
\\	昼[ひる]ご 飯[はん]		
\\	水	
\\	水[みず]		
\\	家	
\\	家[いえ]		
\\	学校	
\\	学校[がっこう]		
\\	朝	
\\	朝[あさ]		
\\	明日	
\\	明日[あした]		
\\	いつ	
\\	いつ		
\\	今晩	
\\	今晩[こんばん]		
\\	毎日	
\\	毎日[まいにち]		
\\	毎晩	
\\	毎晩[まいばん]		
\\	聞く	
\\	聞[き]く		
\\	飲む	
\\	飲[の]む		
\\	話す	
\\	話[はな]す		
\\	読む	
\\	読[よ]む		
\\	起きる	
\\	起[お]きる		
\\	寝る	
\\	寝[ね]る		
\\	来る	
\\	来[く]る		
\\	あまり
\\	あまり
\\	全然
\\	全然[ぜんぜん]		
\\	よく	
\\	よく		
\\	私はよくテレビを観ます。	ます: 
\\	私[わたし]はよくテレビを 観[み]ます。	
\\	美智子さんは時々朝ご飯を食べません。	ません: 
\\	ます) 
\\	美智子[みちこ]さんは 時々[ときどき] 朝[あさ]ご 飯[はん]を 食[た]べません。	
\\	私は明日京都に行きます。	ます: 
\\	私[わたし]は 明日[あした] 京都[きょうと]に 行[い]きます。	
\\	武さんは今日家に帰りません。	ません: 
\\	ます) 
\\	武[たけし]さんは 今日[きょう] 家[うち]に 帰[かえ]りません。	
\\	コーヒーを飲みます。	を: 
\\	コーヒーを 飲[の]みます。	
\\	音楽を聞きます。	
\\	音楽[おんがく]を 聞[き]きます。	
\\	テレビを観ます。	
\\	テレビを 観[み]ます。	
\\	図書館で本を読みます。	"で: 
\\	図書館[としょかん]で 本[ほん]を 読[よ]みます。	
\\	家でテレビを観ます。	
\\	家[うち]でテレビを 観[み]ます。	
\\	私は今日学校に行きません。	に: 
\\	に.	私[わたし]は 今日[きょう] 学校[がっこう]に 行[い]きません。	
\\	私は家に帰ります。	
\\	私[わたし]は 家[うち]に 帰[かえ]ります。	
\\	日曜日に京都に行きます。	
\\	日曜日[にちようび]に 京都[きょうと]に 行[い]きます。	
\\	11時に寝ます。	
\\	時[じ]に 寝[ね]ます。	
\\	11時頃寝ます。	
\\	時[じ] 頃[ごろ] 寝[ね]ます。	
\\	私は今日学校へ行きません。	"へ: 
\\	に 
\\	私[わたし]は 今日[きょう] 学校[がっこう]へ 行[い]きません。	
\\	私は家へ帰ります。	
\\	私[わたし]は 家[うち]へ 帰[かえ]ります。	
\\	日曜日に行きます。	
\\	日曜日[にちようび]に 行[い]きます。	
\\	45に起きます。	
\\	45に 起[お]きます。	
\\	九月に帰ります。	
\\	九月[くがつ]に 帰[かえ]ります。	
\\	明日来ます。	
\\	明日[あした] 来[き]ます。	
\\	毎晩テレビを観ます。	
\\	毎晩[まいばん]テレビを 観[み]ます。	
\\	いつ行きますか。	
\\	いつ 行[い]きますか。	
\\	朝新聞を読みます。	
\\	朝[あさ] 新聞[しんぶん]を 読[よ]みます。	
\\	週末何をしますか。	
\\	週末[しゅうまつ] 何[なに]をしますか。	
\\	昼ご飯を食べませんか。	ませんか: 
\\	昼[ひる]ご 飯[はん]を 食[た]べませんか。	
\\	テニスをしませんか。	
\\	テニスをしませんか。	
\\	うーん、ちょっと。	
\\	うーん、ちょっと。	
\\	私は今日図書館で日本語を勉強します。	
\\	私[わたし]は 今日[きょう] 図書館[としょかん]で 日本語[にほんご]を 勉強[べんきょう]します。	
\\	私はよく7時頃家へ帰ります。	
\\	私[わたし]はよく7 時[じ] 頃[ごろ] 家[うち]へ 帰[かえ]ります。	
\\	私は時々喫茶店に行きます。	
\\	私[わたし]は 時々[ときどき] 喫茶店[きっさてん]に 行[い]きます。	
\\	私は全然テレビを観ません。	
\\	私[わたし]は 全然[ぜんぜん]テレビを 観[み]ません。	
\\	武さんはあまり勉強しません。	
\\	武[たけし]さんはあまり 勉強[べんきょう]しません。	
\\	アルバイト	
\\	アルバイト		
\\	クラス	
\\	クラス		
\\	あなた	
\\	あなた		
\\	パン	
\\	パン		
\\	メール	
\\	メール		
\\	スーパー	
\\	スーパー		
\\	レストラン	
\\	レストラン		
\\	時	
\\	時[とき]		
\\	これは、スーさんの家族の写真ですか?	家族: 
\\	これは、スーさんの 家族[かぞく]の 写真[しゃしん]ですか?		
\\	これです。高校の時は、眼鏡をかけていました。	眼鏡: 
\\	かけて [かける]: 
\\	これです。 高校[こうこう]の 時[とき]は、 眼鏡[めがね]をかけていました。		
\\	可愛いですね。	可愛い: 
\\	可愛[かわい]いですね。		
\\	これは、父です。アメリカの会社に勤めています。	父: 
\\	勤めて [勤める]: 
\\	これは、 父[ちち]です。アメリカの 会社[かいしゃ]に 勤[つと]めています。		
\\	背が高くて、格好いいですね。これは、お姉さんですか?	"背: 
\\	格好いい: 
\\	背[せ]が 高[たか]くて、 格好[かっこう]いいですね。これは、お 姉[ねえ]さんですか?		
\\	ええ、姉は結婚しています。今、ソウルに住んでいます。	姉: 
\\	結婚: 
\\	ソウル: 
\\	住んで [住む]: 
\\	ええ、 姉[あね]は 結婚[けっこん]しています。 今[いま]、ソウルに 住[す]んでいます。		
\\	子供が1人います。3歳です。	1人: 
\\	子供[こども]が一 人[ひとり]います。3 歳[さい]です。	
\\	そうですか。あっ、猫がいますね。	
\\	そうですか。あっ、 猫[ねこ]が いますね。	
\\	でも、ちょっと太っていますね。	
\\	でも、ちょっと 太[ふと]っていますね。	
\\	ええ、よく食べますから。	
\\	ええ、よく 食[た]べますから。	
\\	もしもし、美智子さん。今、何をしていますか?	
\\	もしもし、 美智子[みちこ]さん。 今[いま]、 何[なに]をしていますか?	
\\	別に何もしていません。今、スーさんの写真を見ています。	別に: 
\\	何も: 
\\	別[べつ]に 何[なに]もしていません。 今[いま]、スーさんの 写真[しゃしん]を 見[み]ています。	
\\	そうですか。面白い
\\	を借りましたから、	
\\	そうですか。面白[おもしろ]い 
\\	を 借[か]りましたから、	
\\	良かったら見に来ませんか?	良かったら: 
\\	良[よ]かったら 見[み]に 来[き]ませんか?	
\\	良いですね。スーさんも一緒に行っても良いですか?	
\\	良[い]いですね。スーさんも 一緒[いっしょ]に 行[い]っても 良[い]いですか?	
\\	もちろん。	
\\	もちろん。		
\\	じゃあ、すぐ行きます。	
\\	じゃあ、すぐ 行[い]きます。		
\\	アパート	
\\	アパート		
\\	歌	
\\	歌[うた]		
\\	お爺さん	
\\	お 爺[じい]さん		
\\	男の人	
\\	男[おとこ]の 人[ひと]		
\\	お婆さん	
\\	お 婆[ばあ]さん		
\\	女の人	
\\	女[おんな]の 人[ひと]		
\\	髪	
\\	髪[かみ]		
\\	兄弟	
\\	兄弟[きょうだい]		
\\	口	
\\	口[くち]		
\\	国	
\\	国[くに]		
\\	車	
\\	車[くるま]		
\\	ゲーム	
\\	ゲーム		
\\	コンビニ	
\\	コンビニ		
\\	サークル	
\\	サークル		
\\	食堂	
\\	食堂[しょくどう]		
\\	母	
\\	母[はは]		
\\	目	
\\	目[め]		
\\	頭が良い	
\\	頭[あたま]が 良[よ]い		
\\	背が低い	
\\	背: 
\\	低い: 
\\	背[せ]が 低[ひく]い		
\\	長い	
\\	長[なが]い		
\\	速い	
\\	速[はや]い		
\\	短い	
\\	短[みじか]い		
\\	親切	
\\	親切[しんせつ]		
\\	便利	
\\	便利[べんり]		
\\	歌う	
\\	歌[うた]う		
\\	被る	
\\	被[かぶ]る		
\\	知る	
\\	知[し]る		
\\	知っています。	
\\	知[し]っています。		
\\	知りません。	
\\	知[し]りません。		
\\	履く	
\\	履[は]く		
\\	太っています。	
\\	太[ふと]っています。		
\\	着る	
\\	着[き]る		
\\	痩せる	
\\	痩[や]せる		
\\	痩せています。	
\\	痩[や]せています。		
\\	2人	
\\	2人[ふたり]		
\\	スーさんは、今勉強しています。	
\\	て 
\\	(勉強する 
\\	勉強して) 
\\	います 
\\	スーさんは、 今[いま] 勉強[べんきょう]しています。	
\\	武さんは、英語の本を読んでいます。	
\\	武[たけ]さんは、 英語[えいご]の 本[ほん]を 読[よ]んでいます。	
\\	私は、英語を教えています。	
\\	て 
\\	います 
\\	私[わたし]は、 英語[えいご]を 教[おし]えています。		
\\	メアリーさんは、毎日日本語を勉強しています。	
\\	メアリーさんは、 毎日[まいにち] 日本語[にほんご]を 勉強[べんきょう]しています。	
\\	山下先生は、結婚しています。	結婚: 
\\	山下[やました] 先生[せんせい]は、 結婚[けっこん]しています。	
\\	美智子さんは、窓の近くに座っています。	
\\	美智子[みちこ]さんは、 窓[まど]の 近[ちか]くに 座[すわ]っています。	
\\	スーさんは、お金を沢山持っています。	
\\	スーさんは、お 金[かね]を 沢山[たくさん] 持[も]っています。	
\\	山下先生は、母を知っています。	
\\	山下[やました] 先生[せんせい]は、 母[はは]を 知[し]っています。	
\\	武さんは、ちょっと太っています。	
\\	武[たけし]さんは、ちょっと 太[ふと]っています。	
\\	私の弟は、とっても痩せています。	痩せる: 
\\	私[わたし]の 弟[おとうと]は、とっても 痩[や]せています。	
\\	美智子さんは、
\\	シャツを着ています。	
\\	美智子[みちこ]さんは、 
\\	シャツを 着[き]ています。	
\\	お父さんは、起きています。	
\\	お 父[とう]さんは、 起[お]きています。	
\\	家族は、東京に住んでいます。	家族: 
\\	東京: 
\\	住んで [住む]: 
\\	家族[かぞく]は、 東京[とうきょう]に 住[す]んでいます。	
\\	私の姉は、日本の会社に勤めています。	勤める: 
\\	姉: 
\\	私[わたし]の 姉[あね]は、 日本[にほん]の 会社[かいしゃ]に 勤[つと]めています。	
\\	中国に行っています。	
\\	中国[ちゅうごく]に 行[い]っています。		
\\	家に来ています。	
\\	家[うち]に 来[き]ています。	
\\	美智子さんの髪は、長いです。	
\\	美智子[みちこ]さんの 髪[かみ]は、 長[なが]いです。	
\\	あの店の食べ物は安くて, 美味しいです。	て 
\\	店: 
\\	あの 店[みせ]の 食べ物[たべもの]は 安[やす]くて, 美味[おい]しいです。	
\\	ホテルは綺麗で、良かったです。	
\\	て 
\\	な-
\\	(綺麗な) 
\\	で 
\\	ホテルは 綺麗[きれい]で、 良[よ]かったです。	
\\	山下先生は日本人で、50歳位です。	
\\	です 
\\	で 
\\	[人です]	山下[やました] 先生[せんせい]は 日本人[にほんじん]で、50 歳[さい] 位[ぐらい]です。	
\\	デパートに、鞄を買いに行きました。	"""鞄を買いに
\\	に 
\\	買い.
\\	デパートに、 鞄[かばん]を 買[か]いに 行[い]きました。	
\\	メアリーさんは、日本に日本語を勉強しに来ました。	"""日本語を勉強しに
\\	勉強し.
\\	メアリーさんは、 日本[にほん]に 日本語[にほんご]を 勉強[べんきょう]しに 来[き]ました。	
\\	3人	
\\	3人[さんにん]		
\\	4人	
\\	4人[よにん]		
\\	5人	
\\	5人[ごにん]		
\\	6人	
\\	6人[ろくにん]		
\\	7人	
\\	7人[ななにん]		
\\	8人	
\\	8人[はちにん]		
\\	9人	
\\	9人[きゅうにん]		
\\	10人	
\\	人[にん]		
\\	私のクラスには、スウェーデン人の学生が1人います。	
\\	(スウェーデン人の学生) が 
\\	(1人)	私[わたし]のクラスには、スウェーデン 人[じん]の 学生[がくせい]が1 人[ひとり]います。	
\\	子供の時、よく友達と遊びました。	遊び: 
\\	と: 
\\	と 
\\	子供[こども]の 時[とき]、よく 友達[ともだち]と 遊[あそ]びました。	
\\	先週の週末は、東京に遊びに行きました。	
\\	先週[せんしゅう]の 週末[しゅうまつ]は、 東京[とうきょう]に 遊[あそ]びに 行[い]きました。	
\\	私の家に、遊びに来て下さい。	
\\	私[わたし]の 家[いえ]に、 遊[あそ]びに 来[き]て 下[くだ]さい。	
\\	耳	
\\	耳[みみ]		
\\	鼻	
\\	鼻[はな]		
\\	指	
\\	指[ゆび]		
\\	歯	
\\	歯[は]		
\\	首	
\\	首[くび]		
\\	手	
\\	手[て]		
\\	顔	
\\	顔[かお]		
\\	肩	
\\	肩[かた]		
\\	頭	
\\	頭[あたま]		
\\	足	
\\	足[あし]		
\\	お尻	
\\	お 尻[しり]		
\\	おなか	
\\	おなか		
\\	背中	
\\	背中[せなか]		
\\	胸	
\\	胸[むね]		
\\	武さん、明日みんなで
\\	をしませんか?	みんな: 
\\	武[たけし]さん、 明日[あした]みんなで 
\\	をしませんか?		
\\	良いですね。誰が来ますか?	
\\	良[い]いですね。 誰[だれ]が 来[き]ますか?	
\\	スーさんと、ロバートさんが来ます。	
\\	スーさんと、ロバートさんが 来[き]ます。	
\\	メアリーさんも、来ると思います。	思い [思う]: 
\\	と思います: 
\\	メアリーさんも、 来[く]ると 思[おも]います。		
\\	健さんは、アルバイトがあると言っていました。	"健: 
\\	ある: 
\\	と言っていました: 
\\	と 
\\	健[けん]さんは、アルバイトがあると 言[い]っていました。		
\\	大変ですね。何か持って行きましょうか?	何か: 
\\	大変[たいへん]ですね。 何[なに]か 持[も]って 行[い]きましょうか?		
\\	何もいらないと思います。	何も: 
\\	いらない:	何[なに]もいらないと 思[おも]います。		
\\	上手ですね。ロバートさんは、料理するのが好きですか?	"上手: 
\\	料理: 
\\	する: 
\\	のが好きです: 
\\	の 
\\	上手[じょうず]ですね。ロバートさんは、 料理[りょうり]するのが 好[す]きですか?		
\\	ええ、よく家で作ります。	作り: 
\\	ええ、よく 家[いえ]で 作[つく]ります。		
\\	何か手伝いましょうか?	
\\	何[なに]か 手伝[てつだ]いましょうか?	
\\	じゃあ、トマトを切って下さい。	切って [切る]: 
\\	じゃあ、トマトを 切[き]って 下[くだ]さい。		
\\	始めましょうか?	始め: 
\\	始[はじ]めましょうか?		
\\	あっ、まだ飲まないで下さい。	まだ: 
\\	飲まない [飲む]: 
\\	あっ、まだ 飲[の]まないで 下[くだ]さい。		
\\	メアリーさんも来ると言っていましたから。	
\\	メアリーさんも 来[く]ると 言[い]っていましたから。	
\\	遅くなってすみません。	なって [なる]: 
\\	遅[おそ]くなってすみません。	
\\	じゃあ、乾杯!	乾杯: 
\\	じゃあ、 乾杯[かんぱい]!		
\\	明後日	
\\	明後日[あさって]		
\\	雨	
\\	雨[あめ]		
\\	カメラ	
\\	カメラ		
\\	カラオケ	
\\	カラオケ		
\\	空気	
\\	空気[くうき]		
\\	今朝	
\\	今朝[けさ]		
\\	黒板	
\\	黒板[こくばん]		
\\	天気予報	
\\	天気[てんき] 予報[よほう]		
\\	所	
\\	所[ところ]		
\\	夏	
\\	夏[なつ]		
\\	箸	
\\	箸[はし]		
\\	冬	
\\	冬[ふゆ]		
\\	ホームステイ	
\\	ホームステイ		
\\	毎週	
\\	毎週[まいしゅう]		
\\	下手	
\\	下手[へた]		
\\	有名	
\\	有名[ゆうめい]		
\\	雨が降る	
\\	降る: 
\\	雨[あめ]が 降[ふ]る		
\\	洗う	
\\	洗[あら]う		
\\	じろじろ見る	
\\	じろじろ 見[み]る		
\\	捨てる	
\\	捨[す]てる		
\\	運転する	
\\	運転[うんてん]する		
\\	洗濯する	
\\	洗濯[せんたく]する		
\\	掃除する	
\\	掃除[そうじ]する		
\\	電話する	
\\	電話[でんわ]する		
\\	いつも	
\\	いつも		
\\	について	
\\	について		
\\	どんな音楽を聴く	聴く: 
\\	か, 
\\	どんな 音楽[おんがく]を 聴[き]く		
\\	スーさんは、明日試験があると言っていました。	試験: 
\\	スーさんは、 明日[あした] 試験[しけん]があると 言[い]っていました。		
\\	武さんは、メアリーさんが好きだと思います。	好きだと思います: 
\\	な 
\\	だ 
\\	と思います. 
\\	好き 
\\	な 
\\	[好きな].	武[たけし]さんは、メアリーさんが 好[す]きだと 思[おも]います。	
\\	ここで写真を撮らないで下さい。	
\\	ここで 写真[しゃしん]を 撮[と]らないで 下[くだ]さい。	
\\	日本語を勉強するのが好きです。	
\\	日本語[にほんご]を 勉強[べんきょう]するのが 好[す]きです。	
\\	部屋を掃除するのが嫌いです。	
\\	部屋[へや]を 掃除[そうじ]するのが 嫌[きら]いです。	
\\	ロバートさんは、料理を作るのが上手です。	
\\	ロバートさんは、 料理[りょうり]を 作[つく]るのが 上手[じょうず]です。	
\\	武さんは、英語を話すのが下手です。	
\\	武[たけし]さんは、 英語[えいご]を 話[はな]すのが 下手[へた]です。	
\\	ロバートさんが、沖縄に行きました。	"が: 
\\	が 
\\	ロバートさんが、 沖縄[おきなわ]に 行[い]きました。		
\\	誰が沖縄に行きましたか?	が: 
\\	誰) 
\\	は、
\\	が.	誰[だれ]が 沖縄[おきなわ]に 行[い]きましたか?		
\\	どのクラスが、面白いですか?	
\\	どのクラスが、 面白[おもしろ]いですか?	
\\	日本語のクラスが面白いです。	
\\	日本語[にほんご]のクラスが 面白[おもしろ]いです。	
\\	誰が眼鏡をかけていますか?	
\\	誰[だれ]が 眼鏡[めがね]をかけていますか?	
\\	山下先生が、眼鏡をかけています。	
\\	山下[やました] 先生[せんせい]が、 眼鏡[めがね]をかけています。	
\\	猫が、何か持ってきました。	
\\	猫[ねこ]が、 何[なに]か 持[も]ってきました。	
\\	猫は、何か食べましたか?	
\\	猫[ねこ]は、 何[なに]か 食[た]べましたか?	
\\	いいえ、猫は何も食べませんでした。	何も
\\	いいえ、 猫[ねこ]は 何[なに]も 食[た]べませんでした。	
\\	日本語の勉強は楽しいです。	
\\	日本語[にほんご]の 勉強[べんきょう]は 楽[たの]しいです。	
\\	ロバートさんの料理は、美味しいです。	
\\	ロバートさんの 料理[りょうり]は、 美味[おい]しいです。	
\\	私は、日本語の勉強をしました。	
\\	私[わたし]は、 日本語[にほんご]の 勉強を[べんきょう]しました。	
\\	武さんは、部屋の掃除をしました。	
\\	武[たけ]さんは、 部屋[へや]の 掃除[そうじ]をしました。	
\\	日本語の勉強が好きです。	
\\	日本語[にほんご]の 勉強[べんきょう]が 好[す]きです。	
\\	メアリーさんは、歌舞伎が好きですか?	歌舞伎: 
\\	メアリーさんは、 歌舞伎[かぶき]が 好[す]きですか?		
\\	歌舞伎ですか?あまり知りません。	
\\	歌舞伎[かぶき]ですか?あまり 知[し]りません。	
\\	でも、ロバートさんは面白かったと言っていました。	面白かった: 
\\	面白い	でも、ロバートさんは 面白[おもしろ]かったと 言[い]っていました。		
\\	歌舞伎の切符を2枚貰ったから、見に行きませんか?	貰った [貰う]: 
\\	から: 
\\	歌舞伎[かぶき]の 切符[きっぷ]を2 枚[まい] 貰[もら]ったから、 見[み]に 行[い]きませんか?		
\\	ええ、是非!いつですか?	是非: 
\\	ええ、 是非[ぜひ]!いつですか?		
\\	木曜日です。12時から4時までです。	から: 
\\	まで: 
\\	木曜日[もくようび]です。12 時[じ]から4 時[じ]までです。		
\\	出ている人は、みんな男の人ですよ。	出て [出る]: 
\\	いる: 
\\	出ている人: 
\\	出[で]ている 人[ひと]は、みんな 男[おとこ]の 人[ひと]ですよ。		
\\	本当ですか?	
\\	本当[ほんとう]ですか?	
\\	ええ、ところでもう昼ご飯を食べましたか?	ところで: 
\\	もう: 
\\	ええ、ところでもう 昼[ひる]ご 飯[はん]を 食[た]べましたか?		
\\	いいえ、まだ食べていません。	まだ 
\\	ていません: 
\\	いいえ、まだ 食[た]べていません。		
\\	じゃあ、買いに行きましょう。	
\\	じゃあ、 買[か]いに 行[い]きましょう。	
\\	すみません。お弁当を2つ下さい。	お弁当: 
\\	2つ: 
\\	すみません。お 弁当[べんとう]を 2[ふた]つ 下[くだ]さい。		
\\	それから、お茶を1つとコーヒーを1つ。	
\\	それから、お 茶[ちゃ]を 1[ひと]つとコーヒーを 1[ひと]つ。	
\\	2800円です。どうもありがとうございました。	どうも: 
\\	円[えん]です。どうもありがとうございました。		
\\	いい子	
\\	いい 子[こ]		
\\	色	
\\	色[いろ]		
\\	ギター	
\\	ギター		
\\	薬	
\\	薬[くすり]		
\\	薬を飲む	
\\	薬[くすり]を 飲[の]む		
\\	コンサート	
\\	コンサート		
\\	今度	
\\	今度[こんど]		
\\	作文	
\\	作文[さくぶん]		
\\	試験	
\\	試験[しけん]		
\\	スキー	
\\	スキー		
\\	単語	
\\	単語[たんご]		
\\	ピアノ	
\\	ピアノ		
\\	ピザ	
\\	ピザ		
\\	病気	
\\	病気[びょうき]		
\\	青	
\\	青[あお]		
\\	赤	
\\	赤[あか]		
\\	黒	
\\	黒[くろ]		
\\	寂しい	
\\	寂[さび]しい		
\\	白	
\\	白[しろ]		
\\	若い	
\\	若[わか]い		
\\	意地悪	
\\	意地悪[いじわる]		
\\	踊る	
\\	踊[おど]る		
\\	終わる	
\\	終[お]わる		
\\	人気がある	
\\	人気[にんき]がある		
\\	始まる	
\\	始[はじ]まる		
\\	弾く	
\\	弾[ひ]く		
\\	覚える	
\\	覚[おぼ]える		
\\	運動する	
\\	運動[うんどう]する		
\\	散歩する	
\\	散歩[さんぽ]する		
\\	3つ	3つ: 
\\	つ 
\\	3[みっ]つ		
\\	4つ	4つ: 
\\	つ 
\\	4[よっ]つ		
\\	5つ	5つ: 
\\	つ 
\\	5[いつ]つ		
\\	6つ	6つ: 
\\	つ 
\\	6[むっ]つ		
\\	7つ	7つ: 
\\	つ 
\\	7[なな]つ		
\\	8つ	8つ: 
\\	つ 
\\	8[やっ]つ		
\\	9つ	9つ: 
\\	つ 
\\	9[ここの]つ		
\\	[とう] 
\\	10[とう]		
\\	晩ご飯食べた?	
\\	食べる: 
\\	食べて: 
\\	食べた: 
\\	晩[ばん]ご 飯[はん] 食[た]べた?		
\\	うん。食べた。	
\\	うん。 食[た]べた。		
\\	スーさんは、高校の時、眼鏡をかけていたと言っていました。	かける: 
\\	かけて: 
\\	かけていた: 
\\	スーさんは、 高校[こうこう]の 時[とき]、 眼鏡[めがね]をかけていたと 言[い]っていました。	
\\	トムさんがやったと思います。	やる: 
\\	やって: 
\\	やった: 
\\	トムさんがやったと 思[おも]います。	
\\	スーさんは、日本語を勉強していると言っていました。	
\\	スーさんは、 日本語[にほんご]を 勉強[べんきょう]していると 言[い]っていました。	
\\	あそこで本を読んでいる学生は、美智子さんです。	いる: 
\\	読んでいる学生: 
\\	あそこで 本[ほん]を 読[よ]んでいる 学生[がくせい]は、 美智子[みちこ]さんです。	
\\	あそこで写真を撮っている人は、誰ですか?	
\\	あそこで 写真[しゃしん]を 撮[と]っている 人[ひと]は、 誰[だれ]ですか?	
\\	毎日、運動をする人は、元気です。	運動: 
\\	運動をする人: 
\\	(運動をする) 
\\	(人). 
\\	毎日[まいにち]、 運動[うんどう]をする 人[ひと]は、 元気[げんき]です。	
\\	タバコを吸わない人が好きです。	
\\	タバコを 吸[す]わない 人[ひと]が 好[す]きです。	
\\	去年結婚した友達から、手紙が来ました。	
\\	去年[きょねん] 結婚[けっこん]した 友達[ともだち]から、 手紙[てがみ]が 来[き]ました。	
\\	私は、昨日宿題をしました。	
\\	私[わたし]は、 昨日[きのう] 宿題[しゅくだい]をしました。	
\\	私は、もう宿題をしました。	もう: 
\\	私[わたし]は、もう 宿題[しゅくだい]をしました。	
\\	私は昨日、宿題をしませんでした。	
\\	私[わたし]は 昨日[きのう]、 宿題[しゅくだい]をしませんでした。	
\\	私はまだ、宿題をしていません。	まだ: 
\\	私[わたし]はまだ、 宿題[しゅくだい]をしていません。	
\\	スーさんはまだ、起きていません。	
\\	スーさんはまだ、 起[お]きていません。	
\\	私はまだ、昼ご飯を食べていません。	
\\	私[わたし]はまだ、 昼[ひる]ご 飯[はん]を 食[た]べていません。	
\\	私は、朝ご飯を食べませんでした。忙しかったですから。	
\\	私[わたし]は、 朝[あさ]ご 飯[はん]を 食[た]べませんでした。 忙[いそが]しかったですから。	
\\	明日、試験があるから、私は今晩勉強します。	
\\	明日[あした]、 試験[しけん]があるから、 私[わたし]は 今晩[こんばん] 勉強[べんきょう]します。	
\\	寒かったから、出掛けませんでした。	出掛け [出掛ける]: 
\\	寒[さむ]かったから、 出掛[でか]けませんでした。	
\\	歌舞伎の切符がありますから、一緒に見に行きましょう。	
\\	歌舞伎[かぶき]の 切符[きっぷ]がありますから、 一緒[いっしょ]に 見[み]に 行[い]きましょう。	
\\	黄色	
\\	黄色[きいろ]		
\\	茶色	
\\	茶色[ちゃいろ]		
\\	緑	
\\	緑[みどり]		
\\	紫	
\\	紫[むらさき]		
\\	灰色	
\\	灰色[はいいろ]		
\\	水色	
\\	水色[みずいろ]		
\\	ピンク	
\\	ピンク		
\\	金色	
\\	金色[きんいろ]		
\\	銀色	
\\	銀色[ぎんいろ]		
\\	顔が青いですね。	
\\	顔[かお]が 青[あお]いですね。		
\\	白黒の写真	
\\	白黒[しろくろ]の 写真[しゃしん]		
\\	メアリーさんは、金髪です。	
\\	メアリーさんは、 金髪[きんぱつ]です。		
\\	真っ直ぐ行きます。	
\\	真っ直ぐ: 
\\	真っ直[まっす]ぐ 行[い]きます。		
\\	左に曲がります。	
\\	曲がり [曲がる]: 
\\	左[ひだり]に 曲[ま]がります。		
\\	右に曲がります。	
\\	曲がり [曲がる]: 
\\	右[みぎ]に 曲[ま]がります。		
\\	道を渡ります。	
\\	渡る [渡り]: 
\\	道: 
\\	道[みち]を 渡[わた]ります。		
\\	2つ目の角を左に曲がります。	
\\	2つ目: 
\\	角: 
\\	2[ふた]つ 目[め]の 角[かど]を 左[ひだり]に 曲[ま]がります。		
\\	1つ目の信号を右に曲がります。	
\\	1つ目: 
\\	信号: 
\\	1[ひと]つ 目[め]の 信号[しんごう]を 右[みぎ]に 曲[ま]がります。		
\\	道の左側。	
\\	側: 
\\	道[みち]の 左側[ひだりがわ]。		
\\	道の右側。	
\\	道[みち]の 右側[みぎがわ]。		
\\	北	
\\	北[きた]		
\\	東	
\\	東[ひがし]		
\\	南	
\\	南[みなみ]		
\\	西	
\\	西[にし]		
\\	寒くなりましたね。	なり[なる]: 
\\	寒[さむ]くなりましたね。	
\\	ええ。メアリーさん、冬休みはどうしますか?	冬休み: 
\\	どうしますか: 
\\	ええ。メアリーさん、 冬休[ふゆやす]みはどうしますか?		
\\	韓国か台湾に行くつもりですが、まだ決めていません。	か: 
\\	台湾: 
\\	つもり: 
\\	[行く] 
\\	ですが: 
\\	でも, 
\\	決めて [決める]: 
\\	韓国[かんこく]か 台湾[たいわん]に 行[い]くつもりですが、まだ 決[き]めていません。		
\\	韓国と台湾と、どっちの方が良いと思いますか?	どっち: 
\\	方: 
\\	韓国[かんこく]と 台湾[たいわん]と、どっちの 方[ほう]が 良[い]いと 思[おも]いますか?	
\\	うん、台湾の方が暖かいと思います。	暖かい: 
\\	うん、 台湾[たいわん]の 方[ほう]が 暖[あたた]かいと 思[おも]います。	
\\	でも、スーさんは韓国の食べ物は美味しいと言っていましたよ。	
\\	でも、スーさんは 韓国[かんこく]の 食べ物[たべもの]は 美味[おい]しいと 言[い]っていましたよ。	
\\	そうですか。 ところで、武さんはどこかに行きますか?	ところで: 
\\	どこか: 
\\	そうですか。 ところで、 武[たけ]さんはどこかに 行[い]きますか?		
\\	どこにも行きません。お金がないから、ここに居ます。	どこにも: 
\\	居ます [居る]: 
\\	どこにも 行[い]きません。お 金[かね]がないから、ここに 居[い]ます。		
\\	そうですか。じゃあ、武さんにお土産を買ってきますよ。	"に: 
\\	[お土産] 
\\	買ってきますよ: 
\\	買って 
\\	くる 
\\	そうですか。じゃあ、 武[たけ]さんにお 土産[みやげ]を 買[か]ってきますよ。		
\\	大阪からソウルまで、飛行機の予約をお願いします。	予約: 
\\	大阪[おおさか]からソウルまで、 飛行機[ひこうき]の 予約[よやく]をお 願[ねが]いします。	
\\	午前と午後の便がありますが、、、、、、	便: 
\\	午前[ごぜん]と 午後[ごご]の 便[びん]がありますが、、、、、、	
\\	午前のをお願いします。	"の: 
\\	[便] 
\\	[午前], 
\\	の. 
\\	午前[ごぜん]のをお 願[ねが]いします。		
\\	クレジットカードで払っても良いですか?	で: 
\\	[クレジットカード] 
\\	払って [払う]: 
\\	クレジットカードで 払[はら]っても 良[い]いですか?	
\\	ソウルまで、どのくらい懸りますか?	どのくらい: 
\\	懸り[懸る]: 
\\	ソウルまで、どのくらい 懸[かか]りますか?		
\\	秋	
\\	秋[あき]		
\\	駅	
\\	駅[えき]		
\\	お金持ち	
\\	お 金持[かねも]ち		
\\	季節	
\\	季節[きせつ]		
\\	牛乳	
\\	牛乳[ぎゅうにゅう]		
\\	ケーキ	
\\	ケーキ		
\\	サッカー	
\\	サッカー		
\\	新幹線	
\\	新幹線[しんかんせん]		
\\	寿司	
\\	寿司[すし]		
\\	生活	
\\	生活[せいかつ]		
\\	世界	
\\	世界[せかい]		
\\	地下鉄	
\\	地下鉄[ちかてつ]		
\\	手袋	
\\	手袋[てぶくろ]		
\\	天ぷら	
\\	天[てん]ぷら		
\\	床屋	
\\	床屋[とこや]		
\\	春	
\\	春[はる]		
\\	パンツ	
\\	パンツ		
\\	美容院	
\\	美容[びよう] 院[いん]		
\\	船	
\\	船[ふね]		
\\	野球	
\\	野球[やきゅう]		
\\	有名人	
\\	有名人[ゆうめいじん]		
\\	来学期	
\\	来[らい] 学期[がっき]		
\\	リンゴ	
\\	リンゴ		
\\	遅い	
\\	遅[おそ]い		
\\	涼しい	
\\	涼[すず]しい		
\\	冷たい	
\\	冷[つめ]たい		
\\	眠い	
\\	眠[ねむ]い		
\\	簡単	
\\	簡単[かんたん]		
\\	泊まる	
\\	泊[と]まる		
\\	旅行する	
\\	旅行[りょこう]する		
\\	練習する	
\\	練習[れんしゅう]する		
\\	歩いて	
\\	歩[ある]いて		
\\	一番	
\\	一番[いちばん]		
\\	ヶ月	
\\	ヶ月[かげつ]		
\\	後	
\\	後[のち]		
\\	この頃	
\\	この 頃[ころ]		
\\	週間	
\\	週間[しゅうかん]		
\\	どうやって	
\\	どうやって		
\\	どちら	
\\	どちら		
\\	年	
\\	年[とし]		
\\	早く	
\\	早[はや]く		
\\	中国の方が、日本より大きいです	
\\	[~の方が] 
\\	[~より] 
\\	中国[ちゅうごく]の 方[ほう]が、 日本[にほん]より 大[おお]きいです	
\\	バスと電車と、どっちの方が安いですか?	
\\	バスと 電車[でんしゃ]と、どっちの 方[ほう]が 安[やす]いですか?	
\\	ロシアとフランスと日本の中で、どこが一番寒いですか?	中: 
\\	ロシアとフランスと 日本[にほん]の 中[なか]で、どこが 一番[いちばん] 寒[さむ]いですか?	
\\	ロシアが一番寒いと思います。	
\\	ロシアが 一番[いちばん] 寒[さむ]いと 思[おも]います。	
\\	季節の中で、いつが一番好きですか?	
\\	季節[きせつ]の 中[なか]で、いつが 一番[いちばん] 好[す]きですか?	
\\	秋が一番好きです。	
\\	秋[あき]が 一番[いちばん] 好[す]きです。	
\\	私は、黒いセーターを持っています。	
\\	私[わたし]は、 黒[くろ]いセーターを 持[も]っています。	
\\	赤いのも、持っています。	の 
\\	私は、黒いセーターを持っています。
\\	の 
\\	赤いのも、持っています。
\\	セーター 
\\	赤[あか]いのも、 持[も]っています。	
\\	安い辞書を、買いに行きました。	
\\	安[やす]い 辞書[じしょ]を、 買[か]いに 行[い]きました。	
\\	でも、良いのがありませんでした。	
\\	でも、 良[い]いのがありませんでした。	
\\	アメリカのアイスクリームの方が、日本のより美味しいです。	
\\	アイスクリーム 
\\	日本 
\\	の 
\\	アメリカのアイスクリームの 方[ほう]が、 日本[にほん]のより 美味[おい]しいです。	
\\	週末に、武さんとテニスをするつもりです。	"武さんとテニス: と 
\\	週末[しゅうまつ]に、 武[たけ]さんとテニスをするつもりです。	
\\	山下先生は、明日大学に来ないつもりです。	来ない [来る]: 
\\	山下[やました] 先生[せんせい]は、 明日[あした] 大学[だいがく]に 来[こ]ないつもりです。	
\\	お寺を見に行くつもりでしたが、天気がよくなかったから、行きませんでした。	よく: 
\\	天気がよくなかった: 
\\	お 寺[てら]を 見[み]に 行[い]くつもりでしたが、 天気[てんき]がよくなかったから、 行[い]きませんでした。	
\\	日本語の勉強が、楽しくなりました。	い 
\\	い 
\\	く
\\	なる.	日本語[にほんご]の 勉強[べんきょう]が、 楽[たの]しくなりました。	
\\	日本語の勉強が、好きになりました。	な 
\\	な 
\\	に 
\\	なる好きに [好きな]	日本語[にほんご]の 勉強[べんきょう]が、 好[す]きになりました。	
\\	メアリーさんは、前より日本語が上手になりました。	前より: 
\\	メアリーさんは、 前[まえ]より 日本語[にほんご]が 上手[じょうず]になりました。	
\\	どこかへ行きましたか?	
\\	どこかへ 行[い]きましたか?	
\\	いいえ、どこへも行きませんでした。	
\\	いいえ、どこへも 行[い]きませんでした。	
\\	誰かに会いましたか?	誰か: 
\\	誰[だれ]かに 会[あ]いましたか?	
\\	いいえ、誰にも会いませんでした。	誰にも: 
\\	いいえ、 誰[だれ]にも 会[あ]いませんでした。	
\\	何かしましたか?	何か: 
\\	何[なに]かしましたか?	
\\	いいえ、何もしませんでした。	
\\	いいえ、 何[なに]もしませんでした。	
\\	箸でご飯を食べます。	で: 
\\	箸[はし]でご 飯[はん]を 食[た]べます。	
\\	日本語で話しましょう。	
\\	日本語[にほんご]で 話[はな]しましょう。	
\\	バスで駅まで行きました。	
\\	バスで 駅[えき]まで 行[い]きました。	
\\	テレビで映画を観ました。	
\\	テレビで 映画[えいが]を 観[み]ました。	
\\	普通	
\\	普通[ふつう]		
\\	急行	
\\	急行[きゅうこう]		
\\	特急	
\\	特急[とっきゅう]		
\\	自由席	
\\	自由[じゆう] 席[せき]		
\\	禁煙車	
\\	禁煙[きんえん] 車[しゃ]		
\\	往復	
\\	往復[おうふく]		
\\	片道	
\\	片道[かたみち]		
\\	乗車券	
\\	乗車[じょうしゃ] 券[けん]		
\\	回数券	
\\	回数[かいすう] 券[けん]		
\\	定期券	
\\	定期[ていき] 券[けん]		
\\	学割	
\\	学割[がくわり]		
\\	指定席	
\\	指定[してい] 席[せき]		
\\	番線	
\\	番線[ばんせん]		
\\	切符売り場	
\\	切符[きっぷ] 売り場[うりば]		
\\	改札	
\\	改札[かいさつ]		
\\	ホーム	
\\	ホーム		
\\	売店	
\\	売店[ばいてん]		
\\	出口	
\\	出口[でぐち]		
\\	入り口	
\\	入り口[いりぐち]		
\\	階段	
\\	階段[かいだん]		
\\	終電	
\\	終電[しゅうでん]		
\\	乗り換え	
\\	乗り換[のりか]え		
\\	間もなく出発します。	間もなく: 
\\	出発: 
\\	間[ま]もなく 出発[しゅっぱつ]します。		
\\	電車が参ります。	参り [参る]: 
\\	電車[でんしゃ]が 参[まい]ります。		
\\	次は、... に停まります。	停まり [停まる]: 
\\	次[つぎ]は、... に 停[と]まります。		
\\	ドアが閉まります。ご注意ください。	ご注意: 
\\	ドアが 閉[し]まります。ご 注意[ちゅうい]ください。		
\\	メアリーさん、久しぶりですね。休みはどうでしたか?	久しぶり: 
\\	メアリーさん、 久[ひさ]しぶりですね。 休[やす]みはどうでしたか?		
\\	凄く楽しかったです。韓国で買い物をしたり、韓国料理を食べたりしました。	たり: 
\\	凄[すご]く 楽[たの]しかったです。 韓国[かんこく]で 買い物[かいもの]をしたり、 韓国[かんこく] 料理[りょうり]を 食[た]べたりしました。	
\\	良いですね。私も旅行したいです。	したい: 
\\	たい: 
\\	良[い]いですね。 私[わたし]も 旅行[りょこう]したいです。	
\\	美智子さんの休みは、楽しかったですか?	
\\	美智子[みちこ]さんの 休[やす]みは、 楽[たの]しかったですか?	
\\	まあまあでした。一日だけど、ドライブに行きましたが、毎日アルバイトをしていました。	まあまあ: 
\\	一日: 
\\	だけど: 
\\	まあまあでした。一 日[いちにち]だけど、ドライブに 行[い]きましたが、 毎日[まいにち]アルバイトをしていました。	
\\	美智子さん、友達を紹介します。	紹介: 
\\	美智子[みちこ]さん、 友達[ともだち]を 紹介[しょうかい]します。	
\\	此方はジョンさんです。	此方: 
\\	此方[こちら]はジョンさんです。	
\\	ジョンさんは、先月日本に来ました。	
\\	ジョンさんは、 先月[せんげつ] 日本[にほん]に 来[き]ました。	
\\	初めまして。山川美智子です。	山川: 
\\	初[はじ]めまして。 山川[やまかわ] 美智子[みちこ]です。	
\\	ジョンさん、出身はどこですか?	出身: 
\\	ジョンさん、 出身[しゅっしん]はどこですか?	
\\	オーストラリアのケアンズです。	
\\	オーストラリアのケアンズです。	
\\	美智子さんは、ケアンズに行った事がありますか?	
\\	事がある: 
\\	事がある 
\\	事がありますか, 
\\	あります/ありません 
\\	あります/ありません.
\\	美智子[みちこ]さんは、ケアンズに 行[い]った 事[こと]がありますか?	
\\	山や海があって、綺麗な所ですよ。	や: 
\\	や 
\\	山や海 
\\	山[やま]や 海[うみ]があって、 綺麗[きれい]な 所[ところ]ですよ。	
\\	グレートバリアリーフで有名です。	
\\	グレートバリアリーフで 有名[ゆうめい]です。	
\\	美智子さんは、どこの出身ですか?	
\\	美智子[みちこ]さんは、どこの 出身[しゅっしん]ですか?	
\\	長野です。今度遊びに来て下さい。	長野: 
\\	長野[ながの]です。 今度[こんど] 遊[あそ]びに 来[き]て 下[くだ]さい。	
\\	食べ物も美味しいですよ。	
\\	食べ物[たべもの]も 美味[おい]しいですよ。	
\\	今度の週末は、映画を見たいです。	
\\	今度[こんど]の 週末[しゅうまつ]は、 映画[えいが]を 見[み]たいです。	
\\	いつか、中国に行きたいです。	いつか: 
\\	いつか、 中国[ちゅうごく]に 行[い]きたいです。	
\\	あの人には、会いたくないです。	たくない 
\\	たい]: 
\\	あの 人[ひと]には、 会[あ]いたくないです。	
\\	セーターが買いたかったから、デパートに行きました。	たかった 
\\	たい]: 
\\	セーターが 買[か]いたかったから、デパートに 行[い]きました。	
\\	メアリーさんは、チベットに行きたいと言っていました。	チベット: 
\\	メアリーさんは、チベットに 行[い]きたいと 言[い]っていました。	
\\	メアリーさんは、着物を着たがっています。	たがっています [たがっている]: 
\\	たい.	メアリーさんは、 着物[きもの]を 着[き]たがっています。	
\\	大阪で買い物をして、晩ご飯を食べます。	
\\	大阪[おおさか]で 買い物[かいもの]をして、 晩[ばん]ご 飯[はん]を 食[た]べます。	
\\	大阪で買い物をしたり、晩ご飯を食べたりします。	
\\	大阪[おおさか]で 買い物[かいもの]をしたり、 晩[ばん]ご 飯[はん]を 食[た]べたりします。	
\\	週末は、勉強したり、友達と話したりしました。	
\\	週末[しゅうまつ]は、 勉強[べんきょう]したり、 友達[ともだち]と 話[はな]したりしました。	
\\	踊ったり、音楽を聞いたりするのが好きです。	踊った [踊る]: 
\\	踊[おど]ったり、 音楽[おんがく]を 聞[き]いたりするのが 好[す]きです。	
\\	富士山に登った事があります。	富士山: 
\\	登った [登る]: 
\\	富士山[ふじさん]に 登[のぼ]った 事[こと]があります。	
\\	武さんは、授業を休んだ事がありません。	授業: 
\\	休んだ [休む]: 
\\	武[たけ]さんは、 授業[じゅぎょう]を 休[やす]んだ 事[こと]がありません。	
\\	京都や奈良に行きました。	奈良: 
\\	京都[きょうと]や 奈良[なら]に 行[い]きました。	
\\	山下先生は、テレビを見ますか?	
\\	山下[やました] 先生[せんせい]は、テレビを 見[み]ますか?	
\\	いいえ、テレビは見ません。	は 
\\	は 
\\	が 
\\	を. 
\\	は 
\\	いいえ、テレビは 見[み]ません。	
\\	コーヒーが好きですか?	
\\	コーヒーが 好[す]きですか?	
\\	いいえ、コーヒーは好きじゃないです。	は 
\\	は 
\\	が 
\\	を. 
\\	は 
\\	いいえ、コーヒーは 好[す]きじゃないです。	
\\	英語では、話したくないです。	は 
\\	で 
\\	に.	英語[えいご]では、 話[はな]したくないです。	
\\	広島には、行った事がありません。	は 
\\	で 
\\	に 広島: 
\\	広島[ひろしま]には、 行[い]った 事[こと]がありません。	
\\	私は、その人に1回だけ会った事があります。	1回: 
\\	だけ: 
\\	だけ 
\\	だけ 
\\	私[わたし]は、その 人[ひと]に1 回[いかい]だけ 会[あ]った 事[こと]があります。	
\\	1つだけ、質問があります。	質問: 
\\	1[ひと]つだけ、 質問[しつもん]があります。	
\\	30分だけ寝ました。	
\\	分[ぷん]だけ 寝[ね]ました。	
\\	晩ご飯に、サラダを食べました。	に: 
\\	に 
\\	晩[ばん]ご 飯[はん]に、サラダを 食[た]べました。	
\\	お土産に、絵葉書を買いました。	に: 
\\	に 
\\	絵葉書: 
\\	お 土産[みやげ]に、 絵葉書[えはがき]を 買[か]いました。	
\\	湖まで、ドライブに行きました。	湖: 
\\	ドライブ: 
\\	ドライブに行きました 
\\	ドライブしました	湖[みずうみ]まで、ドライブに 行[い]きました。		
\\	日本で車を運転した事がありますか?	"運転した: 
\\	日本[にほん]で 車[くるま]を 運転[うんてん]した 事[こと]がありますか?	
\\	夕べ、怖い夢を見ました。	夕べ: 
\\	夢: 
\\	見る 
\\	持っている 
\\	ある 
\\	夕[ゆう]べ、 怖[こわ]い 夢[ゆめ]を 見[み]ました。		
\\	夢を持っています。	
\\	夢[ゆめ]を 持[も]っています。	
\\	あなたの将来の夢は何ですか?	
\\	あなたの 将来[しょうらい]の 夢[ゆめ]は 何[なん]ですか?	
\\	京都には、デパートが沢山あります。	には: 
\\	京都[きょうと]には、デパートが 沢山[たくさん]あります。	
\\	どちらでも良いです。	
\\	どちらでも 良[い]いです。		
\\	同じです。	同じ: 
\\	同[おな]じです。	[同じな] な-
\\	大体同じです。	大体: 
\\	大体[だいたい] 同[おな]じです。	
\\	ちょっと違います。	違い: 
\\	ちょっと 違[ちが]います。	
\\	使いません。	
\\	使[つか]いません。		
\\	駄目です。	
\\	駄目[だめ]です。		
\\	手を挙げて下さい。	挙げる: 
\\	手[て]を 挙[あ]げて 下[くだ]さい。	
\\	読んで来て下さい。		読[よ]んで 来[き]て 下[くだ]さい。		
\\	宿題を出して下さい。	出す: 
\\	宿題[しゅくだい]を 出[だ]して 下[くだ]さい。		
\\	教科書を閉じて下さい。	閉じる: 
\\	教科書[きょうかしょ]を 閉[と]じて 下[くだ]さい。		
\\	隣の人に、聞いて下さい。	聞く: 
\\	隣[となり]の 人[ひと]に、 聞[き]いて 下[くだ]さい。		
\\	止めて下さい。	止める: 
\\	止[や]めて 下[くだ]さい。		
\\	今日はこれで終わります。	終わる: 
\\	今日[きょう]はこれで 終[お]わります。		
\\	メアリーさん、元気がありませんね。	
\\	メアリーさん、 元気[げんき]がありませんね。	
\\	うん、ちょっとお腹が痛いんです。	痛い: 
\\	んです: 
\\	うん、ちょっとお 腹[なか]が 痛[いた]いんです。		
\\	どうしたんですか?	どうしたん: 
\\	どうした
\\	んです 
\\	んです 
\\	どうしたんですか?		
\\	昨日、友達と晩ご飯を食べに行ったんです。	
\\	昨日[きのう]、 友達[ともだち]と 晩[ばん]ご 飯[はん]を 食[た]べに 行[い]ったんです。	
\\	多分、食べ過ぎたんだと思います。	多分: 
\\	過ぎ [過ぎる]:
\\	多分[たぶん]、 食[た]べ 過[す]ぎたんだと 思[おも]います。	
\\	ええ。心配しないで下さい。あ....痛い.	心配: 
\\	ええ。 心配[しんぱい]しないで 下[くだ]さい。あ...痛[いた]い	
\\	病院に行った方が良いですよ。	"方が良いです: 
\\	病院[びょういん]に 行[い]った 方[ほう]が 良[い]いですよ。	
\\	先生、喉が痛いんです。昨日は、お腹が痛かったんです。	喉: 
\\	先生[せんせい]、 喉[のど]が 痛[いた]いんです。 昨日[きのう]は、お 腹[なか]が 痛[いた]かったんです。		
\\	ああ、そうですか?熱もありますね。風邪ですね。	熱: 
\\	風邪: 
\\	ああ、そうですか? 熱[ねつ]もありますね。 風邪[かぜ]ですね。		
\\	あの、もうすぐテニスの試合があるので,	試合: 
\\	ので: 
\\	から 
\\	あの、もうすぐテニスの 試合[しあい]があるので、	
\\	3日運動しない方が良いですよ。	
\\	日[にち] 運動[うんどう]しない 方[ほう]が 良[い]いですよ。	
\\	今日は、薬を飲んで早く寝て下さい。	
\\	今日[きょう]は、 薬[くすり]を 飲[の]んで 早[はや]く 寝[ね]て 下[くだ]さい。	
\\	バスが来なかったんです。	んです 
\\	バスが来ませんでした 
\\	バスが 来[こ]なかったんです。	
\\	明日、テストがあります。	
\\	明日[あした]、テストがあります。	
\\	明日、テストがあるんです。	
\\	明日[あした]、テストがあるんです。	
\\	トイレに行きたいです。	
\\	トイレに 行[い]きたいです。	
\\	トイレに行きたいんです。	
\\	トイレに 行[い]きたいんです。	
\\	成績が良くないんです。	成績: 
\\	成績[せいせき]が 良[よ]くないんです。	
\\	試験が終わったんです。	
\\	試験[しけん]が 終[お]わったんです。	
\\	どうして彼と別れたいんですか?	彼: 
\\	別れ [別れる]: 
\\	んですか: 
\\	んです 
\\	どうして 彼[かれ]と 別[わか]れたいんですか?	
\\	彼、全然お風呂に入らないんです。	
\\	彼[かれ]、 全然[ぜんぜん]お 風呂[ふろ]に 入[はい]らないんです。	
\\	猫が死んだんです。	
\\	猫[ねこ]が 死[し]んだんです。	
\\	とっても良い教科書ですね。	
\\	とっても 良[い]い 教科書[きょうかしょ]ですね。	
\\	ええ。私の大学の先生が書いたんです。	んです: 
\\	ええ。 私[わたし]の 大学[だいがく]の 先生[せんせい]が 書[か]いたんです。	
\\	早く起き過ぎました。	過ぎる: 
\\	早[はや]く 起[お]き 過[す]ぎました。	
\\	食べ過ぎてはいけません。	いけません: 
\\	食[た]べ 過[す]ぎてはいけません。	
\\	この本は、高過ぎます。	"過ぎ: 
\\	この 本[ほん]は、 高[たか] 過[す]ぎます。	
\\	この町は、静か過ぎます。	町: 
\\	静か: 
\\	この 町[まち]は、 静[しず]か 過[す]ぎます。	
\\	もっと野菜を食べた方が良いですよ。	方が良いです: 
\\	食べた 
\\	野菜: 
\\	もっと 野菜[やさい]を 食[た]べた 方[ほう]が 良[い]いですよ。	
\\	授業を休まない方が良いですよ。	
\\	授業[じゅぎょう]を 休[やす]まない 方[ほう]が 良[い]いですよ。	
\\	いつも日本語で話すので、日本語が上手になりました。	
\\	ので 
\\	いつも 日本語[にほんご]で 話[はな]すので、 日本語[にほんご]が 上手[じょうず]になりました。	
\\	宿題が沢山あったので、昨日の夜寝ませんでした。	夜: 
\\	宿題[しゅくだい]が 沢山[たくさん]あったので、 昨日[きのう]の 夜[よる] 寝[ね]ませんでした。	
\\	その人は、意地悪なので嫌いです。	意地悪な [意地悪]: 
\\	その 人[ひと]は、 意地悪[いじわる]なので 嫌[きら]いです。	
\\	今日は日曜日なので、銀行は休みです。	
\\	今日[きょう]は 日曜日[にちようび]なので、 銀行[ぎんこう]は 休[やす]みです。	
\\	試験があるから、勉強しなければいけません。	
\\	なければいけません: 
\\	試験[しけん]があるから、 勉強[べんきょう]しなければいけません。	
\\	今朝は6時に起きなきゃいけませんでした。	今朝: 
\\	今朝[けさ]は6 時[じ]に 起[お]きなきゃいけませんでした。	
\\	毎日、練習しなきゃいけないんです。	
\\	毎日[まいにち]、 練習[れんしゅう]しなきゃいけないんです。	
\\	明日は、雨が降るでしょう。	でしょう: 
\\	明日[あした]は、 雨[あめ]が 降[ふ]るでしょう。	
\\	明日は、雨が降らないでしょう。	
\\	明日[あした]は、 雨[あめ]が 降[ふ]らないでしょう。	
\\	北海道は、寒いでしょう。	北海道: 
\\	北海道[ほっかいどう]は、 寒[さむ]いでしょう。	
\\	北海道は、寒くないでしょう。	
\\	北海道[ほっかいどう]は、 寒[さむ]くないでしょう。	
\\	山下先生は、魚が好きでしょう。	
\\	山下[やました] 先生[せんせい]は、 魚[さかな]が 好[す]きでしょう。	
\\	山下先生は、魚が好きじゃないでしょう。	
\\	山下[やました] 先生[せんせい]は、 魚[さかな]が 好[す]きじゃないでしょう。	
\\	あの人は、オーストラリア人でしょう。	
\\	あの 人[ひと]は、オーストラリア 人[じん]でしょう。	
\\	あの人は、オーストラリア人じゃないでしょう。	
\\	あの 人[ひと]は、オーストラリア 人[じん]じゃないでしょう。	
\\	日本語と韓国語と、どっちの方が難しいでしょうか?	でしょうか: 
\\	日本語[にほんご]と 韓国[かんこく] 語[ご]と、どっちの 方[ほう]が 難[むずか]しいでしょうか?	
\\	武さんは、興味があるだろうと思います。	興味: 
\\	だろう: 
\\	だろう 
\\	でしょう. 
\\	武[たけ]さんは、 興味[きょうみ]があるだろうと 思[おも]います。	
\\	ジョン、中国語が分かるでしょう?これ、読んで?	
\\	でしょう 
\\	ジョン、 中国[ちゅうごく] 語[ご]が 分[わ]かるでしょう?これ、 読[よ]んで?	
\\	私、ジョンワンと申します。	と申します: 
\\	私[わたくし]、ジョンワンと 申[もう]します。	
\\	アルバイト募集の広告を見たんですが。	募集: 
\\	広告: 
\\	アルバイト 募集[ぼしゅう]の 広告[こうこく]を 見[み]たんですが。	
\\	そうですか。じゃあ、会って、話しましょうか。	
\\	そうですか。じゃあ、 会[あ]って、 話[はな]しましょうか。	
\\	今日店に来られますか?	来られますか [来られる]: 
\\	今日[きょう] 店[みせ]に 来[こ]られますか?	
\\	今日はちょっと行けないんですが、明日なら行けると思います。	
\\	今日[きょう]はちょっと 行[い]けないんですが、 明日[あした]なら 行[い]けると 思[おも]います。	
\\	そうですか。今日は駄目ですか。	
\\	そうですか。 今日[きょう]は 駄目[だめ]ですか。	
\\	じゃあ、明日の1時頃はどうですか?	
\\	じゃあ、 明日[あした]の1 時[じ] 頃[ごろ]はどうですか?	
\\	1時ですね。分かりました。	
\\	時[じ]ですね。 分[わ]かりました。	
\\	ワンさんは、どうしてこのアルバイトに興味があるんですか?	
\\	ワンさんは、どうしてこのアルバイトに 興味[きょうみ]があるんですか?	
\\	面白そうですから。色々な人に会えるし、日本語も使えるし。	色々: 
\\	面白[おもしろ]そうですから。 色々[いろいろ]な 人[ひと]に 会[あ]えるし、 日本語[にほんご]も 使[つか]えるし。	
\\	レストランで働いた事がありますか?	働い [働く]: 
\\	レストランで 働[はたら]いた 事[こと]がありますか?	
\\	はい。ウエイターならした事があります。	
\\	はい。ウエイターならした 事[こと]があります。	
\\	明日から始められますか?	
\\	明日[あした]から 始[はじ]められますか?	
\\	はい。よろしくお願いします。	よろしくお願いします: 
\\	はい。よろしくお 願[ねが]いします。	
\\	頑張って下さい。	頑張って[頑張る]: 
\\	頑張[がんば]って 下[くだ]さい。	
\\	いらっしゃいませ。あっ、山下先生!	
\\	いらっしゃいませ。あっ、 山下[やました] 先生[せんせい]!	
\\	ジョンさん! ここでアルバイトをしているんですか?	しているん:	ジョンさん! ここでアルバイトをしているんですか?	
\\	ええ。1週間に3日働いています。	
\\	ええ。1 週間[しゅうかん]に3 日[か] 働[はたら]いています。	
\\	そうですか。どれが美味しいですか?	
\\	そうですか。どれが 美味[おい]しいですか?	
\\	このカレーが、1番人気がありますよ。	人気: 
\\	このカレーが、1 番[ばん] 人気[にんき]がありますよ。	
\\	美味しそうですね。じゃあ、食べてみます。	
\\	美味[おい]しそうですね。じゃあ、 食[た]べてみます。	
\\	私は日本語が話せます。	話せます [話せる]: 
\\	私[わたし]は 日本語[にほんご]が 話[はな]せます。	
\\	私は、泳げないんです。	泳げない [泳げる]: 
\\	私[わたし]は、 泳[およ]げないんです。	
\\	雨が降ったので、海に行けませんでした。	行けません [行ける]: 
\\	雨[あめ]が 降[ふ]ったので、 海[うみ]に 行[い]けませんでした。	
\\	どうしてパーティーに来ないんですか?	
\\	どうしてパーティーに 来[こ]ないんですか?	
\\	明日、試験があるから、今日は勉強しなきゃいけないんです。	
\\	明日[あした]、 試験[しけん]があるから、 今日[きょう]は 勉強[べんきょう]しなきゃいけないんです。	
\\	日本語は面白いし、先生は良いし、私は日本語の授業が大好きです。	し: 
\\	し 
\\	から.	日本語[にほんご]は 面白[おもしろ]いし、 先生[せんせい]は 良[い]いし、 私[わたし]は 日本語[にほんご]の 授業[じゅぎょう]が 大好[だいす]きです。	
\\	友達が遊びに来たし、彼と電話で話したし、昨日はとっても良い日でした。	
\\	友達[ともだち]が 遊[あそ]びに 来[き]たし、 彼[かれ]と 電話[でんわ]で 話[はな]したし、 昨日[きのう]はとっても 良[い]い 日[ひ]でした。	
\\	国に帰りたいですか?	
\\	国[くに]に 帰[かえ]りたいですか?	
\\	いいえ。日本の生活は楽しいし、良い友達が居るし、帰りたくないです。	生活: 
\\	いいえ。 日本[にほん]の 生活[せいかつ]は 楽[たの]しいし、 良[い]い 友達[ともだち]が 居[い]るし、 帰[かえ]りたくないです。	
\\	物価が安いし、この町の生活は楽です。	物価: 
\\	し: 
\\	し 
\\	楽: 
\\	物価[ぶっか]が 安[やす]いし、この 町[まち]の 生活[せいかつ]は 楽[らく]です。	
\\	山下先生は、良い先生です。教えるのが上手だし、親切だし。	し: 
\\	し 
\\	し 
\\	だ 
\\	な-
\\	い- 
\\	山下[やました] 先生[せんせい]は、 良[い]い 先生[せんせい]です。 教[おし]えるのが 上手[じょうず]だし、 親切[しんせつ]だし。	
\\	このリンゴは美味しそうです。	~そうです: 
\\	このリンゴは 美味[おい]しそうです。	
\\	明日は、天気が良さそうです。	
\\	明日[あした]は、 天気[てんき]が 良[よ]さそうです。	
\\	メアリーさんは、元気そうでした。	
\\	メアリーさんは、 元気[げんき]そうでした。	
\\	この本は、難しくなさそうです。	
\\	この 本[ほん]は、 難[むずか]しくなさそうです。	
\\	智子さんは、テニスが上手じゃなさそうです。	智子: 
\\	智子[さとこ]さんは、テニスが 上手[じょうず]じゃなさそうです。	
\\	暖かそうなセーターを着ています。	暖か: 
\\	暖[あたた]かそうなセーターを 着[き]ています。	
\\	漢字が分からなかったので、日本人の友達に聞いてみました。	"聞いてみました: 
\\	~て 
\\	みる 
\\	漢字[かんじ]が 分[わ]からなかったので、 日本人[にほんじん]の 友達[ともだち]に 聞[き]いてみました。	
\\	友達が、あの店のケーキは美味しいと言っていましたから、今度食べてみます。		友達[ともだち]が、あの 店[みせ]のケーキは 美味[おい]しいと 言[い]っていましたから、 今度[こんど] 食[た]べてみます。	
\\	ブラジルに行った事がありますか?	
\\	ブラジルに 行[い]った 事[こと]がありますか?	
\\	チリなら行った事がありますが、ブラジルは行った事がありません。	なら: 
\\	チリなら 行[い]った 事[こと]がありますが、ブラジルは 行[い]った 事[こと]がありません。	
\\	日本語が分かりますか?	
\\	日本語[にほんご]が 分[わ]かりますか?	
\\	平仮名なら分かります。	
\\	平仮名[ひらがな]なら 分[わ]かります。	
\\	私は、1週間に3回髪を洗います。	1週間: 
\\	3回: 
\\	私[わたし]は、1 週間[しゅうかん]に3 回[かい] 髪[かみ]を 洗[あら]います。	
\\	私は、1ヶ月に1回家族に電話をかけます。	1ヶ月: 
\\	私[わたし]は、1 ヶ月[かげつ]に1 回[かい] 家族[かぞく]に 電話[でんわ]をかけます。	
\\	父は、1年に2回旅行します。	1年: 
\\	父[ちち]は、1年[いちねん]に2 回[かい] 旅行[りょこう]します。	
\\	ギターを弾く	弾く: 
\\	ギターを 弾[ひ]く		
\\	ピアノを弾く	弾く: 
\\	ピアノを 弾[ひ]く		
\\	サックスを吹く	吹く: 
\\	サックスを 吹[ふ]く		
\\	ドラムを叩く	叩く: 
\\	ドラムを 叩[たた]く		
\\	何か楽器が出来ますか?	楽器: 
\\	出来ます [出来る]: 
\\	何[なに]か 楽器[がっき]が 出来[でき]ますか?	
\\	何か楽器をやりますか?	やり [やる]: 
\\	何[なに]か 楽器[がっき]をやりますか?	
\\	ロバートさんは料理が上手です。	
\\	ロバートさんは 料理[りょうり]が 上手[じょうず]です。	
\\	ロバートさんは、上手に料理ができます。	上手に料理: い-
\\	な-
\\	な-
\\	に 
\\	ロバートさんは、 上手[じょうず]に 料理[りょうり]ができます。	
\\	日本語のクラスが楽しいです。	
\\	日本語[にほんご]のクラスが 楽[たの]しいです。	
\\	毎日、日本語を楽しく勉強しています。	楽しく勉強: い-
\\	な-
\\	い-
\\	い 
\\	く 
\\	毎日[まいにち]、 日本語[にほんご]を 楽[たの]しく 勉強[べんきょう]しています。	
\\	バレンタインのプレゼントは、何が良いと思いますか?	
\\	バレンタインのプレゼントは、 何[なに]が 良[い]いと 思[おも]いますか?	
\\	そうですね。武さんは、いつも同じセーターを着ているから、セーターをあげたらどうですか?	
\\	そうですね。 武[たけ]さんは、いつも 同[おな]じセーターを 着[き]ているから、セーターをあげたらどうですか?	
\\	それは良いかもしれませんね!	
\\	それは 良[い]いかもしれませんね!	
\\	えっ? 僕に? どうもありがとう。	
\\	えっ? 僕[ぼく]に? どうもありがとう。	
\\	開けても良い?	
\\	か 
\\	開[あ]けても 良[い]い?	
\\	わあ、良いね、このセーター。	
\\	わあ、 良[い]いね、このセーター。	
\\	こんなのが欲しかったんだ!	こんな: 
\\	こんなのが 欲[ほ]しかったんだ!	
\\	メアリーが編んだの?	編ん [編む]: 
\\	の: 
\\	の, 
\\	んですか. 
\\	メアリーが 編[あ]んだの?	
\\	うん、小さいかもしれないから着てみて。	着てみて:	うん、 小[ちい]さいかもしれないから 着[き]てみて。	
\\	ちょうどいいよ。ありがとう。	ちょうど: 
\\	ちょうどいいよ。ありがとう。	
\\	これ、メアリーがくれたんです。	
\\	これ、メアリーがくれたんです。	
\\	よく似合っていますよ。	よく: 
\\	似合って [似合う]: 
\\	よく 似合[にあ]っていますよ。	
\\	僕も彼女が欲しいなあ。	彼女: 
\\	僕[ぼく]も 彼女[かのじょ]が 欲[ほ]しいなあ。	
\\	ロバートさんは、チョコレートを10個も貰ったんですよ。	個: 
\\	ロバートさんは、チョコレートを10 個[こ]も 貰[もら]ったんですよ。	
\\	へえ、凄いですね。ジョンさんは?	
\\	へえ、 凄[すご]いですね。ジョンさんは?	
\\	僕は1個しか貰えませんでした。大家さんから。	
\\	しか 
\\	ません. 
\\	貰える 
\\	大家さん: 
\\	僕[ぼく]は1 個[こ]しか 貰[もら]えませんでした。 大家[おおや]さんから。	
\\	寂しいなあ。	
\\	寂[さび]しいなあ。	
\\	でも、ロバートさんはホワイトデーが大変ですよ。	
\\	でも、ロバートさんはホワイトデーが 大変[たいへん]ですよ。	
\\	ええ。男の人は3月14日にお返しをしなきゃいけないんですよ。	お返し: 
\\	ええ。 男[おとこ]の 人[ひと]は 3月[さんがつ]14 日[にち]にお 返[かえ]しをしなきゃいけないんですよ。	
\\	良い漢字の辞書が欲しいです。	"欲しい: 
\\	良[い]い 漢字[かんじ]の 辞書[じしょ]が 欲[ほ]しいです。	
\\	子供の時、ゴジラのオモチャが欲しかったです。	オモチャ: 
\\	子供[こども]の 時[とき]、ゴジラのオモチャが 欲[ほ]しかったです。	
\\	お金はあまり欲しくないです。	
\\	お 金[かね]はあまり 欲[ほ]しくないです。	
\\	ロバートさんは、パソコンが欲しいと言っています。	
\\	ロバートさんは、パソコンが 欲[ほ]しいと 言[い]っています。	
\\	京子さんは、クラシックの
\\	が欲しくないでしょう。	京子: 
\\	京子[きょうこ]さんは、クラシックの 
\\	が 欲[ほ]しくないでしょう。	
\\	智子さんは、英語を習いたがっています。	"習い [習う]: 
\\	たがって: [たがる]: 
\\	""たい
\\	智子[ともこ]さんは、 英語[えいご]を 習[なら]いたがっています。	
\\	トムさんは、友達を欲しがっています。	欲しい 
\\	欲しがって [欲しがる]: 
\\	欲しがっている, 
\\	欲しい, 
\\	を.	トムさんは、 友達[ともだち]を 欲[ほ]しがっています。	
\\	明日は、雨が降るかもしれません。	"かも知れません: 
\\	でしょう, 
\\	明日[あした]は、 雨[あめ]が 降[ふ]るかもしれません。	
\\	田中さんより、鈴木さんの方が背が高いかもしれません。	田中: 
\\	鈴木: 
\\	田中[たなか]さんより、 鈴木[すずき]さんの 方[ほう]が 背[せ]が 高[たか]いかもしれません。	
\\	明日は、天気が良くないかもしれません。	
\\	明日[あした]は、 天気[てんき]が 良[よ]くないかもしれません。	
\\	トムさんは、子供の時、意地悪だったかもしれません。	
\\	トムさんは、 子供[こども]の 時[とき]、 意地悪[いじわる]だったかもしれません。	
\\	私は、その女性に花をあげます。	女性: 
\\	あげます[あげる]: 
\\	私[わたし]は、その 女性[じょせい]に 花[はな]をあげます。	
\\	その女の人は、男の人に時計をあげました。	
\\	その 女[おんな]の 人[ひと]は、 男[おとこ]の 人[ひと]に 時計[とけい]をあげました。	
\\	両親が私に新しい車をくれるかもしれません。	両親: 
\\	くれる: 
\\	両親[りょうしん]が 私[わたし]に 新[あたら]しい 車[くるま]をくれるかもしれません。	
\\	私は姉に古い辞書を貰いました。	貰い: 
\\	私[わたし]は 姉[あね]に 古[ふる]い 辞書[じしょ]を 貰[もら]いました。	
\\	私は姉から、古い辞書を貰いました。	
\\	私[わたし]は 姉[あね]から、 古[ふる]い 辞書[じしょ]を 貰[もら]いました。	
\\	姉が私に、古い辞書をくれました。	
\\	姉[あね]が 私[わたし]に、 古[ふる]い 辞書[じしょ]をくれました。	
\\	トムさんはカナダ人だ。	
\\	です 
\\	だ. 
\\	トムさんはカナダ 人[じん]だ。	
\\	トムさんは、カナダ人かもしれません。	
\\	トムさんは、カナダ 人[じん]かもしれません。	
\\	山下先生は、犬が嫌いだ。	
\\	山下[やました] 先生[せんせい]は、 犬[いぬ]が 嫌[きら]いだ。	
\\	山下先生は、犬が嫌いかもしれません。	
\\	山下[やました] 先生[せんせい]は、 犬[いぬ]が 嫌[きら]いかもしれません。	
\\	もっと勉強したらどうですか?	たらどうですか: 
\\	もっと 勉強[べんきょう]したらどうですか?	
\\	薬を飲んだらどうですか?	
\\	薬[くすり]を 飲[の]んだらどうですか?	
\\	家に来ませんか?	
\\	家[いえ]に 来[き]ませんか?	
\\	私の家には、猫が3匹居ます。	
\\	私[わたし]の 家[うち]には、 猫[ねこ]が 3匹[さんびき] 居[い]ます。	
\\	傘を3本買いました。	
\\	傘[かさ]を 3本[さんぼん] 買[か]いました。	
\\	私の母は、猫を3匹も飼っています。	"飼って [飼う]: 
\\	も: 
\\	私[わたし]の 母[はは]は、 猫[ねこ]を 3匹[さんびき] も飼[か]っています。	
\\	昨日のパーティーには、学生が20人も来ました。	
\\	昨日[きのう]のパーティーには、 学生[がくせい]が20 人[にん]も 来[き]ました。	
\\	私は日本語の辞書を、1冊しか持っていません。	"しか: 
\\	私[わたし]は 日本語[にほんご]の 辞書[じしょ]を、1 冊[いっさつ]しか 持[も]っていません。	
\\	この会社には、パソコンが2台しかありません。	台: 
\\	この 会社[かいしゃ]には、パソコンが2 台[だい]しかありません。	
\\	日本語の先生は優しいなあ。	なあ 
\\	なあ 
\\	日本語[にほんご]の 先生[せんせい]は 優[やさ]しいなあ。	
\\	良い教科書だなあ。	
\\	良[い]い 教科書[きょうかしょ]だなあ。	
\\	お腹が空いたなあ。	空い [空く]: 
\\	お 腹[なか]が 空[す]いたなあ。	
\\	あの人はケチだなあ。	ケチ: 
\\	あの 人[ひと]はケチだなあ。	
\\	日本語が出来ます。	"出来る 
\\	日本語[にほんご]が 出来[でき]ます。	
\\	彼はスキーが出来ません。	
\\	彼[かれ]はスキーが 出来[でき]ません。	
\\	テストであまり出来ませんでした。	
\\	テストであまり 出来[でき]ませんでした。	
\\	晩ご飯が出来ましたよ。	"出来る: 
\\	晩[ばん]ご 飯[はん]が 出来[でき]ましたよ。	
\\	宿題は出来ましたか?	
\\	宿題[しゅくだい]は 出来[でき]ましたか?	
\\	新しい店ができました。	"出来る: 
\\	新[あたら]しい 店[みせ]ができました。	
\\	友達が沢山出来ました。	
\\	友達[ともだち]が 沢山[たくさん] 出来[でき]ました。	
\\	武くん、今度の休み、予定ある?	今度: 
\\	武[たけし]くん、 今度[こんど]の 休[やす]み、 予定[よてい]ある?	
\\	ううん。別に。どうして?	
\\	ううん。 別[べつ]に。どうして?	
\\	美智子さんの長野のうちに行こうと思ってるんだけど、一緒に行かない?	だけど: 
\\	美智子[みちこ]さんの 長野[ながの]のうちに 行[い]こうと 思[おも]ってるんだけど、 一緒[いっしょ]に 行[い]かない?	
\\	うん。美智子さんが、「たけし君も誘って」と言ってたから。	誘って [誘う]: 
\\	うん。 美智子[みちこ]さんが、「たけし 君[くん]も 誘[さそ]って」と 言[い]ってたから。	
\\	じゃあ、行く。電車の時間、調べておくよ。	調べ: 
\\	じゃあ、 行[い]く。 電車[でんしゃ]の 時間[じかん]、 調[しら]べておくよ。	
\\	ありがとう。じゃあ、私、美智子さんに電話しておく。	
\\	ありがとう。じゃあ、 私[わたし]、 美智子[みちこ]さんに 電話[でんわ]しておく。	
\\	早く着いたから、ちょっと観光しない?	着い [着く]: 
\\	観光: 
\\	早[はや]く 着[つ]いたから、ちょっと 観光[かんこう]しない?	
\\	うん。どこに行く?	
\\	うん。どこに 行[い]く?	
\\	善光寺はどう?有名なお寺だよ。	善光寺: 
\\	善光寺[ぜんこうじ]はどう? 有名[ゆうめい]なお 寺[てら]だよ。	
\\	そうだね。昼ご飯は何にする?	
\\	そうだね。 昼[ひる]ご 飯[はん]は 何[なん]にする?	
\\	長野はおそばが美味しいから、そばを食べようよ!	おそば: 
\\	長野[ながの]はおそばが 美味[おい]しいから、そばを 食[た]べようよ!	
\\	すみません、善光寺に行くバスはどれですか?	
\\	すみません、 善光寺[ぜんこうじ]に 行[い]くバスはどれですか?	
\\	善光寺なら、11番のバスですよ。	
\\	善光寺[ぜんこうじ]なら、11 番[ばん]のバスですよ。	
\\	ありがとうございます。この地図、貰っても良いですか?	地図: 
\\	ありがとうございます。この 地図[ちず]、 貰[もら]っても 良[い]いですか?	
\\	ええ、どうぞ。それから、これ、美術館の割引券ですが、良かったらどうぞ。	それから: 
\\	美術館: 
\\	割引券: 
\\	ええ、どうぞ。それから、これ、 美術館[びじゅつかん]の 割引[わりびき] 券[けん]ですが、 良[よ]かったらどうぞ。	
\\	気をつけて。	
\\	気[き]をつけて。		
\\	明日は授業がないから、今晩、どこかに食べに行こう!	行こう: 
\\	ましょう. 
\\	明日[あした]は 授業[じゅぎょう]がないから、 今晩[こんばん]、どこかに 食[た]べに 行[い]こう!	
\\	結婚しようよ!	
\\	結婚[けっこん]しようよ!	
\\	手伝おうか?	
\\	か 
\\	手伝[てつだ]おうか?	
\\	友達が面白いと言っていたから、この映画を見ようか?	
\\	友達[ともだち]が 面白[おもしろ]いと 言[い]っていたから、この 映画[えいが]を 見[み]ようか?	
\\	今度、いつ会おうか?	
\\	今度[こんど]、いつ 会[あ]おうか?	
\\	毎日3時間、日本語を勉強しようと思っています。	
\\	と思っています, 
\\	と思います 
\\	毎日[まいにち]3 時間[じかん]、 日本語[にほんご]を 勉強[べんきょう]しようと 思[おも]っています。	
\\	1万円あげましょう。何に使いますか?	
\\	万[まん] 円[えん]あげましょう。 何[なに]に 使[つか]いますか?	
\\	漢字の辞書を買おうと思います。	
\\	漢字[かんじ]の 辞書[じしょ]を 買[か]おうと 思[おも]います。	
\\	両親から1万円貰ったんですか?何に使うんですか?	
\\	両親[りょうしん]から1 万[まん] 円[えん] 貰[もら]ったんですか? 何[なに]に 使[つか]うんですか?	
\\	漢字の辞書を買おうと思っています。	
\\	漢字[かんじ]の 辞書[じしょ]を 買[か]おうと 思[おも]っています。	
\\	日本の会社で働こうと思います。	
\\	日本[にほん]の 会社[かいしゃ]で 働[はたら]こうと 思[おも]います。	
\\	日本の会社で働くと思います。	
\\	日本[にほん]の 会社[かいしゃ]で 働[はたら]くと 思[おも]います。	
\\	明日試験があるので、今晩勉強しておきます。	おき [おく]: 
\\	て 
\\	おく 
\\	明日[あした] 試験[しけん]があるので、 今晩[こんばん] 勉強[べんきょう]しておきます。	
\\	友達が来るから、部屋を掃除しておかなきゃいけません。	
\\	友達[ともだち]が 来[く]るから、 部屋[へや]を 掃除[そうじ]しておかなきゃいけません。	
\\	ホテルを予約しとくね!	ておく 
\\	とく 
\\	ホテルを 予約[よやく]しとくね!	
\\	これは、去年の誕生日に、彼女がくれた本です。	誕生: 
\\	これは、 去年[きょねん]の 誕生[たんじょう] 日[び]に、 彼女[かのじょ]がくれた 本[ほん]です。	
\\	父が、美里が書いた本をくれました。	美里: 
\\	父[ちち]が、 美里[みさと]が 書[か]いた 本[ほん]をくれました。	
\\	私が一番感動した映画は「生きる」です。	感動: 
\\	生きる: 
\\	私[わたし]が 一番[いちばん] 感動[かんどう]した 映画[えいが]は
\\	生[い]きる」です。	
\\	私は今度の週末に、韓国に行く予定です。	予定です: 
\\	予定です 
\\	私[わたし]は 今度[こんど]の 週末[しゅうまつ]に、 韓国[かんこく]に 行[い]く 予定[よてい]です。	
\\	私の兄は、9月に結婚する予定です。	
\\	私[わたし]の 兄[あに]は、 9月[くがつ]に 結婚[けっこん]する 予定[よてい]です。	
\\	私は、学校に来ない予定です。	
\\	予定です 
\\	私[わたし]は、 学校[がっこう]に 来[こ]ない 予定[よてい]です。	
\\	部屋の窓から、海が見えます。	
\\	見える 
\\	部屋[へや]の 窓[まど]から、 海[うみ]が 見[み]えます。	
\\	どこであの映画が見られますか?	見られる 
\\	どこであの 映画[えいが]が 見[み]られますか?	
\\	今朝、鳥の声が聞こえました。	鳥: 
\\	声: 
\\	聞こえる 
\\	今朝[けさ]、 鳥[とり]の 声[こえ]が 聞[き]こえました。	
\\	インターネットで日本のラジオ番組が聞けます。	番組: 
\\	聞ける: 
\\	インターネットで 日本[にほん]のラジオ 番組[ばんぐみ]が 聞[き]けます。	
\\	失礼します。先生、今日授業に来られなくてすみませんでした。	失礼: 
\\	失礼[しつれい]します。 先生[せんせい]、 今日[きょう] 授業[じゅぎょう]に 来[こ]られなくてすみませんでした。	
\\	実は、朝寝坊して、電車に乗り遅れたんです。すみません。	実: 
\\	朝寝坊: 
\\	乗り遅れる: 
\\	実[じつ]は、 朝寝坊[あさねぼう]して、 電車[でんしゃ]に 乗り遅[のりおく]れたんです。すみません。	
\\	もう3回目ですよ。目覚まし時計を買ったらどうですか?	目覚まし時計: 
\\	もう3 回[かい] 目[め]ですよ。 目覚[めざ]まし 時計[どけい]を 買[か]ったらどうですか?	
\\	はい。あの、先生、宿題を明日まで待っていただけませんか?	
\\	はい。あの、 先生[せんせい]、 宿題[しゅくだい]を 明日[あした]まで 待[ま]っていただけませんか?	
\\	困りましたね。見つかると良いですね。	困る: 
\\	困[こま]りましたね。 見[み]つかると 良[い]いですね。	
\\	すみません。ファイルを亡くしたんですが
\\	亡くす: 
\\	すみません。ファイルを 亡[な]くしたんですが
\\	どんなファイルですか?	どんな: 
\\	どんなファイルですか?	
\\	青くてこのぐらいの大きさです。	このぐらい: 
\\	青[あお]くてこのぐらいの 大[おお]きさです。	
\\	ええと
\\	ちょっと待って下さい。電話して聞いてみます。	ええと: 
\\	ええと
\\	ちょっと 待[ま]って 下[くだ]さい。 電話[でんわ]して 聞[き]いてみます。	
\\	ジョンさん、ファイルはありましたか?	ある: 
\\	ジョンさん、ファイルはありましたか?	
\\	はい、駅員さんが探してくれたんです。	駅員: 
\\	探す: 
\\	はい、 駅員[えきいん]さんが 探[さが]してくれたんです。	
\\	これ、宿題です。遅くなってすみませんでした。	
\\	これ、 宿題[しゅくだい]です。 遅[おそ]くなってすみませんでした。	
\\	いいえ。よく出来ていますね。	
\\	いいえ。よく 出来[でき]ていますね。	
\\	ええ。駅員さんに手伝ってもらいましたから。	
\\	ええ。 駅員[えきいん]さんに 手伝[てつだ]ってもらいましたから。	
\\	私は妹に、お金を貸してあげました。	"貸して [貸す]: 
\\	あげる: 
\\	て-
\\	あげる 
\\	あげる 
\\	私[わたし]は 妹[いもうと]に、お 金[かね]を 貸[か]してあげました。	
\\	私は妹に、お金を貸しました。	
\\	私[わたし]は 妹[いもうと]に、お 金[かね]を 貸[か]しました。	
\\	京子さんはトムさんを、駅に連れて行ってあげました。	連れて [連れる]: 
\\	京子[きょうこ]さんはトムさんを、 駅[えき]に 連[つ]れて 行[い]ってあげました。	
\\	京子さんはトムさんを、駅に連れて行きました。	
\\	京子[きょうこ]さんはトムさんを、 駅[えき]に 連[つ]れて 行[い]きました。	
\\	友達が宿題を手伝ってくれます。	手伝って [手伝う]: 
\\	くれる 
\\	友達[ともだち]が 宿題[しゅくだい]を 手伝[てつだ]ってくれます。	
\\	親戚が、クッキーを送ってくれました。	親戚: 
\\	送って [送る]: 
\\	親戚[しんせき]が、クッキーを 送[おく]ってくれました。	
\\	私は、友達に宿題を手伝ってもらいました。	
\\	て-
\\	もらう 
\\	私[わたし]は、 友達[ともだち]に 宿題[しゅくだい]を 手伝[てつだ]ってもらいました。	
\\	友達が、宿題を手伝ってくれました。	
\\	友達[ともだち]が、 宿題[しゅくだい]を 手伝[てつだ]ってくれました。	
\\	ちょっと手伝っていただけませんか?	て-
\\	いただけませんか: 
\\	ください 
\\	ちょっと 手伝[てつだ]っていただけませんか?	
\\	ちょっと待ってくれませんか?	て-
\\	くれませんか: 
\\	ください. 
\\	ちょっと 待[ま]ってくれませんか?	
\\	それ取ってくれない?	
\\	て-
\\	くれない 
\\	取って [取る]: 
\\	それ 取[と]ってくれない?	
\\	それ取って!	
\\	それ 取[と]って!	
\\	良いアルバイトが見つかると良いですね!	見つかる: 
\\	と良いですね 
\\	と良いね 
\\	良[い]いアルバイトが 見[み]つかると 良[い]いですね!	
\\	雨が降らないと良いね。	
\\	雨[あめ]が 降[ふ]らないと 良[い]いね。	
\\	試験が易しいと良いんですが
\\	と良いんですが 
\\	と良いんだけど 
\\	易しい: 
\\	試験[しけん]が 易[やさ]しいと 良[い]いんですが
\\	8時の電車に乗れると良いんだけど
\\	時[じ]の 電車[でんしゃ]に 乗[の]れると 良[い]いんだけど
\\	大学に行けると良いんですが
\\	といい 
\\	大学[だいがく]に 行[い]けると 良[い]いんですが
\\	チベットに行く時、ビザを取ります。	時 
\\	チベットに 行[い]く 時[とき]、ビザを 取[と]ります。	
\\	チベットに行く時、ビザを取りました。	
\\	チベットに 行[い]く 時[とき]、ビザを 取[と]りました。	
\\	寝る時、コンタクトを取ります。	取る: 
\\	寝[ね]る 時[とき]、コンタクトを 取[と]ります。	
\\	出掛ける時、ドアに鍵をかけました。	鍵をかける: 
\\	出掛[でか]ける 時[とき]、ドアに 鍵[かぎ]をかけました。	
\\	寂しい時、友達に電話します。	
\\	寂[さび]しい 時[とき]、 友達[ともだち]に 電話[でんわ]します。	
\\	寒い時、頭が痛くなります。	
\\	寒[さむ]い 時[とき]、 頭[あたま]が 痛[いた]くなります。	
\\	元気な時、公園を走ります。	公園: 
\\	走る: 
\\	元気[げんき]な 時[とき]、 公園[こうえん]を 走[はし]ります。	
\\	犬が病気の時、病院に連れて行きました。	病気: 
\\	犬[いぬ]が 病気[びょうき]の 時[とき]、 病院[びょういん]に 連[つ]れて 行[い]きました。	
\\	中国に行った時、ウーロン茶を買います。	
\\	中国[ちゅうごく]に 行[い]った 時[とき]、 ウーロン茶[うーろんちゃ]を 買[か]います。	
\\	中国に行った時、ウーロン茶を買いました。	
\\	中国[ちゅうごく]に 行[い]った 時[とき]、 ウーロン茶[うーろんちゃ]を 買[か]いました。	
\\	疲れた時、ゆっくりお風呂に入ります。	疲れた: 
\\	ゆっくり: 
\\	疲[つか]れた 時[とき]、ゆっくりお 風呂[ふろ]に 入[はい]ります。	
\\	宿題を忘れた時、泣いてしまいました。	泣く: 
\\	てしまいました:	宿題[しゅくだい]を 忘[わす]れた 時[とき]、 泣[な]いてしまいました。	
\\	汚い言葉を使って、すみませんでした。	汚い: 
\\	言葉: 
\\	汚[きたな]い 言葉[ことば]を 使[つか]って、すみませんでした。	
\\	デートの約束を忘れて、ごめん!	約束: 
\\	デートの 約束[やくそく]を 忘[わす]れて、ごめん!	
\\	宿題を持ってこなくて、すみませんでした。	
\\	なくて, 
\\	宿題[しゅくだい]を 持[も]ってこなくて、すみませんでした。	
\\	もっと早く言わなくて、ごめん。	言う: 
\\	もっと 早[はや]く 言[い]わなくて、ごめん。	
\\	佐藤さんが、昼ご飯をごちそうしてくれました。	佐藤: 
\\	佐藤[さとう]さんが、 昼[ひる]ご 飯[はん]をごちそうしてくれました。	
\\	週末友達を呼んで、晩ご飯をごちそうしました。	呼ぶ: 
\\	週末[しゅうまつ] 友達[ともだち]を 呼[よ]んで、 晩[ばん]ご 飯[はん]をごちそうしました。	
\\	練習しなきゃいけないんですが....	
\\	なきゃ行けません: 
\\	練習[れんしゅう]しなきゃいけないんですが....		
\\	武さん、久しぶりですね。	久しぶり: 
\\	武[たけし]さん、 久[ひさ]しぶりですね。	
\\	旅行会社に就職したそうですね。	旅行会社: 
\\	就職: 
\\	旅行[りょこう] 会社[かいしゃ]に 就職[しゅうしょく]したそうですね。	
\\	おめでとうございます。	
\\	おめでとうございます。		
\\	もう仕事に慣れましたか?	慣れる: 
\\	もう 仕事[しごと]に 慣[な]れましたか?	
\\	ええ。でも学生の時に比べてすごく忙しくなりました。	比べる: 
\\	ええ。でも 学生[がくせい]の 時[とき]に 比[くら]べてすごく 忙[いそが]しくなりました。	
\\	自分の時間が全然ないんです。	自分: 
\\	自分[じぶん]の 時間[じかん]が 全然[ぜんぜん]ないんです。	
\\	大変ですね。私の友達の会社は休みが多くて、残業をしなくても良いそうですよ。	多く: 
\\	残業: 
\\	大変[たいへん]ですね。 私[わたし]の 友達[ともだち]の 会社[かいしゃ]は 休[やす]みが 多[おお]くて、 残業[ざんぎょう]をしなくても 良[い]いそうですよ。	
\\	羨ましいですよ。	羨ましい: 
\\	羨[うらや]ましいですよ。		
\\	僕の会社は休みも少ないし、給料も安いし、最低です。	少ない: 
\\	給料: 
\\	最低: 
\\	僕[ぼく]の 会社[かいしゃ]は 休[やす]みも 少[すく]ないし、 給料[きゅうりょう]も 安[やす]いし、 最低[さいてい]です。	
\\	会社に入る前にどうしてもっと調べなかったんですか	入る: 
\\	調べる: 
\\	会社[かいしゃ]に 入[はい]る 前[まえ]にどうしてもっと 調[しら]べなかったんですか?	
\\	旅行会社に入ったら、旅行ができると思ったんです。	できる: 
\\	旅行[りょこう] 会社[がいしゃ]に 入[はい]ったら、 旅行[りょこう]ができると 思[おも]ったんです。	
\\	今朝、駅で武さんに会ったよ。	
\\	今朝[けさ]、 駅[えき] で武[たけし]さんに 会[あ]ったよ。	
\\	武さんが卒業してから全然会ってないけど、元気だった?	卒業: 
\\	武[たけし]さんが 卒業[そつぎょう]してから 全然[ぜんぜん] 会[あ]ってないけど、 元気[げんき]だった?	
\\	随分疲れているみたい。	随分: 
\\	疲れる: 
\\	随分[ずいぶん] 疲[つか]れているみたい。	
\\	毎晩4、5時間しか寝ていないそうだよ。	
\\	毎晩[まいばん]4、 
\\	時間[じかん]しか 寝[ね]ていないそうだよ。	
\\	やっぱりサラリーマンは大変だなあ。	やっぱり: 
\\	サラリーマン: 
\\	やっぱりサラリーマンは 大変[たいへん]だなあ。	
\\	それに、忙し過ぎてメアリーとデートする時間もないって。	それに: 
\\	それに、 忙[いそが]し 過[す]ぎてメアリーとデートする 時間[じかん]もないって。	
\\	そうか。僕だったら、仕事より彼女を選ぶけど。	そうか: 
\\	選ぶ: 
\\	けど: 
\\	そうか。 僕[ぼく]だったら、 仕事[しごと]より 彼女[かのじょ]を 選[えら]ぶけど。	
\\	あの2人、大丈夫かなあ。	
\\	あの 
\\	人[ふたり]、 大丈夫[だいじょうぶ]かなあ。	
\\	日本語の授業は楽しいそうです。	"そうです 
\\	日本語[にほんご]の 授業[じゅぎょう]は 楽[たの]しいそうです。	
\\	先生はとても親切だそうです。	
\\	先生[せんせい]はとても 親切[しんせつ]だそうです。	
\\	その日は授業がなかったそうです。	
\\	その 日[ひ]は 授業[じゅぎょう]がなかったそうです。	
\\	天気予報によると、台風が来るそうです。	
\\	によると, 
\\	台風: 
\\	天気[てんき] 予報[よほう]によると、 台風[たいふう]が 来[く]るそうです。	
\\	「今日は忙しいです。明日、試験があるんです。」	
\\	今日[きょう]は 忙[いそが]しいです。 明日[あした]、 試験[しけん]があるんです。」	
\\	メアリーさん、今日は忙しいって。明日、試験があるんだって。	
\\	って 
\\	そうです, 
\\	メアリーさん、 今日[きょう]は 忙[いそが]しいって。 明日[あした]、 試験[しけん]があるんだって。	
\\	明さんは何て言ってた?	
\\	って 
\\	と 
\\	言う.	明[あきら]さんは 何[なん]て 言[い]ってた?	
\\	チョコレートを食べ過ぎたって言ってた。	
\\	チョコレートを 食[た]べ 過[す]ぎたって 言[い]ってた。	
\\	日本に行ったら、着物を買います。	たら 
\\	日本[にほん]に 行[い]ったら、 着物[きもの]を 買[か]います。	
\\	天気が良かったら、散歩に行きます。	散歩: 
\\	天気[てんき]が 良[よ]かったら、 散歩[さんぽ]に 行[い]きます。	
\\	山下先生に会ったら、その事を聞こうと思います。	事: 
\\	山下[やました] 先生[せんせい]に 会[あ]ったら、その 事[こと]を 聞[き]こうと 思[おも]います。	
\\	日本人だったら、この言葉を知っているでしょう。	
\\	日本人[にほじん]だったら、この 言葉[ことば]を 知[し]っているでしょう。	
\\	宝くじに当たったら、アムネスティにお金を送ります。	宝くじ: 
\\	当たる: 
\\	宝[たから]くじに 当[あ]たったら、アムネスティにお 金[かね]を 送[おく]ります。	
\\	今晩、家に帰ったら、電話します。	
\\	たら 
\\	たら 
\\	今晩[こんばん]、 家[うち]に 帰[かえ]ったら、 電話[でんわ]します。	
\\	宿題が終わったら、遊びに行きましょう。	
\\	宿題[しゅくだい]が 終[お]わったら、 遊[あそ]びに 行[い]きましょう。	
\\	私が猫だったら、一日中寝ているでしょう。	一日中: 
\\	たら 
\\	私[わたし]が 猫[ねこ]だったら、一 日[にち] 中[じゅう] 寝[ね]ているでしょう。	
\\	お金があったら、車を買うんですけど。	
\\	お 金[かね]があったら、 車[くるま]を 買[か]うんですけど。	
\\	靴を脱がなくても良いです。	脱ぐ: 
\\	くても良いです
\\	靴[くつ]を 脱[ぬ]がなくても 良[い]いです。	
\\	プレゼントは高くなくても良いです。	
\\	プレゼントは 高[たか]くなくても 良[い]いです。	
\\	あの人はゴリラみたいです。	
\\	あの 人[ひと]はゴリラみたいです。	
\\	あの人はお腹が空いているみたいです。	"みたいです 
\\	あの 人[ひと]はお 腹[なか]が 空[す]いているみたいです。	
\\	雨が降ったみたいです。	
\\	雨[あめ]が 降[ふ]ったみたいです。	
\\	あの人は昨日の夜寝なかったみたいです。	
\\	あの 人[ひと]は 昨日[きのう]の 夜[よる] 寝[ね]なかったみたいです。	
\\	先生は明日学校に来ないみたいです。	
\\	先生[せんせい]は 明日[あした] 学校[がっこう]に 来[こ]ないみたいです。	
\\	国に帰る前に、もう一度東京に行きます。	
\\	前に 
\\	もう一度: 
\\	国[くに]に 帰[かえ]る 前[まえ]に、もう 一度[いちど] 東京[とうきょう]に 行[い]きます。	
\\	日本に来る前に、一学期日本語を勉強しました。	一学期: 
\\	日本[にほん]に 来[く]る 前[まえ]に、一 学期[がっき] 日本語[にほんご]を 勉強[べんきょう]しました。	
\\	勉強してから、友達に手紙を書きました。	
\\	から.	勉強[べんきょう]してから、 友達[ともだち]に 手紙[てがみ]を 書[か]きました。	
\\	健さんが来てから、食べましょう。	
\\	健[けん]さんが 来[き]てから、 食[た]べましょう。	
\\	ジョン、今日は森田君、風邪で来られないそうだ。	森田: 
\\	ジョン、 今日[きょう]は 森田[もりた] 君[くん]、 風邪[かぜ]で 来[こ]られないそうだ。	
\\	はい。頑張ります!	
\\	はい。 頑張[がんば]ります!	
\\	まず、冷蔵庫に野菜が入っているから、出しておいて。	まず: 
\\	冷蔵庫: 
\\	入る: 
\\	出す: 
\\	おく: 
\\	まず、 冷蔵庫[れいぞうこ]に 野菜[やさい]が 入[はい]っているから、 出[だ]しておいて。	
\\	いいえ、ついていません。つけましょうか?	
\\	いいえ、ついていません。つけましょうか?	
\\	うん。そこのスイッチを押すとつくよ。	押す: 
\\	うん。そこのスイッチを 押[お]すとつくよ。	
\\	すみません。醤油を落としちゃったんです。ごめんなさい。	醤油: 
\\	落とす: 
\\	すみません。 醤油[しょうゆ]を 落[お]としちゃったんです。ごめんなさい。	
\\	いいえ、大丈夫です。あっ、スカートが汚れてしまいましたね。	汚れる: 
\\	いいえ、 大丈夫[だいじょうぶ]です。あっ、スカートが 汚[よご]れてしまいましたね。	
\\	本当だ。どうしよう。	どうしよう: 
\\	本当[ほんとう]だ。どうしよう。	
\\	今すぐ、タオルを持ってきます。	今すぐ: 
\\	今[いま]すぐ、タオルを 持[も]ってきます。	
\\	今日は、ジョンのおかげで、助かったよ。	おかげで: 
\\	助かる: 
\\	今日[きょう]は、ジョンのおかげで、 助[たす]かったよ。	
\\	いいえ。でも、本当に忙しかったですね。	
\\	いいえ。でも、 本当[ほんとう]に 忙[いそが]しかったですね。	
\\	明日は学校があるんだろう?	
\\	(ん)だろう 
\\	(ん)でしょう	明日[あした]は 学校[がっこう]があるんだろう?	
\\	ええ。時々、遅刻しちゃうんですよ。	遅刻: 
\\	ええ。 時々[ときどき]、 遅刻[ちこく]しちゃうんですよ。	
\\	僕も学生の時は、よく授業をサボったよ。	サボる: 
\\	僕[ぼく]も 学生[がくせい]の 時[とき]は、よく 授業[じゅぎょう]を サボ[さぼ]ったよ。	
\\	じゃあ、お先に失礼します。お疲れ様でした。	お先に失礼します: 
\\	お疲れ様: 
\\	じゃあ、お 先[さき]に 失礼[しつれい]します。お 疲れ様[つかれさま]でした。	
\\	武さんが、電気をつけました。	電気: 
\\	つける: 
\\	武[たけし]さんが、 電気[でんき]をつけました。	
\\	電気がつきました。	つく: 
\\	電気[でんき]がつきました。	
\\	武さんが、お湯を沸かしました。	お湯: 
\\	沸かす: 
\\	武[たけし]さんが、お 湯[ゆ]を 沸[わ]かしました。	
\\	お湯が沸きました。	沸く: 
\\	お 湯[ゆ]が 沸[わ]きました。	
\\	スーさんは今、電話でお母さんと話しています。	
\\	スーさんは 今[いま]、 電話[でんわ]でお 母[かあ]さんと 話[はな]しています。	
\\	山下先生は、結婚しています。	
\\	山下[やました] 先生[せんせい]は、 結婚[けっこん]しています。	
\\	ロバートさんは、窓を開けています。	開ける: 
\\	ロバートさんは、 窓[まど]を 開[あ]けています。	
\\	ドアが開いています。	
\\	ドアが 開[あ]いています。	
\\	智子さんは、電気を消しています。	消す: 
\\	智子[ともこ]さんは、 電気[でんき]を 消[け]しています。	
\\	テレビは消えています。	
\\	テレビは 消[き]えています。	
\\	ゴジラが町を壊しています。	壊す: 
\\	ゴジラが 町[まち]を 壊[こわ]しています。	
\\	このパソコンは、壊れています。	
\\	このパソコンは、 壊[こわ]れています。	
\\	本を読んでしまいました。	
\\	てしまい 
\\	本[ほん]を 読[よ]んでしまいました。	
\\	電車の中に、鞄を忘れてしまいました。	
\\	てしまい 
\\	電車[でんしゃ]の 中[なか]に、 鞄[かばん]を 忘[わす]れてしまいました。	
\\	宿題を忘れたので、先生は怒ってしまいました。	怒る: 
\\	宿題[しゅくだい]を 忘[わす]れたので、 先生[せんせい]は 怒[おこ]ってしまいました。	
\\	宿題を失くしちゃった!	てしまう 
\\	でします 
\\	ちゃう 
\\	じゃう 
\\	失くす: 
\\	宿題[しゅくだい]を 失[な]くしちゃった!	
\\	私は、その人と話すと元気になる。	
\\	と 
\\	私[わたし]は、その 人[ひと]と 話[はな]すと 元気[げんき]になる。	
\\	道が混んでいると、時間がかかる。	混む: 
\\	時間がかかる: 
\\	道[みち]が 混[こ]んでいると、 時間[じかん]がかかる。	
\\	メアリーさんが国に帰ると、寂しくなります。	
\\	と 
\\	メアリーさんが 国[くに]に 帰[かえ]ると、 寂[さび]しくなります。	
\\	私は子供の時、冬になると風邪を引きました。	引き: 
\\	私[わたし]は 子供[こども]の 時[とき]、 冬[ふゆ]になると 風邪[かぜ]を 引[ひ]きました。	
\\	秋になると、木が赤くなります。	
\\	秋[あき]になると、 木[き]が 赤[あか]くなります。	
\\	夜になると、町が静かになります。	
\\	夜[よる]になると、 町[まち]が 静[しず]かになります。	
\\	私はいつも、音楽を聴きながら日本語を勉強します。	
\\	ながら 
\\	ながら 
\\	ながら, 
\\	私[わたし]はいつも、 音楽[おんがく]を 聴[き]きながら 日本語[にほんご]を 勉強[べんきょう]します。	
\\	武さんは、歌を歌いながら洗濯をしています。	洗濯: 
\\	武[たけし]さんは、 歌[うた]を 歌[うた]いながら 洗濯[せんたく]をしています。	
\\	アルバイトをしながら学校に行くのは大変です。	
\\	アルバイトをしながら 学校[がっこう]に 行[い]くのは 大変[たいへん]です。	
\\	あの時、「愛している」と言えば良かったです。	ば良かったです 
\\	愛す: 
\\	あの 時[とき]、
\\	愛[あい]している」と 言[い]えば 良[よ]かったです。	
\\	彼女と別れなければ良かったです。	
\\	彼女[かのじょ]と 別[わか]れなければ 良[よ]かったです。	
\\	夕方になると忙しくなるから、頼むよ。	夕方: 
\\	頼む: 
\\	夕方[ゆうがた]になると 忙[いそが]しくなるから、 頼[たの]むよ。	
\\	それから、外の電気はついている?	外: 
\\	それから、 外[そと]の 電気[でんき]はついている?	
\\	アルバイトをしながら学校に行くのは、大変だね。	
\\	アルバイトをしながら 学校[がっこう]に 行[い]くのは、 大変[たいへん]だね。	
\\	もっと勉強すれば良かったなあ。	
\\	もっと 勉強[べんきょう]すれば 良[よ]かったなあ。	
\\	後は僕が片付けるから。	後: 
\\	片付ける: 
\\	後[あと]は 僕[ぼく]が 片付[かたづ]けるから。	
\\	友達が書いてくれた地図のおかげで道に迷わなかった。	おかげ (で) 
\\	迷う: 
\\	友達[ともだち]が 書[か]いてくれた 地図[ちず]のおかげで 道[みち]に 迷[まよ]わなかった。	
\\	ええ、おかげさまで、だいぶ慣れました。	おかげさまで 
\\	だいぶ: 
\\	慣れる: 
\\	ええ、おかげさまで、だいぶ 慣[な]れました。	
\end{CJK}
\end{document}