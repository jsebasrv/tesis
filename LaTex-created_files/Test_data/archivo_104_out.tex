\documentclass[8pt]{extreport} 
\usepackage{hyperref}
\usepackage{CJKutf8}
\begin{document}
\begin{CJK}{UTF8}{min}
\\	へ 
\\	の 
\\	んだ 
\\	ん	
\\	か 
\\	昨日何を食べたか忘れた。 「きのうなにをたべたかわすれた」 
\\	どうか	
\\	私の部屋は、きれいで、静かで、とても好き。
\\	で 
\\	から 
\\	うん、時間がなかったから。 
\\	ので 
\\	んで	
\\	(ので/んで) 
\\	ちょっと忙しいので、そろそろ失礼します。
\\	のに	
\\	毎日運動したのに、全然痩せなかった。   
\\	まいにちうんどうしたのに、ぜんぜんやせなかった 
\\	が 
\\	けど	
\\	デパートに行きましたが、何も欲しくなかったです。 
\\	し 
\\	先生だし、年上だし・・・。 
\\	〜た/だり。。。た/だりする	
\\	映画を見たり、本を読んだり、昼寝したりする。 
\\	〜ている	
\\	行っている 
\\	来ている)	"行っている 
\\	来ている 
\\	もう、家に帰っている。
\\	先に行っているよ。 
\\	~てある	
\\	準備はどうですか。 
\\	準備は、もうしてあるよ。 
\\	〜ておく	
\\	晩ご飯を作っておく。 
\\	(いく、くる) 
\\	鉛筆を学校へ持っていく? 
\\	お父さんは、早く帰ってきました。 
\\	られる 
\\	/る	
\\	見る→ 見られる
\\	る 
\\	られる, 
\\	見られる
\\	う 
\\	/る, 
\\	遊ぶー
\\	遊べる
\\	漢字は書けますか?
\\	富士山が登れた。
\\	重い荷物は持てます。 
\\	を 
\\	なる 
\\	に 
\\	彼の日本語が上手になった。 
\\	にする (に 
\\	私は、ハンバーガーとサラダにします。
\\	なる 
\\	去年から背が高くなったね。
\\	ことになる, ことにする, ようになる, ようにする	
\\	こと 
\\	よう. 
\\	海外に行くことになった。
\\	と 
\\	と
\\	だと
\\	ボールを落すと落ちる。
\\	なら(ば)	
\\	「なら」
\\	なら
\\	なら 
\\	みんなが行くなら私も行く。
\\	ならば 
\\	なら.	
\\	ば, ければ	
\\	「ば」
\\	食べる 
\\	食べれ→ 食べれば 
\\	「ない」: 
\\	「い」
\\	「ければ」.
\\	おかしい 
\\	おかしければ 
\\	友達に会えれば、買い物に行きます。
\\	たら(ば)	
\\	ら 
\\	待つ 
\\	待った→ 待ったら 
\\	暇だったら、遊びに行くよ。
\\	そのようにしてもいいのですか。
\\	から	ぎんこうはなんじからですか。 
\\	ぎんこうはごぜんくじからです。 
\\	もし 	
\\	もし時間がないなら、明⽇でもいいよ。 
\\	「だめ」, 「いけない」, 
\\	「ならない」	
\\	は
\\	だめ/いけない/ならない 
\\	ここに入ってはいけません。 
\\	夜、遅くまで電話してはならない。 
\\	「だめ」, 「いけない」, 
\\	「ならない」
\\	「は」 
\\	だめ/いけない/ならない 
\\	「と」 
\\	だめ/いけない/ならない 
\\	ば 
\\	だめ/いけない/ならない 
\\	"毎日学校に行か
\\	なくてはなりません
\\	は 
\\	だめ/いけない/ならない 
\\	"宿題を
\\	しないといけない
\\	と 
\\	だめ/いけない/ならない 
\\	宿題をしなければだめだった。	
\\	ば 
\\	だめ/いけない/ならない 
\\	なくちゃ	
\\	だめ/いけない/ならない  
\\	勉強しなくちゃ。
\\	なきゃ	
\\	だめ/いけない/ならない 
\\	ご飯を食べなきゃ。 
\\	と  
\\	「と」 
\\	だめ/いけない/ならない 
\\	学校に行かないと。 
\\	たは 
\\	ちゃ	
\\	ここに入っちゃだめだよ。
\\	では 
\\	じゃ	
\\	では, 
\\	死んじゃだめだよ!
\\	~て 
\\	もいい	
\\	全部食べてもいいよ。
\\	全部食べなくてもいいよ。
\\	~ていい, ~でいい	
\\	てもいい 
\\	でもいい 
\\	もう帰っていい? 
\\	たい	
\\	たい 
\\	温泉に行きたい。
\\	ケーキ、食べたくないの? 
\\	たい, たくない, たかった, たくなかった	
\\	(たい) 
\\	食べたくなかったけど食べたくなった。	
\\	欲しい「ほしい」	
\\	好き「すき」
\\	欲しい 
\\	全部食べてほしいんだけど・・・。 
\\	部屋をきれいにしてほしいのよ。 
\\	よう 
\\	/う 
\\	る, 
\\	よう
\\	う 
\\	今日は何をしようか? 
\\	テーマパークに行こう!
\\	ましょう	
\\	ましょう 
\\	今日は何をしましょうか?
\\	テーマパークに行きましょう! 
\\	と 
\\	アリスが、「寒い」と言った。
\\	「今⽇は授業がない」と先⽣から聞いたんだけど。
\\	と 
\\	思う「おもう」/考える「かんがえる」	
\\	先生から今日は授業がないと聞いたんだけど。
\\	これは、日本語で何と言いますか。
\\	って	
\\	と 
\\	もうお金がないって。 
\\	今、時間がないって聞いたんだけど、本当? 
\\	って 
\\	は 
\\	明日って、雨が降るんだって。
\\	アリスって、すごくいい人でしょ? 
\\	という	
\\	この魚は、鯛といいます。 
\\	私はアリスといいます。 
\\	主人公が犯人だったというのが一番面白かった  
\\	ルミネ
\\	というデパートはどこにあるか、知っていますか?	"(という 
\\	こう/そう/ああ/どう 
\\	いう	
\\	そういう人と一緒に仕事をするのは、嫌だよね。 
\\	大学に行かないって、どういう意味なの?
\\	"ああいう人と結婚
\\	できたら
\\	、幸せになれると思います。
\\	というか	
\\	お酒は好きというか、ないと生きていけない。 
\\	というか、もう帰らないとだめですけど。 
\\	いうこと	
\\	ということは、みきちゃんは、今彼氏がいないということ? 
\\	そう。そういうこと。
\\	って 
\\	て 
\\	いう	
\\	って, 
\\	いう
\\	来年留学するというのは、智子のこと? ー
\\	来年留学するって智子のこと? 
\\	だって	
\\	だって、時間がないからできないよ。 
\\	だって、みんな行くって。私も行かないと。 
\\	て 
\\	って 
\\	てことは、みきちゃんは、今彼氏がいないてこと? 
\\	ていうか、もう帰らないとだめですけど。
\\	ゆう	
\\	いう 
\\	てゆうか、もう帰らないとだめですけど。 
\\	そうゆうことじゃないって! 
\\	~て 
\\	みる	
\\	切る 
\\	切ってみる、切ってみた、切ってみない、切ってみなかった 
\\	お好み焼きを初めて食べてみたけど、とてもおいしかった! 
\\	お酒を飲んでみましたが、すごく眠くなりました。
\\	よう, 
\\	/う 
\\	とする	
\\	とする. 
\\	見る ー
\\	見ようとする, 行く 
\\	行こうとする 
\\	早く寝ようとしたけど、結局は徹夜した。
\\	無理矢理に部屋に入ろうとしている
\\	(よう, 
\\	/う 
\\	思う	
\\	勉強をなるべく避けようと思った。
\\	(よう, 
\\	/う 
\\	決める	
\\	毎日ジムに行こうと決めた。
\\	あげる	
\\	これは先生にあげる。 
\\	あげる	
\\	車を買ってあげるよ。
\\	やる	
\\	あげる, 
\\	犬に餌をやった? 
\\	くれる	
\\	友達が私にプレゼントをくれた。 
\\	車を買ってくれるの? 
\\	もらう	
\\	私が友達にプレゼントをもらった。
\\	これは友達に買ってもらった。
\\	宿題
\\	をチェック
\\	してもらいたかったけど、時間がなくて無理だった。	
\\	"その時計は私から
\\	もらった
\\	のよ。
\\	千円を貸してくれる?	
\\	千円を貸してもらえる?	
\\	くれる, もらう	
\\	ちょっと静かにしてくれない? 
\\	漢字で書 いてもらえませんか。
\\	ください	
\\	くれる, 
\\	それをください。
\\	それをくれる? 
\\	ください	
\\	漢字で書いてください。
\\	ゆっくり話してください。 
\\	ください	
\\	落書きを書かないでください。
\\	「ください」
\\	日本語で話して。
\\	遠い所に行かないで。
\\	ちょうだい	
\\	ください 
\\	スプーンをちょうだい。
\\	ここに名前を書いてちょうだい。 
\\	~なさい	
\\	なさい, 
\\	飲む→ 飲み→ 飲みなさい, する→ し→ しなさい 
\\	よく聞きなさい!
\\	「さい」
\\	まだいっぱいあるから、たくさん
\\	食べな
\\	~なさい
\\	さい 
\\	「る」
\\	[ろ」 (たべるー
\\	食べろ)
\\	(いく 
\\	いけ)
\\	する 
\\	しろ, くる 
\\	こい, くれる 
\\	くれ	
\\	好きにしろ。	
\\	する-> しろ	
\\	「な」 
\\	行く→ 行くな, する→ するな 
\\	それを食べるな!
\\	変なことを言うな! 
\\	時24 分(いちじ・にじゅうよんぷん)	
\\	午前9 時16 分(ごぜん・くじ・じゅうろっぷん)	
\\	午後4 時10 分(ごご・よじ・じゅっぷん)	
\\	二時間四十分(にじかん・よんじゅっぷん)	
\\	四時, 七時, 九時	
\\	よじ, しちじ, くじ 
\\	じ「時」.	
\\	二十日間(はつかかん)	
\\	二年間(にねんかん)	
\\	「一日」(いちにち)	
\\	「一週間」(いっしゅうかん)	
\\	「八週間」(はっしゅうかん)	
\\	「か」
\\	「⽉」
\\	「げつ」. 
\\	じゃん	
\\	じゃない, 
\\	サラリーマンだから、残業はたくさんするん じゃない? 
\\	駅の近くにカラオケがあるじゃんか。
\\	という 
\\	つ 
\\	という, 
\\	つうか、なんでお前がここにいんのよ! 
\\	宿題で時間がないつってんのに、みきちゃんとデートしにいったらしい。 
\\	"だから、違うんだ
\\	つうの
\\	(という-
\\	つ)	
\\	"明日は試験だぞ。
\\	つっても
\\	、勉強はしてないだろうな。
\\	ってば, ったら	
\\	といえば 
\\	といったら.
\\	もう行くってば!
\\	あなたったら、いつも忘れるんだから。 
\\	なんか	
\\	なにか
\\	今日は、なんか忙しいみたいよ。 
\\	なんかね。お風呂って超気持ちいいよね!
\\	なんかさ。ボブは、私のことなんか本当に好きかな?-
\\	〜やがる	
\\	あんなやつに負けやがって。じゃ、どうすんだよ?
\\	やる気か?だったらさっさと来やがれ! 
\\	かい 
\\	だい 
\\	おい、どこに行くんだい? 
\\	さきちゃんって呼んでもいいかい?
\\	わ 
\\	よ, 
\\	もう時間がないわ。
\\	かな 
\\	今日は雨が降るかな?
\\	かしら	
\\	かな 
\\	いい大学に入れるかしら? 
\\	させる, 
\\	/せる 
\\	全部食べさせた。
\\	「る」
\\	させる
\\	/せる
\\	聞く-
\\	気かせる
\\	させる, 
\\	/せる 
\\	あげる/くれる	
\\	全部食べさせてくれた。
\\	る, 
\\	させる
\\	食べるー
\\	食べさせる, 出す-
\\	出させる 
\\	せる
\\	聞くー
\\	聞かせる, 死ぬー
\\	死なせる
\\	"先生が学生に宿題をたくさん
\\	させた。
\\	"今日は仕事を
\\	休ませて
\\	ください。
\\	させる	
\\	する, 
\\	こさせる	
\\	くる 
\\	る-
\\	さす
\\	食べる-
\\	食べさす 
\\	/す
\\	行く 
\\	行かす 
\\	同じことを何回も言わすな!
\\	お腹空いているんだから、なんか食べさしてくれよ。
\\	さす	
\\	する 
\\	こさす	
\\	くる, 
\\	られる, 
\\	/れる 
\\	ポリッジが誰かに食べられた!
\\	光の速さを超えるのは、不可能だと思われる。
\\	る 
\\	られる 
\\	/れる
\\	られる
\\	/る
\\	られる
\\	/れる
\\	食べられる 
\\	話せる
\\	話される
\\	どうする?	
\\	どうしますか?- 
\\	どうされますか?- 
\\	どうなさいますか?- 
\\	どうなさいますでしょうか?- 
\\	食べる→ 食べさせる→ 食べ
\\	させられる 
\\	行く→ 行かせる→ 行
\\	かせられる 
\\	日本では、お酒を飲ませられることが多い。
\\	行く→ 行か→ 行かす→ 
\\	行かされる 
\\	立つ→ 立た→ 立たす→ 
\\	立たされる 
\\	食べる 
\\	話す
\\	あいつに二時間も待たされた。
\\	こと 
\\	ある	
\\	徹夜して、宿題することはある。
\\	一人で行くことはありません。
\\	しまう 
\\	宿題をやってしまいなさい。
\\	こと 
\\	ある	
\\	日本の映画を観たことないの?
\\	ヨーロッパに行ったことがあったらいいな。 
\\	ところ	
\\	彼は優しいところもあるよ。
\\	これから行くところでした。
\\	もの/もん	
\\	もの 
\\	もん 
\\	あゆあゆじゃないもん) 
\\	授業があったもの。
\\	授業があったもん。 
\\	かも知れない「かもしれない」	
\\	かも	
\\	かもしれない, 
\\	この映画は一回観たことあるかも!
\\	もう逃げられないかもしれんぞ。
\\	でしょう	
\\	明日も雨でしょう。 
\\	学生さんでしょうか。
\\	でしょう/だろう
\\	だから、時間がないって言ったでしょう!
\\	これから食べに行くんでしょ
\\	だろう 
\\	だけ	
\\	これとそれだけ。
\\	この歌だけを歌わなかった。
\\	その人だけが好きだったんだ。
\\	これだけ見ない。
\\	のみ	
\\	だけ
\\	この乗車券は発売当日のみ有効です。 
\\	アンケート対象は大学生のみです。 
\\	しか	
\\	これしかない。
\\	これしか⾒ない。
\\	今日は忙しくて、朝ご飯しか食べられなかった。 
\\	っきゃ 
\\	しか 
\\	これは買うっきゃない! 
\\	こうなったら、もうやるっきゃない! 
\\	ばかり/ばっかり/ばっか 
\\	いやだ。おばさんばっかり。
\\	崇君は漫画ばっかり読んでてさ。かっこ悪い。
\\	ばかり 
\\	しか	ばかり 
\\	しか 
\\	すぎる	
\\	すぎる, 
\\	たべすぎる 
\\	い 
\\	すぎる. 
\\	危ない→ 危なすぎる, 静か→ 静かすぎる 
\\	い 
\\	さすぎる, 
\\	食べない→ 食べなさすぎる,  
\\	ない:  
\\	もったいない→ もったいなさすぎる, 情けない→ 情けなさすぎる	
\\	すぎる 
\\	お酒を飲みすぎないように気をつけてね。
\\	も 
\\	も 
\\	昨日、電話三回もしたよ!
\\	試験のために三時間も勉強した。
\\	今年、十キロも太っちゃった! 
\\	ほど	
\\	寝る時間がないほど忙しい。
\\	今日の天気はそれほど暑くない。
\\	~さ 
\\	高い→ 高さ, 
\\	穏やか→ 穏やかさ 
\\	このビルの高さは何ですか? 
\\	よう 
\\	ように/ような)	
\\	ここには、誰もいないようだ。
\\	映画を観たようです。
\\	の 
\\	な 
\\	学生のようだ。
\\	ここは静かなようだ。 
\\	ような	
\\	あの人を見たような気がした。
\\	(気がする- 
\\	彼は学生のような雰囲気ですね。 
\\	ように 
\\	ちょっと怒ったように聞こえた。
\\	みたい	
\\	もう売り切れみたい。
\\	制服を着ている姿をみると、学生みたいです。
\\	そう	
\\	い), いい
\\	よさ, 
\\	い 
\\	さ 
\\	そう
\\	この辺りにありそうだけどな。あり=
\\	この漬物はおいしそう!
\\	い
\\	おいしい)
\\	もう10 時になったから、来なさそうだね。(なさそう= 
\\	そうだ/そうです/そうでございます	
\\	毎日会いに行ったそうです。
\\	彼は、高校生だそうです。(だ 
\\	~らしい	
\\	こないらしい。
\\	あの子は子供らしくない。
\\	っぽい
\\	っぱい 
\\	あの人はちょっと韓国人っぽいよね。
\\	恭子は全然女っぽくないね。
\\	方(ほう)
\\	ご飯の方がおいしい 
\\	鈴⽊さんの方が若い。
\\	ゆっくり食べた方が健康にいいよ。
\\	怖い映画は観ない方がいいよ。/ こわいい
\\	えいが は みない ほうがいい よ	
\\	方(ほう) 
\\	そんなに飲まなかった方がよかった。 
\\	より	
\\	方 
\\	ご飯の方が、パンよりおいしい。
\\	ゆっくり食べた方が早く食べるよりいい。
\\	より 
\\	誰
\\	何
\\	どこ	
\\	も 
\\	この仕事は誰よりも早くできます。
\\	方(かた)	
\\	方 
\\	行き方(いきかた) 
\\	そういう食べ方は体によくないよ。
\\	パソコンの使い方は、みんな知っているでしょう。
\\	によって 
\\	による	
\\	によって 
\\	人によって話が違う。
\\	による 
\\	それは、裕子によるね。
\\	によると	
\\	によると 
\\	天気予報によると、今日は雨だそうだ。
\\	友達の話によると、朋子はやっとボーイフレンドを見つけたらしい。 
\\	〜やすい	
\\	やすい 
\\	カクテルはビールより飲みやすい。
\\	にくい	
\\	にくい 
\\	その肉は食べにくい。
\\	がたい	
\\	にくい 
\\	とても信じがたい話だが、本当に起こったらしい。
\\	づらい	
\\	にくい 
\\	日本語は読みづらいな。
\\	ないで 
\\	何も
\\	食べないで
\\	寝ました。
\\	先生と相談しないで、この授業を取ることは出来ない。
\\	ず 
\\	ない 
\\	ず するー
\\	せず, くるー
\\	せず
\\	食べるー
\\	たべず、行くー
\\	行かず
\\	彼は何も言わず、帰ってしまった。
\\	勉強せずに東大に入れると思わないな。
\\	/ん	
\\	ないー
\\	なん  知らないー
\\	知らん 
\\	韓国人と結婚しなくてはならん!
\\	~ぬ	
\\	ぬ 
\\	ない 
\\	との	
\\	松川先生
\\	との出会いをきっかけに、私の人生は変わった 
\\	より
\\	のほう	
\\	アメリカ映画よりフランス映画の方が好きです。
\\	犬よりも、猫の方がおすすめです。
\\	わけ (訳)	
\\	つまり、語学には、能力がないという訳か。
\\	(というわけ 
\\	中国語が読めるわけがない。
\\	分かるわけないよ!
\\	わけない	
\\	ここの試験に合格するのはわけない。
\\	この本は面白くなってきた。
\\	食べてきました。
\\	明日は晴れてくるでしょう。
\\	食べてきました。
\\	も 「でも/ても」	
\\	冗談を言っても、彼は笑いません。
\\	みる) 〜てみる	
\\	ってる 
\\	(っている), 
\\	わかってる 
\\	が, 
\\	は, 
\\	は, 
\\	は	
\\	漢字は難しいですが,日本語の文法はあまり難しくないんです 
\\	バターをかいましたか? 
\\	マーガリンはかいましたが、バターは買いませんでした
\\	(はいる)	
\\	コンピューたをもって(は)いますが、まだ使って(は)いません
\\	(はする)	
\\	あのひとを知って(は)いますが、あまり話したこと(は)あまりません. 
\\	が 
\\	この映画は面白いが、長過ぎますね
\\	昨日富士山をはじめてみましたが、きれいでした。
\\	が 
\\	といいが, といいんが	
\\	早く春が来るといいんだが
\\	が 
\\	が	
\\	私は助かろうがかまいません。
\\	が 
\\	が	
\\	私が行こうが行くまいが、あなたには関係ないことです。
\\	が早いか	
\\	窓お開けるが早いか、猫が飛び込んできた
\\	も	
\\	スミスさんは漢字お5000も知っています 
\\	1時間もあれば、ホテルから空港え行けます
\\	とする	
\\	観客
\\	として参加させてもらった。
\\	被害者としては、非常に幸いだった。
\\	明日に行くとする。
\\	ばかり/ばっか	
\\	すみません、今食べたばかりなので、お腹がいっぱいです。
\\	まさか、今起きたばっかなの?
\\	は
\\	より暑い	
\\	より
\\	としたら	
\\	と
\\	する, 
\\	今から行くとしたら、9時に着くと思います。 
\\	として	
\\	観客として参加させてもらった。
\\	被害者としては、非常に幸いだった。
\\	朝ご飯を食べたとしても、もう昼だからお腹が空いたでしょう。 
\\	〜ばかり/ばっか	
\\	すみません、今食べたばかりなので、お腹がいっぱいです。
\\	まさか、今起きたばっかなの?
\\	とたん (に)	
\\	窓を開けたとたんに、猫が跳んでいった。
\\	映画を観たとたんに、眠くなりました。
\\	ながら	
\\	テレビを観ながら、宿題をする。
\\	音楽を聴きながら、学校へ歩くのが好き。
\\	ながら	
\\	彼は、初心者ながらも、実力はプロと同じだ。
\\	貧乏ながらも、高級なバッグを買っちゃったよ。
\\	まくる (まくっている)	
\\	やる 
\\	やりまくっている 
\\	やりまくっていない 
\\	やりまくっていた 
\\	やりまくっていなかった 
\\	アメリカにいた時はコーラを飲みまくっていた。
\\	まま	
\\	このままで宜しいですか? 
\\	半分しか食べてないままで捨てちゃダメ! 
\\	今日だけは悲しいままでいさせてほしい。
\\	っ放し (っぱなし)	
\\	窓が開けっ放しだったので、蚊がいっぱい入ってしまった。
\\	である
\\	である 
\\	吾輩は猫である  
\\	であった,
\\	ではない	
\\	である (は 
\\	それは不公平ではないでしょうか。
\\	ではなかった, 
\\	おり	
\\	封筒には写真が数枚入っており、手紙が添えられていた。
\\	ます 
\\	企業内の顧客データを利用し 
\\	、彼の行方を調べることが出来た。
\\	花火(はなび)は、火薬と金属の粉末を混ぜたものに火を付け 
\\	、燃焼時の火花を楽しむためのもの。
\\	に 
\\	蚊に刺された。
\\	この
\\	都市
\\	は聖
\\	女様
\\	に
\\	守られた
\\	人類最後の聖域であり
\\	~はず	
\\	この料理はおいしいはずだったが、焦げちゃって、まずくなった。
\\	~はずがない, ~はずはない	
\\	そう簡単に直せるはずがないよ。
\\	べき 
\\	行う 
\\	行うべき, する 
\\	するべき 
\\	すべき	
\\	すべき	"する 
\\	する 
\\	するべき 
\\	すべき 
\\	(で) さえ	
\\	金さえあれば、何でも出来るよ。
\\	私でさえ出来れば、あんたには楽ちんでしょう。
\\	*でさえ 
\\	さえ 
\\	する	
\\	教科書をもっとちゃんと
\\	読んでさえ
\\	いれば、合格できたのに。
\\	ビタミンを
\\	食べさえすれば
\\	、健康が保証されますよ。
\\	(で) すら	
\\	この天才の私ですらわからなかった。
\\	おろか	
\\	漢字はおろか、ひらがなさえ読めないよ!
\\	大学はおろか、高校すら卒業しなかった。
\\	どころか	
\\	おろか, 
\\	*どころか 
\\	漢字どころか、ひらがなさえ読めないよ!
\\	がる	
\\	い, 
\\	がる 
\\	怖い 
\\	怖がる,  嫌 
\\	嫌がる 
\\	早くきてよ!何を恥ずかしがっているの?
\\	うちの子供はプールに入るのを理由もなく怖がる。
\\	がる 
\\	がらない, がった, がらなかった	
\\	-たい 
\\	がる, 
\\	ほしい 
\\	がる	
\\	たい 
\\	い 
\\	がる 
\\	家に帰ったら、すぐパソコンを使いたがる。
\\	みんなイタリアに行きたがってるんだけど、私の予算で行けるかどうかはとても怪しい。
\\	がり屋「がりや」	
\\	恥ずかしがり屋
\\	暑がり屋 
\\	私は寒がり屋だから、ミネソタで暮らすのは辛かった。
\\	ん 
\\	ばかり	
\\	ボールは爆発せんばかりに、膨らんでいた。
\\	「あんたとは関係ない」と言わんばかりに彼女は彼を無視していた。
\\	めく	
\\	いつも
\\	皮肉めいた
\\	言い方をしたら、みんなを嫌がらせるよ。
\\	を 
\\	みんなを嫌がらせるよ。
\\	~ざるを得ない「されをえない」	
\\	ない 
\\	ざるを得ない. 
\\	このテレビがこれ以上壊れたら、新しいのを買わざるを得ないな。
\\	~やむを得ない「やむをえない」	
\\	*やむを得ない事由*により手続が遅れた場合、必ずご連絡下さい。
\\	かねる	
\\	*「かねない」
\\	この場ではちょっと決めかねますので、また別途会議を設けましょう。
\\	しかねる	
\\	(するー
\\	しかねる)	
\\	かねない	
\\	がち	
\\	なりがち
\\	父親は病気がちで、みんなが心配している。
\\	なりがちじゃない 
\\	なりがちだった 
\\	なりがちじゃなかった 
\\	つつ	
\\	ながら, 
\\	二日酔いで痛む頭を押さえつつ、トイレに入った。
\\	電気製品の発展につれて、ハードディスクの容量はますます大きくなりつつある。
\\	きらいがある	
\\	多くの大学生は、締切日ぎりぎりまで、宿題をやらないきらいがある。
\\	まい 
\\	見るー
\\	見まい 
\\	まい 
\\	行く ー
\\	行くまい 
\\	まい 
\\	なるー
\\	なりー
\\	なりますー
\\	なりますまい 
\\	するー
\\	するまい 
\\	しまい, くるー
\\	くるまい 
\\	そんな無茶な手段は認めますまい!
\\	くれ	
\\	助けてくれ!
\\	話してくれ 
\\	ようにしている	
\\	肉を食べないようにしている。
\\	ようにする 
\\	やめよう	
\\	やめる, 
\\	明日に行くのをやめよう。
\\	位「くらい/ぐらい」 
\\	まい	
\\	相手は剣の達人だ。そう簡単には勝てまい。
\\	が 
\\	が	
\\	あいつが大学に入ろうが入るまいが、俺とは関係ないよ。
\\	時間があろうがあるまいが、間に合わせるしかない。
\\	~であろう	
\\	もちろん、生徒数減少の現在、学科の新設は困難であろうが...
\\	かろう 
\\	い, 
\\	ではない 
\\	ではなかろう, 早い 
\\	早かろう 
\\	仕事にとりか取りかかろう
\\	に 
\\	ある、いる	に 
\\	だらけ	
\\	間違い 
\\	間違いだらけ 
\\	携帯を2年間使ってたら、傷だらけになった。
\\	まみれ	
\\	血 
\\	血まみれ 
\\	油 
\\	油まみれ 
\\	たった1キロを走っただけで、汗まみれになるのは情けない。
\\	ずくめ	
\\	白ずくめ団体は去年ニュースになっていた。
\\	このシェークは、おいしいし、栄養たっぷりで体にいいですから、いいことずくめですよ。
\\	が早いか	
\\	言う 
\\	言うが早いか, 言う 
\\	言った 
\\	言ったが早いか
\\	彼女は、教授の姿を
\\	見るが早いか
\\	、教室から逃げ出した。
\\	や/や否や「やいなや」	
\\	見る 
\\	見るや 
\\	見るや否や
\\	私の顔を
\\	見るや
\\	、何か言おうとした。
\\	そばから	
\\	教科書を読んだそばから忘れてしまうので勉強ができない。
\\	思いきや	
\\	と, 
\\	あるー
\\	あると思いきゃ このレストランは安いと思いきや、会計は5千円以上だった!
\\	~がてら	
\\	散歩がてら、タバコを買いに行きました。
\\	博物館を見がてらに、お土産を買うつもりです。
\\	あげく	
\\	(の 
\\	先生と相談のあげく、退学をしないことにした。
\\	事情を2時間かけて説明したあげく、納得してもらえなかった。
\\	も 
\\	あの映画は何度も見ました 
\\	このレストランは、いつも混んでいます。 
\\	も 
\\	あの芝居は長くても3時間で終わりますよ 
\\	も	
\\	でも 
\\	でも 
\\	でも 
\\	でも	
\\	どんな 
\\	でも	
\\	でも、
\\	でも、。。。
\\	でも	でも 
\\	とか, 
\\	と 
\\	と	
\\	りんごとみかんとどちらがすきですか。
\\	と
\\	なる	
\\	になる) 
\\	と 
\\	と  
\\	と  
\\	という	
\\	と 
\\	と 
\\	と 
\\	と 
\\	とは 
\\	春とはいえまでさむい 
\\	やいなや 
\\	など, なんか, なぞ, なんぞ	
\\	で	
\\	この本は1時間で読めますよ。
\\	あのテレビは10万円で買える。
\\	で	
\\	で	
\\	病気で旅行に行けなかった。
\\	に 
\\	に 
\\	先生は学生に漢字を書けせました。
\\	子どもたちに本を読ませることはとても大切だ。
\\	に 
\\	学生は先生に漢字を書かされました 
\\	に, 
\\	に, 
\\	に	
\\	と 
\\	[に 
\\	ばかり	
\\	マギーは寝てばかりいる。 
\\	遊んでばかりいてはいけません。 
\\	に 
\\	から 
\\	昨日私は仕事が終わってから買い物をしました。
\\	あの2人が結婚してから20年だそうです。
\\	から 
\\	から 
\\	から。。。
\\	まで 
\\	まで 
\\	子供だけでなく大人まで、そのゲームを楽しんだ。
\\	まで 
\\	今日はここまで。
\\	とりあえずご報告まで。
\\	までもない	
\\	の 
\\	の 
\\	の 
\\	を 
\\	を 
\\	くらい/ぐらい 
\\	ばかり	
\\	ばかりでなく	
\\	だけ 
\\	だけに	
\\	だけのことはある	
\\	だけあって	
\\	だけでなく	
\\	しか 
\\	しか 
\\	何より「なにより」	
\\	京都の桜は何より美しい。
\\	人生において健康が何より大切です。
\\	のみならず 
\\	きり/っきり
\\	きり 
\\	きり 
\\	もので	
\\	けれども、けれど、けど、けども 
\\	けれども、けれど、けど、けども 
\\	けれども、けれど、けど、けども 
\\	だ 
\\	ところで 
\\	のに 
\\	くせに	
\\	くせに 
\\	ものの	
\\	ところが	
\\	できれば	
\\	ば 
\\	ばいい	
\\	たら 
\\	たら 
\\	早くしったら
\\	もうねなさいったら 
\\	たら 
\\	いったら  なら 
\\	ものなら	
\\	ところ	
\\	ところ	
\\	ところ	
\\	ところ 
\\	ものを 
\\	し	
\\	とも 
\\	く 
\\	とも
\\	やら 
\\	やら, 
\\	やら	
\\	やら	
\\	やら 
\\	だの	
\\	だの, 
\\	だの
\\	なり
\\	なり	
\\	なり
\\	なり、
\\	なんなり	
\\	だいなり	
\\	ては(では)
\\	(ては、ては)	
\\	にしては	
\\	ては(では)	
\\	ては(では)こまる	
\\	ては(では)いけない, ては(では)ならない	
\\	ては(では)いられない	
\\	どころ	
\\	として	
\\	(ひとつ、いちど、
\\	として	
\\	というより	
\\	より
\\	より	
\\	より
\\	より
\\	ては(では)
\\	さえ。。。ーたら	
\\	さえ。。。ーば	
\\	すら, する
\\	こちらこそ	
\\	こそ 
\\	もの 
\\	ものか 
\\	もんか	
\\	に 
\\	だろう、でしょう、
\\	ようになる	
\\	日本行ってから、日本に興味を持つようになりました。
\\	ことになっている	
\\	ことになった	
\\	ことになっています	
\\	の) は初[はじ]めてです	
\\	おすしをたべたことがありますか。
\\	いいえ、おすしは初めてです。
\\	ことにする	
\\	来年日本え行くことにしました。
\\	たら, 
\\	たら 
\\	きのう図書館え行ったら、先生に会いました 
\\	とても 
\\	ない 
\\	ません, 
\\	たくさんあった、とても食べきれません 
\\	東京ではとても家なんか買えません。
\\	て初めての
\\	アメリカに着いて初めてのカルチャーショックだった。
\\	それに	
\\	なかなか~ない	
\\	漢字なかなか覚えられなくて、困っています。
\\	かわからない	
\\	いつ日本へ行くかわかりません。
\\	わけです	
\\	わけです 
\\	よくできるわけですね。
\\	に当[あ]たる	
\\	日本語の「こんいちは」は、英語の 
\\	に当たる。
\\	(の)ような 
\\	「どうも」のよう便利なフレーズは、英語にはないだろう。
\\	名前の漢字は、日本人にも読めないような読み方があるから、難しい。
\\	ないで	
\\	ないでください 
\\	ないで, 
\\	ていただけないでしょうか	
\\	中に「中=ちゅう 
\\	じゅう]	
\\	今日中にメールをしてください。
\\	ば 
\\	ほど	
\\	日本語は、勉強すれば勉強するほどおもしろくなると思います。
\\	学生が多ければ多いほど、選べれるのがむずしい。
\\	以外の 
\\	以外に [以外=いがい」	
\\	学期の初めは、強化症[きょうかしょ」以外にいろいろ買うものがある。
\\	必「かなら」ずしも 
\\	というわけではない	
\\	別「べつ」に(~ない)	
\\	いいえ、別に何でもありません。
\\	たしか	
\\	たしか 
\\	今日の人口は、たしか一億に千万ぐらいでしょう。 
\\	たしかに 
\\	は
\\	で有名「ゆうめい」だ	
\\	の/こと) 
\\	ミルウオーキーは、ビールで湯名です 
\\	にする	
\\	私は、すき焼きにします。
\\	いつのまにか	
\\	お金はいつのまにかなくなってしまいます。
\\	外はいつのまにか暗くなっていました。
\\	てくる	
\\	だいぶ日本の生活になれてきました。
\\	日本語がだいぶ話せるようになってきました。
\\	ていく	
\\	私達の生活は、どんどん変わっていくでしょう。
\\	~わけではない	
\\	~わけではない 
\\	あまり英語を話しませんが、英語ができないわけではなりません。
\\	ないで住[す]む	
\\	ないで 
\\	て-
\\	バスがすぐ来たので、あまり待たないで
\\	みました。
\\	~ずつ	
\\	ずつ 
\\	少し) 
\\	きのうは友達の誕生日だったおで、赤いバラとピンクのバラを六本ずつあげました。
\\	私は毎日漢字を五つず覚えることにしています。
\\	~以上(いじょう)	
\\	(~に)気がつく	
\\	間違いに気が付きませんでした。
\\	せっかく	
\\	わけにはいかない	
\\	そうかと言って	
\\	ために	
\\	やせるために毎日運動をしています。
\\	のための 
\\	これは外国人の ための 日本語の辞書です。
\\	〜化 [か]	
\\	~になる 
\\	~にする 
\\	映画化された小説は多い。
\\	~ことは	
\\	でいいです 
\\	でけっこうです	
\\	お茶でけっこうです。
\\	辞書は、一冊でいいでしょうか。
\\	~ことは限[かぎ]らない	
\\	日本の高校生がみんな勉強ばかりしているとは限らない。
\\	時(とき)に	
\\	この辞書は、日本えいった時(に)買いました。
\\	~か~かと心配[しんぱい]した	
\\	か 
\\	かと 
\\	心配する 
\\	迷う 
\\	考える, 
\\	~のではないでしょうか、~のではないだろうか、~のではないか	
\\	むしろ	
\\	会話は、日本語の方が英語おりむしろやさしいかもしいれない。
\\	何「なん」といっても	
\\	~たっけ/だっけ	
\\	去年の夏も暑かったけ。
\\	ようにする	
\\	日本語は毎日勉強するようにしてください。
\\	始[はじ]める	
\\	私が日本語を勉強し始めたのは一年前です。
\\	終わる	
\\	-て 
\\	~をはじめ	
\\	によって	
\\	~的(てき)	
\\	は
\\	くらいです	
\\	わざわざ~	
\\	より仕方がいない	
\\	れる/られる	
\\	ように 
\\	ように 
\\	~にみえる 
\\	くみえる	
\\	いつも元気に見える。
\\	日本人は年より若く見える。
\\	と言っても
\\	といっても, 
\\	クリスマスといっても、日本にはクリスチャンはあまりません。
\\	に限[かぎ]らず, に限[かぎ]らない	
\\	に限[かぎ]らている, に限[かぎ]り	
\\	女性に限り、クラブの入場料は半額です。
\\	どんあ
\\	でも	
\\	分の 
\\	五分の一 
\\	ほど	
\\	ほど 
\\	は)
\\	に限る	
\\	~もんですから 
\\	ものですから	
\\	うちに	
\\	うちに	
\\	が気になる	
\\	の.  
\\	どうしても 
\\	どうしてもわからない時は、先生に聞いてください。
\\	どうしても 
\\	に気をつける	
\\	日本語を話す時は、アクセントに気をつけてください。
\\	~に気がつく	
\\	気になる	
\\	としたら	
\\	て、)
\\	くらいだ	
\\	ものだ	
\\	たて	
\\	ろくに~ない	
\\	~せいか	
\\	せい, 
\\	か, 
\\	せいか, 
\\	おかげで	
\\	てごらんなさい	
\\	てみなさい。
\\	に越[こ]したことはない	
\\	思うように	
\\	たら 
\\	ったら	
\\	よっぱど	
\\	~に違[ちが]いない	
\\	い-
\\	さ
\\	が(に)	
\\	なんか, 
\\	なんて	
\\	~一方[いっぽう]で(は)	
\\	他方[たほう]では 
\\	一方では.
\\	毎日のように	
\\	日 
\\	週、月、年、
\\	せいぜい	
\\	ながら	
\\	それにしても	
\\	~になれる	
\\	の. 
\\	日本の生活になれてきました。
\\	限[かぎ]り	
\\	かえって	
\\	には
\\	の~がある	
\\	1は
\\	1なりに
\\	1は
\\	1なりの
\\	っていう 
\\	という理由で	
\\	を
\\	にまかせる	
\\	たびに
\\	てくれたらと思います	
\\	どっちかと言うと	
\\	~ような気がする	
\\	とか言う	
\\	(と)同様(どうよう)	
\\	単[たん]に~ではない	
\\	の余地がない[のよちがない]	
\\	に(は)かなわない	
\\	に  こだわる 
\\	こだわらない	
\\	こだわらない 
\\	において; 
\\	で 
\\	てからというもの	
\\	ギリシャに来てからというもの、国の家族のことを思わない日はありません。
\\	このタブレット端末たんまつを使ってからというもの、手放てばなせなくなった。
\\	て以来[いらい]	
\\	場合[ばあい]	
\\	火事の場合、段階を使いなさい。
\\	~に備[そな]えて	
\\	~につけ	
\\	につけ
\\	につけ	
\\	喜びにつけ悲しみにつけ 
\\	いりません	
\\	どちにしても	
\\	どちにしても、 帰れなきあでしょう?
\\	不得意[ふとくい](な)	
\\	彼女は物理ぶつりが不得意です。
\\	とおり  
\\	どおり
\\	(通り)	
\\	私がする通りにしてください。 
\\	言った通りにしました。 
\\	み	み 
\\	黒み 
\\	高み 
\\	積り[つもり]	
\\	今日何をするつもり(か)? 
\\	タバコをやめるつもりはない。
\\	もしない	
\\	働きもしないでぐずぐずするな。 
\\	も
\\	今日に至いたるも完成かんせいを見ない。
\\	どんなに多おおくも十人じゅうにんまでだ。
\end{CJK}
\end{document}