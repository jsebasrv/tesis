\documentclass[8pt]{extreport} 
\usepackage{hyperref}
\usepackage{CJKutf8}
\begin{document}
\begin{CJK}{UTF8}{min}
\\	【動】~(気持ち・意見など)を尊重する、~を尊敬する
\\	【動】意思
\\	【名】個人
\\	【定】気楽にやれ
\\	【動】
\\	(人)に~を確約する[保証する]
\\	【動】(最終的に)~という結果になる
\\	【動】~を離す、~を捨てる
\\	【形】悲観的な
\\	【名】(~に対する)見方[態度]、~観
\\	【動】~(関係など)を維持する
\\	【形】積極的な、前向きな
\\	【名】(~に対する)態度
\\	【形】公平な
\\	【前】~にかかわらず
\\	【形】国の、国内の
\\	【名】起源、生まれ
\\	【名】性別
\\	【名】(宗教上の)信条
\\	【名】平等
\\	【動】~を保証する、~を約束する
\\	【名】憲法
\\	【動】
\\	~に寄り掛かる
\\	【名】柱、支柱
\\	【動】(~を)じっと見つめる
\\	【名】彫像
\\	【名】自由
\\	【動】(~の)そばを通り過ぎる
\\	【動】~(臭い・光・熱)を発する
\\	【形】かすかな
\\	【名】香り
\\	【名】香水
\\	【動】
\\	に
\\	を思い出させる
\\	【頭】前の-
\\	【動】(~な)においがする
\\	【形】ひどい
\\	【動】~な味がする
\\	【形】すばらしい
\\	【形】(雨などで)びしょ濡れの
\\	【名】汗
\\	【動】(邪魔・迷惑だから)後ろに下がる
\\	【動】悪臭がする、(ひどく)臭い
\\	【動】シャワーを浴びる
\\	【動】(~で)ひどく興奮する、取り乱す
\\	【副】ほとんど(~し)ない
\\	【動】[通例
\\	を
\\	と区別する
\\	【名】事実、現実
\\	【名】作り話、虚構
\\	【名】(長編)小説(作品)
\\	【動】
\\	と
\\	を組み合わせる
\\	【名】散文
\\	【名】(天賦の)才能
\\	【名】(文学の一分野の)詩
\\	【動】~を出版する
\\	【形】(情報などが)最新の~
\\	【名】(本などの)版
\\	【名】百科事典
\\	【動】(新製品が)発売される、(本が)出版される
\\	【動】(~を)
\\	から
\\	に)翻訳する
\\	【形】(人を)魅了する
\\	【名】妖精、おとぎ話
\\	【名】話、物語
\\	【形】平易な
\\	【形】以下の~
\\	【名】(文章などの)一節
\\	【動】(~を)引用する
\\	【形】有名な
\\	【名】寓話
\\	【動】~を(よく)知っている
\\	【形】現代の
\\	【名】文学
\\	【定】ほとんど~、~の隣に
\\	【副】たまに、時々
\\	【動】
\\	と
\\	を混同する
\\	【名】曲線、(道路などの)カーブ
\\	【動】~を彫る
\\	【形】恥ずかしがりの
\\	【名】発音
\\	【副】だいたい
\\	【形】正しい
\\	【名】記事
\\	【動】~を含んでいる
\\	【名】役立つ情報
\\	【代】~という人たち
\\	【動】しきりに~したがっている
\\	【動】~を増やす
\\	【名】語彙
\\	【形】最新の
\\	【名】作品、仕事、作業
\\	【動】展示されている
\\	【名】市庁舎
\\	【形】すてきな<女性的>
\\	【形】【副】言葉では表せない(ほど)
\\	【動】(人が座れるように)席を詰める
\\	【動】~ということに気付く
\\	【動】(必要以上に)~(空間・時間など)を取る
\\	【形】醜い、不格好[不細工]な
\\	【名】物体、物
\\	【形】1つ[1個]の
\\	【形】抽象的な
\\	【動】(~に)すがってお願いをする
\\	【動】
\\	(人)に
\\	(物・金)を貸す
\\	【名】ドル
\\	【動】
\\	と言う、首を横に振る
\\	【動】(全然)お金がない、すっからかんだ
\\	【動】~しがちである
\\	【副】衝動的に
\\	【副】バーゲン[特売]で
\\	【定】(先行文に答えて)私もそうです[同じです]
\\	【副】(文頭で、残念ながら)実際のところ
\\	【形】普通の
\\	【動】~する余裕がある
\\	【動】~を購入する
\\	【名】贅沢品
\\	【動】~を予期する
\\	【名】失業(状態)
\\	【動】率、割合
\\	【動】(数・量などが)上がる、増える
\\	【名】四半期
\\	【副】連続で
\\	【動】~を減らす
\\	【名】出費
\\	【副】これからは
\\	【動】~の動き[流れ]を常に記録する
\\	【形】毎日の
\\	【副】~という基準[やり方]で
\\	【副】とにかく、いずれにせよ
\\	【名】労働組合
\\	【名】(~について)妥協する
\\	【形】(特定はできないが)ある
\\	【副】~の程度
\\	【形】(要求される一定の)能力[技能]を持つ
\\	【名】(機械の)整備士、修理士
\\	【動】需要がある
\\	【動】~(金)を稼ぐ
\\	【形】かなりよい
\\	【名】(時間・日数などで計算される)賃金
\\	【名】大統領
\\	【動】~を公表[発表]する
\\	【形】(計画などが)具体的な
\\	【動】(計画・任務など)~を実行する
\\	【名】福祉
\\	【名】(制度などの)改革
\\	【名】政策
\\	【副】きっと
\\	【動】~(結果)に至る
\\	【形】(結果・状況などが)気分を暗くさせる
\\	【名】(当然の)結果、影響
\\	【形】(調査などが)徹底的な
\\	【名】見直し
\\	【定】~は言うまでもない
\\	【名】高齢化
\\	【形】避けられない
\\	【動】~をちょっと見る
\\	【名】図、表
\\	【動】~(事実・長考など)を示す
\\	【形】青少年の、少年の
\\	【名】非行
\\	【動】増加している
\\	【形】(悪い事の増加が)深刻な
\\	【副】~なペース[割合]で
\\	【名】地区
\\	【形】たくさんの(数の)
\\	【形】(家・座席などが)誰もいない
\\	【名】1区画
\\	【動】売り物である
\\	【名】人口
\\	【名】密度
\\	【名】大都市
\\	【副】徐々に
\\	【動】減少[低下]する
\\	【副】ほぼ間違いなく
\\	【名】出生率
\\	【動】~し続ける
\\	【動】減少する
\\	【副】着実に
\\	【形】これから来る~
\\	【名】(電気・ガスなどの)エネルギー
\\	【名】工場、発電所
\\	【動】
\\	に
\\	を)供給する
\\	【形】(距離などが)遠く離れた
\\	【名】(米国の)郡
\\	【名】電気
\\	【動】
\\	に~することを許す
\\	【動】~(機械)を動かす[操作する]
\\	【名】装置
\\	【名】(正式な)許可
\\	【副】実際は
\\	【名】住民
\\	【動】
\\	を
\\	にさらす
\\	【名】放射能、放射線
\\	【構】いくら~しすぎてもし過ぎることはない
\\	【動】~を強調[力説]する
\\	【形】潜在的な
\\	【名】危険、危険性
\\	【形】核の、原子力の
\\	【名】(石油・電気などの)エネルギー
\\	【名】
\\	報道陣[機関]
\\	【動】(不注意で)~を見落とす[見過ごす]
\\	【副】むしろ、どちらかと言えば
\\	【動】
\\	に対して~な評価[判断]をする
\\	【前】~だけではなく
\\	【動】~(作物)を栽培する
\\	【名】(集合的に)穀物
\\	【名】(近代的な)農業経営者
\\	【動】~を経営[運営]する
\\	【名】食料雑貨店
\\	【名】(~に関する)研究[調査]
\\	【名】理系の大学、組織
\\	【動】~を設立する
\\	【形】
\\	~後半の
\\	【名】1960年代
\\	【定】~したら?
\\	【動】
\\	に関して)~(専門家など)に相談する
\\	【副】(電話や代理人を使わず)本人と直接会って
\\	【副】(比較級・最上級を強めて)はるかに、間違いなく
\\	【形】(人が)傑出した、著名な
\\	【名】(米)弁護士
\\	【動】(忘れないように)~を書き留める
\\	【副】(文末に用いて)念のため
\\	【動】記憶力が悪い
\\	【動】
\\	(人)に~するのがいいと勧める
\\	【名】(弁護士・設計事務所などの)依頼人、顧客
\\	【形】法的な
\\	【動】(~な)行動を取る
\\	【名】保険
\\	【動】~する権利がある
\\	【動】
\\	(人)の)
\\	(損害)を保証する
\\	【名】けが、負傷
\\	【副】現在までに
\\	【形】~もの(多くの)
\\	【動】~(病気など)で死ぬ
\\	【名】
\\	インフルエンザ
\\	【名】(伝染病の)流行
\\	【名】効果
\\	【名】錠剤
\\	【形】(感情・勢いなどが)激しい、強烈な
\\	【形】短時間の、簡潔な
\\	【動】~する(よくない)癖がある
\\	【動】(~を)噛む
\\	【名】爪
\\	【副】全く
\\	【形】ぞっとするような
\\	【名】おばあちゃん
\\	【動】(無理をして)~(体の一部)を痛める
\\	【名】背中、腰
\\	【動】腰をかがめる
\\	【動】~を(ぎゅっと)抱きしめる
\\	【形】(筋肉などが)凝った
\\	【形】薬草の、漢方の
\\	【名】治療(法)
\\	【形】医学の
\\	【名】(科学・技術などの)飛躍的な発展
\\	【動】~(変化など)をもたらす
\\	【名】(非金銭的な)利益
\\	【名】人類
\\	【形】~を全体[一つ]として
\\	【動】
\\	(理論など)を
\\	に応用[適用]する
\\	【動】~のクローンを作る
\\	【名】(実用的な)技術
\\	【名】(畜産用の)牛
\\	【接】(否定文に続いて)~はなおさら(~でない)
\\	【名】(神・動物・機械に対して)人間
\\	【動】(絶対的な圧力で)~を禁止する
\\	【名】鯨
\\	【動】~を分類する
\\	【名】哺乳動物
\\	【名】理論、学説
\\	【名】進化
\\	【前】~の届かない範囲に
\\	【名】想像力
\\	【名】生物学者
\\	【名】~を誇りに思う
\\	【形】歴史的(に重要)な
\\	【名】発見
\\	【動】~を自慢げに話す
\\	【形】倫理上の、倫理的な
\\	【副】~な理由から
\\	【動】~に反対である
\\	【形】いわゆる
\\	【名】遺伝子
\\	【名】(特殊な)治療、療法
\\	【形】最初の
\\	【名】(病気の)兆候、症状
\\	【名】病気
\\	【名】(病気による)高熱
\\	【形】(炎症などを起こして)痛い
\\	【名】喉
\\	【名】構造
\\	【名】脳
\\	【形】(構造などが)複雑な
\\	【前】~のために、~が理由で
\\	【名】病気
\\	【名】代表(者)
\\	【動】~を欠席する
\\	【形】年に1度の
\\	【名】(公式の大きな)会議
\\	【副】徐々に
\\	【名】娘婿
\\	【動】~(病気)から回復する
\\	【名】胃
\\	【名】癌
\\	【動】~な気分[精神状態]である
\\	【形】偽物の
\\	【形】(物が)本物の
\\	【名】骨董品
\\	【感】まさか(信じられない)
\\	【定】ほかに何か?
\\	【定】(それで)以上です
\\	【定】こちらでお召し上がりですか?それともお持ち帰りですか?
\\	【名】敷物
\\	【前】~を含めて
\\	【名】税金
\\	【定】それ[これ]にします[決めます]
\\	【名】パンツ、ズボン
\\	【動】(服などが)~に似合う
\\	【動】~を試しに着てみる
\\	【動】流行っている
\\	【動】(~(小型車)から)降りる
\\	【名】タクシー
\\	【副】急いで
\\	【定】おつりはいりません
\\	【名】配達
\\	【名】(便利な)サービス
\\	【形】利用[入手・購入]可能な
\\	【名】客
\\	【形】(量・程度が)わずかな
\\	【形】追加の
\\	【名】料金
\\	【定】(この支払は)私のおごりです
\\	【動】~にごちそうする[ごほうびをあげる]、~を扱う
\\	【接】~するときはいつでも
\\	【動】外食する
\\	【動】割り勘にする
\\	【名】伝票
\\	【動】何となく時間をすごす、ぶらぶらする
\\	【名】(巨大な)ショッピングセンター[モール]
\\	【動】~に偶然出会う
\\	【動】~が不足している
\\	【動】~(預金など)を引き出す
\\	【動】~(お金など)を預ける
\\	【名】(銀行)口座
\\	【副】(~に向かう)途中で
\\	【動】
\\	(人)から
\\	(物・金)を奪う
\\	【名】{米}(女性用の)ハンドバッグ
\\	【形】奇妙な
\\	【動】(偶然・不意に)起こる
\\	【動】
\\	を
\\	(人)に説明する
\\	【名】門限
\\	【名】罰
\\	【動】約束を破る
\\	【副】諺にある通り
\\	【名】(最終)目的
\\	【動】~を正当化する
\\	【名】手段
\\	【接】(さあ[いまや])~なのだから
\\	【動】(辛い[困った]時)~に頼る
\\	【動】自立する
\\	【名】(会議などの)議長
\\	【動】~の話を遮る
\\	【動】
\\	を終わりにする
\\	【動】決を採る
\\	【動】~を交換する
\\	【形】(よい意味で)率直な
\\	【名】意見、考え
\\	【名】(会社などの)会議、打ち合わせ
\\	【名】(グループ内の)まとまった意見
\\	【動】まだ~してない
\\	【動】(合計・結論)~に達する
\\	【前】~に関して
\\	【名】事柄、問題
\\	【動】~(人)と同じ意見である
\\	【動】~(人)と意見が異なる
\\	【動】~の味方[側]である
\\	【副】厳密に言うと
\\	【名】考え、見解
\\	【動】~と異なる
\\	【副】いくらか
\\	【動】~(の全体)をよく見る[説明する/考える]
\\	【動】[通例 
\\	~(わかりにくいこと)を理解できる
\\	【動】~を言おうとしている
\\	【動】~を後援[支持]する
\\	【副】最初は
\\	【副】最終的には
\\	【動】
\\	(人)を説得して~させる
\\	【動】~(意見など)に賛成する
\\	【動】~をすばらしい[すごい]と思う、~に感心する
\\	【名】不屈の努力
\\	【名】勇気
\\	【名】知恵
\\	【定】お世辞でもうれしいよ
\\	【前】~する時は
\\	【動】どうするか決める
\\	【動】~に頼る
\\	【定】
\\	ではなく
\\	【名】論理
\\	【名】本能
\\	【動】~(うまい考えなど)を考えだす
\\	【形】(考えなどが)独創的な
\\	【形】(人・判断などが)賢明な
\\	【名】(~に対する)解決策
\\	【副】直ちに
\\	【動】~(考え・計画など)を実行に移す
\\	【形】客観的な
\\	【副】~の見地[観点]からすると
\\	【名】
\\	前者
\\	【動】~よりも劣る
\\	【名】
\\	後者
\\	【動】(聞いたところ)~のようだ
\\	【形】すばらしい
\\	【副】理論的には
\\	【接】しかし(それでも)
\\	【副】(理論ではなく)実際には
\\	【動】
\\	が~することを期待する
\\	【動】~に勝つ、~を負かす
\\	【名】(論争・試合などの)相手、対抗者
\\	【動】(そうすべきなのに)~しない、~し損なう
\\	【動】~の期待に応える
\\	【副】皮肉にも
\\	【前】~にもかかわらず
\\	【名】(壮大な)試み、努力
\\	【名】(政治的・軍事的な)任務、使命
\\	【動】~(結果)に終わる
\\	【形】全くの
\\	【名】失敗
\\	【名】天賦の才能
\\	【動】~(マイナス状態)を埋め合わせる
\\	【名】欠乏、不足
\\	【形】(情報などが)直接得た
\\	【名】経験
\\	【動】~を所有する
\\	【名】(人の)能力
\\	【動】~を克服する
\\	【名】障害
\\	【名】(その時)~がなかったら[いなかった]なら
\\	【形】しっかりした、団結した
\\	【名】支持
\\	【名】取引
\\	【動】~していただろう
\\	【動】(交渉・計画などが)うまくいかない
\\	【動】
\\	のことで)~に感謝している
\\	【形】可燃性の
\\	【名】品目
\\	【名】(飛行機・船などの)客室
\\	【動】(法律・規制で)~(行為)を禁止する
\\	【名】(飛行機の)便
\\	【名】ウィーン
\\	【動】~を遅らせる
\\	【前】~(好ましくないこと)が理由で
\\	【形】(程度・重要度などが)小さな
\\	【名】事故
\\	【名】飛行機
\\	【動】(建物などが)爆発する
\\	【動】
\\	に)勢いよく[激しく]飛び込む
\\	【名】
\\	海
\\	【形】(飛行機などに)乗っている
\\	【副】昼夜休むことなく
\\	【動】~を調査する
\\	【名】原因
\\	【名】(飛行機の)墜落
\\	【副】不幸[不運]にも
\\	【形】(数が)ほとんどない[いない]
\\	【名】(乗り物に)乗る人、乗客
\\	【動】(~(事故など)で)生き残る
\\	【名】大惨事
\\	【名】航空
\\	【名】専門家、評論家
\\	【動】~を分析する
\\	【名】統計
\\	【副】詳細に
\\	【前】~の代表[代理]として
\\	【名】社員、職員(全体)
\\	【動】~を表現する
\\	【名】同情、共感
\\	【名】犠牲者、被害者
\\	【副】ある意味では
\\	【動】
\\	は)
\\	のせいだ
\\	【名】惨事
\\	【名】車両、乗り物
\\	【動】~を(綿密に)検査する
\\	【名】(機械などの)欠陥
\\	【形】~ごとに
\\	【動】~を聞いて[見て](とても)驚く
\\	【名】(公益サービスの)電気、ガス、水道
\\	【名】請求書
\\	【動】(合計が)~になる
\\	【動】~(水栓)を止める
\\	【名】{米}水道の栓
\\	【動】(興奮して)(~と)叫ぶ
\\	【副】激怒して
\\	【動】~をきつく締める
\\	【名】(箱・鍋などの)蓋
\\	【構】
\\	が~するように
\\	【動】(食べ物などが)腐る、いたむ
\\	【動】~(もの)を(無料で)借りる
\\	【名】はさみ
\\	【副】もちろん
\\	【定】(物を渡すときに)はいどうぞ
\\	【動】~を
\\	の)代わりに使う
\\	【形】透明な
\\	【名】接着剤
\\	【名】縫い針
\\	【名】(裁縫用)の糸
\\	【動】~を染める
\\	【名】織物、生地
\\	【動】縮む
\\	【動】~を処分する[捨てる]
\\	【名】革
\\	【動】~をすり切らす
\\	【名】(服)肘の部分
\\	【動】
\\	を
\\	に加える
\\	【名】小麦粉
\\	【名】混合物
\\	【動】~をかき混ぜる
\\	【形】濃い、とろみのある
\\	【動】~をどうにかする[扱う]
\\	【名】(食事の)残り物
\\	【名】冷蔵庫
\\	【副】とりあえず
\\	【名】電子レンジ
\\	【動】~(大金)をつぎ込む
\\	【副】考え直して
\\	【動】~することに決める
\\	【動】~(物・お金など)を取っておく
\\	【副】まさかの時のために
\\	【名】携帯電話
\\	【形】ちょっとした
\\	【名】必需品
\\	【動】~なしでやっていく
\\	【構】
\\	した途端に
\\	が起きる
\\	【動】ゆったりと座る
\\	【動】くつろぐ
\\	【名】
\\	雑用
\\	【動】~を(そっと)置く
\\	【動】(体を休めるために)横になる
\\	【動】~を(収納場所に)片付ける
\\	【名】もの、こと
\\	【副】(~の)邪魔になって
\\	【動】(~の)まねをして遊ぶ
\\	【名】恐竜
\\	【動】
\\	のことで)~(子供)を叱る
\\	【動】~に(ドンと)ぶつかる
\\	【名】棚
\\	【形】とても高価な
\\	【名】陶磁器
\\	【動】粉々になる
\\	【名】(割れた)破片
\\	【動】(万が一のことが起こらないよう)~から目を離さない、~を気をつけて見る
\\	【副】少しの間
\\	【動】(車で)~を迎えに行く
\\	【名】隣の住人
\\	【動】
\\	のことで)~(人)にがみがみ説教をする
\\	【名】散らかっていて汚い状態[もの]
\\	【名】{米}(家の周りの)庭
\\	【動】~に(外見などが)似ている
\\	【名】姉
\\	【副】~の点[意味]で
\\	【動】~(複数の人[もの])を見分ける
\\	【副】概して
\\	【名】双子の一方
\\	【動】(~と)共通点が多い[ない]
\\	【動】~にうんざりしている
\\	【動】食器を洗う
\\	【動】洗濯する
\\	【副】など
\\	【形】(一番の)お気に入りの
\\	【名】余暇の楽しみ
\\	【動】(リラックスして)ぶらぶらと歩く、散歩する
\\	【名】海岸、湖岸
\\	【形】(性格・雰囲気などが)楽しい、心地よい
\\	【動】(~で)疲れ切っている
\\	【動】一休みする
\\	【名】日陰
\\	【定】(勧誘の問に対して)もちろん
\\	【名】自動販売機
\\	【形】調子が悪い、故障中の
\\	【動】~(空腹・退屈など)で死にそうだ、たまらなく~だ
\\	【名】喉の渇き
\\	【動】気を失う
\\	【名】空腹
\\	【名】(倒れる寸前の)疲労
\\	【動】意識を回復する
\\	【副】しばらくして
\\	【名】{米}講堂
\\	【形】(中が)空の
\\	【前】~を除いては
\\	【名】[集合的に]家具
\\	【名】マンション
\\	【動】十分である、間に合う
\\	【接】~である限り
\\	【名】家賃、賃料
\\	【動】~の好みがうるさい
\\	【形】~歳の
\\	【動】~(財産)を相続する
\\	【名】不動産
\\	【定】~はここまでにしましょう
\\	【名】(当たり障りのない)世間話、雑談
\\	【動】(本気で)~(仕事など)に取りかかる
\\	【動】~に腹を立てる
\\	【形】強引な、押しの強い
\\	【名】販売員
\\	【動】(一方的に[無理に])
\\	に~させる
\\	【動】(~に)署名する
\\	【名】契約、契約書
\\	【形】かなりの
\\	【名】(結果や判断に影響を与える)要因、要素
\\	【動】
\\	に~する気をなくさせる
\\	【動】
\\	を)
\\	に)投資する
\\	【名】株、株式
\\	【動】~を輸入する
\\	【形】様々な
\\	【形】生の
\\	【名】(石油などの)原料
\\	【動】~を輸出する
\\	【動】(大規模に)~を製造する
\\	【名】商品、品物
\\	【前】(長い)~の過程に
\\	【形】経済の
\\	【名】発達、発展
\\	【動】~を見捨てる
\\	【形】伝統的な
\\	【名】
\\	価値観
\\	【動】~(大変なこと)に直面する
\\	【形】いまだかつてない
\\	【名】危機
\\	【副】要するに
\\	【名】目的
\\	【名】
\\	規制
\\	【動】~を保護する
\\	【形】国内の
\\	【名】産業
\\	【副】~する目的で
\\	【動】~を維持する
\\	【名】(製品などの)競争力
\\	【動】~に着手する
\\	【形】(変化などが)大幅な
\\	【名】規制緩和[撤廃]
\\	【副】【形】真剣に[な]
\\	【動】~に背を向ける
\\	【形】(人が)わらにもすがる思いの
\\	【動】(~の)危険[危機]にさらされている
\\	【名】議会
\\	【動】
\\	に
\\	を提供する
\\	【名】発展途上国
\\	【形】財政の、財政的な
\\	【名】(財政面・物資面での)援助
\\	【名】君主国、君主政治
\\	【動】~(制度・法律など)を廃止する
\\	【定】(~について)どう思う?
\\	【動】~を探し求める
\\	【名】(悪天候や爆撃などから身を守る)安全な場所
\\	【名】攻撃
\\	【動】隠れる
\\	【名】可能性、見込み
\\	【形】(注目に値する)大きな
\\	【動】
\\	で)~な進歩[進展]をする
\\	【名】軍備縮小
\\	【名】
\\	交渉
\\	【動】(ある状況が)進行中である
\\	【副】間もなく、近いうちに
\\	【動】~に入る
\\	【形】極めて重大[重要]な
\\	【名】(段階的に変化する)局面、段階
\\	【名】奴隷
\\	【動】~したいと切望している
\\	【動】~を(部分的に)変える
\\	【名】運命
\\	【名】貴族(の一人)
\\	【動】~を乱用[悪用]する
\\	【名】特権
\\	【動】思う存分に~する
\\	【名】革命
\\	【副】それ自体は[が]
\\	【動】~(果実)を実らす
\\	【動】(交渉などが)成果を生む
\\	【副】結局、やはり
\\	【動】(~(会話など)に)割り込む
\\	【名】会話
\\	【動】~(意見など)に反対である
\\	【動】~を無視する
\\	【動】~(順番など)を飛ばす
\\	【名】順番
\\	【副】わざと
\\	【動】(~を)いじめる
\\	【動】(~を)見て見ぬふりをする
\\	【動】~で不正行為[カンニング]をする
\\	【動】~を停職[停学]にする、つるす
\\	【前】~はと言うと
\\	【動】~の罪をまんまと逃れる
\\	【形】(~に対して)敵対的な
\\	【動】~を妬んでいる
\\	【名】富
\\	【名】(社会的な高い)地位
\\	【動】~が大嫌い
\\	【動】(~のように)振る舞う
\\	【接】まるで~であるかのように
\\	【名】すごい人、大物
\\	【前】~にもかかわらず、~なのに
\\	【名】ほめ言葉
\\	【動】顔をしかめる
\\	【動】顔をそむける
\\	【形】失礼な
\\	【動】(事実は)~だと主張[断言]する
\\	【形】莫大な
\\	【名】(集合的に)財産
\\	【形】~の自由にできる
\\	比較級 
\\	比較級 
\\	【接】~すればするほど~
\\	【形】頑固な
\\	【形】孤立した
\\	【動】話をじらす
\\	【動】(当然)~するものだと思われている
\\	【定】慌てずに待つしかないよ
\\	【定】時が経てば分かる
\\	【動】~(レポートなど)を提出する
\\	【名】(研究)課題、宿題
\\	【前】~を使って
\\	【定】~までに
\\	【定】とても~とは思えない、決して~ではない
\\	【形】勤勉な
\\	【接】(前の文を受けて)そういうわけで~だ
\\	【動】(~(科目・学科))を落とす
\\	【名】数学
\\	【名】(学位)論文
\\	【動】意味をなす
\\	【副】(文頭で)まず(第一に)
\\	【名】主題
\\	【形】曖昧な
\\	【名】小論文
\\	【形】簡潔な
\\	【形】要領を得た
\\	【動】~(問題など)に対処する、~を処理する
\\	【形】複雑な、込み入った
\\	【動】
\\	(仕事など)を
\\	(人)に任せる
\\	【定】朝飯前だよ
\\	【動】~(書類など)にざっと目を通す
\\	【動】{米}~を提出する
\\	【定】手が放せないんだ
\\	【副】今現在
\\	【定】もし~ならどうなるだろう?
\\	【動】~(基準・要求など)を満たす
\\	【動】期限に間に合わせる
\\	【動】きっと~だ
\\	【動】~(人)に腹を立てる[腹を立てている]
\\	【名】要約
\\	【動】申し分ない
\\	【前】~以外は
\\	【形】(程度などが)ひどい
\\	【名】(手書きの)字
\\	【助】~するべきである
\\	【動】~を(どうしたらいいのか)よく[慎重に]考える
\\	【名】(論理などの)前提
\\	【形】(理由・議論などが)妥当な
\\	【副】とりわけ
\\	【形】科学の
\\	【名】(専門)用語
\\	【動】~を必要とする
\\	【形】厳密な
\\	【名】定義
\\	【副】何よりもまず
\\	【名】(数学など)公式
\\	【動】~を暗記する
\\	【動】~を計算する
\\	【名】体積、容積
\\	【動】
\\	に
\\	を掛ける
\\	【名】縦
\\	【名】横
\\	【名】深さ
\\	【形】とても小さな
\\	【名】ひび
\\	【名】
\\	底
\\	【名】樽
\\	【名】(書物などの)章
\\	【動】~に焦点をあてる
\\	【名】概念、考え
\\	【名】幾何学
\\	【動】(~に)関心[興味]がある
\\	【形】(純粋に)学問的な
\\	【名】主題、テーマ
\\	【形】実用的な
\\	【名】(実用的な)技能[技術]
\\	【動】~(機械)の調子が悪くなる
\\	【名】顕微鏡
\\	【動】(専門家の人に)
\\	を~してもらう
\\	【動】~を(誰かに)修理してもらう
\\	【定】(ちょっと)見せて
\\	【動】~を修理する
\\	【名】(医療・計測用などの)器具
\\	【副】例えば
\\	【名】計器
\\	【前】(例えば)~とか[みたいな]
\\	【名】温度計
\\	【名】気圧計
\\	【形】極めて小さい[細かい]
\\	【名】粒子
\\	【副】何とか~(する)、かろうじて~
\\	【形】目に見える
\\	【名】肉眼
\\	【名】筋肉
\\	【名】(細胞からなる動植物の)組織
\\	【動】~から成る
\\	【形】膨大な
\\	【名】細胞
\\	【名】物質
\\	【副】主に
\\	【動】~から構成されている
\\	【名】水素
\\	【名】酸素
\\	【形】正確な
\\	【名】温度、体温
\\	【名】(温度・角度の)度
\\	【名】セ氏
\\	【動】(~を)見せびらかす[ひけらかす]
\\	【形】完璧な
\\	【名】(女性の)体型[スタイル]
\\	【動】~だったらなあ
\\	【形】(人が)細い、痩せた
\\	【動】~を羨む
\\	【動】ダイエットを始める[中である
\\	【動】(~に)恋をする
\\	【副】一目見ただけで
\\	【動】思い切ってやってみる
\\	【動】(~に)結婚を申し込む
\\	【動】~(人)のお願いごとを断る
\\	【副】偶然に
\\	【形】(人などの性質が)優しい
\\	【形】(人・機械などが)頭のいい
\\	【副】おまけに
\\	【形】独身の
\\	【名】甥
\\	【動】~(子供)を育てる
\\	【形】謙虚な
\\	【形】(他人に対して)思いやりのある
\\	【動】~を
\\	に)紹介する
\\	【名】
\\	両親
\\	【動】~が好きになる
\\	【副】すぐに
\\	【動】(~を)判断する
\\	【構】
\\	というよりもむしろ
\\	【名】
\\	(人・物)の外見、見た目
\\	【名】本当の自分、現在の自分
\\	【構】
\\	こそが~だ
\\	【名】肩書き
\\	【動】重要である
\\	【副】確かに
\\	【動】(~と)語気を強めて言う
\\	【動】~(人)が言うように
\\	【形】(人が)自らの努力で成功[出世]した
\\	【前】
\\	(という性質)を持つ~
\\	【名】先見の明
\\	【動】(社会的・経済的に)成功する
\\	【副】(過程はどうであれ)最後には
\\	【動】~を覚えておく
\\	【名】若さ
\\	【形】永遠の
\\	【動】(よく考えた末に)~を理解する
\\	【副】(疑問文中で)いったい
\\	【名】(~に対する)軽蔑
\\	【動】~せずにはいられない、ついつい~してしまう
\\	【動】~をばかにして笑う
\\	【動】~し続ける
\\	【形】ばかな
\\	【動】間違える、間違いをする
\\	【形】
\\	最も可能性の低い
\\	【動】教訓を得る、懲りる
\\	【副】さらに悪い事に
\\	【動】~を自覚している
\\	【動】~を不快な気分にさせる
\\	【接】~なので
\\	【動】慌てている
\\	【動】~を着る[体の一部につける]
\\	【名】(指が5本に分かれた)手袋
\\	【副】裏返しに
\\	【副】誤って、間違えて
\\	【形】デブの、太った
\\	【形】不器用な、(動きが鈍くて)どじな
\\	【動】(前方に大きく)跳ぶ、飛び越える
\\	【形】(水深・容器などが)浅い
\\	【名】小川
\\	【動】つまずく
\\	【動】~を捻挫する
\\	【名】足首
\\	【動】(~が)怖い
\\	【名】
\\	高所
\\	【名】臆病者、弱虫
\\	【形】(性格などが)消極的な
\\	【副】めったに~ない
\\	【動】自分を表現する
\\	【副】人前で
\\	【動】~(人)を納得させる[わからせる]
\\	【名】迷信
\\	【形】(考えなどが)根拠のない
\\	【動】(体などが)震える
\\	【名】恐怖
\\	【前】~のことを考えると
\\	【名】注射
\\	【形】怠惰な
\\	【副】(気持ちなどが)本当は
\\	【副】頻繁に
\\	【動】~(するべきこと)を怠る
\\	【名】(与えられた)仕事
\\	【形】職場にいて、仕事中の
\\	【形】家にいる
\\	【動】~に夢中になっている
\\	【形】ばかな、くだらない
\\	【名】(主婦を主なターゲットにした)連続メロドラマ
\\	【名】(テレビ・映画の)漫画
\\	【動】
\\	を~(程度)だと思う[評価する]
\\	【副】実際には
\\	【形】退屈な、(人を)うんざりさせる
\\	【動】
\\	(考え)が~(人)に(ふと)浮かぶ
\\	【動】~(真実・感情など)を隠す
\\	【動】(やましいことがない証として)
\\	の目を真っ直ぐ見る
\\	【名】忍耐(力)
\\	【動】(能力などが)尽きる
\\	【動】(~を)(勢いよく[素早く])つかむ
\\	【名】(服の)襟
\\	【動】~をののしる
\\	【動】(仕方なく)~を(事実だと)認める
\\	【動】(~と)口げんかする
\\	【副】たまに、時々
\\	【副】普通は
\\	【動】~と親しい間柄である
\\	【定】~してもいいですか?
\\	【動】(~に)立ち寄る
\\	【定】(相手の問に答えて)いや全然
\\	【定】遠慮しないで
\\	【動】(命令文で)(ちょっと)待ってて
\\	【定】(ここに)すぐ戻って来るよ
\\	【動】(電話を)切る
\\	【動】~(人)を泊める
\\	【副】(その理由の)一つには
\\	【名】お父さん
\\	【動】ふらっと(人に)立ち寄る
\\	【副】時々
\\	【動】じっと立つ[座る/横たわる]
\\	【動】(落ち着きなく)(~(狭い範囲)を)歩く
\\	【副】(2点間を)行ったり来たり
\\	【名】{米}歩道
\\	【動】(と)仲直りする
\\	【動】
\\	を当然のことと思う
\\	【動】終わる
\\	【副】永遠[永久]]に
\\	【定】本気で言っているんだよ
\\	【動】(相手の言ったことに対する軽い抗議のつもりで)おいおい
\\	【形】感情的な
\\	【動】~するつもりである
\\	【動】~の心を傷つける
\\	【動】~についてよく話し合う
\\	【動】~を裏切る
\\	【動】
\\	のことで)~(人)を許してあげる
\\	【動】~(人)に(迷惑などの)埋め合わせをする
\\	【動】(~(人)に)仕返しをする
\\	【動】(~(罪)を)(牧師に)告白する
\\	【名】(宗教上・道徳上の)罪、過ち
\\	【動】~を隠す
\\	【動】約束を守る
\\	【副】(命令・提案などの後で)そうしなければ
\\	【動】(不本意ながら)(~する)結果になる
\\	【動】評判を落とす
\\	【動】~を(あくまでも)守る
\\	【形】道徳的[倫理的]な
\\	【名】主義、信条
\\	【名】誠実、正直
\\	【動】(努力などが)報われる
\\	【副】(長い目で見れば)最後には
\\	【動】元気を出す
\\	【動】~は仕方がない
\\	【動】最善[ベスト]を尽くす
\\	【動】~をくよくよ[長々と]考える[話す]
\\	【名】
\\	過去
\\	【動】一からやり直す[出直す]
\\	【定】お願いがあるんだけど
\\	【動】
\\	(人)を(~まで)乗せて行ってあげる
\\	【名】(四角い)広場
\\	【定】ちょっと無理かなあ
\\	【動】とても忙しい
\\	【副】いつも通り
\\	【動】(待っていた人が)現れる
\\	【副】時間通りに
\\	【形】時間を遵守する
\\	【副】この前
\\	【動】~を使い果たす
\\	【動】{米}ガソリン
\\	【前】~の真ん中に[で]
\\	【形】(通り・場所が)にぎやかな、交通量が多い
\\	【名】(州と州を結ぶ)幹線道路、インターステイト
\\	【副】(位置が)~の前方に
\\	【動】(車が)スリップする
\\	【動】ひっくり返る
\\	【動】~をばらまく[散らす]
\\	【名】(運ばれる)積み荷
\\	【動】~を思い出す
\\	【名】(移動するもの同士の)衝突
\\	【動】~を避ける
\\	【形】全くの
\\	【名】幸運
\\	【前】~の場合には[際には]
\\	【名】緊急事態
\\	【動】(~と)連絡を取る
\\	【名】代理人
\\	【副】すぐに、ただちに
\\	【動】向かう
\\	【動】(道に)迷う[迷っている]
\\	【形】間違った
\\	【名】方向
\\	【動】(車を)路肩に寄せて止める
\\	【形】さびた
\\	【動】(機械などが)故障する
\\	【動】~の進路を妨げる
\\	【名】{米}交差点
\\	【形】(粒などが)細かい
\\	【名】(先が尖ったものでの)引っかき傷
\\	【名】(衝撃を受けてできた小さな)へこみ
\\	【名】表面
\\	【定】~しても無駄だ
\\	【動】(柔らかい布などで)~を磨く
\\	【動】(物が)動かなくなる[動かない]、はまる[はまっている]
\\	【名】交通
\\	【名】交通渋滞
\\	【動】冷静になる
\\	【動】近道をする
\\	【動】~につく
\\	【副】間に合って、予定時刻までに
\\	【動】タイヤがパンクする
\\	【動】~を手助けする
\\	【定】ええ、喜んで
\\	【名】救急車
\\	【形】(機械などが)制御がきかない
\\	【動】危うく~しそうになる
\\	【動】(車などが)~をひく
\\	【名】歩行者
\\	【動】
\\	~だと力説する
\\	【形】原始的な
\\	【名】(集合的に)生物
\\	【動】(空想や噂ではなく)実在する、存在する
\\	【名】火星
\\	【名】乗組員[乗務員](全体)
\\	【動】忙しく~している、~するのに忙しい
\\	【動】~のための準備をする
\\	【名】航海
\\	【名】(大気圏外の)宇宙(空間)
\\	【名】宇宙
\\	【動】~(調査など)を実施する
\\	【形】一連の~
\\	【名】(科学的な)実験
\\	【名】重力
\\	【名】天文学者
\\	【動】
\\	(確証がないが)~だと考える
\\	【名】
\\	宇宙
\\	【動】膨張する
\\	【副】無限に
\\	【名】宇宙飛行士
\\	【動】~に挨拶する、~を迎える
\\	【形】(行動・感情などが)自然に生じる
\\	【名】拍手、称賛
\\	【名】基金
\\	【動】~(委員会・組織など)を設立[設置]する
\\	【前】(将来)~するつもりで
\\	【動】~を(現状のまま)保存[保護]する
\\	【形】(動植物などが)絶滅の危機に瀕した
\\	【名】
\\	地球
\\	【形】極めて大きい
\\	【名】洪水
\\	【動】~を麻痺させる
\\	【形】現地の
\\	【名】交通[輸送]機関
\\	【名】~網
\\	【名】
\\	赤道
\\	【動】
\\	を
\\	に分ける
\\	【名】
\\	地球
\\	【名】半球(体)
\\	【形】残酷な
\\	【名】~(動物)を罠で捕らえる
\\	【名】毛皮製品
\\	【名】団体
\\	【形】最も重要な、主要な
\\	【動】
\\	で~な役割を果たす
\\	【名】[集合的に]野生動物[動物・植物]
\\	【名】保存、保護
\\	【形】(ものが)壊れやすい、(精神的・肉体的に)弱い
\\	【名】(生物の)種
\\	【動】~の寸前である
\\	【名】(生物などの)絶滅
\\	【形】長引く
\\	【名】干ばつ
\\	【形】(被害などが)ひどい
\\	【動】
\\	に~(程度)の損害[被害]を与える
\\	【名】収穫(量)
\\	【名】熱帯雨林
\\	【名】森林
\\	【動】消える
\\	【副】~の規模
\\	【副】一部は
\\	【前】~が原因で
\\	【名】酸性雨
\\	【定】~だと言われている
\\	【名】地球温暖化
\\	【副】(関係が)直接
\\	【動】
\\	を
\\	に[と]関連づける
\\	【名】二酸化炭素
\\	【名】(ガス・熱などの)排出
\\	【名】破壊
\\	【名】オゾン
\\	【名】層
\\	【動】~に影響を及ぼす
\\	【名】
\\	(自然)環境
\\	【名】委員会
\\	【動】
\\	に~(援助など)するように求める
\\	【副】協力し合って
\\	【動】~を抑制する
\\	【名】公害、汚染
\\	【名】人工衛星
\\	【動】~(ロケットなど)を発射する
\\	【名】(惑星などの)軌道
\\	【動】~(の動向[反応])を(常に)監視する
\\	【動】(氷などが)溶ける
\\	【名】氷河
\\	【名】生態学者
\\	【動】(~(人)に)警告する
\\	【副】(しつこく)何度も何度も、繰り返し
\\	【名】石油
\\	【定】
\\	だけでなく
\\	も
\\	【名】天の恵み
\\	【名】災いのもと
\\	【名】大陸
\\	【動】~が豊富である
\\	【名】化石
\\	【名】化石燃料
\\	【名】地方、地域
\\	【副】比較的
\\	【動】~が豊富である
\\	【形】鉱物の
\\	【名】[通例-
\\	資源
\\	【名】小麦
\\	【動】~(割合)を占める
\\	【副】ほぼ、およそ
\\	【名】3分の2
\\	【形】農業の
\\	【名】生産高、生産量
\\	【形】肥沃な
\\	【名】土、土壌
\\	【動】~に不可欠である
\\	【名】収穫高[量]
\\	【形】湿度が高い
\\	【名】気候
\\	【動】~の特徴である
\\	【名】半島
\\	【形】活動している
\\	【名】火山
\\	【動】噴火する
\\	【形】定期的な
\\	【副】~の間隔で
\\	【名】運河
\\	【動】
\\	と
\\	をつなぐ[結ぶ]
\\	【名】
\\	大西洋
\\	【名】
\\	太平洋
\\	【形】地理学上の、地理的な
\\	【名】特徴、特色
\\	【動】~に似ている
\\	【代】~のそれ
\\	【名】(日本・フランスなどの)県、府
\\	【副】たいてい
\\	【名】飢饉
\\	【動】~(物事)が付随して起こる
\\	【名】疫病
\\	【形】壊滅的な
\\	【名】地震
\\	【動】(地震・台風などが)~を襲う
\\	【名】(米国などの)州
\\	【名】首都
\\	【名】住民
\\	【形】家のない
\\	【動】(~に)気をつけろ
\\	【名】天井
\\	【動】崩れ落ちる
\\	【名】天気
\\	【名】予報
\\	【名】台風
\\	【動】~しそうだ
\\	【動】加速する
\\	【動】~に近づく
\\	【名】沿岸(地区)
\\	【定】(どうなるかが)決まっていない
\\	【動】~を中止する
\\	【定】(不愉快・不関心を表して)それが何だって言うんだ?
\\	【動】(~にとっては)どちらでも同じ
\\	【形】霧が濃い
\\	【定】(返答で)全く、本当だ
\\	【定】たぶん~だ
\\	【動】(天気が)よくなる、晴れる
\\	【副】この[その]後に
\\	【副】大部分は、ほとんど、たいてい
\\	【名】近代化
\\	【動】
\\	を
\\	と同じものだと考える
\\	【名】西欧化
\\	【形】都市の、都会の
\\	【名】文化
\\	【動】~のように見える
\\	【形】魅力的な
\\	【副】とりわけ
\\	【動】
\\	は~の略[頭文字]である
\\	【形】人工の
\\	【名】知能
\\	【名】(形容詞的に)家庭の~
\\	【名】電化製品
\\	【形】(ものが)便利な
\\	【副】日々、日に日に
\\	【名】収入
\\	【形】(ある目的には)十分な
\\	【動】~(家庭)を養う
\\	【名】4人家族
\\	【動】~(よくない状況)の中で何とかやっていく
\\	【名】人気
\\	【名】ホームページ、ウェブサイト
\\	【動】~次第である
\\	【名】(形式・表向きに対して)内容、中身
\\	【動】~しなきゃ
\\	【動】~に(遅れずに)ついて行く
\\	【接】いったん~してしまえば
\\	【動】(~に)遅れを取る
\\	【動】(~に)追い付く
\\	【名】(科学)技術
\\	【副】急速に
\\	【動】進歩する
\\	【形】(対処が)難しい
\\	【動】~(速い変化など)に遅れずについていく
\\	【形】ますます多くの
\\	【動】焦る、焦ってする
\\	【動】~を利用[活用]する
\\	【形】双方向性の
\\	【名】性質、自然
\\	【名】媒体
\\	【名】(技術などの)革新(をもたらすこと)
\\	【動】
\\	と(~程度の)関係がある
\\	【名】能力
\\	【動】(~に)気付く
\\	【形】普通ではない、まれな
\\	(複:
\\	【名】現象
\\	【形】たくさんの
\\	【動】~の真似をする
\\	【形】優れた、すばらしい
\\	【名】(一般的に確立された)方法
\\	【形】(場所が)荒涼とした
\\	【名】山間の谷、盆地
\\	【動】
\\	を
\\	へと)一変させる
\\	【形】(事業が)繁栄している
\\	【名】中心(地)、中核
\\	【形】(余るくらい)たくさんの
\\	【名】余地、余裕
\\	【名】改良、改善
\\	【名】発明品
\\	【動】~よりも優れている
\\	【形】従来の
\\	【名】(集合的に)機器、用具
\\	【副】~の点に関して
\\	【名】(人・物の)長所、強み
\\	【名】会社、企業、事務所
\\	【動】
\\	は
\\	に起因する
\\	は
\\	がもたらしたもの]と考える
\\	【形】唯一の、他にはない
\\	【形】~志向の
\\	【名】戦略
\\	【動】
\\	(機能など)を
\\	に)統合させる
\\	【名】範囲
\\	【名】(機械などの)機能
\\	【形】スゴイ
\\	【名】(特殊な)機器
\\	【動】流行る、人気を博す
\\	【副】全米で
\\	【名】道具
\\	【動】(いざという時に)役に立つ
\\	【副】【形】(時間・空間的に)すぐ近くに[の]
\\	【副】一般的に
\\	【名】消費者
\\	【動】
\\	よりも
\\	のほうを選ぶ[好む]
\\	【名】量
\\	【名】質
\\	【副】【形】正反対に[な]
\\	【副】明らかに
\\	【名】広告
\\	【動】(活動などが)~を狙いとしてる
\\	【名】10代の若者
\\	【名】工場
\\	【形】建設中の
\\	【動】~を組み立てる
\\	【名】ビデオデッキ
\\	【名】ユニット、(構成)単位
\\	【前】~当たり
\\	【形】効率的な
\\	【名】(集合的に)機械類
\\	【動】~に取って代わる
\\	【形】(作業などが)手を使う、肉体的な
\\	【名】労働
\\	【副】当時は
\\	【形】巨大な
\\	【名】大企業
\\	【形】事実上の
\\	【名】独占
\\	【形】国内の
\\	【名】商取引、商業
\\	【構】
\\	が
\\	なので
\\	【名】[-通例
\\	(設けられた)制限
\\	【動】~を取り除く
\\	【名】企業
\\	【動】誕生する
\\	【前】~の結果
\\	【形】ばかげた
\\	【名】(新規)事業
\\	【動】(~という)危険な状態にある
\\	【動】倒産[破産]する
\\	【前】~と比べると
\\	【形】(時間・順序的に)前の~、前回の~
\\	【動】(景気などが)上向く
\\	【前】~の点で(は)
\\	【動】儲ける、利益を得る
\\	【動】~を
\\	から)
\\	に転任[転籍]させる
\\	【形】海外の
\\	【動】支店
\\	【動】~がいなくて寂しい
\\	【動】~に(ちょっとした)手紙を書く
\\	【動】(外出で)(~に)いない
\\	【副】仕事で
\\	【動】~を出産する
\\	【副】どうやら[見たところ]~のようだ
\\	【動】~してもいいと思う
\\	【動】~(仕事・責任など)を(さらに)引き受ける
\\	【名】(課せられた大変な)任務、仕事
\\	【動】(行為が負担など)~を増やす[大きくする]
\\	【名】負担
\\	【動】~するつもりである
\\	【動】~を(外食などに)連れて行く、~(異性)とデートする
\\	【動】残業する
\\	【動】~を
\\	ぐらいだと)見積もる
\\	【名】損失
\\	【副】遅くとも~までに
\\	【動】(~を)何とかする
\\	【副】(あれこれ)何とかして
\\	【動】~に頼る
\\	【動】必ず[きちんと]~する
\\	【動】~に感謝する
\\	【定】どういたしまして、いいんですよ
\\	【動】(~の)ふりをする
\\	【形】熱心な
\\	【名】上司
\\	【動】(自分の)近く[そば]にいる
\\	【動】昇進する
\\	【前】~を犠牲にして
\\	【名】(仕事の)同僚
\\	【動】~を恥ずかしく思う
\\	【名】景気後退
\\	【動】(好ましくないことが)始まる
\\	【形】一時的な
\\	【名】臨時社員[短期契約社員]
\\	【動】~を(一時的に)解雇する
\\	【副】次々と
\\	【副】(昔と違って)今日では
\\	【動】失業中である
\\	【副】自らの意思ではなく
\\	【動】~の責任を負うべきである
\\	【形】悲観的な
\\	【動】~を心配している
\\	【動】~を心配している
\\	【形】オフィス労働の
\\	【動】~(大変なこと)に直面する[している]
\\	【名】(経済的)困窮、苦難
\\	(通例 
\\	【名】当局
\\	【動】賢明に努力する
\\	【副】無駄に
\\	【動】~を安定させる
\\	【名】通貨
\\	【動】~という結論に至る
\\	【構】~するよりも
\\	なことはない
\\	【形】緊急の
\\	【動】~を減らす
\\	【形】巨大な
\\	【名】赤字
\\	【名】アンケート用紙
\\	【動】~を配る[配布する]
\\	【副】無作為に
\\	【定】~するだけでいい
\\	【動】~(用紙)に必要事項を記入する
\\	【名】余白、空欄
\\	【動】(~を)ブロック体[楷書]ではっきり書く
\\	【副】黒インクで
\\	【前】~に加えて
\\	【名】割引
\\	【動】~を選ぶ
\\	【形】自由に選べる
\\	【名】日帰りツアー
\\	【副】無料で
\\	【系】{米}往復の
\\	【名】運賃、(乗り物の)料金
\\	【名】目的地
\\	【形】【副】以下の通り(の)
\\	【定】~するのを忘れないで
\\	【動】~を確認する
\\	【名】予約
\\	【副】事前に
\\	【定】必ず~しなさい
\\	【動】(空港で)搭乗手続きをする
\\	【副】(数量が)少なくとも
\\	【前】~より前に
\\	【名】出発
\\	【動】
\\	に)~を取り付ける
\\	【名】(紐などで付ける)札
\\	【名】[集合的に](旅行時の)荷物(全部)
\\	【動】(~から)(自然に)外れる[取れる]
\\	【副】平均すると
\\	【名】{米}小包
\\	【動】~の重さがある
\\	【名】(重量単位)ポンド
\\	【形】ビニール製の
\\	【名】ゴミ
\\	【名】袋、カバン
\\	【動】~(いやなもの)とは無縁である
\\	【形】(化学物質などが)(人体に)有害な
\\	【名】化学物質[薬品]
\\	【名】{米}懐中電灯
\\	【形】(明るさが)薄暗い
\\	【名】(乾)電池
\\	【定】~はどれくらいの頻度ですか?
\\	【名】(飛行機・バスなどの)往復[シャトル]便
\\	【形】{米}(街の)中心地の~、~の市街
\\	【名】モーテル
\\	【動】(施設などが)~(人数)分の十分なスペースがある
\\	など)	【副】~もの…
\\	【名】スイートルーム
\\	【定】
\\	よりも3倍大きい
\\	【名】{米}(分譲)マンション[共同住宅]
\\	【動】(~に)ゴミを投げ捨てる
\\	【名】施設
\\	【動】~を受ける場合がある
\\	【形】最大[最高]の
\\	【名】罰金
\\	【動】(店などが)込んでいる
\\	【動】~した方がいい
\\	【動】予約を取る[する]
\\	【副】事前に
\\	【動】(~を)中断する
\\	【名】食事
\\	【定】相談したいことがあるんだけど
\\	【副】内密に
\\	【動】(後で自分の方から)~(人)に連絡し直す[会いに行く]
\\	【名】料理人
\\	【動】~にたいへん驚く
\\	【形】信じられない
\\	【名】食欲
\\	【定】何が起こっている[行われている]んだい?
\\	【形】(人が)好奇心旺盛な、知りたがっている
\\	【定】(考えたけれど)よくわかりません
\\	【形】(人が)興奮した
\\	【名】(有名人などの)ファン
\\	【動】(~(場所))ぶらぶらする、(集団が)(~(場所)に)群がる[たむろする]
\\	【形】裏の
\\	【名】出口
\\	【動】~を一瞬目にする
\\	【前】~するとすぐに
\\	【動】~に(偶然)出会う
\\	【名】有名人
\\	【動】~を求める[ねだる]
\\	【名】(有名人などの)サイン
\\	【副】海外に[で]
\\	【動】~(病気など)で苦しむ
\\	【名】時差ボケ
\\	【名】下痢
\\	【副】ちょっと、何か
\\	【形】目まいがする
\\	【動】~したいな、~したい気がする
\\	【動】吐く
\\	【形】顔色が悪い、(血の気が引いて)青白い
\\	【名】鎮痛剤
\\	【動】
\\	(人)の
\\	(病気)を治す
\\	【名】~痛
\\	【副】(驚くほど)すぐに
\\	【定】もしかしたら~かなあ
\\	【動】~(風邪など)にかかる、~で体調を崩す
\\	【動】~(日数)の休みを取る
\\	【名】(食料などの)成分、原料
\\	【名】(水以外の)飲み物
\\	【形】(~に対して)有害な
\\	【副】特に、とりわけ
\\	【形】妊娠している
\\	【名】栄養摂取
\\	【形】極めて重要な
\\	【名】乳児、幼児
\\	【名】成長、発展
\\	【形】(程度・量が)適度な
\\	【名】(体の)運動
\\	【動】~を刺激する
\\	【名】循環
\\	【名】血
\\	【名】医師、内科医
\\	【動】
\\	に~するよう助言する[忠告する]
\\	【動】~を控える
\\	【名】酒、アルコール
\\	【副】当分の間は
\\	【助】本当に(~である)、ぜひ(~する)
\\	【動】体調を良くする
\\	【動】体重が減る
\\	【動】(自分の意思で)~しに行く、~を楽しむ
\\	【動】風邪をひく
\\	【形】(病気・けがなどが)ひどい
\\	【動】(遅くまで)夜更かしする
\\	【名】外科医
\\	【動】
\\	(人)を説得して~させる
\\	【動】~(苦痛を伴うこと)を経験する[受ける]
\\	【名】(動植物の)器官、臓器
\\	【名】移植(手術[治療])
\\	【形】(病気・高齢で)寝たきりの
\\	【動】(~を)交代でする
\\	【動】~の面倒を見る
\\	【名】仮説
\\	【形】過度の
\\	【名】量
\\	【名】ジャンクフード
\\	【動】~の一因となる
\\	【名】(避けられない)死ぬべき運命
\\	【名】政府
\\	【形】(行為などが)ふさわしい、適切な
\\	【動】(~な)手段を取る
\\	【動】
\\	に~させない、
\\	が~することを(事前に)防ぐ
\\	【名】(病気・ウイルスの)伝染、感染
\\	【動】(病気・情報などが)広がる
\\	【名】大臣
\\	【動】(ルール・義務なので)~せざるを得ない、~を余儀なくされる
\\	【動】(~を)辞職する[辞任する]
\\	【名】内閣
\\	【動】
\\	を
\\	だと考える
\\	【形】現在の
\\	【形】強制的な、義務の
\\	【名】教育
\\	【形】時代遅れの
\\	【動】
\\	に~するように励ます
\\	【動】~を果たす[実現させる]
\\	【名】(人が感心するぐらい)大きな目標、夢
\\	【動】~(夢・計画など)を実現させる
\\	【動】(~を)退学する
\\	【動】(苦労して[少しずつ])進む
\\	【定】普段は何をなさっているのですか?
\\	【名】大学
\\	【名】大学1年生
\\	【名】(大学での)専攻科目
\\	【名】(学問の)哲学
\\	【名】授業料
\\	【動】~(程度・料金など)を高くする
\\	【副】ほとんど、ほぼ
\\	【名】(単位)パーセント
\\	【前】~(特定の日付)より[から]
\\	【動】
\\	と
\\	の)どちらにしようか悩んでいる
\\	【動】~(仕事・証明書など)を申し込む
\\	【名】奨学金
\\	【動】(あれこれ考えずに)やってみる
\\	【接】~するとすぐに
\\	【動】(~(高校・大学)を)卒業する
\\	【動】(~(もの)を)引き継ぐ
\\	【動】~(職種)に向いている
\\	【名】
\\	軍隊
\\	【前】~の理由で、~が原因で
\\	【形】(規則などが)厳格な
\\	【名】規律
\\	【動】~すると決心[決意]している
\\	【動】生計をたてる
\\	【名】脚本家
\\	【副】どんな犠牲を払っても
\\	【動】大人になる
\\	【動】~できるようになる[するようになる]
\\	【動】~を総合的な視野でとらえる
\\	【名】体制・社会通念に従うこと
\\	【形】(~に)不可欠な、本質に関わる
\\	【名】(基本的な)要素、元素
\\	【形】均質な、同種のものだけで構成された
\\	【名】(地域)社会
\\	【構】(選べるなら)
\\	するより)むしろ
\\	したい
\\	【動】我が道を行く
\\	【動】ひたすら[あくせく]働く
\\	【動】(~に)(ぴったり)合う[はまる]
\\	【名】(決まった)型
\\	【動】(行為などが)(~に)何の進歩[成果]ももたらさない
\\	【動】~(親など)に似ている
\\	【名】人柄、性格
\\	【副】(より明確な理由を加えたい時に)だって、つまり
\\	【形】頑固な
\\	【動】生まれつき~である
\\	【動】(~に)降参する[屈服する]
\\	【形】(機が)熟した
\\	【形】大胆な
\\	【動】~を手に入れようと頑張る
\\	【動】~(困っている友達など)を見捨てない
\\	【動】~(人・権利)を守る、~のために立ち上がる
\\	【動】~が正しいこと[大切なこと]だと信じる
\\	【動】(~に)屈する
\\	【名】脅威、脅迫
\\	【定】~でありさえすればいいのになあ
\\	【動】~(困窮など)を克服する
\\	【名】(性格・性質上の)弱点、短所
\\	【動】それは~(人)次第だ
\\	【動】~(有利な状況)を最大限利用する
\\	【形】珍しい、めったにない
\\	【名】機会、チャンス
\\	【動】~(事実など)をはっきり示す、~を実証する
\\	【名】(~の)才能
\\	【動】~をがっかりさせる
\\	【動】~を信頼している
\\	【動】(人が困難を乗り越えて)成功する
\\	【名】観客
\\	【動】~を感動させる、~の涙を誘う
\\	【動】(動きなどが)優雅な
\\	【名】演技、公演
\\	【動】~を受ける[受け取る]
\\	【名】建築(学)
\\	【形】権威のある
\\	【名】賞
\\	【前】~をものともせず
\\	【名】逆境
\\	【動】~を成し遂げる
\\	【形】並外れた
\\	【名】偉業
\\	【動】~を受けるに値する
\\	【名】称賛(の言葉)
\\	【名】(集合的に)聴衆
\\	【動】~に感銘[感動]を与える
\\	【形】(話などが)説得力のある
\\	【名】講義、講演
\\	【形】庶民の
\\	【名】(人の)育ち、経歴
\\	【動】~を成し遂げる
\\	【形】世界的な
\\	【名】名声
\\	【動】(長時間かけて)~をためる[蓄積する]
\\	【形】(驚くほど)大きな、莫大な
\\	【名】財産
\\	【形】戦後の
\\	【名】時代
\\	【助】かつて~だった
\\	【形】裕福な
\\	【形】気前のよい
\\	【動】その日暮らしをする
\\	【副】決して~しない[ではない]
\\	【動】~に満足する
\\	【名】見返り、報酬
\\	【形】自己中心的な
\\	【形】欲の深い
\\	【動】
\\	(人)に
\\	(金額)の借りがある
\\	【副】全部で
\\	【動】~に
\\	(金額)を)返す
\\	【動】(一時的に)お金に困っている
\\	【定】(あきれて)またそれだ、また始まったよ!
\\	【副】(文を修飾して)(~が)驚いたことに(だ[だった])
\\	【名】(多額の)借金
\\	【動】(合計が)~に達する
\\	【形】(数量・程度などが)かなりの
\\	【名】金額
\\	【動】~にしがみつく
\\	【名】栄光、名誉
\\	【動】~を置き去りにする
\\	【動】~から逃げる
\\	【形】(環境などが)厳しい
\\	【名】現実
\\	【動】~に順応する
\\	【名】(経済的)繁栄
\\	【動】続く
\\	【副】永遠に
\\	【副】つまり
\\	【動】(長く続いてきたことが)終わる
\\	【副】いつかは、近いうちに
\\	【形】(常識的に見て)ふさわしい、適切な
\\	【名】資格、能力
\\	【動】~が必要である
\\	【名】(社内の)ポスト、仕事
\\	【名】コンピューターが使えること
\\	【動】絶対条件である
\\	【動】~を同封する
\\	【名】履歴書
\\	【名】封筒
\\	【動】~(案・申請書など)を
\\	に)提出する
\\	【名】社員(全体)、職員(全体)
\\	【名】(企業・大学などの)部
\\	【形】楽観的な
\\	【動】うまくいく
\\	【副】(程度が)よくても、せいぜい
\\	【形】(可能性などが)五分五分の
\\	【名】登録
\\	【名】(定型の)用紙
\\	【動】~を入手する
\\	【副】無料で
\\	【動】~に電話する
\\	【動】~に参加する
\\	【名】
\\	教室、講習会
\\	【形】なお一層の
\\	【名】問い合わせ、質問
\\	【動】遠慮なくどうぞ
\\	【動】~に(電話・手紙などで)連絡を取る
\\	【形】フリーダイヤルで[の]
\\	【尾】~ゼロの、非~の
\\	【定】最近どう?調子はどうだい?
\\	【定】最高だよ
\\	【動】(仕事・試験などで)うまくいく
\\	【名】仕事、職
\\	【名】(就職などの)面接
\\	【定】よくやった、おめでとう
\\	【名】
\\	指示、取扱説明書
\\	【名】研究生
\\	【動】~が(何が何だか)さっぱりわからない
\\	【副】同時に、一度に
\\	【動】無理だ
\\	【副】一人一人、一つ一つ
\\	【動】~の邪魔をする
\\	【定】おせっかいはやめろ
\\	【動】(~と)ささやく
\\	【動】しつこく~する
\\	【動】~を悩ませる[困らせる]
\\	【動】かっとなる
\\	【構】たいへん
\\	なので
\\	【形】子供じみた
\\	【動】(~(欲求など)を)我慢する
\\	【名】誘惑
\\	【動】(~のことで)騒ぎ立てる
\\	【動】~をいらいらさせる
\\	【動】(~と)叫ぶ
\\	【定】~するのは無駄だ[意味がない]
\\	【動】(~に)口答えする
\\	【名】独裁者、ワンマンな人
\\	【副】言ってみれば、例えて言えば
\\	【形】(人・言動が)礼儀正しい
\\	【副】~のいる前で
\\	【副】実際には
\\	【動】~の陰口を言う
\\	【動】~を軽蔑する[見下す]
\\	【副】それどころか
\\	【動】~を尊敬する
\\	【形】絶え間のない
\\	【名】侮辱
\\	【動】~(怒り・関心など)を呼び起こす
\\	【名】怒り
\\	【動】(~に)うんざりしている
\\	【動】~を仕分け[整理]する
\\	【動】~を注ぐ
\\	【動】(~すると)決心する
\\	【動】(~(仕事)を)辞める
\\	【副】現実には、実際には
\\	【動】~することに苦労している
\\	【動】(収入の範囲内で)何とかやりくりする
\\	【動】(~(金額)で)何とか生活していく
\\	【形】女性の
\\	【名】事務員
\\	【動】
\\	が~するのももっともだ
\\	【動】~について不満[苦情]を言う
\\	【名】日常業務
\\	【副】全然~ない
\\	【形】(困難だが)やりがいのある
\\	【前】~のおかげで
\\	【形】(変化などが)抜本的な
\\	【名】事業の再構築、(組織などの)構造改革
\\	【名】(企業などの)黒字
\\	【動】(数量が)増大する
\\	【尾】~倍の[に]
\\	【動】
\\	を
\\	だと考える[みなす]
\\	【名】忠誠
\\	【形】ばかげた
\\	【名】(漠然とした[根拠のない])考え
\\	【動】(電車などで)通勤する
\\	【副】~(遠い所)からはるばる[わざわざ]
\\	【形】遠い
\\	【名】郊外
\\	【動】~には(ほとほと)うんざりだ
\\	【動】
\\	(時間)を~に費やす
\\	【名】
\\	(~の)残りの部分
\\	【動】~(人・理想など)を追いかける
\\	【名】理想
\\	【名】(定年)退職、引退
\\	【動】
\\	(時間・労力など)を
\\	に捧げる
\\	【動】~の世話をする
\\	【名】孤児
\\	【前】~の(利益・保護の)ために
\\	【形】(生活に)困っている
\\	【動】協力する
\\	【動】~を集める[収集]する
\\	【名】寄付(金)
\\	【名】慈善団体
\\	【動】
\\	の名前を
\\	から取る
\\	【動】~をただであげる、~を寄付する
\\	【副】約、およそ
\\	【名】10億
\\	【名】自叙伝
\\	【動】~(新製品)を発売する
\\	【動】(商品・チケットなどが)完売する
\\	【動】
\\	を
\\	と呼ぶ
\\	【形】高潔さ、誠実さ
\\	【動】~(官職)に立候補する
\\	【名】市長、町長
\\	【名】(立)候補者
\\	【動】(人が)がっかりする
\\	【名】(終わってみて初めてわかる)結果、結末
\\	【名】選挙
\\	【形】市[町]の
\\	【名】議会
\\	【動】~に(神経など)集中する
\\	【形】具体的な
\\	【名】(公の場で議論されるべき)問題(点)
\\	【定】~ということは事実だ
\\	【動】~(収入など)で生活する
\\	【名】年金
\\	【動】
\\	に~するように強いる
\\	【動】~な生活を送る
\\	【動】~を(全体から)除く
\\	【名】データ
\\	【定】~と言うほどに遠い
\\	【形】正確な
\\	【動】~する傾向がある
\\	【動】
\\	から
\\	を連想する
\\	【名】政治家
\\	【名】偽善
\\	【副】第一に、何よりもまず
\\	【動】(人が)~に関心がある
\\	【動】~(人など)から搾取する
\\	【動】~(質・価値など)を高める
\\	【名】生活水準
\\	【動】~(権力など)を行使する
\\	【形】(立場などが)支配的な
\\	【名】(~に対する)影響(力)
\\	【形】保守的な
\\	【名】党、政党
\\	【動】弱まる
\\	【名】報道担当者
\\	【動】(~を)(きっぱりと)断る
\\	【動】(~について)コメントする[見解を述べる]
\\	【形】調査を受けている、調査中の
\\	【名】
\\	警察
\\	【動】(陰で)糸を引く、操る
\\	【副】裏で、密かに
\\	【名】役員、取締役
\\	【動】(~(人)を)もてなす[接待する]
\\	【名】官僚
\\	【接】すなわち[言い換えれば]
\\	【形】高級な、排他[閉鎖]的な
\\	【副】(~の)お返しとして[見返りとして]
\\	【名】秘書
\\	【副】きっぱりと
\\	【動】~したことはないと言う
\\	【動】~(情報など)を漏らす
\\	【形】内密の、機密の
\\	【動】(否定文で)思い切って[勇気を出して]~する
\\	【動】~(人の)意見を否定する
\\	【代】彼、彼女の(/は
\\	の意味)
\\	【名】雇用主
\\	【前】~を恐れて
\\	【動】~をクビにする[解雇する]
\\	【動】~を犠牲にする
\\	【形】前途有望な
\\	【名】(社会人としての)経歴
\\	【動】~を維持する
\\	【名】(人間としての)尊厳
\\	【動】
\\	するくらいなら
\\	するほうがましだ
\\	【動】自殺する
\\	【動】~を仕方なく受け入れる
\\	【名】運命
\\	【動】(間もなく)~するところである
\\	【動】~(犯罪・過失)を犯す
\\	【動】自殺する
\\	【動】~(食べ物など)を飲み込む
\\	【名】毒、毒物
\\	【動】(考え直して)~しないことにする
\\	【形】曖昧な
\\	【名】噂
\\	【動】~であるとわかる
\\	【形】嘘の
\\	【副】(前節を受けて)(だが)それでもやはり
\\	【名】疑い
\\	【動】まだ続いている[残っている]
\\	【形】(役職が)上級(の)
\\	【名】役人、(上級)公務員
\\	【動】
\\	(人)に
\\	の容疑をかける
\\	【動】~を受け取る
\\	【名】賄賂
\\	【形】十分な
\\	【名】証拠
\\	【動】
\\	の容疑で)~を逮捕する
\\	【動】(~の)担当責任者である
\\	【名】計画、事業
\\	【動】~を解雇[解任]する
\\	【名】汚職、不正(行為)、賄賂[収賄]
\\	【副】私の知る限りでは
\\	【動】~(よくないこと)に巻き込まれる、~にかかわる
\\	【名】詐欺(行為)、不正(行為)
\\	【名】企み、陰謀
\\	【動】~(知りたかったもの)を見つけ出す、~を発見する
\\	【名】真理、真実
\\	【動】~を(よく)調べる
\\	【形】(山ほどの)たくさんの
\\	【形】(~と)関連性のある
\\	【名】書類
\\	【名】裁判、審理
\\	【動】(事実などが)明るみに出る
\\	【動】~(悪い事)を隠す[隠蔽する]
\\	【形】違法な
\\	【名】商取引、処理
\\	【名】10年(間)
\\	【名】探偵、刑事
\\	【動】~(人の言葉)を書き取る
\\	【名】証言
\\	【副】その場で、直ちに
\\	【副】一言一句違わずに
\\	【副】初めて
\\	【名】(突然襲う精神的な)~の苦しみ[苦痛]
\\	【名】良心
\\	【名】(困難な状況から)逃れる道[方法]
\\	【動】(嫌な体験などが)~(人)の心[脳裏]に付きまとう
\\	【形】(記憶などが)鮮明な
\\	【名】悪夢
\\	【副】概して
\\	【動】遠慮なく~する
\\	【動】(~(私的な領域)を)侵害する
\\	【名】作家、著者
\\	【動】~を改訂する、~に修正を加える
\\	【名】原稿
\\	【副】何度も(何度も)繰り返して
\\	【動】誰[何]にも劣らない[負けない]
\\	【定】~のこととなると
\\	【動】~の過ちを指摘する、~に文句をつける
\\	【副】概して
\\	【名】[通例 集合的に]エリート(たち)
\\	【動】~(他人の気持ち・意見など)に敏感である
\\	【名】批判、非難
\\	【名】タブロイド紙、大衆ゴシップ紙
\\	【動】~(秘密・真実など)を明かす[示す]
\\	【名】麻薬
\\	【名】中毒者、常用者
\\	【名】(世間の)評判
\\	【副】ほとんど
\\	【動】~(名誉など)を失墜させる
\\	【動】~をよく検討する
\\	【名】(問題などを構成する)側面
\\	【名】防衛
\\	【名】(国・組織による)計画
\\	【動】~を指摘する
\\	【形】(非常に)多くの
\\	【名】(~(機械など)の)欠陥、問題箇所
\\	【動】~に寛容な態度を取る
\\	【動】~(活動など)にかかわる
\\	【名】テロ(行為)
\\	【形】悪名高い
\\	【名】(権力・体制などに対する)反抗者、反乱軍の兵士
\\	【副】最終的には
\\	【動】~を捕らえる
\\	【動】
\\	(人)を
\\	(場所)に閉じ込める
\\	【名】拘置所
\\	【名】
\\	軍隊
\\	【副】素早く
\\	【動】~(専門家など)を呼ぶ
\\	【動】~(暴動など)を鎮圧する
\\	【名】暴動
\\	【動】~(理性など)に訴える
\\	【名】理性
\\	【前】~ではなく(別の)、~(人)の代わりに
\\	【動】(不本意ながら)~という手段を取る
\\	【名】暴力(行為)
\\	【動】(控え目に)~を提案する
\\	【動】~(方法など)を採用する
\\	【形】(考えなどが)柔軟な
\\	【名】(個々の)戦術
\\	【副】今のところは
\\	【名】
\\	(一国の)全軍
\\	【動】~に成功する
\\	【動】~(領土など)を占領する
\\	【形】すべての
\\	【名】領土
\\	【副】このような[そのような]状況下では
\\	【名】
\\	(集合的に)敵軍
\\	【動】~せざるを得ない
\\	【動】(~に)降伏[降参]する
\\	【動】(燃料・抵抗などが)持ちこたえる
\\	【副】これ以上~ない
\\	【副】遅かれ早かれ
\\	【名】人質
\\	【動】(監禁状態などから)~を解放する
\\	【名】(絶対君主の)支配[統治](時代)
\\	【名】専制政治
\\	【形】罪のない
\\	【動】
\\	(人)から
\\	(必要不可欠なもの)を剥奪する[奪う]
\\	【名】市民[公民]権
\\	【名】兵士
\\	【動】~することに気が進まない
\\	【動】(~に)従う
\\	【名】(軍の司令官などの)命令
\\	【名】(非民主的な)政権
\\	【動】きっと~する
\\	【動】崩壊する
\\	【名】同盟国
\\	【動】~を厳しく非難する
\\	【名】侵略
\\	【名】(法・規則などの)違反
\\	【名】決議
\\	【動】~(可能性など)を排除する、~はないと断言する
\\	【名】可能性
\\	【名】内戦、内乱
\\	【動】(戦争・伝染病などが)(突然)発生する
\\	【名】植民地
\\	【動】~を宣言する
\\	【名】独立
\\	【名】共和国
\\	【形】非常に大きい、巨大な
\\	【名】記念碑
\\	【動】~を建てる
\\	【前】~に敬意を表して
\\	【形】(行為などが)崇高な、高潔な
\\	【名】愛国者
\\	【副】ひざをついて
\\	【動】(~を願って)祈りを捧げる
\\	【名】霊、魂
\\	【形】
\\	故人(たち)
\\	【形】(地位・性質などが)最高の
\\	【名】
\\	最高裁判所
\\	【動】位置する
\\	【形】帝国の、天皇[皇帝]の
\\	【名】宮殿
\\	【形】中絶(合法化に)賛成の
\\	【動】~に抗議する
\\	【名】(公式な)(~の)禁止(令)
\\	【名】妊娠中絶
\\	【名】(米国の)上院議員
\\	【動】~のままでいる
\\	【形】(外交関係・議論などで)中立の
\\	【形】(議論などが)激しい、(味が)にがい
\\	【名】論争、議論
\\	【動】~な状況[立場]にある
\\	【副】~の間ずっと
\\	【副】もはや~ない
\\	【動】~できる(能力[技能]はある)
\\	【動】(~に)うまく対処する
\\	【名】
\\	首脳会談[会議]
\\	【名】(数日間にわたる)会談
\\	【助】~する予定である
\\	【動】(~を)放送[放映]する
\\	【副】(全く)同時に
\\	【副】世界中で
\\	【形】外交(上)の
\\	【名】対話
\\	【動】~するのに役立つ[貢献する]
\\	【動】~に終止符を打つ
\\	【名】紛争、衝突
\\	【動】~を任命する
\\	【名】親善、友好
\\	【名】大使
\\	【動】~(関係など)を育てる
\\	【形】相互の
\\	【名】専門家、評論家
\\	【動】~と予測[予言]する
\\	【形】国際的な
\\	【名】(政治的・社会的)緊張(状態)
\\	【動】(徐々に)高まる[増える]
\\	【副】私の考えでは
\\	【形】永久の
\\	【定】~に過ぎない
\\	【名】幻想
\\	【形】民族の
\\	【名】(人種・宗教上の)少数派
\\	【動】(苦しみながら)必死に戦う
\\	【名】偏見
\\	【名】貧困
\\	【名】(人種などに対する)抑圧
\\	【動】努力する
\\	【動】~(制度など)を捨てる[廃止する]
\\	【名】(~に対する)差別
\\	【名】(入国する)移民
\\	【動】(ずっと)~に耐える
\\	【形】肉体[身体]の
\\	【形】精神的な
\\	【名】(体の)痛み、(心の)苦痛
\\	【名】(法律などの)改正案
\\	【動】~(要求・提案など)を拒絶する、~(訴えなど)を却下する
\\	【形】人種の
\\	【名】(人種・意見などの)多様性
\\	【動】~を考慮に入れる
\\	【副】言うまでもなく
\\	【名】難民
\\	【動】~を切望する
\\	【名】自由(の状態)
\\	【前】~(情報源)によると
\\	【名】(世論・動向などの)調査
\\	【形】5つ[5人]のうち3つ[3人]の~
\\	【動】~に無関心である
\\	【形】外国の
\\	【名】
\\	(ある個人[国/時代]の)諸問題、(色々な)出来事
\\	【定】~はどうなってしまうだろう?
\\	【名】
\\	(仕事[外交]上などの)関係
\\	【名】(戦争・犯罪などの)安全の確保[保障](されている状態)
\\	【名】(国家間の)条約
\\	【動】(契約・券などが)期限切れになる
\\	【接】もし~でなければ
\\	【動】~(不要なもの)を取り除く
\\	【形】不公平な
\\	【名】関税
\\	【動】
\\	に)~(税・義務など)を課す
\\	【名】
\\	(国際法違反なおに対する)制裁(措置)
\\	【名】貿易
\\	【名】(関係などの)摩擦
\\	【動】(問題・困難などが)起こる、生じる
\\	【副】(嫌なことが)今にも
\\	【名】優先事項
\\	【動】(~(問題などを)解決する
\\	【名】論争、紛争
\\	【副】きっぱりと
\\	【動】(いつでも)~するつもりだ、~する覚悟ができている
\\	【動】~(人)と妥協する
\\	【形】連邦政府[国家]の
\\	【名】予算(案)の
\\	【副】かろうじて
\\	【動】~を承認する
\\	【名】
\\	(米国の)議会
\\	【動】~だと主張する
\\	【名】政府、政権
\\	【動】~を探す
\\	【形】代わりの~
\\	【名】(発生[入手])源
\\	【名】(国の)歳入
\\	【名】世論調査
\\	【形】(数などが)圧倒的な
\\	【名】過半数、大多数
\\	【動】~(意見など)を支持する
\\	【名】法律
\\	【名】警官、刑事
\\	【動】~(捜し物)を捜す
\\	【名】(~の)手がかり、ヒント
\\	【形】残虐な
\\	【名】殺人
\\	【形】狂気の、正気ではない
\\	【動】
\\	を
\\	と区別する
\\	【名】悪いこと
\\	【名】正しいこと
\\	【名】指紋
\\	【名】兵器、凶器
\\	【動】(~に)一致する
\\	【動】~(人)を驚かす
\\	【形】有名な、著名な
\\	【名】心理学者
\\	【動】
\\	の理由で)(法的に)~を訴える
\\	【名】誘拐
\\	【名】陪審団
\\	【形】(~で)有罪の
\\	【名】(陪審団による)評決
\\	【動】~(よくないこと)を引き起こす
\\	【形】(被害などが)広範囲にわたる
\\	【名】討論、論争
\\	【動】(~に)
\\	を)請願する
\\	【名】慈悲、情状酌量
\\	【動】~(人)に
\\	という)(有罪)判決を下す
\\	【名】刑務所
\\	【名】(法律上の)犯罪(行為)
\\	【名】警官
\\	【形】(兵士・警官・医者などが)職務中の、勤務中の
\\	【動】(五感で)~に気付く
\\	【形】お年寄りの
\\	【動】(自分のほうに)近づいてくる
\\	【動】~をちらっと見かける
\\	【名】泥棒
\\	【動】~しようと試みる
\\	【動】~(建物)に不法侵入する
\\	【動】~を目撃する
\\	【形】(この世のものとは思えないほど)奇妙な
\\	【名】(人々を不安にさせる)出来事
\\	【名】原野
\\	【動】~に気付く
\\	【動】(~を)じっと見つめる
\\	【名】(座席・棚の間の)通路
\\	【動】振り向く
\\	【副】視界に(入って)
\\	【動】(恐怖、興奮などで)悲鳴をあげる
\\	【名】(体が震えるほどの)恐怖
\\	【動】~をつかむ
\\	【動】(正式な形で)
\\	(人)に
\\	を知らせる
\\	【名】(身体・生命の)安全、無事
\\	【動】~を吸う、~を吐く
\\	【名】ため息
\\	【名】安堵
\\	【名】銃弾
\\	【動】(~を)突き抜ける(貫通する)、(~の)奥まで入り込む
\\	【名】胸部(全体)
\\	【形】危機的な、危篤の
\\	【動】~な状態[程度]である
\\	【副】現時点で
\\	【形】(物が)不確実な
\\	【接】
\\	なのかそれとも
\\	なのか
\\	【副】故意に
\\	【副】誤って、過失で
\\	【名】デリ[デリカテッセン]、惣菜売場
\\	【動】火がつく
\\	【動】(建物が)全焼する
\\	【名】(乱雑に積み上げられた)山
\\	【名】[通例-
\\	灰、燃えかす
\\	【副】突然
\\	【名】[通例-
\\	花火
\\	【名】倉庫、貯蔵庫
\\	【動】爆発する
\\	【動】~(火・タバコ)を消す
\\	【名】激しい炎
\\	【名】爆弾
\\	【動】(爆発物が)爆発する
\\	【名】~階建て(の建物)
\\	【副】ひどく
\\	【形】(武器・凶器で)負傷した
\\	【名】貨物
\\	【名】(大型の)船舶
\\	【形】(電車・飛行機などが)(~方面)行きの
\\	【動】沈む
\\	【名】
\\	地中海
\\	【副】何の痕跡も残さずに
\\	【形】無謀な
\\	【動】凍る
\\	【副】~の結果死ぬ、死ぬほど~
\\	【名】(グループによる未開の地への)探検[調査]旅行
\\	【名】
\\	南極(地方)
\\	【名】悲劇的な出来事
\\	【動】(~を)覚えている、(~を)覚えておく
\\	【副】~なるように、(~ならないように)
\\	【動】(~を)繰り返す
\\	【名】展覧会
\\	【動】(物事が)~を提供する[与える]
\\	【形】(意味・教養などが)深い
\\	【名】(~に対する)洞察(力)、深い理解(力)
\\	【形】古代の
\\	【名】文明
\\	【動】
\\	を
\\	(人)の実績として認める[評価する]
\\	【動】~(種)をまく
\\	【名】
\\	(争い・変化などの)種、きっかけ
\\	【名】
\\	ルネッサンス
\\	【動】~を載せる[はめ込む]
\\	【名】肖像画[写真]
\\	【形】装飾された
\\	【名】枠、額縁
\\	【形】【副】上下逆さまの[に]
\\	【形】(外観・景観などが)荘厳な
\\	【名】大聖堂
\\	【動】(起源が)~にまでさかのぼる
\\	【名】
\\	中世
\\	【名】[複合語で]~の宗教、~教
\\	【動】~(人)に創作[行動]の意欲を沸かせる
\\	【動】~(作品など)を創造する
\\	【形】すばらしい
\\	【名】彫刻作品
\\	【動】~(場所)を探検する
\\	【名】砂漠
\\	【前】~を求めて
\\	【動】~を埋める
\\	【名】宝物
\\	【名】大富豪
\\	【動】(~を)強く主張する
\\	【動】(時間・労力などを費やして)~を獲得する
\\	【名】(ある芸術家の)傑作
\\	【副】たとえどんな[どれだけ]~でも
\\	【動】
\\	(人)に)~(金額)の出費[負担]を要求する
\\	【名】神話
\\	【名】伝説
\\	【動】~を
\\	(次の世代)に)伝える
\\	【副】代々
\\	【名】言語学者
\\	【動】~を流暢に話す
\\	【形】いくつかの
\\	【名】(ある地域の)方言
\\	【名】部族
\\	【動】(~(神)を)崇拝する
\\	【名】先祖
\\	【名】(ある国の)言葉
\\	【動】~によく知られている
\\	【名】(文化)人類学者
\\	【名】(社会などの)慣習
\\	【動】(慣習などが)広く行われている[よく見られる]
\\	【名】(自治)州
\\	【形】(性格などが)意地の悪い
\\	【名】魔女
\\	【動】(に)呪文[魔法]をかける
\\	【形】(行為・考えなどが)とても悪い、邪悪な
\\	【名】呪文、魔法
\\	【動】
\\	を
\\	に変える
\\	【名】虫
\\	【動】(先に)進む
\\	【形】(道などの)傾斜が急な
\\	【名】細道、小道
\\	【形】唯一の
\\	【名】(場所などに)到着する方法[入る方法]
\\	【名】国境
\\	【名】(ゴリラ、チンパンジーなどの)猿、類人猿
\\	【動】(姿が)現れる
\\	【名】洞窟
\\	【名】(~を)怖がる、(~に)びっくりする
\\	【動】逃げる
\\	【副】信じてもらえないかもしれないが(本当に)
\\	【名】宇宙人
\\	【動】(突然[完全に])消える
\\	【定】冗談はよしてくれ
\\	【形】かなり多くの(数)
\\	【動】
\\	を
\\	に)招待する
\\	【動】(式などを催して)(~(特別な行事)を)祝う
\\	【名】記念日
\\	【動】~するまでもうすぐだ
\\	【副】その間に
\\	【動】[命令文で]~(食べ物など)をご自由にどうぞ
\\	【名】軽食
\\	【動】~(服)を着ている
\\	【形】(服・色などが)けばけばしい
\\	【形】奇妙な
\\	【名】服装、衣装
\\	【動】目立つ、際立つ
\\	【名】人込み
\\	【動】~(人)を感動させる
\\	【名】(心のこもった)もてなし
\\	【形】(衣装・住居などが)ぼろぼろの、みすぼらしい
\\	【名】衣装、洋服
\\	【動】~にふさわしい
\\	【名】(特別な)行事、イベント
\\	【名】話題
\\	(形容詞) 
\\	【構】~するにはあまりに
\\	過ぎる
\\	【形】とても私的な
\\	【動】
\\	を
\\	に)話す
\\	【形】偶然の、何気ない
\\	【名】知人
\\	【形】酔っ払った
\\	【形】(言葉などが)下品な、卑猥な
\\	【形】(言葉・態度などが)(人を)不快にさせる
\\	【動】~に近づかない
\\	【副】ほとんど~ない
\\	【動】~を理解する
\\	【動】~をほのめかす
\\	【形】どぎまぎした、恥ずかしい
\\	【動】(しつこく)(~を)からかう
\\	【前】~の正面に[で]
\\	【形】曖昧な
\\	【名】返事
\\	【副】なおさら
\\	【動】~をいらいらさせる
\\	【形】臆病な
\\	【動】赤面する
\\	【動】おしゃべりする
\\	【動】~(人)を見下す
\\	【動】~の出身である、~から来る
\\	【形】(都会から離れた)地方の
\\	【定】(~が)どうかしたの
\\	【形】惨めな、とても不幸な
\\	【動】~をほうっておく
\\	【定】あなたには関係ないことだ
\\	【形】【副】変装して
\\	【動】~が誰[何]だかわかる
\\	【副】一目見て
\\	【動】くすくす笑う
\\	【動】~を身に付けている[着ている]
\\	【名】スウェット、トレーナー
\\	【副】{米}(服を)後ろ前に
\\	【動】~が上手だ
\\	【動】(~と)仲良しになる[親しくなる]
\\	【動】自分の殻に閉じこもる
\\	【動】(人が)受け入れられる場所にいる
\\	【副】極度に
\\	【形】不安な
\\	【形】孤独な
\\	【副】かなり
\\	【動】~に順応する
\\	【名】(自分の周囲の)環境
\\	【副】全く
\\	【形】(人・場所などが)場違いの、(人が)浮いてしまって
\\	【形】洗練された
\\	【副】実を言うと
\\	【形】(音楽ジャンルが)クラシックの
\\	【副】しかし一方では
\\	【動】~(趣味など)にはまっている
\\	【動】
\\	から
\\	を引き出す
\\	【名】楽しみ
\\	【動】~を観察する
\\	【名】昆虫
\\	【定】~はいかがですか?
\\	【動】トランプをする
\\	【名】
\\	か何か
\\	【副】気分転換に
\\	【定】そういう気分じゃないよ
\\	【定】お先にどうぞ
\\	【動】先へ進む
\\	【定】どういたしまして
\\	【名】姪
\\	【形】(人が)社交的な
\\	【形】(精神面で)大人になった
\\	【副】年の割には
\\	【動】(本当は違うが)~として通用する
\\	【名】二十歳(かそこら)、二十代(表現をぼかすときに使う)
\\	【定】(別れの挨拶)(相手を気遣って)じゃあね
\\	【定】(別れ際などに)~(人)によろしく
\\	【定】またね
\\	【動】~を楽しみをする
\\	【動】~(人)から(手紙・電話などで)連絡をもらう
\\	【定】敬具
\\	【動】~(時間)を割く
\\	【動】~について話し合う
\\	【形】(~にとって)重要な
\\	【動】~のように思える
\\	【形】些細な
\\	【動】
\\	は~するだけの価値がある
\\	【動】~に注意を払う
\\	【副】
\\	に関しては
\\	【定】何でも受け入れる
\\	【副】それとは対照的に
\\	【形】用心[注意]深い、慎重な
\\	【形】(状況などが)気まずい
\\	【名】(会話などの)間
\\	【動】(どこかへ連れていく目的で)~の手[腕]を取る[つかむ]
\\	【動】~(抵抗するもの)を無理に引っ張って進む
\\	【副】上の階[二階]へ[に、で]
\\	【動】~を慰める
\\	【動】
\\	を
\\	の意味に)解釈する
\\	【形】(関係などが)恋愛の
\\	【名】興味[関心]
\\	【動】~を確かめる
\\	【副】(眠りが)深く
\\	【動】眠っている
\\	【動】(人に気づかれないように)静かに[忍び足で]歩く
\\	【動】(旅行などに)出発する
\\	【副】ここだけの話だけど
\\	【動】~を偶然見つける
\\	【副】情熱的に
\\	【動】(~を)強く抱きしめる
\\	【名】発言
\\	【動】~(批判など)を受けやすい
\\	【名】誤解
\\	【動】~に我慢する
\\	【副】もうこれ以上~しない
\\	【動】落ち着く
\\	【動】そこに行く[相手のいる場所に向かう]
\\	【副】できるだけ~に
\\	【副】文字通りの
\\	【動】~する傾向がある
\\	【動】(~を)大げさに表現する
\\	【動】
\\	のことで)(~に)謝罪する
\\	【動】~をばかにする
\\	【動】(相手に対する態度などが)度を越す
\\	【副】そのうち
\\	【動】機嫌がいい
\\	【動】~(人)とうまくやっていく
\\	【副】時々
\\	【動】~しようかと考える
\\	【動】(~と)離婚する
\\	【定】冗談でしょう
\\	【副】泣きながら
\\	【動】~(薄いもの)を(ずたずたに)引き裂く
\\	【動】~を捨てる
\\	【動】~をだます
\\	【助】~するべきだった(のにしなかった)
\\	【動】(~しないくらいの)分別はある
\\	【動】~を信頼[信用]する
\\	【動】~したことを後悔する
\\	【動】~を無駄にする
\\	【形】(とても)たくさんの~
\\	【定】~する時はいつも[必ず]
\\	をする
\\	【動】(~を)振り返る[回想する]
\\	【名】思春期
\\	【形】憂鬱な
\\	【動】
\\	(人)を見捨てて
\\	(別の人)にいく
\\	【名】男
\\	【動】精神的に不安定である
\\	【定】(返答で)そうでもないよ
\\	【動】~に慣れている
\\	【定】~ということも不思議ではない[当然だ]
\\	【動】(待っていた人が)姿を見せる
\\	【動】~(人)を見送る
\\	【定】どうして?
\\	【動】(異性と)別れる[関係がこわれる]
\\	【定】それはかわいそうに
\\	【副】たった一人で
\\	【名】(罰としての)拷問、(肉体的・精神的な)耐え難い苦痛
\\	【動】~とすすり[むせび]泣きながら言う
\\	【副】正直に言うと
\\	【動】~の熱狂的なファンである
\\	【形】勇敢な
\\	【形】(自分(の能力など)に)自信がある
\\	【動】(~を)怖がる
\\	【動】危険を冒す
\\	【定】私(があなたの立場)だったら~するかな
\\	【動】~(異性)をデートに誘う
\\	【動】~(異性)と付き合う
\\	【副】(長い期間に)したりやめたりを繰り返して
\\	【副】ずいぶんと長い間
\\	【副】相変わらず~
\\	【形】決断力のない、優柔不断な
\\	【構】もう~してもよい時間だ
\\	【動】身を固める
\\	【動】~ということに
\\	(人)が(だんだん)気付いてくる
\\	【動】~にだまされる
\\	【副】最初からずっと
\\	【形】うぶな
\\	【動】~の正体を見抜く
\\	【接】実際に~だけれども(それでも)
\\	【動】~(異性)と付き合っている
\\	【動】(~を)あきらめる
\\	【定】どうだい調子は?
\\	【定】別に
\\	【動】(~と)結婚する
\\	【感】おめでとう!
\\	【動】(相当)大変な[まずい]ことになっている
\\	【動】気分が落ち込んでいる
\\	【動】(困った時には)
\\	の)そばにいる
\\	【名】(大きな)喜び
\\	【名】反対のもの
\\	【名】(深い)悲しみ
\\	【動】(なんと)~さえもする
\\	【名】ばか
\\	【動】~(発言など)を撤回する
\\	【定】当然の報いだ
\\	【動】~(人)を挑発する
\\	【動】~に我慢する[耐える]
\\	【名】緩慢さ
\\	【動】~を悪用する(~を利用する)
\\	【定】黙れ!、うるさい!
\\	【動】慌てない
\\	【形】2、3の
\\	【動】~をよく[深く]考える
\\	【動】~(辛いこと)を乗り切る
\\	【名】厳しい試練
\\	【副】(たった)一人で、自分(だけ)で
\\	【名】ハエ
\\	【名】蚊
\\	【動】(~の)邪魔[妨害]をする
\\	【名】(宗教的な)瞑想
\\	【形】神聖な
\\	【名】(宗教的・伝統的な)儀式
\\	【動】(行事・会合などが)行われる
\\	【動】~の延期を決定する
\\	【動】突然
\\	【名】仏教徒
\\	【名】(キリスト教以外の)聖職者、僧侶
\\	【動】突然~の状態になる
\\	【動】~を台無しにする
\\	【形】厳粛な
\\	【名】雰囲気
\\	【動】咳をする
\\	【動】くしゃみをする
\\	【動】あくびをする
\\	【名】親戚(の一人)
\\	【動】(人が)亡くなる
\\	【定】それは大変だね
\\	【定】それはお気の毒に
\\	【前】~に加えて
\\	【動】~(会議など)に出席する
\\	【名】葬式
\\	【動】段取りをつける
\\	【動】~の気持ちがわかる
\\	【名】(死別などによる)深い悲しみ
\\	【動】徐々に消える
\\	【動】~(病気など)を克服する
\\	【動】(宝物のように)~を大切にする
\\	【形】貴重な、大切な
\\	【名】思い出
\\	【名】少年[少女]時代
\\	【定】~と言えば
\\	【動】(~と)婚約する[している]
\\	【名】実業家
\\	【定】あきれてものが言えないよ
\\	【動】~(関係)が(突然)壊れる[破局する]
\\	【副】近ごろは
\\	【名】動機
\\	【名】結婚
\\	【副】(可能性として)必ずしも~ではない
\\	【形】純粋な
\\	【定】例えば~を見てごらん
\\	【動】~にやさしく接する
\\	【副】(適当な言葉が瞬時に出てこない時に)ほら、あれだよ
\\	【動】~(嫌なこと)を経験する
\\	【形】(人生が)辛い、粗い
\\	【名】期間、時期
\\	【副】最近、この頃
\\	【副】(話題を変えるときに)ところで
\\	【定】(今の)時間わかりますか?、今何時ですか?
\\	【定】え~っと
\\	【定】8時15分前です
\\	【定】今日はこの辺までにしておこう
\\	【動】おなかがペコペコだ
\\	【定】~をおごるよ
\end{CJK}
\end{document}