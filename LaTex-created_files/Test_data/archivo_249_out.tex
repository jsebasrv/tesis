\documentclass[8pt]{extreport} 
\usepackage{hyperref}
\usepackage{CJKutf8}
\begin{document}
\begin{CJK}{UTF8}{min}
\\	綜合大學の圖書
\\	會津八一
\\	小泉八雲といへば、日本人の名であるし、日本人として東京の宅で死んでその全集は日本語で出版されてゐるが、父は英國のアイルランドの軍醫、母はギリシャのリウカヂアの娘、子供の時はフランスの叔母の手で育てられ、青年時代にアメリカへ渡つて文學者となり、日本へ來て出雲松江の中學教師となり、小泉といふ士族の家へ婿入りして、日本人になり、熊本の高校、東京帝大に轉任して英文學の講義をし、おしまひは早稻田へ來て亡くなつた。經歴からが世界的で、作物も世界的にひろく讀まれ、文豪の名が高い。
\\	私もその最後の講義を聞いた一人だが、亡くなられてから、その三人の男の子たちの教育から家事のことまで、いつも夫人の相談を受けてゐた。ある日、夫人が宅へ見えられて、家計上の必要もあるから、八雲ののこした藏書を始末したいから、なるべく早稻田で買つてほしいといふことであつた。その藏書といふのは、日本造りの西大久保の宅で、フランス語の本が一室、英語と日本語の本が一室、純日本風の書物箱に納めて、ぎつしりと並べられてゐた。夫人の言葉では、官立の帝大から早稻田へ移つてから、ほんとに故郷へ歸つたやうに氣樂になつたと喜んでゐたから、書物も早稻田へ納めたい。法政大學では八千圓で頂戴したいといふけれども、早稻田ならその半分でもかまはないといふことであつた。
\\	けれども、その話のまだまとまらぬうちに、夫人がまたやつて來られて、富山の高等學校から校長の南日さんが見えて、一萬圓でもいいから、是非こつちへ戴きたいといふから、どうしたものであらうかといふので、私もそちらに同意して、富山へ送つてしまつた。
\\	その時、南日さんは、日本郵船の株券を額面で一萬圓だけ夫人に渡して、書物を買ふ金は、學校の方で政府から豫算を貰つてゐるのではないが、富山のやうな所に高等學校などを始めても設備が不充分で、ことにいい書物でも無いことには、有爲な青年教授は誰も來てくれるものでない。それでは學生にいい教育が出來ないから、とりあへず私の手持の株券をたんぽに差上げておくのだといふことであつた。南日さんはかういふ風に背水の陣をしいて、富山へ歸つて、縣内の有志の間に熱心に説き囘つて、やうやく一萬圓の耳を揃へて株券を取り戻しに上京されたのであつた。
\\	南日さんといへば、英語の教科書や字書などで大に當てたので、印税の檢印に、家族總がかりで、幾日もかかるといふほどで、學者としては稀な金持であつたかもしれないが、それにしても、その頃の一萬圓は今日の何百萬圓だ。それを自分の勤めてゐる學校のために、一人でぽんと投げ出したところは、ほんとに偉いものだと、私はいつも敬服してゐる。
\\	新潟に、これから出來るのは高等學校ではなく、綜合大學といふもので、いかに「新制」だからといつて、大學は大學である。小泉八雲がいかに偉大な世界的文豪でも、藏書は誰にしても自分の好みに片よる。その程度の藏書を、われわれが今どこかほかで一口や二口見つけて來ても、ただそれだけで、大學の「人文科」とか「文學科」とかいふものの參考書としては、もちろん不足である。アメリカ合衆國は立國が一七七六年で、徳川十代將軍の時だ。その新しいのを羨ましいくらゐであるが、日本は古い國で、佛教が渡つて來てからでも千四百年になるし、支那や印度はもつともつと古いから、日本人が、曲りなりにも東洋の文化を研究をして、世界的の水準に進み出るには、そのために必要な書物は山の如くにある。それを一度に備へつけなければ少しも仕事が出來ないといふのではないが、何もなしに椅子とテーブルだけで、夏季講習會のやうなことをいつまでもやつてゐられるものでない。どうせ建てるなら外よりいい大學が建てたい。南日さんのやうな人が、この場合、新潟にも何百人も出なければならない。
\\	この大學は、政府の役人の計らひでこちらへ建ててくれるのではなく、縣人が一せいに奮起して、熱望したものであつた。街頭で署名もしたであらうが、それはほんとに熱意をこめた署名であつたにちがひない。知事や役人は、何度もそのことで上京して運動もしたであらうが、それは全縣人の熱意を代表して中央へ御使に行つたまでのことであらう。それがいよいよ建つことにきまれば、もうすつかり安心して、一切は御上まかせにしておくといふ法はない。どんなものでも、名さへ大學なら何でもいいといふのではいけない。大學を建てる、ことにいい大學を建てるといふことは容易なことでない。建つことにきまつた今日こそ、縣人はますます奮起もし、奮發もし、犧牲の覺悟もしなければならなくなつた。往來の人を喚び留めて署名をさせる。いはれるままに署名をする。それだけで大學らしい大學が出來るものでないといふことを、よくのみ込んで貰ひたい。
\\	底本:「會津八一全集 第七卷」中央公論社
\\	1982(昭和57)年4月25日初版発行
\\	初出:「夕刊ニイガタ」
\\	1948(昭和23)年5月25日
\\	入力:フクポー
\\	校正:杉浦鳥見
\\	2019年7月30日作成
\\	青空文庫作成ファイル:
\\	このファイルは、インターネットの図書館、青空文庫
\\	で作られました。入力、校正、制作にあたったのは、ボランティアの皆さんです。
\end{CJK}
\end{document}