\documentclass[8pt]{extreport} 
\usepackage{hyperref}
\usepackage{CJKutf8}
\begin{document}
\begin{CJK}{UTF8}{min}
\\	情報セキュリティの
\\	の
\\	は?	
\\	機密性
\\	情報セキュリティの
\\	の
\\	は?	
\\	完全性
\\	情報セキュリティの
\\	の
\\	は?	
\\	可用性
\\	は何性を向上させる?	可用性
\\	セキュリティ構成上の弱点•不備は何という?	脆弱性
\\	情報資産や企業活動に危害や悪影響を与えうる事象は何という?	脅威 (災害、故障、クラッキングとか)
\\	は共通鍵? 公開鍵?	共通鍵
\\	は共通鍵? 公開鍵?	公開鍵
\\	デジタル証明書に
\\	名や有効期間、所有者名は含まれる?	
\\	ユーザの行動がきっかけとなって実行される攻撃は?	受動攻撃
\\	ウイルスの機能3つ	自己伝染機能 潜伏機能 発病機能
\\	他のファイルに寄生しており、ユーザの操作で増殖するウイルスは?	狭義のウイルス
\\	偽装(なりすまし)を意味するセキュリティ上の脅威は?	スプーフィング
\\	脆弱な無線
\\	ポイントを移動しながら探して侵入を試みる行為は?	ウォードライビング
\\	ダイヤルアップ接続に使われるモデムが接続されているか検索する行為は?	ウォーダイヤリング
\\	に脆弱性のあるサイトのフォームに対するスクリプトを埋め込んでおき、クリックすると別のサイトなどに飛ばされたりするのは?	クロスサイトスクリプティング
\\	サイトにログインしたまま罠サイトにアクセスすると、罠サイトがユーザに悪い
\\	を返し、サイトの情報が書き換えられるのは?	クロスサイトリクエストフォージェリ 
\\	メールサーバが転送を行うのを利用して、メールのおくりもとを隠したり、スパムの負荷を肩代わりさせるのは?	不正リレー (不正中継)
\\	ユーザ管理の考え方である
\\	の要素はそれぞれ?	
\\	認証 
\\	承認 
\\	アカウンティング 
\\	監査)
\\	ユーザの身元を確認するのは?	認証 
\\	データに関するアクセス権限をチェックするのは?	承認
\\	データへのアクセスを記録するのは?	
\\	アカウンティング 
\\	監査)
\\	パスワードポリシーやアカウントロックアウトポリシーを変えられるのはどこ?	ローカルセキュリティポリシー
\\	ネットワークを介してアクセスしてくるユーザに適用されるアクセス許可は?	共有アクセス許可
\\	ローカルログオンしたユーザと、ネットワークを介してアクセスしてくるユーザに対しても適応できるアクセス許可は?	
\\	アクセス許可
\\	データへのアクセス制限をする管理リストを
\\	では何と呼ぶ?	
\\	随意アクセス制御リスト
\\	に記載されたエントリのことは?	
\\	監査を記録するために条件を定義するリストは?	
\\	書き込み権限でファイル削除はできる?	できない
\\	ファイルやフォルダにユーザがどのようなアクセス許可を持つか調べるタブは?	有効なアクセス許可
\\	ファイル、フォルダへの操作を記録するには?	「ローカルセキュリティポリシー」の「監査ポリシー」から、「オブジェクトアクセスの監査」を設定し、記録をとるフォルダで監査タブを設定する。
\\	フォルダやファイル単位で行う暗号化機能は?	
\\	暗号化ファイルシステム
\\	が使えるボリュームは?	
\\	ボリューム単位の暗号化機能は?	
\\	リムーバブルメディアの暗号化機能は?	
\\	で
\\	がインストールされているボリュームを暗号化するときの鍵の保存場所は?	
\\	メモリ
\\	でデータボリュームの暗号化を行う場合の鍵は?	パスワード スマートカード
\\	でボリューム保護するときに
\\	を指定する場合どこの設定を変える?	グループポリシーの変更
\\	の鍵は?	パスワード スマートカード
\\	実行するソフトウェアを制御するのは?(2つ)	ソフトウェアの制御ポリシー 
\\	ファイヤーウォールの規則は?	ファイヤウォールルール
\\	ファイアウォールの種類は?	パーソナルファイアウォール
\\	ヘッダーを検査してフレーム通貨の可否を制御するのは?	パケットフィルタ
\\	ヘッダーを検査してフレームの通貨の可否を制御、セッション状態を認識してアクセス制御ができるファイアウォールの機能は?	サーキットレベルゲートウェイ
\\	送信パケットを許可すれば、応答パケットも自動的に許可するファイアウォールの設定は?	ステートフルインスペクション
\\	ファイアウォールの機能+アプリケーションレベルのセキュリティ+
\\	なんかの機器、ソフトは?	
\\	アプリケーションレベルのセキュリティのみを提供する機器、ソフトは?	
\\	スイッチ内で論理的にネットワークを分割する機能は? (同じスイッチに接続していても論理的に異なるネットワークを構成することができる)	
\\	フレームに
\\	のタグ情報をつけてグルーピングする方法は?	タグ
\\	スイッチのポートごとに所属する
\\	を設定するのは?	ポートベース
\\	適切なセキュリティ設定のコンピュータのみを社内ネットワークに接続させる機能は?	
\\	リモートアクセスの認証プロトコル強い順に? 
\\	(つよい) 
\\	(よわい)
\\	で使えるトンネリングプロトコル3つ	
\\	リモートアクセスの認証要求をひとまとめに引き受けるサーバは?	
\\	サーバ
\\	無線
\\	で強い規格は?	
\\	認証で
\\	を使った他の
\\	認証プロトコルより強い奴は?	
\\	認証で証明書、スマートカードなんかを使ったやつは?	
\\	はどの
\\	以降でつかえる?	
\\	を構成するのにいるのは?	ローカルセキュリティポリシー セキュリティが強化された
\\	ファイアウォール
\\	常に最新にするプログラムは?	
\\	複数クライアントに
\\	の自動更新を強制するならどこをかえる?	グループポリシー
\\	フィッシングのアクセスをブロックするのは?	
\\	フィルター
\\	サイト閲覧でクライアント側にダウンロード、実行されるのは?	モバイルコード
\\	コンピュータにおけるセキュリティ構成に問題がないか検査できるのは?	
\\	組織のコンピュータに更新プログラムを配るの2つ	
\\	簡単無料 
\\	詳細高度有料
\\	ではどこ変える?	グループポリシー
\\	許可のない限りアプリケーションを標準ユーザーの権限で動かすことで、不用意にマルウェアなどがオペレーティングシステム 
\\	に変更を加えられないようにするのは	
\\	ユーザアカウント制御
\\	通常は一般、管理者で一時的に昇格するのは?	セカンダリログオン
\\	サービスアカウント3つ	ローカルシステムアカウント 
\end{CJK}
\end{document}