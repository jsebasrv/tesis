\documentclass[8pt]{extreport} 
\usepackage{hyperref}
\usepackage{CJKutf8}
\begin{document}
\begin{CJK}{UTF8}{min}
\\	グレゴール・メンデル
\\	石原純
\\	【テキスト中に現れる記号について】
\\	ルビ
\\	(例)殊《こと》
\\	入力者注 主に外字の説明や、傍点の位置の指定
\\	(例)[#7字下げ]
\\	[#7字下げ]遺伝学のおこり[#「遺伝学のおこり」は中見出し]
\\	ダーウィンの生物進化の説と相並んで、生物学の上で非常に大切な意味をもっているのは、メンデルの遺伝の法則で、今ではこの遺伝に関する学問が大いに進んで、生物をほんとうに研究するには、もちろんそれのいろいろな事がらを知らなくてはなりませんが、そのなかでも殊《こと》に遺伝学の重要であることが認められています。
\\	遺伝というのはごく簡単にいえば、親の性質が子に伝わるということで、これは普通に誰でも知っている事がらです。親と子とは、その顔かたちにしても、どこか似通ったところがあり、気質の上でも大体はそうであるのです。なかには例外もないわけではありませんが、その例外と見られるものも、すぐの親ではなく、それより前の先祖の性質を受け継いでいることも多いのです。それで遺伝ということは、ともかくも確かな事実ですが、しかしどうしてそういう事実が現れるかということについては、科学の上でいろいろ研究を要することにちがいないのです。まず遺伝の場合には、どのような性質が最も多く子孫に伝わるのかということや、そしてそれの伝わり方について、実際にしらべて見なければなりません。これらについて古くからいろいろな考えを持ち出した人々もあったのですが、それよりも大切なのは、実験を行ってそれを事実の上で明らかにすることです。ところで、このような実験を始めて実際に行ったのが、ここでお話ししようとするメンデルなので、それで今日ではこのメンデルの仕事を記念する意味で、遺伝学のことをメンデリズムとも呼んでいるのです。いずれにせよメンデルの遺伝に関する研究は、生物学の上で非常に大きな意味をもっているものにちがいないのです。
\\	[#7字下げ]メンデルの生涯[#「メンデルの生涯」は中見出し]
\\	メンデルの名はグレゴール・ヨハンと云《い》うのですが、一八二一年の七月二十二日にオーストリーのシュレジーエンにあるごく小さな村ハインツェンドルフで生まれました。家は農家でありましたが、中学に当るギムナジウムを卒業してから、ブリュンという処《ところ》にある僧院で神学教育を受け、それを終えて一八四七年にそこの僧院の司祭となりました。そしてそれでともかく一人前の僧侶となったのですが、メンデルにはそのような僧職がどうも十分には気が向かないように感ぜられました。それで何か学問を修《おさ》めたいという心が頻《しき》りに起って来たので、遂《つい》に決心を定めて、一八五一年にオーストリーの首都であるヴィーンに赴《おもむ》き、そこの大学に入って、数学、物理学、および博物学を熱心に学びました。メンデルは、この時もはや三十歳にもなっているので、普通の学生とは年齢の上でもちがうわけですが、ひたすら学問を修《おさ》めたいという心から、一生懸命に勉強したのでした。そして三年の後に、大学を卒業してから、一八五四年にもとのブリュンの町に帰り、そこで或《あ》る実科学校の教師となりました。
\\	ブリュンの町に戻るとなると、僧侶の職の方も勤めないわけにはゆかないので、それは以前のように行っていましたが、大学で修《おさ》めた博物学に大いに興味を感じていたので、それからは僧院のなかに自分でいろいろの動物を飼ったり、また植物を栽培して、それらをこまかく観察することを楽しみとしました。そしてその間に遺伝の問題に不思議を感じ、これを実験して見ようと思い立ったのです。
\\	僧院の庭はさほど広くもなかったのですが、それでも六十坪ほどの土地を利用して、豌豆《えんどう》を栽培して見ました。そして豌豆《えんどう》のいろいろな種類の間に交配を行うと、どんな雑種ができるかを、一々しらべて見ました。メンデルはこの実験を八年間もつづけて行ったということです。そしてその結果が一通りわかって来たので、一八六五年にブリュンの博物学会の会合の席で、これを発表し、その翌年にはこの学会の記要に「雑種植物の研究」という題で、論文を公《おおや》けにしました。これが遺伝の法則を始めて明らかにした大切な論文なのです。この外に、メンデルは柳やたんぽぽのような植物についても、また蜜蜂や鼠《ねずみ》などの動物についてもそれぞれ交配を行わせて遺伝の研究をつづけて居ました。
\\	このようにしてブリュンの僧院には一八六八年まで十五年間を過ごしましたがこの年に僧正の職についたので、その後は自分の研究を進めるだけの暇がなくなってしまったのは、メンデルにとっては遺憾のことであったのでしょう。それにメンデルのそれ迄《まで》の研究についても、今日でこそそれの重大な意味を誰しもが認めているのですが、その頃の人々には一向に顧みられず、そのままに見過ごされていたのでした。これは謂《い》わばメンデルだけが時代に先んじてもいたので、やむを得ないことでもあったのでしょうが、やはり彼にとっては残念な次第でもあったわけです。ところが、そればかりではなく、僧正の職についてその仕事を忠実に行って来たのはよかったにしても、その頃政府が特別の税金をこの僧院に課したので、これを不当であるとしてメンデルは政府と争い、いかにしてもこれに屈しなかったということです。これは一八七二年頃のことでありましたが、その後いろいろと好ましからぬ出来事にであい、もともと快活でもあり友情も並みはずれて深かった性格にまでも影響して、だんだんに世人を嫌うようになったとも云《い》われています。そして一八八四年の一月六日に腎臓炎をわずらって歿くなりました。
\\	メンデルの研究は、かくて世間からは全く知られずに、その後も久しく埋もれていましたが、それがようやく見つけ出されたのは一九〇〇年のことで、メンデルがブリュンの学界でこれを発表してから、実に三十五年も経ってからのことでした。
\\	どうしてメンデルの研究がこのとき発見されたかと云《い》いますと、それにはおもしろい話があるのです。ちょうどその頃同じく遺伝について研究していた三人の学者がありました。それは、ドイツのコレンス、オーストリーのツェルマック、およびオランダのド・フリースであります。この人たちの研究の結果がそれぞれ学会で発表されてみると、ふしぎにもそれらが互いに一致しているので、これは確かな事がらであるとして認められるようになったのでしたが、そうなると、同じ事がらを研究した学者が以前にもありはしなかったかと云《い》うことが、学界の話題となりました。そして古い論文をしらべてゆくうちに、メンデルの研究が見つけ出されたのです。そしてすでに三十五年も前に、メンデルが立派に同じ結果を出して居て、且《かつ》それを詳しく説明していることまで、すっかりわかったのでした。それでこれをメンデルの法則と称《たた》えるようになったのです。メンデルはつまりこのような事を何も知らないで、歿くなったのでしたが、学問の上の仕事は、それが正しければ、立派に残っていて、いつかは見つけ出されて、その偉大な栄誉をになうことのできるものであるということが、この一事によってもみごとに証拠立てられるのです。かくてメンデルは、たとえ不遇のうちに歿したとしても、その名は、科学の歴史の上に限りなく燦然《さんぜん》と輝くことでもありましょう。
\\	[#7字下げ]メンデルの法則[#「メンデルの法則」は中見出し]
\\	メンデルが僧院の庭で長い年月をかけていろいろと苦心した上にようやく見つけ出したメンデルの法則というのは、どういうものかと云《い》うことを、ここでなるべくわかり易く説明して見ましょう。
\\	それはまず親から子に遺伝する性質のなかには、優性と劣性として区別される二種類の性質があって、優性をもっているものと、劣性をもっているものと交配させると、それから生まれてくるものは大体において優性を具《そな》えていると云《い》うのです。しかしこのようにして出来た雑種をもう一度おたがいに交配させると、今度は優性と劣性とが分離して現れ、優性のもの3に対して劣牲のもの1という割合で第二代目の雑種が生ずるのです。これはメンデルの分離の法則と云《い》われていますが、更《さら》にこの第二代目の雑種のうち優性を示している三つの中の一つは純粋の優性でありますけれども、その他の二つには優性と同時に劣性が幾らか含まれているということも明らかにされました。勿論《もちろん》、この場合にも何が優性であり、何が劣性であるかと云《い》うことについては、めいめいの動物や植物についてよく観察してそれを定めてゆかなくてはならないのですが、いつもこのような一般的な法則が成立つということを見つけ出したのは、実にメンデルの偉大な業績であります。
\\	この外に、メンデルは再結合の法則というのを見つけ出しました。メンデルの実験を行った豌豆《えんどう》のなかには、種子が円《まる》くて黄いろい色をしたのと、皺《しわ》があって、緑色をしたのとがありましたが、これ等《ら》を交配させてみると、それから生じた第一代雑種の結んだ実はすべて円《まる》くて黄いろいものでありました。この事で、円《まる》くて黄いろいのは優性で、皺《しわ》があって緑いろなのは劣性であることがわかったのです。ところがこの第一代雑種の種子を蒔《ま》いて、今度はそれについて自花授精をさせてみると、それで出来た第二代雑種のなかには、四種類のちがったものが現れました。この四種類というのは、つまり次の四つで、それらの数の割合は下の数字で示した通りです。
\\	[#ここから2字下げ]
\\	黄いろくて円《まる》いもの    
\\	緑いろで円《まる》いもの     
\\	黄いろくて皺《しわ》のあるもの  
\\	緑いろで皺《しわ》のあるもの   
\\	[#ここで字下げ終わり]
\\	この結果を見ると、優性と劣性との割合はやはり3と1とになっていますが、最初には黄いろいことと円《まる》いこととが伴なって結びついて居り、緑いろと皺《しわ》のあるのともそうであったのに、ここではこれ等《ら》の性質が離れてしまって、却《かえ》って他の性質と結びついて現れることのあるのが、明らかにわかるのです。つまり個々の性質はそれぞれ独立のものであって、それらが分離して再び他のものと結合するということが、これで示されたので、その意味でこれを再結合の法則、または独立結合の法則というのです。
\\	メンデルは遺伝に対してこのような法則のあることを見つけ出した後に、なお進んで、このような法則がいつも成り立つとするなら、それは何によるのかということを考えてみました。そして生物には何かしら遺伝因子というようなものがあって、それが親から子に伝わってゆくのであろうと想像しました。それはうまい考え方で、この事を仮定した上で、それから上に述べた法則を導き出すことはできるのです。しかし実際に遺伝因子というようなものがあるとするなら、それは生殖細胞のなかに含まれていなければならないので、これをはっきりと事実の上でしらべるためには細胞のくわしい研究が必要となるのです。
\\	しかし細胞についての知識は、この頃いくらかずつは進んで来てはいましたが、まだそれらのこまかい事がらは一向にわからなかったのでした。細胞の学問の進んで来たのは、それより後のことで、今ではいろいろの事がらが明らかになって来たばかりでなく、生殖細胞における遺伝因子のことについても、よほどよく知られるようになりました。細胞にはそれを包んでいる膜のなかに原形質と名づけるものがあって、それが細胞の本体を形づくっているということだけは、すでに一八六一年にマクス・シュルツェという人が見出したのでしたが、この原形質がつまり細胞の生命をになっているもので、そのなかに含まれている核が分裂して原形質に境ができると、それで細胞が分れてその数を増し、そのおかげで生物が生長してゆくのであるということも、だんだんにわかって来ました。また細胞にはいろいろの種類があって、遺伝に関係しているのはその中の生殖細胞と名づけるものなのですが、この細胞の核のなかには染色体というものがあって、それが遺伝因子をになっているのだと、今では考えられています。染色体というのは、塩基性の色素で特別に濃く染まるので、そう云《い》われているのですが、生物の種類によってその数や形や大いさ[#「大いさ」はママ]などが異なり、それぞれの種類ではこれらが一定しているのです。
\\	生殖細胞やそのなかの染色体のことについては、今ではさまざまのこまかい研究が行われて、いろいろのおもしろい事実も知られるようになりましたが、遺伝に関するこれらの学問がそれほど進むようになったというのも、その最初はメンデルの研究にあったということを考えるならば、メンデルの仕事の大きな意味が誰にもわかるにちがいありません。またこの遺伝に関する事がらは、実はこの前にお話ししたダーウィンの生物進化説とも密接に関係しているのです。なぜと云《い》えば、生物がだんだんに進化してゆくということも、実際には親から子に性質が遺伝してゆく間にいろいろな変化があらわれてくるからであって、それですから進化の問題を根本的に解くのには、どうしてもまず遺伝に関するすべての事がらを明らかにしておく必要があるのです。
\\	生物に関するこのような問題は、ほんとうはなかなか複雑であって、今でもそれがすっかりわかっていないばかりでなく、まだ私たちに知られていない巧妙な事がらが自然にはたくさんにあるにちがいないのです。ですから、生命や遺伝に関する問題をすっかり解決することは、いつになって出来るかわかりませんけれども、しかしそれらを研究してゆくことは、実に自然の神秘に触れることにもなるという点で、限りなく興味のある事がらでありますから、たくさんの生物学者はこれがために多くの苦心を重ねてもいるのであります。すべて科学の上の問題は、最初にはまるでどのように手をつけてよいかわからないように見えても、やがてそれが解決される日のあることは、これまでの多くの経験から見て恐らく確かなので、ですから生命や遺伝の問題にしても、いつかはその極めて奥ぶかい謎を解くことができるようにならないとは云《い》われないのでしょう。
\\	ここで遺伝に関するメンデルの最初の仕事を思うにつけても、私たちはその学問のますます進むことを心から望まないわけにゆきません。なぜなら、これこそ昔から最も不思議な謎と見られている生命の問題を解く有力な鍵ともなるからであります。
\\	底本:「偉い科學者」實業之日本社
\\	1942(昭和17)年10月10日発行
\\	※「旧字、旧仮名で書かれた作品を、現代表記にあらためる際の作業指針」に基づいて、底本の表記をあらためました。
\\	「先ず」は「まず」に、「漸く」は「ようやく」に、「及び」は「および」に、「之」は「これ」に、「既に」は「すでに」に、「併し」は「しかし」に、「於て」「於ける」は「おいて」「おける」に、置き換えました。
\\	※読みにくい言葉、読み誤りやすい言葉に振り仮名を付しました。底本には振り仮名が付されていません。
\\	※「緑色」と「緑いろ」の混在は、底本通りです。
\\	※国立国会図書館デジタルコレクション
\\	で公開されている当該書籍画像に基づいて、作業しました。
\\	入力:高瀬竜一
\\	校正:
\\	2018年6月27日作成
\\	青空文庫作成ファイル:
\\	このファイルは、インターネットの図書館、青空文庫
\\	で作られました。入力、校正、制作にあたったのは、ボランティアの皆さんです。
\end{CJK}
\end{document}