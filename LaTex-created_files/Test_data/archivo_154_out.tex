\documentclass[8pt]{extreport} 
\usepackage{hyperref}
\usepackage{CJKutf8}
\begin{document}
\begin{CJK}{UTF8}{min}
\\	奥	
\\	おく			
\\	はさみは机の奥にあった。	
\\	奥 
\\	はさみ は つくえ の おく に あった			
\\	得意	
\\	とくい			
\\	彼は歌が得意です。	
\\	得意 
\\	かれ は うた が とくい です			
\\	下がる	
\\	さがる			
\\	やっと熱が下がった。	
\\	下がる 
\\	やっと ねつ が さがった			
\\	レストラン	
\\	レストランでインド料理を食べました。	
\\	レストラン 
\\	れすとらん で いんど りょうり を たべました			
\\	医者	
\\	いしゃ			
\\	私は医者に相談した。	
\\	医者 
\\	わたし は いしゃ に そうだん した			
\\	カード	
\\	支払いはカードでお願いします。	
\\	カード 
\\	しはらい は かーど で おねがい します			
\\	吹く	
\\	ふく			
\\	今日は北風が吹いている。	
\\	吹く 
\\	きょう は きたかぜ が ふいて いる			
\\	信号	
\\	しんごう			
\\	絵	
\\	え			
\\	これは有名な画家の絵です。	
\\	絵 
\\	これ は ゆうめい な がか の え です			
\\	修理	
\\	しゅうり			
\\	車を修理に出した。	
\\	修理 
\\	くるま を しゅうり に だした			
\\	地下鉄	
\\	ちかてつ			
\\	赤ちゃん	
\\	あかちゃん			
\\	面白い	
\\	おもしろい			
\\	この本はすごく面白かった。	
\\	面白い 
\\	この ほん は すごく おもしろかった			
\\	離婚	
\\	りこん			
\\	友人が離婚しました。	
\\	離婚 
\\	ゆうじん が りこん しました			
\\	小学生	
\\	しょうがくせい			
\\	うちの息子は来年、小学生になります。	
\\	小学生 
\\	うち の むすこ は らいねん しょうがくせい に なります			
\\	育つ	
\\	そだつ			
\\	野菜がよく育っている。	
\\	育つ 
\\	やさい が よく そだって いる			
\\	数える	
\\	かぞえる			
\\	いすの数を数えてください。	
\\	数える 
\\	いす の かず を かぞえて ください			
\\	アルバイト	
\\	兄はアルバイトをしています。	
\\	アルバイト 
\\	あに は あるばいと を して います			
\\	買い物	
\\	かいもの			
\\	母は買い物に出かけています。	
\\	買い物 
\\	はは は かいもの に でかけて います			
\\	石	
\\	いし			
\\	私は石につまづいた。	
\\	石 
\\	わたし は いし に つまづいた			
\\	薬	
\\	くすり			
\\	この薬を必ず飲んでください。	
\\	薬 
\\	この くすり を かならず のんで ください			
\\	毎年	
\\	まいねん			
\\	毎年給料が上がる。	
\\	毎年 
\\	まいねん きゅうりょう が あがる			
\\	回す	
\\	まわす			
\\	ねじは左に回すと外れます。	
\\	回す 
\\	ねじ は ひだり に まわす と はずれます			
\\	コピー	
\\	会議で書類のコピーを配った。	
\\	コピー 
\\	かいぎ で しょるい の こぴー を くばった			
\\	ぶつかる	
\\	車が電柱にぶつかった。	
\\	ぶつかる 
\\	くるま が でんちゅう に ぶつかった			
\\	運転手	
\\	うんてんしゅ			
\\	彼はタクシーの運転手です。	
\\	運転手 
\\	かれ は たくしー の うんてんしゅ です			
\\	教会	
\\	きょうかい			
\\	私たちは教会で結婚式をしました。	
\\	教会 
\\	わたしたち は きょうかい で けっこんしき を しました			
\\	フィルム	
\\	旅行のためにたくさんフィルムを買った。	
\\	フィルム 
\\	りょこう の ため に たくさん ふぃるむ を かった			
\\	未来	
\\	みらい			
\\	未来は誰にも分からない。	
\\	未来 
\\	みらい は だれ に も わからない			
\\	校長	
\\	こうちょう			
\\	あの人は高校の校長だ。	
\\	校長 
\\	あの ひと は こうこう の こうちょう だ			
\\	デパート	
\\	私はデパートで靴を買った。	
\\	デパート 
\\	わたし は でぱーと で くつ を かった			
\\	見送る	
\\	みおくる			
\\	彼が外国に行くのを見送りました。	
\\	見送る 
\\	かれ が がいこく に いく の を みおくりました			
\\	向く	
\\	むく			
\\	こっちを向いてください。	
\\	向く 
\\	こっち を むいて ください			
\\	揺れる	
\\	ゆれる			
\\	風で木が揺れています。	
\\	揺れる 
\\	かぜ で き が ゆれて います			
\\	髪	
\\	かみ			
\\	昨日、髪を切りました。	
\\	髪 
\\	きのう かみ を きりました			
\\	箱	
\\	はこ			
\\	この箱は重い。	
\\	箱 
\\	この はこ は おもい			
\\	思い出	
\\	おもいで			
\\	旅行で楽しい思い出ができました。	
\\	思い出 
\\	りょこう で たのしい おもいで が できました			
\\	ベッド	
\\	ベッドで赤ちゃんが眠っています。	
\\	赤ちゃん 
\\	ベッド 
\\	べっど で あかちゃん が ねむって います			
\\	彼はベッドで寝ています。	
\\	ベッド 
\\	かれ は べっど で ねて います			
\\	新幹線	
\\	しんかんせん			
\\	新幹線で京都に行きました。	
\\	新幹線 
\\	しんかんせん で きょうと に いきました			
\\	生む	
\\	うむ			
\\	うちのネコが子猫を生みました。	
\\	生む 
\\	うち の ねこ が こねこ を うみました			
\\	奥さん	
\\	おくさん			
\\	彼の奥さんはきれいな方です。	
\\	奥さん 
\\	かれ の おくさん は きれい な かた です			
\\	階段	
\\	かいだん			
\\	私たちは駅の階段をかけ上がった。	
\\	階段 
\\	わたしたち は えき の かいだん を かけあがった			
\\	招待	
\\	しょうたい			
\\	高校の時の先生を結婚式に招待した。	
\\	招待 
\\	こうこう の とき の せんせい を けっこんしき に しょうたい した			
\\	コート	
\\	寒かったのでコートを着た。	
\\	コート 
\\	さむかった の で こーと を きた			
\\	燃える	
\\	もえる			
\\	山が燃えています。	
\\	燃える 
\\	やま が もえて います			
\\	食堂	
\\	しょくどう			
\\	大学の食堂は安い。	
\\	食堂 
\\	だいがく の しょくどう は やすい			
\\	天井	
\\	てんじょう			
\\	この部屋は天井が高いですね。	
\\	天井 
\\	この へや は てんじょう が たかい です ね			
\\	課	
\\	か			
\\	今から課のミーティングがある。	
\\	課 
\\	いま から か の みーてぃんぐ が ある			
\\	靴	
\\	くつ			
\\	靴が汚れた。	
\\	靴 
\\	くつ が よごれた			
\\	予約	
\\	よやく			
\\	レストランを予約しました。	
\\	レストラン 
\\	予約 
\\	れすとらん を よやく しました			
\\	一昨年	
\\	おととし			
\\	一昨年初めて京都へ旅行しました。	
\\	一昨年 
\\	おととし はじめて きょうと へ りょこう しました			
\\	休日	
\\	きゅうじつ			
\\	休日は家でよくテレビを見ます。	
\\	休日 
\\	きゅうじつ は いえ で よく てれび を みます			
\\	注文	
\\	ちゅうもん			
\\	レストランでピザを注文しました。	
\\	注文 
\\	れすとらん で ぴざ を ちゅうもん しました			
\\	家賃	
\\	やちん			
\\	ここの家賃は12万円です。	
\\	家賃 
\\	ここ の やちん は じゅうにまんえん です			
\\	散歩	
\\	さんぽ			
\\	公園を散歩しようか。	
\\	散歩 
\\	こうえん を さんぽ しよう か			
\\	通勤	
\\	つうきん			
\\	私は地下鉄で通勤しています。	
\\	地下鉄 
\\	通勤 
\\	わたし は ちかてつ で つうきん して います			
\\	毎朝、通勤に30分かかります。	
\\	通勤 
\\	まいあさ つうきん に さんじゅっぷん かかります			
\\	ノート	
\\	ノートを開いてください。	
\\	ノート 
\\	のーと を ひらいて ください			
\\	辞める	
\\	やめる			
\\	彼は会社を辞めます。	
\\	辞める 
\\	かれ は かいしゃ を やめます			
\\	仕方	
\\	しかた			
\\	ファイルのダウンロードの仕方が分かりません。	
\\	仕方 
\\	ふぁいる の だうんろーど の しかた が わかりません			
\\	台所	
\\	だいどころ			
\\	お母さんは台所にいます。	
\\	台所 
\\	おかあさん は だいどころ に います			
\\	建つ	
\\	たつ			
\\	ここに来年、家が建ちます。	
\\	建つ 
\\	ここ に らいねん いえ が たちます			
\\	踏む	
\\	ふむ			
\\	運転手がブレーキを踏んだ。	
\\	運転手 
\\	踏む 
\\	うんてんしゅ が ぶれーき を ふんだ			
\\	配る	
\\	くばる			
\\	会議で書類を配った。	
\\	配る 
\\	かいぎ で しょるい を くばった			
\\	握る	
\\	にぎる			
\\	少女は母親の手を握った。	
\\	握る 
\\	しょうじょ は ははおや の て を にぎった			
\\	練習	
\\	れんしゅう			
\\	娘は今、バイオリンを練習しています。	
\\	練習 
\\	むすめ は いま ばいおりん を れんしゅう して います			
\\	強さ	
\\	つよさ			
\\	風の強さに驚きました。	
\\	強さ 
\\	かぜ の つよさ に おどろきました			
\\	久しぶり	
\\	ひさしぶり			
\\	明日、久しぶりに友達に会います。	
\\	久しぶり 
\\	あした ひさしぶり に ともだち に あいます			
\\	往復	
\\	おうふく			
\\	往復切符をください。	
\\	往復 
\\	おうふく きっぷ を ください			
\\	大勢	
\\	おおぜい			
\\	大勢で食事に出かけました。	
\\	大勢 
\\	おおぜい で しょくじ に でかけました			
\\	親しい	
\\	したしい			
\\	週末、親しい友達を家に呼んだ。	
\\	親しい 
\\	しゅうまつ したしい ともだち を いえ に よんだ			
\\	番号	
\\	ばんごう			
\\	この番号に電話してください。	
\\	番号 
\\	この ばんごう に でんわ して ください			
\\	ワイン	
\\	ワインを少し飲みました。	
\\	ワイン 
\\	わいん を すこし のみました			
\\	巻く	
\\	まく			
\\	彼は頭にタオルを巻いていた。	
\\	巻く 
\\	かれ は あたま に たおる を まいて いた			
\\	青い	
\\	あおい			
\\	ここの海はとても青い。	
\\	青い 
\\	ここ の うみ は とても あおい			
\\	おかしい	
\\	彼の様子がおかしい。	
\\	おかしい 
\\	かれ の ようす が おかしい			
\\	渡る	
\\	わたる			
\\	私たちは歩いて橋を渡った。	
\\	渡る 
\\	わたしたち は あるいて はし を わたった			
\\	牛	
\\	うし			
\\	牛が草を食べています。	
\\	牛 
\\	うし が くさ を たべて います			
\\	氷	
\\	こおり			
\\	グラスに氷を入れてください。	
\\	氷 
\\	ぐらす に こおり を いれて ください			
\\	挨拶	
\\	あいさつ			
\\	彼女は笑顔で挨拶した。	
\\	挨拶 
\\	かのじょ は えがお で あいさつ した			
\\	洗う	
\\	あらう			
\\	早く顔を洗いなさい。	
\\	洗う 
\\	はやく かお を あらいなさい			
\\	尋ねる	
\\	たずねる			
\\	彼女は道を尋ねた。	
\\	尋ねる 
\\	かのじょ は みち を たずねた			
\\	叩く	
\\	たたく			
\\	彼は子供のおしりを叩いた。	
\\	叩く 
\\	かれ は こども の おしり を たたいた			
\\	廊下	
\\	ろうか			
\\	廊下は走らないでください。	
\\	廊下 
\\	ろうか は はしらない で ください			
\\	郵便局	
\\	ゆうびんきょく			
\\	郵便局はどこですか。	
\\	郵便局 
\\	ゆうびんきょく は どこ です か			
\\	割る	
\\	わる			
\\	皿を落として割った。	
\\	割る 
\\	さら を おとして わった			
\\	進学	
\\	しんがく			
\\	彼女は来年、大学に進学する。	
\\	進学 
\\	かのじょ は らいねん だいがく に しんがく する			
\\	トイレ	
\\	トイレを掃除しましたか。	
\\	トイレ 
\\	といれ を そうじ しました か			
\\	咲く	
\\	さく			
\\	桜の花が咲きました。	
\\	咲く 
\\	さくら の はな が さきました			
\\	駐車場	
\\	ちゅうしゃじょう			
\\	車は駐車場に止めてください。	
\\	駐車場 
\\	くるま は ちゅうしゃじょう に とめて ください			
\\	泣く	
\\	なく			
\\	妹はすぐに泣く。	
\\	泣く 
\\	いもうと は すぐ に なく			
\\	屋根	
\\	やね			
\\	屋根にカラスが止まっています。	
\\	屋根 
\\	やね に からす が とまって います			
\\	案内	
\\	あんない			
\\	私が中をご案内します。	
\\	案内 
\\	わたし が なか を ごあんない します			
\\	キャンプ	
\\	友達とキャンプに行った。	
\\	キャンプ 
\\	ともだち と きゃんぷ に いった			
\\	荷物	
\\	にもつ			
\\	彼は大きな荷物を持ってきた。	
\\	荷物 
\\	かれ は おおき な にもつ を もって きた			
\\	迷惑	
\\	めいわく			
\\	人に迷惑をかけてはいけません。	
\\	迷惑 
\\	ひと に めいわく を かけて は いけません			
\\	先月	
\\	せんげつ			
\\	先月、友達の結婚式があった。	
\\	先月 
\\	せんげつ ともだち の けっこんしき が あった			
\\	美術館	
\\	びじゅつかん			
\\	昨日、美術館に行きました。	
\\	美術館 
\\	きのう びじゅつかん に いきました			
\\	破る	
\\	やぶる			
\\	彼は約束を破った。	
\\	破る 
\\	かれ は やくそく を やぶった			
\\	愛	
\\	あい			
\\	彼女は愛をこめて手紙を書いた。	
\\	愛 
\\	かのじょ は あい を こめて てがみ を かいた			
\\	黒	
\\	くろ			
\\	黒のボールペンはありますか。	
\\	黒 
\\	くろ の ぼーるぺん は あります か			
\\	断る	
\\	ことわる			
\\	私は彼のプロポーズを断った。	
\\	断る 
\\	わたし は かれ の ぷろぽーず を ことわった			
\\	品物	
\\	しなもの			
\\	その店は色々な品物を売っている。	
\\	品物 
\\	その みせ は いろいろ な しなもの を うって いる			
\\	プラスチック	
\\	このカップはプラスチックです。	
\\	プラスチック 
\\	この かっぷ は ぷらすちっく です			
\\	貸す	
\\	かす			
\\	私は彼に本を貸しています。	
\\	貸す 
\\	わたし は かれ に ほん を かして います			
\\	沈む	
\\	しずむ			
\\	光る	
\\	ひかる			
\\	波がきらきら光っていました。	
\\	光る 
\\	なみ が きらきら ひかって いました			
\\	右手	
\\	みぎて			
\\	私は右手で字を書きます。	
\\	右手 
\\	わたし は みぎて で じ を かきます			
\\	赤	
\\	あか			
\\	信号が赤に変わりました。	
\\	赤 
\\	しんごう が あか に かわりました			
\\	卵	
\\	たまご			
\\	ニワトリは卵を産みます。	
\\	卵 
\\	にわとり は たまご を うみます			
\\	疲れる	
\\	つかれる			
\\	男の子	
\\	おとこのこ			
\\	男の子たちがサッカーをしている。	
\\	男の子 
\\	おとこのこたち が さっかー を して いる			
\\	泳ぐ	
\\	およぐ			
\\	彼女はダイエットのために泳いでいる。	
\\	泳ぐ 
\\	かのじょ は だいえっと の ため に およいで いる			
\\	内側	
\\	うちがわ			
\\	白線の内側に下がってください。	
\\	内側 
\\	はくせん の うちがわ に さがって ください			
\\	習慣	
\\	しゅうかん			
\\	毎朝コーヒーを飲むのが習慣です。	
\\	習慣 
\\	まいあさ こーひー を のむ の が しゅうかん です			
\\	投げる	
\\	なげる			
\\	ボールをこっちに投げてください。	
\\	投げる 
\\	ぼーる を こっち に なげて ください			
\\	カラー	
\\	カラーコピーは一枚幾らですか。	
\\	カラー 
\\	からーこぴー は いちまい いくら です か			
\\	正月	
\\	しょうがつ			
\\	お正月にはたいてい、家族が集まる。	
\\	正月 
\\	おしょうがつ に は たいてい かぞく が あつまる			
\\	趣味	
\\	しゅみ			
\\	私の趣味は映画とテニスです。	
\\	趣味 
\\	わたし の しゅみ は えいが と てにす です			
\\	盗む	
\\	ぬすむ			
\\	誰かが私のカバンを盗みました。	
\\	盗む 
\\	だれか が わたし の かばん を ぬすみました			
\\	ピアノ	
\\	昔、ピアノを習っていました。	
\\	ピアノ 
\\	むかし ぴあの を ならって いました			
\\	池	
\\	いけ			
\\	池に鯉がいます。	
\\	池 
\\	いけ に こい が います			
\\	大学院	
\\	だいがくいん			
\\	彼は大学院に進みました。	
\\	大学院 
\\	かれ は だいがくいん に すすみました			
\\	何度	
\\	なんど			
\\	あの店には何度も行きました。	
\\	何度 
\\	あの みせ に は なんど も いきました			
\\	不幸	
\\	ふこう			
\\	彼女の家族に不幸があった。	
\\	不幸 
\\	かのじょ の かぞく に ふこう が あった			
\\	恐ろしい	
\\	おそろしい			
\\	昨夜恐ろしい夢を見た。	
\\	恐ろしい 
\\	ゆうべ おそろしい ゆめ を みた			
\\	退院	
\\	たいいん			
\\	母が今日退院します。	
\\	退院 
\\	はは が きょう たいいん します			
\\	付く	
\\	つく			
\\	靴に泥が付いています。	
\\	付く 
\\	くつ に どろ が ついて います			
\\	重さ	
\\	おもさ			
\\	この荷物の重さを計ってください。	
\\	重さ 
\\	この にもつ の おもさ を はかって ください			
\\	商業	
\\	しょうぎょう			
\\	この町では商業が盛んだ。	
\\	商業 
\\	この まち で は しょうぎょう が さかん だ			
\\	スキー	
\\	冬はよくスキーに行きます。	
\\	スキー 
\\	ふゆ は よく すきー に いきます			
\\	敷く	
\\	しく			
\\	生まれて初めて布団を敷いた。	
\\	敷く 
\\	うまれて はじめて ふとん を しいた			
\\	安心	
\\	あんしん			
\\	それを聞いて安心しました。	
\\	安心 
\\	それ を きいて あんしん しました			
\\	糸	
\\	いと			
\\	母は糸を針に通した。	
\\	糸 
\\	はは は いと を はり に とおした			
\\	なかなか	
\\	荷物がなかなか届きません。	
\\	なかなか 
\\	にもつ が なかなか とどきません			
\\	人形	
\\	にんぎょう			
\\	彼女は人形をたくさん持っています。	
\\	人形 
\\	かのじょ は にんぎょう を たくさん もって います			
\\	プール	
\\	私は夏休みにプールに行った。	
\\	プール 
\\	わたし は なつやすみ に ぷーる に いった			
\\	銀	
\\	ぎん			
\\	彼は銀メダルを取った。	
\\	銀 
\\	かれ は ぎんめだる を とった			
\\	匂い	
\\	におい			
\\	いい匂いがする。	
\\	匂い 
\\	いい におい が する			
\\	ホーム	
\\	もうすぐこのホームに電車が来ます。	
\\	ホーム 
\\	もうすぐ この ほーむ に でんしゃ が きます			
\\	交通事故	
\\	こうつうじこ			
\\	彼は交通事故を起こした。	
\\	交通事故 
\\	かれ は こうつうじこ を おこした			
\\	幸せ	
\\	しあわせ			
\\	良い友達がいて私は幸せだ。	
\\	幸せ 
\\	いい ともだち が いて わたし は しあわせ だ			
\\	吸う	
\\	すう			
\\	彼は大きく息を吸った。	
\\	吸う 
\\	かれ は おおきく いき を すった			
\\	浅い	
\\	あさい			
\\	この川は浅いです。	
\\	浅い 
\\	この かわ は あさい です			
\\	見学	
\\	けんがく			
\\	今日、工場の見学に行きました。	
\\	見学 
\\	きょう こうじょう の けんがく に いきました			
\\	濃い	
\\	こい			
\\	私は濃い味が好きだ。	
\\	濃い 
\\	わたし は こい あじ が すき だ			
\\	壊す	
\\	こわす			
\\	彼女が私のケータイを壊した。	
\\	壊す 
\\	かのじょ が わたし の けーたい を こわした			
\\	間に合う	
\\	まにあう			
\\	授業に間に合いました。	
\\	間に合う 
\\	じゅぎょう に まにあいました			
\\	届ける	
\\	とどける			
\\	これを彼に届けてください。	
\\	届ける 
\\	これ を かれ に とどけて ください			
\\	葉	
\\	は			
\\	エレベーター	
\\	エレベーターで下に降りましょう。	
\\	エレベーター 
\\	えれべーたー で した に おりましょう			
\\	メモ	
\\	メモを取ってください。	
\\	メモ 
\\	めも を とって ください			
\\	立派	
\\	りっぱ			
\\	彼は立派な人です。	
\\	立派 
\\	かれ は りっぱ な ひと です			
\\	旅館	
\\	りょかん			
\\	京都では旅館に泊まりました。	
\\	旅館 
\\	きょうと で は りょかん に とまりました			
\\	凄い	
\\	すごい			
\\	凄い雨になった。	
\\	凄い 
\\	すごい あめ に なった			
\\	パン	
\\	朝ご飯にはいつもパンを食べる。	
\\	パン 
\\	あさごはん に は いつも ぱん を たべる			
\\	びっくりする	
\\	大きな音にびっくりしました。	
\\	びっくりする 
\\	おおきな おと に びっくり しました			
\\	便利	
\\	べんり			
\\	インターネットはとても便利です。	
\\	便利 
\\	いんたーねっと は とても べんり です			
\\	砂糖	
\\	さとう			
\\	コーヒーに砂糖は入れますか。	
\\	砂糖 
\\	こーひー に さとう は いれます か			
\\	休み	
\\	やすみ			
\\	今日は休みです。	
\\	休み 
\\	きょう は やすみ です			
\\	録音	
\\	ろくおん			
\\	そのラジオ番組はもう録音しました。	
\\	録音 
\\	その らじお ばんぐみ は もう ろくおん しました			
\\	言い方	
\\	いいかた			
\\	そんな言い方をしてはいけません。	
\\	言い方 
\\	そんな いいかた を して は いけません			
\\	帰り	
\\	かえり			
\\	仕事の帰りにビールを飲んだ。	
\\	帰り 
\\	しごと の かえり に びーる を のんだ			
\\	週末	
\\	しゅうまつ			
\\	週末は家でゆっくりします。	
\\	週末 
\\	しゅうまつ は うち で ゆっくり します			
\\	ズボン	
\\	ズボンが汚れた。	
\\	ズボン 
\\	ずぼん が よごれた			
\\	足りる	
\\	たりる			
\\	お金が足りなくて買えなかった。	
\\	足りる 
\\	おかね が たりなくて かえなかった			
\\	拾う	
\\	ひろう			
\\	道で財布を拾った。	
\\	拾う 
\\	みち で さいふ を ひろった			
\\	祭り	
\\	まつり			
\\	彼女は祭りが大好きです。	
\\	祭り 
\\	かのじょ は まつり が だいすき です 。			
\\	博物館	
\\	はくぶつかん			
\\	昨日、車の博物館に行った。	
\\	博物館 
\\	きのう くるま の はくぶつかん に いった			
\\	青	
\\	あお			
\\	信号が青になった。	
\\	信号 
\\	青 
\\	しんごう が あお に なった			
\\	私の好きな色は青です。	
\\	青 
\\	わたし の すき な いろ は あお です			
\\	売り場	
\\	うりば			
\\	くつ売り場はどこですか。	
\\	売り場 
\\	くつうりば は どこ です か			
\\	おもちゃ	
\\	赤ちゃんが自動車のおもちゃで遊んでいる。	
\\	おもちゃ 
\\	あかちゃん が じどうしゃ の おもちゃ で あそんで いる			
\\	金持ち	
\\	かねもち			
\\	彼は金持ちです。	
\\	金持ち 
\\	かれ は かねもち です			
\\	頑張る	
\\	がんばる			
\\	明日の試験、頑張ってね。	
\\	頑張る 
\\	あした の しけん がんばって ね			
\\	グラム	
\\	ひき肉を200グラムください。	
\\	グラム 
\\	ひきにく を にひゃくぐらむ ください			
\\	コーヒー	
\\	私は毎朝コーヒーを飲みます。	
\\	コーヒー 
\\	わたし は まいあさ こーひー を のみます			
\\	桜	
\\	さくら			
\\	これは桜の葉です。	
\\	葉 
\\	桜 
\\	これ は さくら の は です			
\\	桜は三月か四月に咲きます。	
\\	桜 
\\	さくら は さんがつ か しがつ に さきます			
\\	刺す	
\\	さす			
\\	虫に腕を刺されました。	
\\	刺す 
\\	むし に うで を さされました			
\\	テント	
\\	みんなでテントを張りました。	
\\	テント 
\\	みんな で てんと を はりました			
\\	ボート	
\\	ボートが川に沈んだ。	
\\	沈む 
\\	ボート 
\\	ぼーと が かわ に しずんだ			
\\	池でボートに乗りました。	
\\	ボート 
\\	いけ で ぼーと に のりました			
\\	遭う	
\\	あう			
\\	彼は交通事故に遭った。	
\\	遭う 
\\	かれ は こうつう じこ に あった			
\\	ボール	
\\	彼はボールを投げた。	
\\	投げる 
\\	ボール 
\\	かれ は ぼーる を なげた			
\\	オートバイ	
\\	彼はオートバイに乗っている。	
\\	オートバイ 
\\	かれ は おーとばい に のって いる			
\\	外人	
\\	がいじん			
\\	この町には外人が少ない。	
\\	外人 
\\	この まち に は がいじん が すくない			
\\	猫	
\\	ねこ			
\\	私は猫が大好きです。	
\\	猫 
\\	わたし は ねこ が だいすき です			
\\	隠れる	
\\	かくれる			
\\	太陽が雲に隠れた。	
\\	隠れる 
\\	たいよう が くも に かくれた			
\\	親切	
\\	しんせつ			
\\	親切にしてくださってどうもありがとうございます。	
\\	親切 
\\	しんせつ に して くださって どうも ありがとう ございます			
\\	是非	
\\	ぜひ			
\\	是非、うちに来てください。	
\\	是非 
\\	ぜひ うち に きて ください			
\\	通学	
\\	つうがく			
\\	毎朝、通学に1時間かかる。	
\\	通学 
\\	まいあさ つうがく に いちじかん かかる			
\\	預ける	
\\	あずける			
\\	鍵を彼に預けた。	
\\	預ける 
\\	かぎ を かれ に あずけた			
\\	付ける	
\\	つける			
\\	おまけを付けました。	
\\	付ける 
\\	おまけ を つけました			
\\	習う	
\\	ならう			
\\	私はダンスを習っています。	
\\	習う 
\\	わたし は だんす を ならって います			
\\	組み立てる	
\\	くみたてる			
\\	日曜日に本棚を組み立てます。	
\\	組み立てる 
\\	にちようび に ほんだな を くみたてます			
\\	塗る	
\\	ぬる			
\\	壁にペンキを塗っています。	
\\	塗る 
\\	かべ に ぺんき を ぬって います			
\\	左手	
\\	ひだりて			
\\	彼女は左手で字を書く。	
\\	左手 
\\	かのじょ は ひだりて で じ を かく			
\\	ひどい	
\\	妹とひどい喧嘩をした。	
\\	ひどい 
\\	いもうと と ひどい けんか を した			
\\	別々	
\\	べつべつ			
\\	別々に払いましょう。	
\\	別々 
\\	べつべつ に はらいましょう			
\\	毎週	
\\	まいしゅう			
\\	私は毎週母に電話をします。	
\\	毎週 
\\	わたし は まいしゅう はは に でんわ を します			
\\	気温	
\\	きおん			
\\	今日の気温は26度です。	
\\	気温 
\\	きょう の きおん は にじゅうろくど です			
\\	蹴る	
\\	ける			
\\	ゴールキーパーがボールを蹴った。	
\\	蹴る 
\\	ゴールキーパー が ボール を けった。			
\\	真っ赤	
\\	まっか			
\\	彼の顔は真っ赤でした。	
\\	真っ赤 
\\	かれ の かお は まっか でした			
\end{CJK}
\end{document}