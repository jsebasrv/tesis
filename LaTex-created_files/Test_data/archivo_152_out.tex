\documentclass[8pt]{extreport} 
\usepackage{hyperref}
\usepackage{CJKutf8}
\begin{document}
\begin{CJK}{UTF8}{min}
\\	出口	
\\	でぐち			
\\	出口はあそこです。	
\\	出口 
\\	でぐち は あそこ です			
\\	登る	
\\	のぼる			
\\	私たちは昨年、富士山に登りました。	
\\	登る 
\\	わたしたち は さくねん ふじさん に のぼりました			
\\	真っ白	
\\	まっしろ			
\\	外は雪で真っ白だった。	
\\	真っ白 
\\	そと は ゆき で まっしろ だった			
\\	おじ	
\\	おじは銀行に勤めています。	
\\	おじ 
\\	おじ は ぎんこう に つとめて います			
\\	下ろす	
\\	おろす			
\\	棚からその箱を下ろしてください。	
\\	下ろす 
\\	たな から その はこ を おろして ください			
\\	貸し出す	
\\	かしだす			
\\	その本は貸し出し中です。	
\\	貸し出す 
\\	その ほん は かしだしちゅう です			
\\	サッカー	
\\	彼はサッカーの選手です。	
\\	サッカー 
\\	かれ は さっかー の せんしゅ です			
\\	暖房	
\\	だんぼう			
\\	冬は暖房が必要です。	
\\	暖房 
\\	ふゆ は だんぼう が ひつよう です			
\\	手袋	
\\	てぶくろ			
\\	寒いので手袋をしました。	
\\	手袋 
\\	さむい の で てぶくろ を しました			
\\	留守	
\\	るす			
\\	父は今、留守です。	
\\	留守 
\\	ちち は いま るす です			
\\	嘘	
\\	うそ			
\\	嘘をついてはいけません。	
\\	嘘 
\\	うそ を ついて は いけません			
\\	遠慮	
\\	えんりょ			
\\	私は遠慮します。	
\\	遠慮 
\\	わたし は えんりょ します			
\\	折る	
\\	おる			
\\	祖父は足の骨を折りました。	
\\	折る 
\\	そふ は あし の ほね を おりました			
\\	今夜	
\\	こんや			
\\	今夜は月がとてもきれいです。	
\\	今夜 
\\	こんや は つき が とても きれい です			
\\	スープ	
\\	母がコーンスープを作っている。	
\\	スープ 
\\	はは が こーんすーぷ を つくって いる			
\\	バナナ	
\\	私は毎朝バナナを食べます。	
\\	バナナ 
\\	わたし は まいあさ ばなな を たべます			
\\	うるさい	
\\	この通りは車の音がうるさい。	
\\	うるさい 
\\	この とおり は くるま の おと が うるさい			
\\	可哀相	
\\	かわいそう			
\\	その可哀相な子供たちは食べるものがない。	
\\	可哀相 
\\	その かわいそう な こどもたち は たべる もの が ない			
\\	パトカー	
\\	あそこにパトカーがいる。	
\\	パトカー 
\\	あそこ に ぱとかー が いる			
\\	ハンバーガー	
\\	今日の昼ご飯はハンバーガーでした。	
\\	ハンバーガー 
\\	きょう の ひるごはん は はんばーがー でした			
\\	毛布	
\\	もうふ			
\\	この毛布は暖かい。	
\\	毛布 
\\	この もうふ は あたたかい			
\\	エスカレーター	
\\	3階までエスカレーターで行きましょう。	
\\	エスカレーター 
\\	さんかい まで えすかれーたー で いきましょう			
\\	お嬢さん	
\\	おじょうさん			
\\	お嬢さんはおいくつですか。	
\\	お嬢さん 
\\	おじょうさん は おいくつ です か			
\\	タオル	
\\	私はタオルで顔をふいた。	
\\	タオル 
\\	わたし は たおる で かお を ふいた			
\\	手帳	
\\	てちょう			
\\	新しい手帳を買いました。	
\\	手帳 
\\	あたらしい てちょう を かいました			
\\	乗り換える	
\\	のりかえる			
\\	次の駅で地下鉄に乗り換えます。	
\\	乗り換える 
\\	つぎ の えき で ちかてつ に のりかえます			
\\	売店	
\\	ばいてん			
\\	駅の売店で雑誌を買った。	
\\	売店 
\\	えき の ばいてん で ざっし を かった			
\\	パチンコ	
\\	彼は毎日パチンコをしています。	
\\	パチンコ 
\\	かれ は まいにち ぱちんこ を して います			
\\	美人	
\\	びじん			
\\	彼のお母さんは美人です。	
\\	美人 
\\	かれ の おかあさん は びじん です			
\\	本屋	
\\	ほんや			
\\	駅前に本屋があります。	
\\	本屋 
\\	えきまえ に ほんや が あります			
\\	みかん	
\\	みかんを1つください。	
\\	みかん 
\\	みかん を ひとつ ください			
\\	謝る	
\\	あやまる			
\\	彼は直ぐに謝りました。	
\\	謝る 
\\	かれ は すぐ に あやまりました			
\\	ケーキ	
\\	誕生日にケーキを食べました。	
\\	ケーキ 
\\	たんじょうび に けーき を たべました			
\\	コップ	
\\	私は毎朝コップ一杯の水を飲む。	
\\	コップ 
\\	わたし は まいあさ こっぷ いっぱい の みず を のむ			
\\	四季	
\\	しき			
\\	日本には四季がある。	
\\	四季 
\\	にほん に は しき が ある			
\\	田んぼ	
\\	たんぼ			
\\	この辺は田んぼがたくさんあります。	
\\	田んぼ 
\\	このへん は たんぼ が たくさん あります			
\\	天気予報	
\\	てんきよほう			
\\	明日の天気予報は雨です。	
\\	天気予報 
\\	あした の てんきよほう は あめ です			
\\	ナイロン	
\\	このジャケットはナイロンでできています。	
\\	ナイロン 
\\	この じゃけっと は ないろん で できて います			
\\	変	
\\	へん			
\\	変な音が聞こえます。	
\\	変 
\\	へん な おと が きこえます			
\\	来週	
\\	らいしゅう			
\\	続きは来週やりましょう。	
\\	来週 
\\	つづき は らいしゅう やりましょう			
\\	一生懸命	
\\	いっしょうけんめい			
\\	彼は毎日一生懸命働いている。	
\\	一生懸命 
\\	かれ は まいにち いっしょうけんめい はたらいて いる			
\\	乾く	
\\	かわく			
\\	夏は洗濯物がすぐ乾く。	
\\	乾く 
\\	なつ は せんたくもの が すぐ かわく			
\\	皿	
\\	さら			
\\	空いたお皿をお下げします。	
\\	皿 
\\	あいた おさら を おさげ します			
\\	間違う	
\\	まちがう			
\\	あなたは間違っている。	
\\	間違う 
\\	あなた は まちがって いる			
\\	事務室	
\\	じむしつ			
\\	事務室でコピーを取って来ます。	
\\	事務室 
\\	じむしつ で こぴー を とって きます			
\\	消防車	
\\	しょうぼうしゃ			
\\	消防車が4台も来た。	
\\	消防車 
\\	しょうぼうしゃ が よんだい も きた			
\\	スーツ	
\\	あのスーツはそんなに高くない。	
\\	スーツ 
\\	あの すーつ は そんなに たかく ない			
\\	チケット	
\\	この遊園地のチケットは3000円です。	
\\	チケット 
\\	この ゆうえんち の ちけっと は さんぜんえん です			
\\	チョコレート	
\\	妹はチョコレートが大好きです。	
\\	チョコレート 
\\	いもうと は ちょこれーと が だいすき です			
\\	泥棒	
\\	どろぼう			
\\	近所に泥棒が入った。	
\\	泥棒 
\\	きんじょ に どろぼう が はいった			
\\	脱ぐ	
\\	ぬぐ			
\\	靴を脱いでください。	
\\	脱ぐ 
\\	くつ を ぬいで ください			
\\	年賀状	
\\	ねんがじょう			
\\	昨日、年賀状を出しました。	
\\	年賀状 
\\	きのう ねんがじょう を だしました			
\\	乗り物	
\\	のりもの			
\\	自転車は便利な乗り物です。	
\\	乗り物 
\\	じてんしゃ は べんり な のりもの です			
\\	布団	
\\	ふとん			
\\	役に立つ	
\\	やくにたつ			
\\	私は人々の役に立ちたいと思っています。	
\\	役に立つ 
\\	わたし は ひとびと の やくにたちたい と おもって います			
\\	破れる	
\\	やぶれる			
\\	シャツが破れている。	
\\	破れる 
\\	しゃつ が やぶれて いる			
\\	用事	
\\	ようじ			
\\	父は用事で出掛けています。	
\\	用事 
\\	ちち は ようじ で でかけて います			
\\	写す	
\\	うつす			
\\	彼は友達の答えを写した。	
\\	写す 
\\	かれ は ともだち の こたえ を うつした			
\\	腕時計	
\\	うでどけい			
\\	私の腕時計は遅れている。	
\\	腕時計 
\\	わたし の うでどけい は おくれて いる			
\\	贈り物	
\\	おくりもの			
\\	すてきな贈り物をどうもありがとう。	
\\	贈り物 
\\	すてき な おくりもの を どうも ありがとう			
\\	チャンネル	
\\	テレビのチャンネルを変えてください。	
\\	チャンネル 
\\	てれび の ちゃんねる を かえて ください			
\\	つまらない	
\\	彼の話はつまらないですね。	
\\	つまらない 
\\	かれ の はなし は つまらない です ね			
\\	マッチ	
\\	マッチを持っていますか。	
\\	マッチ 
\\	まっち を もって います か			
\\	いとこ	
\\	従兄弟と私は同じ年です。	
\\	いとこ 
\\	いとこ と わたし は おなじ とし です			
\\	冷やす	
\\	ひやす			
\\	頭を冷やしなさい。	
\\	冷やす 
\\	あたま を ひやしなさい			
\\	片付ける	
\\	かたづける			
\\	早く部屋を片付けなさい。	
\\	片付ける 
\\	はやく へや を かたづけなさい			
\\	乾杯	
\\	かんぱい			
\\	お二人の未来に乾杯しましょう。	
\\	乾杯 
\\	おふたり の みらい に かんぱい しましょう			
\\	ストーブ	
\\	うちではまだストーブを使っています。	
\\	ストーブ 
\\	うち で は まだ すとーぶ を つかって います			
\\	停車	
\\	ていしゃ			
\\	この電車は東京まで停車致しません。	
\\	停車 
\\	この でんしゃ は とうきょう まで ていしゃ いたしません			
\\	鳴く	
\\	なく			
\\	どこかでネコが鳴いている。	
\\	鳴く 
\\	どこか で ねこ が ないて いる			
\\	ガソリンスタンド	
\\	この近くにガソリンスタンドはありますか。	
\\	ガソリンスタンド 
\\	この ちかく に がそりんすたんど は あります か			
\\	片道	
\\	かたみち			
\\	東京まで片道切符を買った。	
\\	片道 
\\	とうきょう まで かたみち きっぷ を かった			
\\	今朝	
\\	けさ			
\\	今朝テレビでそのニュースを知りました。	
\\	今朝 
\\	けさ てれび で その にゅーす を しりました			
\\	交番	
\\	こうばん			
\\	あそこの交番で道を聞きましょう。	
\\	交番 
\\	あそこ の こうばん で みち を ききましょう			
\\	大分	
\\	だいぶ			
\\	大分ピアノが上手くなりました。	
\\	大分 
\\	だいぶ ぴあの が うまく なりました			
\\	梅雨	
\\	つゆ			
\\	梅雨は6月頃です。	
\\	梅雨 
\\	つゆ は ろくがつごろ です			
\\	デート	
\\	デートで遊園地に行きました。	
\\	デート 
\\	でーと で ゆうえんち に いきました			
\\	ふすま	
\\	ふすまを閉めてください。	
\\	ふすま 
\\	ふすま を しめて ください			
\\	真っ直ぐ	
\\	まっすぐ			
\\	この道を真っ直ぐ行ってください。	
\\	真っ直ぐ 
\\	この みち を まっすぐ いって ください			
\\	レモン	
\\	上着	
\\	うわぎ			
\\	暑いので上着を脱ぎました。	
\\	脱ぐ 
\\	上着 
\\	あつい の で うわぎ を ぬぎました			
\\	偉い	
\\	えらい			
\\	彼は偉い学者です。	
\\	偉い 
\\	かれ は えらい がくしゃ です			
\\	書き方	
\\	かきかた			
\\	彼はその漢字の書き方が分からない。	
\\	書き方 
\\	かれ は その かんじ の かきかた が わからない			
\\	炊く	
\\	たく			
\\	母は毎朝ご飯を炊く。	
\\	炊く 
\\	はは は まいあさ ごはん を たく			
\\	チーズ	
\\	チーズを一切れ食べました。	
\\	チーズ 
\\	ちーず を ひときれ たべました			
\\	ドライブ	
\\	今日は群馬までドライブしました。	
\\	ドライブ 
\\	きょう は ぐんま まで どらいぶ しました			
\\	踏切	
\\	ふみきり			
\\	その踏切は長い。	
\\	踏切 
\\	その ふみきり は ながい			
\\	ラケット	
\\	テニスのラケットを買いました。	
\\	ラケット 
\\	てにす の らけっと を かいました			
\\	受付	
\\	うけつけ			
\\	受付は9時からです。	
\\	受付 
\\	うけつけ は くじ から です			
\\	怒る	
\\	おこる			
\\	彼女が嘘をついたので、彼は怒った。	
\\	嘘 
\\	怒る 
\\	かのじょ が うそ を ついた の で かれ は おこった			
\\	化粧	
\\	けしょう			
\\	彼女は化粧が上手い。	
\\	化粧 
\\	かのじょ は けしょう が うまい			
\\	寿司	
\\	すし			
\\	彼女は寿司を初めて食べました。	
\\	寿司 
\\	かのじょ は すし を はじめて たべました 。			
\\	スチュワーデス	
\\	彼女はスチュワーデスになった。	
\\	スチュワーデス 
\\	かのじょ は すちゅわーです に なった			
\\	都合	
\\	つごう			
\\	今日は都合が悪くて行けません。	
\\	都合 
\\	きょう は つごう が わるくて いけません			
\\	テープレコーダー	
\\	テープレコーダーで自分の声を録音しました。	
\\	テープレコーダー 
\\	てーぷれこーだー で じぶん の こえ を ろくおん しました			
\\	ひげ	
\\	父はひげをはやしています。	
\\	ひげ 
\\	ちち は ひげ を はやして います			
\\	本棚	
\\	ほんだな			
\\	これはとても大きな本棚ですね。	
\\	本棚 
\\	これ は とても おおき な ほんだな です ね			
\\	真っ暗	
\\	まっくら			
\\	外は真っ暗です。	
\\	真っ暗 
\\	そと は まっくら です			
\\	沸く	
\\	わく			
\\	お風呂が沸きました。	
\\	沸く 
\\	お ふろ が わきました 。			
\\	売り切れる	
\\	うりきれる			
\\	その本は直ぐ売り切れた。	
\\	売り切れる 
\\	その ほん は すぐ うりきれた			
\\	押し入れ	
\\	おしいれ			
\\	布団を押し入れにしまいました。	
\\	布団 
\\	押し入れ 
\\	ふとん を おしいれ に しまいました			
\\	革	
\\	かわ			
\\	ぐっすり	
\\	赤ちゃんがぐっすり寝ている。	
\\	ぐっすり 
\\	あかちゃん が ぐっすり ねて いる			
\\	紅茶	
\\	こうちゃ			
\\	紅茶にレモンを入れて飲んだ。	
\\	レモン 
\\	紅茶 
\\	こうちゃ に れもん を いれて のんだ			
\\	温かい紅茶が飲みたい。	
\\	紅茶 
\\	あたたかい こうちゃ が のみたい			
\\	邪魔	
\\	じゃま			
\\	邪魔です、どいてください。	
\\	邪魔 
\\	じゃま です どいて ください			
\\	ソース	
\\	ソースはどれですか。	
\\	ソース 
\\	そーす は どれ です か			
\\	タイプライター	
\\	母は古いタイプライターを持っています。	
\\	タイプライター 
\\	はは は ふるい たいぷらいたー を もって います			
\\	足す	
\\	たす			
\\	母は味噌汁に水を足した。	
\\	足す 
\\	はは は みそしる に みず を たした			
\\	トマト	
\\	私はトマトが大好きです。	
\\	トマト 
\\	わたし は とまと が だいすき です			
\\	履く	
\\	はく			
\\	彼女はブーツを履いています。	
\\	履く 
\\	かのじょ は ぶーつ を はいて います			
\\	バター	
\\	パンにバターをぬって食べました。	
\\	バター 
\\	ぱん に ばたー を ぬって たべました			
\\	発車	
\\	はっしゃ			
\\	バスが発車します。	
\\	発車 
\\	ばす が はっしゃ します			
\\	バレーボール	
\\	妹はバレーボールが得意です。	
\\	バレーボール 
\\	いもうと は ばれーぼーる が とくい です			
\\	おかず	
\\	晩ご飯のおかずは何?	
\\	おかず 
\\	ばんごはん の おかず は なに			
\\	カレンダー	
\\	カレンダーに予定を書いた。	
\\	カレンダー 
\\	かれんだー に よてい を かいた			
\\	再来年	
\\	さらいねん			
\\	次のオリンピックは再来年です。	
\\	再来年 
\\	つぎ の おりんぴっく は さらいねん です			
\\	冗談	
\\	じょうだん			
\\	冗談は止めてください。	
\\	冗談 
\\	じょうだん は やめて ください			
\\	近付く	
\\	ちかづく			
\\	女性が近付いてきました。	
\\	近付く 
\\	じょせい が ちかづいて きました			
\\	電話帳	
\\	でんわちょう			
\\	電話帳を見せてください。	
\\	電話帳 
\\	でんわちょう を みせて ください			
\\	ハンドバッグ	
\\	ハンドバッグを忘れました。	
\\	ハンドバッグ 
\\	はんどばっぐ を わすれました			
\\	ベル	
\\	玄関のベルが鳴った。	
\\	ベル 
\\	げんかん の べる が なった			
\\	干す	
\\	ほす			
\\	母が布団を干している。	
\\	布団 
\\	干す 
\\	はは が ふとん を ほして いる			
\\	母は洗濯物を干しています。	
\\	干す 
\\	はは は せんたくもの を ほして います			
\\	インク	
\\	プリンターのインクを買った。	
\\	インク 
\\	ぷりんたー の いんく を かった			
\\	蚊	
\\	か			
\\	蚊に足を刺された。	
\\	蚊 
\\	か に あし を さされた			
\\	改札口	
\\	かいさつぐち			
\\	改札口で会いましょう。	
\\	改札口 
\\	かいさつぐち で あいましょう			
\\	悲しむ	
\\	かなしむ			
\\	父は友だちの死を悲しんでいます。	
\\	悲しむ 
\\	ちち は ともだち の し を かなしんで います			
\\	クーラー	
\\	暑いのでクーラーをつけました。	
\\	クーラー 
\\	あつい の で くーらー を つけました			
\\	靴下	
\\	くつした			
\\	この靴下は3足で1000円です。	
\\	靴下 
\\	この くつした は さんそく で せんえん です			
\\	失礼	
\\	しつれい			
\\	ではそろそろ失礼します。	
\\	失礼 
\\	では そろそろ しつれい します			
\\	西暦	
\\	せいれき			
\\	西暦1964年に東京でオリンピックがあった。	
\\	西暦 
\\	せいれき せんきゅうひゃくろくじゅうよねん に とうきょう で おりんぴっく が あった			
\\	洗濯機	
\\	せんたくき			
\\	新しい洗濯機を買いました。	
\\	洗濯機 
\\	あたらしい せんたっき を かいました			
\\	豆腐	
\\	とうふ			
\\	私は毎日豆腐を食べます。	
\\	豆腐 
\\	わたし は まいにち とうふ を たべます			
\\	昼寝	
\\	ひるね			
\\	私の子供は毎日昼寝をします。	
\\	昼寝 
\\	わたし の こども は まいにち ひるね を します			
\\	不便	
\\	ふべん			
\\	私の家は駅から遠くて不便です。	
\\	不便 
\\	わたし の いえ は えき から とおく て ふべん です			
\\	ミルク	
\\	赤ちゃんにミルクをあげた。	
\\	ミルク 
\\	あかちゃん に みるく を あげた			
\\	向こう側	
\\	むこうがわ			
\\	私の家は川の向こう側にあります。	
\\	向こう側 
\\	わたし の いえ は かわ の むこうがわ に あります			
\\	億	
\\	おく			
\\	世界人口は65億人だ。	
\\	億 
\\	せかい じんこう は ろくじゅうごおくにん だ			
\\	お礼	
\\	おれい			
\\	彼女にお礼の手紙を書きました。	
\\	お礼 
\\	かのじょ に おれい の てがみ を かきました			
\\	着せる	
\\	きせる			
\\	娘に可愛いドレスを着せた。	
\\	着せる 
\\	むすめ に かわいい どれす を きせた			
\\	サラダ	
\\	サラダをたくさん食べました。	
\\	サラダ 
\\	さらだ を たくさん たべました			
\\	八百屋	
\\	やおや			
\\	八百屋でみかんを買いました。	
\\	みかん 
\\	八百屋 
\\	やおや で みかん を かいました			
\\	おじさん	
\\	昨日おじさんに会いました。	
\\	おじさん 
\\	きのう おじさん に あいました			
\\	代える	
\\	かえる			
\\	社長に代えて部長を出席させます。	
\\	代える 
\\	しゃちょう に かえて ぶちょう を しゅっせき させます			
\\	急行	
\\	きゅうこう			
\\	ちょうど急行電車が来た。	
\\	急行 
\\	ちょうど きゅうこう でんしゃ が きた			
\\	交差点	
\\	こうさてん			
\\	次の交差点を左に曲がってください。	
\\	交差点 
\\	つぎ の こうさてん を ひだり に まがって ください			
\\	高等学校	
\\	こうとうがっこう			
\\	弟が高等学校を卒業しました。	
\\	高等学校 
\\	おとうと が こうとうがっこう を そつぎょう しました			
\\	覚める	
\\	さめる			
\\	今朝は6時に目が覚めた。	
\\	覚める 
\\	けさ は ろくじ に め が さめた			
\\	昼休み	
\\	ひるやすみ			
\\	昼休みに公園に行った。	
\\	昼休み 
\\	ひるやすみ に こうえん に いった			
\\	ベルト	
\\	革のベルトを買いました。	
\\	革 
\\	ベルト 
\\	かわ の べると を かいました			
\\	ベルトがきつくなりました。	
\\	ベルト 
\\	べると が きつく なりました			
\\	真っ黒	
\\	まっくろ			
\\	インクで手が真っ黒になった	
\\	インク 
\\	真っ黒 
\\	いんく で て が まっくろ に なった			
\\	皆様	
\\	みなさま			
\\	皆様、こんにちは。	
\\	皆様 
\\	みなさま こんにちは			
\\	ラーメン	
\\	夕食にラーメンを食べました。	
\\	ラーメン 
\\	ゆうしょく に らーめん を たべました			
\\	ライター	
\\	店にライターを忘れました。	
\\	ライター 
\\	みせ に らいたー を わすれました			
\\	ワイシャツ	
\\	彼は白いワイシャツを着ている。	
\\	ワイシャツ 
\\	かれ は しろい わいしゃつ を きて いる			
\\	転ぶ	
\\	ころぶ			
\\	お祖母ちゃんがお風呂場で転んだ。	
\\	転ぶ 
\\	おばあちゃん が おふろば で ころんだ			
\\	冷める	
\\	さめる			
\\	彼への気持ちが冷めた。	
\\	冷める 
\\	かれ へ の きもち が さめた			
\\	シャワー	
\\	朝、急いでシャワーを浴びました。	
\\	シャワー 
\\	あさ いそいで しゃわー を あびました			
\\	扇風機	
\\	せんぷうき			
\\	暑いから扇風機をつけよう。	
\\	扇風機 
\\	あつい から せんぷうき を つけよう			
\\	遅刻	
\\	ちこく			
\\	遅刻しないでください。	
\\	遅刻 
\\	ちこく しない で ください			
\\	釣る	
\\	つる			
\\	昨日大きな魚を釣りました。	
\\	釣る 
\\	きのう おおき な さかな を つりました			
\\	うどん	
\\	私はお昼にうどんを食べました。	
\\	うどん 
\\	わたし は おひる に うどん を たべました			
\\	追い越す	
\\	おいこす			
\\	大きなトラックが私たちを追い越した。	
\\	追い越す 
\\	おおき な とらっく が わたしたち を おいこした			
\\	仮名	
\\	かな			
\\	お名前に仮名を振ってください。	
\\	仮名 
\\	おなまえ に かな を ふって ください			
\\	家内	
\\	かない			
\\	家内は九州出身です。	
\\	家内 
\\	かない は きゅうしゅう しゅっしん です			
\\	記念日	
\\	きねんび			
\\	今日は両親の結婚記念日です。	
\\	記念日 
\\	きょう は りょうしん の けっこん きねんび です			
\\	下り	
\\	くだり			
\\	もうすぐ下りの電車が発車します。	
\\	下り 
\\	もうすぐ くだり の でんしゃ が はっしゃ します			
\\	掲示板	
\\	けいじばん			
\\	掲示板のお知らせを見ましたか。	
\\	掲示板 
\\	けいじばん の おしらせ を みました か			
\\	凍る	
\\	こおる			
\\	寒い朝は道路が凍ります。	
\\	凍る 
\\	さむい あさ は どうろ が こおります			
\\	ごみ	
\\	今日はごみの日だ。	
\\	ごみ 
\\	きょう は ごみ の ひ だ			
\\	消防署	
\\	しょうぼうしょ			
\\	この町には消防署が1つしかありません。	
\\	消防署 
\\	この まち に は しょうぼうしょ が ひとつ しか ありません			
\\	スーパーマーケット	
\\	スーパーマーケットで人参を買った。	
\\	スーパーマーケット 
\\	すーぱーまーけっと で にんじん を かった			
\\	ハンカチ	
\\	ハンカチで手をふきました。	
\\	ハンカチ 
\\	はんかち で て を ふきました			
\\	ビデオテープ	
\\	私はビデオテープを30本持っています。	
\\	ビデオテープ 
\\	わたし は びでおてーぷ を さんじゅっぽん もって います			
\\	ぶらぶら	
\\	朝、近所をぶらぶらした。	
\\	ぶらぶら 
\\	あさ きんじょ を ぶらぶら した			
\\	アナウンサー	
\\	私はアナウンサーになりたい。	
\\	アナウンサー 
\\	わたし は あなうんさー に なりたい			
\\	辞書	
\\	じしょ			
\\	彼女は辞書をよく使います。	
\\	辞書 
\\	かのじょ は じしょ を よく つかいます			
\\	自炊	
\\	じすい			
\\	私は毎日、自炊している。	
\\	自炊 
\\	わたし は まいにち じすい して いる			
\\	止める	
\\	とめる			
\\	車を止めて。	
\\	止める 
\\	くるま を とめて			
\\	スプーン	
\\	カレーライスはスプーンで食べます。	
\\	スプーン 
\\	かれーらいす は すぷーん で たべます			
\\	年下	
\\	としした			
\\	彼は奥さんより年下です。	
\\	年下 
\\	かれ は おくさん より としした です			
\\	治す	
\\	なおす			
\\	早く風邪を治してください。	
\\	治す 
\\	はやく かぜ を なおして ください			
\\	箸	
\\	はし			
\\	箸を上手に使えるよ。	
\\	箸 
\\	はし を じょうず に つかえる よ			
\\	花見	
\\	はなみ			
\\	友達と花見をしました。	
\\	花見 
\\	ともだち と はなみ を しました			
\\	引き出し	
\\	ひきだし			
\\	財布は引き出しの中にあります。	
\\	引き出し 
\\	さいふ は ひきだし の なか に あります			
\\	ブラウス	
\\	彼女は白いブラウスを着ている。	
\\	ブラウス 
\\	かのじょ は しろい ぶらうす を きて いる			
\\	訳	
\\	やく			
\\	その文の訳を読みました。	
\\	訳 
\\	その ぶん の やく を よみました			
\\	リボン	
\\	プレゼントにリボンを付けた。	
\\	リボン 
\\	ぷれぜんと に りぼん を つけた			
\\	いかが	
\\	お加減はいかがですか。	
\\	いかが 
\\	おかげん は いかが です か			
\\	居る	
\\	いる			
\\	今日は一日中家に居ました。	
\\	居る 
\\	きょう は いちにちじゅう うち に いました			
\\	交通費	
\\	こうつうひ			
\\	会社までの交通費は一ヶ月8,000円です。	
\\	交通費 
\\	かいしゃ まで の こうつうひ は いっかげつ はっせんえん です			
\\	ステレオ	
\\	彼はステレオで音楽を聞いた。	
\\	ステレオ 
\\	かれ は すてれお で おんがく を きいた			
\\	滑る	
\\	すべる			
\\	彼は雪の上を滑った。	
\\	滑る 
\\	かれ は ゆき の うえ を すべった			
\\	早口	
\\	はやくち			
\\	彼女は早口だ。	
\\	早口 
\\	かのじょ は はやくち だ			
\\	フォーク	
\\	フォークをもらえますか。	
\\	フォーク 
\\	ふぉーく を もらえます か			
\\	裏返す	
\\	うらがえす			
\\	彼は紙を裏返しました。	
\\	裏返す 
\\	かれ は かみ を うらがえしました			
\\	おば	
\\	おばは大阪に住んでいます。	
\\	おば 
\\	おば は おおさか に すんで います			
\\	角	
\\	かど			
\\	次の角で左に曲がってください。	
\\	角 
\\	つぎ の かど で ひだり に まがって ください			
\\	きらきら	
\\	星がきらきら光っている。	
\\	きらきら 
\\	ほし が きらきら ひかって いる			
\\	三角	
\\	さんかく			
\\	紙を三角に切りました。	
\\	三角 
\\	かみ を さんかく に きりました			
\\	どうぞ	
\\	こちらへどうぞ。	
\\	どうぞ 
\\	こちら へ どうぞ			
\end{CJK}
\end{document}