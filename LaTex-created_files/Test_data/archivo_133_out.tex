\documentclass[8pt]{extreport} 
\usepackage{hyperref}
\usepackage{CJKutf8}
\begin{document}
\begin{CJK}{UTF8}{min}
\\	人	
\\	人[ひと] 
\\	人[じん]	
\\	山	
\\	山[やま]	
\\	川	
\\	川[かわ]	
\\	学	
\\	学[がく] 
\\	学[まな 
\\	ぶ]	
\\	生	
\\	生[せい]	
\\	方	
\\	方[かた]	あの方[あのかた] 
\\	日本の方[にほんのかた] 
\\	大	
\\	大[だい]	
\\	文	
\\	文[ぶん]	
\\	中	
\\	中[なか]	
\\	田	
\\	田[た]	
\\	子	
\\	子[こ]	
\\	先	
\\	先[せん]	
\\	スペイン人	
\\	スペイン人[スペインじん]	
\\	山川	
\\	山川[やまかわ]	
\\	学生	
\\	学生[がくせい]	
\\	スペインの方	
\\	スペインの方[スペインのかた]	
\\	あの人	
\\	あの人[あのひと]	
\\	東京	
\\	東京[とうきょう]	
\\	大学	
\\	大学[だいがく]	
\\	先生	
\\	先生[せんせい]	
\\	中田	
\\	中田[なかた]	
\\	中国	
\\	中国[ちゅうごく]	中[ちゅう] 
\\	国[ごく] 
\\	電話	
\\	電話[でんわ]	電[でん] 
\\	話[わ] 
\\	何番	
\\	何番[なんばん]	何[なん] 
\\	番[ばん] 
\\	紹介	
\\	紹介[しょうかい]	紹[しょう] 
\\	介[かい] 
\\	鉄男	
\\	鉄男[てつお]	鉄[てつ] 
\\	男[お] 
\\	専門	
\\	専門[せんもん]	専[せん] 
\\	門[もん] 
\\	日本語	
\\	日本語[にほんご]	日本[にほん] 
\\	語[ご] 
\\	文学	
\\	文学[ぶんがく]	文[ぶん] 
\\	学[がく] 
\\	番号	
\\	番号[ばんごう]	番[ばん] 
\\	号[ごう] 
\\	名前	
\\	名前[なまえ]	名[な] 
\\	前[まえ] 
\\	京子	
\\	京子[きょうこ]	京[きょう] 
\\	子[こ] 
\\	国	
\\	国[コク 
\\	くに]	天国[てんごく] 
\\	一	
\\	一[いち]	
\\	二	
\\	二[に]	
\\	三	
\\	三[さん]	
\\	四	
\\	四[し 
\\	よん]	
\\	五	
\\	五[ご]	
\\	六	
\\	六[ろく]	
\\	七	
\\	七[しち 
\\	なな]	
\\	八	
\\	八[はち]	
\\	九	
\\	九[く 
\\	きゅう]	
\\	十	
\\	十[じゅう]	
\\	日	
\\	日[にち 
\\	ひ]	日曜日[にちようび] 
\\	本 (本屋)	
\\	本[ほん]	
\\	東	
\\	東[とう 
\\	ひがし]	
\\	京	
\\	京[きょう 
\\	キン]	北京[ペキン] 
\\	教科書	
\\	教科書[きょうかしょ]	
\\	消しゴム	
\\	消しゴム[けしゴム]	
\\	傘	
\\	傘[かさ]	
\\	辞書	
\\	辞書[じしょ]	
\\	昼食	
\\	昼食[ちゅうしょく]	昼[ちゅう] 
\\	食[しょく] 
\\	招待	
\\	招待[しょうたい]	
\\	前菜	
\\	前菜[ぜんさい]	前[ぜん] 
\\	菜[さい 
\\	な] 
\\	野菜	
\\	野菜[やさい]	野[や] 
\\	菜[さい 
\\	な] 
\\	雑誌	
\\	雑誌[ざっし]	
\\	新聞	
\\	新聞[しんぶん]	新[しん] 
\\	聞[ぶん] 
\\	時計	
\\	時計[とけい]	時[と] 
\\	計[けい] 
\\	花火	
\\	花火[はなび]	花[はな] 
\\	火[び] 
\\	星	
\\	星[セイ 
\\	ほし]	
\\	月	
\\	月[つき 
\\	がつ 
\\	げつ]	月曜日[げつようび] 
\\	今月[こんげつ] 
\\	1月[いちがつ] 
\\	1か月[いっかげつ] 
\\	雨	
\\	雨[あめ]	
\\	雲	
\\	雲[くも]	
\\	雪	
\\	雪[ゆき]	
\\	英語	
\\	英語[えいご]	英[えい] 
\\	語[ご] 
\\	歳	
\\	歳[さい]	18歳[じゅうはっさい] 
\\	才	
\\	才[さい]	18才[じゅうはっさい] 
\\	歳.)
\\	何	
\\	何[なん 
\\	なに]	何時[なんじ] 
\\	何歳[なんさい] 
\\	何人[なんにん] 
\\	何番[なんばん] 
\\	何ですか[なんですか] 
\\	何がありますか[なにがありますか] 
\\	何 
\\	今	
\\	今[いま]	
\\	時	
\\	(とき)	時[とき] 
\\	時[ーじ]	好きな時に来てください[すきなときにきてください] 
\\	分	
\\	分[ふん 
\\	ーっぷん]	
\\	火	
\\	火[ひ 
\\	か]	火曜日[かようび] 
\\	火星[かせい] 
\\	水	
\\	水[みず]	水曜日[すいようび] 
\\	水星[すいせい] 
\\	木	
\\	木[き] 
\\	木[モク]	木曜日[もくようび] 
\\	木星[もくせい] 
\\	木版画[もくはんが] 
\\	金	
\\	金[きん]	お金[おかね] 
\\	金曜日[きんようび] 
\\	金星[きんせい] 
\\	土	
\\	土[つち] 
\\	土[ど]	土曜日[どようび] 
\\	土星[どせい] 
\\	出かけます	
\\	出かけます[でかけます]	出[で] 
\\	史	
\\	史[し]	
\\	休み	
\\	休み[やすみ]	休みます[やすみます] 
\\	午	
\\	午[ご]	午前[ごぜん] 
\\	午後[ごご] 
\\	前	
\\	前[まえ] 
\\	前[ぜん]	午前[ごぜん] 
\\	後	
\\	後[あと] 
\\	後[ご]	後で[あとで] 
\\	午後[ごご] 
\\	また後で[またあとで] 
\\	九時	
\\	九時[くじ]	
\\	何時	
\\	何時[なんじ]	
\\	五分	
\\	五分[ごふん]	
\\	半	
\\	半[はん]	九時半[くじはん] 
\\	朝	
\\	朝[あさ]	
\\	午前	
\\	午前[ごぜん]	
\\	午後	
\\	午後[ごご]	
\\	毎日	
\\	毎日[まいにち]	毎[まい] 
\\	授業	
\\	授業[じゅぎょう]	
\\	何曜日	
\\	何曜日[なんようび]	
\\	試験	
\\	試験[しけん]	試[し] 
\\	験[けん] 
\\	晩	
\\	晩[バン]	
\\	店	
\\	店[みせ 
\\	テン]	
\\	時間割	
\\	時間割[じかんわり]	
\\	講義	
\\	講義[こうぎ]	
\\	文学史	
\\	文学史[ぶんがくし]	文[ぶん] 
\\	学[がく] 
\\	史[し] 
\\	図書館	
\\	図書館[としょかん]	
\\	仕事	
\\	仕事[しごと]	
\\	北京	
\\	北京[ペキン]	
\\	開店	
\\	開店[かいてん]	開[かい 
\\	あ] 
\\	店[てん] 
\\	閉店	
\\	閉店[へいてん]	閉[へい] 
\\	店[てん] 
\\	始まります	
\\	始まります[はじまります]	
\\	終わります	
\\	終わります[おわります]	終ります[おわります] 
\\	終[おわり] 
\\	起きます	
\\	起きます[おきます]	
\\	寝ます	
\\	寝ます[ねます]	
\\	勉強します	
\\	勉強します[べんきょうします]	勉強[べんきょう] 
\\	開きます	
\\	開きます[あきます]	店が開きます[みせがあきます] 
\\	開けます[あけます] 
\\	店を開けます[みせをあけます] 
\\	閉まります	
\\	閉まります[しまります]	店が七時に閉まります[みせがしちじにしまります] 
\\	働きます	
\\	働きます[はたらきます]	働き者[はたらきもの] 
\\	休みます	
\\	休みます[やすみます]	休み[やすみ] 
\\	泳ぎます	
\\	泳ぎます[およぎます]	
\\	毎	
\\	毎[まい]	
\\	来	
\\	来[らい]	来ます[きます] 
\\	お茶	
\\	お茶[おちゃ]	
\\	肉	
\\	肉[にく]	
\\	屋	
\\	屋[や] 
\\	屋[ーや]	
\\	図	
\\	図[と 
\\	ず]	
\\	書	
\\	書[ショ]	
\\	館	
\\	館[かん]	
\\	食	
\\	食[しょく]	食べます[たべます] 
\\	飲みます	
\\	飲みます[のみます]	
\\	会います	
\\	会います[あいます]	
\\	帰ります	
\\	帰ります[かえります]	
\\	見ます	
\\	見ます[みます]	
\\	友	
\\	友[とも]	友だち[ともだち] 
\\	家	
\\	家[いえ] 
\\	家[うち]	
\\	昼	
\\	昼[ひる] 
\\	昼[ちゅう]	
\\	行きます	
\\	行きます[いきます]	
\\	買います	
\\	買います[かいます]	
\\	読みます	
\\	読みます[よみます]	
\\	先月	
\\	先月[せんげつ]	
\\	今月	
\\	今月[こんげつ]	
\\	来月	
\\	来月[らいげつ]	
\\	毎月	
\\	毎月[まいつき]	
\\	何月	
\\	何月[なんがつ]	
\\	一日	
\\	一日[ついたち] 
\\	一日[いちにち]	
\\	二日	
\\	二日[ふつか]	
\\	三日	
\\	三日[みっか]	
\\	四日	
\\	四日[よっか]	
\\	五日	
\\	五日[いつか]	
\\	六日	
\\	六日[むいか]	
\\	七日	
\\	七日[なのか]	
\\	八日	
\\	八日[ようか]	
\\	九日	
\\	九日[ここのか]	
\\	十日	
\\	ふりがな 
\\	とおか 
\\	とうか.)	十日[とおか]	
\\	喫茶店	
\\	喫茶店[きっさてん]	
\\	本屋	
\\	本屋[ほんや]	
\\	車	
\\	車[くるま], 車[シャ]	
\\	電車	
\\	電車[でんしゃ]	電[でん] 
\\	車[しゃ] 
\\	地下鉄	
\\	地下鉄[ちかてつ]	地[ち] 
\\	下[か] 
\\	鉄[てつ] 
\\	駅	
\\	駅[えき]	
\\	番線	
\\	番線[ばんせん]	五番線[ごばんせん] 
\\	新宿	
\\	新宿[しんじゅく]	
\\	京都	
\\	京都[きょうと]	
\\	郵便局	
\\	郵便局[ゆうびんきょく]	
\\	切手	
\\	切手[きって]	
\\	食堂	
\\	食堂[しょくどう]	食[しょく] 
\\	堂[どう] 
\\	牛乳	
\\	牛乳[ぎゅうにゅう]	牛[ぎゅう] 
\\	乳[にゅう] 
\\	映画	
\\	映画[えいが]	
\\	誕生日	
\\	誕生日[たんじょうび]	
\\	古事記	
\\	古事記[こじき]	
\\	万葉集	
\\	万葉集[まんようしゅう]	
\\	時間	
\\	時間[じかん]	
\\	上	
\\	上[うえ]	いすの上に本があります[いすのうえにほんがあります] 
\\	犬	
\\	犬[いぬ]	
\\	門	
\\	門[もん]	
\\	間	
\\	間[あいだ] 
\\	間[ま] 
\\	間[かん]	東京ローマ間[とうきょうローマかん] 
\\	一時間[いちじかん] 
\\	口	
\\	口[くち]	
\\	電	
\\	電[でん]	電車[でんしゃ] 
\\	地	
\\	地[ち]	
\\	下	
\\	下[した 
\\	か]	いすの下に新聞があります[いすのしたにしんぶんがあります] 
\\	地下鉄[ちかてつ] 
\\	鉄	
\\	鉄[てつ]	鉄男[てつお] 
\\	部	
\\	部[ぶ 
\\	へ]	部分[ぶぶん] 
\\	学部[がくぶ] 
\\	部屋[へや] 
\\	女	
\\	女[おんな 
\\	じょ]	女の人[おんなのひと] 
\\	男女[だんじょ] 
\\	男	
\\	男[おとこ 
\\	だん 
\\	ーお]	男の人[おとこのひと] 
\\	男女[だんじょ] 
\\	魚	
\\	魚[さかな] 
\\	魚[ぎょ]	金魚[きんぎょ] 
\\	魚屋[さかなや] 
\\	鳥	
\\	鳥[とり] 
\\	鳥[ちょう]	
\\	銀	
\\	銀[ぎん]	
\\	後ろ	
\\	後ろ[うしろ]	テーブルの後ろにいすがあります[テーブルのうしろにいすがあります] 
\\	横	
\\	横[よこ]	
\\	近く	
\\	近く[ちかく]	
\\	間に合います	
\\	間に合います[まにあいます]	
\\	お金	
\\	お金[おかね]	
\\	銀行	
\\	銀行[ぎんこう]	
\\	出口	
\\	出口[でぐち]	
\\	子供	
\\	子供[こども]	子[こ] 
\\	子供たち[こどもたち] 
\\	庭	
\\	庭[にわ]	
\\	公園	
\\	公園[こうえん]	公[こう] 
\\	園[その/エン] 
\\	花園[はなその] 
\\	学園[キャンパス] 
\\	卵	
\\	卵[たまご]	
\\	ご飯	
\\	ご飯[ごはん]	
\\	冷蔵庫	
\\	冷蔵庫[れいぞうこ]	冷[れい] 
\\	蔵[ぞう] 
\\	庫[こ] 
\\	車庫[しゃこ] 
\\	物	
\\	物[もの] 
\\	物[モツ 
\\	ブツ]	
\\	文学部	
\\	文学部[ぶんがくぶ]	学部[がくぶ] 
\\	部屋	
\\	部屋[へや]	
\\	事務室	
\\	事務室[じむしつ]	
\\	教室	
\\	教室[きょうしつ]	
\\	机	
\\	机[つくえ]	
\\	席	
\\	席[せき]	
\\	二つ	
\\	二つ[ふたつ]	
\\	階段	
\\	階段[かいだん]	
\\	海	
\\	海[うみ]	海水[かいすい] 
\\	週	
\\	週[しゅう]	
\\	美	
\\	美[び]	美しい[うつくしい] 
\\	美学[びがく] 
\\	美女[びじょ] 
\\	美男[びなん] 
\\	術	
\\	術[じゅつ]	
\\	夜	
\\	夜[よる 
\\	よ 
\\	や]	月夜[つきよ] 
\\	今夜[こんや] 
\\	社	
\\	社[しゃ 
\\	やしろ]	神社[じんじゃ] 
\\	仕	
\\	仕[し]	
\\	事	
\\	事[じ 
\\	こと 
\\	ごと]	
\\	映	
\\	映[えい]	映画[えいが] 
\\	画	
\\	画[が]	家[か] 
\\	画家[がか] 
\\	漫画家[まんがか] 
\\	入ります	
\\	入ります[はいります]	部屋へ入ります、部屋に入ります[へやへはいります、へやにはいります] 
\\	に 
\\	へ.)
\\	乗ります	
\\	乗ります[のります]	車に乗ります[くるまにのります] 
\\	遊びます	
\\	遊びます[あそびます]	
\\	電気	
\\	電気[でんき]	
\\	宿題	
\\	宿題[しゅくだい]	
\\	食事	
\\	食事[しょくじ]	食事します[しょくじします] 
\\	広場	
\\	広場[ひろば]	場[ば] 
\\	病院	
\\	病院[びょういん]	
\\	美術館	
\\	美術館[びじゅつかん]	美術[びじゅつ] 
\\	館[かん] 
\\	土佐日記	
\\	土佐日記[とさにっき]	土佐[とさ] 
\\	日記[にっき] 
\\	荷物	
\\	荷物[にもつ]	荷[に] 
\\	物[もつ] 
\\	青木	
\\	青木[あおき]	青[あお] 
\\	木[き] 
\\	来週	
\\	来週[らいしゅう]	
\\	箱根	
\\	箱根[はこね]	
\\	会社	
\\	会社[かいしゃ]	
\\	湖	
\\	湖[みずうみ]	
\\	作文	
\\	作文[さくぶん]	作[さく] 
\\	音楽	
\\	音楽[おんがく]	音[おん] 
\\	楽[がく] 
\\	服	
\\	服[ふく]	
\\	少し	
\\	少し[すこし]	
\\	飛行機	
\\	飛行機[ひこうき]	機[き] 
\\	形	
\\	形[かたち 
\\	けい]	て形[てけい] 
\\	て 
\\	ます形[ますけい] 
\\	ます 
\\	乗り換えます	
\\	乗り換えます[のりかえます]	
\\	聞きます	
\\	聞きます[ききます]	先生に聞きます[せんせいにききます] 
\\	話します	
\\	話します[はなします]	
\\	待ちます	
\\	待ちます[まちます]	
\\	切ります	
\\	切ります[きります]	
\\	誘います	
\\	誘います[さそいます]	
\\	浴びます	
\\	浴びます[あびます]	シャワーを浴びます[シャワーをあびます] 
\\	返します	
\\	返します[かえします]	
\\	借ります	
\\	借ります[かります]	
\\	着ます	
\\	着ます[きます]	
\\	飛びます	
\\	飛びます[とびます]	
\\	運びます	
\\	運びます[はこびます]	
\\	死にます	
\\	死にます[しにます]	
\\	立ちます	
\\	立ちます[たちます]	国立[こくりつ] 
\\	作	
\\	作[サク 
\\	つく]	
\\	者	
\\	者[シャ 
\\	もの]	
\\	気	
\\	気[キ]	天気[てんき] 
\\	売	
\\	売[う]	売ります[うります] 
\\	売店[バイてん] 
\\	高	
\\	高[コウ]	高い[たかい] 
\\	校	
\\	校[コウ]	学校[がっこう] 
\\	夕	
\\	夕[ゆう]	夕べ[ゆうべ] 
\\	夕方[ゆうがた] 
\\	昼 
\\	晩]
\\	父	
\\	父[ちち]	お父さん[おとうさん] 
\\	わたしの父[わたしのちち] 
\\	お父さん、きょうどこへ行きますか[おとうさん、きょうどこへいきますか] 
\\	あなたのお父さん[あなたのおとうさん] 
\\	鉄男さんのお父さん[てつおさんのおとうさん] 
\\	母	
\\	母[はは]	お母さん[おかあさん] 
\\	わたしの母[わたしのはは] 
\\	お母さん、こちらに来て[おかあさん、こちらに来て] 
\\	あなたのお母さん[あなたのおかあさん] 
\\	アンナさんのお母さん[アンナさんのおかあさん] 
\\	兄	
\\	兄[あに]	お兄さん[おにいさん] 
\\	姉	
\\	姉[あね]	お姉さん[おねえさん] 
\\	弟	
\\	弟[おとうと]	弟さん[おとうとさん] 
\\	妹	
\\	妹[いもうと]	妹さん[いもうとさん] 
\\	小	
\\	小[ショウ 
\\	ちい]	小さい[ちいさい] 
\\	少	
\\	少[ショウ 
\\	すこ 
\\	すく]	少し[すこし] 
\\	少ない[すくない] 
\\	家族	
\\	家族[かぞく]	
\\	両親	
\\	両親[りょうしん]	両[りょう] 
\\	親[しん] 
\\	兄弟	
\\	兄弟[きょうだい]	
\\	一人	
\\	一人[ひとり]	
\\	二人	
\\	二人[ふたり]	
\\	小学校	
\\	小学校[しょうがっこう]	
\\	中学校	
\\	中学校[ちゅうがっこう]	
\\	高校	
\\	高校[こうこう]	
\\	研究	
\\	研究[けんきゅう]	
\\	内容	
\\	内容[ないよう]	
\\	南	
\\	南[みなみ 
\\	ナン]	南イタリア[みなみイタリア] 
\\	東南アジア[とうなんアジア] 
\\	大使館	
\\	大使館[たいしかん]	大使[たいし] 
\\	自動車	
\\	自動車[じどうしゃ]	自[じ] 
\\	動[どう] 
\\	車[しゃ] 
\\	着物	
\\	着物[きもの]	
\\	劇場	
\\	劇場[げきじょう]	
\\	工場	
\\	工場[こうじょう]	
\\	作者	
\\	作者[さくしゃ]	
\\	何人	
\\	何人[なんにん]	
\\	侍	
\\	侍[さむらい]	七人の侍[しちにんのさむらい] 
\\	源氏物語	
\\	源氏物語[げんじものがたり]	物語[ものがたり] 
\\	紫式部	
\\	紫式部[むらさきしきぶ]	
\\	枕草子	
\\	枕草子[まくらのそうし]	
\\	清少納言	
\\	清少納言[せいしょうなごん]	
\\	売ります	
\\	売ります[うります]	
\\	作ります	
\\	作ります[つくります]	
\\	教えます	
\\	教えます[おしえます]	
\\	持ちます	
\\	持ちます[もちます]	
\\	勤めます	
\\	勤めます[つとめます]	わたしは郵便局に勤めています[わたしはゆうびんきょくにつとめています] 
\\	知ります	
\\	知ります[しります]	知っています[しっています] 
\\	中田さんを知っています[なかたさんをしっています] 
\\	住みます	
\\	住みます[すみます]	ローマに住んでいます[ローマにすんでいます] 
\\	習います	
\\	習います[ならいます]	日本語を習います[にほんごをならいます] 
\\	通います	
\\	通います[かよいます]	
\\	結婚します	
\\	結婚します[けっこんします]	ダリオさんはアンナさんと結婚します[ダリオさんはアンナさんとけっこんします] 
\\	去	
\\	去[キョ]	
\\	年	
\\	年[ネン] 
\\	年[とし]	生年月日[せいねんがっぴ] 
\\	虫	
\\	虫[むし]	
\\	勉強	
\\	勉強[ベンキョウ]	強い[つよい] 
\\	勉強します[べんきょうします] 
\\	空	
\\	空[そら] 
\\	空[から]	空きます[あきます] 
\\	席が空きます[せきがあきます] 
\\	席が空いています[せきがあいています] 
\\	席が空きました[せきがあきました] 
\\	席が空いていますか[せきがあいていますか] 
\\	窓	
\\	窓[まど]	
\\	言	
\\	言[こと]	言います[いいます] 
\\	一言[ひとこと] 
\\	言葉[ことば] 
\\	語	
\\	語[ゴ] 
\\	語[かたり]	物語[ものがたり] 
\\	語ります[かたります] 
\\	英	
\\	英[エイ]	
\\	代	
\\	代[ダイ]	時代[じだい] 
\\	80年代[80ねんだい] 
\\	電気代[でんきだい] 
\\	貸します	
\\	貸します[かします]	
\\	手	
\\	手[て]	
\\	紙	
\\	紙[かみ]	
\\	続きます	
\\	続きます[つづきます]	
\\	考えます	
\\	考えます[かんがえます]	
\\	手紙	
\\	手紙[てがみ]	
\\	お酒	
\\	お酒[おさけ]	日本酒[にほんしゅ] 
\\	言葉	
\\	言葉[ことば]	葉[は] 
\\	自転車	
\\	自転車[じてんしゃ]	
\\	意味	
\\	意味[いみ]	
\\	料理	
\\	料理[りょうり]	
\\	歴史	
\\	歴史[れきし]	歴[れき] 
\\	日本の歴史[にほんのれきし] 
\\	日本史[にほんし] 
\\	学歴[がくれき] 
\\	僕	
\\	僕[ぼく]	
\\	仲間	
\\	仲間[なかま]	
\\	最初	
\\	最初[さいしょ]	最初の授業[さいしょのじゅぎょう] 
\\	質問	
\\	質問[しつもん]	
\\	時代	
\\	時代[じだい]	
\\	次	
\\	次[つぎ]	
\\	私	
\\	私[わたし]	
\\	答え	
\\	答[こた]え, 答[トウ]	答案[とうあん] 
\\	約	
\\	約[やく]	
\\	通り	
\\	通り[とおり]	
\\	去年	
\\	去年[きょねん]	
\\	彼女	
\\	彼女[かのじょ]	
\\	和歌	
\\	和歌[わか]	和ー[わー] 
\\	和室[わしつ] 
\\	学会	
\\	学会[がっかい]	
\\	本箱	
\\	本箱[ほんばこ]	
\\	美容院	
\\	美容院[びよういん]	
\\	一か月	
\\	一か月[いっかげつ]	
\\	一週間	
\\	一週間[いっしゅうかん]	
\\	一度	
\\	一度[いちど]	
\\	奈良	
\\	奈良[なら]	奈良時代[ならじだい] 
\\	平安	
\\	平安[へいあん]	平安時代[へいあんじだい] 
\\	古今集	
\\	古今集[こきんしゅう]	古い[ふるい] 
\\	集めます[あつめます] 
\\	冊	
\\	冊[さつ]	
\\	台	
\\	台[だい]	台所[だいどころ] 
\\	枚	
\\	枚[まい]	ここに紙が二枚あります[ここにかみがにまいあります] 
\\	匹	
\\	匹[ひき]	
\\	回	
\\	回[かい]	
\\	止まります	
\\	止まります[とまります]	止めます[とめます] 
\\	車を止めます[くるまをとめます] 
\\	車が止まります[くるまがとまります] 
\\	書きます	
\\	書きます[かきます]	
\\	登録します	
\\	登録します[とうろくします]	
\\	消します	
\\	消します[けします]	
\\	出します	
\\	出します[だします]	本をかばんから出します[ほんをかばんからだします] 
\\	メールを出します[メールをだします] 
\\	手紙を出します[てがみをだします] 
\\	登ります	
\\	登ります[のぼります]	富士山に登ります[ふじさんにのぼります] 
\\	入れます	
\\	入れます[いれます]	本をかばんに入れます[ほんをかばんにいれます] 
\\	曇ります	
\\	曇ります[くもります]	
\\	晴れます	
\\	晴れます[はれます]	
\\	並びます	
\\	並びます[ならびます]	
\\	走ります	
\\	走ります[はしります]	
\\	急ぎます	
\\	急ぎます[いそぎます]	
\\	両	
\\	両[リョウ]	
\\	親	
\\	親[シン 
\\	おや]	親友[しんゆう] 
\\	新	
\\	新[シン 
\\	あたら]	新しい[あたらしい] 
\\	近い	
\\	近い[ちかい]	
\\	所	
\\	所[ショ 
\\	ところ]	
\\	切	
\\	切[セツ]	切ります[きります] 
\\	重い	
\\	重い[おもい]	重力[じゅうりょく] 
\\	動	
\\	動[うご]きます, 動[ドウ]	"動きます[うごきます] 
\\	「それでも動いています」[「それでもうごいています」] 
\\	白	
\\	白[ハク]	白い[しろい] 
\\	百	
\\	百[ひゃく]	
\\	安	
\\	安[あん 
\\	やす]	安い[やすい] 
\\	安心[あんしん] 
\\	安心な[あんしんな] 
\\	安心します[あんしんします] 
\\	品	
\\	品[ヒン 
\\	しな]	品物[しなもの] 
\\	上品な[じょうひんな] 
\\	下品な[げひんな] 
\\	下品な言葉[げひんなことば] 
\\	円	
\\	円[エン]	
\\	万	
\\	万[マン]	
\\	千	
\\	千[セン]	
\\	元	
\\	元[ゲン]	
\\	自	
\\	自[じ]	自分[じぶん] 
\\	自立[じりつ] 
\\	自由[じゆう] 
\\	有名な	
\\	有名な[ゆうめいな]	
\\	複雑な	
\\	複雑な[ふくざつな]	
\\	静かな	
\\	静かな[しずかな]	"静かに[しずかに] 
\\	きのう、教室で先生が「静かに」と言いました[きのう、きょうしつでせんせいが「しずかに」といいました] 
\\	学生は静かではありませんでしたから[がくせいはしずかではありませんでしたから] 
\\	親切な	
\\	親切な[しんせつな]	
\\	元気な	
\\	元気な[げんきな]	元気[げんき] 
\\	お元気ですか[おげんきですか] 
\\	中田さんは元気ではありません[なかたさんはげんきではありません] 
\\	便利な	
\\	便利な[べんりな]	
\\	不便な	
\\	不便な[ふべんな]	
\\	高い	
\\	高い[たかい]	
\\	遠い	
\\	遠い[とおい]	
\\	少ない	
\\	少ない[すくない]	
\\	冷たい	
\\	冷たい[つめたい]	冷たいコーヒー[つめたいコーヒー] 
\\	冷たい手[つめたいて] 
\\	海の水は冷たいです[うみのみずはつめたいです] 
\\	寒い.)
\\	熱い	
\\	熱い[あつい]	熱いお茶[あついおちゃ] 
\\	このコーヒーは熱いです[このコーヒーはあついです] 
\\	暑い.)
\\	大きい	
\\	大きい[おおきい]	
\\	小さい	
\\	小さい[ちいさい]	
\\	茶色い	
\\	茶色い[ちゃいろい]	
\\	寒い	
\\	寒い[さむい]	きょうは寒いです[きょうはさむいです] 
\\	冷たい.)
\\	短い	
\\	短い[みじかい]	短い休み[みじかいやすみ] 
\\	短いスカート[みじかいスカート] 
\\	短気な[たんきな] 
\\	短所[たんしょ] 
\\	豆[まめ] 
\\	強い	
\\	強い[つよい]	
\\	良い	
\\	良い[よい] 
\\	良い[いい]	
\\	広い	
\\	広い[ひろい]	
\\	作品	
\\	作品[さくひん]	
\\	白鳥	
\\	白鳥[はくちょう]	
\\	探します	
\\	探します[さがします]	京子さんはアパートを探しています[きょうこさんはアパートをさがしています] 
\\	随筆	
\\	随筆[ずいひつ]	
\\	語彙	
\\	語彙[ごい]	
\\	事務所	
\\	事務所[じむしょ]	
\\	下宿	
\\	下宿[げしゅく]	
\\	台所	
\\	台所[だいどころ]	キッチン[キッチン] 
\\	家賃	
\\	家賃[やちん]	
\\	日本書紀	
\\	日本書紀[にほんしょき]	
\\	北山	
\\	北山[きたやま]	
\\	上野	
\\	上野[うえの]	
\\	横浜	
\\	横浜[よこはま]	
\\	本 (一本)	
\\	本[ほん]	
\\	目	
\\	目[め]	目で見ます[めでみます] 
\\	耳	
\\	耳[みみ]	
\\	古い	
\\	古い[ふるい]	
\\	舌	
\\	舌[した]	
\\	話	
\\	話[はなし]	会話[かいわ] 
\\	日本語を話します[にほんごをはなします] 
\\	友だちと話します[ともだちとはなします] 
\\	友だちに話します[ともだちにはなします] 
\\	背	
\\	背[せ 
\\	せい]	背が高い[せがたかい] 
\\	田中さんは背が高いです[たなかさんはせがたかいです] 
\\	背中[せなか] 
\\	春	
\\	春[はる]	春分[しゅんぶん] 
\\	春が来ました[はるがきました] 
\\	春になりました[はるになりました] 
\\	夏	
\\	夏[なつ]	
\\	秋	
\\	秋[あき]	秋分[しゅうぶん] 
\\	冬	
\\	冬[ふゆ]	
\\	法	
\\	法[ほう]	法学部[ほうがくぶ] 
\\	音	
\\	音[オン]	音読み[おんよみ] 
\\	楽しい	
\\	楽しい[たのしい]	音楽[おんがく] 
\\	多い	
\\	多い[おおい]	
\\	好きな	
\\	好きな[すきな]	
\\	上手な	
\\	上手な[じょうずな]	
\\	下手な	
\\	下手な[へたな]	
\\	得意な	
\\	得意な[とくいな]	僕はサッカーが得意です[ぼくはサッカーがとくいです] 
\\	残念な	
\\	残念な[ざんねんな]	残念です[ざんねんです] 
\\	残念ですが・・・[ざんねんですが・・・] 
\\	弱い	
\\	弱い[よわい]	
\\	長い	
\\	長い[ながい]	
\\	悪い	
\\	悪い[わるい]	"悪[あく] 
\\	悪人[あくにん] 
\\	悪魔[あくま] 
\\	「天気は悪いです」[「てんきはわるいです」] 
\\	暑い	
\\	暑い[あつい]	
\\	速い	
\\	速い[はやい]	
\\	面積	
\\	面積[めんせき]	
\\	歌	
\\	歌[うた]	
\\	今晩	
\\	今晩[こんばん]	
\\	用事	
\\	用事[ようじ]	
\\	小説	
\\	小説[しょうせつ]	
\\	生活	
\\	生活[せいかつ]	
\\	物価	
\\	物価[ぶっか]	東京の物価は高いです[とうきょうのぶっかはたかいです] 
\\	豚	
\\	豚[ぶた]	豚肉[ぶたにく] 
\\	人口	
\\	人口[じんこう]	
\\	鼻	
\\	鼻[はな]	
\\	頭	
\\	頭[あたま]	"「春子さんは頭がいいです」[「はるこさんはあたまがいいです」] 
\\	「頭が痛いです」[「あたまがいたいです」] 
\\	髪	
\\	髪[かみ]	
\\	首	
\\	首[くび]	
\\	旅行	
\\	旅行[りょこう]	
\\	野球	
\\	野球[やきゅう]	
\\	字	
\\	字[じ]	
\\	絵	
\\	絵[え]	絵本[えほん] 
\\	葉書[はがき] 
\\	絵葉書[えはがき] 
\\	絵画[かいが] 
\\	絵画館[かいがかん] 
\\	運転	
\\	運転[うんてん]	
\\	趣味	
\\	趣味[しゅみ]	
\\	成績	
\\	成績[せいせき]	
\\	文法	
\\	文法[ぶんぽう]	
\\	皆	
\\	皆[みな 
\\	みんな]	
\\	哲学	
\\	哲学[てつがく]	
\\	暇な	
\\	暇な[ひまな]	暇な時[ひまなとき] 
\\	東洋	
\\	東洋[とうよう]	
\\	発音	
\\	発音[はつおん]	"発音する[はつおんする] 
\\	「日本語の発音がいいです」[「にほんごのはつおんがいいです」] 
\\	出発[しゅっぱつ] 
\\	出発する[しゅっぱつする] 
\\	北海道	
\\	北海道[ほっかいどう]	
\\	今度	
\\	今度[こんど]	
\\	曜	
\\	曜[よう]	曜日[ようび] 
\\	平	
\\	平[ヘイ 
\\	たい] 
\\	平[たいら]	平らな[たいらな] 
\\	有	
\\	有[ユウ]	所有[しょゆう] 
\\	有名な[ゆうめいな] 
\\	有料[ゆうりょう] 
\\	無料[むりょう] 
\\	有 
\\	あ 
\\	あります.)
\\	名	
\\	名[メイ 
\\	な]	
\\	族	
\\	族[ぞく]	
\\	長	
\\	長[おさ 
\\	チョウ]	駅長[えきちょう] 
\\	長い[ながい] 
\\	明るい	
\\	明るい[あかるい]	明日[あした] 
\\	冷	
\\	冷[レイ 
\\	つめ]	冷たい[つめたい] 
\\	無口な	
\\	無口な[むくちな]	無[む] 
\\	忙しい	
\\	忙しい[いそがしい]	今日は忙しくないです[きょうはいそがしくないです] 
\\	きのうは忙しい一日でした[きのうはいそがしいいちにちでした] 
\\	田中さんは忙しい人です[たなかさんはいそがしいひとです] 
\\	危ない	
\\	危ない[あぶない]	"「危ない!」[「あぶない!」] 
\\	涼しい	
\\	涼しい[すずしい]	
\\	若い	
\\	若い[わかい]	若さ[わかさ] 
\\	若者[わかもの] 
\\	軽い	
\\	軽い[かるい]	
\\	女性	
\\	女性[じょせい]	男性[だんせい] 
\\	性[せい] 
\\	(女性 
\\	男性 
\\	女の人 
\\	男の人.)
\\	数学	
\\	数学[すうがく]	数[すう 
\\	かず] 
\\	独身	
\\	独身[どくしん]	
\\	訪問	
\\	訪問[ほうもん]	
\\	社長	
\\	社長[しゃちょう]	
\\	留守	
\\	留守[るす]	留守番電話[るすばんでんわ] 
\\	留守電[るすでん] 
\\	アンナちゃんは留守です[あんなちゃんはるすです] 
\\	動物	
\\	動物[どうぶつ]	
\\	銀座	
\\	銀座[ぎんざ]	
\\	藤原	
\\	藤原[ふじはら]	藤原佐為[ふじはらさい] 
\\	は 
\\	原 
\\	わ.)
\\	色	
\\	色[いろ]	
\\	体	
\\	体[からだ]	"「僕は体が強いです」[「ぼくはからだがつよいです」] 
\\	「煙草は体に悪いです」[「たばこはからだにわるいです」] 
\\	お願いします	
\\	お願いします[おねがいします]	
\\	通ります	
\\	通ります[とおります]	公園を通っています[こうえんをとおっています] 
\\	特に	
\\	特に[とくに]	
\\	主	
\\	主[シュ]	主人公[しゅじんこう] 
\\	主な登場人物[おもなとうじょうじんぶつ] 
\\	貝	
\\	貝[かい]	
\\	暗い	
\\	暗い[くらい]	暗い部屋[くらいへや] 
\\	工	
\\	工[コウ]	工員[こういん] 
\\	大工[だいく] 
\\	工学[こうがく] 
\\	場	
\\	場[ジョウ 
\\	ば]	乗り場[のりば] 
\\	青い	
\\	青い[あおい]	青い海[あおいうみ] 
\\	青い目[あおいめ] 
\\	赤い	
\\	赤い[あかい]	赤[あか] 
\\	赤ちゃん[あかちゃん] 
\\	黒い	
\\	黒い[くろい]	
\\	漢字	
\\	漢字[カンジ]	
\\	荷	
\\	荷[に]	
\\	運	
\\	運[ウン]	"運転[うんてん] 
\\	運動[うんどう] 
\\	「僕は運がいいです」[「ぼくはうんがいいです」] 
\\	「田中さんは運が悪いです」[「たなかさんはうんがわるいです」] 
\\	運びます[はこびます] 
\\	イタリア製	
\\	イタリア製[イタリアせい]	日本製の車[にほんせいのくるま] 
\\	能	
\\	能[のう]	"能力[のうりょく] 
\\	才能[さいのう] 
\\	「あの人は才能があります」[「あのひとはさいのうがあります」] 
\\	「あの人は数学の才能があります」[「あのひとはすうがくのさいのうがあります」] 
\\	空港	
\\	空港[くうこう]	港[みなと] 
\\	娘	
\\	娘[むすめ]	"娘は五歳です[むすめはごさいです] 
\\	お嬢さん[おじょうさん] 
\\	お嬢さんはおいくつですか[おじょうさんはおいくつですか] 
\\	中田さんのお嬢さんは銀行員です[なかたさんのおじょうさんはぎんこういんです] 
\\	(「お嬢さん」 
\\	人形	
\\	人形[にんぎょう]	
\\	息子	
\\	息子[むすこ]	息子は結婚しています[むすこはけっこんしています] 
\\	ロッシさんの息子さんは大学生です[ロッシさんのむすこさんはだいがくせいです] 
\\	坊ちゃん.)
\\	海岸	
\\	海岸[かいがん]	
\\	洋服	
\\	洋服[ようふく]	和服[わふく] 
\\	通訳	
\\	通訳[つうやく]	
\\	歌人	
\\	和歌	歌人[かじん]	
\\	翻訳	
\\	翻訳[ほんやく]	翻訳家[ほんやくか] 
\\	翻訳をします[ほんやくをします] 
\\	意見	
\\	意見[いけん]	
\\	留学	
\\	留学[りゅうがく]	"留学します[りゅうがくします] 
\\	「卒業してから、日本に留学します」[「そつぎょうしてから、にほんにりゅうがくします」] 
\\	二階建て	
\\	二階建て[にかいだて]	一階[いっかい] 
\\	二階[にかい] 
\\	三階[さんかい] 
\\	地下一階[ちかいっかい] 
\\	主人	
\\	主人[しゅじん]	"店の主人[みせのしゅじん] 
\\	「主人はエンジニアです」[「しゅじんはエンジニアです」] 
\\	ご主人[ごしゅじん] 
\\	春子さんのご主人[はるこさんのごしゅじん] 
\\	緑	
\\	緑[みどり]	緑色のかばん[みどりいろのかばん] 
\\	彼	
\\	彼[かれ]	彼はパーティーに来ました[かれはパーティーに来ました] 
\\	夏子ちゃんの彼はイタリア人です[なつこちゃんのかれはイタリアじんです] 
\\	彼女[かのじょ] 
\\	彼女はパーティーに来ません[かのじょはパーティーにきません] 
\\	鉄男君の彼女は歌手です[てつおくんのかのじょはかしゅです] 
\\	画家	
\\	画家[がか]	
\\	会社員	
\\	会社員[かいしゃいん]	
\\	歌手	
\\	歌手[かしゅ]	
\\	リチャード君	
\\	(君 
\\	リチャード君[リチャードくん]	
\\	猫	
\\	猫[ねこ]	
\\	西村	
\\	西村[にしむら]	
\\	伊勢	
\\	伊勢[いせ]	
\\	大鏡	
\\	大鏡[おおかがみ]	鏡[かがみ] 
\\	迎えます	
\\	迎えます[むかえます]	"「彼女を駅まで迎えに行きます」[「かのじょをえきまでむかえにいきます」] 
\\	拾います	
\\	拾います[ひろいます]	海岸でこの貝を拾いました[かいがんでこのかいをひろいました] 
\\	受かります	
\\	受かります[うかります]	試験に受かります[しけんにうかります] 
\\	試験を受けます[しけんをうけます] 
\\	送ります	
\\	送ります[おくります]	"手紙を送ります[てがみをおくります] 
\\	「彼女を車で駅まで送りました」[「かのじょをくるまでえきまでおくりました」] 
\\	出します.)
\\	思います	
\\	思います[おもいます]	
\\	変えます	
\\	変えます[かえます]	家を変えます[うちをかえます] 
\\	意見を変えます[いけんをかえます] 
\\	気に入ります	
\\	気に入ります[きにいります]	"「コンサートが気に入りました」[「コンサートがきにいりました」] 
\\	「音楽が好きです」[「おんがくがすきです」] 
\\	頼みます	
\\	頼みます[たのみます]	田中さんに頼みます[たなかさんにたのみます] 
\\	ダリオさんに引越しの手伝いを頼みます[ダリオさんにひっこしのてつだいをたのみます] 
\\	引っ越します	
\\	引っ越します[ひっこします]	
\\	決めます	
\\	決めます[きめます]	"「引越しの日を土曜日に決めました」[「ひっこしのひをどようびにきめました」] 
\\	疲れます	
\\	疲れます[つかれます]	"「疲れています」[「つかれています」] 
\\	「疲れました」[「つかれました」] 
\\	式	
\\	式[しき]	入学式[にゅうがくしき] 
\\	卒業式[そつぎょうしき] 
\\	馬	
\\	馬[うま]	馬車[ばしゃ] 
\\	駐車	
\\	駐車[ちゅうしゃ]	"駐[ちゅう] 
\\	駐車します[ちゅうしゃします] 
\\	駐車場[ちゅうしゃじょう] 
\\	「ここに車を駐車することはできませんよ」[「ここにくるまをチュウシャすることはできませんよ」] 
\\	走る	
\\	走る[はしる]	
\\	起きる	
\\	起きる[おきる]	"「僕はベッドから起きる」[「ぼくはベッドからおきる」] 
\\	洗う	
\\	洗う[あらう]	"洗面台[せんめんだい] 
\\	洗面所[せんめんじょ] 
\\	「手を洗ってください」[「てをあらってください」] 
\\	泳	
\\	泳[エイ 
\\	およ]	泳ぐ[およぐ] 
\\	水泳[すいえい] 
\\	国際	
\\	国際[こくさい]	国際電話[こくさいでんわ] 
\\	際[きわ] 
\\	何語	
\\	何語[なにご]	「何語を話しますか」[「なにごをはなしますか」] 
\\	衛星放送	
\\	衛星放送[えいせいほうそう]	衛星[えいせい] 
\\	放送[ほうそう] 
\\	衛[えい] 
\\	星[せい] 
\\	放ります[ほうります] 
\\	送ります[おくります] 
\\	天気	
\\	天気[てんき]	"「天気は良いです」[「てんきはいいです」] 
\\	「今日は悪い天気です」[「きょうはわるいてんきです」] 
\\	両替	
\\	両替[りょうがえ]	"替え[かえ] 
\\	両替します[りょうがえします] 
\\	「ユーロを円に両替しました」[「ユーロをえんにりょうがえしました」] 
\\	面接	
\\	面接[めんせつ]	"接[せつ] 
\\	面[めん] 
\\	能面[のうめん] 
\\	「面接を受けます」[「めんせつをうけます」] 
\\	「面接をします」[「めんせつをします」] 
\\	打ち合わせ	
\\	打ち合わせ[うちあわせ]	打ち合わせします[うちあわせします] 
\\	免許	
\\	免許[めんきょ]	免[めん] 
\\	許[きょ] 
\\	車の免許を持っています[くるまのめんきょをもっています] 
\\	免許があります[めんきょがあります] 
\\	健康	
\\	健康[けんこう]	"健[けん] 
\\	康[こう] 
\\	「健康です」[「けんこうです」] 
\\	買い物	
\\	買い物[かいもの]	買い物する[かいものする] 
\\	水泳	
\\	水泳[すいえい]	"「ダリオ君は水泳が得意です」[「ダリオくんはすいえいがとくいです」] 
\\	柔道	
\\	柔道[じゅうどう]	柔らかな[やわらかな] 
\\	道[みち] 
\\	空手	
\\	空手[からて]	"空[から] 
\\	手[て] 
\\	飛ぶ	
\\	飛ぶ[とぶ]	"「飛行機で空を飛びます」[「ひこうきでそらをとびます」] 
\\	を)
\\	取る	
\\	取る[とる]	
\\	止める	
\\	止める[とめる]	鉄男さんは自転車を止めました[てつおさんはじてんしゃをとめました] 
\\	糸	
\\	糸[いと]	
\\	左	
\\	左[ひだり]	"左手[ひだりて] 
\\	「左利きです」[「ひだりききです」] 
\\	左翼[さよく] 
\\	「彼は左翼の人です」[「かれはさよくのひとです」] 
\\	右	
\\	右[みぎ]	"右手[みぎて] 
\\	「左利きではありません。右利きです。」[「ひだりききではありません。みぎききです。」] 
\\	右翼[うよく] 
\\	「左右を見てください」[「さゆうをみてください」] 
\\	右 
\\	花	
\\	花[はな]	"花屋[はなや] 
\\	開花[かいか] 
\\	桜の花[さくらのはな] 
\\	「お花見に行きましょう」[「おはなみにいきましょう」] 
\\	生け花[いけばな] 
\\	花びん[かびん] 
\\	化けます[ばけます] 
\\	一段] お化け[おばけ] 
\\	化学[かがく] 
\\	心	
\\	心[こころ 
\\	シン]	中心[ちゅうしん] 
\\	心中[しんじゅう] 
\\	心中物[しんじゅう物] 
\\	思う 
\\	思います	
\\	思う 
\\	思います[おもう 
\\	おもいます]	"考える 
\\	考えます[かんがえる 
\\	かんがえます] 
\\	「春子さんはスウェーデンへ行きたいと思っています」[「はるこさんはスウェーデンへ行きたいとおもっています」] 
\\	忘れる 
\\	忘れます	
\\	忘れる 
\\	忘れます[わすれる 
\\	わすれます]	"亡[ぼう] 
\\	「この漢字の読み方を忘れました」[「このかんじのよみかたをわすれました」] 
\\	鳴る 
\\	鳴ります	
\\	鳴る 
\\	鳴ります[なる 
\\	なります]	"「非常ベルが鳴りました」[「ひじょうベルがなりました」] 
\\	歩く 
\\	歩きます	
\\	歩く 
\\	歩きます[あるく 
\\	あるきます]	散歩[さんぽ] 
\\	散歩します[さんぽします] 
\\	一歩[いっぽ] 
\\	覚える 
\\	覚えます	
\\	覚える 
\\	覚えます[おぼえる 
\\	おぼえます]	"覚えています[おぼえています] 
\\	覚める 
\\	覚めます[さめる 
\\	さめます] 
\\	「夕べは二十ぐらい漢字を覚えました」[「ゆうべはにじゅうぐらいかんじをおぼえました」] 
\\	「映画のタイトルを覚えていますか」[「えいがのタイトルをおぼえていますか」] 
\\	目が覚めます[めがさめます] 
\\	目覚まし時計[めざましどけい] 
\\	押す 
\\	押します	
\\	押す 
\\	押します[おす 
\\	おします]	"「ドアを押して中に入ります」[「ドアをおしてなかにはいります」] 
\\	「危ないですから、押さないでください」[「あぶないですから、おさないでください」] 
\\	卒論	
\\	卒論[そつろん]	
\\	使う 
\\	使います	
\\	使う 
\\	使います[つかう 
\\	つかいます]	「難しい言葉を使わないでください」[「むずかしいことばをつかわないでください」] 
\\	右側	
\\	右側[みぎがわ]	左側[ひだりがわ] 
\\	消える 
\\	消えます	
\\	消える 
\\	消えます[きえる 
\\	きえます]	電気が消える[でんきがきえる] 
\\	電気を消す[でんきをけす] 
\\	道	
\\	道[みち]	
\\	非常	
\\	非常[ひじょう]	非常口[ひじょうぐち] 
\\	非常に[ひじょうに] 
\\	非ー[ひー] 
\\	煙	
\\	煙[けむり]	工場は煙を出します[こうじょうはけむりをだします] 
\\	鳴く 
\\	鳴きます	
\\	鳴く 
\\	鳴きます[なく 
\\	なきます]	猫が鳴きます[ねこがなきます] 
\\	鳥が鳴いています[とりがないています] 
\\	心配する 
\\	心配します	
\\	心配する 
\\	心配します[しんぱいする 
\\	しんぱいします]	"「心配です」[「しんぱいです」] 
\\	「今度の試験が心配です」[「こんどのしけんがしんぱいです」] 
\\	曲がる 
\\	曲がります	
\\	曲がる 
\\	曲がります[まがる 
\\	まがります]	"「左に曲がってくださいねー」[「ひだりにまがってくださいねー」] 
\\	曲[きょく] 
\\	「この曲が好きです」[「このきょくがすきです」] 
\\	早く	
\\	早く[はやく]	"早い[はやい] 
\\	「まだ早いです」[「まだはやいです」] 
\\	「毎朝早く起きます」[「まいあさはやくおきます」] 
\\	調べる 
\\	調べます	
\\	調べる 
\\	調べます[しらべる 
\\	しらべます]	"「辞書で漢字の意味を調べます」[「じしょでかんじのいみをしらべます」] 
\\	辺	
\\	辺[へん]	海辺[うみべ] 
\\	この辺[このへん] 
\\	お皿	
\\	お皿[おさら]	
\\	割る 
\\	割ります	
\\	割[わ]る 
\\	割[わ]ります	"「お皿を割りました」[「おさらをわりました」] 
\\	「お皿を割ってしまいました」[「おさらをわってしまいました」] 
\\	先輩	
\\	先輩[せんぱい]	
\\	機械	
\\	機械[きかい]	
\\	煙草を吸う	
\\	煙草を吸う[たばこをすう]	煙草[たばこ] 
\\	煙[けむり] 
\\	草[くさ] 
\\	吸う 
\\	吸います[すう 
\\	すいます] 
\\	乗り場	
\\	乗り場[のりば]	
\\	写真	
\\	写真[しゃしん]	写真を写します[しゃしんをうつします] 
\\	写真を撮ります[しゃしんをとります] 
\\	写す 
\\	写します	
\\	写す 
\\	写します[うつす 
\\	うつします]	"「この手紙を写します」[「このてがみをうつします」] 
\\	外	
\\	外[ガイ 
\\	そと]	"内と外[うちとそと] 
\\	「外はまだ明るいです」[「そとはまだあかるいです」] 
\\	「窓から外を見ています」[「まどからそとをみています」] 
\\	竹	
\\	竹[たけ]	"『竹取物語』[『たけとりものがたり』] 
\\	合う	
\\	合う[あう]	"合わせる[あわせる] 
\\	一段] 「彼に手を合わせて頼みました」[「かれにてをあわせてたのみました」] 
\\	試合[しあい] 
\\	答える	
\\	答[こた]える	質問に答えます[しつもんにこたえます] 
\\	練習	
\\	練習[れんしゅう]	
\\	林	
\\	林[はやし]	"「後で林の中を散歩します」[「あとではやしのなかをさんぽします」] 
\\	登る	
\\	登る[のぼる]	山に登る[やまにのぼる] 
\\	登山[とざん] 
\\	案内	
\\	案内[アンナイ]	"案内所[あんないじょ] 
\\	案内します[あんないします] 
\\	名案[めいあん] 
\\	内容[ないよう] 
\\	家内[かない] 
\\	「家内は今出かけています」[「かないはいまでかけています」] 
\\	正しい	
\\	正しい[ただしい]	正しく[ただしく] 
\\	お正月[おしょうがつ] 
\\	正式な[せいしきな] 
\\	正式に[せいしきに] 
\\	草	
\\	草[くさ]	"「草が生えます」[「くさがはえます」] 
\\	漢詩	
\\	漢詩[カンシ]	
\\	歌舞伎	
\\	歌舞伎[カブキ]	
\\	送別会	
\\	送別会[ソウベツカイ]	
\\	顔	
\\	顔[かお]	"「顔が赤いです」[「かおがあかいです」] 
\\	別人	
\\	別人[べつじん]	"「南さんはお酒を飲むと、別人になります」[「みなみさんはおさけをのむと、べつじんになります」] 
\\	「あれは主人じゃありません。別人です。」[「あれはしゅじんじゃありません。べつじんです。」] 
\\	自分	
\\	自分[ジブン]	"「彼は図書館より自分の部屋で勉強したいと思っています」[「かれはとしょかんよりじぶんのへやでべんきょうしたいとおもっています」] 
\\	「自分で運転しました」[「じぶんでうんてんしました」] 
\\	外国	
\\	外国[ガイコク]	外国人[がいこくじん] 
\\	外人[がいじん] 
\\	試合	
\\	試合[しあい]	
\\	運動場	
\\	運動場[ウンドウジョウ]	"「学校の運動場でサッカーの試合をしました」[「がっこうのウンドウジョウでサッカーのしあいをしました」] 
\\	球	
\\	球[キュウ 
\\	たま]	ボール[ボール] 
\\	テニスと野球とゴルフの球[テニスとやきゅうとゴルフのたま] 
\\	サッカーのボール[サッカーのボール] 
\\	病気	
\\	病気[ビョウキ]	山田さんは病気になりました[やまださんはビョウキになりました] 
\\	私は今病気で、寝ています[わたしはいまビョウキで、ねています] 
\\	客	
\\	客[きゃく]	今、家にお客さんが来ています[いま、うちにおきゃくさんがきています] 
\\	店の中に日本人のお客がおおぜいいます[みせのなかににほんじんのおきゃくがおおぜいいます] 
\\	薬	
\\	薬[くすり]	"「この薬を飲むと、風邪が治ります」[「このくすりをのむと、かぜがなおります」] 
\\	熱	
\\	熱[ねつ]	"熱がある[ねつがある] 
\\	熱が出る[ねつがでる] 
\\	「熱が高いです」[「ねつがたかいです」] 
\\	「彼女は風邪で、熱が出ました」[「かのじょはかぜで、ねつがでました」] 
\\	医者	
\\	医者[いしゃ]	"「医者に行ったほうがいいですよ」[「いしゃにいったほうがいいですよ」] 
\\	宝くじ	
\\	宝くじ[たからくじ]	宝[たから] 
\\	大阪	
\\	関西 (かんさい)	大阪[おおさか]	
\\	近松	
\\	近松[ちかまつ]	近松門左衛門[ちかまつもんざえもん] 
\\	正	
\\	正[ただし]	
\\	お箸	
\\	お箸[おはし]	
\\	投げる	
\\	[一段, 
\\	投げる[なげる]	球を投げる[たまをなげる] 
\\	打つ	
\\	打つ[うつ]	ピンポンの球を打つ[ピンポンのたまをうつ] 
\\	相談する	
\\	[不規則]	相談する[そうだんする]	友だちと相談します[ともだちとそうだんします] 
\\	先生に相談します[せんせいにそうだんします] 
\\	卒業する	
\\	[不規則]	卒業する[そつぎょうする]	"「東京大学を卒業しました」[「とうきょうだいがくをそつぎょうしました」] 
\\	案内する	
\\	[不規則]	案内する[あんないする]	"「春子さんにボルゲーセ公園を案内します」[「はるこさんにボルゲーセこうえんをあんないします」] 
\\	注意する	
\\	[不規則]	注意する[ちゅういする]	"「車に注意してください」[「くるまにちゅういしてください」] 
\\	「先生が学生に注意しました」[「せんせいががくせいにちゅういしました」] 
\\	入学する	
\\	[不規則]	入学する[ニュウガクする]	"「ローマ大学に入学しました」[「ローマだいがくににゅうがくしました」] 
\\	踊る	
\\	[五段]	踊る[おどる]	"「サンバを私と踊りますか」[「サンバをわたしとおどりますか」] 
\\	急ぐ	
\\	急ぐ[いそぐ]	
\\	選ぶ	
\\	選ぶ[えらぶ]	"「きれいな服を選びました」[「きれいなふくをえらびました」] 
\\	残る	
\\	残る[のこる]	"「家に残ります」[「うちにのこります」] 
\\	風邪を引く	
\\	風邪を引く[かぜをひく]	
\\	当たる	
\\	[五段, 
\\	当たる[あたる]	"「当たった!」[「あたった!」] 
\\	宝くじ[たからくじ] 
\\	宝くじに当たる[たからくじにあたる] 
\\	生える	
\\	[一段]	生える[はえる]	"「いつ草が生えますか」[「いつくさがはえますか」] 
\\	「娘は最初の歯が生えました」[「むすめはさいしょのはがはえました」] 
\\	生まれる	
\\	[一段]	生まれる[うまれる]	"「子供はもう生まれましたか」[「こどもはもううまれましたか」] 
\\	落ちる	
\\	[一段]	落ちる[おちる]	試験に落ちる[しけんにおちる] 
\\	海に落ちる[うみにおちる] 
\\	窓から落ちる[まどからおちる] 
\\	痛い	
\\	痛い[いたい]	"「頭が痛いです」[「あたまがいたいです」] 
\\	「どこが痛いですか。」[「どこがいたいですか。」] 
\\	「この歯が痛いです。」[「このはがいたいです。」] 
\\	丈夫な	
\\	丈夫な[ジョウブな]	
\\	時々	
\\	時々[ときどき]	"「時々朝早く目が覚めます」[「ときどきあさはやくめがさめます」] 
\\	遅く	
\\	遅く[おそく]	遅い[おそい] 
\\	遅く寝ます[おそくねます] 
\\	初めて	
\\	初めて[はじめて]	"「きのう初めて馬に乗りました」[「きのうはじめてうまにのりました」] 
\\	「どうでしたか」[「どうでしたか」] 
\\	「落ちて、とても痛かったです」[「おちて、とてもいたかったです」] 
\end{CJK}
\end{document}