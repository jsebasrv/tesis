\documentclass[8pt]{extreport} 
\usepackage{hyperref}
\usepackage{CJKutf8}
\begin{document}
\begin{CJK}{UTF8}{min}
\\	彼が私の目を見ようとしなかったので、何かを隠していると思った。	
\\	~を隠す 
\\	の目をまっすぐ見る
\\	彼女は我慢しきれなくなり、襟首をつかんで彼をののしった。	
\\	尽きる 
\\	襟 
\\	~をののしる
\\	確かに、私たちは時々口げんかもしますっが、普段は仲良くやっています。	
\\	口げんかする 
\\	たまに、時々 
\\	~と親しい間柄である
\\	「君の家によってもいいかい?」 「もちろんいいわよ。気兼ねなく!」	
\\	~に立ち寄る 
\\	いや全然(そんなことありません) 
\\	遠慮しないで
\\	ちょっと待ってて。すぐ戻るから。切らないでよ!	
\\	うちには泊められないわ。一つには、時々お父さんがふらっとやって来るから。	
\\	~を泊める 
\\	ふらっと立ち寄る 
\\	時々
\\	ジェニファーがじっと立っている間、ボブは歩道を行ったり来たりしていた。	
\\	じっと立つ 
\\	(落ち着きなく)歩く 
\\	行ったり来たり 
\\	歩道
\\	「仲直りしようよ、リサ。」 「私がいて当然だなんて思わないで。私たち、今度は永遠に終わりよ。本気なんだから!」	
\\	仲直りする 
\\	を当然のことと思う 
\\	を
\\	だとおもう 
\\	終わる 
\\	永遠に 
\\	本気で言っているんだよ
\\	おい!リサ、そう感情的になるなよ。君を傷つけるつもりじゃなかったんだ。じっくり話し合おう。	
\\	おいおい 
\\	感情的な 
\\	~するつもりである 
\\	~についてよく話し合う
\\	「私を裏切ったわね!」 「許してくれよ。どうしたら埋め合わせができるかなあ?」 「仕返ししてやるっ!」	
\\	~の埋め合わせをする 
\\	仕返しする
\\	約束は守りなさい。そうしないと評判を落としますよ。	
\\	約束を守る 
\\	結果になる 
\\	評判を落とす
\\	ボブは自分の道徳的信条を曲げない男だ。正直でいれば最後には報われると信じている。	
\\	~を(あくまでも)守る 
\\	報われる
\\	「元気を出して。仕方ないわ。」 「やれるだけのことはやったんだ。」 「わかってる。過ぎたことにくよくよしないで。やり直せるわよ!」	
\\	~は仕方がない 
\\	~をくよくよ考える 
\\	一からやり直す
\\	「ジェーン、タイムズ・スクェアーまで乗せて行って欲しいんだけど。」 「そうしてあげたいのは山々だけど。今、全然時間がないの。」	
\\	お願いがあるんだけど 
\\	を~まで乗せて行ってあげる 
\\	覚えるべし!!!めっちゃ大事 
\\	ちょっと無理かなあ、 
\\	とても忙しい
\\	いつものように、マイクは時間通りに現れた。本当に時間に正確な人だ。	
\\	いつも通り 
\\	現れる 
\\	時間通りに 
\\	時間を厳守する
\\	この前、混雑した幹線道路の真ん中でガス欠してしまった。	
\\	~を使い果たす 
\\	にぎやかな、交通量が多い(形容詞) 
\\	幹線道路、インターステイト
\\	私の前を走っていたトラックがスリップして横転し、積み荷を路面いっぱいに散乱させた。	
\\	スリップする 
\\	ひっくり返る
\\	衝突を避けられたのは運が良かっただけ、と彼は回想した。	
\\	全くの~
\\	緊急の際には、すぐに私の代理人と連絡を取って下さい。	
\\	(~と)連絡を取る 
\\	すぐに、ただちに
\\	「ジェーン、俺たちはどこに向かってるんだ?」 「迷ったみたい。違う方向に来ちゃったわ。」 「何だよ。車を止めてくれ。」	
\\	向かう 
\\	迷う 
\\	(車を)路肩に寄せて止める
\\	僕のさびついたフォードが故障して、交差点をふさいでしまった。	
\\	さびた 
\\	~の進路を妨げる 
\\	交差点
\\	表面に細かい傷やへこみがたくさんあるから、磨いても無駄だよ。	
\\	細かい(形容詞) 
\\	へこみ
\\	「おい。渋滞にはまっちゃったよ!」 「熱くならないで、ジョー。近道しましょう。時間までには着けるわ。」	
\\	動かなくなる(動かない)、はまる(はまっている) 
\\	冷静になる 
\\	~に着く 
\\	間に合って、予定時刻までに
\\	「パンクしちゃったよ。手を貸してくれる?」 「喜んで。」	
\\	タイヤがパンクする 
\\	~を手助けする 
\\	ええ、喜んで
\\	ハンドルがきかなくなった救急車が、危うく歩行者をひきそうになった。	
\\	救急車 
\\	制御がきかない 
\\	危うく~しそうになる 
\\	~をひく
\end{CJK}
\end{document}