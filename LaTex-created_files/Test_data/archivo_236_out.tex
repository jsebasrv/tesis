\documentclass[8pt]{extreport} 
\usepackage{hyperref}
\usepackage{CJKutf8}
\begin{document}
\begin{CJK}{UTF8}{min}
\\	菊の根分をしながら
\\	會津八一
\\	【テキスト中に現れる記号について】
\\	ルビ
\\	(例)軈《やが》て
\\	入力者注 主に外字の説明や、傍点の位置の指定
\\	(例)[#ここから2字下げ]
\\	二倍の踊り字(「く」を縦に長くしたような形の繰り返し記号)
\\	(例)いろ/\
\\	濁点付きの二倍の踊り字は
\\	昨日が所謂彼岸の中日でした。吾々のやうに田舎に住むものの生活が、これから始まるといふ時です。私も東京の市中を離れた此の武蔵野の畑の最中に住んで居るから、今日は庭の隅に片寄せてある菊の鉢を取り出して、この秋を楽しむ為に菊の根分をしようとして居るところです。実は私は久しいこと菊を作つて居るのであるが、此二三年間は思ふ所あつて試にわざと手入れをしないで投げやりに作つて見た。一体菊と云ふものは其栽培法を調べて見ると、或は菊作りの秘伝書とか植木屋の口伝とかいふものがいろ/\とあつて、なか/\面倒なものです。これほど面倒なものとすれば、到底素人には作れないと思ふほどやかましいものです。そして此色々な秘訣を守らなければ、存分に立派な菊が作られないといふことになつて居る。ところが私は昨年も一昨年もあらゆる菊作りの法則を無視して作つて見た。たとへば春早く根分けをすること、植ゑる土には濃厚な肥料を包含せしめなければならぬこと、鉢はなるべく大きなものを用ゐること、五月、七月、九月の芽を摘まなければならぬこと、日当りをよくすること、水は毎日一回乃至数回与へなければならぬこと、秋になつて又肥料を追加し、雑草を除くことなどと、まだ/\いろ/\の心得があるのにも拘らず、二三年の間は私はまるで之をやらなかつた。根分もやらず、小さい鉢に植ゑた儘で、土を取り替へもせず、芽も摘まず、勿論水も途絶え勝であつた。云はゞあらゆる虐待と薄遇とを与へたのだ。それでも秋になると菊は菊らしくそれ/
\\	に蕾が出て、綺麗な色で、相当に優しい花を見せてくれた。それで考へて見れば菊の栽培といつても絶対的に必須なものでもないらしい。手入れをすれば勿論よろしい。しかし手入れが無くとも咲く、植木屋などがよく文人作りなどと名をつけて売つて居るのは私などから見れば、いつも少し出来過ぎて居て、かへつて面白くない。私の庭の隅に咲いた菊の花の天然の美しさにより多く心が惹かれぬでもない。
\\	併し考へて見ると、世間で観賞されて居る多数の植物の中では温室の中で一定の化学的成分を含んだ肥料を施さなければ生長しないもの、湿度や温度を綿密に塩梅しなければ出来ない物、特別な光線を与へなければならぬものとか色々なものがある。保護が無ければすぐ枯れて仕舞ふ。斯ういふ植物と、虐待、欠乏の中にあつて、尚強い根強い力を振り起して何時までも生き長へて美しい花を開く私の庭の菊の如きものと比較して見ると、無限の感慨が生ずるのである。之を人にたとへて云ふならば名望のある富貴の家に生れて、健全な父母を保護者として育ち、求め得ざるは無く、欲して遂げざるはなく、教育も思う儘に受けられ、何一つ事を欠かぬといふ人もあらう。又相当に艱苦にも、欠乏にも堪へて行かなければならぬ人もあらう。一体今の世の中には、放置せられて居て、なほ自分自身の根強い力を振り起して軈《やが》ては美しい花を咲かせるだけの意気込のある少年が多いであらうか。文化の進むにつれて、温室の中の植物のやうな人が、漸く増加して来るのは免れ難い傾向である。試みに田舎の少年と都会の少年とを比較すると、其間にももう斯うした区別が認められる。世の中が複雑になつて、文明の設備を利用することが多くなれば、自ら人間の性質が変化して天然の素質が幾分かづゝ弱つたり失はれたりして行くことも止を得ないかも知れないけれども、吾々は吾々として最も大切なものを失つてはならぬ。それは吾々の心の底に横《よこたわ》つて居る根強い力である。
\\	今日のやうな日に縁側から外を眺めて、暖かな太陽の光を浴びて、大地の底や枯れたやうな老木の幹から、輝いた鮮かな芽が萌え出て居るのを見る時に、私は其中に何とも云はれない一種の力を感ずる。そして草木にも吾々人間にも天然に与へられてある此力を限りなく頼もしく思はずに居られない。
\\	そも/\吾々が生れ出て勉強して世の中へ出て暮らして行くのは人に頼まれてのことではない。吾々自身が持つて生れた力、これを自分の境遇に応じて、時としては境遇以上にも伸びるだけ伸ばして行く為めである。吾々が貧困の家に生れて欠乏の中に生長し、如何程の苦学を重ねても、自分の心の底に此力を見出して進んで行かねばならぬ。それにつけては独立自恃の精神ほど大切なものは無いのである。
\\	しかし私は今年は菊を作るのにこれまでとは全く方針をかへて、根分も、採光も、肥料も、剪定も、灌水も出来るだけの優遇を与へて昨年よりは一層美しい花を見たいと思つて居る。独立自恃の精神のあるものは容易に他の援助や庇護を希はない。しかし援助を与へて庇護を加へらるべき第一の資格は此の独立自恃の精神の存在である。一昨年以来菊が私に示した悲壮な態度、その元気の頼もしさに私も心から栽培を促されるのである。同情や援助といふものは求めても無暗に与へられるものではない。猥りに左様いふものを求めざる人こそ与へらるべきであるのだ。
\\	それから又、いよ/\菊の苗を分けようとするときに、如何なる苗を選ぶべきであらうか。勿論吾々は最も有望な苗を選ばなければならぬ。一株の古根からは幾十本となく若い芽が吹き出して居る。それが一様に生気に満ちたもののやうに見える。しかし経験のある栽培家は思ひもかけぬほど遠い所へ顔を出して居る芽を択ぶのである。親木のわきに在る芽はどうしても弱い。よくよく自分の活力に自信のあるのが親木をたよらずに遠くまで行く、其意気を栽培家は壮なりとするのである。私も今年は勿論そのつもりである。
\\	世に云ひ古された、「今日になりて菊作らうと思ひけり」といふ俳句、是は格別文学的でもないかもしれぬが、秋を迎へてから他人の作つた菊の花を見て、羨しく思つて眺める気持を詠んだもので誰にも経験しさうな事であるだけに有名な句になつてゐる。しかし此句を修養的に味つてみようとする人は、秋になつたらもう遅い。此句を誦みながら庭なり畑なりへ下り立つて季節を失はずに、しかも自分で土いぢりを始めるならば、やがては其花の如く美しい将来が、其人の身の上にも展開して来るであらう。
\\	私はさきにもいふやうに落合村の百姓で、歌人でも何でも無いけれども、今日はあまりに気候の心地よさに、歌のやうなものが少しばかり出来た。それを此所で御披露に及ぶといふことにしよう。
\\	[#ここから2字下げ]
\\	さ庭べの菊の古根も打ち返へし分ち植うべき春は来にけり
\\	菊植うと下り立つ庭の木の間ゆもたま/\遠き鶯の声
\\	取り持てばもろ手にあふれ籠に盛れば籠にあふれたる菊の苗かも
\\	十の指土にまみれて狭庭べに菊植うる日ぞ人な訪ひそね
\\	今植うる菊の若草白妙に庭を埋めて咲かずしもあらず
\\	今植うる菊の草むら咲き出でて小蜂群れ飛ぶ秋の日をおもふ
\\	武蔵野の木ぬれを茂み白菊の咲きて出づとも人知らめやも
\\	武蔵野の霞める中にしろ妙の富士の高根に入日さす見ゆ
\\	[#ここで字下げ終わり]
\\	[#地付き](大正十二年三月二十三日談)
\\	底本:「花の名随筆3 三月の花」作品社
\\	1999(平成11)年2月10日初版第1刷発行
\\	底本の親本:「會津八一全集 第七巻」中央公論社
\\	1982(昭和57)年4月発行
\\	入力:門田裕志
\\	校正:仙酔ゑびす
\\	2006年11月18日作成
\\	青空文庫作成ファイル:
\\	このファイルは、インターネットの図書館、青空文庫
\\	で作られました。入力、校正、制作にあたったのは、ボランティアの皆さんです。
\end{CJK}
\end{document}