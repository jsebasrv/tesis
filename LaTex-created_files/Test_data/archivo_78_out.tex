\documentclass[8pt]{extreport} 
\usepackage{hyperref}
\usepackage{CJKutf8}
\begin{document}
\begin{CJK}{UTF8}{min}
\\	~あっての
\\	あってのことだ	
\\	これはインターネットあってのビジネスだ。 
\\	家族のお蔭で今日まで頑張ることができました。家族あっての私だと感謝しています。 
\\	大手のGM自動車と雖も(いえども)、部品を作るのは下請け会社だ。下請けあってのGM社である 
\\	~ができたのは
\\	のような友人たちの支援があってのことだ 
\\	彼があんなことを言ったのも、きっと何か理由があってのことに違いない。 
\\	がなければ 
\\	、Bは存在できない。AはBの必要条件。Aには、人や組織が多く用いられる。 
\\	あっての+N2	ならではの~ ~なくてはならない ~があるからこそ・いるからこそ
\\	~いかんだ (である)・ ~いかんによっては (如何だ・如何に依って)	
\\	その応募者を採用するかどうかは、人事部長の判断いかんだ 
\\	テストの成績いかんによっては、進級できない学生も出る 
\\	燃料電池自動車が普及するかどうかは価格とパワーいかんである 
\\	実験の結果いかんによっては、大発見になる可能性もある 
\\	賞味期限のある食品類は開封・未開封のいかんに関わらず返品対象外となります 
\\	いかんで、B 
\\	はAいかんだ 
\\	がどうであるかによって、B(結果)がいろいろに変わる 
\\	いかん 
\\	(の)+いかん 
\\	~次第  ~にかかっている この会社に採用されるかどうかは、面接の結果いかん/次第だ
\\	~(の)いかんによらず ・ ~いかんにかかわらず	
\\	一度納めた入学金は、理由のいかんによらず返金されない 
\\	試験の結果いかんにかかわらず、必ず私に連絡してください 
\\	いかんによらずB  
\\	いかんにかかわらず、B (の) 
\\	いかんによらず 
\\	の+いかんによらず 
\\	~に関係ない・~に関係なく 「好むと好まざるとにかかわらず、友達は助けなくてはいけないよ」。 
\\	いかん(だ) 
\\	いかんで 
\\	いかんによって 今年の新車の売れ行き(いかんで/いかんによって)この会社の将来が決まりそうだ 
\\	このテストの結果いかんによっては今卒業できないかもしれない 
\\	実験の結果いかんによっては、大発見にはる可能性もある 
\\	(どんなに) 
\\	1うが  
\\	1うが 
\\	1まいが ・ 
\\	1うと 
\\	1まいと 
\\	父はどんなに遅く帰ろうが、寝る前に必ず日記を書く(に書き込む/を付ける) 
\\	結婚しようがしまいが、本人の自由です。 
\\	君がそこへ行こうと行くまいと、私には関係のないことだ 
\\	たとえ~しても 
\\	しても、しなくても 
\\	、Bは同じ動詞 
\\	意志形+が、V辞書形+まいが・まいと	
\\	うまく行くにつけ、いかないにつけ 
\\	どんなに/いかに 遅く帰っても
\\	行くべきかどうか、
\\	スラングを使うか否かは 
\\	買おうか迷っています 
\\	~うにも 
\\	~ない 
\\	~うとしない 
\\	~うものなら 
\\	ものならず	
\\	倒れた木に妨害されたて、車は通ろうにも通れない 
\\	この本は難しすぎて、読もうにも読めない 
\\	~したくてもできない したいこと、しようと思うことを邪魔する何かがある	~うとしない=絶対に~しない 
\\	~うものなら=~したら(大変だ) 
\\	ものならず=
\\	~かぎりだ 
\\	豪華船で世界一周旅行をするんですって。羨ましいかぎりですね。 行ったことのない国へ一人で行かなければならない。心細いかぎりだが、仕事なので、行かざるを得ない。 
\\	非常に~だと感じる ~は、感情を表す形容詞	の極みだ
\\	~が最後、	
\\	自然は一度破壊されたが最後、元の状態に戻すことはできない 
\\	彼は飲みだしたが最後、一晩中飲み続けるから、酒は飲ませないこと 
\\	いったん~したら、それっきり 
\\	が最後B  
\\	したら、必ずB。  
\\	はよくないことを表す分が多い	ちょっと~でも~ば(悪いことが起きる)
\\	が早いか
\\	ベルが鳴るが早いか、学生は机の上の本やノートを鞄に仕舞った。
\\	息子は「只今」と言って、鞄を置くが早いか、外に飛び出していった。 
\\	其の知らせを聞くが早いか、彼女は真っ青になって、立ち上がった。	~と、直ぐ 殆ど同時に ~や否や 
\\	(私、僕、俺),
\\	~たらすぐ 
\\	が 
\\	辞書形+が早いか	~かと思うと=空が暗くなったかと思うと、ぱらぱらと雨が降ってきた ~か~ないかの内に=一つの仕事が終わるか終わらないかのうちに、次の仕事が来る。 
\\	ですぐ(に) 
\\	とすぐ 
\\	したらすぐ 会社に着いたらすぐに電話して やいなや 
\\	家を出るや否や激しく雨が降り出した。
\\	かいなか 
\\	た「過去形」とたん 
\\	「彼らは顔を合わせたとたん、直ぐに仲良くなった」。 
\\	~からある	
\\	一日に100件からある電話相談に、一人だけで対応するのは無理だ。 
\\	その俳優は5ページからある長い台詞を直ぐに覚えてしまった。 
\\	50キロからある荷物を運ばなければならない仕事だよ。君にできるかなあ 
\\	~もある 
\\	~は(ある・するEtc
\\	=~以上 少なくとも あのお相撲さんは大きいね。体重が150キロはあるだろう。 そのダイアは100万はするだろう! ~からあるの反対の表現=~もない、~以下、多くとも 彼女は40キロもないんじゃない。
\\	~きらいがある 
\\	ともすると・ともすれば。。 とかく・とかくすれば。。ややもすると・ややもすれば~する 
\\	彼は慎重すぎて、決断が遅れるきらいがある 
\\	確かにコンピューターは便利なものではあるが、現代人は、ともするとコンピューターに頼り過ぎる嫌いがある。 
\\	最近のテレビ番組製作者は、よい番組を作ることより、視聴率を上げることのほうを重視(じゅうし)する嫌いがあ 
\\	この会社には男性優遇のきらいがあるのではないでしょうか。
\\	今の子共たちは、夜遅くまでテレビゲームなどをして寝ないきらいがある 
\\	彼女が親切な人ですが、知らずに人を傷つけることを言うきらいがある 
\\	日本の教育は子共を社会から保護しすぎるきらいがある 
\\	~(の)傾向がある 
\\	病気のきらがある 
\\	病気がちだ 
\\	とかく
\\	ともすると
\\	若い人はとかくそんなふうに行動するものだ	とかく~する ・ 人間にはとかく怠けたがる傾向があることは否定できません。 ともすると ・ ともすれば~する ・ 冒険には、ともすると危険が伴う。 ややもすると・ややもすれば~する 
\\	私たちは、ややもすれば異質なものを排除しようとする。 ーがち 
\\	~極まる ・ ~極まりない 
\\	極めて、とびきり、実に	
\\	客に対して失礼極まる態度で接した其の店員は、直ぐに辞めさせられた。 
\\	その小説を読んでみたが、平凡極まるストーリーで、がっかりした。 
\\	電車が近づいているのに踏切を渡るとは、危険極まりない。 
\\	健康セミナーに出て、医者の話を聞き、「酒の飲み過ぎやタバコの吸い過ぎは不健康極まりない」ということを再確認させられた。	非常に~ 
\\	極めて、とびきり、実に=とても 
\\	~ごとき ・ ~ごとく (如き・如く・如し)	
\\	指導員は、私のごとき初心者にも、丁寧によく教えてくれた。 私如き者がこんな立派な賞を頂くとは、感激の極みでございます。 鳥のごとく空を飛ぶこと、これは人間の夢である。	~(の)ような・ように 
\\	*ごとい 
\\	悪夢の如き出来事=ような 悪夢のごとく恐ろしい=ように その恐ろしさ悪夢のごとし=ようだ
\\	~ことなしに	
\\	上司に一言も相談することなしに企画を進めてしまった。まずかったと反省している。 
\\	ベストを尽くそう。そして、たとえ失敗しても、後悔することなしに一からやり直せばいい。 
\\	精神を磨き高めることなしに人としての成長は有り得ない。 
\\	~しないで ・ ~せずに 
\\	辞書形+ことなしに	
\\	相談しないで 
\\	高めなくちゃ
\\	{果ては}[動詞]~始末だ (しまつ) 、この始末だ	
\\	最近、弟は金に困っていて生活が苦しいらしく、友達に借金までするしまつだ。困ったものだ。
\\	論文の締め切りが迫っていて、2晩も徹夜する始末だ。もっと早く書き始めればよかったのだが。
\\	この計算、また間違っているよ。全く彼にやらせるといつもこのしまつだ。後始末は誰がするんだ。 
\\	後始末 【あとしまつ】 
\\	ゴキブリを見てキャーと叫んだだけだったのに、近所から人が寄ってきて、果てはおまわりさんまで来る始末[だった] 
\\	~ということになってしまった 
\\	~有様だ=~状況・状態だ 最近弟はお金に困っていて、友達に借金までする有様だ。 少年たちのアパートは筆舌に尽くしがたい有様だった。 
\\	筆舌に尽くし難い 【ひつぜつにつくしがたい】 
\\	~ずくめ (の
\\	(尽くめ)	
\\	規則ずくめの寮生活にはもううんざりしてしまった。 
\\	娘が結婚したし、息子は就職が決まったし、今年はいい事ずくめだった。 
\\	そんな、うまい事ずくめの話なんて信じられない。たぶん眉唾物(でたらめな話)だから、気を付けた方がよさそうだね。 
\\	~ばかり たくさんある ~は 形容詞+ことの形が多い	~づくし 
\\	久しぶりに故郷の実家に帰った。母の心づくしの手料理が食べられてうれしかった
\\	~ずにはおかない	
\\	そんなひどい詐欺にあったら、誰でも相手を訴えずにはおかないだろう。 長引く
\\	不況の波は中小企業に打撃を与えずにはおかなかった。 彼らは、攻撃されたら報復
\\	せずにはおかない。斯うして殺し合いが果てしなく続く。 技術の発達は社会環境の変化をもたらさずにはおかない 
\\	この映画は観る者に感動を与えずにはおかない 
\\	この小説は我々に人間とは何かを考えさせずにはおかない 
\\	この事故は日本とアメリカの関係に影響を及ぼさずにはおかないだろう 
\\	絶対に~する ・必ず~する ~ないではおかない 
\\	彼は必ず来る 
\\	必ず/絶対に/間違いなく/きまって/きっと~ だろう ~ずにはいられない 
\\	彼女は周りにいる者を魅了せずにはおかない, 
\\	周りの者は彼女に魅了されずにはいられない 
\\	~ずにではおかない
\\	~ずにはすまない (せずには済まない)	
\\	車のドアに付けた小さな傷とはいえ
\\	、私がやったことなのだから、修理代を弁償せずにはすまないだろう。 この件については、正直に上司に話さずにはすまないと思う。 相手にあれだけ迷惑を掛けたのだから、お詫びをせずにはすみませんよ。 危ういところで死なずに済んだ。 
\\	必ず~しなければならない ~しないで済めばいいけれど駄目だ。やはり、しなければならないという意味	~ないわけにはいかない メンバーが全員出席するそうだから、私も行かないわけには行かない。 ~ないではすまない
\\	~かたがた~ (旁; 旁々) (二つの名所の間)	
\\	先日は大変失礼致しました。今日はお詫び旁々ご相談に伺いました。 来春結婚する孫が、結婚の報告かたがた婚約者を連れてやって来た。 この度は大変を世話になりました。来週にでもお礼旁々ご報告に参ります 仕事かたがた京都見物にゆく 
\\	~のついでに ~をかねて 
\\	兼ねて 
\\	仕事を兼ねて京都見物にゆく ついでに がてら
\\	~かたわら (傍ら)	
\\	姉は会社に勤めるかたわら、夜は専門学校で英語を勉強しています。 アンさんは大学院で日本文学を研究するかたわら、地域のボランティア活動(無償奉仕・社会奉仕)もやっている。 
\\	企画は本来の広告業務のかたわら、新たにレジャー(娯楽)産業への進出を計画している。	~しながら、同時に	二足のわらじを履く
\\	私の姉は日本人に英語を教える傍ら、外国人に日本語を教える仕事もしている。二足の草鞋を履いているわけだ
\\	~がてら
\\	散歩がてら皇居前広場を通って東京駅から電車に乗ろうよ。 花見がてら上野へ買い物に行ったが、余りの人出に驚いて、桜もゆっくり見ずに帰ってきた。 
\\	「ドライブがてら道順を覚えたらどうですか」 「じゃ、助手席に乗せてください」	~をするときに 
\\	がてらB=Aをするとき、その機会を利用してBもする 
\\	は移動を表す動詞、「~に行く」の意味を持つ名詞が多い。 
\\	旅行がてら 
\\	散歩がてら	~にかこつけて=~を口実にして 
\\	社長は視察にかこつけて海外へ旅びに行くことが多い 彼女は叔父さんの病気見舞いにかこつけて、昨日から故郷の実家へ帰っている。
\\	の)
\\	こととて	
\\	休日の午後のこととて、その遊園地は家族連れで賑わっていた。
\\	何分慣れぬこととて、ミスが多く、まことに申し訳ございません。
\\	初めてのこととて、やり方を覚えるまで随分苦労しました。 
\\	その事件については、かなり以前のこととて、ほとんど記憶しておりませんが。
\\	~なので ~だから 
\\	とて 
\\	ということで(も)=って(も)	とて 
\\	私たちが環境面の問題を耳にしないで過ぎる日は1日とてない。 
\\	天井のことをはなしたとて、どうして信じるでしょう。 
\\	~すら ・ ~ですら	
\\	草一本すら生えない砂漠でも、生き物は必ずいるものだ。 手術後しばらくの間は水すら飲めず、辛かった。 これは幼稚園の子供ですら理解できる簡単なことだ 昨日駅伝競走に参加したが、今日は全身が筋肉痛で,歩くことすらおぼつかない。
\\	覚束ない 【おぼつかない】 
\\	草一本さえない砂漠でも生物は必ずいる 
\\	草が一本さえない砂漠で生えている 
\\	手術後は水さえ飲めず、辛かった。 
\\	子供だって理解できる簡単なことだ 草一本さえない砂漠でも、生物はいるもんだ。
\\	~そばから 
\\	をする)	
\\	育ち盛り(そだちざかり;
\\	の息子は、私が料理を作ってテーブルの上に置くそばから食べてしまう。 「近頃はとっても忘れっぽくなって、何でも聞いたそばから忘れてしまうの」とよく祖母が言っている。 
\\	氏は売れっ子
\\	の作家。彼が原稿を書くそばから、それを出版社の社員が持って帰るのだそうだ。	~すると、そのすぐ後で、また 
\\	(散らかす、汚す、台無しにする、めちゃくちゃにする) 
\\	辞書形・た形+そばから	~でも~でも 追っても追っても、カラスはごみを荒らしに来る。 
\\	度に (たびに) 彼女は来るたびにお土産を持って来てくれる 彼は上京するたびに私の家に泊まる 稲妻が走る(轟く)たびに犬はベッドの下に隠れた
\\	(後は・Xついてて)  ただ~のみ	
\\	練習は十分やった。後はただ試合の日を待つのみ/ばかりだ。 彼女からは、ただ短い電話があったのみで、その後ぜんぜん連絡がない。どうしたのかと心配だ。 彼の離婚については、ただ噂を聞いたのみで、本当のことは知らない。 今度のトラブルは、すべて私の不注意によるものです。本当に申し訳ないと思っており、ただ反省あるのみです。	~だけ  ~しかない 
\\	ただ~(する)ばかり 練習は十分やった。後はただ試合の日を待つばかりだ。
\\	ただ
\\	のみならず 
\\	も	
\\	"ただ人間のみならず動物もストレスを感じるというのは、驚くべきことだ。 
\\	あいつの話は眉唾物(まゆつばもの)だ。 ただしゃくに障るのみならず まるっきり信用もなくしたよ。 
\\	この株は、ただ安定性が高いのみならず、近い将来値上がりも期待できるので推奨株といえる 
\\	外交面のみならず軍事面においても可能な限り(できらだけ)合衆国を支援するべきだ  
\\	「このリスク(血栓症)はフライト後のみならず、あらゆる交通手段において、旅行者が長い時間全く動かないまま着席している場合でも高いのです」  
\\	~だけでなく  ひとり~のみならず 
\\	ばかりか、Bも 
\\	だけでなく、Bも 
\\	ばかりでなく
\\	も 男だけならともかく、女性もマスクをかけているのです 
\\	[いくら/どんなに]
\\	ところで 
\\	どうせ・どっちみち)	
\\	-しても 
\\	ここから駅まで15分はかかるから、どんなに急いだところで、9時の特急には間に会わない。 
\\	定年まで毎日一生懸命働いたところで、私の安月給
\\	では小さい家を一軒買うこともできないだろう。 
\\	もう手遅れだそうだ。手術をしたところで助からないだろう。もっと早く検診を受けておけばよかったのに 
\\	君がいくら話をしたところで私を納得させることはできないよ。 
\\	友人に相談したところで、同じだろう 
\\	喧嘩したところでどうにもならないぜ 
\\	彼は消極的でちゃらんぽらんな態度だから、いくら間違ったことを咎めたところで、いやな眼差しを投げかけて、何か出任せを言うに決まっているだろう 
\\	~でも  ~たとて~としても 
\\	どうせ 「プロポーズしたところで、どうせ断られるに決まっているよ」 どっちみち 隠したところで、どっちみちばれてしまうことだ。ばれる前に公表したほうがいいんじゃないか。 (彼は)どうせ遅れるに決まってるじゃん! 
\\	~だに	
\\	この過密
\\	都市東京で、万一大地震が起こったら、どうなるだろう。考えるだに恐ろしいことだ。 私の体脂肪率は今どれくらいあるんだろうか。考えるだに、恐ろしい 彼は何か心配事があるのか、青い顔をして、こちらを一瞥だにしなかった	
\\	想像する,考慮する、注意する、微動する、
\\	一瞥する (いちべつ 
\\	、一考する
\\	微動だにしない 
\\	思うだに悲しい 
\\	考えるだに
\\	想像するだに 
\\	聞くだに 
\\	~だけで、だけでも、さえ、すら
\\	~たりとも	
\\	「倹約しろ。、一円たりとも駄目には使うな」というのが父の口癖だ。 手術中、医師は一瞬たりとも、気を抜くことができない。手術が終わった後はくたくたになってしまう。 会社にタイムレコーダーがある限り、一分たりとも遅れられない。機械は決して待っては呉れないのだから。	~も  たった~さえも 
\\	一
\\	として~ない
\\	~たる(者・人・方、Etc)	
\\	警察官たる者、本来なら
\\	飲酒運転などするはずがないのだが。 試合までに5キロ減量すること。プロのスポーツ選手たる者が自分の体重ぐらい自分でコントロールできなくてどうするのだ。 社長たる者、そのような個人的な感情に流されて会社を危うくする
\\	べきではない。 警官たる者、そのような犯罪にかかわってはいけない	~である ~という立場にある 
\\	確たる
\\	=確かな 確たる証拠もないのに、彼が犯人だとは言えません。
\\	つ
\\	つ	
\\	分かりにくい場所だった。地図を見ながら行きつ戻りつ30分も歩き回って、やっとその店を見つけることができた 日本自動車と東京自動車とは、追いつ追われつ、激しい売り上げ競争をしている 昨日のマラソン競技会では、抜きつ抜かれつの激しい戦いの末、A選手がかろうじて優勝した	~たり~たりしながら 
\\	ためつすがめつ 矯めつ眇めつ 
\\	双子の兄弟に会った。ためつすがめつ見たけれど違いは分からなかった。 やっと出来上がった彫刻をためつすがめつ眺めて楽しんでいます。
\\	っぱなし	
\\	あ、いけない。図書館の本が借りっぱなしだ。返さなくちゃ。 
\\	大阪から東京まで、3時間立ちっぱなしで、足が痛くなった。 
\\	このドアは開けっ放しにしておいて、部屋の空気を入れ換えますから。 重要書類が机の上に出しっぱなしになっているよ。すぐにしまいなさい	~たまま 
\\	~たきり=~た後 そのまま それが最後で 村田さんとは5年ぐらい前に会ったきり、ずっとあっていない。元気だろうか これっきり、もうこれっきり?
\\	であれ
\\	であれ
\\	であれ 
\\	1のであれ、
\\	2のであれ 
\\	かれADJかれ 
\\	かろうと(難しかろうと優しかろうと) 
\\	平日であれ、週末であれ、健太は仕事に追われている 
\\	言語学者は多かれ少なかれチョムスキーの影響を受けている たとえ全社員が反対であれ、社長は今回のプロジェクトを進める
\\	つもりだそうだ。 成功率が10%以下であれ、成功する可能性があるのなら、手術を受けようと思う。 新しくできた遊園地「アミューズパーク」の人気はすごい。休日であれ、平日であれ、人波が絶える事がない。 進学するのであれ、しないのであれ、どちらにしても卒業まできちんと勉強しなさい。	~でも ~でも ~の場合でも 
\\	~にしても~にしても=~ても~ても 
\\	であろうと、Yであろうと、Xでも、Yでも、Xと言わず、Yと言わず 行くにしても行かないにしても、前もって連絡をすること。 ~にせよ~にせよ ~にしろ~にしろ 進学するにせよ、就職するにせよ、面接の練習をしておいたほうがいいよ 人間にしろ動物にしろ、子孫を残すことが生きる目的の一つである
\\	てからというもの 
\\	~てからでないと・~てからでなければ	
\\	子供が生まれてからというもの、生活のすべてが子供中心になっている。 
\\	課長に昇進してからというもの、夫はすっかり多忙になり、帰宅時間も遅くなってしまった。健康が心配だ。 
\\	パソコンを買い、Eメールを使い始めてからというもの、手紙をほとんど書かなくなった。 
\\	恋に落ちいてからというもの、君のことを考えるのをやめられない 
\\	~てから、ずっと  ~た後、ずっと	~てからでないと・~てからでなければ 
\\	まず十分にリサーチをしてからでないと、計画を実行に移すことはできません。 ~て以来 大学を卒業して社会人になって以来、運動などする暇のない生活を続けている。 心臓の手術を受けて以来、冬になると風邪を引きやすくなったような気がする
\\	~でなくてなんだろう 
\\	にほかならない/ほかならぬ~ 
\\	以外の何ものでもない	
\\	ミスを繰り返す、遅刻が多い、これが怠慢でなくてなんだろう。 
\\	あの大統領は国民の意見は無視して何でも自分一人で決定する。これが独裁者
\\	でなくてなんだろう 役所は「情報公開」と言いながら、大切なデータは決して公開しない。これが「秘密主義」
\\	でなくてなんだろう。	まさに~だ これこそ~だ 
\\	~にほかならない=まさに~だ 出生率の低下は、育児に対する国民の不安感の表れにほかならない
\\	~以外の何ものでもない=これこそ~だ グループ全員の意見を無視して事を進めるのは横暴以外の何ものでもない
\\	~ではあるまいし ~な訳でもあるまいし	
\\	子供ではあるまいし、こんな簡単な漢字もかけないなんて、恥ずかしいこどだ。
\\	大金持ちではあるまいし、世界一周クルーズなんて、とてもできませんよ。
\\	江戸時代ではあるまいし、毎日着物を着て生活している日本人など、滅多に居ない。 オリンピック選手じゃあるまいし、100メートルを9秒台で走れるはずがない。 バナナのたたき売りじゃあるまいし、あれも、これもただでやるといっちゃいかん!
\\	パニクするなよ。世の中が終わるわけじゃあるまいし
\\	~ではないのだから 
\\	~な訳でもあるまいし 
\\	子供じゃないんだから、こんな簡単な漢字も書けないんじゃ恥ずかしいよ。 
\\	そんなことをするなんてばかじゃあるまいか。
\\	~てやまない (やみません) 
\\	祈って;期待して;願って;求めて;
\\	やむことがない 
\\	やむことなく 
\\	~したいのがやまやまですが、	
\\	一体いつ地球上から戦争がなくなるのだろうか。平和を祈ってやまない。 新事業のスタートにあたり、社員諸君の一層の活躍を期待してやみません。 花嫁、花婿、お二人の末長いお幸せを祈ってやみません。どうぞ素晴らしい家庭を築いてください。 やむことがない 
\\	大きな魚に狙われ、逃走の日々はやむことがないでしょう。 
\\	やむことなく 
\\	雪がやむことなく一週間降り続けた。 
\\	~したいのがやまやまですが、 そうしたいのはやまやまなのですが、そのための時間がないのです。  
\\	(私は)いつまでも~ている こころから~ている 
\\	と
\\	と(が)相まって (~とあいまって) 
\\	が
\\	と相まって 
\\	相反	
\\	持って生まれた才能が日頃の努力と相まって、彼のピアノ演奏のレベルはプロ並みに達しつつある
\\	素晴らしい料理とサービスの良さとが相まって、このレストランの評判は上がる一方だ。 不況と円高とが相まって、中小企業の経営が苦しくなっている	~と一緒になって 
\\	相反 【あいはん】 
\\	一致しない、食い違う その問題については、相反する二つの意見がある。ある人は「テロをなくすためには攻撃あるのみだ」と言い、ある人は「攻撃より話し合いをすべきだと言う
\\	~とあって 
\\	連休の最後の日とあって、遊園地には家族連れが多かった。 
\\	ドルトン先生から始めて数字を習うとあって、スティーブは早速良い子ぶり、 今やっているところを 先生に教えたり、小さな子供を諭すような口ぶりで、捻った問題をあれこれ説明し出したけれど 、ドルトロン先生は ちっとも気にしなかった。 人気ナンバーワンの歌手のコンサートとあって、会場には溢れんばかりの客が詰まった。
\\	町のチームが全国大会で初優勝
\\	したとあって、地元の人たちの喜びは大きかった 
\\	この新製品は撮影後わずか1分で写真を見ることができるとあって、大きな注目を集めました。 
\\	スポーツ界の2大スターの結婚式とあって、豪華絢爛になりそうです 
\\	~だから ~と言う状況だから 社会的、客観的なことに使う。個人的なことには余り用いられない 
\\	連休の最後の日だったんで、家族連れが多かった
\\	~とあれば 
\\	ため)	
\\	ため) 
\\	ぜひにとあれば仕方がない。 
\\	「子供の幸せのためとあればどんな苦労をしてもいい」と言う親が多 
\\	ワールドカップ出場のためとあれば、どんなに厳しい練習も苦にはならない
\\	上司の命令とあれば、長期出張も海外出張もせざるを得ない 
\\	~なら 「~ためとあれば」の形が多い 「なら、なんでもする」という意味の文が多い 
\\	子供の幸せのためなら、どんな苦労をしてもいい、という親が多いです。 上司の命令なら、。。。
\\	~といい~といい	
\\	色といいデザインといい、このデザイナーの服は本当に素敵ですね。 味といい香りといい、今年のワインは最高の出来だ。 
\\	君は、学業成績といい授業態度といい、どうしようもない学生だ。退学処分にせざるを得ない。 知能と言い美貌と言い、彼女は本当に素敵ですね	~も~も 
\\	といいBtoii,C]
\\	について言うと、Cだ」と評価する。 
\\	~といえば~けれど=~といえるが、しかし 受験するわが子のために、早い受験番号を取ろうとする親で、毎年校門の前には長い列ができる。ばかばかしいといえばばかばかしいけど、これも子供を思う親の気持ちの表れだろう。 ~といわず~といわず 
\\	手といわず足といわず、引っかき傷だらけです。
\\	~というところだ・ ~といったところだ 
\\	いうところの	
\\	単純な仕事だから、時給は800円から900円というところです。 
\\	彼の成績はクラスで上の中といったところだ。国立大学を狙うなら、もう少し頑張らなければならない。 
\\	新しい法律に対する人々の反応は、「賛成6・反対4」といったところだ 
\\	だいたい~ぐらいだ 
\\	言うところの=いわゆる 彼は、言うところの「おたく」で、ほとんど外出せずに部屋でゲームばかりしてるそうだ。 兄は、いうところの「ばついち」で、また独身にもどっているんです。
\\	~といえども
\\	(と雖も)	
\\	医者といえども、時に病気になることもある いかなる名人といえども、時には失敗することもある 
\\	東京に住んでいるといえども知らない都内の名所は沢山ある 
\\	最近の子供はしっかりしている。幼稚園児といえども、自分のしたいことをはっきり言う 
\\	暫く試合から離れていたと雖も、やはり元チャンピオンだ。観客は彼の勝利に惜しみない
\\	拍手を送った。 裁判官といえども、神様ではなく人間なのだから、常に完全な判断ができると言うわけではない 
\\	同時通訳は一瞬といえども気が抜けない 
\\	当たらずといえども遠からず 
\\	いかに日本での生活が長いといえども、言葉の問題には常に遭遇する 
\\	たとえ 
\\	いかなる名人でも時には失敗することもある. 
\\	難しい仕事でも、します;ここからでも富士山が見える;この図書館は誰でも入れる.
\\	とは言うものの 【とはいうものの】 
\\	っていっても  ったって ・ っても 幼稚園児って言っても 裁判官ったって
\\	~といったらない ・ ~といったらありはしない ・ ~といったらありゃしない ・ ~ったらありゃしない
\\	あんなにおろおろしちゃって恥ずかしいったらありゃしない。馬鹿みたいだったわよ。
\\	いたずらばかりして親を困らせるけど、こうしてぐっすり寝ている子供の顔の可愛さといったらない 
\\	水平線から昇る日の出の美しさといったらない 
\\	近所の工場の騒音のひどさと言ったらない。家に居られないくらいだ。 
\\	その花嫁さんのドレスの豪華さといったらなかった。噂だは数百万円もするドレスだそうだ 
\\	たった150円、計算が合わないからと、残業させられた。馬鹿馬鹿しいといったらありゃしなかった。 
\\	あたふたしちゃって、みっともないったらありゃしない。
\\	そのときの悔しさといったらなかったと伊東さんは言う 
\\	本当に~だ 
\\	ありゃしない; ありはしない 
\\	~たらない=とても~ 私が病気になったときの、母の心配たらなかった。(とても心配した) 最近の肥り方ったらないね 
\\	~と思いきや	
\\	静かだから、勉強していると思いきや、息子はゲームに夢中になっていた。 今回の試合は相手が新人だから簡単に勝てると思いきや、最後まで苦戦を強いられた。 彼女はまだ10代に見えたので、独身だと思いきや、実は子供が二人いるそうだ。	と思ったのだが、そうではなく 
\\	とばっかり思ってたら ・ とばっかり思ったら ・ てっきり。。。。と思ったら。。。
\\	~ときたら 
\\	~時に限って)	
\\	課長ときたら、こっちが忙しいときに限って
\\	面倒な仕事をやらせるんだ。 あいつときたら、酒を飲みだすと止まらないんだ。一緒に飲みに行ったら大変だよ あの店ときたら、客に挨拶もしないんだもん。あれじゃあ、其のうち潰れるわ、きっと。	
\\	のことになると 
\\	に関しては 
\\	については 政治のことになると、君は、論理とか理性とかおかまいなしになるようだね。 
\\	(=に関しては,については) 
\\	に関して ~ったら 
\\	あなたったら、そんなことも知らないの。もっとちゃんと勉強しなさい このコピー機ったら、故障ばかりしてて駄目だ
\\	~ところを 
\\	今日は、お忙しいところをお越しくださいまして
\\	、ありがとうございます。 
\\	お休み中のところをお邪魔してすみません。どうしてもご報告しなければならないことがありまして 
\\	ご多忙のところを恐縮ですが
\\	、一寸ご相談したいことがありまして 
\\	~ときなのに  ~というじょうきょうなのに 
\\	~ところに ~ところで 電話をかけようと思っていたところに、彼のひょうからかかってきた 物語は、主人公の一家が苦難の末に国を脱出するところで終わります
\\	~としたところで ・ ~としたって ・ ~にしたところで ・ ~にしたって 
\\	彼女は私の提案に反対しているが、彼女にしたところで、良い案があるわけではない 
\\	暴力を振るう生徒を一体どのように導けばよいのか、教師としたところで困り抜いている
\\	のが現状だ
\\	大企業にしたって、この不況の影響を受けないわけがないだろう 
\\	芸能人と言うのは浮き草
\\	のようなものだ。人気スターにしたって、いつ人気が落ち目になる
\\	か分からない	~も  ~でも  ~の場合も 
\\	に於いては ~にしても 彼は提案に反対しているが、彼にしても名案
\\	がある訳ではないんです 人気スターにしても、いつ人気がなくなるか分からないものだ 
\\	~にすれば、~にしたら ~にしてみれば
\\	~とは	
\\	しんじられない;おもいもよらない;
\\	あんなに優しそうな人が逃亡中の殺人犯だったとは、信じられない。 
\\	普段は大人しい彼があんなに興奮するとは、思いもよらなかった
\\	まだ18歳の若さで市場試験に合格するとは、すごい秀才だ 
\\	携帯電話でテレビ放送まで見られるとは、技術の進歩は大したものだ 
\\	~ということは ~なんて 
\\	とは 
\\	思わなかった 
\\	~とはいえ ~とはいうものの	
\\	不況とはいえ、急成長している企業もあるのだ 
\\	少ないとはいえ、定収入があるのは有難いことだ 
\\	3月に入り、昼間はだいぶ暖かくなってきたとはいえ、まだまだ朝と晩は冷え込む 医者の話によると、たった20分歩くだけでかなりの運動効果があるそうだ。とは言え、毎日休まずに続けるのは簡単ではないだろう 
\\	魔が差したとはいえ、してはならないことをしてしまった。
\\	(「魔が差す」
\\	~とはいっても、しかし 
\\	とはいえ 
\\	といえども 
\\	とはいえ 
\\	といえども, 
\\	といえども 
\\	とはいえ 
\\	とはいえ. 
\\	新入社員とはいえ、手際がいい 
\\	新入社員といえども、手際がいい 
\\	たとえ, いかに, どんなに 
\\	いくら. 
\\	といえども	~とはいうものの 
\\	絶対に世界旅行をするぞ。とは言うものの、先立つものがないのだ
\\	もうそろそろ結婚したいな。とはいうものの、相手がいない しかしながら 
\\	彼は元気で、やる気があって、なかなかいい社員だ。しかしながら、少々行儀が悪いのが玉に傷だ
\\	"""~""とばかりに ~てばかりいる ~てばかりはいられない ・ ~てばかりもいられない
\\	友人と居酒屋で飲んでいたら、「もう店を閉める。早く帰れ」とばかりに店員が掃除を始めた。 内の犬は散歩が大好きだ。夕方になると、「散歩に行こうよ」とばかりに私の顔を見て吠える 私が「お孫さんは元気?」と言うと、彼女はバッグから写真を出して、「可愛いでしょう」とばかりに私の前に置いた。	
\\	ばかりに 
\\	まるで~と言うように 
\\	といわんばかりに	~てばかりいる=ずっと~している 先生は朝から怒ってばかりいる。今日は機嫌が悪いらしい ~てばかりはいられない ・ ~てばかりもいられない 
\\	試験に失敗したからといって、落ち込んでばかりもいられない。次のチャンスを目指して、また新しい気持ちで頑張ろう ~ばかりは=~だけは ほかの場合と違って~は ~んばかり
\\	ともなく ・ ~ともなしに 
\\	1ともなく
\\	からともなく	
\\	休日は、一日中何をするともなく、ぶらぶらしていることが多い。 朝、起きると直ぐテレビを付けるが、見ようと思って見るわけではない。天気予報などを、聞くともなく聞いているだけだ。 喫茶店で、コーヒーを飲みながら見るともなしに外を見ると、兄が綺麗な女性と歩いているのが目に入った どこへ行くともなくぶらつく 
\\	特に~しようと思うのではなく なんとなく、無意識に 
\\	だれ・どこ・いつ+からともなく 打ち合わせが終わった後、誰からともなく「飲みに行こう」ということになった。 どこからともなく甘い香りが流れてくる。バラの花でもあるのかなと思い、周りを見回した
\\	ともなると ・ 
\\	ともなれば	
\\	ここは人口わずかに先人の小さな村だが、夏祭りともなると、日本全国から観光客が押し寄せるそうだ 
\\	この辺は静かなところだが、夜ともなると、バイクで若者が集まってくるので、煩くてかなわない. 
\\	学生の間はシャツとジーンズでもいいが社会人ともなれば、スーツ一着ぐらいは持っていないと困ることがある 
\\	先人 せんじん 
\\	~となると=~場合は 他人の作ったものを批評するのは簡単だが、いざ自分で作るとなると、なかなか難しい。 三人の子供を私立大学に行かせるとなると、親は大変な負担を強いられる 
\\	ないまでも 
\\	までも	
\\	彼女は、天才とは言えないまでも、相当優秀な人だと評判です 
\\	三年勉強すれば、日本人のようには話せないまでも、意志を伝えることぐらいはできます 
\\	こんな生活を続けていれば、死なないまでも、病気になってしまいますよ 
\\	彼は、ずるいとは言えないまでも、あまり信用できる人ではないので、なるべく付き合いたくないんだ 
\\	とは言わないけど、 
\\	ほどじゃないけど
\\	~ものでもない {読めないものでもない(ですが..)} 
\\	~もので(も/は)ない {読めるものではない}	
\\	頼まれれば、
\\	引き受けないものでもないんですが、出来ればやりたくないんです 
\\	大変な仕事ですが、皆で力を合わせて一生懸命やれば、出来ないものでもないから、やって見ましょうね 
\\	通訳ですか。やれないものでもないんですが、まだ力不足だなあ。一寸不安ですね 
\\	始めから「だめだ」と諦めずに、社長に頼んでみたらどうですか?「うん」と言わないものでもないと思いますが。 
\\	~ものではない 
\\	~というものではない 
\\	読めるものではない 
\\	戦争は軽々しくするものではないよ 
\\	夜のこんな時間に電話をかけるものではない。 
\\	哲学は6ヶ月で学べるものではない。 
\\	~とは言い切れない
\\	~なくはない ~ないこともない	
\\	断れなくはないけどね。断ると後が怖いから、いきますよ。
\\	~ながらに
\\	(乍らに) 
\\	~ながらの
\\	彼は、涙ながらに苦しかった戦争の思い出を語った。
\\	飛行機事故で生き残った人が涙ながらにテレビのインタビューに答えている
\\	優勝した選手は、涙ながらに喜びを語った
\\	彼は生まれながらの悪人だといわざるを得ない
\\	京都では昔ながらの街並みを保存する運動が盛んです
\\	涙ながらに=泣いている状態で 生まれながら「昔ながら」=そのままのかわらない状態	及ばずながら 
\\	及ばずながら尽力
\\	いたつもりです 
\\	陰ながら 
\\	陰ながらご成功を祈っています 
\\	かげながら案じておりました 
\\	うれし涙 
\\	悔し涙 
\\	すずめの涙 
\\	血も涙もない 
\\	鬼の目にも涙 
\\	~ながらも (乍も)	
\\	女ながらも彼女は勇敢だった 
\\	この家は狭いながらも庭もあるし、駅にも近いので、満足しています 
\\	この車はデザインが古いながらも、性能がいい
\\	ので、人気があります 
\\	彼はぶつぶついいながらも、結構楽しそうに仕事をしていた 
\\	彼は、貧しいながらも温かい家族で育った 
\\	~が ~しかし ~けれど	
\\	~なくして ・ ~なくしては 
\\	彼の活躍なくしてチームの優勝はあり得なかった。さすがキャプテンだ 愛なくして、結婚生活は続けられない 日々の努力なくして、成功は望めない。諦めずにこつこつ努力することだ 真の勇気なくしては正義を貫くことはできない。 「君なくしては今日のわが社はない」と社長に言われていたのに、突然リストラされてしまった。	~ない場合は ~がなくては ~がなければ 
\\	~なくちゃ、~なしには
\\	~なしに ・ ~なしには 
\\	無きにしも非ず	
\\	関係者以外は、許可なしにこの建物に入ることは出来ません. 
\\	あなたの意見に文句なしに賛成と言うわけではありません。でも、一応は賛成です。 
\\	どんなに有能な人でも、家族の協力なしに仕事をすることは出来ない。 
\\	専門的知識があるあなたの協力なしには、この仕事の成功は考えられない。 
\\	~がないと ~がなければ 
\\	~を措いて 無きにしも非ず=ないこともない=少しある 
\\	野党が次回の選挙で勝つ可能性はなきにしもあらずだと言われている。 
\\	~ならでは ・ ~ならではの	
\\	こんな絵が描けるのは子供ならではですね。大人にはとてもこんな絵は描けませんね 彼女は母親ならではの細かい心遣いで子供のお弁当を作っている 秋ならではの、澄んだ空気のもとで、運動会が開かれた。 こんな斬新な発想は若者ならではのものです 
\\	「ハイビジョンならではの美しい画面をお楽しみください」とNHKが宣伝している。 それは、彼ならではの発言だ。 
\\	=~でなくてはできない ~でなくては(~し)ない ~以外にはない 
\\	あっての ~がなくては 
\\	彼女は、母親にしかできない細かい。。。 せっかく外国生活をするんだから、その国にしかない文化に触れたいですね。
\\	なり 
\\	なんとかマラソンコースを走りぬいた彼女だが、ゴールインするなり倒れこんでしまった。大丈夫だろうか。 
\\	よほど疲れていたのだろう。夫は帰ってくるなり食事もせずに寝てしまった 
\\	彼は電車のドアが開くなり飛び出し、改札口に向かって走りだした。遅刻しそうなのだろう。 
\\	弓なり 
\\	山なりに 
\\	道なりに曲がる 
\\	大の字なりに 
\\	卵なりの 
\\	それなりに 
\\	自分なりに 
\\	上司の言うなりになる 
\\	=いずれでも 
\\	大なり小なり 
\\	~なりと(も) 
\\	親なり友人なりに 
\\	=したまま  部屋に閉じこもったなり 
\\	海外に行ったなり 
\\	申込書に記入したなり 
\\	なり, 
\\	なり 
\\	するなりしないなり 
\\	何か困った事があったら先生なり、お母さんなりに相談しなさい 
\\	人間には、誰にでも大なり小なりの欠点があるものだ。 
\\	スキーツアー、どうしますか。参加するなり、しないなり、はっきり決めてください。旅館の予約をしなければなりませんから 
\\	ストレスが溜まった時は、カラオケで歌うなり、旅行をするなり、気分転換をするといいですよ 
\\	~とか~とか 
\\	したりBしたりする 
\\	(私;自分;彼;など)~なりに	
\\	常識に反するかもしれませんが、これは私なりに考えて出した結論です。 彼は彼なりに頑張っているのだから、見守っていてあげよう。 子供は幼いなりによく自分の周りの人間関係を理解している 彼女は彼女なりにとてもきれいだ。
\\	あんな車でも、それなりに役に立っているんだ。
\\	~にあったように ~にふさわしく 
\\	なり 
\\	これと言う~(は/が/も)ない ・ これといって~ない ・ これといった~ない	
\\	これといった特徴のない 
\\	彼にはこれといった道楽(興味)もない 
\\	これといってほしいものはない。 
\\	これといった解決策がなくて、手探り状態です。
\\	だれに投票するか決めた?どの人もこれといって特徴がないのよね。
\\	海外でクリスマスと言えばターキーだけど、日本では特にこれという食べ物はないな。
\\	いまだに~ない	
\\	この雑誌の売れ行きはいまだに衰えていない 
\\	彼は30にもなるのに、いまだにガールフレンドの一人もいないんです。
\\	一概に~
\\	一概にそうとばかりは言えない 
\\	本に書いてあることを一概に信じ込むのは危険だ 
\\	そうですね。一概には言えませんね。 
\\	一概に~と決め付ける 
\\	あながち
\\	一向に~ない	
\\	そのことは一向に存じません 
\\	風邪を引いてもう一週間になる。薬を飲んでいるが、一向に治らない。 議論が空回りして一向に発展しなかった。
\\	他人が私のことをどう思おうと、私はいっこうに気にしない	ぜんぜん~ない いつまでたっても~ない	
\\	ろくに~ない	
\\	ろくに見もしないで買ってしまった 
\\	昨夜はろくに眠れなかった 
\\	彼女はフランス語は言うに及ばず母国語すらろくに話せない 
\\	彼女はろくに微笑みをみせず/かけず /返さず 答えた 
\\	言うに及ばず 【いうにおよばず】 
\\	一二に及ばず 【いちににおよばず】 
\\	是非に及ばず 【ぜひにおよばず】 
\\	及ばず乍ら 【およばずながら】 
\\	及ぶ 
\\	及ばない) 
\\	及び 
\\	且	
\\	~にあたらない ・ ~にはあたらない 
\\	彼は事故を起こして悩んでいるが、全く同情するにあたらない。飲酒運転したんだから、自業自得だ 政治家の汚職
\\	なんて今さら驚くにあたらない。起こるのは政治が腐敗している証拠だ。 あの監督の最新の作品は賞賛するには当たらない。平凡な恋愛映画だ	~(する)ことはない ~(する)必要はない 
\\	驚くことじゃないよ 同情することなんかないですよ ~にあたる=~と同じだ 「女のくせに」と言う発言はセクハラにあたるのだそうです。
\\	~にあって(は/も) 
\\	期、時、中、状況、など) 
\\	いかなる状況にあっても穏やかでいる 
\\	現在A国は混乱期にあって、国民は苦戦を強いられている。 
\\	国家の非常時にあっていかに行動するべきか考えるのが、国民の義務です 
\\	彼女は苦しみの中にあって、なお自分を見失うことなく生きている 
\\	多忙な現代にあって温泉療法の重要性はますます増やしている 
\\	弱小球団にあって彼一人はなかなかの活躍であった。 
\\	この川は古代にあっては重要な交易(こうえき 
\\	路だった 
\\	~において ~の状況で 
\\	~にあり=~にあって
\\	~に至る ・ から~に至るまで	
\\	今に至るまで何の連絡もない 
\\	彼が身につけているものは眼鏡から靴紐にいたるまで妻の好みに合うように選び抜かれている
\\	努力に努力を重ねて、ついに新薬を開発するに至った。 
\\	この病気は死に至る病として人々に恐れられている。 
\\	自営業から大企業に至るまで、すべての産業が不況に陥った 
\\	原子力発電所で火災が起きた。一時はどうなるかと心配したが、大事に至らず
\\	、担当者はほっと胸をなでおろした (撫で下ろす 【なでおろす】 
\\	~まで行く  ~まで及ぶ 
\\	至らない 
\\	至らないところ 
\\	~に至って ・ ~に至っては ・ ~に至っても	
\\	首脳会議の最終日に至って、やっと結論が出た
\\	先進国首脳会議(サミット)の最終日に至っても、なかなか結論が出なかった。 
\\	わが社は赤字が100億円まで膨らんでしまった。事ここに至っては、もう手の打ちようがない、遠からずわが社は倒産るすだろう。 
\\	はっきりした証拠が、裁判所に提出された。彼の罪は明白だ。しかし、この場に至っても、彼はまだ真実を話そうとはしない。 
\\	その段階・状況になって・なっても ~は、極端な状態	に至るまで いたって=きわめて、とても
\\	~にかかわる (こと/問題/など)  
\\	係わり合いになる)	
\\	会社の信用にかかわることは絶対にしてはいけないと、社長に言われています 
\\	これは人命にかかわる問題ですから、議論を尽くさなければなりません 
\\	これは我々の名誉にかかわる問題だから、黙っているわけには行かない。裁判に訴えようと思っている 
\\	年金の問題は生活にかかわることだけに、国民の関心は高い 
\\	~に関係する  ~影響がある	係わり合いになる=関係をもつ あんな悪人とは、係わり合いにならないほうがいいですよ。 ~にかかわらず=関係なく 賃金は、男女にかかわらず平等に支払われるべきです。
\\	~にかたくない 
\\	原子力発電所でまた事故があった。周辺住民の不安は想像にかたくない。 看護師の仕事がいかに重労働で苦労が多いか、想像にかたくない。 市民の力で暴力を追放した喜びは、想像するにかたくない。 子供を殺された両親の無念さは、察するにかたくない。 市民が協力して町から暴力を追放した喜びは、察するにかたくない。	容易に~できる 簡単に~できる 
\\	~がたい=難しい 
\\	去年の新嘉坡への旅行は忘れがたい思い出となった。
\\	~にして 
\\	はじめて) 
\\	(可能/できる/など)	
\\	彼はノーベル賞に値する研究を成し遂げた。これは、偉大な科学者にして初めてできることであろう。 
\\	超音速ジェット機は経験豊かな人にして初めて操縦できる飛行機で、とても新人には操縦できない。 
\\	彼女は、ハワイに別荘を買ったそうだ。羨ましい話だが、これは彼女のような資産家(富豪)にして初めて可能なことだ 
\\	~だけ ~だからこそ 
\\	言語は、文字と音と意味が合致して初めて成り立つものだ。
\\	女房の尻にしかれている亭主は、幸せな独り者に話してはじめて、自分の問題を認識することがたびたびだ。 
\\	この研究は、あの偉大な科学者だからできたことで、普通の人にはできませんよ。 超音速ジェット機はベテランだけに操縦できる飛行機で新米にはとても無理ですよ ~にしては=~なのに 今日は、真冬にしては暖かいですね。春のようです。 
\\	過度の恐怖や心労は一夜にして白髪をつくる 
\\	彼女は45歳にして初めて子共ができた 
\\	彼女は頭脳明晰にして端麗な秘書です 
\\	首相の容態は不安定にして、予断を許さない 
\\	~にして 
\\	推して知るべしだ	
\\	あの慎重な彼にしてこのような失敗をするのだから、私たちが失敗するのは当然だ。 
\\	首相にしてこの程度の理解だから、他の議員は推して知るべしだ 
\\	この高級車は中古にしてこの値段だ。新車なら、3000万ぐらいするそうだ。 
\\	12月にしてすでに積雪
\\	が1メートルを越した。これでは最も寒い2月が思いやられる 
\\	先生があの程度なら、習っている生徒の力は推して知るべしだ 
\\	島には電気も水道もない、その生活は推して知るべし  
\\	~でも ~なのに ~さえ 
\\	推す 
\\	(推測する) 
\\	(推薦する) 彼女の服装から推してから... 
\\	私の推した作品が... 
\\	として(も)) 
\\	政治家にしては若い方だ。) 冬にしては暖かい日だった 
\\	新人にしてはずいぶんよくやった 
\\	彼女は10年も日本に住んでいるにしては日本語がへただ 
\\	山へ行くにしては軽装じゃないかな 
\\	交渉はもとまるにしても時間が掛かるだろう 
\\	彼女は演出家にして俳優である 
\\	この父にして、この子あり 
\\	~に即して ・ ~に即しては ・ ~に即しても ・ ~に即した	
\\	国の作成したカリキュラム(履修課程)は古い、時代に即した新しいカリキュラムを作成しなければならない 政府はダム建設の現地を視察
\\	して、実情に即した政策を立てるべきだ。 
\\	歌舞伎の物語は、江戸時代の社会状況に即して書かれている 
\\	憲法改正については、国際情勢
\\	に即した憲法にするべきか、国民の考えに即したものにするべきかが議論されている	~の通り ~に沿った ~に合う ~は 事実、事情 などが多い 
\\	に則して 
\\	基準に従って 犯罪は法律に則して処罰される。
\\	~にたえる(もの) ~にたえない(もの)	
\\	これは素人が書いた小説だが、十分読むにたえる作品だ。 バッハの作品は万人の鑑賞にたえる名曲ばかりだ。 国会の論議は聞くにたえないものが多い。国民が聞くにたえるような内容の濃い論議をしてほしいものだ。	(の)価値がある  ~ができる 大丈夫だ。~できる」という意味がある 
\\	に負けない 持ちこたえる
\\	彼は多くの厳しい批判にたえて、自分の意思を最後まで変えなかった。
\\	~にたえない	
\\	社長は大のカラオケ好きだが、その歌ときたら聞くに堪えない。 なにしろ酷い音痴なのだ (何しろ) 
\\	これは芸術作品どころか、見るに堪えないほどの出来だ この安物のアクセサリは人前でつけるにたえないものですね。
\\	読むにたえない週刊誌が飛ぶように売れているとは、日本の将来が心配だ 
\\	我慢できない ~には、見る、聞く、読むなどの感覚・知識の動詞が入ることが多い 
\\	・・・するにたえない 
\\	それは同情にたえない 感にたえない 
\\	~に足る 
\\	足ることを知る 
\\	取るに足りない	
\\	賞賛するにたる 
\\	6万を収容
\\	するに足る施設 
\\	多くの画期的な発明をした彼は、尊敬に足る人物である。 
\\	「この作品は完璧とは言えないまでも誇るにたる出来だ」と、彼は自画自賛している
\\	信頼するに足る人物だと思っていたその人にだまされてしまった 
\\	一見するに足る  
\\	ほとんど驚くに足らず 
\\	取るに足りない(やつ・損失・もの・言説・駄作・連中・動機・存在・質問) 
\\	画期的
\\	劃期的 【かっきてき】 
\\	足ることを知る 
\\	用がすむ 
\\	たりる 
\\	満足する 
\\	~にたえる 
\\	のに 
\\	に]ひきかえ(て)	
\\	反面	公立の学校に引き換え、私立の学校は学費が高すぎる 
\\	寒かった去年にひきかえ、今年は暖かいですね 
\\	社が環境保護を優先させているのに引き換え、B社は利益しか追求しない 
\\	金融業界
\\	が伸び悩んでいるのに引き換え、製造業は売り上げを伸ばしている。 
\\	私が引っ込み思案なのにひきかえ、妹は社交的だ。
\\	彼女の優秀なのに引き換えて弟はさっぱりだ。 
\\	引き換えて 
\\	商品券を商品と引き換える 
\\	それに引き換え 
\\	~反面 去年に比べると今年は寒いね 私立の学校と違って、公立は学費が安い。 ~対照してみると  ・  ~の対照が
\\	にもまして(増して) 
\\	のにもまして(増して)	
\\	よりもずっと
\\	のほうが。。。	わが社の業績は去年にもまして悪化してきた。 
\\	世界はこれからどう変わるのだろうか。自分のことにも増して子供達の将来が心配だ。 
\\	この国では電気も水も不足している。それにもまして不足しているのは医薬品だ 
\\	其の花は色が美しいのにもまして、香りが素晴らしい。 
\\	~以上 
\\	まして=だから~の場合はもっと~だ 入院するのは、大人でさえも心細いのだから、まして、子供はどんなに心細いことだろう。 
\\	~の至り	
\\	社長のお宅のパーティーに招待していただき、光栄の至りでございます。 
\\	結婚式のスピーチで新婦の名前を間違えてしまうなんて、赤面の至りだ。
\\	20代には若気の至りで悪いことをいろいろして両親に迷惑をかけてしまった	とても、最高の、~の結果 
\\	光栄 
\\	若気 
\\	若気の至り 
\\	で いたって=とても 
\\	家族一同、お蔭様でいたって元気にしております 息子のかげはいたって順調に回復していて、私もほっとしています
\\	~の極み (である)	
\\	30年前に別かれた母親との再会は感激の極みであった 
\\	手術ミスで患者を死なせてしまったという今回の不祥事
\\	は、遺憾
\\	の極みであります 
\\	父の顔色が良くない。先月からの残業続きで疲労の極みに達しているようだ。 あのテレビ番組は低俗
\\	の極みだ。テレビ局には良識(りょうしき 
\\	ある人がいないのか 
\\	一番~  最高に~	
\\	この上ない 
\\	食事にセールスの電話がかかってくるが、迷惑この上ない   面倒なことこの上ない;無責任この上ない;卑劣この上ない;この上ない解放感に満たされている;この上ないお褒めの言葉をいただいた 
\\	うるさいことこの上ない 
\\	~極まり ~極まりない
\\	非常に、実に、極めて 口を極めてののしる=相手を強く非難する 
\\	~の骨頂だ
\\	~はおろか 
\\	男性はおろか女性もアルコール依存症になる
\\	父の会社が倒産して、私たちは会社のビルはおろか自宅までも人に取られてしまった 
\\	彼女の名前は、日本はおろか世界中に知れ渡っている。何しろあの世界的ビッグスター[名優/花形役者]と結婚したのだから。
\\	ジョンは簡単な漢字はおろか、ひらがなも正確に読めません。 
\\	もう直ぐ海外旅行に行くというのに切符の手配はおろか、パスポート[旅券]も用意していない 
\\	アメリカでは、隣町まで100マイル、途中ガソリンスタンドはおろか、民家さえないという場所がたくさんある 英語を話すのがすきなだけで、英米の文学はおろか日本の文学もろくに読んでいない学生が英米文学専門に入ってくると悲惨だ	~はまちろん ~どころか ~は言うまでもなく
\\	はいうのもおろかだ 
\\	1はおろか
\\	2(も/さえ) 
\\	言うに及ばず=
\\	どころか 最近は男性だけじゃなくて、女性もアルコール依存症になるんだね。
\\	うちの息子はガールフレンドどころか女の子と話すこともできないんですよ。 いうまでもなく はもちろん
\\	~ばこそ 
\\	(のだ)	
\\	両親の助けがあればこそ、安心して学生生活が送れるのです。 
\\	あなたのためを思えばこそ、厳しくしているのです 
\\	「雨が嫌い」なんて言ってはいけません。雨が降ればこそ作物も実るのですから。 
\\	「将来は大学に残って教授になるか、弁護士になるか迷っている」なんて、頭がよければこその悩みですね 
\\	100円ショップのように、不景気ならばこそ売り上げが伸びる商売もある。 
\\	人は一人でいるのが寂しければこそペットを飼うのだ 
\\	言論が自由であればこそ民主主義は育っていくのでる 
\\	~からこそ 理由を強調する表現 
\\	英語が上手に(話せないからこそ/
\\	はなせなければこそ)人より余計に練習しなければならない	
\\	こそすれ 
\\	僕がどんなにいい仕事をしても、同僚は妬みこそすれ、ほめることはない。(妬むだけで、ほめない) 彼は朝から晩まで動きこそすれ、怠けることは全くなかった。それなのに、昇進しないのはなにかおかしい。(よく働いて、怠けなかった)
\\	(どんなに
\\	をしても)、
\\	ばそれまでだ 
\\	丈夫な鍵を作っても、鍵を掛け忘れればそれまでだ どんなに良く勉強しても、試験開始時間に遅れてしまえばそれまでだ。それまでの努力は水の泡になる
\\	=骨折り損 (ほねおりぞん 
\\	いくらお金を貯めても、使わずに死んでしまえばそれまでだ。 帰国したくないのだが、ビザが延長されなければそれまでだ。 私のほうで望んでも、彼女が「イエス」と言わなければそれまでだ。この結婚は諦めるしかない。	~ば、それで終わりだ 他の条件は整っていても、~ば、もうだめだ	~たらそれまでだ ~ならそれまでだ=~ばそれまでだ 無駄骨を折る 
\\	ひとり~だけでなく ・ ひとり~のみならず	
\\	以前は春闘での電力業界のベースアップの幅が、ひとり電力業界だけではなく、日本中の企業の賃金に影響したものだ。
\\	産学協同研究体制が整い、ひとり大学だけでなく産業界も優秀な研究者を求めている。
\\	国の核の問題は、ひとりアジアのみならず、世界の安全も脅かす。
\\	ワインを保管する際は、ひとり温度のみならず、湿度にも気を配らねばならない(=配らなければならない)。
\\	単に~だけでなく 
\\	~ばかりじゃなくて、Bも... ~に限らず 
\\	これは、企業に限らず映画や翻訳の世界でも増えている。
\\	~べからず	
\\	「ごみを捨てるべからず」「煙草のポイ捨てをするべからず」という看板か駅前に立っている。 「危険。関係者以外は許可なく立ち入るべからず」 「人から受けた恩を忘れるべからず」と祖母はいつも言っていた。 盗作をす(る)べからず 
\\	盗作 
\\	倒錯 
\\	[性的
\\	電車の中で携帯電話を使うべからず	~してはいけない 「禁止」を表す古き言い方 
\\	児童虐待などは許すべからざる/べきではない行為だ	するな;を禁ず(る);禁止;とはいけない;ないでください;ないようにしてください. 「べき」の形の変化 否定形 べきではない べからず=してはいけない 命令形 べし =しなければいけない 福祉の計 べく(~する) 
\\	するために(目的)
\\	~べからざる  
\\	行為、もの、など)	
\\	動機が何であれ、殺人は許すべからざる行為である 
\\	原子力発電所が爆発炎上
\\	した。事故の原因は、犯すべからざる初歩的なミスであった。 人の命を救う医者として、医療ミスはなす
\\	べからざる行為である 許すべからざる戦争のために、多くの人々の命が失われていく。 生きていく上で欠くべからざるものは「愛」である。	~ことができない~  ~てはいけない~ 「べからざる+名詞」の形で用いられる。 
\\	~べく 
\\	来年イギリスに留学をすべく、現在準備をしている。 
\\	営業不振
\\	の支社を立て直すべく、有能な部長が社外から迎え入れられた
\\	彼女は、病気の父を見舞うべく、故郷に帰った。 
\\	戦地で傷ついた人を救うべく、国際赤十字が設立された。 
\\	当社はいち早くを客さまの多様なニーズに対応すべく、最新の設備を導入してまいりました 
\\	~ために ~(する)つもりで ~(し)ようと思って 
\\	は、目的の行為 
\\	留学しようと思って、今準備をしております。 立て直すために、有能な部長。。 見舞いをするために故郷へ帰りました。 ~んがため
\\	~まじき 
\\	行為/こと	
\\	無断で会議に欠席するなど、会社員としてあるまじき行為だ。 
\\	警察官が交通違反者から賄賂を受け取るなど、許すまじき事である
\\	駅のホームでたばこを吸うとは、高校生としてあるまじきことです。 学生にあるまじき行為だ 
\\	~までだ ・ ~までのことだ	
\\	「私達の要求に応じなければ、人質を殺すまでだ」とテロリストは言った。 「家庭より仕事が大切だと言うのなら、あなたとは離婚するまでです」と妻に言われてしまったので、今日は残業しないで家へ帰ろう。 「強盗を捕まえてくれて本当にありがとう」「警察官として当然のことをしたまでのことですから、お礼には及びません」 結婚できなかったら、一生独身で仕事をするまでのことだ。 バスが来ないんなら歩いて帰るまでだ 夕食にどうぞと言われたから彼女のアパートに言ったまでだ。とやかく言われる筋合いはない
\\	彼は日本語が面白そうだったので、取ったまで(のこと)だ 
\\	ここに名前と住所を書くだけのことだ
\\	~までもない ・ ~までもなく 
\\	わざわざ)	
\\	薬を飲むまでもありません。ただの風邪ですから、暖かくして寝ていてください 
\\	電話一本で済むことですから、わざわざ行くまでもありません 
\\	あれは、わざわざ映画館まで見に行くまでもないつまらない映画です。 
\\	調べるまでもなく、このダイヤが偽物であることは、誰の目にも明らかだ。 
\\	言うまでもなく、彼は日本 で有名な流行歌手だ 
\\	~ことはない ~する必要はない	
\\	いくことないよ 買う必要なんかないよ。 いかなくてもいい。
\\	~んがため ・ ~んがために ・ ~んがための 
\\	私は社長の信頼を得んがため、朝早くから深夜まで働き続けた。 
\\	社は競争相手をけ落とさんがため
\\	、様々な手段をとった。 
\\	勝たんがための練習しか考えない部活動を嫌う学生が多いそうだ。 
\\	生き残らんがために、動物は敵から身を守る方法を考える 
\\	雇用を守らんがために、労働組合は会社と交渉を続けている。 
\\	~たいために ~ようと思って 
\\	せんがため	
\\	得ようと思って 勝つための練習 生き残ろうとして 
\\	べく
\\	~んばかりだ ・ ~んばかりに ・ んばかりの	
\\	犬は噛みつかんばかりに僕に向かって吠えた 
\\	時折、木の枝も折らんばかりの強風が吹いていた 
\\	驚いて腰を抜かさんばかりだった 
\\	引っ越しの時あまり重いものを持ったので、腕が肩から抜けんばかりだった 
\\	子供を失った両親は、悲しみの余り気も狂わんばかりだった。 怒った彼は、さっさと帰れと叫ばんばかりにドアを開けた。
\\	地震で家が壊れんばかりに揺れ、おそろしくて生きた心地がしなかった. 
\\	周囲の人は、まるで私が犯人だと言わんばかりの態度を取った 
\\	赤ちゃんの零れんばかりの笑顔は、見る人を幸せにする。
\\	今にも~しそう ・ ~とばかりに 
\\	気が狂いそうに悲しんでいる 壊れそうなくらいに振れて 犯人だと言うような態度をとりました。
\\	~をよそに (余所)	
\\	彼は親の心配をよそに遊んでばかりいたから、今年も進級できないらしい
\\	ファンの期待をよそに、あのチームはすでに10連敗だ 
\\	(れんぱい) 
\\	国民の不満をよそに、消費税引き上げの法案が可決された。 
\\	市民の貧困をよそに、特権階級
\\	は贅沢な生活を続けていた。
\\	我がの不安をよそに、物価は上がる一方である。
\\	よそ見をする=ちゃんと前を見ないで 横を見る わき見をする 
\\	余所行きの服 
\\	よそよそしい 
\\	~にもかかわらず 経験がないにもかかわらず、彼は有罪判決を受けた ~割には
\\	~を余儀なくされる ・ ~を余儀なくさせる	
\\	彼女は家庭の事情で退職を余儀なくされた 
\\	電力不足で、工場は一時閉鎖を余儀なくされている。 
\\	火山噴火で土地や家を失った人々は、移転を余儀なくされた。 
\\	彼は高校生の時結核を患い
\\	、半年間の入院生活を余儀なくされた 
\\	大統領に方針の変更を余儀なくさせる 
\\	(人)に避難を余儀なくさせる 
\\	やむをえないので~する しかたがないので~する 余儀なく 【よぎなく】 
\\	せざるを得ない 
\\	その計画を中止せざるを得ない 
\\	強いられる 
\\	妥協を強いられる 
\\	むりやり~させる 
\\	独裁者が部族に対しその降伏条件に無理矢理同意させた。
\\	止むを得ない 
\\	彼女は少しあいまいな態度だったのはやむをえなかった。 
\\	我々はやむを得ない事情で、その会合を延期した 
\\	この若人に対してねたましく思う気持ちを、君はやむを得ないものと受け入れなければならない。 
\\	~を限りに 
\\	あらん限りの 
\\	その場限りの 
\\	(時間ことば)+をかけて	
\\	できるだけの 
\\	部長は今年7月を限りに退職なさるそうです。 
\\	この店は3月20日を限りに閉店させていただきます。
\\	みんなで声を限りに応援したのに、日本のサッカーチームは負けてしまった 
\\	いよいよ今週を限りに、しばらくのお休みです。 この堅苦しい国を脱出したい。 
\\	声を限りに 
\\	~を最後に ~の限界まで 
\\	あらん限りの=あるだけ全部の 
\\	手にあらん限りの力を込める 
\\	医者はあらん限りの手段を尽くして患者の命を救おうとした。 
\\	暴風雨があらん限りの猛威を振るった。 
\\	その場限り=そのときだけであとのことを考えない 
\\	その場限りのことを言う。 
\\	(時間ことば)+をかけて 
\\	~を皮切りに ・ ~を皮切りにして ・ ~を皮切りとして	
\\	"かわりに「ある程度」「多くの」「しばしば」という言葉を使い、「私の経験では」「間違っていたら申し訳ないのですが」「多くの例では」のような言葉を皮切りに話を始めなさい。 
\\	それを皮切りとして欧州の詩や文学を多数紹介するようになりました。 
\\	彼女は、ハリウッド映画に出演したのを皮切りに、次々と外国の映画に出演するようになった 
\\	彼のコンサートは東京を皮切りにして、全国10都市で行われる予定です。
\\	アメリカで学生によるデモが行われたのを皮切りとして、反戦デモは世界に広がった. 
\\	~から始まって 
\\	面(つら)の皮が厚い  
\\	欲の皮が突っ張っている 
\\	よくのかわがつっぱっている
\\	を禁じえない (禁じ得ない) 
\\	を禁じえないでいる 
\\	突然の解雇に、社員達は怒りを禁じえないでいる 
\\	戦争に巻き込まれていく子供達の姿を目の当たりにして
\\	、私は涙を禁じえなかった 
\\	苦しかった彼女の一生を考えると、同情を禁じえない。 
\\	子供たちが遊ぶ無邪気な様子に、微笑みを禁じえない。 
\\	~をおさえることができない, ~ずにいられない 
\\	怒り、同情、驚き、失望、涙、笑い	
\\	驚きの念を禁じえない 
\\	~に同情を禁じえない 
\\	~をもって 
\\	[をもちまして」	
\\	"1/商品の発送をもって当選者の発言にかえさせていただきます 
\\	合格、不合格は(合否)、後日書面
\\	をもって通知いたします 
\\	純粋な努力をもって 、初優勝した 
\\	「あなたの罪、死をもって償うべきです」と裁判官は犯人に死刑を宣告した (誅 
\\	この試合をもって、バファローズは55年の歴史に幕を下ろしました。 
\\	日本の営業は9時をもって閉店とさせていただきます 
\\	昭和64年をもって激動の昭和は幕を閉じ、平成がスタートした。 
\\	これをもちまして、第20回定期総会を終了いたします。 
\\	本日をもって 
\\	~で ・ ~によって 
\\	持つ 
\\	それをもって 
\\	を身をもって知る 
\\	を身をもって示す 
\\	思いやりをもって扱う 
\\	~を通じて
\\	~をものともせずに 
\\	ものの数ではない 
\\	もの寂しい・もの静か	
\\	激しい吹雪をものともせず、登山家たちは出発した 
\\	氷のような水をものともせずに彼は川へ飛びこんだ 
\\	そのお転婆嬢(おてんばむすめ)は危険をものともせずにその木に登った 
\\	一行は危険をものともせず前進した 
\\	寒さをものともせずに、彼は薄着で外出した 
\\	~に負けずに立ち向かって	ものの数ではない=たいしたことはない 
\\	ものの数秒で大使館は占拠されてしまった。
\\	もの寂しい・もの静か=雰囲気が寂しいとか 秋の夕暮れはなんとなくもの寂しい気分がします
\\	~を措いて (おいて) 
\\	ない 
\\	於いて=
\\	"あなたをおいて、この会社の社長の適任者
\\	はいない 
\\	大学3年の夏休み、この時期をおいて長期の海外旅行をすることは無理だろう。 
\\	古い家だが、ここをおいて、ゆっくり休めるところはない 
\\	「この本を書ける人は、あなたをおいていない」と煽てられて(おだてられて)書き始めたのですが 
\\	税(金)還付をおいて、テレビを買う余裕はないだろう 
\\	~を除いて  ~以外に	
\\	をおいてBはない=AがBに最適だ 
\\	をおいて
\\	~ゆえ ・ ~ゆえに ・ ~ゆえの	
\\	戦争中のことゆえ、子供達はまともな教育を受けることができなかった 
\\	家が貧しいゆえに学校へ行けない子供が世界にはたくさんいる 其の国は独立してまだ間もない
\\	。ゆえに、国の機構がまだ完全には整っていない
\\	彼女はその美貌(の)故(に)、悩みも多い 
\\	結果にこだわりすぎる(が)故(に)小さなことに心を奪われることになる 
\\	「我思う、ゆえに我あり
\\	といったのは、デカルトという哲学者です 彼は子供のときに盗みをしたことがあった。それは貧しさゆえの過ちだった
\\	地震のためにビルが壊れた. 
\\	そういう考え方しかできないのは、視野が狭いゆえである	ゆえない 
\\	ために 
\\	から 
\\	~まみれ	
\\	雨の中での試合となり、子供たちのユニフォームは泥まみれだった。 
\\	交通事故で病院に運ばれた(運用された)男のシャツは血まみれだった 
\\	年末は、ほこりまみれになって、家族全員で大掃除をした。 
\\	その子は、食事中にソースまみれの手をテーブルクロスで拭いて、お母さんに大目玉に食らった
\\	間違いだらけ	~にまみれて=~まみれになって(~する) 子供たちは海辺で砂にまみれて遊んでいる ~だらけ 
\\	野良猫がゴミ袋からごみを引っ張り出すので、町中がごみだらけだ
\\	~めく (めいた)
\\	毎年3月には花が咲き始めて、山は春めく、4月になると、すっかり春になる 社長の皮肉(セクハラ)めいた言い方を/からして、社員は不快に感じている 冗談めいた話に聞こえるかも知れませんが、すべて本当のことです 彼は、「もう直ぐ世界の終わりが来る」と言って、みんなからばかにされていた。しかし、世界のあちこちで戦争や地震や新種ウイルスによる病気が起きて、彼の話が真実めいてきた	
\\	春、秋、謎、冗談、皮肉 
\\	ざわめく=ざわざわする・騒がしくなる 色めきたつ=緊張・興奮した様子になる  
\\	きらめく ひらめく  (きらきら、ひらにら)
\\	もさることながら (も然る事乍ら 
\\	あんなやつの話を真に受けるのもさることながら、(しかも)金まで貸すとは、愚の骨頂(こっちょう
\\	この図書館は蔵書の多きもさることながら、サービスが実にすばらしい 
\\	下から眺めるテレビ塔やイルミネーションの美しさもさることながら、展望台に登って眺める市もまた格別です 
\\	彼女の留学は、本人の希望もさることながら、親の意向もあって実現することになった 
\\	彼は、英語もさることながら、中国語もかなり上手だ 
\\	駅前にできた新しいホテルは、便利さもさることながら、眺めも素晴らしいので人気があるそうだ 
\\	あの女優は容貌(ようぼう)もさることながら、演技力といわず、歌唱力といわず(が)素晴らしい。
\\	といい...といい) 
\\	もさることながらB」Aを肯定しながらBに重点を置いた表現 
\\	もさることながら	(の)にもまして。。 然る事乍ら 【さることながら】 
\\	親の意向ももちろんあるけど、留学は本人の強い希望なんです 彼は、素質
\\	はもちろんですが、あの厳しい練習で金メダルを勝ち取ったんですよ さること=さ(そう)+ある+こと=さること=そうであること(そう)
\\	~ものを 
\\	~をものにした 
\\	~ものになった 
\\	ものをいう	
\\	人手が足りなかったんですって?ちょっと連絡してくれれば、手伝ってあげたものを。 
\\	嫌だったら断ればいいものを、何でも引き受けてしまう。だから、彼はオーバーワークになって疲れてしまうんだ 
\\	居留守を使えばいい物を、彼女は気が弱いのでそうもできず、いやいや人に会うことになってしまうのだ 
\\	電話で予約すれば簡単に済むものを、彼はわざわざ店まで行ったんだって 
\\	正直に本当のことを言えばいいものを、彼はつまらない嘘をついて信用を失ってしまった 
\\	私のことが嫌いなら嫌いだともっと早くはっきり言ってくれたらよかったものを 
\\	~のに ~すればいいのに(しない)という話者の軽い不満の気持ちを表す 
\\	もの
\\	、ものだ
\\	、ものがある
\\	、 ものか
\\	、 ものなら
\\	、 ものの
\\	、 たいものだ ものにする=完全にする 
\\	・彼らは苦しい練習を重ねた末、勝利をものにした。 努力の結果、とうとう日本語をものにした ものになる=完全になる 
\\	「最近の若者は、苦労して仕事を教えて、やっとものになったと思うころに辞めてしまう」と部長が嘆いている。 いろいろ習い事をしたんですが、どれ一つものになりませんでした ものをいう=効果を発揮する 
\\	希望の会社に採用された。長い経験がものをいったようだ その国では、何をするにも親の地位がものをいうのです
\\	果たして 
\\	果ては	
\\	果たして本当だろうか? 
\\	彼は果たして成功した。 
\\	経営陣は果たして本気で我々のボーナスについて考えようとしてるのか、それとも単に口先だけだったのだろうか。 
\\	だめだろうと思っていたが果たして失敗だった。 
\\	果てはどこまで行くの? 
\\	彼女は不平を言ったり、罵ったり、懇願したりして、果ては離婚すると脅迫しだす始末だった。
\\	やっぱり ー 結局 
\\	初めて
\\	従って
\\	却って
\\	極めて
\\	至って
\\	わけても 
\\	取り分けて
\\	とりたてて
\\	ず(ば)ぬけて 
\\	そうじて 
\\	努めて 
\\	曲げて 
\\	引いては 
\\	おって 
\\	(こと)改めて 
\\	あわせて 
\\	好んで 
\\	加えて 
\\	続けて 
\\	せめて 
\\	目立って 
\\	優れて 
\\	かねて 
\\	かねて主張してきた立場 
\\	ひるがえって 
\\	翻って考えるに 
\\	しめて 
\\	しめて20数万円を1泊2日で使う。
\\	いかにして 
\\	問題はいかにして資金を集めるかである。 
\\	かくして 
\\	かくして、彼らに脅威を与えるほかの動物に発見されないですむ。
\\	しかして 
\\	=こうして こうしてから彼女は部屋から出ていった。
\\	こぞって 
\\	(挙る
\\	彼らはこぞってその候補者を支援した。 
\\	使い果たして 
\\	私達は石炭を使い果たしてしまった。 
\\	使い切る 
\\	いかにも~ (如何にも)~ 
\\	らしい、~そう、~よう、など)	
\\	あの鞄はいかにも重そうだが、中身はからで、1キロしかない。 
\\	そのような行いはいかにもメグらしい。
\\	彼はいかにも「営業」って感じだね。
\\	あの人は見かけ(上辺だけで)はいかにもこわそうだが、本当はとても優しい人です。 
\\	いかにも 
\\	いかにも 
\\	いかなる~でも	
\\	いかなる意見でも、必ず聞き入れます。 
\\	いかなる人でも入ることはできません。 
\\	いかなる状況でも部屋を離れてはならない。
\\	どんなに いかなる~か 
\\	いかなる政策をえらぶべきか いかなる~も:いかなる図書館にも 
\\	いかなるNでも/だれ/であろうと(も): いかなる人でも いかなるN+Prtても: いかなる大学で教えても	いかなる場合でも 【いかなるばあいでも】 
\\	~かけた ・ ~かけの ・ ~かけだ	
\\	作文は今日中に書かなければならないのにまだ書きかけだ。 
\\	冷蔵庫の中にある飲みかけの物を捨てて下さい。 
\\	おぼれかけた人は息を吹き返した。 
\\	半分忘れかけた音楽が彼の心の中で踊りました。 
\\	彼が眠りかけたとき名前が呼ばれるのを耳にした。 
\\	彼はその仕事に飽きて、やりかけたままで放っておいた。
\\	言いかけ (いいかけ) 
\\	本を読みかけて、寝てしまいました 
\\	食べかけのりんご 
\\	書きかけだ 
\\	(なら)まだしも 
\\	よりまだしもA 
\\	ほうがまだしも、Bより(いいなど)	
\\	一度ならまだしも、ここまで六回その言葉を間違えって書いた。 
\\	この方がまだしもあれよりよい。
\\	こんなひどい目に合うよりまだしも死んだほうがよい。 
\\	すうじつならまだしも10日も滞納している。すぐ支払わないと立ち退かせるまでです。(追い立てる) 
\\	なくした金が私のだったから、まだしもよかった 
\\	食事ぐらいならまだしも、泊めてくれとはずうずうしい	まだしも
\\	ーだけでもましだ だけまだましだ
\\	~兼ねる ・ 
\\	~兼ねない 
\\	その少年は盗みをしかねない。 
\\	冬に暖かい服装をしていないと、ひどいかぜをひくことになりかねない。 
\\	こういう事情ですから、残念ながら、せっかくのご招待をお受けいたしかねるのです。 
\\	ご期待に添いかねます。 
\\	その点は賛成しかねる。 
\\	仕兼ねない 【しかねない】 
\\	私は勉強と遊びを兼ねている仕事に就きたい。 
\\	ならいざ知らず ・ ならともかく 
\\	辞書があるならいざ知らず、何もなくてはこの文章はとても読めるものではない。 
\\	専門家ならいざ知らず、素人ではこの化学品を作ることができない。 
\\	他の人ならいざしらず、私はそんなことができない。 
\\	こんな大きな下駄、相撲取りならいざ知らず、普通の人で履く人はいないだろう。 
\\	若いころならいざ知らず、今はとてもそんなに速くは走れない。 晴れた日ならいざ知らず、そんなものを一日で乾かそうなんて無理な話だ。
\\	一時代前ならいざ知らず、今は高卒が当たり前だ。	名詞句 
\\	ならいざ知らず, 
\\	いざ知らず 人はいざ知らず私は彼の言葉を信じる 
\\	ならともかく= 
\\	君ならともかく、俺はそんなことをしない。 
\\	外人ならともかく、日本人はそんなことは絶対にしない。
\\	いざ 
\\	いざというときに 
\\	いざ~する段になると ・ 
\\	いざ撮影する段になってカメラにフィルムが入ってないのに気が ついた。 いざ~となると・となったら・とすると 読めるのだが、いざ書ことすると書けない漢字がたくさんある 
\\	いざとなれば 
\\	(やむを得なければ) 
\\	~に相当する 
\\	この日本語に相当する英語はない
\\	鳥の羽は人間の腕と足に相当する。 
\\	日本語の「義理」に相当する英語は見つからない。 
\\	その当時1ドルは大体110円に相当していました。 
\\	彼らが一日に食べる食べ物の量は、彼らの体重に相当している。 
\\	彼の月給は、私の半年分の給料に相当します。 
\\	その語に相当するようなフランス語を知っていますか。 
\\	"それに等しい 等価 
\\	同等 
\\	一等賞または同等品=
\\	均等 
\\	均等論=
\\	さながら「の」~; さながら~のように  (宛ら)	
\\	本場ニューヨークさながらのジャズナイト 
\\	実際の試験さながらの模擬テスト 
\\	さまざまなトラブルを想定した本番さながらの訓練 
\\	少女たちはさながら蝶のように軽やかに踊っていた。 
\\	3人の少年はさながら猿のように木に登った。 
\\	まるで 
\\	あたかも 
\\	そびれる	
\\	借りたお金を返しそびれた。 
\\	飲みすぎて終電に乗りそびれた。 
\\	かれは「ありがとう」と言いそびれるやつだ	
\\	聳える 
\\	損なう 返し損なう 
\\	やり損なう 
\\	~隙に (すきに)	
\\	皆さん、ジョンさんがいないすきに、サプライズパーティーを整えましょう。 
\\	皆が前を向いているすきにぎゅっと手を引いてほしいんだ。 
\\	先生が余所見をしているすきに、ラッブレターをきみが渡してくれたんだな。 
\\	彼女は気がつかないすきに、彼女にキスした。 
\\	間に~毎日コツコツ日本語を勉強すれば、知らない間に上達していると思うよ。
\\	いない間に、郵便と新聞がたまってしまった。
\\	しばらく会わない間に、ずいぶん額が広くなったね。
\\	うちに~マッサージをしてもらっている内に、疲れが取れていくのを感じた 
\\	(つけいる機会) 
\\	逃げる隙がなかった 
\\	隙をつく (隙を乗じる) 
\\	隙をねらう(うかがう) 
\\	隙を与えない 
\\	隙を見せない
\\	~ただでさえ 
\\	もうこの近所はただでさえうるさい[うざい]ところが、今週は近くで工事が始まったため、夜は一睡も出来なくなった。 
\\	もうただでさえ迷惑をかけているのに、もうこりごりだよ! 
\\	これは ただでさえおいしいものが、さらに我が隠し味を入れたら美味しくてたまらなくなるよ。 
\\	この夏はただださえ暑かったが、うんと暑くなってきたよ。道路に陽炎(かげろう)が見えるよ!この猛暑(は/を?)どうやって凌げばいいだろう
\\	今日はただでさえ忙しいのに、母に買い物に行かされた 
\\	ただでさえ、日本人は下に見られているので、このような場面においてはかなりきついことを言われる。 
\\	のに、ところが、が、けど、
\\	たかだか	
\\	この仕事だったらたかだか三日もあればできる。 
\\	精精(せいぜい) 今度は徹夜になりそうだ。今のうちにせいぜい休んでおこう。 
\\	ここから歩いてせいぜい10分のところです。 
\\	せいぜい20人しかそのパーティーには来ないだろう。 
\\	~からすると;~からすれば ~から判断すると;~から判断すれば・して	
\\	彼のメールから判断すると馬鹿な人です。
\\	この空模様からすると、雨になりそうだ。 
\\	話し方からすると、かれは、東京の人ではないようだ。 
\\	部長のあの表情からすると、この企画はあんまり進めたくないようだ。 
\\	買い手の立場からするとこれらのCDプレーヤーの値段は高すぎる。 
\\	~から推して 
\\	彼女の服装から推して、彼女は自分で言うほど金持ちではなさそうだ。 
\\	以下)という事実から察すると
\\	まさかの(時)~	
\\	まさかの時、すぐお知らせ下さい。 
\\	まさかのときに備えて貯蓄せよ。 
\\	まさかの時にはあなたを助けてやろう。 
\\	よもや 
\\	よもやそんなことは有り得ない 
\\	よもやそんなことはあるまい。 
\\	~としたら ・ ~とすれば ・ ~とすると	
\\	温泉に行くとしたらどこがいいでしょうね。 
\\	この質問を答えるとしたら何が貰いますか。 
\\	君が僕の立場としたら、この場合どうする? 
\\	このまま不況から抜け出せないとすれば、日本の将来に希望はない。 
\\	仮に 
\\	としたら・とすれば・とすると いA 
\\	としたら・とすれば・とすると なA・N 
\\	だ 
\\	としたら・とすれば・とすると	
\\	金持ちに生まれていたら生活がもっと豊かだったろうに 
\\	大統領であるとしたら、飢えた人々を救済したであろうに
\\	~ようもない ・ ~ようがない 
\\	~ようがある	
\\	ひどい虫歯ですね。治しようがありません。もう抜くしかありません。 
\\	彼の作品はたとえようもないほど素晴らしい。 
\\	スプーンがないことに気がついた。これでは食べようがない。
\\	彼の発言は他に解釈のしようがない。 
\\	もっと他に言いようがあるだろう。
\\	これだけストレスが溜まると、どうにも解消しようがない 
\\	ようがない 食べます 
\\	食べようがない します 
\\	しようがない	-かた; 食べ方 
\\	他の直し方がない. 
\\	~を問わず ~のいかんを問わず	
\\	学歴、経験は問わず、やる気のある社員を募集します。 
\\	試合は晴雨
\\	=晴天でも雨天でも)を問わず行われます。 
\\	年齢を問わず人々はこの歌が好きだ。 
\\	肌の色のいかんを問わず、彼は万人の言論の自由を擁護した
\\	多少を問わず 
\\	老若男女を問わず 
\\	ことのいかんを問わず 
\\	洋の東西(とうざい)を問わず 
\\	京都には季節問わず観光客が来る		問わず語り(とわずがたり)
\\	問わず語りにそう言った 
\\	を顧みず...する
\\	~をもと{基}にして (いる) ~をもとに 
\\	~を踏まえた	
\\	この映画は小説をもとにしている。
\\	以前は経験を元にして設計していましたが、 今や経験と勘だけではレースに勝てるような マシンを作ることはできません。 
\\	ファンの人気投票をもとに審査し、今年の歌のベストテンが決まります。 
\\	山川民の調査結果をもとに、今後の方針を決めようと思う。 
\\	ように 
\\	通りに 
\\	に沿って 
\\	に基づいて 
\\	を基にして 
\\	のもとで= 
\\	を中心に 
\\	~を踏まえる  (ふまえる) 
\\	根拠とする 
\\	経験を踏まえた助言 
\\	事実を踏まえた考察 
\\	それまでの実績を踏まえてオリンピック出場選手を決める 
\\	彼の疑念には根拠がある 
\\	根拠になる資料がない 
\\	~をめぐって	
\\	先生たちはその問題をめぐって賛否が分かれた。 
\\	その土地をどうするかをめぐって反目
\\	が生じた。 
\\	国会は、盗聴法案を巡ってまさに侃々諤々の体であった。 
\\	をめぐり 
\\	~をはじめ(として)	
\\	本日、退社することになりました社長を初め皆様には本当にお世話になりました メルセデスベンツをはじめとして、輸入車を販売しています。 
\\	携帯電話をはじめ、あらゆる組み込み
\\	機器の
\\	を
\\	により実現する。	を始め
\\	をはじめ、
\\	かれの下品の言い方からして、すごく嫌な人です 
\\	~を
\\	の)中心に (動く)	
\\	中国を海外展開計画の中心に議論しましょう 
\\	その家のことはすべてメアリーおばさんを中心に動いていた。 
\\	世界経済はアメリカ経済を中心に動いているという事実は誰も否定できない。 
\\	非核武装を中心にした話し合い。 
\\	を 
\\	の 中心に する 
\\	を 
\\	の 中心に する 中国を中心に議論をする 
\\	を中心に 
\\	を 
\\	~を通じて [つうじて]	
\\	仕事を通じて会いました 
\\	会話を通じて日本語覚えてます 
\\	言葉(読書)を通じて多くの言葉が習得される。 
\\	この日はスポーツを通じて健康を増進するために制定されました 
\\	貨幣は銀行制度を通じて流通する。 
\\	1年を通じて続く 
\\	1990年代を通じて見られた傾向が、2000年代にはさらに加速するものと思われる 
\\	にもわたり、にわたって(亘る) ~中;
\\	つつ 
\\	つつある	
\\	思い 
\\	いい 
\\	食べつつパソコンも使う 
\\	昨夜、久しぶりに会った友達と酒を飲みつつ語り合った。 
\\	往時(おうじ)を語りつつ夜を明かした。 (語り明かした) 
\\	このレポートでは、調査の結果を分析しつつ、今後の問題点を明らかにしてみたいと思う。 
\\	もらった手紙に返事を書かなければと思いつつ、もう10日も経ったしまった。 
\\	運動不足は体に悪いと知れつつ、時間がなくて運動ができない。 
\\	今ダイエット中だからなんとか言いつつお菓子食べてるし。あきれるよ。 
\\	彼って忙しい忙しいって言いつつ、超暇そうなんだけど。 
\\	彼女の新しい連絡先を控えておこうと思いつつ、ついつい忘れていました。 
\\	やりたい事をやりつつお金を貯める。  
\\	最近勉強をするのが嫌いになりつつある 
\\	死亡により人口は減少しつつある。
\\	市の中心地に新しい博物館が建造されつつある。 
\\	~ながら
\\	~だけのことはある 「だけあって」	
\\	明子は韓国語がとても上手だ。さすがに2年間ソウル大学に留学していただけのことはある 毎日1キロ以上水泳をしてきただけのことはあって、彼は90歳の今もかくしゃくとしている。
\\	隣の家の子供たちは実に行儀がいい。親のしつけが厳しいだけのことはある。 幸い日本で日本語を勉強して、かなり話せるようになりました。日本に行っただけのことはありました。
\\	このメロン、甘くて、とってもおいしい。高かっただけのことはあるね。 
\\	その映画は面白かった。
\\	が作っただけのことはある。 
\\	その映画は面白かった。
\\	が作っただけのことはある。 
\\	が作った映画だけあって、面白かった。 
\\	だけあって・だけに、B。」=
\\	。Aだけのことはある。」 
\\	だけのことはある 
\\	だけのことはある いA 
\\	だけのことはある なA 
\\	な・だった 
\\	だけのことはある
\\	""このメロンはおいしい。高かっただけのことはある。
\\	""このメロンは高かっただけあって、おいしい。” 
\\	彼女が人気女優だけあって、すごいうちに住んでいる
\\	聡子は小さい時からバレイをしているだけあって、体がしなやかで柔らかい 彼が自慢するだけあって、オフィスからの眺めはちょっとほかでは見られないものだった 
\\	伊達(だて) に~ ない 
\\	だてに包まれているわけじゃねぇよ 
\\	~だけの+N 
\\	どうぞ 
\\	できるだけ	
\\	お好きな だけどうぞ。 
\\	できる だけのことはした。 
\\	12人座れるだけの椅子がありますか。 
\\	彼に必要なだけのお金を貸してやりなさい。 
\\	だけに 
\\	苦難の途を歩んできただけにできたひとだ。 
\\	フランスで勉強しただけに、彼女のフランス語は発音がきれいだ。 
\\	一生懸命働いただけに、彼は昇進を自分の物にした(ついた) 
\\	ビルは元フットボール選手だけに足が速い 
\\	彼は長い間日本に住んでいただけに日本のことをよく知っています, 
\\	彼は仕事の覚えが遅いだけに昇進も遅かった 
\\	だけに 
\\	だけあって 
\\	彼が自慢するだけあって彼のオフィスからの眺めはちょっとほかでは見られないものだった 
\\	吉岡さんが薦めるだけあって、ここの料理はちょっとそこらでは食べられない 
\\	ばかりに 
\\	[さすが] だけあって	
\\	さすが毎日トレーニングしているだけあってすごい体力。 
\\	さすが評判のいいレストランだけあって料理もサービスもすばらしかった。 
\\	専門家だけあって彼はその分野に詳しい。 
\\	彼女は、美術学校を出ただけあって、絵が上手です。 
\\	この壺はさすが名工の作品だけあって色や形に気品がある 
\\	あのレストランは藤木さんのお薦めだけあっておいしかったです 
\\	だけあって 
\\	だけに 
\\	さすが(に) 
\\	だけあって 
\\	だけのことはあって;だけに
\\	ばかりに 
\\	新車を買ったばかりにお金がありません (一文なしだ。/破産だ。/すっからかんだ。) 
\\	ドイツ語の試験に落ちたばかりに、来月に卒業できない。
\\	一言変なことを言ったばからに関係が悪くなった 
\\	あの子は弱いばかりにいじめられている 
\\	アメリカ人である/であった/だった/ばかりに僕(しもべ)のような感じに扱われた。どうしていいのか途方に暮れて何らかの疎外感やストレスを感じた。 
\\	会いたいばかりにはるばるやって来た 
\\	雉(きじ)はその時鳴いたばかりに見つかって撃たれてしまった 
\\	大の凝り性で細かいことにうるさいばかりに、馬鹿にされて、うっとうしく思われている 
\\	完全主義者
\\	彼の申し出を受けたばかりに、軽はずみだと批判された。 
\\	(と) 
\\	今がチャンスとばかりに彼女に言い寄った 
\\	絶好 の機会とばかりに飛びついた。 
\\	だけ 
\\	(もっぱら~する一方) 
\\	してばかりいる そればかりでなく 
\\	才能ばかりでは成功できない 
\\	何でも上げるがこの絵ばかりは上げられない 
\\	正直な人ばかりはいない 
\\	赤ちゃんがただ泣くばかりだった。 
\\	およそ 
\\	50歳ばかりの男 
\\	それだけの原因で 
\\	ほとんど同然なさま 
\\	=~と言わんばかり ただ生きているというばかり 
\\	どうやら借金せずにいるばかりさ 
\\	彼は驚いて腰を抜かさんばかりだった 
\\	ご飯は食べるばかりになっている。 
\\	~して間もない 
\\	口をあけたばかりのワイン 
\\	今朝着いたばかりです 
\\	まるでそう言うように;強調 ここぞとばかりに力をこめた 
\\	姉は待ってましたとばかり手を出してきた 
\\	ばかりか,
\\	柔道は若者の健康によいばかりか、人格形成にもおおいに役立つ。 
\\	気温ばかりか湿度も高い。 
\\	この報告書は字が汚いばかりか間違いがたくさんある 
\\	彼女は美人であるばかりか気立てもいい。 
\\	そればかりか... 
\\	これは効果的でないばかりか、プロらしくないやり方です。
\\	昨今(さっこん)、(帰国者は)面接官の質問に英語で正しく答えることができないばかりか、日本語がきちんと話せないので、外国人スタッフに通訳するにも困ってしまうケースがある 
\\	「ばかりでなく」の硬い表現。そして、「程度が大きい」ということを強調する。 
\\	ばかりでなく 
\\	ばかりではなく だけじゃなくて
\\	挙げ句 [あげく] 挙句の果てに あげくに	
\\	挙げ句の果てに彼は刑務所行きになった。 
\\	迷ったあげく彼の誘いを断ってしまった。 
\\	コンピュータを物色して
\\	歩いたあげく、デイヴィッドより200ドル安い値段で手に入れた。 
\\	さんざん文句を言ったあげくに、出て行ったんです。 
\\	さんざん考えたあげくに進学した 悩んだあげくの決断	
\\	悩んだ挙句、
\\	上で、末に、結果
\\	~はともかく	
\\	店は雰囲気はともかく味はいい。 
\\	いつにするかはともかく、どこにしましょうか 
\\	冗談はともかく、頭が痛いのなら医者にみせるべきだ 
\\	私は話すことはともかく、 書くことは苦手だ 
\\	はともかく、
\\	は良い) 
\\	味はともかく、値段が安い 
\\	費用の問題はともかく 
\\	はさておき ~ならともかく
\\	~はさておき	
\\	それはさておき本題に入ることにします。 
\\	それはさておき、他に必要なものはないか。 
\\	金の問題はさておき、そういう旅行は疲れるよ。 
\\	仕事の話はさておき、今日は思い切り楽しみましょう。 
\\	それはさておき、本題に移ります。 
\\	(それは今は置いておきましょう)		~はともかく
\\	~反面 [~はんめん]	
\\	普通兄は優しい反面、卑しいところもある。 
\\	その仕事は十分にお金になるが、その反面1日に12時間働かなくてはならない。 
\\	卒業できるのはうれしい反面、今日で皆とお別れだと思うと淋しいです。 
\\	携帯電話は便利な反面、わずらわしいこともある 一人暮らしは気楽な反面、寂しくもある 
\\	孫はかわいいが、その反面(において)、疲れるから預かるのは大変だ。 
\\	反面教師 
\\	その一方 
\\	他方は
\\	一方は...他方は... 一方はその仕事は十分にお金になる。他方は1日に12時間働かなくてはならない。 でも、かえって 
\\	反対に 
\\	タクシーで値切った事はないけど、反対にぼられそうになったことはある。
\\	にたいして 
\\	にはんして 
\\	我が社はテレビの売れ行きが伸びているのに反して、オーディオ製品の売れ行きが下がっている
\\	を契機に [をけいきに]	
\\	仕事の失敗を契機に、彼は自分の経歴を検討するようになった。 
\\	退職を契機に知識を深める (ふかめる 
\\	大統領が中国に行ったのを契機に万里の長城と紫禁城を訪問しておりました。 
\\	オンデマンドを契機にダンボールから印刷までサービス 
\\	~をいいことに…する ~をうまく[巧みに]利用する、~を生かす、~を活用[駆使]する、~の特典を生かす 
\\	~を悪用する、~に乗じる、~に便乗する 
\\	金持ちはいつも貧乏人を食い物にしているように見える。
\\	つもりで	
\\	ボナスは買い物に使ったつもりで銀行にお金を入れた。 
\\	父親になったつもりで学生の世話をした。 
\\	有名な人になったつもりで、道で写真を撮らせた。 
\\	威厳を込めたつもりで言う 
\\	自分が客になったつもりで考える 
\\	~て ならない 
\\	でならない[なりません] 
\\	で ならない
\\	戦争が始まる(勃発する)かもしれない。国民は不安でならない様子だ 
\\	夫が単身赴任
\\	しているので、寂しくてなりません。 明日のプレゼンのことが心配でならない。眠れないよ~。 
\\	毎日が楽しくはずなのに、気分が滅入ってならない (めいる 
\\	へこんでいる) 
\\	実家が思い出されてならない 
\\	嬉しくてならない 
\\	不安でならない 
\\	ならない 
\\	痛い、嫌い、好き.	
\\	掛かる [かかる]	
\\	ドアが開き かかったが、また閉まってしまった。 
\\	僕は車の事故で死に掛かった。 
\\	森の上に月が出かかっていた。 
\\	月曜日に始めた仕事も金曜日には出来かかっていた。 
\\	=そうになる 車に轢かれそうになる。 犬の糞を踏みそうになった。 
\\	もうじき会議始まる 
\\	あの2人、結婚間近なんだって 
\\	掛け
\\	~を目当て(にして/で/の)	
\\	何が目当てで私に会いに来たのだろう 
\\	(相撲)今日のお目当ての一番 
\\	金目当てに結婚する 
\\	何の目当てもなしに 
\\	探検隊は北極星を目当てにして進んだ 
\\	ため
\\	~に応えて (にこたえて)	
\\	"わが社は消費者のニーズに応えて新製品を開発しました。 
\\	社員の要求にこたえて労働時間を短縮した。 
\\	「医者になってほしい。」という親の期待に応えて、彼は医者になった。 
\\	~ものの、B とはいうものの	
\\	パソコンを買ったものの, 使い方がぜんぜんわからない。 
\\	熱は下がったものの、まだ咳が止まらない。 
\\	軽いけがで済んだからいいようなものの、次回からは気をつけて下さい。 
\\	あのホワイト・ライオンの子どもはおとなしいとはいうものの(=とは言えども)、なんといっても獣ですから、気を許さないで下さい。 
\\	いろいろな説はあるものの、恐竜がなぜ突然絶滅してしまったのかはまだ謎のままである。 
\\	パーティーに行くことにしたものの、気がはずまない。 
\\	意気揚々
\\	と就職活動をしてみたものの、ことごとく断られてしまった。 
\\	パソコンを買った[ものの]、使い方が全然わからない。 
\\	とはいうものの、
\\	先生とは言うものの、
\\	が、けれども 
\\	ものの 
\\	ほんの 
\\	ものの200メートルも行ったころ 
\\	ものの10分も経たぬうちに 
\\	ものの数ではない 
\\	ものの数秒で 
\\	~ものがある	
\\	彼女の小説には人を勇気づけるものがある。 
\\	家族に会うと私の心に複雑なものがある。 
\\	彼の学問的な業績には感銘
\\	を与えるものがある。 
\\	彼の功績は特筆大書すべきものがある。 
\\	特筆大書 【とくひつたいしょ】 彼の性格には単なる正直以上のものがある。 
\\	彼女には何か磁力のようなものがある。
\\	物には限界というものがある。 
\\	僕の人生には何か欠けているものがあるような気がするんだ。 
\\	問題の中には外国人に不公平なものがある。 
\\	~どころではなく 
\\	~どころじゃない	
\\	お金がなくて、買い物どころではない。
\\	休暇を取るどころではなく、毎日忙しい。
\\	飲みに行こうか?
\\	それどころじゃないんだ。妻が入院したんだ。
\\	今日は非常に暑かった。もう暖かいどころではない。
\\	ものすごい排気ガスを吸っているから、健康どころではない 怪我をしてテニスどころではない。
\\	どころか 
\\	彼女、日本語はなせるか?いや、話せるどころ(じゃない。/か、)一度も勉強したことがないんだ 
\\	どころのはなしではない
\\	~からには 
\\	~からには 
\\	約束したからには、やらなければなりません。
\\	名指しで非難されたからには、受けて立つしかないでしょう。 
\\	きみがそう言うからには、正しいに違いない 
\\	同僚に英作文頼まれたんだけど、やっぱり回答するからには、ちゃんと正しい答えを教えたい。 
\\	いったん引き受けたからには最後まで責任をもってやります 
\\	裁判というからには弁護人が不可欠だ 
\\	写真をカラーで撮るからには、色にこだわりたい 
\\	一度すると言ったからには 
\\	函館に来たからは。。まずはやっぱり函館ラーメン
\\	以上は 
\\	酒を飲み続けている以上
\\	からには)病気は治らない 
\\	からして 
\\	彼の表情からして、何か心配事があるとわかった 
\\	(から見て、
\\	から言って)
\\	~こと ・ ~ことだ ・~ことではない 
\\	ことだ ・ ~ことだから 
\\	ことで 
\\	ことから	
\\	遅れないこと。
\\	旅をしたかったら、まず英語を勉強することだ。
\\	彼のことだから遅れるだろう 
\\	彼が何か言ったことで、喧嘩になった。
\\	明るい子供が欲しかったことから陽子と名付けた。
\\	に相違ない (そういない)	
\\	彼は犯人に相違ない。 
\\	相違ないこと  
\\	途中で彼女に何事か起こったに相違ありません 
\\	それに相違ないか? 
\\	注文内容をご確認いただき、出荷予定の品物に相違ないかどうかご連絡ください。 
\\	に違いない
\\	=差、違い、異なっていること. 
\\	相異 
\\	ちがいない;まちがいない 相違なく 【そういなく】 
\\	相違点 
\\	イギリス英語とアメリカ英語との間にはいくらかの相違点がある 
\\	~につけ
\\	~に付け込む	
\\	彼を見るにつけ、私の兄を思い出す。
\\	うまく行くにつけ、いかないにつけ、いつも文句を言う。
\\	~に付け込む 
\\	不安定な情勢
\\	不安感
\\	不幸
\\	弱点 
\\	待てなし・歓待
\\	無知 
\\	良さ 
\\	純朴さ
\\	うそにしろ本当にしろ 認めようが認めまいが 
\\	しようがしまいが 
\\	であろうとなかろうと するかどうか.. か否か
\\	しかるに (然るに)	
\\	何回も思い出させた。しかるに(それなのに)彼は忘れた。 
\\	彼は非常に尊敬されていた、しかるに彼の息子は非常に軽蔑されていた。 
\\	彼は非常に誉められた[賞賛された]がしかるに私は非常にしかられた[叱責された]。 
\\	この実験にはこれまでかなりの費用がかかった。然りにいまだ見るべき結果が出ていない。 
\\	地球上では多くの人々が飢餓状態にある。しかるに現代日本は大量の食べ残しを捨てている 
\\	反面 
\\	一方 
\\	~である一方で、~だがところが ~であるのに対して[反して]
\\	~のなんの[って]	
\\	あの映画、どうだった? もう!面白かったのなんのって! もう眠いのなんのって もう、うまいのなんのって! この茄子、おじいが作ってるんだってね 「この靴履いてると、足が痛いのなんのって」 
\\	うちのカミさん、寝相が悪いのなんのって。
\\	[上さん 【かみさん】 
\\	=女主人、女性の大家	
\\	って 
\\	~につき (に就き) 
\\	病気療養中につき欠席いたします。 
\\	出入り口につき駐車禁止 
\\	目的地天候不良につき、325便は本日欠航となります 
\\	一人につき、について800円 
\\	セールスマンへの報奨金は売り上げ10万円につき3000円だ 
\\	~のおかげで ~のせいで ~のため ~により、~によって ~が原因で、~に起因して、~の理由から ーずつ
\\	~(も)かえりみず (顧みず)	
\\	カメラマンは自らの命もかえりみず戦場に向かった 
\\	個人的な利益を顧みず 
\\	危険を顧みず 
\\	失礼をも顧みず~する 
\\	有権者の意を顧みずに 
\\	自分の幸福を顧みずに 
\\	自分の身の安全を顧みずに 
\\	身の危険も顧みず 
\\	(人)のプライバシーも顧みずに 
\\	~の危険を顧みず立ち向かう 
\\	省みず(反省) 
\\	顧みず(振り返る) 
\\	をものともせずに 
\\	省みる 
\\	誰かが本当に苦しんでいる時に、自らを省みず助けの手を差し出せるのはとても特別な人です。 
\\	失礼も省みず電話を差し上げる 
\\	~以上 
\\	君はそれを見た以上、我々に話さなければいけない 
\\	生きている以上は何か仕事をせねばならない 
\\	卒業した以上親の厄介にはなれない 
\\	いったんやろうと決心した以上は、あくまでも遣り通せ 
\\	参加する以上は是非とも優勝したいと思います 
\\	秘密を知られた以上、お前を生かして帰すわけには行かない 
\\	約束した以上は、必ず実行してもらおう 
\\	40以上100まで 
\\	想像以上である 
\\	~てからには ~てから(には) ~のなら いったん~したら
\\	折に
\\	をする) 
\\	折に触れて 
\\	折も折(とて) 
\\	ついでの折に 
\\	ようとしてる折も折 
\\	折もあろうに 
\\	おり悪しく (あしく) 
\\	折々(の/に) (季節折々の) 
\\	またとない折 又と無い機会	
\\	出かけた折に買い物をした 
\\	京都へ行った折に彼を訪ねました 
\\	ついでの折にでもお寄りください 
\\	季節の折々に食器を替えている 折々に買った骨董が数になってしまった	
\\	時, 
\\	またとない折 
\\	良い折だ 
\\	おりあしく (折悪しく) 
\\	おりよく おりもおり(とて) 
\\	テートに出かけようとしていた折も折り、会社から電話で呼び出された 
\\	折もあろうに 
\\	なんて、とは、
\\	折に触れて 
\\	折に触れて書きとめたエッセイ 
\\	そのことは折に触れて社員全員に注意していたんだが、残念ながら効き目がなかった 
\\	折の悪い 
\\	こんなときに彼が来合わせるとは折の悪い話だ 
\\	その折私も其処に居合わせた 
\\	あいつは折りさえあれば酒を飲んでいる 
\\	~ず もがな(=無くもがな 
\\	[ずもがな] 
\\	新しい電気製品なんて、買わずもがなで、買うと碌なものがない。 
\\	あの人の書く文章には無くもがなといえる言葉が多い。
\\	あんまり腹が立ったので、つい言わずもがなのことを言ってしまった。 
\\	~(しようとした)矢先に(やさき) 
\\	やさきに/の 
\\	~間際 (まぎわ)	
\\	こちらから電話しようとしていた矢先に電話が鳴ったのです 
\\	俺もそろそろ独立しなくちゃなと思っていた矢先に親父(おやじ)が入院した 
\\	事故は役場が防止策に乗り出した先矢の出来事だった
\\	不祥事が露顕
\\	したのは彼が新党結成に動き始めた矢先だった 
\\	彼女は死ぬ間際にそう言い残した 
\\	卒業間際になって単位の足りないことが分かった 
\\	発車間際の駆け込みは危険極まりない 
\\	(やじり) 
\\	(やおもて)
\\	~もってのほか(だ・です) (以ての外) 
\\	まして..もってのほか	
\\	私は中古車を買う余裕などなく、[まして]新車などはもってのほかだ。 
\\	新車を買うどころか,中古車(も/さえ/すら)もってのほかだ。
\\	現在のところ、3万種類の遺伝子が発見されていますが、3万の遺伝子で既知の病気を網羅することはできません。ましてや人間の特質すべてなど、もってのほかです。
\\	雨の中自転車に乗るなんてばかだし、乗りながら友達とメールをしてるなんて、もってのほかでしょう。 
\\	彼女は英語を話せない、ましてやフランス語なんてもってのほかだ。 
\\	既知 【きち】 
\\	網羅 【もうら】 
\\	況して{や} (ましてや)、 
\\	はお手の物だ/思いもよらない/不可能/はずがない/なおさらだ/もっとX	
\\	彼はスペイン語が話せる、ましてや英語はお手のものだ。 
\\	私はギターを弾くことはできない。ましてやバイオリンは弾けない。
\\	コンピューターの買い方さえ、さっぱり分からない、ましてや利用するなんて 
\\	いまだに狩猟というものが理解できません。ましてや殺人などなおさらです。 
\\	アパートの家賃を払うのが精一杯なのに、まして(や)家を買うなんて夢物語だ 
\\	彼女は日本語で講演もできる。まして(や)日常会話など問題ではない 
\\	この小説は大人にも読みにくいのに、まして小学生に読めるはずがない。 
\\	彼は簡単な英語すら話せない。ましてや、書くことなど思いもよらない。 
\\	僕は自転車さえ買えない。まして、車を買うなんて不可能だ 
\\	国内での単身赴任でもつらい、ましてや、外国での単身赴任となるともっとつらい 
\\	この数学の問題は教授でも解けないのだから、もして、学生の僕に解けるはずがない 
\\	それらの星は肉眼で見える、まして望遠鏡ならなおさらである。 
\\	彼は自動車を買う余裕はない。まして家は買えない。 
\\	だれでも自由を享受する権利を持つ。まして人生を享受する権利はなおさらだ。 
\\	フランス語は読むことさえ出来ないのです。まして話すことなどとても出来ません。 
\\	大人でさえ知らない事は、 まして子供は知っているはずがない。 
\\	ましてや~なおさら 
\\	ましてや~するほど~なものはない
\\	なおさら [どころか] 家を買うどころか、アパートの家賃を払うのが精一杯だ いわんや 
\\	ましてや 
\\	おろか 
\\	ヴァイオリンはおろか、ギターはとても弾けない。
\\	なおさら (尚更)	
\\	なおさら効果的 
\\	淑やかなのでなおさら(人)が好きである 
\\	醜い女が豪華に着飾れば、なおさら醜くなってしまう。
\\	英語を書くのは難しいが、正確さが要求される科学論文を書くのはなおさらだ 
\\	このチーズは搾りたての新鮮な牛乳で作るから、尚更おいしい	しとやかさ (淑やか)
\\	着飾る 【きかざる】 
\\	以前(として) 
\\	まだ 
\\	一層、もっと、さらに	余計 
\\	寝過ぎると余計眠くなるって言うよね。= 寝過ぎるとなおさら眠くなるって言うよね。
\\	一層 
\\	熱い風呂に入ったので、それだけいっそう気分がいい。 
\\	なお 
\\	日本へ行く前から日本が好きだったが、日本へ行ったらなお好きになった  噴火はなお続いている 
\\	なおきれいになる 
\\	なお10年 
\\	~をめどに (目処・目途)	
\\	10年後をめどに 
\\	年をめどに 
\\	年をめどに完全実施される 
\\	月をめどに合併する予定 
\\	当面_年間をめどにした接続料金引き下げの幅 
\\	~末をめどに 
\\	彼は、結婚するためにイランに帰るかもしれないと言いながらも(言いつつ)、日本から帰った後の彼の計画は依然としてめどが立っていない。 
\\	目安 めやす 
\\	~に見合う
\\	~に見合うだけの
\\	このお金であなたの要求に見合いますか。 
\\	費用に見合う価値がない 
\\	払った金に見合うもの 
\\	(見合うだけのもの) 働きたい人々に見合うだけの十分な雇用がない。
\\	消費者物価の上昇に見合うベースアップを要求する 
\\	そのリスクに見合うだけの利益ですか? 
\\	三人の少年たちは互いに顔を見合った。 
\\	相当する ーかいがある
\\	~における(於ける)	
\\	カメラ製造における日本の競争力は揺るぎない。 
\\	揺るぎない 【ゆるぎない】 
\\	男女関係における嫉妬はしばしば信頼の欠落によってもたらされる(齎す 
\\	この町における犯罪の発生率は高い。
\\	中東における現在の情勢は、緊迫した政情不安となっている。
\\	(政情=政治の情勢) パキスタンにおける核兵器の開発は、その地域を不安定にする可能性がある 
\\	心配なのは、同時に実施された意識調査における学習意欲の問題だ。 
\\	殊に (ことに)	
\\	この数年は異常高温が続いているが、今年の夏はことに暑く、熱射病による死者もたくさん出た。 
\\	この辺りはハイキングコースとして人気が高く、殊に、紅葉の季節にはハイカーたちで賑わう 
\\	特に 取り分け 
\\	特別に 際立って(きわだって) 著しく(いちじるしく)  めっきり
\\	めっきり寒くなった。 
\\	こともなげに 
\\	何気なく,3 不注意に,4 偶然に 
\\	さりげなく,6 思いがけなく	
\\	彼は頭がいいから、先生の難しい質問にもこともなげに答える その映画の主人公はこともなげに人を殺す。子供には見せたくない映画だ なにげなく 入ったのだが、意外に感じのいい店だった 。 他人に不注意に携帯電話の番号などを教えてはいけない 。 旅行中に思いがくなく 昔の友達にバッタリ会った。 失敗の原因を全て偶然に帰するわけにはいかない。		何気なく=ふと、不注意に 
\\	さりげなく 
\\	(然り気無い) 偶然に 
\\	思いがけなく 
\\	こともあろうに	
\\	暮れは大晦日までずっと晴天
\\	が続いていたが、事もあろうに、元日の朝から大雨となってしまった。 
\\	姉は受験勉強に励んでいたが、こともあろうに入学試験の日に高熱を出して、受験が受けられなくなってしまった 
\\	励む 【はげむ】
\\	折悪しく 因果なことには 
\\	折りよく 折もあろうに 
\\	ことごとく(悉く)	
\\	計画が悉く失敗してしまい、関係者はショックを受けている。 
\\	最近うちの子は親の言うことにことごとく反抗する。反抗期なのだろうか 
\\	こと
\\	ごとし 
\\	ことごとい 
\\	ことによると	
\\	検査の結果はまだ出ていませんが、ことによると手術するかもしれません。
\\	「そんなこと、ただの噂に過ぎないでしょう」「いや、ことによると事実かもしれませんよ」 
\\	ことによるとね、でも思い切った行動が必要なんだ。 
\\	私はジョンと同じくらい、ことによるとジョン以上に、あなたを愛しています。 
\\	~かもしれない	
\\	によると 星占いによると彼女と僕はあまり相性がよくないようだ。 天気予報によると今日の午後は季節はずれの雪が降るらしい。 新しい移民法が先月末から適用されましたが、それによるとかなり規制がゆるくなった気がします。
\\	「未確認の報告によると、地震による死者の数は120人ということです」。
\\	よう取り計らいなさい	
\\	この手紙が彼女に渡すよう取り計らいなさい 彼らが十分食べられるよう取り計らっていただけませんか?	
\\	再び間違いをしないように気をつけなさい よう面倒をみてください
\\	当初の
\\	当初の予定通りに、10分程度の面接に臨みました 
\\	当初の出発点に戻る 
\\	原点 その会合は当初の予定どおりに開催された。 
\\	いや、当初の考えはそうでした。 
\\	そもそも,初めから
\\	あえて~	
\\	あえて述べる あえて計画を立てる あえて異議を唱える あえて言えば;あえて反対はしないよ	
\\	思い切って) 
\\	思い切って
\\	あくまで(も)  
\\	{飽くまで}	
\\	あくまでも自説を変えなかった 
\\	あくまでも自分に正直だった 
\\	私はあくまでも彼の無実 を信じる 
\\	彼はあくまでも日本の参戦に反対を唱え続けた 
\\	ケイトはあくまでも自分の外見にこだわった 
\\	これは、あくまでも小説であって実話ではない 
\\	これはあくまでも私の直感なんですが、山田さんは近々この会社をやめるんじゃないかと思う 
\\	空はあくまでも青く澄み切っていた 
\\	4か月という筋はあくまで目安ですから、もっと長くかかるかもしれない 
\\	あくまでも君自身の人生なのだから進路は君自身で決めなさい 
\\	スポーツはあくまで楽しみであって、苦行の場であってはならない。
\\	くぎょう 
\\	二十歳とはいえ、まだあくまで学生でしょう。お酒は謹みなさい 
\\	(最後に、どこまでも、絶対) 僕は明日のマラソンで絶対/最後まで走りぬくつもりだ
\\	あながち~ない 
\\	あながち
\\	とばかりは言えない)	
\\	推理はあながち間違っていない 
\\	この俳優の演技はうまいが、時々違和感を覚えるのは、あながち演出のせいとばかりは言えない 
\\	審判もみんな賄賂を受け取ったと考えてもあながちこじつけ(た話)ではない。 
\\	金持ちは必ずしも傲慢ではない 
\\	あの人は、傲慢だといわれているけど、僕の観察では時々非常に優しい面を見せる。 あながち/必ずしも傲慢ではないのだ 
\\	(必ずしも~ない) まんざら~でおもない
\\	あるいは 
\\	あるいは~かもれしれない	
\\	政治学あるいは経済学を専攻した学生を採用します この会社は大規模なリストラをするか、あるいは、倒産するかしなない 将来、あるいは、アメリカのように二大政党になるかもしれない 
\\	マウス(か)あるいは/または/もしくは/
\\	それとも/キーボードで入力してください
\\	ご注文は電話(か)あるいは/または/もしくは/
\\	それとも/インタネットでお願します
\\	(または;もしくわ;それとも;また;ないし(は))
\\	あたかも 
\\	(かの)ように/な
\\	あたかも責任者のように振る舞う あたかも氷の上をすべる(かの)ように走る あたかも現場にいる(かの)ような感覚で操縦する あたかも専門家である/であった(かの)ように言う/ような言い方 
\\	感じ/感覚/印象/気がする 
\\	時はあたかも就職シーズンだ/時あたかも桜が満開の時だった	(まるで;~かのように/ごとし)
\\	べくもない 
\\	私が一億円の家を買うなど望むべくもない 
\\	最近は多忙で本を読む時間など期待すべくもない 
\\	わけがない 
\\	はずがない 
\\	のぞむ、知る、勝つ 
\\	~べくして 
\\	~べく 
\\	~べくして~
\\	彼は大統領になるべくしてなった。 
\\	二人の性格、趣味、学歴がすべて一致している。二人は結婚するべくして結婚したのだ 
\\	私は自宅が海のすぐ近くのため、釣り好きになるべくしてなったと言っていいでしょう 
\\	彼は嫌われるべくして嫌われている 
\\	なる、起こる、変わる、入る、勝つ、倒産する. 
\\	結婚する. 
\\	結ばれる,
\\	やっぱり/やはり 
\\	ちなみに (因みに)	
\\	彼の小説はポストモダン的で大変気に入っている、ちなみに、去年の作品はベストセラー第一位だった	
\\	ところで 
\\	だけ 
\\	たべるだけたべる 
\\	できるだけ	
\\	ほど) 
\\	若い内に本を読めるだけ読んでおいたほうがいい 
\\	食べるだけ食べると、礼も言わずに帰ってしまっ 
\\	食べたいだけ食べるのが僕の主義です 
\\	うちの子は背が伸びるだけ伸びた感じです 
\\	昨日は寿司屋で寿司を食べられるだけ食べた。 
\\	その学会にはできるだけ出席するようにします 
\\	さあ、飲みたいだけ飲んでくれ  
\\	言うだけ言ったらすっきりした。 
\\	借金の話ならお断りだよ! まあ、聞くだけ聞いてくれ 
\\	迷惑をかけるだけかけておいて挨拶なしとは不届き千万
\\	迷惑千万 
\\	千万かたじけない 
\\	だけ
\\	ほど
\\	だの 
\\	庭には菫だの、チューリップだのつつじだの、花がたくさん咲いていた 受験生たちは塾だのテストだのと忙しい毎日を送っている 
\\	とか、や 
\\	とか, など, や
\\	~ども 
\\	ども
\\	ども 
\\	この二つの文章は言い方は違えども、言おうとしていることは同じです 
\\	行けども行けども人家
\\	が見えてこない 
\\	知識はあれども学才がなく、今後学問の世界でやっていけるとは思えない。
\\	行けども 
\\	みれども 
\\	すれども 
\\	来れども(くれども) 
\\	食べれども 
\\	食べても 
\\	といえども 
\\	ど 
\\	待てど暮らせど 
\\	とも 
\\	何が起ころうとも驚かない 
\\	ても 
\\	行っても行っても、間違っても、
\\	1が
\\	1なら.. 
\\	会社が会社なら、もう少しましな給料を出してくれる 
\\	友達が友達なら、役に立つ助言をしてくれたかも知れない,
\\	親が親なら、子供も子もだ 
\\	医者が医者なら、看護師も看護師だ 
\\	ひいては~(延いては)	
\\	イラクへの攻撃はアメリカの孤立を招き、ひいてはイスラム世界全体との戦争に発展していく恐れがある 
\\	彼は同僚にライバル意識、ひいては殺意すら抱いていた (いたく) 
\\	(ライバル意識を抱き、ひいては...) 
\\	ことは、
\\	ば この状況が続けば、ひいては..	さらに(は) 
\\	一
\\	として
\\	[は]ない 
\\	一冊として。。。本はない	
\\	うちの課には一人としてろくに英語を話せる者はいない 
\\	この資料室には一冊として役に立つ本がない 
\\	剣道の審判は一秒として気を抜くことは許されない 
\\	たりとも=一人たりとも 
\\	一切~ない	
\\	この製品には遺伝子組み換え大豆は一切使用していません 
\\	私はその汚職事件とは一切関係ない 
\\	一切の外交関係を断ち切った 
\\	盗難に関しては当社は一切責任を負いません 謝金は一切いただくわけにはまいりません 
\\	ぜんぜん~ない まったく~ない
\\	言ってみれば 
\\	よう/みたい/など)	
\\	"日本の文化には厳しいしきたりがあって、言ってみれば、それは宗教のようなものだ 
\\	彼女の家は、言ってみれば、お城のように作られている 
\\	このブログは言ってみれば僕の絵日記のようなものだ 
\\	言ってみれば、その国のあらゆるメディアは、政府にコントロールされている状態です。 
\\	「写真は言ってみれば俳句のようなものです。
\\	言わば 
\\	たとえて言えば、たとえていうならば、
\\	ある意味では
\\	甲斐(かい/がい) (がある) 
\\	がい 
\\	かい 
\\	そのかいがって	
\\	日本語をよく勉強したかいがあって、日本関係の会社に就職することができた 社員一同の努力のかいがあって... 
\\	あなたにとって生きがいとはなんですか? 
\\	寒い中待っていたかいがあって、見事な日の出を見ることができた
\\	頂上からは下が全然見えず、昇ったかいがなかった, 
\\	私の料理をおいしく食べてくれる人がいるから、作りがいがある 長い間入院してしまった。そのかいがあって、今では元どうり歩けるようになった 辛抱のかいがあって、運が開けた 
\\	価値がある・意味がある (期待)の成果を挙げる、利 益[効果]を生む[もた らす・上げる]、うまくいく、成功する、 報われる むくわれる・ 報いる  むく.いる 元が取れる 功を奏する 
\\	長年の努力がついに報われて、彼 らは金持ちになった。 
\\	私は耐え忍ん だ甲斐があった。 
\\	勉強の成果です よ。 
\\	努力は功を奏する。 
\\	勉強のために使ったお金は元を取り つつある。 
\\	投資は回収可 能でなければならない。 
\\	練習の成果を遺憾 なく発揮できた。
\\	か否か(が/は)	
\\	このソフトが役に立つか否かは保証の限りではありません 
\\	子供に携帯を持たせるべきか否かは議論が分かれる 
\\	家を探す場合、その地域が安全であるか否かが一番大事なポイントになる 
\\	容疑者に遺体を見せると、真犯人か否かが分かることがある 
\\	(良い) 
\\	大学を出てすぐ就職するのがいいか悪いかは一概に言えない 
\\	冗談か否かの一線を越えた 
\\	~か~ないか  ~かどうか 
\\	否が応でも、否でも応でも 
\\	や否や 
\\	~かな	
\\	悲しいかな、年を取って往年の記憶力がなくなってしまった
\\	(往年 【おうねん】 
\\	惜しいかな、コンピューターを買うお金がないんだ 画面では絵も動きます。チャットもできます。惜しいかな、全部英語ですが 途中に教会があったので期待して入ってみると、果せるかな、来る素ます礼拝で荘厳な雰囲気であった(そうごん=
\\	果たせるかな
\\	案の定) 
\\	愉快、幸福、痛快
\\	、壮烈
\\	、憂鬱 ことに 
\\	ことに 面白いことに、日本人はやたらと英語の単語を使いたがる。 
\\	ことか 
\\	テキサス州は何と太陽が明るいことか 日本語が読めるようになるためにどれぐらい勉強したことか!
\\	~から言って 
\\	燃費からいって 
\\	評価からいって	
\\	の点で 
\\	からすると	燃費から言って{の点で}、この車は非常に経済的だ 
\\	学生の評価から言って{からすると}、鈴木先生はいい先生に違いない 
\\	見地から言って 
\\	学的見地から言って 
\\	コスト面からいってこの問題を解決することは難しい。 
\\	の点で. 
\\	からすると	からして、からすると、の点で、から判断すると
\\	~からなる (からなっている) ~とかるなっている	
\\	"英語の”
\\	はラテン語のRE(再び)と”LAXARE”(緩む)とからなっている 
\\	蛋白質
\\	はアミノ酸からなっている 
\\	フィリピンは多くの島々からなっている. 
\\	なる 
\\	成る 
\\	~で構成されている 
\\	~からできている 
\\	(もしも)仮に (すれば/にすると/にすれば/したら/としても)	
\\	仮に1億円 貰う/貰った としたら/とすれば 何に使いますか? 仮に手術が成功する/した としても、高齢だから父の体が元に戻ることはないだろう 
\\	仮に高い/高かった と したら/すれば 
\\	仮に話しても 仮に話す/話した としても 
\\	仮に日本に一年ぐらい行って日本語を勉強すれば、君たちの日本語はすごく上達するよ 
\\	見るからに	
\\	太陽が光をたっぷり浴びたミカンは見るからにおいしそうだ 
\\	彼女が身につけている物は見るからに高級品だった このパンは見るからに手作りという感じですね 
\\	その青年は見るからに聡明そうだった 
\\	【そうめい】 
\\	このスープは野菜たっぷりで、見るからに体に良さそう!	
\\	聞く =聞くからに 
\\	"あきらかに 
\\	いかにも 
\\	みるからに 
\\	彼の顔は青黒くむくみ、(見るからに/いかにも)健康を害していた 浮腫 
\\	も
\\	ば 
\\	もする 
\\	も
\\	ば
\\	も 
\\	こともあれば、
\\	こともある	
\\	私だって、ロックも聞けば、クラシックもする 岡山全部が担当地域なので、津山市へも行けば、岡山市へも行く 交通事情によって30分で行けることもあれば、1時間以上かかることもある アイデアが突然湧き上がってくることこもあれば、まったく出てこない時もある 担当の先生が来られない時は、代わりの先生を頼むこともあれば、やむをえず休講にすることもある 
\\	人生は晴れの日もあれば雨の日もある	
\\	円満な夫婦は口論したり、不満を言いあったりする 
\\	(時に)
\\	円満 【えんまん】 
\\	ものではない (もんじゃない=
\\	ものではない (もんじゃない=
\\	ものではない (もんじゃない=
\\	夜遅く人に電話をかけるものではない 
\\	彼の演奏は聴けたものではない 
\\	そんな時間のかかる仕事は引き受けられるものではない 
\\	女性に年齢を聞くものではない 
\\	こんなつまらない小説、読めたもんじゃない 
\\	辞書などは、そう簡単に書けるものじゃない 
\\	当然だ、当たり前だ 
\\	私は2、3日ではとうていその仕事を終えることはできない。 
\\	その女性の司会者は彼の傲慢さがどうしても我慢できなかった。 
\\	その建物は右側にあります。見逃しっこありません。 君がいくのはまっぴらだ。
\\	ものか 
\\	もの(だろう)か	
\\	分かるもんか 
\\	構うもんか 
\\	彼は親切ですか? 親切ものか。 
\\	もう少し広いアパートに住めないもの(だろう)か 
\\	もう少しまじめにこの問題に取り組んでくれないもの(だろう)か 
\\	日本語を勉強したいんですが、どの大学を選んだものかと考えているところだ 
\\	なる、変わる、痩せる、膨れる、呉れる、降る,
\\	ものだろうか, 
\\	思う、考える、
\\	迷う、思案する ""と
\\	子共が反抗期に入っている。その期間を問題なく乗り越えるにはどうしたものかと思っている 
\\	どころの段じゃない! 
\\	-たらいいのに、たらいいのだが 書き終われたらいいものか/書き終われたらいいのに 2たらいいか、べきか 決めたものか(と)迷っている/決めたらいいか(と)迷っている 決めるべきか(と)迷っている
\\	ものなら 
\\	(もし)一戸建ての家が買えるものなら、買いたい 一人でやれるものならやってみろ 
\\	この曲、やさしいと思うか?弾けるものなら弾いてみなさい 
\\	もしこの本は図書館で借りられるものなら買わずに済むのに、まだ入っていないようだ 
\\	一戸建ての家が買えるものなら、買いたかったのだが、
\\	~ようものなら
\\	(より)むしろ	
\\	夏休みはアルバイトをするよりもむしろ海外旅行がしたい 
\\	僕は寿司よりむしろ刺身(のほう)が好きだ 
\\	彼は秀才というよりはむしろ努力家だ 
\\	無為の生活をするぐらいなら、むしろ死んだ方がいい 
\\	英語は話すよりもむしろ読むのが好きだ 今年は2月よりもむしろ3月の方が寒かった	
\\	無為 【むい】 
\\	かえって 
\\	なす (なされている) 
\\	災い(わざわい)(を)転じて福となす 
\\	比喩に関しては現在多くの研究がなされている 
\\	人間というものは、悪をなす存在である 
\\	民主主義はアメリカ合衆国の基盤をなす思想である 
\\	醜を美となす 
\\	外国語教育がなされる 
\\	善をなす 
\\	インターネットに関してこのような質問がなされる可能性は高い 
\\	色をなす 
\\	群れをなす 
\\	恐れをなす;
\\	重きをなす 
\\	""ここを根拠地となす/する 
\\	環境保護の活動が盛んに(なされている/行われている/されている) 
\\	の基本(をなす/になる/となる)考え方は、さまさまな科学の分野で重要になってくると思われる
\\	~だけでもまし 
\\	だけでもいい 
\\	だけまだまし	
\\	シャワーは水だったが、水が出るだけまだましだった。 
\\	ボーナスが出るだけでも増しだ。 
\\	火事は災難だったが死者が出なかっただけでもよかった。
\\	それぐらいで済んでよかった	
\\	たった~しか、せめて~だけでも、せめて(~くらいは)、~であること(だけ)は確か、一度は、一応、曲がりなりにも、何はともあれ、とにかく、いずれにせよ 子どもの頃は映画を見れば多少なりとも感動していたような気がする。 
\\	段違い(である) 
\\	はBとは段違いである。 段違いに
\\	コミュニケーションを図る上で、段違いの差が生まれる 
\\	相手の実力が段違いだから、試合にならなかった 
\\	今の彼女の真剣なことといったら、以前とは段違いだ 
\\	段違いに強い 
\\	彼は(周囲の人とは)段違いに高価な時計を身につけていた 
\\	段 
\\	(一段=
\\	一続きの階段= 
\\	段々 
\\	二段ずつ階段を上る 
\\	押入れの上の段 
\\	(層) 髪に段をつける 
\\	3段の記事 
\\	(九九の) 五の段まで全部覚えた 
\\	(文章の一区切り) 
\\	初段 
\\	段が違う 
\\	(程度) 寒いどころの段じゃない! 
\\	うれしいとか言う段じゃない 
\\	いざ) いざ筆をとる段になると、頭に何も浮かんでこない。 
\\	雲泥の差がある 
\\	と
\\	では比べものにならない 
\\	と
\\	では比較できるような対象ではない  
\\	~に対等なものがない、~にかなわない 
\\	と
\\	には雲泥の差がある
\\	~(する)に越したことはない それに越したことはない	
\\	あの男には用心するに越したことはない 
\\	切らず(手術せず)に治せるなら、それに越したことはない 
\\	多い{早い}に越したことはない 
\\	多ければ多いほどいい なりすまし犯罪に関しては、用心深くなるに越したことはない。 
\\	(個人情報泥棒・身元詐称 など)身元=
\\	身元を証明する 
\\	確認するにこしたことはないよ 
\\	風邪はひかぬにこしたことはない 
\\	超す 
\\	私の店の売上が100万円を超したことはない 
\\	露(つゆ)....ない 露程も 
\\	知らず、疑わず)	
\\	とはつゆ知らず 
\\	私は彼女の言葉をつゆ疑わなかった[勘ぐらなかった] 
\\	露 
\\	露程も つゆほど(も) 
\\	僕は教師になるとはつゆほども思わなかった 
\\	自分が同じ行動をしていたとはつゆほども気がついていないながら、僕に注意した。 
\\	これっぽちも 一切~ない いっさい、いっこう、とうてい、すこしも、とても、いささかも
\\	において (に於いて)  における (に於ける)	
\\	第18回オリンピック大会が東京において開催された 
\\	少子化による学校の統廃合が都会において進んでいる 
\\	(統合
\\	廃合) この大会において... 
\\	オリンピック競技において最も大切なことは勝つことではなく参加することである 
\\	(その機関、役職で)(誰かの側で) 当方において.. 
\\	格自体において定める 
\\	(その時、場合に) 
\\	近い将来において 
\\	平安時代においては 
\\	その時点においては、警察はまだ彼が容疑者だと思っていなかった 
\\	(事柄に関して) 
\\	彼女は経歴において問題がある 
\\	使いやすさにおいてはこれが一番だ 
\\	外交面のみならず軍事面においても可能な限り(できらだ け)合衆国を支援するべきだ 
\\	機能性においても経済性においても、和服より洋服のほうが勝っている 
\\	学校生活において何か、不満なことでもあるのか? 
\\	-ーーーーOKERU 人生における大きな喜びは、あなたはできないと人が言うことをすることである。 
\\	その政治家は選挙における支持に対して一同に大いに感謝している、と述べた。 
\\	我々の同社における持分
\\	は、60%です。 
\\	日本における今年の賃上げはなだらかなものだった。 
\\	エイズ調査の分野におけるすべての変化を追い続けることはできない。 
\\	今の子供たちは人生におけるどのような問題でもコンピューターゲームのようにリセットできると思っている。 
\\	赤道における経度1度当たりの長さ。 (反対語は緯度) 
\\	その事故は彼の車の運転における不注意によるものであった。 
\\	近年における医学の進歩はめざましい。 
\\	で、の点で=から言って	に で にあって には
\\	~ようにする ~ように
\\	する	
\\	日本語が上手になるようにしています 
\\	病気にならないようにしましょう 
\\	上手になるように朝から晩まで勉強しています 
\\	見るようにします 
\\	間に合うようにします 
\\	~ために 
\\	やってみる 
\\	ようになる
\\	の 
\\	あげくに 挙句の果てに	
\\	悩んだあげくの決断 
\\	挙句の果てには 
\\	働いても給料がもらえず、あげくの果てに首になった 
\\	適切な応募者をさんざん探したあげくに。。。
\\	口論の挙句に殴り合いになった。 
\\	上で、末に、結果
\\	[の]に[は]持って来いのY 
\\	はYにもってこいだ 
\\	秋は読書にもってこいの季節だ。 
\\	日本の温泉は裸で泳ぐのにもってこいだ。 
\\	これらのりんごは料理用にもってこいだ。 
\\	ピクニックにはもってこいの天気だ。 
\\	打って付け ~にとって申し分のない、~に理想的な[適している], 
\\	にうってつけの
\\	子どもの遊び場にうってつけの場所 新婚さん[夫婦]にぴったり[おあつらえ向き]であるもの 忙しい人たちにとって理想的な食事
\\	まんざら~でもない/じゃない   まんざらでもないX	
\\	まんざらでもない顔をしている 
\\	まんざら夢ではない 
\\	初めのうちラジオ体操にはうんざりしたが、今ではまんざらいやでもない。 
\\	彼の演説はまんざら捨てたものではなかった。 
\\	~に対するまんざらでもない気持ち 
\\	それは、まんざら悪い知らせでもなさそうだ。 
\\	それはまんざら悪いことでもなさそうだ。 
\\	彼女は彼に愛してるといわれて、まんざらでもなさそうだった 
\\	満更	すべてが~なわけではない、全く~というわけではない (何から何まで)
\\	~にあたって (に当たって)(の)
\\	"友人を選ぶにあたってはいくら注意してもしすぎることはない  {選ぶときは} 
\\	(用心するに越したことはない 中学受験の勉強を始めるにあたって、多くの塾は「学習習慣を身につけよう」と呼びかけます 
\\	{始めるとき} 面接に臨むにあたっての心構え  
\\	{臨むうえでの/臨むときの} ~に当たっての一連の指針 
\\	についての ~に当たっての留意/勤勉 
\\	にかんする ~に当たってストレス/欲求不満を発散する 
\\	について ~に当たって細心の注意を払う 
\\	~に細心の注意 ~に当たって(人)をもてなす 
\\	{において} (人)に当たって援助を求める 
\\	{人に援助} 建設に当たって国からの補助金が増額されるように便宜を図る 
\\	{建設のために} ~するに当たって多くの障害に出くわす 
\\	するに対して ~するに当たって細心の注意を払う必要がある 
\\	するときに ~での最後の年に当たって現在非常に忙しい日々を送っている 
\\	最後の年で ~で働くに当たって何を望んでいるのか他の人々に尋ねる 
\\	働く上で ~に関する法律を作るに当たっての問題 
\\	作る上での ~の採用に当たっての将来の方向性を決定する 
\\	採用における、採用についての ~の設計に当たって補正される 
\\	設計により ~を勉強するに当たっての(人)の目標 
\\	する上での
\\	さほど~ない	
\\	ごく最近まで、先進国の人々は環境にさほど気にかけてはいなかった。 
\\	彼女は真剣なので、自室で勉強していても、さほど心配しませんでした 
\\	その当時は、偽造文書で学校に入学することがさほど難しいことではなかったのです。 
\\	通りの向かいに、さほど見栄えのしないレストランがあった。 
\\	然程 
\\	左程	あまり(ない) 特に(ない)
\\	曲りなりにもX (まがりなりにも)	
\\	"曲がりなりにも彼は原稿を書き終えた 
\\	彼は苦学して曲がりなりにも大学を出た 
\\	「まがりなりにも通じている」ということと「正しい英語を使っている」ということには雲泥[大きい]の差があります 
\\	何とか~する 
\\	(思いがけず)~してしまう 
\\	木の根につまずいてしまった!
\\	~っこ (する) 
\\	っこない	
\\	ご) 
\\	ない] 
\\	[子]) 
\\	お互いに同じ動作をすること、 ボールを投げっこする 
\\	テストの点を見せっこした 
\\	当てっこ 
\\	慣れっこ (になる) 
\\	このぐらいの寒さは慣れっこだ いたずらっ子 
\\	恨みっこ 
\\	恨みっこのないように 
\\	売れっ子 
\\	あまっこ  
\\	いじめっ子 
\\	追いかけっ子(をする) 
\\	おしっこ 
\\	抱っこ  
\\	可愛い子ぶりっこ 
\\	鬼ごっこ 
\\	ーごっこ 
\\	お飯事=
\\	比べっ子 (競争) 
\\	誰かが一番早く漢字を書けるか比べっこしよう! 潜りっこ 
\\	にらめっこ 
\\	ひぞっこ 
\\	っこない 
\\	っこい 
\\	の/
\\	あまり (あんまり/あんまりにも) 
\\	私はその晩興奮のあまり寝られなかった 
\\	この教科書は文法を重視するあまり面白くないものになってしまった 
\\	秋子は恐怖のあまり声も出なかった。 
\\	私は喜びのあまり思わず隣の人に抱きついてしまった。 
\\	日本の英語教育は文法が強調されるあまり会話力の養成がおろそかになっているようだ 
\\	今度の会合は形式を重んずるあまり内容が乏しくなってしまった。 
\\	子供を失った両親は、悲しみの余り気も狂わんばかりだった 
\\	ので 
\\	ために: その教科書はあまり文法を重視したので(/ために)面白くないものになった。 
\\	あまりにも 
\\	彼の話はあまりにも馬鹿げていたので誰も信じなかった。
\\	会社の経営があまりにも順風満帆で、かえって恐いぐらいだ。
\\	あまる 
\\	のあまり 
\\	悲しみのあまり涙も出なかった 
\\	悲しいあまり) 心配のあまり食事ものどを通らない様子だった 
\\	心配なあまり)	
\\	すぎる 
\\	私はその番興奮しすぎて寝られなかった。 
\\	ビールを飲み余り頭が痛い
\\	の・である}かぎり 
\\	のかぎり、Xでないかぎり) 出来る限り/知ってる限り
\\	力の続くかぎり漕いだ. 
\\	アメリカ人でない限り、この仕事には就けない 
\\	事態が変わらない限り、今以上の援助は不可能だ。 
\\	できる限りやっています 
\\	私は知っている限り彼は正直者です。 
\\	この事件に関する限り、彼は無実だ。 
\\	間は、内は 
\\	先生はここにいっらしゃる{限り/うちは/間は}心配は無用です。 
\\	{面白い 
\\	便利な}{うちは 
\\	間は}続けるつもりだ。 
\\	かぎり 
\\	助詞)	
\\	今度限りで彼のパーティーには行かないつもりだ 
\\	切符は一人二枚限りです。 
\\	貸し出しは一回三冊限りです。 
\\	セールは明日かぎりだ。 
\\	限り 
\\	限り 
\\	~からして 
\\	彼の下品な言い方からして、すごく嫌な人です 
\\	彼女はその話し方からして気に食わないんだ 
\\	社長からして会社再建の意欲がないんだから社員の意気が上がらないのは当たり前だ 
\\	プロの銀行員からして初歩的な経済の分析に必要な数学に弱い 
\\	(あの男が自分の子供
\\	からして/でさえ/ですら/殴るよ) 
\\	からして 
\\	彼の表情からして、何か心配事があるとわかった 
\\	(から見て、
\\	から言って)	
\\	から見て 
\\	から言って
\\	て 
\\	で)かなわない ~て適わない ~には適わない	
\\	その試験は難しくてかなわなかった 
\\	隣の大喧嘩はうるさくてかなわない 
\\	私は退屈でかなわない 
\\	僕は数学では彼にかなわないことがよくわかった。 
\\	料理ということになれば、誰もメアリーにはかなわない。 
\\	知力では誰も彼にはかなわない。 
\\	テニスでは、彼は私にかなわない。 
\\	て 
\\	~てたまらない 
\\	~に匹敵 する~はない 
\\	シェークスピアに匹敵する劇作家はいない。 
\\	て)たまるか 
\\	堪まる 
\\	て)たまるも[の/ん]か 
\\	て)たまるも[の/ん]じゃない	
\\	負けてたまるか 
\\	あいつを許してたまるか 
\\	幽霊なんかいてたまるか 
\\	そんな事があってたまるものか 
\\	このくらいでへこたれてたまるもんか 
\\	たまったもんじゃないね 
\\	そんな事をされてはたまったもんじゃない 
\\	っこない 
\\	堪らない 
\\	くもなんともない(よ 
\\	(なんか)なんともない		彼はかっこよくもなんともないよ 
\\	その知らせなんかなんともない 
\\	その雑誌はおもしろくもなんともない 
\\	[~く]もクソもない。	
\end{CJK}
\end{document}