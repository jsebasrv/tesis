\documentclass[8pt]{extreport} 
\usepackage{hyperref}
\usepackage{CJKutf8}
\begin{document}
\begin{CJK}{UTF8}{min}
\\	左千夫先生への追憶
\\	石原純
\\	【テキスト中に現れる記号について】
\\	ルビ
\\	(例)左千夫《さちお》
\\	ルビの付く文字列の始まりを特定する記号
\\	(例)茶室|唯真閣《ゆいしんかく》
\\	入力者注 主に外字の説明や、傍点の位置の指定
\\	(例)[#ここから1字下げ]
\\	左千夫《さちお》先生のことを憶《おも》うと、私にはいかにも懐かしい気分が湧《わ》いてくる。あの大きな肥《ふと》った身体《からだ》、そしてみなりなどにかまわない素朴な態度、その平淡ななかに言い知れぬ深いところを湛《たた》えて我々に接せられたことなどに対し、私はどんなに懐かしさを感じているかわからないほどである。
\\	「馬酔木《あしび》」がはじめて発刊せられたのは明治三十六年のことであった。それ以前から根岸派《ねぎしは》の歌に親しんでいた私はこれを嬉しく思いながら、先生のことを想像していた。その時分は大学の学生であったが、まだ見知らぬ人をいきなり尋《たず》ねて行ってよいかどうかを思いまどいながら数箇月を過ごしてしまった。そのうちに毎月の歌会が先生の宅で開かれるようになったので、この年の秋過ぐるころに、私ははじめてその歌会の日に訪ねて行った。牛乳屋の硝子《ガラス》戸のはまった入口のかたわらに、少し奥まったところに格子戸《こうしど》の玄関が別にあった。そこで案内を乞いながら私ははじめて先生のあの懐かしい面に接したのであった。この折りに見た炉《ろ》をきった座敷や、愛蔵せられていた茶釜や、無一塵《むいちじん》の額面や、それらは今でも私の眼前にちらついて見えるようである。そして先生のおもかげと結びついて私の脳裡《のうり》に消されずにのこっている。
\\	本所《ほんじょ》茅場町《かやばちょう》の先生の家は、もう町はずれの寂しいところであった。庭さきの墻《かき》の外にはひろい蓮沼《はすぬま》があって、夏ごろは蛙《かわず》が喧《やか》ましいように鳴いていた。五位鷺《ごいさぎ》や葭切《よしき》りのなく声などもよく聞いた。そこで牛を飼っていながら、茶を楽しみ、歌や文学や絵画を論じていられた先生は、実に高尚な趣味に徹した人であった。雑然たる都会のなかに住んでいた私には、暇を見つけては先生のもとに行って、その閑寂《かんせき》な趣味のなかに浸《ひた》ることのできるのを、この上なく嬉しく思ったことであった。いつもあまりながく話して、知らない間に夜をふかしてしまうこともしばしばあった。まだ電車などまるでなかったころであったから、本郷《ほんごう》の家まで帰るのに、もうひっそりと寝しずまった町々を歩いて来たのであったが、時々はあまりに遅い時間になってしまって、そのまま泊《と》めていただいたことなどもかなりにあった。
\\	趣味に徹していた先生は、そうであるからと言って趣味に溺《おぼ》れる人では決してなかった。閑寂をもとめ平淡を愛しながら、なお決して世を離れるような退嬰《たいえい》的な態度をとらしめるに至らなかった所以《ゆえん》はここにあると私は思う。あれほど淡雅な趣味を好んでいた先生が、他面においてはなはだ進取的な若々しい気分や、執拗《しつよう》な強い自信をもって、実世間につき進んでゆかれたことなど思うと、むしろ不思議なほどである。この性格において私は先生の偉大さを切実に認めるとともに、そこに少しの厭味《いやみ》をも伴うことなく、どこまでも懐かしさを感ぜしめることを、まことに貴とくも思うのである。
\\	歌論に対する先生の自信はおそらくすべての人々が異常な感をもってそれに対したほどであった。先生のこころにはそれが絶対のものであったので、当時世間でもてはやされていた歌などには、まるでその価値を認めずに罵倒《ばとう》された。その議論に熱烈であったことはまことに驚くべきほどである。私はあぶらぎって肥《こ》えていた先生の体格が、この強い確信を燃えたたしめる素質となっていたのだと思っている。正岡子規子《まさおかしきし》の没後、先生がひとりその門弟のなかに抽《ぬき》んでて、根岸派歌会の中心となってそれを背負ってゆかれたことも、年齢などの関係もあったには違いないが、また主としてこの強味をもたれていたからであると思う。中年になってから、あれだけの小説を書かれたのも、やはり同様の性格に基づくもので、そのころの小説に対する自信もかなりの程度のものであった。
\\	先生が我々よりも二十年も年上でありながら若い気分をもっていられたことは随所に見られた。本当に友だちのように我々を遇せられていた。歌会のときなど、席上の歌作に苦しんでいると、いつも先生は元気な声で、「そんなことではだめだ、僕はもう数首できたよ」と言っては、我々を励まされた。また私が大学で物理学を専攻していたので、先生はよく物質の分子とか電子とかラジウムとか、それから地球や天体のことなどを、非常な興味をもって私に尋ねられるのであった。そしてそれらのふしぎな現象をいろいろと心に描きながら、自然の幽幻なありさまや、人間の知識の究極するところの深さに感嘆しておられた。これらのことは、一面には先生が近代教育を受けない素朴な性質をもっておられたことにもよるが、それでありながら先生が熱心にこのような知識を解しようとせられたところに、実に若々しい進取的な気質を私は観取しないわけにはゆかないのであった。
\\	自然に対する驚異、それは本当に敬虔《けいけん》な心から生まれる。なまなかの学問をしたものはかえってそういう心を失って、自分の浅薄な知識にたよりたがるのである。先生にはそういうことが絶対になかったので、最も深く自然を愛し、これを讃美せられた。明治四十三年五月にかの有名なハリー彗星《すいせい》が太陽に近づき、遠くその尾をひいて、それがわが地球にも触れると言われたとき、先生はちょうどその折りにできあがった茶室|唯真閣《ゆいしんかく》に我々を待って、このまれな日の感慨を深められた。そのとき書かれた文には次の句がある。
\\	[#ここから1字下げ]
\\	(五月十九日) 七十五年ごとに現わるべき彗星のこの世界に最も近づくという日である。わが方丈の一室もようやく工を竣《お》え、この日はじめて諸友をここに会した。……十九日はもとより我々の忘るることあたわざる日である。今またこの日をもってこの会をなす。今後|予《よ》をしてさらにこの日を親しましめるであろう。予は永久に毎月この日をもってこの一室に諸友の来遊を待つことと定めた。
\\	彗星来降の実況は晴天なるにかかわらずついに何ごとをも感ずることができなかった。夜に入ってはただ月白く風|爽《さわや》かに、若葉青葉の薫《かお》りが夜気に揺《ゆ》らぐを覚《おぼ》ゆるのみである。会は実におもしろかりし楽しかりし。
\\	[#ここで字下げ終わり]
\\	ここで十九日は我々の忘るることあたわざる日であると書かれたのは、正岡子規子の命日に当たるからである。このとき我々は夜を徹するばかりに語りふけって、それから月明のふけわたった静かな街路を、何ものかの変異を心に予感しようとしながら、それぞれの家に向けて帰ったのであった。
\\	偏僻《へんぴ》なところにあった先生の家のすぐ前には、汽車の高架線があって、錦糸堀《きんしぼり》の停車場の構内になっていた。夜分静かに話にふけっていると、汽車がごうごうと通り過ぎてゆく。沼地につづいたこのあたりの軟らかい地面を揺らがして、地震のようにぐらぐらする。私はいつもの寂《さ》びた心地のなかに、急に近代的の刺戟《しげき》を感じさせられるようにも思った。しかしそれにも慣《な》れてくると、今度はかえってそれもなくてはならぬもののように平気になってしまった。先生の立てられた渋い茶を味わって、こうして我々は現代に生きていたのである。世の人たちは万葉崇拝《まんようすうはい》をいたずらに古めかしい趣味ででもあるように見なしていた。先生は万葉精神の体現はたとえ一般人には認められなくとも、それを理想とする少数の我々がここにあるということは、やはり現代思潮の一部として否定すべからざる事実であるとも言われていた。それを今思うと感慨がふかい。
\\	「馬酔木《あしび》」時代には、雑誌の編集はほとんど先生一人の仕事であった。それに対しては非常に熱心でいられたのにかかわらず、発行の遅れないときはないほどであった。きょうはぜひやってしまわなくてはならないと言いながら、訪問者でもあると、それを断わりきれずに、やはりゆっくりと茶を飲んで話していられた。先生のゆったりした、しかも愛情のみちた性格がこういうところに遺憾《いかん》なく覗《うかが》われる。第二巻、第三巻のころには印刷所が京橋にあったので、雑誌のできあがった日には、そこへ出かけて行って雑誌を自分でうけとり、それから私の本郷の寓居《ぐうきょ》へ立ちよって、一緒《いっしょ》に発送をするのを例とせられていた。
\\	真間《まま》で歌会をやって手古奈《てこな》の祠《ほこら》に詣でたことや、千葉の瀬川氏の別荘へ行って歌をつくったことや、東京湾の観艦式《かんかんしき》を見るのに川崎におもむいてそこで泊った折りのことや、多摩川《たまがわ》べりの寺内で鮎《あゆ》を賞したときのことなど、私には忘れられない記憶となって残っている。そして袴《はかま》の股《もも》だちをとって田舎道《いなかみち》を歩いてゆかれた先生の姿など眼のまえに浮かんでくる。甲州|御嶽《みたけ》の歌会には私の都合で行をともにすることのできなかったのを、今でも遺憾に思っている。
\\	明治四十五年の三月に私が欧州へ向けて留学の旅に出かける折りに、送別の会を先生のもとで開いていただいた。先生の健康な身体をそのとき限り見ることができなくなろうとは、かりにも予想し得ないことであった。翌年先生の訃報《ふほう》を私はスイスのチューリッヒで受けとったのであったが、そのとき私はそこの山腹の下宿の高い窓から、呆然《ぼうぜん》として町の向こうの青い湖水の面を見おろしながら、孤《ひと》り離れて遠い思いに浸らないわけにはゆかなかった。私はやがて故国に帰って先生に話そうと思っていたいろいろな事がらを、そのままにしなくてはならないようになってしまったことを、その時どんなに憾《うら》んだかしれない。
\\	西洋の文字を知らなかった先生は、欧州にいる私に対する手紙の宛名を書くのに、いつも斎藤君を煩《わずら》わさねばならなかったが、そういう面倒をあえてしては、いつも真情のこもった手紙をはるかに送られたことを、私はまことにありがたいと思っている。ドイツから送った私の歌に対して、「アララギ」第六巻第三号で「歌の潤《うるお》い」という歌論のもとで、大いに褒《ほ》められ、それが先生の最後に近い歌論ともなったことは、私にとってまことに感銘のふかいところである。それはこのころ斎藤君などが新らしい道に進もうとされて、先生からいくらか離れるようにも見えることを寂しく思われたのにもよることと思うが、ともかく私はこのことを忘れるわけにはゆかない。
\\	先生が逝《ゆ》かれて、もう七年も過ぎたかと思うと、今さらに年月の経つのがはやい気がする。先生がいままで達者でいられたならどんなであろうなどとも思っていると、近眼鏡を二重にかけた先生のおもかげが眼前にありありと見える気がする。
\\	[#地付き](一九一九年六月「アララギ」)
\\	底本:「随想全集 第九巻」尚学図書
\\	1969(昭和44)年11月5日発行
\\	底本の親本:「随筆集 夾竹桃」文明社
\\	1943(昭和18)年7月20日発行
\\	初出:「アララギ」アララギ発行所
\\	1919(大正8)年7月
\\	※「本所茅場町」は当時の東京市本所区茅場町、「本郷」は同じく東京市本郷区です。
\\	入力:高瀬竜一
\\	校正:フクポー
\\	2018年6月27日作成
\\	青空文庫作成ファイル:
\\	このファイルは、インターネットの図書館、青空文庫
\\	で作られました。入力、校正、制作にあたったのは、ボランティアの皆さんです。
\end{CJK}
\end{document}