\documentclass[8pt]{extreport} 
\usepackage{hyperref}
\usepackage{CJKutf8}
\begin{document}
\begin{CJK}{UTF8}{min}
\\	ジョンさんは まいにち テレビを みます。 ジョンさんは まいにち テレビを みる。	
\\	まりこさん、あした すしを たべましょう。 まりこさん、あした すしを たべよう。	
\\	ジョンさん、えいがを みましょうか。 ジョンさん、えいがを みようか。	
\\	まりこさんは あした レストランで たべる。	
\\	だいどころで ねずみを みた。	
\\	ひるごはんを いちじはんに たべた。	
\\	わたしは スポーツの テレビを みない。	
\\	ジョンさんは やさいを たべない。	
\\	あつい。	
\\	あつくない。	
\\	あつくない。	
\\	あつくなかった。	
\\	みない。	
\\	みなかった。	
\\	ジョンさんは まいにち ラーメンを たべる。	
\\	あした まりこさんは えいがを みる。	
\\	きのう スーパーマーケットで せんせいを みた。	
\\	おにいちゃんは わたしの アイスクリームを たべた!	
\\	ジョンさんは すしを たべない。	
\\	わたしと テレビを みない?	
\\	とりは ねこを みなかった。	
\\	しかし、ねこは とりを たべなかった。	
\\	まりこさんは やきとりを たべなかった?	
\\	あの えいがを みた?	
\\	あそこに ぜいかんの ひとが います。	
\\	ジョンさんは パスポートを みせる。	
\\	パスポートを みせて ください。	
\\	ジョンさんは ぜいかんの ひとに パスポートを みせた。	
\\	わあ、がいじんが たくさん いますね!	
\\	え?ここは ドイツ ですよ。ここでは、すずきさんが がいじん ですよ。	
\\	ゲートの ばんごうを しっていますか。	
\\	15ばん	
\\	ゲートは なんばん ですか。	
\\	23ばん です。	
\\	すずきさんの しゅっぱつは なんじ ですか。	
\\	ごご はちじに しゅっぱつ します。	
\\	なかたさんの とうちゃくは いつ でしたか。	
\\	きんようびに にほんに とうちゃく しました。	
\\	すみません。あんないじょは どこ ですか。	
\\	あんないの カウンターは あそこに あります。	
\\	ジョンさん、にもつを みつけた?	
\\	みつけたよ。	
\\	にもつが さんこ あります。	
\\	いしが あります。	
\\	とりが います。	
\\	まりこさんに にもつが いっこ あります。	
\\	ジョンさんに かのじょが います。	
\\	まりこさんは にもつが なんこ ありますか?	
\\	にこ あります。	
\\	まりこさんは いもうとさんが いますか。	
\\	いいえ、いません。	
\\	ああ、そう ですか。	
\\	こう しましょう。	
\\	これを しましょう。 こう しましょう。	
\\	どう しましょうか。	
\\	なにを しましょうか。 どう しましょうか。	
\\	どこに ありますか。	
\\	ここに あります。	
\\	どれ ですか。	
\\	これ です。	
\\	そう しましょう。	
\\	やっと とうちゃく しました。	
\\	ちょくせつ スーパーに いきましょうか。	
\\	まりこさんは ジョンさんを つめたく みた。	
\\	ほんとうに そう ですか。	
\\	バスが やっと きました。	
\\	ロンドンから とうきょうまで ちょくせつ とびました。	
\\	くうこうで ほんを よみました。	
\\	バスで がっこうに いきました。	
\\	まりこさんは かみのけが ちゃいろで、めが くろい。	
\\	これで いきましょう。	
\\	それで だいじょうぶ です。	
\\	これで いい ですか。	
\\	それで、どこに いきましたか。	
\\	それで、どう しましょうか。	
\\	それでは	
\\	アイスクリームを たべますか。それとも やさいを たべる?	
\\	アイスクリームを たべますか。やさいを たべますか。	
\\	ジョンさんは つかれる。	
\\	まいしゅう すいようびに 
\\	00まで しごとが あります。ほんとうに つかれます。	
\\	つかれました!	
\\	ああ、きょうは ながかった。つかれた!	
\\	すごく つかれました。	
\\	スポーツは たのしい けど つかれますね。	
\\	わたしは つかれませんよ。	
\\	まりこさんは どう ですか。つかれましたか。	
\\	つかれました。	
\\	まりこさん、にもつを みつけた?	
\\	みつけたよ。	
\\	よし!いきましょう。	
\\	わあ、やっと とうちゃく しました! ひこうきは ながかったね!	
\\	そう ですね。ひこうきを でて、にもつを みつけて、とうちゃく ロビーに でた。つかれた!	
\\	ほんとうに つかれましたね。それで、どう しましょうか。ちょくせつ ホテルに いきましょうか。それとも ここで たべようか。	
\\	たべたいね。ここに レストランが たくさん あります。ここで ちょっと たべて いきましょう。	
\\	わたしも たべたいね。そう しましょう。	
\\	よし!おいしい レストランを みつけましょう。	
\\	まりこさん、ばんごはんを たべた?	
\\	たべた。	
\\	ジョンさん、つかれた?	
\\	つかれた。	
\\	つかれた。	
\\	つかれたね!	
\\	くうこう	
\\	なりた  くうこう	
\\	ひこうき	
\\	ターミナル	
\\	ゲート	
\\	パスポート	
\\	ぜいかん	
\\	みせる	
\\	がいこくじん	
\\	がいじん	
\\	ばんごう	
\\	~ばん	
\\	なんばん	
\\	わあ	
\\	え?	
\\	とうちゃく	
\\	しゅっぱつ	
\\	とうちゃく  する	
\\	しゅっぱつ  する	
\\	あんない	
\\	あんないじょ	
\\	カウンター	
\\	ホテル	
\\	ロビー	
\\	にもつ	
\\	みつける	
\\	いりぐち	
\\	でぐち	
\\	エレベーター	
\\	エスカレーター	
\\	こう	
\\	そう	
\\	ああ	
\\	どう	
\\	やっと	
\\	ちょくせつ	
\\	だいじょうぶ	
\\	それで	
\\	それとも	
\\	だめ	
\\	よし!	
\\	つかれる	
\\	ジョンさんは 水を のんでいます。	
\\	りんごが 一つ あります。	
\\	こんげつは 一月 です。	
\\	一時 です。	
\\	一がつ	
\\	一月のパーティーは、一時から です。	
\\	オレンジを 二つ ください。	
\\	二月に おきなわに いきたい です。	
\\	ケーキを 三つ たべました。	
\\	あそこに せんせいが 三人 います。	
\\	一、二、三!	
\\	この ほんを よもう。	
\\	おかあさんに てがみを かこう。	
\\	その かみに わたしの なまえを かこう か。	
\\	はなこちゃんは まいにち ほんを よむ。	
\\	すずきさんは ゆうこさんと はなさない。	
\\	にほんごが わからない。	
\\	かわで およがなかった。	
\\	この すうがくが わからなかった。	
\\	ジョンさんは てがみに なまえを かかなかった。	
\\	まりこさんは カナダに とばない。	
\\	かわない	
\\	かわなかった	
\\	わかる	
\\	ふる	
\\	おる	
\\	わたしは えいごが わからない。	
\\	ジャパニーズ ハズ ア ロット オブ 「カタカナ イングリッシュ」。	
\\	チェックイン	
\\	チェックアウト	
\\	フォーク	
\\	フィード	
\\	かのう ですか。	
\\	チェックインは なんじ から かのう ですか。	
\\	から かのう です。	
\\	チェックアウトは なんじ ですか。	
\\	チェックアウト したい です。	
\\	キャンセルの ポリシーは なん ですか。	
\\	わたしの よやくを キャンセル したい です。	
\\	わたしの よやくを かくにん したい です。	
\\	412ごうしつ	
\\	102ごうしつ	
\\	チェックアウト したい ですが。	
\\	チェックアウト したい ですが、いい ですか。	
\\	よやくを かくにん したい ですが。	
\\	レストランは 二階に ある。	
\\	三階に なにが ありますか。	
\\	ジョンさんは いま よんかいに いる。	
\\	この ビルは 18かいだて です。	
\\	ソニーの ビルは なんかいだて ですか。	
\\	はちじゅうごかいだて です。	
\\	二かいだての レストラン	
\\	はちかいだての えいがかん	
\\	ここは なんかい ですか。	
\\	じゅっかい です。	
\\	おふろは なんかいに ありますか。	
\\	一階に あります。	
\\	エレベーターで いきましょう。	
\\	すずきさんは エスカレーターで よんかいに いきました。	
\\	10かい ですよ。かいだんで いきたくない。	
\\	エレベーターに のりましょう。	
\\	エレベーターに のって、7かいに いきました。	
\\	エスカレーターを つかいましょうか。	
\\	かいだんを つかって 312ごうしつに いきました。	
\\	この ホテルは、一泊 いくら ですか。	
\\	一泊 8000えん です。	
\\	二泊の りょこう	
\\	らいげつ、ななはくの りょこうに いきます。	
\\	なかたさんは 十一泊の りょこうに いきました。	
\\	ななはく ようか の りょこう	
\\	この ホテルに しゅくはく したい です。	
\\	この ホテルに 10ぱく しゅくはく したい です。	
\\	この ホテルに 11がつ 15にち から 10ぱく しゅくはく したい です。	
\\	この ホテルに 11がつ 15にち かようび から 10ぱく しゅくはく したい です。	
\\	スミスさま ですか。	
\\	すみません。みずを ください。	
\\	かしこまりました。	
\\	これで よろしい ですか。	
\\	はるに なりましたね。	
\\	あさに なりました。	
\\	その ひとは べんごしに なった。	
\\	わたしは せんせいに なりたい です。	
\\	さむく なりましたね。	
\\	この へやは あつく なったね。	
\\	この まちは うつくしく なりました。	
\\	なく なる。	
\\	おかねが なく なった。	
\\	わからなく なった。	
\\	みちが わからなく なった。	
\\	これが せいきゅうしょに なります。 こちらは せいきゅうしょに なります。	
\\	プールは エレベーターの うしろに なります。	
\\	チェックアウトは 
\\	00に なります。	
\\	おへやは こちら です。	
\\	おてあらいは どちら ですか。	
\\	こちらは すずきさん です。	
\\	こんばんは。チェックイン したい ですが。	
\\	いらっしゃいませ。よやくは ありますか。	
\\	はい、あります。なまえは スミス です。	
\\	かしこまりました。しょうしょう おまちください。はい、スミスさま ですね。	
\\	よやくは ろっぱくの しゅくはく ですね。チェックアウトは らいしゅうの げつようびに なります。りょうきんは 一泊 9000えん です。こちらで よろしい ですか。	
\\	はい、それで だいじょうぶ です。	
\\	ありがとうございます。おへやは 405ごうしつに なります。4かいに あります。エレベーターは あそこに あります。どうぞ、おたのしみください。	
\\	つくえの うえに、でんわが 四つ ありました。	
\\	四月に にほんに いきます。	
\\	いもうとは 四歳 です。	
\\	かわの なかに、いしが 五つ ありました。	
\\	五時から しごとが あります。	
\\	へやに いすが 六つ ある。	
\\	六月に えいがを たくさん みた。	
\\	きが 六本 ある。	
\\	えいがを 一本 みた。	
\\	一分 まって ください。	
\\	ホテル	
\\	りょかん	
\\	おふろ	
\\	おんせん	
\\	ゆかた	
\\	フロント	
\\	ロビー	
\\	チェックイン	
\\	チェックアウト	
\\	キャンセル	
\\	ポリシー	
\\	プール	
\\	よやく	
\\	かのう	
\\	かくにん	
\\	りょうきん	
\\	せいきゅうしょ	
\\	まち	
\\	~ごうしつ	
\\	ゼロ	
\\	まる	
\\	~かい	
\\	~だて	
\\	ビル	
\\	いっかい  (一階)	
\\	にかい  (二階)	
\\	さんかい  (三階)	
\\	よんかい	
\\	ごかい	
\\	ろっかい	
\\	ななかい	
\\	はちかい	
\\	きゅうかい	
\\	じゅっかい	
\\	なんかい	
\\	かいだん	
\\	のる	
\\	つかう	
\\	~はく	
\\	いっぱく  (一泊)	
\\	にはく  (二泊)	
\\	さんはく  (三泊)	
\\	よんはく	
\\	ごはく	
\\	ろくはく	
\\	ななはく	
\\	はちはく	
\\	きゅうはく	
\\	じゅっぱく	
\\	なんはく	
\\	しゅくはく	
\\	りょこう	
\\	~さま	
\\	いらっしゃいませ	
\\	かしこまる	
\\	しょうしょう	
\\	おまち  ください	
\\	たのしむ	
\\	なる	
\\	よろしい	
\\	どうぞ	
\\	らいねん、イタリアに いきます。 らいねん、イタリアへ いきます。	
\\	ちゅうごくへ いく。	
\\	わたしは にほんへ いきました。	
\\	おじいさんが えきへ あるいた。	
\\	かれは きたへ あるいた。	
\\	二人は ひがしへ はしった。	
\\	そのひとは、ごぜん 
\\	00まで にしへ うんてん した。	
\\	みち	
\\	コンクリートの みち	
\\	とうきょうの みち	
\\	とうきょうへの みち	
\\	おおさかへの びん	
\\	おかあさんへの プレゼント	
\\	まりこさんへの てがみ	
\\	まりこさんへ てがみを かいた。 てがみを まりこさんへ かいた。	
\\	[まりこさんへの てがみ] です。	
\\	シアトルへの びんに のりましょう。	
\\	おとうさんへの プレゼントを かった?	
\\	これは せんせいへの てがみ です。	
\\	いぬは いもを 七つ たべた。	
\\	いとこは らいねん、七歳に なる。	
\\	らいげつは 七月 です。	
\\	わたしの 一番 すきな えいがは「七人の さむらい」です。	
\\	きのう、ジョンさんは みかんを 八つ たべた。	
\\	八月が 一番 すきだ。	
\\	はなこちゃんは 八歳に なりました。	
\\	いちごを 九つ たべた。	
\\	まりこさんは 九時に かえりました。	
\\	この へやに くつが 九つ ある。	
\\	十月に マフラーを 十 かった。	
\\	きた	
\\	ひがし	
\\	みなみ	
\\	にし	
\\	おおさか	
\\	びん	
\\	プレゼント	
\\	コンクリート	
\\	かえる	
\\	まりこさんは どこに いった?	
\\	わたしと さんぽに いかない?	
\\	はやく いこう!	
\\	スーパーに いって、にんじんを かいましょう。	
\\	どう する?	
\\	きょう、なにを しようか。	
\\	きのう、いえを そうじ した。	
\\	いまから そう する。	
\\	すずきさん、らいしゅうの パーティーに くる?	
\\	あ、まりこさんが きた。	
\\	わたしの いえに こない?	
\\	ゆうべ、なぜ こなかった?	
\\	ここに また こよう。	
\\	ちょっと きて ください。	
\\	これは わたしの くるま だ。	
\\	それは ジョンさんの にもつ だった。	
\\	ほんとうに そう だった?	
\\	あしたは いい てんき だろう。	
\\	これは えんぴつで、これは ペンだ。	
\\	りんご では ありません。 りんご では ない。	
\\	ジョンさんは おかねが ありません。 ジョンさんは おかねが ない。	
\\	おかねが ある?	
\\	おかねが ない。	
\\	りんご では ない。	
\\	りんご では なかった。	
\\	そう では ありません。 そう じゃ ない。	
\\	きのう わたしは しごとに いきません でした。 きのう わたしは しごとに いかなかった。	
\\	その ひとの なまえは すずき でした。 その ひとの なまえは すずき だった。	
\\	わたしの なまえは ゆうこ です。 わたしの なまえは ゆうこ だ。	
\\	これは りんご。	
\\	あ、まりこさん、だいじょうぶ?	
\\	だいじょうぶ。ありがとう。	
\\	りんご だった。	
\\	りんご じゃない。	
\\	この ほんは みどり だけど、その ほんは ちゃいろ だ。 この ほんは みどり だけど、その ほんは ちゃいろ。	
\\	これは きれいな はな だね。	
\\	これは きれいな はな ね。	
\\	だいじょうぶ だよ。	
\\	だいじょうぶ よ。	
\\	ジョンさんは スーパーに いったよ。	
\\	きょうは あついね。	
\\	ジョンさんの にもつは ここ よ。	
\\	「ゆうこ」は いい なまえ ね。	
\\	ジョンさんは どこに いきましたか。	
\\	ジョンさんは どこに いった?	
\\	ゆうこさんは くうこうに いった?	
\\	ジョンさんは エレベーターに のった?	
\\	ホテルの チェックインは なんじ から かのう だった?	
\\	えいがを みようか?	
\\	なにを たべようか?	
\\	ホテルに いこうか?	
\\	その ぼうしは きれい だね。かう?	
\\	しゅっぱつの ゲートを みつけた?	
\\	きたの ターミナル だった?それとも みなみの ターミナル だった?	
\\	まりこさん、だいじょうぶ ですか。 まりこさん、だいじょうぶ?	
\\	まりこさん、だいじょうぶ か。	
\\	これは ジョンさんの くるま?	
\\	これは ジョンさんの くるまか?	
\\	わたしの ズボンは どこか?	
\\	わたしの ズボンは どこ?	
\\	すずきさんは どんな ひと?	
\\	すずきさんは どんな ひとか?	
\\	おなまえを かいて ください。	
\\	ピアノを れんしゅう して ください。	
\\	これを みて。	
\\	はやく きて!	
\\	また きてね。	
\\	一杯 たべてね。	
\\	たのしんでね。	
\\	ぼくは いぬが だいすき だ。	
\\	ここは ぼくの へやだ。	
\\	あたしも いきたい!	
\\	きみの にもつは どれ?	
\\	きみが だいすきだよ。	
\\	ゆうこが だいすきだよ。	
\\	おい、おれの じてんしゃは どこ?	
\\	おまえ、どこから きたか。	
\\	おまえ、おれの ビールを のんだか。	
\\	みなさん、ありがとうございました。	
\\	みなさんは どこから ですか。	
\\	みんなは どこに いった?	
\\	きょう、みんなは なにを したい?	
\\	日が でました。	
\\	あの日は たのしかったね。	
\\	青い。	
\\	青かった。	
\\	日は、その くもの うしろに ある。	
\\	本	
\\	日本	
\\	らいげつ、日本に いきます。	
\\	日のまる	
\\	あしたは 休日 です。	
\\	がっこうは 十六日から です。	
\\	七月 二十五日に フランスに いく。	
\\	ゆうこさんは 二十日に きます。	
\\	きのうは 三月 一日 でした。	
\\	きょうは 八月 二日 です。	
\\	あしたは 十月 六日 です。	
\\	きのうは 九月 七日 でした。	
\\	曜日	
\\	日曜日	
\\	日曜日は、五日ですか。それとも六日ですか。	
\\	日本では、五月 四日は 休日 です。	
\\	あしたは 何曜日ですか。	
\\	「日曜日」の「にち」を しっています。	
\\	「休日」の「じつ」を かきたい です。	
\\	月は とても きれい だね。	
\\	八月か 九月に、日本に いきたい。	
\\	先月は四月でした。今月は五月です。来月は六月です。	
\\	あしたは 月曜日 です。	
\\	火を つくりましょう。	
\\	こんしゅうの 火曜日に えいがを みようか?	
\\	あの いえは 火事に なった。	
\\	わたし	
\\	あなた	
\\	みなさん	
\\	日	
\\	日本	
\\	休日	
\\	一日	
\\	二日	
\\	三日	
\\	十日	
\\	十一日	
\\	二十日	
\\	何曜日	
\\	何日	
\\	毎日	
\\	月	
\\	一月	
\\	二月	
\\	先月	
\\	今月	
\\	来月	
\\	月曜日	
\\	火	
\\	火曜日	
\\	火事	
\\	セーターは たかかった。	
\\	かいません でした。	
\\	セーターは たかかった から かいません でした。	
\\	よしこさんは アイスクリームが すきだ。	
\\	毎日 たべます。	
\\	よしこさんは アイスクリームが すき だから 毎日 たべます。 よしこさんは アイスクリームが すき だから、毎日 たべます。	
\\	よしこさんは アイスクリームが すき ですから 毎日 たべます。	
\\	セーターは たかかった ですから かいません でした。	
\\	セーターは むらさき だった。 セーターは むらさき でした。	
\\	セーターは むらさき だったから かいません でした。 セーターは むらさき でしたから かいません でした。	
\\	来月 日本に いくから 日本語を べんきょう しています。	
\\	あした ともだちが くるから いえを そうじ した。	
\\	そのレストランの たべものは おいしくないから いきたくない。	
\\	すずきさんは いま じむしょに いませんから  あした きて ください。	
\\	日本が だいすきだ。だから、すごく いきたい。	
\\	なかたさんは インフルエンザです。ですから、きょうは じむしょに こない でしょう。	
\\	ジョンさんは ラーメンが だいすきだ。だから まいしゅう たべている。	
\\	すずきさんは 毎日 やさいを たべている。だから すごく つよい。	
\\	来月 日本に いくから。	
\\	来月 日本に いくから です。	
\\	おかねが なかったから です。	
\\	いきたくなかったから だ。	
\\	そんな えいがが すき じゃないから です。	
\\	みどりの ドレスが すき だったから だ。	
\\	すみません。水が のみたい  です。	
\\	水曜日が 一番 きらいな日 です。	
\\	みずうみの まえに、木が ある。	
\\	木の コップが すごく すき です。	
\\	木曜日に 日本に いく。	
\\	ジョンさんは お金が ない。	
\\	かわの なかに、金魚が いる。	
\\	金曜日だから、お金が ある。うれしい!	
\\	これは 金のイヤリング です。	
\\	土が いいから、ここの やさいが おいしい。	
\\	土曜日に なにを したい?	
\\	日本の 土地は とても たかい。	
\\	この 土地は せまいけど、土が いい。	
\\	から	
\\	だから	
\\	水	
\\	水曜日	
\\	水圧	
\\	木	
\\	木曜日	
\\	木材	
\\	お金	
\\	金曜日	
\\	金魚	
\\	金属	
\\	土	
\\	土曜日	
\\	土地	
\\	水曜日 まで して ください。	
\\	水曜日 までに して ください。	
\\	ジョンさんは ねて いる?	
\\	八時半に おきた。	
\\	00までに おきてね。	
\\	00までに ねて ください。	
\\	まりこさんは いつも 
\\	00に おきて いる。	
\\	はやく おきて ください。	
\\	はい、はい、おきるよ。	
\\	ねむいから おきたくないよ。	
\\	おそいね。ねむい!	
\\	おきて、かおを せっけんで あらった。	
\\	まりこさんは いま かみのけを あらって いる。	
\\	わたしは まいあさ はを みがく。	
\\	イギリスに いって、えいごを みがきたい です。	
\\	はを みがいて ください。	
\\	はみがきを して ください。	
\\	けさ シャワーを あびた。	
\\	つかれたから シャワーを あびたい。	
\\	おふろに はいりたい。	
\\	おかあさんは いま おふろに はいっている。	
\\	鈴木さんは スーパーに はいって、はなを かった。	
\\	まりこさんは かみのけを ドライヤーで かわかして いる。	
\\	ジョンさんは かおを タオルで ふいた。	
\\	テーブルの うえを タオルで ふいて ください。	
\\	てが きたない から タオルで よく ふいて ください。	
\\	きょう、けしょうを する?	
\\	きょう まりこさんは けしょうを して いない。	
\\	ジョンさんも けしょうを して いない。	
\\	ジョンさんは トイレで ひげを そって いる。	
\\	まりこさんは シャワーの なかで あしの けを そった。	
\\	この ねこの ひげが ながいね!	
\\	かみのけを くしで とかした。	
\\	まりこさんは かみのけを ブラシで とかして いる。	
\\	トイレを つかいたい。	
\\	まりこさんは かみそりを つかって、うでの けを そった。	
\\	ジョンさんは わたしの でんわを つかって いる。	
\\	こんばんの ゆうしょくは なにを たべたい?	
\\	きょうの おべんとうを つくりましょう。	
\\	あした、ジョンさんが ちょうしょくを つくる。	
\\	ジョンさん、わたしが ちょうしょくを つくる から、コーヒーを いれて ください。	
\\	鈴木さんは だいどころで ケーキを やいて います。	
\\	ジョンさんは フライパンで たまごを やいて いる。	
\\	せんせい からの てがみを やいた。	
\\	ジョンさん、しごとから なんじに かえる?	
\\	まりこさんは 
\\	00に かえった。	
\\	なにを もって いる?	
\\	おじいさんは しんぶんを もって いた。	
\\	ジョンさん、お金を もっている?	
\\	もっていない。	
\\	かばんを もって いく。	
\\	おべんとうを しごとに もって いきました。	
\\	きょう、あめが ふるから かさを もって いってね。	
\\	おかあさんは パーティーに ケーキを もって いきました。	
\\	ともだちは ほんを もって きた。	
\\	こんばんの パーティーに くる?	
\\	こんばんの パーティーに いく?	
\\	きょう しごとに かさを もっていく。	
\\	わたしの いえは 左に あります。	
\\	左側に はなこちゃんの がっこうが ある。	
\\	右へ いって ください。	
\\	右側に ある。	
\\	れいぞうこの 上に ねこが いる。	
\\	つくえの 上に しんぶんが ある。	
\\	上がる。	
\\	上がります。上がった。上がらない。	
\\	まりこさんの ねつが 上がった。	
\\	この レストランの ねだんは あしたから 上がります。	
\\	てを 上げて ください。	
\\	あそこの みせは ねだんを 上げた。	
\\	まりこさんは テニスが 上手だ。	
\\	ジョンさんは 日本語が とても 上手です。	
\\	くるまの 下に お金が  ある。	
\\	鈴木さんは 木の下に すわっていた。	
\\	まりこさんの ねつが 下がっている。	
\\	あしを テーブルから 下げて 下さい。	
\\	あした、かわを 下りたい。	
\\	二人は やまを 下った。	
\\	下がる。下げる。	
\\	うまは やまを 下る。	
\\	お金を 下さい。	
\\	土曜日に きて 下さい。	
\\	きょうは 月が 明るいね!	
\\	きのうは 明るかった。あしたは 明るくない だろう。	
\\	明日	
\\	明日は 何曜日ですか。	
\\	明日は 月曜日だけど、地下鉄で とうきょうに いく?	
\\	~までに	
\\	ねる	
\\	おきる	
\\	ねむい	
\\	いつも	
\\	せっけん	
\\	あらう	
\\	みがく	
\\	はみがき	
\\	はブラシ	
\\	シャワー	
\\	あびる	
\\	はいる	
\\	シャンプー	
\\	リンス	
\\	ドライヤー	
\\	かわかす	
\\	タオル	
\\	ふく	
\\	けしょう	
\\	けしょうを  する	
\\	ひげ	
\\	そる	
\\	かみそり	
\\	ブラシ	
\\	くし	
\\	とかす	
\\	ちょうしょく	
\\	ちゅうしょく	
\\	ゆうしょく	
\\	こんばん	
\\	べんとう	
\\	つくる	
\\	コーヒー	
\\	いれる	
\\	ケーキ	
\\	フライパン	
\\	やく	
\\	かさ	
\\	しんぶん	
\\	かばん	
\\	かえる	
\\	もつ	
\\	もって  いく	
\\	もって  くる	
\\	左	
\\	左側	
\\	左折	
\\	右	
\\	右側	
\\	右折	
\\	左右	
\\	ねだん	
\\	上	
\\	上がる	
\\	上げる	
\\	上手	
\\	下	
\\	下がる	
\\	下げる	
\\	下る	
\\	下さい	
\\	地下	
\\	地下鉄	
\\	下品	
\\	明るい	
\\	説明	
\\	説明する	
\\	すごいね!	
\\	すごいな。	
\\	明日 鈴木さんと えいがを みる。	
\\	わたしも いきたいな。	
\\	いいね!	
\\	いいな。	
\\	わたしも いきたかったな。	
\\	ジョンさんは ラッキーだな。	
\\	あそこの ビルは たかいな。	
\\	ジョンさんは すごい くるまを うんてん しているな。	
\\	まりこさんの アパートは ひろいな。	
\\	いこうか。	
\\	いこうかな。	
\\	ちょっと たべようかな。	
\\	おそいから ねようかな。	
\\	シャワーを あびようかな。	
\\	どこで たべよう かな。	
\\	どこが いいかな。	
\\	まりこさんは どこへ いった かな。	
\\	わたしの 一番 すきな かさは どこかな。	
\\	そう かな。	
\\	どう かな。	
\\	あ、これ かな。	
\\	ジョンさんの アメリカの いえは 大きい。	
\\	あの いぬは 大きくないね。	
\\	ゆうこさん、どこの 大学に いっていますか。	
\\	これは とても 大切な てがみです。	
\\	大変だ!ジョンさんは かぜを ひいた!	
\\	大変 いい 大学 です。	
\\	あの 金魚は 大変 大きい。	
\\	水の中に 金魚が いる。	
\\	真ん中	
\\	ジョンさんが かわの 真ん中に たっていた。	
\\	一日中	
\\	一日中 ほんを よんでいた。	
\\	はなこちゃんの ゆびは とても 小さい。	
\\	ゆうべの 月は 小さかったね。	
\\	わあ、ここは 田んぼが たくさん ある!	
\\	中田さんは どこ ですか。	
\\	田中さんは 田んぼを みている。	
\\	山田	
\\	田村	
\\	村田	
\\	本田	
\\	豊田	
\\	松田	
\\	鈴木さんは 力が あるね。	
\\	カタカナの 力	
\\	みんなの 協力が 大切 です。	
\\	わたしと 協力して 下さい。	
\\	な	
\\	大きい	
\\	大学	
\\	大切	
\\	大変	
\\	大丈夫	
\\	大好き	
\\	中	
\\	真ん中	
\\	中学校	
\\	中国	
\\	小さい	
\\	小学校	
\\	小説	
\\	力	
\\	水力	
\\	協力	
\\	協力する	
\\	ちょっと まって 下さい。	
\\	けさ、かおを あらって、はを みがいて、がっこうに いった。	
\\	まりこさんは めが みどり で、かみのけが ちゃいろ です。	
\\	まりこさんは ゆうしょくを たべて います。	
\\	ジョンさんは テレビを みて いた。	
\\	ジョンさんは プログラマー だ。マイクロソフトで はたらいて いる。	
\\	ジョンさんは プログラマー で、マイクロソフトで はたらいて いる。	
\\	あの えいがは へん で、えいがかんも きたなかった。	
\\	田中さんは せんせい で、こうこうで はたらいて いる。	
\\	わたしは ソニーで はたらいて いて、たのしんで いる。	
\\	あの えいがは へん で、えいがかんも きたない。	
\\	わたしは ソニーで はたらいて いる。たのしんで いる。	
\\	金曜日のあさ、べんとうを つくって、かさを もって、がっこうに いった。	
\\	ビルさんは カナダじんで、よく りょこう している。	
\\	えいこくに いって、えいごを べんきょう しました。	
\\	まりこさんには いもうとさんが いて、わたしには おとうとが いる。	
\\	ちょうしょくを たべて、しごとに いった。	
\\	みちは せまい。うるさい。	
\\	みちは せまくて うるさい。	
\\	みちは せまくて うるさかった。	
\\	かばんは くろくて 大きい。	
\\	鈴木さんの アパートは せまくて きたない。	
\\	あそこの みせの ラーメンは、やすくて おいしい。	
\\	鈴木さんの アパートは しずか です。きれい です。	
\\	鈴木さんの アパートは しずか で、きれい です。	
\\	この テストは かんたん で、みじかい。	
\\	この テストは みじかくて、かんたん だ。	
\\	まりこさんは ピアノが 上手 で、あたまが いい。	
\\	この ほんは ながくて、むずかしかった。	
\\	ジョンさんは あたまが よくて、すてき です。	
\\	田中さんは せが たかくて、かおが うつくしくて、すてき です。	
\\	中田さんは おいしゃで、毎日 おそく  まで はたらいて います。	
\\	中田さんは おいしゃ だから、毎日 おそく まで はたらいて います。	
\\	田中さんは インフルエンザ で、きょう かいしゃに こない。	
\\	ジョンさんは お金が たくさん あって、よく りょこうに いって いる。	
\\	ラーメンが すき で、よく たべる。	
\\	小説が すき で、いつも よんで いる。	
\\	しごとが たくさん あって、日曜日も じむしょに いった。	
\\	この しごとが むずかしくて、すごく つかれる。	
\\	ああ!この オレンジジュースは つめたくて おいしい。	
\\	この スープは あたたかくて おいしい。	
\\	きみの へやは ひろくて いいね。	
\\	あの おんがくが うるさくて きらい だ。	
\\	まりこさんは ほそくて、せが たかくて、うつくしいね。	
\\	ビルさんは あるいて かえった。	
\\	小さい とりは とんで いった。	
\\	へんな さかなが およいで きた。	
\\	おとこの ひとは みなみへ およいで いった。	
\\	まりこさんは はしって きた。	
\\	わたしたちは びょういんに うんてん して いった。	
\\	いぬは かえって きた。	
\\	健二さんは いすに すわって、ほんを よんでいた。	
\\	ジョンさんの ねつが 下がって きた。	
\\	ねつが 上がって いった。	
\\	まりこさんは、目が きれいだな。	
\\	一日中、目が いたかった。	
\\	あなたの 目的は なんですか。	
\\	なんの 目的で 日本に いきたい?	
\\	耳が いたい。	
\\	そとが さむいから、耳が つめたい!	
\\	中田さんは、耳が 大変 小さい。	
\\	わたしの 口の中に 水が ある。	
\\	田中さんは 目が 大きいけど、口が 小さい。	
\\	このビルの 入り口は とても 大きい。	
\\	すみません。出口は どこ ですか。	
\\	とうきょうの 人口は なんにん ですか。	
\\	ジョンさんは 手が 大きい。	
\\	鈴木さんは 空手を べんきょう している。	
\\	明日 ともだちに 手紙を かこうかな。	
\\	まりこさんは 歌手に 手紙を 手で かいた。	
\\	ながい 足が 大好きだ。	
\\	足は 大丈夫 ですか。いたくない ですか。	
\\	水が 足りない。	
\\	ゆうこさん、大丈夫?お金は 足りる?	
\\	おいしい チキンを 一杯 たべて、満足 した。	
\\	目	
\\	目的	
\\	口	
\\	入り口	
\\	出口	
\\	人口	
\\	手	
\\	空手	
\\	手紙	
\\	歌手	
\\	足	
\\	足りる	
\\	満足	
\\	満足する	
\\	でんしゃで いこう。	
\\	えきは どこ ですか。	
\\	ここは しぶや えき です。	
\\	しんじゅく ゆき	
\\	しんじゅく ゆきの でんしゃ	
\\	これは、しんじゅく ゆきの でんしゃ ですか。	
\\	この でんしゃは しんじゅく ゆき ですか。	
\\	この でんしゃは しぶや えきに とまりますか。	
\\	はい、とまりますよ。	
\\	すみません。しぶや ゆきの でんしゃは、こちらがわ ですか。	
\\	いいえ、こちらがわは 上野 ゆき です。しぶや ゆきは せんろの はんたいがわ です。	
\\	はんたいがわ ですか。わかりました。ありがとうございます。	
\\	まりこさんと ジョンさんは でんしゃに のって います。	
\\	鈴木さんは うまに のって いる。	
\\	でんしゃに のって、しごとに いきました。	
\\	はやく、でんしゃに のりましょう。	
\\	のった!	
\\	ちょっと まって よ。ジョンさんは? のった?	
\\	しまった!ジョンさんは のっていない!ホームに たっている!	
\\	ふつう でんしゃ	
\\	かいそく でんしゃ	
\\	とっきゅう でんしゃ	
\\	すみません。かいそく でんしゃは 成田 えきに とまりますか。	
\\	ふつうの コーヒーが のみたい です。	
\\	ふつうの レストランに いきたい。	
\\	二つ目	
\\	三人目	
\\	四日目	
\\	三つ目の えき	
\\	一個目の にもつ	
\\	二人目の こども	
\\	日比谷 せん	
\\	千代田 せん	
\\	すみません。あさくさに いきたい です けど・・・	
\\	あさくさ ですか。わかりました。ここから 日比谷 せんで 上野 えき まで いって 下さい。そこから ぎんざ せんに のって 下さい。あさくさは 三つ目の えき です。	
\\	アナウンスを きいた?	
\\	きょう せんせいの いえに まいります。	
\\	まもなく、でんしゃが まいります。	
\\	きいろい せんの うちがわ まで お下がりください。	
\\	まりこさん、きょうの ひる、うちに こない?	
\\	うちは シアトルに あります。	
\\	ジョンさんの おうちは どこ ですか。	
\\	うちは ラーメンを たべない。	
\\	うちは テレビを もっていない。	
\\	鈴木さんの おうちでは、まいしゅう 金曜日に ラーメンを たべる。	
\\	まもなく、八ばんせんに でんしゃが まいります。	
\\	まもなく、上野 ゆきの ふつう でんしゃが まいります。	
\\	まもなく、成田 くうこう ゆきの かいそく でんしゃが まいります。	
\\	まもなく、10ばんせんに とうきょう ゆきの とっきゅう でんしゃが まいります。きいろい せんの うちがわまで お下がりください。	
\\	まもなく、でんしゃが はっしゃ します。	
\\	ドアに ちゅうい して 下さい。 ドアに ごちゅうい 下さい。	
\\	くるまに ちゅうい して いなかった。	
\\	ごちゅうい	
\\	ごちゅうい 下さい。	
\\	まもなく、はっしゃ します。ドアが しまります。ごちゅうい 下さい。	
\\	あの人は だれ ですか。	
\\	地下鉄の 中に 人が たくさん いた。	
\\	その 大学に、日本人と、中国人と、イタリア人が いました。	
\\	いえの 中に、ともだちが 五人 いた。	
\\	中田さんの いえの 人口は、三人です。	
\\	三人	
\\	一人	
\\	二人	
\\	わたしは 人間 だ。きみは ねこ だ。	
\\	あなたは 大人ですか。	
\\	「まりこ」は、女の子の なまえ です。	
\\	明日、田中さんの 子どもが うちに くる。	
\\	真理子さんは、裕子さんの 子どもを みた。	
\\	あの こうえんに 女の人が たっている。	
\\	花子ちゃんは とても 明るい 女の子 です。	
\\	女子 女の子	
\\	女子のトイレ	
\\	でんしゃの中に、女性が 二人 のっていた。	
\\	男の子が 何人 いますか。	
\\	鈴木さんは、女の人ですか。男の人ですか。	
\\	男子のトイレは どこ ですかね。	
\\	その男性は ピアノが 上手 だった?	
\\	ちかてつ  (地下鉄)	
\\	でんしゃ	
\\	えき	
\\	きっぷ	
\\	うりば	
\\	しぶや	
\\	しんじゅく	
\\	じどう  はんばいき	
\\	えきいん	
\\	まどぐち  (窓口)	
\\	ちず	
\\	おとな  (大人)	
\\	こども	
\\	かいさつぐち  (改札口)	
\\	あさくさ	
\\	ぎんざ	
\\	ホーム	
\\	~ゆき	
\\	とまる	
\\	せんろ	
\\	はんたい	
\\	~がわ	
\\	うえの  (上野)	
\\	しまった	
\\	ふつう	
\\	かいそく	
\\	とっきゅう	
\\	~め  (~目)	
\\	せん	
\\	~せん	
\\	アナウンス	
\\	まもなく	
\\	まいる	
\\	さがる  (下がる)	
\\	うち	
\\	うちがわ	
\\	はっしゃ	
\\	はっしゃ  する	
\\	しまる	
\\	ご~	
\\	ちゅうい	
\\	人	
\\	~人	
\\	人間	
\\	人口	
\\	何人	
\\	子ども	
\\	光子	
\\	椅子	
\\	女	
\\	女性	
\\	男	
\\	男子	
\\	男性	
\\	長男	
\\	たべたい。	
\\	たべたくない。	
\\	ラーメンを たべたい。 ラーメンが たべたい。	
\\	りんごが ほしい。	
\\	やさいが ほしくない。	
\\	ビールが ほしかった。	
\\	ぎゅうにゅうが ほしくなかった。	
\\	レモネードが すき だ。レモネードが ほしい。	
\\	やさいが きらい だ。やさいが ほしくない。	
\\	わたしは 日本人の 彼女が ほしい。	
\\	あなたは アメリカ人の かれが ほしい?	
\\	先月 くるまを かった。わたしの じてんしゃが ほしい?	
\\	ほんとう?ほしい!	
\\	あかい くるまが ほしい です。	
\\	あかい くるまが ほしかった です。	
\\	ジョンさんは ほしがって いる。	
\\	真理子さんは えいがに いきたがって いる。	
\\	ジョンさんは さむがって いる。	
\\	真理子さんは うれしがって いる。	
\\	真理子さんは えいがに いきたかった。 真理子さんは えいがに いきたがって いた。	
\\	ジョンさんは りんごを ほしがって いる。	
\\	真理子さんは いい しごとを ほしがって いる。	
\\	中田さんは はやい オートバイを ほしがって いる。	
\\	ジョンさんは ラーメンを たべたがって いる。	
\\	真理子さんは ほんを よみたがって いる。	
\\	鈴木さんは えいがを みたがって いる。	
\\	たべて ほしい。	
\\	いって ほしい。	
\\	きて ほしい。	
\\	スーパーに いって ほしい。	
\\	こんばん、うちに きて ほしい。	
\\	わたしの やさいを たべて ほしい。	
\\	あの みせで 帽子を かって ほしい。	
\\	あなたの へやを そうじ して ほしい?	
\\	べんごしに なって ほしい?	
\\	びょういんに いって ほしい?	
\\	真理子さんに きて ほしい。	
\\	鈴木さんに して ほしい。	
\\	ジョンさんに りょうりを して ほしい。	
\\	いもうとに あそこに たって ほしい。	
\\	いって ほしくない。	
\\	ジョンさんに カナダに かえって ほしくない。	
\\	きて ほしくない?	
\\	パーティーに きて ほしくない?	
\\	いかないで ほしい。	
\\	手を あらわないで ほしい。	
\\	でんしゃに のらないで ほしい。	
\\	そこに たたないで ほしい。	
\\	この ほんを よまないで 下さい。	
\\	ぼくの ちゅうしょくを たべないで 下さい。	
\\	みないで ね。	
\\	いかないで。	
\\	さしみは、生の さかな ですよ。	
\\	明日のよる、生の おんがくが ききたいね。	
\\	裕子さんは 男子を 生みました。	
\\	ジョンさんは いつ 生まれた?	
\\	10月 14日に 生まれた。	
\\	あの おじいさんは 九十九さいまで 生きた。	
\\	ジョンさんは やきとりと ビールで 生きていた。	
\\	裕子さんは 大学生です。	
\\	ここには 中学生と 小学生が います。	
\\	中田さんは 大学 三年生 だ。	
\\	一生、日本語を べんきょう したい。	
\\	きょう わたしは 
\\	00まで しごとが あるから、先に たべて 下さい。	
\\	50メートル 先に トイレが ある。	
\\	出口は 100メートル 先 です。	
\\	お先に、どうぞ。	
\\	先月は 五月 でした。	
\\	先生は 手を 上げた。	
\\	先日、男性の 先生が、一日中、田んぼの中に たっていた。	
\\	私は 日本人 です。	
\\	私の 先生は 中田先生です。	
\\	君は なにを している?	
\\	ジョン君と 田中君が 金曜日に くる。	
\\	ほしい	
\\	~がる	
\\	生	
\\	生む	
\\	生まれる	
\\	生きる	
\\	~生	
\\	大学生	
\\	一年生	
\\	一生	
\\	先	
\\	先生	
\\	先日	
\\	先週	
\\	先月	
\\	メートル	
\\	私	
\\	私服	
\\	君	
\\	~君	
\\	これで いい ですか。	
\\	いい ですよ。	
\\	いっても	
\\	みても	
\\	しても	
\\	いっても いい ですか。	
\\	えいがを みても いい ですか。	
\\	テニスを しても いい ですか。	
\\	トイレを つかっても いい?	
\\	コーヒーを のんでも いい?	
\\	真理子さんの きっぷを みても いい?	
\\	シャワーを あびても いいですか。	
\\	いい ですよ。	
\\	ここで ちょうしょくを つくっても いい?	
\\	いいよ。	
\\	ねえ、れいぞうこに チキンが あるけど、たべても いい?	
\\	駄目 だよ。それは きょうの ばんごはん だ。	
\\	この くるまは ほんとうに 駄目 だ。	
\\	ああ、まずい!ぎゅうにゅうは 駄目に なった!	
\\	かれは 駄目な 人 です。	
\\	おかあさん、アイスクリームが ほしい!	
\\	駄目 だよ。いま から ばんごはんを たべる から。	
\\	でも、おかあさ~ん。	
\\	駄目!	
\\	だれが 私の チキンを たべた?!	
\\	ごめんなさい。私 です。	
\\	きょうの ばんごはんに たべたかったよ。	
\\	ごめん。	
\\	真理子さんの かばんを かりても いい?	
\\	ごめん。明日 しごとに もっていく。	
\\	おお、このほんが よみたかった。かりても いい?	
\\	その ほん ですか。ごめんなさい 。私の ではない。	
\\	明日 真理子さんの くるまで いっても いい ですか。	
\\	かまいませんよ。	
\\	この ベルトは きれい だね。かりても いい?	
\\	うん、かまわないよ。	
\\	ジョンさん、先週 ジョンさんの ほんを かりた けど、すごく おもしろかった から、私の ともだちにも よんで ほしい。その人に かしても いい?	
\\	うん、かしても かまわないよ。	
\\	はいって いい?	
\\	ここに すわって いい?	
\\	ともだちと バスケットボールを して いい?	
\\	明日 いかなくても いい?	
\\	こんしゅう、くるまを あらわなくても いい ですか。	
\\	けさ、ひげを そらなくても いい かな。	
\\	私の かばんを もって いかなくても いい ですか。	
\\	ここで とまらなくても いい。	
\\	私を またなくても いいよ。	
\\	大きい	
\\	赤い。	
\\	赤くない。	
\\	ジョンさんの 目が 赤かった。	
\\	この赤ちゃんは、先月 生まれた。	
\\	赤十字	
\\	あそこの 田んぼは 青いね。	
\\	その 女子の 目が 青かった。	
\\	なにいろが 一番 すき?	
\\	赤と 青が すき だな。	
\\	白い くもが そらに 一杯 ある。	
\\	月が 白くて、大きくて、きれい だった。	
\\	火曜日に、スーパーで 米を かって 下さい。	
\\	お米が 足りない!	
\\	田中君の 子どもは 米国で 生まれた。	
\\	亜米利加	
\\	日本	
\\	韓国	
\\	中国	
\\	米国	
\\	いい	
\\	だめ  (駄目)	
\\	~ても	
\\	~ても  いい	
\\	ねえ	
\\	かりる	
\\	かす	
\\	おお	
\\	かまう	
\\	おもしろい	
\\	赤い	
\\	赤ちゃん	
\\	青い	
\\	青年	
\\	白い	
\\	白人	
\\	白鳥	
\\	米	
\\	白米	
\\	鈴木さんと 裕子さんは 一緒の がっこうに いった。	
\\	これと それは 一緒の ねだん です。	
\\	真理子さんと ジョンさんは 一緒 です。	
\\	一緒に えいがを みようか?	
\\	私と 一緒に いきませんか?	
\\	私と 一緒に びじゅつかんに いきませんか?	
\\	一緒に いきたい?	
\\	つかれたね!一緒に おちゃでも のまない?	
\\	えいがでも みない?	
\\	ケーキでも たべる?	
\\	ほんでも よみたい?	
\\	この へんに えいがかんが ありますか。	
\\	ぎんこうの あたり	
\\	ぎんこうの あたりに レストランが ありますか。	
\\	この ちかく	
\\	この ちかくに トイレが ありますか。	
\\	この ちかくに いい レストランが ある。	
\\	えきの ちかくに ゆうびんきょくが ある。	
\\	ぼくの がっこうの ちかくに、カラオケの みせが ある。	
\\	いい ところを しっているよ。	
\\	ここは いい ところ だね。	
\\	真理子さんは たぶん、いま いえに いる。	
\\	私は たぶん、らいねん 日本に いく。	
\\	一日中 かんこう したね。ちょっと、コーヒーでも のみますか。	
\\	いいね。この へんに きっさてんでも ある?	
\\	うん、この ちかくに いい ところが あるよ。たぶん・・・この あたり かな・・・あ、みつけた。あそこの みせだ。	
\\	すてきな ところ じゃない!	
\\	むかしから、ともだちと この きっさてんに きているよ。	
\\	じゃあ、はいりましょう。	
\\	すてきな ドレス じゃない ですか!	
\\	いい ところ じゃない!	
\\	しかし、面白かった じゃない!	
\\	わあ、すごい くるま じゃん!	
\\	すごい じゃん。	
\\	あの 人は すごく 人気 です。	
\\	もちろん いきたいよ。	
\\	この ビールは うまい!	
\\	真理子さんは やきゅうが うまい。	
\\	ジョンさんは ピアノが うまい。	
\\	私だけ です。	
\\	これは 水だけ だ。	
\\	ちょっと だけ たべようかな。	
\\	私は コーヒーに します。	
\\	真理子さんは おちゃに しました。	
\\	鈴木さんは、ビールに する?ワインに する?	
\\	どれに しますか。	
\\	どれに しよう かな。	
\\	ここは なにが 人気 ですか?	
\\	コーヒーは もちろん おいしい けど、ここの りょくちゃも すごく うまい。	
\\	そうですか。じゃあ、りょくちゃに しようかな。	
\\	私も りょくちゃに する。たべものは?	
\\	ちょっと だけ たべようかな。	
\\	私も ちょっと だけ たべたい。なにかを シェア しようか。	
\\	いいね。あの ドーナツに きょうみが あるな。それに して いい?	
\\	いいよ。じゃ、ちゅうもんするね。	
\\	ドーナツを 一個 たべたい。	
\\	土曜日には この みせは すごく こむ。	
\\	みちは こんでいるね。	
\\	みせは なんじに あく?	
\\	この せきは あいて いますか。	
\\	こんど、私の ばん です。	
\\	こんどの 木曜日に 一緒に とうきょうに いかない?	
\\	パーティーは こんどの 日曜日 です。	
\\	こんどは、てんぷらに しよう かな。	
\\	ああ、うまかった。でも こんどは やきとりを たべたい。	
\\	みせは こんでいるね。	
\\	うん、テーブルが ないね。	
\\	あ、あった。うしろに 一つ だけ あいている。	
\\	しまった。こんどは テーブルが あるけど、椅子が ない。	
\\	ちょっと まってね。あそこの テーブルから 椅子を かりて くる。	
\\	すみません。この 二つの 椅子を とっても いいですか。	
\\	あ、いいですよ。どうぞ。	
\\	ありがとうございます。	
\\	ちょっと まってね。手を あらって くる。	
\\	おかあさん、こんばん、ともだちと たべて くる。	
\\	ちょっと、トイレに いって くるね。	
\\	日本に 白い 山が たくさん ある。	
\\	富士山は とても きれいだ。	
\\	エベレスト山は すごく たかい。	
\\	真理子さんは 川で およいでいる。	
\\	子どもたちが 森の中で あそんでいた。	
\\	私は、日本の 青い森が すごく すき です。	
\\	青森	
\\	森田	
\\	山田	
\\	青森さんと 森田先生と 山田君は、山の中の 森で、ハイキングを した。	
\\	明日、雨が ふるかな。	
\\	雨が ふっているから コーヒーでも のみましょうか。	
\\	森の中の 花と木が きれい だった。	
\\	アメリカでは、七月 四日に、花火が 上がる。	
\\	いっしょ  (一緒)	
\\	~でも	
\\	おちゃ	
\\	へん	
\\	ちかい	
\\	ちかく	
\\	ところ	
\\	かんこう	
\\	かんこう  する	
\\	たぶん	
\\	むかし	
\\	にんき  (人気)	
\\	もちろん	
\\	うまい	
\\	だけ	
\\	~に  する	
\\	ドーナツ	
\\	なにか	
\\	シェア  する	
\\	ちゅうもん	
\\	ちゅうもん  する	
\\	こむ	
\\	あく	
\\	あった!	
\\	せき	
\\	こんど	
\\	ばん	
\\	とる	
\\	山	
\\	~山	
\\	川	
\\	森	
\\	雨	
\\	梅雨	
\\	花	
\\	花火	
\\	花瓶	
\\	鈴木さんは くるまを とめる。	
\\	くるまが とまる。	
\\	先生は クラスを はじめる。	
\\	クラスが はじまる。	
\\	とまった。	
\\	とめた。	
\\	レストランの ねだんが 上がった。	
\\	レストランは ねだんを 上げた。	
\\	手を 上げて 下さい。	
\\	ねつが 下がりました。	
\\	でんしゃを とめて 下さい。	
\\	でんしゃが とまった。	
\\	あそこの まどを あけて 下さい。	
\\	ジョンさんは ドアを あけました。	
\\	ドアが あいた。	
\\	みせは、なんじから あきますか。	
\\	このみせは、ごご 
\\	00に しまります。	
\\	いま まどを しめた。	
\\	ジョンさんは コーヒーに ぎゅうにゅうを いれた。	
\\	かばんを くるまの 中に いれて 下さい。	
\\	でんしゃは えきに はいりました。	
\\	二人は 森に はいった。	
\\	いえに はいって 下さい。	
\\	ごみを だして 下さい。	
\\	裕子さん、いぬを そとに だして 下さい。	
\\	ジョンさんは くるまの 中に いた。まどから かさを だした。	
\\	ジョンさんは くるまの そとに いた。くるまから かさを だした。	
\\	ねこは そとに でた。	
\\	ねこは トイレから だいどころに でた。	
\\	ねこは だいどころに でた。	
\\	ねこは そとに でて いった。	
\\	ねこは そとに でて きた。	
\\	明日 八時に いえを でましょう。	
\\	うれしい!日が でた!	
\\	その えいがは、らいしゅう でる。	
\\	先月 真理子さんは 大学を でた。	
\\	行く。	
\\	明日、山に 行きたい。	
\\	先月、とうきょう 大学に 行った。	
\\	私と 一緒に 川に 行かない?	
\\	まもなく、成田行きの ふつう でんしゃが まいります。	
\\	裕子さんは 銀行で はたらいている。	
\\	森田君の 行動は とても いい。	
\\	花火を 見た?	
\\	先生は 田中君の 行動を 見ていた。	
\\	ジョンさんは 真理子さんに 花を 見せた。	
\\	それを 見せて 下さい。	
\\	出す。	
\\	出る。	
\\	ごみを 出して 下さい。	
\\	ねこを そとに 出そうか。	
\\	日が 出た。	
\\	鈴木君は みせから 出て きました。	
\\	来月、面白い えいがが 出る。	
\\	この 水を れいぞうこに 入れても いい ですか。	
\\	お米を くるまの 中に 入れた。	
\\	田中さんは 森に 入りました。	
\\	花を 花瓶に 入れて 下さい。	
\\	すみません。入っても いい ですか。	
\\	はい、入って 下さい。	
\\	なまえを 入力して 下さい。	
\\	コンピューターは 田中さんの なまえを カタカナで 出力した。	
\\	分かりました。	
\\	この すうがくが 分からない。	
\\	日本語が 分かりますか。	
\\	5分 まって 下さい。	
\\	真理子さんは 一分 まちました。	
\\	真理子さんは、自分で その しごとを しました。	
\\	ジョンさんは 自分の 足を 見ている。	
\\	私は 自分の お金を もっている。	
\\	自分の べんとうを もって きた。	
\\	とめる	
\\	とまる	
\\	はじめる	
\\	はじまる	
\\	あげる  (上げる)	
\\	あがる  (上がる)	
\\	さげる  (下げる)	
\\	さがる  (下がる)	
\\	あける	
\\	あく	
\\	しめる	
\\	しまる	
\\	いれる	
\\	はいる	
\\	だす	
\\	でる	
\\	行く	
\\	飛行機	
\\	銀行	
\\	行動	
\\	見る	
\\	見せる	
\\	見学	
\\	出す	
\\	出る	
\\	出口	
\\	出力	
\\	出力する	
\\	出発	
\\	入れる	
\\	入る	
\\	入り口	
\\	入力	
\\	分かる	
\\	~分	
\\	自分	
\\	よんだ ことが ある。	
\\	行った ことが あります。	
\\	見た ことが ある?	
\\	私は [くるまが] ある。	
\\	私は [うんてん した ことが] ある。	
\\	その ほんを よんだ ことが ある。	
\\	日本に 行った ことが あります。	
\\	この えいがを 見た ことが ありますか?	
\\	日本に 行った ことが ない。	
\\	やきとりを たべた ことが ありません。	
\\	真理子さんは テニスを した ことが ない。	
\\	私の くるまを 見た ことが ない?	
\\	真理子さんは、うちに きた ことが ありませんか。	
\\	ジョンさん、とうきょうに 行った ことが ありますか。	
\\	ありますよ。 あるよ。	
\\	日本の ラーメンを たべた ことが ある?	
\\	でんしゃに のった ことが ない?	
\\	ない。	
\\	飛行機で とんだ ことが ありませんか。	
\\	ありますよ。	
\\	日本の でんしゃに のった こと、ある?	
\\	日本の おふろに 入った こと、ない!	
\\	この コーヒー、うまいね!	
\\	ぼくの きっぷ、どこ かな。	
\\	私、日本に 行ったこと、ない。	
\\	なにを 見ている? なにを 見ていますか。 なにを 見てる? なにを 見てますか。	
\\	えいがを 見てた。 えいがを 見てました。	
\\	ほんを よんでない。 ほんを よんでません。	
\\	川の 中に 大きい 石が ある。	
\\	石の 上に 男性が 一人 すわっている。	
\\	石田	
\\	石川	
\\	石山	
\\	日本に 行きたい!	
\\	本が 大好き です。	
\\	この くるまは 本田ですか。それとも 豊田ですか。	
\\	ビールを 三本 下さい!	
\\	私は、ハワイに 行ったことが あるけど、アメリカの 本土に 行ったことが ない。	
\\	スープの もとを かって 下さい。	
\\	足下に ごちゅうい 下さい。	
\\	足元	
\\	山本	
\\	石本	
\\	松本	
\\	これは 正しい ですか。	
\\	いいえ、正しくない。	
\\	そらが 白い雲で 一杯 です。	
\\	雲が あるから、雨が ふる でしょう。	
\\	この本は、900円 です。	
\\	1000円さつが テーブルの 上に ある。	
\\	円形の お金は ふつうだけど、円形の ビルは 面白いね。	
\\	石	
\\	石炭	
\\	磁石	
\\	本	
\\	本土	
\\	日本	
\\	山本	
\\	正しい	
\\	正月	
\\	正直	
\\	訂正	
\\	雲	
\\	星雲	
\\	~円	
\\	円形	
\\	あたためますか。	
\\	大丈夫 です。	
\\	はい、おねがい します。	
\\	いらっしゃいませ。	
\\	これで おねがい します。	
\\	おにぎり、一つ。かんコーヒー、一本。カレーパン、一つ。	
\\	以上で よろしい ですか。	
\\	おかいけいは 768円に なります。	
\\	1000円を おあずかり します。	
\\	おつりは 232円に なります。	
\\	ありがとうございました。また おこしください。	
\\	おはようございます。これで おねがい します。	
\\	かんコーヒー、一本。カレーライス、一つ。おにぎり、2こ。以上で よろしい ですか。	
\\	はい、大丈夫 です。	
\\	おかいけいは 1170円に なります。	
\\	(ジョンは レジの 人に 1000円さつを 二まい わたす。)	
\\	2000円を おあずかり します。・・・おつりは 830円に なります。	
\\	(レジの 人は ジョンに おつりを わたす。)	
\\	ありがとうございます。また おこしください。	
\\	私は フランスの 王です。	
\\	鈴木君は、自分の いえでは 王子だ。	
\\	川の音が すきだ。	
\\	いま、音が した。	
\\	ラジオの 音が する。	
\\	ジョンさんの へやから、ラジオの 音が する。	
\\	そとで、雨の音が していた。	
\\	ジョンさん、何を 見てる?	
\\	何月 ですか。	
\\	明日の パーティーに 何人 きますか。	
\\	真理子さんは ことしの 夏、中国に 行く。	
\\	ここは 私の家です。	
\\	ジョンさんの お家は どこ ですか。	
\\	私の 家族は アメリカに いる。	
\\	田中家は みんなが 面白いね!	
\\	夏も 冬も すきです。	
\\	冬の 何月に 森に 行きましょうか。	
\\	くまは、何月から 冬眠に 入りますか。	
\\	ここの くまは 十一月から 四月まで 冬眠 している。	
\\	おにぎり	
\\	のり	
\\	うめぼし	
\\	しゃけ	
\\	ツナマヨ	
\\	ぎゅうにく	
\\	べんとう	
\\	カレーライス	
\\	あたためる	
\\	おねがい  する	
\\	かんコーヒー	
\\	こうちゃ	
\\	ガム	
\\	おかし  (お菓子)	
\\	かしパン  (菓子パン)	
\\	カレーパン	
\\	レジ	
\\	まんが	
\\	ざっし	
\\	いじょう  (以上)	
\\	かいけい	
\\	あずかる	
\\	おあずかり  します。	
\\	おつり	
\\	わたす	
\\	こす	
\\	おこし  ください  (お越し下さい)	
\\	王	
\\	王子	
\\	王国	
\\	音	
\\	音楽	
\\	発音	
\\	何	
\\	何月	
\\	何人	
\\	夏	
\\	初夏	
\\	家	
\\	家族	
\\	ご家族	
\\	~家	
\\	冬	
\\	冬眠	
\\	冬眠する	
\\	熊	
\\	ジョンさんは いま 本を よんで いる。	
\\	真理子さん、ばんごはんを たべている?	
\\	たべて いない。	
\\	べんとうを あたためて、たべた。	
\\	家を そうじ して、ちゅうしょくを たべて、テレビを 見た。	
\\	ばんごはんを たべて、いる。	
\\	ばんごはんを たべて、いない。	
\\	ばんごはんを たべて、いた。	
\\	こんばん、ラーメンを たべようか?	
\\	あ、ごめん。いま たべている。	
\\	鈴木君、べんきょうを した?	
\\	いま している!	
\\	さっき たべた。	
\\	鈴木さんは さっき でんわ した。	
\\	もう たべた。	
\\	もう その 本を よんで いる。	
\\	べんきょうは?もう した?	
\\	健二君は もう、がっこうに 行って いる。	
\\	真理子さんは まだ テレビを 見ている。	
\\	ジョンさんは まだ ビールを のんで いる?	
\\	私は まだ 大学 一年生 です。	
\\	真理子さんは まだ 大学を 出て いない。	
\\	ビルさんは まだ 日本に 行った ことが ない。	
\\	ねつは まだ 下がって いない。	
\\	ジョンさん、モスバーガーで たべた こと、ありますか?	
\\	まだ です。行きましょう!	
\\	まだ です。= まだ モスバーガーで たべた こと、ありません。	
\\	たべている。	
\\	ジョンさん、もう ばんごはんを たべている?	
\\	いいえ、まだ たべて いないよ。	
\\	ごめん。もう たべている。	
\\	あ、さっき たべた。	
\\	いま たべている。	
\\	まだ ばんごはんを たべて いない?	
\\	もう ばんごはんを たべた?	
\\	もう ばんごはんを たべている?	
\\	鈴木さんは、もう 行った?	
\\	まだ 行って いないよ。	
\\	真理子さんは インフルエンザの ちゅうしゃを うけた?	
\\	うん、もう うけて いるよ。	
\\	きょう、真理子さんに 会いました。	
\\	裕子さんに 会ったこと、ある?	
\\	いいえ、まだ 会ってないけど、会いたい。	
\\	どこで 会いましょうか。	
\\	成田えきの 入り口で 会いましょう。	
\\	海で およいだ ことが ない。	
\\	海の 音が 大好きだ。	
\\	海水を のまないでね。	
\\	三水	
\\	ねこを 外に 出した?	
\\	中田君は 外で あそんでいる。	
\\	私は 外国人だ。	
\\	何を 学びたい?	
\\	いま、日本語を 学んでいる。	
\\	この 大学に 学生が 2000人 います。	
\\	その 学生は 何を 学んでいる?	
\\	花子ちゃんは 小学生 です。	
\\	田中さんの 家のまえに 門が ある。	
\\	門の中に 真理子さんが たっていた。	
\\	さっき	
\\	もう	
\\	まだ	
\\	会う	
\\	会話	
\\	会話する	
\\	会議	
\\	海	
\\	海水	
\\	海上	
\\	外	
\\	外国人	
\\	外国	
\\	外科	
\\	学ぶ	
\\	大学	
\\	学生	
\\	数学	
\\	真理子さんの でんわ ばんごうを おぼえた。	
\\	私の なまえを おぼえて いる?	
\\	すみません。おぼえて いない。	
\\	テレビで 中国の 人口を しりました。	
\\	田中さんを しっている。	
\\	真理子さんの 家を しっている。	
\\	鈴木君の でんわ ばんごうを しってる?	
\\	その人を しらない。	
\\	日本の 人口を しらない?	
\\	ジョンさん、明日の パーティーに くる?	
\\	まだ 分からない。	
\\	何 ですか。しりません。	
\\	何 だか。 何か。	
\\	何か、しりません。 何だか、しりません。	
\\	どこ でしたか。しりません。	
\\	どこ だったか。	
\\	どこ だったか しりません。	
\\	何時に かえりましたか。おぼえていません。	
\\	何時に かえったか おぼえていません。	
\\	何を のんだか おぼえていません。	
\\	どこで 見たか わすれた。	
\\	なぜ そう したか 分からない。	
\\	何時に 家を 出たか おぼえていますか。	
\\	何を ちゅうもん したか わすれましたか。	
\\	何だか しりたい ですか。	
\\	どこに 行ったか おぼえてない?	
\\	どこで 学んだか わすれた?	
\\	私は [山田さんが いつ かえったか] しらない。 [山田さんが いつ かえったか]  しらない。	
\\	ジョンさんが どこに 行ったか おぼえてない。	
\\	真理子さんは、ジョンさんが 何を かったか、しっている。	
\\	私は、裕子さんが どこで 生まれたか、しってる。	
\\	真理子さんが そのセーターを どこで かったか、しらない。	
\\	ジョンさんが なぜ 行きたかったか、わすれた。	
\\	私の かばんが どこか、しってる?	
\\	行くか 分からない。	
\\	見たか おぼえてない。	
\\	行くか どうか	
\\	行くかどうか 分からない。	
\\	土曜日に みせが あいて いるかどうか しらない。	
\\	でんしゃが もう 出発 したかどうか、分からない。	
\\	花子ちゃんが 大学に 入ったかどうか、しってる?	
\\	この あたりに きっさてんが あるかどうか、しってる?	
\\	真理子さんが まだ 家に いるかどうか、分からない。	
\\	会議が もう はじまっているかどうか しらない。	
\\	ジョンさんが かんコーヒーが すきかどうか おぼえてない。	
\\	田んぼの 真ん中に、牛が いる。	
\\	犬が 門のまえに いる。	
\\	どんな 犬が すき?	
\\	先月、真理子さんの 家で、白い 子犬が 五ひき 生まれた。	
\\	日本の 柴犬は うつくしいね。	
\\	馬に のったこと、ある?	
\\	裕子さんは 馬が 大好き です。	
\\	300馬力の くるまです。	
\\	みずうみの 上に 白い鳥が たくさん いる。	
\\	川の 真ん中で 魚が およいでいる。	
\\	さしみは、生の 魚だ。たべたこと、ある?おいしいよ!	
\\	おぼえる	
\\	しる	
\\	わすれる	
\\	~かどうか	
\\	牛	
\\	牛乳	
\\	牛肉	
\\	水牛	
\\	犬	
\\	子犬	
\\	番犬	
\\	馬	
\\	木馬	
\\	トロイの木馬	
\\	馬力	
\\	鳥	
\\	白鳥	
\\	魚	
\\	金魚	
\\	ジョンさん、ごはんを たべた?	
\\	おなかが すいた。	
\\	真理子さん、おなかが すきましたか。	
\\	うん、すきましたね。	
\\	のどが かわきました。	
\\	ジョンさん、のどが かわいた?	
\\	うん、かわいた。	
\\	のどが かわいて いる。	
\\	真理子さん、おなかが すいて いる?	
\\	この みちは すいて いますね。	
\\	その ワイシャツは まだ かわいて いません。	
\\	真理子さんは、足が ながい。	
\\	真理子さんは おなかが すいてる。	
\\	ジョンさんは のどが かわいてる?	
\\	かわいてないよ。	
\\	ああ、おいしかった。おなかが 一杯 だ。	
\\	おなかが 一杯に なっている。	
\\	しょくじ しましょうか。	
\\	しょくじを しましょうか。	
\\	りょうりを しました。	
\\	べんきょうを しました。	
\\	きょうの べんきょうを しました。	
\\	私は タイりょうりが 一番 すき だ。	
\\	ここの インドりょうりは すごく うまい。	
\\	ジョンさん、日本で 中華料理を たべたこと、ある? おいしいよ!	
\\	たべに 行く。	
\\	たべに 行った。	
\\	たべに 行きたい。	
\\	たべに 行かない?	
\\	インドりょうりを たべに 行きたい。	
\\	真理子さん、私と 一緒に 中華料理を たべに 行きたい?	
\\	えいがを 見に 行きましょう。	
\\	ジョンさんと 真理子さんは のみに 行った。	
\\	真理子さんは えいごを べんきょう しに 行った。	
\\	真理子さんは アメリカに えいごを べんきょう しに 行った。	
\\	鈴木さんは ホテルの よやくを かくにん しに 行った。	
\\	真理子さんは かいものを しに 行った。	
\\	ジョンさんは 鈴木君に 会いに 行った。	
\\	こんばん 健二君が うちに たべに くる。	
\\	うちに あそびに くる?	
\\	たべやすい	
\\	のみにくい	
\\	ききにくい	
\\	のりやすい	
\\	この 牛肉は たべやすいね。	
\\	のみにくい くすりが きらい です。	
\\	ききにくい ラジオ だね。	
\\	裕子ちゃんの じてんしゃは すごく のりやすかった。	
\\	この ラーメンは たべにくい。	
\\	この へやが うるさくて、べんきょう しにくい。	
\\	うんてん しやすい くるまが ほしい。	
\\	この くつは はきにくいね。	
\\	おいしくて のみやすい ワイン ですね。	
\\	たべて 見る。	
\\	おお、この アイスクリームは おいしいよ。たべて 見る?	
\\	たべて 見たい!	
\\	日本の さけを のんだ ことが ない。のんで 見たい。	
\\	日本の ケーキは おいしいよ。たべて 見て 下さい。	
\\	本当の 日本人と 日本語で はなして 見たい。	
\\	イギリス人と えいごで はなして 見た けど、むずかしかった!	
\\	真理子さん、水が ちょっと つめたい けど、川で およいで 見ない?	
\\	ジョンさんが パーティーに 行く かどうか、きいて 見て。	
\\	その本を 見て みたい。	
\\	牛乳が ある かどうか、見て みて 下さい。	
\\	あそこの 鳥を 見て みて!	
\\	ジョンさん、もう 
\\	00だけど、おなかが すいた?	
\\	すいたね。こんばん、何を たべようか。	
\\	本当の ラーメンを たべたこと、ある?	
\\	ないよ。おいしい?たべて 見たいね。	
\\	おいしいよ。こんばん、たべに 行こうか?	
\\	いいね。	
\\	ああ、真理子さん、お元気ですか。	
\\	元気だよ!	
\\	明日は どんな 天気 だろう。	
\\	ジョンさんは どんな 気分ですか。	
\\	いい 気分だよ。	
\\	気をつけてね!	
\\	気をつけているよ。	
\\	くるまに 気をつけて 下さい。	
\\	さむいから からだに 気をつけて 下さい。	
\\	真理子さんは 家で 休んでいる。	
\\	休みの 日が すきです。	
\\	夏休みは 何月 何日から ですか。	
\\	冬休みに ジョンさんに 会って、えいがを 見た。	
\\	日本では 休日が たくさん ある。	
\\	京都に 行ったことが ありますか。	
\\	東京には 外人が 一杯 いるね。	
\\	ジョンさんは ビリヤードの玉を もっている。	
\\	ジョンさん、100円玉を もっている?	
\\	テーブルの 上に、五十円玉が 二まい あったよ。	
\\	ああ、目玉が いたい!	
\\	空が 青くて、きれいだね。	
\\	ジョンさんの コップは 空っぽ です。	
\\	森の中の 空気は きれいだね。	
\\	ごはん	
\\	こめ  (米)	
\\	みそ	
\\	みそしる	
\\	おかず	
\\	どんぶり	
\\	ぎゅうどん  (牛丼)	
\\	おやこどん  (親子丼)	
\\	かつどん	
\\	とうがらし  (唐辛子)	
\\	そば	
\\	やきそば	
\\	うどん	
\\	ラーメン	
\\	みそラーメン	
\\	とんこつラーメン	
\\	しょうゆラーメン	
\\	ぎょうざ  (餃子)	
\\	すし	
\\	にぎりずし	
\\	まきずし	
\\	ぐんかん	
\\	しゃけ	
\\	まぐろ	
\\	さば	
\\	いか	
\\	たこ	
\\	さしみ	
\\	わさび  (山葵)	
\\	おなか	
\\	のど	
\\	すく	
\\	かわく	
\\	しょくじ	
\\	しょくじ  する	
\\	りょうり	
\\	ちゅうか  りょうり  (中華料理)	
\\	インド  りょうり	
\\	タイ  りょうり	
\\	イタリア  りょうり	
\\	~に  いく	
\\	~に  くる	
\\	~やすい	
\\	~にくい	
\\	元気	
\\	天気	
\\	気分	
\\	電気	
\\	気をつける	
\\	体	
\\	休む	
\\	休み	
\\	休日	
\\	東京	
\\	京都	
\\	玉	
\\	~玉	
\\	目玉	
\\	水玉	
\\	空	
\\	空っぽ	
\\	空気	
\\	空港	
\\	よむ 人	
\\	よむ。	
\\	よむ 人	
\\	かく 人	
\\	はなす ねこ	
\\	[しんぶんを よむ] 人	
\\	[しんぶんを よんだ] 人	
\\	[しんぶんを よんでいる]  人	
\\	[しんぶんを よまない]  人	
\\	[しんぶんを よまなかった]  人	
\\	[土曜日に えいがを 見た]  人	
\\	人	
\\	[見た] 人	
\\	[えいがを 見た] 人	
\\	[土曜日に えいがを 見た] 人	
\\	ねこ です。	
\\	[ねずみを たべる] ねこ です。	
\\	[ねずみを たべる] ねこが きらい です。	
\\	真理子さんは ねこが きらい です。	
\\	真理子さんは [ねずみを たべる] ねこが きらい です。	
\\	[この 本を かいた] 人は 私の ちち です。	
\\	あなたの アイスクリームを たべた 人は ここに いません。	
\\	あそこに たっている サラリーマンは だれ ですか。	
\\	青い シャツを きている 女の人は いもうと です。	
\\	レストランの 中で すしを たべている おじいさんは 田中さん です。	
\\	毎日 べんきょう している 子どもたちは あたまが いい。	
\\	子どもたちは 毎日 べんきょう していて、あたまが いい。	
\\	毎日 べんきょう している 子どもと、べんきょう しない 子どもが いる。毎日 べんきょう している 子どもたちは あたまが いい。	
\\	そう ですか?	
\\	そう じゃないよ。	
\\	テストは むずかしかった?	
\\	かんたん だったよ。	
\\	ジョンさん、まだ たべて いない?	
\\	いいえ、さっき たべたよ。	
\\	ジョンさんは すしが きらい?	
\\	大好き だよ。	
\\	ジョンさん、カナダに 行ったことが ありますか。	
\\	ありますよ。	
\\	日本語を べんきょう していて、たのしい?	
\\	たのしいよ。日本語が 大好き だ。	
\\	おなまえは 何ですか。	
\\	ジョン です。	
\\	金曜日に、どこに 行きたい?	
\\	海に 行きたいね。	
\\	あの えいがは、本当に 面白かったね。	
\\	あの えいがは、本当に 面白かったよ。	
\\	ああ、おいしかったよ。	
\\	おいしかったよね。	
\\	カラオケは たのしいよね。	
\\	ゆうべは たのしかったよね。	
\\	あそこの みせの ラーメンは おいしいよね。	
\\	真理子さんは もう かえったよね。	
\\	馬鹿馬鹿しいね。	
\\	馬鹿馬鹿しいよ。	
\\	それは まったく ごんごどうだん だよ。	
\\	見て 下さいよ。	
\\	ちょっと まってよ。	
\\	行こうよ!	
\\	ぜったいに しようよ。	
\\	何を している?	
\\	何を してる?	
\\	何を してるのよ!	
\\	何を してるんだよ!	
\\	ジョンさん、どう 思う?	
\\	そう 思っていた。	
\\	不思議な 男性が 外で まっている。	
\\	私は 一日中 考えていた。	
\\	ジョンさんは きっさてんの 中で すわって、考えていた。	
\\	ジョンさん、何を 考えている?	
\\	夏休みに 何を するか、考えている。	
\\	おお、それは いい 考え だね。	
\\	真理子さんが そう 言いました。	
\\	いくつの 言語を しっていますか。	
\\	真理子さんは 石田君と 話している。	
\\	そんな話を きいたことが ない。	
\\	その えいがは どんな 話 ですか。	
\\	まったく	
\\	ばか  (馬鹿)	
\\	ばかばかしい  (馬鹿馬鹿しい)	
\\	ごんごどうだん	
\\	思う	
\\	不思議	
\\	考え	
\\	思考	
\\	思考力	
\\	言う	
\\	言語	
\\	無言	
\\	話す	
\\	話	
\\	会話	
\\	電話	
\\	「鳥 です」と 言った。	
\\	「日本に 行きたい」と 思った。	
\\	「この ジュースは おいしいね」と 言った。	
\\	「やった!」と 言った。	
\\	ビルさんは、「らいねんの夏、日本に 行く!」と 言った。	
\\	「あの人は、あたまが いいね」と、私は 思った。	
\\	「日本語は むずかしくないよ」と、先生は 金曜日に 言った。	
\\	先生は 金曜日に「日本語は むずかしくないよ」と 言った。	
\\	鈴木さんは、「やった!」と 思った。	
\\	私は 真理子さんに、「さんぽに 行かない?」と きいた。	
\\	「ここは 私の へや です」と、鈴木さんは 説明 した。	
\\	真理子さんは、「いたい!」と さけんだ。	
\\	ジョンさんは、「大丈夫?」と きいた。	
\\	真理子さんは、「うん、すごく いたかった けど、大丈夫 だ」と こたえた。	
\\	真理子さんは、「こんばんの パーティーに 行きます」と 言った。	
\\	真理子さんは、こんばんの パーティーに 行く と 言った。	
\\	ジョンさんは、かんこくに 行ったことが ない と こたえた。	
\\	鈴木さんは、きのう すしを たべた と 言いました。	
\\	真理子さんは、ねこが すきだ と 説明 しました。	
\\	私は、先生に なりたい と 思った。	
\\	家が 大きい と 思いました。	
\\	パーティーに 行く?	
\\	パーティーに 行くか。	
\\	ジョンさんは 真理子さんに パーティーに 行くか と きいた。	
\\	先生に アメリカに 行ったことが あるか と きいた。	
\\	ジョンさんに お金が ないか と きいた。	
\\	大丈夫 ですか。	
\\	大丈夫か。	
\\	ジョンさんは 真理子さんに 「大丈夫 ですか」と きいた。	
\\	ジョンさんは 真理子さんに 大丈夫か と きいた。	
\\	真理子さんは、「大丈夫!」と こたえた。	
\\	真理子さんは 大丈夫だ と こたえた。	
\\	真理子さんは、「ビールが きらいだ」って 言った。	
\\	真理子さんは、ビールが きらいだって。	
\\	真理子さんは、ビールが きらいだって(言う)。	
\\	真理子さん、裕子さんが えいがを 見に 行かないか って。	
\\	おなかが すいたね。たべに 行かない?	
\\	真理子さんに きく。ちょっと まってね。真理子さん、裕子さんが たべに 行かないか って。	
\\	何 りょうりが たべたいか、きいて。	
\\	真理子さんは、何の りょうりが たべたいか って。	
\\	そうだね。インドりょうりを たべに 行こうか。おいしい レストランを しっているよ。	
\\	裕子さんは、インドりょうりが たべたいって。おいしい レストランを しってるって。	
\\	ごめん。私は インドりょうりが 駄目だ。	
\\	裕子さん?真理子さんは インドりょうりが きらいだ って。	
\\	じゃあ、イタリアりょうりは どう?	
\\	真理子さん、イタリアりょうりは どうか って。	
\\	ああ、それが いいね。	
\\	真理子さんは イタリアりょうりが いいって。	
\\	よし!それで 行きましょう。	
\\	話の元には 何が ありますか。	
\\	足元に 気をつけて 下さい。	
\\	お元気 ですか。	
\\	明日は 元日です。	
\\	その女子は 大学の ビルの 戸口に たっていた。	
\\	二人は とても 古い ビルに 入りました。	
\\	私の くるまは 古い。	
\\	裕子さんは 中古のくるまを かった。	
\\	ジョンさんは 大学で 工学を べんきょう している。	
\\	鈴木君は アルバイトで 工事の しごとを している。	
\\	50メートル先の みちは、工事中です。	
\\	この へやは 光が たくさん 入って、いいね。	
\\	川の上で、月が きれいに 光っていた。	
\\	川の真ん中に、光っている石が ある。金かな?	
\\	と	
\\	いう  (言う)	
\\	おもう  (思う)	
\\	こたえる	
\\	さけぶ	
\\	おしえる	
\\	せつめい  (説明)	
\\	せつめい  する  (説明する)	
\\	って	
\\	と, 
\\	いう	
\\	元	
\\	元気	
\\	元日	
\\	戸口	
\\	戸籍	
\\	古い	
\\	中古	
\\	考古学	
\\	工学	
\\	工事	
\\	工事中	
\\	光	
\\	光る	
\\	光子	
\\	光年	
\\	たべよう と思う。	
\\	ドイツに 行こう と 思っている。	
\\	10の でんしゃに のろうと 思っている。	
\\	裕子さんに 話そうと 思っています。	
\\	らいねん から、工学を べんきょう しよう と 思っている。	
\\	私は その つもり です。	
\\	行く つもり です。	
\\	たべる つもり です。	
\\	先生に 会う つもり です。	
\\	らいねん、日本に 行く つもり だ。	
\\	明日、真理子さんと 会話を する つもり です。	
\\	日本で えいごを おしえる つもり です。	
\\	らいしゅう、真理子さんと えいがを 見に行く つもり だ。	
\\	明日、学校に 行く つもり じゃない。	
\\	明日、学校に 行かない つもり だ。	
\\	行く つもり ですか。	
\\	どこへ 行く つもり ですか。	
\\	たべる つもり だった?	
\\	私の アイスクリームを たべる つもり だった?	
\\	これを かう つもり?	
\\	真理子さんは、明日 行く つもりだ と 言いました。	
\\	きょうの ごご、ジョンさんが くるまを あらう つもり かな。	
\\	鈴木さんは パーティーに くる つもり かな。	
\\	真理子さん、こんばん、よていが ある?	
\\	よていが あります。	
\\	ごめん。もう よていが 入っている。	
\\	じゃ、土曜日は?よていが 入っていますか。	
\\	でんしゃで 行く よてい です。	
\\	ともだちと レストランで たべる よてい です。	
\\	お金が ない。	
\\	お金が ぜんぜん ない。	
\\	行きたくない。	
\\	ぜんぜん 行きたくない。	
\\	たべた ことが ない。	
\\	ぜんぜん たべた ことが ない。	
\\	こんしゅうまつ、よていが ある?	
\\	ぜんぜん ないよ。	
\\	あそこの レストランは おいしい?	
\\	ぜんぜん。	
\\	え?フランスに 行きたくないって?	
\\	きょうは とくべつな よていが あります。	
\\	裕子さんは とても とくべつな 人 です。	
\\	よていは とくに ない。	
\\	コーヒーは べつに すき じゃない。	
\\	べつに そう ではないよ。	
\\	カラオケには とくに よく 行ってませんが、すき ですよ。	
\\	真理子さんの 学校は どこですか。	
\\	この金魚を かいたいけど、高い。	
\\	高いビルが すき じゃない。	
\\	日本の 高校に 行ったことが ありますか。	
\\	日本の高校生と 会話した ことが ありますか。	
\\	中国と 米国が 会議を している。	
\\	この国では、日本語で 話す。	
\\	一番 ひろい国は どこですか。	
\\	ジョンさんの くるまは 黒い。	
\\	先月の 休日に 海に 行って、黒い鳥を 見た。	
\\	何色が 一番 すき ですか。青?赤?それとも 黒?	
\\	今から 学校に 行く。	
\\	すみません。今 何時 ですか。	
\\	夏休みは 今月の 11日から だ。	
\\	今日、何を したい?	
\\	今日は!	
\\	今日、山田先生に「こんにちは!」と言いました。	
\\	つもり	
\\	よてい	
\\	しゅうまつ	
\\	こんしゅうまつ	
\\	ぜんぜん	
\\	とくべつ	
\\	とくに	
\\	べつに	
\\	学校	
\\	小学校	
\\	中学校	
\\	高い	
\\	高校	
\\	高校生	
\\	国	
\\	中国	
\\	米国	
\\	国語	
\\	黒い	
\\	黒板	
\\	黒人	
\\	今	
\\	今月	
\\	今週	
\\	今日	
\\	いらっしゃいませ。何名様 ですか。	
\\	どうぞ・・・	
\\	こちらへ どうぞ。	
\\	ゆっくり 話して 下さい。	
\\	ゆっくり して 下さい。	
\\	しごとで すごく つかれた。ちょっと、おふろで ゆっくり したい。	
\\	ごゆっくり、どうぞ。	
\\	今 かくにん いたします。	
\\	今晩 行く レストランが きまった。	
\\	何を たべたいか、きまった?	
\\	ばんごはんが きまった。	
\\	かいたい テレビが きまらない。	
\\	行きたい 大学が きまった。	
\\	おきまり ですか。	
\\	いただきます!	
\\	いらっしゃいませ。何名様 ですか。	
\\	二人 です。	
\\	お二人さま ですね。こちらへ どうぞ。・・・どうぞ、こちらに おすわり 下さい。	
\\	こちらは メニューに なります。ごゆっくり、どうぞ。	
\\	(ジョンと 真理子は メニューを 見る。)	
\\	はい、きまりました。私は 牛丼 セットで おねがい します。	
\\	私は そば セットで。	
\\	はい、分かりました。おのみものは?	
\\	私は、ビールで、ジョンさんは?	
\\	私も ビールで 行こうかな。	
\\	かしこまりました。ごちゅうもんを くりかえします。	
\\	牛丼 セット、一つ。そば セット、一つ。それに ビール、二本。以上で よろしい ですか。	
\\	はい、大丈夫 です。	
\\	ありがとう ございました。	
\\	現金で はらいました。	
\\	クレジットカードで はらっても いいですか。	
\\	そろそろ かえりましょうか。	
\\	そろそろ 行こうか。	
\\	おいしかった!ごちそうさまでした!	
\\	ありがとう ございました!	
\\	ごちそうさま でした!	
\\	ああ、おまたせ!	
\\	おまたせ しました。	
\\	おまたせ いたしました。	
\\	ああ、おいしかったね。	
\\	そうだね。日本の たべものは 最高だね。	
\\	そろそろ 行きましょうか。えいがは 
\\	00から だから。	
\\	そうだね。ところで、きのうは 真理子さんが はらったから、今日は 私が はらうよ。すみません、会計を おねがい します!	
\\	はい、おまたせ いたしました。こちらが お会計に なります。	
\\	ここは クレジットカードで はらっても いい ですか。	
\\	はい、よろしい ですよ。	
\\	じゃ、この カードで おねがい します。	
\\	かしこまりました。しょうしょう、おまちください。・・・おまたせ いたしました。今日は、ありがとう ございました。	
\\	今日、何を作る つもりですか。	
\\	これを 自分で 作った。	
\\	ジョンさんの ともだちは 作家 です。	
\\	何の作家ですか。	
\\	セーターから 糸が 出ているよ。	
\\	紙を 三まい 下さい。	
\\	あそこの、漢字を べんきょう している 男性が ジョンさんです。	
\\	真理子さんは 字が きれいだね。	
\\	わあ、鈴木君の 字が きたない!	
\\	花子ちゃんは テーブルで 数字の れんしゅうを している。	
\\	日本に 行った時を おぼえています。	
\\	あの時は たのしかったね。	
\\	今 何時 ですか。	
\\	九時です。	
\\	時間が ない!	
\\	一緒に かいものを する 時間が ある?	
\\	レストラン	
\\	ファミレス	
\\	ウエートレス	
\\	ウエーター	
\\	メニュー	
\\	セット	
\\	なんめいさま  (何名様)	
\\	どうぞ	
\\	ゆっくり	
\\	ごゆっくり	
\\	おのみもの	
\\	いたす	
\\	きめる	
\\	きまる	
\\	くりかえす	
\\	ごちゅうもん	
\\	いただきます	
\\	はらう	
\\	げんきん  (現金)	
\\	クレジットカード	
\\	カード	
\\	そろそろ	
\\	さいこう  (最高)	
\\	ところで	
\\	作る	
\\	作家	
\\	糸	
\\	毛糸	
\\	紙	
\\	手紙	
\\	表紙	
\\	字	
\\	漢字	
\\	数字	
\\	時	
\\	~時	
\\	時間	
\\	どんな もの ですか。	
\\	へんな ものが れいぞうこの 中に ある。	
\\	おいしい ものが たべたい。	
\\	言いたい ことが ある。	
\\	そんな ことは 私に 分かりません。	
\\	分かりにくい ことが たくさん ある。	
\\	私は ピアノが 下手 です。	
\\	そんな こと、ないよ!	
\\	スーパーで すごい ことを 見た。	
\\	今日 学校で 面白い ことを きいた。	
\\	何を 見たか、話して 下さい。	
\\	見た ことを 話して 下さい。	
\\	見た ものを 話して 下さい。	
\\	考えている ことを 言って 下さい。	
\\	おかあさんが 言った ことを おぼえて いる?	
\\	あなたが 手紙に かいた ことを おぼえて いる。	
\\	真理子さんが 私に 話した ことを よく 考えました。	
\\	考えていた ことを 言わなかった。	
\\	たべる こと が すき です。	
\\	かく ことが すき です。	
\\	まつ ことが きらい だ。	
\\	べんきょうする ことが 大嫌い です。	
\\	よむ の が すき です。	
\\	はたらく の が すき だ。	
\\	はしるのが 大好き だ。	
\\	この しごとは かんたん です。	
\\	およぐ こと は かんたん です。	
\\	まつのは たのしくない。	
\\	うんてんすることが 大好きだ。	
\\	この数学を 説明することが むずかしい。	
\\	よむことは たのしい。	
\\	りょうりする ことは むずかしくない です。	
\\	きめるのが むずかしい。	
\\	かえることは かなしい です。	
\\	べんきょうするのが すき だ。	
\\	ラーメンを たべるのが すきだ。	
\\	まいしゅう 日曜日に えいがを 見るのが すきだ。	
\\	まいしゅう 日曜日に えいがを 見る…	
\\	まいしゅう 日曜日に えいがを 見ること…	
\\	…が すきだ。	
\\	あさ コーヒーを のむのが 大好き だ。	
\\	毎日 ピアノを れんしゅうすることが とても 大切 です。	
\\	くすりを のむのが きらい です。	
\\	えきで ごご10:00まで あなたを まつのは たのしくなかった。	
\\	人の たべものを たべることは よくない。	
\\	しっている。	
\\	あなたは 私の アイスクリームを たべた。	
\\	あなたが 私の アイスクリームを たべたこと	
\\	[あなたが 私の アイスクリームを たべたこと] を しっている。	
\\	その人を しっている。	
\\	今日 鈴木さんが しごとに こなかったことを しっている。	
\\	鈴木さんが みずうみで あそんでいるのを 見た。	
\\	鈴木さんが 病気 ではないことを しっている。	
\\	鈴木さんは 多分 くびに なる
\\	でしょう。	
\\	ジョンさんが ラジオを きいているのを 見た。	
\\	真理子さんが もう その えいがを 見たことを しっている。	
\\	らいねん から 日本語を べんきょうすることを きめた。	
\\	ジョンさんが はを みがくのを 真理子さんは 見た。 真理子さんは、ジョンさんが はを みがくのを 見た。	
\\	真理子さんが へやに 入るのを きいた。	
\\	およぐことは むずかしい。 およぐのは むずかしい。	
\\	らいねん イギリスに 行くことに した。	
\\	今日 から お菓子を たべないことに した。	
\\	本を かくことに した!	
\\	東京 大学に 入ることに した。	
\\	日本に かえる ことに なった。	
\\	来月 から 京都に 行くことに なった。	
\\	明日 学校で この本を よむことに なった。	
\\	アメリカに 行くことに なった!	
\\	もの	
\\	こと	
\\	こと	
\\	~ことに  する	
\\	~ことに  なる	
\\	行く。 私は 行く。 私が 行く。	
\\	真理子さん、お元気 ですか。	
\\	元気 ですよ。ジョンさんは?	
\\	私も 元気です。	
\\	私は?行く。	
\\	漢字が すき です。	
\\	それが ほしい。	
\\	足が ながい。	
\\	真理子さんが 行った ことを しっている。	
\\	だれが 行く?	
\\	どこが 一番 面白い?	
\\	私が 行く。	
\\	京都が 一番 面白い。	
\\	私が。	
\\	私は 行く。	
\\	ラーメンは、たべた。	
\\	私の ラーメンは どこ?	
\\	たべた。	
\\	ラーメンは?たべた。	
\\	れいぞうこの 中に あった ラーメンと 餃子を たべた?	
\\	ラーメンは たべた けど、餃子は たべて いない。	
\\	ラーメンは・・・	
\\	たべた けど、	
\\	餃子は・・・	
\\	たべて いない。	
\\	すしは すきだ けど、さしみは きらい だ。	
\\	鈴木さんは、足は ながい けど、うでは みじかい。	
\\	アメリカには まだ 行って いない。	
\\	それは、私には 分からない。	
\\	それでは よくない。	
\\	京都に 行った ことは ある けど、でんしゃでは 行った ことが ない。	
\\	のどが かわいて いるな。ビール、ある?	
\\	ごめん。ビールは ない。	
\\	そんな ことは ない!	
\\	りんご ではない。 りんご でない。	
\\	りんご では?・・・ない。	
\\	先生は いない。	
\\	れいぞうこの 中に 牛乳は ない。	
\\	かえりたくは ない けど、明日は しごと だからね。	
\\	たべては いない けど、おそいから かえるよ。	
\\	たべては・・・	
\\	いる?いない?どうかな。	
\\	・・・いないけど、おそいから かえるよ。	
\\	うちに ねこが いる。	
\\	その ねこは、黒い。	
\\	みずうみが あそこに ある。	
\\	みずうみは あそこに ある。	
\\	君の 青い車で 海へ 行こう。	
\\	黒い車が 門のまえに とまっている。	
\\	ああ、秋が すきだな。	
\\	秋の月が 大きくて きれいだ。	
\\	春は 今月から だよ。森でハイキングしようか。	
\\	ここの 春巻きは 高いけど おいしいよね。	
\\	明日、先生に 手紙を 書きます。	
\\	山田君は へんな 話を 書くのが すきだ。	
\\	ジョンさんの 車を 少し うんてん して 見たい。	
\\	少しの 時間が ある。	
\\	日本では、中国人が 少ない。	
\\	シアトルでは 夏の日が 少ない。	
\\	車	
\\	自転車	
\\	自動車	
\\	秋	
\\	秋分	
\\	春	
\\	青春	
\\	青春期	
\\	書く	
\\	文書	
\\	辞書	
\\	少し	
\\	少ない	
\\	少女	
\\	少年	
\\	水が いる。	
\\	水が いらない。	
\\	私は 馬が すき だ。	
\\	私は 馬が ほしい。	
\\	私は 馬が いる。	
\\	ジョンさんは お金が いる。	
\\	真理子さんは くつが いる。	
\\	鈴木君は この 本が いらないと 言っている。	
\\	この 車に ガソリンが たくさん いる。	
\\	ジョンさんに お金が いる。	
\\	私は、真理子さんに お金が いる ことを しらなかった。	
\\	水が いりますか。	
\\	今から おちゃを 入れる けど、いりますか。	
\\	いります!	
\\	コーヒーが いる 人は?	
\\	コーヒーが いる 人は、だれ ですか。	
\\	コーヒーが いる 人は、手を 上げて 下さい。	
\\	いる!	
\\	私も いる!	
\\	ジョンさんは?おちゃは?いらない?	
\\	あ、私は いらない。ありがとう。	
\\	馬が いる。	
\\	馬が いない。	
\\	馬が いらない。	
\\	馬が います。	
\\	馬が いります。	
\\	ひつようが あります。	
\\	ひつようは ない。	
\\	ひつような しごと でした。	
\\	ひつような こと じゃない。	
\\	ひつような お金が ない。	
\\	馬が ひつよう です。	
\\	私に 馬が ひつよう です。	
\\	これに お金が ひつよう です。	
\\	旅行するのに お金が ひつよう だ。	
\\	小説を 書くのに たくさんの 時間が ひつよう です。	
\\	車を かうのに お金が ひつよう だ。	
\\	上手に なるのに 時間が ひつよう だ。	
\\	行く ひつよう	
\\	書く ひつよう	
\\	一杯 たべる ひつよう	
\\	一杯 たべる ひつようが ある。	
\\	一杯 たべる ひつようは ない。	
\\	少し 考える ひつようが ある。	
\\	会計を はらう ひつようが ある。	
\\	はやく きめる ひつようが あった。	
\\	今日は ばんごはんを 作る ひつようは ない。	
\\	毎日 行く ひつようが、べつに ないよ。	
\\	ああ、この色が 大好き。	
\\	何色ですか。	
\\	私は 色々な 漢字を しっています。	
\\	ジョンさんは 色々な 言語を しっている。	
\\	真理子さんは 心が ひろいね。	
\\	あたまでは 鈴木君が すきだけど、心では 健二君が すき。	
\\	車の中では、小さい声で 話して下さい。	
\\	少し、声を出して 見てね。	
\\	鈴木さんは 東京大学に入る、と声明した。	
\\	鈴木さんに この牛肉を 切って ほしい。	
\\	その女性は ナイフで 自分の 足を 切りました。	
\\	とても 大切な 手紙です。	
\\	明日、雪がふるかな?	
\\	今日、大雪がふった。	
\\	いる	
\\	ひつよう	
\\	色	
\\	色々	
\\	景色	
\\	一色	
\\	心	
\\	小心	
\\	心配	
\\	心配する	
\\	声	
\\	声明	
\\	声明する	
\\	切る	
\\	大切	
\\	親切	
\\	雪	
\\	大雪	
\\	明日 しごとが ある。	
\\	明日 しごとが あるの	
\\	明日 しごとが あるの です。	
\\	ごめん。雪が すき じゃないの です。	
\\	お金が ないの だ。	
\\	あの 手紙が とても 大切 だったの です。	
\\	すごく 行きたいの だ。	
\\	お金が ないの だ。 お金が ないの。 お金が ないん だ。	
\\	今から 旅行に 行く んだ。 今から 旅行に 行く の。	
\\	この漢字が 書きにくい の。 この漢字が 書きにくい んだ。	
\\	私は テニスが 上手 じゃない の。 ぼくは テニスが 上手 じゃない んだ。	
\\	あさごはんは もう 作ったんです。	
\\	ここの ねだんが すごく 高いんです。	
\\	おなかが すいているん です。	
\\	そう だ。	
\\	そう な のだ。	
\\	そう なんだ。 そう なの。	
\\	ここは ぼくの へや なんだ。 ここは 私の へや なの。	
\\	田中さんは 高校の 先生 なの。	
\\	すみませんが、にんにくが きらい なんです。	
\\	えいがが 大好き なんだ。	
\\	そう なんだ? そう なの?	
\\	今晩、ラーメンが たべたいん ですか。	
\\	今から かえるん ですか。	
\\	ここは ジョンさんの うち なん ですか。	
\\	何を 心配しているの?	
\\	本当 なの?	
\\	うれしくないの?	
\\	これが ジョンさんの 車 なの?	
\\	あの 人は 真理子さんの おかあさん だったの?	
\\	なぜ 私の アイスクリームを たべたの?	
\\	どう したの?	
\\	それは 本当 なの?	
\\	鈴木さんの おにいさんは、ピアノが 上手 なの?	
\\	ここから 西へ 300メートル うんてん して 下さい。	
\\	西海岸にある シアトルで 生まれた。	
\\	東では 雪が ふっている。	
\\	ニューヨークは 東海岸 ですか。それとも西海岸ですか。	
\\	東京から 中東へ 飛行機で とんだ。	
\\	日本の 南にある 海が きれい。	
\\	南米の 国を 五つ 言って見て 下さい。	
\\	東南アジアに 行って見たい ですね。	
\\	私の家は 東京の 北に ある。	
\\	この 辞書は 千円です。	
\\	西	
\\	西洋	
\\	西海岸	
\\	東	
\\	東京	
\\	中東	
\\	東海岸	
\\	南	
\\	東南	
\\	西南	
\\	南米	
\\	北	
\\	東北	
\\	西北	
\\	北海道	
\\	千	
\\	千鳥	
\\	私の なまえは、「真理子」と 言う。	
\\	はじめまして。「真理子」と 言います。	
\\	これを「スポーツカー」と 言う。	
\\	これは、「スポーツカー」と 言う。	
\\	私の なまえは、「真理子」と 言う。	
\\	はじめまして。「真理子」と 言います。	
\\	この 音楽は 「フリージャズ」と言う。	
\\	あそこに たっている 人は、「田中」と言います。	
\\	うちの 犬は 「しろ」と言う。	
\\	この のみものは、「ザ・デストロイヤー」と言う。 うまいよ。のんで 見て!	
\\	これは、「ペン」と言う。	
\\	「ペン」と 言う。	
\\	何と 言う?	
\\	これは 何と 言う?	
\\	この ビルは 何と 言いますか。	
\\	この ビルは 「六本木ヒルズ」と 言います。	
\\	この 本は 何と 言うの?	
\\	この たべものは  すごく おいしい。何と 言うの?	
\\	はしる 人	
\\	「橋本」と言う 人	
\\	「日本 毎日」と言う しんぶん	
\\	六本木 と言う ところ	
\\	日本 と言う 国	
\\	私は、「日本 毎日」と言う しんぶんを 毎日 よんでいる。	
\\	六本木 と言う ところを しっていますか。	
\\	日本 と言う 国は、とても うつくしい 国 です。	
\\	すみませんが、真理子さん と言う 人が 電話で まっていますよ。	
\\	この えきは、何と 言いますか。	
\\	ここは、何と 言う えき ですか。	
\\	ここは、成田 と 言う えき です。	
\\	何と 言う 犬 ですか。	
\\	何と 言う まち ですか。	
\\	金曜日に、何と 言う レストランに 行く よてい ですか。	
\\	電話で まっている 人は、何と 言う 人 ですか。	
\\	私は こう 思う。	
\\	本当に そう 思う?	
\\	どう 思う?	
\\	こう 言う。	
\\	真理子さんは、そう 言った。	
\\	ハワイは どう 言う ところ ですか。	
\\	日本は どう 言う 国 ですか。	
\\	鈴木さんは どう 言う 人 ですか。	
\\	どう 言う しごとが ほしい?	
\\	どう 言う こと ですか。	
\\	今日、先生に なりたいと きめた。	
\\	え?どう 言う こと ですか?	
\\	こう 言う こと です。	
\\	そう 言う こと です。	
\\	きのう から、こう 言う ことを 考えている。	
\\	今晩、どう 言う えいがが 見たい?	
\\	鈴木さんの おとうさんは どう 言う 人 ですか。	
\\	鈴木さんの おとうさんは 何 と言う 人 ですか。	
\\	米国の 東海岸は どう言う ところ だった?	
\\	ぼくは「たろう」と言います。	
\\	ジョンさん、さっき 裕子さんが きらいだと 言ったけど、どう 言う こと だったの?	
\\	東京から見た 夕日は きれいだね。	
\\	二人は 夕日の 光を 背中に あびて、東へ あるいた。	
\\	鈴木さんが 黒板の 前に たっている。	
\\	私の 上の名前は 田中です。	
\\	鈴木さんの 下の名前は 何ですか。	
\\	今朝、早く おきた。	
\\	明日、しごとが はじまるのは 早い。	
\\	鈴木君は、まいあさ 学校に 行くのが 早い。	
\\	鈴木さんは 東へ 走っていた。	
\\	私は 走るのが 大好きです。	
\\	春の 青い草の上で 走るのが だいすきだ。	
\\	先日、東の 草原に 大雪が ふった。	
\\	夕方	
\\	夕食	
\\	夕日	
\\	前	
\\	名前	
\\	前日	
\\	早い	
\\	早春	
\\	走る	
\\	競走	
\\	草	
\\	草原	
\\	あら、そう なの?	
\\	へ~、本当に?	
\\	へえ、君は ピアノも ひくんだ!	
\\	これ、おいしい でしょ?	
\\	すごい だろ?	
\\	いい お天気 ですね。	
\\	いい お天気 でしょう。	
\\	いい お天気 でしょ?	
\\	ジョンさんは 上手 でしょ?	
\\	この えいが、面白い だろ?	
\\	じつは、海に 行った ことが ない。	
\\	じつは、まだ 真理子さんに 話していない。	
\\	あまり おいしくない。	
\\	あまり すき じゃない。	
\\	あまり お金が ない。	
\\	うめぼしを たべて 見たけど、あんまり すき じゃなかった。	
\\	とても 行きたくない。	
\\	あまり 行きたくない。	
\\	カラオケが すき?	
\\	やった ことが ないから 分からない。	
\\	じゃ、一ど やって 見て!	
\\	どうやって 分かるの?	
\\	この まどを どうやって あけるの?	
\\	こう やって あける。	
\\	金曜日の よる、何を したい?	
\\	あ、さっき、鈴木君から メールが あったけど、カラオケに 行かないかって。どう 思う?行きたい?	
\\	カラオケですか・・・それは どうかな。	
\\	あら、カラオケは きらいなの?	
\\	いや、じつは、行ったことが ないけど、うたうのは あまり 上手 じゃないんだよ。	
\\	大丈夫だよ。私も 下手だから 心配する ひつよう、ないよ。	
\\	でも、みんなの 前で うたう ん だろ?	
\\	何を言ってるの?カラオケボックスだよ。ともだち だけで 小さな へやに 入って うたうのよ。	
\\	へえ、そう なんだ。それは しらなかった。じゃ、やって 見ようかな。	
\\	ぜったいに たのしいよ。行こう!	
\\	やあ、真理子さん!かいものでも してるの?	
\\	この レストランは 高いよ。メニューを 見てよ。ほら、高い でしょ?	
\\	大丈夫 かい?	
\\	鈴木君も くる かい?	
\\	ねずみや ねこや 犬が いる。	
\\	ねずみとか、ねことか、犬とか、いる。	
\\	日本の どんな たべものが すき ですか。	
\\	どんぶりとか、すごく すき です。	
\\	この 
\\	で どの きょくが 一番 すき?	
\\	はじめて です。	
\\	はじめて ですか。	
\\	カラオケは はじめて です。	
\\	日本は はじめて ですか。	
\\	はじめての カラオケは いつ だった?	
\\	はじめて 日本に 行ったのは いつ だった?	
\\	よう、ジョンさん、カラオケは はじめてだ って きいたけど、本当 かい?	
\\	本当 なんだよ。多分 下手 だから ごめんね。	
\\	いやいや、そんな こと、ないよ。みんな、下手 だから、たのしいよ。日本語の べんきょうにも なる と思うよ。	
\\	じゃ、どうやって やるの?	
\\	この 本から、すきな きょくを えらぶ。そしたら、リモコンを つかって ばんごうを 入力する。そして、自分の ばんを まつ だけ だ。かんたん でしょ?	
\\	なるほど。でも、日本語の きょくを しらないけど。	
\\	大丈夫。えいごの きょくも 一杯 ある。ほら。	
\\	本当だ。じゃ、何が いいかな。ビートルズとか あるかな。	
\\	すきな のみものを えらんで 下さい。	
\\	どう だった?気に 入った?	
\\	ラーメンが 気に 入った。	
\\	その レストランが すごく 気に 入っている。	
\\	おにぎりは あんまり 気に 入らない。	
\\	やっぱり 雨が ふった。	
\\	日本に 行って 見たけど、さむらいは やっぱり いなかったよ。	
\\	ジョンさん、どんな 音楽が すき ですか?	
\\	やっぱり、マドンナとか、ビートルズとか、そんな もの かな。	
\\	あの えいがは やっぱり 面白かった。	
\\	ジョンさんは こないと 思ったけど、やっぱり きた。	
\\	ジョンさん、どの わしょくが 一番 すき?	
\\	むずかしいな。すしも、ラーメンも すきだけど、やっぱり すしかな。	
\\	やっぱり そう ですか。	
\\	ジョンさん、アメリカ人は やっぱり 毎日 ハンバーガーを たべている?	
\\	裕子さんは やっぱり あたまが いいね。	
\\	真理子さんが 言った とおり だった。	
\\	鈴木さんが 言った とおり、先生は 学校に いなかった。	
\\	学校で 学んだ とおり、日本では まんがが 人気 です。	
\\	また ちかいうちに 会いましょう。	
\\	ちかいうちに 真理子さんと テニスを したい。	
\\	真理子さんは ねこの 名前を よんだ。	
\\	明日の パーティーに だれを よんでいる?	
\\	鈴木君と 裕子さんを よんだ。	
\\	ジョンさん、はじめての カラオケは どうだった? 気に 入った?	
\\	すごく 面白かった。やっぱり たのしいね!	
\\	ほら、たのしいって 言った でしょ?	
\\	うん。鈴木君が 言った とおり、日本語の べんきょうにも なるね。また ちかいうちに 行きたいね。	
\\	ぼくも カラオケが 大好き だから、いつでも よんでね。	
\\	自転車に のっている 人が 多いね。	
\\	ジョンさんは、多分、夕食を たべていないと 思う。	
\\	多分 行くよ。	
\\	この古い池に、何が いるんだろう。	
\\	ああ、私の 電話の 電池が 切れた!しまった!	
\\	日本の土地は 高いね。	
\\	私の家では、私のへやが 地下にある。	
\\	鈴木君の ともだちの 名前を 知っている?	
\\	いいえ、知らない。	
\\	その女性は しごとからの 知人です。	
\\	あそこの 池の右に 竹の森が ある。	
\\	真理子さんは 爆竹が きらいだと 言っている。	
\\	あら	
\\	へえ	
\\	いや	
\\	でしょ	
\\	じつは	
\\	あまり	
\\	やる	
\\	うたう	
\\	うた	
\\	メール	
\\	よう	
\\	ほら	
\\	かい	
\\	とか	
\\	そしたら	
\\	きょく	
\\	えらぶ	
\\	はじめて	
\\	なるほど	
\\	ばん	
\\	リモコン	
\\	ビートルズ	
\\	マドンナ	
\\	きに  いる  (気に入る)	
\\	やはり	
\\	やっぱり	
\\	~とおり	
\\	ちかい  うちに	
\\	いつでも	
\\	よぶ	
\\	多い	
\\	多分	
\\	多少	
\\	池	
\\	電池	
\\	切れる	
\\	土地	
\\	地下	
\\	地下鉄	
\\	知る	
\\	知人	
\\	竹	
\\	爆竹	
\\	犬 より	
\\	私 より	
\\	犬 より ねこが すき です。	
\\	私 より あなたが やさしい。	
\\	あなたは やさしい。 あなたが やさしい。	
\\	だれが やさしい?	
\\	私 より あなたが やさしい。	
\\	犬 より 馬が つよい。	
\\	ねこ より ねずみが 小さい。	
\\	人参 より 餃子が おいしい。	
\\	タクシーより 地下鉄が はやい。	
\\	こちらの ほうへ どうぞ。	
\\	私は 北の ほうから きました。	
\\	私は 東京の ほうから きた。	
\\	明日 から おとうさんは アメリカの ほうへ 行く。	
\\	フランスご より イタリアごの ほうが かんたん だ。	
\\	コーヒーより おちゃの ほうが すき です。	
\\	大阪の ラーメンより 東京の ラーメンの ほうが おいしい。	
\\	ふつう 電車 より しんかんせんの ほうが はやい。	
\\	私より あなたの あたまが いい。	
\\	裕子さんより 真理子さんの 足が ながい。 裕子さんより 真理子さんの ほうが 足が ながい。	
\\	ジョンさんより 裕子さんの せが 高い。	
\\	鈴木君より ジョンさんの うでが つよい。	
\\	真理子さんより 鈴木さんの ほうが あたまが いい。 鈴木さんの ほうが 真理子さんより あたまが いい。	
\\	ラーメンの ほうが すき です。	
\\	ジョンさんの ほうが 面白い。	
\\	日本の 土地の ほうが 高い。	
\\	日本 より あつい。	
\\	きのう より さむい。	
\\	しごと より たのしい。	
\\	マクドナルドは どっち?	
\\	こっち だよ。	
\\	真理子さんは あっちの ほうへ 行った。	
\\	犬と ねこと 鳥が いた。	
\\	犬と ねこと 鳥と、どっちの ほうが すき ですか。	
\\	さけと ワインと、どっちの ほうが すき ですか。	
\\	さけの ほうが すきです。	
\\	わしょくと ようしょくと、どっちの ほうが おいしい?	
\\	わしょくの ほうが おいしいと 思う。	
\\	これと それと、どっちの ほうが 高い?	
\\	そっちの ほうが 高い。	
\\	バスと 電車と、どっちの ほうが はやい?	
\\	電車の ほうが ぜったいに はやい。	
\\	日本では、モスバーガーと、マクドナルドと、どっちが 多い?	
\\	モスバーガーの ほうが 多いかな。	
\\	おはしと、ナイフと フォークと、どっちの ほうが つかいやすい?	
\\	わるい けど、ナイフと フォークのほうが つかいやすい。ごめんね、日本!	
\\	ああ、わるかった。	
\\	えいごと 日本語と、どっちの ほうが いい?	
\\	日本語の ほうが いい。	
\\	オレンジジュースと レモネードと、どっちのほうが いい?	
\\	オレンジジュースの ほうが いい。	
\\	おにぎりは、しゃけと ツナマヨと、どっちの ほうが いいですか?	
\\	ツナマヨの ほうが いい です。	
\\	水と おちゃと、どっちが いい?	
\\	おちゃが いい。 おちゃが いいです。	
\\	真理子さん、今日の ひるごはんは、ラーメンと うどんと、どっちが いい?	
\\	うどんが いい!	
\\	北海道と カナダと、どっちが さむい?	
\\	カナダが さむい。	
\\	コーヒーは? のみますか。	
\\	いいです。	
\\	コーヒーは いいです。	
\\	ああ、そうですか。分かりました。	
\\	これが いいです。	
\\	これで いいです。	
\\	ビールを のみませんか。	
\\	いい ですね。	
\\	おお、これが いいね。	
\\	コーヒーが いい。	
\\	それは いいです。	
\\	いいです。	
\\	コーヒーは いいです。	
\\	どっちが いいか、きまりましたか。	
\\	これが いいと 思います。	
\\	ジョンさん、日本の さけを のむ?	
\\	いいね。	
\\	ジョンさん、うめぼしを たべる?	
\\	いい。	
\\	しごとに 行くのと、テレビゲームを するのと、どっちが いい?	
\\	テレビゲームを するのが いい。	
\\	テレビゲームが いい。	
\\	テレビゲームが したい。	
\\	家が きたないね。今日 そうじ するのと、明日の あさ そうじ するのと、どっちが いい?	
\\	明日の あさに しよう。	
\\	おなかが すきましたね。うちで たべるのと、レストランに たべに 行くのと、どっちが いい ですか?	
\\	ここで たべましょう。	
\\	土曜日に、えいがを 見に 行くのと、ハイキングを するのと、どっちが いい?	
\\	ハイキングが いいね。	
\\	私のへやに 虫が いる!	
\\	虫歯が なくて うれしい。	
\\	東京の北に、「さわ」と言う、小さな村が ある。	
\\	中村	
\\	木村	
\\	西村	
\\	田村	
\\	この町は 小さいけど、本当に きれい だね。	
\\	東京は たのしい 町だよ。	
\\	鈴木君は 足が長いね。	
\\	来月までの 時間が 長くない。	
\\	私は大学一年生です。	
\\	いい 年 だったね。	
\\	今年は 田舎の 村に 行って見たい。	
\\	より	
\\	ほう	
\\	こっち	
\\	そっち	
\\	あっち	
\\	どっち	
\\	虫	
\\	虫歯	
\\	昆虫	
\\	村	
\\	農村	
\\	町	
\\	町内	
\\	長い	
\\	町長	
\\	校長先生	
\\	社長	
\\	年	
\\	一年生	
\\	今年	
\\	来年	
\\	あの 時は たのしかったね。	
\\	たべた ラーメン 私が たべた ラーメン	
\\	私が たべた 時	
\\	あの虫を 見た時	
\\	私たちが 日本に 行った時	
\\	真理子さんと 一緒に カラオケに 行った時	
\\	私たちが 日本に 行った時は、たのしかったね。	
\\	真理子さんと 一緒に カラオケに 行った時を おぼえている。	
\\	去年 日本に 行った 時、真理子さんに 会いました。	
\\	はじめて すしを たべた時、おいしいと 思わなかった。	
\\	おとうさんが 家に かえった時、私は もう ねていた。	
\\	私が ドアを あけた 時に、ねこは 出て 行った。	
\\	鈴木君、私が 電話 した時、何を していた?	
\\	ともだちが 家に かえった 時、私は もう そこに いた。	
\\	真理子さんは、ジョンさんが トイレに 行った時、ジョンさんの 電話を 見ました。	
\\	私は、しごとに 行く時、いつも かさを 持って行く。	
\\	日本に 行く時、コンビニで おにぎりを かうのが すき です。	
\\	ともだちと あそぶ 時、よく えいがを 見に 行ってる。	
\\	えきの 前に、ゆうびんきょくが あります。	
\\	夕食の 前に、手を あらった。	
\\	フランスに 行く 前に、フランスごを 学びたい。	
\\	夕食を たべる 前に、手を あらった。	
\\	はを みがく 前に、シャワーを あびた。	
\\	電車に のる 前に、切符を かった。	
\\	その きょくを うたう 前に、ビールを のんだ。	
\\	学校の あとで、何を したい?	
\\	夕食の あとで、えいがを 見ようか。	
\\	しごとから かえった あとで、おふろに 入りました。	
\\	校長先生と 話した あとで、学校を 出た。	
\\	明日の あさ、ごはんを たべた あとで、はを みがく つもり です。	
\\	カラオケの あと、のみに 行こうか。	
\\	たべた あと、何を しようか。	
\\	日本に いる うちに	
\\	日本に いる うちに 日本語を 一杯 学びたい。	
\\	やすい うちに かいましょう。	
\\	雨が ふらない うちに	
\\	わすれない うちに	
\\	雨が ふらない うちに  行きましょう。	
\\	わすれない うちに 話したい ことが ある。	
\\	時間が ない。	
\\	スーパーに 行く 時間が ありません。	
\\	テレビを 見る 時間が ない。	
\\	うたを うたう 時間が ない。	
\\	四日かん	
\\	十日かん	
\\	30日かん	
\\	四日かん 中国に いました。	
\\	ジョンさんは 二日かん ねた。	
\\	180日かん 旅行 した。	
\\	きのう、真理子さんは 一日中 テレビを 見た。	
\\	ジョンさんは 一晩中 テレビゲームを していた。	
\\	一時かん	
\\	二時かん うたいました。	
\\	一しゅうかん	
\\	三しゅうかん 日本に 行くことに なっている。	
\\	六しゅうかん フランスに いる よてい です。	
\\	一年かん	
\\	日本に 行く前に、5年かん 日本語を べんきょう しました。	
\\	一ヶ月かん	
\\	四ヶ月かん	
\\	10ヶ月かん	
\\	一ヶ月かん イタリアに いました。	
\\	ちちは しごとで 韓国に 三ヶ月かん 行く ことに なりました。	
\\	私は 6ヶ月かん 日本で えいごを おしえる よてい だ。	
\\	朝の 空気が おいしいね。	
\\	今朝、何を たべたい?	
\\	毎朝の 朝食は 何を たべている?	
\\	今日は どんな 天気かな?	
\\	今晩、天の川が きれい だね。	
\\	今朝、校長先生が 道を あるいているのを 見た。	
\\	北海道に 行ったこと、ある?	
\\	今日は いい天気になると、今朝のしんぶんで 読んだ。	
\\	私は 本を読むのが 大好きです。	
\\	とき  (時)	
\\	まえ  (前)	
\\	まえに  (前に)	
\\	あと	
\\	あとで	
\\	うち	
\\	じかん  (時間)	
\\	~かん	
\\	いちにちじゅう  (一日中)	
\\	ひとばんじゅう  (一晩中)	
\\	~じかん  (~時間)	
\\	~しゅうかん	
\\	~かげつかん  (~ヶ月間)	
\\	~ねんかん  (~年間)	
\\	朝	
\\	今朝	
\\	朝一	
\\	早朝	
\\	天	
\\	天気	
\\	天才	
\\	天の川	
\\	道	
\\	水道	
\\	北海道	
\\	読む	
\\	読書	
\\	読者	
\\	すみません。ラーメンは どの れつに あるん でしょうか。	
\\	ラーメンは 3ばんの れつに あります。	
\\	やさいの コーナーは どこ ですか。	
\\	かんづめの しゃけを うっていますか。	
\\	うっていますよ。10ばんの れつに あります。	
\\	すみません。しょうゆを さがして いるん ですが、どこ でしょうか。	
\\	あ、見つけた。	
\\	アイスクリームは 牛乳の となりに ある。	
\\	パンのコーナーは、やさいのコーナーの よこに ある。	
\\	魚の コーナーは おくに ある。	
\\	かんづめの やさいは レジの 手前の れつに ある。	
\\	花は みせの むこうがわに あります。	
\\	すみません。この しょくパンは 薄切り ですが、厚切りの パンも ありますか。	
\\	朝、クロワッサンを たべるのが 大好き。	
\\	ハヤシライスの ルーを かいたいん ですが、どこ でしょうか。	
\\	今日 カレーライスを 作る 時間が ないから、カップラーメンで いいかな。	
\\	この本は、半ば まで 読んでいる。	
\\	そのえいがは 半ば まで 見たけど、面白くなかったから えいがかん から 出た。	
\\	今 2時半 です。	
\\	今朝 六時半に 家を 出た。	
\\	裕子さんが 私の 春巻きを 半分 たべた。	
\\	しょくパンは 二百円 です。	
\\	その みせの中に 多分、百人 いると 思います。	
\\	私の父は この町の 町長です。	
\\	お父さん、来月 北海道に 行くって きいたけど、本当なの?	
\\	私の母は 字がきれい だ。	
\\	お母さん、何を読んでる?	
\\	ジョンさんの 母国語は 何ですか。	
\\	今日、風が つよい。	
\\	明日、東京に 台風が くる。	
\\	れつ	
\\	コーナー	
\\	びん	
\\	はこ	
\\	かん	
\\	かんづめ	
\\	たな	
\\	さがす	
\\	うる	
\\	となり	
\\	よこ	
\\	おく	
\\	てまえ  (手前)	
\\	はしっこ	
\\	むこう	
\\	しょくパン	
\\	あつい	
\\	うすい	
\\	あつぎり  (厚切り)	
\\	うすぎり  (薄切り)	
\\	クロワッサン	
\\	ピザ	
\\	ヨーグルト	
\\	にく	
\\	ぎゅうにく  (牛肉)	
\\	ぶた	
\\	ぶたにく	
\\	とりにく	
\\	ラム	
\\	ハム	
\\	えび  (海老)	
\\	カレーライス	
\\	ハヤシライス	
\\	ルー	
\\	カップラーメン	
\\	ほうれんそう  (ほうれん草)	
\\	なす  (茄子)	
\\	ピーマン	
\\	きゅうり	
\\	いも	
\\	じゃがいも	
\\	さつまいも	
\\	かぼちゃ	
\\	半ば	
\\	一時半	
\\	半分	
\\	百	
\\	百円	
\\	父	
\\	お父さん	
\\	母	
\\	お母さん	
\\	母国	
\\	母国語	
\\	風	
\\	風車	
\\	台風	
\\	私は ともだちに クロワッサンを 上げた。	
\\	手を 上げて 下さい。	
\\	私は 大統領に りんごを 差し上げました。 大統領に りんごを 差し上げた。	
\\	犬に ごはんを やった。	
\\	裕子さん、私は きのう りんごを 大統領に 差し上げました。	
\\	ねこに 水を やりました。	
\\	山田さん、これを 差し上げます。	
\\	ともだちが このコップを くれた。	
\\	先生が この本を 下さった。	
\\	会社の 社長が 私に プレゼントを 下さった。	
\\	真理子さん、水を 少し くれる?	
\\	今朝、校長先生が ぼくに この本を 下さった。	
\\	むかし、お母さんが 私に この自転車を くれた。	
\\	となりの家の 石田さんが この 茄子を 下さった。	
\\	私は 帽子を もらった。	
\\	トランペットを いただいた。	
\\	お金を もらった。	
\\	先生から えんぴつを いただいた。	
\\	ともだちから シャツを もらった。	
\\	先日、知人に 子犬を もらった。	
\\	ともだちに プレゼントを いただいた。	
\\	ともだちが 私に お金を くれた。	
\\	私は ともだちから お金を もらった。	
\\	真理子さんは 健二君に プレゼントを 上げた。	
\\	社長が 私の つまに ペンを 下さった。	
\\	真理子さんは ともだちから プレゼントを もらった。	
\\	鈴木君は しごとの どうりょうに プレゼントを もらった。	
\\	主人は 大統領から 電話を いただいた。	
\\	先生が 私の 息子に ペンを くれた。 先生が 私の 息子に ペンを 下さった。	
\\	今 この本を読んでいるけど、この文が 分からなかった。	
\\	鈴木君は 大学で 日本の文学を べんきょう している。	
\\	文子さんは 文学が 大好きだと 言った。	
\\	車を 早く 止めて 下さい。	
\\	電車が 川の上で 止まった。	
\\	先生が 今日のクラスを 中止した。	
\\	どっちの方が いい?	
\\	女性は 東の方へ 走って行った。	
\\	すみません。日本の方ですか。	
\\	木村さんは とても親切な方です。	
\\	君の名は何? 君の名前は何?	
\\	何時に 帰るつもり ですか。	
\\	真理子さんは もう 帰った。	
\\	ジョンさん、いつ 帰国するんですか。	
\\	あげる  (上げる)	
\\	さしあげる  (差し上げる)	
\\	やる	
\\	くれる	
\\	くださる  (下さる)	
\\	もらう	
\\	いただく	
\\	文	
\\	文学	
\\	文明	
\\	文化	
\\	止める	
\\	止まる	
\\	中止	
\\	中止する	
\\	方	
\\	方	
\\	夕方	
\\	方言	
\\	名前	
\\	名人	
\\	有名	
\\	帰る	
\\	帰り道	
\\	帰国	
\\	帰国する	
\\	この本を だれかに 上げたい。	
\\	その人は 何かを たべている。	
\\	いつか 日本に 行って 見たいな。	
\\	鈴木君は どこかに 行ったと 思う。	
\\	ジョンさんは なぜか 田中さんと 話している。	
\\	どうにか、来年 イタリアに 行きたい。	
\\	どこかに 行きたい。	
\\	面白い ところに 行きたい。	
\\	どこか、面白い ところに 行きたい。	
\\	おいしい ものが たべたいね。	
\\	何か、おいしい ものが たべたいね。	
\\	コーヒーを のみたい 人が いますか。	
\\	だれか、コーヒーを のみたい 人が いますか。	
\\	時間が ある 時、ジョンさんと ハイキングに 行きたい。	
\\	いつか、時間が ある 時、ジョンさんと ハイキングに 行きたい。	
\\	有名な 人に 会いたいな。	
\\	だれか、有名な 人に 会いたいな。	
\\	私も 行きたい。	
\\	ラーメンを たべる。ピザも たべる。	
\\	こうえんに 行った。えきにも 行った。	
\\	ねずみが いた。ねこも 犬も 馬も いた。	
\\	鳥も 魚も 犬も いた。わにも いた。	
\\	私は 何も たべていない。	
\\	裕子さんは だれも すき じゃない。	
\\	何も ない。	
\\	山田さんは 何も しない。	
\\	ここに だれも いない。	
\\	ジョンさんは いつも やさいを たべないね。	
\\	だれも こない。	
\\	どれも すき じゃない。	
\\	だれにも 上げない。	
\\	だれにも 言わない。	
\\	真理子さんに 上げない。裕子さんにも 上げない。だれにも 上げない。	
\\	学校に 行かない。しごとにも 行かない。どこにも 行かない。	
\\	花子ちゃんは だれとも 話さないね。	
\\	だれにも 見せない。	
\\	だれにも 会わない つもり です。	
\\	土曜日は いつも 車を あらっている。	
\\	鈴木君は いつも 家に いない。	
\\	どこも 面白い。	
\\	どこも 面白くない。	
\\	これを たべても いい ですか。	
\\	ジョンさんの 家に 行っても、ジョンさんは いません でした。	
\\	その 本を 読んでも、分からなかった。	
\\	分からない ことが あっても、大丈夫 です。	
\\	パーティーに 知らない 人が いても、たのしんで 下さい。	
\\	何を たべても・・・	
\\	パンを たべても、ごはんを たべても、何を たべても・・・	
\\	どこを 見ても	
\\	だれと 話しても	
\\	どう 考えても	
\\	いつ 行っても	
\\	どこを 見ても、私の犬が いない。	
\\	だれと 話しても、面白い。	
\\	どう 考えても、分からない。	
\\	いつ 行っても、りょうりが うまい。	
\\	おちゃ でも	
\\	コーヒーでも	
\\	何でも	
\\	だれでも	
\\	いつでも	
\\	どこでも	
\\	私は 何でも、たべる。	
\\	いつでも 私の 家に きて 下さい。	
\\	だれでも ジョンさんの ことが すき です。	
\\	どれでも 大丈夫 だ。	
\\	だれと でも	
\\	真理子さんは だれとでも かんたんに 話す。	
\\	どこへでも	
\\	どこへでも 鈴木君と 行くよ。	
\\	だれにでも	
\\	ジョンさんは だれにでも お金を 上げる。	
\\	真理子さんは 髪の毛は 何色ですか。	
\\	裕子さんは 髪の毛が 長いね。	
\\	家が さむいから 毛布を くれる?	
\\	ここで 一時間も まっていたよ。	
\\	日本に 二ヶ月間 いるつもり です。	
\\	その間、何をしていたの?	
\\	ジョンさんが スーパーに 行っている間、私は 車を あらうね。	
\\	間もなく、電車が まいります。	
\\	今日の夜、台風が くるって きいたよ。	
\\	夜中に おきた時、お母さんが 家に いなかった。	
\\	今夜、えいがを 見ようか。	
\\	今夜、帰り道で 友達に 会いました。	
\\	君は とても 大切な 友人だ。	
\\	この会社は 風車を 用いている。	
\\	あの人は 色々なものを 用いて、音楽を 作っている。	
\\	子ども用の 本が ありますか。	
\\	この ズボンは 男性用じゃないの?	
\\	だれか	
\\	なにか  (何か)	
\\	いつか	
\\	どこか	
\\	なぜか	
\\	どうにか	
\\	だれも	
\\	なにも  (何も)	
\\	どれも	
\\	いつも	
\\	どこも	
\\	毛	
\\	毛虫	
\\	毛布	
\\	間	
\\	間もなく	
\\	時間	
\\	一年間	
\\	夜	
\\	夜中	
\\	今夜	
\\	友達	
\\	用いる	
\\	用紙	
\\	~用	
\\	用心	
\\	用心する	
\\	ジョンさんに プレゼントを 上げた。	
\\	ジョンさんに プレゼントを かって 上げた。	
\\	花子ちゃんに 本を 読んで 上げた。	
\\	一緒に 行って 上げるよ。	
\\	ジョンさんに 数学を おしえて 上げている。	
\\	あなたが しごとばから 帰るのは おそい だろうけど、まって 上げるよ。	
\\	裕子さんは 花子ちゃんに 本を 読んで 上げた。	
\\	鈴木君は ジョンさんに お金を かして 上げた。	
\\	真理子さんの ため	
\\	真理子さんの ために	
\\	鈴木さんは 真理子さんの ために 車を あらって 上げた。	
\\	父が 私に 車を かって くれた! 父が 車を かって くれた!	
\\	父が 車を かった。	
\\	父が 車を かって 上げた。	
\\	父が 車を かって くれた。	
\\	友達が 面白い 本を かして くれた。	
\\	ジョンさんが レストランの 会計を はらって くれた。	
\\	私が 病気 だったから、真理子さんが 家を そうじ して くれた。	
\\	先生、りんごを 下さい。	
\\	先生は りんごを 下さった。	
\\	先生、まって 下さい。	
\\	先生は まって 下さった。	
\\	おい、まって くれ。	
\\	書いて くれる?	
\\	一緒に べんきょう して くれる?	
\\	ジョンさん、ちょっと まって くれる?	
\\	真理子さん、私に 日本語を おしえて くれる?	
\\	レモネードを のまない?	
\\	これを かって くれない?	
\\	私と 一緒に 行って くれない?	
\\	一緒に えいがを 見ない?	
\\	ビールを のまない?	
\\	ロビーまで きて 下さる? ロビーまで きて 下さいますか。	
\\	また らいしゅう きて 下さいませんか。	
\\	もう 少し ゆっくり 話して 下さいませんか。	
\\	私は ジョンさんに 本を もらった。	
\\	私は ジョンさんに 本を かって もらった。	
\\	私に ジョンさんが 本を かって くれた。	
\\	お父さんに 自転車を かって もらった。	
\\	私が 書いた本を、国語の 先生に 見て いただいた。	
\\	真理子さんに きめて もらう。	
\\	真理子さんに 大学で 学んでいる 数学を おしえて もらった。	
\\	ジョンさんに ビートルズの きょくを うたって もらった。	
\\	校長先生に 説明して いただきました。	
\\	ちょっと まって もらって いい?	
\\	これを 見て もらって いい?	
\\	これを 見て いただいても よろしい ですか。	
\\	ちょっと まって いただいても よろしい ですか。	
\\	ジョンさんの 友人は 今夜のパーティーに 来る?	
\\	ちょっと コンビニに 行って来るね。	
\\	私が 長い間 まっていた 手紙が やっと 来た!	
\\	来月 雪が ふるかな?	
\\	一人の男性が 門の前に 立っている。	
\\	あそこに 立っている人を知っていますか。	
\\	ここは 国立 こうえん です。	
\\	この大学は 国立ですか。それとも私立ですか。	
\\	川の前に、林がある。	
\\	林田	
\\	林	
\\	小林	
\\	「りんご」の漢字は、「林檎」だって、知ってた?	
\\	林檎?	
\\	悪いけど、ちょっと ここに 来てくれる?	
\\	わあ、この 餃子は 悪くないね。	
\\	ああ、私は いつも 悪運が つよい。	
\\	あのレストランは 安いけど、たべものは 悪くないよ。	
\\	学校の帰り道に、安くておいしい ラーメンやを 見つけた。	
\\	気をつけるから、どうぞ、安心して 下さい。	
\\	ため	
\\	くださいます  (下さいます)	
\\	来る	
\\	来月	
\\	来年	
\\	立つ	
\\	私立	
\\	国立	
\\	林	
\\	森林	
\\	林学	
\\	悪い	
\\	悪運	
\\	安い	
\\	安心	
\\	安心する	
\\	安全	
\\	お水を のむ? 水を のむ?	
\\	明日は いい お天気 でしょう。 明日は いい 天気 だろう。	
\\	私と えいがを 見に 行かない? ぼくと えいがを 見に 行かないか。	
\\	あなた、どう したの?	
\\	これで 大丈夫。	
\\	これで 大丈夫よ。	
\\	これで 大丈夫だよ。	
\\	あの人は きれいね。	
\\	あの人は きれい だね。	
\\	今から 帰るよ。	
\\	すごいね。	
\\	ジョンさんは どこに いる かしら。	
\\	裕子さんは 本当に そう 言った かしら。	
\\	裕子さんは 大丈夫 かしら。	
\\	ジョンさんの お母さんは どんな 人 かしら。	
\\	つかれたから、帰りますわ。	
\\	裕子さんの ドレスは すごく きれいだわ。	
\\	ああ、たのしかったわ。ありがとう。	
\\	そう ですわ。	
\\	すごく 行きたいのだ。 すごく 行きたいの。 すごく 行きたいんだ。 すごく 行きたいんだよ。	
\\	すごく 行きたいのよ。	
\\	音楽は とても 良かった。	
\\	石田さんは とても 良い人です。	
\\	私は 悪いことを したから 良心が いたい。	
\\	この ちかくに 駅が ありますか。	
\\	東京駅は どこ ですか。	
\\	駅は ここから 近いですか。	
\\	この近くに、有名な 町がある。	
\\	裕子さんは 最近 何をしている?	
\\	ここから 駅は 遠いですか。	
\\	いいえ、遠くないよ。あるいて 5分 です。	
\\	この お菓子を 永遠に たべても いい。	
\\	今日どこで 食べたい?	
\\	朝食を 食べる時間が なかった。	
\\	一緒に 食事を しようか。	
\\	かしら	
\\	わ	
\\	良い	
\\	良心	
\\	駅	
\\	東京駅	
\\	近い	
\\	近道	
\\	最近	
\\	遠い	
\\	永遠	
\\	食べる	
\\	朝食	
\\	夕食	
\\	食事	
\\	ジョンさん、おてらに 行ったこと、ある?	
\\	ないね。この 近くに あるの?	
\\	うん、すごく 古くて 有名な おてらが あるよ。	
\\	何と言う おてらなの?	
\\	「成田山」と言って、成田 と言う 町の 近くに ある。今日 行って見る?	
\\	前から おてらに 行って見たいと 思ってたから、いいね。	
\\	一時間 ぐらい	
\\	東京まで、30分くらい です。	
\\	まず、真理子さんが 何を したいか、きいて 見たい。	
\\	まずは、スーパーに 行く。それから 銀行 だ。	
\\	ジョンさんは 明日の 何時に 家に つく?	
\\	30に しごとに ついた。	
\\	(出発する じゅんびをした あと・・・)	
\\	良し!行きましょう。ところで、成田山には、どうやって 行く?	
\\	まずは 電車で、JR 成田駅 まで 行く。そこから、20分くらい あるいて 成田山に つく。成田山までの 道も 面白い から、カメラを わすれないでね。	
\\	大丈夫だよ。ほら、もってる。	
\\	今日、お天気が すごく いいから、一杯 たのしもう。	
\\	鈴木君の 家までの 道を 知っている?	
\\	道が 分からなく なった。	
\\	子どもの ころ	
\\	大学生の ころ	
\\	子どもの ころ、お菓子が 大好き だった。	
\\	アメリカに いた ころ、毎日 えいがを 見てた。	
\\	まっすぐ 行って 下さい。	
\\	右へ まがって 下さい。	
\\	あそこの かどで、右へ まがって 下さい。	
\\	あそこの かどを、左へ まがって 下さい。	
\\	そのまま、食べても いいよ。	
\\	そのまま、すわって いて 下さい。	
\\	そのまま、ちょっと まって もらっても いい?	
\\	もう 行きました。	
\\	もう 少し です。	
\\	もう ちょっと 食べたい。	
\\	もう 少し 考えても いい?	
\\	もう 一つ	
\\	もう 一つ 食べたい。	
\\	もう 一ど 言って 下さい。	
\\	ここが 成田か。名前は きいたことが あったけど、来たことは なかった。きれいな 町だね。	
\\	そうだね。私は 子どもの ころ、おじいさんと 一緒に 成田山を 見に来た。	
\\	それで、ここから どう 行くんだ?	
\\	私は 道を よく 知っているけど、ジョンさんのために この 地図を 持って来て 上げた。	
\\	(地図を 見る)	
\\	今 私たちは ここ。成田山は ここ。この かどを 左に まがって、そこから まっすぐ あるくの。	
\\	なるほど。	
\\	そしたら この かどで 右に まがって、さかを 下る。	
\\	なるほどね。	
\\	そのまま もう 少し あるく。そしたら、おてらは 左側に ある。遠くないよ。	
\\	いいね。じゃ、行こう!	
\\	わあ、本当に 面白い 道だね。	
\\	かわいい おみせが たくさん ある でしょ?	
\\	そうだね。	
\\	お土産でも かいたい?	
\\	いいね。あそこの みせを ちょっと 見て 見よう。	
\\	(成田山に つく 時・・・)	
\\	ほ!これが 成田山か。すごいね!	
\\	きれい でしょ。このおてらは 千年前に「かんちょう」と言う 人が 設立 したの。	
\\	千年も 前に?	
\\	そう。その時は 小さな おてら だった。あとで ひろく なった。中を 見て 見たい?	
\\	見たい!	
\\	花子ちゃんは おにぎりを 4こ 食べた。	
\\	4こも 食べた?!	
\\	鈴木君は ビールを 7本も のんだ。明日の 朝、あたまが いたい だろうね。	
\\	(口を きよめるばしょで・・・)	
\\	日本では、おてらに 入る前に、口と手を きよめる しゅうかんが ある。	
\\	口と手を きよめる?	
\\	そうだよ。水を くんで、まずは 手をあらう。それから、口を ゆすぐの。私が 今 するから 見てね。・・・どう?分かった?	
\\	分かったと 思う。	
\\	ジョンさんも して 見てね。	
\\	(ジョンが 手と口を きよめる)	
\\	そう、そう。	
\\	こう 言う しゅうかんが あると きいた こと、なかった。面白い。	
\\	おてらに 行くことは、「おまいり」と言う。毎朝 おまいりに 来る人も いるよ。	
\\	そうか。	
\\	おてらの ビルは、「どう」と言う。入る時、くつを ぬぐの。入って 見たい?	
\\	うん、入りたい。	
\\	(あとで・・・)	
\\	すべての おてらは そう じゃないけど、成田山は、うしろの 方に うつくしい こうえんが ある。	
\\	こうえんも あるの? かんどうするな。ここは 本当に すごい ところ だね。	
\\	この 池に、こいとか かめとか、色んな どうぶつが いる。しずかで すてきな ばしょ でしょ?	
\\	うん、本当に そうだね。成田山は こんなに きれいな ところ だと 知らなかった。	
\\	気に入って くれて 良かった。すごく うれしい。	
\\	じんじゃ	
\\	しんとう  (神道)	
\\	おてら	
\\	ぶっきょう	
\\	とりい  (鳥居)	
\\	ふるい  (古い)	
\\	じゅんび	
\\	じゅんび  する	
\\	ところで	
\\	まず	
\\	~くらい	
\\	カメラ	
\\	わすれる	
\\	つく	
\\	ころ	
\\	かど	
\\	まがる	
\\	まっすぐ	
\\	さか	
\\	まま	
\\	もう  すこし  (もう  少し)	
\\	おみやげ  (お土産)	
\\	かわいい	
\\	せつりつ  (設立)	
\\	せつりつ  する  (設立する)	
\\	ほ	
\\	きよめる	
\\	しゅうかん	
\\	くむ	
\\	ゆすぐ	
\\	ばしょ	
\\	ぬぐ	
\\	おまいり	
\\	かんどう	
\\	かんどう  する	
\\	すべて	
\\	こい	
\\	かめ	
\\	どうぶつ	
\\	いろんな  (色んな)	
\end{CJK}
\end{document}