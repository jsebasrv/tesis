\documentclass[8pt]{extreport} 
\usepackage{hyperref}
\usepackage{CJKutf8}
\begin{document}
\begin{CJK}{UTF8}{min}
\\	それ	
\\	それはとってもいい話だ。	
\\	それ 
\\	それ は とっても いい はなし だ			
\\	見る	
\\	みる			
\\	私は絵を見るのが好きです。	
\\	見る 
\\	わたし は え を みる の が すき です			
\\	円	
\\	えん			
\\	そこに大きな円を描いて。	
\\	円 
\\	そこ に おおき な えん を かいて			
\\	多い	
\\	おおい			
\\	京都にはお寺が多い。	
\\	多い 
\\	きょうと に は おてら が おおい			
\\	家	
\\	うち			
\\	家に遊びに来てください。	
\\	家 
\\	うち に あそび に きて ください			
\\	これ	
\\	これをください。	
\\	これ 
\\	これ を ください			
\\	新しい	
\\	あたらしい			
\\	私	
\\	わたし			
\\	私が行きましょう。	
\\	私 
\\	わたし が いきましょう			
\\	仕事	
\\	しごと			
\\	3月は仕事が忙しい。	
\\	仕事 
\\	さんがつ は しごと が いそがしい			
\\	始める	
\\	はじめる			
\\	テストを始めてください。	
\\	始める 
\\	てすと を はじめて ください			
\\	なる	
\\	彼は医者になりました。	
\\	なる 
\\	かれ は いしゃ に なりました			
\\	する	
\\	出る	
\\	でる			
\\	今朝は早く家を出ました。	
\\	出る 
\\	けさ は はやく いえ を でました			
\\	使う	
\\	つかう			
\\	このパソコンを使ってください。	
\\	使う 
\\	この ぱそこん を つかって ください			
\\	ところ	
\\	私は友達のところに泊まった。	
\\	ところ 
\\	わたし は ともだち の ところ に とまった			
\\	思う	
\\	おもう			
\\	私もそう思います。	
\\	思う 
\\	わたし も そう おもいます			
\\	持つ	
\\	もつ			
\\	私は車を持っています。	
\\	持つ 
\\	わたし は くるま を もって います			
\\	時間	
\\	じかん			
\\	今は時間がありません。	
\\	時間 
\\	いま は じかん が ありません			
\\	今年	
\\	ことし			
\\	今年はイタリアに旅行したい。	
\\	今年 
\\	ことし は いたりあ に りょこう したい			
\\	知る	
\\	しる			
\\	誰もその話を知らない。	
\\	知る 
\\	だれ も その はなし を しらない			
\\	作る	
\\	つくる			
\\	今、朝ご飯を作っています。	
\\	作る 
\\	いま あさごはん を つくって います			
\\	同じ	
\\	おなじ			
\\	彼の日本語のレベルは私と同じ位だ。	
\\	同じ 
\\	かれ の にほんご の れべる は わたし と おなじ くらい だ			
\\	今	
\\	いま			
\\	彼は今、勉強しています。	
\\	今 
\\	かれ は いま べんきょう して います			
\\	続く	
\\	つづく			
\\	工事は3月まで続きます。	
\\	続く 
\\	こうじ は さんがつ まで つづきます			
\\	国	
\\	くに			
\\	未だ	
\\	まだ			
\\	やる	
\\	一緒に宿題をやろう。	
\\	やる 
\\	いっしょ に しゅくだい を やろう			
\\	言う	
\\	いう			
\\	上司が「一杯、飲もう。」と言った。	
\\	言う 
\\	じょうし が いっぱい のもう と いった			
\\	少ない	
\\	すくない			
\\	今年は雨が少ないです。	
\\	少ない 
\\	ことし は あめ が すくない です 。			
\\	行く	
\\	いく			
\\	決める	
\\	きめる			
\\	帰国することに決めました。	
\\	決める 
\\	きこく する こと に きめました			
\\	高い	
\\	たかい			
\\	これがこの町で一番高いビルです。	
\\	高い 
\\	これ が この まち で いちばん たかい びる です			
\\	考える	
\\	かんがえる			
\\	よく考えてください。	
\\	考える 
\\	よく かんがえて ください			
\\	感じる	
\\	かんじる			
\\	膝に痛みを感じます。	
\\	感じる 
\\	ひざ に いたみ を かんじます			
\\	子供	
\\	こども			
\\	電車で子供が騒いでいた。	
\\	子供 
\\	でんしゃ で こども が さわいで いた			
\\	そう	
\\	私もそう思います。	
\\	そう 
\\	わたし も そう おもいます			
\\	大学	
\\	だいがく			
\\	もう	
\\	彼はもう帰りました。	
\\	もう 
\\	かれ は もう かえりました 。			
\\	近く	
\\	ちかく			
\\	駅の近くで食事をした。	
\\	近く 
\\	えき の ちかく で しょくじ を した			
\\	入れる	
\\	いれる			
\\	彼はかばんに手帳を入れた。	
\\	入れる 
\\	かれ は かばん に てちょう を いれた			
\\	学生	
\\	がくせい			
\\	彼は真面目な学生です。	
\\	学生 
\\	かれ は まじめ な がくせい です			
\\	午後	
\\	ごご			
\\	明日の午後、お客様が来る。	
\\	午後 
\\	あす の ごご おきゃくさま が くる			
\\	長い	
\\	ながい			
\\	彼女の髪はとても長い。	
\\	長い 
\\	かのじょ の かみ は とても ながい			
\\	本	
\\	ほん			
\\	本を1冊買いました。	
\\	本 
\\	ほん を いっさつ かいました			
\\	よく	
\\	彼女はよく旅行に行きます。	
\\	よく 
\\	かのじょ は よく りょこう に いきます			
\\	彼女	
\\	かのじょ			
\\	彼女は
\\	です。	
\\	彼女 
\\	かのじょ は おーえる です			
\\	聞く	
\\	きく			
\\	彼女はラジオを聞いています。	
\\	聞く 
\\	かのじょ は らじお を きいて います			
\\	どう	
\\	あなたはどう思いますか。	
\\	どう 
\\	あなた は どう おもいます か			
\\	生きる	
\\	いきる			
\\	皆一生懸命生きている。	
\\	生きる 
\\	みんな いっしょうけんめい いきて いる			
\\	言葉	
\\	ことば			
\\	顔	
\\	かお			
\\	彼はタオルで顔を拭きました。	
\\	顔 
\\	かれ は たおる で かお を ふきました			
\\	どこ	
\\	あの本をどこに置きましたか。	
\\	どこ 
\\	あの ほん を どこ に おきました か			
\\	一つ	
\\	ひとつ			
\\	それを一つください。	
\\	それ 
\\	一つ 
\\	それ を ひとつ ください			
\\	あげる	
\\	この本、あなたにあげます。	
\\	あげる 
\\	この ほん あなた に あげます			
\\	こう	
\\	こう小さい字は読めない。	
\\	こう 
\\	こう ちいさい じ は よめない			
\\	学校	
\\	がっこう			
\\	学校は8時半に始まります。	
\\	学校 
\\	がっこう は はちじはん に はじまります			
\\	くれる	
\\	友達が誕生日プレゼントをくれた。	
\\	くれる 
\\	ともだち が たんじょうび ぷれぜんと を くれた			
\\	起きる	
\\	おきる			
\\	私は毎朝6時に起きます。	
\\	起きる 
\\	わたし は まいあさ ろくじ に おきます			
\\	春	
\\	はる			
\\	今年の春は暖かいね。	
\\	春 
\\	ことし の はる は あたたかい ね			
\\	午前	
\\	ごぜん			
\\	午前9時のニュースです。	
\\	午前 
\\	ごぜん くじ の にゅーす です			
\\	買う	
\\	かう			
\\	郵便局で切手を買いました。	
\\	買う 
\\	ゆうびんきょく で きって を かいました			
\\	別	
\\	べつ			
\\	終わる	
\\	おわる			
\\	宿題は未だ終わっていません。	
\\	未だ 
\\	終わる 
\\	しゅくだい は まだ おわって いません			
\\	会議は4時に終わります。	
\\	終わる 
\\	かいぎ は よじ に おわります			
\\	若い	
\\	わかい			
\\	彼はまだ若いです。	
\\	若い 
\\	かれ は まだ わかい です			
\\	かなり	
\\	彼はかなり英語が上手です。	
\\	かなり 
\\	かれ は かなり えいご が じょうず です			
\\	置く	
\\	おく			
\\	彼はかばんをいすの上に置きました。	
\\	置く 
\\	かれ は かばん を いす の うえ に おきました			
\\	住む	
\\	すむ			
\\	彼は会社の近くに住んでいる。	
\\	住む 
\\	かれ は かいしゃ の ちかく に すんで いる			
\\	戻る	
\\	もどる			
\\	今、会社に戻ります。	
\\	戻る 
\\	いま かいしゃ に もどります			
\\	働く	
\\	はたらく			
\\	後	
\\	あと			
\\	仕事の後、映画を見た。	
\\	見る 
\\	仕事 
\\	後 
\\	しごと の あと えいが を みた			
\\	先生	
\\	せんせい			
\\	私は日本語の先生になりたいです。	
\\	先生 
\\	わたし は にほんご の せんせい に なりたい です			
\\	立つ	
\\	たつ			
\\	彼はステージに立った。	
\\	立つ 
\\	かれ は すてーじ に たった			
\\	気	
\\	き			
\\	彼は意外に気が小さい。	
\\	気 
\\	かれ は いがい に き が ちいさい			
\\	呼ぶ	
\\	よぶ			
\\	ウェイターを呼びましょう。	
\\	呼ぶ 
\\	うぇいたー を よびましょう			
\\	もっと	
\\	もっと近くに来てください。	
\\	近く 
\\	もっと 
\\	もっと ちかく に きて ください			
\\	大学に行ってもっと勉強したいです。	
\\	大学 
\\	もっと 
\\	だいがく に いって もっと べんきょう したい です			
\\	帰る	
\\	かえる			
\\	家に帰ろう。	
\\	家 
\\	帰る 
\\	うち に かえろう			
\\	気持ち	
\\	きもち			
\\	彼の気持ちが分からない。	
\\	気持ち 
\\	かれ の きもち が わからない			
\\	飛ぶ	
\\	とぶ			
\\	鳥が飛んでいます。	
\\	飛ぶ 
\\	とり が とんで います			
\\	車	
\\	くるま			
\\	彼の車は新しい。	
\\	新しい 
\\	車 
\\	かれ の くるま は あたらしい			
\\	弟が車を買った。	
\\	車 
\\	おとうと が くるま を かった			
\\	変える	
\\	かえる			
\\	旅行の日程を変えました。	
\\	変える 
\\	りょこう の にってい を かえました			
\\	広い	
\\	ひろい			
\\	彼の家はとても広い。	
\\	広い 
\\	かれ の いえ は とても ひろい			
\\	分かる	
\\	わかる			
\\	この言葉の意味が分かりません。	
\\	言葉 
\\	分かる 
\\	この ことば の いみ が わかりません			
\\	数	
\\	かず			
\\	グラスの数が足りません。	
\\	数 
\\	ぐらす の かず が たりません			
\\	近い	
\\	ちかい			
\\	そこ	
\\	走る	
\\	はしる			
\\	彼は毎晩3キロ走っています。	
\\	走る 
\\	かれ は まいばん さんきろ はしって います			
\\	売る	
\\	うる			
\\	彼は家を売った。	
\\	売る 
\\	かれ は いえ を うった			
\\	朝	
\\	あさ			
\\	気持ちのいい朝です。	
\\	気持ち 
\\	朝 
\\	きもち の いい あさ です			
\\	教える	
\\	おしえる			
\\	彼は数学を教えています。	
\\	教える 
\\	かれ は すうがく を おしえて います			
\\	歩く	
\\	あるく			
\\	駅まで歩きましょう。	
\\	歩く 
\\	えき まで あるきましょう			
\\	直ぐ	
\\	すぐ			
\\	直ぐ行きます。	
\\	行く 
\\	直ぐ 
\\	すぐ いきます			
\\	会う	
\\	あう			
\\	また会いましょう。	
\\	会う 
\\	また あいましょう			
\\	部屋	
\\	へや			
\\	私の部屋は2階にあります。	
\\	部屋 
\\	わたし の へや は にかい に あります			
\\	起こる	
\\	おこる			
\\	町で大事件が起こりました。	
\\	起こる 
\\	まち で だいじけん が おこりました			
\\	秋	
\\	あき			
\\	難しい	
\\	むずかしい			
\\	この本は難しいですね。	
\\	本 
\\	難しい 
\\	この ほん は むずかしい です ね			
\\	送る	
\\	おくる			
\\	彼の家に荷物を送りました。	
\\	送る 
\\	かれ の いえ に にもつ を おくりました			
\\	死ぬ	
\\	しぬ			
\\	乗る	
\\	のる			
\\	駅からはタクシーに乗ってください。	
\\	乗る 
\\	えき からは たくしー に のって ください			
\\	安い	
\\	やすい			
\\	この服はとても安かった。	
\\	安い 
\\	この ふく は とても やすかった			
\\	書く	
\\	かく			
\\	彼に手紙を書きました。	
\\	書く 
\\	かれ に てがみ を かきました			
\\	話す	
\\	はなす			
\\	母と電話で話しました。	
\\	話す 
\\	はは と でんわ で はなしました			
\\	頭	
\\	あたま			
\\	今朝から頭が痛い。	
\\	頭 
\\	けさ から あたま が いたい			
\\	体	
\\	からだ			
\\	私は体が丈夫だ。	
\\	体 
\\	わたし は からだ が じょうぶ だ			
\\	ここ	
\\	駅はここから近いです。	
\\	近い 
\\	ここ 
\\	えき は ここ から ちかい です			
\\	ここに本があります。	
\\	ここ 
\\	ここ に ほん が あります			
\\	待つ	
\\	まつ			
\\	あなたが来るのを待っています。	
\\	待つ 
\\	あなた が くる の を まって います			
\\	低い	
\\	ひくい			
\\	彼は背が低い。	
\\	低い 
\\	かれ は せ が ひくい			
\\	もらう	
\\	彼女からプレゼントをもらいました。	
\\	もらう 
\\	かのじょ から ぷれぜんと を もらいました			
\\	食べる	
\\	たべる			
\\	昨日タイカレーを食べました。	
\\	食べる 
\\	きのう たいかれー を たべました			
\\	早い	
\\	はやい			
\\	まだ学校へ行くには早い時間です。	
\\	早い 
\\	まだ がっこう へ いく に は はやい じかん です			
\\	兄	
\\	あに			
\\	兄は水泳が得意です。	
\\	兄 
\\	あに は すいえい が とくい です			
\\	名前	
\\	なまえ			
\\	あなたの名前を教えてください。	
\\	教える 
\\	名前 
\\	あなた の なまえ を おしえて ください			
\\	夫	
\\	おっと			
\\	私の夫はサラリーマンです。	
\\	夫 
\\	わたし の おっと は さらりーまん です			
\\	一	
\\	いち			
\\	一から始めましょう。	
\\	始める 
\\	一 
\\	いち から はじめましょう			
\\	結婚	
\\	けっこん			
\\	彼女は秋に結婚します。	
\\	する 
\\	秋 
\\	結婚 
\\	かのじょ は あき に けっこん します			
\\	彼女は来月結婚します。	
\\	結婚 
\\	かのじょ は らいげつ けっこん します			
\\	親	
\\	おや			
\\	親の愛は有り難い。	
\\	親 
\\	おや の あい は ありがたい			
\\	古い	
\\	ふるい			
\\	私は古い車が好きです。	
\\	古い 
\\	わたし は ふるい くるま が すき です			
\\	美しい	
\\	うつくしい			
\\	この絵は美しいです。	
\\	美しい 
\\	この え は うつくしい です			
\\	足	
\\	あし			
\\	彼は足が長い。	
\\	足 
\\	かれ は あし が ながい			
\\	次	
\\	つぎ			
\\	起こす	
\\	おこす			
\\	明日、6時に起こしてください。	
\\	起こす 
\\	あす ろくじ に おこして ください			
\\	見せる	
\\	みせる			
\\	別の本も見せてください。	
\\	別 
\\	見せる 
\\	べつ の ほん も みせて ください			
\\	その写真を見せてください。	
\\	見せる 
\\	その しゃしん を みせて ください			
\\	娘	
\\	むすめ			
\\	私の娘はアメリカにいます。	
\\	娘 
\\	わたし の むすめ は あめりか に います			
\\	とても	
\\	この本はとてもおもしろい。	
\\	とても 
\\	この ほん は とても おもしろい			
\\	少し	
\\	すこし			
\\	私の国について少しお話しましょう。	
\\	国 
\\	少し 
\\	わたし の くに に ついて すこし おはなし しましょう			
\\	少し疲れました。	
\\	少し 
\\	すこし つかれました			
\\	質問	
\\	しつもん			
\\	質問の意味は分かりましたか。	
\\	分かる 
\\	質問 
\\	しつもん の いみ は わかりました か			
\\	質問のある方はどうぞ。	
\\	質問 
\\	しつもん の ある かた は どうぞ			
\\	楽しむ	
\\	たのしむ			
\\	色	
\\	いろ			
\\	すてきな色のセーターですね。	
\\	色 
\\	すてき な いろ の せーたー です ね			
\\	皆	
\\	みんな			
\\	みんなにお菓子をあげましょう。	
\\	みんな 
\\	みんな に おかし を あげましょう			
\\	取る	
\\	とる			
\\	テストでいい点を取った。	
\\	取る 
\\	てすと で いい てん を とった			
\\	勉強	
\\	べんきょう			
\\	私は日本語を勉強しています。	
\\	勉強 
\\	わたし は にほんご を べんきょう して います			
\\	出来る	
\\	できる			
\\	彼女はイタリア語が出来ます。	
\\	出来る 
\\	かのじょ は いたりあご が できます			
\\	短い	
\\	みじかい			
\\	彼は足が短い。	
\\	短い 
\\	かれ は あし が みじかい			
\\	落ちる	
\\	おちる			
\\	猿も木から落ちる。	
\\	落ちる 
\\	さる も き から おちる			
\\	息子	
\\	むすこ			
\\	うちの息子は大学1年生です。	
\\	息子 
\\	うち の むすこ は だいがく いちねんせい です			
\\	白い	
\\	しろい			
\\	彼は歯が白い。	
\\	白い 
\\	かれ は は が しろい			
\\	誰	
\\	だれ			
\\	誰と会ってみたいですか。	
\\	誰 
\\	だれ と あって みたい です か			
\\	いつも	
\\	飛行機	
\\	ひこうき			
\\	息子は飛行機のおもちゃが好きです。	
\\	息子 
\\	飛行機 
\\	むすこ は ひこうき の おもちゃ が すき です			
\\	勝つ	
\\	かつ			
\\	今日はヤンキースが勝った。	
\\	勝つ 
\\	きょう は やんきーす が かった			
\\	病気	
\\	びょうき			
\\	祖父が病気になった。	
\\	なる 
\\	病気 
\\	そふ が びょうき に なった			
\\	犬が病気で死にました。	
\\	死ぬ 
\\	病気 
\\	いぬ が びょうき で しにました。			
\\	冬	
\\	ふゆ			
\\	カナダの冬はとても寒いです。	
\\	冬 
\\	かなだ の ふゆ は とても さむい です			
\\	年	
\\	とし			
\\	新しい年が始まりました。	
\\	新しい 
\\	年 
\\	あたらしい とし が はじまりました			
\\	重い	
\\	おもい			
\\	このかばんは重いです。	
\\	重い 
\\	この かばん は おもい です			
\\	胸	
\\	むね			
\\	胸に少し痛みがあります。	
\\	胸 
\\	むね に すこし いたみ が あります			
\\	読む	
\\	よむ			
\\	彼女は雑誌を読んでいます。	
\\	読む 
\\	かのじょ は ざっし を よんで います			
\\	払う	
\\	はらう			
\\	私が払いましょう。	
\\	払う 
\\	わたし が はらいましょう			
\\	軽い	
\\	かるい			
\\	この靴はとても軽い。	
\\	軽い 
\\	この くつ は とても かるい			
\\	見つける	
\\	みつける			
\\	新しい仕事を見つけました。	
\\	見つける 
\\	あたらしい しごと を みつけました。			
\\	いつ	
\\	次はいつ会いましょうか。	
\\	会う 
\\	次 
\\	いつ 
\\	つぎ は いつ あいましょう か			
\\	彼女はいつ来ますか。	
\\	いつ 
\\	かのじょ は いつ きます か			
\\	友達	
\\	ともだち			
\\	私には友達がたくさんいます。	
\\	友達 
\\	わたし に は ともだち が たくさん います			
\\	忘れる	
\\	わすれる			
\\	約束を忘れないでください。	
\\	忘れる 
\\	やくそく を わすれない で ください			
\\	酒	
\\	さけ			
\\	彼女は酒に強い。	
\\	酒 
\\	かのじょ は さけ に つよい			
\\	どちら	
\\	肉と魚とどちらが好きですか。	
\\	どちら 
\\	にく と さかな と どちら が すき です か			
\\	姉	
\\	あね			
\\	姉は銀行で働いています。	
\\	働く 
\\	姉 
\\	あね は ぎんこう で はたらいています 。			
\\	姉は大学生です。	
\\	姉 
\\	あね は だいがくせい です			
\\	覚える	
\\	おぼえる			
\\	妹は平仮名を全部覚えました。	
\\	覚える 
\\	いもうと は ひらがな を ぜんぶ おぼえました			
\\	狭い	
\\	せまい			
\\	私の部屋は狭いです。	
\\	部屋 
\\	狭い 
\\	わたし の へや は せまい です			
\\	好き	
\\	すき			
\\	私はワインが好きです。	
\\	好き 
\\	わたし は わいん が すき です			
\\	建てる	
\\	たてる			
\\	私たちは来年、家を建てます。	
\\	建てる 
\\	わたしたち は らいねん いえ を たてます			
\\	どれ	
\\	この中でどれが好きですか。	
\\	どれ 
\\	この なか で どれ が すき です か			
\\	赤い	
\\	あかい			
\\	赤いバラを買いました。	
\\	買う 
\\	赤い 
\\	あかい ばら を かいました			
\\	今日	
\\	きょう			
\\	今日は一人の時間を楽しみたい。	
\\	楽しむ 
\\	今日 
\\	きょう は ひとり の じかん を たのしみたい			
\\	今日は仕事がありません。	
\\	今日 
\\	きょう は しごと が ありません			
\\	着る	
\\	きる			
\\	今日はスーツを着ています。	
\\	着る 
\\	きょう は すーつ を きて います			
\\	笑う	
\\	わらう			
\\	赤ん坊が笑っています。	
\\	笑う 
\\	あかんぼう が わらって います			
\\	一番	
\\	いちばん			
\\	彼はクラスで一番背が高い。	
\\	高い 
\\	一番 
\\	かれ は くらす で いちばん せ が たかい			
\\	授業	
\\	じゅぎょう			
\\	今日は日本語の授業があります。	
\\	授業 
\\	きょう は にほんご の じゅぎょう が あります			
\\	週	
\\	しゅう			
\\	その次の週は空いてますか。	
\\	次 
\\	週 
\\	その つぎ の しゅう は あいてます か			
\\	漢字	
\\	かんじ			
\\	漢字は中国から来ました。	
\\	漢字 
\\	かんじ は ちゅうごく から きました			
\\	自転車	
\\	じてんしゃ			
\\	毎日、駅まで自転車で行きます。	
\\	自転車 
\\	まいにち えき まで じてんしゃ で いきます			
\\	電車	
\\	でんしゃ			
\\	私は電車で通学しています。	
\\	電車 
\\	わたし は でんしゃ で つうがく して います			
\\	探す	
\\	さがす			
\\	彼は郵便局を探していました。	
\\	探す 
\\	かれ は ゆうびんきょく を さがして いました			
\\	紙	
\\	かみ			
\\	紙と鉛筆はありますか。	
\\	紙 
\\	かみ と えんぴつ は あります か			
\\	歌う	
\\	うたう			
\\	私たちは大きな声で歌いました。	
\\	歌う 
\\	わたしたち は おおき な こえ で うたいました			
\\	遅い	
\\	おそい			
\\	前の車はとても遅い。	
\\	遅い 
\\	まえ の くるま は とても おそい			
\\	首	
\\	くび			
\\	きりんの首は長い。	
\\	首 
\\	きりん の くび は ながい			
\\	速い	
\\	はやい			
\\	彼は走るのが速い。	
\\	走る 
\\	速い 
\\	かれ は はしる の が はやい			
\\	一緒に	
\\	いっしょに			
\\	一緒に帰りましょう。	
\\	帰る 
\\	一緒に 
\\	いっしょに かえりましょう			
\\	友達と一緒に宿題をした。	
\\	する 
\\	一緒に 
\\	ともだち と いっしょ に しゅくだい を した			
\\	日記	
\\	にっき			
\\	私は毎日、日記を付けています。	
\\	日記 
\\	わたし は まいにち にっき を つけて います			
\\	今月	
\\	こんげつ			
\\	今月はとても忙しい。	
\\	今月 
\\	こんげつ は とても いそがしい			
\\	遊ぶ	
\\	あそぶ			
\\	子供たちが公園で遊んでいる。	
\\	遊ぶ 
\\	こどもたち が こうえん で あそんで いる			
\\	遠い	
\\	とおい			
\\	家から学校までは遠いです。	
\\	遠い 
\\	いえ から がっこう まで は とおい です			
\\	弱い	
\\	よわい			
\\	その子は体が少し弱い。	
\\	体 
\\	弱い 
\\	その こ は からだ が すこし よわい			
\\	耳	
\\	みみ			
\\	彼女は耳がよく聞こえません。	
\\	耳 
\\	かのじょ は みみ が よく きこえません			
\\	座る	
\\	すわる			
\\	そこに座ってください。	
\\	そこ 
\\	座る 
\\	そこ に すわって ください			
\\	私は窓側の席に座った。	
\\	座る 
\\	わたし は まどがわ の せき に すわった			
\\	右	
\\	みぎ			
\\	右のポケットにハンカチが入っています。	
\\	右 
\\	みぎ の ぽけっと に はんかち が はいって います			
\\	浴びる	
\\	あびる			
\\	私は朝、シャワーを浴びます。	
\\	私 
\\	朝 
\\	浴びる 
\\	わたし は あさ しゃわー を あびます			
\\	肩	
\\	かた			
\\	肩が凝りました。	
\\	肩 
\\	かた が こりました			
\\	飲む	
\\	のむ			
\\	友達とお酒を飲んでいます。	
\\	友達 
\\	酒 
\\	飲む 
\\	ともだち と おさけ を のんで います			
\\	寝る	
\\	ねる			
\\	もう寝よう。	
\\	寝る 
\\	もう ねよう			
\\	消す	
\\	けす			
\\	昼間は電気を消してください。	
\\	消す 
\\	ひるま は でんき を けして ください			
\\	元気	
\\	げんき			
\\	彼女はいつも元気だ。	
\\	いつも 
\\	元気 
\\	かのじょ は いつも げんき だ			
\\	おかげさまで元気です。	
\\	元気 
\\	おかげさま で げんき です			
\\	負ける	
\\	まける			
\\	私たちのチームはその試合で負けた。	
\\	負ける 
\\	わたしたち の ちーむ は その しあい で まけた			
\\	全部	
\\	ぜんぶ			
\\	それ、全部ください。	
\\	全部 
\\	それ ぜんぶ ください			
\\	去年	
\\	きょねん			
\\	私は去年フランスへ行った。	
\\	去年 
\\	わたし は きょねん ふらんす へ いった			
\\	引く	
\\	ひく			
\\	このドアは引いてください。	
\\	引く 
\\	この どあ は ひいて ください			
\\	図書館	
\\	としょかん			
\\	日曜日は図書館に行きます。	
\\	行く 
\\	図書館 
\\	にちようび は としょかん に いきます			
\\	図書館で料理の本を借りた。	
\\	図書館 
\\	としょかん で りょうり の ほん を かりた			
\\	上げる	
\\	あげる			
\\	彼は荷物をあみだなに上げた。	
\\	上げる 
\\	かれ は にもつ を あみだな に あげた			
\\	緑	
\\	みどり			
\\	この町には緑がたくさんあります。	
\\	緑 
\\	この まち に は みどり が たくさん あります			
\\	腕	
\\	うで			
\\	彼の腕は太い。	
\\	腕 
\\	かれ の うで は ふとい			
\\	ドア	
\\	ドアを開けてください。	
\\	ドア 
\\	どあ を あけて ください			
\\	閉じる	
\\	とじる			
\\	教科書を閉じてください。	
\\	閉じる 
\\	きょうかしょ を とじて ください			
\\	時	
\\	とき			
\\	時の経つのは早い。	
\\	時 
\\	とき の たつ の は はやい			
\\	米	
\\	こめ			
\\	日本人はお米が大好きです。	
\\	米 
\\	にほんじん は おこめ が だいすき です			
\\	切る	
\\	きる			
\\	この紙を半分に切ってください。	
\\	切る 
\\	この かみ を はんぶん に きって ください			
\\	線	
\\	せん			
\\	赤い線を2本引いて下さい。	
\\	線 
\\	あかい せん を に ほん ひいて ください			
\\	決まる	
\\	きまる			
\\	旅行の日程が決まりました。	
\\	決まる 
\\	りょこう の にってい が きまりました			
\\	楽しい	
\\	たのしい			
\\	彼はとても楽しい人です。	
\\	楽しい 
\\	かれ は とても たのしい ひと です			
\\	服	
\\	ふく			
\\	昨日、新しい服を買った。	
\\	服 
\\	きのう あたらしい ふく を かった			
\\	後ろ	
\\	うしろ			
\\	後ろを向いて。	
\\	後ろ 
\\	うしろ を むいて			
\\	嬉しい	
\\	うれしい			
\\	彼に会えて嬉しかった。	
\\	嬉しい 
\\	かれ に あえて うれしかった			
\\	腰	
\\	こし			
\\	昨日から腰が痛い。	
\\	腰 
\\	きのう から こし が いたい			
\\	日曜日	
\\	にちようび			
\\	日曜日は海に行きました。	
\\	日曜日 
\\	にちようび は うみ に いきました			
\\	開ける	
\\	あける			
\\	窓を開けてください。	
\\	開ける 
\\	まど を あけて ください			
\\	昼	
\\	ひる			
\\	私は昼のドラマを毎日見ます。	
\\	昼 
\\	わたし は ひる の どらま を まいにち みます			
\\	お母さん	
\\	おかあさん			
\\	お母さんによろしくお伝えください。	
\\	お母さん 
\\	おかあさん に よろしく おつたえ ください			
\\	留学生	
\\	りゅうがくせい			
\\	彼は留学生です。	
\\	留学生 
\\	かれ は りゅうがくせい です			
\\	大学生	
\\	だいがくせい			
\\	姉は大学生です。	
\\	大学生 
\\	あね は だいがくせい です			
\\	終わり	
\\	おわり			
\\	夏休みももう終わりだ。	
\\	終わり 
\\	なつやすみ も もう おわり だ			
\\	背	
\\	せ			
\\	彼女は背が高い。	
\\	背 
\\	かのじょ は せ が たかい			
\\	手伝う	
\\	てつだう			
\\	私が手伝いましょう。	
\\	手伝う 
\\	わたし が てつだいましょう			
\\	鼻	
\\	はな			
\\	鼻がかゆいです。	
\\	鼻 
\\	はな が かゆい です			
\\	起きる	
\\	おきる			
\\	近所で盗難事件が起きました。	
\\	起きる 
\\	きんじょ で とうなん じけん が おきました			
\\	熱い	
\\	あつい			
\\	このスープはとても熱い。	
\\	熱い 
\\	この すーぷ は とても あつい			
\\	載せる	
\\	のせる			
\\	皿にケーキを載せました。	
\\	載せる 
\\	さら に ケーキ を のせました 。			
\\	悲しい	
\\	かなしい			
\\	その映画はとても悲しかった。	
\\	悲しい 
\\	その えいが は とても かなしかった			
\\	愛する	
\\	あいする			
\\	私は家族を愛しています。	
\\	愛する 
\\	わたし は かぞく を あいして います			
\\	しゃべる	
\\	彼女はよくしゃべるね。	
\\	しゃべる 
\\	かのじょ は よく しゃべる ね			
\\	近く	
\\	ちかく			
\\	私の家は駅の近くです。	
\\	近く 
\\	わたし の いえ は えき の ちかく です			
\\	甘い	
\\	あまい			
\\	このイチゴは甘い。	
\\	甘い 
\\	この いちご は あまい			
\\	テーブル	
\\	新しいテーブルを買いました。	
\\	テーブル 
\\	あたらしい てーぶる を かいました			
\\	食べ物	
\\	たべもの			
\\	日本の食べ物はとても美味しいです。	
\\	食べ物 
\\	にっぽん の たべもの は とても おいしい です			
\\	始まる	
\\	はじまる			
\\	新しい仕事が始まりました。	
\\	始まる 
\\	あたらしい しごと が はじまりました			
\\	ゲーム	
\\	私たちはビデオゲームをした。	
\\	ゲーム 
\\	わたしたち は びでおげーむ を した			
\\	十	
\\	じゅう			
\\	その子は指で十数えました。	
\\	十 
\\	その こ は ゆび で じゅう かぞえました			
\\	天気	
\\	てんき			
\\	今日はいい天気ですね。	
\\	天気 
\\	きょう は いい てんき です ね			
\\	暑い	
\\	あつい			
\\	今日はとても暑い。	
\\	暑い 
\\	きょう は とても あつい			
\\	太い	
\\	ふとい			
\\	彼女は足が太い。	
\\	太い 
\\	かのじょ は あし が ふとい			
\\	晩	
\\	ばん			
\\	晩ご飯は食べましたか。	
\\	晩 
\\	ばんごはん は たべました か			
\\	土曜日	
\\	どようび			
\\	土曜日の夜はクラブに行きます。	
\\	土曜日 
\\	どようび の よる は くらぶ に いきます			
\\	神	
\\	かみ			
\\	彼は神を信じている。	
\\	神 
\\	かれ は かみ を しんじて いる			
\\	痛い	
\\	いたい			
\\	今日は頭が痛いです。	
\\	痛い 
\\	きょう は あたま が いたい です			
\\	お父さん	
\\	おとうさん			
\\	お父さんは会社員です。	
\\	お父さん 
\\	おとうさん は かいしゃいん です			
\\	多分	
\\	たぶん			
\\	彼女は多分家で寝ています。	
\\	多分 
\\	かのじょ は たぶん いえ で ねて います			
\\	時計	
\\	とけい			
\\	時計を見たらちょうど3時だった。	
\\	時計 
\\	とけい を みたら ちょうど さんじ だった			
\\	泊まる	
\\	とまる			
\\	今日はこのホテルに泊まります。	
\\	泊まる 
\\	きょう は この ほてる に とまります			
\\	どうして	
\\	どうして泣いているの。	
\\	どうして 
\\	どうして ないて いる の			
\\	掛ける	
\\	かける			
\\	夫の服をハンガーに掛けた。	
\\	掛ける 
\\	おっと の ふく を はんがー に かけた			
\\	曲がる	
\\	まがる			
\\	そこを左に曲がってください。	
\\	曲がる 
\\	そこ を ひだり に まがって ください			
\\	おなか	
\\	ミーティング	
\\	朝9時からミーティングが始まった。	
\\	ミーティング 
\\	あさ 
\\	じ から ミーティング が はじまった			
\\	閉める	
\\	しめる			
\\	ちゃんとドアを閉めてよ。	
\\	閉める 
\\	ちゃんと どあ を しめてよ			
\\	嫌い	
\\	きらい			
\\	私はタバコが嫌いです。	
\\	嫌い 
\\	わたし は たばこ が きらい です			
\\	金曜日	
\\	きんようび			
\\	要る	
\\	いる			
\\	予約は要りません。	
\\	要る 
\\	よやく は いりません			
\\	無い	
\\	ない			
\\	ここには何も無い。	
\\	無い 
\\	ここ に は なに も ない			
\\	黄色い	
\\	きいろい			
\\	彼女に黄色いバラを買いました。	
\\	黄色い 
\\	かのじょ に きいろい ばら を かいました			
\\	風邪	
\\	かぜ			
\\	私は風邪を引きました。	
\\	風邪 
\\	わたし は かぜ を ひきました 。			
\\	優しい	
\\	やさしい			
\\	彼はとても優しい人です。	
\\	優しい 
\\	かれ は とても やさしい ひと です			
\\	晴れる	
\\	はれる			
\\	明日は晴れるといいですね。	
\\	晴れる 
\\	あした は はれる と いい です ね			
\\	大好き	
\\	だいすき			
\\	私は犬が大好きだ。	
\\	大好き 
\\	わたし は いぬ が だいすき だ			
\\	茶色	
\\	ちゃいろ			
\\	彼女は茶色の靴を履いています。	
\\	茶色 
\\	かのじょ は ちゃいろ の くつ を はいて います			
\\	汚い	
\\	きたない			
\\	彼の部屋はとても汚い。	
\\	汚い 
\\	かれ の へや は とても きたない			
\\	空く	
\\	すく			
\\	おなかが空きました。	
\\	おなか 
\\	空く 
\\	おなか が すきました			
\\	レストランは空いていました。	
\\	空く 
\\	れすとらん は すいて いました			
\\	上る	
\\	のぼる			
\\	猫が屋根に上っている。	
\\	上る 
\\	ねこ が やね に のぼって いる			
\\	ご飯	
\\	ごはん			
\\	私はパンよりご飯が好きだ。	
\\	ご飯 
\\	わたし は ぱん より ごはん が すき だ			
\\	日	
\\	にち			
\\	私たちは先月11日に結婚しました。	
\\	日 
\\	わたしたち は せんげつ じゅういち にち に けっこん しました			
\\	髪の毛	
\\	かみのけ			
\\	髪の毛が伸びたね。	
\\	髪の毛 
\\	かみのけ が のびた ね			
\\	つける	
\\	電気をつけてください。	
\\	つける 
\\	でんき を つけて ください			
\\	月曜日	
\\	げつようび			
\\	月曜日に会いましょう。	
\\	月曜日 
\\	げつようび に あいましょう			
\\	入る	
\\	はいる			
\\	寒いので中に入ってください。	
\\	入る 
\\	さむい の で なか に はいって ください			
\\	片仮名	
\\	かたかな			
\\	お名前を片仮名で書いてください。	
\\	片仮名 
\\	おなまえ を かたかな で かいて ください			
\\	今週	
\\	こんしゅう			
\\	今週は日本語のテストがあります。	
\\	今週 
\\	こんしゅう は にほんご の てすと が あります			
\\	開く	
\\	ひらく			
\\	32ページを開いてください。	
\\	開く 
\\	さんじゅうにぺーじ を ひらいて ください			
\\	水	
\\	みず			
\\	水を一杯ください。	
\\	水 
\\	みず を いっぱい ください			
\\	あれ	
\\	二	
\\	に			
\\	その人には二回会った。	
\\	二 
\\	その ひと に は に かい あった			
\\	閉まる	
\\	しまる			
\\	お店はもう閉まっていました。	
\\	閉まる 
\\	おみせ は もう しまって いました			
\\	締める	
\\	しめる			
\\	彼はシートベルトを締めた。	
\\	締める 
\\	かれ は しーとべると を しめた			
\\	まずい	
\\	ここの料理はまずい。	
\\	まずい 
\\	ここ の りょうり は まずい			
\\	平仮名	
\\	ひらがな			
\\	私は平仮名を全部読めます。	
\\	平仮名 
\\	わたし は ひらがな を ぜんぶ よめます			
\\	曇る	
\\	くもる			
\\	明日は昼頃から曇るでしょう。	
\\	昼 
\\	曇る 
\\	あす は ひるごろ から くもる でしょう			
\\	触る	
\\	さわる			
\\	絵に触らないでください。	
\\	触る 
\\	え に さわらない で ください			
\\	駄目	
\\	だめ			
\\	書き直す	
\\	かきなおす			
\\	この書類を書き直してください。	
\\	書き直す 
\\	この しょるい を かきなおして ください			
\\	飲み物	
\\	のみもの			
\\	木曜日	
\\	もくようび			
\\	木曜日は仕事が休みです。	
\\	木曜日 
\\	もくようび は しごと が やすみ です			
\\	曜日	
\\	ようび			
\\	曜日を間違えました。	
\\	曜日 
\\	ようび を まちがえました			
\\	そば	
\\	そばにいて下さい。	
\\	そば 
\\	そば に いて ください			
\\	こっち	
\\	こっちに来て下さい。	
\\	こっち 
\\	こっち に きて ください			
\\	火曜日	
\\	かようび			
\\	火曜日に会議があります。	
\\	火曜日 
\\	かようび に かいぎ が あります			
\\	渇く	
\\	かわく			
\\	喉が渇きました。	
\\	渇く 
\\	のど が かわきました			
\\	三	
\\	さん			
\\	彼女は三人の子供の母親だ。	
\\	三 
\\	かのじょ は さんにん の こども の ははおや だ。			
\\	水曜日	
\\	すいようび			
\\	水曜日はバイトがあります。	
\\	水曜日 
\\	すいようび は ばいと が あります			
\\	二つ	
\\	ふたつ			
\\	ソフトクリームを二つください。	
\\	二つ 
\\	そふとくりーむ を ふたつ ください			
\\	今晩	
\\	こんばん			
\\	今晩のパーティーは何時からですか。	
\\	今晩 
\\	こんばん の ぱーてぃー は なんじ から です か			
\\	千	
\\	せん			
\\	六つ	
\\	むっつ			
\\	息子は六つになりました。	
\\	六つ 
\\	むすこ は むっつ に なりました			
\\	六日	
\\	むいか			
\\	六日前に日本に帰ってきました。	
\\	六日 
\\	むいかまえ に にほん に かえって きました			
\\	お姉さん	
\\	おねえさん			
\\	昨日、あなたのお姉さんに会ったよ。	
\\	お姉さん 
\\	きのう あなた の おねえさん に あった よ			
\\	直る	
\\	なおる			
\\	クーラーはまだ直りません。	
\\	直る 
\\	くーらー は まだ なおりません			
\\	易しい	
\\	やさしい			
\\	この問題はかなり易しいです。	
\\	易しい 
\\	この もんだい は かなり やさしい です			
\\	辺	
\\	へん			
\\	彼はこの辺に住んでいます。	
\\	辺 
\\	かれ は このへん に すんで います			
\\	よう	
\\	彼女は眠いようです。	
\\	よう 
\\	かのじょ は ねむい よう です			
\\	お兄さん	
\\	おにいさん			
\\	あなたのお兄さんは何歳?	
\\	お兄さん 
\\	あなた の おにいさん は なんさい			
\\	辛い	
\\	からい			
\\	彼は辛いものが好きです。	
\\	辛い 
\\	かれ は からい もの が すき です			
\\	八	
\\	はち			
\\	りんごを八個ください。	
\\	八 
\\	りんご を はちこ ください			
\\	三日	
\\	みっか			
\\	手紙が届くのに三日かかりました。	
\\	三日 
\\	てがみ が とどく の に みっか かかりました			
\\	あそこ	
\\	あそこにバス停があります。	
\\	あそこ 
\\	あそこ に ばすてい が あります			
\\	来る	
\\	くる			
\\	彼は昼過ぎに来ます。	
\\	来る 
\\	かれ は ひるすぎ に きます			
\\	前	
\\	まえ			
\\	その店の前で会いましょう。	
\\	前 
\\	その みせ の まえ で あいましょう			
\\	五日	
\\	いつか			
\\	五月五日は祝日です	
\\	五日 
\\	ごがつ いつか は しゅくじつ です			
\\	一杯	
\\	いっぱい			
\\	プールは人で一杯です。	
\\	一杯 
\\	ぷーる は ひと で いっぱい です			
\\	九	
\\	きゅう			
\\	野球は九人で1チームです。	
\\	九 
\\	やきゅう は きゅうにん で ひとちーむ です			
\\	酸っぱい	
\\	すっぱい			
\\	このぶどうは酸っぱいです。	
\\	酸っぱい 
\\	この ぶどう は すっぱい です			
\\	違う	
\\	ちがう			
\\	答えが違います。	
\\	違う 
\\	こたえ が ちがいます			
\\	細い	
\\	ほそい			
\\	彼女は指が細いですね。	
\\	細い 
\\	かのじょ は ゆび が ほそい です ね			
\\	三つ	
\\	みっつ			
\\	コップを三つ買いました。	
\\	三つ 
\\	こっぷ を みっつ かいました			
\\	八日	
\\	ようか			
\\	八日からイギリスに行きます。	
\\	八日 
\\	ようか から いぎりす に いきます			
\\	高校生	
\\	こうこうせい			
\\	私の弟は高校生です。	
\\	高校生 
\\	わたし の おとうと は こうこうせい です			
\\	上手	
\\	じょうず			
\\	妹は歌が上手です。	
\\	上手 
\\	いもうと は うた が じょうず です			
\\	強い	
\\	つよい			
\\	今日は風が強い。	
\\	強い 
\\	きょう は かぜ が つよい			
\\	七	
\\	なな			
\\	バナナが七本あります。	
\\	七 
\\	ばなな が ななほん あります			
\\	二十日	
\\	はつか			
\\	来月の二十日は弟の誕生日です。	
\\	二十日 
\\	らいげつ の はつか は おとうと の たんじょうび です			
\\	左	
\\	ひだり			
\\	そこを左に曲がってください。	
\\	曲がる 
\\	左 
\\	そこ を ひだり に まがって ください			
\\	二日	
\\	ふつか			
\\	私は二日待った。	
\\	二日 
\\	わたし は ふつか まった			
\\	四つ	
\\	よっつ			
\\	私は腕時計を四つ持っています。	
\\	四つ 
\\	わたし は うでどけい を よっつ もって います			
\\	暖かい	
\\	あたたかい			
\\	このコートはとても暖かい。	
\\	暖かい 
\\	この コート は とても あたたかい。			
\\	有る	
\\	ある			
\\	私の机の上に書類がたくさん有ります。	
\\	有る 
\\	わたし の つくえ の うえ に しょるい が たくさん あります			
\\	良い	
\\	いい			
\\	彼女は良い友達です。	
\\	良い 
\\	かのじょ は いい ともだち です			
\\	五つ	
\\	いつつ			
\\	桃を五つください。	
\\	五つ 
\\	もも を いつつ ください			
\\	妹	
\\	いもうと			
\\	私の妹は小学生です。	
\\	妹 
\\	わたし の いもうと は しょうがくせい です			
\\	上	
\\	うえ			
\\	上を向いて。	
\\	上 
\\	うえ を むいて			
\\	動く	
\\	うごく			
\\	動かないで。	
\\	動く 
\\	うごかない で			
\\	駅	
\\	えき			
\\	駅はどこですか。	
\\	駅 
\\	えき は どこ です か			
\\	おいしい	
\\	このケーキはおいしいね。	
\\	おいしい 
\\	この けーき は おいしい ね			
\\	大きい	
\\	おおきい			
\\	あの大きい建物は何ですか。	
\\	大きい 
\\	あの おおきい たてもの は なん です か			
\\	弟	
\\	おとうと			
\\	弟は野球が好きです。	
\\	弟 
\\	おとうと は やきゅう が すき です			
\\	男	
\\	おとこ			
\\	彼は駄目な男だ。	
\\	駄目 
\\	男 
\\	かれ は だめ な おとこ だ			
\\	男の人が私たちに話しかけた。	
\\	男 
\\	おとこ の ひと が わたしたち に はなしかけた			
\\	大人	
\\	おとな			
\\	お酒は大人になってから。	
\\	大人 
\\	おさけ は おとな に なって から			
\\	お願い	
\\	おねがい			
\\	お願いがあります。	
\\	お願い 
\\	おねがい が あります			
\\	女	
\\	おんな			
\\	店員は若い女の人でした。	
\\	女 
\\	てんいん は わかい おんな の ひと でした			
\\	金	
\\	かね			
\\	これはかなり金がかかった。	
\\	金 
\\	これ は かなり かね が かかった			
\\	彼	
\\	かれ			
\\	彼は私の上司です。	
\\	彼 
\\	かれ は わたし の じょうし です			
\\	昨日	
\\	きのう			
\\	昨日、友達に会った。	
\\	昨日 
\\	きのう ともだち に あった			
\\	奇麗	
\\	きれい			
\\	彼女はとても奇麗だ。	
\\	奇麗 
\\	かのじょ は とても きれい だ			
\\	口	
\\	くち			
\\	口を大きく開けてください。	
\\	口 
\\	くち を おおきく あけて ください			
\\	五	
\\	ご			
\\	五人で旅行に行きました。	
\\	五 
\\	ごにん で りょこう に いきました			
\\	九日	
\\	ここのか			
\\	九日に荷物が届きます。	
\\	九 
\\	九日 
\\	ここのか に にもつ が とどきます			
\\	九つ	
\\	ここのつ			
\\	娘は明日九つになります。	
\\	九つ 
\\	むすめ は あす ここのつ に なります			
\\	答える	
\\	こたえる			
\\	私の質問に答えてください。	
\\	答える 
\\	わたし の しつもん に こたえて ください			
\\	先	
\\	さき			
\\	お先にどうぞ。	
\\	先 
\\	おさきに どうぞ			
\\	寒い	
\\	さむい			
\\	この部屋は寒いです。	
\\	寒い 
\\	この へや は さむい です			
\\	四	
\\	し			
\\	四月に大学に入学しました。	
\\	四 
\\	しがつ に だいがく に にゅうがく しました			
\\	下	
\\	した			
\\	財布は机の下にあった。	
\\	下 
\\	さいふ は つくえ の した に あった			
\\	大丈夫	
\\	だいじょうぶ			
\\	大丈夫ですか。	
\\	大丈夫 
\\	だいじょうぶ です か			
\\	出す	
\\	だす			
\\	彼はかばんから教科書を出した。	
\\	出す 
\\	かれ は かばん から きょうかしょ を だした			
\\	小さい	
\\	ちいさい			
\\	小さい花が咲いています。	
\\	小さい 
\\	ちいさい はな が さいて います			
\\	父	
\\	ちち			
\\	私は父が大好きです。	
\\	父 
\\	わたし は ちち が だいすき です			
\\	一日	
\\	ついたち			
\\	来月の一日は空いていますか。	
\\	一日 
\\	らいげつ の ついたち は あいています か			
\\	月	
\\	つき			
\\	今夜は月がとてもきれいです。	
\\	月 
\\	こんや は つき が とても きれい です			
\\	妻	
\\	つま			
\\	今日は妻の誕生日だ。	
\\	妻 
\\	きょう は つま の たんじょうび だ			
\\	冷たい	
\\	つめたい			
\\	冷たい飲み物をください。	
\\	飲み物 
\\	冷たい 
\\	つめたい のみもの を ください			
\\	手	
\\	て			
\\	分かった人は手を上げてください。	
\\	手 
\\	わかった ひと は て を あげて ください			
\\	十日	
\\	とおか			
\\	十日後に帰ります。	
\\	十日 
\\	とおかご に かえります			
\\	時々	
\\	ときどき			
\\	彼は時々遅刻します。	
\\	時々 
\\	かれ は ときどき ちこく します			
\\	夏	
\\	なつ			
\\	私は夏が大好き。	
\\	大好き 
\\	夏 
\\	わたし は なつ が だいすき			
\\	七つ	
\\	ななつ			
\\	この子は今年七つになります。	
\\	七つ 
\\	この こ は ことし ななつ に なります			
\\	何	
\\	なに			
\\	あれは何ですか。	
\\	あれ 
\\	何 
\\	あれ は なん です か			
\\	夕食には何を食べたいですか。	
\\	何 
\\	ゆうしょく に は なに を たべたい です か			
\\	七日	
\\	なのか			
\\	先月の七日に孫が生まれました。	
\\	七日 
\\	せんげつ の なのか に まご が うまれました			
\\	母	
\\	はは			
\\	昨日、母と話をしました。	
\\	母 
\\	きのう はは と はなし を しました			
\\	人	
\\	ひと			
\\	彼は優しい人です。	
\\	優しい 
\\	人 
\\	かれ は やさしい ひと です			
\\	一人	
\\	ひとり			
\\	そこには私一人しかいなかった。	
\\	一人 
\\	そこ に は わたし ひとり しか いなかった			
\\	百	
\\	ひゃく			
\\	私の祖母は百才です。	
\\	百 
\\	わたし の そぼ は ひゃくさい です			
\\	方	
\\	ほう			
\\	彼は私の方を見ました。	
\\	方 
\\	かれ は わたし の ほう を みました			
\\	他	
\\	ほか			
\\	他に方法がありません。	
\\	他 
\\	ほか に ほうほう が ありません			
\\	僕	
\\	ぼく			
\\	僕は学生です。	
\\	僕 
\\	ぼく は がくせい です			
\\	欲しい	
\\	ほしい			
\\	僕は新しい靴が欲しいです。	
\\	僕 
\\	欲しい 
\\	ぼく は あたらしい くつ が ほしい です			
\\	何か飲み物が欲しいな。	
\\	飲み物 
\\	欲しい 
\\	なに か のみもの が ほしい な 。			
\\	万	
\\	まん			
\\	この靴は1万円です。	
\\	万 
\\	この くつ は いちまんえん です			
\\	見える	
\\	みえる			
\\	ここから富士山がよく見えます。	
\\	見える 
\\	ここ から ふじさん が よく みえます			
\\	店	
\\	みせ			
\\	私はこの店によく来ます	
\\	店 
\\	わたし は この みせ に よく きます			
\\	道	
\\	みち			
\\	この道を真っ直ぐ行くと駅です。	
\\	道 
\\	この みち を まっすぐ いく と えき です			
\\	目	
\\	め			
\\	彼女は青い目をしています。	
\\	目 
\\	かのじょ は あおい め を して います			
\\	八つ	
\\	やっつ			
\\	あの家には時計が八つあります。	
\\	時計 
\\	八つ 
\\	あの いえ に は とけい が やっつ あります			
\\	止める	
\\	やめる			
\\	話すのを止めてください。	
\\	止める 
\\	はなす の を やめて ください			
\\	四日	
\\	よっか			
\\	新学期は来月の四日からです。	
\\	四日 
\\	しんがっき は らいげつ の よっか から です			
\\	夜	
\\	よる			
\\	金曜日の夜は友達と出かけます。	
\\	金曜日 
\\	夜 
\\	きんようび の よる は ともだち と でかけます			
\\	きのうの夜は家にいました。	
\\	夜 
\\	きのう の よる は いえ に いました			
\\	ハワイは四回目です。	
\\	四 
\\	四 
\\	はわい は よんかいめ です			
\\	来年	
\\	らいねん			
\\	来年一緒に旅行しましょう。	
\\	来年 
\\	らいねん いっしょ に りょこう しましょう			
\\	六	
\\	ろく			
\\	彼には子供が六人います。	
\\	六 
\\	かれ に は こども が ろくにん います			
\\	悪い	
\\	わるい			
\\	たばこは体に悪い。	
\\	悪い 
\\	たばこ は からだ に わるい			
\\	お手洗い	
\\	おてあらい			
\\	お手洗いはどこですか。	
\\	お手洗い 
\\	おてあらい は どこ です か			
\\	ご主人	
\\	ごしゅじん			
\\	ご主人はお元気ですか。	
\\	ご主人 
\\	ごしゅじん は おげんき です か			
\\	本当に	
\\	ほんとうに			
\\	あなたが本当に好きです。	
\\	本当に 
\\	あなた が ほんとうに すき です			
\\	自分	
\\	じぶん			
\\	宿題は自分でやりなさい。	
\\	自分 
\\	しゅくだい は じぶん で やりなさい			
\\	なる	
\\	柿の木に実がたくさんなっています。	
\\	なる 
\\	かきのき に み が たくさん なって います			
\\	ため	
\\	これは医者のためのサイトです。	
\\	ため 
\\	これ は いしゃ の ため の さいと です			
\\	円	
\\	えん			
\\	千円貸してください。	
\\	千 
\\	円 
\\	せんえん かして ください			
\\	カレーライスは700円です。	
\\	円 
\\	かれーらいす は ななひゃくえん です			
\\	場合	
\\	ばあい			
\\	分からない場合は私に聞いてください。	
\\	場合 
\\	わからない ばあい は わたし に きいて ください			
\\	休む	
\\	やすむ			
\\	明日、会社を休みます。	
\\	休む 
\\	あした かいしゃ を やすみます			
\\	見つかる	
\\	みつかる			
\\	メガネが見つかりません。	
\\	見つかる 
\\	めがね が みつかりません			
\\	問題	
\\	もんだい			
\\	問題が一つあります。	
\\	問題 
\\	もんだい が ひとつ あります			
\\	開発	
\\	かいはつ			
\\	ダムの開発に住民は反対しています。	
\\	開発 
\\	だむ の かいはつ に じゅうみん は はんたい して います			
\\	事件	
\\	じけん			
\\	その事件の犯人はまだ捕まっていない。	
\\	事件 
\\	その じけん の はんにん は まだ つかまって いない			
\\	必要	
\\	ひつよう			
\\	私にはたくさんのお金が必要だ。	
\\	必要 
\\	わたし に は たくさん の おかね が ひつよう だ			
\\	企業	
\\	きぎょう			
\\	彼女はアメリカの企業で働いています。	
\\	企業 
\\	かのじょ は あめりか の きぎょう で はたらいて います			
\\	行う	
\\	おこなう			
\\	その会社は来月、キャンペーンを行う。	
\\	行う 
\\	その かいしゃ は らいげつ きゃんぺーん を おこなう			
\\	現在	
\\	げんざい			
\\	現在の気温は30度です。	
\\	現在 
\\	げんざい の きおん は さんじゅうど です			
\\	利用	
\\	りよう			
\\	私はよく図書館を利用します。	
\\	利用 
\\	わたし は よく としょかん を りよう します			
\\	首相	
\\	しゅしょう			
\\	持つ	
\\	もつ			
\\	この車はよく持っているね。	
\\	持つ 
\\	この くるま は よく もって いる ね			
\\	政府	
\\	せいふ			
\\	そのデモについて政府は何もしなかった。	
\\	政府 
\\	その でも に ついて せいふ は なにも しなかった			
\\	発表	
\\	はっぴょう			
\\	合格者が発表された。	
\\	発表 
\\	ごうかくしゃ が はっぴょう された			
\\	開く	
\\	あく			
\\	電車のドアが開きました。	
\\	開く 
\\	でんしゃ の どあ が あきました			
\\	受ける	
\\	うける			
\\	彼は就職試験を受けた。	
\\	受ける 
\\	かれ は しゅうしょく しけん を うけた			
\\	より	
\\	これから、より一層努力します。	
\\	より 
\\	これから より いっそう どりょく します			
\\	昨年	
\\	さくねん			
\\	昨年は地震が多い年でした。	
\\	昨年 
\\	さくねん は じしん が おおい とし でした			
\\	情報	
\\	じょうほう			
\\	学生たちはインターネットでいろいろな情報を集めた。	
\\	情報 
\\	がくせいたち は いんたーねっと で いろいろな じょうほう を あつめた			
\\	示す	
\\	しめす			
\\	彼は新製品に興味を示している。	
\\	示す 
\\	かれ は しんせいひん に きょうみ を しめして いる			
\\	建設	
\\	けんせつ			
\\	新しいビルの建設が始まった。	
\\	建設 
\\	あたらしい びる の けんせつ が はじまった			
\\	販売	
\\	はんばい			
\\	前売券は窓口で販売しています。	
\\	販売 
\\	まえうりけん は まどぐち で はんばい して います			
\\	治る	
\\	なおる			
\\	けがはもう治りましたか。	
\\	治る 
\\	けが は もう なおりました か			
\\	機能	
\\	きのう			
\\	このソフトにはいろいろな機能があります。	
\\	機能 
\\	この そふと に は いろいろな きのう が あります			
\\	研究	
\\	けんきゅう			
\\	彼は何年も地震の研究をしている。	
\\	研究 
\\	かれ は なんねん も じしん の けんきゅう を して いる			
\\	一方	
\\	いっぽう			
\\	ここは一方通行です。	
\\	一方 
\\	ここ は いっぽう つうこう です			
\\	ドル	
\\	この服は300ドルしました。	
\\	ドル 
\\	この ふく は さんびゃくどる しました			
\\	システム	
\\	全てのシステムが停止した。	
\\	システム 
\\	すべて の しすてむ が ていし した			
\\	頃	
\\	ころ			
\\	子供の頃、よくその公園で遊んだ。	
\\	頃 
\\	こども の ころ よく その こうえん で あそんだ			
\\	対する	
\\	たいする			
\\	その質問に対する答えが見つからなかった。	
\\	対する 
\\	その しつもん に たいする こたえ が みつからなかった			
\\	方法	
\\	ほうほう			
\\	いい方法を思いつきました。	
\\	方法 
\\	いい ほうほう を おもいつきました			
\\	以上	
\\	いじょう			
\\	飛行機が1時間以上遅れた。	
\\	以上 
\\	ひこうき が いちじかん いじょう おくれた			
\\	最近	
\\	さいきん			
\\	それは最近話題の本ですね。	
\\	最近 
\\	それ は さいきん わだい の ほん です ね			
\\	今後	
\\	こんご			
\\	今後ともよろしくお願いします。	
\\	今後 
\\	こんご とも よろしく おねがい します			
\\	世界	
\\	せかい			
\\	私は世界旅行をしたい。	
\\	世界 
\\	わたし は せかい りょこう を したい			
\\	用いる	
\\	もちいる			
\\	彼はその詩を用いて自分の気持ちを伝えた。	
\\	用いる 
\\	かれ は その し を もちいて じぶん の きもち を つたえた			
\\	語る	
\\	かたる			
\\	今夜は大いに語りましょう。	
\\	語る 
\\	こんや は おおいに かたりましょう			
\\	コンピューター	
\\	新しいコンピューターを買った。	
\\	コンピューター 
\\	あたらしい こんぴゅーたー を かった			
\\	中心	
\\	ちゅうしん			
\\	ここが建物の中心です。	
\\	中心 
\\	ここ が たてもの の ちゅうしん です			
\\	やる	
\\	やる			
\\	犬にえさをやった。	
\\	やる 
\\	いぬ に えさ を やった			
\\	意味	
\\	いみ			
\\	それはどういう意味ですか。	
\\	意味 
\\	それ は どういう いみ です か			
\\	計画	
\\	けいかく			
\\	彼は一人旅の計画を立てた。	
\\	計画 
\\	かれ は ひとりたび の けいかく を たてた			
\\	影響	
\\	えいきょう			
\\	私は彼から大きな影響を受けました。	
\\	受ける 
\\	影響 
\\	わたし は かれ から おおき な えいきょう を うけました			
\\	増える	
\\	ふえる			
\\	この町は人口が増えた。	
\\	増える 
\\	この まち は じんこう が ふえた			
\\	選ぶ	
\\	えらぶ			
\\	良い家を選ぶのは難しい。	
\\	選ぶ 
\\	よい いえ を えらぶ の は むずかしい			
\\	経済	
\\	けいざい			
\\	大学で経済を勉強しました。	
\\	経済 
\\	だいがく で けいざい を べんきょう しました			
\\	導入	
\\	どうにゅう			
\\	会社で新しいシステムを導入した。	
\\	導入 
\\	かいしゃ で あたらしい しすてむ を どうにゅう した			
\\	生活	
\\	せいかつ			
\\	日本での生活は楽しいです。	
\\	生活 
\\	にほん で の せいかつ は たのしい です			
\\	進める	
\\	すすめる			
\\	早く授業を進めましょう。	
\\	進める 
\\	はやく じゅぎょう を すすめましょう			
\\	関係	
\\	かんけい			
\\	彼はその事件に関係がない。	
\\	関係 
\\	かれ は その じけん に かんけい が ない			
\\	続ける	
\\	つづける			
\\	仕事を続けてください。	
\\	続ける 
\\	しごと を つづけて ください			
\\	ロボット	
\\	彼はロボットの研究をしています。	
\\	ロボット 
\\	かれ は ロボット の けんきゅう を しています 。			
\\	ほとんど	
\\	お金がほとんどありません。	
\\	ほとんど 
\\	おかね が ほとんど ありません			
\\	国内	
\\	こくない			
\\	この携帯電話が使えるのは国内だけです。	
\\	国内 
\\	この けいたい でんわ が つかえる の は こくない だけ です			
\\	分野	
\\	ぶんや			
\\	音楽は彼の得意な分野です。	
\\	分野 
\\	おんがく は かれ の とくい な ぶんや です			
\\	会社	
\\	かいしゃ			
\\	彼は小さな会社に勤めています。	
\\	会社 
\\	かれ は ちいさ な かいしゃ に つとめて います			
\\	関する	
\\	かんする			
\\	その問題に関する記事を読みました。	
\\	関する 
\\	その もんだい に かんする きじ を よみました			
\\	状態	
\\	じょうたい			
\\	ここは道の状態がとても悪いです。	
\\	状態 
\\	ここ は みち の じょうたい が とても わるい です			
\\	大統領	
\\	だいとうりょう			
\\	フランスの大統領は誰ですか。	
\\	大統領 
\\	ふらんす の だいとうりょう は だれ です か			
\\	対象	
\\	たいしょう			
\\	このアンケートは大学生が対象です。	
\\	対象 
\\	この あんけーと は だいがくせい が たいしょう です			
\\	処理	
\\	しょり			
\\	事務的な処理に1週間かかります。	
\\	処理 
\\	じむてき な しょり に いっしゅうかん かかります			
\\	女性	
\\	じょせい			
\\	そのパーティーに女性は何人来ますか。	
\\	女性 
\\	その ぱーてぃー に じょせい は なんにん きます か			
\\	家	
\\	いえ			
\\	ここが私の家です。	
\\	家 
\\	ここ が わたし の いえ です			
\\	パソコン	
\\	彼はパソコンを2台持っています。	
\\	パソコン 
\\	かれ は ぱそこん を にだい もって います			
\\	解決	
\\	かいけつ			
\\	トラブルがやっと解決した。	
\\	解決 
\\	とらぶる が やっと かいけつ した			
\\	実現	
\\	じつげん			
\\	夢を実現するには努力が必要です。	
\\	必要 
\\	実現 
\\	ゆめ を じつげん する に は どりょく が ひつよう です			
\\	多く	
\\	おおく			
\\	毎年多くの人が海外へ旅行する。	
\\	多く 
\\	まいとし おおく の ひと が かいがい へ りょこう する			
\\	動き	
\\	うごき			
\\	話	
\\	はなし			
\\	あなたの話は面白いね。	
\\	話 
\\	あなた の はなし は おもしろい ね			
\\	価格	
\\	かかく			
\\	ガソリンの価格がどんどん上がっている。	
\\	価格 
\\	がそりん の かかく が どんどん あがって いる			
\\	上がる	
\\	あがる			
\\	私たちは2階に上がった。	
\\	上がる 
\\	わたしたち は にかい に あがった			
\\	もう	
\\	コーヒーをもう一杯ください。	
\\	もう 
\\	こーひー を もう いっぱい ください			
\\	集める	
\\	あつめる			
\\	弟は切手を集めています。	
\\	集める 
\\	おとうと は きって を あつめて います			
\\	声	
\\	こえ			
\\	彼は大きな声で話した。	
\\	声 
\\	かれ は おおき な こえ で はなした			
\\	初めて	
\\	はじめて			
\\	東京に来るのは初めてです。	
\\	初めて 
\\	とうきょう に くる の は はじめて です 。			
\\	明らか	
\\	あきらか			
\\	明らかに彼が悪い。	
\\	明らか 
\\	あきらか に かれ が わるい			
\\	一部	
\\	いちぶ			
\\	計画を一部変更しましょう。	
\\	計画 
\\	一部 
\\	けいかく を いちぶ へんこう しましょう			
\\	調査	
\\	ちょうさ			
\\	私たちがその問題を調査しています。	
\\	調査 
\\	わたしたち が その もんだい を ちょうさ して います			
\\	党	
\\	とう			
\\	変わる	
\\	かわる			
\\	信号が青に変わりました。	
\\	変わる 
\\	しんごう が あお に かわりました。			
\\	社長	
\\	しゃちょう			
\\	おばは小さな会社の社長です。	
\\	社長 
\\	おば は ちいさ な かいしゃ の しゃちょう です			
\\	全国	
\\	ぜんこく			
\\	次は全国のお天気です。	
\\	全国 
\\	つぎ は ぜんこく の おてんき です			
\\	地域	
\\	ちいき			
\\	この地域は雨が多い。	
\\	地域 
\\	この ちいき は あめ が おおい			
\\	重要	
\\	じゅうよう			
\\	これは重要な書類です。	
\\	重要 
\\	これ は じゅうよう な しょるい です			
\\	姿	
\\	すがた			
\\	遠くに彼女の姿が見えた。	
\\	姿 
\\	とおく に かのじょ の すがた が みえた			
\\	先ず	
\\	まず			
\\	帰ったら先ず手を洗いましょう。	
\\	先ず 
\\	かえったら まず て を あらいましょう 。			
\\	昭和	
\\	しょうわ			
\\	私の両親は昭和生まれです。	
\\	昭和 
\\	わたし の りょうしん は しょうわうまれ です			
\\	社会	
\\	しゃかい			
\\	これは大きな社会問題になっている。	
\\	社会 
\\	これ は おおき な しゃかい もんだい に なって いる			
\\	プログラム	
\\	受付でプログラムを受け取った。	
\\	プログラム 
\\	うけつけ で ぷろぐらむ を うけとった			
\\	期待	
\\	きたい			
\\	みんな私たちに期待しています。	
\\	期待 
\\	みんな わたしたち に きたい して います			
\\	力	
\\	ちから			
\\	今の首相はあまり力がない。	
\\	首相 
\\	力 
\\	いま の しゅしょう は あまり ちから が ない			
\\	お相撲さんは、みんな力持ちだ。	
\\	力 
\\	おすもうさん は みんな ちからもち だ			
\\	除く	
\\	のぞく			
\\	部長を除く全員が土曜日も働いた。	
\\	除く 
\\	ぶちょう を のぞく ぜんいん が どようび も はたらいた			
\\	一般	
\\	いっぱん			
\\	一般の方はこちらの席へどうぞ。	
\\	一般 
\\	いっぱん の かた は こちら の せき へ どうぞ			
\\	今回	
\\	こんかい			
\\	まあ今回は許してあげよう。	
\\	今回 
\\	まあ こんかい は ゆるして あげよう			
\\	普及	
\\	ふきゅう			
\\	ゴミのリサイクルが普及している。	
\\	普及 
\\	ごみ の りさいくる が ふきゅう して いる			
\\	表現	
\\	ひょうげん			
\\	彼は歌で自分の気持ちを表現した。	
\\	表現 
\\	かれ は うた で じぶん の きもち を ひょうげん した			
\\	作業	
\\	さぎょう			
\\	作業するにはもっと広いスペースが必要だ。	
\\	作業 
\\	さぎょう する に は もっと ひろい すぺーす が ひつよう だ			
\\	予定	
\\	よてい			
\\	今日の予定を教えてください。	
\\	予定 
\\	きょう の よてい を おしえて ください			
\\	工場	
\\	こうじょう			
\\	彼は食品工場で働いています。	
\\	工場 
\\	かれ は しょくひん こうじょう で はたらいて います			
\\	よく	
\\	よく答えが分かりましたね。	
\\	よく 
\\	よく こたえ が わかりました ね			
\\	生産	
\\	せいさん			
\\	ボルドーはワインの生産で有名だ。	
\\	生産 
\\	ぼるどー は わいん の せいさん で ゆうめい だ			
\\	政権	
\\	せいけん			
\\	政権が交代した。	
\\	政権 
\\	せいけん が こうたい した			
\\	技術	
\\	ぎじゅつ			
\\	彼は非常に高い技術を持っている。	
\\	技術 
\\	かれ は ひじょう に たかい ぎじゅつ を もって いる			
\\	まま	
\\	電気がついたままですよ。	
\\	まま 
\\	でんき が ついた まま です よ			
\\	完成	
\\	かんせい			
\\	新しいホームページが完成した。	
\\	完成 
\\	あたらしい ほーむぺーじ が かんせい した			
\\	変化	
\\	へんか			
\\	今年は変化の多い年でした。	
\\	変化 
\\	ことし は へんか の おおい とし でした			
\\	存在	
\\	そんざい			
\\	宇宙人は存在すると思いますか。	
\\	存在 
\\	うちゅうじん は そんざい する と おもいます か			
\\	参加	
\\	さんか			
\\	明日は市民マラソンに参加します。	
\\	参加 
\\	あす は しみん まらそん に さんか します			
\\	テレビ	
\\	土地	
\\	とち			
\\	ここは父の土地です。	
\\	土地 
\\	ここ は ちち の とち です			
\\	減る	
\\	へる			
\\	体重がかなり減りました。	
\\	減る 
\\	たいじゅう が かなり へりました			
\\	検討	
\\	けんとう			
\\	今日中にこの問題を検討してください。	
\\	検討 
\\	きょう じゅう に この もんだい を けんとう して ください			
\\	政治	
\\	せいじ			
\\	私は政治に関心がある。	
\\	政治 
\\	わたし は せいじ に かんしん が ある			
\\	実施	
\\	じっし			
\\	現在、スペシャルキャンペーンを実施中です。	
\\	実施 
\\	げんざい すぺしゃるきゃんぺーん を じっしちゅう です			
\\	開始	
\\	かいし			
\\	運動会は9時開始です。	
\\	開始 
\\	うんどうかい は くじ かいし です			
\\	備える	
\\	そなえる			
\\	災害に備えて大量の水を買い込んだ。	
\\	備える 
\\	さいがい に そなえて たいりょう の みず を かいこんだ			
\\	輸入	
\\	ゆにゅう			
\\	これはイタリアから輸入した服です。	
\\	輸入 
\\	これ は いたりあ から ゆにゅう した ふく です			
\\	最大	
\\	さいだい			
\\	これは世界最大の船です。	
\\	世界 
\\	最大 
\\	これ は せかい さいだい の ふね です			
\\	競争	
\\	きょうそう			
\\	2社は互いに競争している。	
\\	競争 
\\	にしゃ は たがい に きょうそう して いる			
\\	部分	
\\	ぶぶん			
\\	この部分は問題ないです。	
\\	問題 
\\	部分 
\\	この ぶぶん は もんだい ない です			
\\	消える	
\\	きえる			
\\	突然、電気が消えた。	
\\	消える 
\\	とつぜん でんき が きえた			
\\	家族	
\\	かぞく			
\\	うちは五人家族です。	
\\	家族 
\\	うち は ごにん かぞく です			
\\	内容	
\\	ないよう			
\\	この本の内容を説明してください。	
\\	内容 
\\	この ほん の ないよう を せつめい して ください			
\\	比べる	
\\	くらべる			
\\	今月と先月の売上を比べた。	
\\	比べる 
\\	こんげつ と せんげつ の うりあげ を くらべた			
\\	生まれる	
\\	うまれる			
\\	姉夫婦に男の子が生まれました。	
\\	生まれる 
\\	あね ふうふ に おとこのこ が うまれました			
\\	与える	
\\	あたえる			
\\	そのニュースは彼に大きなショックを与えた。	
\\	与える 
\\	その にゅーす は かれ に おおき な しょっく を あたえた			
\\	選挙	
\\	せんきょ			
\\	彼は選挙に出るつもりだ。	
\\	選挙 
\\	かれ は せんきょ に でる つもり だ			
\\	対立	
\\	たいりつ			
\\	その2社は対立しています。	
\\	対立 
\\	その にしゃ は たいりつ して います			
\\	理由	
\\	りゆう			
\\	遅れた理由を教えてください。	
\\	理由 
\\	おくれた りゆう を おしえて ください			
\\	構成	
\\	こうせい			
\\	システムの構成を変えてみました。	
\\	構成 
\\	しすてむ の こうせい を かえて みました			
\\	管理	
\\	かんり			
\\	彼女が私のスケジュールを管理している。	
\\	管理 
\\	かのじょ が わたし の すけじゅーる を かんり して いる			
\\	増加	
\\	ぞうか			
\\	島の人口は年々増加しています。	
\\	増加 
\\	しま の じんこう は ねんねん ぞうか して います			
\\	改革	
\\	かいかく			
\\	彼は行政を改革したいと思っている。	
\\	改革 
\\	かれ は ぎょうせい を かいかく したい と おもって いる			
\\	会談	
\\	かいだん			
\\	会談の内容が発表されました。	
\\	発表 
\\	内容 
\\	会談 
\\	かいだん の ないよう が はっぴょう されました			
\\	会長	
\\	かいちょう			
\\	初めに会長が挨拶した。	
\\	会長 
\\	はじめ に かいちょう が あいさつ した			
\\	ソフト	
\\	このソフトで日本語を勉強することができます。	
\\	ソフト 
\\	この そふと で にほんご を べんきょう する こと が できます			
\\	提供	
\\	ていきょう			
\\	彼がパーティー会場を提供してくれました。	
\\	提供 
\\	かれ が ぱーてぃー かいじょう を ていきょう して くれました			
\\	ただ	
\\	この温泉はただです。	
\\	ただ 
\\	この おんせん は ただ です			
\\	これら	
\\	今日はこれらの問題について話し合います。	
\\	これら 
\\	きょう は これら の もんだい に ついて はなしあいます			
\\	広がる	
\\	ひろがる			
\\	留学してから私の世界が広がった。	
\\	広がる 
\\	りゅうがく して から わたし の せかい が ひろがった			
\\	特徴	
\\	とくちょう			
\\	この曲線がこの車の特徴です。	
\\	特徴 
\\	この きょくせん が この くるま の とくちょう です			
\\	調べる	
\\	しらべる			
\\	この単語の意味を辞書で調べましょう。	
\\	調べる 
\\	この たんご の いみ を じしょ で しらべましょう			
\\	強調	
\\	きょうちょう			
\\	彼は良いところだけを強調した。	
\\	強調 
\\	かれ は よい ところ だけ を きょうちょう した			
\\	制度	
\\	せいど			
\\	来年から新しい制度が始まります。	
\\	制度 
\\	らいねん から あたらしい せいど が はじまります			
\\	方針	
\\	ほうしん			
\\	今後の方針が決まった。	
\\	方針 
\\	こんご の ほうしん が きまった			
\\	従来	
\\	じゅうらい			
\\	このプリンターは従来のものより速い。	
\\	従来 
\\	この ぷりんたー は じゅうらい の もの より はやい			
\\	いずれ	
\\	いずれまたお会いしましょう。	
\\	いずれ 
\\	いずれ また おあい しましょう			
\\	事故	
\\	じこ			
\\	彼は事故で怪我をしました。	
\\	事故 
\\	かれ は じこ で けが を しました			
\\	電話	
\\	でんわ			
\\	あとで電話します。	
\\	電話 
\\	あとで でんわ します			
\\	外国	
\\	がいこく			
\\	母はまだ外国に行ったことがありません。	
\\	外国 
\\	はは は まだ がいこく に いった こと が ありません			
\\	目的	
\\	もくてき			
\\	彼が来た目的が分かりません。	
\\	目的 
\\	かれ が きた もくてき が わかりません			
\\	自由	
\\	じゆう			
\\	今日は自由な時間が多い。	
\\	自由 
\\	きょう は じゆう な じかん が おおい			
\\	資金	
\\	しきん			
\\	私たちは今、結婚資金を貯めています。	
\\	資金 
\\	わたしたち は いま けっこん しきん を ためて います			
\\	銀行	
\\	ぎんこう			
\\	銀行は3時まで開いています。	
\\	銀行 
\\	ぎんこう は さんじ まで あいて います			
\\	事実	
\\	じじつ			
\\	それは全て事実ですか。	
\\	事実 
\\	それ は すべて じじつ です か			
\\	決定	
\\	けってい			
\\	会議で重要な決定がありました。	
\\	決定 
\\	かいぎ で じゅうよう な けってい が ありました			
\\	方向	
\\	ほうこう			
\\	あの人たちは皆、同じ方向を見ている。	
\\	方向 
\\	あの ひとたち は みな おなじ ほうこう を みて いる			
\\	住宅	
\\	じゅうたく			
\\	ここは静かな住宅地だ。	
\\	住宅 
\\	ここ は しずかな じゅうたくち だ			
\\	代表	
\\	だいひょう			
\\	党の代表が質問に答えました。	
\\	党 
\\	代表 
\\	とう の だいひょう が しつもん に こたえました			
\\	彼がクラスの代表だ。	
\\	代表 
\\	かれ が くらす の だいひょう だ			
\\	激しい	
\\	はげしい			
\\	激しい雨が降っています。	
\\	激しい 
\\	はげしい あめ が ふって います			
\\	株	
\\	かぶ			
\\	最近株を始めました。	
\\	最近 
\\	株 
\\	さいきん かぶ を はじめました			
\\	交渉	
\\	こうしょう			
\\	今、値段を交渉しています。	
\\	交渉 
\\	いま ねだん を こうしょう して います			
\\	結果	
\\	けっか			
\\	試合の結果を早く知りたい。	
\\	結果 
\\	しあい の けっか を はやく しりたい			
\\	全体	
\\	ぜんたい			
\\	全体の80パーセントが完成しました。	
\\	完成 
\\	全体 
\\	ぜんたい の はちじゅっぱーせんと が かんせい しました			
\\	後	
\\	のち			
\\	後に彼は総理大臣になりました。	
\\	後 
\\	のち に かれ は そうり だいじん に なりました			
\\	命令	
\\	めいれい			
\\	彼女は命令に従わなかった。	
\\	命令 
\\	かのじょ は めいれい に したがわなかった			
\\	十分	
\\	じゅうぶん			
\\	お金はまだ十分あります。	
\\	十分 
\\	おかね は まだ じゅうぶん あります			
\\	あまり	
\\	あまり			
\\	私はテレビをあまり見ません。	
\\	テレビ 
\\	あまり 
\\	わたし は てれび を あまり みません			
\\	このビールはあまり美味しくありません。	
\\	あまり 
\\	この びーる は あまり おいしく ありません			
\\	経営	
\\	けいえい			
\\	我が社の経営はうまくいっています。	
\\	経営 
\\	わがしゃ の けいえい は うまく いって います			
\\	原因	
\\	げんいん			
\\	この事故の原因は何ですか。	
\\	事故 
\\	原因 
\\	このじこ の げんいん は なん です か			
\\	写真	
\\	しゃしん			
\\	写真は良い思い出になります。	
\\	写真 
\\	しゃしん は いい おもいで に なります			
\\	立場	
\\	たちば			
\\	彼女は自分の立場を分かっていない。	
\\	立場 
\\	かのじょ は じぶん の たちば を わかって いない			
\\	対策	
\\	たいさく			
\\	一緒に対策を考えましょう。	
\\	対策 
\\	いっしょ に たいさく を かんがえましょう			
\\	繰り返す	
\\	くりかえす			
\\	彼女は同じ間違いを繰り返した。	
\\	繰り返す 
\\	かのじょ は おなじ まちがい を くりかえした			
\\	評価	
\\	ひょうか			
\\	最近、彼の評価が上がった。	
\\	評価 
\\	さいきん かれ の ひょうか が あがった			
\\	なお	
\\	なお、雨の場合は中止です。	
\\	なお 
\\	なお あめ の ばあい は ちゅうし です			
\\	計算	
\\	けいさん			
\\	その計算は間違っている。	
\\	計算 
\\	その けいさん は まちがって いる			
\\	投資	
\\	とうし			
\\	私は4つの会社に投資しています。	
\\	投資 
\\	わたし は よっつ の かいしゃ に とうし して います			
\\	可能	
\\	かのう			
\\	20キロのダイエットは可能だと思いますか。	
\\	可能 
\\	にじゅっきろ の だいえっと は かのう だ と おもいます か			
\\	加える	
\\	くわえる			
\\	塩、コショウを加えてください。	
\\	加える 
\\	しお こしょう を くわえて ください			
\\	種類	
\\	しゅるい			
\\	バラには色々な種類があります。	
\\	種類 
\\	ばら に は いろいろ な しゅるい が あります			
\\	強化	
\\	きょうか			
\\	国は国語教育を強化しています。	
\\	強化 
\\	くに は こくご きょういく を きょうか して います			
\\	意見	
\\	いけん			
\\	あなたの意見が聞きたいです。	
\\	意見 
\\	あなた の いけん が ききたい です			
\\	伴う	
\\	ともなう			
\\	その仕事は危険を伴う。	
\\	伴う 
\\	その しごと は きけん を ともなう			
\\	新聞	
\\	しんぶん			
\\	今日の新聞、どこに置いた?	
\\	新聞 
\\	きょう の しんぶん どこ に おいた			
\\	文章	
\\	ぶんしょう			
\\	彼は文章がとてもうまい。	
\\	文章 
\\	かれ は ぶんしょう が とても うまい			
\\	目立つ	
\\	めだつ			
\\	彼女の大きな帽子はとても目立つ。	
\\	目立つ 
\\	かのじょ の おおき な ぼうし は とても めだつ			
\\	設計	
\\	せっけい			
\\	兄は船の設計をしています。	
\\	設計 
\\	あに は ふね の せっけい を して います			
\\	構造	
\\	こうぞう			
\\	この建物の構造は複雑です。	
\\	構造 
\\	この たてもの の こうぞう は ふくざつ です			
\\	国会	
\\	こっかい			
\\	国会が再開した。	
\\	国会 
\\	こっかい が さいかい した			
\\	ほぼ	
\\	仕事がほぼ終わりました。	
\\	ほぼ 
\\	しごと が ほぼ おわりました			
\\	相手	
\\	あいて			
\\	まず相手の動きを見ましょう。	
\\	動き 
\\	相手 
\\	まず あいて の うごき を みましょう			
\\	試合の相手は誰ですか。	
\\	相手 
\\	しあい の あいて は だれ です か			
\\	発生	
\\	はっせい			
\\	交差点で事故が発生した。	
\\	発生 
\\	こうさてん で じこ が はっせい した			
\\	採用	
\\	さいよう			
\\	その会社は女性を多く採用している。	
\\	女性 
\\	採用 
\\	その かいしゃ は じょせい を おおく さいよう して いる			
\\	残る	
\\	のこる			
\\	料理がたくさん残りました。	
\\	残る 
\\	りょうり が たくさん のこりました			
\\	段階	
\\	だんかい			
\\	この段階では、決断するのはまだ早い。	
\\	段階 
\\	この だんかい で は けつだん する の は まだ はやい			
\\	報告	
\\	ほうこく			
\\	昨日の会議について報告があります。	
\\	報告 
\\	きのう の かいぎ に ついて ほうこく が あります			
\\	商品	
\\	しょうひん			
\\	この商品はよく売れている。	
\\	よく 
\\	商品 
\\	この しょうひん は よく うれて いる			
\\	都市	
\\	とし			
\\	東京は日本一大きな都市です。	
\\	都市 
\\	とうきょう は にほんいち おおき な とし です			
\\	病院	
\\	びょういん			
\\	病院はどこですか。	
\\	病院 
\\	びょういん は どこ です か			
\\	規模	
\\	きぼ			
\\	この動物園は日本一の規模です。	
\\	規模 
\\	この どうぶつえん は にっぽんいち の きぼ です			
\\	業者	
\\	ぎょうしゃ			
\\	引っ越しを業者に頼んだ。	
\\	業者 
\\	ひっこし を ぎょうしゃ に たのんだ			
\\	経験	
\\	けいけん			
\\	今日の試合はいい経験になりました。	
\\	経験 
\\	きょう の しあい は いい けいけん に なりました			
\\	能力	
\\	のうりょく			
\\	彼は能力のある社員です。	
\\	能力 
\\	かれ は のうりょく の ある しゃいん です			
\\	最初	
\\	さいしょ			
\\	5ページの最初を見てください。	
\\	最初 
\\	ごぺーじ の さいしょ を みて ください			
\\	国際	
\\	こくさい			
\\	ここで国際会議が開かれます。	
\\	国際 
\\	ここ で こくさい かいぎ が ひらかれます			
\\	議員	
\\	ぎいん			
\\	彼は国会議員です。	
\\	議員 
\\	かれ は こっかい ぎいん です			
\\	記者	
\\	きしゃ			
\\	彼は新聞記者です。	
\\	記者 
\\	かれ は しんぶん きしゃ です			
\\	場所	
\\	ばしょ			
\\	会社の場所を教えてください。	
\\	場所 
\\	かいしゃ の ばしょ を おしえて ください			
\\	運動	
\\	うんどう			
\\	彼はもっと運動した方がいい。	
\\	運動 
\\	かれ は もっと うんどう した ほう が いい			
\\	サービス	
\\	サービス			
\\	この店はサービスがいい。	
\\	サービス 
\\	この みせ は サービス が いい 。			
\\	集まる	
\\	あつまる			
\\	駅前に人が集まっています。	
\\	集まる 
\\	えきまえ に ひと が あつまって います			
\\	政策	
\\	せいさく			
\\	新しい政策はあまり良いとは思えません。	
\\	政策 
\\	あたらしい せいさく は あまり いい と は おもえません			
\\	車	
\\	くるま			
\\	この道は車が多い。	
\\	車 
\\	この みち は くるま が おおい			
\\	拡大	
\\	かくだい			
\\	この図を拡大コピーしてください。	
\\	拡大 
\\	この ず を かくだい こぴー して ください			
\\	進む	
\\	すすむ			
\\	前に進んでください。	
\\	進む 
\\	まえ に すすんで ください			
\\	達する	
\\	たっする			
\\	気温は35度に達した。	
\\	達する 
\\	きおん は さんじゅうごど に たっした			
\\	会議	
\\	かいぎ			
\\	今日の午後、大事な会議があります。	
\\	会議 
\\	きょう の ごご だいじ な かいぎ が あります			
\\	装置	
\\	そうち			
\\	実験にはこの装置を使います。	
\\	装置 
\\	じっけん に は この そうち を つかいます			
\\	主張	
\\	しゅちょう			
\\	上司は私の主張を受け入れた。	
\\	主張 
\\	じょうし は わたし の しゅちょう を うけいれた			
\\	記録	
\\	きろく			
\\	マラソンで世界記録が出た。	
\\	記録 
\\	まらそん で せかい きろく が でた			
\\	年度	
\\	ねんど			
\\	売り上げは年度によって違います。	
\\	年度 
\\	うりあげ は ねんど に よって ちがいます			
\\	産業	
\\	さんぎょう			
\\	背景	
\\	はいけい			
\\	事件の背景に何があったのだろう。	
\\	背景 
\\	じけん の はいけい に なに が あった の だろう			
\\	材料	
\\	ざいりょう			
\\	サラダの材料をそろえました。	
\\	材料 
\\	さらだ の ざいりょう を そろえました			
\\	要求	
\\	ようきゅう			
\\	彼は私の要求にこたえた。	
\\	要求 
\\	かれ は わたし の ようきゅう に こたえた			
\\	市民	
\\	しみん			
\\	世紀	
\\	せいき			
\\	新しい世紀の始まりです。	
\\	世紀 
\\	あたらしい せいき の はじまり です			
\\	大幅	
\\	おおはば			
\\	計画を大幅に変更した。	
\\	大幅 
\\	けいかく を おおはば に へんこう した			
\\	加工	
\\	かこう			
\\	この工場では魚を加工している。	
\\	加工 
\\	この こうじょう で は さかな を かこう して いる			
\\	実行	
\\	じっこう			
\\	彼はその計画を実行した。	
\\	実行 
\\	かれ は その けいかく を じっこう した			
\\	判断	
\\	はんだん			
\\	彼の判断は正しい。	
\\	判断 
\\	かれ の はんだん は ただしい			
\\	組織	
\\	そしき			
\\	彼はある組織のリーダーだ。	
\\	組織 
\\	かれ は ある そしき の りーだー だ			
\\	輸出	
\\	ゆしゅつ			
\\	彼の会社は車を輸出しています。	
\\	輸出 
\\	かれ の かいしゃ は くるま を ゆしゅつ して います			
\\	警察	
\\	けいさつ			
\\	警察を呼んでください。	
\\	警察 
\\	けいさつ を よんで ください			
\\	伝える	
\\	つたえる			
\\	みんなにこのことを伝えてください。	
\\	伝える 
\\	みんな に この こと を つたえて ください			
\\	公開	
\\	こうかい			
\\	その映画は今日、公開されます。	
\\	公開 
\\	その えいが は きょう こうかい されます			
\\	向かう	
\\	むかう			
\\	今、会社に向かっています。	
\\	向かう 
\\	いま かいしゃ に むかって います			
\\	直す	
\\	なおす			
\\	私がそれを直しました。	
\\	直す 
\\	わたし が それ を なおしました			
\\	具体的	
\\	ぐたいてき			
\\	具体的な例をいくつか見せてください。	
\\	具体的 
\\	ぐたいてき な れい を いくつか みせて ください			
\\	願う	
\\	ねがう			
\\	彼が元気になるよう願っています。	
\\	願う 
\\	かれ が げんき に なる よう ねがって います			
\\	グループ	
\\	グループ			
\\	店に学生のグループが来た。	
\\	グループ 
\\	みせ に がくせい の ぐるーぷ が きた			
\\	個人	
\\	こじん			
\\	これは私個人の意見です。	
\\	個人 
\\	これ は わたし こじん の いけん です			
\\	基づく	
\\	もとづく			
\\	この話は真実に基づいています。	
\\	基づく 
\\	この はなし は しんじつ に もとづいて います			
\\	全く	
\\	まったく			
\\	私には全く分かりません。	
\\	全く 
\\	わたし に は まったく わかりません			
\\	見方	
\\	みかた			
\\	彼に対する見方が変わりました。	
\\	見方 
\\	かれ に たいする みかた が かわりました			
\\	姿勢	
\\	しせい			
\\	あの子はいつも姿勢が悪い。	
\\	姿勢 
\\	あの こ は いつも しせい が わるい			
\\	協力	
\\	きょうりょく			
\\	このプロジェクトにはみんなの協力が必要です。	
\\	協力 
\\	この ぷろじぇくと に は みんな の きょうりょく が ひつよう です			
\\	担当	
\\	たんとう			
\\	私はセールスを担当しています。	
\\	担当 
\\	わたし は せーるす を たんとう して います			
\\	提案	
\\	ていあん			
\\	そのアイデアは彼の提案です。	
\\	提案 
\\	その あいであ は かれ の ていあん です			
\\	最後	
\\	さいご			
\\	今日が夏休み最後の日だ。	
\\	最後 
\\	きょう が なつやすみ さいご の ひ だ			
\\	招く	
\\	まねく			
\\	両親を食事に招いた。	
\\	招く 
\\	りょうしん を しょくじ に まねいた			
\\	ホテル	
\\	今ホテルに着きました。	
\\	ホテル 
\\	いま ほてる に つきました			
\\	再び	
\\	ふたたび			
\\	彼は再びここに戻ってきた。	
\\	再び 
\\	かれ は ふたたび ここ に もどって きた			
\\	上昇	
\\	じょうしょう			
\\	地球の気温は上昇している。	
\\	上昇 
\\	ちきゅう の きおん は じょうしょう して いる			
\\	機械	
\\	きかい			
\\	新しい機械が壊れた。	
\\	機械 
\\	あたらしい きかい が こわれた			
\\	注目	
\\	ちゅうもく			
\\	私たちはその会社に注目している。	
\\	注目 
\\	わたしたち は その かいしゃ に ちゅうもく して いる			
\\	過去	
\\	かこ			
\\	それは過去の話だ。	
\\	過去 
\\	それ は かこ の はなし だ			
\\	作品	
\\	さくひん			
\\	この絵はゴッホの作品だ。	
\\	作品 
\\	この え は ごっほ の さくひん だ			
\\	直接	
\\	ちょくせつ			
\\	彼に直接お願いしなさい。	
\\	直接 
\\	かれ に ちょくせつ おねがい しなさい			
\\	批判	
\\	ひはん			
\\	彼は同僚を批判した。	
\\	批判 
\\	かれ は どうりょう を ひはん した			
\\	文化	
\\	ぶんか			
\\	私はこの国の文化を勉強しています。	
\\	文化 
\\	わたし は この くに の ぶんか を べんきょう して います			
\\	施設	
\\	しせつ			
\\	そのホテルにはレジャー施設がたくさんある。	
\\	施設 
\\	その ほてる に は れじゃー しせつ が たくさん ある			
\\	成功	
\\	せいこう			
\\	実際	
\\	じっさい			
\\	彼は実際にはあまり背が高くない。	
\\	実際 
\\	かれ は じっさい に は あまり せ が たかく ない			
\\	当然	
\\	とうぜん			
\\	彼女が怒るのも当然だ。	
\\	当然 
\\	かのじょ が おこる の も とうぜん だ			
\\	年間	
\\	ねんかん			
\\	簡単	
\\	かんたん			
\\	この料理はとても簡単です。	
\\	簡単 
\\	この りょうり は とても かんたん です			
\\	迫る	
\\	せまる			
\\	締め切りが迫っています。	
\\	迫る 
\\	しめきり が せまって います			
\\	まとめる	
\\	彼女は荷物をまとめて出て行った。	
\\	まとめる 
\\	かのじょ は にもつ を まとめて でて いった			
\\	占める	
\\	しめる			
\\	私に届くメールのうち、迷惑メールが7割を占めている。	
\\	占める 
\\	わたし に とどく めーる の うち めいわく めーる が ななわり を しめて いる			
\\	残す	
\\	のこす			
\\	彼女はメッセージを残しました。	
\\	残す 
\\	かのじょ は めっせーじ を のこしました			
\\	言語	
\\	げんご			
\\	彼はアジアの言語を研究している。	
\\	言語 
\\	かれ は あじあ の げんご を けんきゅう して いる			
\\	教育	
\\	きょういく			
\\	彼は海外で教育を受けました。	
\\	教育 
\\	かれ は かいがい で きょういく を うけました			
\\	他	
\\	た			
\\	留学生はアジア人が多く、その他は3割です。	
\\	他 
\\	りゅうがくせい は あじあじん が おおく そのた は さんわり です			
\\	取引	
\\	とりひき			
\\	私たちは中国の会社と取引しています。	
\\	取引 
\\	わたしたち は ちゅうごく の かいしゃ と とりひき して います			
\\	自動車	
\\	じどうしゃ			
\\	日本の自動車産業は世界的に有名だ。	
\\	産業 
\\	自動車 
\\	にほん の じどうしゃ さんぎょう は せかいてき に ゆうめい だ			
\\	彼は自動車会社に就職した。	
\\	自動車 
\\	かれ は じどうしゃ がいしゃ に しゅうしょく した			
\\	反対	
\\	はんたい			
\\	私は反対です。	
\\	反対 
\\	わたし は はんたい です			
\\	若者	
\\	わかもの			
\\	最近の若者は本を読まない。	
\\	若者 
\\	さいきん の わかもの は ほん を よまない			
\\	分	
\\	ぶん			
\\	このケーキはあなたの分です。	
\\	分 
\\	この けーき は あなた の ぶん です			
\\	課題	
\\	かだい			
\\	夏休みの課題は何ですか。	
\\	課題 
\\	なつやすみ の かだい は なん です か			
\\	通常	
\\	つうじょう			
\\	通常は夜8時まで営業しています。	
\\	通常 
\\	つうじょう は よる はちじ まで えいぎょう して います			
\\	同様	
\\	どうよう			
\\	私たちは彼を家族同様に思っている。	
\\	同様 
\\	わたしたち は かれ を かぞく どうよう に おもって いる			
\\	程度	
\\	ていど			
\\	この程度の怪我なら大丈夫です。	
\\	程度 
\\	この ていど の けが なら だいじょうぶ です			
\\	当たる	
\\	あたる			
\\	ボールが彼の頭に当たった。	
\\	当たる 
\\	ぼーる が かれ の あたま に あたった			
\\	訴える	
\\	うったえる			
\\	彼女は会社を訴えた。	
\\	訴える 
\\	かのじょ は かいしゃ を うったえた			
\\	事情	
\\	じじょう			
\\	あなたの事情はよく分かりました。	
\\	事情 
\\	あなた の じじょう は よく わかりました			
\\	発言	
\\	はつげん			
\\	会議で全員が発言した。	
\\	発言 
\\	かいぎ で ぜんいん が はつげん した			
\\	傾向	
\\	けいこう			
\\	最近の若者は難しい本を読まない傾向がある。	
\\	若者 
\\	傾向 
\\	さいきん の わかもの は むずかしい ほん を よまない けいこう が ある			
\\	開催	
\\	かいさい			
\\	京都で国際会議が開催された。	
\\	国際 
\\	開催 
\\	きょうと で こくさいかいぎ が かいさい された			
\\	文字	
\\	もじ			
\\	壁に文字が書いてあった。	
\\	文字 
\\	かべ に もじ が かいて あった			
\\	子	
\\	こ			
\\	その子は日本語が分からない。	
\\	子 
\\	その こ は にほんご が わからない			
\\	述べる	
\\	のべる			
\\	上司が意見を述べた。	
\\	述べる 
\\	じょうし が いけん を のべた			
\\	活動	
\\	かつどう			
\\	彼は地域の活動に参加した。	
\\	活動 
\\	かれ は ちいき の かつどう に さんか した			
\\	期間	
\\	きかん			
\\	テスト期間は10日から15日までだ。	
\\	期間 
\\	てすと きかん は とおか から じゅうごにち まで だ			
\\	一定	
\\	いってい			
\\	申し込むには一定の資格が要る。	
\\	一定 
\\	もうしこむ に は いってい の しかく が いる			
\\	住民	
\\	じゅうみん			
\\	地域の住民が集まって話合いをした。	
\\	住民 
\\	ちいき の じゅうみん が あつまって はなしあい を した			
\\	届く	
\\	とどく			
\\	昨日、父から手紙が届いた。	
\\	届く 
\\	きのう ちち から てがみ が とどいた			
\\	やはり	
\\	彼はやはり遅刻しました。	
\\	やはり 
\\	かれ は やはり ちこく しました			
\\	効果	
\\	こうか			
\\	この薬には胃を守る効果がある。	
\\	効果 
\\	この くすり に は い を まもる こうか が ある			
\\	状況	
\\	じょうきょう			
\\	この状況では出発は難しいです。	
\\	状況 
\\	この じょうきょう で は しゅっぱつ は むずかしい です			
\\	確認	
\\	かくにん			
\\	もう一度、予約を確認した。	
\\	確認 
\\	もう いちど よやく を かくにん した			
\\	位置	
\\	いち			
\\	私の町は東京の北に位置します。	
\\	位置 
\\	わたし の まち は とうきょう の きた に いち します			
\\	はっきり	
\\	はっきり			
\\	今日は山がはっきり見える。	
\\	はっきり 
\\	きょう は やま が はっきり みえる			
\\	つまり	
\\	つまり、あなたは何も知らないのですね。	
\\	つまり 
\\	つまり あなた は なに も しらない の です ね			
\\	通過	
\\	つうか			
\\	次の駅は通過します。	
\\	通過 
\\	つぎ の えき は つうか します			
\\	深い	
\\	ふかい			
\\	あの池はとても深い。	
\\	深い 
\\	あの いけ は とても ふかい			
\\	候補	
\\	こうほ			
\\	会長の候補は3人います。	
\\	候補 
\\	かいちょう の こうほ は さんにん います			
\\	選手	
\\	せんしゅ			
\\	彼はプロのサッカー選手だ。	
\\	選手 
\\	かれ は ぷろ の さっかー せんしゅ だ			
\\	音	
\\	おと			
\\	雨の音が聞こえる。	
\\	音 
\\	あめ の おと が きこえる			
\\	大会	
\\	たいかい			
\\	夏には川辺で花火大会があります。	
\\	大会 
\\	なつ に は かわべ で はなびたいかい が あります			
\\	基本	
\\	きほん			
\\	今、ジャズダンスの基本を習っています。	
\\	基本 
\\	いま じゃず だんす の きほん を ならって います			
\\	契約	
\\	けいやく			
\\	その選手は新しいチームと契約した。	
\\	契約 
\\	その せんしゅ は あたらしい ちーむ と けいやく した			
\\	発見	
\\	はっけん			
\\	新しい星が発見された。	
\\	発見 
\\	あたらしい ほし が はっけん された			
\\	当時	
\\	とうじ			
\\	彼女は当時、まだ3才だった。	
\\	当時 
\\	かのじょ は とうじ まだ さんさい だった			
\\	自宅	
\\	じたく			
\\	自宅に電話を下さい。	
\\	自宅 
\\	じたく に でんわ を ください			
\\	空気	
\\	くうき			
\\	ここは空気がきれいです。	
\\	空気 
\\	ここ は くうき が きれい です			
\\	設ける	
\\	もうける			
\\	授業の終わりに復習の時間を設けた。	
\\	設ける 
\\	じゅぎょう の おわり に ふくしゅう の じかん を もうけた			
\\	歴史	
\\	れきし			
\\	私は歴史に興味があります。	
\\	歴史 
\\	わたし は れきし に きょうみ が あります			
\\	各国	
\\	かっこく			
\\	各国の代表がニューヨークに集まった。	
\\	各国 
\\	かっこく の だいひょう が にゅーよーく に あつまった			
\\	特に	
\\	とくに			
\\	特に質問はありません。	
\\	特に 
\\	とくに しつもん は ありません			
\\	実験	
\\	じっけん			
\\	ついに実験が成功した。	
\\	成功 
\\	実験 
\\	ついに じっけん が せいこう した			
\\	科学の授業で実験をした。	
\\	実験 
\\	かがく の じゅぎょう で じっけん を した			
\\	訪れる	
\\	おとずれる			
\\	年間5万人がここを訪れます。	
\\	年間 
\\	訪れる 
\\	ねんかん ごまんにん が ここ を おとずれます			
\\	私は夏に京都を訪れました。	
\\	訪れる 
\\	わたし は なつ に きょうと を おとずれました			
\\	建物	
\\	たてもの			
\\	これは日本一古い建物です。	
\\	建物 
\\	これ は にほんいち ふるい たてもの です			
\\	ビル	
\\	私の会社はあのビルの8階です。	
\\	ビル 
\\	わたし の かいしゃ は あの びる の はちかい です			
\\	彼ら	
\\	かれら			
\\	彼らはバスケットの選手です。	
\\	彼ら 
\\	かれら は ばすけっと の せんしゅ です			
\\	少年	
\\	しょうねん			
\\	少年たちがサッカーをしている。	
\\	少年 
\\	しょうねんたち が さっかー を して いる			
\\	諸国	
\\	しょこく			
\\	彼はヨーロッパ諸国を旅行した。	
\\	諸国 
\\	かれ は よーろっぱ しょこく を りょこう した			
\\	もちろん	
\\	もちろん一緒に行きます。	
\\	もちろん 
\\	もちろん いっしょ に いきます			
\\	事務所	
\\	じむしょ			
\\	後で事務所に来てください。	
\\	事務所 
\\	あとで じむしょ に きて ください			
\\	責任	
\\	せきにん			
\\	彼は失敗の責任を取って、会社を辞めた。	
\\	責任 
\\	かれ は しっぱい の せきにん を とって かいしゃ を やめた			
\\	かつて	
\\	かつて			
\\	かつて私が学生だった頃のことです。	
\\	かつて 
\\	かつて わたし が がくせい だった ころ の こと です			
\\	限る	
\\	かぎる			
\\	このサービスは週末に限ります	
\\	限る 
\\	この さーびす は しゅうまつ に かぎります			
\\	海外	
\\	かいがい			
\\	彼は海外での生活が長いです。	
\\	海外 
\\	かれ は かいがい で の せいかつ が ながい です			
\\	条件	
\\	じょうけん			
\\	この条件では厳し過ぎます。	
\\	条件 
\\	この じょうけん で は きびし すぎます			
\\	家庭	
\\	かてい			
\\	彼は家庭を大切にしている。	
\\	家庭 
\\	かれ は かてい を たいせつ に して いる			
\\	教授	
\\	きょうじゅ			
\\	彼は化学の教授です。	
\\	教授 
\\	かれ は かがく の きょうじゅ です			
\\	人口	
\\	じんこう			
\\	その国の人口はどのくらいですか。	
\\	人口 
\\	その くに の じんこう は どの くらい です か			
\\	海	
\\	うみ			
\\	海は広くて大きい。	
\\	海 
\\	うみ は ひろく て おおきい			
\\	基準	
\\	きじゅん			
\\	判断の基準が示された。	
\\	基準 
\\	はんだん の きじゅん が しめされた			
\\	攻撃	
\\	こうげき			
\\	2003年にアメリカはイラクを攻撃した。	
\\	攻撃 
\\	にせんさんねん に あめりか は いらく を こうげき した			
\\	時代	
\\	じだい			
\\	今は便利さとスピードの時代だ。	
\\	時代 
\\	いま は べんりさ と すぴーど の じだい だ			
\\	時期	
\\	じき			
\\	今はあなたにとって大事な時期です。	
\\	時期 
\\	いま は あなた に とって だいじ な じき です			
\\	製造	
\\	せいぞう			
\\	この工場ではエアコンを製造しています。	
\\	製造 
\\	この こうじょう で は えあこん を せいぞう して います			
\\	大型	
\\	おおがた			
\\	大型のテレビを買った。	
\\	大型 
\\	おおがた の てれび を かった			
\\	周辺	
\\	しゅうへん			
\\	この周辺には大学が多い。	
\\	周辺 
\\	この しゅうへん に は だいがく が おおい			
\\	異なる	
\\	ことなる			
\\	彼と私はいつも意見が異なる。	
\\	異なる 
\\	かれ と わたし は いつも いけん が ことなる			
\\	逮捕	
\\	たいほ			
\\	逃げていた犯人が逮捕されました。	
\\	逮捕 
\\	にげて いた はんにん が たいほ されました			
\\	関連	
\\	かんれん			
\\	関連のセクションへ連絡した。	
\\	関連 
\\	かんれん の せくしょん へ れんらく した			
\\	貿易	
\\	ぼうえき			
\\	父は貿易の仕事をしています。	
\\	貿易 
\\	ちち は ぼうえき の しごと を して います			
\\	社員	
\\	しゃいん			
\\	彼は優秀な社員です。	
\\	社員 
\\	かれ は ゆうしゅう な しゃいん です			
\\	結合	
\\	けつごう			
\\	ファイルを結合して1つにしました。	
\\	結合 
\\	ふぁいる を けつごう して ひとつ に しました			
\\	自然	
\\	しぜん			
\\	スポーツ	
\\	あなたは何かスポーツをしていますか。	
\\	スポーツ 
\\	あなた は なにか すぽーつ を して います か			
\\	患者	
\\	かんじゃ			
\\	患者は眠っています。	
\\	患者 
\\	かんじゃ は ねむって います			
\\	なぜ	
\\	なぜ来なかったの。	
\\	なぜ 
\\	なぜ こなかった の			
\\	映画	
\\	えいが			
\\	彼はよく映画を見ます。	
\\	映画 
\\	かれ は よく えいが を みます			
\\	並ぶ	
\\	ならぶ			
\\	ここに並んでください。	
\\	並ぶ 
\\	ここ に ならんで ください			
\\	生徒	
\\	せいと			
\\	このクラスの生徒は30人です。	
\\	生徒 
\\	この くらす の せいと は さんじゅうにん です			
\\	そのまま	
\\	そのままお待ちください。	
\\	そのまま 
\\	そのまま おまち ください			
\\	今度	
\\	こんど			
\\	今度はどこに行きたいですか。	
\\	今度 
\\	こんど は どこ に いきたい です か			
\\	分ける	
\\	わける			
\\	お菓子をみんなで分けました。	
\\	分ける 
\\	おかし を みんな で わけました			
\\	運ぶ	
\\	はこぶ			
\\	いすを二階に運んでください。	
\\	運ぶ 
\\	いす を にかい に はこんで ください			
\\	複雑	
\\	ふくざつ			
\\	このプログラムはとても複雑です。	
\\	複雑 
\\	この ぷろぐらむ は とても ふくざつ です			
\\	回る	
\\	まわる			
\\	高校	
\\	こうこう			
\\	妹は高校に通っています。	
\\	高校 
\\	いもうと は こうこう に かよって います			
\\	工事	
\\	こうじ			
\\	工事の音がうるさい。	
\\	工事 
\\	こうじ の おと が うるさい			
\\	毎日	
\\	まいにち			
\\	私たちは毎日散歩をします。	
\\	毎日 
\\	わたしたち は まいにち さんぽ を します			
\\	帰国	
\\	きこく			
\\	彼は帰国しました。	
\\	帰国 
\\	かれ は きこく しました			
\\	雨	
\\	あめ			
\\	雨が降っています。	
\\	雨 
\\	あめ が ふって います			
\\	経つ	
\\	たつ			
\\	あれから14年が経ちました。	
\\	経つ 
\\	あれ から じゅうよねん が たちました			
\\	花	
\\	はな			
\\	きれいな花が咲きました。	
\\	花 
\\	きれい な はな が さきました			
\\	出席	
\\	しゅっせき			
\\	午後は会議に出席します。	
\\	出席 
\\	ごご は かいぎ に しゅっせき します			
\\	外国人	
\\	がいこくじん			
\\	日本に住む外国人が増えています。	
\\	外国人 
\\	にほん に すむ がいこくじん が ふえて います			
\\	道路	
\\	どうろ			
\\	この道路は3年前にできました。	
\\	道路 
\\	この どうろ は さんねんまえ に できました			
\\	地球	
\\	ちきゅう			
\\	月は地球のまわりを回っています。	
\\	回る 
\\	地球 
\\	つき は ちきゅう の まわり を まわって います			
\\	地球は丸い。	
\\	地球 
\\	ちきゅう は まるい			
\\	英語	
\\	えいご			
\\	あなたは英語が話せますか。	
\\	英語 
\\	あなた は えいご が はなせます か			
\\	危険	
\\	きけん			
\\	その地域は今、危険だ。	
\\	危険 
\\	その ちいき は いま きけん だ			
\\	普通	
\\	ふつう			
\\	彼女は普通の女の子だ。	
\\	普通 
\\	かのじょ は ふつう の おんな の こ だ			
\\	説明	
\\	せつめい			
\\	この単語の意味を説明してください。	
\\	説明 
\\	この たんご の いみ を せつめい して ください			
\\	半分	
\\	はんぶん			
\\	お菓子を友達に半分あげた。	
\\	半分 
\\	おかし を ともだち に はんぶん あげた			
\\	注意	
\\	ちゅうい			
\\	車に注意してください。	
\\	注意 
\\	くるま に ちゅうい して ください			
\\	安全	
\\	あんぜん			
\\	市民の安全は大切だ。	
\\	市民 
\\	安全 
\\	しみん の あんぜん は たいせつ だ			
\\	安全が第一です。	
\\	安全 
\\	あんぜん が だいいち です			
\\	手紙	
\\	てがみ			
\\	友人から手紙をもらいました。	
\\	手紙 
\\	ゆうじん から てがみ を もらいました			
\\	努力	
\\	どりょく			
\\	もっと努力しよう。	
\\	努力 
\\	もっと どりょく しよう			
\\	手術	
\\	しゅじゅつ			
\\	父は胸の手術をした。	
\\	手術 
\\	ちち は むね の しゅじゅつ を した			
\\	準備	
\\	じゅんび			
\\	明日の会議の準備をした。	
\\	準備 
\\	あす の かいぎ の じゅんび を した			
\\	守る	
\\	まもる			
\\	みんなで自然を守りましょう。	
\\	自然 
\\	守る 
\\	みんな で しぜん を まもりましょう			
\\	彼は約束を守る人です。	
\\	守る 
\\	かれ は やくそく を まもる ひと です			
\\	驚く	
\\	おどろく			
\\	彼は血を見て驚いた。	
\\	驚く 
\\	かれ は ち を みて おどろいた			
\\	地震	
\\	じしん			
\\	日本は地震が多いです。	
\\	地震 
\\	にほん は じしん が おおい です			
\\	話し合う	
\\	はなしあう			
\\	私たちはよく話し合いました。	
\\	話し合う 
\\	わたしたち は よく はなしあいました			
\\	木	
\\	き			
\\	台風で木が倒れた。	
\\	木 
\\	たいふう で き が たおれた			
\\	許す	
\\	ゆるす			
\\	私は彼を許しました。	
\\	許す 
\\	わたし は かれ を ゆるしました			
\\	雪	
\\	ゆき			
\\	クリスマスに雪が降りました。	
\\	雪 
\\	くりすます に ゆき が ふりました			
\\	動物	
\\	どうぶつ			
\\	私は動物が大好きです。	
\\	動物 
\\	わたし は どうぶつ が だいすき です			
\\	合う	
\\	あう			
\\	この靴は私の足に合っている。	
\\	合う 
\\	この くつ は わたし の あし に あって いる			
\\	平和	
\\	へいわ			
\\	この国は平和です。	
\\	平和 
\\	この くに は へいわ です			
\\	非常に	
\\	ひじょうに			
\\	これは非常に重要です。	
\\	非常に 
\\	これ は ひじょう に じゅうよう です			
\\	遅れる	
\\	おくれる			
\\	今朝彼女は学校に遅れました。	
\\	遅れる 
\\	けさ かのじょ は がっこう に おくれました			
\\	もし	
\\	もし			
\\	もし雨が降ったら、行きません。	
\\	もし 
\\	もし あめ が ふったら いきません			
\\	卒業	
\\	そつぎょう			
\\	私は去年、大学を卒業した。	
\\	卒業 
\\	わたし は きょねん だいがく を そつぎょう した			
\\	移る	
\\	うつる			
\\	彼女は新しい会社に移った。	
\\	移る 
\\	かのじょ は あたらしい かいしゃ に うつった			
\\	農業	
\\	のうぎょう			
\\	私は農業を勉強しています。	
\\	農業 
\\	わたし は のうぎょう を べんきょう して います			
\\	空港	
\\	くうこう			
\\	空港までリムジンバスで行った。	
\\	空港 
\\	くうこう まで りむじん ばす で いった			
\\	珍しい	
\\	めずらしい			
\\	昨日珍しい果物を食べました。	
\\	珍しい 
\\	きのう めずらしい くだもの を たべました			
\\	急ぐ	
\\	いそぐ			
\\	私たちは駅へ急ぎました。	
\\	急ぐ 
\\	わたしたち は えき へ いそぎました			
\\	つもり	
\\	明日からタバコを止めるつもりです。	
\\	つもり 
\\	あした から たばこ を やめる つもり です			
\\	必ず	
\\	かならず			
\\	必ずシートベルトを着けて下さい。	
\\	必ず 
\\	かならず しーとべると を つけて ください			
\\	数字	
\\	すうじ			
\\	数字は苦手です。	
\\	数字 
\\	すうじ は にがて です			
\\	やっと	
\\	やっと仕事が終わりました。	
\\	やっと 
\\	やっと しごと が おわりました			
\\	旅行	
\\	りょこう			
\\	彼女は旅行が好きです。	
\\	旅行 
\\	かのじょ は りょこう が すき です			
\\	音楽	
\\	おんがく			
\\	私は音楽を聞くのが好きだ。	
\\	音楽 
\\	わたし は おんがく を きく の が すき だ			
\\	川	
\\	かわ			
\\	小さな川を渡りました。	
\\	川 
\\	ちいさ な かわ を わたりました			
\\	例えば	
\\	たとえば			
\\	例えば、このソフトで日本語を勉強することができます。	
\\	例えば 
\\	たとえば この そふと で にほんご を べんきょう する こと が できます			
\\	小学校	
\\	しょうがっこう			
\\	家の近くに小学校があります。	
\\	小学校 
\\	いえ の ちかく に しょうがっこう が あります			
\\	壁	
\\	かべ			
\\	壁に絵が掛かっている。	
\\	壁 
\\	かべ に え が かかって いる			
\\	流れる	
\\	ながれる			
\\	ラジオから美しい音楽が流れています。	
\\	音楽 
\\	流れる 
\\	らじお から うつくしい おんがく が ながれて います			
\\	夢	
\\	ゆめ			
\\	私の夢は先生になることです。	
\\	夢 
\\	わたし の ゆめ は せんせい に なる こと です			
\\	逃げる	
\\	にげる			
\\	過ぎる	
\\	すぎる			
\\	時が過ぎるのは速い。	
\\	過ぎる 
\\	とき が すぎる の は はやい 。			
\\	首都	
\\	しゅと			
\\	東京は日本の首都です。	
\\	首都 
\\	とうきょう は にっぽん の しゅと です			
\\	教師	
\\	きょうし			
\\	彼は高校教師だ。	
\\	教師 
\\	かれ は こうこう きょうし だ			
\\	相談	
\\	そうだん			
\\	相談したいことがあります。	
\\	相談 
\\	そうだん したい こと が あります			
\\	用意	
\\	ようい			
\\	食事の用意ができました。	
\\	用意 
\\	しょくじ の ようい が できました			
\\	ニュース	
\\	夜のニュースを見ましたか。	
\\	ニュース 
\\	よる の にゅーす を みました か			
\\	法律	
\\	ほうりつ			
\\	新しい法律ができた。	
\\	法律 
\\	あたらしい ほうりつ が できた			
\\	大変	
\\	たいへん			
\\	大変なことが起こりました。	
\\	大変 
\\	たいへん な こと が おこりました			
\\	戦争	
\\	せんそう			
\\	2003年にイラクで戦争があった。	
\\	戦争 
\\	にせんさんねん に いらく で せんそう が あった			
\\	ずっと	
\\	父は休みの日はずっとテレビを見ている。	
\\	ずっと 
\\	ちち は やすみ の ひ は ずっと てれび を みて いる			
\\	自然	
\\	しぜん			
\\	親が子供を守るのは自然なことだ。	
\\	自然 
\\	おや が こども を まもる の は しぜん な こと だ			
\\	島	
\\	しま			
\\	日本は島国です。	
\\	島 
\\	にっぽん は しまぐに です			
\\	窓	
\\	まど			
\\	鳥が窓から逃げたよ。	
\\	逃げる 
\\	窓 
\\	とり が まど から にげた よ			
\\	窓を開けてください。	
\\	窓 
\\	まど を あけて ください			
\\	考え	
\\	かんがえ			
\\	それは良い考えです。	
\\	考え 
\\	それ は いい かんがえ です			
\\	ビデオ	
\\	私はその番組をビデオに撮った。	
\\	ビデオ 
\\	わたし は その ばんぐみ を びでお に とった			
\\	マンション	
\\	彼はマンションに住んでいます。	
\\	マンション 
\\	かれ は まんしょん に すんで います			
\\	健康	
\\	けんこう			
\\	私は健康な生活を送っています。	
\\	健康 
\\	わたし は けんこう な せいかつ を おくって います			
\\	通り	
\\	とおり			
\\	この通りはにぎやかですね。	
\\	通り 
\\	この とおり は にぎやか です ね			
\\	喜ぶ	
\\	よろこぶ			
\\	彼女はとても喜びました。	
\\	喜ぶ 
\\	かのじょ は とても よろこびました			
\\	入院	
\\	にゅういん			
\\	昨日、母が入院しました。	
\\	入院 
\\	きのう はは が にゅういん しました			
\\	増やす	
\\	ふやす			
\\	あの町は緑を増やしています。	
\\	増やす 
\\	あの まち は みどり を ふやして います			
\\	育てる	
\\	そだてる			
\\	彼女は三人の子を育てました。	
\\	育てる 
\\	かのじょ は さんにん の こ を そだてました			
\\	温度	
\\	おんど			
\\	今、部屋の温度は25度だ。	
\\	温度 
\\	いま へや の おんど は にじゅうごど だ			
\\	雑誌	
\\	ざっし			
\\	この雑誌はよく売れています。	
\\	雑誌 
\\	この ざっし は よく うれて います			
\\	私たち	
\\	わたしたち			
\\	私たちは来月結婚します。	
\\	私たち 
\\	わたしたち は らいげつ けっこん します 。			
\\	食事	
\\	しょくじ			
\\	今日は上司と食事をする。	
\\	食事 
\\	きょう は じょうし と しょくじ を する			
\\	風	
\\	かぜ			
\\	今日は風が強いです。	
\\	風 
\\	きょう は かぜ が つよい です			
\\	早く	
\\	はやく			
\\	なるべく早く来て下さい。	
\\	早く 
\\	なるべく はやく きて ください			
\\	明るい	
\\	あかるい			
\\	彼女は明るい性格です。	
\\	明るい 
\\	かのじょ は あかるい せいかく です			
\\	運転	
\\	うんてん			
\\	父は安全運転だ。	
\\	運転 
\\	ちち は あんぜん うんてん だ			
\\	しばらく	
\\	そこでしばらく休んでいます。	
\\	しばらく 
\\	そこで しばらく やすんで います			
\\	高さ	
\\	たかさ			
\\	富士山の高さは3,776メートルです。	
\\	高さ 
\\	ふじさん の たかさ は さんぜんななひゃく-ななじゅうろくめーとる です			
\\	借りる	
\\	かりる			
\\	彼にビデオを借りました。	
\\	借りる 
\\	かれ に びでお を かりました			
\\	放送	
\\	ほうそう			
\\	就職	
\\	しゅうしょく			
\\	最近、若い人たちの就職が難しくなっています。	
\\	就職 
\\	さいきん わかい ひとたち の しゅうしょく が むずかしく なって います			
\\	大きさ	
\\	おおきさ			
\\	この大きさの封筒が欲しいのですが。	
\\	大きさ 
\\	この おおきさ の ふうとう が ほしい の です が			
\\	困る	
\\	こまる			
\\	ケータイをなくして困っています。	
\\	困る 
\\	けーたい を なくして こまって います			
\\	通る	
\\	とおる			
\\	毎日、この道を通ります。	
\\	通る 
\\	まいにち この みち を とおります			
\\	ガス	
\\	地震でガスが止まった。	
\\	ガス 
\\	じしん で がす が とまった			
\\	夫婦	
\\	ふうふ			
\\	その夫婦はとても仲がいい。	
\\	夫婦 
\\	その ふうふ は とても なか が いい			
\\	立てる	
\\	たてる			
\\	彼はケーキにろうそくを立てた。	
\\	立てる 
\\	かれ は けーき に ろうそく を たてた			
\\	伸びる	
\\	のびる			
\\	髪がだいぶ伸びたね。	
\\	伸びる 
\\	かみ が だいぶ のびた ね			
\\	聞こえる	
\\	きこえる			
\\	殺す	
\\	ころす			
\\	私は生き物を殺すのが嫌いだ。	
\\	殺す 
\\	わたし は いきもの を ころす の が きらい だ			
\\	静か	
\\	しずか			
\\	今年の夏休みには静かな所へ行きたい。	
\\	静か 
\\	ことし の なつやすみ に は しずか な ところ へ いきたい			
\\	分かれる	
\\	わかれる			
\\	グループの中で意見が分かれました。	
\\	分かれる 
\\	ぐるーぷ の なか で いけん が わかれました			
\\	詳しい	
\\	くわしい			
\\	もっと詳しく説明してください。	
\\	詳しい 
\\	もっと くわしく せつめい して ください			
\\	正確	
\\	せいかく			
\\	彼の計算は正確です。	
\\	正確 
\\	かれ の けいさん は せいかく です			
\\	うまい	
\\	彼は野球がうまい。	
\\	うまい 
\\	かれ は やきゅう が うまい			
\\	打つ	
\\	うつ			
\\	転んでひざを打ちました。	
\\	打つ 
\\	ころんで ひざ を うちました			
\\	工業	
\\	こうぎょう			
\\	そこは工業都市だ。	
\\	工業 
\\	そこ は こうぎょう とし だ			
\\	サラリーマン	
\\	父はサラリーマンです。	
\\	サラリーマン 
\\	ちち は さらりーまん です			
\\	有名	
\\	ゆうめい			
\\	彼は有名な歌手です。	
\\	有名 
\\	かれ は ゆうめい な かしゅ です			
\\	捨てる	
\\	すてる			
\\	ゴミを捨ててください。	
\\	捨てる 
\\	ごみ を すてて ください			
\\	移す	
\\	うつす			
\\	机を窓の傍に移しました。	
\\	移す 
\\	つくえ を まど の そば に うつしました			
\\	正しい	
\\	ただしい			
\\	それは正しい答えです。	
\\	正しい 
\\	それ は ただしい こたえ です			
\\	売れる	
\\	うれる			
\\	今年の夏はクーラーがよく売れた。	
\\	売れる 
\\	ことし の なつ は くーらー が よく うれた			
\\	番組	
\\	ばんぐみ			
\\	その番組は来週放送されます。	
\\	放送 
\\	番組 
\\	その ばんぐみ は らいしゅう ほうそう されます			
\\	私はこの番組が好きです。	
\\	番組 
\\	わたし は この ばんぐみ が すき です			
\\	代わる	
\\	かわる			
\\	上司に代わって会議に出た。	
\\	代わる 
\\	じょうし に かわって かいぎ に でた			
\\	着く	
\\	つく			
\\	午後8時に大阪に着きます。	
\\	着く 
\\	ごご はちじ に おおさか に つきます			
\\	バス	
\\	バスで行こう。	
\\	バス 
\\	ばす で いこう			
\\	中学	
\\	ちゅうがく			
\\	息子は中学に通っています。	
\\	中学 
\\	むすこ は ちゅうがく に かよって います			
\\	歌	
\\	うた			
\\	私はその歌を知らなかった。	
\\	歌 
\\	わたし は その うた を しらなかった 。			
\\	薄い	
\\	うすい			
\\	この電子辞書はとても薄い。	
\\	薄い 
\\	この でんし じしょ は とても うすい			
\\	減らす	
\\	へらす			
\\	最近、食事を減らしています。	
\\	減らす 
\\	さいきん しょくじ を へらして います			
\\	科学	
\\	かがく			
\\	科学は常に進歩している。	
\\	科学 
\\	かがく は つねに しんぽ して いる			
\\	約束	
\\	やくそく			
\\	約束は守ります。	
\\	約束 
\\	やくそく は まもります			
\\	出発	
\\	しゅっぱつ			
\\	あと15分で出発です。	
\\	出発 
\\	あと じゅうごふん で しゅっぱつ です			
\\	動かす	
\\	うごかす			
\\	部屋の家具を動かしました。	
\\	動かす 
\\	へや の かぐ を うごかしました			
\\	紹介	
\\	しょうかい			
\\	両親に彼女を紹介した。	
\\	紹介 
\\	りょうしん に かのじょ を しょうかい した			
\\	倒れる	
\\	たおれる			
\\	強風で木が倒れた。	
\\	倒れる 
\\	きょうふう で き が たおれた			
\\	席	
\\	せき			
\\	この席、空いてますか。	
\\	席 
\\	この せき あいてます か			
\\	熱	
\\	ねつ			
\\	昨日の夜、熱が出ました。	
\\	熱 
\\	きのう の よる ねつ が でました			
\\	値段	
\\	ねだん			
\\	このベルトの値段は2500円でした。	
\\	値段 
\\	この べると の ねだん は にせんごひゃくえん でした			
\\	長さ	
\\	ながさ			
\\	このケーブルの長さは1メートルです。	
\\	長さ 
\\	この けーぶる の ながさ は いちめーとる です			
\\	クラス	
\\	この学校は1クラス30人です。	
\\	クラス 
\\	この がっこう は 
\\	クラス 
\\	にん です 。			
\\	隣	
\\	となり			
\\	隣の部屋からテレビの音が聞こえる。	
\\	聞こえる 
\\	隣 
\\	となり の へや から てれび の おと が きこえる			
\\	隣の家には犬がいます。	
\\	隣 
\\	となり の いえ に は いぬ が います			
\\	トラック	
\\	トラックを運転できますか。	
\\	トラック 
\\	とらっく を うんてん できます か			
\\	大使館	
\\	たいしかん			
\\	彼は大使館に勤めています。	
\\	大使館 
\\	かれ は たいしかん に つとめて います			
\\	亡くなる	
\\	なくなる			
\\	おととい、昔の友人が亡くなった。	
\\	亡くなる 
\\	おととい むかし の ゆうじん が なくなった			
\\	パーティー	
\\	明日、うちでパーティーを開きます。	
\\	パーティー 
\\	あした うち で ぱーてぃー を ひらきます			
\\	犬	
\\	いぬ			
\\	この犬はとても賢い。	
\\	犬 
\\	この いぬ は とても かしこい			
\\	うまい寿司屋を見つけたよ。	
\\	うまい 
\\	うまい すしや を みつけた よ			
\\	特別	
\\	とくべつ			
\\	あなたは私にとって特別な人です。	
\\	特別 
\\	あなた は わたし に とって とくべつ な ひと です			
\\	教室	
\\	きょうしつ			
\\	私の教室は3階にあります。	
\\	教室 
\\	わたし の きょうしつ は さんがい に あります			
\\	ガラス	
\\	クリスタルガラスの花瓶を買いました。	
\\	ガラス 
\\	くりすたるがらす の かびん を かいました			
\\	振る	
\\	ふる			
\\	犬がしっぽを振っている。	
\\	振る 
\\	いぬ が しっぽ を ふって いる			
\\	落とす	
\\	おとす			
\\	途中で財布を落としました。	
\\	落とす 
\\	とちゅう で さいふ を おとしました			
\\	夕方	
\\	ゆうがた			
\\	夕方、雨が降りました。	
\\	夕方 
\\	ゆうがた あめ が ふりました			
\\	コース	
\\	私は日本語コースを取っています。	
\\	コース 
\\	わたし は にほんご こーす を とって います			
\\	アパート	
\\	彼はアパートに住んでいます。	
\\	アパート 
\\	かれ は あぱーと に すんで います			
\\	小説	
\\	しょうせつ			
\\	私は月に3冊くらい小説を読みます。	
\\	小説 
\\	わたし は つき に さんさつ くらい しょうせつ を よみます			
\\	レコード	
\\	ジャズのレコードをかけました。	
\\	レコード 
\\	じゃず の れこーど を かけました			
\\	どんどん	
\\	ドアをどんどんとたたいた。	
\\	どんどん 
\\	どあ を どんどん と たたいた			
\\	通う	
\\	かよう			
\\	私はジムに通っています。	
\\	通う 
\\	わたし は じむ に かよって います			
\\	秘密	
\\	ひみつ			
\\	これは秘密です。	
\\	秘密 
\\	これ は ひみつ です			
\\	料理	
\\	りょうり			
\\	母は料理が得意です。	
\\	料理 
\\	はは は りょうり が とくい です			
\\	苦しい	
\\	くるしい			
\\	食べ過ぎておなかが苦しい。	
\\	苦しい 
\\	たべすぎて おなか が くるしい			
\\	信じる	
\\	しんじる			
\\	彼はキリストを信じている。	
\\	信じる 
\\	かれ は きりすと を しんじて いる			
\\	何か	
\\	なにか			
\\	道に何か落ちています。	
\\	何か 
\\	みち に なにか おちて います			
\\	カメラ	
\\	カメラが壊れた。	
\\	カメラ 
\\	かめら が こわれた			
\\	下げる	
\\	さげる			
\\	少し音量を下げてください。	
\\	下げる 
\\	すこし おんりょう を さげて ください			
\\	留学	
\\	りゅうがく			
\\	1年間、アメリカに留学しました。	
\\	留学 
\\	いちねんかん あめりか に りゅうがく しました			
\\	無理	
\\	むり			
\\	5時までに家に帰るのは無理です。	
\\	無理 
\\	ごじ まで に いえ に かえる の は むり です			
\\	飾る	
\\	かざる			
\\	テーブルの上に花を飾りました。	
\\	飾る 
\\	てーぶる の うえ に はな を かざりました			
\\	北	
\\	きた			
\\	ロシアは日本の北にあります。	
\\	北 
\\	ろしあ は にっぽん の きた に あります			
\\	訪ねる	
\\	たずねる			
\\	彼はニューヨークの友達を訪ねた。	
\\	訪ねる 
\\	かれ は にゅーよーく の ともだち を たずねた			
\\	テープ	
\\	彼女はその会話をテープに録音した。	
\\	テープ 
\\	かのじょ は その かいわ を てーぷ に ろくおん した			
\\	馬	
\\	うま			
\\	彼は牧場で馬に乗った。	
\\	馬 
\\	かれ は ぼくじょう で うま に のった			
\\	長男	
\\	ちょうなん			
\\	彼の長男は8才です。	
\\	長男 
\\	かれ の ちょうなん は はっさい です			
\\	近所	
\\	きんじょ			
\\	近所にカナダ人が住んでいる。	
\\	近所 
\\	きんじょ に かなだじん が すんで いる			
\\	数学	
\\	すうがく			
\\	兄は数学の先生です。	
\\	数学 
\\	あに は すうがく の せんせい です			
\\	入学	
\\	にゅうがく			
\\	妹は九月にアメリカの大学に入学します。	
\\	入学 
\\	いもうと は くがつ に あめりか の だいがく に にゅうがく します			
\\	黒い	
\\	くろい			
\\	彼女は黒いドレスを着ています。	
\\	黒い 
\\	かのじょ は くろい どれす を きて います			
\\	頼む	
\\	たのむ			
\\	私はハンバーガーを頼みました。	
\\	頼む 
\\	わたし は はんばーがー を たのみました			
\\	並べる	
\\	ならべる			
\\	私は料理をテーブルに並べた。	
\\	並べる 
\\	わたし は りょうり を てーぶる に ならべた			
\\	受け取る	
\\	うけとる			
\\	彼からメールを受け取りました。	
\\	受け取る 
\\	かれ から めーる を うけとりました			
\\	太陽	
\\	たいよう			
\\	太陽が雲に隠れた。	
\\	太陽 
\\	たいよう が くも に かくれた			
\\	年寄り	
\\	としより			
\\	あの村にはお年寄りがたくさん住んでいます。	
\\	年寄り 
\\	あの むら に は おとしより が たくさん すんで います			
\\	味	
\\	あじ			
\\	この料理は味が薄い。	
\\	薄い 
\\	味 
\\	この りょうり は あじ が うすい			
\\	思い出す	
\\	おもいだす			
\\	大切な用事を思い出しました。	
\\	思い出す 
\\	たいせつ な ようじ を おもいだしました			
\\	細かい	
\\	こまかい			
\\	彼女は細かいことにうるさい。	
\\	細かい 
\\	かのじょ は こまかい こと に うるさい			
\\	成績	
\\	せいせき			
\\	成績が上がりました。	
\\	成績 
\\	せいせき が あがりました			
\\	涙	
\\	なみだ			
\\	母は涙を流していた。	
\\	涙 
\\	はは は なみだ を ながして いた			
\\	横	
\\	よこ			
\\	横の長さは1メートルです。	
\\	横 
\\	よこ の ながさ は いちめーとる です			
\\	穴	
\\	あな			
\\	靴下に穴が開いている。	
\\	穴 
\\	くつした に あな が あいて いる			
\\	季節	
\\	きせつ			
\\	私の一番好きな季節は春です。	
\\	季節 
\\	わたし の いちばん すき な きせつ は はる です			
\\	ビール	
\\	夏はビールがとても美味しい。	
\\	ビール 
\\	なつ は びーる が とても おいしい			
\\	故障	
\\	こしょう			
\\	冷蔵庫が故障しました。	
\\	故障 
\\	れいぞうこ が こしょう しました			
\\	裏	
\\	うら			
\\	会社の裏に公園があります。	
\\	裏 
\\	かいしゃ の うら に こうえん が あります			
\\	ページ	
\\	迎える	
\\	むかえる			
\\	姉が空港まで迎えに来てくれます。	
\\	空港 
\\	迎える 
\\	あね が くうこう まで むかえに きて くれます			
\\	野菜	
\\	やさい			
\\	私は毎日たくさん野菜を食べます。	
\\	野菜 
\\	わたし は まいにち たくさん やさい を たべます			
\\	渡す	
\\	わたす			
\\	彼に手紙を渡しました。	
\\	渡す 
\\	かれ に てがみ を わたしました			
\\	贈る	
\\	おくる			
\\	母に花を贈った。	
\\	贈る 
\\	はは に はな を おくった			
\\	合格	
\\	ごうかく			
\\	娘が入学試験に合格しました。	
\\	合格 
\\	むすめ が にゅうがく しけん に ごうかく しました			
\\	グラフ	
\\	彼は売上をグラフにした。	
\\	グラフ 
\\	かれ は うりあげ を ぐらふ に した			
\\	公園	
\\	こうえん			
\\	公園に小さな池があります。	
\\	公園 
\\	こうえん に ちいさ な いけ が あります			
\\	戸	
\\	と			
\\	部屋の戸が開いています。	
\\	戸 
\\	へや の と が あいて います			
\\	中学生	
\\	ちゅうがくせい			
\\	息子は中学生です。	
\\	中学生 
\\	むすこ は ちゅうがくせい です			
\\	玄関	
\\	げんかん			
\\	玄関に花を飾りました。	
\\	飾る 
\\	玄関 
\\	げんかん に はな を かざりました			
\\	撮る	
\\	とる			
\\	写真をたくさん撮りました。	
\\	撮る 
\\	しゃしん を たくさん とりました			
\\	夏休み	
\\	なつやすみ			
\\	夏休みにお祖母ちゃんの家へ行きました。	
\\	夏休み 
\\	なつやすみ に おばあちゃん の うち へ いきました			
\\	毎年	
\\	まいとし			
\\	私は毎年、海外旅行に行きます。	
\\	毎年 
\\	わたし は まいとし かいがい りょこう に いきます			
\\	地図	
\\	ちず			
\\	地図を見て来てください。	
\\	地図 
\\	ちず を みて きて ください			
\\	海岸	
\\	かいがん			
\\	海岸を散歩しましょう。	
\\	海岸 
\\	かいがん を さんぽ しましょう			
\\	急に	
\\	きゅうに			
\\	急に用事を思い出した。	
\\	思い出す 
\\	急に 
\\	きゅうに ようじ を おもいだした			
\\	女の子	
\\	おんなのこ			
\\	あの女の子を知っていますか。	
\\	女の子 
\\	あの おんなのこ を しって います か			
\\	ポスト	
\\	手紙をポストに入れました。	
\\	ポスト 
\\	てがみ を ぽすと に いれました			
\\	暗い	
\\	くらい			
\\	東の空が暗いです。	
\\	暗い 
\\	ひがし の そら が くらい です			
\\	両方	
\\	りょうほう			
\\	チョコレートとケーキを両方ください。	
\\	両方 
\\	ちょこれーと と けーき を りょうほう ください			
\\	向こう	
\\	むこう			
\\	友達は向こうにいます。	
\\	向こう 
\\	ともだち は むこう に います			
\\	忙しい	
\\	いそがしい			
\\	忙しいので手伝ってください。	
\\	忙しい 
\\	いそがしい の で てつだって ください			
\\	給料	
\\	きゅうりょう			
\\	来年から給料が上がります。	
\\	給料 
\\	らいねん から きゅうりょう が あがります			
\\	テスト	
\\	明日は理科のテストです。	
\\	テスト 
\\	あした は りか の てすと です			
\\	あちこち	
\\	私たちは朝からあちこち散歩しました。	
\\	あちこち 
\\	わたしたち は あさ から あちこち さんぽ しました			
\\	切れる	
\\	きれる			
\\	このはさみはよく切れますね。	
\\	切れる 
\\	この はさみ は よく きれます ね			
\\	野球	
\\	やきゅう			
\\	友達と野球をしました。	
\\	野球 
\\	ともだち と やきゅう を しました			
\\	試験	
\\	しけん			
\\	彼は採用試験に合格した。	
\\	試験 
\\	かれ は さいよう しけん に ごうかく した			
\\	台風	
\\	たいふう			
\\	台風が近づいている。	
\\	台風 
\\	たいふう が ちかづいて いる			
\\	規則	
\\	きそく			
\\	あの会社の規則は厳しいです。	
\\	規則 
\\	あの かいしゃ の きそく は きびしい です			
\\	遠く	
\\	とおく			
\\	遠くに船が見えます。	
\\	遠く 
\\	とおく に ふね が みえます			
\\	ゴルフ	
\\	兄はゴルフを始めました。	
\\	ゴルフ 
\\	あに は ごるふ を はじめました			
\\	看護婦	
\\	かんごふ			
\\	母は看護婦です。	
\\	看護婦 
\\	はは は かんごふ です			
\\	ラジオ	
\\	私は毎朝ラジオを聴きます。	
\\	ラジオ 
\\	わたし は まいあさ らじお を ききます			
\\	逢う	
\\	あう			
\\	ついに素晴らしい女性に逢えた。	
\\	逢う 
\\	ついに すばらしい じょせい に あえた 。			
\\	タクシー	
\\	タクシーを呼んでください。	
\\	タクシー 
\\	たくしー を よんで ください			
\\	来月	
\\	らいげつ			
\\	来月から大学生になります。	
\\	来月 
\\	らいげつ から だいがくせい に なります			
\\	教科書	
\\	きょうかしょ			
\\	教科書の36ページを開いてください。	
\\	ページ 
\\	教科書 
\\	きょうかしょ の さんじゅうろくぺーじ を ひらいて ください			
\\	日本語の教科書を忘れた。	
\\	教科書 
\\	にほんご の きょうかしょ を わすれた			
\\	電気	
\\	でんき			
\\	急	
\\	きゅう			
\\	急な坂道を上った。	
\\	急 
\\	きゅう な さかみち を のぼった			
\\	ゆっくり	
\\	もっとゆっくり話してください。	
\\	ゆっくり 
\\	もっと ゆっくり はなして ください			
\\	兄弟	
\\	きょうだい			
\\	彼は3人兄弟です。	
\\	兄弟 
\\	かれ は さんにん きょうだい です			
\\	牛肉	
\\	ぎゅうにく			
\\	夕食に牛肉を買った。	
\\	牛肉 
\\	ゆうしょく に ぎゅうにく を かった			
\\	確か	
\\	たしか			
\\	彼の昇進は確かだ。	
\\	確か 
\\	かれ の しょうしん は たしか だ			
\\	慣れる	
\\	なれる			
\\	新しい家にはもう慣れましたか。	
\\	慣れる 
\\	あたらしい いえ に は もう なれました か			
\\	厚い	
\\	あつい			
\\	その辞書はとても厚い。	
\\	厚い 
\\	その じしょ は とても あつい			
\\	返す	
\\	かえす			
\\	図書館に本を返した。	
\\	返す 
\\	としょかん に ほん を かえした			
\\	中学校	
\\	ちゅうがっこう			
\\	息子の中学校は家から5分です。	
\\	中学校 
\\	むすこ の ちゅうがっこう は いえ から ごふん です			
\\	森	
\\	もり			
\\	私は森を歩くのが好きです。	
\\	森 
\\	わたし は もり を あるく の が すき です			
\\	降りる	
\\	おりる			
\\	次の駅で降ります。	
\\	降りる 
\\	つぎ の えき で おります			
\\	曲	
\\	きょく			
\\	私はこの曲が大好きです。	
\\	曲 
\\	わたし は この きょく が だいすき です			
\\	遊び	
\\	あそび			
\\	お正月には色々な遊びをします。	
\\	遊び 
\\	おしょうがつ に は いろいろ な あそび を します			
\\	奥	
\\	おく			
\\	はさみは机の奥にあった。	
\\	奥 
\\	はさみ は つくえ の おく に あった			
\\	得意	
\\	とくい			
\\	彼は歌が得意です。	
\\	得意 
\\	かれ は うた が とくい です			
\\	下がる	
\\	さがる			
\\	やっと熱が下がった。	
\\	下がる 
\\	やっと ねつ が さがった			
\\	レストラン	
\\	レストランでインド料理を食べました。	
\\	レストラン 
\\	れすとらん で いんど りょうり を たべました			
\\	医者	
\\	いしゃ			
\\	私は医者に相談した。	
\\	医者 
\\	わたし は いしゃ に そうだん した			
\\	カード	
\\	支払いはカードでお願いします。	
\\	カード 
\\	しはらい は かーど で おねがい します			
\\	吹く	
\\	ふく			
\\	今日は北風が吹いている。	
\\	吹く 
\\	きょう は きたかぜ が ふいて いる			
\\	信号	
\\	しんごう			
\\	絵	
\\	え			
\\	これは有名な画家の絵です。	
\\	絵 
\\	これ は ゆうめい な がか の え です			
\\	修理	
\\	しゅうり			
\\	車を修理に出した。	
\\	修理 
\\	くるま を しゅうり に だした			
\\	地下鉄	
\\	ちかてつ			
\\	赤ちゃん	
\\	あかちゃん			
\\	面白い	
\\	おもしろい			
\\	この本はすごく面白かった。	
\\	面白い 
\\	この ほん は すごく おもしろかった			
\\	離婚	
\\	りこん			
\\	友人が離婚しました。	
\\	離婚 
\\	ゆうじん が りこん しました			
\\	小学生	
\\	しょうがくせい			
\\	うちの息子は来年、小学生になります。	
\\	小学生 
\\	うち の むすこ は らいねん しょうがくせい に なります			
\\	育つ	
\\	そだつ			
\\	野菜がよく育っている。	
\\	育つ 
\\	やさい が よく そだって いる			
\\	数える	
\\	かぞえる			
\\	いすの数を数えてください。	
\\	数える 
\\	いす の かず を かぞえて ください			
\\	アルバイト	
\\	兄はアルバイトをしています。	
\\	アルバイト 
\\	あに は あるばいと を して います			
\\	買い物	
\\	かいもの			
\\	母は買い物に出かけています。	
\\	買い物 
\\	はは は かいもの に でかけて います			
\\	石	
\\	いし			
\\	私は石につまづいた。	
\\	石 
\\	わたし は いし に つまづいた			
\\	薬	
\\	くすり			
\\	この薬を必ず飲んでください。	
\\	薬 
\\	この くすり を かならず のんで ください			
\\	毎年	
\\	まいねん			
\\	毎年給料が上がる。	
\\	毎年 
\\	まいねん きゅうりょう が あがる			
\\	回す	
\\	まわす			
\\	ねじは左に回すと外れます。	
\\	回す 
\\	ねじ は ひだり に まわす と はずれます			
\\	コピー	
\\	会議で書類のコピーを配った。	
\\	コピー 
\\	かいぎ で しょるい の こぴー を くばった			
\\	ぶつかる	
\\	車が電柱にぶつかった。	
\\	ぶつかる 
\\	くるま が でんちゅう に ぶつかった			
\\	運転手	
\\	うんてんしゅ			
\\	彼はタクシーの運転手です。	
\\	運転手 
\\	かれ は たくしー の うんてんしゅ です			
\\	教会	
\\	きょうかい			
\\	私たちは教会で結婚式をしました。	
\\	教会 
\\	わたしたち は きょうかい で けっこんしき を しました			
\\	フィルム	
\\	旅行のためにたくさんフィルムを買った。	
\\	フィルム 
\\	りょこう の ため に たくさん ふぃるむ を かった			
\\	未来	
\\	みらい			
\\	未来は誰にも分からない。	
\\	未来 
\\	みらい は だれ に も わからない			
\\	校長	
\\	こうちょう			
\\	あの人は高校の校長だ。	
\\	校長 
\\	あの ひと は こうこう の こうちょう だ			
\\	デパート	
\\	私はデパートで靴を買った。	
\\	デパート 
\\	わたし は でぱーと で くつ を かった			
\\	見送る	
\\	みおくる			
\\	彼が外国に行くのを見送りました。	
\\	見送る 
\\	かれ が がいこく に いく の を みおくりました			
\\	向く	
\\	むく			
\\	こっちを向いてください。	
\\	向く 
\\	こっち を むいて ください			
\\	揺れる	
\\	ゆれる			
\\	風で木が揺れています。	
\\	揺れる 
\\	かぜ で き が ゆれて います			
\\	髪	
\\	かみ			
\\	昨日、髪を切りました。	
\\	髪 
\\	きのう かみ を きりました			
\\	箱	
\\	はこ			
\\	この箱は重い。	
\\	箱 
\\	この はこ は おもい			
\\	思い出	
\\	おもいで			
\\	旅行で楽しい思い出ができました。	
\\	思い出 
\\	りょこう で たのしい おもいで が できました			
\\	ベッド	
\\	ベッドで赤ちゃんが眠っています。	
\\	赤ちゃん 
\\	ベッド 
\\	べっど で あかちゃん が ねむって います			
\\	彼はベッドで寝ています。	
\\	ベッド 
\\	かれ は べっど で ねて います			
\\	新幹線	
\\	しんかんせん			
\\	新幹線で京都に行きました。	
\\	新幹線 
\\	しんかんせん で きょうと に いきました			
\\	生む	
\\	うむ			
\\	うちのネコが子猫を生みました。	
\\	生む 
\\	うち の ねこ が こねこ を うみました			
\\	奥さん	
\\	おくさん			
\\	彼の奥さんはきれいな方です。	
\\	奥さん 
\\	かれ の おくさん は きれい な かた です			
\\	階段	
\\	かいだん			
\\	私たちは駅の階段をかけ上がった。	
\\	階段 
\\	わたしたち は えき の かいだん を かけあがった			
\\	招待	
\\	しょうたい			
\\	高校の時の先生を結婚式に招待した。	
\\	招待 
\\	こうこう の とき の せんせい を けっこんしき に しょうたい した			
\\	コート	
\\	寒かったのでコートを着た。	
\\	コート 
\\	さむかった の で こーと を きた			
\\	燃える	
\\	もえる			
\\	山が燃えています。	
\\	燃える 
\\	やま が もえて います			
\\	食堂	
\\	しょくどう			
\\	大学の食堂は安い。	
\\	食堂 
\\	だいがく の しょくどう は やすい			
\\	天井	
\\	てんじょう			
\\	この部屋は天井が高いですね。	
\\	天井 
\\	この へや は てんじょう が たかい です ね			
\\	課	
\\	か			
\\	今から課のミーティングがある。	
\\	課 
\\	いま から か の みーてぃんぐ が ある			
\\	靴	
\\	くつ			
\\	靴が汚れた。	
\\	靴 
\\	くつ が よごれた			
\\	予約	
\\	よやく			
\\	レストランを予約しました。	
\\	レストラン 
\\	予約 
\\	れすとらん を よやく しました			
\\	一昨年	
\\	おととし			
\\	一昨年初めて京都へ旅行しました。	
\\	一昨年 
\\	おととし はじめて きょうと へ りょこう しました			
\\	休日	
\\	きゅうじつ			
\\	休日は家でよくテレビを見ます。	
\\	休日 
\\	きゅうじつ は いえ で よく てれび を みます			
\\	注文	
\\	ちゅうもん			
\\	レストランでピザを注文しました。	
\\	注文 
\\	れすとらん で ぴざ を ちゅうもん しました			
\\	家賃	
\\	やちん			
\\	ここの家賃は12万円です。	
\\	家賃 
\\	ここ の やちん は じゅうにまんえん です			
\\	散歩	
\\	さんぽ			
\\	公園を散歩しようか。	
\\	散歩 
\\	こうえん を さんぽ しよう か			
\\	通勤	
\\	つうきん			
\\	私は地下鉄で通勤しています。	
\\	地下鉄 
\\	通勤 
\\	わたし は ちかてつ で つうきん して います			
\\	毎朝、通勤に30分かかります。	
\\	通勤 
\\	まいあさ つうきん に さんじゅっぷん かかります			
\\	ノート	
\\	ノートを開いてください。	
\\	ノート 
\\	のーと を ひらいて ください			
\\	辞める	
\\	やめる			
\\	彼は会社を辞めます。	
\\	辞める 
\\	かれ は かいしゃ を やめます			
\\	仕方	
\\	しかた			
\\	ファイルのダウンロードの仕方が分かりません。	
\\	仕方 
\\	ふぁいる の だうんろーど の しかた が わかりません			
\\	台所	
\\	だいどころ			
\\	お母さんは台所にいます。	
\\	台所 
\\	おかあさん は だいどころ に います			
\\	建つ	
\\	たつ			
\\	ここに来年、家が建ちます。	
\\	建つ 
\\	ここ に らいねん いえ が たちます			
\\	踏む	
\\	ふむ			
\\	運転手がブレーキを踏んだ。	
\\	運転手 
\\	踏む 
\\	うんてんしゅ が ぶれーき を ふんだ			
\\	配る	
\\	くばる			
\\	会議で書類を配った。	
\\	配る 
\\	かいぎ で しょるい を くばった			
\\	握る	
\\	にぎる			
\\	少女は母親の手を握った。	
\\	握る 
\\	しょうじょ は ははおや の て を にぎった			
\\	練習	
\\	れんしゅう			
\\	娘は今、バイオリンを練習しています。	
\\	練習 
\\	むすめ は いま ばいおりん を れんしゅう して います			
\\	強さ	
\\	つよさ			
\\	風の強さに驚きました。	
\\	強さ 
\\	かぜ の つよさ に おどろきました			
\\	久しぶり	
\\	ひさしぶり			
\\	明日、久しぶりに友達に会います。	
\\	久しぶり 
\\	あした ひさしぶり に ともだち に あいます			
\\	往復	
\\	おうふく			
\\	往復切符をください。	
\\	往復 
\\	おうふく きっぷ を ください			
\\	大勢	
\\	おおぜい			
\\	大勢で食事に出かけました。	
\\	大勢 
\\	おおぜい で しょくじ に でかけました			
\\	親しい	
\\	したしい			
\\	週末、親しい友達を家に呼んだ。	
\\	親しい 
\\	しゅうまつ したしい ともだち を いえ に よんだ			
\\	番号	
\\	ばんごう			
\\	この番号に電話してください。	
\\	番号 
\\	この ばんごう に でんわ して ください			
\\	ワイン	
\\	ワインを少し飲みました。	
\\	ワイン 
\\	わいん を すこし のみました			
\\	巻く	
\\	まく			
\\	彼は頭にタオルを巻いていた。	
\\	巻く 
\\	かれ は あたま に たおる を まいて いた			
\\	青い	
\\	あおい			
\\	ここの海はとても青い。	
\\	青い 
\\	ここ の うみ は とても あおい			
\\	おかしい	
\\	彼の様子がおかしい。	
\\	おかしい 
\\	かれ の ようす が おかしい			
\\	渡る	
\\	わたる			
\\	私たちは歩いて橋を渡った。	
\\	渡る 
\\	わたしたち は あるいて はし を わたった			
\\	牛	
\\	うし			
\\	牛が草を食べています。	
\\	牛 
\\	うし が くさ を たべて います			
\\	氷	
\\	こおり			
\\	グラスに氷を入れてください。	
\\	氷 
\\	ぐらす に こおり を いれて ください			
\\	挨拶	
\\	あいさつ			
\\	彼女は笑顔で挨拶した。	
\\	挨拶 
\\	かのじょ は えがお で あいさつ した			
\\	洗う	
\\	あらう			
\\	早く顔を洗いなさい。	
\\	洗う 
\\	はやく かお を あらいなさい			
\\	尋ねる	
\\	たずねる			
\\	彼女は道を尋ねた。	
\\	尋ねる 
\\	かのじょ は みち を たずねた			
\\	叩く	
\\	たたく			
\\	彼は子供のおしりを叩いた。	
\\	叩く 
\\	かれ は こども の おしり を たたいた			
\\	廊下	
\\	ろうか			
\\	廊下は走らないでください。	
\\	廊下 
\\	ろうか は はしらない で ください			
\\	郵便局	
\\	ゆうびんきょく			
\\	郵便局はどこですか。	
\\	郵便局 
\\	ゆうびんきょく は どこ です か			
\\	割る	
\\	わる			
\\	皿を落として割った。	
\\	割る 
\\	さら を おとして わった			
\\	進学	
\\	しんがく			
\\	彼女は来年、大学に進学する。	
\\	進学 
\\	かのじょ は らいねん だいがく に しんがく する			
\\	トイレ	
\\	トイレを掃除しましたか。	
\\	トイレ 
\\	といれ を そうじ しました か			
\\	咲く	
\\	さく			
\\	桜の花が咲きました。	
\\	咲く 
\\	さくら の はな が さきました			
\\	駐車場	
\\	ちゅうしゃじょう			
\\	車は駐車場に止めてください。	
\\	駐車場 
\\	くるま は ちゅうしゃじょう に とめて ください			
\\	泣く	
\\	なく			
\\	妹はすぐに泣く。	
\\	泣く 
\\	いもうと は すぐ に なく			
\\	屋根	
\\	やね			
\\	屋根にカラスが止まっています。	
\\	屋根 
\\	やね に からす が とまって います			
\\	案内	
\\	あんない			
\\	私が中をご案内します。	
\\	案内 
\\	わたし が なか を ごあんない します			
\\	キャンプ	
\\	友達とキャンプに行った。	
\\	キャンプ 
\\	ともだち と きゃんぷ に いった			
\\	荷物	
\\	にもつ			
\\	彼は大きな荷物を持ってきた。	
\\	荷物 
\\	かれ は おおき な にもつ を もって きた			
\\	迷惑	
\\	めいわく			
\\	人に迷惑をかけてはいけません。	
\\	迷惑 
\\	ひと に めいわく を かけて は いけません			
\\	先月	
\\	せんげつ			
\\	先月、友達の結婚式があった。	
\\	先月 
\\	せんげつ ともだち の けっこんしき が あった			
\\	美術館	
\\	びじゅつかん			
\\	昨日、美術館に行きました。	
\\	美術館 
\\	きのう びじゅつかん に いきました			
\\	破る	
\\	やぶる			
\\	彼は約束を破った。	
\\	破る 
\\	かれ は やくそく を やぶった			
\\	愛	
\\	あい			
\\	彼女は愛をこめて手紙を書いた。	
\\	愛 
\\	かのじょ は あい を こめて てがみ を かいた			
\\	黒	
\\	くろ			
\\	黒のボールペンはありますか。	
\\	黒 
\\	くろ の ぼーるぺん は あります か			
\\	断る	
\\	ことわる			
\\	私は彼のプロポーズを断った。	
\\	断る 
\\	わたし は かれ の ぷろぽーず を ことわった			
\\	品物	
\\	しなもの			
\\	その店は色々な品物を売っている。	
\\	品物 
\\	その みせ は いろいろ な しなもの を うって いる			
\\	プラスチック	
\\	このカップはプラスチックです。	
\\	プラスチック 
\\	この かっぷ は ぷらすちっく です			
\\	貸す	
\\	かす			
\\	私は彼に本を貸しています。	
\\	貸す 
\\	わたし は かれ に ほん を かして います			
\\	沈む	
\\	しずむ			
\\	光る	
\\	ひかる			
\\	波がきらきら光っていました。	
\\	光る 
\\	なみ が きらきら ひかって いました			
\\	右手	
\\	みぎて			
\\	私は右手で字を書きます。	
\\	右手 
\\	わたし は みぎて で じ を かきます			
\\	赤	
\\	あか			
\\	信号が赤に変わりました。	
\\	赤 
\\	しんごう が あか に かわりました			
\\	卵	
\\	たまご			
\\	ニワトリは卵を産みます。	
\\	卵 
\\	にわとり は たまご を うみます			
\\	疲れる	
\\	つかれる			
\\	男の子	
\\	おとこのこ			
\\	男の子たちがサッカーをしている。	
\\	男の子 
\\	おとこのこたち が さっかー を して いる			
\\	泳ぐ	
\\	およぐ			
\\	彼女はダイエットのために泳いでいる。	
\\	泳ぐ 
\\	かのじょ は だいえっと の ため に およいで いる			
\\	内側	
\\	うちがわ			
\\	白線の内側に下がってください。	
\\	内側 
\\	はくせん の うちがわ に さがって ください			
\\	習慣	
\\	しゅうかん			
\\	毎朝コーヒーを飲むのが習慣です。	
\\	習慣 
\\	まいあさ こーひー を のむ の が しゅうかん です			
\\	投げる	
\\	なげる			
\\	ボールをこっちに投げてください。	
\\	投げる 
\\	ぼーる を こっち に なげて ください			
\\	カラー	
\\	カラーコピーは一枚幾らですか。	
\\	カラー 
\\	からーこぴー は いちまい いくら です か			
\\	正月	
\\	しょうがつ			
\\	お正月にはたいてい、家族が集まる。	
\\	正月 
\\	おしょうがつ に は たいてい かぞく が あつまる			
\\	趣味	
\\	しゅみ			
\\	私の趣味は映画とテニスです。	
\\	趣味 
\\	わたし の しゅみ は えいが と てにす です			
\\	盗む	
\\	ぬすむ			
\\	誰かが私のカバンを盗みました。	
\\	盗む 
\\	だれか が わたし の かばん を ぬすみました			
\\	ピアノ	
\\	昔、ピアノを習っていました。	
\\	ピアノ 
\\	むかし ぴあの を ならって いました			
\\	池	
\\	いけ			
\\	池に鯉がいます。	
\\	池 
\\	いけ に こい が います			
\\	大学院	
\\	だいがくいん			
\\	彼は大学院に進みました。	
\\	大学院 
\\	かれ は だいがくいん に すすみました			
\\	何度	
\\	なんど			
\\	あの店には何度も行きました。	
\\	何度 
\\	あの みせ に は なんど も いきました			
\\	不幸	
\\	ふこう			
\\	彼女の家族に不幸があった。	
\\	不幸 
\\	かのじょ の かぞく に ふこう が あった			
\\	恐ろしい	
\\	おそろしい			
\\	昨夜恐ろしい夢を見た。	
\\	恐ろしい 
\\	ゆうべ おそろしい ゆめ を みた			
\\	退院	
\\	たいいん			
\\	母が今日退院します。	
\\	退院 
\\	はは が きょう たいいん します			
\\	付く	
\\	つく			
\\	靴に泥が付いています。	
\\	付く 
\\	くつ に どろ が ついて います			
\\	重さ	
\\	おもさ			
\\	この荷物の重さを計ってください。	
\\	重さ 
\\	この にもつ の おもさ を はかって ください			
\\	商業	
\\	しょうぎょう			
\\	この町では商業が盛んだ。	
\\	商業 
\\	この まち で は しょうぎょう が さかん だ			
\\	スキー	
\\	冬はよくスキーに行きます。	
\\	スキー 
\\	ふゆ は よく すきー に いきます			
\\	敷く	
\\	しく			
\\	生まれて初めて布団を敷いた。	
\\	敷く 
\\	うまれて はじめて ふとん を しいた			
\\	安心	
\\	あんしん			
\\	それを聞いて安心しました。	
\\	安心 
\\	それ を きいて あんしん しました			
\\	糸	
\\	いと			
\\	母は糸を針に通した。	
\\	糸 
\\	はは は いと を はり に とおした			
\\	なかなか	
\\	荷物がなかなか届きません。	
\\	なかなか 
\\	にもつ が なかなか とどきません			
\\	人形	
\\	にんぎょう			
\\	彼女は人形をたくさん持っています。	
\\	人形 
\\	かのじょ は にんぎょう を たくさん もって います			
\\	プール	
\\	私は夏休みにプールに行った。	
\\	プール 
\\	わたし は なつやすみ に ぷーる に いった			
\\	銀	
\\	ぎん			
\\	彼は銀メダルを取った。	
\\	銀 
\\	かれ は ぎんめだる を とった			
\\	匂い	
\\	におい			
\\	いい匂いがする。	
\\	匂い 
\\	いい におい が する			
\\	ホーム	
\\	もうすぐこのホームに電車が来ます。	
\\	ホーム 
\\	もうすぐ この ほーむ に でんしゃ が きます			
\\	交通事故	
\\	こうつうじこ			
\\	彼は交通事故を起こした。	
\\	交通事故 
\\	かれ は こうつうじこ を おこした			
\\	幸せ	
\\	しあわせ			
\\	良い友達がいて私は幸せだ。	
\\	幸せ 
\\	いい ともだち が いて わたし は しあわせ だ			
\\	吸う	
\\	すう			
\\	彼は大きく息を吸った。	
\\	吸う 
\\	かれ は おおきく いき を すった			
\\	浅い	
\\	あさい			
\\	この川は浅いです。	
\\	浅い 
\\	この かわ は あさい です			
\\	見学	
\\	けんがく			
\\	今日、工場の見学に行きました。	
\\	見学 
\\	きょう こうじょう の けんがく に いきました			
\\	濃い	
\\	こい			
\\	私は濃い味が好きだ。	
\\	濃い 
\\	わたし は こい あじ が すき だ			
\\	壊す	
\\	こわす			
\\	彼女が私のケータイを壊した。	
\\	壊す 
\\	かのじょ が わたし の けーたい を こわした			
\\	間に合う	
\\	まにあう			
\\	授業に間に合いました。	
\\	間に合う 
\\	じゅぎょう に まにあいました			
\\	届ける	
\\	とどける			
\\	これを彼に届けてください。	
\\	届ける 
\\	これ を かれ に とどけて ください			
\\	葉	
\\	は			
\\	エレベーター	
\\	エレベーターで下に降りましょう。	
\\	エレベーター 
\\	えれべーたー で した に おりましょう			
\\	メモ	
\\	メモを取ってください。	
\\	メモ 
\\	めも を とって ください			
\\	立派	
\\	りっぱ			
\\	彼は立派な人です。	
\\	立派 
\\	かれ は りっぱ な ひと です			
\\	旅館	
\\	りょかん			
\\	京都では旅館に泊まりました。	
\\	旅館 
\\	きょうと で は りょかん に とまりました			
\\	凄い	
\\	すごい			
\\	凄い雨になった。	
\\	凄い 
\\	すごい あめ に なった			
\\	パン	
\\	朝ご飯にはいつもパンを食べる。	
\\	パン 
\\	あさごはん に は いつも ぱん を たべる			
\\	びっくりする	
\\	大きな音にびっくりしました。	
\\	びっくりする 
\\	おおきな おと に びっくり しました			
\\	便利	
\\	べんり			
\\	インターネットはとても便利です。	
\\	便利 
\\	いんたーねっと は とても べんり です			
\\	砂糖	
\\	さとう			
\\	コーヒーに砂糖は入れますか。	
\\	砂糖 
\\	こーひー に さとう は いれます か			
\\	休み	
\\	やすみ			
\\	今日は休みです。	
\\	休み 
\\	きょう は やすみ です			
\\	録音	
\\	ろくおん			
\\	そのラジオ番組はもう録音しました。	
\\	録音 
\\	その らじお ばんぐみ は もう ろくおん しました			
\\	言い方	
\\	いいかた			
\\	そんな言い方をしてはいけません。	
\\	言い方 
\\	そんな いいかた を して は いけません			
\\	帰り	
\\	かえり			
\\	仕事の帰りにビールを飲んだ。	
\\	帰り 
\\	しごと の かえり に びーる を のんだ			
\\	週末	
\\	しゅうまつ			
\\	週末は家でゆっくりします。	
\\	週末 
\\	しゅうまつ は うち で ゆっくり します			
\\	ズボン	
\\	ズボンが汚れた。	
\\	ズボン 
\\	ずぼん が よごれた			
\\	足りる	
\\	たりる			
\\	お金が足りなくて買えなかった。	
\\	足りる 
\\	おかね が たりなくて かえなかった			
\\	拾う	
\\	ひろう			
\\	道で財布を拾った。	
\\	拾う 
\\	みち で さいふ を ひろった			
\\	祭り	
\\	まつり			
\\	彼女は祭りが大好きです。	
\\	祭り 
\\	かのじょ は まつり が だいすき です 。			
\\	博物館	
\\	はくぶつかん			
\\	昨日、車の博物館に行った。	
\\	博物館 
\\	きのう くるま の はくぶつかん に いった			
\\	青	
\\	あお			
\\	信号が青になった。	
\\	信号 
\\	青 
\\	しんごう が あお に なった			
\\	私の好きな色は青です。	
\\	青 
\\	わたし の すき な いろ は あお です			
\\	売り場	
\\	うりば			
\\	くつ売り場はどこですか。	
\\	売り場 
\\	くつうりば は どこ です か			
\\	おもちゃ	
\\	赤ちゃんが自動車のおもちゃで遊んでいる。	
\\	おもちゃ 
\\	あかちゃん が じどうしゃ の おもちゃ で あそんで いる			
\\	金持ち	
\\	かねもち			
\\	彼は金持ちです。	
\\	金持ち 
\\	かれ は かねもち です			
\\	頑張る	
\\	がんばる			
\\	明日の試験、頑張ってね。	
\\	頑張る 
\\	あした の しけん がんばって ね			
\\	グラム	
\\	ひき肉を200グラムください。	
\\	グラム 
\\	ひきにく を にひゃくぐらむ ください			
\\	コーヒー	
\\	私は毎朝コーヒーを飲みます。	
\\	コーヒー 
\\	わたし は まいあさ こーひー を のみます			
\\	桜	
\\	さくら			
\\	これは桜の葉です。	
\\	葉 
\\	桜 
\\	これ は さくら の は です			
\\	桜は三月か四月に咲きます。	
\\	桜 
\\	さくら は さんがつ か しがつ に さきます			
\\	刺す	
\\	さす			
\\	虫に腕を刺されました。	
\\	刺す 
\\	むし に うで を さされました			
\\	テント	
\\	みんなでテントを張りました。	
\\	テント 
\\	みんな で てんと を はりました			
\\	ボート	
\\	ボートが川に沈んだ。	
\\	沈む 
\\	ボート 
\\	ぼーと が かわ に しずんだ			
\\	池でボートに乗りました。	
\\	ボート 
\\	いけ で ぼーと に のりました			
\\	遭う	
\\	あう			
\\	彼は交通事故に遭った。	
\\	遭う 
\\	かれ は こうつう じこ に あった			
\\	ボール	
\\	彼はボールを投げた。	
\\	投げる 
\\	ボール 
\\	かれ は ぼーる を なげた			
\\	オートバイ	
\\	彼はオートバイに乗っている。	
\\	オートバイ 
\\	かれ は おーとばい に のって いる			
\\	外人	
\\	がいじん			
\\	この町には外人が少ない。	
\\	外人 
\\	この まち に は がいじん が すくない			
\\	猫	
\\	ねこ			
\\	私は猫が大好きです。	
\\	猫 
\\	わたし は ねこ が だいすき です			
\\	隠れる	
\\	かくれる			
\\	太陽が雲に隠れた。	
\\	隠れる 
\\	たいよう が くも に かくれた			
\\	親切	
\\	しんせつ			
\\	親切にしてくださってどうもありがとうございます。	
\\	親切 
\\	しんせつ に して くださって どうも ありがとう ございます			
\\	是非	
\\	ぜひ			
\\	是非、うちに来てください。	
\\	是非 
\\	ぜひ うち に きて ください			
\\	通学	
\\	つうがく			
\\	毎朝、通学に1時間かかる。	
\\	通学 
\\	まいあさ つうがく に いちじかん かかる			
\\	預ける	
\\	あずける			
\\	鍵を彼に預けた。	
\\	預ける 
\\	かぎ を かれ に あずけた			
\\	付ける	
\\	つける			
\\	おまけを付けました。	
\\	付ける 
\\	おまけ を つけました			
\\	習う	
\\	ならう			
\\	私はダンスを習っています。	
\\	習う 
\\	わたし は だんす を ならって います			
\\	組み立てる	
\\	くみたてる			
\\	日曜日に本棚を組み立てます。	
\\	組み立てる 
\\	にちようび に ほんだな を くみたてます			
\\	塗る	
\\	ぬる			
\\	壁にペンキを塗っています。	
\\	塗る 
\\	かべ に ぺんき を ぬって います			
\\	左手	
\\	ひだりて			
\\	彼女は左手で字を書く。	
\\	左手 
\\	かのじょ は ひだりて で じ を かく			
\\	ひどい	
\\	妹とひどい喧嘩をした。	
\\	ひどい 
\\	いもうと と ひどい けんか を した			
\\	別々	
\\	べつべつ			
\\	別々に払いましょう。	
\\	別々 
\\	べつべつ に はらいましょう			
\\	毎週	
\\	まいしゅう			
\\	私は毎週母に電話をします。	
\\	毎週 
\\	わたし は まいしゅう はは に でんわ を します			
\\	気温	
\\	きおん			
\\	今日の気温は26度です。	
\\	気温 
\\	きょう の きおん は にじゅうろくど です			
\\	蹴る	
\\	ける			
\\	ゴールキーパーがボールを蹴った。	
\\	蹴る 
\\	ゴールキーパー が ボール を けった。			
\\	真っ赤	
\\	まっか			
\\	彼の顔は真っ赤でした。	
\\	真っ赤 
\\	かれ の かお は まっか でした			
\\	外国語	
\\	がいこくご			
\\	外国語を習うのは難しい。	
\\	外国語 
\\	がいこくご を ならう の は むずかしい			
\\	着物	
\\	きもの			
\\	彼女は着物がよく似合います。	
\\	着物 
\\	かのじょ は きもの が よく にあいます			
\\	牛乳	
\\	ぎゅうにゅう			
\\	私は毎朝、牛乳を飲む。	
\\	牛乳 
\\	わたし は まいあさ ぎゅうにゅう を のむ			
\\	住所	
\\	じゅうしょ			
\\	この住所に行ってください。	
\\	住所 
\\	この じゅうしょ に いって ください			
\\	先週	
\\	せんしゅう			
\\	先週は海に行った。	
\\	先週 
\\	せんしゅう は うみ に いった			
\\	飛行場	
\\	ひこうじょう			
\\	バスが飛行場に着きました。	
\\	飛行場 
\\	ばす が ひこうじょう に つきました			
\\	救急車	
\\	きゅうきゅうしゃ			
\\	誰か救急車を呼んでください。	
\\	救急車 
\\	だれか きゅうきゅうしゃ を よんで ください			
\\	下りる	
\\	おりる			
\\	そこの階段を下りてください。	
\\	下りる 
\\	そこ の かいだん を おりて ください			
\\	果物	
\\	くだもの			
\\	デザートに果物を食べましょう。	
\\	果物 
\\	でざーと に くだもの を たべましょう			
\\	間違い	
\\	まちがい			
\\	この文には間違いがあります。	
\\	間違い 
\\	この ぶん に は まちがい が あります			
\\	あなた	
\\	これはあなたのカバンですか。	
\\	あなた 
\\	これ は あなた の かばん です か			
\\	映画館	
\\	えいがかん			
\\	彼と近くの映画館に行きました。	
\\	映画館 
\\	かれ と ちかく の えいがかん に いきました			
\\	踊る	
\\	おどる			
\\	彼女はクラブで踊るのが好きです。	
\\	踊る 
\\	かのじょ は くらぶ で おどる の が すき です			
\\	怖い	
\\	こわい			
\\	私は犬が怖いです。	
\\	怖い 
\\	わたし は いぬ が こわい です			
\\	鏡	
\\	かがみ			
\\	彼女は鏡を見て髪を直した。	
\\	鏡 
\\	かのじょ は かがみ を みて かみ を なおした			
\\	草	
\\	くさ			
\\	庭に草が生えている。	
\\	草 
\\	にわ に くさ が はえて いる			
\\	敬語	
\\	けいご			
\\	お客様には敬語を使いなさい。	
\\	敬語 
\\	おきゃくさま に は けいご を つかいなさい			
\\	欠席	
\\	けっせき			
\\	風邪のため今日は欠席します。	
\\	欠席 
\\	かぜ の ため きょう は けっせき します			
\\	スイッチ	
\\	彼はカーラジオのスイッチを入れた。	
\\	スイッチ 
\\	かれ は かーらじお の すいっち を いれた			
\\	掃除	
\\	そうじ			
\\	週末は部屋の掃除をしました。	
\\	掃除 
\\	しゅうまつ は へや の そうじ を しました			
\\	畳	
\\	たたみ			
\\	そのホテルには畳の部屋がある。	
\\	畳 
\\	その ほてる に は たたみ の へや が ある			
\\	速さ	
\\	はやさ			
\\	新幹線の速さはどれ位ですか。	
\\	速さ 
\\	しんかんせん の はやさ は どれ くらい です か			
\\	林	
\\	はやし			
\\	私たちは林の中に入っていった。	
\\	林 
\\	わたしたち は はやし の なか に はいって いった			
\\	プレゼント	
\\	誕生日にプレゼントをもらいました。	
\\	プレゼント 
\\	たんじょうび に ぷれぜんと を もらいました			
\\	弁当	
\\	べんとう			
\\	今日は弁当を持ってきました。	
\\	弁当 
\\	きょう は べんとう を もって きました			
\\	いつでも	
\\	いつでもうちに来てください。	
\\	いつでも 
\\	いつでも うち に きて ください			
\\	寂しい	
\\	さびしい			
\\	これは寂しい曲ですね。	
\\	寂しい 
\\	これ は さびしい きょく です ね			
\\	少しも	
\\	すこしも			
\\	あなたは少しも悪くない。	
\\	少しも 
\\	あなた は すこしも わるく ない			
\\	尊敬	
\\	そんけい			
\\	祖父は家族みんなに尊敬されています。	
\\	尊敬 
\\	そふ は かぞく みんな に そんけい されて います			
\\	テニス	
\\	彼らはよくテニスをしています。	
\\	テニス 
\\	かれら は よく てにす を して います			
\\	広さ	
\\	ひろさ			
\\	その家の広さはどれ位ですか。	
\\	広さ 
\\	その いえ の ひろさ は どれ くらい です か			
\\	読み	
\\	よみ			
\\	母に読み書きを習いました。	
\\	読み 
\\	はは に よみかき を ならいました			
\\	植える	
\\	うえる			
\\	庭にバラを植えました。	
\\	植える 
\\	にわ に ばら を うえました			
\\	我慢	
\\	がまん			
\\	彼のわがままには我慢できません。	
\\	我慢 
\\	かれ の わがまま に は がまん できません			
\\	国籍	
\\	こくせき			
\\	私は日本国籍です。	
\\	国籍 
\\	わたし は にほん こくせき です			
\\	こちら	
\\	受付はこちらです。	
\\	こちら 
\\	うけつけ は こちら です			
\\	恥ずかしい	
\\	はずかしい			
\\	とても恥ずかしかった。	
\\	恥ずかしい 
\\	とても はずかしかった			
\\	ボタン	
\\	ボタンを押してください。	
\\	ボタン 
\\	ぼたん を おして ください			
\\	洋服	
\\	ようふく			
\\	今日は洋服を買いに行きます。	
\\	洋服 
\\	きょう は ようふく を かい に いきます			
\\	油	
\\	あぶら			
\\	水と油は混ざらない。	
\\	油 
\\	みず と あぶら は まざらない			
\\	おかしい	
\\	彼の話はおかしかった。	
\\	おかしい 
\\	かれ の はなし は おかしかった			
\\	余る	
\\	あまる			
\\	夕食の料理がたくさん余った。	
\\	余る 
\\	ゆうしょく の りょうり が たくさん あまった			
\\	コート	
\\	新しいコートでテニスをしました。	
\\	コート 
\\	あたらしい こーと で てにす を しました			
\\	正直	
\\	しょうじき			
\\	彼女はとても正直だ。	
\\	正直 
\\	かのじょ は とても しょうじき だ			
\\	ドラマ	
\\	私はドラマを見るのが好きです。	
\\	ドラマ 
\\	わたし は どらま を みる の が すき です			
\\	ビザ	
\\	学生ビザを持っています。	
\\	ビザ 
\\	がくせい びざ を もって います			
\\	水道	
\\	すいどう			
\\	東京は水道の水が不味い。	
\\	水道 
\\	とうきょう は すいどう の みず が まずい			
\\	店員	
\\	てんいん			
\\	あの店員はとても親切です。	
\\	店員 
\\	あの てんいん は とても しんせつ です			
\\	電池	
\\	でんち			
\\	新しい電池を入れましょう。	
\\	電池 
\\	あたらしい でんち を いれましょう			
\\	ポケット	
\\	財布をポケットにしまった。	
\\	ポケット 
\\	さいふ を ぽけっと に しまった			
\\	握手	
\\	あくしゅ			
\\	彼らは握手をした。	
\\	握手 
\\	かれら は あくしゅ を した			
\\	固い	
\\	かたい			
\\	私の上司は頭が固い。	
\\	固い 
\\	わたし の じょうし は あたま が かたい			
\\	そろそろ	
\\	そろそろ始めましょうか。	
\\	そろそろ 
\\	そろそろ はじめましょう か			
\\	ぶどう	
\\	私はぶどうが好きです。	
\\	ぶどう 
\\	わたし は ぶどう が すき です			
\\	迷う	
\\	まよう			
\\	道に迷いました。	
\\	迷う 
\\	みち に まよいました			
\\	押さえる	
\\	おさえる			
\\	ドアを押さえてください。	
\\	押さえる 
\\	どあ を おさえて ください			
\\	見物	
\\	けんぶつ			
\\	皆で東京見物をした。	
\\	見物 
\\	みんな で とうきょう けんぶつ を した			
\\	知らせる	
\\	しらせる			
\\	皆に会議の日にちを知らせた。	
\\	知らせる 
\\	みんな に かいぎ の ひにち を しらせた			
\\	神社	
\\	じんじゃ			
\\	京都には神社がたくさんある。	
\\	神社 
\\	きょうと に は じんじゃ が たくさん ある			
\\	注射	
\\	ちゅうしゃ			
\\	彼は注射があまり好きではありません。	
\\	注射 
\\	かれ は ちゅうしゃ が あまり すき で は ありません			
\\	めったに	
\\	彼女はめったに怒りません。	
\\	めったに 
\\	かのじょ は めったに おこりません 。			
\\	恋人	
\\	こいびと			
\\	彼は恋人を失った。	
\\	恋人 
\\	かれ は こいびと を うしなった			
\\	財布	
\\	さいふ			
\\	会社に財布を忘れた。	
\\	財布 
\\	かいしゃ に さいふ を わすれた			
\\	作文	
\\	さくぶん			
\\	日本語で作文を書きました。	
\\	作文 
\\	にほんご で さくぶん を かきました			
\\	鳴る	
\\	なる			
\\	今朝、5時に電話が鳴った。	
\\	鳴る 
\\	けさ ごじ に でんわ が なった			
\\	真面目	
\\	まじめ			
\\	彼は真面目な人です。	
\\	真面目 
\\	かれ は まじめ な ひと です			
\\	申し込む	
\\	もうしこむ			
\\	彼女はそのセミナーに申し込んだ。	
\\	申し込む 
\\	かのじょ は その せみなー に もうしこんだ			
\\	冷蔵庫	
\\	れいぞうこ			
\\	飲み物は冷蔵庫にあります。	
\\	冷蔵庫 
\\	のみもの は れいぞうこ に あります			
\\	気に入る	
\\	きにいる			
\\	新しい靴がとても気に入りました。	
\\	気に入る 
\\	あたらしい くつ が とても きにいりました			
\\	クリスマス	
\\	クリスマスにはケーキを食べます。	
\\	クリスマス 
\\	くりすます に は けーき を たべます			
\\	ネクタイ	
\\	父の日にネクタイをプレゼントした。	
\\	ネクタイ 
\\	ちちのひ に ねくたい を ぷれぜんと した			
\\	厚さ	
\\	あつさ			
\\	私は板の厚さを測った。	
\\	厚さ 
\\	わたし は いた の あつさ を はかった			
\\	受け付ける	
\\	うけつける			
\\	郵便物は5時まで受け付けています。	
\\	受け付ける 
\\	ゆうびんぶつ は ごじ まで うけつけて います			
\\	誕生日	
\\	たんじょうび			
\\	彼女の誕生日は7月16日です。	
\\	誕生日 
\\	かのじょ の たんじょうび は しちがつ じゅうろくにち です			
\\	独身	
\\	どくしん			
\\	私の兄はまだ独身です。	
\\	独身 
\\	わたし の あに は まだ どくしん です			
\\	苦い	
\\	にがい			
\\	私は苦いコーヒーが好きです。	
\\	苦い 
\\	わたし は にがい こーひー が すき です			
\\	バイオリン	
\\	彼女はバイオリンを習っています。	
\\	バイオリン 
\\	かのじょ は ばいおりん を ならって います			
\\	パスポート	
\\	パスポートが見つからない。	
\\	パスポート 
\\	ぱすぽーと が みつからない			
\\	バッグ	
\\	私は黒いバッグを持っています。	
\\	バッグ 
\\	わたし は くろい ばっぐ を もって います			
\\	今まで	
\\	いままで			
\\	今までどこにいたのですか。	
\\	今まで 
\\	いままで どこ に いた の です か			
\\	切符	
\\	きっぷ			
\\	東京までの切符を買った。	
\\	切符 
\\	とうきょう まで の きっぷ を かった			
\\	ゴム	
\\	ゴムが伸びてしまった。	
\\	ゴム 
\\	ごむ が のびて しまった			
\\	ジェット機	
\\	ジェットき			
\\	ジェット機が飛んでいる。	
\\	ジェット機 
\\	じぇっとき が とんで いる			
\\	少々	
\\	しょうしょう			
\\	塩を少々入れてください。	
\\	少々 
\\	しお を しょうしょう いれて ください			
\\	随分	
\\	ずいぶん			
\\	随分早く終わったね。	
\\	随分 
\\	ずいぶん はやく おわった ね			
\\	引っ張る	
\\	ひっぱる			
\\	娘が私の手を引っ張った。	
\\	引っ張る 
\\	むすめ が わたし の て を ひっぱった			
\\	帽子	
\\	ぼうし			
\\	暑いので帽子を被りましょう。	
\\	帽子 
\\	あつい の で ぼうし を かぶりましょう			
\\	焼ける	
\\	やける			
\\	肉が焼けました。	
\\	焼ける 
\\	にく が やけました			
\\	鉛筆	
\\	えんぴつ			
\\	鉛筆を貸して下さい。	
\\	鉛筆 
\\	えんぴつ を かして ください			
\\	会話	
\\	かいわ			
\\	親子の会話は大切です。	
\\	会話 
\\	おやこ の かいわ は たいせつ です			
\\	ギター	
\\	彼女はギターが得意です。	
\\	ギター 
\\	かのじょ は ぎたー が とくい です			
\\	茶	
\\	ちゃ			
\\	暑さ	
\\	あつさ			
\\	今年の夏は暑さが厳しい。	
\\	暑さ 
\\	ことし の なつ は あつさ が きびしい			
\\	胃	
\\	い			
\\	昨日から胃が痛い。	
\\	胃 
\\	きのう から い が いたい			
\\	椅子	
\\	いす			
\\	そのお年寄りは椅子に座った。	
\\	椅子 
\\	その お としより は いす に すわった 。			
\\	お茶	
\\	おちゃ			
\\	私たちは毎日お茶を飲みます。	
\\	茶 
\\	お茶 
\\	わたしたち は まいにち おちゃ を のみます			
\\	お茶を入れましょうか。	
\\	お茶 
\\	おちゃ を いれましょう か			
\\	踊り	
\\	おどり			
\\	この踊りは易しいですよ。	
\\	踊り 
\\	この おどり は やさしい です よ			
\\	禁煙	
\\	きんえん			
\\	彼は今、禁煙しています。	
\\	禁煙 
\\	かれ は いま きんえん して います			
\\	次男	
\\	じなん			
\\	次男は今、海外にいます。	
\\	次男 
\\	じなん は いま かいがい に います			
\\	セーター	
\\	このセーターはウールだ。	
\\	セーター 
\\	この せーたー は うーる だ			
\\	引っ越す	
\\	ひっこす			
\\	来月、大阪に引っ越します。	
\\	引っ越す 
\\	らいげつ おおさか に ひっこします			
\\	丸い	
\\	まるい			
\\	地球は丸い。	
\\	丸い 
\\	ちきゅう は まるい。			
\\	写る	
\\	うつる			
\\	このカメラはよく写りますよ。	
\\	写る 
\\	この かめら は よく うつります よ			
\\	喫茶店	
\\	きっさてん			
\\	喫茶店でコーヒーを飲んだ。	
\\	喫茶店 
\\	きっさてん で こーひー を のんだ			
\\	全然	
\\	ぜんぜん			
\\	この本は全然面白くなかった。	
\\	全然 
\\	この ほん は ぜんぜん おもしろく なかった			
\\	やっぱり	
\\	やっぱり旅が大好きだ。	
\\	やっぱり 
\\	やっぱり たび が だいすき だ			
\\	週刊誌	
\\	しゅうかんし			
\\	その週刊誌はゴシップばかりだ。	
\\	週刊誌 
\\	その しゅうかんし は ごしっぷ ばかり だ			
\\	棚	
\\	たな			
\\	大きな棚はとても便利です。	
\\	棚 
\\	おおき な たな は とても べんり です			
\\	震える	
\\	ふるえる			
\\	彼は寒くて震えていました。	
\\	震える 
\\	かれ は さむくて ふるえて いました			
\\	磨く	
\\	みがく			
\\	靴を磨いてください。	
\\	磨く 
\\	くつ を みがいて ください			
\\	カーテン	
\\	朝、カーテンを開けた。	
\\	カーテン 
\\	あさ かーてん を あけた			
\\	火事	
\\	かじ			
\\	火事です。119番に電話してください。	
\\	火事 
\\	かじ です ひゃくじゅうきゅうばん に でんわ して ください			
\\	シャツ	
\\	このシャツはアイロンが必要だ。	
\\	シャツ 
\\	この しゃつ は あいろん が ひつよう だ			
\\	ナイフ	
\\	ナイフで手を切った。	
\\	ナイフ 
\\	ないふ で て を きった			
\\	皆さん	
\\	みなさん			
\\	皆さんにお話があります。	
\\	皆さん 
\\	みなさん に おはなし が あります			
\\	次女	
\\	じじょ			
\\	うちの次女は春から中学生です。	
\\	次女 
\\	うち の じじょ は はる から ちゅうがくせい です			
\\	貯金	
\\	ちょきん			
\\	貯金は十分にあります。	
\\	貯金 
\\	ちょきん は じゅうぶん に あります			
\\	バイク	
\\	兄はバイクが大好きです。	
\\	バイク 
\\	あに は ばいく が だいすき です			
\\	預かる	
\\	あずかる			
\\	荷物を預かってください。	
\\	預かる 
\\	にもつ を あずかって ください			
\\	いつか	
\\	僕はいつかアフリカに行きたい。	
\\	いつか 
\\	ぼく は いつか あふりか に いきたい			
\\	オレンジ	
\\	私はオレンジが好きです。	
\\	オレンジ 
\\	わたし は おれんじ が すき です			
\\	噛む	
\\	かむ			
\\	もっとよく噛みなさい。	
\\	噛む 
\\	もっと よく かみなさい			
\\	きつい	
\\	このシャツは少しきついです。	
\\	きつい 
\\	この しゃつ は すこし きつい です			
\\	削る	
\\	けずる			
\\	彼は家族のために仕事の時間を削った。	
\\	削る 
\\	かれ は かぞく の ため に しごと の じかん を けずった			
\\	越える	
\\	こえる			
\\	私たちは高い山を越えました。	
\\	越える 
\\	わたしたち は たかい やま を こえました			
\\	小包	
\\	こづつみ			
\\	フランスの友達から小包が届いた。	
\\	小包 
\\	ふらんす の ともだち から こづつみ が とどいた			
\\	新年	
\\	しんねん			
\\	新年明けましておめでとうございます。	
\\	新年 
\\	しんねん あけまして おめでとう ございます			
\\	配達	
\\	はいたつ			
\\	彼は新聞配達をしている。	
\\	配達 
\\	かれ は しんぶん はいたつ を して いる			
\\	動物園	
\\	どうぶつえん			
\\	昨日子供たちと動物園に行きました。	
\\	動物園 
\\	きのう こどもたち と どうぶつえん に いきました			
\\	封筒	
\\	ふうとう			
\\	その手紙を封筒に入れた。	
\\	封筒 
\\	その てがみ を ふうとう に いれた			
\\	危ない	
\\	あぶない			
\\	その道は車が多くて危ない。	
\\	危ない 
\\	その みち は くるま が おおくて あぶない			
\\	可愛い	
\\	かわいい			
\\	彼女の赤ちゃんは可愛いです。	
\\	可愛い 
\\	かのじょ の あかちゃん は かわいい です			
\\	さっき	
\\	さっきの話を続けましょう。	
\\	さっき 
\\	さっき の はなし を つづけましょう			
\\	宿題	
\\	しゅくだい			
\\	宿題、手伝ってくれない?	
\\	宿題 
\\	しゅくだい てつだって くれない			
\\	テキスト	
\\	テキストを読んでください。	
\\	テキスト 
\\	てきすと を よんで ください			
\\	ビニール	
\\	ゴミはそのビニール袋に入れてください。	
\\	ビニール 
\\	ごみ は その びにーるぶくろ に いれて ください			
\\	りんご	
\\	りんごを一つください。	
\\	りんご 
\\	りんご を ひとつ ください			
\\	アルバム	
\\	私は彼のアルバムを見た。	
\\	アルバム 
\\	わたし は かれ の あるばむ を みた			
\\	お祖父さん	
\\	おじいさん			
\\	私のお祖父さんは毎日散歩します。	
\\	お祖父さん 
\\	わたし の おじいさん は まいにち さんぽ します 。			
\\	水泳	
\\	すいえい			
\\	母は健康のために水泳をしている。	
\\	水泳 
\\	はは は けんこう の ため に すいえい を して いる			
\\	混む・込む	
\\	こむ			
\\	電車が込んでいる。	
\\	混む・込む 
\\	でんしゃ が こんで いる			
\\	柔道	
\\	じゅうどう			
\\	私は柔道を習っています。	
\\	柔道 
\\	わたし は じゅうどう を ならって います			
\\	スカート	
\\	彼女はあまりスカートははかない。	
\\	スカート 
\\	かのじょ は あまり すかーと は はかない			
\\	取れる	
\\	とれる			
\\	シャツのボタンが取れた。	
\\	ボタン 
\\	取れる 
\\	しゃつ の ぼたん が とれた			
\\	ペン	
\\	ペンを貸してください。	
\\	ペン 
\\	ぺん を かして ください			
\\	物	
\\	もの			
\\	彼女の家には物がたくさんあります。	
\\	物 
\\	かのじょ の いえ に は もの が たくさん あります			
\\	柔らかい	
\\	やわらかい			
\\	布団がとても柔らかい。	
\\	柔らかい 
\\	ふとん が とても やわらかい			
\\	郵便	
\\	ゆうびん			
\\	さっき郵便が届きました。	
\\	さっき 
\\	郵便 
\\	さっき ゆうびん が とどきました			
\\	割れる	
\\	われる			
\\	コップが落ちて割れた。	
\\	割れる 
\\	こっぷ が おちて われた			
\\	駅員	
\\	えきいん			
\\	駅員に聞きましょう。	
\\	駅員 
\\	えきいん に ききましょう			
\\	おしゃべり	
\\	妹はとてもおしゃべりです。	
\\	おしゃべり 
\\	いもうと は とても おしゃべり です			
\\	切手	
\\	きって			
\\	ここに切手を貼ってください。	
\\	切手 
\\	ここ に きって を はって ください			
\\	緑色	
\\	みどりいろ			
\\	彼は緑色のジャケットを着ています。	
\\	緑色 
\\	かれ は みどりいろ の じゃけっと を きて います			
\\	毎朝	
\\	まいあさ			
\\	私は毎朝ジョギングをします。	
\\	毎朝 
\\	わたし は まいあさ じょぎんぐ を します			
\\	曲げる	
\\	まげる			
\\	ひざを曲げてください。	
\\	曲げる 
\\	ひざ を まげて ください			
\\	幼稚園	
\\	ようちえん			
\\	娘は幼稚園に通っています。	
\\	幼稚園 
\\	むすめ は ようちえん に かよって います			
\\	きっと	
\\	明日はきっと雨が降ります。	
\\	きっと 
\\	あした は きっと あめ が ふります			
\\	理科	
\\	りか			
\\	私は理科が得意です。	
\\	理科 
\\	わたし は りか が とくい です			
\\	下着	
\\	したぎ			
\\	私は下着を手で洗う。	
\\	下着 
\\	わたし は したぎ を て で あらう			
\\	涼しい	
\\	すずしい			
\\	夕方は涼しくなりますよ。	
\\	涼しい 
\\	ゆうがた は すずしく なります よ			
\\	太る	
\\	ふとる			
\\	私の姉はすぐ太ります。	
\\	太る 
\\	わたし の あね は すぐ ふとります			
\\	冷房	
\\	れいぼう			
\\	冷房を入れてください。	
\\	冷房 
\\	れいぼう を いれて ください			
\\	別れる	
\\	わかれる			
\\	駅で友だちと別れました。	
\\	別れる 
\\	えき で ともだち と わかれました			
\\	空く	
\\	あく			
\\	後ろの席が空いています。	
\\	空く 
\\	うしろ の せき が あいて います			
\\	鍵	
\\	かぎ			
\\	出かける時は鍵を掛けてください。	
\\	鍵 
\\	でかける とき は かぎ を かけて ください			
\\	傘	
\\	かさ			
\\	電車に傘を忘れた。	
\\	傘 
\\	でんしゃ に かさ を わすれた			
\\	出掛ける	
\\	でかける			
\\	主人はもう出掛けました。	
\\	出掛ける 
\\	しゅじん は もう でかけました			
\\	毎晩	
\\	まいばん			
\\	姉は毎晩日記を書いています。	
\\	毎晩 
\\	あね は まいばん にっき を かいて います			
\\	真ん中	
\\	まんなか			
\\	道の真ん中に人が立っている。	
\\	真ん中 
\\	みち の まんなか に ひと が たって いる			
\\	指輪	
\\	ゆびわ			
\\	彼女に指輪をプレゼントしました。	
\\	指輪 
\\	かのじょ に ゆびわ を ぷれぜんと しました			
\\	幾つ	
\\	いくつ			
\\	娘さんは幾つになりましたか。	
\\	幾つ 
\\	むすめさん は いくつ に なりました か			
\\	腐る	
\\	くさる			
\\	リンゴが腐っている。	
\\	腐る 
\\	りんご が くさって いる			
\\	貧乏	
\\	びんぼう			
\\	彼は昔は貧乏だった。	
\\	貧乏 
\\	かれ は むかし は びんぼう だった			
\\	間違える	
\\	まちがえる			
\\	電話番号を間違えました。	
\\	間違える 
\\	でんわ ばんごう を まちがえました			
\\	メニュー	
\\	メニューをください。	
\\	メニュー 
\\	めにゅー を ください			
\\	アイスクリーム	
\\	弟はアイスクリームが大好きです。	
\\	アイスクリーム 
\\	おとうと は あいすくりーむ が だいすき です			
\\	おしゃれ	
\\	あの子はおしゃれだね。	
\\	おしゃれ 
\\	あの こ は おしゃれ だ ね			
\\	休憩	
\\	きゅうけい			
\\	少し休憩しましょう。	
\\	休憩 
\\	すこし きゅうけい しましょう			
\\	舌	
\\	した			
\\	舌を出してください。	
\\	舌 
\\	した を だして ください			
\\	ジュース	
\\	このジュースは甘すぎる。	
\\	ジュース 
\\	この じゅーす は あま すぎる			
\\	素晴らしい	
\\	すばらしい			
\\	素晴らしい景色ですね。	
\\	素晴らしい 
\\	すばらしい けしき です ね			
\\	ゼロ	
\\	今日の交通事故はゼロです。	
\\	ゼロ 
\\	きょう の こうつう じこ は ぜろ です			
\\	朝刊	
\\	ちょうかん			
\\	今日の朝刊に面白い記事があった。	
\\	朝刊 
\\	きょう の ちょうかん に おもしろい きじ が あった			
\\	丁寧	
\\	ていねい			
\\	彼女はいつも丁寧に仕事をします。	
\\	丁寧 
\\	かのじょ は いつも ていねい に しごと を します			
\\	苦手	
\\	にがて			
\\	私は料理が苦手です。	
\\	苦手 
\\	わたし は りょうり が にがて です			
\\	日	
\\	ひ			
\\	夏は日が長い。	
\\	日 
\\	なつ は ひ が ながい			
\\	引っ越し	
\\	ひっこし			
\\	去年、引っ越ししました。	
\\	引っ越し 
\\	きょねん ひっこし しました			
\\	歩道	
\\	ほどう			
\\	歩道を歩きましょう。	
\\	歩道 
\\	ほどう を あるきましょう			
\\	折れる	
\\	おれる			
\\	強風で木の枝が折れた。	
\\	折れる 
\\	きょうふう で き の えだ が おれた			
\\	見送り	
\\	みおくり			
\\	駅に友人の見送りに行きました。	
\\	見送り 
\\	えき に ゆうじん の みおくり に いきました			
\\	夕刊	
\\	ゆうかん			
\\	そのニュースは夕刊で見ました。	
\\	夕刊 
\\	その にゅーす は ゆうかん で みました			
\\	この間	
\\	このあいだ			
\\	この間彼女に会った。	
\\	この間 
\\	このあいだ かのじょ に あった			
\\	冷える	
\\	ひえる			
\\	クーラーで体が冷えた。	
\\	冷える 
\\	くーらー で からだ が ひえた			
\\	ピンク	
\\	娘がピンクのドレスを着ている。	
\\	ピンク 
\\	むすめ が ぴんく の どれす を きて いる			
\\	グラス	
\\	これはきれいなグラスですね。	
\\	グラス 
\\	これ は きれい な ぐらす です ね			
\\	毛	
\\	け			
\\	猫の毛がセーターに付いた。	
\\	毛 
\\	ねこ の け が せーたー に ついた			
\\	黒板	
\\	こくばん			
\\	答えを黒板に書いてください。	
\\	黒板 
\\	こたえ を こくばん に かいて ください			
\\	ダンス	
\\	彼女はダンスが大好きです。	
\\	ダンス 
\\	かのじょ は だんす が だいすき です			
\\	積もる	
\\	つもる			
\\	雪が積もっています。	
\\	積もる 
\\	ゆき が つもって います			
\\	半年	
\\	はんとし			
\\	日本に来て半年になります。	
\\	半年 
\\	にっぽん に きて はんとし に なります			
\\	出口	
\\	でぐち			
\\	出口はあそこです。	
\\	出口 
\\	でぐち は あそこ です			
\\	登る	
\\	のぼる			
\\	私たちは昨年、富士山に登りました。	
\\	登る 
\\	わたしたち は さくねん ふじさん に のぼりました			
\\	真っ白	
\\	まっしろ			
\\	外は雪で真っ白だった。	
\\	真っ白 
\\	そと は ゆき で まっしろ だった			
\\	おじ	
\\	おじは銀行に勤めています。	
\\	おじ 
\\	おじ は ぎんこう に つとめて います			
\\	下ろす	
\\	おろす			
\\	棚からその箱を下ろしてください。	
\\	下ろす 
\\	たな から その はこ を おろして ください			
\\	貸し出す	
\\	かしだす			
\\	その本は貸し出し中です。	
\\	貸し出す 
\\	その ほん は かしだしちゅう です			
\\	サッカー	
\\	彼はサッカーの選手です。	
\\	サッカー 
\\	かれ は さっかー の せんしゅ です			
\\	暖房	
\\	だんぼう			
\\	冬は暖房が必要です。	
\\	暖房 
\\	ふゆ は だんぼう が ひつよう です			
\\	手袋	
\\	てぶくろ			
\\	寒いので手袋をしました。	
\\	手袋 
\\	さむい の で てぶくろ を しました			
\\	留守	
\\	るす			
\\	父は今、留守です。	
\\	留守 
\\	ちち は いま るす です			
\\	嘘	
\\	うそ			
\\	嘘をついてはいけません。	
\\	嘘 
\\	うそ を ついて は いけません			
\\	遠慮	
\\	えんりょ			
\\	私は遠慮します。	
\\	遠慮 
\\	わたし は えんりょ します			
\\	折る	
\\	おる			
\\	祖父は足の骨を折りました。	
\\	折る 
\\	そふ は あし の ほね を おりました			
\\	今夜	
\\	こんや			
\\	今夜は月がとてもきれいです。	
\\	今夜 
\\	こんや は つき が とても きれい です			
\\	スープ	
\\	母がコーンスープを作っている。	
\\	スープ 
\\	はは が こーんすーぷ を つくって いる			
\\	バナナ	
\\	私は毎朝バナナを食べます。	
\\	バナナ 
\\	わたし は まいあさ ばなな を たべます			
\\	うるさい	
\\	この通りは車の音がうるさい。	
\\	うるさい 
\\	この とおり は くるま の おと が うるさい			
\\	可哀相	
\\	かわいそう			
\\	その可哀相な子供たちは食べるものがない。	
\\	可哀相 
\\	その かわいそう な こどもたち は たべる もの が ない			
\\	パトカー	
\\	あそこにパトカーがいる。	
\\	パトカー 
\\	あそこ に ぱとかー が いる			
\\	ハンバーガー	
\\	今日の昼ご飯はハンバーガーでした。	
\\	ハンバーガー 
\\	きょう の ひるごはん は はんばーがー でした			
\\	毛布	
\\	もうふ			
\\	この毛布は暖かい。	
\\	毛布 
\\	この もうふ は あたたかい			
\\	エスカレーター	
\\	3階までエスカレーターで行きましょう。	
\\	エスカレーター 
\\	さんかい まで えすかれーたー で いきましょう			
\\	お嬢さん	
\\	おじょうさん			
\\	お嬢さんはおいくつですか。	
\\	お嬢さん 
\\	おじょうさん は おいくつ です か			
\\	タオル	
\\	私はタオルで顔をふいた。	
\\	タオル 
\\	わたし は たおる で かお を ふいた			
\\	手帳	
\\	てちょう			
\\	新しい手帳を買いました。	
\\	手帳 
\\	あたらしい てちょう を かいました			
\\	乗り換える	
\\	のりかえる			
\\	次の駅で地下鉄に乗り換えます。	
\\	乗り換える 
\\	つぎ の えき で ちかてつ に のりかえます			
\\	売店	
\\	ばいてん			
\\	駅の売店で雑誌を買った。	
\\	売店 
\\	えき の ばいてん で ざっし を かった			
\\	パチンコ	
\\	彼は毎日パチンコをしています。	
\\	パチンコ 
\\	かれ は まいにち ぱちんこ を して います			
\\	美人	
\\	びじん			
\\	彼のお母さんは美人です。	
\\	美人 
\\	かれ の おかあさん は びじん です			
\\	本屋	
\\	ほんや			
\\	駅前に本屋があります。	
\\	本屋 
\\	えきまえ に ほんや が あります			
\\	みかん	
\\	みかんを1つください。	
\\	みかん 
\\	みかん を ひとつ ください			
\\	謝る	
\\	あやまる			
\\	彼は直ぐに謝りました。	
\\	謝る 
\\	かれ は すぐ に あやまりました			
\\	ケーキ	
\\	誕生日にケーキを食べました。	
\\	ケーキ 
\\	たんじょうび に けーき を たべました			
\\	コップ	
\\	私は毎朝コップ一杯の水を飲む。	
\\	コップ 
\\	わたし は まいあさ こっぷ いっぱい の みず を のむ			
\\	四季	
\\	しき			
\\	日本には四季がある。	
\\	四季 
\\	にほん に は しき が ある			
\\	田んぼ	
\\	たんぼ			
\\	この辺は田んぼがたくさんあります。	
\\	田んぼ 
\\	このへん は たんぼ が たくさん あります			
\\	天気予報	
\\	てんきよほう			
\\	明日の天気予報は雨です。	
\\	天気予報 
\\	あした の てんきよほう は あめ です			
\\	ナイロン	
\\	このジャケットはナイロンでできています。	
\\	ナイロン 
\\	この じゃけっと は ないろん で できて います			
\\	変	
\\	へん			
\\	変な音が聞こえます。	
\\	変 
\\	へん な おと が きこえます			
\\	来週	
\\	らいしゅう			
\\	続きは来週やりましょう。	
\\	来週 
\\	つづき は らいしゅう やりましょう			
\\	一生懸命	
\\	いっしょうけんめい			
\\	彼は毎日一生懸命働いている。	
\\	一生懸命 
\\	かれ は まいにち いっしょうけんめい はたらいて いる			
\\	乾く	
\\	かわく			
\\	夏は洗濯物がすぐ乾く。	
\\	乾く 
\\	なつ は せんたくもの が すぐ かわく			
\\	皿	
\\	さら			
\\	空いたお皿をお下げします。	
\\	皿 
\\	あいた おさら を おさげ します			
\\	間違う	
\\	まちがう			
\\	あなたは間違っている。	
\\	間違う 
\\	あなた は まちがって いる			
\\	事務室	
\\	じむしつ			
\\	事務室でコピーを取って来ます。	
\\	事務室 
\\	じむしつ で こぴー を とって きます			
\\	消防車	
\\	しょうぼうしゃ			
\\	消防車が4台も来た。	
\\	消防車 
\\	しょうぼうしゃ が よんだい も きた			
\\	スーツ	
\\	あのスーツはそんなに高くない。	
\\	スーツ 
\\	あの すーつ は そんなに たかく ない			
\\	チケット	
\\	この遊園地のチケットは3000円です。	
\\	チケット 
\\	この ゆうえんち の ちけっと は さんぜんえん です			
\\	チョコレート	
\\	妹はチョコレートが大好きです。	
\\	チョコレート 
\\	いもうと は ちょこれーと が だいすき です			
\\	泥棒	
\\	どろぼう			
\\	近所に泥棒が入った。	
\\	泥棒 
\\	きんじょ に どろぼう が はいった			
\\	脱ぐ	
\\	ぬぐ			
\\	靴を脱いでください。	
\\	脱ぐ 
\\	くつ を ぬいで ください			
\\	年賀状	
\\	ねんがじょう			
\\	昨日、年賀状を出しました。	
\\	年賀状 
\\	きのう ねんがじょう を だしました			
\\	乗り物	
\\	のりもの			
\\	自転車は便利な乗り物です。	
\\	乗り物 
\\	じてんしゃ は べんり な のりもの です			
\\	布団	
\\	ふとん			
\\	役に立つ	
\\	やくにたつ			
\\	私は人々の役に立ちたいと思っています。	
\\	役に立つ 
\\	わたし は ひとびと の やくにたちたい と おもって います			
\\	破れる	
\\	やぶれる			
\\	シャツが破れている。	
\\	破れる 
\\	しゃつ が やぶれて いる			
\\	用事	
\\	ようじ			
\\	父は用事で出掛けています。	
\\	用事 
\\	ちち は ようじ で でかけて います			
\\	写す	
\\	うつす			
\\	彼は友達の答えを写した。	
\\	写す 
\\	かれ は ともだち の こたえ を うつした			
\\	腕時計	
\\	うでどけい			
\\	私の腕時計は遅れている。	
\\	腕時計 
\\	わたし の うでどけい は おくれて いる			
\\	贈り物	
\\	おくりもの			
\\	すてきな贈り物をどうもありがとう。	
\\	贈り物 
\\	すてき な おくりもの を どうも ありがとう			
\\	チャンネル	
\\	テレビのチャンネルを変えてください。	
\\	チャンネル 
\\	てれび の ちゃんねる を かえて ください			
\\	つまらない	
\\	彼の話はつまらないですね。	
\\	つまらない 
\\	かれ の はなし は つまらない です ね			
\\	マッチ	
\\	マッチを持っていますか。	
\\	マッチ 
\\	まっち を もって います か			
\\	いとこ	
\\	従兄弟と私は同じ年です。	
\\	いとこ 
\\	いとこ と わたし は おなじ とし です			
\\	冷やす	
\\	ひやす			
\\	頭を冷やしなさい。	
\\	冷やす 
\\	あたま を ひやしなさい			
\\	片付ける	
\\	かたづける			
\\	早く部屋を片付けなさい。	
\\	片付ける 
\\	はやく へや を かたづけなさい			
\\	乾杯	
\\	かんぱい			
\\	お二人の未来に乾杯しましょう。	
\\	乾杯 
\\	おふたり の みらい に かんぱい しましょう			
\\	ストーブ	
\\	うちではまだストーブを使っています。	
\\	ストーブ 
\\	うち で は まだ すとーぶ を つかって います			
\\	停車	
\\	ていしゃ			
\\	この電車は東京まで停車致しません。	
\\	停車 
\\	この でんしゃ は とうきょう まで ていしゃ いたしません			
\\	鳴く	
\\	なく			
\\	どこかでネコが鳴いている。	
\\	鳴く 
\\	どこか で ねこ が ないて いる			
\\	ガソリンスタンド	
\\	この近くにガソリンスタンドはありますか。	
\\	ガソリンスタンド 
\\	この ちかく に がそりんすたんど は あります か			
\\	片道	
\\	かたみち			
\\	東京まで片道切符を買った。	
\\	片道 
\\	とうきょう まで かたみち きっぷ を かった			
\\	今朝	
\\	けさ			
\\	今朝テレビでそのニュースを知りました。	
\\	今朝 
\\	けさ てれび で その にゅーす を しりました			
\\	交番	
\\	こうばん			
\\	あそこの交番で道を聞きましょう。	
\\	交番 
\\	あそこ の こうばん で みち を ききましょう			
\\	大分	
\\	だいぶ			
\\	大分ピアノが上手くなりました。	
\\	大分 
\\	だいぶ ぴあの が うまく なりました			
\\	梅雨	
\\	つゆ			
\\	梅雨は6月頃です。	
\\	梅雨 
\\	つゆ は ろくがつごろ です			
\\	デート	
\\	デートで遊園地に行きました。	
\\	デート 
\\	でーと で ゆうえんち に いきました			
\\	ふすま	
\\	ふすまを閉めてください。	
\\	ふすま 
\\	ふすま を しめて ください			
\\	真っ直ぐ	
\\	まっすぐ			
\\	この道を真っ直ぐ行ってください。	
\\	真っ直ぐ 
\\	この みち を まっすぐ いって ください			
\\	レモン	
\\	上着	
\\	うわぎ			
\\	暑いので上着を脱ぎました。	
\\	脱ぐ 
\\	上着 
\\	あつい の で うわぎ を ぬぎました			
\\	偉い	
\\	えらい			
\\	彼は偉い学者です。	
\\	偉い 
\\	かれ は えらい がくしゃ です			
\\	書き方	
\\	かきかた			
\\	彼はその漢字の書き方が分からない。	
\\	書き方 
\\	かれ は その かんじ の かきかた が わからない			
\\	炊く	
\\	たく			
\\	母は毎朝ご飯を炊く。	
\\	炊く 
\\	はは は まいあさ ごはん を たく			
\\	チーズ	
\\	チーズを一切れ食べました。	
\\	チーズ 
\\	ちーず を ひときれ たべました			
\\	ドライブ	
\\	今日は群馬までドライブしました。	
\\	ドライブ 
\\	きょう は ぐんま まで どらいぶ しました			
\\	踏切	
\\	ふみきり			
\\	その踏切は長い。	
\\	踏切 
\\	その ふみきり は ながい			
\\	ラケット	
\\	テニスのラケットを買いました。	
\\	ラケット 
\\	てにす の らけっと を かいました			
\\	受付	
\\	うけつけ			
\\	受付は9時からです。	
\\	受付 
\\	うけつけ は くじ から です			
\\	怒る	
\\	おこる			
\\	彼女が嘘をついたので、彼は怒った。	
\\	嘘 
\\	怒る 
\\	かのじょ が うそ を ついた の で かれ は おこった			
\\	化粧	
\\	けしょう			
\\	彼女は化粧が上手い。	
\\	化粧 
\\	かのじょ は けしょう が うまい			
\\	寿司	
\\	すし			
\\	彼女は寿司を初めて食べました。	
\\	寿司 
\\	かのじょ は すし を はじめて たべました 。			
\\	スチュワーデス	
\\	彼女はスチュワーデスになった。	
\\	スチュワーデス 
\\	かのじょ は すちゅわーです に なった			
\\	都合	
\\	つごう			
\\	今日は都合が悪くて行けません。	
\\	都合 
\\	きょう は つごう が わるくて いけません			
\\	テープレコーダー	
\\	テープレコーダーで自分の声を録音しました。	
\\	テープレコーダー 
\\	てーぷれこーだー で じぶん の こえ を ろくおん しました			
\\	ひげ	
\\	父はひげをはやしています。	
\\	ひげ 
\\	ちち は ひげ を はやして います			
\\	本棚	
\\	ほんだな			
\\	これはとても大きな本棚ですね。	
\\	本棚 
\\	これ は とても おおき な ほんだな です ね			
\\	真っ暗	
\\	まっくら			
\\	外は真っ暗です。	
\\	真っ暗 
\\	そと は まっくら です			
\\	沸く	
\\	わく			
\\	お風呂が沸きました。	
\\	沸く 
\\	お ふろ が わきました 。			
\\	売り切れる	
\\	うりきれる			
\\	その本は直ぐ売り切れた。	
\\	売り切れる 
\\	その ほん は すぐ うりきれた			
\\	押し入れ	
\\	おしいれ			
\\	布団を押し入れにしまいました。	
\\	布団 
\\	押し入れ 
\\	ふとん を おしいれ に しまいました			
\\	革	
\\	かわ			
\\	ぐっすり	
\\	赤ちゃんがぐっすり寝ている。	
\\	ぐっすり 
\\	あかちゃん が ぐっすり ねて いる			
\\	紅茶	
\\	こうちゃ			
\\	紅茶にレモンを入れて飲んだ。	
\\	レモン 
\\	紅茶 
\\	こうちゃ に れもん を いれて のんだ			
\\	温かい紅茶が飲みたい。	
\\	紅茶 
\\	あたたかい こうちゃ が のみたい			
\\	邪魔	
\\	じゃま			
\\	邪魔です、どいてください。	
\\	邪魔 
\\	じゃま です どいて ください			
\\	ソース	
\\	ソースはどれですか。	
\\	ソース 
\\	そーす は どれ です か			
\\	タイプライター	
\\	母は古いタイプライターを持っています。	
\\	タイプライター 
\\	はは は ふるい たいぷらいたー を もって います			
\\	足す	
\\	たす			
\\	母は味噌汁に水を足した。	
\\	足す 
\\	はは は みそしる に みず を たした			
\\	トマト	
\\	私はトマトが大好きです。	
\\	トマト 
\\	わたし は とまと が だいすき です			
\\	履く	
\\	はく			
\\	彼女はブーツを履いています。	
\\	履く 
\\	かのじょ は ぶーつ を はいて います			
\\	バター	
\\	パンにバターをぬって食べました。	
\\	バター 
\\	ぱん に ばたー を ぬって たべました			
\\	発車	
\\	はっしゃ			
\\	バスが発車します。	
\\	発車 
\\	ばす が はっしゃ します			
\\	バレーボール	
\\	妹はバレーボールが得意です。	
\\	バレーボール 
\\	いもうと は ばれーぼーる が とくい です			
\\	おかず	
\\	晩ご飯のおかずは何?	
\\	おかず 
\\	ばんごはん の おかず は なに			
\\	カレンダー	
\\	カレンダーに予定を書いた。	
\\	カレンダー 
\\	かれんだー に よてい を かいた			
\\	再来年	
\\	さらいねん			
\\	次のオリンピックは再来年です。	
\\	再来年 
\\	つぎ の おりんぴっく は さらいねん です			
\\	冗談	
\\	じょうだん			
\\	冗談は止めてください。	
\\	冗談 
\\	じょうだん は やめて ください			
\\	近付く	
\\	ちかづく			
\\	女性が近付いてきました。	
\\	近付く 
\\	じょせい が ちかづいて きました			
\\	電話帳	
\\	でんわちょう			
\\	電話帳を見せてください。	
\\	電話帳 
\\	でんわちょう を みせて ください			
\\	ハンドバッグ	
\\	ハンドバッグを忘れました。	
\\	ハンドバッグ 
\\	はんどばっぐ を わすれました			
\\	ベル	
\\	玄関のベルが鳴った。	
\\	ベル 
\\	げんかん の べる が なった			
\\	干す	
\\	ほす			
\\	母が布団を干している。	
\\	布団 
\\	干す 
\\	はは が ふとん を ほして いる			
\\	母は洗濯物を干しています。	
\\	干す 
\\	はは は せんたくもの を ほして います			
\\	インク	
\\	プリンターのインクを買った。	
\\	インク 
\\	ぷりんたー の いんく を かった			
\\	蚊	
\\	か			
\\	蚊に足を刺された。	
\\	蚊 
\\	か に あし を さされた			
\\	改札口	
\\	かいさつぐち			
\\	改札口で会いましょう。	
\\	改札口 
\\	かいさつぐち で あいましょう			
\\	悲しむ	
\\	かなしむ			
\\	父は友だちの死を悲しんでいます。	
\\	悲しむ 
\\	ちち は ともだち の し を かなしんで います			
\\	クーラー	
\\	暑いのでクーラーをつけました。	
\\	クーラー 
\\	あつい の で くーらー を つけました			
\\	靴下	
\\	くつした			
\\	この靴下は3足で1000円です。	
\\	靴下 
\\	この くつした は さんそく で せんえん です			
\\	失礼	
\\	しつれい			
\\	ではそろそろ失礼します。	
\\	失礼 
\\	では そろそろ しつれい します			
\\	西暦	
\\	せいれき			
\\	西暦1964年に東京でオリンピックがあった。	
\\	西暦 
\\	せいれき せんきゅうひゃくろくじゅうよねん に とうきょう で おりんぴっく が あった			
\\	洗濯機	
\\	せんたくき			
\\	新しい洗濯機を買いました。	
\\	洗濯機 
\\	あたらしい せんたっき を かいました			
\\	豆腐	
\\	とうふ			
\\	私は毎日豆腐を食べます。	
\\	豆腐 
\\	わたし は まいにち とうふ を たべます			
\\	昼寝	
\\	ひるね			
\\	私の子供は毎日昼寝をします。	
\\	昼寝 
\\	わたし の こども は まいにち ひるね を します			
\\	不便	
\\	ふべん			
\\	私の家は駅から遠くて不便です。	
\\	不便 
\\	わたし の いえ は えき から とおく て ふべん です			
\\	ミルク	
\\	赤ちゃんにミルクをあげた。	
\\	ミルク 
\\	あかちゃん に みるく を あげた			
\\	向こう側	
\\	むこうがわ			
\\	私の家は川の向こう側にあります。	
\\	向こう側 
\\	わたし の いえ は かわ の むこうがわ に あります			
\\	億	
\\	おく			
\\	世界人口は65億人だ。	
\\	億 
\\	せかい じんこう は ろくじゅうごおくにん だ			
\\	お礼	
\\	おれい			
\\	彼女にお礼の手紙を書きました。	
\\	お礼 
\\	かのじょ に おれい の てがみ を かきました			
\\	着せる	
\\	きせる			
\\	娘に可愛いドレスを着せた。	
\\	着せる 
\\	むすめ に かわいい どれす を きせた			
\\	サラダ	
\\	サラダをたくさん食べました。	
\\	サラダ 
\\	さらだ を たくさん たべました			
\\	八百屋	
\\	やおや			
\\	八百屋でみかんを買いました。	
\\	みかん 
\\	八百屋 
\\	やおや で みかん を かいました			
\\	おじさん	
\\	昨日おじさんに会いました。	
\\	おじさん 
\\	きのう おじさん に あいました			
\\	代える	
\\	かえる			
\\	社長に代えて部長を出席させます。	
\\	代える 
\\	しゃちょう に かえて ぶちょう を しゅっせき させます			
\\	急行	
\\	きゅうこう			
\\	ちょうど急行電車が来た。	
\\	急行 
\\	ちょうど きゅうこう でんしゃ が きた			
\\	交差点	
\\	こうさてん			
\\	次の交差点を左に曲がってください。	
\\	交差点 
\\	つぎ の こうさてん を ひだり に まがって ください			
\\	高等学校	
\\	こうとうがっこう			
\\	弟が高等学校を卒業しました。	
\\	高等学校 
\\	おとうと が こうとうがっこう を そつぎょう しました			
\\	覚める	
\\	さめる			
\\	今朝は6時に目が覚めた。	
\\	覚める 
\\	けさ は ろくじ に め が さめた			
\\	昼休み	
\\	ひるやすみ			
\\	昼休みに公園に行った。	
\\	昼休み 
\\	ひるやすみ に こうえん に いった			
\\	ベルト	
\\	革のベルトを買いました。	
\\	革 
\\	ベルト 
\\	かわ の べると を かいました			
\\	ベルトがきつくなりました。	
\\	ベルト 
\\	べると が きつく なりました			
\\	真っ黒	
\\	まっくろ			
\\	インクで手が真っ黒になった	
\\	インク 
\\	真っ黒 
\\	いんく で て が まっくろ に なった			
\\	皆様	
\\	みなさま			
\\	皆様、こんにちは。	
\\	皆様 
\\	みなさま こんにちは			
\\	ラーメン	
\\	夕食にラーメンを食べました。	
\\	ラーメン 
\\	ゆうしょく に らーめん を たべました			
\\	ライター	
\\	店にライターを忘れました。	
\\	ライター 
\\	みせ に らいたー を わすれました			
\\	ワイシャツ	
\\	彼は白いワイシャツを着ている。	
\\	ワイシャツ 
\\	かれ は しろい わいしゃつ を きて いる			
\\	転ぶ	
\\	ころぶ			
\\	お祖母ちゃんがお風呂場で転んだ。	
\\	転ぶ 
\\	おばあちゃん が おふろば で ころんだ			
\\	冷める	
\\	さめる			
\\	彼への気持ちが冷めた。	
\\	冷める 
\\	かれ へ の きもち が さめた			
\\	シャワー	
\\	朝、急いでシャワーを浴びました。	
\\	シャワー 
\\	あさ いそいで しゃわー を あびました			
\\	扇風機	
\\	せんぷうき			
\\	暑いから扇風機をつけよう。	
\\	扇風機 
\\	あつい から せんぷうき を つけよう			
\\	遅刻	
\\	ちこく			
\\	遅刻しないでください。	
\\	遅刻 
\\	ちこく しない で ください			
\\	釣る	
\\	つる			
\\	昨日大きな魚を釣りました。	
\\	釣る 
\\	きのう おおき な さかな を つりました			
\\	うどん	
\\	私はお昼にうどんを食べました。	
\\	うどん 
\\	わたし は おひる に うどん を たべました			
\\	追い越す	
\\	おいこす			
\\	大きなトラックが私たちを追い越した。	
\\	追い越す 
\\	おおき な とらっく が わたしたち を おいこした			
\\	仮名	
\\	かな			
\\	お名前に仮名を振ってください。	
\\	仮名 
\\	おなまえ に かな を ふって ください			
\\	家内	
\\	かない			
\\	家内は九州出身です。	
\\	家内 
\\	かない は きゅうしゅう しゅっしん です			
\\	記念日	
\\	きねんび			
\\	今日は両親の結婚記念日です。	
\\	記念日 
\\	きょう は りょうしん の けっこん きねんび です			
\\	下り	
\\	くだり			
\\	もうすぐ下りの電車が発車します。	
\\	下り 
\\	もうすぐ くだり の でんしゃ が はっしゃ します			
\\	掲示板	
\\	けいじばん			
\\	掲示板のお知らせを見ましたか。	
\\	掲示板 
\\	けいじばん の おしらせ を みました か			
\\	凍る	
\\	こおる			
\\	寒い朝は道路が凍ります。	
\\	凍る 
\\	さむい あさ は どうろ が こおります			
\\	ごみ	
\\	今日はごみの日だ。	
\\	ごみ 
\\	きょう は ごみ の ひ だ			
\\	消防署	
\\	しょうぼうしょ			
\\	この町には消防署が1つしかありません。	
\\	消防署 
\\	この まち に は しょうぼうしょ が ひとつ しか ありません			
\\	スーパーマーケット	
\\	スーパーマーケットで人参を買った。	
\\	スーパーマーケット 
\\	すーぱーまーけっと で にんじん を かった			
\\	ハンカチ	
\\	ハンカチで手をふきました。	
\\	ハンカチ 
\\	はんかち で て を ふきました			
\\	ビデオテープ	
\\	私はビデオテープを30本持っています。	
\\	ビデオテープ 
\\	わたし は びでおてーぷ を さんじゅっぽん もって います			
\\	ぶらぶら	
\\	朝、近所をぶらぶらした。	
\\	ぶらぶら 
\\	あさ きんじょ を ぶらぶら した			
\\	アナウンサー	
\\	私はアナウンサーになりたい。	
\\	アナウンサー 
\\	わたし は あなうんさー に なりたい			
\\	辞書	
\\	じしょ			
\\	彼女は辞書をよく使います。	
\\	辞書 
\\	かのじょ は じしょ を よく つかいます			
\\	自炊	
\\	じすい			
\\	私は毎日、自炊している。	
\\	自炊 
\\	わたし は まいにち じすい して いる			
\\	止める	
\\	とめる			
\\	車を止めて。	
\\	止める 
\\	くるま を とめて			
\\	スプーン	
\\	カレーライスはスプーンで食べます。	
\\	スプーン 
\\	かれーらいす は すぷーん で たべます			
\\	年下	
\\	としした			
\\	彼は奥さんより年下です。	
\\	年下 
\\	かれ は おくさん より としした です			
\\	治す	
\\	なおす			
\\	早く風邪を治してください。	
\\	治す 
\\	はやく かぜ を なおして ください			
\\	箸	
\\	はし			
\\	箸を上手に使えるよ。	
\\	箸 
\\	はし を じょうず に つかえる よ			
\\	花見	
\\	はなみ			
\\	友達と花見をしました。	
\\	花見 
\\	ともだち と はなみ を しました			
\\	引き出し	
\\	ひきだし			
\\	財布は引き出しの中にあります。	
\\	引き出し 
\\	さいふ は ひきだし の なか に あります			
\\	ブラウス	
\\	彼女は白いブラウスを着ている。	
\\	ブラウス 
\\	かのじょ は しろい ぶらうす を きて いる			
\\	訳	
\\	やく			
\\	その文の訳を読みました。	
\\	訳 
\\	その ぶん の やく を よみました			
\\	リボン	
\\	プレゼントにリボンを付けた。	
\\	リボン 
\\	ぷれぜんと に りぼん を つけた			
\\	いかが	
\\	お加減はいかがですか。	
\\	いかが 
\\	おかげん は いかが です か			
\\	居る	
\\	いる			
\\	今日は一日中家に居ました。	
\\	居る 
\\	きょう は いちにちじゅう うち に いました			
\\	交通費	
\\	こうつうひ			
\\	会社までの交通費は一ヶ月8,000円です。	
\\	交通費 
\\	かいしゃ まで の こうつうひ は いっかげつ はっせんえん です			
\\	ステレオ	
\\	彼はステレオで音楽を聞いた。	
\\	ステレオ 
\\	かれ は すてれお で おんがく を きいた			
\\	滑る	
\\	すべる			
\\	彼は雪の上を滑った。	
\\	滑る 
\\	かれ は ゆき の うえ を すべった			
\\	早口	
\\	はやくち			
\\	彼女は早口だ。	
\\	早口 
\\	かのじょ は はやくち だ			
\\	フォーク	
\\	フォークをもらえますか。	
\\	フォーク 
\\	ふぉーく を もらえます か			
\\	裏返す	
\\	うらがえす			
\\	彼は紙を裏返しました。	
\\	裏返す 
\\	かれ は かみ を うらがえしました			
\\	おば	
\\	おばは大阪に住んでいます。	
\\	おば 
\\	おば は おおさか に すんで います			
\\	角	
\\	かど			
\\	次の角で左に曲がってください。	
\\	角 
\\	つぎ の かど で ひだり に まがって ください			
\\	きらきら	
\\	星がきらきら光っている。	
\\	きらきら 
\\	ほし が きらきら ひかって いる			
\\	三角	
\\	さんかく			
\\	紙を三角に切りました。	
\\	三角 
\\	かみ を さんかく に きりました			
\\	どうぞ	
\\	こちらへどうぞ。	
\\	どうぞ 
\\	こちら へ どうぞ			
\\	図書室	
\\	としょしつ			
\\	図書室で勉強した。	
\\	図書室 
\\	としょしつ で べんきょう した			
\\	入場券	
\\	にゅうじょうけん			
\\	入場券は3000円です。	
\\	入場券 
\\	にゅうじょうけん は さんぜんえん です			
\\	ネックレス	
\\	妻にネックレスをプレゼントしました。	
\\	ネックレス 
\\	つま に ねっくれす を ぷれぜんと しました			
\\	パジャマ	
\\	このパジャマを着て。	
\\	パジャマ 
\\	この ぱじゃま を きて			
\\	ボールペン	
\\	ボールペンで名前を書いてください。	
\\	ボールペン 
\\	ぼーるぺん で なまえ を かいて ください			
\\	万年筆	
\\	まんねんひつ			
\\	父に万年筆をプレゼントしました。	
\\	万年筆 
\\	ちち に まんねんひつ を ぷれぜんと しました			
\\	柔らか	
\\	やわらか			
\\	彼の声は柔らかだ。	
\\	柔らか 
\\	かれ の こえ は やわらか だ			
\\	覚ます	
\\	さます			
\\	子供が目を覚ました。	
\\	覚ます 
\\	こども が め を さました			
\\	すいか	
\\	夏はすいかが美味しい。	
\\	すいか 
\\	なつ は すいか が おいしい			
\\	そちら	
\\	夕方そちらに着きます。	
\\	そちら 
\\	ゆうがた そちら に つきます			
\\	そば	
\\	お昼にそばを食べた。	
\\	そば 
\\	おひる に そば を たべた			
\\	大抵	
\\	たいてい			
\\	朝食は大抵7時頃に食べます。	
\\	大抵 
\\	ちょうしょく は たいてい しちじごろ に たべます			
\\	特急	
\\	とっきゅう			
\\	東京まで特急で3時間かかります。	
\\	特急 
\\	とうきょう まで とっきゅう で さんじかん かかります			
\\	どっち	
\\	どっちの色が好きですか。	
\\	どっち 
\\	どっち の いろ が すき です か 。			
\\	糊	
\\	のり			
\\	糊で2枚の紙を貼り合わせた。	
\\	糊 
\\	のり で にまい の かみ を はり あわせた			
\\	バスケットボール	
\\	姉はバスケットボールの選手です。	
\\	バスケットボール 
\\	あね は ばすけっとぼーる の せんしゅ です			
\\	晴れ	
\\	はれ			
\\	明日の天気は晴れです。	
\\	晴れ 
\\	あす の てんき は はれ です			
\\	欲しがる	
\\	ほしがる			
\\	子供がジュースを欲しがっています。	
\\	欲しがる 
\\	こども が じゅーす を ほしがって います			
\\	味噌汁	
\\	みそしる			
\\	私は毎日味噌汁を飲みます。	
\\	味噌汁 
\\	わたし は まいにち みそしる を のみます 。			
\\	読み方	
\\	よみかた			
\\	この漢字の読み方を教えてください。	
\\	読み方 
\\	この かんじ の よみかた を おしえて ください			
\\	嫌	
\\	いや			
\\	私は待つのが嫌だ。	
\\	嫌 
\\	わたし は まつ の が いや だ			
\\	入り口	
\\	いりぐち			
\\	入り口は向こうです。	
\\	入り口 
\\	いりぐち は むこう です			
\\	色々	
\\	いろいろ			
\\	彼は色々なことを知っている。	
\\	色々 
\\	かれ は いろいろ な こと を しって いる			
\\	降ろす	
\\	おろす			
\\	彼は車から荷物を降ろした。	
\\	降ろす 
\\	かれ は くるま から にもつ を おろした			
\\	掛かる	
\\	かかる			
\\	壁に大きな時計が掛かっています。	
\\	掛かる 
\\	かべ に おおき な とけい が かかって います			
\\	サンドイッチ	
\\	今日のお昼はサンドイッチです。	
\\	サンドイッチ 
\\	きょう の おひる は さんどいっち です			
\\	四角	
\\	しかく			
\\	紙を四角に切ってください。	
\\	四角 
\\	かみ を しかく に きって ください			
\\	四角い	
\\	しかくい			
\\	こっちの四角いテーブルを買おうよ。	
\\	四角い 
\\	こっち の しかくい てーぶる を かおうよ			
\\	突き当たる	
\\	つきあたる			
\\	突き当たったら右に曲がってください。	
\\	突き当たる 
\\	つきあたったら みぎ に まがって ください			
\\	定期券	
\\	ていきけん			
\\	定期券は1万2千円でした。	
\\	定期券 
\\	ていきけん は いちまんにせんえん でした			
\\	にこにこ	
\\	彼女はいつもにこにこしています。	
\\	にこにこ 
\\	かのじょ は いつも にこにこ して います			
\\	馬鹿	
\\	ばか			
\\	私は馬鹿だった。	
\\	馬鹿 
\\	わたし は ばか だった			
\\	名字	
\\	みょうじ			
\\	あなたの名字は何ですか。	
\\	名字 
\\	あなた の みょうじ は なん です か			
\\	下る	
\\	くだる			
\\	小さな船が川を下っています。	
\\	下る 
\\	ちいさ な ふね が かわ を くだって います			
\\	ウール	
\\	このウールのセーターは暖かい。	
\\	ウール 
\\	この うーる の せーたー は あたたかい			
\\	コーラ	
\\	私の兄はコーラが好きだ。	
\\	コーラ 
\\	わたし の あに は こーら が すき だ			
\\	トイレットペーパー	
\\	トイレットペーパーを交換しました。	
\\	トイレットペーパー 
\\	といれっとぺーぱー を こうかん しました			
\\	どなた	
\\	失礼ですが、どなたですか。	
\\	どなた 
\\	しつれい です が 、 どなた です か 。			
\\	乗り換え	
\\	のりかえ			
\\	次の駅で乗り換えです。	
\\	乗り換え 
\\	つぎ の えき で のりかえ です			
\\	ハイキング	
\\	昨日は友人とハイキングに行きました。	
\\	ハイキング 
\\	きのう は ゆうじん と はいきんぐ に いきました			
\\	灰皿	
\\	はいざら			
\\	灰皿をください。	
\\	灰皿 
\\	はいざら を ください			
\\	バス停	
\\	ばすてい			
\\	次のバス停で降ります。	
\\	バス停 
\\	つぎ の ばすてい で おります			
\\	ハム	
\\	ハムサンドをください。	
\\	ハム 
\\	はむ さんど を ください			
\\	ボーイフレンド	
\\	私のボーイフレンドはハンサムな方でした。	
\\	ボーイフレンド 
\\	わたし の ぼーいふれんど は はんさむな かた でした			
\\	床	
\\	ゆか			
\\	床がぬれている。	
\\	床 
\\	ゆか が ぬれて いる			
\\	例文	
\\	れいぶん			
\\	例文を3つ作ってください。	
\\	例文 
\\	れいぶん を みっつ つくって ください			
\\	ローマ字	
\\	ろーまじ			
\\	ローマ字で名前を書いてください。	
\\	ローマ字 
\\	ろーまじ で なまえ を かいて ください			
\\	アイロン	
\\	彼女はシャツにアイロンをかけた。	
\\	アイロン 
\\	かのじょ は しゃつ に あいろん を かけた			
\\	苺	
\\	いちご			
\\	この苺はとても甘い。	
\\	苺 
\\	この いちご は とても あまい			
\\	カレーライス	
\\	子供はカレーライスが好きです。	
\\	カレーライス 
\\	こども は かれーらいす が すき です			
\\	キャッシュカード	
\\	キャッシュカードでお金を下ろした。	
\\	キャッシュカード 
\\	きゃっしゅかーど で おかね を おろした			
\\	曇り	
\\	くもり			
\\	今日は一日曇りでした。	
\\	曇り 
\\	きょう は いちにち くもり でした			
\\	魚	
\\	さかな			
\\	塩	
\\	しお			
\\	もうちょっと塩を入れて。	
\\	塩 
\\	もう ちょっと しお を いれて			
\\	停留所	
\\	ていりゅうじょ			
\\	バスの停留所で10分待ちました。	
\\	停留所 
\\	ばす の ていりゅうじょ で じゅっぷん まちました			
\\	床屋	
\\	とこや			
\\	昨日床屋で髪を切った。	
\\	床屋 
\\	きのう とこや で かみ を きった			
\\	歯医者	
\\	はいしゃ			
\\	私は歯医者が嫌いです。	
\\	歯医者 
\\	わたし は はいしゃ が きらい です			
\\	歯磨き	
\\	はみがき			
\\	歯磨きはしましたか。	
\\	歯磨き 
\\	はみがき は しました か			
\\	早起き	
\\	はやおき			
\\	祖父は早起きです。	
\\	早起き 
\\	そふ は はやおき です			
\\	冬休み	
\\	ふゆやすみ			
\\	冬休みにお祖父ちゃんの家に行きます。	
\\	冬休み 
\\	ふゆやすみ に おじいちゃん の うち に いきます			
\\	持って行く	
\\	もっていく			
\\	水を持って行きましょう。	
\\	持って行く 
\\	みず を もっていきましょう			
\\	夕飯	
\\	ゆうはん			
\\	夕飯は寿司でした。	
\\	夕飯 
\\	ゆうはん は すし でした			
\\	酔っ払い	
\\	よっぱらい			
\\	例	
\\	れい			
\\	一つ例をあげてください。	
\\	例 
\\	ひとつ れい を あげて ください			
\\	イヤリング	
\\	彼女はすてきなイヤリングをしている。	
\\	イヤリング 
\\	かのじょ は すてき な いやりんぐ を して いる			
\\	横断歩道	
\\	おうだんほどう			
\\	あそこに横断歩道があります。	
\\	横断歩道 
\\	あそこ に おうだんほどう が あります			
\\	お辞儀	
\\	おじぎ			
\\	皆、社長にお辞儀をした。	
\\	お辞儀 
\\	みんな しゃちょう に おじぎ を した			
\\	お宅	
\\	おたく			
\\	先生のお宅はどちらですか。	
\\	お宅 
\\	せんせい の おたく は どちら です か			
\\	ガールフレンド	
\\	昔は彼女のことをガールフレンドと言っていました。	
\\	ガールフレンド 
\\	むかし は かのじょ の こと を がーるふれんど と いって いました			
\\	カセットテープ	
\\	昔はカセットテープを使っていました。	
\\	カセットテープ 
\\	むかし は カセットテープ を つかっていました 。			
\\	かゆい	
\\	背中がかゆいです。	
\\	かゆい 
\\	せなか が かゆい です			
\\	乾かす	
\\	かわかす			
\\	ぬれた服を乾かした。	
\\	乾かす 
\\	ぬれた ふく を かわかした			
\\	休校	
\\	きゅうこう			
\\	学校は今週は休校です。	
\\	休校 
\\	がっこう は こんしゅう は きゅうこう です			
\\	ぐらぐら	
\\	地震で家がぐらぐらと揺れた。	
\\	ぐらぐら 
\\	じしん で いえ が ぐらぐら と ゆれた			
\\	航空便	
\\	こうくうびん			
\\	航空便で書類が届きました。	
\\	航空便 
\\	こうくうびん で しょるい が とどきました			
\\	ジャム	
\\	このイチゴでジャムを作りましょう。	
\\	ジャム 
\\	この いちご で じゃむ を つくりましょう			
\\	スリッパ	
\\	スリッパをはいてください。	
\\	スリッパ 
\\	すりっぱ を はいて ください			
\\	大嫌い	
\\	だいきらい			
\\	私はテストが大嫌い。	
\\	大嫌い 
\\	わたし は てすと が だいきらい			
\\	畳む	
\\	たたむ			
\\	布団を畳んでください。	
\\	畳む 
\\	ふとん を たたんで ください			
\\	トランプ	
\\	友達とトランプをして遊びました。	
\\	トランプ 
\\	ともだち と とらんぷ を して あそびました			
\\	万歳	
\\	ばんざい			
\\	勝った、万歳!	
\\	万歳 
\\	かった ばんざい			
\\	番地	
\\	ばんち			
\\	その建物の番地は分かりますか。	
\\	番地 
\\	その たてもの の ばんち は わかります か			
\\	日陰	
\\	ひかげ			
\\	暑いので日陰で休みましょう。	
\\	日陰 
\\	あつい の で ひかげ で やすみましょう			
\\	美容院	
\\	びよういん			
\\	父は美容院で髪を切ります。	
\\	美容院 
\\	ちち は びよういん で かみ を きります			
\\	文房具	
\\	ぶんぼうぐ			
\\	新しい文房具を買いました。	
\\	文房具 
\\	あたらしい ぶんぼうぐ を かいました			
\\	免許証	
\\	めんきょしょう			
\\	免許証を見せてください。	
\\	免許証 
\\	めんきょしょう を みせて ください			
\\	訳す	
\\	やくす			
\\	この文を訳してください。	
\\	訳す 
\\	この ぶん を やくして ください			
\\	よろしい	
\\	レポートはこれでよろしいですか。	
\\	よろしい 
\\	れぽーと は これ で よろしい です か			
\\	ラッシュアワー	
\\	私はラッシュアワーの電車が嫌いです。	
\\	ラッシュアワー 
\\	わたし は らっしゅあわー の でんしゃ が きらい です			
\\	両替	
\\	りょうがえ			
\\	1万円を両替してください。	
\\	両替 
\\	いちまんえん を りょうがえ して ください			
\\	明日	
\\	あした			
\\	では、また明日。	
\\	明日 
\\	では 、 また あした。			
\\	送り仮名	
\\	おくりがな			
\\	その送り仮名は間違っています。	
\\	送り仮名 
\\	その おくりがな は まちがって います			
\\	片付く	
\\	かたづく			
\\	仕事が大分片付いた。	
\\	片付く 
\\	しごと が だいぶ かたづいた			
\\	カップ	
\\	水を1カップ入れてください。	
\\	カップ 
\\	みず を いちかっぷ いれて ください			
\\	消しゴム	
\\	けしごむ			
\\	消しゴムを貸して下さい。	
\\	消しゴム 
\\	けしごむ を かして ください			
\\	主人	
\\	しゅじん			
\\	主人は今、留守です。	
\\	主人 
\\	しゅじん は いま るす です			
\\	スケート	
\\	湖でスケートをした。	
\\	スケート 
\\	みずうみ で すけーと を した			
\\	たくさん	
\\	昨日はたくさん泳ぎました。	
\\	たくさん 
\\	きのう は たくさん およぎました			
\\	中	
\\	なか			
\\	鍋	
\\	なべ			
\\	鍋にスープが入っているよ。	
\\	鍋 
\\	なべ に すーぷ が はいって いる よ			
\\	怠ける	
\\	なまける			
\\	怠けていないで、手伝って。	
\\	怠ける 
\\	なまけて いない で てつだって			
\\	二階	
\\	にかい			
\\	兄は二階にいます。	
\\	二階 
\\	あに は にかい に います			
\\	寝坊	
\\	ねぼう			
\\	今朝は寝坊しました。	
\\	寝坊 
\\	けさ は ねぼう しました			
\\	鋏	
\\	はさみ			
\\	この鋏はよく切れる。	
\\	鋏 
\\	この はさみ は よく きれる			
\\	畑	
\\	はたけ			
\\	母は畑仕事が好きだ。	
\\	畑 
\\	はは は はたけ しごと が すき だ			
\\	ピクニック	
\\	今日はピクニックに行きましょう。	
\\	ピクニック 
\\	きょう は ぴくにっく に いきましょう			
\\	真っ青	
\\	まっさお			
\\	空が真っ青です。	
\\	真っ青 
\\	そら が まっさお です			
\\	汚す	
\\	よごす			
\\	彼は服を汚した。	
\\	汚す 
\\	かれ は ふく を よごした			
\\	酔っ払う	
\\	よっぱらう			
\\	彼はただの酔っ払いです。	
\\	酔っ払い 
\\	酔っ払う 
\\	かれ は ただ の よっぱらい です			
\\	昨夜は酔っ払いました。	
\\	酔っ払い 
\\	酔っ払う 
\\	ゆうべ は よっぱらいました			
\\	忘れ物	
\\	わすれもの			
\\	学校に忘れ物をしました。	
\\	忘れ物 
\\	がっこう に わすれもの を しました			
\\	あちら	
\\	あちらに行ってみよう。	
\\	あちら 
\\	あちら に いって みよう			
\\	掛け算	
\\	かけざん			
\\	弟は掛け算を習っている。	
\\	掛け算 
\\	おとうと は かけざん を ならって いる			
\\	臭い	
\\	くさい			
\\	納豆は臭い。	
\\	臭い 
\\	なっとう は くさい			
\\	自習	
\\	じしゅう			
\\	明日は自習の時間がある。	
\\	自習 
\\	あす は じしゅう の じかん が ある			
\\	突き当たり	
\\	つきあたり			
\\	この先は突き当たりです。	
\\	突き当たり 
\\	この さき は つきあたり です			
\\	どきどき	
\\	彼女に会うとどきどきします。	
\\	どきどき 
\\	かのじょ に あう と どきどき します			
\\	無くす	
\\	なくす			
\\	今日、鍵を無くしました。	
\\	無くす 
\\	きょう かぎ を なくしました			
\\	納豆	
\\	なっとう			
\\	私は納豆をよく食べます。	
\\	納豆 
\\	わたし は なっとう を よく たべます			
\\	二十歳	
\\	はたち			
\\	妹は来年、二十歳になります。	
\\	二十歳 
\\	いもうと は らいねん はたち に なります			
\\	東	
\\	ひがし			
\\	日は東から上ります。	
\\	東 
\\	ひ は ひがし から のぼります			
\\	まあまあ	
\\	彼の成績はまあまあです。	
\\	まあまあ 
\\	かれ の せいせき は まあまあ です			
\\	マフラー	
\\	彼女は首にマフラーをまいていた。	
\\	マフラー 
\\	かのじょ は くび に まふらー を まいて いた			
\\	桃	
\\	もも			
\\	私の一番好きな果物は桃です。	
\\	桃 
\\	わたし の いちばん すき な くだもの は もも です			
\\	アクセサリー	
\\	このアクセサリーは素敵ね。	
\\	アクセサリー 
\\	この あくせさりー は すてき ね			
\\	あくび	
\\	父があくびをした。	
\\	あくび 
\\	ちち が あくび を した			
\\	飴	
\\	あめ			
\\	缶に飴が入っています。	
\\	飴 
\\	かん に あめ が はいって います			
\\	売り切れ	
\\	うりきれ			
\\	チケットはもう売り切れだって。	
\\	売り切れ 
\\	ちけっと は もう うりきれ だって			
\\	お巡りさん	
\\	おまわりさん			
\\	あのお巡りさんに道を聞きましょう。	
\\	お巡りさん 
\\	あの おまわりさん に みち を ききましょう			
\\	からから	
\\	おしゃべりして喉がからからになりました。	
\\	からから 
\\	おしゃべり して のど が からから に なりました			
\\	黄色	
\\	きいろ			
\\	信号は黄色でした。	
\\	黄色 
\\	しんごう は きいろ でした			
\\	誘う	
\\	さそう			
\\	彼女をデートに誘った。	
\\	誘う 
\\	かのじょ を でーと に さそった			
\\	速達	
\\	そくたつ			
\\	これを速達で送りたいのですが。	
\\	速達 
\\	これ を そくたつ で おくりたい の です が			
\\	歯	
\\	は			
\\	私の歯は丈夫です。	
\\	歯 
\\	わたし の は は じょうぶ です			
\\	引き算	
\\	ひきざん			
\\	娘は学校で引き算を習っている。	
\\	引き算 
\\	むすめ は がっこう で ひきざん を ならって いる			
\\	昼前	
\\	ひるまえ			
\\	昼前に会議があった。	
\\	昼前 
\\	ひるまえ に かいぎ が あった			
\\	ふらふら	
\\	私は疲れてふらふらです。	
\\	ふらふら 
\\	わたし は つかれて ふらふら です			
\\	目覚まし時計	
\\	めざましどけい			
\\	7時に目覚まし時計が鳴りました。	
\\	目覚まし時計 
\\	しちじ に めざましどけい が なりました			
\\	予習	
\\	よしゅう			
\\	明日の予習をしましょう。	
\\	予習 
\\	あした の よしゅう を しましょう			
\\	こぼす	
\\	ソファにワインをこぼした。	
\\	こぼす 
\\	そふぁ に わいん を こぼした			
\\	ランチ	
\\	一緒にランチに行きませんか。	
\\	ランチ 
\\	いっしょ に らんち に いきません か			
\\	レシート	
\\	レシートを財布に入れました。	
\\	レシート 
\\	れしーと を さいふ に いれました			
\\	持って来る	
\\	もってくる			
\\	そのいすを持って来てください。	
\\	持って来る 
\\	その いす を もって きて ください			
\\	気を付ける	
\\	きをつける			
\\	体に気を付けてください。	
\\	気を付ける 
\\	からだ に き を つけて ください			
\\	刺身	
\\	さしみ			
\\	私は刺身は食べません。	
\\	刺身 
\\	わたし は さしみ は たべません			
\\	洗面所	
\\	せんめんじょ			
\\	洗面所で顔を洗った。	
\\	洗面所 
\\	せんめんじょ で かお を あらった			
\\	足し算	
\\	たしざん			
\\	娘は学校で足し算を習った。	
\\	足し算 
\\	むすめ は がっこう で たしざん を ならった			
\\	段々	
\\	だんだん			
\\	段々仕事が楽しくなってきました。	
\\	段々 
\\	だんだん しごと が たのしく なって きました			
\\	チョーク	
\\	包む	
\\	つつむ			
\\	プレゼントをきれいな紙で包みました。	
\\	包む 
\\	プレゼント を きれい な かみ で つつみました			
\\	ティッシュ	
\\	ティッシュを取ってください。	
\\	ティッシュ 
\\	てぃっしゅ を とって ください			
\\	止まる	
\\	とまる			
\\	今朝、事故で電車が止まりました。	
\\	止まる 
\\	けさ じこ で でんしゃ が とまりました			
\\	無くなる	
\\	なくなる			
\\	もうお金が無くなりました。	
\\	無くなる 
\\	もう おかね が なくなりました			
\\	西	
\\	にし			
\\	太陽は西に沈みます。	
\\	西 
\\	たいよう は にし に しずみます			
\\	ノック	
\\	入る時はドアをノックしてください。	
\\	ノック 
\\	はいる とき は どあ を のっく して ください			
\\	歯ブラシ	
\\	はぶらし			
\\	新しい歯ブラシが必要だ。	
\\	歯ブラシ 
\\	あたらしい はぶらし が ひつよう だ			
\\	暇	
\\	ひま			
\\	明日は暇ですか。	
\\	暇 
\\	あした は ひま です か			
\\	豚	
\\	ぶた			
\\	豚はすごく鼻が良い。	
\\	豚 
\\	ぶた は すごく はな が いい			
\\	二人	
\\	ふたり			
\\	今日は妻と二人で食事をします。	
\\	二人 
\\	きょう は つま と ふたり で しょくじ を します			
\\	褒める	
\\	ほめる			
\\	頑張ったので褒められました。	
\\	褒める 
\\	がんばった の で ほめられました			
\\	本箱	
\\	ほんばこ			
\\	雑誌を本箱に入れました。	
\\	本箱 
\\	ざっし を ほんばこ に いれました			
\\	止む	
\\	やむ			
\\	雨が止みました。	
\\	止む 
\\	あめ が やみました			
\\	よろしく	
\\	よろしくお願いします。	
\\	よろしく 
\\	よろしく おねがい します			
\\	沸かす	
\\	わかす			
\\	お湯を沸かしてください。	
\\	沸かす 
\\	おゆ を わかして ください			
\\	明後日	
\\	あさって			
\\	明後日は休日です。	
\\	明後日 
\\	あさって は きゅうじつ です 。			
\\	幾ら	
\\	いくら			
\\	この靴は幾らですか。	
\\	幾ら 
\\	この くつ は いくら です か			
\\	絵葉書	
\\	えはがき			
\\	友達から絵葉書が届きました。	
\\	絵葉書 
\\	ともだち から えはがき が とどきました			
\\	暖める	
\\	あたためる			
\\	今、車を暖めています。	
\\	暖める 
\\	いま くるま を あたためています			
\\	一昨日	
\\	おととい			
\\	一昨日彼から電話がありました。	
\\	一昨日 
\\	おととい かれ から でんわ が ありました			
\\	換える	
\\	かえる			
\\	車のタイヤを換えた。	
\\	換える 
\\	くるま の たいや を かえた			
\\	書留	
\\	かきとめ			
\\	これを書留で送りたいのですが。	
\\	書留 
\\	これ を かきとめ で おくりたい の です が			
\\	がらがら	
\\	映画館はがらがらでした。	
\\	がらがら 
\\	えいがかん は がらがら でした			
\\	缶詰	
\\	かんづめ			
\\	缶詰のフルーツはとても甘い。	
\\	缶詰 
\\	かんづめ の ふるーつ は とても あまい			
\\	ぎらぎら	
\\	太陽がぎらぎらしている。	
\\	ぎらぎら 
\\	たいよう が ぎらぎら して いる			
\\	九	
\\	く			
\\	私は九月に行く予定です。	
\\	九 
\\	わたし は くがつ に いく よてい です			
\\	怪我	
\\	けが			
\\	彼女は腕を怪我した。	
\\	怪我 
\\	かのじょ は うで を けがした			
\\	寒さ	
\\	さむさ			
\\	今日は厳しい寒さになるでしょう。	
\\	寒さ 
\\	きょう は きびしい さむさ に なる でしょう			
\\	祝日	
\\	しゅくじつ			
\\	5月3日は祝日だ。	
\\	祝日 
\\	ごがつ みっか は しゅくじつ だ			
\\	すき焼き	
\\	すきやき			
\\	昨日の夜はすき焼きを食べた。	
\\	すき焼き 
\\	きのう の よる は すきやき を たべた			
\\	そっち	
\\	そっちが私の部屋です。	
\\	そっち 
\\	そっち が わたし の へや です			
\\	葉書	
\\	はがき			
\\	家族に葉書を書いています。	
\\	葉書 
\\	かぞく に はがき を かいています 。			
\\	掃く	
\\	はく			
\\	床をほうきで掃きました。	
\\	床 
\\	掃く 
\\	ゆか を ほうき で はきました			
\\	昼間	
\\	ひるま			
\\	昼間は仕事で忙しいです。	
\\	昼間 
\\	ひるま は しごと で いそがしい です			
\\	ピンポン	
\\	昨日の夕方、友達とピンポンをした。	
\\	ピンポン 
\\	きのう の ゆうがた ともだち と ぴんぽん を した			
\\	深さ	
\\	ふかさ			
\\	このプールの深さは2メートルです。	
\\	深さ 
\\	この ぷーる の ふかさ は にめーとる です			
\\	不親切	
\\	ふしんせつ			
\\	その店員は不親切だった。	
\\	不親切 
\\	その てんいん は ふしんせつ だった			
\\	ぺこぺこ	
\\	私はおなかがぺこぺこです。	
\\	ぺこぺこ 
\\	わたし は おなか が ぺこぺこ です			
\\	ぺらぺら	
\\	彼女はドイツ語がぺらぺらです。	
\\	ぺらぺら 
\\	かのじょ は どいつご が ぺらぺら です			
\\	横書き	
\\	よこがき			
\\	この本は横書きです。	
\\	横書き 
\\	この ほん は よこがき です			
\\	レインコート	
\\	雨なのでレインコートを着ました。	
\\	レインコート 
\\	あめ な の で れいんこーと を きました			
\\	アイス	
\\	暑いのでアイスを食べました。	
\\	アイス 
\\	あつい の で あいす を たべました			
\\	いつごろ	
\\	いつごろ夏休みを取りますか。	
\\	いつごろ 
\\	いつごろ なつやすみ を とります か			
\\	ウィスキー	
\\	このウィスキーは強いよ。	
\\	ウィスキー 
\\	この うぃすきー は つよい よ			
\\	エアメール	
\\	彼にエアメールを送りました。	
\\	エアメール 
\\	かれ に えあめーる を おくりました			
\\	字	
\\	じ			
\\	チョークで黒板に字を書きました。	
\\	チョーク 
\\	字 
\\	ちょーく で こくばん に じ を かきました			
\\	もっと大きく字を書いてください。	
\\	字 
\\	もっと おおきく じ を かいて ください			
\\	塩辛い	
\\	しおからい			
\\	海の水は塩辛い。	
\\	塩辛い 
\\	うみ の みず は しおからい			
\\	七	
\\	しち			
\\	そのグループのメンバーは全部で七人だ。	
\\	七 
\\	その ぐるーぷ の めんばー は ぜんぶ で しちにん だ			
\\	シャープペンシル	
\\	私のシャープペンシルがありません。	
\\	シャープペンシル 
\\	わたし の しゃーぷぺんしる が ありません			
\\	セロテープ	
\\	セロテープはありますか。	
\\	セロテープ 
\\	せろてーぷ は あります か			
\\	外側	
\\	そとがわ			
\\	白線の外側を歩かないでください。	
\\	外側 
\\	はくせん の そとがわ を あるかない で ください			
\\	そば屋	
\\	そばや			
\\	昼はそば屋に行きました。	
\\	そば 
\\	そば屋 
\\	ひる は そばや に いきました 。			
\\	たばこ屋	
\\	たばこや			
\\	私はたばこ屋でライターを買った。	
\\	たばこ屋 
\\	わたし は たばこや で らいたー を かった			
\\	庭	
\\	にわ			
\\	庭に白い猫がいます。	
\\	庭 
\\	にわ に しろい ねこ が います			
\\	流行る	
\\	はやる			
\\	去年はスニーカーが流行りました。	
\\	流行る 
\\	きょねん は すにーかー が はやりました			
\\	左側	
\\	ひだりがわ			
\\	画面の左側を見てください。	
\\	左側 
\\	がめん の ひだりがわ を みて ください			
\\	袋	
\\	ふくろ			
\\	袋はいりません。	
\\	袋 
\\	ふくろ は いりません			
\\	風呂屋	
\\	ふろや			
\\	昨日、友達とお風呂屋さんに行った。	
\\	風呂屋 
\\	きのう ともだち と おふろやさん に いった			
\\	骨	
\\	ほね			
\\	彼は足の骨を折りました。	
\\	骨 
\\	かれ は あし の ほね を おりました			
\\	右側	
\\	みぎがわ			
\\	彼女はいつも私の右側を歩きます。	
\\	右側 
\\	かのじょ は いつも わたし の みぎがわ を あるきます			
\\	眼鏡	
\\	めがね			
\\	彼は眼鏡をかけています。	
\\	眼鏡 
\\	かれ は めがね を かけて います			
\\	やけど	
\\	彼は手にやけどをしました。	
\\	やけど 
\\	かれ は て に やけど を しました			
\\	汚れ	
\\	よごれ			
\\	靴の汚れを落としました。	
\\	汚れ 
\\	くつ の よごれ を おとしました			
\\	割り算	
\\	わりざん			
\\	割り算は小学校で習います。	
\\	割り算 
\\	わりざん は しょうがっこう で ならいます			
\\	アイスコーヒー	
\\	アイスコーヒーをください。	
\\	アイスコーヒー 
\\	あいすこーひー を ください			
\\	間	
\\	あいだ			
\\	雲の間から月が出た。	
\\	間 
\\	くも の あいだ から つき が でた			
\\	顎	
\\	あご			
\\	顎が痛い。	
\\	顎 
\\	あご が いたい			
\\	朝御飯	
\\	あさごはん			
\\	七時に朝御飯を食べました。	
\\	朝御飯 
\\	しちじ に あさごはん を たべました			
\\	暖まる	
\\	あたたまる			
\\	まだ部屋が暖まらない。	
\\	暖まる 
\\	まだ へや が あたたまらない。			
\\	あっち	
\\	郵便局はあっちです。	
\\	あっち 
\\	ゆうびんきょく は あっち です			
\\	あんなに	
\\	あんなにいい人はいません。	
\\	あんなに 
\\	あんなに いい ひと は いません			
\\	行き	
\\	いき			
\\	行きは新幹線で行った。	
\\	行き 
\\	いき は しんかんせん で いった			
\\	いたずら	
\\	その子はいたずらが大好きだ。	
\\	いたずら 
\\	その こ は いたずら が だいすき だ			
\\	ウェートレス	
\\	ウェートレスを呼んだ。	
\\	ウェートレス 
\\	うぇーとれす を よんだ			
\\	お祝い	
\\	おいわい			
\\	彼の就職のお祝いをしよう。	
\\	お祝い 
\\	かれ の しゅうしょく の おいわい を しよう			
\\	お菓子	
\\	おかし			
\\	お母さんと一緒にお菓子を作りました。	
\\	お菓子 
\\	おかあさん と いっしょ に おかし を つくりました			
\\	お釣り	
\\	おつり			
\\	母はお釣りを確かめた。	
\\	お釣り 
\\	はは は おつり を たしかめた			
\\	大人しい	
\\	おとなしい			
\\	私の彼女はとても大人しいです。	
\\	大人しい 
\\	わたし の かのじょ は とても おとなしい です			
\\	お祖母さん	
\\	おばあさん			
\\	彼女はお祖母さんと住んでいる。	
\\	お祖母さん 
\\	かのじょ は おばあさん と すん でいる 。			
\\	おばさん	
\\	おばさん一家が遊びに来ました。	
\\	おばさん 
\\	おばさん いっか が あそび に きました			
\\	お土産	
\\	おみやげ			
\\	父はお土産にお菓子を買ってきた。	
\\	お土産 
\\	ちち は おみやげ に おかし を かって きた			
\\	表	
\\	おもて			
\\	表に人が来ています。	
\\	表 
\\	おもて に ひと が きて います			
\\	書き取る	
\\	かきとる			
\\	話しのポイントを書き取った。	
\\	書き取る 
\\	はなし の ぽいんと を かきとった			
\\	泊める	
\\	とめる			
\\	友達をうちに泊めてあげました。	
\\	泊める 
\\	ともだち を うち に とめて あげました			
\\	歌手	
\\	かしゅ			
\\	風邪薬	
\\	かぜぐすり			
\\	この風邪薬を飲みなさい。	
\\	風邪薬 
\\	この かぜぐすり を のみなさい			
\\	方	
\\	かた			
\\	次の方、どうぞ。	
\\	方 
\\	つぎ の かた どうぞ			
\\	形	
\\	かたち			
\\	その椅子は変わった形をしている。	
\\	形 
\\	その いす は かわった かたち を して いる			
\\	鞄	
\\	かばん			
\\	その黒い鞄は僕のです。	
\\	鞄 
\\	その くろい かばん は ぼく の です			
\\	被る	
\\	かぶる			
\\	帽子を被って外出した。	
\\	被る 
\\	ぼうし を かぶって がいしゅつ した			
\\	雷	
\\	かみなり			
\\	雷が鳴っています。	
\\	雷 
\\	かみなり が なって います			
\\	上がる	
\\	あがる			
\\	彼は人前だと上がってしまう。	
\\	上がる 
\\	かれ は ひとまえ だ と あがって しまう			
\\	空手	
\\	からて			
\\	彼は空手を習っています。	
\\	空手 
\\	かれ は からて を ならって います			
\\	缶	
\\	かん			
\\	コーヒーはその缶に入っています。	
\\	缶 
\\	こーひー は その かん に はいって います			
\\	元旦	
\\	がんたん			
\\	元旦に彼から年賀状が来た。	
\\	元旦 
\\	がんたん に かれ から ねんがじょう が きた 。			
\\	着替える	
\\	きがえる			
\\	彼はきれいな服に着替えた。	
\\	着替える 
\\	かれ は きれい な ふく に きがえた			
\\	厳しい	
\\	きびしい			
\\	私の上司はとても厳しい。	
\\	厳しい 
\\	わたし の じょうし は とても きびしい			
\\	客	
\\	きゃく			
\\	その店は若い客が多いです。	
\\	客 
\\	その みせ は わかい きゃく が おおい です			
\\	金	
\\	きん			
\\	彼女は金のネックレスをしています。	
\\	金 
\\	かのじょ は きん の ねっくれす を して います			
\\	くし	
\\	彼はくしで髪をとかした。	
\\	くし 
\\	かれ は くし で かみ を とかした			
\\	雲	
\\	くも			
\\	今日は雲が多い。	
\\	雲 
\\	きょう は くも が おおい			
\\	クリーニング	
\\	スーツをクリーニングに出しました。	
\\	クリーニング 
\\	すーつ を くりーにんぐ に だしました			
\\	景色	
\\	けしき			
\\	ここは景色がきれいですね。	
\\	景色 
\\	ここ は けしき が きれい です ね			
\\	喧嘩	
\\	けんか			
\\	喧嘩はやめて。	
\\	喧嘩 
\\	けんか は やめて			
\\	答え	
\\	こたえ			
\\	彼はその問題の答えが分からない。	
\\	答え 
\\	かれ は その もんだい の こたえ が わからない			
\\	ごちそう	
\\	テーブルの上にごちそうが並んでいる。	
\\	ごちそう 
\\	てーぶる の うえ に ごちそう が ならんで いる			
\\	ごちそうする	
\\	今日の夕食は私がごちそうします。	
\\	ごちそう 
\\	ごちそうする 
\\	きょう の ゆうしょく は わたし が ごちそう します			
\\	この頃	
\\	このごろ			
\\	この頃、よく肩が凝る。	
\\	この頃 
\\	このごろ よく かた が こる			
\\	こぼれる	
\\	彼女の目から涙がこぼれた。	
\\	こぼれる 
\\	かのじょ の め から なみだ が こぼれた			
\\	壊れる	
\\	こわれる			
\\	会社のパソコンが壊れた。	
\\	壊れる 
\\	かいしゃ の ぱそこん が こわれた			
\\	こんなに	
\\	こんなに面白い本は初めて読んだ。	
\\	こんなに 
\\	こんなに おもしろい ほん は はじめて よんだ			
\\	ざあざあ	
\\	雨がざあざあ降っている。	
\\	ざあざあ 
\\	あめ が ざあざあ ふって いる			
\\	再来月	
\\	さらいげつ			
\\	再来月まで予約で一杯です。	
\\	再来月 
\\	さらいげつ まで よやく で いっぱい です			
\\	再来週	
\\	さらいしゅう			
\\	再来週は忙しいです。	
\\	再来週 
\\	さらいしゅう は いそがしい です			
\\	騒ぐ	
\\	さわぐ			
\\	電車の中で騒がないでください。	
\\	騒ぐ 
\\	でんしゃ の なか で さわがない で ください			
\\	残念	
\\	ざんねん			
\\	その試合は残念な結果になった。	
\\	残念 
\\	その しあい は ざんねん な けっか に なった			
\\	叱る	
\\	しかる			
\\	父親が子供を叱っている。	
\\	叱る 
\\	ちちおや が こども を しかって いる			
\\	仕舞う	
\\	しまう			
\\	彼女は大切な書類を机に仕舞った。	
\\	仕舞う 
\\	かのじょ は たいせつ な しょるい を つくえ に しまった。			
\\	氏名	
\\	しめい			
\\	ここに住所と氏名を書いてください。	
\\	氏名 
\\	ここ に じゅうしょ と しめい を かいて ください			
\\	ジャガイモ	
\\	ポテトサラダを作るから、ジャガイモを買ってきて。	
\\	ジャガイモ 
\\	ぽてと さらだ を つくる から じゃがいも を かって きて			
\\	車道	
\\	しゃどう			
\\	車道の工事が始まりました。	
\\	車道 
\\	しゃどう の こうじ が はじまりました			
\\	丈夫	
\\	じょうぶ			
\\	祖母は身体が丈夫だ。	
\\	丈夫 
\\	そぼ は からだ が じょうぶ だ			
\\	醤油	
\\	しょうゆ			
\\	もう少し醤油を足してください。	
\\	醤油 
\\	もうすこし しょうゆ を たして ください 。			
\\	白	
\\	しろ			
\\	白は雪の色です。	
\\	白 
\\	しろ は ゆき の いろ です			
\\	心配	
\\	しんぱい			
\\	明日のプレゼンテーションが心配だ。	
\\	心配 
\\	あす の ぷれぜんてーしょん が しんぱい だ			
\\	石鹸	
\\	せっけん			
\\	石鹸で手を洗ってください。	
\\	石鹸 
\\	せっけん で て を あらって ください			
\\	洗濯	
\\	せんたく			
\\	一週間、洗濯をしていない。	
\\	洗濯 
\\	いっしゅうかん せんたく を して いない			
\\	掃除機	
\\	そうじき			
\\	掃除機が壊れた。	
\\	掃除機 
\\	そうじき が こわれた			
\\	袖	
\\	そで			
\\	このシャツは袖が短い。	
\\	袖 
\\	この しゃつ は そで が みじかい			
\\	外	
\\	そと			
\\	外は暑いよ。	
\\	外 
\\	そと は あついよ			
\\	祖父	
\\	そふ			
\\	祖父は元気です。	
\\	祖父 
\\	そふ は げんき です			
\\	ソファー	
\\	このソファーは気持ちがいい。	
\\	ソファー 
\\	この そふぁー は きもち が いい			
\\	祖母	
\\	そぼ			
\\	祖母は京都で生まれました。	
\\	祖母 
\\	そぼ は きょうと で うまれました			
\\	空	
\\	そら			
\\	空の写真を撮るのが好きです。	
\\	空 
\\	そら の しゃしん を とる の が すき です			
\\	剃る	
\\	そる			
\\	父は毎日髭を剃ります。	
\\	剃る 
\\	ちち は まいにち ひげ を そります			
\\	台	
\\	だい			
\\	そこにちょうど良い台がある。	
\\	台 
\\	そこ に ちょうど いい だい が ある			
\\	大事	
\\	だいじ			
\\	お体をお大事に。	
\\	大事 
\\	おからだ を おだいじ に			
\\	大切	
\\	たいせつ			
\\	これは母が大切にしていた指輪です。	
\\	大切 
\\	これ は はは が たいせつ に して いた ゆびわ です			
\\	倒す	
\\	たおす			
\\	そのスキーヤーはポールを倒した。	
\\	倒す 
\\	その すきーやー は ぽーる を たおした			
\\	助ける	
\\	たすける			
\\	彼女は病気の犬を助けた。	
\\	助ける 
\\	かのじょ は びょうき の いぬ を たすけた			
\\	たばこ	
\\	彼はたばこを吸いません。	
\\	たばこ 
\\	かれ は たばこ を すいません			
\\	誰か	
\\	だれか			
\\	誰かに聞いてみてください。	
\\	誰か 
\\	だれか に きいて みて ください			
\\	血	
\\	ち			
\\	血が出ていますよ。	
\\	血 
\\	ち が でて います よ			
\\	茶わん	
\\	ちゃわん			
\\	茶わんにご飯をよそいました。	
\\	茶わん 
\\	ちゃわん に ごはん を よそいました			
\\	昼食	
\\	ちゅうしょく			
\\	昼食に寿司を食べました。	
\\	昼食 
\\	ちゅうしょく に すし を たべました			
\\	長女	
\\	ちょうじょ			
\\	うちの長女は10歳です。	
\\	長女 
\\	うち の ちょうじょ は じゅっさい です			
\\	朝食	
\\	ちょうしょく			
\\	朝食に納豆を食べました。	
\\	朝食 
\\	ちょうしょく に なっとう を たべました			
\\	ちょうど	
\\	値段はちょうど3万円です。	
\\	ちょうど 
\\	ねだん は ちょうど さんまんえん です			
\\	ちょっと	
\\	ちょっとだけ待っていてください。	
\\	ちょっと 
\\	ちょっと だけ まって いて ください			
\\	つく	
\\	部屋の電気がついています。	
\\	つく 
\\	へや の でんき が ついて います			
\\	机	
\\	つくえ			
\\	新しい机を買ってもらいました。	
\\	机 
\\	あたらしい つくえ を かって もらいました			
\\	勤める	
\\	つとめる			
\\	私は銀行に勤めています。	
\\	勤める 
\\	わたし は ぎんこう に つとめて います			
\\	爪	
\\	つめ			
\\	爪が伸びています。	
\\	爪 
\\	つめ が のびて います			
\\	寺	
\\	てら			
\\	あそこに古いお寺があります。	
\\	寺 
\\	あそこ に ふるい おてら が あります			
\\	てんぷら	
\\	私はえびのてんぷらが好きです。	
\\	てんぷら 
\\	わたし は えび の てんぷら が すき です			
\\	年上	
\\	としうえ			
\\	彼は私より年上です。	
\\	年上 
\\	かれ は わたし より としうえ です			
\\	鳥	
\\	とり			
\\	あそこにきれいな鳥がいます。	
\\	鳥 
\\	あそこ に きれい な とり が います			
\\	とんとん	
\\	母の肩をとんとんたたいた。	
\\	とんとん 
\\	はは の かた を とんとん たたいた			
\\	長袖	
\\	ながそで			
\\	今日は寒いので長袖を着ました。	
\\	長袖 
\\	きょう は さむい ので ながそで を きました 。			
\\	茄子	
\\	なす			
\\	夕飯に茄子の天ぷらを食べました。	
\\	茄子 
\\	ゆうはん に なすび の てんぷら を たべました 。			
\\	何	
\\	なん			
\\	質問は何ですか。	
\\	何 
\\	しつもん は なん です か			
\\	兄さん	
\\	にいさん			
\\	兄さん、おめでとう。	
\\	兄さん 
\\	にいさん おめでとう			
\\	賑やか	
\\	にぎやか			
\\	浅草は賑やかな街です。	
\\	賑やか 
\\	あさくさ は にぎやか な まち です			
\\	肉	
\\	にく			
\\	昨日の夜は肉をたくさん食べました。	
\\	肉 
\\	きのう の よる は にく を たくさん たべました			
\\	鶏	
\\	にわとり			
\\	その鶏は毎朝鳴きます。	
\\	鶏 
\\	その にわとり は まいあさ なきます			
\\	人気	
\\	にんき			
\\	このバンドはとても人気があるよ。	
\\	人気 
\\	この ばんど は とても にんき が ある よ			
\\	人間	
\\	にんげん			
\\	人間の心は複雑です。	
\\	人間 
\\	にんげん の こころ は ふくざつ です			
\\	にんじん	
\\	私はにんじんが嫌いだ。	
\\	にんじん 
\\	わたし は にんじん が きらい だ			
\\	ぬるい	
\\	濡れる	
\\	ぬれる			
\\	雨で濡れてしまった。	
\\	濡れる 
\\	あめ で ぬれて しまった			
\\	姉さん	
\\	ねえさん			
\\	姉さん、ごめんね。	
\\	姉さん 
\\	ねえさん ごめん ね			
\\	眠い	
\\	ねむい			
\\	今日はとても眠いです。	
\\	眠い 
\\	きょう は とても ねむい です			
\\	眠る	
\\	ねむる			
\\	昨日は8時間眠りました。	
\\	眠る 
\\	きのう は はちじかん ねむりました			
\\	上り	
\\	のぼり			
\\	これは上り電車です。	
\\	上り 
\\	これ は のぼり でんしゃ です			
\\	パーセント	
\\	この村の80パーセントは老人です。	
\\	パーセント 
\\	この むら の はちじゅっぱーせんと は ろうじん です			
\\	灰	
\\	はい			
\\	火事ですべて灰になった。	
\\	灰 
\\	かじ で すべて はい に なった			
\\	灰色	
\\	はいいろ			
\\	今日の空は灰色だ。	
\\	灰色 
\\	きょう の そら は はいいろ だ			
\\	拍手	
\\	はくしゅ			
\\	大きな拍手が上がった。	
\\	拍手 
\\	おおき な はくしゅ が あがった			
\\	橋	
\\	はし			
\\	あの橋は日本で一番長い。	
\\	橋 
\\	あの はし は にほん で いちばん ながい			
\\	初め	
\\	はじめ			
\\	初めは上手くできませんでした。	
\\	初め 
\\	はじめ は うまく できません でした 。			
\\	柱	
\\	はしら			
\\	この家の柱は太い。	
\\	柱 
\\	この いえ の はしら は ふとい			
\\	旗	
\\	はた			
\\	旗が風に揺れている。	
\\	旗 
\\	はた が かぜ に ゆれて いる			
\\	貼る	
\\	はる			
\\	机にシールを貼った。	
\\	机 
\\	貼る 
\\	つくえ に しーる を はった			
\\	半袖	
\\	はんそで			
\\	今日は、半袖のシャツを着よう。	
\\	半袖 
\\	きょう は 、 はんそで の シャツ を きよう 。			
\\	半月	
\\	はんつき			
\\	半月前に日本に来ました。	
\\	半月 
\\	はんつきまえ に にほん に きました			
\\	半日	
\\	はんにち			
\\	今日は半日だけ仕事だ。	
\\	半日 
\\	きょう は はんにち だけ しごと だ			
\\	火	
\\	ひ			
\\	火を消して。	
\\	火 
\\	ひ を けして			
\\	弾く	
\\	ひく			
\\	彼はギターを弾きます。	
\\	弾く 
\\	かれ は ぎたー を ひきます			
\\	ひざ	
\\	スキーでひざを怪我しました。	
\\	ひざ 
\\	すきー で ひざ を けがしました			
\\	ひじ	
\\	彼はひじに怪我をした。	
\\	ひじ 
\\	かれ は ひじ に けが を した			
\\	人々	
\\	ひとびと			
\\	あの村の人々はとても親切です。	
\\	人々 
\\	あの むら の ひとびと は とても しんせつ です			
\\	ひも	
\\	靴のひもが切れた。	
\\	ひも 
\\	くつ の ひも が きれた			
\\	この表を見てください。	
\\	表 
\\	この ひょう を みて ください			
\\	昼過ぎ	
\\	ひるすぎ			
\\	今日は昼過ぎまで寝ていた。	
\\	昼過ぎ 
\\	きょう は ひるすぎ まで ねて いた			
\\	瓶	
\\	びん			
\\	瓶ビールを注文した。	
\\	瓶 
\\	びんびーる を ちゅうもん した			
\\	拭く	
\\	ふく			
\\	タオルで体を拭きました。	
\\	拭く 
\\	たおる で からだ を ふきました			
\\	復習	
\\	ふくしゅう			
\\	昨日の復習をしましたか。	
\\	復習 
\\	きのう の ふくしゅう を しました か			
\\	豚肉	
\\	ぶたにく			
\\	豚肉は美味しいです。	
\\	豚肉 
\\	ぶたにく は おいしい です			
\\	船便	
\\	ふなびん			
\\	アメリカの友人から船便が届いた。	
\\	船便 
\\	あめりか の ゆうじん から ふなびん が とどいた			
\\	船	
\\	ふね			
\\	私たちは船に乗った。	
\\	船 
\\	わたしたち は ふね に のった			
\\	降る	
\\	ふる			
\\	明日は雨が降るでしょう。	
\\	降る 
\\	あす は あめ が ふる でしょう			
\\	風呂	
\\	ふろ			
\\	風呂がぬるかった。	
\\	ぬるい 
\\	風呂 
\\	ふろ が ぬるかった			
\\	父は今お風呂に入っています。	
\\	風呂 
\\	ちち は いま お ふろ に はいって います			
\\	平成	
\\	へいせい			
\\	彼女は平成3年生まれです。	
\\	平成 
\\	かのじょ は へいせい さんねん うまれ です			
\\	下手	
\\	へた			
\\	その歌手は歌が下手だ。	
\\	歌手 
\\	下手 
\\	その かしゅ は うた が へた だ			
\\	私は歌が下手だ。	
\\	下手 
\\	わたし は うた が へた だ			
\\	返事	
\\	へんじ			
\\	手紙の返事を出しました。	
\\	返事 
\\	てがみ の へんじ を だしました			
\\	便所	
\\	べんじょ			
\\	便所はそこです。	
\\	便所 
\\	べんじょ は そこ です			
\\	星	
\\	ほし			
\\	今夜は星がよく見えます。	
\\	星 
\\	こんや は ほし が よく みえます			
\\	本当	
\\	ほんとう			
\\	その話は本当ですか。	
\\	本当 
\\	その はなし は ほんとう です か			
\\	毎月	
\\	まいつき			
\\	私は毎月貯金をしています。	
\\	毎月 
\\	わたし は まいつき ちょきん を して います			
\\	まく	
\\	日本では、二月に豆をまく行事がある。	
\\	まく 
\\	にほん で は にがつ に まめ を まく ぎょうじ が ある			
\\	枕	
\\	まくら			
\\	私は低い枕が好きです。	
\\	枕 
\\	わたし は ひくい まくら が すき です			
\\	孫	
\\	まご			
\\	昨日、孫が生まれました。	
\\	孫 
\\	きのう まご が うまれました			
\\	又	
\\	また			
\\	明日、また来ます。	
\\	又 
\\	あした また きます			
\\	町	
\\	まち			
\\	ここは小さな町です。	
\\	町 
\\	ここ は ちいさ な まち です			
\\	丸	
\\	まる			
\\	紙に大きな丸を書きました。	
\\	丸 
\\	かみ に おおき な まる を かきました			
\\	港	
\\	みなと			
\\	港に船が着きました。	
\\	港 
\\	みなと に ふね が つきました			
\\	南	
\\	みなみ			
\\	私の家は町の南にあります。	
\\	南 
\\	わたし の いえ は まち の みなみ に あります			
\\	紫	
\\	むらさき			
\\	彼女は紫のドレスを着ていた。	
\\	紫 
\\	かのじょ は むらさき の どれす を きて いた			
\\	姪	
\\	めい			
\\	私の姪は3才です。	
\\	姪 
\\	わたし の めい は さんさい です			
\\	木綿	
\\	もめん			
\\	彼女は木綿のシャツを着ています。	
\\	木綿 
\\	かのじょ は もめん の しゃつ を きて います			
\\	やかん	
\\	やかんでお湯を沸かしました。	
\\	やかん 
\\	やかん で おゆ を わかしました			
\\	焼く	
\\	やく			
\\	今、魚を焼いています。	
\\	焼く 
\\	いま さかな を やいて います			
\\	痩せる	
\\	やせる			
\\	私は少し痩せました。	
\\	痩せる 
\\	わたし は すこし やせました			
\\	山	
\\	やま			
\\	山の空気はきれいだ。	
\\	山 
\\	やま の くうき は きれい だ			
\\	湯	
\\	ゆ			
\\	お風呂のお湯が一杯です。	
\\	湯 
\\	おふろ の おゆ が いっぱい です			
\\	夕食	
\\	ゆうしょく			
\\	夕食は7時です。	
\\	夕食 
\\	ゆうしょく は しちじ です			
\\	昨夜	
\\	ゆうべ			
\\	昨夜、流れ星を見ました。	
\\	昨夜 
\\	ゆうべ ながれぼし を みました			
\\	東京行きの列車に乗った。	
\\	行き 
\\	とうきょうゆき の れっしゃ に のった			
\\	指	
\\	ゆび			
\\	彼は指が太い。	
\\	指 
\\	かれ は ゆび が ふとい			
\\	緩い	
\\	ゆるい			
\\	このズボンは緩いです。	
\\	緩い 
\\	この ずぼん は ゆるい です			
\\	良い	
\\	よい			
\\	彼は良い人です。	
\\	良い 
\\	かれ は よい ひと です			
\\	ようこそ	
\\	日本へようこそ。	
\\	ようこそ 
\\	にっぽん へ ようこそ			
\\	汚れる	
\\	よごれる			
\\	エプロンをしないと服が汚れます。	
\\	汚れる 
\\	えぷろん を しない と ふく が よごれます			
\\	四つ角	
\\	よつかど			
\\	あそこの四つ角を左に曲がってください。	
\\	四つ角 
\\	あそこ の よつかど を ひだり に まがって ください			
\\	夜中	
\\	よなか			
\\	夜中に電話がありました。	
\\	夜中 
\\	よなか に でんわ が ありました			
\\	両親	
\\	りょうしん			
\\	私の両親は大阪に住んでいます。	
\\	両親 
\\	わたし の りょうしん は おおさか に すんで います			
\\	零	
\\	れい			
\\	今ちょうど零時です。	
\\	ちょうど 
\\	零 
\\	いま ちょうど れいじ です			
\\	曲がり角	
\\	まがりかど			
\\	ポストはそこの曲がり角にあります。	
\\	曲がり角 
\\	ぽすと は そこ の まがりかど に あります			
\\	ウェーター	
\\	ウェーターが水を運んできました。	
\\	ウェーター 
\\	うぇーたー が みず を はこんで きました			
\\	お医者さん	
\\	おいしゃさん			
\\	熱があるのでお医者さんに行った。	
\\	お医者さん 
\\	ねつ が ある の で おいしゃさん に いった			
\\	細か	
\\	こまか			
\\	それは細かな問題です。	
\\	細か 
\\	それ は こまか な もんだい です			
\\	絶対に	
\\	ぜったいに			
\\	絶対に駄目です。	
\\	絶対に 
\\	ぜったいに だめ です			
\\	そんなに	
\\	一度にそんなにたくさんはできない。	
\\	そんなに 
\\	いちど に そんなに たくさん は できない			
\\	連れて行く	
\\	つれていく			
\\	私も連れて行ってください。	
\\	連れて行く 
\\	わたし も つれて いって ください			
\\	連れて来る	
\\	つれてくる			
\\	息子が友達を連れて来ました。	
\\	連れて来る 
\\	むすこ が ともだち を つれて きました			
\\	似ている	
\\	にている			
\\	私は母に似ています。	
\\	似ている 
\\	わたし は はは に にて います			
\\	音	
\\	おと			
\\	このピアノは音があまり良くない。	
\\	音 
\\	この ぴあの は おと が あまり よく ない			
\\	始めに	
\\	はじめに			
\\	始めにスープが出ます。	
\\	始めに 
\\	はじめに すーぷ が でます			
\\	一人で	
\\	ひとりで			
\\	今日は一人で映画を見ます。	
\\	一人で 
\\	きょう は ひとりで えいが を みます			
\\	不動産屋	
\\	ふどうさんや			
\\	私の父は不動産屋です。	
\\	不動産屋 
\\	わたし の ちち は ふどうさんや です			
\\	郵便屋さん	
\\	ゆうびんやさん			
\\	郵便屋さんはもう来ましたか。	
\\	郵便屋さん 
\\	ゆうびんやさん は もう きました か			
\\	可	
\\	か			
\\	このアルバイトは「学生可」ですね。	
\\	可 
\\	この あるばいと は がくせい か です ね			
\\	でしょう	
\\	かもしれません	
\\	ずっと	
\\	きょう (今日)	
\\	同じ (おなじ)	
\\	そっくり	
\\	美しいい	
\\	立つ	
\\	奴隷 (どれい)	
\\	ばあい	
\\	変わる (かわる)	
\\	作る (つくる)	
\\	木 (き)	
\\	新しい (あたらしい)	
\\	わかい	
\\	古い (ふるい)	
\\	生きる (いきる)	
\\	難しい (むずかしい)	
\\	しか ない	
\\	お願い (おねがい)	
\\	とんでもない	
\\	止める (とめる 
\\	やめる)	
\\	働く (はたらく), 仕事 (しごと)	
\\	動く (うごく)	
\\	動かす (うごかす)	
\\	戦争 (せんそう)	
\\	戦闘 (せんとう)	
\\	始める (はじめる)	
\\	女性 (じょせい)	
\\	男性 (だんせい)	
\\	紳士 (しんし)	
\\	マジで	
\\	安い (やすい)	
\\	ーくて 
\\	よくて 赤い リンゴは	
\\	優しい (やさしい) 新設 (しんせつ)	
\\	全部 (ぜんぶ)	
\\	すべて, すべての	
\\	多く、たくさん	
\\	何ー
\\	(なんー
\\	(お)いくら	
\\	どれくらい	
\\	覚ます (さます)	
\\	やっと	
\\	最後に (さいごに) 、ついに	
\\	眼、目 (め)	
\\	ーますかい、ーるかい	
\\	ーなのに	
\\	ーんだ -のです	
\\	合わせる (あわせる)	
\\	合わせて (あわせて)	
\\	やる	
\\	時間 (じかん)	
\\	唯一の (ゆいいつ)	
\\	時 (とき)	
\\	生まれる (うまれる)	
\\	昔から (むかし)	
\\	遥か(な) (はるか(な))	
\\	探す (さがす) 探し (さがし)	
\\	遅い (おそい)	
\\	身体 (からだ)	
\\	何でもない (なんでもない)	
\\	永遠に (えいえんに)	
\\	ずっと	
\\	まるで 
\\	(の)ようだ ・に 
\\	消える きえる	
\\	怒る (おこる)	
\\	少ない (すくない)	
\\	ふりがな	
\\	現象 (げんしょう)	
\\	手紙 (てがみ)	
\\	ーと申します (もうします)	
\\	住む (すむ) 、住んでいます	
\\	州 (しゅう)	
\\	念のために (ねんのために)	
\\	雨が降りそうです (あめがふりそうです)	
\\	悲しい (かなしい)	
\\	嬉しい (うれしい)	
\\	親 (おや)	
\\	閉じる (とじる)、閉める (しめる)	
\\	開く (ひらく)、開ける (あける)	
\\	閉まる (しまる)	
\\	-ほど	
\\	~と同じくらい(に)~	
\\	のように	
\\	飛ぶ 、飛ばす (飛ぶ、飛ばす)	
\\	分 (ふん)	
\\	に 
\\	行けるだけ	
\\	助ける (たすける)	
\\	一度 (いちど)	
\\	すごく	
\\	眼鏡 (めがね)	
\\	料理 (りょうり)	
\\	取る (とる)	
\\	箸 (はし)	
\\	また(ね)	
\\	それなら	
\\	付き合う (つきあう)	
\\	ただ(の)	
\\	関係 (関係)	
\\	見つめる	
\\	ー前に ー(まえに)	
\\	ー後に (-あとに)	
\\	成るべく (なるべく)	
\\	信じる (しんじる)	
\\	のための のために 
\\	無用 (むよう)	
\\	絶対に (ぜったいに)	
\\	―の意見では  (のいけんでは)	
\\	上がる (あがる)	
\end{CJK}
\end{document}