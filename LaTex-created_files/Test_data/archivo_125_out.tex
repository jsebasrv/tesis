\documentclass[8pt]{extreport} 
\usepackage{hyperref}
\usepackage{CJKutf8}
\begin{document}
\begin{CJK}{UTF8}{min}
\\	あなた は 先[せん] 生[せい] ですか?	あなた は せんせい ですか?	
\\	あなた 
\\	は 
\\	先[せん] 生[せい] 
\\	です 
\\	か 
\\	彼[かれ] は エンジニア[えんじにあ] です。	かれ は えんじにあ です。	
\\	かれ 
\\	は 
\\	エンジニア[えんじにあ] 
\\	です 
\\	あなた は 田[た] 中[なか]さん ですか。	あなた は たなかさん ですか?	
\\	あなた 
\\	は 
\\	さん 
\\	です 
\\	か 
\\	私[わたし] は 英[えい] 語[ご] の 先[せん] 生[せい] です。	わたし は えいご の せんせい です。	
\\	私[わたし] 
\\	は 
\\	英語[えいご] 
\\	の 
\\	の
\\	先生[せんせい] 
\\	です 
\\	私[わたし] は カ[か]ナ[な]ダ[だ] 人[じん] です。	わたし は かなだじん です。	
\\	私 
\\	は 
\\	カナダ人 
\\	(カナダ 
\\	人 
\\	です 
\\	私[わたし] は 木[き]村[むら] です。	わたし は きむら です。	
\\	私[わたし] 
\\	は 
\\	です 
\\	さん 
\\	私[わたし] は 日[に]本[ほん]語[ご] の 先[せん]生[せい] じゃありません。	わたし は にほんご の せんせい じゃありません。	
\\	私[わたし] 
\\	は 
\\	日本語[にほんご] 
\\	(日本=
\\	語=
\\	の 
\\	の
\\	先生[せんせい] 
\\	じゃ(でわ)ありません 
\\	私 は 田[た]中[なか] じゃありません。	わたし は たなか じゃありません。	
\\	私 
\\	は 
\\	じゃ(でわ)ありません 
\\	さん 
\\	私[わたし] は カ[か]ナ[な]ダ[だ] から 来[き]ました	わたし は かなだ から きました。	
\\	私[わたし] 
\\	は 
\\	から 
\\	から 
\\	来[き]る 
\\	[くる] 
\\	[き])	ました 
\\	田[た]中[なか]さん は 日[に]本[ほん]語[ご] の 先[せん]生[せい] ですか。	たなかさん は にほんご の せんせい ですか?	
\\	さん 
\\	は 
\\	日本語[にほんご] 
\\	(日本=
\\	語=
\\	の 
\\	の
\\	先生[せんせい] 
\\	ですか 
\\	彼女[かのじょ] は 学[がく]生[せい] です。	かのじょ は がくせい です。	
\\	かのじょ 
\\	は 
\\	学生[がくせい] 
\\	です 
\\	田[た]中[なか]さん も カ[か]ナ[な]ダ[だ]人[じん] です。	たなかさん も かなだじん です。	
\\	さん 
\\	も 
\\	も 
\\	カナダ人 
\\	(カナダ 
\\	人 
\\	です 
\\	あの 人[ひと] は だれ ですか?	あの ひと は だれ ですか?	
\\	あの 
\\	は 
\\	人[ひと] 
\\	だれ 
\\	ですか 
\\	これ は 傘[かさ] です。	これ は かさ です。	
\\	これ 
\\	は 
\\	です 
\\	あれ は 時[と]計[けい] です。	あれ は とけい です。	
\\	あれ 
\\	は 
\\	です 
\\	時計[とけい]		
\\	それ は 何[なん] ですか。	それ は なん ですか。	
\\	それ 
\\	は 
\\	何[なん] 
\\	ですか 
\\	その 本[ほん] は あなた の です。	その ほん は あなた の です。	
\\	その 
\\	は 
\\	あなた 
\\	の 
\\	です 
\\	それ は 私[わたし] の 帽子[ぼうし] です。	それ は わたし の ぼうし です。	
\\	それ 
\\	は 
\\	私[わたし] 
\\	の 
\\	です 
\\	それ は パ[ぱ]ト[と]リ[り]ッ[っ]ク[く]さん の です。	それ は ぱとりっくさん の です。	
\\	それ 
\\	は 
\\	さん 
\\	の 
\\	です 
\\	この 車[くるま] は 木村[きむら]さん の です。	この くるま は きむらさん の です。	
\\	この 
\\	は 
\\	さん 
\\	の 
\\	です 
\\	これ は ケ[け]イ[い]ト[と]リ[り]ン[ん]さん の ペン です。	これ は けいとりんさん の ペン です。	
\\	これ 
\\	は 
\\	さん 
\\	の 
\\	です 
\\	これ は 私[わたし] の です。	これ は わたし の です。	
\\	これ 
\\	は 
\\	私[わたし] 
\\	の 
\\	です 
\\	この 鍵[かぎ] は だれ の ですか。	この かぎ は だれ の ですか。	
\\	この 
\\	は 
\\	だれ 
\\	の 
\\	ですか 
\\	あれ は だれ の  カメラ[かめら] ですか。	あれ は だれ の かめら です か。	
\\	あれ 
\\	は 
\\	だれ 
\\	の 
\\	ですか 
\\	それ は 誰[だれ] の ですか。	それ は だれ の ですか。	
\\	それ 
\\	は 
\\	だれ 
\\	の 
\\	ですか 
\\	あなた の  鞄[かばん] は どれ ですか?	あなた の かばん は どれ ですか?	
\\	あなた 
\\	は 
\\	どれ 
\\	の 
\\	ですか 
\\	これ は あなた の  財布[さいふ] ですか?	これは あなた の さいふ ですか?	
\\	これ 
\\	は 
\\	あなた 
\\	の 
\\	ですか 
\\	今日[きょう] は  何曜日[なんようび] ですか?	きょう は なんようび ですか?	
\\	今日 
\\	(今 
\\	日 
\\	何曜日 
\\	(何 
\\	曜日 
\\	ですか 
\\	明日[あした] は  火曜日[かようび] です。	あした は かようび です。	
\\	明日 
\\	は 
\\	火曜日 
\\	です 
\\	明日[あした] は  何月[なんがつ]  何日[なんにち] ですか?	あした は なんがつ なんにち ですか?	
\\	明日 
\\	は 
\\	何月 
\\	(何 
\\	月 
\\	何日 
\\	(何 
\\	日 
\\	ですか 
\\	今日[きょう] は  四月[しがつ]  三日[みっか] です。	きょう は しがつ みっか です。	
\\	今日 
\\	は 
\\	四月 
\\	三日 
\\	です 
\\	今[いま]  何時[なんじ] ですか?	いま なんじ ですか?	
\\	今 
\\	何時 
\\	(何 
\\	時 
\\	ですか 
\\	今[いま]  八時[はちじ] です。	いま はちじ です。	
\\	今 
\\	八 
\\	時 
\\	です 
\\	あなた の  誕生日[たんじょうび] は  何月[なんがつ]  何日[なんにち] ですか?	あなた の たんじょうび は なんがつ なんにち ですか?	
\\	何月 
\\	(何 
\\	月 
\\	何日 
\\	(何 
\\	日 
\\	ですか 
\\	の 
\\	ですか 
\\	何時[なんじ] まで です か?	なんじ まで ですか?	
\\	何時 
\\	(何 
\\	時 
\\	まで 
\\	まで 
\\	ですか 
\\	何時[なんじ] から ですか?	なんじ から ですか?	
\\	何時 
\\	(何 
\\	時 
\\	から 
\\	から 
\\	ですか 
\\	私[わたし] は  三時半[さんじはん] まで 働[はたら]きます。	わたし は さんじはん まで はたらきます。	
\\	半 
\\	まで 
\\	まで 
\\	働く= 
\\	ます 
\\	この  レストラン[れすとらん] は  九時[くじ] から  三時[さんじ] まで です。	この れすとらん は くじ から さんじ まで です。	
\\	この 
\\	時 
\\	から 
\\	から 
\\	まで 
\\	まで 
\\	仕事[しごと] は  何時[なんじ] に  始[はじ]まりますか?	しごと は なんじ に はじまりますか?	
\\	何時[なんじ] 
\\	(何 
\\	時 
\\	に 
\\	始[はじ]まる 
\\	ます 
\\	か 
\\	仕事[しごと] は  何時[なんじ] に  終[お]わりますか?	しごと は なんじ に おわりますか?	
\\	何時 
\\	(何 
\\	時 
\\	に 
\\	終[お]わる 
\\	ます 
\\	か 
\\	私[わたし] は 七[しち]時[じ]十[じゅう]五[ご]分[ふん] に 起[お]きます。	わたし は しちじ じゅうごふん に おきます。	
\\	七時十五分 
\\	(七 
\\	時 
\\	十五 
\\	分 
\\	に 
\\	起[お]く 
\\	ます 
\\	私[わたし] は  十時[じゅじ] に  寝[ね]ません。	わたし は じゅじ に ねません。	
\\	十時 
\\	に 
\\	寝[ね]る 
\\	る)	ません 
\\	私[わたし] は  銀行[ぎんこう] へ  行[い]きます。	わたし は ぎんこう へ いきます。	
\\	へ 
\\	行[い]く 
\\	ます 
\\	あなた は どこ へ  行[い]きますか?	あなた は どこ へ いきますか?	
\\	へ 
\\	行く 
\\	ます 
\\	か 
\\	木村[きむら]さん は 病院[びょういん] へ  行[い]きます。	きむらさん は びょういん へ いきます。	
\\	へ 
\\	行く 
\\	ます 
\\	病院[びょういん] 
\\	私[わたし] は 電車[でんしゃ] で セ[せ]ン[ん]ター[たあ]へ 来[き]ます。	わたし は でんしゃ で せんたあ へ きます。	
\\	へ 
\\	来[く]る 
\\	[くる] 
\\	[き])	ます 
\\	電車[でんしゃ] 
\\	私[わたし]たち は 自転車[じてんしゃ] で 川崎駅[かわさきえき] へ 来[き]ます。	わたしたち は じてんしゃ で かわさきえき へ きます。	
\\	私たち 
\\	で 
\\	へ 
\\	来る[く] 
\\	[くる] 
\\	[き])	ます 
\\	あなた は 何[なん]で ここ へ 来[き]ますか?	あなた は なんで ここ へ きますか?	
\\	何で 
\\	へ 
\\	来る 
\\	[くる] 
\\	[き])	ます 
\\	か 
\\	私[わたし] は 来年[らいねん] 国[くに] へ 帰[かえ]ります。	わたし は らいねん くに へ かえります。	
\\	来年[らいねん] 
\\	へ 
\\	帰[かえ]る 
\\	ます 
\\	私[わたし] は 妻[つま] と 家[うち] へ 帰[かえ]ります。	わたし は つま と うち へ かえります。	
\\	と 
\\	へ 
\\	帰[かえ]る 
\\	ます 
\\	あなた は いつ 家[うち] へ 帰[かえ]りますか?	あなた は いつ うち へ かえります。	
\\	いつ 
\\	へ 
\\	帰[かえ]る 
\\	ます 
\\	か 
\\	私[わたし] は 川崎駅[かわさき] へ 電車[でんしゃ] で 行[い]きます。	わたし は かわさき へ でんしゃ で いきます。	
\\	で 
\\	へ 
\\	行[い]く 
\\	ます 
\\	私[わたし] は 家[うち] へ 飛行機[ひこうき] で 来[き]ません。	わたし は うち へ ひこうき で きません。	
\\	で 
\\	へ 
\\	来[く]る 
\\	[くる] 
\\	[き])			
\\	私[わたし] は 鶴見駅[つるみえき] から 潮田中学[うしおだちゅうがく] まで 歩[ある]きました。	わたし は つるみえき から うしおだちゅうがく まで あるきました。	
\\	歩[ある]く= 
\\	ました 
\\	から 
\\	から 
\\	まで 
\\	まで 
\\	私[わたし] は 一昨年[おととし] 日本[にほん] へ 来[き]ました。	わたし は おととし にほん へ きました。	
\\	一昨年[おととし] 
\\	へ 
\\	来る 
\\	[くる] 
\\	[き])	ました 
\\	あなた は いつ 日本[にほん] へ 来[き]ましたか?	あなた は いつ にほん へ きましたか?	
\\	いつ 
\\	へ 
\\	来る 
\\	[くる] 
\\	[き])	ました 
\\	か 
\\	あなた は 昨日[きのう] どこ へ 行[い]きましたか?	あなた は きのう どこ へ いきましたか?	
\\	どこ 
\\	へ 
\\	行[い]く 
\\	ました 
\\	か 
\\	私[わたし] は どこ も 行[い]きませんでした。	わたし は どこも いきませんでした。	
\\	どこも 
\\	行[い]く 
\\	ませんでした 
\\	私[わたし] は 本[ほん] を 読[よ]みます。	わたし は ほん を よみます。	
\\	を 
\\	読[よ]む 
\\	ます 
\\	私[わたし] は ゲー[げえ]ム[む] を しました。	わたし は げえむ を しました。	
\\	を 
\\	する 
\\	する 
\\	し)	ました 
\\	私[わたし] は コー[こお]ヒー[ひい] を 飲[の]みません。	わたし は コーヒー を のみません。	
\\	を 
\\	飲[の]む 
\\	ません 
\\	あなた は 日本語[にほんご] を 勉強[べんきょう]しますか?	あなた は にほんご を べんきょうしますか?	
\\	を 
\\	勉強[べんきょう]する 
\\	する 
\\	し)	ます 
\\	か 
\\	私[わたし] は 居酒屋[いざかや] で 焼き鳥[やきとり] を 食[た]べます。	わたし は いざかや で やきとり を たべます。	
\\	で 
\\	を 
\\	食[た]べる 
\\	る)	ます 
\\	ケ[け]イ[い]ト[と]リ[り]ン[ん]さん は コ[こ]ン[ん]ビ[び]ニ[に] で おにぎり を 買[か]います。	ケイトリンさん は こんびに で おにぎり を かいます。	
\\	で 
\\	を 
\\	買[か]う 
\\	ます 
\\	あなた は どこ で ビー[びい]ル[る] を 飲[の]みますか?	あなた は どこ で ビール を のみますか?	
\\	で 
\\	を 
\\	飲[の]む 
\\	ます 
\\	か 
\\	日本人[にほんじん] は 橋[はし] で ご 飯[はん] を 食[た]べます。	にほんじん は はし で ごはん を たべます。	
\\	で 
\\	を 
\\	食[た]べる 
\\	る)	ます 
\\	私[わたし] は はさみ で 髪[かみ] を 切[き]ります	わたし は はさみ で かみ を きります。	
\\	で 
\\	を 
\\	ます 
\\	髪[かみ] 
\\	切[き]る 
\\	あなた は 何語[なにご] で 手紙[てがみ] を 書[か]きますか?	あなた は なにご で てがみ を かきますか?	
\\	で 
\\	を 
\\	書[か]く 
\\	ます 
\\	か 
\\	家[うち] で 日本語[にほんご] の 映画[えいが] を 見[み]ました。	うち で にほんご の えいが を みました。	
\\	で 
\\	を 
\\	見[み]る 
\\	る)	ました 
\\	毎朝[まいあさ] ケ[け]イ[い]ト[と]リ[り]ン[ん] は コー[こお]ヒー[ひい] を 買[か]います。	まいあさ けいとりんさん は こおひい を かいます。	
\\	毎朝[まいあさ] 
\\	(毎 
\\	朝 
\\	を 
\\	買[か]う 
\\	ます 
\\	学生[がくせい] が います。	がくせい が います。	
\\	が 
\\	いる 
\\	る)	ます 
\\	公園[こうえん] に 犬[いぬ] が います。	こうえん に いぬ が います。	
\\	に 
\\	いる 
\\	る)	ます 
\\	ここ に だれも が いません。	ここ に だれも が いません。	
\\	に 
\\	ここ 
\\	いる 
\\	る)	ません 
\\	だれも 
\\	そこ に 韓国[かんこく]の レ[れ]ス[す]ト[と]ラ[ら]ン[ん] が あります。	そこ に かんこく の れすとらん が あります。	
\\	に 
\\	そこ 
\\	(そこ 
\\	ある 
\\	ます 
\\	韓国[かんこく]の 
\\	(韓国[かんこく] 
\\	の 
\\	机[つくえ]の 上[うえ] に パ[ぱ]ス[す]コ[こ]ン[ん] が あります。	つくえ の うえ に ぱすこん が あります。	
\\	の 
\\	の 
\\	に 
\\	が 
\\	ある 
\\	ます 
\\	上[うえ] 
\\	木[き] の 下[した] に 猫[ねこ] が います。	き の した に ねこ が います。	
\\	の 
\\	の 
\\	に 
\\	が 
\\	いる 
\\	る)	ます 
\\	下[した] 
\\	冷蔵庫[れいぞうこ] の 中[なか] に ビー[びい]ル[る] が あります。	れいぞうこ の なか に びいる が あります。	
\\	の 
\\	の 
\\	に 
\\	が 
\\	ある 
\\	ます 
\\	家[いえ] の 前[まえ] に 車[くるま] が あります。	いえ の まえ に くるま が あります。	
\\	の 
\\	の 
\\	に 
\\	が 
\\	ある 
\\	ます 
\\	前[まえ] 
\\	本屋[ほんや] の 後[うし]ろ に 花屋[はなや] が あります。	ほんや の うしろ に はなや が あります。	
\\	の 
\\	の 
\\	に 
\\	が 
\\	ある 
\\	ます 
\\	後[うし]ろ 
\\	猫[ねこ] と 犬[いぬ] の 間[あいだ] に 鼠[ねずみ] が います。	ねこ と いぬ の あいだ に ねずみ が います。	
\\	の 
\\	の 
\\	に 
\\	が 
\\	いる 
\\	る)	ます 
\\	間[あいだ] 
\\	川崎大師[かわさきだいし] の 近[ちか]く に 私[わたし] の 家[いえ] が あります。	かわさきだいし の ちかく に わたしの いえ が あります。	
\\	の 
\\	の 
\\	に 
\\	が 
\\	ある 
\\	ます 
\\	私[わたし] の 隣[となり] に パ[ぱ]ト[と]リ[り]ッ[っ]ク[く] が います。	わたし の となり に パトリック が います。	
\\	の 
\\	の 
\\	に 
\\	が 
\\	いる 
\\	る)	ます 
\\	の 右[みぎ] に 
\\	が あります。	
\\	の みぎ に 
\\	が あります。	
\\	の 
\\	の 
\\	に 
\\	が 
\\	ある 
\\	ます 
\\	右[みぎ] 
\\	の 左[ひだり] に 
\\	が あります。	
\\	の ひだり に 
\\	が あります。	
\\	の 
\\	の 
\\	に 
\\	が 
\\	ある 
\\	ます 
\\	左[ひだり] 
\\	富士山[ふじさん] は 高[たか]い です。	ふじさん は たかい です。	
\\	高[たか]いい 
\\	です。	富士山[ふじさん] 
\\	日本語[にほんご] は 難[むすか]しいです。	にほんご は むすかしい です。	
\\	難[むすか]しい 
\\	です。	日本語[にほんご] 
\\	この クラス は 楽[たの]しくないです。	この クラス は たのし くない です。	
\\	この 
\\	楽[たの]しい 
\\	'い
\\	くない です。			
\\	今日[きょう] は 暑[あつ]くないです。	きょう は あつ くない です。	
\\	今日[きょう] 
\\	暑[あつ]くい 
\\	'い
\\	くない です。			
\\	桜[さくら] は 綺麗[きれい]な です。	さくら は きれい です。	
\\	綺麗[きれい]な 
\\	'な
\\	です。	桜[さくら] 
\\	横浜駅[よこはまえき] は 賑[にぎ]やかです。	よこはまえき は にぎやか です。	
\\	賑[にぎ]やかな 
\\	'な
\\	です。				
\\	私[わたし] は 朝[あさ] 元気[げんき]じゃありません。	わたし は あさ げんき じゃありません。	
\\	元気[げんき]な 
\\	'な
\\	じゃありません。	朝[あさ] 
\\	川崎[かわさき] は 便利[べんり]じゃありません。	かわさき は べんり じゃありません。	
\\	便利[[べんり]]な 
\\	'な
\\	じゃありません。				
\\	これ は 思[おも]い 鞄[かばん] です。	これ は おもい かばん です。	
\\	思[おも]い 
\\	な/い	鞄[かばん] 
\\	キ[き]ウィ[うぃ] は 可愛[かわい]い 猫[ねこ]です。	きうぃ は かわいい ねこ です。	
\\	可愛[かわい]い 
\\	な/い	猫[ねこ] 
\\	京都[きょうと] は 静[しず]かな 町[まち] です。	きょうと は しずかな まち です。	
\\	静かな 
\\	な/い	町[まち] 
\\	寿司[すし] は 有名[ゆうめい]な 日本[にほん] の  食[た]べ物[もの] です。	すし は ゆうめいな にほん の たべもの です。	
\\	有名[ゆうめい]な 
\\	な/い	食[た]べ物[もの] 
\\	(食[た]べ 
\\	物[もの] 
\\	漢字[かんじ] は 優[やさ]しい ですか?	かんじ は やさしい です か?	
\\	優しい 
\\	です。	か 
\\	あなた の 奥[おく]さん は 異例[きれい] ですか?	あなた の おくさん は きれい ですか?	
\\	異例[きれい]な 
\\	'な
\\	です。	か 
\\	美味[おい]しかったです。	おいしかった です。	
\\	味しい 
\\	い, 
\\	かった です。				
\\	映画[えいが] は 面白[おもしろ]かった です。	えいが は おもしろかった です。	
\\	面白[おもしろ]い 
\\	い, 
\\	かった です。				
\\	祭[まつ]り は 賑[にぎ]やか でした。	まつり は にぎやか でした。	
\\	賑[にぎ]やかな 
\\	な, 
\\	でした。	祭[まつ]り 
\\	水[みず] は 寒[さむ]くなかった です。	みず は さむくなかった です。	
\\	寒い 
\\	い, 
\\	くなかった です。	水[みず] 
\\	パーティ は 楽[たの]しくなかった です。	パーティ は たのしくなかった です。	
\\	楽[たの]しい 
\\	い, 
\\	くなかった です。				
\\	私[わたし] は 今朝[けさ] 元気[げんき] じゃありません でした。	わたし は けさ げんき じゃありませんでした。	
\\	元気[げんき]な 
\\	な, 
\\	じゃありませんでした。	今朝[けさ] 
\\	(今 
\\	朝 
\\	鈴木[すずき] は 水木[みずき] に 花[はな] を 上[あ]げます。	すずき は みずき に はな を あげます。	
\\	を 
\\	に 
\\	上げる 
\\	ます 
\\	花[はな] 
\\	私[わたし] は 美玖[みか]ちゃん  に チョ[ちょ]コ[こ]レー[れえ]ト[と] を 上[あ]げました。	わたし は みかちゃん に チョコレート を あげました。	
\\	は 
\\	を 
\\	上げる 
\\	ました 
\\	水木[みずき] は 鈴木[すずき] に(から) 花[はな] を 貰[もら]います。	みずき は すずき に(から) はな を もらいます。	
\\	は 
\\	から 
\\	から 
\\	貰う 
\\	ます 
\\	花[はな] 
\\	美玖[みく]ちゃん は 私[わたし] に(から) チョ[ちょ]コ[こ]レー[れえ]ト[と] を 貰[もら]いました。	みくちゃん は わたし に(から) ちょこれえと を もらいました。	
\\	は 
\\	から 
\\	から 
\\	貰[もら]う 
\\	ました 
\\	母[はは] は 私[わたし] に お 金[かね] を くれました。	はは は わたし に おかね を くれました。	
\\	は 
\\	に 
\\	くれる 
\\	る)	ます 
\\	伊勢[いせ] は 美玖[みく] に(から)鉛筆[えんぴつ] を 借[か]ります。	いせ は みく に(から) えんぴつ を かります。	
\\	は 
\\	から 
\\	から 
\\	借[か]りる 
\\	る)	ます 
\\	伊[い]勢[せ] は 美[み]玖[く] に 消[け]し ゴ[ご]ム[む] を 貸[か]します。	いせ は みく に けしごむ を かします。	
\\	は 
\\	に 
\\	貸す 
\\	る)	ます 
\\	消[け]しゴム 
\\	(消[け] 
\\	ゴム 
\\	私[わたし] は 車[くるま] が 欲[ほ]しいです。	わたし は くるま が ほしいです。	
\\	が 
\\	欲しい 
\\	車[くるま] 
\\	私[わたし] は 新[あたら]しい カ[か]メ[め]ラ[ら] が 欲[ほ]しいです。	わたし は あたらしい かめら が ほしい です。	
\\	が 
\\	新しい 
\\	な/い	欲しい 
\\	あなた は 何[なに] が 欲[ほ]しい ですか?	あなた は なに が ほしい ですか?	
\\	が 
\\	欲しい 
\\	か 
\\	私[わたし] は 本[ほん] を 読[よ]みたい です。	わたし は ほん を よみたい です。	
\\	を 
\\	読む 
\\	たい 
\\	私[わたし] は 納豆[なっと] が 食[た]べたくない です。	わたし は なっと が たべたくない です。	
\\	が 
\\	食べる 
\\	る)	
\\	たくない 
\\	納豆[なっと] 
\\	私[わたし] は 何[なに]も したくな いです。	わたし は なにも したくない です。	
\\	何も 
\\	する 
\\	する 
\\	し)	
\\	たくない 
\\	私[わたし] は 明日[あした] 試験[しけん] が あります。	わたし は あした しけん が あります。	
\\	が 
\\	ある 
\\	ます 
\\	明日[あした] 
\\	試験[しけん] 
\\	私[わたし] は 用事[しけん] が ありません。	わたし は しけん が ありません。	
\\	が 
\\	ある 
\\	ます 
\\	用事[しけん] 
\\	あなた は 時間[じかん] が ありますか?	あなた は じかん が ありますか?	
\\	が 
\\	ある 
\\	ます 
\\	か 
\\	時間[じかん] 
\\	(時 
\\	間 
\\	犬[いぬ] が 七匹[ななひき] います。	いぬ が ななひき います。	
\\	が 
\\	匹 
\\	いる 
\\	る)	ます 
\\	事務室[じむしつ] に 学生[がくせい] が 二人[ふたり] います。	じむしつ に がくせい が ふたり います。	
\\	が 
\\	に 
\\	人 
\\	いる 
\\	る)	ます 
\\	私[わたし] は 焼[や]き 鳥[とり] が 四本[よんほん] 買[か]います。	わたし は やきとり が よんほん かいます。	
\\	が 
\\	本 
\\	買う 
\\	ます 
\\	私[わたし] は ビー[びい]ル[る] が 三倍[さんばい] 飲[の]みます。	わたし は びいる が さんばい のみます。	
\\	が 
\\	倍 
\\	のむ 
\\	ます 
\\	私[わたし] は  ア[あ]ニ[に]メ[め] を  五時間[ごじかん]  見[み]ました。	わたし は あにめ を ごじかん みました。	
\\	を 
\\	時間 
\\	見る 
\\	る)	ました 
\\	私[わたし] は  日本語[にほんご] を  一週間[いっしゅうかん] に  一回[いっかい]  勉強[べんきょう]します。	わたし は にほんご を いっしゅうかん に いっかい べんきょうします。	
\\	を 
\\	に 
\\	回 
\\	勉強する 
\\	勉強します 
\\	する 
\\	し)	ます 
\\	永井[ながい] は  一年[いちねん] に  一[いっ]回[かい]  沖縄[おきなわ] へ  行[い]きました。	ながい は いちねん に いっかい おきなわ へ いきました。	
\\	に 
\\	へ 
\\	回 
\\	行く 
\\	ます 
\\	私[わたし] は  寿司[すし] が  好[す]き です。	わたし は すし が すき です。	
\\	が 
\\	好きな 
\\	'な
\\	です。	
\\	私[わたし] は  野菜[やさい] が  嫌[きら]い です。	わたし は やさい が きらい です。	
\\	が 
\\	嫌い 
\\	です。	野菜[やさい] 
\\	諭介[ゆすけ] は  歌[うた] が  上手[じょうず] です。	ゆすけ は うた が じょうず です。	
\\	が 
\\	上手な 
\\	'な
\\	です。	歌[うた] 
\\	陽向[ひなた] は  料理[りょうり] が 下手[へた]です。	ひなた は りょうり が へた です。	
\\	が 
\\	下手な 
\\	'な
\\	です。	料理[りょうり] 
\\	恵美[えみ] は 制[せい] が 高[た]いです。	えみ は せい が たかい です。	
\\	が 
\\	高い 
\\	です。	制[せい] 
\\	港[みなと] は 目[め] が 大[おお]きい です。	みなと は め が おおきい です。	
\\	が 
\\	大きい 
\\	です。	目[め] 
\\	私[わたし] は 英語[えいご] が 分[わか]かります。	わたし は えいご が わかります。	
\\	が 
\\	分かる 
\\	ます 
\\	英語[えいご] 
\\	私[わたし] はひらがな が よく 分[わ]かります。	わたし は ひらがな が よく わかります。	
\\	が 
\\	分かる 
\\	ます 
\\	私[わたし] は カ[か]タ[た]カ[か]ナ[な] が 大体[だいたい] 分[わ]かります。	わたし は かたかな が だいたい わかります。	
\\	が 
\\	分かる 
\\	ます 
\\	体[だいたい] 
\\	私[わたし] は 日本語[にほんご] が 少[すこ]し 分[わ]かります。	わたし は にほんご が すこし わかります。	
\\	が 
\\	分かる 
\\	ます 
\\	少[すこ]し 
\\	私[わたし] は 漢字[かんじ] が あまり 分[わ]かりません。	わたし は かんじ が あまり わかりません。	
\\	が 
\\	分かる 
\\	ません 
\\	あまり 
\\	私[わたし] は 中国語[ちゅうごくご] が 全然[ぜんぜん] 分[わ]かりません。	わたし は ちゅうごくご が ぜんぜん わかりません。	
\\	が 
\\	分かる 
\\	ません 
\\	全然[ぜんぜん] 
\\	中国語[ちゅうごくご] 
\\	一緒[いしょ]に 餃子[ぎょうざ] を 作[つく]りませんか?	いしょに ぎょうざ を つくりませんか?	
\\	を 
\\	作る 
\\	ません 
\\	か 
\\	ええ、 餃子[ぎょうざ] を 作[つく]りましょう!	ええ、ぎょうざ を つくりましょう!	
\\	を 
\\	ええ 
\\	作る 
\\	ましょう 
\\	か 
\\	一緒[いっしょ]に この 歌[うた] を 歌[うた]いませんか?	いっしょ に この うた を うたいませんか?	
\\	を 
\\	歌う 
\\	ません 
\\	か 
\\	この 歌[うた] ですか?この 歌[うた]は ちょっと。。。	この うた ですか? この うた は ちょっと。。。	
\\	この 
\\	か 
\\	ちょうと 
\\	北海道[ほっかいどう] は 寒[さむ]い です から、コート が 要[い]ります。	ほっかいどう は さむい です から、コート が いります。	
\\	が 
\\	寒い 
\\	です。	から 
\\	から 
\\	お金[かね] が ありません から、 ハ[は]ワ[わ]イ[い] へ 行[い]きません。	おかね が ありません から、はわい へ いきません。	
\\	が 
\\	ある 
\\	行く 
\\	ません 
\\	から 
\\	から 
\\	どうして 先週[せんしゅう] 休[やす]み ましたか?	どうして せんしゅう やすみましたか?	
\\	どうして 
\\	先週 
\\	(先 
\\	週 
\\	休む 
\\	ました 
\\	か 
\\	熱[ねつ] が ありました から。	ねつ が ありました から。	
\\	が 
\\	ある 
\\	ました 
\\	から 
\\	から 
\\	豚肉[ぶたにく] は 牛肉[ぎゅにく] より 安[やす]いです。	ぶたにく は ぎゅにく より やすい です。	
\\	より 
\\	より 
\\	安い 
\\	です。			
\\	土曜日[どようび] は 水曜日[すいようび] より 暇[ひま]です。	どようび は すいようび より ひま です。	
\\	より 
\\	より 
\\	暇な 
\\	'な
\\	です。			
\\	赤[あか]ワイン と 白[しろ]ワイン と どっち が 好[す]き ですか?	あかわいん と しろわいん と どっち が すきですか?	
\\	と 
\\	どちら 
\\	どっち 
\\	好きな 
\\	'な
\\	です。	か 
\\	と 
\\	どっちも 好[す]きです。	どっちも すき です。	
\\	どっちも 
\\	好きな 
\\	'な
\\	です。			
\\	どっちも 好[す]き じゃありません。	どっちも すき じゃありません。	
\\	どっちも 
\\	好きな 
\\	な, 
\\	じゃ(でわ)ありません	
\\	川崎[かわさき] と 東京[とうきょう] と どちら が 人[ひと]が 少[すく]ない ですか?	かわさき と とうきょう と どちら が ひとがすくない ですか?	
\\	と 
\\	どちら 
\\	どっち 
\\	少ない 
\\	です。	か 
\\	と 
\\	川崎[かわさき] のほうが 人[ひと]が 少[すく]ない です。	かわさき のほうが ひと が すくない です。	
\\	のほうが 
\\	のほうが 
\\	少ない 
\\	です。	
\\	これ のほうが 良[い]いです。	これ のほうが いい です。	
\\	のほうが 
\\	のほうが 
\\	良い 
\\	です。	
\\	猫[ねこ] が 一番[いちばん] 可愛[かわいい] いです。	ねこ が いちばん かわいい です。	
\\	一番 
\\	(一 
\\	番 
\\	可愛い 
\\	です。	猫[ねこ] 
\\	漢字[かんじ] が 一番[いちばん] 難[むず]しいです。	かんじ が いちばん むずかしい です。	
\\	一番 
\\	(一 
\\	番 
\\	難しい 
\\	です。			
\\	乗[の]り 物[もの] で 何[なに] が 一番[いちばん] 早[はや]い ですか?	のりもの で なに が いちばん はやい ですか?	
\\	一番 
\\	(一 
\\	番 
\\	早い 
\\	です。	か 
\\	何[なに] 
\\	家族[かぞく] で 誰[だれ] が 一番[いちばん] 性[せい] が 高[た]い ですか?	かぞく で だれ が いちばん せい が たかい ですか?	
\\	一番 
\\	(一 
\\	番 
\\	高い 
\\	です。	か 
\\	誰[だれ] 
\\	日本[にほん] で どこ が 一番[いちばん] 寒[さ]いですか?	にほん で どこ が いちばん さむい ですか?	
\\	一番 
\\	(一 
\\	番 
\\	寒い 
\\	です。	か 
\\	どこ 
\\	一週間[いちしゅうかん] で いつ が 一番[いちばん] 熱[あつ]い ですか?	いちしゅうかん で いつか いちばん あつい ですか?	
\\	一番 
\\	(一 
\\	番 
\\	熱い 
\\	です。	か 
\\	いつ 
\\	一週間[いちしゅうかん 
\\	(一 
\\	週 
\\	間 
\\	月曜日[げつようび] と 木曜日[もくようび] と 土曜日[どようび] で どれ が 一番[いちばん] 都合[つごう] が 良[い]いですか?	げつようび と もくようび と どようび で どれ が いちばん つごう が いい ですか?	
\\	一番 
\\	(一 
\\	番 
\\	都合 
\\	良い 
\\	です。	どれ 
\\	と 
\\	で 
\\	ルー[るう]フィ[ふぃ] と イ[い]チ[ち]ゴ[ご] と ナ[な]ル[る]ト[と] で 誰[だれ] が 一番[いちばん] 強[つよ]いですか?	るうふぃ と いちご と なると で だれ が いちばん つよい ですか?	
\\	一番 
\\	(一 
\\	番 
\\	強い 
\\	です。	か 
\\	と 
\\	で 
\\	どれ 
\\	座[す]って 下[くだ]さい。	すわって ください。	
\\	座る 
\\	て 
\\	[る/つ/う] 
\\	[って])	下さい 
\\	*て 
\\	急[おそ]いで 下[くだ]さい。	おそいで ください。	
\\	急ぐ 
\\	て 
\\	[く/ぐ] 
\\	[いて/いで])	下さい 
\\	*て 
\\	ゆっくり 話[はな]して 下[くだ]さい。	ゆっくり はなして ください。	
\\	話す 
\\	て 
\\	[す] 
\\	[して])	下さい 
\\	*て 
\\	私[わたし] は シャ[しゃ]ワー[わあ] を 浴[あ]びて います。	わたし は しゃわあ を あびて います。	
\\	浴びる 
\\	る, 
\\	て)	*~ています 
\\	彼[かれ] は 本[ほん] を 読[よ]んで います。	かれ は ほん を よんで います。	
\\	読む 
\\	て 
\\	[む/ぶ/ぬ] 
\\	[んで])	*~ています 
\\	彼[かれ] が 私[わたし] の 家[うち] へ 来[き]て います。	かれ は わたし の うち へ きて います。	
\\	くる 
\\	て 
\\	[くる] 
\\	[きて])	*~ています 
\\	私[わたし] は 日本語[にほんご] を 勉強[べんきょう]して います。	わたし は にほんご を べんきょうして います。	
\\	勉強する 
\\	勉強します 
\\	て 
\\	[する] 
\\	[して])	*~ています 
\\	日本語[にほんご] 
\\	スープ の 作[つく]り 方[かた] が 分[わ]かります。	すーぷ の つくり かた が わかります。	
\\	作る 
\\	方 
\\	の
\\	方 
\\	分かる 
\\	ます 
\\	漢字[かんじ] の 読[よ]み 方[かた] が 分[わ]かりません。	かんじ の よみ かた が わかりません。	
\\	読む 
\\	方 
\\	の
\\	方 
\\	分かる 
\\	ます 
\\	丸[まる] は 眠[ねむ]って います	まる は ねむって います。	
\\	眠る 
\\	て 
\\	[る/つ/う] 
\\	[って])	*~ています 
\\	私[わたし] は 結婚[けっこん]して いません。	わたし は けっこんして いません。	
\\	結婚する 
\\	て 
\\	[する] 
\\	[して])	*~ています 
\\	高校[こうこう] で 働[はたら]いて います	こうこう で はたらいて います。	
\\	働く 
\\	て 
\\	[く/ぐ] 
\\	[いて/いで])	*~ています 
\\	私[わたし] は 川崎[かわさき] に 住[す]んで います。	わたし は かわさき に すんで います。	
\\	住む 
\\	て 
\\	[む/ぶ/ぬ] 
\\	[んで])	*~ています 
\\	自転車[じてんしゃ] を 持[も]って います。	じてんしゃ を もって います。	
\\	持つ 
\\	て 
\\	[る/つ/う] 
\\	[って])	*~ています 
\\	インタラック の 電話番号[でんわばんごう] を 知[し]って いますか?	いんたらっく の でんわばんごう を しって いますか?	
\\	知る 
\\	て 
\\	[る/つ/う] 
\\	[って])	*~ています 
\\	はい、 知[し]っています。	はい、しって います。	
\\	知る 
\\	て 
\\	[る/つ/う] 
\\	[って])	*~ています 
\\	いいえ、  知[し]りません。	いいえ、 しりまえん。	
\\	知る 
\\	て 
\\	[る/つ/う] 
\\	[って])	
\\	知る 
\\	分かる 
\\	ていません 
\\	若命さん[わかめ] の 息子[むすこ] は 幼稚園[ようちえん] に 通[かよ]って います。	わかめさん の むすこ は ようちえん に かよって います。	
\\	通る 
\\	て 
\\	[る/つ/う] 
\\	[って])	*~ています 
\\	息子[むすこ] 
\\	ポ[ぽ]ケ[け]モ[も]ン[ん]セ[せ]ン[ん]ター[たあ] で  ポ[ぽ]ケ[け]モ[も]ン[ん] の ぬいぐる み を 売[う]っています。	ぽけもんせんたあ で ぽけもん の ぬいぐるみ を うって います。	
\\	売る 
\\	て 
\\	[る/つ/う] 
\\	[って])	*~ています 
\\	私 は 毎朝[まいあさ] 顔[かお] を 洗[あら]って、 歯[は] を 磨[みが]いて、 服[ふく] を 着[き]ります。	わたし は まいあさ かお を あらって、は を みがいて、ふく を きります。	
\\	洗う 
\\	て 
\\	[る/つ/う] 
\\	[って])	磨く 
\\	て 
\\	[く/ぐ] 
\\	[いて/いで])	着る 
\\	*て,て,~ます。 
\\	水曜日[すいようび] 元住吉駅[もとすみよしえき] へ 行[い]って、 晩御飯[ばんごはん] を 食[た]べて から、 日本語[にほんご] を 勉強[べんきょう]します。	すいようび もとすみよし へ いって、ばんごはん を たべて、 にほんご を べんきょうします。	
\\	行く 
\\	て 
\\	[く] 
\\	[って])	食べる 
\\	て 
\\	る, 
\\	て)	勉強する 
\\	勉強します 
\\	て 
\\	[する] 
\\	[して])	*て,て,~ます。 
\\	家[うち] へ 帰[かえ]って から、アイスクリーム を 食[た]べます。	うち へ かえって から、 あいすくりいむ を たべます。	
\\	へ 
\\	帰る 
\\	て 
\\	[る/つ/う] 
\\	[って])	てから, 
\\	*て、てから、~ます。 
\\	昨日[きのう] 
\\	を 見[み]てから、 寝[ね]ました。	きのう 
\\	を みて から、 ねました。	
\\	見る 
\\	て 
\\	る, 
\\	て)	寝る 
\\	る)	ました 
\\	てから, 
\\	鈴木著駅[すずきちょえき] で 電車[でんしゃ] に 乗[の]って、 川崎駅[かわさきえき] で 乗[の]り 換[か]えて から、  鶴見駅[つるみえき] で 降[お]ります。	すずきちょえき で でんしゃ に のって、かわさきえき で のりかえて から、つるみえき に おります。	
\\	乗る 
\\	て 
\\	[る/つ/う] 
\\	[って])	乗り換える 
\\	て 
\\	る 
\\	て)	降りる 
\\	る)	*て、てから、~ます。 
\\	七時半[しちじはん] に 起[お]きて、 歯[は] を 磨[みが]いて から、 仕事[しごと] へ 行[い]きます。	しちじはん に おきて、は を みがいて から、しごと へ いきます。	
\\	起きる 
\\	て 
\\	る 
\\	て)	磨く 
\\	て 
\\	[く/ぐ] 
\\	[いて/いで])	仕事 
\\	行く 
\\	半 
\\	おにぎり は 安[やす]くて、 美味[おい]しい です。	おにぎり は やすくて、おいしい です。	
\\	安い 
\\	-て 
\\	[い] 
\\	[くて]	美味しい 
\\	*て/で, です 
\\	この ゲーム は 楽[たの]しくて、 簡単[かんたん] です。	この ゲーム は たのしくて、かんたん です。	
\\	楽しい 
\\	て 
\\	[い] 
\\	[くて]	簡単[かんたん]な 
\\	*て/で, です 
\\	この ゲーム は 簡単[かんたん] で、 楽[たの]しいです。	この ゲーム は かんたん で、たのしい です。	
\\	簡単な 
\\	て 
\\	[な] 
\\	[で]	楽しい 
\\	*て/で, です 
\\	日本人[にほんじん] は 丁寧[ていねい]で、 親切[しんせつ]です。	にほんじん は ていねい で、しんせつ です。	
\\	丁寧[ていねい]な 
\\	て 
\\	[な] 
\\	[で]	親切[しんせつ]な 
\\	*て/で, です 
\\	私[わたし] は カナダ 人[じん] で、 英語[えいご] の 先生[せんせい] です。	わたし は かなだ じん で、えいご の せんせい です。	
\\	*て/で, です 
\\	英語[えいご] の 先生[せんせい] 
\\	人[じん] 
\\	焼肉[やきにく] は 美味[おい]しい ですが、 高[たか]いです。	やきにく は おいしい ですが、たかい です。	
\\	美味しい 
\\	高い 
\\	が 
\\	焼肉[やきにく] 
\\	この ゲーム は 簡単[かんたん] ですが、 詰[つ]まらない です。	この ゲーム は かんたん ですが、つまらない です。	
\\	親切[かんたん]な 
\\	詰[つ]まらない 
\\	が 
\\	写真[しゃしん] を 撮[と]って もいいです。	しゃしん を とって もいいです。	
\\	を 
\\	撮[と]る 
\\	て 
\\	[る/つ/う] 
\\	[って])	
\\	て もいいです 
\\	写真[しゃしん] 
\\	ペット を 飼[か]って もいいです。	ぺっと を かって もいいです。	
\\	を 
\\	飼う 
\\	て 
\\	[る/つ/う] 
\\	[って])	
\\	て もいいです 
\\	自転車[じてんしゃ] を 止[と]めて もいいですか?	じてんしゃ を とめて もいいですか?	
\\	を 
\\	止める 
\\	て 
\\	る 
\\	て)	
\\	て もいいです 
\\	自転車[じてんしゃ] 
\\	ここ で タ[た]バ[ば]コ[こ] を 吸[す]って は いけません。(禁煙[きんえん])	ここ で たばこ を すって はいけません。(きんえん)	
\\	吸う 
\\	て 
\\	[る/つ/う] 
\\	[って])	
\\	は いけません 
\\	ここ で 入[はい]って は いけません。 (立入り禁止[たちいりきんし])	ここ で はいって は いけません。(たちいりきんし)	
\\	入る 
\\	て 
\\	[る/つ/う] 
\\	[って])	
\\	は いけません 
\\	ここ で 止[と]めて は いけません。 (駐車禁止[ちゅうしゃきんし])	ここ で とめて は いけません。(ちゅうしゃきんし)	
\\	止める 
\\	て 
\\	る 
\\	て)	
\\	は いけません 
\\	ここ で 写真[しゃしん] を 撮[と]って は いけません。 (撮影禁止[さつえいきんし])	ここ で しゃしん を とって は いけません。(さつえいきんし)	
\\	撮[と]る 
\\	て 
\\	[る/つ/う] 
\\	[って])	
\\	は いけません 
\\	写真[しゃしん] 
\\	押[おさ]さない で下さい。	おさない でください。	
\\	押す 
\\	ない 
\\	[ない])	
\\	ない でください 
\\	私[わたし] の カメラ を 触[さわ]らない で 下[くだ]さい。	わたし の かめら を さわらない でください。	
\\	触[さわ]る 
\\	ない 
\\	[ない])	
\\	ない でください 
\\	遅刻[ちこく]しない で下さい。	ちこくしない でください。	
\\	遅刻する 
\\	[する] 
\\	[しない])	
\\	ない でください 
\\	忘[わす]れないで下さい。	わすれない でください。	
\\	忘[わす]れる 
\\	ない 
\\	[る], 
\\	[ない])	
\\	ない でください 
\\	心配[しんぱい]しない で 下[くだ]さい。	しんぱいしない でください。	
\\	心配[しんぱい]する 
\\	[する] 
\\	[しない])	
\\	ない でください 
\\	行[い]きましょうか?	いきましょうか?	
\\	行く 
\\	ましょう 
\\	寒[さむ]いです。 窓[まど] を 閉[し]めましょうか?	さむいです。まど を しめましょうか?	
\\	寒い 
\\	閉める 
\\	ましょう 
\\	はい、お 願[ねが]いします。	はい、おねがいします。	
\\	いいえ、 結構[けっこう]です。ありがとうございます。	いいえ、けっこう です。ありがとうございます。	
\\	結構[けっこう] 
\\	結構です 
\\	ありがとうございます.				
\\	払[はら]わな ければなりません。	はらわな ければなりません。	
\\	払[はら]わる 
\\	ない 
\\	[な])	
\\	な ければなりません 
\\	パスポート を 見[み]せな ければなりません。	パスポート を みせな ければなりません。	
\\	見せる 
\\	ない 
\\	[る], 
\\	[な])	
\\	な ければなりません 
\\	信号[しんごう] が 赤[あか]ですから、 車[くるま] を 止[と]まらな ければなりません。	しんごう が あか です から、くるま を とまらな ければなりません。	
\\	止まる 
\\	ない 
\\	[な])	から 
\\	から 
\\	な ければなりません 
\\	子供[こども] は 払[はら]わな くてもいいです。	こども は はらわな くてもいいです。	
\\	払わる 
\\	ない 
\\	[な])	
\\	な くてもいいです 
\\	子供[こども] 
\\	心配[しんぱい]しな くてもいいです。	しんぱいしな くてもいいです。	
\\	心配[しんぱい]する 
\\	[する] 
\\	[しない])	
\\	な くてもいいです 
\\	時間[じかん] が ありますから、 急[いそ]がなくて もいいです。	じかん が あります から、いそがな くてもいいです。	
\\	時間[じかん] 
\\	から 
\\	から 
\\	急[いそ]ぐ 
\\	ない 
\\	[な])	
\\	な くてもいいです 
\\	私[わたし] の 趣味[しゅみ] は 旅行[りょこう]する こと です。	わたし の しゅみ は りょこうする こと です。	
\\	趣味[しゅみ] 
\\	旅行[りょこう]する 
\\	こと 
\\	私[わたし] の 趣味[しゅみ] は フルート を 吹[ふ]く こと です。	わたし の しゅみ は フルート を ふく こと です。	
\\	趣味[しゅみ] 
\\	吹[ふ]く 
\\	こと 
\\	あなた の 趣味[しゅみ] は 何[なん]ですか?	あなた の しゅみ は なんですか?	
\\	趣味 
\\	何[なん] 
\\	食事[しょくじ] の 前[まえ]に、 手[て] を 洗[あら]います。	しょくじ の まえ に て を あらいます。	
\\	食事[しょくじ] 
\\	前[まえ] 
\\	洗[あら]う 
\\	の前に 
\\	寝[ね]る 前[まえ]に、 何[なに] を しますか?	ねる まえ に、なに を しますか?	
\\	寝[ね]る 
\\	前[まえ] 
\\	する 
\\	する 
\\	し)	
\\	の前に 
\\	何[なに] 
\\	車[くるま] を 運転[うんてん]する の 前[まえ]に、お 酒[さけ] を 飲[の]んで は いけません。	くるま を うんてんする の まえに、おさけ を のんで は いけません。	
\\	前 
\\	の前に 
\\	飲[の]む/ 
\\	て 
\\	[む/ぶ/ぬ] 
\\	[んで])	
\\	は いけません 
\\	四年[よんねん] 前[まえ]に, 結婚[けっこん]しました。	よんねん まえに、けっこんしました。	
\\	の前に 
\\	結婚[けっこん]する 
\\	する 
\\	し)				
\\	二年[にねん] 前[まえ]に, 日本[にほん] へ 行[い]きました。	にねん まえに、にほん へ いきました。	
\\	の前[まえ]に 
\\	行[い]く 
\\	ます 
\\	車[くるま]の 運転[うんてん] が できます。	くるま の うんてん が できます。	
\\	車[くるま]の 運転[うんてん] 
\\	できる 
\\	る)	ます 
\\	英語[えいご] が できますか?	えいご が できますか?	
\\	できる 
\\	る)	ます 
\\	歌[うた]う こと が できます。	うたう こと が できます。	
\\	歌[うた]う 
\\	こと 
\\	できる 
\\	る)	ます 
\\	漢字[かんじ] を 読[よ]む こと が できます。	かんじ を よむ こと が できます。	
\\	読む 
\\	こと 
\\	できる 
\\	る)	ません 
\\	泳[およ]ぐ こと が できますか?	およぐ こと が できますか?	
\\	泳ぐ 
\\	こと 
\\	できる 
\\	る)	ます 
\\	現金[げんきん] を 払[はら]う こと が できますが、カード を 払[はら]う こと が できません。	げんきん を はらう こと ができますが、カード を はらう こと が できません。	
\\	払う 
\\	こと 
\\	が 
\\	できる 
\\	る)	ます 
\\	現金[げんきん] 
\\	クラス の 後[あと]で, 居酒屋[いざかや] へ 行[い]きます。	クラス の あとで、いざかや へ いきます。	
\\	後[あと] 
\\	行[い]く 
\\	ます 
\\	仕事[しごと] の 後[あと]で、お 酒[さけ] を 飲[の]んで もいいです。	しごと の あとで、おさけ を のんで もいいです。	
\\	仕事[しごと] 
\\	後[あと] 
\\	飲[の]む 
\\	て 
\\	[む/ぶ/ぬ] 
\\	[んで])	
\\	てもいいです 
\\	お酒 
\\	ゲーム を した 後[あと]で, 寝[ね]ます。	ゲーム を した あとで、ねます。	
\\	後[あと] 
\\	する 
\\	します 
\\	て 
\\	[する] 
\\	[して])	寝る 
\\	る)	ません 
\\	カナダ へ 帰[かえ]った 後[あと]で, 家族[かぞく] に 会[あ]います。	かなだ へ かえった あとで、かぞく に あいます。	
\\	後[あと] 
\\	帰[かえ]る 
\\	た 
\\	[る/つ/う] 
\\	[った])	会[あ]う 
\\	ません 
\\	日光[にっこう] へ 紅葉[こうよう] を 見[み] に 行[い]きます。	にっこう へ こうよう を み に いきます。	
\\	紅葉[こうよう] 
\\	見[み]る 
\\	る)	行[い]く 
\\	ます 
\\	センター へ 日本語[にほんご] を 習[なら]い に 行[い]きます。	センター へ にほんご を ならい に いきます。	
\\	習[なら]う 
\\	行[い]く 
\\	ます 
\end{CJK}
\end{document}