\documentclass[8pt]{extreport} 
\usepackage{hyperref}
\usepackage{CJKutf8}
\begin{document}
\begin{CJK}{UTF8}{min}
\\	私はイギリス人ではありません	
\\	私はイギリス人です	
\\	私は韓国人ではありません	
\\	私は中国人です	
\\	私は日本人です	
\\	私は韓国人です	
\\	私はアメリカ人ではありません	
\\	私は日本人ではありません	
\\	私はアメリカ人です	
\\	あなたは日本人ですか?	
\\	あの人は留学生ですか?	
\\	私は学生です	
\\	私は先生ではありません	
\\	あなたは学生ですか?	
\\	あの人は先生ですか?	
\\	あの人は中国人ではありません	
\\	私は会社員です	
\\	あなたは留学生ですか?	
\\	私は医者です	
\\	あなたはイギリス人ですか?	
\\	あの人は医者ではありません	
\\	あなたは会社員ですか?	
\\	あの人は日本人です	
\\	私は会社員ではありません	
\\	あの男の人は李さんですか	
\\	田中さんも先生です	
\\	田中さんは日本人ですか?	
\\	王さんは学生ですか	
\\	あの人は田中さんです	
\\	あの女の人はだれですか	
\\	あの男の人はアメリカ人です	
\\	私も会社員ではありません	
\\	李さんも学生です	
\\	あの人は王さんではありません	
\\	王さんも中国人です	
\\	鈴木さんも日本人ですか?	
\\	鈴木さんは留学生ではありません	
\\	あの男の人はだれですか	
\\	あなたもアメリカ人ですか?	
\\	あの女の人は鈴木さんではありません	
\\	田中さんも医者ではありません	
\\	あの人はだれですか	
\\	あの女の人は学生ですか	
\\	李さんは中国人です	
\\	あの人もイギリス人ですか?	
\\	いいえ、あの人はアメリカ人ではありません	
\\	王さんは先生ですか	
\\	はい、私は会社員です	
\\	はい、私は日本人です	
\\	いいえ、田中さんは学生ではありません	
\\	田中さんは学生ですか?	
\\	いいえ、王さんは先生ではありません	
\\	あの人はアメリカ人ですか	
\\	鈴木さんは会社員ですか?	
\\	はい、私は中国人です	
\\	鈴木さんはイギリス人ですか?	
\\	はい、私は学生です	
\\	李さんは中国人ですか?	
\\	いいえ、鈴木さんはイギリス人ではありません	
\\	これは新聞です	
\\	それは雑誌です	
\\	これは本です	
\\	あれは何ですか	
\\	あれは新聞です	
\\	これは新聞ですか?	
\\	それは手紙ですか	
\\	それは新聞ではありません	
\\	これは本ですか	
\\	これは本ではありません	
\\	それは何ですか	
\\	あれは手紙です	
\\	あれは雑誌です	
\\	それは手紙です	
\\	これは手紙です	
\\	あれは雑誌ではありません	
\\	それは雑誌ですか	
\\	はい、そうです、それは本です	
\\	これはテレビです	
\\	これは椅子ですか?	
\\	これは本ですか	
\\	あれは机ですか?	
\\	あれは椅子です	
\\	これはテレビですか?カメラですか?	
\\	いいえ、ちがいます。これはカメラです。	
\\	田中さんは会社員です	
\\	いいえ、ちがいます。それは雑誌です。	
\\	はい、そうです。これはテレビです	
\\	田中さんは会社員ですか?先生ですか?	
\\	いいえ、ちがいます。それは机です。	
\\	それはテレビですか?カメラですか?	
\\	あれは机ですか?椅子ですか	
\\	それはテレビですか	
\\	はい、そうです。あれは机です	
\\	それはカメラです	
\\	それはテレビですか?	
\\	これは中国語の新聞です	
\\	それは映画の雑誌です。	
\\	これは日本語の本です	
\\	それは日本語の新聞です	
\\	あれは英語の新聞です	
\\	これは何の雑誌ですか?	
\\	あれは英語の本です	
\\	あれはカメラの本です	
\\	それはカメラの雑誌です	
\\	あれは何の本ですか?	
\\	これはカメラの雑誌です	
\\	それは何の新聞ですか	
\\	あれはだれの辞書ですか?	
\\	王さんの机はどれですか?	
\\	これは私の本です	
\\	あれは李さんのです	
\\	これはあの人の時計ですか?	
\\	私のパソコンはどれですか?	
\\	それは王さんの傘ですか?	
\\	鈴木さんの本はどれですか?	
\\	田中さんのハンカチはどれですか?	
\\	これはだれのかばんですか?	
\\	それは田中さんのです。	
\\	これは私の時計です。	
\\	これは私のです。	
\\	それは鈴木さんのカメラですか	
\\	李さんの傘はどれですか?	
\\	それはだれの傘ですか?	
\\	それは田中さんの雑誌ですか?	
\\	田中さんの時計はどれですか?	
\\	あれは李さんの椅子ではありません。	
\\	そのハンカチは田中さんのですか?	
\\	そのかばんは鈴木さんのですか?	
\\	その辞書は私のです	
\\	このパソコンは李さんのですか?	
\\	あのパソコンは私のではありません。	
\\	あのカメラはだれのですか?	
\\	この辞書はだれのですか?	
\\	この雑誌は田中さんのです	
\\	あの時計は王さんのです。	
\\	その手紙はだれのですか?	
\\	あのハンカチはだれのですか?	
\\	このかばんは鈴木さんのではありません。	
\\	ここは事務室ですか?	
\\	あそこも会議室ですか	
\\	ここはトイレです	
\\	そこは教室です	
\\	ここは学校です	
\\	そこはどこですか?	
\\	ここは図書館ではありません	
\\	あそこは事務室ですか?	
\\	あそこは学校ではありません	
\\	そこは学校です	
\\	そこは図書館ですか?	
\\	あそこは教室です	
\\	あそこはどこですか?	
\\	ここは図書館です	
\\	ここはどこですか?	
\\	そこは会議室です	
\\	ここは教室です	
\\	そこはトイレではありません	
\\	あそこは教室です	
\\	トイレはそこです	
\\	レストランはここです	
\\	事務室はここです	
\\	会議室はここです	
\\	銀行はどこですか	
\\	本屋はどこですか?	
\\	図書館はそこです。	
\\	銀行はあそこですか?	
\\	病院はどこですか?	
\\	郵便局はどこですか?	
\\	銀行はそこです。	
\\	映画館はどこですか?	
\\	私の学校はここではありません	
\\	喫茶店はどこですか?	
\\	レストランはあそこではありません	
\\	本屋はそこですか?	
\\	病院はあそこです	
\\	映画館はあそこです。	
\\	映画館はそこです	
\\	デパートはここです。	
\\	デパートはそこです	
\\	公園に花と石があります	
\\	庭に何がありますか?	
\\	この部屋に傘と時計があります。	
\\	デパートに何がありますか?	
\\	部屋にテレビがありません	
\\	公園に花がありません	
\\	部屋に机と椅子があります。	
\\	本屋に何がありますか?	
\\	庭に石がありません。	
\\	あそこに何がありますか?	
\\	あそこに英語の新聞があります	
\\	あそこに英語の新聞がありません	
\\	教室にだれがいますか?	
\\	病院にだれがいますか?	
\\	教室に学生がいます。	
\\	居間にだれがいますか?	
\\	庭にだれがいますか?	
\\	病院に医者がいます	
\\	学校に先生がいます。	
\\	病院に犬がいません。	
\\	居間に猫がいません	
\\	居間に猫と犬がいます	
\\	公園に子供がいません	
\\	ベッドの上に子供がいます	
\\	椅子の下に猫がいます	
\\	箱の中に薬があります	
\\	箱の隣に猫がいます	
\\	机の上に本と雑誌があります	
\\	李さんの後ろに子供がいます	
\\	銀行の隣に学校があります	
\\	田中さんの前に犬がいます	
\\	田中さんのハンカチはかばんの中にありません	
\\	リンゴは冷蔵庫にありません	
\\	茶碗はどこにありますか?	
\\	私の辞書はどこにありますか?	
\\	紅茶はどこにありますか	
\\	日本語の雑誌は先生の事務室にあります	
\\	リンゴはどこにありますか?	
\\	ビールはどこにありますか?	
\\	グラスはどこにありますか?	
\\	私のカメラは箱の上にあります。	
\\	ビールは冷蔵庫の中にあります	
\\	鳥は魚の横にいません	
\\	田中先生は車の後ろにいません	
\\	田中先生は車の横にいます	
\\	鈴木さんは家の前にいます	
\\	魚はどこにいますか?	
\\	子供たちはどこにいますか?	
\\	田中先生はどこにいますか?	
\\	私の猫はその喫茶店にいません	
\\	魚は木の上にいません	
\\	子供は木の下にいません	
\\	子供たちは部屋の外にいます	
\\	猫は犬の横にいます	
\\	デパートに何もありません	
\\	映画館の前にだれもいません	
\\	部屋の中にだれもいません	
\\	木の下にだれもいません	
\\	私のペットはどこにもいません	
\\	ベッドの下に何もありません	
\\	居間に何もありません	
\\	社長の財布はどこにもありません	
\\	私の消しゴムはどこにもありません	
\\	財布の中に何もありません	
\\	田中先生のペンはどこにもありません	
\\	社長はどこにもいません	
\\	銀行にだれもいません	
\\	四十七	
\\	八	
\\	七十四	
\\	二	
\\	四	
\\	五	
\\	六	
\\	九	
\\	九十二	
\\	七	
\\	三十八	
\\	二十九	
\\	五十六	
\\	十	
\\	十一	
\\	三	
\\	六十五	
\\	八十三	
\\	一	
\\	十時半	
\\	今午後八時十五分です	
\\	今午前十時半です	
\\	二時十分	
\\	四時	
\\	八時十五分	
\\	今午前二時十分です	
\\	今午後六時三十分です	
\\	今午後九時です	
\\	今午後四時です	
\\	六時三十分	
\\	今何時ですか?	
\\	九時	
\\	今午前九時です	
\\	今午前七時です	
\\	七時	
\\	明日は金曜日です	
\\	今日は月曜日です	
\\	明日は火曜日です	
\\	あさっては水曜日ではありません	
\\	今日は金曜日ではありません	
\\	あさっては何曜日ですか?	
\\	明日は月曜日ではありません	
\\	今日は火曜日です	
\\	明日は何曜日ですか?	
\\	今日は木曜日です	
\\	今日は何曜日ですか?	
\\	あさっては木曜日です	
\\	今日は水曜日です	
\\	あの女の子は何歳ですか?	
\\	あの男の子は何歳ですか?	
\\	あの男の人は三十七歳です	
\\	あなたは何歳ですか?	
\\	私は七歳です	
\\	私は十ハ歳です	
\\	私は二十一歳です	
\\	李さんは二十三歳です	
\\	あの女の人四十五歳です	
\\	あの男の子は何歳ですか	
\\	あの男の子は十歳です	
\\	あの人は五十歳です	
\\	あの女の子は二十歳です	
\\	その子供は何歳ですか	
\\	王さんは十九歳です	
\\	あの男の人は三十七歳です	
\\	あのカメラはいくらですか?	
\\	この机は8000円です	
\\	四千	
\\	百	
\\	二百	
\\	この雑誌はいくらですか?	
\\	三百	
\\	四百	
\\	このリンゴは二百円です	
\\	五百	
\\	一万	
\\	そのリンゴはいくらですか?	
\\	七百	
\\	七千	
\\	ハ百	
\\	千	
\\	九百	
\\	このパソコンは85000円です	
\\	六百	
\\	その時計は26,000円です	
\\	三千	
\\	六千	
\\	ハ千	
\\	その薬は600円です	
\\	おばあさん今どこにいますか?	
\\	お兄さん、あの人はだれですか?	
\\	あの人は私の母です	
\\	李さんのお父さんはアメリカにいます	
\\	お母さん、今何時ですか?	
\\	今王さんのお姉さんは銀行にいます	
\\	お父さん、私の猫はどこにいますか?	
\\	あの女の子は私の姉です	
\\	娘はアメリカにいます	
\\	田中さんのお母さんは今日本にいません	
\\	息子は今日本にいます	
\\	あの男の子は私の兄です	
\\	今鈴木さんのお兄さんは車の中にいます	
\\	あの人は私の父です	
\\	この雑誌はお姉さんのですか?	
\\	あの方は李さんのおじさんです	
\\	あの人は田中さんのおばあさんです	
\\	おじいさん、これは何ですか?	
\\	今日は曇りです	
\\	昨日は晴れでした	
\\	昨日は雪でした	
\\	土曜日はいい天気ではありませんでした。	
\\	明日はいい天気ではありません	
\\	明日は曇りです	
\\	今日は晴れです	
\\	昨日は晴れではありませんでした	
\\	明日の天気はどうですか	
\\	今日の天気はどうですか	
\\	今日はいい天気です	
\\	昨日は曇りでした	
\\	明日は雨です	
\\	昨日の天気はどうでしたか?	
\\	ここはレストランではありませんでした	
\\	日曜日は雨ではありませんでした	
\\	先週は暑かったです	
\\	今年の春は暖かくないです	
\\	先週の火曜日は暑くなかったです	
\\	今年の春は暖かかったです	
\\	今年の冬は寒くなかったです	
\\	冬の日本は寒かったです	
\\	先月は涼しくなかったです	
\\	月曜日は寒かったです	
\\	去年の春は暖かくなかったです	
\\	今年の冬の天気はどうでしたか?	
\\	今年の夏は暑くなかったです	
\\	先週は寒かったです	
\\	アメリカの秋は涼しくなかったです	
\\	秋は涼しかったです	
\\	昨日は涼しかったです	
\\	秋は涼しいです	
\\	明日は涼しくないです	
\\	去年の夏は暑かったです	
\\	秋は暑くないです	
\\	夏は暑いです	
\\	冬は寒いです	
\\	東京の冬は寒くないです	
\\	来年の秋	
\\	冬は暖かくないです	
\\	春は暖かいです	
\\	今年の春	
\\	夏は涼しくないです	
\\	春は寒くないです	
\\	東京の冬	
\\	今日は暑くないです	
\\	去年の冬は暖かかったです	
\\	去年の夏	
\\	私の馬は大きいです	
\\	鳥は軽いです	
\\	パンダは重いです	
\\	虎は大きいです	
\\	虎は重いです	
\\	パンダはかわいいです	
\\	私の猫はかわいいです	
\\	この猫は小さいです	
\\	この魚は軽いです	
\\	あの犬は小さいです	
\\	田中さんの荷物は重くなかったです	
\\	ライオンは小さい動物ではありません	
\\	うさぎは重い動物ではありません	
\\	去年の冬、あの鳥は重かったです	
\\	昨日パンダはかわいくなかったです	
\\	パンダはかわいい動物です	
\\	サルは大きい動物ではありません	
\\	兄の猫はかわいかったです	
\\	この鳥は軽くないです	
\\	あの象は重い象です	
\\	昔犬は大きくなかったです	
\\	私はかわいいサルです	
\\	そのサルは小さくないです。	
\\	あのライオンは重くないです	
\\	去年娘のウサギは小さかったです	
\\	あのパンダはかわいくないです	
\\	私のかばんは軽かったです	
\\	あの象は小さくないです	
\\	昔サルは軽くなかったです	
\\	このウサギは大きくないです	
\\	刺身は赤い色です	
\\	ビールは白い色ではありません	
\\	バナナは青い色ではありません	
\\	寿司は白い色です	
\\	ジュウスは黒い色ではありません	
\\	その青いコートはとても短いです	
\\	緑のセーターはちょっと長いです	
\\	その黒い靴は少し大きいです	
\\	田中さんの洋服は全然安くないです	
\\	この白い洋服は少し小さいです	
\\	あの赤いネクタイはあまり安くないです	
\\	あの黒いズボンはちょっと短かったです	
\\	去年の夏はちょっと暑かったです	
\\	青いコートは全然高くないです	
\\	今年の春はとても暖かいです	
\\	白い靴は全然安くないです	
\\	あの白いシャツはとても長いです	
\\	この靴下はあまり高くないです	
\\	あの緑のセーターは少し大きいです	
\\	あのオレンジ色の上着はちょっと短いです	
\\	この赤いかばんはとても大きいです	
\\	東京の冬は少し寒いです	
\\	そのジーンズはあまり高くないです	
\\	コンビニは便利です	
\\	英語は簡単でした	
\\	日本語は簡単ではありません	
\\	この町は便利ではありません	
\\	昔この町は静かではありませんでした	
\\	賑やか	あのレストランはにぎやかでした	
\\	図書館は静かです	
\\	図書館はにぎやかではありません	
\\	昔東京はにぎやかではありませんでした	
\\	この日本語の本は簡単です	
\\	あのコンビニは便利ではありませんでした	
\\	図書館は静かでした	
\\	あの人は親切な人ではありません	
\\	東京はどんなところですか?	
\\	あのお酒はどんなお酒ですか?	
\\	田中さんは親切な先生です	
\\	これは簡単な料理ではありません	
\\	彼女はどんな人ですか?	
\\	便利なところです	
\\	彼は有名な医者です	
\\	東京はきれいな町です	
\\	田中さんはどんな先生ですか?	
\\	有名のお酒です	
\\	きれいな人です	
\\	北京は静かな町ではありません	
\\	親切な先生です	
\\	牛乳を買う	
\\	兄は毎朝新聞を読みます	
\\	夫はラジオを聞きません	
\\	妻はお茶を飲みません	
\\	母は毎晩本を書きます	
\\	母は毎日パンを買います	
\\	コーヒーを飲む	
\\	息子は毎晩パンを買います	
\\	ラジオを聞く	
\\	ご主人はラジオを聞きますか?	
\\	本を読む	
\\	姉は毎日雑誌を読みます	
\\	音楽を聞く	
\\	奥さんはお茶を飲みますか?	
\\	手紙を書く	
\\	娘は毎朝牛乳を買います	
\\	姉は毎日お茶を飲みます	
\\	兄は毎晩牛乳を飲みます	
\\	兄は毎日手紙を書きます	
\\	お茶を飲む	
\\	父は毎朝コーヒーを飲みます	
\\	パンを買う	
\\	父は毎日音楽を聞きます	
\\	姉は毎日ラジオを聞きます	
\\	娘は毎日十二時ごろ昼ご飯を食べます	
\\	七時に起きる	
\\	おばあさんは魚を食べますか?	
\\	朝ごはんを食べる	
\\	子供たちは卵を食べますか?	
\\	今日奥さんは出かけますか?	
\\	姉は毎週映画を見ます	
\\	子供たちは毎晩十時に寝ます	
\\	昼ご飯を食べる	
\\	妻は毎日九時に出かけます	
\\	おばあさんは魚を食べません	
\\	子供たちは卵を食べません	
\\	今日妻は出かけません	
\\	兄は午後六時半に晩ご飯を食べます	
\\	父は毎晩九時半に寝ます	
\\	晩ご飯を食べる	
\\	母は毎日父を出迎えます	
\\	兄は毎日テレビを見ます	
\\	母は毎朝七時に起きます	
\\	九時に寝る	
\\	夫は毎日七時ごろ朝ご飯を食べます	
\\	あさって田中さんと奥さんは中国へ来ますか?	
\\	会社へ来る	
\\	中国へ来る	
\\	明日友達は私の家来ます	
\\	土曜日会社へ来ますか?	
\\	来年旅行しません	
\\	父は毎日散歩します	
\\	日曜日両親は出勤しません	
\\	日本へ来る	
\\	母は毎朝掃除します	
\\	姉は毎晩ジョギングしません	
\\	来年両親は日本へ来ます	
\\	今日は出かけませんでした	
\\	私はあの映画を見ませんでした	
\\	昨日母は掃除しませんでした	
\\	先月彼はテレビを買いました	
\\	先月田中先生は学校へ来ませんでした	
\\	昨日兄の犬はパンを食べませんでした	
\\	先週息子は日本語の本を読みました	
\\	昨日娘は日本料理を食べました	
\\	去年両親は旅行しませんでした	
\\	先週兄は出勤しませんでした	
\\	金曜日息子はラジオを聞きませんでした	
\\	私は中国語の手紙を書きました	
\\	先週息子は何の本を読みましたか?	
\\	土曜日は部屋を掃除しました	
\\	先月田中さんは何を買いましたか?	
\\	昨日娘は何を食べましたか?	
\\	夫は毎週スーパーでビールを買います	
\\	私は毎日ハ時間働きます	
\\	李さんは毎晩どこで泳ぎますか	
\\	何ヶ月間日本語を勉強しましたか?	
\\	昨日どこでこの本を読みましたか?	
\\	月曜日から金曜日まで五日間旅行します	
\\	十分間休む	
\\	スーパーでビールを買う	
\\	プールで泳ぐ	
\\	毎日何時間働きますか?	
\\	昨日どこでカレーを食べましたか	
\\	一ヶ月間日本語を勉強する	
\\	私は一ヶ月間日本語を勉強しました	
\\	四時間遊ぶ	
\\	何日間旅行しますか?	
\\	李さんは何年間日本にいましたか?	
\\	十分間休みました	
\\	私は毎日喫茶店でコーヒーを飲みます	
\\	昨日食堂でカレーを食べました	
\\	妻は毎日家でお茶を飲みます	
\\	ハ時間働く	
\\	昨日子供たちは四時間遊びました	
\\	喫茶店でコーヒーを飲む	
\\	食堂でカレーを食べる	
\\	何分間休みましたか?	
\\	李さんは二年間日本にいました	
\\	木で部屋を作ります	
\\	ナイフで鶏肉を切ります	
\\	紙で飛行機を作ります	
\\	スプーンでスープを飲みます	
\\	箸でラーメンを食べます	
\\	私は兄弟が三人います	
\\	公園に蛇が三匹います	
\\	リンゴを一つ食べました	
\\	本を十冊読みました	
\\	車を一台買いました	
\\	そこに魚が一匹います	
\\	はがきを二枚出しました	
\\	馬を二頭売りました	
\\	猫を二匹買いました	
\\	この動物園にライオンが一頭います	
\\	ラーメンを三杯食べました	
\\	黒い馬が三頭います	
\\	ビールを三本飲みました	
\\	来年は東京へ行きません	
\\	私は昨日友達と一緒に美術館へ行きました	
\\	来週田中さんは大阪へ行きます	
\\	だれとイギリスへ行きますか	
\\	友達と一緒にフランスへ行きますか	
\\	去年妻と一緒にオーストリアへ来ました	
\\	明日家族と一緒にドイツへ帰ります	
\\	友達の李さんと一緒にドイツへ行きます	
\\	昨日美術館へ行きましたか?	
\\	来月兄は北京へ来ます	
\\	両親と一緒にイギリスへ行きます	
\\	いつ国へ帰りますか	
\\	先週彼は中国へ行きませんでした	
\\	東京から大阪まで飛行機で行きます	
\\	去年中国から日本まで船で来ました	
\\	寮から映画館まで自転車でどのぐらいかかりますか?	
\\	ここから駅まで歩いて十五分ぐらいかかります	
\\	ホテルから空港までタクシーで一時間半かかります	
\\	家から会社までバスで五十分ぐらいかかります	
\\	映画館から図書館まで自転車で来ました	
\\	ここから大阪まで船で三日間かかります	
\\	ホテルから美術館までタクシーで行きました	
\\	駅から家までバスで帰ります	
\\	空港からホテルまで電車で行きます	
\\	美術館からお寺まで地下鉄でどのぐらいかかりますか?	
\\	彼はいつも図書館で日本語を勉強します	
\\	私は一週間に二回友達とテニスをします	
\\	娘はほとんどスポーツをしません	
\\	時々友達とサッカーをします	
\\	私はあまりデパートへ行きません	
\\	この花は十年に一度咲きます	
\\	私はあまり新聞を読みません	
\\	一週間に四日アルバイトをします	
\\	彼女はあまりフランス語を勉強しません	
\\	李さんはよく日本へ行きます	
\\	姉は一ヶ月に三回映画を見ます	
\\	私は時々ジョギングをします	
\\	たまに外で晩ご飯を食べます	
\\	田中さんは全然外で晩ご飯を食べません	
\\	私はよく音楽を聞きます	
\\	奥さんはいつもスーパーへ行きますか?	
\\	王さんはほとんど日本料理を食べません	
\\	夫はたまにコーヒーを飲みます	
\\	田中さんは一週間に三日授業があります	
\\	家族は四ヶ月に一回旅行します	
\\	毎日一緒に日本語を勉強しに行きませんか?	
\\	一緒に花を見に行きませんか	
\\	土曜日一緒に買い物に行きましょう	
\\	田中さんは北京へ旅行に来ました	
\\	レストランへ昼ご飯を食べに行きます	
\\	郵便局へはがきを出しに行きました	
\\	春に花を見に行きました	
\\	一緒にデパートへ買い物に行きましょう	
\\	来月一緒に東京へ旅行に行きましょう	
\\	この夏休みは一緒に旅行に行きませんか?	
\\	一緒に昼ご飯を食べに行きませんか	
\\	東京へ日本語を勉強しに来ます	
\\	映画館へ映画を見に行きます	
\\	アメリカへ留学に行きます	
\\	明日一緒に映画を見に行きましょう	
\\	家へ帰りたくなかったです	
\\	車を運転しながら、歌を歌います	
\\	鍋を食べながら、お酒を飲みます	
\\	旅行したいです	
\\	コーヒーを飲みながら、日本語を勉強します	
\\	長いスカートがほしかったです	
\\	何がほしいですか	
\\	おいしい食べ物がほしいです	
\\	何もほしくないです	
\\	掃除しながら、テレビを見ます	
\\	音楽を聞きながら、本を読みます	
\\	山に登りたくないです	
\\	素敵なかばんがほしいです	
\\	日本語の本を買いたかったです	
\\	日本人の友達がほしくなかったです	
\\	学校へ来たくなかったです	
\\	ジュースを飲みたいです	
\\	映画を見たくないです	
\\	日本へ行きたかったです	
\\	きれいな服がほしいです	
\\	中国語は分かりやすいです	
\\	中国の山は登りにくいです	
\\	地図は見にくいです	
\\	このパンは書きやすいです	
\\	箸は使いにくいです	
\\	大きい字は見やすいです	
\\	富士山は登りやすいです	
\\	この車は運転しやすいです	
\\	あの薬は飲みにくいです	
\\	日本語は勉強しにくいです	
\\	この薬は飲みやすいです	
\\	このパンは書きにくいです	
\\	私のパソコンは使いやすいです	
\\	野菜のサラダはまずいです	
\\	この本はすごくつまらないです	
\\	飴はどうですか?	
\\	昨日の映画はどうでしたか?	
\\	とてもうるさかったです	
\\	晩ご飯の海老はおいしかったです	
\\	野菜のサラダはどうですか?	
\\	飴は甘いです	
\\	この本はどうですか?	
\\	すごく面白かったです	
\\	昨日子供たちはどうでしたか?	
\\	いいえ、おいしくなかったです	
\\	このお菓子はおいしかったですか?	
\\	晩ご飯の海老はどうでしたか?	
\\	おじいさんはジョギングが嫌いです	
\\	私はテニスが好きです	
\\	太極拳は中国で人気があります	
\\	夫はバドミントンが嫌いです	
\\	息子はバスケットボールが上手です	
\\	子供たちはスポーツが好きです	
\\	兄は水泳が好きです	
\\	野球は日本でとても人気があります	
\\	バスケットボールは日本で人気がありません	
\\	サッカーはイギリスでとても人気があります	
\\	彼らはハイキングが嫌いです	
\\	友達はスキーが下手です	
\\	彼女は卓球が下手です	
\\	バドミントンは当社でとても人気があります	
\\	兄はスキーが上手です	
\\	テニスは中国であまり人気がありません	
\\	私は野球が上手です	
\\	太極拳はアメリカで人気がありません	
\\	彼らはサッカーが下手です	
\\	東京は物価が高いです	
\\	母は髪が短いです	
\\	私の犬は足が細いです	
\\	娘は耳が痒いです	
\\	王さんは頭が悪いです	
\\	田中さんは背が高いです	
\\	大阪は食べ物がおいしいです	
\\	兄は足が痒いです	
\\	私はお腹が痛いです	
\\	北海道は冬が寒いです	
\\	李さんは頭がいいです	
\\	兄は背が低いです	
\\	父は頭が痛いです	
\\	北京は人が多いです	
\\	昔沖縄は人が少なかったです	
\\	李さんは目が小さいです	
\\	子供は手が小さいです	
\\	猫は尻尾が長いです	
\\	妹は手が痛いです	
\\	象は足が太いです	
\\	象は鼻が長いです	
\\	テニスよりサッカーのほうが好きです	
\\	妹は姉より料理が下手です	
\\	バナナはイチゴほど値段が高くないです	
\\	白いスカートより赤いスカートのほうが高いです	
\\	中国は日本よりずっと広いです	
\\	自転車は電車ほど速くないです	
\\	サラダは刺身ほどおいしくないです	
\\	日本語より英語のほうが易しいです	
\\	中国より日本のほうが狭いです	
\\	リンゴはバナナより値段が高いです	
\\	私は田中さんより若いです	
\\	田中さんより私のほうが若いです	
\\	飛行機より電車のほうが遅いです	
\\	ブスは自転車より少し速いです	
\\	兄は私より背が高いです	
\\	英語は日本語ほど難しくないです	
\\	私は李さんより日本語が上手です	
\\	東京は京都ほど人が多くないです	
\\	イギリスはアメリカほど広くないです	
\\	中国語は日本語より難しいです	
\\	花の中で何が一番きれいですか	
\\	アメリカのほうが広いです	
\\	リンゴと海老とどちらが高いですか?	
\\	アメリカと日本とどちらが広いですか?	
\\	果物の中で何が一番好きですか?	
\\	海老のほうが高いです	
\\	コーヒーとお茶とどちらが好きですか?	
\\	コーヒーのほうが好きです	
\\	スポーツの中で何が一番上手ですか?	
\\	バドミントンが一番上手です	
\\	一年の中でいつが一番暑いですか?	
\\	スイカが一番好きです	
\\	桜が一番きれいです	
\\	動物の中でどれが一番かわいいですか?	
\\	中国でどこが一番にぎやかですか?	
\\	家族の中でどれが一番優しいですか?	
\\	英語と日本語とどちらが易しいですか?	
\\	携帯電話の操作は簡単で便利です	
\\	暖かくて軽い帽子がほしいです	
\\	彼女は私の友達で、日本語の先生です	
\\	李さんは元気で親切な人です	
\\	この本は面白いです。そして、易しいです	
\\	東京はにぎやかで面白い町です	
\\	北京は人が多くて、物価が高いとしです	
\\	この店はおいしくて安いです	
\\	私の出身は北海道で、仕事は医者です	
\\	私は大学二年生で、北京の出身です	
\\	田中さんは親切で優しいです	
\\	この料理は簡単でおいしいです	
\\	あの人は親切です。そして、ハンサムです	
\\	李さんは東京大学の学生で、二十歳です	
\\	公園は広いです。そして、静かです	
\\	この本は易しくて面白いです	
\\	田中さんは日本人です。そして、日本語の先生です	
\\	この公園は静かできれいです	
\\	田中さんは日本人で、教師です	
\\	あの人は面白くて、ハンサムな人です	
\\	彼は私のクラスメートで、恋人です	
\\	彼は優しくて明るいです	
\\	私の部屋は狭くて暗いです	
\\	北京は人が多いです。そして、物価が高いです	
\\	娘が教師になりました	
\\	娘を教師にしました	
\\	空を高く飛びます	
\\	ラジオの音が小さくなりました	
\\	天気が寒くなりました	
\\	問題を簡単にしました	
\\	テレビの音を小さくしました	
\\	運動場を速く走ります	
\\	あの女の子がきれいになりました	
\\	お菓子をおいしく食べます	
\\	会議室を禁煙にします	
\\	この部屋が会議室になりました	
\\	雑誌を静かに読みます	
\\	息子を医者にしました	
\\	部屋をきれいにしました	
\\	日本語をきれいに書きます	
\\	牛乳を冷たくしました	
\\	私は日本語が上手になりました	
\\	今この問題が簡単になりました	
\\	ラジオを聞いて日本語を勉強します	
\\	夕べ音楽を聞いてから、寝ました	
\\	郵便局に行って、手紙を出します	
\\	洗濯してから、部屋を掃除しました	
\\	電話を掛けて説明しました	
\\	傘を持って行きます	
\\	図書館へ本を借りに行ってから、帰りました	
\\	歌を歌って出迎えました	
\\	家に帰って、部屋を掃除します	
\\	着物を着て出かけます	
\\	パソコンを使って資料を調べます	
\\	牛乳を飲んでから、卵を一つ食べました	
\\	毎朝起きて、顔を洗います	
\\	スーパーへ行って、買い物します	
\\	夕べ音楽を聞いて、お風呂に入って、寝ました。	
\\	毎朝起きてから、コーヒーを飲みます	
\\	友達に会ってから、郵便局へ行きました	
\\	ジュースを飲んで、リンゴを二つ食べました	
\\	東京に来てから、友達に会いました	
\\	ここに座ってください	
\\	この角を右へ曲がってください	
\\	窓を開けてください	
\\	信号から西へ曲がってください	
\\	ここから南へ行ってください	
\\	速くしてください	
\\	交差点を右に曲がってください	
\\	窓を閉める	
\\	名前を書く	
\\	角を曲がる	
\\	ドアを閉めてください	
\\	ドアを開ける	
\\	ここに座る	
\\	この本を読んでください	
\\	公園から北へ行ってください	
\\	この道をまっすぐ行ってください	
\\	ちょっと待ってください	
\\	ここに名前を書いてください	
\\	部屋を掃除してください	
\\	静かにしてください	
\\	道を左に曲がってください	
\\	郵便局から東へ曲がってください	
\\	いいえ、それはちょっと。。	
\\	写真を撮る	
\\	ここで写真を撮ってもいいですか?	
\\	パソコンを使ってもいいですか?	
\\	写真を撮ってはいけません	
\\	荷物を置く	
\\	はい、いいですよ	
\\	あそこにごみを捨ててはいけません	
\\	クーラーをつける	
\\	この部屋に入ってはいけません	
\\	電気をつけてはいけません	
\\	ここに座ってもいいですか?	
\\	ここでタバコを吸ってはいけません	
\\	ここにごみを捨ててもいいですか?	
\\	電気を消してもいいですか?	
\\	ごみを捨てる	
\\	電気を消す	
\\	ここで電話をしてはいけません	
\\	ここに車を止めてはいけません	
\\	窓を開けてはいけません	
\\	ここに荷物を置いてはいけません	
\\	タバコを吸ってもいいですか?	
\\	タバコを吸う	
\\	クーラーをつけてもいいですか?	
\\	ここに荷物を置いてもいいですか?	
\\	鳥が飛んでいます	
\\	私は今本を読んでいます	
\\	どんな仕事をしていますか?	
\\	英語の教師をしています	
\\	李さんは今音楽を聞いています	
\\	娘は歯を磨いています	
\\	あの子はお母さんを待っています	
\\	田中さんは毎日公園を散歩しています	
\\	雪が降っています	
\\	桜が咲いています	
\\	私はあの会社で働いています	
\\	私は日本語の学校で日本語を勉強しています	
\\	私は一週間に二回料理の学校に通っています	
\\	子供は公園で遊んでいます	
\\	田中さんは今喫茶店でコーヒーを飲んでいます	
\\	お母さんは今日本料理を作っています	
\\	あの二人は日本語で話しています	
\\	車が走っています	
\\	王さんは今手紙を書いています	
\\	李さんは英語を教えています	
\\	魚が泳いでいます	
\\	みんなマスコをつけています	
\\	そのことを知っていますか?	
\\	あの人は眼鏡をかけています	
\\	そのことを知りません	
\\	李さんは結婚しています	
\\	お母さんは黒いコートを着ています	
\\	いいえ、田中さんを知りません	
\\	王さんは出かけています	
\\	今、中国に住んでいます	
\\	田中さんを知っていますか?	
\\	みんな帽子をかぶっています	
\\	授業を始まっています	
\\	私は車を持っています	
\\	田中さんは今日傘を持っていますか?	
\\	田中さんはアメリカへ行っています	
\\	娘は赤いセーターをきています	
\\	友達を持っています	
\\	今どこに住んでいますか?	
\\	お母さんは今起きていますか?	
\\	子供たちは公園で遊んでいました	
\\	娘はまだ起きていません	
\\	あの二人はずっと日本語で話していました	
\\	まだそのお歌を聞いていません	
\\	携帯電話を持っていました	
\\	去年王さんは東京に住んでいました	
\\	李さんはまだ結婚しません	
\\	あの女の子はずっとお母さんを待っていました	
\\	私はその会社で働いていました	
\\	私はまだこの本を読んでいません	
\\	まだ昼ご飯を食べていません	
\\	朝九時から十二時まで日本語を勉強していました	
\\	授業はまだ始まっていません	
\\	今日はまだ英語を勉強していません	
\\	私は昨日本を読んでいました	
\\	コートを着ないで出かけました	
\\	靴を脱がなくてもいいです	
\\	明日学校へ行かなくてもいいです	
\\	ここに車を止めないでください	
\\	教室で日本語を話さなければなりません	
\\	プールで泳がないでください	
\\	傘を持たないで学校へ行きました	
\\	私の家に来ないでください	
\\	今日はお風呂に入らないで寝ました	
\\	片仮名を覚えなければなりません	
\\	携帯電話を使わないでください	
\\	英語を勉強しなければなりません	
\\	一人で旅行しないでください	
\\	野菜を食べなければなりません	
\\	廊下を走らないでください	
\\	スーツを着なければなりません	
\\	切手を貼らないで手紙を出しました	
\\	今日は平仮名を覚えなくてもいいです	
\\	九時に学校へ行かなければなりません	
\\	子供は何も食べないで家を出ました	
\\	今日は会社へ来なくてもいいです	
\\	はじめのページに何も書かないでください	
\\	窓を閉めなくてもいいです	
\\	クーラーを消さなくてもいいです	
\\	ここに座らないでください	
\\	バスに乗らないで歩いて友達の家に行きます	
\\	買い物しないでください	
\\	この川を渡らなければなりません	
\\	部屋を掃除しなくてもいいです	
\\	タバコを吸わないでください	
\\	一人で旅行したことがあります	
\\	今日は早く帰ったほうがいいです	
\\	すぐに病院へ行ったほうがいいです	
\\	フランス語を勉強したことがあります	
\\	夏休みは山に登ったり船に乗ったりします	
\\	お金をお母さんに渡したほうがいいです	
\\	一度だけ日本へ行ったことがあります	
\\	スーパーへ行った後で友達のところへ行きます	
\\	いいえ、一度もありません	
\\	はい、一度だけあります	
\\	週末は家で本を読んだり、テレビを見たりします	
\\	寿司を食べたことがあります	
\\	お母さんはパンを作ったり、ケーキを作ったりします	
\\	泳いだ後で水を飲みます	
\\	朝ご飯を食べた後で出勤します	
\\	日本へ行ったことがありますか?	
\\	休みの日映画を見たり美術館へ行ったりします	
\\	子供たちは飛んだり叫んだりします	
\\	タクシーを呼んだほうがいいです	
\\	毎日牛乳を飲んだ後で寝ます	
\\	窓を閉めだ後でクーラーをつけます	
\\	車を運転したことがあります	
\\	パンダを見たことがありますか?	
\\	靴を脱いだほうがいいです	
\\	パンを買った後で家へ帰ります	
\\	洗濯した後で掃除します	
\\	娘は泣いたり、笑ったりします	
\\	家族と相談したほうがいいです	
\\	明日九時に学校へ来たほうがいいです	
\\	馬に乗ったことがあります	
\\	いいえ、日本料理を作ることができません	
\\	はい、私は英語を話すことができます	
\\	私の夢はあの山に登ることです	
\\	このアパートでペットを飼うことができません	
\\	夢はアメリカへ留学することです	
\\	日本語で手紙を書くことができます	
\\	赤ちゃんは泳ぐことができます	
\\	日本料理を作ることができますか?	
\\	息子の夢は医者になることです	
\\	英語を話すことができますか?	
\\	趣味は車を買うことです	
\\	私の趣味はテニスをすることです	
\\	着物を一人で着ることができません	
\\	八百屋で果物を買うことができますか?	
\\	姉の趣味は音楽を聞くことです	
\\	寝る前にお風呂に入ります	
\\	服を買う前に試着します	
\\	結婚したとき仕事を辞めました	
\\	東京へ行くときカバンをたくさん買いました	
\\	雪が降るとき家へ帰りました	
\\	結婚する前にあの会社で働きました	
\\	出かける前にクーラーを消してください	
\\	学校へ来るとき駅で日本語の先生に会いました	
\\	学校へ来たとき廊下で日本語の先生に会いました	
\\	結婚するとき仕事を辞めました	
\\	学校へ行く前に朝ご飯を食べました	
\\	東京へ行ったときかばんをたくさん買いました	
\\	アメリカへ旅行する前にパスポートを忘れないでください	
\\	どうして野菜をあまり食べませんでしたか?	
\\	今日は風邪を引きましたから	
\\	リンゴは安いですから、たくさん買いました	
\\	あのスカートが高いです。だから、買いません	
\\	明日は日曜日ですから、出勤しません	
\\	寒いです。だから、窓を閉めました	
\\	今日はなぜお酒を飲みませんでしたか?	
\\	どうして新しい車を買いましたか?	
\\	休みの日ですから、会社に誰もいません	
\\	野菜が嫌いですから	
\\	父はお酒が好きです。だから、毎日お酒を飲みます	
\\	上手ですから、スキーが好きです	
\\	今日はなぜ会社を休みましたか?	
\\	今の車はとても古いですから	
\\	車で来ましたから	
\\	私は音楽が好きですが、歌うことができません	
\\	春になると、暖かくなります	
\\	ケーキはおいしくなくても、食べます	
\\	このボタンを押すと、天気が明るくなります	
\\	風邪でも会社へ行きます	
\\	雨が降っても、家へ帰ります	
\\	彼は魚を食べます。でも、鶏肉は食べません	
\\	冬になると、寒くなります	
\\	この車は値段が安くても、買いません	
\\	私は友達と遊びたいです。でも、病気になりました	
\\	彼はお酒を飲むと、顔が赤くなります	
\\	その帽子が好きです。でも、値段がとても高いです	
\\	風邪を引いても、学校を休みません	
\\	雨でも家へ帰ります	
\\	映画を見に行きたいですが、暇がありません	
\\	春になると、花が咲きます	
\\	母わ日本人ですが、私は日本語ができません	
\\	このボタンを押すと、水が出ます	
\\	私は友達にお土産をもらいました	
\\	私は田中さんにお土産をあげました	
\\	あのおばあさんは私にお菓子をくれました	
\\	私は彼女に手紙をもらいました	
\\	親切な人からお金をもらいました	
\\	田中さんは私に傘をくれました	
\\	私は鈴木さんに本をあげました	
\\	王さんは私においしいお酒をくれました	
\\	私は姉からプレゼントをもらいました	
\\	田中さんに何ももらいませんでした	
\\	鈴木さんから緑の財布をもらいました	
\\	王さんのお母さんから、セーターをもらいました	
\\	父は母にネックレスをあげました	
\\	私は母にリンゴをあげました	
\\	鈴木さんから何をもらいましたか?	
\\	李さんは私に切符をくれました	
\\	妹は李さんにお誕生日のプレゼントをあげました	
\\	お母さんは私に玩具をくれました	
\\	お父さんは私に何もくれませんでした	
\end{CJK}
\end{document}