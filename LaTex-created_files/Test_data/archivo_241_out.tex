\documentclass[8pt]{extreport} 
\usepackage{hyperref}
\usepackage{CJKutf8}
\begin{document}
\begin{CJK}{UTF8}{min}
\\	雨粒
\\	石原純
\\	【テキスト中に現れる記号について】
\\	ルビ
\\	(例)驟雨《しゅうう》
\\	そろそろさみだれの季節がやって来る。
\\	同じく雨ではあっても、ふしぎに季節や環境によってその感じは非常にちがっている。それで我国では雨にいろいろな名まえがつけられている。春さめ、さみだれ、しぐれ、驟雨《しゅうう》、ゆうだち、霧雨、小糠雨《こぬかあめ》、その外《ほか》にもなおあるであろう。そう云う雨のいろいろな感じのなかには、雨の音がかなりな役目をはたらいている。さみだれの静かに降りそそぐ音とか、ゆうだちの激しくものを撃つ音とか、音もなくひっそりと濡らしてゆく小糠雨とか、みんなそれぞれの趣きをそなえているのである。
\\	ものしずかに雨の音を聞いていると、いろいろな記憶が心のなかによみがえって来るのも、一つのなつかしげな風情である。
\\	ところで、ちょっと見方を変えて、雨というのはたくさんの水粒が空から生れて、地上に落ちて来るものだと考え出すと、恰《あたか》もそれらが人間の運命を象徴しているようにも思われる。こういう水の粒にもいろいろの大いさのものがある。眼で見ると、雨は普通に細い線につながって見える。雨の落ちるのはそんなに速いのではないが、それでも人間の眼はその粒を見分けるわけにはゆかない。
\\	そこで雨粒の大いさを測るのにはどうしたらよいか。気象学では、そのためにちょっとおもしろい方法をつかっている。それは雨粒の落ちるのを吸取紙で受けて、紙の上に滲《にじ》み拡がる面積を測るのである。それから別に半径のわかっている水粒を同質の吸取紙に滲ませてその面積を雨の場合と比較すれば、これから雨粒の大いさを知ることができようと云うのである。
\\	この方法は科学的にはさほど精密だとは云い難いが、雨粒の大いさなどは個々にそれ程精密に知る必要はないのであるし、大体の平均がわかればよいのだから、これでも十分に間に合うのであろう。私がそれをおもしろいと云うのは、雨が何事もなく落ちている間は、人間の眼でその大いさなどはっきりとわからないのに、紙で受けとるとそれがはっきり見えるようになると云うことである。生きているうちはさほどとも思われない人間が死ぬと急にその偉さが世間に認められると云うことなども、之《これ》と似ている。
\\	すべてもの事はこれと同じである。たいした変り方もなく続いている間は、そう云う事があると承知していながら、人間はとかくぼんやりと見過ごすだけである。そしてそこに何かの事変が起ると、始めてその正体を認めて、今更のように慌て驚くことすらある。また病気などの場合でも、身体のなかに潜んでいる間は、たとえ自分で承知していても、まあ、どうにかなってゆくと、たいして気にも留めずにいるが、それが何かの反応を起すようになると、これではならぬと、今更その重大さを気にするようになる。これらは人間の通性で仕方のない事だと云ってしまえば、それ迄《まで》である。併《しか》し事の起らない先にその赴くところをはっきりと見究めることこそ、社会や人生や、その他すべての仕事にとってどれ程大切であるかわからない。
\\	雨粒の大いさを吸取紙で調べるなどは、謂《い》わば昔風な観測法である。もっと近代的な方法としては、雨粒の落ちているのを瞬間的に写真にとればよい、そうすれば大いさもわかるし、形などもはっきりする。普通に人間の眼がぼんやりと見過ごしているのを写真はもっと鋭敏に印してくれる。つまり何事に対しても、表面的な感覚的観察に終らせることなしに、もっと科学的な方法をそこに利用することが必要なのではないか。雨をただ直線的に降るものと呑気に見ているだけではいけない。何かしら大事なことだと察したなら、それを出来るだけ科学的に突きつめる近代的な方法を講ずることが大切なのである。この頃のはやり言葉で云えば、認識というのであろうが、写真のレンズが歪んでいると、とんだまちがった認識を結果しないとも限らないから、それも十分に注意しなくてはならない。
\\	雨粒のようなものは直接に写真にとってその形を見ることができるけれども、もっと小さいものになると、それができなくなる。例えば物理学で取り扱う放射性物質からの放射線のようなものである。これらも雨粒を吸取紙で受けとるように、何かの物質に当ててその作用で調べることはできるが、これでは途中の有様がわからない。そこで、ウィルソン霧箱と云うものをつかって巧妙な方法でその途筋を写真にとると、途中の通路がはっきりわかる。ウィルソン霧箱というのは、物理学の書物を見れば説明してあるが、つまり水蒸気を過飽和にする一種の装置で、そのなかへ放射線を通すと、それが通過した場処にはイオンが出来、その周りに水蒸気が凝結して水粒となるから、これを写真にとると放射線の通路が示されるのである。ちょうど雨粒を糸につらねた恰好でこれもあながち雨粒の話と縁がないとは云われない。
\\	このウィルソン霧箱の方法というのは、説明を聞けば何でもないが、いかにも巧妙なものだと云わなくてはならない。放射線をつくっているα《アルファ》粒子だとか電子だとかは、どんなに小さなものであるかは、物理学ではっきりわかっている。そんな眼に見えぬ小さなものの通路を写真に見せるなどと云うことは実に驚くべきことだと思われる。そこで普通にはとても正体のわからないと考えられる事柄でも、何かの工夫をしてせめてその輪廓をでも明らかにすると云うことが、このウィルソン霧箱のように出来たなら、実にすばらしい事ではあるまいか。
\\	雨粒を見ながら私はこんな事をいろいろ考えていた。この頃のように世間の人心が何かしら不安に襲われているときに、衝撃につき当るまでぼんやり待っていると云うだけでは、まことに心細いものである。
\\	それから今ではどこの学校にも試験があって、試験間際になると、学生が頻《しき》りに頭をなやますのであるが、これもちょうど雨粒の大きさを吸取紙に滲ませるようなもので、そんな事をしないで、ふだんから何かもう少し賢明な方法で学力を検《しら》べるようにするのが近代的な教育であると私は思う。雨粒の身になって見れば、足許《あしもと》に吸取紙が見えてから急に騒ぎ出したって、本当に眼のある人から見られたら、それはいかにも愚かな仕業にうつるであろう。
\\	底本:「日本近代随筆選 1出会いの時〔全3冊〕」岩波文庫、岩波書店
\\	2016(平成28)年4月15日第1刷発行
\\	2016(平成28)年6月15日第2刷発行
\\	底本の親本:「石原純随筆集」
\\	叢書、エスコム出版
\\	2011(平成23)年11月
\\	初出:「新短歌」新短歌発行所
\\	1939(昭和14)年11月20日
\\	※「大いさ」と「大きさ」の混在は、底本通りです。
\\	入力:岡村和彦
\\	校正:高瀬竜一
\\	2017年12月26日作成
\\	青空文庫作成ファイル:
\\	このファイルは、インターネットの図書館、青空文庫
\\	で作られました。入力、校正、制作にあたったのは、ボランティアの皆さんです。
\end{CJK}
\end{document}