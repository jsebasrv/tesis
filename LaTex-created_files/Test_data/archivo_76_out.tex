\documentclass[8pt]{extreport} 
\usepackage{hyperref}
\usepackage{CJKutf8}
\begin{document}
\begin{CJK}{UTF8}{min}
\\	褒めます	今日 私は 部長に 褒められました。 。。。何ついて褒められましたか。 お客さんと 会議が 成功しました。 。。。いいですね	褒[ほ]めます			
\\	人1は
\\	2に
\\	人1
\\	い→かれます;
\\	られます;
\\	されます-こられます)
\\	誘います	弟を誘ったので、チケットは無駄になりませんでした 。。。。	誘[さそ]います			
\\	人1は
\\	2に
\\	人1
\\	い→かれます;
\\	られます;
\\	されます-こられます)
\\	注意します	今朝 部長に 呼ばれました。 。。。何か あったんですか 出張のレポートの 書き方について 注意されました	注意[ちゅうい]します			
\\	人1は
\\	2に
\\	人1
\\	い→かれます;
\\	られます;
\\	されます-こられます)
\\	盗ります	私は 誰かに 旅行中に 財布を 盗られて、困りました 。。。大変です。どうしましたか。 。。。。	盗[と]ります			
\\	人1は 
\\	2に 
\\	3を 
\\	人1 
\\	い→かれます;
\\	られます;
\\	されます-こられます)
\\	踏みます	ラッシュの電車で 足を 踏まれました 。。。そうですか。こんど 気をつきます	踏[ふ]みます			
\\	人1は 
\\	2に 
\\	3を 
\\	人1 
\\	い→かれます;
\\	られます;
\\	されます-こられます)
\\	壊します	この美術館は 壊されます。 。。。えぇ。どうしてですか しりませんね	壊[こわ]します			
\\	は・が
\\	い→かれます;
\\	られます;
\\	されます-こられます)
\\	汚します	私は子どもに 服を汚されました。 。。。それは大変ね	汚[よご]します				人1は 
\\	2に 
\\	3を 
\\	人1 
\\	い→かれます;
\\	られます;
\\	されます-こられます)
\\	行います	大阪で 国際会議が 行われますか。 。。。どんな会議ですか。 ボランティアのかいぎですね。 。。。ええ、いいですね。この会議は成功するでしょうか。 6か月前から 準備していましたから。きっと 成功するでしょう。 。。。はい、わかりました	行[おこな]います			
\\	は・が
\\	い→かれます;
\\	られます;
\\	されます-こられます)
\\	輸出します	日本の車は いろいろな国へ 輸出されていますね。 。。。そうですね。日本の車は いいし、値段も そんなに 高くないですから	輸出[ゆしゅつ]します			
\\	は・が
\\	い→かれます;
\\	られます;
\\	されます-こられます)
\\	輸入します	今ベトナムは ほとんど電子品が輸入されています	輸入[ゆにゅう]します				
\\	は・が
\\	い→かれます;
\\	られます;
\\	されます-こられます)
\\	翻訳します	この小説は いろいろな国言葉に 翻訳されています。 。。。そうですか。すごいですね	翻訳[ほんやく]します				
\\	は・が
\\	い→かれます;
\\	られます;
\\	されます-こられます)
\\	発明します	インスタントラーメンは 1958年に「あんどうもふく」によって 発明されました。どんぶりに めんと お湯を 入れて、 3分待つと、食べられるので、とても便利な食べ物です。	発明[はつめい]します				
\\	は 人によって 
\\	(書かれます、発明されます、発見されます、)	
\\	い→かれます;
\\	られます;
\\	されます-こられます)
\\	発見しま	また 新しい大陸が 発見されました。 。。。そうですか。 だれが 発見されましたか 
\\	が 発見されました.	発見[はっけん]しま			
\\	は・が
\\	い→かれます;
\\	られます;
\\	されます-こられます)
\\	麦	ビールは 麦から 造られます。これが 原料の麦です 。。。これが ビールに なるんですね	麦[むぎ]				
\\	1は 
\\	から
\\	・で
\\	造られます。
\\	い→かれます;
\\	られます;
\\	されます-こられます)
\\	原料	ビールは 麦から 造られます。これが 原料の麦です 。。。これが ビールに なるんですね	原料[げんりょう]			
\\	1は 
\\	から
\\	・で
\\	造られます。
\\	い→かれます;
\\	られます;
\\	されます-こられます)
\\	インスタントラーメン	インスタントラーメンは 1958年に「あんどうもふく」によって 発明されました。どんぶりに めんと お湯を 入れて、 3分待つと、食べられるので、とても便利な食べ物です。	インスタントラーメン				
\\	は 人によって 
\\	(書かれます、発明されます、発見されます、)		
\\	い→かれます;
\\	られます;
\\	されます-こられます)
\\	泥棒	私は 泥棒に カメラを 盗られました。 。。。残念ですね	泥棒[どろぼう]			
\\	人1は 
\\	2に 
\\	3を 
\\	人1 
\\	い→かれます;
\\	られます;
\\	されます-こられます)
\\	警官	私は警官に ここに車を止めるなと 言われました。	警官[けいかん]				人1は
\\	2に
\\	人1
\\	い→かれます;
\\	られます;
\\	されます-こられます)
\\	~中	この本は 世界中で 読まれています。 。。。そうですか。ぜひ 読もうと思っています そうします	
\\	中[ちゅう]			
\\	は・が
\\	い→かれます;
\\	られます;
\\	されます-こられます)
\\	だれか	どうしたんですか 。。。誰かに 傘を 間違えられたんです	だれか			
\\	人1は 
\\	2に 
\\	3を 
\\	人1 
\\	い→かれます;
\\	られます;
\\	されます-こられます)
\\	オリンピック	1964年に 東京で オリンピックが 行われました。 。。。そうですか	オリンピック				
\\	は・が
\\	い→かれます;
\\	られます;
\\	されます-こられます)
\\	ワールドカップ	2002年に 日本と韓国で サッカーのワールドカップが 開かれました	ワールドカップ				
\\	は・が
\\	い→かれます;
\\	られます;
\\	されます-こられます)
\\	ポルトガル	ブラジルでは 何語が 使われて いますか。 。。。ポルトガル語が 使われています そうですか。	ポルトガル			
\\	は・が
\\	い→かれます;
\\	られます;
\\	されます-こられます)
\\	江戸時代		江戸[えど] 時代[じだい]			
\\	ロシア	この人形はロシアで 作られています	ロシア				
\\	は・が
\\	い→かれます;
\\	られます;
\\	されます-こられます)
\\	焼けます	金閣寺は 14世紀に 建てられました。 1950年に 一度 焼けて しまいましたが。 その後 新しい建物が 建てられて、1994年に 世界遺産になりました。京都で 人気がある お寺の一つです	焼[や]けます				
\\	は・が
\\	い→かれます;
\\	られます;
\\	されます-こられます)
\\	その後	金閣寺は 14世紀に 建てられました。 1950年に 一度 焼けて しまいましたが。 その後 新しい建物が 建てられて、1994年に 世界遺産になりました。京都で 人気がある お寺の一つです	その 後[ご]				
\\	は・が
\\	い→かれます;
\\	られます;
\\	されます-こられます)
\\	世界遺産	金閣寺は 14世紀に 建てられました。 1950年に 一度 焼けて しまいましたが。 その後 新しい建物が 建てられて、1994年に 世界遺産になりました。	世界[せかい] 遺産[いさん]				
\\	は・が
\\	い→かれます;
\\	られます;
\\	されます-こられます)
\\	美しい	紅葉がきれいですね。 。。。ええ、金閣寺は 紅葉と 雪の 季節が 特に 美しいと言われています	美[うつく]しい			
\\	は・が
\\	い→かれます;
\\	られます;
\\	されます-こられます)
\\	豪華	お城は 17世紀の初めに建てられました。建物は 豪華で 有名ですが、建物の中にも 有名な彫刻や 絵が あります	豪華[ごうか]			
\\	は・が
\\	い→かれます;
\\	られます;
\\	されます-こられます)
\\	彫刻	お城は 17世紀の初めに建てられました。建物は 豪華で 有名ですが、建物の中にも 有名な彫刻や 絵が あります	彫刻[ちょうこく]			
\\	は・が
\\	い→かれます;
\\	られます;
\\	されます-こられます)
\\	言い伝え		言い伝[いいつた]え					
\\	眠ります	子どもは眠っていますが。騒がないでください。 。。。はい、わかりました	眠[ねむ]ります			
\\	彫ります	眠り猫は眠っている猫の彫刻で、左甚五郎が 彫りました	彫[ほ]ります			
\\	は・が
\\	い→かれます;
\\	られます;
\\	されます-こられます)
\\	しかし		しかし						
\\	鼠	内弁慶の外鼠	鼠[ねずみ]			
\\	育てます	きれいな庭ですね。 。。。ありがとうございます。夫は 花を 育てるのが上手なんですよ 子どもを育てるのが大変です	育[そだ]てます			
\\	運びます	一人でこの 荷物を 運ぶのは 無理です 。。。ええ。手伝いましょうか。 ええ、お願いします	運[はこ]びます			
\\	退院します	明日 田中さんが 退院するのを 知っていますか。 。。。いいえ。田中さんは 治りますか 息子さんは まだ スポーツが できますか。 。。。先週 退院したばかりなので、まだ スポーツが できません	退院[たいいん]します			
\\	る、ない・
\\	い、な・
\\	の
\\	はずです、~。
\\	入れます	このコピー機の使い方を教えていただけませんか。 。。。ええ、いいですよ。まず 電源を 入れて、3分待ちます。それから、安心番号を入力します そうですか。ありごとうございます	入[い]れます			
\\	電源[でんげん]を~	
\\	切ります	あ、いけない。 。。。どう したんですか。 コンピューターの電源を 切るのを 忘れてしまいました。すみませんが、お先に どうぞ 。。。お先に 失礼します	切[き]ります			
\\	掛けます	あ、いけない。 。。。どう したんですか。 机のかぎを 掛けるのを 忘れてしまいました。すみませんが、お先に どうぞ 。。。お先に 失礼します	掛[か]けます			
\\	うそをつきます	息子は うそをつくのが 下手です	うそをつきます			
\\	気持ちがいい	朝 早く散歩するのは 気持ちがいいですよ。 。。。そうですか、明日 一緒に 散歩しよう うん。いいよ	気持[きも]ちがいい			
\\	赤ちゃん	兄の赤ちゃんが 生まれたのを知っていますか。 。。。いいえ、知りませんでした。いつですか 1か月ぐらい いまえです	赤[あか]ちゃん			
\\	気持ちが悪い		気持[きも]ちが 悪[わる]い			
\\	小学校	初めて好きになった人のことを 覚えていますか。 。。。ええ。 彼女に 初めて 会ったのは 小学校の教室です。彼女は 音楽の先生です	小学校[しょうがっこう]			
\\	中学校	大学の先生になりたいんですか。 。。。いいえ、小学校の先生になりたいんです	中学校[ちゅうがっこう]			
\\	海岸	私は 海岸を 散歩するのが 好きですよ	海岸[かいがん]			
\\	指輪	結婚の指輪が 高いですので、今から 一生懸命 働いてください 。。。そうですね。ありがとうございます	指輪[ゆびわ]					
\\	電源	あ、いけない。 。。。どう したんですか。 コンピューターの電源を 切るのを 忘れてしまいました。すみませんが、お先に どうぞ 。。。お先に 失礼します	電源[でんげん]			
\\	健康	一番大切なのは家族の健康です	健康[けんこう]			
\\	原爆ドーム	すみません。この旅行に ついて 聞きたいんですが 。。。はい。どうぞ 広島で見学したいんですが。どこか いい所が あちませんか。 。。。原爆ドームと 自動車工場がいいと思います そうですか	原爆[げんばく]ドーム			
\\	あ、いけない。	あ、いけない。 。。。どうしたんですか 車の窓を 閉めるのを忘れました。	あ、いけない。			
\\	お先に失礼します	あ、いけない。 。。。どう したんですか。 机のかぎを 掛けるのを 忘れてしまいました。すみませんが、お先に どうぞ 。。。お先に 失礼します	お 先[さき]に 失礼[しつれい]します			
\\	回覧	おはようございます 。。。あ、本を 持って来るのを忘れてしまいました。 大丈夫ですよ。でも、回覧に はんこを 押すのを忘れないでください。先月も 押して ありませんでしたよ。	回覧[かいらん]			
\\	きちんと	先生の研究室は いつも きれいですね。 。。。私は 片付けるのが 好きなんです 整理するのが 上手なんですね。 。。。昔 「上手な整理の方法」という 本を書いたことがあるんです へえ。すごいですね 。。。あまり 売れませんでしたけどね。よかったら、1冊 持って来ましょうか	きちんと			
\\	整理します	先生の研究室は いつも きれいですね。 。。。私は 片付けるのが 好きなんです 整理するのが 上手なんですね。 。。。昔 「上手な整理の方法」という 本を書いたことがあるんです へえ。すごいですね 。。。あまり 売れませんでしたけどね。よかったら、1冊 持って来ましょうか	整理[せいり]します			
\\	方法	先生の研究室は いつも きれいですね。 。。。私は 片付けるのが 好きなんです 整理するのが 上手なんですね。 。。。昔 「上手な整理の方法」という 本を書いたことがあるんです へえ。すごいですね 。。。あまり 売れませんでしたけどね。よかったら、1冊 持って来ましょうか どの 方法が 一番 いいか、考えましょう	方法[ほうほう]			
\\	判子	おはようございます 。。。あ、本を 持って来るのを忘れてしまいました。 大丈夫ですよ。でも、回覧に 判子を 押すのを忘れないでください。先月も 押して ありませんでしたよ。	判子[はんこ]			
\\	双子	「しずか」と 「あすか」 は 双子の姉妹です。 今10歳です。小学校5年生です 顔は ほんとうによく似ていますが、性格はずいぶん違います	双子[ふたご]			
\\	姉妹	「しずか」と 「あすか」 は 双子の姉妹です。 今10歳です。小学校5年生です 顔は ほんとうによく似ていますが、性格はずいぶん違います	姉妹[しまい]					
\\	似ています	「しずか」と 「あすか」 は 双子の姉妹です。 今10歳です。小学校5年生です 顔は ほんとうによく似ていますが、性格はずいぶん違います	似[に]ています			
\\	性格	「しずか」と 「あすか」 は 双子の姉妹です。 今10歳です。小学校5年生です 顔は ほんとうによく似ていますが、性格はずいぶん違います	性格[せいかく]					
\\	おとなしい	姉は おとなしくて、優しい女の子です。本を読んだり、犬を世話したり するのが好きです。 特に 外国の小説が好きで、本を読んでいると、時間がたつのを忘れてしまいます	おとなしい			
\\	世話します	姉は おとなしくて、優しい女の子です。本を読んだり、犬を世話したり するのが好きです。 特に 外国の小説が好きで、本を読んでいると、時間がたつのを忘れてしまいます	世話[せわ]します			
\\	時間がたちます	姉は おとなしくて、優しい女の子です。本を読んだり、犬を世話したり するのが好きです。 特に 外国の小説が好きで、本を読んでいると、時間がたつのを忘れてしまいます	時間[じかん]がたちます			
\\	気が強い	妹は 外で遊ぶのが大好きです。試験は いつも 50点ぐらいですが、走るのはクラスで 一番速いです。 それに 気も強いです。男の子と けんかしても、 負けません	気[き]が 強[つよ]い					
\\	けんかします	妹は 外で遊ぶのが大好きです。試験は いつも 50点ぐらいですが、走るのはクラスで 一番速いです。 それに 気も強いです。男の子と けんかしても、 負けません 子どもの声が しますね。 。。。子どもたちが けんかしているようですよ	けんかします			
\\	い、な・
\\	の
\\	ようです。
\\	どうも
\\	声・音・におい・味が します。
\\	上から二段目	どこに 青いフィールをしまいましたか。 。。。よく 覚えてないな。机の上にある引き出しだ、下から二段目です。 見当たらないわ。 。。。おかしい。	上[うえ]から二 段[だん] 目[め]			
\\	手前	どこに 青いフィールをしまいました。 。。。よく 覚えてないな。机の上にある引き出しだ、下から二段目です。 見当たらないわ。 。。。おかしい。 ああ、机の引き出しにあった。 。。。よかった。どこですか 上から3段目、手前にあった	手前[てまえ]					
\\	前から二列目	前から二列目,左から二番目をおねがいします	前[まえ]から二 列[れつ] 目[め]					
\\	斜め前	私はどこに 座ってもいいですか。 。。。ミンさんは 私の斜め前いすに座ってください	斜[なな]め 前[まえ]			
\\	斜め後ろ	ダムさんは 
\\	1のいすに座ってください。ミンさんは ダムさんの斜め後ろに座ってください。 。。。はい、分かりました	斜[なな]め 後[うし]ろ			
\\	三行目	3行目に 名前書いてくだいさい。	三 行[ぎょう] 目[め]						
\\	答えます	質問が 難しき手、答えられませんでした。 。。。そうですか。	答[こた]えます				
\\	質問[しつもん]に~	
\\	通ります	自転車が 邪魔で、通れません。 。。。そうですね。手伝いましょうか。 ええ、お願いします この道、通れますか。 いいえ、台風で 木がたくさん 倒れたので、通れません はい、わかりました	通[とお]ります				
\\	な:
\\	で・
\\	い: 
\\	くて・
\\	て、
\\	びっくりします	地震のニュースを 見て、びっくりしました	びっくりします			
\\	な:
\\	で・
\\	い: 
\\	くて・
\\	て、
\\	がっかりします	お見合いは どうでしたか。 。。。写真を見た時は、すてきな人だと 思いましたが、会って、がっかりしました。 どうしてですか。	がっかりします			
\\	安心します	メールを読んで、安心しました 母の元気の声を 聞いて、安心しました	安心[あんしん]します			
\\	な:
\\	で・
\\	い: 
\\	くて・
\\	て、
\\	離婚します	フランスの大統領が 離婚したのを 知っていますか。 。。。ええ。テレビを見て、びっくりしました。 お金の問題だと 言っていましたね。 。。。えええ。	離婚[りこん]します			
\\	太ります	最近 太ったので、ダイエットを しようと 思っています。 。。。そうですか。頑張って。	太[ふと]ります			
\\	やせます	今 私は やせたいんですが、いいスポーツクラブが ありませんか。 。。。CALIFORNIAクラブが いいと思います	やせます			
\\	複雑	昨日の 映画は どうでしたか。 。。。話が 複雑で、よく わかりませんでした	複雑[ふくざつ]				
\\	な:
\\	で・
\\	い: 
\\	くて・
\\	て、
\\	邪魔	最近、布団で 寝ているんですが、便利ですね。 。。。ベッドは どうしたんですか 部屋が狭くて、 邪魔なので、友達に あげました。	邪魔[じゃま]			
\\	「な」	
\\	硬い	歯が 痛くて、硬い物が 食べられません。 。。。ええ、やわらかい物を 食べたほうがいいですよ。例えばね、ケーキとか、スープとか。。。。	硬[かた]い			
\\	な:
\\	で・
\\	い: 
\\	くて・
\\	て、
\\	軟らかい	歯が 痛くて、硬い物が 食べられません。 。。。ええ、やわらかい物を 食べたほうがいいですよ。例えばね、ケーキとか、スープとか。。。。	軟[やわ]らかい			
\\	汚い	この辺の海は 汚いので、泳がないほうがいいですよ 。。。そうと思います	汚[きたな]い			
\\	うれしい	
\\	3の試験に 合格して、うれしかったです	うれしい			
\\	恥ずかしい	スピーチが 上手に できなくて、はずかしかったです。 。。。頑張ってね。 会社を 替わるということは もはや 恥ずかしいことでは ない	恥[は]ずかしい			
\\	首相	首相が 入院したのを 知っていますか 。。。ええ、ニュースを聞いて、びっくりしました、 胃の病気で入院したと 言っていましたね 。。。ええ	首相[しゅしょう]					
\\	津波	津波で 人が大勢 死にました	津波[つなみ]				
\\	で~
\\	台風	大きい台風9号日本へ来て、日本中 交通が 止まります	台風[たいふう]				
\\	な:
\\	で・
\\	い: 
\\	くて・
\\	て、
\\	雷	雷で 町の電気が ほとんど 消えました	雷[かみなり]			
\\	で~
\\	火事	1950年に 火事で 金閣寺が 焼けました 火事や地震の場合は、エレベーターを 使わないでください。 。。。はい、わかりました。	火事[かじ]			
\\	な:
\\	で・
\\	い: 
\\	くて・
\\	て、
\\	る、ない、た・
\\	い、な・
\\	の
\\	場合は、~。
\\	事故	遅くなって、すみません。 。。。どう したんですか 事故で バスが 遅れたんです 交通事故に会った 場合は、すぐ 110番に連絡してください	事故[じこ]					
\\	お見合い	お見合いは どうでしたか。 。。。写真を見た時は、すてきな人だと 思いましたが、会って、がっかりしました。 どうしてですか。	見合[みあ]い			
\\	て、
\\	操作	このカメラは 操作が 簡単なので、だれでも 使えます 。。。いいですね。	操作[そうさ]			
\\	会場	コンサートは ありますか。 。。。いいえ、昨日の雨と雷で 会場が 使えません ああ、そうですか	会場[かいじょう]					
\\	~代	電気代が高いので、エアコンを つけません。 。。。そうします。日本の値段が とても高いです	
\\	代[だい]			
\\	フロント	はい、フロントです 。。。417号室ですが、シャワーのお湯が 出ないので、見に来て いただけませんか 417号室ですね、 はい、 すぐ 伺います 。。。お願いします	フロント						
\\	タオル	はい、フロントです 。。。417号室ですが、タオルと 石鹸がないので、持って来て いただけませんか 417号室ですね、 はい、 すぐ 伺います 。。。お願いします	タオル						
\\	石鹸	はい、フロントです 。。。417号室ですが、タオルと 石鹸がないので、持って来て いただけませんか 417号室ですね、 はい、 すぐ 伺います 。。。お願いします	石鹸[せっけん]			
\\	トラック	課長、遅れてすみません 。。。ミラーさん、どう したんですか 実は 来る途中で 事故があって、バスが 遅れてしまいました 。。。バスの事故ですか。 いいえ、交差点で トラックと 車が ぶつかって、バスが 動かなかったんです 。。。それは、大変ですね	トラック				
\\	る 途中で ~。
\\	ぶつかります	課長、遅れてすみません 。。。ミラーさん、どう したんですか 実は 来る途中で 事故があって、バスが 遅れてしまいました 。。。バスの事故ですか。 いいえ、交差点で トラックと 車が ぶつかって、バスが 動かなかったんです 。。。それは、大変ですね	ぶつかります			
\\	る 途中で ~。
\\	大人	昔、日本人は大人も 子どもも みんな 毎日 着物を着て生活していた	大人[おとな]			
\\	おもしろい	日本のことを 知るのが おもしろいです	おもしろい			
\\	うらやましい	うらやましい!これからは、あらゆる種類の楽しい番組が 見られるように なるのね そんなに くだらない本に 夢中になれるなんて、 うらやましいね	うらやましい			
\\	うっとりします	
\\	の映画にうっとりするので、教室に 遅刻してしまいました。 。。。これはいけませんね。 
\\	の声にうっとりしまいました	うっとりします			
\\	いらいらします	時間がなくて、いらいらします	いらいらします			
\\	どきどきします	夜遅く歩いて どきどきします。 初めて インタビューして、 どきどきしま	どきどきします			
\\	洋服	しかし、着物を着るのは 難しいし、時間もかかって、大変だ。 また、歩くときや、仕事をするときも、着物は不便なので、みんな 洋服を着るように なった。	洋服[ようふく]			
\\	西洋化します	洋服は 着るのが簡単だ。それに日本人の生活にも西洋化したので、着物より洋服のほうが生活に合う。	西洋[せいよう] 化[か]します			
\\	会います	お見合いは どうでしたか。 。。。写真を見た時は、すてきな人だと 思いましたが、会って、がっかりしました。 どうしてですか。	会[あ]います			
\\	成人式	今では 着物は 結婚式、葬式、成人式、正月など 特別な機会にだけ 着る着物になってしまった	成人[せいじん] 式[しき]					
\\	伝統的	日本の伝統的な服は 着物です。では 和服について 話してください	伝統[でんとう] 的[てき]			
\\	残念	旅行に行けなくて、残念です	残念[ざんねん]			
\\	な:
\\	で・
\\	い: 
\\	くて・
\\	て、
\\	数えます	何を 数えているんですか。 。。。ビールが 何本 残っているか、数えているんです はい、いいですね	数[かぞ]えます			
\\	い 
\\	な
\\	だ・
\\	だ 
\\	か、~。
\\	測ります	山の高さは どうやって 測るか、知っていますか。 。。。さああ、インタネットで 調べましょう	測[はか]ります			
\\	い 
\\	な
\\	だ・
\\	だ 
\\	か、~。
\\	い 
\\	い 
\\	さ 
\\	量ります	何をして いるんですか 。。。荷物の重さが 20キロ以下かどうか、量っているです	量[はか]ります			
\\	い 
\\	な
\\	だ・
\\	だ 
\\	かどうか、~。
\\	確かめます	忘年会に 出席できるか どうか、メールで 確かめてください 。。。はい、わかりました	確[たし]かめます			
\\	い 
\\	な
\\	だ・
\\	だ 
\\	かどうか、~。
\\	「サイズが~」合います	この服、デザインが すてきですよ。 。。。サイズが 合うかどうか、着てみていただけませんか	合[あ]います			
\\	出発する	飛行機が何時に 出発するか、インターネットで調べています。 道が込みそうですから、出発しましょう。	出発[しゅっぱつ]する			
\\	到着します	しR107便は 何時に 到着するか、知っていますか。 。。。ええ、ちょっと お待ちください。調べますから	到着[とうちゃく]します					
\\	酔います	二次会は どこへ 行きましたか。 。。。酔っていたので、どこへ 行ったか、全然 覚えていないんです。カラオケへ行ったそうです。	酔[よ]います			
\\	いまくいきます	発表が いまくいくか、 心配します。 。。。心配しないでください。1週間前から 準備していましたので、きっと 成功するでしょう スピーチは うまくいきましたか。 。。。いいえ、一生懸命練習して 覚えたのに、途中で 忘れて しまいました。	いまくいきます			
\\	い 
\\	な
\\	だ・
\\	だ 
\\	か、~。
\\	い、な・
\\	な
\\	のに、~。
\\	相談します	何を 相談しているんですか。 。。。夏休みに どこへ 行くか、相談しているんです	相談[そうだん]します			
\\	い 
\\	な
\\	だ・
\\	だ 
\\	か、~。
\\	必要	カードを 申し込む時は、判子が 必要かどうか、調べたほうがいいです。 。。。わかりました、どうも	必要[ひつよう]				な	
\\	天気予報	何をして いるんですか。 。。。明日は 晴れるかどうか、天気予報見ているんです	天気[てんき] 予報[よほう]			
\\	い 
\\	な
\\	だ・
\\	だ 
\\	かどうか、~。
\\	忘年会	忘年会に 出席できるか どうか、メールで 返事を ください。 。。。はい、わかりました どうしたんですか。 。。。忘年会で お酒を 飲みすぎて、頭が痛いんです。 それは いけませんですね。お大事に。	忘年会[ぼうねんかい]			
\\	い 
\\	な
\\	だ・
\\	だ 
\\	かどうか、~。
\\	ます 
\\	ます・
\\	い
\\	い・
\\	な
\\	な
\\	すぎます。
\\	新年会	新年会に 出席できるか どうか、メールで 確かめてください 。。。はい、わかりました	新年[しんねん] 会[かい]			
\\	い 
\\	な
\\	だ・
\\	だ 
\\	かどうか、~。
\\	二次会	二次会は どこへ 行きましたか。 。。。酔っていたので、どこへ 行ったか、全然 覚えていないんです。カラオケへ行ったそうです。	二次会[にじかい]			
\\	い 
\\	な
\\	だ・
\\	だ 
\\	か、~。
\\	発表会	来月日本語の 発表会が あります。 。。。どんな 発表を するんですか。 日本の若い人は 将来に ついて どう 思っているか、調べ、発表します 。。。そうですか、頑張ってください	発表[はっぴょう] 会[かい]			
\\	い 
\\	な
\\	だ・
\\	だ 
\\	か、~。
\\	マラソン	スピーチコンテストに 出るか どうか、決めましたか。 。。。いいえ、まだ 決めていません。 申し込みは明後日までですよ 。。。はい、わかりました	マラソン			
\\	い 
\\	な
\\	だ・
\\	だ 
\\	かどうか、~。
\\	コンテスト	マラソン大学に 出るか どうか、決めましたか。 。。。いいえ、まだ 決めていません。 申し込みは明後日までですよ 。。。はい、わかりました	コンテスト			
\\	い 
\\	な
\\	だ・
\\	だ 
\\	かどうか、~。
\\	表	この 
\\	サッツは どちら 表ですか、裏です 。。。ええ、私は どちらが 表か、わかりません おかしいな	表[おもて]			
\\	い 
\\	な
\\	だ・
\\	だ 
\\	か、~。
\\	裏	この 
\\	サッツは どちら 表ですか、裏です 。。。ええ、私は どちらが 表か、わかりません おかしいな	裏[うら]			
\\	い 
\\	な
\\	だ・
\\	だ 
\\	か、~。
\\	傷	借りたパソコンは 傷が ないかどうか、確かめてください 。。。はい、わかった	傷[きず]			
\\	い 
\\	な
\\	だ・
\\	だ 
\\	か、~。
\\	長さ	
\\	橋の長さかどうか、知っていますか	長[なが]さ			
\\	重さ	何をして いるんですか 。。。荷物の重さが 20キロ以下かどうか、量っているです 荷物の重さは 何キロか、 量ってみていただけませんか。	重[おも]さ				
\\	い 
\\	な
\\	だ・
\\	だ 
\\	かどうか、~。
\\	高さ	山の高さは どうやって 測るか、知っていますか。 。。。さああ、インタネットで 調べましょう	高[たか]さ				
\\	い 
\\	な
\\	だ・
\\	だ 
\\	か、~。
\\	い 
\\	い 
\\	さ 
\\	~便	しR107便は 何時に 到着するか、知っていますか。 。。。ええ、ちょっと お待ちください。調べますから	
\\	便[びん]			
\\	い 
\\	な
\\	だ・
\\	だ 
\\	か、~。
\\	~本	何を 数えているんですか。 。。。ビールが 何本 残っているか、数えているんです はい、いいですね	本[ほん]			
\\	い 
\\	な
\\	だ・
\\	だ 
\\	か、~。
\\	~杯	ビールは何杯が飲めますか。 。。。気分がいい日は10杯が飲めます	杯[はい]			
\\	キロ	日本の 北から 南まで 何キロあるか、知っていますか。 。。。いいえ、インタネットで 調べましょう	キロ			
\\	い 
\\	な
\\	だ・
\\	だ 
\\	か、~。
\\	グラム		グラム						
\\	ミリメートル		ミリメートル						
\\	以上	何をして いるんですか 。。。荷物の重さが 20キロ以上かどうか、量っているです	以上[いじょう]					
\\	以下	何をして いるんですか 。。。荷物の重さが 20キロ以下かどうか、量っているです	以下[いか]			
\\	い 
\\	な
\\	だ・
\\	だ 
\\	かどうか、~。
\\	どうでしょうか。	先生、
\\	ちゃんは 小学校で どうでしょうか。友達が できたか どうか、心配なんですが 。。。大丈夫ですよ、
\\	ちゃんは クラスで とても 人気があります。 そうですか。安心しました。	どうでしょうか。			
\\	い 
\\	な
\\	だ・
\\	だ 
\\	かどうか、~。
\\	テスト	勉強は どうですか。 漢字が 大変だと 言って いますが。 。。。毎日 漢字のテストを していますが。
\\	ちゃんは いい成績ですよ そうですか。ありがとうございます	テスト			
\\	成績	勉強は どうですか。 漢字が 大変だと 言って いますが。 。。。毎日 漢字のテストを していますが。
\\	ちゃんは いい成績ですよ そうですか。ありがとうございます 成績がよくないですが。息子を 塾に通わせています	成績[せいせき]			
\\	は人を
\\	は人に
\\	を
\\	は人に
\\	を
\\	い→かせます;
\\	させます;
\\	されます-こさせます)
\\	ところで	ところで、もうすぐ 運動会ですが。お父さんも いらっしゃいますか 。。。ええ。 
\\	ちゃんは 小学校で どんな 様子か、ぜひ 見てください。 。。。わかりました。 これからも よろしく お願いします	ところで			
\\	い 
\\	な
\\	だ・
\\	だ 
\\	か、~。
\\	様子	ところで、もうすぐ 運動会ですが。お父さんも いらっしゃいますか 。。。ええ。 
\\	ちゃんは 小学校で どんな 様子か、ぜひ 見てください。 。。。わかりました。 これからも よろしく お願いします	様子[ようす]			
\\	事件	この書類は 3億円事件です 。。。では、この事件について 話しましょう	事件[じけん]					
\\	オートバイ	1968年に 12月10日午前9時半ごろ。銀行の車が お金を 運んでました。そのとき、後ろから 景観が 白いオートバイに乗って、走って来ました。	オートバイ						
\\	積みます	警官の車を止めました。そして、車に爆弾が 積まれているかもしれないといいました。運転手と 銀行員は急ぎで 降りて、離れたところに 逃げました	積[つ]みます			
\\	運転手	警官の車を止めました。そして、車に爆弾が 積まれているかもしれないといいました。運転手と 銀行員は急ぎで 降りて、離れたところに 逃げました	運転[うんてん] 手[しゅ]			
\\	犯人	警官は 銀行の車に 乗って、中を調べていましたが、急に車を 動かして、言ってしまいました。警察は 一生懸命 犯人を捜しましたが、見つかりませんでした	犯人[はんにん]			
\\	手に入れます	日本中の人が、犯人は どんな男か、3億円をどう 使うか、どうやって 警官の服とオートバイを 手に入れたか、話しました。 昔の事件ですが、有名な事件で、映画やドラマにも なりました	手[て]に 入[い]れます			
\\	面積		面積[めんせき]					
\\	体積	この 箱の体積かどうか、よくわかりません	体積[たいせき]					
\\	立法センチメートル	この 箱の体積は 
\\	3です	立法[りっぽう]センチメートル			
\\	ミリリットル		ミリリットル						
\\	シーシー	今何シーシーの血液が必要か、知っていますか	シーシー						
\\	トン		トン		トン				
\\	たす		たす						
\\	ひきます		ひきます			
\\	計算	計算するかどうか、知っていますか	計算[けいさん]					
\\	割ります		割[わ]ります			
\\	イコール		イコール		イコール				
\\	直線	直線の長さを測ってください	直線[ちょくせん]					
\\	曲線	曲線をどうやって 描くか、 教えて くださいませんか。	曲線[きょくせん]					
\\	点線		点線[てんせん]					
\\	三角形	三角形をどうやって 描くか、 教えて くださいませんか。	三角形[さんかくけい]					
\\	正方形	正方形の 面積は どうやって 計算するか、 教ていただけませんか。	正方形[せいほうけい]			
\\	い 
\\	な
\\	だ・
\\	だ 
\\	か、~。
\\	四角形	四角形の 面積は どうやって 計算するか、 教ていただけませんか。	四角形[しかっけい]			
\\	い 
\\	な
\\	だ・
\\	だ 
\\	か、~。
\\	親切にします。	夏休みの ホームステイは どうでしたか。 。。。楽しかったです。家族の皆さんが とても 親切に してくださいました。	親切[しんせつ]にします。			
\\	人1は 人2(私
\\	に 
\\	て)くれます・くれます・くださいます。
\\	人1
\\	人2. (くれます:
\\	(くれます:
\\	-(くださいます:
\\	珍しい	珍しい鳥ですね。 。。。ええ、とても大変なんですよ。毎日水を換えてやらなければなりませんから。 そうですか	珍[めずら]しい			
\\	お祝い	きれいな お皿ですね。 。。。ええ。結婚のお祝いに 田中さんが くださいます	お 祝[いわ]い			
\\	人1は 人2(私
\\	に 
\\	て)くれます・くれます・くださいます。
\\	人1
\\	人2. (くれます:
\\	(くれます:
\\	-(くださいます:
\\	興味	サッカーを 見ることに興味がありますが。毎週末 サッカーを見ることが 2時間かかります	興味[きょうみ]			
\\	情報	初めて日本へ 来たとき、たいへんだったでしょう? 。。。ええ、でも ボ友達は 町のいろいろな 情報を教えてくださいました。 そうですか 。。。日本語や 日本料理を 教えてくださいました それはよかったですね	情報[じょうほう]			
\\	人1は 人2(私
\\	に 
\\	て)くれます・くれます・くださいます。
\\	人1
\\	人2. (くれます:
\\	(くれます:
\\	-(くださいます:
\\	文法	文法と発音を勉強するのがとても大切です。日本語が上手になる人は毎日練習したほうがいいですよ	文法[ぶんぽう]			
\\	発音	文法と発音を勉強するのがとても大切です。日本語が上手になく 人は毎日練習したほうがいいですよ.	発音[はつおん]			
\\	猿	お母さん、猿に お菓子を やっても いい? 。。。だまよ。あそこに 餌を やっては いけないと 書いてあるでしょう。	猿[さる]			
\\	人1(私)は 人2に 
\\	て)やります・あげます・さしあげます。
\\	人1
\\	人2
\\	(やります:
\\	(あげます:
\\	-(さしあげます:
\\	餌	お母さん、猿に お菓子を やっても いい? 。。。だまよ。あそこに 餌を やっては いけないと 書いてあるでしょう。	餌[えさ]			
\\	人1(私)は 人2に 
\\	て)やります・あげます・さしあげます。
\\	人1
\\	人2
\\	(やります:
\\	(あげます:
\\	-(さしあげます:
\\	おもちゃ	田舎へ帰るとき、孫におもちゃを書いてやっています。	おもちゃ			
\\	人1(私)は 人2に 
\\	て)やります・あげます・さしあげます。
\\	人1
\\	人2
\\	(やります:
\\	(あげます:
\\	-(さしあげます:
\\	絵本	私は 息子に 絵本を読んで やりました。	絵本[えほん]			
\\	人1(私)は 人2に 
\\	て)やります・あげます・さしあげます。
\\	人1
\\	人2
\\	(やります:
\\	(あげます:
\\	-(さしあげます:
\\	絵葉書	先月 高山先生に 日本の 絵葉書をいただきました。	絵葉書[えはがき]					
\\	ドライバ	棚を組み立てたいんですが、 ちょっとドライバを貨してくださいませんか。	ドライバ			
\\	て ください 
\\	~てくれ。
\\	~て!)。 
\\	て いださいませんか。
\\	ていただけませんか 
\\	ハンカチ	きれいな ハンカチですね。 。。。ええ。課長に いただいたんです	ハンカチ			
\\	人1(私)は 人2に 
\\	て)もらいます・もらいます・いただきます。
\\	人1
\\	人2. (もらいます:
\\	(もらいます:
\\	-(いただきます:
\\	手袋	きれいな手袋ですね。 。。。ええ、おばに もらいます。	手袋[てぶくろ]			
\\	人1(私)は 人2に 
\\	て)もらいます・もらいます・いただきます。
\\	人1
\\	人2. (もらいます:
\\	(もらいます:
\\	-(いただきます:
\\	幼稚園	母は孫に幼稚園まで迎えに 来てやっています。	幼稚園[ようちえん]				人1(私)は 人2に 
\\	て)やります・あげます・さしあげます。
\\	人1
\\	人2
\\	(やります:
\\	(あげます:
\\	-(さしあげます:
\\	暖房	ちょっと寒いですが。 暖房の温度 上げてくださいませんか。	暖房[だんぼう]			
\\	て ください 
\\	~てくれ。
\\	~て!)。 
\\	て いださいませんか。
\\	ていただけませんか 
\\	冷房	この 新しい冷房の 使い方が よくわからないんですが。ちょっと 教えて くださいませんか。 。。。いいですよ	冷房[れいぼう]			
\\	て ください 
\\	~てくれ。
\\	~て!)。 
\\	て いださいませんか。
\\	ていただけませんか 
\\	温度	ちょっと寒いですが。 暖房の温度 上げてくださいませんか。	温度[おんど]			
\\	て ください 
\\	~てくれ。
\\	~て!)。 
\\	て いださいませんか。
\\	ていただけませんか 
\\	体温	今朝から  私の 体温が  高かったので、 風を 引きでしょう。	体温[たいおん]					
\\	祖父	十年前に祖父に 
\\	へ送ってあげました	祖父[そふ]			
\\	人1(私)は 人2に 
\\	て)やります・あげます・さしあげます。
\\	人1
\\	人2
\\	(やります:
\\	(あげます:
\\	-(さしあげます:
\\	祖母	子どものとき 私は毎年祖母に セーターを  作って もらいました。	祖母[そぼ]				人1(私)は 人2に 
\\	て)もらいます・もらいます・いただきます。
\\	人1
\\	人2. (もらいます:
\\	(もらいます:
\\	-(いただきます:
\\	孫	来月の22日 
\\	へ孫を連れて行ってやります。	孫[まご]			
\\	人1(私)は 人2に 
\\	て)やります・あげます・さしあげます。
\\	人1
\\	人2
\\	(やります:
\\	(あげます:
\\	-(さしあげます:
\\	叔父	先週
\\	へ帰ったとき、 叔父と 叔母が迎えに来て もらいました	叔父[おじ]				人1(私)は 人2に 
\\	て)もらいます・もらいます・いただきます。
\\	人1
\\	人2. (もらいます:
\\	(もらいます:
\\	-(いただきます:
\\	叔母	先週 
\\	へ 帰ったとき、叔父と叔母が  迎えに  来て もらいました	叔母[おば]			
\\	人1(私)は 人2に 
\\	て)もらいます・もらいます・いただきます。
\\	人1
\\	人2. (もらいます:
\\	(もらいます:
\\	-(いただきます:
\\	管理人	管理人さん、すみません 。。。はい、なんですか 廊下の電気が つからないんですが。ちょっと見て くださいませんか。 。。。。ええ、いいですよ	管理[かんり] 人[じん]			
\\	お宅	日本のお宅は とても 美しいですよ。	お 宅[たく]					
\\	昔話	子どものとき、毎晩 私は 祖母に 昔話をしてもらいました	昔話[むかしばなし]				人1(私)は 人2に 
\\	て)もらいます・もらいます・いただきます。
\\	人1
\\	人2. (もらいます:
\\	(もらいます:
\\	-(いただきます:
\\	ある所	昔、ある所に ヒエップという 若い男がいました。ある日ヒエップさんは 子どもたちに いじめられている亀を 助けてやりました。 亀は 「助けていただいて、ありがとうございました」と 言って、ヒエップさんを 海のお城へ 連れて行ってくれました	ある 所[ところ]			
\\	子どもたち	毎日母は兄の子どもを世話しています	子[こ]どもたち					
\\	いじめます	昔、ある所に ヒエップという 若い男がいました。ある日ヒエップさんは 子どもたちに いじめられている亀を 助けてやりました。 亀は 「助けていただいて、ありがとうございました」と 言って、ヒエップさんを 海のお城へ 連れて行ってくれました	いじめます			
\\	亀	昔、ある所に ヒエップという 若い男がいました。ある日ヒエップさんは 子どもたちに いじめられている亀を 助けてやりました。 亀は 「助けていただいて、ありがとうございました」と 言って、ヒエップさんを 海のお城へ 連れて行ってくれました	亀[かめ]			
\\	助けます	昔、ある所に ヒエップという 若い男がいました。ある日ヒエップさんは 子どもたちに いじめられている亀を 助けてやりました。 亀は 「助けていただいて、ありがとうございました」と 言って、ヒエップさんを 海のお城へ 連れて行ってくれました	助[たす]けます					
\\	お姫様	そこには、 とてもきれいで、優しいお姫様がいました。ヒエップさんは 毎日 楽しく暮らしていましたが。家へ 帰りたくなりました。 帰るとき、お姫様は お土産に 箱をくれました。でも 絶対に 箱を 開けては いけないと 言いました。	お 姫様[ひめさま]			
\\	陸	ヒエップさんは 陸へ帰りましたが、どこにも 家はありませんでした。 道で会った人が300年ぐらい前に、 ヒエップのうちが あったと 教えてくれました。 ヒエップは 悲しくなって、お土産の箱を 開けました。すると、中から白い煙が 出て、ヒエップは 髪が 真っ白な おじいさんになりました	陸[りく]					
\\	煙	ヒエップさんは 陸へ帰りましたが、どこにも 家はありませんでした。 道で会った人が300年ぐらい前に、 ヒエップのうちが あったと 教えてくれました。 ヒエップは 悲しくなって、お土産の箱を 開けました。すると、中から白い煙が 出て、ヒエップは 髪が 真っ白な おじいさんになりました	煙[けむり]					
\\	真っ白	ヒエップさんは 陸へ帰りましたが、どこにも 家はありませんでした。 道で会った人が300年ぐらい前に、 ヒエップのうちが あったと 教えてくれました。 ヒエップは 悲しくなって、お土産の箱を 開けました。すると、中から白い煙が 出て、ヒエップは 髪が 真っ白な おじいさんになりました	真っ白[まっしろ]				な	
\\	中身	お土産の箱の中身は白い煙でした。	中身[なかみ]					
\\	包みます	赤いふろしきはプレゼントを包むのに使います	包[つつ]みます			
\\	る・
\\	の 
\\	のに~。
\\	使います、いいです、便利です、役に立ちます、時間がかかります。	
\\	ふろしき	ふろしきは 結婚式のお祝いを包むのに 使います	ふろしき				
\\	る・
\\	の 
\\	のに~。
\\	使います、いいです、便利です、役に立ちます、時間がかかります。		
\\	沸かします	この薬缶はお湯を沸かすのに使います。 。。。はい、これは沸かすのに便利ですね	沸[わ]かします			
\\	る・
\\	の 
\\	のに~。
\\	使います、いいです、便利です、役に立ちます、時間がかかります。	
\\	薬缶	この薬缶はお湯を沸かすのに使います。 。。。はい、これは沸かすのに便利ですね	薬缶[やかん]			
\\	る・
\\	の 
\\	のに~。
\\	使います、いいです、便利です、役に立ちます、時間がかかります	
\\	混ぜます	あのミキサーは  
\\	の 材料を混ぜるのに使います.	混[ま]ぜます			
\\	る・
\\	の 
\\	のに~。
\\	使います、いいです、便利です、役に立ちます、時間がかかります。	
\\	ミキサー	あのミキサーは  
\\	の 材料を まぜるのに  使います.	ミキサー			
\\	る・
\\	の 
\\	のに~。
\\	使います、いいです、便利です、役に立ちます、時間がかかります。		
\\	材料	あのミキサーは
\\	の材料を混ぜるのに使います.	材料[ざいりょう]			
\\	る・
\\	の 
\\	のに~。
\\	使います、いいです、便利です、役に立ちます、時間がかかります。	
\\	電卓	この電卓は三角の面積を計算するのに使ってください。	電卓[でんたく]			
\\	る・
\\	の 
\\	のに~。
\\	使います、いいです、便利です、役に立ちます、時間がかかります。	
\\	並びます	コンサートのチケットを買うために、並んでいるんですが。	並[なら]びます			
\\	る・
\\	の 
\\	ために、~。
\\	ために)
\\	アパート	今私は4S
\\	のアパートに住んでいます。家賃は少し安いですので、給料は  半分で貯金できます	アパート				
\\	る・
\\	の 
\\	ために、~。
\\	ために)
\\	弁護士	海外で大学に 通う 目的は何ですか。 いい 弁護士に なくために、海外で法律を勉強したいです	弁護士[べんごし]			
\\	る・
\\	の 
\\	ために、~。
\\	ために)
\\	音楽家	将来どんな 仕事を したいですか。 私は 歌曲をつくるのが大好きで から。 音楽家になるために、 今から
\\	の音楽を  研究して います。	音楽家[おんがくか]			
\\	る・
\\	の 
\\	ために、~。
\\	ために)
\\	自然	なぜ田舎に引越しするんですか。 自然の中で子どもを育てます	自然[しぜん]					
\\	教育	へえ、独身の人はいいですね。全部 自分のために、使えて。私は家のローンを 払って、子どもの教育のために、貯金したら。ほとんど 残りませんよ	教育[きょういく]			
\\	る・
\\	の 
\\	ために、~。
\\	ために)
\\	文化	日本へ来た目的は何ですか。 日 。。。本の芸術を勉強するのために、 大学で文化と日本語を研究にきました。	文化[ぶんか]			
\\	る・
\\	の 
\\	ために、~。
\\	ために)
\\	言語学	日本[にっぽん]へ  来[き]た  目的[もくてき]は  何[なに]ですか。 言語[げんご] 学[がく]のために、 大学[だいがく]で  文化[ぶんか]と 日本語[にほんご]を  研究[けんきゅう]に きました。	言語[げんご] 学[がく]					
\\	社会	今では ベトナムの社会がとても複雑ですよ。 国のために、私たちは一生懸命 勉強しなければなりません。	社会[しゃかい]			
\\	る・
\\	の 
\\	ために、~。
\\	ために)
\\	政治	
\\	大統領は偉い政治家ですよ。首相になくために、政治学を勉強しています。	政治[せいじ]				
\\	る・
\\	の 
\\	ために、~。
\\	ために)
\\	戦争	1954年にベトナムの戦争は とても激しいです。	戦争[せんそう]			
\\	平和	来週 どんな仕事を したいんですが。 。。。世界の平和のために、国連の仕事を したいんです そうですか、いいですね	平和[へいわ]			
\\	る・
\\	の 
\\	ために、~。
\\	ために)
\\	目的	地球へ来た目的は何ですか 私は世界の平和を守るために来ました。	目的[もくてき]			
\\	る・
\\	の 
\\	ために、~。
\\	ために)
\\	論文	論文を書くために、資料を たくさん集めていますよ。 。。。それは大変ですね。	論文[ろんぶん]			
\\	る・
\\	の 
\\	ために、~。
\\	ために)
\\	楽しみ	今楽しみは 何ですか。 私の楽しみは日本教室へ来て、みんなと 
\\	先に会います。	楽[たの]しみ			
\\	ふた	まず どんぶりに めん、卵、炒めた野菜を入れます。 次にどんぶりに お湯を入れて、ふたをして、3分間待ちます。 3分だったら、ふたを開けて、細かく切ったねぎをのせて、できあります	ふた			
\\	栓抜き	この瓶を 開ける物は何です。 その栓抜きは開けるのに使います。	栓[せん] 抜[ぬ]き			
\\	る・
\\	の 
\\	のに~。
\\	使います、いいです、便利です、役に立ちます、時間がかかります。	
\\	缶詰	この  
\\	缶詰は どうやって  開けるか、教えて ください。 あの 缶切りは 開けすのに 使ってください	缶詰[かんづめ]			
\\	る・
\\	の 
\\	のに~。
\\	使います、いいです、便利です、役に立ちます、時間がかかります。	
\\	のし袋	日本人の結婚のお祝いは何にいいですか。 お祝いのお金がいいです。 ここ のし袋はお祝いのお金を入れるのに使います	のし 袋[ぶくろ]				
\\	る・
\\	の 
\\	のに~。
\\	使います、いいです、便利です、役に立ちます、時間がかかります。	
\\	体温計	何で 体温を測りますか。 体温計は測るのに使います	体温計[たいおんけい]			
\\	る・
\\	の 
\\	のに~。
\\	使います、いいです、便利です、役に立ちます、時間がかかります。	
\\	どのくらい	大学の 論文を書くのにどのくらいかかりますか。 四か月必要です。	どのくらい			
\\	る・
\\	の 
\\	のに~。
\\	使います、いいです、便利です、役に立ちます、時間がかかります。		
\\	国連	将来どんな 仕事を したいんですか。 世界の平和のために国連の仕事を働きたいです。	国連[こくれん]			
\\	る・
\\	の 
\\	ために、~。
\\	ために)
\\	ボーナスが出ます	林さん、ボーナスは いつ 出るんですか。 。。。来週です。鈴木さんの 会社は? 明日です。楽しみですね	ボーナスが 出[で]ます			
\\	ローン	へえ、独身の人はいいですね。全部 自分のために、使えて。私は家のローンを 払って、子どもの教育のために、貯金したら。ほとんど 残りませんよ	ローン			
\\	る・
\\	の 
\\	ために、~。
\\	ために)
\\	カップめん	そして、1971年に カップめんが 生まれました。お湯が あれば、いつでも どこででも 作れて、忙しい人が食べるのにとても便利です	カップ[かっぷ]めん			
\\	る・
\\	の 
\\	のに~。
\\	使います、いいです、便利です、役に立ちます、時間がかかります。		
\\	どんぶり	まず どんぶりに めん、卵、炒めた野菜を入れます。 次にどんぶりに お湯を入れて、ふたをして、3分間待ちます。 3分だったら、ふたを開けて、細かく切ったねぎをのせて、できあります	どんぶり						
\\	い→かれます;
\\	られます;
\\	されます-こられます)
\\	めん	インスタントラーメンは 1958年に「あんどうもふく」によって 発明されました。どんぶりに めんと お湯を 入れて、 3分待つと、食べられるので、とても便利な食べ物です。	めん				
\\	は 人によって 
\\	(書かれます、発明されます、発見されます、)		
\\	い→かれます;
\\	られます;
\\	されます-こられます)
\\	広めます	あんどうさんは このチキンラーメンを 世界中に 広めたいと思いました。そして、市場調査のために、アメリカへ 行きました	広[ひろ]めます			
\\	る・
\\	の 
\\	ために、~。
\\	ために)
\\	市場調査	あんどうさんは このチキンラーメンを 世界中に 広めたいと思いました。そして、市場調査のために、アメリカへ 行きました	市場[しじょう] 調査[ちょうさ]				
\\	る・
\\	の 
\\	ために、~。
\\	ために)
\\	注ぎます	このワインは栓抜きであけたら、コップに注いでくださいませんか。	注[そそ]ぎます			
\\	ホッチキス	この書類を 閉じていただけませんか。 。。。はい、いいですよ あぁ、このホッチキスは閉じるのに使って ください。	ホッチキス			
\\	る・
\\	の 
\\	のに~。
\\	使います、いいです、便利です、役に立ちます、時間がかかります。		
\\	閉じます	この書類を 閉じていただけませんか。 。。。はい、いいですよ あぁ、このホッチキスは閉じるのに使って ください。	閉[と]じます			
\\	る・
\\	の 
\\	のに~。
\\	使います、いいです、便利です、役に立ちます、時間がかかります。	
\\	クリップ	この会議の資料をクリップで挟みましょうか。	クリップ			
\\	る・
\\	の 
\\	のに~。
\\	使います、いいです、便利です、役に立ちます、時間がかかります。		
\\	挟みます	この 会議の資料をクリップで挟みましょうか。	挟[はさ]みます			
\\	る・
\\	の 
\\	のに~。
\\	使います、いいです、便利です、役に立ちます、時間がかかります。		
\\	画びょう	ここに時刻表をとめてください。 ああ、その画びょうは留めるのにつかいます。	画[が]びょう			
\\	る・
\\	の 
\\	のに~。
\\	使います、いいです、便利です、役に立ちます、時間がかかります。		
\\	留めます。	ここに  時刻表を留めて ください。ああ、 その画びょうは留めるのに使います。	留[と]めます。				
\\	る・
\\	の 
\\	のに~。
\\	使います、いいです、便利です、役に立ちます、時間がかかります	
\\	カッター	はさみは ふろしきを切るのに使わないで ください。カッターはだけ切るのに 使います	カッター			
\\	る・
\\	の 
\\	ために、~。
\\	ために)
\\	鋏	鋏はふろしきを切るのに使わないでください。カッターは だけ切るのに使ってもいいです。	鋏[はさみ]う			
\\	る・
\\	の 
\\	のに~。
\\	使います、いいです、便利です、役に立ちます、時間がかかります。		
\\	セロテープ	セロテープで箱をはっていただけませんか。糊は はるのに使ってもいいですか。 。。。はい、どちらもいいです。	セロテープ				
\\	る・
\\	の 
\\	のに~。
\\	使います、いいです、便利です、役に立ちます、時間がかかります。		
\\	糊	セロテープで箱をはっていただけませんか。糊は はるのに使ってもいいですか。 。。。はい、どちらもいいです。	糊[のり]			
\\	る・
\\	の 
\\	のに~。
\\	使います、いいです、便利です、役に立ちます、時間がかかります。	
\\	貼ります	セロテープで箱をはっていただけませんか。糊は はるのに使ってもいいですか。 。。。はい、どちらもいいです。	貼[は]ります			
\\	る・
\\	の 
\\	のに~。
\\	使います、いいです、便利です、役に立ちます、時間がかかります。	
\\	ガムテープ	ガムテープは怪我を貼るのに使ったほうがいいですか。 はい、いいです。	ガムテープ				
\\	る・
\\	の 
\\	のに~。
\\	使います、いいです、便利です、役に立ちます、時間がかかります。		
\\	鉛筆削り	試験に出るとき 鉛筆削りを持っていきます。鉛筆削りは削るのに使ってください。 。。。はい、分かりました。	鉛筆[えんぴつ] 削[けず]り			
\\	る・
\\	の 
\\	のに~。
\\	使います、いいです、便利です、役に立ちます、時間がかかります。	
\\	定規	四角の面積を計算するために、定規はあの線を測るのに使います。 。。。はい、分かりました	定規[じょうぎ]			
\\	る・
\\	の 
\\	のに~。
\\	使います、いいです、便利です、役に立ちます、時間がかかります。	
\\	のこぎり	のこぎりは あの机を切るのに使ます	のこぎり			
\\	る・
\\	の 
\\	のに~。
\\	使います、いいです、便利です、役に立ちます、時間がかかります。		
\\	金づち	金づちは 本棚に釘を打つのに使います	金[かな]づち			
\\	る・
\\	の 
\\	のに~。
\\	使います、いいです、便利です、役に立ちます、時間がかかります。	
\\	ペンチ	ペンチは鉄線索を曲げるのに使います。	鉄線[てっせん] 策[さく]				
\\	る・
\\	の 
\\	のに~。
\\	使います、いいです、便利です、役に立ちます、時間がかかります。	
\\	消しゴム	日本消しゴムは 美術を勉強するのに とても便利です	消しゴム[けしごむ]			
\\	る・
\\	の 
\\	のに~。
\\	使います、いいです、便利です、役に立ちます、時間がかかります。	
\\	下がります	
\\	年の 4月に  
\\	に入って、うれしいでしたね。 初めは  将来が 明るいでしたが、でも 2014年末に 石油の値段が 下がったから、これからは 仕事が つまらないでした。 それ 以来 キャリアが 暗くてなりました。	下[さ]がります			
\\	増えます。	ベトナムには いい方法は稲を植えるのに使ったので、 米の 輸出が増えそうですね。 これからも 東京の人口は 増えているんです。 。。。はい、最近のデータによると、増えているんだそうです	増[ふ]えます。			
\\	ます 
\\	ます そうです。
\\	今にも もうすぐ これから)	
\\	減ります	2014年から アメリカとISの 戦争があるので、  石油の 輸出が減りますね。	減[へ]ります			
\\	上がります	ベトナムでは 今にもガソリンの 値段が上がると、 何でもの 値段が 上がります。 でも ガソリンが下がったとき、 他の値段を そのままにします。 ベトナムは とても へんな 国ですね。	上[あ]がります			
\\	変	ベトナムでは 今にもガソリンの 値段が上がると、 何でもの 値段が 上がります。 でも ガソリンが下がったとき、 他の値段を そのままにします。 ベトナムは とても へんな 国ですね。	変[へん]			
\\	切れます	
\\	さん。 。。。はい もうすぐ かばんの紐が切れそうですよ。 。。。あ、ほんとうだ。気がつけませんでした	切[き]れます			
\\	ます 
\\	ます そうです。
\\	今にも もうすぐ これから)	
\\	とれます	君、このボタンを仕立て直してくれませんか。  これは とれそうですよ。	とれます				
\\	ます 
\\	ます そうです。
\\	今にも もうすぐ これから) 
\\	て くれませんか。
\\	ください、
\\	いただけませんか。		
\\	落ちます。	
\\	さん、 。。。はい、何ですか。 バイクから 荷物が 落ちそうですよ 。。。あ、本当だ。気がつきませんでした。どうも	落[お]ちます。			
\\	ます 
\\	ます そうです。
\\	今にも もうすぐ これから)	
\\	なくなります	ガソリンがなくなりそうです。ガソリンの店を探して、 速く入れましょう 。。。そうですね 皆さんは ここに 時々 靴を置いて、 教室が 終わったら、 靴が なくなってしましました。 泥棒に 注意して ください。	なくなります			
\\	ます 
\\	ます そうです。
\\	今にも もうすぐ これから)		
\\	幸せ。	あの二人、幸せそうですね。 。。。ええ、先月 結婚したんです そうですか	幸[しあわ]せ。			
\\	い 
\\	い・
\\	な 
\\	な 
\\	そうです。
\\	楽	会社が終わったら、楽に 風呂に入って、のんびりしたいな 。。。はい、いいですね	楽[らく]			
\\	うまい	だれの声が うまそうですね。 。。。HIEU先生のようですよ。 あの 妻が 作ったご飯ですね。どうぞ。 。。。へい、 うまそうですね。 ええ ありがとうごさいます。	うまい			
\\	い 
\\	い・
\\	な 
\\	な 
\\	そうです。
\\	つまらない	あの 本は つまらなそうですね。いくらなんですか。 。。。そうですか。	つまらない			
\\	い 
\\	い・
\\	な 
\\	な 
\\	そうです。
\\	火	あのピクニックの 火が  消えそうですね。すぐ ガソリンに入れてください。	火[ひ]			
\\	ます 
\\	ます そうです。
\\	今にも もうすぐ これから) 
\\	て くれませんか。
\\	ください、
\\	いただけませんか。	
\\	パンフレット	ちょっと 市役所へ 行ってきます。 。。。じゃ、日本語の パンフレットを 持ってきて くれませんか いいですよ	パンフレット			
\\	~て来ます。
\\	て くれませんか。
\\	ください、
\\	いただけませんか。		
\\	明るい。	
\\	年の 4月に  
\\	に入って、うれしいでしたね。 初めは  将来が 明るいでしたが、でも 2014年末に 石油の値段が 下がったから、これからは 仕事が つまらないでした。 それ 以来 キャリアが 暗くてなりました。	明[あか]るい。					
\\	暗い。	
\\	年の 4月に  
\\	に入って、うれしいでしたね。 初めは  将来が 明るいでしたが、でも 2014年末に 石油の値段が 下がったから、これからは 仕事が つまらないでした。 それ 以来 キャリアが 暗くてなりました。	暗[くら]い。					
\\	おとなしい。	
\\	先生について どう 思いますか。  
\\	先生は おとなしい 先生ですね。いつも 困ったことがあった 学生を  熱心に手伝っています。とても 下手な学生を 気が 長く 教えま。	おとなしい			
\\	厳しい	
\\	先生は 厳しい先生だそうですね。 。。。ええ。でも、授業は とても おもしろいですよ。	厳[きび]しい				
\\	そうです。
\\	によると ~そうです)	
\\	気が長い.	
\\	先生について どう 思いますか。  
\\	先生は おとなしい 先生ですね。いつも 困ったことがあった 学生を  熱心に手伝っています。とても 下手な学生を 気が 長く 教えま。	気[き]が 長[なが]い				い	
\\	気が短い	子どものときは 仕事で よく気が 短いです。 。。。それは よくないでしたね。	気[き]が 短[みじか]い			
\\	気が弱い	
\\	さんは 優しくて、 活発です。でも、彼は とても 気が弱いですから、 虫と暗がりを怖ります。	気[き]が 弱[よわ]い			
\\	活発	
\\	さんを 知って いますか。
\\	さんは優しくて、 活発です。仕事で  極めて まじめです	活発[かっぱつ]				な	
\\	気が強い。	
\\	さんは 気が 強いですね。2年間癌があったんですが。でも、 毎日いつも楽しみです。	気[き]が 強[つよ]い。			
\\	誠実	
\\	さんについて どう 思いますか。 
\\	さんは 優しくて、気が 長くて、それに とても  誠実ですね。 今は  
\\	会社で  働いて います。	誠実[せいじつ]					
\\	わがまま	
\\	さんは わがままな人ですね。	わがまま			
\\	な		
\\	まじめ	
\\	さんを 知って いますか。
\\	さんは優しくて、 活発です。仕事で  極めて まじめです	まじめ						
\\	ふまじめ	
\\	さんは 教室へ 来たとき、いつも すっと寝ましたね、勉強するのが ふまじめでした。でも  試験の点は いつも高いでした。	ふまじめ						
\\	頑固	
\\	さんは 頑固な人ですよ。 
\\	先生の アドバイスのとおりに,何も しません。	頑固[がんこ]				な	
\\	意地悪	
\\	さんは 意地悪な人ですから。HOAさんと友達にならなかった方がいいですよ	意地悪[いじわる]			
\\	勝気な	中学校のとき、学生の 中で 
\\	さんは  一番勝気な 学生でしたね。 相談するとき、彼は いつも  見解を 守って、 とても おもしろいでした。	勝気[かちき]な					
\\	神経質	
\\	さんは 神経質な 人ですよ。BAOさんは教室で  私たちが  女性について  話して、 がっかりしたら、すぐ やめました	神経質[しんけいしつ]			
\\	な	
\\	笑います	泣いているんですか 。。。いいえ、笑いすぎて、涙がでたんです そうですか。	笑[わら]います			
\\	ます 
\\	ます
\\	にくい・やすいです。
\\	泣きます	眠そうですね。 。。。ええ、実は 子どもが毎晩 泣いて、寝られないんです。 そうですか。それは 大変ですね	泣[な]きます				
\\	い 
\\	い・
\\	な 
\\	な 
\\	そうです。
\\	乾きます	この間 天気予報を 見て、 明日雨が降るでしょう、また あのコートが厚すぎですよ。それで 明日は あのコートが乾きにくいですから。 今日洗濯してください。 このタオルは 洗濯したばかりなので、まだ 乾いていません	乾[かわ]きます			
\\	ます 
\\	ます
\\	にくい・やすいです。
\\	ます 
\\	ます・
\\	い
\\	い・
\\	な
\\	な
\\	すぎます。
\\	た ばかりです。
\\	起きます	事故は 雨の日に起きやすいです。気をつけてください 。。。はい、わかりました 火事が起きた場合は、すぐ 119番に 連絡します。 。。。はい、分かりました	起[お]きます			
\\	起こります	
\\	ます 
\\	ます
\\	にくい・やすいです。
\\	る、ない、た・
\\	い、な・
\\	の
\\	場合は、~。
\\	調節します	このテーブルはとても便利ですね。 大きさと高さが  調節できて、 使いやすいですよ。 。。。いいですね	調節[ちょうせつ]します				
\\	ます 
\\	ます
\\	にくい・やすいです。
\\	安全	この コップは 丈夫で 割れにくいですよ。 。。。子どもが 使うのに 安全で、いいですね。	安全[あんぜん]				
\\	ます 
\\	ます
\\	にくい・やすいです。
\\	薄い	この 字は薄いので、 濃くしてください 日本の 料理の 味は 薄いので、体にいいです。ベトナムの 料理は濃いので、体に  悪いと 思います	薄[うす]い				
\\	を
\\	い
\\	い
\\	く。
\\	な
\\	な
\\	に。
\\	に)します。
\\	濃い	日本の 料理の 味は 薄いので、体にいいです。ベトナムの 料理は濃いので、体に  悪いと 思います この 字は薄いので、 濃くしてください	濃[こ]い				
\\	を
\\	い
\\	い
\\	く。
\\	な
\\	な
\\	に。
\\	に)します。
\\	厚い	明日雨が降るでしょう、また あのコートが厚すぎですよ。それで 明日は あのコートが乾きにくいですから。 今日洗濯してください。 厚いタオルは 乾きにくいです	厚[あつ]い				
\\	ます 
\\	ます
\\	にくい・やすいです。
\\	ます 
\\	ます・
\\	い
\\	い・
\\	な
\\	な
\\	すぎます。
\\	太い	すみません、ちょっと 教えて くれませんか。 。。。ええ、何ですか。この 線を 太くしたいんですが。どうすれば いいですか。 。。。ここをクリックすれば いいですよ。 なるほど、どうも	太[ふと]い				
\\	を
\\	い
\\	い
\\	く。
\\	な
\\	な
\\	に。
\\	に)します。
\\	空気	田舎と 町と どちらが 住みやすいですか。 。。。田舎のほうが 住みやすいと 思います。物価も安いし、空気も きれいですから そうですか。私は そうと おもいます	空気[くうき]				
\\	ます 
\\	ます
\\	にくい・やすいです。
\\	涙	泣いているんですか。 。。。いいえ、笑いすぎて、 涙が出たんです。 そうですか	涙[なみだ]				
\\	ます 
\\	ます・
\\	い
\\	い・
\\	な
\\	な
\\	すぎます。
\\	和食	昼ごはんは どこか 行きませんか。 和食と 洋食どちらに しますか。 。。。ええと、洋食にします。ミラーさんは? 私は 和食にしますが。私は 和食を 食べたことがありませんが 。。。ああ、ミラーさは和食になります。和食に行きましょう	和食[わしょく]				
\\	にします。
\\	になる。
\\	洋食	昼ごはんは どこか 行きませんか。 和食と 洋食どちらに しますか。 。。。ええと、洋食にします。ミラーさんは? 私は 和食にしますが。私は 和食を 食べたことがありませんが 。。。ああ、ミラーさは和食になります。和食に行きましょう	洋食[ようしょく]				
\\	にします。
\\	になる。
\\	おかず	ベトナムには家族の 食事は ご飯とおかずが 必ず あります	おかず						
\\	量	体によくなりたいんですが。コーヒーの 量を  半分に しなければなりません。	量[りょう]				
\\	を
\\	い
\\	い
\\	く。
\\	な
\\	な
\\	に。
\\	に)します。
\\	ー倍	すみません、ちょっと 教えて くれませんか。 。。。ええ、何ですか。この 字を 2倍にしたいんですが。どうすれば いいですか。 。。。ここをクリックすれば いいですよ。 なるほど、どうも	ー 倍[ばい]				
\\	を
\\	い
\\	い
\\	く。
\\	な
\\	な
\\	に。
\\	に)します。
\\	シングル	
\\	年前TOANさんと 
\\	へ 旅行に  行きました。TOANさんは シングルに なりたいんでした。でも  彫金したいので、ツインに しました。30 分ごろ けんかしましたから、 最後にツインに 泊まりました。	シングル				
\\	にします。
\\	になる。
\\	ツイン	
\\	年前TOANさんと 
\\	へ 旅行に  行きました。TOANさんは シングルに なりたいんでした。でも  彫金したいので、ツインに しました。30 分ごろ けんかしましたから、 最後にツインに 泊まりました。	ツイン				
\\	にします。
\\	になる。
\\	洗濯物	毎週末洗濯物が たくさんありますよ。 。。。なぜですか。 ウィクーデーは  横着して、 洗濯しなかったので、たくさんあります。	洗濯[せんたく] 物[ぶつ]					
\\	嫌がります。	結婚式のスピーチを 頼まれたことがりますか。スピーチは 長すぎると、皆さんに嫌がられました。 また、短すぎると、お祝いの気持ちがうまく伝えられません。難しいですね	嫌[いや]がります。				
\\	ます 
\\	ます・
\\	い
\\	い・
\\	な
\\	な
\\	すぎます。
\\	また。	明日雨が降るでしょう、また あのコートが厚すぎですよ。それで 明日は あのコートが乾きにくいですから。 今日洗濯してください。	また。			
\\	ます 
\\	ます
\\	にくい・やすいです。
\\	ます 
\\	ます・
\\	い
\\	い・
\\	な
\\	な
\\	すぎます。
\\	順序	明日 会社人に  大切なレポートを  発表するのがありますが。レポートの 順序をまちがえないたり、 忘れないたりするために、 今晩 綿密に  準備しておいて ください。	順序[じゅんじょ]					
\\	別れります	結婚式スピーチは 使ってはいけないことばがあります。 例えば 
\\	別れる」とか、
\\	切れる」とかです。これらは  縁起が 悪いので、 使いません。気をつけましょう	別[わか]れります					
\\	縁起が悪い	もうすぐ 
\\	になって、 縁起が 悪い物を  捨てた ほうがいいですよ。 。。。たとえばね 
\\	とか、
\\	とかです	縁起[えんぎ]が 悪[わる]い					
\\	掃除機	新しい掃除機は軽くて、 使いやすいですよ.	掃除[そうじ] 機[き]				
\\	ます 
\\	ます
\\	にくい・やすいです。
\\	苦い	このスープは苦すぎて、 食べにくいですよ。	苦[にが]い				
\\	ます 
\\	ます・
\\	い
\\	い・
\\	な
\\	な
\\	すぎます。
\\	ます 
\\	ます
\\	にくい・やすいです。
\\	管理します	夏は 食べ物が腐りやすいですから、 冷蔵庫で  管理したほうがいいです。 。。。はい、わかりました。どうも	管理[かんり]します			
\\	ます 
\\	ます
\\	にくい・やすいです。
\\	煩い	教室には 煩いので、 静かにしてください。	煩[うるさ]い				
\\	を
\\	い
\\	い
\\	く。
\\	な
\\	な
\\	に。
\\	に)します。
\\	信じます	田中さん、今度 結婚するんですよ。 。。。えっ、信じられませんね。あんなに 独身のほうが いいと言っていたのに そうですか	信[しん]じます				
\\	い、な・
\\	な
\\	のに、~。
\\	キャンセルします	来週用事がありますが。ホテルの 予約をキャンセルしたい場合は、どうすれば いいですか。 。。。はい、メールで ホテルのキャンセルについて 依頼して、返事をもらいました後で、確認の 番号をインポートしてください。 確認が完成すると、キャンセルできます	キャンセルします			
\\	知らせます	忘年会へ 家族を 連れて行くかどか、3日前 書記に メールで 知らせてください	知[し]らせます					
\\	保証書	これが この洗濯機の 保証書です。調子が悪い場合はこの番号に連絡してください。 。。。はい、わかりました	保証書[ほしょうしょ]			
\\	る、ない、た・
\\	い、な・
\\	の
\\	場合は、~。
\\	領収書	あのう、この図書館では コピーの領収書が もらえますか。 。。。ええ、必要な場合は、言ってください。	領収[りょうしゅう] 書[しょ]			
\\	る、ない、た・
\\	い、な・
\\	の
\\	場合は、~。
\\	キャンプ	キャンプの予定は以上です。何か質問がありますか。 。。。急に都合が悪く場合は、どうしたらいいですか。 その場合は電話で連絡しください 。。。はい、わかりました	キャンプ					
\\	中止	新聞によると来週
\\	に大雨が降るそうですから。楽しみにしていたのに、家族と旅行に 行くのが 中止になってしまいました。	中止[ちゅうし]			
\\	い、な・
\\	な
\\	のに、~。
\\	点	一生懸命 日本語を 勉強したのに、N4の試験の点が 悪かったです。	点[てん]			
\\	い、な・
\\	な
\\	のに、~。
\\	急に	キャンプの 予定は 以上です。何か 質問が ありますか。 。。。急に 都合が悪く場合は、 どう したら いいですか。 その場合は 電話で 連絡し ください 。。。はい、わかりました。	急[きゅう]			
\\	る、ない、た・
\\	い、な・
\\	の
\\	場合は、~。
\\	楽しみにしています	新聞によると来週
\\	に大雨が降るそうですから。楽しみにしていたのに、家族と旅行に 行くのが 中止になってしまいました。	楽[たの]しみに しています			
\\	い、な・
\\	な
\\	のに、~。
\\	以上です	今 キャンプの予定を 発表するところです。 明日 5時に 工科大学の前に 来て、バスに乗って、出発します。 キャンプの予定は 以上です。何か 質問が ありますか	以上[いじょう]です					
\\	係員	コピー機の調子が悪い場合は、すぐ係員に連絡してください	係員[かかりいん]			
\\	運動会	来週
\\	会社の運動会がありますが。皆さん、私たちは運動会を参加する目的は健康ですから、またけがをしないように、無理をしないでください。 もし気分が悪くなった場合は、すぐ 係員に言ってください。	運動会[うんどうかい]				
\\	い、な・
\\	な
\\	のに、~。
\\	2位	テニス会はどうでしたか。 。。。2位でした 2位だったんですか。すごいですね。 。。。いいえ、一生懸命 練習したのに、優勝できなくて、残念です。 来年が ありますが、頑張れ!	
\\	位[い]				
\\	い、な・
\\	な
\\	のに、~。
\\	優勝します	テニス会はどうでしたか。 。。。2位でした 2位だったんですか。すごいですね。 。。。いいえ、一生懸命 練習したのに、優勝できなくて、残念です。 来年が ありますが、頑張れ!	優勝[ゆうしょう]します				
\\	い、な・
\\	な
\\	のに、~。
\\	悩み	僕の悩みは体重を下げるようにできません。 毎日早く起きて、バイクをして、プールへ泳ぎに行っているんです。食事は果物と野菜だけです。 一生懸命運動したのに、体重をそのままにしてしました。どうすれば いいですか。	悩[なや]み			
\\	い、な・
\\	な
\\	のに、~。
\\	目覚まし時計	3つの目覚まし時計は 違う時間に 鳴るように、セットして、いろいろな所に 置いておきます。 時計が鳴ると、起きて、止めに 行かなければならないので、目が覚めますよ。	目覚[めざ]まし 時計[どけい]					
\\	目が覚めます	夜は 早く寝るようにしていますが、朝起きられるかどうか、心配で、なかなか 眠れません。 隣の部屋の友達は「毎朝 君の目覚ましで 目が覚める」と 言っていますが、僕は 気がつきません 気がついても、止めて、また ねてしまうんです。	目[め]が 覚[さ]めます					
\\	回答	毎日、山田先生の授業は 私たちが ある問題について 相談して、回答して、とても おもしろいです	回答[かいとう]					
\\	鳴ります	3つの目覚まし時計は 違う時間に 鳴るように、セットして、いろいろな所に 置いておきます。 時計が鳴ると、起きて、止めに 行かなければならないので、目が覚めますよ。	鳴[な]ります				
\\	い、な・
\\	な
\\	のに、~。
\\	セットします	3つの目覚まし時計は 違う時間に 鳴るように、セットして、いろいろな所に 置いておきます。 時計が鳴ると、起きて、止めに 行かなければならないので、目が覚めますよ。	セットします				
\\	い、な・
\\	な
\\	のに、~。
\\	それでも	それでもだめな場合は、妻に起こしてもらってください	それでも			
\\	る、ない、た・
\\	い、な・
\\	の
\\	場合は、~。
\\	交通事故	交通事故に会った場合は、 警察に  連絡してください 交通事故が 一番多いのは 何月ですか。 。。。警察の発表によると、12月だそうです	交通[こうつう] 事故[じこ]				
\\	る、ない、た・
\\	い、な・
\\	の
\\	場合は、~。
\\	そうです。
\\	によると ~そうです)	
\\	調子	コピー機の調子が悪い場合は、係員に 連絡してください	調子[ちょうし]				
\\	る、ない、た・
\\	い、な・
\\	の
\\	場合は、~。
\\	夫	夫が 料理が上手なのに、 あまり作ってくれません。 夫は さっき帰って来たばかりなので、まだ 晩御飯を食べていません	夫[おっと]				
\\	る、ない・
\\	い、な・
\\	の
\\	はずです、~。
\\	押します	火急の 場合には、この 赤いボタンを  押して, 安全な 所へ  行ってください。	押[お]します					
\\	火急	工場の中で 矢印のとおりに、言ってください。 火急の場合には、 この赤いボタンを押して, 安全な所へ行け。 。。。はい、分かりました	火急[かきゅう]				
\\	い、な・
\\	な
\\	のに、~。
\\	間に合います	東京便は 間に合わない場合は、どう したら いいですか。 。。。すぐ 空港員に連絡してください	間に合[まにあ]います					
\\	以下.	ベトナム便のビジネスのチケットは 荷物が 40キロ以下です。 40キロ以上の場合は、どうしたら いいですか  持って行きたければ、罰金を 払わなければなりません	以下[いか]					
\\	熱	いい薬を飲んだのに、なかなか熱が下がりません	熱[ねつ]				
\\	い、な・
\\	な
\\	のに、~。
\\	保険証	入院するとき、保険証が 必要なので、持って 来るのを 忘れないで ください。 。。。はい、わかりました 病院へ行くとき、保険証を忘れないでください 。。はい、わかりました。ありがとうございます。	保険[ほけん] 証[しょう]				
\\	る・た
\\	とき、~。
\\	る 
\\	た
\\	汚い.	この床は とても汚いので、きれいにしてください	汚[きたな]い					
\\	スイッチを入れます	スイッチを入れたのに、パソコンが動かないんですが. どうすれば いいですか	スイッチを 入[い]れます				
\\	い、な・
\\	な
\\	のに、~。
\\	足ります	
\\	さん、結婚式の飲み物は足りましたか。 いいえ、たくさん買っておいたのに、足りませんでした。	足[た]ります				
\\	い、な・
\\	な
\\	のに、~。
\\	渡します	来週 
\\	さんは 東京へ 着たばかりなので、道が 分かるでしょうか。 。。。分かるはずですよ。昨日 地図を 渡しましたから。 じゃ 大丈夫ですね。	渡[わた]します				
\\	た ばかりです。
\\	帰ってきます	妻 さっき 帰って来たばかりなので、晩ご飯を食べていません。今 私は料理を作るところなんです。後で話しましょう。 。。。はい、わかりました	帰[かえ]ってきます				
\\	た ばかりです。
\\	バスが出ます	8時のバスは もう 来ましたか。 。。。たった今 出たところです	バスが 出[で]ます				
\\	た ところです。 
\\	(たったいま)	
\\	届きます	郵便局から メールをもらいますから、荷物は 明日 届くはずです。 荷物は もう 届きましたか。 。。。たった今 着いたところです	届[とど]きます			
\\	る、ない・
\\	い、な・
\\	の
\\	はずです、~。
\\	た ところです。 
\\	(たったいま)	
\\	入学します	先週 弟は 工科大学に入学したばかりなので、まだ 大学に慣れていません。	入学[にゅうがく]します				
\\	た ばかりです。
\\	卒業	私は 2014年に 大学を卒業したばかりです。 。。。そうですか。2017までの3年間で、何をしていますか	卒業[そつぎょう]				
\\	た ところです。 
\\	(たったいま)	
\\	焼きます	新年会は 始まりましたか。 。。。いいえ、ちょうど 始まる所です、早く来てください。 肉が焼けましたか。 。。。いいえ、私たちは 焼いている ところです いい匂いがしますね。 。。。ケーキを 焼いているようですよ。	焼[や]きます				
\\	ている ところです。 
\\	(今) 
\\	い、な・
\\	の
\\	ようです。
\\	どうも
\\	声・音・におい・味が します。
\\	肉が焼けます	新年会は 始まりましたか。 。。。いいえ、ちょうど 始まる所です、早く来てください。 肉が焼けましたか。 。。。いいえ、私たちは 焼いている ところです	焼[や]けます				
\\	ている ところです。 
\\	(今)	
\\	留守	留守のとき、犬に留守番させて、安心です 返事が ありませんね。 。。。留守のようですよ	る 図[ず]				
\\	は人を
\\	は人に
\\	を
\\	い、な・
\\	の
\\	ようです。
\\	どうも
\\	い→かせます;
\\	させます;
\\	されます-こさせます)
\\	留守番	留守のとき、犬に留守番させて、安心です	留守番[るすばん]				
\\	は人を
\\	は人に
\\	を
\\	い→かせます;
\\	させます;
\\	されます-こさせます)
\\	宅配便	荷物は 明日 着きますか。 。。。今朝 宅配便で 送りましたら、着くはずです	宅配[たくはい] 便[びん]				
\\	る、ない・
\\	い、な・
\\	の
\\	はずです、~。
\\	原因	ヒエップさん、昨日の事故の原因は もうわかりましたか。 。。。いいえ、今調べて いるところなんです。原因がわかったら、すぐ ニュースで 知らせます。 。。。はい、わかりました。よろしく お願いします	原因[げんいん]				
\\	ている ところです。 
\\	(今)	
\\	こちら	もしもし、
\\	です。今いいですか。(ちょっと ようろしいでしょうか) 。。。あ、
\\	さん、今 電車に 乗る ところなんです。 東京に着いたら、こちらから かけます。 わかりました、お願いします	こちら			
\\	る ところです。
\\	(これから、 ちょうど、いまから)		
\\	半年	
\\	さん 日本に 長く住んで いるんですか。 。。。いいえ、半年前に、働きに 来たばかりです	半年[はんとし]				
\\	た ばかりです。
\\	ちょうど	試合は まだ やって いますか。 。。。ちょうど 終わったところです	ちょうど				
\\	た ところです。 
\\	(たったいま)		
\\	たった今	渡辺さんは いますか。 。。。あ、たった今 帰ったところです。まだ エレベーターのところに 要るかもしれません	たった 今[いま]			
\\	た ところです。 
\\	(たったいま)		
\\	ちょっとようろしいでしょうか	しもし、
\\	です。ちょっと ようろしいでしょうか 。。。あ、
\\	さん、今 会議が始まるところなんです。 終わったら、こちらから かけます。 わかりました、お願いします	ちょっとようろしいでしょうか				
\\	る ところです。
\\	(これから、 ちょうど、いまから)		
\\	ガスレンジ	もしもし、大阪ガスサービスセンターですか。 。。。はい、大阪ガスサービスセンターで ございます あのう、ガスレンジの 調子が おかしいんですが、、 。。。どんな 具合ですか 先週 直して もらったばかりなのに、また 火が きえてしまうんです。危ないので、 すぐ 来てくれませんか。	ガスレンジ				
\\	た ばかりです。
\\	具合	もしもし、大阪ガスサービスセンターですか。 。。。はい、大阪ガスサービスセンターで ございます あのう、ガスレンジの 調子が おかしいんですが、、 。。。どんな 具合ですか 先週 直して もらったばかりなのに、また 火が きえてしまうんです。危ないので、 すぐ 来てくれませんか。	具合[ぐあい]					
\\	申し訳ありません	もしもし、五時まで エアコンを 直しに 来ると いったのに、いま 8時ごろ まだ きません。 。。。申し訳 ありません。どちら様でしょうか。 
\\	です。 。。。少々 お待ち ください、係員に 連絡しますから。 。。。。。。。。 。。。お待たせしました、たった今 そちらに 向かって いた ところなんですよ。あと 10分ほど お待ちください。ほんとうに 遅れて すみません。	申し訳[もうしわけ]ありません					
\\	どちらさまでしょうか	もしもし、五時まで エアコンを 直しに 来ると いったのに、いま 8時ごろ まだ きません。 。。。申し訳 ありません。どちら様でしょうか。 
\\	です。 。。。少々 お待ち ください、係員に 連絡しますから。 。。。。。。。。 。。。お待たせしました、たった今 そちらに 向かって いた ところなんですよ。あと 10分ほど お待ちください。ほんとうに 遅れて すみません。	どちらさまでしょうか						
\\	お待たせしました	もしもし、五時まで エアコンを 直しに 来ると いったのに、いま 8時ごろ まだ きません。 。。。申し訳 ありません。どちら様でしょうか。 
\\	です。 。。。少々 お待ち ください、係員に 連絡しますから。 。。。。。。。。 。。。お待たせしました、たった今 そちらに 向かって いた ところなんですよ。あと 10分ほど お待ちください。ほんとうに 遅れて すみません。	お 待[ま]たせしました					
\\	ついています	今日は ついていない日だった。 彼女に もらったばかりの手袋を 駅のトイレに 忘れてしまった。気がついて、取りに行った。 ちょうど 掃除しているところだった。床がぬれていたので、滑って、転んでしまって。足も痛いし、服も汚れるし、大変だった。	ついています						
\\	出来事	妻、今日は ほんとうに ついていない日だった。 。。。ついて いないかった日の 出来事は 何ですか? 今朝 会社へ 来たところでしたが。部長に しかられました。またたくさん レポトをさせました。12時まで 昼ごはんを 食べに 行けられました、でも 食堂へ来たが、 何も なってしまいました。 。。。大変だったよ。 どこか 食べに 行きませんか。	出来事[できごと]			
\\	床	今日は ついていない日だった。 彼女に もらったばかりの手袋を 駅のトイレに 忘れてしまった。気がついて、取りに行った。 ちょうど 掃除しているところだった。床がぬれていたので、滑って、転んでしまって。足も痛いし、服も汚れるし、大変だった。	床[ゆか]					
\\	転びます	今日は ついていない日だった。 彼女に もらったばかりの手袋を 駅のトイレに 忘れてしまった。気がついて、取りに行った。 ちょうど 掃除しているところだった。床がぬれていたので、滑って、転んでしまって。足も痛いし、服も汚れるし、大変だった。	転[ころ]びます					
\\	ベル	母から 宅配便が 届くはずだったので、いつもより 早く家へ帰った。玄関のベルが 鳴ったとき、お風呂に入るところだった。慌てて服を着て、ドアを 開けたが、だれもいなかった。	ベル						
\\	慌てて	母から 宅配便が 届くはずだったので、いつもより 早く家へ帰った。玄関のベルが 鳴ったとき、お風呂に入るところだった。慌てて服を着て、ドアを 開けたが、だれもいなかった。	慌[あわ]てて			
\\	着きます	
\\	部長のレポトは もう 届きましたか。 。。。たった今 着きました。 はい、読んで しまいまいたら、内容をまとめさせてください。それから 私に 伝えてください。	着[つ]きます				
\\	た ところです。 
\\	(たったいま)	
\\	込みます	あの レストランは おいしくないのに、いつも 込んで います。なぜですか。 。。。大阪の中で一番 有名レストランですから。 ああ、わかりました。	込[こ]みます			
\\	い、な・
\\	な
\\	のに、~。
\\	大丈夫	ミラーさん、今日 来るでしょうか。 。。。来るはすですよ。きのう 電話が ありましたから じゃ、大丈夫ですね	大丈夫[だいじょうぶ]			
\\	見ます	毎晩 アパートへ帰ったら、宿題をしたり、本を読んだり、テレビを見たりしているだ	見[み]ます					
\\	診ます	私は 2週間ぐらい 頭が痛いのが続きますが。薬を飲んでも、治さないですよ。いい医者を紹介していたかけませんか。 。。。
\\	先生に診てみてください。 はい、わかった。どうも	診[み]ます					
\\	探します	急がしそうだよ。何を探しているんだ?手伝いましょうか。 。。。ああ、たった今 ここに会議資料を置いたところだよ。だれかに間違えされたな。 その隅にあったよ。 。。。よかった!どうもありがと	探[さが]します					
\\	捜します	卒業したら、何をするつもりですか。 。。。今両親と住みたいんだが、大きな部屋を捜そうと思っている。それから、いい仕事を捜すつもりだ	捜[さが]します					
\\	時間に 遅れます	去年の4月から12月まで 
\\	さんが よくクーラスに遅れていたことは よくないです 彼女は「論文を書くので、時間におくれていたんだった」と言っていました	時間[じかん]に  遅[おく]れます					
\\	時間に 間に合います	
\\	さん この間 クーラスの時間に間に合いますね。暇そですね。 はい、昨日から 会社をやめましたので、日本語だけを勉強しています 。。。	時間[じかん]に  間に合[まにあ]います					
\\	やります	今
\\	のケーキをやりたいだが。どうやって するか、教えていただけませんか	やります						
\\	拾います	今朝 公園で ジョギングしているとき、可愛い猫を拾いました 私は可愛い猫を飼っています。また、名前を考えています	拾[ひろ]います					
\\	連絡します	ボランティアをしたいんですが。どうしたらいですか 市役所に連絡したほうが いいですよ。 そうですか。どうも	連絡[れんらく]します					
\\	気分がいい		気分[きぶん]がいい					
\\	気分が悪い	どうしたんですか。 今日は気分が悪いんです	気分[きぶん]が 悪[わる]い					
\\	運動会.	すみません。東京の運動会はいつ 行いますか 私はチケットを買いたいんです。 
\\	来月のようです。インタネットでしらべましょうか。 はい、わかりました	運動会[うんどうかい]					
\\	盆踊り	盆踊りはどうでしたか。 とても 楽しかったです。どうして 来なかったんですか。 ちょっと用事があったんです	盆踊[ぼんおど]り					
\\	フリーマーケット	いいそろばんなんです。どこで 買ったんですか フリーマーケットで買ったんです	フリーマーケット						
\\	場所	日にち 12月三日(土)午後4時~8時 場所  桜大学体育館 *パーティーは無料です	場所[ばしょ]					
\\	ボランティア	ボランティアをしたいんですが。どうしたらいですか 直接市役所に連絡したほうが いいですよ。 そうですか。どうも	ボランティア						
\\	財布	その財布は いいデザインなんですね。どこか買ったんですか。 
\\	自分で作ったんです。 えっ、ほんとうですか。すごいですね	財布[さいふ]					
\\	ごみ	今日のごみを捨てましょうか。 すみません。お願いします ああ、ごみの置き場はどこですか。 階段の横です	ごみ						
\\	国会議事堂	国会議事堂の日本語クーラスに参加したいんですが。どうしたらいですか。 直接 この電話番ごに 連絡してください はい、ありがとうございました	国会[こっかい] 議事堂[ぎじどう]					
\\	平日	大阪城を見学したいんですが。どうしたらいいですか 直接 係員に連絡したらいいです。平日 いつでも 見学することができます	平日[へいじつ]					
\\	大阪弁	
\\	さんは 時々 
\\	弁を使いますね 
\\	市に 住んでいたんですか。 いいえ、父は 
\\	人です。それで 
\\	弁を話せます	大阪[おおさか] 弁[べん]					
\\	ずいぶん	ずいぶん 暇ですね。何をしているんですか。 今日本語を勉強しています。 はい、なぜ?	ずいぶん			
\\	直接	大阪城を見学したいんですが。どうしたらいいですか 直接 係員に連絡したらいいです。平日 いつでも 見学することができます	直接[ちょくせつ]					
\\	いつでも	大阪城を見学したいんですが。どうしたらいいですか 直接 係員に連絡したらいいです。平日 いつでも 見学することができます	いつでも			
\\	どこでも	ベトナムには どこでも きれいな所です。 ほんとうですか。 はい、来月ベトナム中を一緒に旅行しましょう	どこでも			
\\	だれでも	私の同僚は だれでも いいひとですね	だれでも			
\\	何でも	母の犬は 何でも食べられます	何[なに]でも			
\\	荷物が 片付きます	昨日は 引越したのが大変ですか。 はい、とても大変だね。荷物を片付けたのが一番大変だ。	荷物[にもつ]が  片付[かたづ]きます				
\\	燃えるごみ	はい、燃えるごみは 火曜日と木曜日に出してください。 瓶や缶はいつですか 土曜日です	燃[も]えるごみ					
\\	置き場	
\\	さん、ごみの置き場はどこですか。 駐車場の横にあります	置き場[おきば]					
\\	横	
\\	さん、ごみの置き場はどこですか。 駐車場の横にあります	横[よこ]					
\\	宇宙	先月子どもは 宇宙飛行士が宇宙船の外で飛ぶことを見た。 ですから宇宙学に興味があるよ。	宇宙[うちゅう]					
\\	宇宙船	先月子どもは 宇宙船の外で飛ぶことを見た。 ですから宇宙学に興味があるよ。	宇宙船[うちゅうせん]					
\\	宇宙ステーション	毎晩 インタネットで 宇宙ステーションや宇宙船を研究しています。	宇宙[うちゅう]ステーション					
\\	宇宙飛行士	先月子どもは 宇宙船の外で飛ぶことを見た。 ですから宇宙学に興味があるよ。 毎晩 インタネットで 宇宙ステーションや宇宙船を研究しています。 子どもは 宇宙飛行士になりたいと言いました。 しかし、実が違って、子どもが暗いが怖くて、背が低くて、心配していますが。 どうしたらいいですか	宇宙[うちゅう] 飛行[ひこう] 士[し]					
\\	はっきり	言いたい事は はっきり 言ったほうがいい。 私の部屋から 
\\	ビルが はっきり 見えます	はっきり			
\\	マンション	この マンションでペットが飼えますか 。。。小さい鳥や 魚は 飼えますが、 犬や猫は飼えません	マンション						
\\	飼います	この マンションでペットが飼えますか 。。。小さい鳥や 金魚は 飼えますが、 犬や猫は飼えません	飼[か]います					
\\	ペット	この マンションでペットが飼えますか 。。。小さい鳥や 魚は 飼えますが、 犬や猫は飼えません	ペット						
\\	花火	家から 
\\	の花火が見えます。とてもきれいですよ	花火[はなび]					
\\	見えます	家から 
\\	の花火が見えます。とてもきれいですよ	見[み]えます			
\\	修理できます	子どもから 初めて自転車を研究したので、自分で修理できます	修理[しゅうり]できます					
\\	走ります	今走れません、足にけがをしたので、来週の運動会に参加できません	走[はし]ります					
\\	時刻表	どこで 新幹線の時刻表をもらえますか。 きっぷ売り場にもらえますよ	時刻[じこく] 表[ひょう]					
\\	美術館	商店街に豆腐屋さんの横に美術館ができます	美術館[びじゅつかん]					
\\	クリーニング	クリーニングは いつ できますか。 水曜日できます	クリーニング			
\\	簡単	簡単な料理しか 作れませんので、料理を習いに行きたいです	簡単[かんたん]			
\\	(しか) 
\\	~しか~ません。
\\	他の	外国語が話せますか。 英語は話せますが、他の言葉は話せません	他[た]の			
\\	キッチン	あのう、 こちらで料理教室が開けますか。 。。。ええ、3階に キッチンがあります 道具も借りられますか 。。。ええ、借りられます	キッチン						
\\	パーティルーム	あのう、 こちらでパーティーができますか。 。。。ええ、3階に パーティルームがあります カラオケも使えますか 。。。ええ、使えます	パーティルーム						
\\	1週間後	靴の修理、お願いします 。。。はい、 どのくらい かかりますか。 。。。1週間後にできます じゃ、お願いします。	
\\	週間[しゅうかん] 後[ご]			
\\	富士山が見えます	明るくて、いい部屋ですね。 。。。ええ、天気がいい日はこれから 富士山が見えます	富士山[ふじさん]が 見[み]えます					
\\	本棚	この本棚はおもしろいデザインですね。どこで買ったんですか 。。。道具は何でも自分で作ったですよ えっ、ほんとうですか 。。。ええ、趣味は自分で家具をつくることなんです すばらしい趣味ですね	本棚[ほんだな]					
\\	家具	この本棚はおもしろいデザインですね。どこで買ったんですか 。。。家具は何でも自分で作ったですよ えっ、ほんとうですか 。。。ええ、趣味は自分で家具をつくることなんです すばらしい趣味ですね	家具[かぐ]			
\\	すばらしい	この本棚はおもしろいデザインですね。どこで買ったんですか 。。。家具は何でも自分で作ったんですよ えっ、ほんとうですか 。。。ええ、趣味は自分で家具をつくることなんです すばらしい趣味ですね	すばらしい			
\\	主人公	これは「ドラえもん」です。日本の子どもたちはドラえもんが大好きです。マンガの主人公で、猫の形のロボットです	主人公[しゅじんこう]			
\\	不思議な	ドラえもんは不思議なポケットを持っていて、いろいろなもの出せます。 例えば、「タケコプタイ」や「タイムテレビ」 「タケコプタイ」を頭に付けると、自由に空を飛べます	不思議[ふしぎ]な			
\\	ポケット	ドラえもんは不思議なポケットを持っていて、いろいろなもの出せます。 例えば、「タケコプタイ」や「タイムテレビ」 「タケコプタイ」を頭に付けると、自由に空を飛べます	ポケット			
\\	付ける	ドラえもんは不思議なポケットが 付いていますね、いろいろなもの出せます。 例えば、「タケコプタイ」や「タイムテレビ」 「タケコプタイ」を頭に付けると、自由に空を飛べます	付[つ]ける			
\\	自由に	ドラえもんは不思議なポケットを持っていて、いろいろなものが出せます。 例えば、「タケコプタイ」や「タイムテレビ」。 「タケコプタイ」を頭に付けると、自由に空を飛べます	自由[じゆう]に					
\\	ガムをかみます	眠いとき、ガムをかみながら 運転します 。。。そうですか、私は車を止めて、しばらく寝ます	かみます			
\\	しばらく寝ます	眠いとき、ガムをかみながら 運転します 。。。そうですか、私は車を止めて、しばらく寝ます	しばらく 寝[ね]ます			
\\	偉い	彼は 働きながら 勉強しています 。。。そうですか。偉いですね	偉[えら]い			
\\	たいてい	休みの日は 何をしていますか。 。。。そうですね、たいてい 絵を描いています	たいてい			
\\	熱心	
\\	先生は 熱心だし、おもしろいし、それに経験もあります。 いい先生ですね	熱心[ねっしん]			
\\	経験	
\\	先生は 熱心だし、おもしろいし、それに経験もあります。 いい先生ですね	経験[けいけん]					
\\	値段も安いし	よく この鮨屋へ来るんですか。 。。。ええ、ここは 値段も安いし、魚も新しいし、味もいいし、よく 食べに来ます	値段[ねだん]も 安[やす]いし			
\\	選んだんですか。	どうして 富士大学は 選んだんですか。 。。。富士大学は 有名だし、いい先生も多いし、寮もありますから	選[えら]んだんですか。					
\\	息子	週末は 何をしていますか。 。。。たいてい息子と釣りに行っています。時々 娘と買い物に行ったり、映画を見たり するあります。	息子[むすこ]			
\\	娘	週末は 何をしていますか。 。。。たいてい息子と釣りに行っています。時々 娘と買い物に行ったり、映画を見たり するあります。	娘[むすめ]			
\\	力	弟は 若いし、体も大きいし、それに力も強いです	力[ちから]			
\\	まじめな	ニャムさんは まじめな生徒だし、日本語も上手だし、それに 経験もある	まじめな			
\\	メモします	約束の時間を 忘れないように、メモしてください。 。。。はい、わかりました	メモします				
\\	1る・
\\	1ない
\\	ように、
\\	ピアノを弾きます	ミンさんはよく ピアノを弾きながら 歌を歌っています	ピアノを 弾[はじ]きます			
\\	ゲーム	暇なとき、いつも 何をしていますか。 。。。パソコンでゲームをしています	ゲーム					
\\	味	よく この鮨屋へ来るんですか。 。。。ええ、ここは 値段も安いし、魚も新しい、味もいいし、よく 食べに来ます 変な味が しますね。 。。。しょうゆとソースを 間違えたようですよ	味[あじ]				
\\	い、な・
\\	の
\\	ようです。
\\	どうも
\\	声・音・におい・味が します。
\\	番組	毎晩 どんな番組を 見ていますか。 。。。ニュースやドラマが好きなんです。それで 毎晩 見ています	番組[ばんぐみ]			
\\	ドラマ	毎晩 どんな番組を 見ていますか。 。。。ニュースやドラマが好きなんです。それで 毎晩 見ています	ドラマ			
\\	~機	この掃除機は小さいし、軽いし、それに音も静かです	
\\	機[き]					
\\	景色の冬がきれいだし	なぜ 冬には 北海道へ旅行に行きますか。 北海道は涼しいし、景色の冬がきれいだし、それに食べ物がおいしいです	景色[けしき]の 冬[ふゆ]がきれいだし					
\\	美容院	あの美容院は 医者が上手だし、速いし、それに安いです。	美容[びよう] 院[いん]					
\\	形がいい	この車は形もいいし、色もきれだし、それに 値段も そんなに高くないです。	形[かたち]がいい					
\\	品物	この店は安いし、品物も多いし、いつもここで買い物をしています	品物[しなもの]			
\\	~し、~し、
\\	ちょうどいい	なぜ この靴をかったんですか。 。。。デザインもすてきだし、サイズもちょうどいいし。	ちょうどいい						
\\	歌手	どうして あの歌手は 若い人に人気があるんですか。 。。。声もいいし、ダンスが上手だし。	歌手[かしゅ]					
\\	人気があるんです	どうして あの歌手は 若い人に人気があるんですか。 。。。声もいいし、ダンスが上手だし。	人気[にんき]があるんです					
\\	緑	どうして 田舎に住みたいんですか。 。。。緑も多いし、食べ物もおいしいし。	緑[みどり]					
\\	給料	どうして 石油の会社で働きたいんですか。 。。。ボーナスがよく出るし、給料が 高いし。 先週給料を もらったばかりなのに、もう 使ってしまいました。	給料[きゅうりょう]				
\\	た ばかりです。
\\	る、ない・
\\	い、な・
\\	の
\\	はずです、~。
\\	売れます	どうして パワー電気の製品はよく売れるんです 。。。値段も安いし、デザインもいいし、それに故障も少ないですから、	売[う]れます					
\\	小説家	将来の夢は何ですか。 。。。そうですね、いつか小説家になりたいんです。 素晴らしい夢ですね。 。。。それで 今はアルバイトをしながら小説を書いています そうですか。頑張ってください	小説[しょうせつ] 家[か]					
\\	ちょっと おねがいがあるんですが。	ニャムさん、ちょっと おねがいがあるんですが。 。。。何ですか 実は8月 オーストラリア へ ホームステイに行くんです 。。。ホームステイですか。いいですね ええ、それで 今は友達と 英語を勉強しているんですが。 。。。ええ なかなか 上手にならないんです。 先生もいないし、英語で話すチャンスもないし。ニャムさん、会話の先生になっていただけませんか。	ちょっと おねがいがあるんですが。						
\\	実は	ニャムさん、ちょっと おねがいがあるんですが。 。。。何ですか 実は8月 オーストラリア へ ホームステイに行くんです 。。。ホームステイですか。いいですね ええ、それで 今は友達と 英語を勉強しているんですが。 。。。ええ なかなか 上手にならないんです。 先生もいないし、英語で話すチャンスもないし。ニャムさん、会話の先生になっていただけませんか。	実[じつ]は					
\\	ホームステイに行くんです	ニャムさん、ちょっと おねがいがあるんですが。 。。。何ですか 実は8月 オーストラリア へ ホームステイに行くんです 。。。ホームステイですか。いいですね ええ、それで 今は友達と 英語を勉強しているんですが。 。。。ええ なかなか 上手にならないんです。 先生もいないし、英語で話すチャンスもないし。ニャムさん、会話の先生になっていただけませんか。	ホームステイに 行[い]くんです					
\\	踊ります	いろいろな国の料理を食べながら、日本人と友達になりましょう。 踊るし、すてきなプレゼントもあるし。皆さん、ぜひ 参加してください ビデオを見た とおりに、踊ってください	踊[おど]ります				
\\	た・
\\	の
\\	とおりに、V2
\\	参加します	いろいろな国の料理を食べながら、日本人と友達になりましょう。 踊るし、すてきなプレゼントもあるし。皆さん、ぜひ 参加してください	参加[さんか]します				パーティーに参加します	
\\	体育館	
\\	体育館[たいいくかん]					
\\	無料	日にち 12月三日(土)午後4時~8時 場所  桜大学体育館 *パーティーは無料です	無料[むりょう]					
\\	故障	このパソコンは 使っても いいですか。 。。。それは 故障して いますから、あちらのを 使って ください。 変な音がしますね。 。。。エンジンが故障のようですよ。 ちょっと 見ましょう	故障[こしょう]				
\\	い、な・
\\	の
\\	ようです。
\\	どうも
\\	声・音・におい・味が します。
\\	先に	一緒に帰りましょう。 。。。すみません。このメールを書いてしまいますから。お先にどうぞ。 
\\	そろそろ 帰りませんか。 。。。これを片付けてしまいますから、お先にどうぞ。	先[さき]に			
\\	~てしまう。 。。。
\\	そろそろ~ませんか。:
\\	間違えます	約束の時間に間に合いますか。 。。。いいえ、遅れてしまいました。道を間違えたんです。 時間を 間違えないように してください	間違[まちが]えます			
\\	~てしまう ー
\\	る・
\\	ない
\\	ように してください。
\\	拭きます	部屋の床が汚れていますが。拭いていただけませんか	拭[ふ]きます					
\\	取り替えします	時計がとまっていますので、電池を取り替えてください	取り替[とりか]えします			
\\	洗います	洗濯機が壊れていますから、手で洗わなければなりません	洗[あら]います					
\\	ポッケとが付きます 
\\	ポッケとを付けます	ドラえもんは不思議なポッケとが付いていますね。いろいろな物が出せます タケコプタイを頭に付けると、自由に空を飛べます	ポッケとが 付[つ]きます 
\\	ポッケとを 付[つ]けます					
\\	スピーチ	スピーチは もう 覚えてしまいましたか。 。。。まだ 覚えていません できるだけて 明日の朝前に覚えてください 。。。はい、わかった	スピーチ				~てまう:
\\	まだ~ていません:
\\	もう~ましたか:
\\	返事	そろそろ帰りませんか 。。。メールの返事を書いてしまいますから、お先にどうぞ。	返事[へんじ]				~てしまう。
\\	そろそろ~ませんか。:
\\	封筒	すみません、この封筒、使ってもいいですか。 。。。あ、その封筒は破れていますから。こちらのを使ってください。 すみません	封筒[ふうとう]			
\\	交番	どう したんですか。 。。。どこかで財布を落として しまったんです。 それは大変ですね。すぐ交番に 行かないと。 。。。ええ	交番[こうばん]			
\\	書類	どう したんですか。 。。。電車に大切な書類を忘れてしまったんです それは大変ですね。すぐ駅員に 言わないと。 。。。ええ	書類[しょるい]			
\\	駅員	どう したんですか。 。。。電車に大切な書類を忘れてちゃったんだ。 それは大変ですね。すぐ駅員に 言わないと。 。。。ええ	駅員[えきいん]				
\\	ないと。
\\	ないといけません 
\\	なければなりません) 
\\	枝	強い風が吹いているので、木の枝が折れています	枝[えだ]			
\\	お皿	今お皿やお茶碗が割ってあります。コンビニで買いに行っていただけませんか	お 皿[さら]			
\\	お茶碗	今お皿やお茶碗を割ってあります。コンビにに買いに行っていただけませんか	お 茶碗[ちゃわん]					
\\	今の電車	すみません、今の電車に忘れ物をしてしまったんですが。。。 。。。何を忘れたんですか。 青いかばんです。このくらい。。。。外側に大きいポケットが付いています	今[いま]の 電車[でんしゃ]			
\\	このくらい。。。。	すみません、今の電車に忘れ物をしてしまったんですが。。。 。。。何を忘れたんですか。 青いかばんです。このくらい。。。。外側に大きいポケットが付いています	このくらい。。。。			
\\	網棚	どの辺に置きましたか。 。。。よく 覚えていません。でも、網棚の上に置きました 中に何が入っていますか。 。。。えーと、確か 本と傘が入っています じゃ、調べますから、ちょっと 待っていてください。	網棚[あみだな]					
\\	確か	どの辺に置きましたか。 。。。よく 覚えていません。でも、網棚の上に置きました 中に何が入っていますか。 。。。えーと、確か 本と傘が入っています じゃ、調べますから、ちょっと 待っていてください。	確[たし]か			
\\	地震	今朝5時46分に 大きい地震がありますが 地震でビルが 倒れました	地震[じしん]				
\\	で~
\\	壁	今私は駅の前にいます。駅の建物は壊れて、壁の時計は止まっています。時計の針は5時46分を指しています。電車は動いていません	壁[かべ]					
\\	針	今私は駅の前にいます。駅の建物は壊れて、壁の時計は止まっています。時計の針は5時46分を指しています。	針[はり]			
\\	指します	今私は駅の前にいます。駅の建物は壊れて、壁の時計は止まっています。時計の針は5時46分を指しています。電車は動いていません	指[さ]します			
\\	倒れます	古いビルが 駅前の広い道に倒れています。倒れていないビルも窓のガラスが割れていますが	倒[たお]れます					
\\	危ない	ビルの中を見ると、いろいろな物が壊れています。危ないですから、入ることができません	危[あぶ]ない					
\\	燃えます	駅の西の方では今もうちが燃えています 変な匂いがしますね。 。。。ええ、何が 燃えているようです	燃[も]えます			
\\	い、な・
\\	の
\\	ようです。
\\	どうも
\\	声・音・におい・味が します。
\\	地	駅の西の方では今もうちが燃えています	地[ち]					
\\	レポーター	レポーターになって、この絵に ついて話してください	レポーター					
\\	はります	交番に町の地図がはってあります	はります			
\\	1に
\\	2が
\\	てあります。
\\	壁に	駅の新しいトイレ、おもしろいですね。 。。。え?そうですか 壁に花や動物の絵が掛けてあるんです	壁[かべ]に				
\\	1に
\\	2が
\\	てあります。
\\	しまいます	見当たらないわね。 。。。おかしいな、大切なものだから、引き出しにしまって。。。 やっぱり最初ところかしら、あ、奥にあった	しまいます			
\\	1に
\\	2が
\\	てあります。
\\	引き出し	見当たらないわね。 。。。おかしいな、大切なものだから、引き出しにしまって。。。 やっぱり最初ところかしら、あ、奥にあった	引き出[ひきだ]し				
\\	1に
\\	2が
\\	てあります。
\\	出張	来月の出張ですが。ホテルの予約しておきましょうか。 。。。ええ、お願いします	出張[しゅっちょう]			
\\	ておく。
\\	元の所	はさみは 使ったら、元の所に 戻しておいてください。 。。。はい、わかりました	元[もと]の 所[ところ]				
\\	ておく。
\\	戻します	はさみは 使ったら、元の所に 戻しておいてください。 。。。はい、わかりました	戻[もど]します			
\\	ておく。
\\	そのままにします	資料を片付けても いいですか。 。。。いいえ、そのままに して おいてください。まだ 使っています	そのままにします				
\\	ておく。
\\	まだ~
\\	ています。
\\	カレンダー	カレンダーに今月の予定が書いてあります. 。。。はい、わかった	カレンダー				
\\	1に
\\	2が
\\	てあります。
\\	予定	カレンダーに今月の予定が書いてあります. 。。。はい、わかった	予定[よてい]				
\\	1に
\\	2が
\\	てあります。
\\	絵が掛けます	部屋の壁にきれいな絵が掛けてありますね。	掛[か]けます					
\\	並べます	このテーブルにお皿が並べてあります	並[なら]べます			
\\	ごみ箱	ごみ箱は部屋の隅に置いてあります。	ごみ 箱[ばこ]			
\\	隅	ごみ箱は部屋の隅に置いてあります。	隅[すみ]					
\\	予定表	予定表はドアの右にはってあります	予定[よてい] 表[ひょう]			
\\	決めます	お子さんの名前は もう 決めましたか。 。。。いいえ、考えているところです。	決[き]めます					
\\	真ん中	娘の部屋の真ん中に人形が飾ってあります	真ん中[まんなか]					
\\	飾ります	娘 の部屋の真ん中に人形が飾ってあります	飾[かざ]ります			
\\	鏡	ここに鏡が掛けてあるのに便利です	鏡[かがみ]					
\\	花瓶	明日スーパーへ花瓶を買いに行って部屋に飾るつもりです	花瓶[かびん]					
\\	周り	池の周りに木が植えてある。	周[まわ]り					
\\	植えます	池の周りに木が植えてあります	植[う]えます					
\\	冷蔵庫	君、買い物のメモはどこですか。 。。。冷蔵庫にはってあります	冷蔵庫[れいぞうこ]					
\\	まとめます	田舎へ帰る前に 荷物をまとめておいてください 来週の会議の前に、レポートをまとめておいてください。	まとめます			
\\	授業	
\\	先生の授業の前に 予習しておいてください。	授業[じゅぎょう]				
\\	ておく。
\\	予習します	
\\	先生の授業の前に 予習しておいてください。	予習[よしゅう]します			
\\	ておく。
\\	復習します	来週の試験の前に、復習しておいてください	復習[ふくしゅう]します				
\\	ておく。
\\	ガイドプック	日本の旅行の前に ガイドプックを読んでおいてださい	ガイドプック					
\\	講義	来週の講義の前に、この本を全部 読んでおいてください	講義[こうぎ]				
\\	ておく。
\\	準備	7時の前に、食事の準備しておいてください	準備[じゅんび]				
\\	ておく。
\\	考えます	次の会議前に、この問題について考えておいてください。	考[かんが]えます				
\\	ておく。
\\	ポスター	あそこに ポスターがはってありますね。あれは何ですか 。。。スポーツ教室のお知らせです あ、そうですか	ポスター					
\\	お知らせ	あそこに ポスターがはってありますね。あれは何ですか 。。。柔道や 剣道教室のお知らせです あ、そうですか	お 知[し]らせ					
\\	廊下	廊下に本が並べてありますね。どんな本ですか 。。。日本語のほんです。	廊下[ろうか]			
\\	玄関	玄関に箱が置いてありますね。それはなんですか。 。。。古い本や雑誌を入れる箱です	玄関[げんかん]					
\\	ミーティング	来週のミーティングの前に 何をして  おいたら いいですか 。。。そうですね、この資料をコピーして おいて ください はい、わかりました	ミーティング					
\\	リュック	大きいリュックが 置いてありますね。山へ行くんですか 。。。いいえ、非常袋なんですよ。 。。。非常時に使うものを 入れて 置く袋です。 。。。電気やガスが止まっても、3日ぐらい 生活できるものが入れてあるんです	リュック					
\\	非常時	大きいリュックが 置いてありますね。山へ行くんですか 。。。いいえ、非常袋なんですよ。 。。。非常時に使うものを 入れて 置く袋です。 。。。電気やガスが止まっても、3日ぐらい 生活できるものが入れてあるんです	非常時[ひじょうじ]					
\\	非常袋	大きいリュックが 置いてありますね。山へ行くんですか 。。。いいえ、非常袋なんですよ。 。。。非常時に使うものを 入れて 置く袋です。 。。。電気やガスが止まっても、3日ぐらい 生活できるものが入れてあるんです	非常[ひじょう] 袋[ぶくろ]					
\\	生活します	大きいリュックが 置いてありますね。山へ行くんですか 。。。いいえ、非常袋なんですよ。 。。。非常時に使うものを 入れて 置く袋です。 。。。電気やガスが止まっても、3日ぐらい 生活できるものが入れてあるんです	生活[せいかつ]します					
\\	懐中電灯	非常袋には いろいろな物があります。例えば みずや食べ物や 懐中電灯とか、、、、。	懐中[かいちゅう] 電灯[でんとう]			
\\	温泉	お正月は何をしますか。 。。。家族と温泉に行こうと思っているよ。ゆっくり お風呂に入って、のんびりしたいな	温泉[おんせん]			
\\	ようと 思っている。
\\	のんびりします	お正月は何をしますか。 。。。家族と温泉に行こうと思っているよ。ゆっくり お風呂に入って、のんびりしたいな	のんびりします			
\\	続けます	国へ帰っても、日本語の勉強を続けますか 。。。はい、続けるつもりです	続[つづ]けます			
\\	る・ない つ りです。
\\	る
\\	ない)
\\	試験を受けます	夏休みは国へ帰らないんですか 。。。ええ、大学院の試験を 受けますから、今年は帰らないつもりです	試験[しけん]を 受[う]けます				る・ない つ りです。
\\	る
\\	ない)
\\	予定です	明日からニューヨークへ出張します 。。。そうですか。いつ帰りますか。 来週の金曜日に帰る予定です	予定[よてい]です				名詞の・辞書形
\\	予定です
\\	散歩します	公園を散歩しよう。 。。。うん、そう しよう。	散歩[さんぽ]します				
\\	よう。
\\	ましょう	
\\	い→おう;
\\	よう;
\\	しよう-こよう)
\\	見つけます	今仕事を見つけようと思っています。	見[み]つけます			
\\	捜します	
\\	ようと 思っている。
\\	支店	明日 部長は支店へ行く予定です	支店[してん]				名詞・辞書形
\\	予定です
\\	休憩します	疲れましたから、ちょっと休憩しよう 。。。うん、そう しよう。	休憩[きゅうけい]します				
\\	よう。
\\	ましょう	
\\	疲れます	疲れましたから、ちょっと休憩しよう 。。。うん、そう しよう。	疲[つか]れます			
\\	よう。
\\	ましょう	
\\	~の方	皆さん、よく 見えませんから、前の方に座ろうよ 。。。はい、かしこまりました よく 見えるように、 前の方に 座れましょう	~の 方[ほう]				
\\	よう。
\\	ましょう 
\\	1る・
\\	1ない
\\	ように、
\\	動物園	明日は 休みですから、動物園へ行こうよ。 。。。うん、いいよ	動物[どうぶつ] 園[えん]				
\\	よう。
\\	ましょう		
\\	教会	今度の日曜日は 何をしますか。 。。。家族と教会へ行こうと思っています	教会[きょうかい]			
\\	ようと 思っている。
\\	連休	今度の連休は何をしますか。 。。。妻と沖縄へ旅行に行くつもりです おきなわ?そうですか。沖縄の景色は本当にきれいです	連休[れんきゅう]			
\\	る・ない つ りです。
\\	る
\\	ない)
\\	お菓子	ミラーさにもらったお菓子はおいしかったですか。 。。。まだ 食べていませんから、わかりません	お 菓子[かし]			
\\	まだ ~ていません。
\\	申し込みます	北海道の旅行は もう 申し込みましたか。 。。。いいえ、まだ 申し込んでいません。明日 申し込もうと 思っています	申し込[もうしこ]みます			
\\	もう~
\\	ました 
\\	まだ ~ていません。
\\	ようと 思っている。
\\	作文	作文は もう 書きましたか。 。。。まだ 書いていません。今晩 書こうと思います。	作文[さくぶん]				もう~
\\	ました 
\\	まだ ~ていません。
\\	ようと 思っている。
\\	返します	図書館の本は もう 返しましたか。 。。。いいえ、まだ 返していません。明後日 返そうと思っています	返[かえ]します			
\\	もう~
\\	ました 
\\	まだ ~ていません。
\\	ようと 思っている。
\\	展覧会	自動車の展覧会は もう見に行きましたか。 。。。いいえ、まだ見ていません。今度の日曜日 見ようと思っています 大阪で展覧会が 開かれます	展覧[てんらん] 会[かい]				もう~
\\	ました 
\\	まだ ~ていません。
\\	ようと 思っている。
\\	は・が
\\	い→かれます;
\\	られます;
\\	されます-こられます)
\\	お子さん	お子さんの名前は もう 決めましたか 。。。いいえ、まだ 決めていません。顔を見てから、決めようと思っています	お 子[こ]さん				もう~
\\	ました 
\\	まだ ~ていません。
\\	ようと 思っている。
\\	い→おう;
\\	よう;
\\	しよう-こよう)
\\	結婚式	結婚式はどこで しますか 。。。神社に するつもりです	結婚式[けっこんしき]				る・ない つ りです。
\\	る
\\	ない)
\\	取ります	夏休みを 取りますか。 。。。いいえ、取らないつもりです。秋に休みを取ろうと思っています。	取[と]ります			
\\	る・ない つ りです。
\\	る
\\	ない)
\\	駅に着きます	何時に新大阪に着きますか。 。。。9時36分に 着く予定です	着[つ]きます			
\\	名詞の・辞書形
\\	予定です
\\	帰り	帰りの新幹線は どこから 乗りますか。 。。。東京駅で 乗ります	帰[かえ]り			
\\	発表します	発表の準備はできましたか。 。。。えーと、まだ 資料をコピーしていません。これから コピーしようと思っています わかりました	発表[はっぴょう]します				まだ ~ていません。
\\	ようと 思っている。
\\	い→おう;
\\	よう;
\\	しよう-こよう)
\\	本社	出張の準備はできましたか。 。。。えーと、まだ 本社に資料を送っていません。これから 資料を送くろうと思っています わかりました	本社[ほんしゃ]				まだ ~ていません。
\\	ようと 思っている。
\\	転勤します	来週から 独身です。 。。。えっ? 実は大阪の本社に転勤なんです 。。。本社ですか。それはおめでと ございます 高橋さんは アメリカへ転勤するそうですよ。 。。。それで、英語を勉強しているんですね。	転勤[てんきん]			
\\	そうです。
\\	によると ~そうです)	
\\	残ります	妻と子どもは東京に残るんです 。。。えっ、一緒に行かないんですか 息子は 来年 大学の入学試験があるから。東京に残ると言うし。 妻も今の会社をやめたくないと言うんです 。。。別々に住むんですね。	残[のこ]ります			
\\	入学試験	妻と子どもは東京に残るんです 。。。えっ、一緒に行かないんですか 息子は 来年 大学の入学試験があるから。東京に残ると言うし。 妻も今の会社をやめたくないと言うんです 。。。別々に住むんですね。	入学[にゅうがく] 試験[しけん]					
\\	月に	ええ、月に2,3回 週末に 帰るつもりです 。。。大変ですね。	月[つき]に			
\\	る・ない つ りです。
\\	る
\\	ない)
\\	村	私は九州の小さい村で生まれました	村[むら]			
\\	嫌な	田舎にいた時は、映画館もないし、レストランもないし、田舎の生活は嫌だと思った。 でも最近 疲れた時や 寂しい時、よく 田舎の青い空や 緑の山を思い出します	嫌[いや]な			
\\	目を閉じます	目を閉じると、友達と泳いだ川の音が聞こえます	目[め]を 閉[と]じます			
\\	運動します	このごろ エレベタに乗らないで、階段を使っていますよ。 。。。いいうんどうになりますたね	運動[うんどう]します				ない・た ほうがいいです。
\\	生徒	生徒のアリバイとに ついて どう思いますか。 。。。いいと思いますよ。若いときは、いろいろな経験をしたほうがいいですから	生徒[せいと]			
\\	ない。た ほうがいいです。
\\	十分な	1か月ぐらい ヨーロッパへ遊びに行きたいんですが。40万円でたりますか 。。。十分だと思います。でも、現金を持って行かないほうがいいですよ	十分[じゅうぶん]な			
\\	ない。た ほうがいいです。
\\	経済	先生、日本の経済は どう なるでしょうか。 。。。そうですか。まだ しばらく よく ならないでしょう。	経済[けいざい]				
\\	い・
\\	なだ・
\\	だ
\\	でしょう。
\\	きっと
\\	、たぶん
\\	インフルエンザ	先生、わたくしはインフルエンザですか 。。。はい、インフルエンザです。2,3日 高い熱が続くかもしれませんが。心配しなくてもいいですよ	インフルエンザ			
\\	い・
\\	なだ・
\\	だ
\\	かもしれません。
\\	もうしかしたら
\\	熱が続きます	先生、わたくしはインフルエンザですか 。。。はい、インフルエンザです。2,3日 高い熱が続くかもしれませんが。心配しなくてもいいですよ	熱[ねつ]が 続[つづ]きます			
\\	い・
\\	なだ・
\\	だ
\\	かもしれません。
\\	もうしかしたら
\\	おかしい	エンジンの音がおかしいですね 。。。そうですね。もうしかしたら 故障かもしれません。ちょっと 調べましょう	おかしい			
\\	い・
\\	なだ・
\\	だ
\\	かもしれません。
\\	もうしかしたら
\\	ます
\\	ましょう
\\	年を取ります	年を取ったら、車を運転しないほうがいいです 60歳になったら、仕事をやめます もし年を取ったら、働きたいです	年[とし]を 取[と]ります			
\\	たら、~。
\\	ない。た ほうがいいです。
\\	た
\\	ら,~。
\\	もし
\\	~たら、~ 
\\	たら、~。
\\	(~てから、~)(~た・
\\	の
\\	あとで、)		
\\	せきが出ます	昨日から せきが出るんですが 。。。病院へ行ったほうがいいです	せきが 出[で]ます			
\\	ない。た ほうがいいです。
\\	やけど	やけどをしたんです 。。。すぐ 水道の水で 冷やしたほうがいいです	やけど			
\\	ない。た ほうがいいです。
\\	冷やします	やけどをしたんです 。。。すぐ 水道の水で 冷やしたほうがいいです	冷[ひ]やします			
\\	ない。た ほうがいいです。
\\	けがをします	足にけがをしたんです 。。。すぐ練習を休んだほうがいいです	けがをします			
\\	ない。た ほうがいいです。
\\	風が引きます	風が引いているんですが。 。。。今日は出かけないほうがいいです	風[かぜ]が 引[ひ]きます			
\\	ない。た ほうがいいです。
\\	無理をします	ちょっと体の調子が悪いんです。 。。。あまり 無理をしないほうがいいです	無理[むり]をします			
\\	ない。た ほうがいいです。
\\	雨がやみます	たぶん 夕方には雨がやむでしょう	雨[あめ]がやみます			
\\	い・
\\	なだ・
\\	だ
\\	でしょう。
\\	きっと
\\	、たぶん
\\	晴れます	明日は晴れるでしょうか 。。。ええ、西の空が赤いですから。たぶん晴れるでしょう	晴[は]れます			
\\	い・
\\	なだ・
\\	だ
\\	でしょう。
\\	きっと
\\	、たぶん
\\	曇ります	きっと午後は曇るでしょう。	曇[くも]ります			
\\	い・
\\	なだ・
\\	だ
\\	でしょう。
\\	きっと
\\	、たぶん
\\	道がすきます	8時の新幹線に間に合うでしょうか。 。。。ええ、道がすいていますから。たぶん 間に合うでしょう	道[みち]がすきます			
\\	い・
\\	な
\\	な
\\	だ・
\\	だ
\\	でしょう。
\\	きっと
\\	、たぶん
\\	試合に出ます	彼は試合に出るでしょうか。 。。。いいえ、けがをしましたから、試合にでないでしょう。	試合[しあい]に 出[で]ます			
\\	い・
\\	なだ・
\\	だ
\\	でしょう。
\\	きっと
\\	、たぶん
\\	南	南側に高いビルがありますね。 。。。ええ、冬は寒いかもしれません	南[みなみ]				
\\	い・
\\	なだ・
\\	だ
\\	かもしれません。
\\	もうしかしたら
\\	うるさい	隣はカラオケの店がたくさんあります 。。。うるさいかもしれません	うるさい			
\\	い・
\\	なだ・
\\	だ
\\	かもしれません。
\\	もうしかしたら
\\	家賃	家賃がとても安いですね。 。。。何か問題があるかもしれません	家賃[やちん]			
\\	い・
\\	なだ・
\\	だ
\\	かもしれません。
\\	もうしかしたら
\\	道が込みます	今 道が込んでいますね 。。。ええ、もうしかしたら 約束の時間に間に合わないかもしれませんから。電話をかけましょう	道[みち]が 込[こ]みます			
\\	い・
\\	なだ・
\\	だ
\\	かもしれません。
\\	もうしかしたら
\\	ます
\\	ましょう
\\	水道	どうしたんですか。 。。。やけどをしたんですが じゃ、すぐ 水道の水で 冷やしたほうがいいです 。。。ええ、そう します	水道[すいどう]				ない。た ほうがいいです。
\\	試験に合格します	もうすぐ 入学試験ですね 。。。タワポンさんは合格するでしょうか。 よく 勉強していましたから、きっと 合格するでしょう	試験[しけん]に 合格[ごうかく]します			
\\	い・
\\	なだ・
\\	だ
\\	でしょう。
\\	きっと
\\	、たぶん
\\	チーム	もうすぐ サッカーの試合ですね 。。。
\\	のチームは勝つでしょうか。 あんなに練習していましたから、きっと 勝つでしょう.	チーム				
\\	い・
\\	なだ・
\\	だ
\\	でしょう。
\\	きっと
\\	、たぶん
\\	国際	もうすぐ 国際ボランティア会議ですね 。。。会議は成功するでしょうか。 6か月前から準備していましたから、きっと成功するでしょう.	国際[こくさい]				
\\	い・
\\	なだ・
\\	だ
\\	でしょう。
\\	きっと
\\	、たぶん
\\	成功します	もうすぐ 国際ボランティア会議ですね 。。。会議は成功するでしょうか。 6か月前から準備していましたから、きっと成功するでしょう.	成功[せいこ]				
\\	い・
\\	なだ・
\\	だ
\\	でしょう。
\\	きっと
\\	、たぶん
\\	紅葉	京都の紅葉を見たことがありますか 。。。ええ、 来週 行こうと 思っているんですが 。。。きれいですよ。でも 京都はちょっと 寒いかもしれませんから。セーターを持って行った方がいいですよ そうですか、ありがとう ございます	紅葉[こうよう]			
\\	たことがあります。
\\	ようと 思っている。
\\	い・
\\	なだ・
\\	だ
\\	かもしれません。
\\	もうしかしたら
\\	ない。た ほうがいいです。
\\	元気	ニャムさん、元気がありませんね。どうしたんですか 。。。最近 体の調子が悪いんです。時々頭や胃が痛くなるんですよ	元気[げんき]				
\\	くなる。
\\	それはいけません。	それはいけませんね。仕事が忙しいんですか 。。。ええ、残業が多いんです もうしかしたらストレスかもしれませんね。一度病院で診てもらったほうがいいですよ	それはいけません。			
\\	い・
\\	なだ・
\\	だ
\\	かもしれません。
\\	もうしかしたら
\\	ない。た ほうがいいです。
\\	宝くじ	今月はいくらお金を使っても、困らないでしょう、 宝くじを買うと、当たるかもしれません	宝[たから]くじ			
\\	い・
\\	なだ・
\\	だ
\\	でしょう。
\\	きっと
\\	、たぶん
\\	い・
\\	なだ・
\\	だ
\\	かもしれません。
\\	もうしかしたら
\\	宝くじが当たります	今月はたくさんお金を使っても、困らないでしょう、 宝くじを買うと、当たるかもしれません	宝[たから]くじが 当[あ]たります			
\\	い・
\\	なだ・
\\	だ
\\	でしょう。
\\	きっと
\\	、たぶん
\\	い・
\\	なだ・
\\	だ
\\	かもしれません。
\\	もうしかしたら
\\	困ります	今月はたくさんお金を使っても、困らないでしょう、 宝くじを買うと、当たるかもしれません	困[こま]ります			
\\	い・
\\	なだ・
\\	だ
\\	でしょう。
\\	きっと
\\	、たぶん
\\	い・
\\	なだ・
\\	だ
\\	かもしれません。
\\	もうしかしたら
\\	困ります	今月はたくさんお金を使っても、困らないでしょう、 宝くじを買うと、当たるかもしれません	困[こま]ります					
\\	立入禁止	立入禁止は 入るな という意味です	立入禁止[たちいりきんし]			
\\	~という意味です.
\\	(る
\\	な) 
\\	だめだ	だめだ。 もう 走れない。 。。。頑張れ	だめだ			
\\	い→え;
\\	られ;
\\	でき-こられ) 
\\	い→え;
\\	ろ;
\\	しろ-こい)
\\	あきらめます	もう 時間がない。 。。。まだ 1分ある。あきらめるな	あきらめます			
\\	(る
\\	な)
\\	池	この池の周りで 遊んではいけません。あそこに「入るな」と書いてあります	池[いけ]			
\\	ては いけません。
\\	(る
\\	な)
\\	マーク	このマークは どういう 意味ですか 。。。洗濯機で 洗える という 意味です	マーク			
\\	~という 意味です。
\\	どういう 意味ですか) 
\\	い→え;
\\	られ;
\\	でき-こられ) 
\\	(る
\\	な)
\\	禁煙	あの 漢字は 何と 読むんです 。。。「きんえん」です。タバコを吸っては いけない という 意味です	禁煙[きんえん]			
\\	~と読みます。
\\	これは何と読みますか?) ~という 意味です。
\\	どういう 意味ですか)	
\\	戻ります	ミラーさんは どこに いますか。 。。。たった今 出かけたところです。30分ぐらいで 戻ると 言ってしました。 わかった。	戻[もど]ります			
\\	た ところです。 
\\	(たったいま) 
\\	は
\\	と 言っています。
\\	逃げます	
\\	くん、母は入っているよ。すぐ逃げろ。	逃[に]げます			
\\	い→え;
\\	ろ;
\\	しろ-こい)
\\	守ります	
\\	くん、車を運転しているとき。交通規則を守れ。	守[まも]ります			
\\	い→え;
\\	ろ;
\\	しろ-こい)
\\	騒ぎます	日本には電車の中で 子どもたちが 騒ぐな。	騒[さわ]ぎます			
\\	(る
\\	な)
\\	入口	あの 漢字は 「いりぐち」と 読みます。 。。。これは どう いう意味ですか。 このドアで 部屋の中に 入れます	入口[いりぐち]			
\\	~と読みます。
\\	これは何と読みますか?) ~という 意味です。
\\	どういう 意味ですか)	
\\	い→え;
\\	られ;
\\	でき-こられ)
\\	リサイクル	箱の上にあるマークは どういう意味ですか。 。。。これは リサイクルできると いう意味です	リサイクル			
\\	~という 意味です。
\\	どういう 意味ですか)		
\\	い→え;
\\	られ;
\\	でき-こられ)
\\	締め切り	
\\	さんは 締め切りに 間に合わないと いっていました。 高山さんに 締め切りが 明後日だと 伝えて いただけませんか。	締め切[しめき]り			
\\	は
\\	と 言っています。
\\	に
\\	と 伝えて いただけませんか。
\\	~ほど	松浦に 30分ほど 遅れると 伝えて いただけませか。	~ほど			
\\	に
\\	と 伝えて いただけませんか。
\\	投げます	兄、母は「道に ごみを投げるな」と言っていました	投[な]げます			
\\	は
\\	と 言っています。
\\	(る
\\	な)
\\	出口	あの漢字は 何と読むんですか。 。。。「でぐち」と読みます。	出口[いでぐち]			
\\	~と読みます。
\\	これは何と読みますか?)	
\\	非常口	これは 何と 読むんですか。 。。。「ひじょうぐち」と読みます 非常口は どこに あるか、確かめておきてください	非常口[ひじょうぐち]			
\\	~と読みます。
\\	これは何と読みますか?) 
\\	い 
\\	な
\\	だ・
\\	だ 
\\	か、~。
\\	自動販売機	これは 何と 読むんですか。 。。。「じどうはんばいき」と読みます	自動[じどう] 販売[はんばい] 機[き]			
\\	~と読みます。
\\	これは何と読みますか?)	
\\	曲がります	あの信号は どう いう意味ですか。 。。。右へ曲がるな という 意味です。	曲[ま]がります			
\\	~という 意味です。
\\	どういう 意味ですか)	
\\	(る
\\	な)
\\	徐行	あの信号は どう いう意味ですか。 。。。ゆっくり 行け という 意味です。	徐行[じょこう]			
\\	~という 意味です。
\\	どういう 意味ですか)	
\\	い→え;
\\	ろ;
\\	しろ-こい)
\\	使用禁止	これは どう いう意味ですか。 。。。使っては いけない という 意味です。	使用[しよう] 禁止[きんし]			
\\	~という 意味です。
\\	どういう 意味ですか) ~ては いけません。
\\	使用中	これは どう いう意味ですか。 。。。今誰かが 使っている という 意味です。	使用[しよう] 中[ちゅう]			
\\	~という 意味です。
\\	どういう 意味ですか) ~ては いけません。
\\	危ないです	これは どう いう意味ですか。 。。。危険がありますが、使うときは 注意してください。	危[あぶ]ない			
\\	~という 意味です。
\\	どういう 意味ですか)	
\\	出席します	
\\	さんは 何と 言っていましたか。 。。。「ミーティングに 出席できません」と言っていました。 彼女は 来るでしょうか。 。。。出席の返事もらいましたから、来るはずです 部長は 佐藤さんを 会議に 出席させました	出席[しゅっせき]します			
\\	に
\\	と 伝えて いただけませんか。
\\	る、ない・
\\	い、な・
\\	の
\\	はずです、~。
\\	は人を
\\	は人に
\\	を
\\	は人に
\\	を
\\	い→え;
\\	られ;
\\	でき-こられ) 
\\	い→かせます;
\\	させます;
\\	されます-こさせます)
\\	警察	
\\	さんは 何と言っていましたか 。。。警察に 行くと 言っていました。 交通事故が 一番多いのは 何月ですか。 。。。警察の発表によると、12月だそうです	警察[けいさつ]			
\\	は
\\	と 言っています。
\\	そうです。
\\	によると ~そうです)	
\\	割引	すみません。これは 何と 書いてありますか。 。。。「学生割引」です どう いう意味ですか 。。。学生なら 安くなると いう意味です わかりました。どうも	割引[わりびき]			
\\	~と書いてあります。
\\	これは何と書いてありますか?) ~という 意味です。
\\	どういう 意味ですか) 
\\	い
\\	い
\\	く・
\\	な
\\	な
\\	に・
\\	に 
\\	なります。
\\	飲み放題	すみません。これは 何と 書いてありますか。 。。。「飲み放題」です どう いう意味ですか 。。。飲み物を いくら 飲んでも、値段が同じだと いう意味です わかりました。どうも	飲[の]み 放題[ほうだい]			
\\	~と書いてあります。
\\	これは何と書いてありますか?) ~という 意味です。
\\	どういう 意味ですか) 
\\	い
\\	い
\\	く・
\\	な
\\	な
\\	に・
\\	に 
\\	なります。
\\	いくら~も	すみません。これは 何と 書いてありますか。 。。。「飲み放題」です どう いう意味ですか 。。。飲み物を いくら のんでも、値段は同じだと いう意味です わかりました。どうも	いくら~も			
\\	~と書いてあります。
\\	これは何と書いてありますか?) ~という 意味です。
\\	どういう 意味ですか) 
\\	い
\\	い
\\	く・
\\	な
\\	な
\\	に・
\\	に 
\\	なります。
\\	同じ	すみません。これは 何と 書いてありますか。 。。。「飲み放題」です どう いう意味ですか 。。。飲み物を いくら 飲んでも、値段が同じだ という意味です わかりました。どうも	同[おな]じ			
\\	~と書いてあります。
\\	これは何と書いてありますか?) ~という 意味です。
\\	どういう 意味ですか) 
\\	い
\\	い
\\	く・
\\	な
\\	な
\\	に・
\\	に 
\\	なります。
\\	磨きます	寝る前に、何をしていますか。 。。。寝る前に 本を読んでから、歯を磨きます。 はい、いいですね 御飯を 食べた後で、よく 歯を磨きます 朝 靴を磨いたばかりなのに、 もう 汚れてしまいました	磨[みが]きます			
\\	る・
\\	の・
\\	まえに、
\\	た・Nの
\\	あとで、V2
\\	た ばかりです。
\\	る、ない・
\\	い、な・
\\	の
\\	はずです、~。
\\	ロボット	これは 新しいロボットです 。。。どんな ロボットですか。 人がした とおりに、 何でも するんです	ロボット				
\\	た・
\\	の
\\	とおりに、V2
\\	組み立てます	この自転車は 自分で 組み立てるんですか 。。。ええ、 説明書の とおりに 組み立ててください。 はい、わかりました	組み立[くみたて]てます			
\\	た・
\\	の
\\	とおりに、V2
\\	スポーツクラブ	仕事の後で、飲みに行きませんか。 。。。すみません。今日は スポーツクラブへ 行く日なんです。 また 今度お願いします	スポーツクラブ			
\\	た・Nの
\\	あとで、V2
\\	スーツ	友達の結婚式に 何を着て 行ったらいいですか。 。。。そうですね。日本では 男の人 は 黒か 紺のスーツを着て、ネクタイをして 行きます。 。。。はい、わかりました。どうも	スーツ			
\\	て・Vないで 
\\	か
\\	紺	友達の結婚式に 何を着て 行ったらいいですか。 。。。そうですね。日本では 男の人 は 黒か 紺のスーツを着て、ネクタイをして 行きます。 。。。はい、わかりました。どうも	紺[こん]			
\\	て・Vないで 
\\	ソース	これは ソースを つけるんですか。 。。。いいえ、何も つけないで 食べてください そうですか、ありがとう	ソース			
\\	て・Vないで 
\\	しょうゆをつけます	これは ソースを つけるんですか。 。。。いいえ、何も つけないで 食べてください そうですか、ありがとう	ソース			
\\	て・Vないで 
\\	エレベーター	最近 エレベーターに 乗らないで、階段を 使っているんです 。。。いい 運動に なりますね。 はい、健康にいいことに、しています。	エレベーター			
\\	ないで、V2. 
\\	さっき	さっき 書いた とおりに、漢字を 書いてください	さっき			
\\	た・
\\	の
\\	とおりに、V2
\\	歯医者	歯医者に 教えてもらった とおりに、歯を磨いて ください	歯医者[はいしゃ]			
\\	た・
\\	の
\\	とおりに、V2
\\	図	図のとおりに、紙を折ってください 。。。はい、わかりました	図[ず]			
\\	た・
\\	の
\\	とおりに、V2
\\	線	線のとおりに、紙を切ってください 。。。そうですか、どうも	線[せん]			
\\	た・
\\	の
\\	とおりに、V2
\\	矢印	工場の中では 矢印のとおりに、行ってください 。。。はい、わかりました、どうも	矢印[やじるし]			
\\	た・
\\	の
\\	とおりに、V2
\\	お城	お城を見学した後で、 一緒に写真を 撮りましょう。 。。。はい、いいですね。	お 城[しろ]			
\\	た・Nの
\\	あとで、V2
\\	質問します	説明を聞いた後で、質問して ください。	質問[しつもん]します			
\\	た・Nの
\\	あとで、V2
\\	見つかります	新しいのを買った後で、なくした時計が 見つかりました	見[み]つかります			
\\	た・Nの
\\	あとで、V2
\\	忘れ物に気がつきます	いつ 忘れ物に 気がつきましたか。 。。。バスを降りた後で、気がつきした。	忘れ物[わすれもの]に 気[き]がつきます			
\\	た・Nの
\\	あとで、V2
\\	傘を差します	傘を差して 歩きます。 傘を差さないで 歩きます	傘[かさ]を 差[さ]します			
\\	て・Vないで 
\\	お客さん	課長、ちょっと 出張のレポートを 見て いただけませんか。 。。。お客さんが 来ますから。お客さんが 帰った後で、見ます。 お願いします	お 客[きゃく]さん			
\\	た・Nの
\\	あとで、V2
\\	ゆっくり	明日は 休みですね。どこか 行きますか。 。。。いいえ、どこも 出かけないで、うちで ゆっくり休みます。田中さんは? 子どもをプールへ 連れて 行こうと思っています。 。。。そうですか。	ゆっくり			
\\	ないで、V2. 
\\	茶道	一度 茶道を 見たいんですが。。。 。。。じゃ、来週の土曜日 一緒に 行きませんか。	茶道[さどう]					
\\	お茶をたてます	渡辺さん、お茶をたてて ください。 クララさん、お菓子を どうぞ 。。。えっ、先に お菓子を食べるんですか ええ。甘い お菓子を 食べた後で、お茶を飲むと、おいしいんです 。。。そうですか	お 茶[ちゃ]をたてます			
\\	た・Nの
\\	あとで、V2
\\	では~	では お茶を飲みましょう。まず 右手で お茶碗を 取って、左手に 載せます。次に お茶を2回 回して、それから 飲みましょう 。。。はい	では~			
\\	て、
\\	て、~)・(い
\\	、い
\\	・(な
\\	な
\\	載せます	では お茶を飲みましょう。まず 右手で お茶碗を 取って、左手に 載せます。次に お茶を2回 回して、それから 飲みましょう 。。。はい	載[の]せます			
\\	(手に~: 
\\	て、
\\	て、~)・(い
\\	、い
\\	・(な
\\	な
\\	これで いいですか	これで いいですか 。。。はい。 。。。いかがですか 少し 苦いですが、おいしいです	これで いいですか			
\\	いかがですか	これで いいですか 。。。はい。 。。。いかがですか。 少し 苦いですが、おいしいです	いかがですか			
\\	個	材料(一人分) 鳥肉(50グラム)、卵(1個)、たまねぎ(4分の1
\\	個) 調味料: (しょうゆ、砂糖、酒)、御飯	個[こ]			
\\	調味料	材料(一人分) 鳥肉(50グラム)、卵(1個)、たまねぎ(4分の1
\\	個) 調味料: (しょうゆ、砂糖、酒)、御飯	調味[ちょうみ] 料[りょう]			
\\	しょうゆ	材料(一人分) 鳥肉(50グラム)、卵(1個)、たまねぎ(4分の1
\\	個) 調味料: (しょうゆ、砂糖、酒)、御飯	しょうゆ						
\\	適当	鳥肉、たまねぎを 適当な大きさに 切ります	適当[てきとう]			
\\	なべ	なべに 調味料と 水を 入れて、火にかけます	なべ			
\\	火にかけます	なべに 調味料と 水を 入れて、火にかけます	火[ひ]にかけます			
\\	煮ます	調味料が 熱くなったら、材料を入れて、煮ます	煮[に]ます			
\\	煮えます	肉が煮えたら、卵を入れます。卵を入れた後で、火を消します	煮[に]えます			
\\	どんぶりの御飯	1分ぐらい だったら、どんぶりの御飯の上に載せます	どんぶりの 御飯[ごはん]			
\\	咲きます	春に なれば、東京には 桜が 咲きます。 。。。一緒に 花見しよう。 いいよ	咲[さ]きます			
\\	ると、~	
\\	い→え
\\	ば;
\\	れば;
\\	すれば-くれば)― 
\\	い
\\	い 
\\	ければ 
\\	な 
\\	な
\\	なら 
\\	なら
\\	向こう	天気がよければ、向こうに 島が 見えます。 。。。いいですね。何の島ですか 沖縄の島です	向[む]こう			
\\	い→え
\\	ば;
\\	れば;
\\	すれば-くれば)― 
\\	い
\\	い 
\\	ければ 
\\	な 
\\	な
\\	なら 
\\	なら
\\	島	天気がよければ、向こうに 島が 見えます。 。。。いいですね。何の島ですか 沖縄です 橋ができれば、島まで 20分で 行けます。 。。。いいですね。	島[しま]			
\\	ると、~ 
\\	で ~。
\\	い→え
\\	ば;
\\	れば;
\\	すれば-くれば)― 
\\	い
\\	い 
\\	ければ 
\\	な 
\\	な
\\	なら 
\\	なら
\\	港	神戸港に旅行に行きたいんですが。いつ 行ったらいいですか 。。。神戸港旅行なら、いつも 行ったらいいです	港[みなと]			
\\	なら、~。
\\	なら、
\\	い→え
\\	ば;
\\	れば;
\\	すれば-くれば)― 
\\	い
\\	い 
\\	ければ 
\\	な 
\\	な
\\	なら 
\\	なら
\\	開きます	車の窓が 開かないんですが。 。。。ええ、見ましょう。ああ、そのボタンを 押せば、開きますよ。 えぇ、なるほど、わかりました。どうも	開[あ]きます			
\\	ると、~	
\\	い→え
\\	ば;
\\	れば;
\\	すれば-くれば)― 
\\	い
\\	い 
\\	ければ 
\\	な 
\\	な
\\	なら 
\\	なら
\\	なければ	他に 意見がありますか。 。。。いいえ、特に ありません。 なければ、 これで 終わりましょう	なければ			
\\	い→え
\\	ば;
\\	れば;
\\	すれば-くれば)― 
\\	い
\\	い 
\\	ければ 
\\	な 
\\	な
\\	なら 
\\	なら
\\	出さなければ なりませんか	明日までに レポートを 出さなければ なりませんか。 。。。無理なら、金曜日 までに 出してください わかりました。どうも	出[だ]さなければ なりませんか			
\\	い→え
\\	ば;
\\	れば;
\\	すれば-くれば)― 
\\	い
\\	い 
\\	ければ 
\\	な 
\\	な
\\	なら 
\\	なら
\\	借ります	本を借りたいんですが。どうすれば いいですか 。。。受付で カードを 作って もらって ください ありがとう ございます	借[か]ります			
\\	いいですか (どう すれば いいですか)。
\\	い→え
\\	ば;
\\	れば;
\\	すれば-くれば)― 
\\	い
\\	い 
\\	ければ 
\\	な 
\\	な
\\	なら 
\\	なら
\\	花見	花見に 行こうと思っているんですが。どこか いい所は ありませんか。 。。。そうですね。花見なら、吉野山が いいと思います あぁ、吉野山ですか。ありがとうございます。	花見[はなみ]			
\\	~はありませんか。
\\	なら、~。
\\	なら、
\\	と 思います。
\\	でしょう、かもしれません)	
\\	い→え
\\	ば;
\\	れば;
\\	すれば-くれば)― 
\\	い
\\	い 
\\	ければ 
\\	な 
\\	な
\\	なら 
\\	なら
\\	炊飯器	炊飯器を買いたいんですが。近所に いい電気屋は ありませんか。 。。。そうですね。炊飯器なら、
\\	屋が いいと思います あぁ、
\\	屋ですか。ありがとうございます。	炊飯[すいはん] 器[き]			
\\	~はありませんか。
\\	なら、~。
\\	なら、
\\	と 思います。
\\	でしょう、かもしれません)	
\\	い→え
\\	ば;
\\	れば;
\\	すれば-くれば)― 
\\	い
\\	い 
\\	ければ 
\\	な 
\\	な
\\	なら 
\\	なら
\\	近所	炊飯器を買いたいんですが。近所に いい電気屋は ありませんか。 。。。そうですね。炊飯器なら、
\\	屋が いいと思います あぁ、
\\	屋ですか。ありがとうございます。	近所[きんじょ]			
\\	~はありませんか。
\\	なら、~。
\\	なら、
\\	と 思います。
\\	でしょう、かもしれません)	
\\	い→え
\\	ば;
\\	れば;
\\	すれば-くれば)― 
\\	い
\\	い 
\\	ければ 
\\	な 
\\	な
\\	なら 
\\	なら
\\	急ぎます	急げば、9時のバスに 間に合うでしょう。 。。。はい、急ぎましょう。	急[いそ]ぎます			
\\	い→え
\\	ば;
\\	れば;
\\	すれば-くれば)― 
\\	い
\\	い 
\\	ければ 
\\	な 
\\	な
\\	なら 
\\	なら
\\	葉	秋になれば、木の葉の色が変わります。 。。。もちろんよ	葉[は]			
\\	ると、~	
\\	い→え
\\	ば;
\\	れば;
\\	すれば-くれば)― 
\\	い
\\	い 
\\	ければ 
\\	な 
\\	な
\\	なら 
\\	なら
\\	変わります	秋になれば、木の葉の色が変わります。 。。。もちろんよ	変[か]わります			
\\	ると、~	
\\	い→え
\\	ば;
\\	れば;
\\	すれば-くれば)― 
\\	い
\\	い 
\\	ければ 
\\	な 
\\	な
\\	なら 
\\	なら
\\	機会	機会が あれば、アフリカへ 行きたいです 。。。いいですね	機会[きかい]			
\\	~たら、~
\\	い→え
\\	ば;
\\	れば;
\\	すれば-くれば)― 
\\	い
\\	い 
\\	ければ 
\\	な 
\\	な
\\	なら 
\\	なら
\\	都合が悪い	明日 都合が悪ければ、明後日 来てもいいです。 。。。はい、わかりました。 今度の土曜日に みんなで ハイキングに 行くんですが 一緒に 行きませんか。 。。。すみません。土曜日は ちょっと 都合が悪くて、行けないんです	都合[つごう]が 悪[わる]い			
\\	て、
\\	い→え
\\	ば;
\\	れば;
\\	すれば-くれば)― 
\\	い
\\	い 
\\	ければ 
\\	な 
\\	な
\\	なら 
\\	なら
\\	山登り	明日 いい天気なら、一緒に 山登りに 行きませんか。 。。。いいよ。そうしよう	山登[やまのぼ]り			
\\	~たら、~
\\	い→え
\\	ば;
\\	れば;
\\	すれば-くれば)― 
\\	い
\\	い 
\\	ければ 
\\	な 
\\	な
\\	なら 
\\	なら
\\	動きます	ロボットが動かないんですが。どう すれば いいですか。 。。。「動け」と 言えば、動きますよ。 あぁ、ありがとうございます	動[うご]きます			
\\	ると、~	
\\	い→え
\\	ば;
\\	れば;
\\	すれば-くれば)― 
\\	い
\\	い 
\\	ければ 
\\	な 
\\	な
\\	なら 
\\	なら
\\	お金が戻ります	お金が なかなか 戻らないんですが。どうすればいいですか。 。。。白いボタンを押せば、戻りますよ	お 金[かね]が 戻[もど]ります			
\\	いいですか(どう・どうやって すればいいですか 
\\	ると、~	
\\	い→え
\\	ば;
\\	れば;
\\	すれば-くれば)― 
\\	い
\\	い 
\\	ければ 
\\	な 
\\	な
\\	なら 
\\	なら
\\	熱い	お湯が なかなか 熱くならないんですが。どうすればいいですか 。。。レバーを左へ 回せば、熱くなりますよ はい、わかりました。どうも	熱[あつ]い			
\\	いいですか(どう・どうやって すればいいですか 
\\	ると、~	
\\	い→え
\\	ば;
\\	れば;
\\	すれば-くれば)― 
\\	い
\\	い 
\\	ければ 
\\	な 
\\	な
\\	なら 
\\	なら
\\	カーテン	カーテンが なかなか 閉まらないんですが。どうすればいいですか 。。。横の紐を 引けば、閉まりますよ ああ、なるほど ありがとう ございます	カーテン			
\\	いいですか(どう・どうやって すればいいですか 
\\	ると、~		
\\	い→え
\\	ば;
\\	れば;
\\	すれば-くれば)― 
\\	い
\\	い 
\\	ければ 
\\	な 
\\	な
\\	なら 
\\	なら
\\	紐	カーテンが なかなか 閉まらないんですが。どうすればいいですか 。。。横の紐を 引けば、閉まりますよ ああ、なるほど ありがとう ございます	紐[ひも]			
\\	いいですか(どう・どうやって すればいいですか 
\\	ると、~	
\\	い→え
\\	ば;
\\	れば;
\\	すれば-くれば)― 
\\	い
\\	い 
\\	ければ 
\\	な 
\\	な
\\	なら 
\\	なら
\\	眼鏡	眼鏡を かけなければ、辞書の字が読めません	眼鏡[めがね]			
\\	い→え
\\	ば;
\\	れば;
\\	すれば-くれば)― 
\\	い
\\	い 
\\	ければ 
\\	な 
\\	な
\\	なら 
\\	なら
\\	治します	薬を飲まなければ、治しません 。。。はい、わかりました・	治[なお]します			
\\	い→え
\\	ば;
\\	れば;
\\	すれば-くれば)― 
\\	い
\\	い 
\\	ければ 
\\	な 
\\	な
\\	なら 
\\	なら
\\	海外	英語ができなければ、海外旅行のとき 困るかもしれません。 。。。はい、わかった、これから、英語を勉強するつもりです	海外[かいがい]			
\\	い→え
\\	ば;
\\	れば;
\\	すれば-くれば)― 
\\	い
\\	い 
\\	ければ 
\\	な 
\\	な
\\	なら 
\\	なら
\\	正しい	答えが正しければ、丸を付けてください。 。。。はい、わかりました。どうも	正[ただ]しい			
\\	~たら、~
\\	い→え
\\	ば;
\\	れば;
\\	すれば-くれば)― 
\\	い
\\	い 
\\	ければ 
\\	な 
\\	な
\\	なら 
\\	なら
\\	丸を付けます	答えが正しければ、丸を付けてください。 。。。はい、わかりました。どうも カレンダーの丸は どういう意味ですか。 。。。ごみの日です。忘れないように、付けてあるんです はい、わかりました	丸[まる]を 付[つ]けます			
\\	~たら、~
\\	1る・
\\	1ない
\\	ように、
\\	い→え
\\	ば;
\\	れば;
\\	すれば-くれば)― 
\\	い
\\	い 
\\	ければ 
\\	な 
\\	な
\\	なら 
\\	なら
\\	都合がいいです	都合がよければ、ぜひ 来てください。 。。。もちろんよ。	都合[つごう]がいいです			
\\	~たら、~
\\	い→え
\\	ば;
\\	れば;
\\	すれば-くれば)― 
\\	い
\\	い 
\\	ければ 
\\	な 
\\	な
\\	なら 
\\	なら
\\	許可	寮で パーティーができますか。 。。。ええ、許可を もらえば、できます	許可[きょか]			
\\	い→え
\\	ば;
\\	れば;
\\	すれば-くれば)― 
\\	い
\\	い 
\\	ければ 
\\	な 
\\	な
\\	なら 
\\	なら
\\	屋上	このビルの屋上から 港が見えますか。 。。。ええ、天気がよければ、見えるでしょう	屋上[おくじょう]			
\\	い→え
\\	ば;
\\	れば;
\\	すれば-くれば)― 
\\	い
\\	い 
\\	ければ 
\\	な 
\\	な
\\	なら 
\\	なら
\\	お葬式	お葬式に行くとき、何を着て 行けば いいですか。 。。。日本には 男の人は 黒か白のスーツを着て行けばいいです はい、わかりました。どうも	お 葬式[そうしき]			
\\	いいですか(どう・どうやって すればいいですか 
\\	い→え
\\	ば;
\\	れば;
\\	すれば-くれば)― 
\\	い
\\	い 
\\	ければ 
\\	な 
\\	な
\\	なら 
\\	なら
\\	入力します	すみません。日本語を 入力したいんですが。 。。。ここを クリックして、日本語を 選べば、入力できますよ そうですか。どうも	入力[にゅうりょく]します			
\\	い→え
\\	ば;
\\	れば;
\\	すれば-くれば)― 
\\	い
\\	い 
\\	ければ 
\\	な 
\\	な
\\	なら 
\\	なら
\\	クリックします	すみません。日本語を 入力したいんですが。 。。。ここを クリックして、日本語を 選べば、入力できますよ そうですか。どうも	クリックします			
\\	い→え
\\	ば;
\\	れば;
\\	すれば-くれば)― 
\\	い
\\	い 
\\	ければ 
\\	な 
\\	な
\\	なら 
\\	なら
\\	キー	すみません。ひらがなをカタカナに 換えたいんですが。 。。。換えたい事を 選んで、このキーを押せば、換えられますよ そうですか。どうも	キー			
\\	い→え
\\	ば;
\\	れば;
\\	すれば-くれば)― 
\\	い
\\	い 
\\	ければ 
\\	な 
\\	な
\\	なら 
\\	なら 
\\	い→え;
\\	られ;
\\	でき-こられ)
\\	ふりがな	すみません。漢字にふりがなを付けたいんですが。 。。。漢字を選んで、ここを クリックすれば、付けられますよ そうですか。どうも	ふりがな						
\\	い→え
\\	ば;
\\	れば;
\\	すれば-くれば)― 
\\	い
\\	い 
\\	ければ 
\\	な 
\\	な
\\	なら 
\\	なら 
\\	い→え;
\\	られ;
\\	でき-こられ)
\\	マンガ	日本のマンガについて 調べたいんですが、どこへ 行けばいいですか 。。。そうですね。京都のマンガミュージアムへ 行けばいいですよ そうですか。ありがとうございました	マンガ			
\\	いいですか(どう・どうやって すればいいですか 
\\	い→え
\\	ば;
\\	れば;
\\	すれば-くれば)― 
\\	い
\\	い 
\\	ければ 
\\	な 
\\	な
\\	なら 
\\	なら
\\	歴史	この町の歴史について 調べたいんですが、だれに 聞けばいいですか 。。。そうですね。田中さんのおじいさんに聞けば いいですよ そうですか。ありがとうございました	歴史[れきし]				
\\	いいですか(どう・どうやって すればいいですか 
\\	い→え
\\	ば;
\\	れば;
\\	すれば-くれば)― 
\\	い
\\	い 
\\	ければ 
\\	な 
\\	な
\\	なら 
\\	なら
\\	設備	歯科を探しているんですが、どこか いい所が ありませんか。 。。。そうですね。歯科なら、
\\	歯科は どうですか。設備がいいですよ そうですか。場所を 教えていただけませんか 。。。ええ。いいですよ。。。。。。。。	設備[せつび]			
\\	~はありませんか。
\\	なら、~。
\\	なら、
\\	朱	「朱に交われば、赤くなる」です。 これは ことわざです  悪い友達と 仲良くしないほうがいいという意味です	朱[しゅ]			
\\	ことわざ	「朱に交われば、赤くなる」です。 これは ことわざです。 悪い友達と 仲良くしないほうがいいという意味です	ことわざ			
\\	交わります	「朱に交われば、赤くなる」です。 これは ことわざです	交[まじ]わります			
\\	(~に交わります)		
\\	関係	白いワインに 赤いワインを 入れると、ワインは赤くなってしまいます。人間関係も同じです。 いい友達と 仲良くすれば、いい人になりますが。悪い友達と 仲良くすると、悪いことをします。	関係[かんけい]			
\\	仲良くします	白いワインに 赤いワインを 入れると、ワインは赤くなってしまいます。人間関係も同じです。 いい友達と 仲良くすれば、いい人になりますが。悪い友達と 仲良くすると、悪いことをします。	仲良[なかよ]くします			
\\	(~と仲良くします)		
\\	それなら	すずきさん、冬休みに 友達と スキーに 行きますが。どこか いい所が ありませんか 何日ぐらいの予定ですか 。。。3日ぐらいです。 それなら、草津 か 志賀高原が いいと 思いますよ。温泉もあるし。。。。	それなら			
\\	夜行バス	どうやって行くんですか。 。。。JRでも 行けますが、夜行バスなら、朝 着きますから、便利です。 そうですか、どちらが 安いんですか。 。。。さあ、 旅行社へ 行けば、もっと 詳しいことが わかりますよ	夜行[やこう]バス			
\\	旅行社	どうやって行くんですか。 。。。JRでも 行けますが、夜行バスなら、朝 着きますから、便利です。 そうですか、どちらが 安いんですか。 。。。さあ、 旅行社へ 行けば、もっと 詳しいことが わかりますよ	旅行[りょこう] 社[しゃ]			
\\	詳しい	どうやって行くんですか。 。。。JRでも 行けますが、夜行バスなら、朝 着きますから、便利です。 そうですか、どちらが 安いんですか。 。。。さあ、 旅行社へ 行けば、もっと 詳しいことが わかりますよ	詳[くわ]しい			
\\	やっと	やっと 自転車に 乗れるようになりました。	やっと			
\\	るように・
\\	ない
\\	い 
\\	く なります。
\\	(なりました) 
\\	い
\\	い
\\	く・
\\	な
\\	な
\\	に・
\\	に 
\\	なります。
\\	を
\\	い
\\	い
\\	く。
\\	な
\\	な
\\	に。
\\	に)します。
\\	電子辞書	それは 電子辞書ですか。 。。。ええ。知らない言葉が あったら、すぐ 調べられるように、持っているんです そうですか。	電子[でんし] 辞書[じしょ]			
\\	1る・
\\	1ない
\\	ように、
\\	慣れます	もう 和食に慣れましたか。 。。。はい、初めは 食べられませんでした。今は 何でも食べられるようになりました。 。。。そうですか	慣[な]れます			
\\	~(~に慣れます)	
\\	るように・
\\	ない
\\	い 
\\	く なります。
\\	(なりました) 
\\	い
\\	い
\\	く・
\\	な
\\	な
\\	に・
\\	に 
\\	なります。
\\	を
\\	い
\\	い
\\	く。
\\	な
\\	な
\\	に。
\\	に)します。
\\	曲	ショパンの曲が 弾けるように なりましたか。 。。。いいえ、まだ 引けません。早く 弾けるように なりたいです ええ、頑張ってくれ。	曲[きょく]			
\\	るように・
\\	ない
\\	い 
\\	く なります。
\\	(なりました) 
\\	い
\\	い
\\	く・
\\	な
\\	な
\\	に・
\\	に 
\\	なります。
\\	を
\\	い
\\	い
\\	く。
\\	な
\\	な
\\	に。
\\	に)します。
\\	できるだけ	酸っぱい物は 食べられるんですか。 。。。ええ、できるだけ 食べるように しているんです はい、頑張ってくれ。	できるだけ			
\\	る・
\\	ない
\\	ように します
\\	・しています
\\	絶対に	試験は 9時からです。絶対に遅れないように してください。遅れたら、入れませんから。 。。。はい、わかりました。 絶対に パスポートを なくさないように してください	絶対[ぜったい]に			
\\	る・
\\	ない
\\	ように してください。
\\	ラッシュ	ラッシュにあわないように、早く家を出ます 。。。はい、わかりました	ラッシュ			
\\	1る・
\\	1ない
\\	ように、
\\	かなり	番組の日本語が かなり わかるようになりました。 。。。はい、いいですね。どうやって 練習しますか。 毎日 アニメを見ていました	かなり			
\\	るように・
\\	ない
\\	い 
\\	く なります。
\\	(なりました) 
\\	い
\\	い
\\	く・
\\	な
\\	な
\\	に・
\\	に 
\\	なります。
\\	を
\\	い
\\	い
\\	く。
\\	な
\\	な
\\	に。
\\	に)します。
\\	必ず	休むときは、必ず 連絡するように してください。 。。。はい、わかりました	必[かなら]ず			
\\	る・
\\	ない
\\	ように してください。
\\	練習します	試合に 出られるように、毎日 練習しています	練習[れんしゅう]します			
\\	1る・
\\	1ない
\\	ように、
\\	貯金します	年を取ったら、困らないように、今から 貯金して おいてください。 。。。そうですか。わかりました	貯金[ちょきん]します			
\\	1る・
\\	1ない
\\	ように、
\\	ておく。
\\	腐ります	食べ物が腐らないように、冷蔵庫に 入れておいてください。 。。。はい、わかりました。 夏は 食べ物が腐りやすいですから、 冷蔵庫で  管理したほうがいいです。 。。。はい、わかりました。どうも	腐[くさ]ります				
\\	1る・
\\	1ない
\\	ように、
\\	ておく。
\\	ます 
\\	ます
\\	にくい・やすいです。
\\	ほとんど	かたかなが 書けますか。 。。。ほとんど 書けるようになりました。 いいですね	ほとんど			
\\	るように・
\\	ない
\\	い 
\\	く なります。
\\	(なりました) 
\\	い
\\	い
\\	く・
\\	な
\\	な
\\	に・
\\	に 
\\	なります。
\\	を
\\	い
\\	い
\\	く。
\\	な
\\	な
\\	に。
\\	に)します。
\\	このごろ	日本語で 電話が かけられますか。 。。。このごろ やっと かけられるように なりました	このごろ			
\\	るように・
\\	ない
\\	い 
\\	く なります。
\\	(なりました) 
\\	い
\\	い
\\	く・
\\	な
\\	な
\\	に・
\\	に 
\\	なります。
\\	を
\\	い
\\	い
\\	く。
\\	な
\\	な
\\	に。
\\	に)します。
\\	規則	規則を守るようにして ください。 。。。はい、わかりました	規則[きそく]			
\\	る・
\\	ない
\\	ように します
\\	・しています
\\	る・
\\	ない
\\	ように してください。
\\	過ぎます	夜 12時を過ぎたら、勉強しないようにしてください 。。。はい、わかりました	過[す]ぎます			
\\	る・
\\	ない
\\	ように します
\\	・しています
\\	る・
\\	ない
\\	ように してください。
\\	剣道	剣道は上手になりましたね。 。。。いいえ、まだまだです。早く 試合に出られるようになりたいです	剣道[けんどう]					
\\	特別	こんにちは。 。。。お元気ですね。何か 特別なことを して いらっしゃいますか。 毎日 運動するようにしています 。。。どんな 運動ですか ダンスとか、水泳とか。。。最近 500メートル 泳げるように なりました	特別[とくべつ]			
\\	乗り物	今日 15世紀の乗り物について 話します。 。。。えぇ、すごいですね	乗り物[のりもの]			
\\	招待します	私は 木村さんに 結婚式を 招待されました 初めて日本へ 来たとき、たいへんだったでしょう? 。。。ええ、でも ボランティアの方が 親切に してくださいました。 そうですか 。。。日本語や 日本料理を 教えてくださいました それはよかったですね	招待[しょうたい]します			
\\	人1は
\\	2に
\\	人1
\\	人1は 人2(私
\\	に 
\\	て)くれます・くれます・くださいます。
\\	人1
\\	人2. (くれます:
\\	(くれます:
\\	-(くださいます:
\\	い→かれます;
\\	られます;
\\	されます-こられます)
\\	途中	課長、遅れてすみません 。。。ミラーさん、どう したんですか 実は 来る途中で 事故があって、バスが 遅れてしまいました 。。。バスの事故ですか。 いいえ、交差点で トラックと 車が ぶつかって、バスが 動かなかったんです 。。。それは、大変ですね	途中[とちゅう]			
\\	る 途中で ~。
\\	怒ります	きのう私は 家へ 遅く帰って、妻が 怒ります	怒[おこ]ります			
\\	幸せ	毎日 妻と 一緒に 何でも いているので、幸せです	幸[しあわ]せ			
\\	問題が出ます	どんな問題が 試験に 出るか 知っていますか。 。。。いいえ、山田先生は 問題について 考えていますから。 はい、わかりましたた	問題[もんだい]が 出[で]ます			
\\	い 
\\	な
\\	だ・
\\	だ 
\\	か、~。
\\	間違い	レポートが 間違いがある かどうか、一度 見ていただけませんか。 。。。いいですよ、会議が終わったら、見ます はい、お願いします	間違[まちが]い				
\\	い 
\\	な
\\	だ・
\\	だ 
\\	かどうか、~。
\\	お年寄り	来月日本語の 発表会が あります。 。。。どんな 発表を するんですか。 日本のお年寄りは どうして 元気か、調べ、発表します 。。。そうですか、頑張ってください	お 年寄[としよ]り			
\\	い 
\\	な
\\	だ・
\\	だ 
\\	か、~。
\\	爆弾	警官の車を止めました。そして、車に爆弾が 積まれているかもしれないといいました。運転手と 銀行員は急ぎで 降りて、離れたところに 逃げました	爆弾[ばくだん]					
\\	直します	私は 先生に 漢字の間違いを直していただきました。 。。。いいですね。	直[なお]します			
\\	人1(私)は 人2に 
\\	て)もらいます・もらいます・いただきます。
\\	人1
\\	人2. (もらいます:
\\	(もらいます:
\\	-(いただきます:
\\	あ		あ				人1(私)は 人2に 
\\	て)やります・あげます・さしあげます。
\\	人1
\\	人2
\\	(やります:
\\	(あげます:
\\	-(さしあげます:
\\	人1(私)は 人2に 
\\	て)もらいます・もらいます・いただきます。
\\	人1
\\	人2. (もらいます:
\\	(もらいます:
\\	-(いただきます:
\\	人1は 人2(私
\\	に 
\\	て)くれます・くれます・くださいます。
\\	人1
\\	人2. (くれます:
\\	(くれます:
\\	-(くださいます:
\\	て ください 
\\	~てくれ。
\\	~て!)。 
\\	て いださいませんか。
\\	ていただけませんか 
\\	紙飛行機	私は 息子に 紙飛行機を 作って やました	紙[かみ] 飛行機[ひこうき]			
\\	人1(私)は 人2に 
\\	て)やります・あげます・さしあげます。
\\	人1
\\	人2
\\	(やります:
\\	(あげます:
\\	-(さしあげます:
\\	ディズニーランド	連休は 何をしますか。 。。。子どもを  ディズニーランドへ 連れて行ってやります	ディズニーランド				人1(私)は 人2に 
\\	て)やります・あげます・さしあげます。
\\	人1
\\	人2
\\	(やります:
\\	(あげます:
\\	-(さしあげます:
\\	相撲	相撲を 見に行ったことが ありますか。 。。。ええ。この間 部長に 連れて 行って いただきました。とても おもしろかったです	相撲[すもう]				人1(私)は 人2に 
\\	て)もらいます・もらいます・いただきます。
\\	人1
\\	人2. (もらいます:
\\	(もらいます:
\\	-(いただきます:
\\	お年玉	お正月には 
\\	さんが お年玉をくれました。 。。。いいですね	お 年玉[としだま]			
\\	人1は 人2(私
\\	に 
\\	て)くれます・くれます・くださいます。
\\	人1
\\	人2. (くれます:
\\	(くれます:
\\	-(くださいます:
\\	取り替えます	セーターのサイズを まちがえましたので。取り替えしてくださいませんか 。。。いいですよ	取り替[とりか]えます			
\\	て ください 
\\	~てくれ。
\\	~て!)。 
\\	て いださいませんか。
\\	ていただけませんか 
\\	ひとこと	次にこちらの皆さんに ひとことずつ お願いします。 。。。 どうぞお幸せに	ひとこと			
\\	どうぞお幸せに	次にこちらの皆さんに ひとことずつ お願いします。 。。。 どうぞお幸せに	どうぞお 幸[しあわ]せに			
\\	ます 
\\	ます そうです。
\\	今にも もうすぐ これから) 
\\	い 
\\	い・
\\	な 
\\	な 
\\	そうです。
\\	~て来ます。
\\	て くれませんか。
\\	ください、
\\	いただけませんか。		
\\	素直	兄の子どもは 3人います。子どもたちは 素直で、活発です	素直[すなお]			
\\	破れます	ミラーさん 。。。はい、なんですか 袋が破れそうですよ。 。。。あ、ほんとうだ。気がつきませんでした。どうも	破[やぶ]れます			
\\	ます 
\\	ます そうです。
\\	今にも もうすぐ これから)	
\\	結婚を申し込みます	うれしそうですね。 。。。ええ、実は 昨日 結婚を申し込まれたんです そうですか。それは よかったですね	結婚[けっこん]を 申し込[もうしこ]みます			
\\	い 
\\	い・
\\	な 
\\	な 
\\	そうです。
\\	い→かれます;
\\	られます;
\\	されます-こられます)
\\	楽しい	楽しそうですね。 。。。ええ、実は 明日 海外旅行 に行くんです。 そうですか。それは 楽しみですね	楽[たの]しい			
\\	い 
\\	い・
\\	な 
\\	な 
\\	そうです。
\\	ます 
\\	ます・
\\	い
\\	い・
\\	な
\\	な
\\	すぎます。
\\	ます 
\\	ます
\\	にくい・やすいです。
\\	を
\\	い
\\	い
\\	く。
\\	な
\\	な
\\	に。
\\	に)します。
\\	にします。
\\	になる。
\\	無理に	地震で 電車が 止まった場合は、無理に 帰らないで、会社に泊まってください 。。。はい、わかりました。	無理[むり]に			
\\	(無理に~しない)	
\\	る、ない、た・
\\	い、な・
\\	の
\\	場合は、~。
\\	梅	冬なのに、桜が 咲いていますね。 。。。えっ、あれは さくらじゃ ありません。梅です	梅[うめ]			
\\	い、な・
\\	な
\\	のに、~。
\\	そうです。
\\	によると ~そうです) 
\\	い、な・
\\	の
\\	ようです。
\\	どうも
\\	声・音・におい・味が します。
\\	て みます。
\\	~て来ます。
\\	る・
\\	の 
\\	のに~。
\\	使います、いいです、便利です、役に立ちます、時間がかかります。 
\\	る、ない・
\\	い、な・
\\	の
\\	はずです、~。
\\	大会	新聞によると 1月に 日本語のスピーチ大会が あるそうですよ。ミラーさんも 出てみませんか。 。。。そうですね。考えてみます	大会[たいかい]			
\\	そうです。
\\	によると ~そうです) 
\\	て みます。
\\	パトカー	人が 大勢 集まっていますね。 。。。事故のようですね。 パトカーと 救急車が 来ていますよ	パトカー			
\\	い、な・
\\	の
\\	ようです。
\\	どうも
\\	救急車	人が 大勢 集まっていますね。 。。。事故のようですね。 パトカーと 救急車が 来ていますよ	救急[きゅうきゅう] 車[しゃ]			
\\	い、な・
\\	の
\\	ようです。
\\	どうも
\\	台風が来ます	天気予報によると、明日は 台風が 東京の東へ 来るそうです 。。。そうですか。	台風[たいふう]が 来[き]ます			
\\	そうです。
\\	によると ~そうです)	
\\	人口	新聞で 見ましたが、世界の人口は 2100年い 100億人になるそうです。 。。。そうですか。	人口[じんこう]			
\\	そうです。
\\	によると ~そうです)	
\\	科学	アメリカの科学雑誌で 読みましたが、新しい星が 発見されたそうです 。。。そうですか。いいですね	科学[かがく]			
\\	そうです。
\\	によると ~そうです)	
\\	実験	実験は どうでしたか。 。。。昼のニュースによると、失敗したそうです。 そうですか。残念ですよ	実験[じっけん]			
\\	そうです。
\\	によると ~そうです)	
\\	データ	これからも 東京の人口は 増えているんです。 。。。はい、最近のデータによると、増えているんだそうです	データ			
\\	そうです。
\\	によると ~そうです)		
\\	賛成	首相は 大統領の意見に 賛成ですか。 。。。いいえ、今朝のニュースによると、反対だそうです	賛成[さんせい]			
\\	そうです。
\\	によると ~そうです)	
\\	反対	首相は 大統領の意見に 賛成ですか。 。。。いいえ、今朝のニュースによると、反対だそうです	反対[はんたい]			
\\	そうです。
\\	によると ~そうです)	
\\	別れます	あの 二人は 別れたそうですよ。 。。。最近 彼が 元気が ありません。 そうですね	別[わか]れます			
\\	そうです。
\\	によると ~そうです)	
\\	匂い	変な匂いがしますね。 。。。ええ、何が 燃えているようです	匂[にお]い			
\\	い、な・
\\	の
\\	ようです。
\\	どうも
\\	声・音・におい・味が します。
\\	吹きます	木の葉が たくさん 落ちていますね。 。。。強い風が 吹いたようですよ	吹[ふ]きます			
\\	い、な・
\\	の
\\	ようです。
\\	どうも
\\	ワンワン	ワンワンのほえる声がしますね 。。。だれか 来たようですから。ちょっと 見てきますね 犬は留守番をするのにいいですね	ワンワン				
\\	い、な・
\\	の
\\	ようです。
\\	どうも
\\	~て来ます。
\\	る・
\\	の 
\\	のに~。
\\	使います、いいです、便利です、役に立ちます、時間がかかります。	
\\	ニャーニャー	ニャーニャーの鳴る声がしますね 。。。
\\	が帰ったようです。ちょっと 見てきますね ああ、かえりなさい	ワンワン			
\\	い、な・
\\	の
\\	ようです。
\\	どうも
\\	声・音・におい・味が します。
\\	~て来ます。
\\	ゴロゴロ	ゴロゴロの鳴る音がしますね 。。。大きい雷の音ですね。たぶん 大きい雨が降るでしょう そうですね。	ゴロゴロ			
\\	声・音・におい・味が します。
\\	い・
\\	なだ・
\\	だ
\\	でしょう。
\\	きっと
\\	、たぶん
\\	ピュービュー	ゴロゴロの吹く音がしますね 。。。強い風ですね。もうしかしたら 台風が来るかもしれませんね  そうですね。	ピュービュー			
\\	声・音・におい・味が します。
\\	い・
\\	な
\\	なだ・
\\	だ
\\	かもしれません。
\\	もうしかしたら
\\	ザーザー	ザーザーの降る音がしますね 。。。今日は もうしかしたら  ずっと雨が降るかもしれません そうですね。	ザーザー			
\\	声・音・におい・味が します。
\\	い・
\\	な
\\	なだ・
\\	だ
\\	かもしれません。
\\	もうしかしたら
\\	カーカー	外に カーカーの鳴る声がしますね 。。。烏が来たようですね。	ザーザー			
\\	声・音・におい・味が します。
\\	(からす)	
\\	げらげら	部屋の中にげらげらの笑う声がしますね。 。。。きっと 子どもたちが まだ 寝ていません	ザーザー			
\\	声・音・におい・味が します。
\\	まだ ~ていません。
\\	(わらう)	
\\	しくしく	田中さんの部屋から しくしくの泣く声がしますね。 。。。そうですか。恋人と別れるようですね	しくしく			
\\	声・音・におい・味が します。
\\	い、な・
\\	の
\\	ようです。
\\	どうも
\\	(なきます)	
\\	きょろきょろ	彼は きょろきょろ 見ていますね。 。。。たぶん 財布でも 落ちてしまいました	きょろきょろ			
\\	ぱくぱく	彼は その料理を パクパク 食べていますね。 。。。この料理は おいしいはずです	ぱくぱく			
\\	る、ない・
\\	い、な・
\\	の
\\	はずです、~。
\\	ぐうぐう	子どもの部屋から ぐうぐうの音がしますね。 。。。たぶん 寝てしまったはずです	ぐうぐう			
\\	メールが来ます	木村さんから メールが来たんですが、ワットさんと 結婚したそうですよ。 。。。えっ、いつですか。 四月一日だそうです。 。。。じゃ、お祝いを しないと。。。。	メールが 来[き]ます			
\\	そうです。
\\	によると ~そうです) なければなりません・ないといけません
\\	ないと)。
\\	お見舞い	山田さんから メールが来たんですが、奥さんが 入院したそうですよ。 。。。えっ、いつですか。 四月一日だそうです。 。。。じゃ、お見舞いに 行かないと。。。。	お 見舞[みま]い			
\\	そうです。
\\	によると ~そうです) なければなりません・ないといけません
\\	ないと)。
\\	亡くなります	松本さんから メールが来たんですが、お母さんが 亡くなったそうですよ。 。。。えっ、いつですか。 四月一日だそうです。 。。。じゃ、みんなに 連絡しないと。。。。	亡[な]くなります			
\\	そうです。
\\	によると ~そうです) なければなりません・ないといけません
\\	ないと)。
\\	怖い	昼のニュースを見ましたか。オーストラリアで 山火事が あったそうですよ 。。。本当ですか 原因は タバコの火だそうです 。。。怖いですね	怖[こわ]い			
\\	そうです。
\\	によると ~そうです)	
\\	地震が起こります	昼のニュースを見ましたか。インドネシアで 地震が起こったそうですよ。 。。。本当ですか 大きい津波が来たそうです 。。。怖いですね	地震[じしん]が 起[お]こります			
\\	そうです。
\\	によると ~そうです)	
\\	津波が来ます	昼のニュースを見ましたか。インドネシアで 地震が起こったそうですよ。 。。。本当ですか 大きい津波が来たそうです 。。。怖いですね	津波[つなみ]が 来[き]ます			
\\	そうです。
\\	によると ~そうです)	
\\	婚約します	渡辺さん、このごろ 早く 帰りますね。 どうも 恋人が できたようですね 。。。あ、知らないんですか。この間 婚約したそうですよ えっ、だれですか、相手は? 。。。
\\	の鈴木さんですよ	婚約[こんやく]します			
\\	い、な・
\\	の
\\	ようです。
\\	どうも
\\	そうです。
\\	によると ~そうです)	
\\	相手	渡辺さん、このごろ 早く 帰りますね。 どうも 恋人が できたようですね 。。。あ、知らないんですか。この間 婚約したそうですよ えっ、だれですか、相手は? 。。。
\\	の鈴木さんですよ	相手[あいて]			
\\	い、な・
\\	の
\\	ようです。
\\	どうも
\\	そうです。
\\	によると ~そうです)	
\\	どうも	渡辺さん、このごろ 早く 帰りますね。 どうも 恋人が できたようですね 。。。あ、知らないんですか。この間 婚約したそうですよ えっ、だれですか、相手は? 。。。
\\	の鈴木さんですよ	どうも			
\\	い、な・
\\	の
\\	ようです。
\\	どうも
\\	そうです。
\\	によると ~そうです)		
\\	化粧	私の友達は お年寄りの世話をするボランティアを しています。 彼女は 1か月に 
\\	3回、女性の髪を きれいに したり、化粧を したり しに 行っています	化粧[けしょう]			
\\	長生き	日本では 男性より 女性のほうが 元気で、長生きです。理由は いろいろあると 思いますが。  化粧も理由の一つかもしれません	長生[ながい]き			
\\	は人を
\\	は人に
\\	を
\\	は人に
\\	を
\\	私は人に
\\	てもらいます。
\\	人
\\	ていただけませんか。
\\	い→かせます;
\\	させます;
\\	させます-こさせます)
\\	留学します	息子を イギリスへ 留学させます	留学[りゅうがく]します			
\\	は人を
\\	は人に
\\	を
\\	い→わせます;
\\	させます;
\\	させます-こさせます)
\\	そろそろ 失礼します	そろそろ 失礼します 。。。あ、ちょっと 待ってください。息子に 駅まで 送らせますから すみません、けっこうです	そろそろ  失礼[しつれい]します			
\\	は人を
\\	は人に
\\	を
\\	い→かせます;
\\	させます;
\\	させます-こさせます)
\\	サッカー	このサッカー教室は 練習が 厳しいそうです 。。。ええ、毎日 子どもたちを 1キロ 走らせます。 大変ですが、上手に 早くなるはすです	サッカー			
\\	は人を
\\	は人に
\\	を
\\	い→かせます;
\\	させます;
\\	させます-こさせます)
\\	柔道	ハンス君は 学校の勉強の他に、何か 習っていますか。 。。。ええ、柔道を したいと 言ったので、柔道教室に 行かせています いい運動になりますね	柔道[じゅうどう]			
\\	は人を
\\	は人に
\\	を
\\	は人に
\\	を
\\	い→かせます;
\\	させます;
\\	させます-こさせます)
\\	意見	伊藤先生は どんな先生ですか。 。。。いい先生ですよ。生徒に 好きな本 読ませて、自由に 意見を 言わせるんです	意見[いけん]				
\\	は人を
\\	は人に
\\	を
\\	は人に
\\	を
\\	い→かせます;
\\	させます;
\\	させます-こさせます)
\\	車	すみません。 しばらく ここに 車を止めさせて いただけませんか。 。。。いいですよ	車[くるま]				私は人に
\\	てもらいます。
\\	人
\\	ていただけませんか。
\\	い→かせます;
\\	させます;
\\	させます-こさせます)
\\	お忙しいですか	課長、今 お忙しいですか。 。。。いいえ、どうぞ ちょっと お願いがあるんですが。 。。。何ですか あの来週 7日(なのか)から 10日(とうか)ほど 休みを 取らせていただけませんか	お 忙[いそが]しいですか			
\\	私は人に
\\	てもらいます。
\\	人
\\	ていただけませんか。
\\	い→かせます;
\\	させます;
\\	させます-こさせます)
\\	営業	実は アメリカの友達が 結婚するんです。 。。。そうですか。えーと、来月 20日(はつか) 営業会議が ありますが。 それ前に 帰れますね はい 。。。じゃ、かまいませんよ。楽しんで来てください ありがとうございます	お 忙[いそが]しいですか			
\\	い→え;
\\	ろ;
\\	しろ-こい)
\\	かまいません	実は アメリカの友達が 結婚するんです。 。。。そうですか。えーと、来月 20日(はつか) 営業会議が ありますが。 それ前に 帰れますね はい 。。。じゃ、かまいませんよ。楽しんで来てください ありがとうございます	かまいません			
\\	い→え;
\\	ろ;
\\	しろ-こい)
\\	楽しみます	実は アメリカの友達が 結婚するんです。 。。。そうですか。えーと、来月 20日(はつか) 営業会議が ありますが。 それ前に 帰れますね はい 。。。じゃ、かまいませんよ。楽しんで来てください ありがとうございます	楽[たの]しみます			
\\	い→え;
\\	ろ;
\\	しろ-こい)
\\	買い物	私は 息子を 買い物に 行かせます	買い物[かいもの]				
\\	は人を
\\	は人に
\\	を
\\	は人に
\\	を
\\	い→かせます;
\\	させます;
\\	されます-こさせます)
\\	立ちます	わたしは息子を 電車の中で 立たせます	立[た]ちます				
\\	は人を
\\	は人に
\\	を
\\	は人に
\\	を
\\	い→かせます;
\\	させます;
\\	されます-こさせます)
\\	プール	私は 子どもを ブールで 泳がせます	プール			
\\	は人を
\\	は人に
\\	を
\\	は人に
\\	を
\\	い→かせます;
\\	させます;
\\	されます-こさせます)
\\	塾	私は 娘を 塾に通わせます	塾[じゅく]			
\\	は人を
\\	は人に
\\	を
\\	は人に
\\	を
\\	い→かせます;
\\	させます;
\\	されます-こさせます)
\\	届けます		届[とど]けます			
\\	は人を
\\	は人に
\\	を
\\	は人に
\\	を
\\	い→かせます;
\\	させます;
\\	されます-こさせます)
\\	嫌	お子さんが 高校を やめたいと 言ったら、どうしますか。 。。。そうですね。本当に 勉強が嫌だったら、やめさせます そうですか	嫌[いや]			
\\	は人を
\\	は人に
\\	を
\\	は人に
\\	を
\\	い→かせます;
\\	させます;
\\	されます-こさせます)
\\	録音します	先生、ちょっと お願いがあるんですが。 。。。はい、何ですか。 来週の金曜日に 先生の講義を録音させて いただけませんか。 。。。いいですよ	録音[ろくおん]します			
\\	私は人に
\\	てもらいます。
\\	人
\\	ていただけませんか。
\\	い→かせます;
\\	させます;
\\	されます-こさせます)
\\	録画します	先生、ちょっと お願いがあるんですが。 。。。はい、何ですか。 来週の金曜日に 先生の講義を録画させて いただけませんか。 。。。いいですよ	録画[ろくが]します			
\\	私は人に
\\	てもらいます。
\\	人
\\	ていただけませんか。
\\	い→かせます;
\\	させます;
\\	されます-こさせます)
\\	そろばん	そろばんは 普通の生活では もう使われないが、そろばんを 習うと、計算が速くできるようになる。 それで、 日本では 今も 人気がある。	そろばん			
\\	い→われます;
\\	られます;
\\	されます-こられます)
\\	一人で旅行します	どこもに 何をさせますか。。。。子どもがいた時、私は子どもを 一人で旅行させます。いろいろ経験できますから。	一 人[にん]で 旅行[りょこう]します			
\\	は人を
\\	は人に
\\	を
\\	は人に
\\	を
\\	い→かせます;
\\	させます;
\\	させます-こさせます)
\\	自然の中で遊びます	どこもに 何をさせますか。。。。子どもがいた時、私は子どもを 自然の中で遊ばせます。いろいろ経験できだし、自分で 世話できます	自然[しぜん]の 中[なか]で 遊[あそ]びます			
\\	は人を
\\	は人に
\\	を
\\	は人に
\\	を
\\	い→かせます;
\\	させます;
\\	させます-こさせます)
\\	いい本をたくさん読みます	どこもに 何をさせますか。。。。子どもがいた時、私は子どもを いい本をたくさん読ませます。いろいろ経験できだし、本から いいことが勉強できます	いい 本[ほん]をたくさん 読[よ]みます			
\\	は人を
\\	は人に
\\	を
\\	は人に
\\	を
\\	い→かせます;
\\	させます;
\\	させます-こさせます)
\\	ボランティアに参加します	どこもに 何をさせますか。。。。子どもがいた時、私は子どもを ボランティアに参加させます。いろいろ経験できだし、ボランティアから いいことが勉強できます	ボランティアに 参加[さんか]します			
\\	は人を
\\	は人に
\\	を
\\	は人に
\\	を
\\	い→かせます;
\\	させます;
\\	させます-こさせます)
\\	自分がやりたいことをやります	どこもに 何をさせますか。。。。子どもがいた時、私は子どもを 自分がやりたいことをやらせます。いろいろ経験できだし、ボランティアから いいことが勉強できます	自分[じぶん]がやりたいことをやります自分がやりたいことをやります			
\\	は人を
\\	は人に
\\	を
\\	は人に
\\	を
\\	い→かせます;
\\	させます;
\\	させます-こさせます)
\\	責任を持ちます	どこもに 何をさせますか。。。。子どもがいた時、私は子どもを 責任を持たせます。	責任[せきにん]を 持[も]ちます			
\\	は人を
\\	は人に
\\	を
\\	は人に
\\	を
\\	い→かせます;
\\	させます;
\\	させます-こさせます)
\\	なさいます	部長は アメリカへ 出張なさいます お仕事は何を なさっていますか。 。。。銀行員です。アップル銀行に勤めています	なさいます				
\\	します
\\	お出かけになります	部長は どちらですか 。。。さっき お出かけになります	お 出[で]かけになります			
\\	お
\\	(ます
\\	ます)に なります。
\\	見ます、寝ます。。。)		
\\	い→われます;
\\	られます;
\\	されます-こられます)	
\\	お
\\	(ます
\\	ます)に なります。
\\	見ます、寝ます。。。) 
\\	「お」
\\	「ご」
\\	~ください。 「お」
\\	「ご」
\\	言います
\\	ご覧になります	よく 映画を ご覧になりますか。 。。。そうですか。たまに 妻と見に行きます 中村課長は 歌舞伎を ご覧になりますか。 。。。はい、ご覧になります	ご 覧[らん]になります			
\\	見ます。
\\	ご存知です	小川さんの娘さんが 桜大学に 合格したのを ご存知ですか。 。。。いいえ、知りませんでした 日本の首相の 名前はを ご存知ですか 。。。いいえ、ご存知じゃ ありません。	ご 存知[ぞんじ]ですか			
\\	知っています。
\\	おっしゃいます	お名前は何と おっしゃいますか。 。。。ワットと言います 社長は 何と おっしゃいましたか。 。。。会議に 出席なさいます お名前は何と おっしゃいますか。 。。。ミラーと 言います	おっしゃいます				
\\	言います
\\	勤めます	お仕事は何を なさっていますか。 。。。銀行員です。アップル銀行に勤めています わたくしは 
\\	の会社に 勤めております	勤[つと]めます			
\\	(会社に勤めます)ー(会社で働きます)	
\\	「お」
\\	「ご」
\\	します
\\	いらっしゃいます	松本部長は どちらに いらっしゃいますか。 。。。ええ、こちらの 部屋です。どうぞ お入り ください。 松本部長は もう いらっしゃいましたか。 。。。はい、いらっしゃいました 田中さんはどちらに いらっしゃいますか。 。。。ロビーに いらっしゃいます 社長は どちらへ いらっしゃいましたか。 。。。銀行へ いらっしゃいました。	いらっしゃいます				
\\	「お」
\\	「ご」
\\	~ください。	
\\	行きます・来ます・います
\\	昨日の新年会に 出られますか。 。。。はい、 出ました	
\\	い→われます;
\\	られます;
\\	されます-こられます)
\\	吉野山に お花見に行かれますか。 。。。はい、5月に 行きました	
\\	い→われます;
\\	られます;
\\	されます-こられます)
\\	部長は どのくらい 夏休みを 取られますか。 。。。2週間ぐらい 取るつもりです	
\\	い→われます;
\\	られます;
\\	されます-こられます)
\\	部長は いつ 大阪に 引っ越しされますか。 。。。来週の日曜日です	
\\	い→われます;
\\	られます;
\\	されます-こられます)
\\	バス停	バス停の場所は お分かりになりますか。 。。。いいえ、分かりません	バス停[ばすてい]			
\\	お
\\	(ます
\\	ます)に なります。
\\	見ます、寝ます。。。)	
\\	旅館	日光では どんな所に お泊まりになりましたか。 。。。古い旅館に 泊まりました	旅館[りょかん]			
\\	お
\\	(ます
\\	ます)に なります。
\\	見ます、寝ます。。。)	
\\	日光市	日光では どんな所に お泊まりになりましたか。 。。。古い旅館に 泊まりました	日光[にっこう] 市[し]			
\\	お
\\	(ます
\\	ます)に なります。
\\	見ます、寝ます。。。)	
\\	ロビー	田中さんはどちらに いらっしゃいますか。 。。。ロビーに いらっしゃいます	ロビー			
\\	行きます・来ます・います
\\	召し上がります	ワット先生は お酒を 召し上がりますか。 。。。いいえ、召し上がりません	おっしゃいます				
\\	飲みます・食べます
\\	過ごします	いい週末を お過ごし ください 日本で本当に お世話になりました。日本で過ごした2年間は とても 速く過ぎました。 日本へ行ったばかりの時は、分からないことや 慣れないことが 多くて、皆様に 迷惑を おかけしましたが、本当に 親切にして いただきました。	過[す]ごします			
\\	「お」
\\	「ご」
\\	ください。	
\\	利用します	どなたでも ご利用してください。	利用[りよう]します			
\\	「お」
\\	「ご」
\\	~ください。	
\\	いすに掛けます	すみません、このいすに 掛けられますか。 。。。いいえ、どうぞ ありがとう ございます	いすに 掛[か]けます			
\\	い→われます;
\\	られます;
\\	されます-こられます)
\\	挨拶します	注意されたら 隣の子は ちゃんと 挨拶するように なった	挨拶[あいさつ]します			
\\	よろしくお伝えください	実は ハンスが 夕べ 熱を 出しまして、今朝も まだ 下がらないんです 。。。それは いけませんね。 それで 今日は 学校を 休ませますので、先生に よろしく お伝えください 。。。分かりました。どうぞ お大事に。 ありがとうございます。失礼いたします	よろしくお 伝[つた]えください			
\\	目指します	山中伸先生の経歴を紹介します山中伸先生は 1962年に大阪で お生まれになりました。1987年に 大学の医学部を 卒業して、医者になられましたが、研究者に目指して、大学院に進まれました。	目指[めざ]します			
\\	お
\\	(ます
\\	ます)に なります。
\\	見ます、寝ます。。。)	
\\	い→われます;
\\	られます;
\\	されます-こられます)
\\	細胞	その後、日本へ帰ってから、IPS細胞を 開発する研究を 始められました。 2006に マウスでIPS細胞を作られました。2007年に ヒトで成功されました。2012年に ノーベル賞を 受賞されました	細胞[さいぼう]				
\\	お
\\	(ます
\\	ます)に なります。
\\	見ます、寝ます。。。) 「お」
\\	「ご」
\\	い→われます;
\\	られます;
\\	されます-こられます)
\\	ノーベル賞	その後、日本へ帰ってから、IPS細胞を 開発する研究を 始められました。 2006に マウスでIPS細胞を作られました。2007年に ヒトで成功されました。2012年に ノーベル賞を 受賞されました	ノーベル 賞[しょう]			
\\	お
\\	(ます
\\	ます)に なります。
\\	見ます、寝ます。。。) 「お」
\\	「ご」
\\	い→われます;
\\	られます;
\\	されます-こられます)
\\	受賞します	その後、日本へ帰ってから、IPS細胞を 開発する研究を 始められました。 2006に マウスでIPS細胞を作られました。2007年に ヒトで成功されました。2012年に ノーベル賞を 受賞されました	受賞[じゅしょう]します			
\\	お
\\	(ます
\\	ます)に なります。
\\	見ます、寝ます。。。) 「お」
\\	「ご」
\\	い→われます;
\\	られます;
\\	されます-こられます)
\\	スケジュール	今月のスケジュールを お送りします。 。。。はい、いいです。このに 置いてください	スケジュール			
\\	「お」
\\	「ご」
\\	~します		
\\	伺います	明日 3時に 伺います 明日 3時に お宅へ 伺います おもしろい お話を 伺いました。 だれが 手伝いに 来て くれますか。 。。。わたくしが 手伝いに伺いますよ	伺[うかが]います			
\\	「お」
\\	「ご」
\\	~します 
\\	聞きます、行きます
\\	参りました	わたくしは ヒエップと申します。ベトナムから 参りました。ご結婚しております。	参[まい]りました			
\\	「お」
\\	「ご」
\\	~します 
\\	行きます、来ます
\\	重そうですね。お持ちしましょうか。 。。。すみません。お願いします	
\\	「お」
\\	「ご」
\\	~します 
\\	ガイド	ガイドさん、ここを 見た後で、どこへ 行くんですか。 。。。江戸東京博物館へ ご案内します	ガイド			
\\	「お」
\\	「ご」
\\	~します 
\\	博物館	ガイドさん、ここを 見た後で、どこへ 行くんですか。 。。。江戸東京博物館へ ご案内します	博物館[はくぶつかん]			
\\	「お」
\\	「ご」
\\	~します 
\\	私	グプタさんの 到着は 2時ですね。誰か 迎えに行くんですか。 。。。はい、私が 参りました	私[わたくし]			
\\	「お」
\\	「ご」
\\	~します 
\\	行きます、来ます
\\	拝見します	ちょっと 切符を 拝見します。 。。。はい。 どうも ありがとう ございました ご家族の写真を 拝見しました	拝見[はいけん]します			
\\	「お」
\\	「ご」
\\	~します 
\\	見ます
\\	申します	こちらは ヒエップです 。。。初めまして、ヒエップと 申します。どうぞ よそしくお願いします 。。。。。。こちらこそ よろしくお願いします	申[もう]します			
\\	「お」
\\	「ご」
\\	~します 
\\	言います
\\	おります	ご家族が どちらに いらっしゃいますか。 。。。ベトナムに おります	おります				「お」
\\	「ご」
\\	~します 
\\	います
\\	だれが このコーヒーを入れましたか。 。。。わたくしは お入れしました	
\\	「お」
\\	「ご」
\\	~します 
\\	わたくしが 今日の予定を ご説明します	
\\	「お」
\\	「ご」
\\	~します 
\\	お目にかかります	来週 部長に お目にかかります。 今日は 部長の奥様に お目にかかりました	お 目[め]にかかります			
\\	「お」
\\	「ご」
\\	~します 
\\	会います
\\	お手伝いします	
\\	「お」
\\	「ご」
\\	~します 
\\	よていを お知らせします	
\\	「お」
\\	「ご」
\\	~します 
\\	用意します	お弁当は こちらで ご用意します	用意[ようい]します			
\\	「お」
\\	「ご」
\\	~します 
\\	初めに	初めに 伊藤先生を ご紹介します	初[はじ]めに			
\\	「お」
\\	「ご」
\\	~します 
\\	この本が おもしろそうですね。 。。。お貸ししましょうか。 はい、ありがとうございます。	
\\	「お」
\\	「ご」
\\	~します 
\\	ちょっと 疲れしましたね。 。。。コーヒーを お入れしましょうか。 すみません、お願いします	
\\	「お」
\\	「ご」
\\	~します 
\\	いたします	だれが 運転して くれますか。 。。。わたくしが いたします	いたします				「お」
\\	「ご」
\\	~します 
\\	します
\\	メールアドレス	社長の メールアドレスを 知っていますか。 。。。いいえ、存じません	メールアドレス				「お」
\\	「ご」
\\	~します 
\\	存じます	社長の メールアドレスを 知っていますか。 。。。いいえ、存じません	存[ぞん]じます				「お」
\\	「ご」
\\	~します 
\\	知っています
\\	いただきます	わたくしが 部長の奥様に いただいた日本お料理を いただいてしまいました わたくしが ドイツのビールを いただいたこと あります	いただきます				「お」
\\	「ご」
\\	~します 
\\	食べます・飲みます・もらいます
\\	優勝	優勝 おめでとう ございます。すばらしい スピーチでした 。。。ありがとう ございます 緊張なさいましたか。 。。。はい、とても 緊張いたしました	優勝[ゆうしょう]				「お」
\\	「ご」
\\	~します 
\\	しますー
\\	なさいますー
\\	いたします
\\	緊張します	優勝 おめでとう ございます。すばらしい スピーチでした 。。。ありがとう ございます 緊張なさいましたか。 。。。はい、とても 緊張いたしました	緊張[きんちょう]します			
\\	「お」
\\	「ご」
\\	~します 
\\	しますー
\\	なさいますー
\\	いたします
\\	賞金	練習 大変でしたか 。。。ええ。忙しくて、なかなか 練習の時間 ありませんでした。 賞金は 何に お使いに なりますか 。。。そうですね。私は 動物が 好きで、子どもの 時から アフリカへ 行くのが夢でした	賞金[しょうきん]			
\\	お
\\	(ます
\\	ます)に なります。
\\	見ます、寝ます。。。)	
\\	かないます	じゃ、アフリカへ 行かれますか。 。。。はい、アフリカの自然の中で キリンや 像を 見たいと思います 子どものころの 夢が かなうんですね 。。。はい、うれしいです。応援して くださった 皆様に 心から 感謝いたします。 どうも ありがとう ございます	かないます			
\\	い→われます;
\\	られます;
\\	されます-こられます)
\\	応援します	じゃ、アフリカへ 行かれますか。 。。。はい、アフリカの自然の中で キリンや 像を 見たいと思います 子どものころの 夢が かなうんですね 。。。はい、うれしいです。応援して くださった 皆様に 心から 感謝いたします。 どうも ありがとう ございます	応援[おうえん]します			
\\	い→われます;
\\	られます;
\\	されます-こられます)
\\	感謝します	じゃ、アフリカへ 行かれますか。 。。。はい、アフリカの自然の中で キリンや 像を 見たいと思います 子どものころの 夢が かなうんですね 。。。はい、うれしいです。応援して くださった 皆様に 心から 感謝いたします。 どうも ありがとう ございます	感謝[かんしゃ]します			
\\	い→われます;
\\	られます;
\\	されます-こられます)
\\	迷惑	日本で本当に お世話になりました。日本で過ごした2年間は とても 速く過ぎました。 日本へ行ったばかりの時は、分からないことや 慣れないことが 多くて、皆様に 迷惑を おかけしましたが、本当に 親切にして いただきました。	迷惑[めいわく]			
\\	しょうゆをかけます	ベトナムの御飯は しょうゆをかけるんですか。 。。。はい、ナンプラを かけて 食べてください	しょうゆをかけます			
\\	て・Vないで 
\end{CJK}
\end{document}