\documentclass[8pt]{extreport} 
\usepackage{hyperref}
\usepackage{CJKutf8}
\begin{document}
\begin{CJK}{UTF8}{min}
\\	〔過去〕
\\	〔過分〕
\\	1【自】〔完了
\\	①(乗り物で)出発する 
\\	【自】〔完了
\\	旅立つ、(旅へ)出発する
\\	【副】①(相手の問い・疑いに対し強い肯定を示して)もちろん 
\\	君はピアノが弾けるの?―もちろんだよ! ③(先行する発言内容を限定して)ただし、もっとも 
\\	彼はロシア語が話せる、もっとも流ちょうではないが。
\\	【副】①(先行する発言内容を受けて)それでは、ということは 
\\	それではあしたお会いすることにしましょう! ②(より具体的に説明する導入として)つまり、要するに ③(理由を表わして)それゆえ、だから
\\	〔比較〕
\\	〔最上〕
\\	【形】① 年とった、老いた
\\	②〔数量〕
\\	歳の 
\\	古い
\\	君は何歳?-18歳だ。 
\\	年齢
\\	〔過去〕
\\	〔過分〕
\\	【他】①
\\	に飲食物など⁴を〕差し出す、勧める 
\\	に…⁴を〕提案する ④〔商品など⁴を〕売りに出す 
\\	〔過去〕
\\	〔過分〕
\\	1【他】①〔…⁴を〕始める、〔…⁴に〕取りかかる2【自】① 始まる
\\	を〕始める、
\\	に〕取りかかる
\\	1【自】〔完了
\\	到着する、着く
\\	〔参考〕3・4格支配の前置詞と結び付く場合、前置詞は3格支配 
\\	町に着く
\\	到着(〔略〕
\\	列車の到着を待つ
\\	〔過去〕
\\	〔過分〕
\\	1【他】①《4格》〔…⁴に〕電話をかける、電話をする
\\	2【自】電話をかける
\\	1【他】③〔…⁴を〕雇う
\\	答え、返事
\\	答える,返事する
\\	①(新聞などの)広告
\\	仕事
\\	(単2)‐
\\	/(複1・2・4)
\\	(複3)
\\	医者、医師
\\	/〔女性〕
\\	1【他】①〔…⁴を〕上演する、演奏する、上映する
\\	チャップリンの映画を上映する 2【再】
\\	様態》〔…の〕態度をとる、ようにふるまう
\\	課せられたこと、課題
\\	、②〔ふつう複数で〕宿題、練習問題
\\	1【他】①⦅口語⦆〔…⁴を〕開ける、開く
\\	②⦅口語⦆〔店など⁴を〕(新しく)開店する、開業する 2【自】⦅口語⦆(店などが新しく)開店する 3【再】
\\	出発する
\\	〔過去〕
\\	〔過分〕
\\	【自】①〔完了
\\	(ベッドなどから)起きる、起床する
\\	②〔完了
\\	(座った・寝た状態から)立ち上がる、起き上がる
\\	【他】①〔用紙など⁴に〕記入する
\\	浴室、バスルーム
\\	1【自】①入浴する、ふろに入る
\\	②(楽しみで)泳ぐ、水浴びをする
\\	駅(〔略〕
\\	【副】① まもなく、じきに
\\	⦅口語⦆じゃあね!、またね!
\\	1【他】①〔…⁴を〕建てる
\\	②〔機械など⁴を〕組み立てる、制作する
\\	急ぐ 
\\	[派生] 
\\	急ぎ
\\	〔現在〕
\\	〔過去〕
\\	〔過分〕
\\	1【再】①
\\	方向》〔…へ〕行く 
\\	出会い、遭遇
\\	〔過去〕
\\	〔過分〕
\\	〔接Ⅱ〕
\\	1【他】①〔…⁴を〕もらう、受け取る
\\	子供ができる ②〔…⁴を〕(努力して)手に入れる、獲得する 2【自】〔完了
\\	の〕体に合う、健康に良い 
\\	【他】〔…⁴を〕利用する、使う
\\	忙しい
\\	【他】〔…⁴を〕見学する、見物する
\\	〔過去〕
\\	〔過分〕
\\	〔接Ⅱ〕
\\	1【他】①〔試験など⁴に〕合格する
\\	2【自】① 存在する 
\\	から〕成る、構成されている 
\\	「定める」の過去分詞) 【形】①(数値などが)定められた、決められた Ⅲ【副】きっと、確かに 
\\	【他】①〔…⁴の〕代金を支払う
\\	タクシー代〈ホテルの部屋代〉を払う
\\	図書館
\\	蔵書 発音:
\\	【形】①(値段の)安い 
\\	〔過去〕
\\	〔過分〕
\\	〔接Ⅱ〕
\\	【自】〔完了
\\	①〔…に〕とどまる、滞在する 
\\	②《状態》〔…の〕ままである
\\	土地;土壌 
\\	地面;床 
\\	屋根裏部屋
\\	【他】①〔…⁴を〕必要とする
\\	②〔時間⁴が〕かかる、〔時間⁴を〕要する 
\\	〔過去〕
\\	〔過分〕
\\	〔接Ⅱ〕
\\	【他】②
\\	方向》〔…⁴を…へ〕連れて行く ③〔…⁴を〕もたらす 
\\	(単2)‐
\\	/(複1・2・4)
\\	(複3)
\\	兄、弟;兄弟
\\	バス 
\\	【副】当時
\\	【副】①(順番を表わして)それから
\\	③(条件を表わして)それならば 
\\	天気が良ければ私たちは遠足に行くつもりです。
\\	1【自】①〔…の間〕続く、かかる
\\	〖非人称の
\\	を主語として〗
\\	どれくらい時間がかかるのでしょうか?
\\	手紙 
\\	喫茶店 アクセント:
\\	、発音:カフェー
\\	【他】①〔…⁴を〕印刷する
\\	①〔複数なし〕愚かさ、無知
\\	シャワー
\\	(ホテルで)シャワー付きのシングルルームはありますか?
\\	Ⅰ【副】やっと、ついに
\\	Ⅰ【副】① ほんとうは (〔略〕
\\	本来は、もともとは ③〖疑問文で〗そもそも、ところで 
\\	そもそもあなたは何をお望みなのですか? Ⅱ【形】〖付加語的に〗① ほんとうの
\\	入口 
\\	②〔複数なし〕⦅文語⦆(郵便物の)到着
\\	〔現在〕
\\	〔過去〕
\\	〔過分〕
\\	【他】①〔…⁴を〕招待する、招く
\\	私は友人たちをコンサート〈食事〉に招待した 
\\	招待、誘い
\\	1【副】1度、1回
\\	再び;もう一度[だけ] 2【副】① 昔、かつて 
\\	昔々王様がいました
\\	1【他】①〔…⁴を〕包む、包装する
\\	〔過去〕
\\	〔過分〕
\\	【自】〔完了
\\	①(乗り物に)乗る、乗車する
\\	1①〔複数なし〕終わり
\\	【他】①〔…⁴を〕発見する、(偶然)見つける
\\	〔過去〕
\\	〔過分〕
\\	【再】
\\	決心する、決める
\\	私は旅行に行くことにした。 
\\	地震 
\\	成功
\\	1【再】
\\	〔…⁴を〕思い出す、覚えている
\\	私は祖母のことをよく覚えている。 2【他】
\\	〔…⁴に…⁴を〕思い出させる
\\	1【再】
\\	風邪を引く
\\	1【他】①〔…⁴を〕説明する
\\	【他】②〔…⁴に〕達する、着く、至る 
\\	(飛行機などが)1万メートルの高度に達する 
\\	まにあう 
\\	列車にまにあう ④〔…⁴と〕連絡がつく
\\	Ⅰ【数】《序数》(順序が)第1の、最初の
\\	/基数
\\	Ⅲ【副】① 最初に ②〖時点を表わす語句と〗やっと、ようやく
\\	ほんの、まだ
\\	〔現在〕
\\	〔過去〕
\\	〔過分〕
\\	〔接Ⅱ〕
\\	1【他】①〔…⁴を〕食べる
\\	①(学問的な)実験 
\\	1⦅地名⦆ヨーロッパ、欧州
\\	アクセント:
\\	、発音:オイローパ
\\	【形】①
\\	をする〕能力がある、
\\	をすることが〕できる
\\	彼はこの仕事をこなす能力がある。 
\\	有能な、才能ある
\\	乗車〈船〉券、切符
\\	(バス・市電などの)乗車券、切符
\\	【形】① 間違った、誤った
\\	〔過去〕
\\	〔過分〕
\\	〔接Ⅱ〕
\\	1【他】①〔…⁴を〕捕らえる、捕まえる;逮捕する
\\	怠惰
\\	[派生] 
\\	怠惰な, 怠け者の
\\	【複】①(学校などの)休暇
\\	〔過去〕
\\	〔過分〕
\\	〔接Ⅱ〕
\\	1【他】①〔…⁴を〕(偶然に・努力して)見つける
\\	②〔欲しいもの⁴を〕得る 
\\	状態》〔…⁴を…だと〕思う 2【再】①
\\	見つかる
\\	【形】① 勤勉な、熱心な
\\	発音:
\\	(印刷された)書き込み用紙、届出〈申告・申請〉用紙
\\	発音:
\\	問い、質問
\\	質問する 
\\	私はあなたに聞きたいことがあります。 
\\	問題
\\	1【他】①《4格》〔…⁴に〕尋ねる
\\	私は彼に名前を尋ねた。
\\	1【再】①
\\	〔…⁴を〕喜ぶ 
\\	〔先のこと⁴を〕楽しみにしている 2【他】〔…⁴を〕喜ばせる
\\	(単2)‐
\\	/(複1・2・4)‐
\\	(複3)‐
\\	友達 
\\	/〔女性〕
\\	ボーイフレンド(㊟ふつう所有冠詞と)
\\	春 
\\	2【再】①
\\	状態》(自分が)〔…だと〕感じる 
\\	ご気分はいかがですか? 
\\	英雄になったような気になる
\\	(単2)‐
\\	/(複1・2・4)
\\	(複3)
\\	客
\\	バイオリン
\\	①〔複数なし〕金(かね)
\\	(単2)‐
\\	/(複1・2・4)‐ (複3)‐
\\	絵画、(特に)油絵
\\	Ⅰ【副】① 今、ちょうど ②(時間的に)たった今 
\\	私は今出かけようと思っていた。 
\\	その時 
\\	発音:
\\	器具、用具
\\	〔比較〕
\\	〔最上〕
\\	【副】喜んで、好んで 
\\	あなたはワインがお好きですか?
\\	店 
\\	商売、取り引き 
\\	贈り物、プレゼント
\\	法律 
\\	法に従って 
\\	法則;原理、原則
\\	1【他】①〔…⁴と〕思う、信じる
\\	㊟ふつう
\\	文か
\\	不定詞句と) 2【自】②
\\	〔…⁴の存在を〕信じる 
\\	〔…⁴を〕信頼する 
\\	Ⅰ【形】① 同じ 
\\	Ⅱ【副】① すぐに 
\\	②すぐ近くに
\\	…番線、…番ホーム
\\	Ⅰ【形】① 幸福な、幸せな
\\	②【付加語的に】幸運な、運のいい
\\	〔比較〕
\\	〔最上〕
\\	Ⅰ【形】① 大きい
\\	②〔数量〕〔…の〕大きさの 
\\	彼女の身長は1メートル75センチだ。
\\	【形】① 有利な、好都合な
\\	この値段は実際手ごろだ。 
\\	〔比較〕
\\	〔最上〕
\\	【形】①(質的に)良い、上質の 
\\	Ⅱ【副】(時刻を表わして)…半 
\\	4時半だ。
\\	激しい、強烈な 
\\	激しい嵐
\\	【形】① 熱い
\\	暑い
\\	〔過去〕
\\	〔過分〕
\\	〔接Ⅱ〕
\\	1【自】①〖述語と〗(…という)名である、(…と)呼ばれている ②〖述語と〗(…を)意味する 
\\	秋
\\	【副】① きょう、本日 
\\	【副】① ここに、ここで
\\	結婚式 
\\	思いつき;考え
\\	私にいい考えがある。
\\	【副】① いつも、常に
\\	②〖比較級と〗ますます、いっそう 
\\	彼らはますます大声で話した。 ③〖疑問詞と〗たとえ・・・であろうと 
\\	1【再】
\\	〔…⁴に〕興味を持つ 
\\	ペーターはクラシック音楽に興味がある 2【他】《4格》〔…⁴に〕興味を起こさせる
\\	年 
\\	【副】① 今、現在 
\\	コーヒー 
\\	【形】⦅口語⦆① 壊れた、故障した
\\	発音:
\\	(単2)‐/(複1)‐
\\	ふつう複数で ジャガイモ
\\	レジ 
\\	レジでお支払いください。
\\	デパート、百貨店
\\	ピアノ
\\	コンサート 発音:
\\	【他】①〔…⁴を〕複写する、コピーする 
\\	【形】① 病気の
\\	病院 
\\	台所、キッチン
\\	②〔複数なし〕料理、料理法
\\	〔比較〕
\\	〔最上〕
\\	【形】①(空間的に)短い 
\\	すばやい
\\	【副】残念ながら、あいにく
\\	〔過去〕
\\	〔過分〕
\\	〔接Ⅱ〕
\\	【他】①〔…⁴を〕読む 
\\	⑥〔…⁴を〕摘み取る、拾い集める 
\\	ブドウを摘む
\\	【形】① 最後の、最終の 
\\	最近の;この前の  
\\	先月に 
\\	(または
\\	最近
\\	【複】① 人々 
\\	部下、従業員 
\\	(単2)‐
\\	/(複1・2・4)‐
\\	(複3)‐
\\	歌 
\\	【他】①〔…⁴を〕ほめる、賞賛する
\\	飲食店
\\	うそ 
\\	1②〔…⁴を〕する、行なう
\\	1【他】①〔絵など⁴を〕(絵の具で)描く、かく
\\	【副】ときどき 
\\	市(いち)、市場(いちば)
\\	機械 
\\	、 発音:
\\	海
\\	家賃 
\\	〔過去〕
\\	〔過分〕
\\	【自】〔完了〕
\\	いっしょに来る、いっしょに行く 
\\	いっしょに来るかい?
\\	昼、正午
\\	昼ご飯を食べる
\\	1【他】
\\	に…⁴を〕伝える、知らせる
\\	【副】① あす、明日
\\	【形】① 疲れた
\\	音楽 
\\	発音:
\\	知らせ;ニュース  ②〔複数で〕(ラジオ・テレビの)ニュース番組
\\	Ⅰ【形】②〖付加後的に〗(時間的に)次の
\\	Ⅰ【副】① もちろん 
\\	Ⅱ【形】②(成り行きとして)当然の
\\	〔過去〕
\\	〔過分〕
\\	〔接Ⅱ〕
\\	【他】③〔…⁴を〕使う、利用する 
\\	私は次の列車に乗ります。  ④(検討して)〔…⁴に〕決める、(店などで)〔…⁴に〕する 
\\	(店で)私はこの靴にします。
\\	〔比較〕
\\	〔最上〕
\\	【形】① 新しい、新品の
\\	番号、ナンバー(〔略〕
\\	複数
\\	〔過去〕
\\	〔過分〕
\\	1【他】〔…⁴を〕開ける
\\	【副】しばしば、たびたび
\\	手術 アクセント:
\\	、発音:オペラツィオーン
\\	1【他】①〔…⁴を〕手術する
\\	1【自】〔完了
\\	①(不都合なことが)起こる 
\\	2【他】①〔…⁴を〕通過する、越える
\\	平鍋、フライパン 
\\	計画、プラン ③(一地域の)地図
\\	郵便局
\\	値段、価格
\\	賞 
\\	等賞
\\	【形】〔格語尾は付かない〕①⦅口語⦆すばらしい、すてきな
\\	問題 
\\	平気だよ!;いいよ!
\\	試験 
\\	【形】時間どおりの 
\\	1【他】①〔…⁴を〕(こすって・ふいて)きれいにする、磨く
\\	平方メートル(〔記号〕
\\	発音:
\\	雨
\\	1[非人称]① 雨が降る 
\\	激しく雨が降っている。 
\\	【形】① 金持ちの、裕福な
\\	豊富な、豊かな 
\\	3と〗
\\	この国には湖が多い。 
\\	米;ご飯
\\	【他】〔…⁴を〕修理する
\\	レストラン
\\	Ⅰ【形】① 正しい 
\\	まともな、ちゃんとした ④〖付加語的に〗ほんとうの、本物の、真の 
\\	長編小説 発音:
\\	【形】〖述語的に〗残念な 
\\	あなたがもう行かなければならないとは残念です。
\\	【再】
\\	恥じる、恥ずかしく思う
\\	彼は自分のうそを恥じる。
\\	俳優、役者
\\	/〔女性〕-
\\	離婚 
\\	〔過去〕
\\	〔過分〕
\\	〔接Ⅱ〕
\\	【自】①
\\	不定詞句》〔…のように〕見える ②(主に天体が)輝く、照る
\\	〔過去〕
\\	〔過分〕
\\	〔接Ⅱ〕
\\	1【自】① 眠る、寝る 
\\	Ⅰ【形】①(質において)悪い
\\	〔過去〕
\\	〔過分〕
\\	〔接Ⅱ〕
\\	1【他】①〔…⁴を〕閉める、閉じる
\\	②〔店など⁴を〕閉める
\\	(単2)‐
\\	/(複1・2・4)
\\	(複3)
\\	城
\\	〔過去〕
\\	〔過分〕
\\	〔接Ⅱ〕
\\	1【他】①〔…⁴を〕切る 
\\	【形】① 速い 
\\	できるだけ急いで来い!
\\	〔過去〕
\\	〔過分〕
\\	〔接Ⅱ〕
\\	1【他】①〔文字など⁴を〕書く 2【自】② 手紙を書く 【3格/
\\	4格と】
\\	彼は母親に〈会社あてに〉手紙を書いた
\\	(単2)‐
\\	/(複1・2・4)‐(複3)‐
\\	生徒
\\	/〔女性〕
\\	Ⅰ【形】① 重い
\\	難しい 
\\	姉、妹;姉妹
\\	私の姉〈妹〉 
\\	湖
\\	湖で泳ぐ
\\	【副】とても、非常に
\\	秘書
\\	/〔女性〕‐
\\	1【再】①
\\	方向》〔…に〕座る
\\	ベンチに座る
\\	Ⅰ【形】① 安全な
\\	3格と】
\\	ここにいれば攻撃を受ける心配はない ②(信頼性において)確かな、信頼できる Ⅱ【副】きっと、確かに
\\	〔過去〕
\\	〔過分〕
\\	〔接Ⅱ〕
\\	1【他】①〔…⁴を〕歌う 
\\	2【自】① 歌う ②(鳥が)さえずる
\\	ソファー
\\	【副】すぐに、ただちに
\\	夏 
\\	①〔複数なし〕太陽
\\	②〔複数なし〕日光 
\\	日なたに座っている
\\	【形】①(時間的に)遅い 
\\	彼はよく遅刻する。
\\	1【他】①〔球技など⁴を〕する
\\	②〔…⁴をして〕遊ぶ、〔ゲーム⁴を〕する ③〔…⁴を〕演奏する、弾く 2【自】⑤ 賭け事をする 
\\	金を賭けてゲームをする
\\	〔過去〕
\\	〔過分〕
\\	〔接Ⅱ〕
\\	1【他】①〔言語⁴を〕話す 
\\	彼は流ちょうにドイツ語を話す。 2【自】③
\\	〔…4について〕話をする、話し合う
\\	(単2)‐/(複1・2・4)
\\	(複3)
\\	町 
\\	町へ行く ③〔複数なし〕市当局
\\	〔過去〕
\\	〔過分〕
\\	〔接Ⅱ〕
\\	1【自】①〔…に〕立っている
\\	③《場所》〔…に〕書いてある、載っている ⑨《3格》
\\	に〕似合う
\\	〔過去〕
\\	〔過分〕
\\	〔接Ⅱ〕
\\	1【他】①〔…⁴を〕盗む 
\\	から時間など⁴を〕奪う
\\	1【他】①《4格
\\	方向》〔…⁴を…に〕立てる、(立てて)置く
\\	〔過去〕
\\	〔過分〕
\\	〔接Ⅱ〕
\\	1【自】〔完了
\\	死ぬ 
\\	70歳で死ぬ
\\	1【他】①〔…⁴の〕邪魔をする
\\	②〔…⁴を〕妨害する、乱す ③〔…⁴の〕気に障る、気に掛かる
\\	(単2)
\\	/(複1・2・4)
\\	(複3)
\\	①(一人用で背もたれがある)いす
\\	時間(60分)
\\	1【他】①〔…⁴を〕探す
\\	②〔…⁴を〕求める、得ようとする 
\\	けんかしたがる
\\	【形】①(味・香りが)甘い 
\\	かわいい
\\	①(0時から24時までの)1日、日
\\	(単2)‐
\\	/(複1・2・4)‐(複3)‐
\\	犯人(〔女性〕‐
\\	〔過去〕
\\	〔過分〕
\\	【自】①
\\	に〕参加する
\\	電話、電話機(〔略〕
\\	②(診察などの)予約 
\\	医者に予約をしてある 発音:
\\	【形】① 値段の高い、高価な
\\	〔現在〕
\\	〔過去〕
\\	〔過分〕
\\	〔接Ⅱ〕
\\	1【他】①《4格》〔…⁴に〕(申し合わせて・偶然)会う 
\\	〖相互的に(4格)〗
\\	どこで会いましょうか? ②《4格》〔…⁴に〕命中させる 
\\	猟師はその鳥をしとめた ③《4格》〔…⁴に〕(弾丸などが)命中する、当たる 
\\	〖特定の名詞と〗(…を)する 
\\	決定をする 
\\	選択する 3【再】
\\	と〕(申し合わせて)会う
\\	〔過去〕
\\	〔過分〕
\\	〔接Ⅱ〕
\\	1【他】①〔…⁴を〕飲む 
\\	2【自】②
\\	〔…⁴のために〕乾杯する 
\\	〔過去〕
\\	〔過分〕
\\	【自】〔完了
\\	乗り換える 
\\	〔過去〕
\\	〔過分〕
\\	1【自】〔完了
\\	引っ越す、移転する 
\\	新しい家に引っ越す 3【再】
\\	着替える
\\	署名、サイン
\\	署名をする
\\	調査 
\\	診察 
\\	調べる;診察する
\\	(会社・官庁・軍隊に勤める人の)休暇
\\	休暇をとる〈申請する・もらう〉 
\\	スイスで休暇を過ごす 
\\	〔過去〕
\\	〔過分〕
\\	1【他】
\\	に…⁴を〕禁じる、禁止する
\\	1【他】①〔…⁴を〕売る
\\	2【再】①
\\	様態》〔…に〕売れる 
\\	この本はよく売れている
\\	【他】①〔…⁴を〕要求する;〔代金⁴を〕請求する
\\	このトマトはいくらになりますか? 
\\	3格と〗
\\	その警官は私に身分証明書の提示を求めた 2【自】
\\	を〕求める、ほしがる
\\	【再】
\\	遅れる、遅刻する 
\\	バスがまたしても遅れた。
\\	遅れ;遅延
\\	飛行機は1時間遅れている。
\\	〔過去〕
\\	〔過分〕
\\	〔接Ⅱ〕
\\	1【他】①
\\	に…⁴を〕約束する
\\	[派生] 
\\	約束
\\	〔比較〕
\\	〔最上〕
\\	Ⅰ【形】1①〖複数名詞と〗多くの、たくさんの 
\\	②〖単数名詞と〗多くの ①1格と4格では格語尾は付かないことがある。②物質・抽象名詞の前では格語尾は付かないが、定冠詞や所有冠詞があれば付く 
\\	たくさんのお金
\\	【副】① ひょっとしたら、もしかしたら
\\	②〖依頼を表わす疑問文で〗もしや、よろしければ
\\	午前 
\\	1【他】①
\\	に…⁴を〕紹介する
\\	〔…⁴を〕想像する 2【再】①
\\	自己紹介する 
\\	自動車、車
\\	馬車;荷車
\\	品物、商品
\\	1【自】①
\\	〔…⁴を〕待つ
\\	1①(物質としての)水 
\\	【副】⦅口語⦆①〖動詞
\\	と〗なくなった、消えた 
\\	列車はもう出てしまった。 
\\	離れて、去って
\\	クリスマス 
\\	ワイン、ブドウ酒 
\\	〔過去〕
\\	〔過分〕
\\	/受動 
\\	〔接Ⅱ〕
\\	Ⅰ【自】〔完了〕
\\	①〖述語と〗(…に)なる
\\	(質的に変わって)
\\	に〕なる、変わる 
\\	から〕生じる Ⅱ1《未来の助動詞》 2《受動の助動詞》
\\	〔過去〕
\\	〔過分〕
\\	〔接Ⅱ〕
\\	1【他】①
\\	方向》〔…⁴を…へ・から〕投げる
\\	彼は本を壁に投げつけた。 3【再】②
\\	〔…⁴に〕飛びかかる、襲いかかる
\\	作品
\\	②〔複数なし〕仕事 
\\	行為、行ない;しわざ
\\	①〔複数なし〕天気、天候、気象
\\	【形】① 重要な、重大な
\\	もったいぶった、偉そうな 
\\	【副】①(繰り返しを表わして)再び、また、もう一度
\\	②(元の状態への回帰を表わして)元どおりに
\\	〔過去〕
\\	〔過分〕
\\	【他】①《4格》〔…⁴に〕再会する ②〔…⁴を〕再び見る、再び訪れる
\\	風
\\	冬 
\\	〔過去〕
\\	〔過分〕
\\	〔接Ⅱ〕
\\	1【他】①〔…⁴を〕(知識として)知っている
\\	週、週間
\\	週末、ウイークエンド
\\	週末に
\\	【自】①〔…に〕住んでいる、住む
\\	彼女は両親の家に住んでいる。 
\\	①(マンションなどの)住まい、住居
\\	辞書 
\\	Ⅰ【形】① すばらしい、みごとな
\\	①〔複数なし〕(経過していく)時、時間
\\	時代
\\	雑誌 
\\	新聞 
\\	(単2)‐
\\	/(複1・2・4)‐ (複3)‐
\\	①(家・ホテルなどの)部屋・室
\\	(単2)‐
\\	/(複1・2・4)
\\	(複3)
\\	列車、電車
\\	⑩(チェスなどの)指し手
\\	〔現在〕
\\	〔過去〕
\\	〔過分〕
\\	【他】②〔…⁴を〕(正しいと)認める 
\\	それが真実だということは認めるよ。
\\	未来、将来
\\	1【他】⦅口語⦆①〔…⁴を〕閉める、閉じる
\\	〔過去〕
\\	〔過分〕
\\	【他】①〔…⁴を〕返す、返却する
\\	【副】① いっしょに 
\\	全部で、合わせて、合計して 
\\	晩、夕方 
\\	問い合わせ
\\	申し出、提案
\\	特価;特売品 
\\	1【自】答える、返事する 
\\	4と〗
\\	問い合わせに折り返し返事する
\\	1【自】① 働く;勉強する
\\	【形】失業している、仕事のない
\\	(単2)‐
\\	/(複1・2・4)‐ (複3)‐
\\	外国人
\\	/〔女性〕‐
\\	[派生] 
\\	外国
\\	証明書、身分証明書
\\	(単2)‐
\\	/(複1・2・3・4)‐
\\	自動車、車
\\	ベンチ
\\	銀行
\\	例;見本
\\	【形】①(多くの人に)知られた、有名な
\\	職業
\\	あなたはどんな職業ですか?-私の職業はパン屋です。
\\	【形】有名な 
\\	ベッド 
\\	鉛筆
\\	(単
\\	/(複1・2・4)-
\\	(複
\\	視線 
\\	4格 
\\	(または
\\	…⁴に目をやる③ 眺め、見晴らし
\\	【形】① 幅の広い 
\\	②〔数量〕〔…の〕幅のある 
\\	その道路は20メートルの幅である。
\\	(単2)‐
\\	(単3・4)‐
\\	/(複1・2・3・4)‐
\\	文字
\\	(単2)‐
\\	/(複1・2・3・4)‐
\\	(会社の課・部・局などの)長(〔女性〕‐
\\	(単
\\	/(複
\\	コンピュータ
\\	1【自】①《3格》
\\	に〕感謝する 
\\	あなたの手助けに感謝いたします。
\\	【副】①(建物などの)外に、外で
\\	①(二つの線または面が出会う)角、隅
\\	②(二本の通りが出合う)角、街角
\\	1【自】買い物をする 2【他】①〔食料品など⁴を〕買う
\\	Ⅰ1【他】①〔…⁴を〕(聞いたり読んだりして)知る ②〔…⁴を〕経験する 2【自】
\\	を〕聞き知るⅡ【形】経験を積んだ、熟練した
\\	〔過去〕
\\	〔過分〕
\\	①〔…⁴を〕発明する、考案する
\\	②〔作り話など⁴を〕考え出す、でっち上げる
\\	1【他】〔…⁴を〕物語る、話す
\\	工場
\\	(単2)‐
\\	/(複1・2・4)‐ (複3)‐
\\	運転手、ドライバー
\\	/〔女性〕‐
\\	(単2)‐
\\	/(複1・2・4)
\\	(複3)
\\	自転車
\\	〔過去〕
\\	〔過分〕
\\	【自】テレビを見る
\\	(単2)‐
\\	/(複1・2・3・4)
\\	空港
\\	朝食 
\\	1【自】朝食をとる
\\	(単2)‐
\\	/(複1・2・3・4)
\\	庭 
\\	〔過去〕
\\	〔過分〕
\\	〔接Ⅱ〕
\\	1【他】②
\\	に…⁴を〕与える
\\	〔過去〕
\\	〔過分〕
\\	〔接Ⅱ〕
\\	Ⅰ1【自】《3格》
\\	の〕気に入る
\\	〔過去〕
\\	〔過分〕
\\	〔接Ⅱ〕
\\	1【自】①〔…へ〕行く、出かける
\\	(単2)‐
\\	/(複1・2・4)
\\	(複3)
\\	理由、根拠;動機 
\\	4格 
\\	…⁴の理由を挙げる ②〔複数なし〕基礎 ④〔複数なし〕底、底面 
\\	根本的には、結局のところ
\\	(単2)‐/(複1・2・4)
\\	(複3)
\\	手
\\	1【他】①〔…⁴を〕生産する、製造する
\\	助け、手伝い 
\\	〔比較〕
\\	〔最上〕
\\	㊟格語尾が付く場合は 
\\	Ⅰ【形】① 高い
\\	②〔数量〕〔…の〕高さの 
\\	北方の
\\	【形】① 興味深い、おもしろい
\\	インターネット
\\	上着、ジャケット
\\	日本人
\\	/〔女性〕‐
\\	発音:
\\	〔比較〕
\\	〔最上〕
\\	【形】① 若い 
\\	カード
\\	トランプ
\\	トランプをする 
\\	はがき  
\\	地図 
\\	クラス 
\\	【形】① 小さい
\\	1【他】①〔…⁴を〕作曲する
\\	戦争 
\\	戦士、兵士
\\	(単2)‐
\\	/(複1・2・4)‐ (複3)‐
\\	ボールペン
\\	弱変化;(単2・3・4)
\\	/(複1・2・3・4)
\\	1①(商店などの)客 
\\	/〔女性〕
\\	【自】① 笑う 
\\	私たちはパーティーでたくさん笑った。
\\	(単2)‐
\\	/(複1・2・4)
\\	(複3)
\\	国
\\	田舎 
\\	彼は田舎に住んでいる。 ⑥(ドイツ・オーストリアの)州
\\	〔比較〕
\\	〔最上〕
\\	Ⅰ【形】①(空間的に)長い 
\\	②〔数量〕〔…の〕長さの
\\	(単2)‐
\\	/(複1・2・4)‐ (複3)‐
\\	教師、先生
\\	/〔女性〕
\\	Ⅰ【形】① 軽い 
\\	易しい、簡単な 
\\	ちょっとした、わずかな
\\	1【他】〔…⁴を〕学ぶ 
\\	〔過去〕
\\	〔過分〕
\\	〔接Ⅱ〕
\\	【自】②〔…に〕ある、置いてある 
\\	その本は机の上に置かれている。 ③《場所/状態》(町などが)〔…に〕位置している、ある
\\	【副】①(数式で)掛ける(〔記号〕×) ②〖命令文で;文アクセントなしで〗⦅口語⦆(命令を和らげて)ちょっと 
\\	ちょっとおいで! ③⦅口語⦆ 以前に[一度]、かつて
\\	(単2)‐
\\	/(複1・2・4)
\\	(複3)
\\	①(成年の)男、男性
\\	夫
\\	Ⅰ【形】② たいていの、大部分の
\\	中央、真ん中
\\	②(時間・年齢などの)中ごろ、半ば
\\	(単2)‐
\\	/(複1・2・4)‐ (複3)‐
\\	①〔ふつう複数で〕家具
\\	(暦の)月 
\\	Ⅰ【形】① 親切な、感じのいい 
\\	すてきな、かわいい
\\	Ⅰ【副】① まだ 
\\	②(残り・限定などを表わして)あと
\\	【副】① ただ…だけ
\\	⑤【補足疑問文で;文アクセントなしで】(強く答えを求めて)
\\	(単2)‐
\\	/(複1・2・4)
\\	(複3)
\\	広場 
\\	場所 
\\	席
\\	①(宣伝用の)パンフレット、カタログ 
\\	請求書;勘定
\\	Ⅰ【副】① 右に、右側に
\\	旅行代理店、旅行案内所
\\	【他】〔…⁴を〕予約する 
\\	【他】①
\\	に…⁴を〕言う
\\	(単2)‐
\\	/(複1・2・4)‐(複3)‐
\\	歌手(〔女性〕-
\\	1【他】①〔…⁴を〕送る、届ける
\\	鍵(かぎ)、キー
\\	靴
\\	①(施設・建物などを表わして)学校
\\	【形】① 黒い
\\	〔過去〕
\\	〔過分〕
\\	〔接Ⅱ〕
\\	1【他】①〔…⁴が〕見える
\\	②〔…⁴を〕(意識的に)見る ④〔…⁴が〕わかる
\\	〔過去〕
\\	〔過分〕
\\	〔接Ⅱ〕
\\	1【自】① 座っている
\\	遊び 
\\	試合
\\	場所、場 
\\	職 
\\	彼は経理の職に就いている。 
\\	立場 
\\	私が君の立場だったらそれはしないだろう。
\\	(単2)‐
\\	/(複1・2・4)
\\	(複3)
\\	②(バラ・ブドウなどの)低木 
\\	2(建物の)階
\\	私たちは2階〈3階〉に住んでいる。 〔参考〕階はふつう2階から数え始めるので、例えば
\\	「2番目の階」は日本語では「3階」に当たる。なお、「1階」は
\\	または
\\	という。
\\	弱変化;(単2・3・4)‐
\\	/(複1・2・3・4)‐
\\	大学生、男子学生(〔女性〕‐
\\	1【他】①〔…⁴を〕大学で学ぶ、専攻する ②〔…⁴を〕研究する
\\	1【他】①〔…⁴を〕分ける、分割する
\\	②〔…⁴を〕分け合う、分配する 
\\	3と〗
\\	彼女はリンゴを妹と分け合った ③〔…⁴を〕共用する ④〔考え・感情など⁴を〕分かち合う、共にする
\\	テーブル、机
\\	②〖無冠詞で〗食事
\\	【形】①⦅口語⦆ すてきな、すごくいい
\\	それはすばらしいアイデアだ!
\\	〔過去〕
\\	〔過分〕
\\	〔接Ⅱ〕
\\	1【他】①〔…⁴を〕する、行なう 
\\	2【自】〔…の〕ふりをする、〔…のように〕ふるまう
\\	時計 
\\	②〔不変化〕(時間表示の単位として)…時
\\	【他】①〔…⁴を〕稼ぐ、もうける
\\	②〔称賛など⁴に〕値する、〔…⁴を〕受けて当然である
\\	【他】〔車・住居など⁴を〕賃貸する
\\	4分の1;15分[間]
\\	③(街の特定の)地区、地域
\\	Ⅰ【副】まったく、完全に 
\\	まったく君の言うとおりだ。
\\	午前に、午前中に(〔略〕
\\	Ⅱ【副】少ししか…でない、あまり…でない 
\\	私はあまり映画を見に行かない。
\\	Ⅰ【副】① ほんとうに 
\\	センチメートル(〔記号〕
\\	90センチメートルの布
\\	【他】①〔…⁴を〕破壊する
\\	【形】満足した、満ち足りた
\\	3格と】
\\	私は新しい車にとても満足している 
\\	〔現在〕
\\	〔過去〕
\\	〔過分〕
\\	【自】①《状態》〔…のように〕見える
\\	⦅口語⦆
\\	のように〕思われる、見える
\\	(単2)‐
\\	/(複1・2・4)
\\	(複3)
\\	腹
\\	1【他】①〔…⁴を〕固定する 
\\	絵を壁に固定する
\\	〔過去〕
\\	〔過分〕
\\	【他】①
\\	に…⁴を〕教える
\\	【他】①〔…⁴を〕注文する 
\\	②〔…⁴を〕呼ぶ、来させる 
\\	タクシーを呼ぶ
\\	花
\\	(単2)‐
\\	/(複1・2・4)
\\	(複3)
\\	本 
\\	③⦅口語⦆医者
\\	【副】あそこに、あそこで
\\	のどの渇き
\\	(単2)‐
\\	(時に‐
\\	/(複1・2・4)‐
\\	(複3)‐
\\	卵
\\	③〔ふつう複数で〕⦅口語⦆ 睾丸
\\	Ⅰ【形】① 簡単な、単純な
\\	④(切符などが)片道の 
\\	(駅の切符売場で)ベルリンまで片道をください! Ⅱ【副】ただ、とにかく 
\\	君はとにかくすばらしい
\\	入場券
\\	【複】両親 
\\	1①〔複数なし〕食事
\\	〔過去〕
\\	〔過分〕
\\	〔接Ⅱ〕
\\	1【自】〔完了
\\	①(乗り物で)行く 2【他】②〔乗り物⁴を〕運転する
\\	1【自】①《3格》
\\	に〕欠けている、足りない、
\\	見当たらない 
\\	彼にはその勇気がない。 ③《3格》
\\	の〕(体の)具合が悪い 
\\	どこの具合が悪いのですか? 2【非人称】
\\	にとって…
\\	が〕不足している
\\	①(会社などの)終業 
\\	(1日の)仕事を終える
\\	(単2)‐
\\	/(複1・2・4)‐ (複3)‐
\\	窓
\\	①(病気の)熱
\\	映画
\\	Ⅰ【副】① 以前  Ⅱ【形】② 以前の、かつての、昔の 
\\	(単2)‐
\\	/(複1・2・4)
\\	(複3)
\\	足 
\\	③(山の)ふもと
\\	誕生日 
\\	手荷物、(旅行の)荷物
\\	【副】きのう、昨日
\\	1①(温度の)度(〔記号〕゚ 
\\	③(緯度・経度の)度
\\	1【自】① 結婚する
\\	〔過去〕
\\	〔過分〕
\\	〔接Ⅱ〕
\\	【自】①《3格》
\\	を〕手伝う、助ける
\\	3と〗
\\	彼は私の引っ越し〈洗濯〉の手伝いをしてくれた 
\\	(単
\\	/(複1・2・4)‐
\\	(複
\\	犬
\\	/〔女性〕
\\	②⦅俗語⦆(特に男について)人、やつ
\\	空腹、飢え
\\	私は腹ぺこだ。
\\	技師、エンジニア(〔略〕
\\	/〔女性〕‐
\\	発音:
\\	①〔…⁴を〕買う 
\\	〔過去〕
\\	〔過分〕
\\	〔接Ⅱ〕
\\	【他】①〔…⁴を〕知っている
\\	子供 
\\	(単2)‐
\\	/(複1・2・4)‐
\\	(複3)‐
\\	ワンピース、ドレス②〔複数で〕衣服、服
\\	1【他】①〔…⁴を〕煮る、ゆでる 
\\	卵をゆでる 2【自】① 料理する 
\\	(単2)‐
\\	/(複1・2・4)‐ (複3)‐
\\	スーツケース、トランク
\\	①(顔も含めた)頭
\\	①〔…⁴を〕もらう、受け取る
\\	【副】① 長い間
\\	〔現在〕
\\	〔過去〕
\\	〔過分〕
\\	〔接Ⅱ〕
\\	1【自】〔完了
\\	走る
\\	歩く ④(機械などが)動いている ⑦(事が)進行する、進行中である ⑧(映画・番組が)上映される 
\\	映画館では何をやっていますか?
\\	解答 ④⦅化学⦆ 溶解;溶液
\\	弱変化;(単2・3・4)‐
\\	/(複1・2・3・4)‐
\\	人間
\\	③(呼ぶかけで)おい;(驚きを表わして)うわー
\\	〔過去〕
\\	〔過分〕
\\	①〔…⁴を〕持って行く 
\\	その本を旅行に持って行く ②〔…⁴を〕連れて行く
\\	〔過去〕
\\	〔過分〕
\\	〔接Ⅱ〕
\\	1【他】①
\\	〔…⁴を…⁴と〕名づける
\\	〔…⁴を…⁴という名で〕呼ぶ ④〔名・理由など⁴を〕挙げる、言う 
\\	方向》〔…⁴を…に〕詰める
\\	③〔…⁴を〕ぎゅっとつかむ 
\\	彼女は彼の手をつかんだ ⑥⦅口語⦆〔仕事など⁴を〕うまくやってのける
\\	①〔複数なし〕紙 
\\	書類 発音:
\\	【自】〔完了〕
\\	旅をする、旅行する 
\\	指輪
\\	⑤(ボクシングの)リング
\\	【他】①
\\	に…⁴を〕贈る、プレゼントする
\\	(単2)‐
\\	/(複1・2・4)
\\	(複3)
\\	①〔ふつう単数で〕終了、終わり
\\	秒(時間の単位)(〔略〕
\\	/〔記号〕
\\	(単2)‐
\\	/(複1・2・4)
\\	(複3)
\\	息子
\\	鏡 
\\	(単2)‐ 
\\	(複1・2・4)
\\	(複3)
\\	娘 
\\	ドア
\\	〔過去〕
\\	〔過分〕
\\	〔接Ⅱ〕
\\	1【他】①〔持ち物⁴を〕なくす
\\	②〔…⁴を〕失う ③〔…⁴を〕見失う ④〔試合など⁴に〕負ける
\\	(単2)‐/(複1・2・4)
\\	(複3)
\\	壁
\\	〔過去〕
\\	〔過分〕
\\	【自】①(体の一部が)痛い、痛む 
\\	私は頭が痛い。 ②《3格》
\\	に〕痛い思いをさせる、
\\	を〕傷つける
\\	夕食 
\\	Ⅰ【形】〖述語的に〗①(他の人がいなくて)一人きりで 
\\	〔過去〕
\\	〔過分〕
\\	1【他】①〔衣服⁴を〕着る、〔靴・ズボン⁴を〕はく
\\	〖3格と〗
\\	子供に靴下をはかせる ④〔…⁴の心を〕引きつける 2【再】
\\	服を着る
\\	(単2)‐
\\	/(複1・2・4)
\\	(複3)
\\	リンゴ 
\\	腕 
\\	貧しい,貧乏な
\\	目 
\\	とっとと消え失せろ!
\\	【他】①〔火・電灯など⁴を〕消す;⦅口語⦆〔…⁴の〕スイッチを切る 
\\	②〔…⁴を〕取り決める ③〔…⁴の〕決着をつける
\\	【形】① 現金の 
\\	現金 
\\	現金でお支払いになりますか?
\\	【他】①〔…⁴を〕意味する 
\\	㊟受動文を作らない) 
\\	などと〗(…の)価値を持つ 【3格と】
\\	この本は私にとって非常に重要だ〈それほど重要ではない〉
\\	〔過去〕
\\	〔過分〕
\\	〔接Ⅱ〕
\\	1【自】① 始まる 
\\	【形】①〖複数で;冠詞や代名詞を伴って〗二人の、二つの 
\\	②〖複数で;冠詞なしで文アクセントが置かれて〗両方とも、二つとも 
\\	山 
\\	【他】①〔…⁴を〕訪問する
\\	③〔学校など⁴に〕通う 
\\	ビール 
\\	【形】① 青い、青色の 
\\	③⦅俗語⦆ひどく酔った
\\	パン 
\\	オフィス、事務所 
\\	バター 
\\	1【他】①〔…⁴を〕押す
\\	④〔…⁴を〕(物理的・精神的に)圧迫する 
\\	この靴は私にはきつい。
\\	1【自】〔完了
\\	シャワーを浴びる
\\	【形】① 急いでいる 
\\	招待、誘い 
\\	〔過去〕
\\	〔過分〕
\\	〔接Ⅱ〕
\\	1【他】①
\\	に…⁴を〕勧める
\\	1【他】①〔製品・方法など⁴を〕開発する 
\\	②〔…⁴を〕養成する、育成する 2【再】①
\\	発展する、成長する 
\\	3格と〗
\\	彼は一人前の男になった 
\\	生じる、発生する
\\	【形】②⦅口語⦆すばらしい、すてきな
\\	色
\\	絵の具;ペンキ
\\	1【他】①〔…⁴を〕祝う 
\\	②〔パーティーなど⁴を〕催す 2【自】① パーティーをする
\\	祝日 
\\	【形】① できあがった 
\\	用意のできた 
\\	終わった、済ませた
\\	祭り
\\	火 
\\	魚 
\\	瓶 
\\	①(体の一部としての)肉 
\\	食肉
\\	①〖無冠詞で〗フランス語 
\\	発音:
\\	万年筆 
\\	誕生 
\\	野菜 
\\	飲み物 
\\	山頂、頂上 
\\	グラス〔単位としては(複1・2・3・4)
\\	ミルク1杯
\\	大きさ、(服などの)サイズ
\\	【形】① 緑色の 
\\	髪の毛、髪 
\\	【副】後ろに;奥に
\\	(単2)‐
\\	/(複1・2・3・4)‐
\\	趣味
\\	1【他】①〔…⁴が〕(自然に)聞こえる 
\\	②〔…⁴を〕(聞こうとして)聞く 2【自】④
\\	〔…⁴の〕言うことを聞く
\\	カメラ 
\\	チーズ
\\	猫
\\	幼稚園
\\	弱変化;(単2・3・4)
\\	/(複1・2・3・4)
\\	(役所・会社・学校などの)同僚、仲間
\\	/〔女性〕
\\	〔過去〕
\\	〔過分〕
\\	〔接Ⅱ〕
\\	【自】〔完了
\\	来る
\\	②(話し相手またはお互いの関心が向いている方向へ)行く 
\\	〔…⁴を〕思いつく、思い出す
\\	(銀行の)口座 (〔略〕
\\	【他】①〔…⁴の〕値段である
\\	②〔時間・労力⁴が〕かかる ⑤〔…⁴の〕味をみる 【4格なしで】
\\	ちょっと味見をしてみるかい? 
\\	ネクタイ 
\\	発音:
\\	ケーキ 
\\	冷蔵庫 
\\	1【再】①
\\	〔…⁴の〕めんどうを見る、世話をする 
\\	〔…⁴を〕気にかける、かまう 2【他】〔…⁴に〕かかわりがある 
\\	それが私になんのかかわりがあるっていうんだ?;そんなことどうでもいい
\\	(単2)‐
\\	/(複1・2・3・4)
\\	店 
\\	Ⅰ【形】② 騒々しい、うるさい
\\	Ⅱ【前】〖2格または3格支配〗…によれば、…に従って ([略]
\\	〔過去〕
\\	〔過分〕
\\	〔接Ⅱ〕
\\	【他】①
\\	に…⁴を〕(ふつうただで)貸す
\\	〔…⁴を〕(ふつうただで)借りる 
\\	【形】①(音・声が)小さい;静かな 
\\	【付加語的に】かすかな、わずかな
\\	①〔複数なし〕光、明かり 
\\	電灯
\\	Ⅰ【副】① 左に、左側に
\\	ミルク、牛乳 
\\	分(時間の単位)(〔略〕
\\	/〔記号〕
\\	〔過去〕
\\	〔過分〕
\\	【他】①〔…⁴を〕持って来る、持参する ②〔…⁴を〕連れて来る
\\	1【他】①〔服など⁴を〕縫う 
\\	(単2)‐
\\	/(単3・4)‐
\\	(複1・2・3・4)‐
\\	名前 
\\	【形】① ふつうの、通常の
\\	〖集合的に〗果物 
\\	公園 
\\	Ⅰ【形】② 実用的な
\\	①(大学の)教授 (〔略〕
\\	/〔女性〕‐
\\	発音:
\\	修理 発音:
\\	〔過去〕
\\	〔過分〕
\\	〔接Ⅱ〕
\\	1【自】①〔…な〕においがする 
\\	3と〗
\\	彼女の髪はタバコのにおいがする。 2【他】①〔…⁴の〕においをかぐ、においがする 
\\	ガスのにおいがする
\\	〔比較〕
\\	〔最上〕
\\	【形】① 赤い 
\\	(単2)‐
\\	/(複1・2・4)
\\	(複3)
\\	ジュース 
\\	①〔複数なし〕塩、食塩 
\\	①(豚のもも肉の)ハム
\\	1【自】①〔…な〕味がする
\\	これはおいしい。 ②〔3格〕
\\	にとって〕おいしい
\\	(単2)‐
\\	/(複1・2・4)
\\	(複3)
\\	戸棚
\\	作家 
\\	/〔女性〕‐
\\	〔過去〕
\\	〔過分〕
\\	〔接Ⅱ〕
\\	1【自】①〔〔完了〕
\\	または
\\	泳ぐ 
\\	④〔完了
\\	に〕浸っている
\\	【副】① そのほかに 
\\	さもなければ;そうしないと 
\\	ふだんは;いつもは 
\\	①〔…⁴を〕寄付する 
\\	②〔…⁴を〕与える
\\	道路、通り (〔略〕
\\	おば、叔母 
\\	1【自】① 踊る
\\	①(飲み物としての)茶、(特に)紅茶 
\\	〔現在〕
\\	〔過去〕
\\	〔過分〕
\\	〔接Ⅱ〕
\\	1【他】①〔…⁴を〕運ぶ 
\\	②〔衣服・装飾品など⁴を〕身につけている 
\\	帽子をかぶっている
\\	〔過去〕
\\	〔過分〕
\\	【他】①〔お金⁴を〕振り込む
\\	(単2)‐
\\	/(複1・2・4)
\\	(複3)
\\	父 
\\	①(男性の)店員(〔女性〕‐
\\	〔過去〕
\\	〔過分〕
\\	1【他】①〔…⁴を〕先に延ばす、延期する
\\	②〔…⁴の位置を〕ずらす 
\\	さまざまな、いろいろな
\\	〔過去〕
\\	〔過分〕
\\	〔接Ⅱ〕
\\	1【他】①〔…⁴を〕理解する 
\\	②〔…⁴が〕聞き取れる、〔…⁴の〕言うことが聞き取れる 2【再】①
\\	自明である、当然である 
\\	それは自明のことだ
\\	講義 
\\	【自】〔完了
\\	ハイキングをする 
\\	②《方向/場所》〔…を〕(特に目的もなく)歩き回る;動き回る ③《方向/場所》(その場にとどまらずに)〔…を〕移動する、移る
\\	〔現在〕
\\	〔過去〕
\\	〔過分〕
\\	〔接Ⅱ〕
\\	1【他】①〔…⁴を〕洗う、洗濯する 
\\	洗濯機
\\	Ⅰ【形】白い 
\\	Ⅰ【形】②(空間的に)遠い
\\	Ⅰ【副】① かなり 
\\	Ⅱ【形】〖付加語的に〗⦅口語⦆ かなりの、相当の
\\	【副】現在、目下のところ(〔略〕
\\	発音:
\\	①〔ふつう単数で〕出発 
\\	〔現在〕
\\	〔過去〕
\\	〔過分〕
\\	1【他】①〔…⁴を〕手渡す 
\\	に…⁴を〕分け与える
\\	【他】①〔…⁴を〕迎えに行く 
\\	娘を学校まで迎えに行く ②〔…⁴を〕取りに行く;取って来る
\\	注意 
\\	(場内放送で)皆様、重要なお知らせがあります!
\\	【接】①〖定動詞後置〗…したときに 
\\	②〖比較級と〗…より 
\\	…として
\\	年齢 
\\	高齢
\\	【形】①(…ではない)別の、ほかの 
\\	(単2)‐
\\	/(複1・2・4)
\\	(複3)
\\	①〔ふつう単数で〕始め 
\\	年の始めに 
\\	…⁴を始める;始まる
\\	【他】〔…⁴に〕×印を付ける 
\\	①(入学・参加などの)申し込み;届け出 [派生] 
\\	…⁴を申し込む
\\	(単2)‐
\\	(複1・2・4)..
\\	(複3)..
\\	スーツ 
\\	1【自】〔完了
\\	終わる 
\\	を〕やめる 2【他】
\\	不定詞句》〔…するのを〕やめる 
\\	彼女は歌うのをやめなかった
\\	(単2)‐
\\	/(複1・2・4).. 
\\	(複3).. 
\\	出口 
\\	情報、案内 
\\	【副】(それだけではなく)そのうえ、更に 
\\	〔過去〕
\\	〔過分〕
\\	【自】〔完了〕
\\	①(乗り物から)降りる 
\\	アウトバーン、高速道路 
\\	自動販売機 
\\	(単2)‐
\\	(複1・2・4)
\\	(複3)
\\	木 
\\	〔形容詞変化〕知人、知り合い 
\\	絵 
\\	写真 
\\	⑤〔ふつう単数で〕イメージ
\\	【形】① 褐色の、茶色の 
\\	郵便切手 
\\	(単2)‐
\\	/(複1・2・4)..
\\	(複3)..
\\	連邦州(連邦を構成する州)
\\	ご婦人、女性 
\\	③(トランプ・チェスの)クイーン
\\	【副】1① そのあとで 発音:
\\	(単2)‐
\\	/(複1・2・3・4)
\\	日付 
\\	Ⅰ【接】①《並列》というのは…だから
\\	Ⅱ【副】④〖補足疑問文で;文アクセントが置かれて〗それでは 
\\	そんなことやりたくないね。-それじゃ何をやりたいんだ?
\\	Ⅱ【副】①〖前置詞句と〗(空間的・時間的に)すぐ 
\\	(単2)‐
\\	/(複1・2・4)
\\	(複3)
\\	村 
\\	【数】《序数》〔形容詞変化〕第3の、3番目の 
\\	/基数
\\	【名詞化して】
\\	第三者
\\	入ること、入場 
\\	【自】①(時間的・空間的に)終わる 
\\	〔過去〕
\\	〔過分〕
\\	〔接Ⅱ〕
\\	1【他】①〔…⁴を〕(はっきりと)見分ける、識別する ②(特徴から)〔…⁴だと〕わかる 
\\	3と〗
\\	声で君だとわかる ③《4格》〔誤りなど⁴に〕気づく
\\	1【他】①〔…⁴を〕(補って)完全なものにする;〔…⁴の〕不足分を補う 
\\	〔形容詞変化〕大人、成人 
\\	家族 
\\	【副】ほとんど 
\\	誤り、間違い 
\\	(単2)‐/(複1・2・3・4)
\\	会社 (〔略〕
\\	〔過去〕
\\	〔過分〕
\\	〔接Ⅱ〕
\\	1【自】〔完了
\\	②(鳥・昆虫・飛行機などが)飛ぶ
\\	飛行機、航空機 
\\	写真 
\\	⦅文法⦆疑問詞
\\	Ⅱ【副】⦅口語⦆ ひどく、非常に 
\\	ひどく怒る
\\	の過去分詞)【形】①《〔場所〕/〔時期〕》〔…で・に〕生まれた 
\\	②〖付加語的に〗旧姓…の (〔略〕
\\	マリアブラウン、旧姓シュミット
\\	出生地
\\	地方;地域 
\\	1【自】①《3格》
\\	の〕ものである 
\\	その本は私のものだ。 
\\	の〕一部〈一員〉である、
\\	に〕属している 2【再】
\\	ふさわしい、適当である 
\\	【副】まっすぐに 
\\	発音:
\\	①(男女の)性 
\\	会話 
\\	①〔複数なし〕重さ、重量 
\\	幸運 
\\	…とは幸運だ
\\	グループ、集団 
\\	携帯電話 
\\	(単2)‐
\\	(複1・2・4)
\\	(複3)
\\	①(建物としての)家 
\\	【他】①〔…⁴を〕行って持って来る 
\\	取りに行く ②〔…⁴を〕(電話などで急いで)呼び寄せる、〔…⁴に〕来てもらう 
\\	ホテル 発音:
\\	【数】《基数》① 100、百 
\\	【副】どこかで、どこかに 
\\	【形】① 寒い 
\\	冷たい ④(家賃が)暖房費抜きの 
\\	その部屋は暖房費抜きで700ユーロだ 
\\	キログラム (〔記号〕
\\	発音:
\\	〔比較〕
\\	〔最上〕
\\	【形】① 利口な、賢い 
\\	【他】①〔…⁴を〕訂正する、修正する 
\\	Ⅰ【形】①(速度が)遅い、ゆっくりとした 
\\	Ⅱ【副】だんだん、そろそろ 
\\	そろそろ君は学校へ行く時間だ 
\\	1【自】① 生きている、生きる 
\\	③〔…に〕住んでいる 
\\	彼は都会〈田舎〉に住んでいる 
\\	で〕生活する;
\\	を〕食べて生きていく 
\\	(または
\\	(別れのあいさつとして)ご機嫌よう!、さようなら!
\\	人生 
\\	生活、暮らし
\\	【形】独身の、未婚の (〔略〕
\\	1【他】①
\\	方向》〔…⁴を…に〕置く 
\\	2【再】①
\\	方向》〔…に〕横になる
\\	【他】①〔…⁴を〕愛する 
\\	[相互的に(4格)] 
\\	彼らは愛し合っている 
\\	【形】① 愉快な、陽気な 
\\	おもしろい
\\	女の子、少女 
\\	【形】①(人・動物が)雄の、男の 
\\	1【他】①《4格》〔…⁴に〕印を付ける 
\\	【他】①〔…⁴に〕気づく 
\\	〔…⁴を〕覚えておく 
\\	(単2)‐
\\	/(複1・2・3・4)
\\	博物館、美術館 
\\	隣国 
\\	夜 
\\	〔比較〕
\\	〔最上〕
\\	Ⅰ【形】①(空間的に)近い 
\\	①(空間的に)近く、近所 
\\	近くに〈で〉
\\	【副】決して…ない 
\\	1【他】①〔…⁴を〕書き留める、メモする 
\\	【副】① 上に 
\\	1【他】①〔…⁴を〕(一定の順序に)整理する、並べる
\\	パーティー 
\\	(または
\\	パーティーに行く
\\	(単2)‐
\\	(複1・2・4)
\\	(複3)
\\	パスポート、旅券 
\\	峠、峠道
\\	休憩 
\\	⦅文法⦆複数[形] (〔略〕
\\	警察 
\\	発音:
\\	④(開業医の)診療所
\\	市役所 
\\	1【他】〔タバコ・パイプなど⁴を〕吸う、吹かす 
\\	2【自】① タバコを吸う
\\	(単2)‐
\\	/(複1・2・4)
\\	(複3)
\\	部屋 
\\	④〔複数なし〕空間
\\	旅、旅行 
\\	(単2)‐
\\	/(複1・2・4)
\\	(複3)
\\	スカート 
\\	Ⅰ【形】① 静かな 
\\	静かな部屋 
\\	平穏な 
\\	落ち着いた、冷静な
\\	(単2)‐
\\	/(複1・2・4)
\\	(複3)
\\	文 
\\	一組、1セット
\\	表示板、標識
\\	1[非人称] 
\\	雪が降る 
\\	Ⅰ【形】① 美しい、きれいな 
\\	すばらしい
\\	【副】① すでに、もう 
\\	きっと、確かに ⑤〖命令文で;文アクセントなしで〗さっさと、いい加減に 
\\	の形で〕(ましてや…にいたっては)全然…でない
\\	〖定冠詞と〗⦅国名⦆スイス
\\	難しい、困難な 
\\	⦅国名⦆スペイン[王国]
\\	①〔複数なし〕スポーツ、運動 
\\	スポーツをする
\\	は間違い。) 
\\	言語、言葉 
\\	〔比較〕
\\	〔最上〕
\\	Ⅰ【形】①(体・意志などが)強い 
\\	濃い
\\	【形】毎日の (〔略〕
\\	バッグ、かばん 
\\	②(衣服の)ポケット
\\	劇場 
\\	発音:
\\	【形】① 死んでいる 
\\	階段 
\\	【自】〔…に〕宿泊する、泊まる
\\	練習 
\\	練習問題
\\	【副】① 下に 
\\	授業 
\\	⦅文法⦆動詞 
\\	1【他】①〔…⁴を〕消費する、使う 
\\	消費
\\	協会、クラブ 
\\	〔過去〕
\\	〔過分〕
\\	1【他】①〔…⁴を〕比較する、比べる 
\\	の過去分詞)【形】結婚している、既婚の 
\\	ビザ 
\\	発音:
\\	(姓に対し)名[前] 
\\	〔比較〕
\\	〔最上〕
\\	【形】① 暖かい 
\\	温かい
\\	【形】①(人間・動物が)雌の 
\\	①〔複数なし〕(地球全体を指して)世界 
\\	(または
\\	生まれる
\\	1【他】①〔発言⁴を〕繰り返す 
\\	③〔…⁴を〕復習する 
\\	(単2)‐
\\	(複1・2・4)
\\	(複3)
\\	語、単語 
\\	(単2)‐
\\	(複1・2・4)‐
\\	(複3)‐
\\	言葉
\\	に幸運・成功など⁴を〕祈る ③〔…⁴を〕願う、望む 
\\	数 (〔略〕
\\	(単2)‐
\\	/(複1・2・3・4)
\\	①(空間的な意味での)中心、中央 
\\	(紙巻きの)タバコ 
\\	発音:
\\	(単2)‐
\\	(複1・2・4)
\\	(複3)
\\	関税 
\\	(または
\\	関税を払う
\\	【副】① 最初に 
\\	発音:
\\	【自】(注意深く)聴く、耳を傾ける 
\\	〖3格と〗
\\	先生の話〈ラジオ放送〉に耳を傾ける
\\	〔過去〕
\\	〔過分〕
\\	1【他】①〔…⁴を〕取り外す 
\\	帽子を脱ぐ 2【自】
\\	減る、減少する;和らぐ、衰える 
\\	体重が減る、やせる
\\	〔過去〕
\\	〔過分〕
\\	1【他】①《4格》〔…⁴に〕鍵を掛ける ②〔…⁴を〕締結する 
\\	条約を締結する;契約を結ぶ ③〔…⁴を〕完了する 
\\	【形】心地よい、快適な 
\\	(贈り物などに対する)うれしい驚き
\\	(一定の)数 
\\	①〔複数なし〕やり方 
\\	種類
\\	〔過去〕
\\	〔過分〕
\\	1【他】①〔…⁴を〕あきらめる、断念する 
\\	1【他】〔…⁴を〕かたづける、整頓する 2【自】①
\\	を〕取り除く
\\	〔過去〕
\\	〔過分〕
\\	【自】〔完了
\\	①(特に夕方、楽しみのために)外出する、出かける 
\\	①〔ふつう単数で〕発音 
\\	バナナ
\\	【他】《4格》〔…⁴に〕答える 
\\	報告、報告書 
\\	…⁴を報告する
\\	〔過去〕
\\	〔過分〕
\\	【他】①《4格》〔山など⁴に〕登る;〔馬など⁴に〕またがる 
\\	【他】①〔座席・旅行など⁴を〕(客が)予約する 
\\	〔過去〕
\\	〔過分〕
\\	〔接Ⅱ〕
\\	1【再】①
\\	決める、決断する 
\\	ペーターはジーンズを買うことに決める。 2【他】〔…⁴を〕決定する
\\	1【他】①〔要求など⁴を〕満たす 2【再】
\\	(希望などが)かなう、実現する
\\	専門家 
\\	/〔女性〕
\\	(単2)‐
\\	/(複1・2・4)
\\	(複3)
\\	フライト 
\\	余暇、暇
\\	外国語 
\\	機会 
\\	機会があれば
\\	【形】① 天才の 発音:
\\	料理 
\\	①〔複数なし〕味、風味 
\\	②〔複数なし〕味覚 
\\	好み、趣味 
\\	それは私の好みではない。
\\	〔比較〕
\\	〔最上〕
\\	【形】① 健康な 
\\	健康に役立つ、体にいい
\\	習慣
\\	①〔複数なし〕文法 
\\	発音:
\\	Ⅰ【形】徹底的な
\\	1【他】②《4格》〔…⁴に〕あいさつする 
\\	彼は私にていねいにあいさつした。
\\	【他】〔…⁴を〕憎む;嫌悪する、嫌がる 
\\	【形】たびたびの、しばしばの、頻繁の 
\\	【副詞的に】
\\	彼女はしばしば遅れて来る。
\\	故郷
\\	【形】礼儀正しい、丁寧な 
\\	(単2)‐
\\	(複1・2・4)
\\	(複3)
\\	1①(縁のある)帽子 
\\	2(単
\\	/複数なし 
\\	用心 
\\	3格 
\\	⦅文語⦆ 
\\	に用心する
\\	地下室、地下貯蔵室 
\\	【他】①〔…⁴と〕知り合う
\\	【複】費用 
\\	(単
\\	/(複1・2・4)-
\\	(複
\\	針路;コース 
\\	講習、講座
\\	①〔複数なし〕(何かしたい)気、気持ち
\\	メロディー 
\\	発音:
\\	大量、多数 
\\	彼はお金をたくさん持っている。 
\\	数量 ③〔複数なし〕群衆
\\	朝 
\\	毎朝 
\\	①〔ふつう複数で〕覚え書き、メモ 
\\	メモを取る
\\	Ⅰ【形】① きちんとした、整頓された 
\\	きちょうめんな
\\	義務、責務 
\\	[派生] 
\\	…⁴に…
\\	を義務づける
\\	郵便番号 (〔略〕
\\	華やかな 
\\	すばらしい、みごとな、りっぱな 
\\	【自】①《3格》
\\	を〕害する、損なう、傷つける 
\\	喫煙は健康に悪い。 ②(ある事が)ためにならない、害になる (㊟否定文で) 
\\	彼に聞いてみても悪くはあるまい
\\	スイッチ 
\\	②(銀行・郵便局などの)窓口
\\	(食事用の)ナプキン 
\\	発音:
\\	会議 
\\	散歩 
\\	散歩をする
\\	(単
\\	/(複1・2・4)‐
\\	(複
\\	星 
\\	浜辺、浜 
\\	Ⅰ【形】① 南の 
\\	【形】好感の持てる、感じの良い 発音:
\\	カップ 
\\	1杯のコーヒー (㊟
\\	と
\\	は同格) 
\\	【自】電話で話をする 
\\	彼女は今電話中だ
\\	(単2)‐
\\	(複1・2・4)
\\	(複3)
\\	夢 
\\	2【他】①〔…⁴を〕練習する 
\\	彼は毎日ギターのけいこをする
\\	Ⅰ【他】〔…⁴を〕よく考える、熟慮する 
\\	Ⅱ【形】① 優れている、優勢である 
\\	【他】
\\	〔…⁴を…
\\	するように〕説得する 
\\	大学 (〔略〕
\\	発音:
\\	【形】① 不可能な 
\\	〔現在〕
\\	〔過去〕
\\	〔過分〕
\\	〔接Ⅱ〕
\\	1【他】①〔…⁴を〕忘れる 
\\	〔過去〕
\\	〔過分〕
\\	〔接Ⅱ〕
\\	Ⅰ1【他】①〔場所⁴を〕去る 
\\	②〔…⁴のもとを〕去る、出る、離れる ③〔…⁴を〕見捨てる 2【再】
\\	〔…⁴を〕頼りにする、あてにする Ⅱ【形】① 人のいない
\\	〔過去〕
\\	〔過分〕
\\	〔接Ⅱ〕
\\	【他】〔困難・面倒など⁴を〕避ける、回避する 
\\	〔過去〕
\\	〔過分〕
\\	【他】〔…⁴を〕黙っている、秘密にする [派生] 
\\	黙っている,黙る
\\	約束 
\\	1【他】①〔…⁴を〕試みる、試す 
\\	アンケは日本にいる兄に電話しようと試みる。
\\	1【他】〔夢・計画など⁴を〕実現する 
\\	1【他】①〔…⁴の〕準備をする、用意をする 
\\	2【再】①
\\	〔…⁴に対する〕準備をする;心構えをする 
\\	用意のできた
\\	〔過去〕
\\	〔過分〕
\\	【他】〔…⁴を〕予定している、するつもりである 
\\	アンケは大学卒業後に旅行をするつもりだ。
\\	提案 
\\	…⁴を提案する
\\	(単2)‐
\\	(複1・2・4)
\\	(複3)
\\	願望、望み、願い 
\\	1【他】①
\\	に…⁴を〕見せる 
\\	に道・方法など⁴を〕教える 2【自】〔方向〕〔…を〕指し示す、指す
\\	砂糖 
\\	〔比較〕
\\	〔最上〕
\\	㊟格語尾が付く場合は 
\\	【形】① 暗い 
\\	あいまいな、はっきりしない
\\	空 
\\	天国
\\	①(交通の)信号機
\\	1【他】①〔…⁴を〕変える、変更する 
\\	それは変えようがない;それはどうしようもない 2【再】
\\	変わる、変化する
\\	(単2)‐/(複1・2・4)
\\	(複3)
\\	不安
\\	【副】暗記して 
\\	例文を暗記する
\\	建築、建設 
\\	2(単
\\	/(複1・2・3・4)
\\	(大きな)建物
\\	〔過去〕
\\	〔過分〕
\\	〔接Ⅱ〕
\\	1【再】①
\\	場所》〔…に〕ある、いる
\\	1【他】〔…⁴を〕報告する 
\\	2【自】
\\	〔…4・3について〕報告する 
\\	記述、描写 
\\	【形】① 特別な、格別の 
\\	【他】①〔…⁴を〕調達する ②〔…⁴を〕済ます、処理する
\\	1【他】①〔…⁴を〕動かす 
\\	風が木の葉を揺らす。
\\	(単2)‐
\\	/(複1・2・4)
\\	(複3)
\\	1①〔ふつう複数で〕物 
\\	②〔複数で〕事、事柄 
\\	⦅口語⦆ 悪事を働く
\\	【再】①
\\	休養する 
\\	休暇中に十分休養する
\\	【他】①
\\	に…⁴を〕許す、許可する 
\\	許可 
\\	【他】①〔…⁴の〕到来を待つ、待ち望む 
\\	ペーターは恋人からの電話を待っている。 ②〔…⁴を〕予期する、期待する 
\\	3と〗
\\	君が私たちを手伝ってくれることを期待している
\\	【自】①〔完了
\\	《3格》
\\	の〕あとについて行く;
\\	の〕あとを追う 
\\	④《3格》
\\	に〕従う、ならう;応じる
\\	【形】① 喜んでいる、うれしい 
\\	明けましておめでとう!
\\	Ⅰ【副】① 完全に、まったく ③〖文アクセントなしで〗まあまあ
\\	〔過去〕
\\	〔過分〕
\\	①《2格》⦅文語⦆
\\	のことを〕思い出す,しのぶ 
\\	戦争犠牲者をしのぶ 
\\	不定詞句》〔…する〕つもりである 
\\	詩 
\\	【形】互いの 
\\	現在;現代 
\\	③⦅文法⦆ 現在形 
\\	〔現在〕
\\	〔過去〕
\\	〔過分〕
\\	〔接Ⅱ〕
\\	1【自】①(証明書・規則などが)効力がある、有効である、通用する 
\\	4格と】
\\	その法律はすべての市民に適用される 
\\	〔…と〕見なされている 3[非人称]①
\\	不定詞句》〔…することが〕重要である 
\\	①〔複数なし〕歴史 
\\	物語 ③⦅口語⦆ 出来事;事柄 
\\	速度、スピード 
\\	顔 
\\	〖集合的に〗穀物 
\\	〔過去〕
\\	〔過分〕
\\	〔接Ⅱ〕
\\	1【他】①《4格》〔…⁴に〕勝つ 
\\	③〔…⁴を〕(努力して)獲得する
\\	【形】① 寛大な 
\\	気前のいい
\\	〔過去〕
\\	〔過分〕
\\	〔接Ⅱ〕
\\	1【他】③〔…⁴を〕保つ 
\\	状態》〔…⁴を…に〕保つ 
\\	〔…⁴を…と〕思う2【自】①止まる、停車する
\\	1【自】①
\\	の〕商売〈売買〉をする 
\\	〔題材など3・4を〕取り扱う 
\\	その本は何をテーマにしているのですか? 
\\	行動する、ふるまう 2【他】〔商品・株など⁴を〕(市場で)売る、売買〈取り引き〉する (㊟ふつう受動文で) 
\\	4格 …⁴が問題である、重要である(㊟
\\	は形式主語)
\\	…さん (〔略〕
\\	単数3・4格
\\	主人、支配者;領主、君主 (〔女性〕
\\	【形】① こっけいな 
\\	②⦅口語⦆奇妙な、変な
\\	読者 
\\	/〔女性〕‐
\\	革
\\	③〔問題など⁴を〕解く
\\	Ⅱ【副】①
\\	の比較級)より以上に ②〖否定を表わす語句と〗もう(…ない)、もはや(…ない)
\\	私はもう時間がない。
\\	音楽家 
\\	Ⅰ【副】① すなわち 
\\	というのは
\\	【代】《不定》何も…ない 
\\	何か見つけたの?-いいえ、なんにも。
\\	Ⅰ【副】① 今 
\\	今度は君の番だ。 
\\	さて、ところで
\\	【形】① プライベートな、私的な 
\\	個人的な 発音:
\\	①〔…⁴を〕試食〈試飲〉する  ②〔…⁴を〕試す
\\	(単2)‐
\\	/(複1・2・4)
\\	(複3)
\\	①〔複数なし〕助言 
\\	(単2)‐
\\	/(複1・2・4)..
\\	(複3)..
\\	弁護士 (〔女性〕..
\\	政府 
\\	統治、支配
\\	【形】② ロマンチックな 発音:
\\	〔過去〕
\\	〔過分〕
\\	〔接Ⅱ〕
\\	1【他】①〔…⁴を〕呼ぶ 
\\	タクシーを呼ぶ 2【自】②
\\	を〕(来るように)呼ぶ
\\	サラダ 
\\	〔現在〕
\\	〔過去〕
\\	〔過分〕
\\	〔接Ⅱ〕
\\	1【自】〔完了
\\	(熱により)溶ける 
\\	雪 
\\	状況
\\	心配 
\\	(単2)‐
\\	/(複1・2・4)
\\	(複3)
\\	②〔複数なし〕楽しさ、楽しみ 
\\	(遊びに行く人に対して)おおいに楽しんでおいで!
\\	ことわざ、格言 
\\	(単2)‐
\\	/(複1・2・4)
\\	(複3)
\\	幹 
\\	〔過去〕
\\	〔過分〕
\\	〔接Ⅱ〕
\\	【自】〔完了
\\	①〔…に〕登る 
\\	ある山に登る 
\\	②〔方向〕〔…から〕下りる、降りる、〔…へ〕下りて行く
\\	日記[帳] 
\\	(単2)‐
\\	(複1・2・4)
\\	(複3)
\\	音 
\\	③〔複数なし〕口調、言葉の調子
\\	①(予期せぬことによる)驚き 
\\	予期せぬ出来事 発音:
\\	〔現在〕
\\	〔過去〕
\\	〔過分〕
\\	【再】①
\\	(周りを)見回す、見て回る 
\\	を〕探す
\\	【他】①〔…⁴を〕追う、追跡する 
\\	③〔…⁴を〕目で追う、注視する
\\	①〔複数なし〕過去 
\\	③⦅文法⦆過去時制
\\	の過去分詞)【形】① 絶望した、絶望的な 
\\	必死の、死にもの狂いの
\\	〔過去〕
\\	〔過分〕
\\	【他】①〔…⁴を〕提案する 
\\	①(ふつう舗装されていない)道 
\\	経路;道のり 
\\	方法、手段
\\	Ⅰ【副】① 更に先へ 
\\	先へ進め!;(話を)続けろ! Ⅱ【形】
\\	の比較級)②【付加語的に】これ以上の、これ以外の
\\	【副】③〖文アクセントなしで〗おそらく、たぶん
\\	〔過去〕
\\	〔過分〕
\\	【自】〔完了〕
\\	帰って来る、(元の場所に)戻る 
\\	〔過去〕
\\	〔過分〕
\\	【自】〔完了
\\	後ろに下がる、後退する 
\\	辞める、辞任する
\\	【副】晩に、夕方に 
\\	部門 (〔略〕
\\	住所 (〔略〕
\\	発音:
\\	⦅文法⦆ 4格 発音:
\\	Ⅰ【形】不安そうな;臆病な 
\\	[派生] 
\\	不安;心配
\\	〔過去〕
\\	〔過分〕
\\	1【他】①《4格》〔…⁴に〕話しかける 
\\	②〔テーマ・問題など⁴に〕言及する ③〔…⁴に〕頼む 
\\	4と〗
\\	彼はその政治家にそのテーマについて意見を求めた
\\	【自】〔完了
\\	目覚める 
\\	真夜中に目を覚ます
\\	エレベーター 
\\	(単2)‐
\\	/(複1・2・4).. 
\\	(複3).. 
\\	ハイキング、遠足 
\\	山へハイキングに行く 
\\	【他】①〔…⁴を〕選び出す 
\\	彼は妻のために指輪を選ぶ 
\\	3と】
\\	好きなものを選んでいいですか?
\\	パン屋、パン職人 
\\	/〔女性〕‐
\\	鉄道
\\	バルコニー 
\\	発音:
\\	①〔複数なし〕サービス 
\\	[派生] 
\\	…⁴に応対する
\\	①《4格》〔…⁴に〕同行する 
\\	【自】①(犬などが)ほえる 
\\	1【再】①
\\	〔…⁴を得ようと〕努力する 
\\	銀行に勤め口を得ようと努力する 
\\	努力,骨折り
\\	【形】〖述語的に〗① 用意のできた
\\	をする〕心構えができた 
\\	私はどんなことでもする覚悟ができている。
\\	1【他】①〔…⁴を〕制限する、限定する 
\\	見学
\\	〔過去〕
\\	〔過分〕
\\	〔接Ⅱ〕
\\	【他】①〔…⁴を〕所有している 
\\	〔過去〕
\\	〔過分〕
\\	1【他】①〔…⁴について〕話し合う 
\\	舞台、ステージ 
\\	(単2)‐
\\	/(複1・2・4)
\\	(複3)
\\	屋根 
\\	〔過去〕
\\	〔過分〕
\\	〔接Ⅱ〕
\\	1【他】①〔…⁴と〕思う、考える 
\\	様態》〔…⁴を…だと〕思う、想像する 2【自】②
\\	〔…⁴のことを〕思う、考える;思い出す、覚えている
\\	【副】あそこへ、そこへ
\\	(単2)‐
\\	/(複1・2・4)..
\\	(複3)..
\\	②〔複数なし〕歓迎、出迎え 
\\	歓迎会、レセプション
\\	1階 (〔略〕
\\	の過去分詞)【形】喜んでいる 
\\	お会いできてうれしく思います!;初めまして!
\\	〔過去〕
\\	〔過分〕
\\	〔接Ⅱ〕
\\	Ⅰ1【他】(結果として)〔…⁴を〕もたらす 
\\	アンケートははっきりした選挙予測をもたらした 2【再】①
\\	(結果として)生じる 
\\	降伏する 
\\	(もたらす)
\\	1【他】①〔仕事など⁴を〕済ます、かたづける
\\	【他】〔…⁴を〕習得する
\\	Ⅰ【副】① 別に 
\\	暖房費をあなたは別に払わなければならない 
\\	余分に
\\	②〔複数なし〕
\\	不定詞句と〗(…する)能力 [派生] 
\\	をする能力がある
\\	廊下 
\\	玄関ホール
\\	1【他】②〔…⁴ではないかと〕心配する、懸念する (㊟ふつう
\\	文か
\\	不定詞句と)2【再】
\\	恐れる、怖がる 
\\	暗やみの中で怖がる
\\	フォーク 
\\	ガレージ、車庫 
\\	発音:
\\	(単2)‐
\\	/(複1・2・4)
\\	(複3)
\\	給料 
\\	【形】① 居心地のよい 
\\	心情、心
\\	【複】(男女合わせた)きょうだい
\\	【自】《3格》
\\	に〕お祝いを言う、おめでとうと言う 
\\	(単2)‐
\\	/(複1・2・3・4)
\\	港 
\\	〔過去〕
\\	〔過分〕
\\	〔接Ⅱ〕
\\	Ⅰ【自】①〔…に〕掛かっている 
\\	Ⅱ〔過去〕
\\	〔過分〕
\\	【他】①
\\	方向》〔…⁴を…に〕掛ける、つるす 
\\	1【他】①〔病気・傷⁴を〕治す 
\\	2【自】〔完了
\\	(病気・傷が)治る、治癒する 
\\	傷は早く治った 
\\	1【他】〔…⁴を〕望む 
\\	希望 
\\	【形】① 空腹の 
\\	季節 
\\	寒い〈暖かい〉季節
\\	(単2)‐
\\	/(複1・2・4)
\\	(複3)
\\	運河 
\\	発音:
\\	会議 
\\	発音:
\\	頭痛 
\\	情勢、状況 
\\	位置、場所
\\	スプーン 
\\	①〔複数なし〕不足、欠乏 
\\	②〔ふつう複数で〕欠陥、欠点
\\	ジャム 
\\	Ⅰ【他】①〔…⁴と〕思う 
\\	②〔…⁴のことを指して〕言っている;〔…⁴を〕意図〈意味〉する
\\	ナイフ 
\\	(単
\\	/(複1・2・4)- (複
\\	従業員 (〔女性〕‐
\\	従業員130人の企業 
\\	構成員;メンバー 
\\	可能性 
\\	【副】
\\	の最上級)① できる限り、可能な限り
\\	【形】毎月の、月々の 
\\	記念碑 発音:
\\	(単2)‐
\\	/(複1・2・4)
\\	(複3)
\\	①(人間の)口 
\\	世間の評判になっている
\\	〔過去〕
\\	〔過分〕
\\	1【他】①〔…⁴を〕(辞書などで)調べる 
\\	【形】① 裸の 
\\	神経質な 
\\	発音:
\\	北 (〔略〕
\\	【形】必要な 
\\	ペンション、民宿 発音:
\\	【他】①〔…⁴を〕計画する 
\\	[派生] 
\\	計画
\\	Ⅰ【副】① 突然 
\\	領収書 
\\	(単
\\	/(複1・2・4)‐ (複
\\	謎々 発音:
\\	1【自】① 計算する 
\\	を〕見込む、予期する
\\	1【自】① 足りる 
\\	②〔…に〕達する 2【他】
\\	に…⁴を〕手渡す、差し出す
\\	【自】〔完了
\\	失敗する 
\\	痛み 
\\	Ⅰ【形】① 恐ろしい 
\\	②⦅口語⦆ ひどい 
\\	【副】① それどころか 
\\	それは簡単ではない、それどころか非常に難しい。 
\\	…でさえ、…ですら
\\	1【他】①〔お金など⁴を〕蓄える、ためる 
\\	②〔…⁴を〕節約する 
\\	(相談のための)面会時間、診察時間
\\	国家、国 
\\	活動 
\\	①〔複数なし〕科学技術 
\\	テクニック、技術
\\	皿 
\\	トイレ 
\\	門 
\\	【他】①〔…⁴を〕翻訳する 
\\	【副】ところで、それはそうと、ちなみに
\\	Ⅰ【副】およそ、約
\\	〔現在〕
\\	〔過去〕
\\	〔過分〕
\\	①〔…⁴を〕(一時)中断する 
\\	交通 
\\	試み 
\\	実験 
\\	【副】① 通り過ぎて
\\	終わって 
\\	戦争は終わった。 発音:
\\	(単2)‐
\\	/(複1・2・4)..
\\	(複3)..
\\	①(厚地の)カーテン 
\\	真実 
\\	1【他】①〔…⁴を〕(別の物・新しい物と)取り替える、交換する、変える 
\\	②〔…⁴を〕交わす 
\\	友人と文通する〈言葉を交わす〉
\\	1【自】 泣く 
\\	曜日
\\	居間、リビングルーム 
\\	雲 
\\	傷 
\\	奇跡;驚嘆すべきもの〈人〉 
\\	奇跡を信じる
\\	1【他】①〔絵⁴を〕描く、〔図面など⁴を〕かく 
\\	肖像画を描く
\\	目的地 
\\	目標、目的
\\	【形】① 類似した、同じような 
\\	②《3格》
\\	に〕似ている 
\\	彼女は母親によく似ている。
\\	【形】① 礼儀正しい
\\	【形】目立つ
\\	〔現在〕
\\	〔過去〕
\\	〔過分〕
\\	【自】〔完了
\\	登場する;(映画・舞台などに)出演する 
\\	現われる;(突然)発生する、起こる 
\\	そのとき問題が生じた
\\	〔現在〕
\\	〔過去〕
\\	〔過分〕
\\	【自】〔完了
\\	①《場所/様態》〔…で・に〕成長する 
\\	赤ん坊、赤ちゃん 発音:
\\	②《4格》〔…⁴に〕注意を払う [派生] 
\\	〔…⁴に〕注意を払う
\\	【副】例えば 
\\	【形】職業の、職業上の 
\\	職業
\\	【他】
\\	〔…⁴を…
\\	から〕守る 
\\	(単2)‐
\\	/(複1・2・4)‐(複3)‐
\\	訪問者 
\\	/〔女性〕‐
\\	名称 
\\	…⁴を…と呼ぶ,言う
\\	【形】① カラフルな 
\\	【副】② それによって 発音:
\\	【副】③ それに加えて 
\\	それに対して 発音:
\\	【形】① 唯一の、ただ一つの 
\\	①〖無冠詞で〗英語 
\\	1【他】①〔…⁴を〕染める 
\\	2【再】
\\	色づく、染まる
\\	テレビ 
\\	1【他】①〔…⁴を〕形作る 
\\	Ⅰ【副】①〖否定を表わす語句と〗全然(…ない) 
\\	それはまったく真実ではない
\\	天才 
\\	音楽の天才 発音:
\\	⦅文法⦆2格、属格 (〔略〕
\\	発音:
\\	①〔複数なし〕(キリスト教の)神 
\\	/〔女性〕
\\	主婦 
\\	指示、指摘 
\\	ヒント、手掛り、兆候
\\	(単2)‐
\\	(複1・2・4)
\\	(複3)
\\	中庭 
\\	宮廷、宮殿
\\	ズボン 
\\	(または
\\	⦅口語⦆ ひどく不安がっている
\\	弱変化;(単2・3・4)‐
\\	/(複1・2・3・4)‐
\\	男の子、少年 
\\	①(ドイツ・オーストリアの)首相
\\	/〔女性〕‐
\\	出世、(職業上の)成功 発音:
\\	1【自】①(とんとんと)たたく、ノックする 
\\	彼は私の背中をぽんとたたいた。 ②(心臓が)鼓動する、(脈が)打つ
\\	紛争、衝突 
\\	発音:
\\	【形】① 批判の;批判的な 
\\	②(局面などが)重大な、決定的な
\\	①〔政府・対策など⁴を〕批判する 
\\	好きな食べ物、[大]好物 
\\	【形】① 強大な、権力のある 
\\	力強い ④⦅口語⦆ ものすごい [派生] 
\\	力;権力 
\\	画家 
\\	/〔女性〕‐
\\	ペンキ屋
\\	(国家・州政府の)大臣 (〔女性〕‐
\\	①〔複数なし〕(天体の)月 
\\	⑤⦅文語⦆(暦の)月 
\\	国籍 
\\	発音:
\\	【副】隣に、隣接して 
\\	隣の家
\\	政党、党 
\\	④(アパートに入っている)世帯 発音:
\\	【形】① 完璧な 
\\	①(1人、2人と数える場合の)人 発音:
\\	弱変化;(単2・3・4)‐
\\	/(複1・2・3・4)‐
\\	哲学者 
\\	/〔女性〕‐
\\	物理学 
\\	発音:
\\	政治家 (〔女性〕‐
\\	[派生] 
\\	政治 発音:
\\	【形】① 政治の、政治的な 
\\	発音:
\\	〔過去〕
\\	〔過分〕
\\	〔接Ⅱ〕
\\	1【他】①
\\	に…⁴を〕助言する、忠告する 
\\	②〔…⁴を〕言い当てる、推測する
\\	③(パンなどの)スライス 
\\	1枚の薄切りのパン
\\	〔過去〕
\\	〔過分〕
\\	〔接Ⅱ〕
\\	1【自】① 叫ぶ 
\\	肩 
\\	Ⅰ【副】① めったに…ない 
\\	Ⅱ【形】珍しい
\\	(単2)‐
\\	/(複1・2・4)
\\	(複3)
\\	格言
\\	戦略 発音:
\\	環境 
\\	〔過去〕
\\	〔過分〕
\\	【他】①《4格》〔…⁴に〕下線〈アンダーライン〉を引く 
\\	財産
\\	【他】①〔機会など⁴を〕逃す、逸する;〔列車など⁴に〕乗り遅れる 
\\	②〔…⁴に〕会い損なう 
\\	⦅口語⦆
\\	にいやなもの⁴を〕与える
\\	【形】① いっぱいの、満ちた 
\\	利点、長所 
\\	【副】① 前へ、前方へ 
\\	1【他】①〔…⁴を〕選ぶ 
\\	②〔…⁴を〕選挙する
\\	【形】① 真実の、ほんとうの 
\\	【副】少なくとも 
\\	それは少なくとも4週間はかかる
\\	〔過去〕
\\	〔過分〕
\\	〔接Ⅱ〕
\\	1【他】①〔…⁴を〕引く、引っ張る 
\\	2【自】①〔完了
\\	《方向》〔…へ〕引っ越す、移る ②〔完了
\\	《方向》〔…へ〕(特に集団で)移動する
\\	〔過去〕
\\	〔過分〕
\\	【自】〔完了
\\	戻る、引き返す 
\\	〔…⁴にまで〕さかのぼる
\\	1【再】
\\	腹を立てる、怒る 2【他】〔…⁴を〕怒らせる
\\	著者 
\\	Ⅰ【副】① ただ…だけ 
\\	③〖命令文で〗とにかく Ⅱ【形】① ただの
\\	【副】それゆえに 
\\	①〔複数なし〕地球;世界 
\\	地球は1年に1回太陽の周りを回る ④〔ふつう単数で〕土 
\\	Ⅰ【代】《不定》① あるもの、あること;何か 
\\	Ⅱ【副】少し、いくらか 
\\	①(鳥の)翼 
\\	1【自】ささやく、小声で話す 
\\	木の葉がさらさら鳴る
\\	建物 
\\	(単2)‐
\\	/(複1・2・4)
\\	(複3)
\\	①〔複数なし〕使用;使い方 
\\	基礎 
\\	(単2)‐ 
\\	(複1・2・4)
\\	(複3)
\\	①〔複数なし〕皮膚、肌 
\\	(または
\\	⦅口語⦆ 何もしない;ぶらぶらしている
\\	【副】せいぜい 
\\	その本はせいぜい200ページしかない。
\\	②(話・本などの)内容 
\\	内容, 中身)
\\	円 
\\	仲間、サークル 
\\	範囲
\\	【形】① 生きている 
\\	生き物 
\\	生き生きとした、活気に満ちた 発音:
\\	おとぎ話、童話、昔話 
\\	弱変化;(単2・3・4)‐
\\	/(複1・2・3・4)‐
\\	隣人 
\\	鼻 
\\	⦅口語⦆ ずっとまっすぐに
\\	1【自】① うなずく 
\\	1【他】〔…⁴を〕(有効に)利用する、使う 2【自】役に立つ 【3格と】
\\	それは私の役に立たない
\\	おじ、叔父 
\\	弱変化;(単2・3・4)‐
\\	/(複1・2・3・4)‐
\\	警官 
\\	/〔女性〕‐
\\	[派生] 
\\	警察;警察署 発音:
\\	煙 
\\	[派生] 
\\	タバコを吸う 
\\	背、背中 
\\	砂 
\\	(単2)‐
\\	/(複1・2・4)
\\	(複3)
\\	尾、しっぽ 
\\	④⦅俗語⦆ ペニス 
\\	黒い
\\	①(物体の)面 
\\	ページ (〔略〕
\\	⑥(人・集団の)側、方
\\	種類、品種 
\\	①(鉛筆・ペンなどで引いた)線 
\\	線を引く
\\	(単2)‐
\\	/(複1・2・3・4)
\\	テーマ 
\\	理論 
\\	発音:
\\	【形】① 乾いた、乾燥した 
\\	⑥(酒が)辛口の 
\\	…⁴を乾かす;乾く
\\	準備 
\\	…⁴の準備をする,用意をする
\\	【形】用心深い、慎重な 
\\	気をつけて!
\\	ブドウ栽培
\\	【形】 ごく小さな 
\\	【形】① 威厳のある、品位のある;厳かな 
\\	ふさわしい、値する 〔2格〕
\\	に値する;ふさわしい
\\	【副】① 最後に 
\\	発音:
\\	舌 
\\	枝 
\\	弱変化;(単2・3・4)‐
\\	/(複1・2・3・4)‐
\\	農民 
\\	/〔女性〕
\\	【自】② せがむ
\\	あめ玉、キャンディー 
\\	発音:
\\	①⦅口語⦆ 汚物、ごみ 
\\	【他】①〔…⁴を〕尊敬する
\\	〔現在〕
\\	〔過去〕
\\	〔過分〕
\\	〔接Ⅱ〕
\\	1【他】①〔…⁴を〕受け取る;手に入れる 
\\	知らせ〈手紙〉を受け取る 
\\	〔現在〕
\\	〔過去〕
\\	〔過分〕
\\	〔接Ⅱ〕
\\	【自】〔完了
\\	落ちる 
\\	倒れる、転ぶ
\\	〔現在〕
\\	〔過去〕
\\	〔過分〕
\\	1【他】①(動物が)〔…⁴を〕食べる 
\\	喜び 
\\	〔過去〕
\\	〔過分〕
\\	1【自】① 寒がる;凍える 
\\	②〔完了
\\	凍る、凍結する 2【非人称】①《4格》〔…⁴が〕寒がる;凍える 
\\	私は寒い。
\\	〔過去〕
\\	〔過分〕
\\	〔接Ⅱ〕
\\	1【自】①(つかむために)〔…に〕手を伸ばす 
\\	2【他】①〔…⁴を〕つかむ
\\	【副】(中からこちらに向かって)外へ 
\\	発音:
\\	【副】①(外から向こうの)中へ 
\\	発音:
\\	島 
\\	(単2)‐
\\	/(複1・2・4)
\\	(複3)
\\	穴 
\\	①〔複数なし〕空気 
\\	(単2)‐
\\	/(複1・2・3・4)
\\	ストーブ 
\\	(単2)‐
\\	/(複1・2・4)‐
\\	(複3)‐
\\	馬
\\	①〔複数なし〕哲学 
\\	哲学者 発音:
\\	弱変化;(単2・3・4)‐
\\	/(複1・2・3・4)‐
\\	大統領 
\\	/〔女性〕‐
\\	発音:
\\	【形】(速度が)速い;(決断などが)すばやい
\\	1【自】① 話す、しゃべる 
\\	チョコレート 
\\	発音:
\\	エプロン 
\\	1【他】〔…⁴を〕振る 
\\	①(料理やデザートなどにかける)ソース 
\\	の現在分詞)【形】はらはらさせる、わくわくさせる 
\\	〔過去〕
\\	〔過分〕
\\	〔接Ⅱ〕
\\	1【自】①〔完了
\\	跳ぶ;跳ねる 
\\	方向》〔…⁴を…に〕差し込む、突っ込む 2〔過去〕
\\	(まれに
\\	〔過分〕
\\	【自】〔完了
\\	(ドイツ南部・オーストリア・スイスでは
\\	も)〕 ②⦅口語⦆〔…に〕いる、ある 
\\	彼はいったいどこにいるんだ? 
\\	①(意図的な)行為 
\\	トマト 
\\	発音:
\\	(単2)‐
\\	/(複1・2・4)
\\	(複3)
\\	森 
\\	〔過去〕
\\	〔過分〕
\\	1【他】①〔…⁴を〕書き写す ④⦅口語⦆〔…⁴を〕ないものとしてあきらめる
\\	労働者 
\\	/〔女性〕
\\	【形】① 怒った、腹を立てた 
\\	Ⅱ【副】よりによって 
\\	よりによってきょう病気になるなんて、大事な約束があるというのに!
\\	Ⅰ【前】〖3格支配〗①(除外)…を除いて、…以外
\\	1【他】①〔…⁴を〕主張する 
\\	2【再】①
\\	方向》〔…の方に〕身をかがめる
\\	【形】① 厚い 
\\	太い 
\\	太った
\\	Ⅰ【接】《並列》だが、しかし ※文と文を結ぶ。定動詞の位置に影響を与えない。
\\	2【再】①
\\	回る、回転する 
\\	〔比較〕
\\	〔最上〕
\\	【形】① ばかな、愚かな、無知な 
\\	愚かさ,無知
\\	〔現在〕
\\	〔過去〕
\\	〔過分〕
\\	【自】〔完了
\\	①《3格》
\\	の〕念頭にうかぶ、
\\	に〕思いつく 
\\	彼はいつも口実を思いつく。
\\	Ⅰ【形】⑥⦅口語⦆ すてきな、すばらしい
\\	(単2)‐
\\	/(複1・2・4)..
\\	(複3)..
\\	墓地
\\	反対、逆 
\\	(単2)‐
\\	/(複1・2・4)..
\\	(複3)..
\\	①〔複数なし〕歌うこと、歌唱;歌声
\\	弱変化;(単2・3・4)‐
\\	/(複1・2・3・4)‐
\\	ウサギ 
\\	/〔女性〕
\\	【副】回って 発音:
\\	〔過去〕
\\	〔過分〕
\\	〔接Ⅱ〕
\\	【自】①(鐘などが)鳴る、響く 
\\	③《様態》〔…のように〕聞こえる
\\	唇 
\\	弱変化;(単2・3・4)‐
\\	/(複1・2・3・4)‐
\\	ライオン 
\\	/〔女性〕
\\	〔現在〕
\\	〔過去〕
\\	〔過分〕
\\	〔接Ⅱ〕
\\	Ⅰ【他】①〔…⁴が〕好きだ
\\	【代】《不定》だれも…ない 
\\	【接】《従属》〖定動詞後置〗…にもかかわらず、…だけれども 
\\	(単2)‐
\\	/(複1・2・3・4)‐
\\	耳;聴覚 
\\	Ⅰ【数】《不定》①
\\	の形で〗二三の、いくつかの 
\\	権利 
\\	①〔複数で〕物 
\\	事、事柄
\\	1【他】①〔…⁴を〕収集する 
\\	③〔資源・情報・資金など⁴を〕集める 2【再】①
\\	(人が)集まる
\\	船 
\\	〔現在〕
\\	〔過去〕
\\	〔過分〕
\\	〔接Ⅱ〕
\\	1【他】①
\\	方向》〔…⁴を…へ・から〕投げる 
\\	【自】勝つ 
\\	⦅文法⦆単数[形](〔略〕
\\	Ⅰ【接】《並列》そうではなくて ※先行する否定に呼応して。定動詞の位置に影響を与えない。 
\\	【形】① 誇りを持った、誇らしげな 
\\	と〗
\\	彼は自分の息子を誇りにしている。
\\	①(全体に対する)部分;かけら ②(一つの)まとまり、塊り 
\\	(または
\\	チーズを塊りのまま買う
\\	①(表示・伝達のための)板、黒板
\\	【副】①〖否定を表わす語句と〗全然(…ない)、まったく(…ない) 
\\	〔疑問・非難の口調を強めて〕大体[からして]、いったいぜんたい、そもそも
\\	【形】残っている 
\\	ケーキはまだ少し残っている 
\\	普通の
\\	【自】
\\	を〕疑う 
\\	〔現在〕
\\	〔過去〕
\\	〔過分〕
\\	【他】①
\\	〔…⁴を〕(じっくり)見る;見物する 
\\	映画を見る ②〔…⁴を〕見る、眺める 
\\	〔比較〕
\\	〔最上〕
\\	【形】① 貧しい、貧乏な 
\\	かわいそうな、哀れな、不幸な 
\\	腕
\\	1【再】
\\	休息する、休養する
\\	(植物の)病害虫 
\\	【形】① 快適な、楽な 
\\	発音:
\\	地域、区域 
\\	領域、分野
\\	①〔ふつう単数で〕(川・道などの)幅 
\\	②〔複数なし〕⦅地理⦆緯度 
\\	(単2)‐/(複1・2・4)
\\	(複3)
\\	胸;肺 
\\	⦅地名・国名⦆中国 発音:
\\	【副】② その際に
\\	【副】それゆえに、それだから 
\\	(単2)‐
\\	/(複1・2・4)..
\\	(複3)..
\\	ばか者、愚か者 
\\	このばかが!
\\	【副】同じく、同様に
\\	1【他】①〔ラジオなど⁴の〕スイッチを入れる 
\\	氷 
\\	アイスクリーム
\\	【形】吐き気を催すような、不快きわまる;嫌でたまらない 
\\	〔現在〕
\\	〔過去〕
\\	〔過分〕
\\	②〔…⁴を〕迎える、歓迎する
\\	(単2)‐/(複1)‐
\\	⦅文法⦆語尾;接尾辞 
\\	発展 
\\	[派生語] 
\\	発展する
\\	(単2)..
\\	/(複1・2・4)..
\\	(複3)..
\\	出来事、事件 
\\	【副】① 約、ほぼ、およそ 
\\	③〖疑問文・条件文で〗(懸念・驚きなどを表わして)ひょっとして、場合によって[は]
\\	①〔複数なし〕存在 
\\	1【他】①〔…⁴を〕連れて行く、導く;案内する 
\\	②〔…⁴を〕運営〈経営〉する 
\\	店を経営する ⑫〖特定の名詞と〗(…を)行なう、する 
\\	3格 
\\	と話をする2【自】①
\\	の〕結果になる、
\\	を〕結果としてもたらす ②《方向》(道などが)〔…へ〕通じる、至る 
\\	地域、地帯 
\\	【副】まったく同じように
\\	(単2)‐
\\	/(複1・2・4)..
\\	(複3)..
\\	実業家、ビジネスマン 
\\	/〔女性〕
\\	(一定区画の)土地
\\	ノート 
\\	【副】(外からこちらへ向かって)中へ 
\\	発音:
\\	(単2)‐
\\	(複1・2・4)
\\	(複3)
\\	①〔複数は種類を表わすときのみ〕木、木材 
\\	2【自】① かむ 
\\	3格と〗
\\	(いらいらして)つめをかむ
\\	(単2)‐
\\	/(複1・2・4)‐ (複3)‐
\\	ウエーター、ボーイ 
\\	/〔女性〕‐
\\	①(果実の)種
\\	【自】②(ベル・電話などが)鳴る〖非人称の
\\	を主語として〗
\\	ベルが鳴った。
\\	1【他】
\\	〔…⁴を…
\\	と〕組み合わせる、結合させる 
\\	病気、疾病 
\\	①〔複数なし〕癌(がん)
\\	【形】①(鉤形に)曲がった
\\	文化 
\\	電灯、ランプ 
\\	②〔ふつう単数で〕(時間的な)長さ
\\	①〔…⁴を〕教える 
\\	〔…⁴に…⁴を〕教える
\\	①〔複数なし〕力 
\\	②〔複数なし〕権力
\\	大量、多数 
\\	大衆;群衆
\\	【副】たいてい 
\\	Ⅰ【形】① 現代的な、モダンな 
\\	〔過去〕
\\	〔過分〕
\\	【自】
\\	〔…⁴について〕熟考する、じっくり考える
\\	⦅国名⦆ニュージーランド(首都はウェリントン)
\\	水準、レベル 
\\	発音:
\\	(単2)‐
\\	(複1・2・4)‐ (複3)‐
\\	犠牲 
\\	犠牲者、被害者
\\	キノコ  ②⦅植物・医学⦆菌類
\\	(単2)‐
\\	/(複1)‐または..
\\	⦅文法⦆代名詞 発音:
\\	⦅植物⦆ブドウ;⦅文語⦆ブドウの木
\\	(数段からなる)棚、本棚 発音:
\\	①(ある種の特徴をもった比較的広い)地域、地方 発音:
\\	成熟 ②(中・高等学校の)修了、卒業 
\\	中学校卒業;ギムナジウム第6学年修了
\\	(単2)‐/(複1・2・3・4)
\\	リズム 発音:
\\	酸素 (〔記号〕
\\	【形】① 傾いた、斜めになった
\\	〔現在〕
\\	〔過去〕
\\	〔過分〕
\\	〔接Ⅱ〕
\\	1【他】①〔…⁴を〕打つ、殴る、たたく 
\\	〔過去〕
\\	〔過分〕
\\	〔接Ⅱ〕
\\	【自】① 黙っている、黙る
\\	Ⅰ【副】あとで、のちに 
\\	ではまたあとで!
\\	Ⅱ【形】
\\	の比較級)② のちの
\\	(一品一品の)料理 
\\	南 (〔略〕
\\	タバコ 
\\	発音:
\\	部分、(全体の中の)一部 
\\	温度;気温 
\\	(単2)‐/(複1・2・3・4)‐
\\	ブドウ、ブドウの房 
\\	1【自】①〔完了
\\	〔…へ・から〕滴り落ちる 
\\	相違、違い 
\\	花瓶 
\\	〔現在〕
\\	〔過去〕
\\	〔過分〕
\\	〔接Ⅱ〕
\\	1【自】〔完了
\\	①(食品が)傷む、腐る、駄目になる2【他】①〔…⁴を〕台無しにする
\\	【他】①
\\	に…⁴を〕してみせる 
\\	に…⁴を〕ほんとうだと思わせる
\\	成長、生長 
\\	1【再】
\\	抵抗する、逆らう 
\\	1【他】①〔…⁴の〕数を数える、〔…⁴を〕数える 
\\	2【自】②
\\	に〕数えられる、
\\	の〕一つ〈一人〉とみなされる
\\	(単2)‐
\\	(複1・2・4)
\\	(複3)
\\	歯 
\\	地帯、地域 
\\	森林地帯
\\	【副】少なくとも 発音:
\\	〔過去〕
\\	〔過分〕
\\	〔接Ⅱ〕
\\	1【他】①
\\	〔…⁴に…
\\	を〕強いる、強制する 
\\	装置、設備
\\	①〔複数なし〕重要性 
\\	意味 
\\	【自】〔完了
\\	①《3格》
\\	に〕(偶然)出会う 
\\	様態》⦅文語⦆
\\	に…の態度で〕接する、応対する 
\\	〔過去〕
\\	〔過分〕
\\	【他】〔…⁴を〕理解する、把握する 
\\	主張 
\\	…⁴を主張する
\\	①(人間・動物の)足、脚 
\\	【副】ほとんど;危うく、もう少しのところで 
\\	少年はほとんど母親と同じくらいの背の高さだ 発音:
\\	【副】すでに、もう 
\\	【形】①
\\	を〕意識している;気づいている 
\\	彼は自分の義務を自覚している 
\\	彼は自分の過ちに気づく。 
\\	故意の、意識的な
\\	①(樹木の)花 
\\	③〔複数なし〕最盛期 
\\	(または
\\	当時経済は最盛期を迎えていた。
\\	Ⅰ【副】①(話者から少し離れた場所を身振りなどで指して)そこに、そこで 
\\	それで 
\\	〘文脈などで示される状況・理由・条件など〙そういうわけで Ⅱ【接】《従属》【定動詞後置】① …だから、…なので 
\\	(〔参考〕
\\	は相手も知っていることを理由とする。) 
\\	【接】…するために ・副文を作る(定動詞は後置される)Ⅰ【副】1①(道具・手段)それでもって 
\\	【副】④ それについて 発音:
\\	⦅文語⦆① 存在 
\\	神の存在を否定する
\\	の現在分詞)【形】① 緊急の、差し迫った
\\	Ⅰ【副】① ちょっと前に、たった今 ⑤〖文アクセントなしで〗(その状況が変えられないことを表わして)なにしろ、とにかく Ⅱ【形】平らな、平坦な 
\\	〔過去〕
\\	〔過分〕
\\	1【他】①〔身体の一部⁴を〕引っ込める 2【自】〔完了
\\	入居する 
\\	①〔複数なし〕鉄道 
\\	征服、占領 
\\	…⁴を征服する,攻略する,占領する
\\	(単2)‐
\\	/(複1・2・4)
\\	(複3)
\\	川 
\\	Ⅰ【形】① 正確な 
\\	⦅口語⦆(同意を表わして)そのとおり!
\\	〔現在〕
\\	〔過去〕
\\	〔過分〕
\\	〔接Ⅱ〕
\\	㊟物事を表わす名詞のみが主語になる 【自】〔完了
\\	起こる 
\\	(お礼に対して)どういたしまして
\\	①〔ふつう単数で〕社会 
\\	④〔ふつう単数で〕(いっしょにいる)仲間
\\	Ⅰ【形】① ふつうの 
\\	半分、2分の1 
\\	家庭教師 (〔女性〕‐
\\	〔過去〕
\\	〔過分〕
\\	〔接Ⅱ〕
\\	1【他】①〔…⁴を〕上げる、持ち上げる 
\\	【形】① きょうの  
\\	今日(こんにち)の
\\	【副】特に、とりわけ 
\\	発音:
\\	世紀 (〔略〕
\\	Ⅰ【副】しかし、けれども 発音:
\\	映画館 
\\	映画を見に行く
\\	【自】〔完了
\\	①(両手足を使って)よじ登る 
\\	①(教科書の)課 
\\	発音:
\\	〖定冠詞と〗⦅海名⦆ 地中海
\\	午後 
\\	午後に
\\	①〔複数なし〕自然 
\\	②〔ふつう単数で〕本性 発音:
\\	Ⅰ【副】② ときどき、何度か
\\	油、オイル 
\\	Ⅰ【形】① 東の、東方〈東部〉の 
\\	【形】① 個人の 
\\	発音:
\\	コショウ 
\\	①〔ふつう単数で〕質、品質 
\\	発音:
\\	〔過去〕
\\	〔過分〕
\\	〔接Ⅱ〕
\\	1【他】①〔…⁴を〕こする 
\\	その犬は頭を彼女のひざにすりつけた。
\\	残り;余り 
\\	Ⅰ【形】① 丸い 
\\	Ⅱ【副】およそ、だいたい、約 (略:
\\	(ボール紙などで作られたふた付きの)箱
\\	Ⅰ1【他】①〔…⁴を〕やり遂げる、なし遂げる 
\\	Ⅱ〔過去〕
\\	〔過分〕
\\	〔接Ⅱ〕
\\	【他】②〔状況・状態など⁴を〕作り出す
\\	運命 
\\	【副】② 結局のところ
\\	【接】1《並列》及び、並びに 
\\	※語句と語句を結び付ける。定動詞の位置に影響を与えない。 発音:
\\	【副】遅くとも
\\	【自】②
\\	に〕由来する
\\	鉛筆、色鉛筆
\\	⦅口語⦆街娼(街頭で客をさそう売春婦。)
\\	(単2)‐
\\	(複1・2・4)
\\	(複3)
\\	谷 
\\	動物 
\\	2【自】① 夢を見る 
\\	のことを〕夢見る、夢に描く
\\	【形】① 悲しい 
\\	1【他】①〔ニュース・うわさなど⁴を〕広める、広く伝える 
\\	(または
\\	ニュースをラジオを通じて伝える 2【再】①
\\	(ニュース・うわさなどが)広まる、流布する
\\	【形】② さまざまな、いろいろな
\\	〔過去〕
\\	〔過分〕
\\	【他】〔…⁴を〕許す 
\\	Ⅰ【形】① 西の、西方〈西部〉の 
\\	1【再】①
\\	〔予想外のこと⁴に〕驚く 
\\	彼は私の決心に驚いている 
\\	【形】多数の、たくさんの
\\	【副】〖後続の
\\	などと〗確かに 
\\	確かに彼女はひどい風邪をひいていたが、大学へは行った。
\\	2【自】① 魚釣りをする
\\	1【再】
\\	努力する、頑張る
\\	(単2)‐
\\	/(複1・2・4)
\\	(複3)
\\	①(幹から直接出ている)枝
\\	【自】①
\\	〔…⁴に〕(不都合が生じないように)気をつける 
\\	子供〈荷物〉を見ていてくれ! 
\\	注意を払う 
\\	気をつけて!;いいかい(よく聞いて)!
\\	発言
\\	〔現在〕
\\	(まれに
\\	(まれに
\\	〔過去〕
\\	(まれに
\\	〔過分〕
\\	〔接Ⅱ〕
\\	1【他】①〔パン・ケーキなど⁴を〕(オーブンなどで)焼く 
\\	〔現在〕
\\	〔過去〕
\\	〔過分〕
\\	①〔…⁴を〕持っておく、とっておく、手元におく 
\\	②〔…⁴を〕保つ、保持する
\\	〔現在〕
\\	〔過去〕
\\	〔過分〕
\\	1【自】
\\	に〕貢献する、寄与する 
\\	世界平和の維持に貢献する
\\	ガソリン 
\\	発音:
\\	(ある地域の)住民;人口 
\\	【形】①(顔などが)青白い;顔色が悪い 
\\	②(色などが)薄い、淡い
\\	【自】①(花が)咲いている 
\\	2【自】① 掘る 
\\	指で鼻をほじくる
\\	【副】② それについて 発音:
\\	Ⅰ【形】①(偽物ではなく)本物の 
\\	Ⅱ【副】ほんとうに
\\	導入、採用 
\\	①(労働力などの)投入、動員;(警察などの)出動②(機械などの)使用 
\\	賭金;保証金
\\	経験 
\\	【他】①〔…⁴を〕体験する、経験する 
\\	【形】新たな、再度の
\\	3の過去分詞)【形】驚いた
\\	【形】ヨーロッパの 
\\	【形】① 怠惰な、怠け者の 
\\	祝典;祝賀会、パーティー 
\\	祝賀会〈式〉に出席する
\\	Ⅰ【形】①(空間的に)遠い 
\\	〔現在〕
\\	〔過去〕
\\	〔過分〕
\\	②〔…⁴を〕放さない 
\\	母親の腕をつかんで放さない ③〔…⁴を〕(写真・文章などの形にして)残しておく、保存しておく 3【再】
\\	に〕しっかりつかまる
\\	(単2)‐
\\	/(複1・2・4)‐ (複3)‐
\\	(手の)指 
\\	【形】① 自由な 
\\	②(席・部屋などが)空いている
\\	危険 
\\	危険な状態にある
\\	1【他】①〔部屋など⁴を〕暖める、暖房する 
\\	【形】① 明るい 
\\	②⦅口語⦆(人間が)大声で泣く、泣きわめく
\\	【副】①(時間的に)あとから
\\	後ろから
\\	①〔複数なし〕輸入 
\\	発音:
\\	皇帝、(日本の)天皇 
\\	/〔女性〕‐
\\	(単2)‐
\\	/(複1・2・3・4)..
\\	(まれに複
\\	専門用語として..
\\	気候 
\\	の過去分詞)【形】① 複雑な、込み入った
\\	王、国王 
\\	/〔女性〕‐
\\	1【他】①〔…⁴を〕ひっかく、かく 
\\	銅 (〔記号〕
\\	【形】① 退屈な、つまらない 
\\	〔現在〕
\\	〔過去〕
\\	〔過分〕
\\	〔接Ⅱ〕
\\	Ⅰ【助動】《使役の助動詞》①《4格》
\\	のない不定詞と]〔…⁴に〕…させる
\\	Ⅱ1【他】①〔…⁴を〕やめる、やめておく ②《4格
\\	場所》〔…⁴を…に〕置いておく;残しておく 
\\	状態》〔…⁴を…の〕ままにしておく
\\	〔過去〕
\\	〔過分〕
\\	〔接Ⅱ〕
\\	1【自】①
\\	〔病気
\\	に〕かかっている 
\\	彼は不眠症だ。 
\\	に〕苦しむ 
\\	病気;苦しみ
\\	リットル (〔記号〕
\\	小麦粉
\\	⦅植物⦆うどん粉病
\\	【冠】《不定数量》① いくつかの、二三の 
\\	意見, 考え 
\\	(または
\\	私の考えでは
\\	【副】少なくとも 
\\	手段
\\	薬 
\\	平均値 
\\	平均[値]では 
\\	中央、真ん中
\\	①〔複数なし〕窮乏、困窮 
\\	苦境、窮地 
\\	Ⅰ【形】① 必要な、不可欠な 
\\	東 (〔略〕
\\	ポスター 
\\	①〔複1・2・3・4
\\	パーセント (〔略〕
\\	/〔記号〕
\\	発音:
\\	⦅口語⦆ くだらないこと、ばかばかしいこと 
\\	ばかばかしい!;くだらない!
\\	ラジオ
\\	(単2)‐
\\	/(複1・2・4)
\\	(複3)
\\	縁、端 
\\	②(ページの)余白、欄外
\\	⦅昆虫⦆ブドウネアブラムシ
\\	Ⅰ【形】①〖付加語的に〗右の、右側の 
\\	③〖述語的に〗《3格》
\\	にとって〕都合のよい
\\	1【他】③〔…⁴を〕感動させる 3【再】
\\	動く 
\\	ここは身動きできない
\\	寝室 
\\	(単2)‐
\\	/(複1・2・4)
\\	(複3)
\\	膝(ひざ)(座ったときの腰から膝頭までの部分) 
\\	⦅化学⦆硫黄 (〔記号〕
\\	石けん 
\\	1【自】ため息をつく 
\\	散歩する 
\\	統計 
\\	発音:
\\	〔過去〕
\\	〔過分〕
\\	【自】(行事・催し物などが)開催される、行なわれる 
\\	【自】驚く;驚嘆〈感嘆〉する
\\	石 
\\	3格+
\\	は心の重荷が下りる
\\	【形】① 静かな 
\\	静かにしなさい!
\\	2【自】① 正しい、事実に合う 
\\	に由来する
\\	1【自】うめく、うめき声を発する、うなる 
\\	①〔ふつう複数で〕涙 
\\	【副】いたるところで、どこでも 
\\	①(大規模な)事故、災害 
\\	②〔複数なし〕不幸、不運
\\	(単2)‐
\\	/(複1・2・4)
\\	(複3)
\\	鳥 
\\	(単2)‐
\\	/(複1・2・4)
\\	(複3)
\\	民族 
\\	②〔複数なし〕(市民の集合体としての)国民、人民
\\	〔過去〕
\\	〔過分〕
\\	【自】①
\\	を〕通り過ぎる、通りかかる 
\\	⦅口語⦆
\\	の所に〕立ち寄る
\\	〔現在〕
\\	〔過去〕
\\	〔過分〕
\\	【他】〔…⁴を〕読んで聞かせる 
\\	私は子供たちに物語をいくつか読んでやった。
\\	①〔複数なし〕選択
\\	Ⅰ【副】たぶん、おそらく 
\\	Ⅱ【形】ありそうな、考えられる
\\	〔過去〕
\\	〔過分〕
\\	【自】〔完了
\\	立ち去る ②⦅口語⦆ 出かける、外出する
\\	しばらくの間 
\\	①〔複数なし〕⦅口語⦆(つまらない)もの②〔複数なし〕⦅口語⦆くだらないこと;くだらない話 
\\	たわごとを言うな!
\\	【副】①(…と)異なって、別なふうに
\\	変化、変更 
\\	〔過去〕
\\	〔過分〕
\\	①〔…⁴を〕書き留める、控える
\\	職業教育、専門教育 
\\	選択 
\\	①〔複数なし〕助言、忠告
\\	〔過去〕
\\	〔過分〕
\\	1【他】①(審議・考量のあとに)〔…⁴を〕決定する 
\\	〔…⁴を〕採決する、議決する
\\	連邦議会は新しい法案を議決した 
\\	【形】① かなりの、相当の 
\\	化学 
\\	発音:
\\	の複数) 2【複】データ、資料
\\	【形】①(物事が)明らかな、明白な 
\\	はっきりした、明瞭な
\\	【形】最終的な 
\\	風邪 
\\	(単2)‐
\\	/(複1・2・4)..
\\	(複3)..
\\	ふつう複数で ⦅文語⦆① 収穫高、収量
\\	②〔農産物など⁴を〕生産する、産出する
\\	平地、平野 
\\	面積
\\	研究 
\\	[派生] 
\\	研究する
\\	大陸 
\\	発音:
\\	【形】文化の、文化的な 
\\	発音:
\\	【形】① 愛する、好きな;親愛なる 
\\	心の優しい、親切な 
\\	感じのよい、かわいい
\\	②(二枚貝の)貝殻
\\	患者 (〔女性〕‐
\\	発音:
\\	①〔ふつう単数で〕生産、製造 
\\	列 
\\	一連
\\	Ⅰ【形】① 純粋な 
\\	純金  ③〖付加語的に〗⦅口語⦆ まったくの、純然たる 
\\	それは狂気も同然だ ④(音・色が)澄んだ
\\	【形】① 宗教の、宗教上の、宗教的な 
\\	発音:
\\	ショーウインドー
\\	【形】統計の、統計上の 
\\	発音:
\\	立ち止まる 
\\	【形】系統立った、体系的な 
\\	発音:
\\	インク 
\\	〔現在〕
\\	〔過去〕
\\	〔過分〕
\\	〔接Ⅱ〕
\\	1【自】①〔…を〕ける 
\\	彼女は私の腹をけった ④〔完了
\\	〔…へ・から〕歩み出る〈入る〉 
\\	①(目的を達するための)試み、行為② 企業、会社 発音:
\\	催し物 
\\	…⁴を催す、開催する
\\	1【他】①〔…⁴を〕改善する 
\\	〔過去〕
\\	〔過分〕
\\	〔接Ⅱ〕
\\	【自】〔完了
\\	①(視界から)消える、姿を消す
\\	【形】手元にある、存在している 
\\	発音:
\\	目覚まし時計 
\\	【形】①〔…⁴の〕価値のある 
\\	〔…2・4に〕値する 
\\	(お礼のことばに対する返事で)どういたしまして(←それは語るに値しないことだ)
\\	学問、科学[の一分野]
\\	(単2)‐
\\	/(複1・2・4)..
\\	(複3)..
\\	ごみ 
\\	【他】
\\	②(不用のものなどを)わきへ置く;(一時的に)しまう、格納する (③〔車など⁴を〕止めておく) 
\\	【他】〔資料・データなど⁴を〕(最新の状態に)更新する 
\\	並木道 
\\	発音:
\\	の現在分詞)【形】骨の折れる、きつい
\\	薬局 
\\	弱変化;(単2・3・4)‐
\\	/(複1・2・3・4)‐
\\	建築家 発音:
\\	①(新聞などの)記事
\\	品物、商品 ③⦅文法⦆冠詞 
\\	②(乗り物用の)出口 
\\	①⦅文法⦆アクセント[を置くこと] 
\\	①(布・皮・ビニール製の小さな)袋 
\\	略奪品,戦利品
\\	西洋なし 
\\	電球
\\	めがね 
\\	【他】①〔…⁴の〕つづりを言う 
\\	あなたの名前のつづりを言ってください!
\\	市民、国民 
\\	/〔女性〕-
\\	①(ふつう円筒形でふた付きの小さな)容器② 缶;缶詰め 
\\	魚の缶詰め1個
\\	三角形 
\\	①(乗り物の)乗り入れ口 
\\	【他】〔…⁴を〕中に入れる
\\	1(卵の)白身、卵白 
\\	たんぱく質
\\	⦅植物⦆ イチゴ
\\	酢 
\\	乗り物(車両・列車・船舶など)
\\	脂肪
\\	【形】① 新鮮な 
\\	理髪師、美容師、床屋 
\\	/〔女性〕‐
\\	発音:
\\	【形】危険な、危ない 
\\	遺伝子技術
\\	②⦅文法⦆(名詞の)性 発音:
\\	グラム (〔記号〕
\\	【形】① 有効な、通用している 
\\	ホール 
\\	【形】①(品物・食品が)長持ちする、持ちの良い 
\\	Ⅰ【形】① 固い、硬い 
\\	厳しい 
\\	つらい
\\	暖房装置 
\\	(単
\\	(複
\\	(複1・2・3・4)‐
\\	心臓 
\\	心 
\\	中心部、心臓部 
\\	ハート形のもの
\\	【副】…だとよいが 
\\	彼女が来てくれるとよいが。
\\	⦅文法⦆不定詞
\\	(本の)章 (〔略〕
\\	発音:
\\	(単2)‐
\\	/(複1・2・3・4)
\\	[まれに(複1)‐] 
\\	箱、ケース 
\\	ネックレス ④(商店・ホテルなどの)チェーン店
\\	芸術、美術 
\\	技術、技;テクニック
\\	宿営地、野営地④(商品の)倉庫 
\\	寝床 
\\	【形】おいしい、おいしそうな
\\	〔過去〕
\\	〔過分〕
\\	〔接Ⅱ〕
\\	1【自】うそをつく 
\\	殺人、殺害 
\\	勇気 
\\	⦅文法⦆名詞
\\	救急医
\\	(単2)‐/(複1・2・3・4)‐
\\	①〔ふつう複数で〕ヌードル(西洋風の麺類)
\\	①(商品の)一包み、一箱 
\\	1パックの卵
\\	①〔複数なし〕⦅口語⦆不運 
\\	運が悪い 
\\	(単2)‐
\\	/(複1・2・4)
\\	(複3)
\\	②(ビール瓶などの)保証金、デポジット 
\\	植物 
\\	ピザ 
\\	位置
\\	Ⅰ【前】〖4格支配〗…当たり、…につき
\\	1日に2回
\\	1①(形としての)点 
\\	ピリオド 
\\	論点 
\\	得点、点数、ポイント
\\	⦅商業⦆値引き、割引 発音:
\\	規則、ルール 
\\	スパゲッティ 
\\	発音:
\\	⦅植物⦆アスパラガス 
\\	(単2)‐
\\	/(複1・2・3・4)‐
\\	[まれに(複1)‐
\\	①(活動などの)開始
\\	声
\\	①〔複数なし〕大学での勉強 
\\	スープ 
\\	料金〈運賃〉[表] 発音:
\\	〔過去〕
\\	〔過分〕
\\	【他】〔…⁴を〕殺害する、殺す 
\\	(単2)‐
\\	/(複1・2・4)..
\\	(複3)..
\\	事故 
\\	【形】① 不幸な 
\\	不運な
\\	⦅植物⦆バニラ;(香料の)バニラ 発音:
\\	〔過去〕
\\	〔過分〕
\\	〔接Ⅱ〕
\\	1【他】②〔複数の物・場所⁴を〕結ぶ、つなぐ ③〔…⁴を〕縛る
\\	⦅言語⦆母音 発音:
\\	【形】① 目が覚めている 
\\	①〔複数なし〕〖集合的に〗洗濯物 
\\	【形】① 柔らかい、軟らかい 
\\	優しい、柔和な 
\\	レモン 
\\	〔現在〕
\\	〔過去〕
\\	〔過分〕
\\	【自】〔完了〕
\\	①(乗り物で)帰る、戻る
\\	【形】追加の
\\	略語 (〔略〕
\\	Ⅰ【形】意図的な、故意の 
\\	〔現在〕
\\	〔過去〕
\\	〔過分〕
\\	【他】①〔…⁴を〕洗ってきれいにする 
\\	顔〈食器〉を洗う
\\	1【再】④
\\	⦅コンピュータ⦆ログインする 2【他】①〔…⁴の入学・参加などを〕申し込む 
\\	(または
\\	子供の入学願書を出す 
\\	1【他】①〔…⁴を〕掛ける、つるす 
\\	コートを洋服掛けに掛ける 2【再】
\\	首つり自殺をする
\\	1【他】①〔…⁴を〕解明する、解決する 
\\	〔過去〕
\\	〔過分〕
\\	【自】〔完了
\\	②(乗り物に)乗る 
\\	バイクに乗る
\\	(単2)‐
\\	/(複1・2・4)..
\\	(複3)..
\\	依頼 
\\	電池、バッテリー 
\\	①(ふつう取っ手や脚のない)コップ;(プラスチックなどの)容器 
\\	ミルクをコップ1杯飲む 
\\	初め、最初 
\\	発言、コメント (〔略〕
\\	[派生] 
\\	②…⁴を述べる
\\	【他】①〔…⁴を〕観察する、監視する 
\\	所有者、オーナー 
\\	/〔女性〕‐
\\	…⁴を所有している
\\	【他】①〔…⁴を〕じっと見る、観察する 
\\	彼は私を物珍しそうにじっと見る。 
\\	〔…⁴を…と〕みなす 
\\	〔現在〕
\\	〔過去〕
\\	〔過分〕
\\	①《4格》〔部屋・建物⁴に〕入る
\\	【形】① 苦い 
\\	橋 
\\	【他】《4格》〔…⁴に〕アイロンをかける 
\\	1【自】議論する、討論する 
\\	正直な、誠実な 
\\	ほんとうの
\\	【形】① 個々の 
\\	説明 
\\	【副】第一に、まず
\\	1【他】〔…⁴を〕続ける、続行する 
\\	継続,続行
\\	(単2)..
\\	/(複1・2・4)..
\\	(複3)..
\\	秘密 
\\	〔過去〕
\\	〔過分〕
\\	〔接Ⅱ〕
\\	【他】①〔…⁴を〕楽しむ 
\\	②〔信頼・名声など⁴を〕得ている
\\	【副】十分に;かなり 
\\	①〖集合的に〗食器
\\	スパイス、香辛料 
\\	【形】① すばらしい、立派な 
\\	1【自】⦅口語⦆①〔方向〕〔…を〕見る、のぞく 〖〔方向〕なしで〗
\\	見てごらん!
\\	首 
\\	のど 
\\	【形】① 秘密の、ひそかな 
\\	ヘルメット 
\\	⦅文法⦆ 命令法
\\	(単
\\	/(複1)..
\\	市の中心部;都心、中心街 
\\	【副】全部で、合計して;全体として 
\\	【副】① いずれにせよ、ともかく 
\\	少なくとも
\\	2【他】①《4格》〔…⁴にとって〕かゆい 
\\	または〖非人称の
\\	を主語として〗
\\	私は背中がかゆい。
\\	①⦅電気⦆ケーブル、電線;(電気器具の)コード 
\\	ケーブルを敷設する
\\	【副】① ほとんど…ない 
\\	サクランボ、チェリー 
\\	クッション、座ぶとん 
\\	まくら 
\\	〖集合的に〗衣服、服 
\\	(単2)‐
\\	/(複1・2・4)
\\	(複3)
\\	コック、調理師、料理人 
\\	/〔女性〕
\\	消費 
\\	かご、ざる 
\\	【形】① 空(から)の;空いた 
\\	【他】①〔…⁴を〕配達する 
\\	線 
\\	透き間、空白 
\\	①〔複数なし〕医学 
\\	薬 発音:
\\	【冠】《定冠詞類》 ①〖単数名詞と〗かなりの数の ②〖複数名詞と〗一部の 
\\	コート 
\\	3格+
\\	にコートを着せて〈脱がせて〉やる
\\	1【他】①〔…⁴を〕混ぜる 
\\	【形】① 可能な 
\\	ありうる、起こりうる 
\\	ごみバケツ 
\\	弱変化;(単2・3・4)‐
\\	/(複1・2・3・4)‐
\\	甥 
\\	【形】好奇心の強い、知りたがっている 
\\	⦅文法⦆1格、主格(〔略〕
\\	2【自】駐車する
\\	⦅文法⦆接頭辞、前つづり
\\	など)
\\	(単
\\	/(複1・2・4)..
\\	(複
\\	助言、忠告 
\\	1【他】①〔数量など⁴を〕減らす、〔程度など⁴を〕下げる 
\\	失業者の数を減らす
\\	雨傘
\\	方向 
\\	【形】① 生の 
\\	(単2)‐
\\	/(複1・2・4)..
\\	(複3)..
\\	リュックサック
\\	(単
\\	/(複1・2・4)
\\	(複
\\	袋
\\	(単
\\	/(複1・2・3・4)
\\	損害、被害 
\\	《3格》…
\\	を害する,損なう,傷つける
\\	Ⅰ【形】① 鋭い、鋭利な;とがった 
\\	③(味・においが)刺激性の 
\\	(または
\\	このソースは辛い。
\\	にわか雨、にわか雪、夕立 
\\	【形】①(身なりなどが)粋な、しゃれた ②⦅口語⦆最新流行の 
\\	【形】① 汚れた、汚い 
\\	【形】①(体が)弱い 
\\	【形】奇妙な、不思議な 
\\	安楽いす、ひじ掛けいす 
\\	【自】⦅口語⦆ ショッピングをする 
\\	【接】《従属》〖定動詞後置〗① …の間は、…しているうちは 
\\	発音:
\\	①〔複数なし〕〖集合的に〗おもちゃ 
\\	②(小さな)駅;停留所
\\	2【自】止まる、停止する
\\	合計 
\\	①〔ふつう複数で〕(チョコレート・キャンディーなどの)甘いもの、お菓子
\\	(単2)‐
\\	/(複1・2・4)..
\\	(複3)..
\\	ハンカチ 
\\	Ⅰ【副】② ほんとうに
\\	(単2)‐
\\	/(複1・2・4)..
\\	(複3)..
\\	引っ越し、移転 
\\	【形】普通でない、並外れた 
\\	【形】① 直接の 
\\	②(空間・時間的に)すぐの
\\	【形】異なった、さまざまな、いろいろな
\\	【他】①〔…⁴を〕調べる、調査する 
\\	(単2)‐
\\	/(複3)‐
\\	②〔ふつう複数で〕⦅映画⦆字幕
\\	1【他】②〔…⁴を〕延長する;更新する 
\\	…⁴を要求する
\\	〔現在〕
\\	〔過去〕
\\	〔過分〕
\\	1【他】①〔…⁴を〕裏切る 
\\	②〔秘密・計画など⁴を〕漏らす、そっと教える
\\	1【他】〔…⁴を〕隠す 
\\	3格と】
\\	彼女は日記を両親に見つからないように隠した 2【再】
\\	隠れる
\\	「使う」の過去分詞) Ⅱ【形】① 親類の、親戚の 
\\	似通った、類似した 
\\	〔過去〕
\\	〔過分〕
\\	または〔過去〕
\\	〔過分〕
\\	〔接Ⅱ〕
\\	1【他】〔…⁴を〕使う、用いる 
\\	〔現在〕
\\	〔過去〕
\\	〔過分〕
\\	〔接Ⅱ〕
\\	1【自】〔完了
\\	①(人・動植物が)成長する、育つ;(植物が)生える 
\\	【形】全世界にわたる、世界的な
\\	平日
\\	1【自】効果がある、作用する 2【他】①⦅文語⦆〔奇跡など⁴を〕行なう 
\\	奇跡を行なう
\\	①(植物の)根 
\\	根底;根源、原因
\\	(単
\\	/(複1・2・4)‐ (複
\\	紙片、紙切れ 
\\	何も書いてない紙きれ 
\\	①(不安・寒さなどで)震える 
\\	⦅文法⦆ 形容詞
\\	平日 
\\	②〔複数なし〕(単調な)日常[生活]
\\	②〔農作物⁴を〕栽培する 
\\	〔過去〕
\\	〔過分〕
\\	2【自】〔完了
\\	②(料理が)焦げつく 
\\	【他】①〔…⁴の〕スイッチを入れる 
\\	〔現在〕
\\	〔過去〕
\\	〔過分〕
\\	目立つ、人目につく、人目〈注意〉を引く 【3格と】
\\	この車のどこが君の目を引きますか? 
\\	目立つ
\\	1【再】
\\	興奮する;憤慨する 
\\	4格と】
\\	彼らは彼のふるまいに憤慨した
\\	例外 
\\	〔過去〕
\\	〔過分〕
\\	【自】〔完了
\\	(種族・家系が)死に絶える、絶滅する 
\\	〔過去〕
\\	〔過分〕
\\	①〔…⁴を〕描写する、記述する 
\\	〔道⁴を〕歩む、たどる
\\	【副】今まで[は]、これまで[は]
\\	【形】①(道徳的に)悪い 
\\	嫌な 
\\	怒っている
\\	【形】キリスト教の 
\\	(男性の)いとこ 
\\	/〔女性〕
\\	発音:
\\	テーブルクロス  
\\	天井
\\	(単
\\	/(複1・2・4)‐(複
\\	①(缶・鍋・木箱などの)ふた 
\\	②(本の)表紙
\\	【自】①《3格》
\\	に〕仕える、奉仕する 
\\	②《3格》
\\	の〕役に立つ 
\\	【形】⦅口語⦆① 汚れた、汚い 
\\	【形】①〖付加語的に〗自分[自身]の 
\\	②《3格》
\\	に〕特有の、固有の
\\	1【再】
\\	〔…⁴に〕ふさわしい、適している
\\	1【他】①〔…⁴を〕はめ込む、入れる ②〔部隊・力など⁴を〕投入する、動員する 
\\	〔…⁴を…4・3に〕任命する、指名する
\\	1【他】①〔…⁴を〕閉じ込める ②⦅口語⦆〔…⁴を〕投獄する 2【再】
\\	閉じこもる
\\	(単
\\	/(複1・2・3・4)
\\	糸 
\\	ライター 
\\	感情、気持ち 
\\	雑音、物音 
\\	1【他】①〔…⁴を〕創設する、設立する 
\\	(単
\\	/(複1・2・4)
\\	(複
\\	ハンマー 
\\	(単
\\	/(複1・2・4)..
\\	(複
\\	首都 (〔略〕
\\	Ⅰ【形】① きれいな、かわいい 
\\	③⦅口語⦆ かなりの、相当な
\\	ヨーグルト 
\\	(単
\\	/(複1・2・4)‐
\\	(複
\\	檻(おり) 
\\	1【他】①⦅ドイツ南部で⦆〔部屋など⁴を〕掃く 
\\	ろうそく 
\\	(単
\\	/(複1・2・4)
\\	(複
\\	力、能力
\\	カラス 
\\	⦅魚⦆サケ 
\\	〖集合的に〗木の葉
\\	1【再】
\\	…するに値する 
\\	この古い時計はもう直してもらう価値はない。
\\	〔現在〕
\\	〔過去〕
\\	〔過分〕
\\	【他】①〔つかんでいたもの⁴を〕放す、離す
\\	(単
\\	/(複1・2・3・4)
\\	または‐ 胃;おなか 
\\	⦅口語⦆私のおなかがぐうぐう鳴っている;私は腹ぺこだ。 
\\	【副】何度か、何度も
\\	〖ふつう無冠詞で〗真夜中 
\\	⦅文法⦆ 話法の助動詞 
\\	硬貨、コイン 
\\	母語、母国語 
\\	(単
\\	/(複1・2・4)
\\	(複
\\	くぎ 
\\	つめ
\\	【副】先日、この間 
\\	(単
\\	/(複1・2・4)
\\	(複
\\	駐車場
\\	(特に船・飛行機などの)乗客 
\\	ポンド(重量単位, 500グラム)(〔略〕
\\	試験、(学校での)テスト 
\\	【形】① 汁気の多い;みずみずしい 
\\	【形】① 清潔な、きれいな 
\\	1【他】①〔…⁴の〕皮をむく 
\\	蝶結び;リボン
\\	豚 
\\	シリーズ
\\	視界 
\\	見方、観点、見解 
\\	視界に入っている〈入る〉、見える〈見えてくる〉
\\	1【自】①
\\	〔…⁴の〕世話をする、めんどうを見る 
\\	〔…⁴(の達成のため)に〕配慮する 2【再】
\\	〔…⁴のことを〕心配する、気遣う 
\\	母は常に私の〈私の健康の〉ことを心配している
\\	【形】① 先のとがった、先の細い 
\\	⦅料理⦆ステーキ
\\	(単
\\	/(複1・2・4)
\\	(複
\\	(長めの)靴下;ストッキング 
\\	【形】②(鉛筆などが)とがっていない、先端が丸くなった
\\	【形】数時間にわたる
\\	【自】〔完了
\\	①《方向》〔…へ・から〕転落する、墜落する 
\\	転倒する
\\	①⦅文法⦆ 主語
\\	表、一覧表
\\	参加者、出席者 
\\	/〔女性〕‐
\\	に参加する
\\	助言、ヒント 
\\	(単
\\	/(複1・2・4)
\\	(複
\\	深鍋(なべ) 
\\	【形】完全な、全面的な
\\	1【再】①
\\	別れる、離れる2【他】①
\\	〔…⁴を…
\\	から〕分ける、切り離す ④〔…⁴を〕区別する
\\	(円すい形の)紙袋;(買い物などを入れる)ビニール袋 
\\	〔現在〕
\\	〔過去〕
\\	〔過分〕
\\	1【他】
\\	に…⁴を〕譲る
\\	【副】あさって、明後日
\\	【形】① 不親切な、無愛想な 
\\	ばかげたこと、くだらないこと 
\\	間違い 
\\	誤って、うっかりして
\\	【副】おととい、一昨日
\\	【副】つい今し方、さっき
\\	Ⅱ【副】⦅文語⦆ 主に、主として 
\\	〔現在〕
\\	〔過去〕
\\	〔過分〕
\\	1【他】〔…⁴を〕投げ捨てる、捨てる 
\\	①〔ふつう単数で〕交代;変化、移り変わり 
\\	①〔複数なし〕宣伝、広告 
\\	テレビで自動車の広告を打つ
\\	繰り返し;復習 
\\	【副】《疑問》なぜ、どうして 
\\	1【自】②《3格》 
\\	に〕来るように合図する 
\\	ウエーター〈タクシー〉に合図して呼ぶ 
\\	【他】①〔…⁴を〕飼育する;品種改良する
\\	(単
\\	/(複1・2・4)
\\	(複
\\	①〔ふつう複数で〕要求 
\\	【他】①〔…⁴を〕(あいさつをして)迎える、〔…⁴に〕(歓迎の)あいさつをする 
\\	〔過去〕
\\	〔過分〕
\\	〔接Ⅱ〕
\\	1【他】①
\\	〔…⁴を…⁴に〕結び付ける、縛り付ける 
\\	②〔…⁴を〕結ぶ
\\	熱中、熱意
\\	【形】① まじめな、真剣な 
\\	重大な、深刻な 
\\	(単
\\	(単3・4)‐
\\	/(複1・2・3・4)‐
\\	考え、思考 
\\	考えをまとめる
\\	Ⅰ【形】① 滑らかな、つるつるした;(しわのない)すべすべした 
\\	(単
\\	/(複1・2・4)‐
\\	(複
\\	レンジ、かまど(熱板・オーブンなどの付いた一体型調理設備を指すことが多い) 
\\	英雄、勇士
\\	⦅口語⦆① 飲み屋、居酒屋
\\	①〔複数なし〕機嫌 
\\	方法 
\\	【形】口頭の、口頭による 
\\	1【他】①〔…⁴の〕世話をする
\\	②〔…⁴の〕手入れをする
\\	酸っぱい、酸味の強い 
\\	夏は牛乳がすぐ酸っぱくなる〈腐る〉 ②⦅口語⦆不機嫌な、怒っている
\\	①〔複数なし〕心、精神 
\\	(単
\\	/複数なし 許し 
\\	すみません;ごめんなさい! [派生] 
\\	…⁴を許す
\\	【副】(あいさつ・感謝・謝罪の程度を高めて)幾重にも、重ね重ね 
\\	幾重にもお詫び申し上げます。
\\	動物園 
\\	【形】① 物おじする、内気な
\\	【形】⦅口語⦆① 頑固な 
\\	一気に
\\	①【自】
\\	〔…⁴に〕注意を払う
\\	自由
\\	政治 
\\	政界入りする
\\	はさみ
\\	①〔複数なし〕尊厳;威厳
\\	【接】《従属》[定動詞後置] 
\\	もし…ならば
\\	②⦅口語⦆ …の場合に備えて
\\	【形】無料の 
\\	【他】①〔ある金額⁴を〕支払う
\\	私はそれに100ユーロ払った 
\\	4格 
\\	…⁴を心から愛する 
\\	【他】①⦅ドイツ南部・オーストリア・スイスで⦆〔…⁴を〕見る
\\	〔…⁴を〕じっくり見る
\\	注文
\\	【形】① 孤独な、寂しい
\\	人里離れた 
\\	体験、経験
\\	【形】けちな 
\\	彼女はとてもけちだ 
\\	【自】① 咳(せき)をする
\\	①〔複数なし〕興味、関心
\\	3格と〗
\\	私はイタリアに行ってみたい 
\\	利益;利害 
\\	世界平和のために 
\\	おいしい 
\\	②(アイディアなどが)とてもおもしろい、愉快な 
\\	【他】①〔火・明かりなど⁴を〕消す 
\\	明かりを消す 
\\	肺 
\\	【副】昼に 
\\	巣 
\\	【形】かわいらしい
\\	【形】①(戸・本などが)開いている
\\	①〔複数なし〕現実
\\	1【自】⦅ドイツ南部・オーストリア・スイスで⦆①《方向》〔…に〕目を向ける 2【他】⦅ドイツ南部・オーストリア・スイスで⦆①〔…⁴を〕(意識して)見る、眺める 
\\	(または
\\	ほら;ねえ
\\	Ⅰ【代】《指示》〔不変化〕⦅口語⦆[自分]自身 
\\	Ⅱ【副】…すら、…さえ
\\	Ⅰ【副】それにもかかわらず
\\	【形】①(感覚的に)嫌な、不快な 
\\	嫌なにおい
\\	【副】① ただで、無料で  
\\	無駄に 
\\	【副】① 最初に、第一に、まず
\\	【複】食品
\\	【形】有害な 
\\	【形】伝統的な
\\	場所
\\	【形】① 困った
\\	嫌な、不快な
\\	㊚ (単
\\	/(複1・2・4)‐(複
\\	指導者、リーダー 
\\	/〔女性〕-
\\	㊛ (単2)‐/(複1・2・3・4)‐
\\	はしご
\\	【自】① ねらう 
\\	球
\\	弾丸
\\	【自】②
\\	⦅文語⦆
\\	を〕続ける
\\	【他】①〔災害など⁴を〕生き延びる
\\	彼女はその空襲を生き延びた
\\	①〔複数なし〕逃走
\\	(単2)‐
\\	/(複1・2・4)
\\	(複3)
\\	戦い 
\\	敵
\\	/〔女性〕-
\\	〔過去〕
\\	〔過分〕
\\	【自】〔完了
\\	走る 
\\	(単2)‐
\\	/(複1・2・4)‐(複3)‐
\\	敵
\\	/〔女性〕-
\\	(単2)‐/複数なし 忍耐、我慢
\\	忍耐強い、我慢強い
\\	トンネル
\\	【形】①⦅口語⦆どうでもよい
\\	序文 
\\	【形】① 永遠の 
\\	(単2)‐/(複1・2・3・4)‐
\\	武器 
\\	(単2)‐/(複1・2・3・4)‐
\\	警告
\\	Ⅰ【副】おそらく、多分
\\	(単2)‐
\\	/(複1・2・4)‐(複3)‐
\\	住民
\\	【他】①〔…⁴を〕小さくする、縮小する
\\	【副】(道・川・海・国境などを挟んだ)向こう側に、向こう側で 
\\	命令
\\	⦅軍事⦆(命令に対して)了解!
\\	【形】必要な
\\	板
\\	状態
\\	良好な〈最善の〉状態にある
\\	改善、改良
\\	丸薬、錠剤
\\	【自】① 揺れる 
\\	連続 
\\	結果 
\\	これは…という結果を生んだ
\\	(お礼に対して)どういたしまして
\\	(単2)‐/(複1・2・3・4)‐
\\	①〔複数で〕財政
\\	②〔複数なし〕金融
\\	(単2)‐
\\	/(複1・2・4)
\\	(複3)
\\	射撃、発砲
\\	瓦礫
\\	【他】①〔…⁴を〕(床・地面などに接したままで)押して動かす、ずらす
\\	ネズミ
\\	1【他】①
\\	方向》〔…⁴を…に〕挟む 3【自】(ドア・引き出しなどが引っかかって)動かない
\\	知識 
\\	(単2)‐
\\	/(複1・2・4)..
\\	(複3)..
\\	①〔ふつう単数で〕背景
\\	【他】①〔…⁴を〕(絵画などが)表わす
\\	やり方;方法
\\	【他】〔…⁴に〕言及する
\\	講演の中で彼はそのことにわずかに触れただけだった
\\	教育
\\	①(一定のまとまりとしての)システム
\\	結果、成果
\\	出来事、事件
\\	路地 
\\	〔過去〕
\\	〔過分〕
\\	〔接Ⅱ〕
\\	【自】① 撃つ
\\	ピストルを撃つ 
\\	だれかが警官たちをねらって発砲した 
\\	閉める、閉じる
\\	蓄え 
\\	〔過去〕
\\	〔過分〕
\\	【他】②〔シーツ・皮など⁴を〕はぐ 
\\	掛けぶとんカバーをはぐ【自】〔完了
\\	退く 
\\	軍、軍隊 
\\	軍人である
\\	【自】①〔完了
\\	はう 
\\	【形】① 疑わしい、不確かな
\\	におい、香り
\\	⦅口語⦆(肯定の答えを強めて)もちろんだよ!
\\	【他】①〔…⁴を〕大きくする、拡大〈拡張〉する
\\	【副】決して…ない
\\	(単
\\	/(複1・2・4)
\\	(複
\\	墓
\\	4格+
\\	⦅口語⦆ …⁴をいらいらさせる、怒らせる
\\	①〔複数なし〕(過失などに対する)責任 ③〔ふつう複数で〕借金、負債 
\\	(または
\\	⦅文語⦆ 彼に深い恩義を感じている
\\	【他】〔…⁴を〕救う、救助する 
\\	(単
\\	/複数なし 急ぎ 
\\	その件は急ぎません
\\	(単
\\	/複数なし 用心、注意 
\\	攻撃
\\	1【他】①〔…⁴に〕けがをさせる、〔…⁴を〕負傷させる 
\\	〔体の一部⁴に〕けがをする 
\\	心配するな! 
\\	の過去分詞)【形】⦅口語⦆① いまいましい 
\\	足跡 
\\	Ⅱ【再】①
\\	広がる 
\\	(または
\\	霧が一帯に広がった
\\	例えば(〔略〕
\\	影響
\\	…⁴に影響を及ぼす
\\	Ⅰ【自】〔完了
\\	〔問題など⁴を〕取り上げる 
\\	問題を取り上げる
\\	①〔ふつう単数で〕経済 
\\	役割 
\\	役割を演じる;重要である
\\	まず第一に
\\	⦅文語⦆ 計画 
\\	「一致する」の現在分詞)【形】ふさわしい、適した 
\\	気候に合った服装 
\\	Ⅱ【前】〖3格支配;名詞のあとに置かれることが多い〗…に応じて、…に従って
\\	⦅口語⦆(人に何かを聞かれて)全然わかりません!;まさかそんな!
\\	弱変化;(単2・3・4)‐
\\	/(複1・2・3・4)‐
\\	熊 
\\	/〔女性〕-
\\	(単2)‐/(複1・2・4)‐
\\	(複3)‐
\\	ネズミ
\\	カモ;アヒル
\\	ペット 
\\	(単2)‐/(複1・2・4)
\\	(複3)
\\	雌牛
\\	ミツバチ
\\	クモ 
\\	甲虫(カブト虫・テントウ虫・コガネ虫など)
\\	深い
\\	【形】① 澄んだ 
\\	明晰な
\\	さようなら!
\\	⦅口語⦆よろしい、オーケー
\\	①(服の)ボタン
\\	②(スイッチなどの)ボタン
\\	しみ
\\	①(下着の上に着る)シャツ、ワイシャツ
\\	英雄,勇士
\\	①(ネックレス・指輪などの)装身具、アクセサリー
\\	金のアクセサリーをつけている
\\	山岳地帯 
\\	①(成人の)女、女性
\\	妻
\\	①〔複数なし〕(芸術のジャンルとしての)絵画
\\	画家,絵かき
\\	パン屋
\\	①(計算・計量などの)結果
\\	【形】① 典型的な 
\\	外国
\\	【他】《4格》〔…⁴に〕キスをする
\\	【形】① ある地方の
\\	ローカルニュース
\\	世代
\\	大学
\\	〔参考〕
\\	「総合大学」を含めた総称として用いる場合と単科大学のみを指す場合とがある
\\	アンケート、世論調査
\\	サラリーマン、社員、職員 
\\	銀行員
\\	騒音
\\	【形】① 機械の 
\\	①(楽しい)会話、おしゃべり
\\	【副】① 特に 
\\	【形】① 問題のある
\\	落ち着く
\\	【形】① 液体の 
\\	【形】氷のように冷たい
\\	南極 
\\	道具、工具
\\	⑤⦅機械⦆ 装置、仕掛け
\\	オレンジ
\\	Ⅱ【他】⦅文語⦆ ①〔…⁴を〕送る
\\	【副】もう一度
\\	【他】②〔データ⁴を〕記憶させる、保存〈セーブ〉する 
\\	処置
\\	乾杯!
\\	①〔複数なし〕狩り、狩猟
\\	(単2)‐
\\	/(複1・2・4)
\\	(複3)
\\	講演
\\	講演を聞きに行く
\\	【形】電子工学の;電子的な
\\	【形】① だらしない、いい加減な 
\\	いい加減な仕事
\\	1【自】
\\	〔…3・4を〕信頼する、信用する
\\	われわれは彼を無条件で信用することができる [派生] 
\\	《3格》…
\\	を信用する,信頼する
\\	【他】〔お金・時間・労力など⁴を〕無駄に使う、浪費する
\\	(または
\\	その仕事に多大な労力を浪費する
\\	1【自】① 交渉する 
\\	【副】いつでも;いつなんどき(…かも知れない)
\\	是が非でも;どんな犠牲を払ってでも
\\	(単2)‐
\\	/(複1・2・4)..
\\	(複3)..
\\	買い物、ショッピング 
\\	買い物をする
\\	人質 
\\	1【再】①
\\	近づく 
\\	夏が近づく 〖3格と〗
\\	目的地〈終わり〉に近づく
\\	雄猫 ②⦅口語⦆ 二日酔い
\\	1【他】〔…⁴を〕(本質・外観において)変える
\\	1【自】① 失敗する 
\\	彼は仕事に失敗した 
\\	1【他】〔…⁴を〕落ち着かせる
\\	2【再】
\\	落ち着く;静まる
\\	1【他】②〔申し出など⁴を〕受け入れる 
\\	招待に応じる ④〔…⁴と〕推測する
\\	【形】① 不気味な 
\\	不気味な人影 ②⦅口語⦆ ものすごい、大変な
\\	(単2)‐
\\	/(複1・2・4)..
\\	(複3)..
\\	売ること 
\\	…⁴を売る
\\	【他】〔…⁴を〕傷つける
\\	過料、罰金 
\\	4格 
\\	⦅口語⦆…⁴をそっとしておく 
\\	そっとしておいてくれ!
\\	3格 
\\	を叱責する(「叱責」は、責任者が下の者の失敗や過ちをきつく非難すること。
\\	辞書)
\\	【他】⦅口語⦆〔…⁴を〕理解する、わかる 
\\	わかったかい?
\\	(単2)‐
\\	/(複1・2・4)..
\\	(複3)..
\\	①(個人的な)主義、信条
\\	原則、原理
\\	Ⅰ【形】① 一般的な([略]
\\	伝説 
\\	【他】〔伝統など⁴を〕(後世に)伝える、伝承する
\\	(単2)‐
\\	/(複1・2・4)‐(複3)‐
\\	ロバ
\\	/[女性]-
\\	放送 
\\	放送番組
\\	【自】①《3格》
\\	に〕服従する、従う
\\	Ⅰ【副】絶対に
\\	(単2)‐/(複1・2・3・4)‐
\\	①〔複数なし〕教育;教養
\\	〔過去〕
\\	〔過分〕
\\	【他】①〔…⁴を〕だます、欺く
\\	⦅文法⦆ 3格
\\	1【他】①〔…⁴を〕調べる、検査する
\\	商品を検査する 
\\	(単2)‐/複数なし 理性 
\\	①〔複数なし〕(肉体に対する)精神
\\	霊;亡霊、幽霊 3格+
\\	をいらいらさせる
\\	【他】①〔…⁴を〕操縦する、運転する
\\	(単2)‐
\\	/(複1・2・4)..
\\	(複3)..
\\	①(長くてゆったりした)衣服、衣装 
\\	衣服を身にまとう〈脱ぐ〉 
\\	(単2)‐
\\	/(複1・2・4)‐(複3)‐
\\	芸術家 
\\	/[女性]-
\\	名人、達人
\\	〔現在〕
\\	〔過去〕
\\	〔過分〕
\\	〔接Ⅱ〕
\\	1【他】①〔…⁴を〕折る
\\	2【自】①〔完了
\\	折れる
\\	(単2)‐/複数なし 
\\	正義 
\\	(単2)‐/(まれに複1‐
\\	孤独 
\\	1【他】①〔…⁴を〕刺す
\\	弱変化;(単2・3・4)‐
\\	/(複1・2・3・4)‐
\\	①⦅文語⦆ 男の子、少年
\\	(単2)‐/(複1・2・3・4)‐
\\	①〔複数なし〕継続、続行 
\\	②(小説・テレビ番組などの)続き、続編 
\\	…⁴を続ける、続行する
\\	(単2)‐/複数なし 
\\	青春時代;青春
\\	(単2)‐
\\	/(複1・2・4)..
\\	(複3)..
\\	車輪 
\\	【他】①〔…⁴の〕到来を待つ
\\	【他】《4格》〔不快なこと⁴に〕耐える
\\	私はその痛み〈運命〉に耐えられない 
\\	境界、国境 
\\	②〔ふつう複数で〕限度、限界
\\	1【自】祈る 
\\	1【他】①〔…⁴を〕一つにまとめる、統合する
\\	(単2)‐
\\	/(複1・2・4)‐
\\	(複3)‐
\\	①⦅昆虫⦆ チョウ 
\\	〔過去〕
\\	〔過分〕
\\	1【自】① 争う、けんかする 
\\	2【再】
\\	と〕争いをする、けんかする 〖相互的に(4格)〗
\\	こんな些細なことでけんかをするつもりはない
\\	(単2)‐/(複1・2・3・4)‐
\\	音符 ②(成績の)評点
\\	【形】過剰な、余分な
\\	【自】① 光る、輝く 
\\	ランプがともっている
\\	【他】①〔…⁴を〕持ち上げる
\\	信念
\\	信仰
\\	ケーキ屋(ふつう喫茶店を兼ねる)
\\	(単2)‐
\\	/(複1・2・4)
\\	(複3)
\\	香り、(快い)におい
\\	【副】同じくらい、同様に
\\	夜明け;夕暮れ
\\	【自】〔完了
\\	墜落する、転落する
\\	(単2)‐
\\	/(複1・2・4)
\\	(複3)
\\	①〔ふつう単数で〕騒音
\\	【他】①《4格》〔…⁴に〕耐える 
\\	【他】①〔…⁴を〕拒否する 
\\	【形】① 成功した 
\\	(単
\\	/(複1・2・4)-
\\	(複
\\	①〔複数なし〕(言葉・文章などの表わす)意味 
\\	③〔ふつう複数で〕感覚、知覚
\\	ヘビ(蛇) 
\\	長蛇の列
\\	(単2)‐/複数なし 人類 
\\	(単2)‐
\\	/(複1・2・4)‐(複3)‐
\\	発明者
\\	/[女性]-
\\	(単2)‐
\\	/複数なし 
\\	えさ
\\	【形】① 野生の 
\\	訪問 
\\	…⁴を訪問する
\\	【他】①〔部屋など⁴を〕換気する ④〔秘密など⁴を〕明かす
\\	[物]語ること
\\	私は彼の話に耳を傾けた 
\\	物語 [派生] 
\\	…⁴を物語る,話す
\\	(単2)‐
\\	/複数なし 
\\	血、血液
\\	1【他】①
\\	〔…⁴を…
\\	から〕守る 
\\	(または2/
\\	一種の…1・2・3、…1・2・3のようなもの
\\	血管 
\\	②⦅地学⦆ 鉱脈
\\	【自】機能する;作動する
\\	(単2)‐
\\	/(複1・2・4)
\\	(複3)
\\	音 
\\	発見 
\\	Ⅰ【形】① 目の見えない、盲目の
\\	に対して目を閉ざす〈耳をふさぐ〉
\\	1【他】①〔…⁴を〕燃やす 
\\	3【再】
\\	やけどする
\\	別れ
\\	にとって…⁴が重要である(㊟
\\	は形式主語)
\\	1【他】①〔…⁴を〕形作る、作る2【再】①
\\	できる、生じる
\\	【他】〔…⁴を〕阻止する
\\	1【他】②〔数値・価格など⁴を〕下げる
\\	①〔複数なし〕(果物・料理などの)酸味
\\	⦅生物⦆(無性増殖の)クローン 
\\	〔過去〕
\\	〔過分〕
\\	【自】《3格》
\\	に〕似ている 
\\	彼の気性は父親にそっくりだ
\\	(単2)‐/(複1・2・4)
\\	(複3)
\\	果実;果物
\\	【他】①〔…⁴を〕起こす
\\	②〔感情など⁴を〕目覚めさせる、呼び起こす
\\	うわさ 
\\	【形】① 親切な、好意的な;友好的な
\\	親切な人
\\	【形】信頼できる、頼りになる
\\	1【他】①〔…⁴を〕元の場所へ戻す
\\	【他】①
\\	〔…⁴に…⁴をくれるよう〕頼む
\\	決定 
\\	(単2)‐
\\	/(複1・2・3・4)
\\	または
\\	事典 
\\	②⦅口語⦆辞典、辞書
\\	1【他】〔…⁴を〕納得させる、確信させる
\\	3と〗
\\	私は彼に私の意見が正しいということを納得させた 2【再】
\\	について〕納得する、確信する
\\	酵母 
\\	(単2)‐
\\	/(複1・2・4)..
\\	(複3)..
\\	金額
\\	①(映画・演劇の)場面、シーン 
\\	1【他】②〔…⁴を〕引き開ける 
\\	カーテンを開ける 2【自】〔完了
\\	①(雷雨などが)近づいて来る
\\	湿度
\\	(単2)‐
\\	/(複1・2・4)-
\\	(複3)-
\\	薬 
\\	1【他】①〔…⁴を〕引き止める  
\\	これ以上お引き止め〈お邪魔〉はいたしません ②〔災害・発展など⁴を〕抑える 2【再】①
\\	場所》〔…に〕滞在する
\\	(単2)‐
\\	/(複1・2・4)-
\\	(複3)-
\\	進歩 
\\	(単2)‐
\\	/(複1・2・4)‐(複3)‐
\\	浴室、バスルーム
\\	【形】普通の 
\\	いつものように
\\	①(豚の)脂身;ベーコン 
\\	1【他】①〔飲食物⁴を〕出す 
\\	仕事 
\\	定職がない
\\	2【再】
\\	(自分の行為を)弁明する
\\	2【再】①
\\	謝る
\\	麻薬 
\\	分析 
\\	1【他】①《4格》〔…⁴に〕鍵を掛ける 
\\	【他】①〔…⁴を〕攻撃する 
\\	(単2)‐
\\	/(複1・2・3・4)‐
\\	(まれに(複1・2・4)-
\\	(複3)-
\\	(精神的な)ショック
\\	【他】②〔…⁴を〕(攻撃などから)守る
\\	【形】忍耐強い、我慢強い 
\\	(単2)‐
\\	/(複1・2・4)
\\	(複3)
\\	垣根、フェンス 
\\	【再】
\\	身をかがめる 
\\	【他】①〔…⁴を〕つかむ 
\\	1【形】① 乏しい、不十分な 
\\	少ない給料 
\\	かろうじて足りる、ぎりぎりの ④【付加語的に】《数量》
\\	弱の、
\\	足らずの
\\	【他】〔…⁴を〕殺す 
\\	1【他】①⦅文語⦆〔…⁴を〕取り除く 
\\	2【再】
\\	遠ざかる、離れる
\\	2【再】①
\\	起き上がる
\\	(単2)‐
\\	/(複1・2・4)-
\\	(複3)-
\\	勝利 
\\	勝つ
\\	【形】粗暴な、残忍な 
\\	粗暴な振舞い 
\\	1【他】①〔…⁴を〕爆破する 
\\	(単2)‐
\\	/(複1・2・4)
\\	(複3)
\\	行進 
\\	Ⅰ【副】同時に Ⅱ【形】同時の
\\	①(一国の)軍隊 
\\	知力;分別 
\\	正気である
\\	弱変化;(単2・3・4)‐
\\	/(複1・2・3・4)‐
\\	作曲家 
\\	/[女性]-
\\	(単2)‐
\\	/(複1・2・4)-
\\	(複3)-
\\	才能 
\\	幼年時代、子供の時〈ころ〉 
\\	激しい怒り、激怒 
\\	【形】勇敢な、勇ましい 
\\	④〔任務・責任など⁴を〕引き受ける 
\\	【自】〔完了
\\	⦅文語⦆ 帰る、戻る 
\\	【他】①〔…⁴を〕脅す 
\\	【副】今回は 
\\	今回彼のやり方は正しかった
\\	①〔複数なし〕見張り
\\	(単2)‐
\\	/(複1・2・4)-
\\	(複3)-
\\	小包
\\	(単2)‐
\\	/(複1・2・4)..
\\	(複3)..
\\	出入り、立ち入り
\\	【自】〔完了
\\	①〔…に〕着く 
\\	【他】①〔…⁴と〕推測する、(…だろうと)思う 
\\	(または
\\	…であろうと推測される
\\	【形】① 気絶した、失神した 
\\	気絶する
\\	【形】内気な、はにかみやの、シャイな 
\\	内気な女の子
\\	(単2)‐
\\	/(複1・2・4)..
\\	(複3)..
\\	起源、由来 
\\	1【他】①〔…⁴が〕正しいと認める 
\\	②⦅文語⦆〔手紙・商品などの受領⁴を〕通知する
\\	非分離 【他】〔家など⁴を〕くまなく探す、調べつくす 
\\	麻薬がないかと住まいをくまなく捜索する
\\	1【自】〔完了
\\	通り抜ける、通過する 
\\	の過去分詞)【形】決心した
\\	【形】① 公共の 
\\	公共の福祉
\\	【形】(自分の気持ちに)正直な、誠実な 
\\	…に対して誠実である 
\\	【他】①〔…⁴を〕(経済的・社会的に)破滅させる 
\\	2【他】①〔…⁴を〕調べる、確かめる
\\	(単2)‐
\\	/(複1・2・4)-
\\	(複3)-
\\	証明、証拠 
\\	②〔複数なし〕暴力
\\	1【他】①〔…⁴を〕こじ開ける 2【自】〔完了
\\	出発する
\\	1【自】〔完了
\\	逃げる 
\\	3格と】
\\	敵から逃げる
\\	復讐 
\\	思い出;記憶 
\\	…⁴を思い出す,覚えている
\\	死 
\\	【自】〔完了
\\	(兵士などが)行進する、行軍する、進軍する 
\\	①〔複数なし〕統一、まとまり 
\\	③⦅軍事⦆ 部隊
\\	【他】①〔…⁴を〕(神のご加護があるように)祝福する 
\\	神のご加護がありますように!(接続法?)
\\	世話  
\\	手入れ [派生] 
\\	…⁴の世話をする;…⁴の手入れをする
\\	①(一定期間の)天気、天候 ②⦅比喩⦆(動物などの)におい
\\	(単2)‐
\\	/(複1・2・4)..
\\	(複3)..
\\	(略 
\\	複数
\\	…年生まれの人  
\\	…年産のワイン
\\	【形】①
\\	に〕(成果などが)依存した、左右される 
\\	⦅文語⦆ 
\\	傾向 
\\	それにもかかわらず、それでもやはり 
\\	1【他】①〔…⁴を〕定める、決める 
\\	【副】① それゆえ、だから、したがって 
\\	成長、生長 
\\	たとえ…であっても
\\	生理的な;生理学[上]の 
\\	関連、つながり 
\\	【形】① 集中的な、徹底的な;⦅農業⦆集約的な 
\\	強烈な、激しい
\\	描くこと、描写 
\\	風景を描写する難しさ
\\	(単2)‐
\\	/(複1・2・4)-
\\	(複3)-
\\	シンボル、象徴 
\\	【他】①
\\	〔…⁴を…と〕呼ぶ、言う 
\\	彼は私を裏切り者と呼ぶ 
\\	名称
\\	【形】①
\\	に〕依存していない、左右されない 
\\	②〔…⁴を〕栽培する 
\\	情報
\\	1①【複数名詞と】
\\	いくつかの、二三の
\\	(単2)‐
\\	/(複1・2・4)-
\\	(複3)-
\\	牛
\\	2(ある用途向けに加工された)布 
\\	1【他】①〔…⁴を〕表現する 
\\	②〔…⁴を〕表わす
\\	(単2)‐
\\	/(複1・2・4)-
\\	(複3)-
\\	概念 
\\	理解
\\	【形】①(空間的に)上の、上方の 
\\	(単2)‐/(複1・2・3・4)
\\	基礎、基盤 
\\	努力 
\\	高さ 
\\	高み
\\	【副】①(あることに)先立って、前もって 
\\	原因を前もって解明しておく
\\	【他】〔難解・複雑なもの⁴を〕説明する 
\\	【形】全体の、全部の 
\\	全部で,合計して
\\	Ⅰ【形】本質的な 
\\	[派生] 
\\	本質
\\	【形】毎年の 
\\	(または
\\	ふつうは;通例は
\\	1【他】①〔…⁴を〕実行する、実施する 
\\	③[ダ・ゲーゲン] それに反して 
\\	彼女はビールは飲む、けれどもワインは飲まない
\\	新芽、若枝 
\\	【他】①《4格》〔…⁴に〕関係する 
\\	「決める」の現在分詞)【形】決定的な、重大な
\\	Ⅰ【前】(期間)…の間 
\\	・2格支配 Ⅱ【接】《従属》〖定動詞後置〗② …であるのに対して、一方…
\\	…も…も
\\	弱変化;(単2・3・4)‐
\\	/(複1・2・3・4)‐
\\	候補者 
\\	/〔女性〕-
\\	②(国家試験などの)受験者
\\	形 
\\	規則 
\\	規則, 規定, 法規)
\\	(単2)‐
\\	/(複1・2・3・4)
\\	エンジン、モーター
\\	2【自】給油する
\\	(単2)‐
\\	/(複1・2・4)-
\\	(複3)-
\\	出版社 
\\	(単2)‐
\\	/複数なし ⦅化学⦆水素 (〔記号〕
\\	(単2)‐/複数なし 嫉妬 
\\	恋敵への嫉妬
\\	【定冠詞と】⦅川名⦆ザール川(モーゼル川の支流)
\\	(単2)‐/(複1)‐
\\	ふつう複数で ⦅俗語⦆(女性の)乳房、おっぱい
\\	(単2)‐
\\	/(複1)-
\\	未発酵の果汁(特にぶどう汁) 
\\	音無	(単2)‐
\\	/(複1)..
\\	ブドウ園
\\	1【他】①〔…⁴を〕表彰する ②〔…⁴を〕際立たせる、目だたせる 
\\	表彰;賞
\\	表彰 
\\	賞 
\\	…⁴を表彰する
\\	(ワインの)自家醸造元詰
\\	②(特にワインを)試飲する
\\	ぶどうの収穫 
\\	条件 
\\	②〔ふつう複数で〕状況
\\	(単2)‐
\\	/(複1・2・4)
\\	(複3)
\\	契約
\\	【自】〔完了
\\	〔…⁴に〕達する、迫る;手が届く、〔入手困難な物⁴を〕手に入れる 
\\	混んでいてレジまで行けなかった
\\	(単2)‐
\\	/(複1・2・4)..
\\	(複3)..
\\	楽しみ、喜び ②〔複数なし〕⦅文語⦆飲食 
\\	【形】湿った、湿っぽい 
\\	〔過去〕
\\	〔過分〕
\\	【自】〔完了
\\	①(戸・窓が)開く ②(月・太陽などが)昇る 
\\	脂肪分の多い;脂っこい 
\\	②⦅軽蔑して⦆(人が)でぶの
\\	〔現在〕
\\	〔過去〕
\\	〔過分〕
\\	2【再】①
\\	道に迷う 
\\	電球 
\\	【他】〔部品など⁴を〕取り替える、交換する 
\\	お大事に!
\\	(相手の行為に対して)気になさらないでください;たいしたことありませんよ!
\\	(単2)‐
\\	/(複1・2・4)‐(複3)‐
\\	①⦅口語⦆ボーイ、ウエーター 
\\	ボーイさん!
\\	【副】
\\	の強調)そうですとも、もちろんですとも;かしこまりました
\\	⦅口語⦆(あれやこれやの仕事で)多忙をきわめている
\\	⦅口語⦆大目に見る(←片目をつぶる)
\\	3格 
\\	(または
\\	⦅口語⦆…
\\	の成功を祈る
\\	(単2)‐
\\	/(複1・2・4)‐(複3)‐
\\	所有者、持ち主(〔女性〕-
\\	歯に衣(きぬ)を着せない
\\	(または
\\	一部リクエスト中	⦅口語⦆かんかんに怒る
\\	(食事のときのあいさつで)どうぞ召し上がれ!;いただきます!
\\	【形】① 共通の 
\\	共同の、いっしょの
\\	(単2)‐
\\	/(複1・2・4)‐(複3)‐
\\	被雇用者、従業員 
\\	/〔女性〕-
\\	外国人労働者
\\	(単2)‐
\\	/(複1・2・4)‐(複3)‐
\\	雇用主、経営者 
\\	(単2)‐
\\	/(複1・2・4)
\\	(複3)
\\	広間、ホール 
\\	1【再】
\\	場所》〔…に〕集まる、集合する
\\	【自】① 研究する 
\\	(単2)‐/(複1・2・3・4)‐
\\	リスト、一覧表、名簿 
\\	(単2)‐
\\	/(複1・2・4)‐(複3)‐
\\	冒険 
\\	【形】① しらふの 
\\	彼は酔いがさめた 
\\	食事をとっていない 
\\	冷静な、客観的な
\\	【形】①(高さが)低い 
\\	(単2)‐
\\	/複数なし 血圧 
\\	【形】①(水深が)浅い 
\\	(単2)‐
\\	/(複1・2・4)..
\\	(複3)..
\\	表現 
\\	(単2)
\\	/(複1・2・4)
\\	(複3)
\\	キス、口づけ 
\\	①〔ふつう複数で〕関係 
\\	(単2)‐
\\	/(複1・2・4)
\\	(複3)
\\	性格 
\\	【他】《4格》〔…⁴に〕影響を及ぼす 
\\	3格 
\\	⦅文語⦆…
\\	に別れを告げる
\\	(単2)‐
\\	/(複1・2・4)‐(複3)‐
\\	大みそか
\\	(単2)‐
\\	/複数なし 新年、元旦
\\	(単2)‐
\\	/(複1・2・4)-
\\	(複3)-
\\	企業;会社、工場 
\\	(単2)‐
\\	/(複1・2・4)‐(複3)‐
\\	勝者;優勝者 
\\	/〔女性〕-
\\	【形】① 情熱的な 
\\	熱狂的な
\\	①(商品の)特売、安売り;特価品 
\\	行動、活動 
\\	行動を起こす
\\	⦅植物⦆ソバ 
\\	湯船、浴槽
\\	【他】①〔時⁴を〕過ごす 
\\	仏教 
\\	欲望、欲 
\\	原因;理由 
\\	(単2)‐
\\	/(複1)‐ ①(長期に及ぶ)病気 
\\	苦しみ
\\	1【他】①〔…⁴を〕自由にする、解放する 
\\	〔…⁴から不快なもの・厄介なもの
\\	を〕取り除く
\\	【他】①〔…⁴が〕いなくて寂しい、なくて残念だ 
\\	2【再】①
\\	リラックスする 
\\	緩む;休養〈息抜きを〉する
\\	それに反して
\\	(単2)‐
\\	/複数なし ふるまい、態度 
\\	の過去分詞)【形】①〖付加語的に〗遠く離れた、へんぴな ②〖述語的に〗《数量》〔…の〕距離にある 
\\	町はここから10キロメートル離れた所にある
\\	①〔ふつう複数で〕慣習、しきたり 
\\	(単2)‐
\\	/(まれに複
\\	語彙 
\\	(単2)‐
\\	/(複1・2・4)‐(複3)‐
\\	聞き手;(ラジオの)聴取者 
\\	/〔女性〕-
\\	【形】機械の、機械による 
\\	(単2)‐
\\	/(複1)-
\\	食卓 
\\	1【他】①〔…⁴を〕組み上げる、設営する 
\\	足場を組む
\\	満腹した、腹がいっぱいの
\\	1【他】①〔…⁴を〕溶かす 
\\	錠剤 
\\	(単2)..
\\	/(複1・2・4)..
\\	(複3)..
\\	誤解、勘違い
\\	(単2)‐
\\	(単3・4)‐
\\	/(まれに複1‐
\\	意志 
\\	〔過去〕
\\	〔過分〕
\\	〔接Ⅱ〕
\\	1【他】〔…⁴を〕かむ 
\\	【自】
\\	〔権利・要求など⁴を〕断念する 
\\	参加を断念する
\\	〔現在〕
\\	〔過去〕
\\	〔過分〕
\\	1【他】〔…⁴を〕含む、含有する 
\\	この瓶には2リットルのワインが入っている
\\	(重要な意味を持つ)事柄、用件 
\\	機会、チャンス
\\	【再】
\\	(事故・不幸などが)起こる、生じる 
\\	その爆発は早朝に起こった 
\\	1【自】
\\	〔…⁴を〕指し示す;指摘する 
\\	手でバスの停留所を指し示す
\\	苦労 
\\	弱変化;(単2・3・4)‐
\\	/(複1)‐
\\	消費者 
\\	【副】1《疑問》なぜ 
\\	(単2)‐/(複1・2・3・4)
\\	神話 
\\	(単2)‐/複数なし 
\\	生産、製造 
\\	【形】① 革命の 
\\	革命的な
\\	変化、変更 
\\	1【他】①〔職務・命令など⁴を〕実行する、遂行する、執行する 
\\	任務を遂行する 2【再】
\\	(変化などが)起こる
\\	【他】①〔提案・条件など⁴を〕受け入れる
\\	弱変化;(単2・3・4)‐
\\	/(複1・2・3・4)‐
\\	王子 
\\	/〔女性〕..
\\	1【再】
\\	と〕会う約束をする
\\	⦅文法⦆副文 
\\	【他】②〔薬⁴を〕服用する 
\\	③⦅文語⦆〔飲食物⁴を〕とる 
\\	朝食をとる
\\	ちょっと待ってください!
\\	【他】①
\\	〔…⁴に…
\\	を〕要請する
\\	【副】① その前に、以前に 
\\	【形】① 柔らかい 
\\	気分 
\\	②(その場の)雰囲気、ムード
\\	(単
\\	/(複
\\	(休暇用の)別荘 
\\	〔過去〕
\\	〔過分〕
\\	【自】〔完了
\\	①(太陽・月などが)沈む 
\\	②(水中に)沈む、沈没する
\\	1【自】〔完了
\\	①(時が)経過する、過ぎる 
\\	【他】〔…⁴を〕殺害する 
\\	【形】① 平和な;平和的な 
\\	穏やかな
\\	(単
\\	/(複1・2・4)
\\	(複
\\	慣習、習慣 
\\	Ⅰ【形】明白な、まぎれもない 
\\	Ⅱ【副】明らかに…らしい、どうやら…のようだ
\\	(単
\\	/(複1・2・4)‐(複
\\	案内人、ガイド 
\\	案内書、ガイドブック
\\	「見る」の過去分詞)【形】名声〈声望〉のある 
\\	その男はいたるところで名声を得ている
\\	業績、成績;成果 
\\	性能、能力;出力
\\	(単
\\	/(複1・2・4)
\\	(複
\\	畑 
\\	(単
\\	/(複1・2・3・4)‐(ドイツ南部では複
\\	も) トラック 
\\	〔現在〕
\\	〔過去〕
\\	〔過分〕
\\	【他】①
\\	〔…⁴を…⁴に〕積む 
\\	(単
\\	/(複
\\	批評家、評論家 (〔女性〕-
\\	1【他】①
\\	に…⁴を〕差し出す、提供する ②〔…⁴を〕紹介する
\\	【自】〔完了
\\	輝く、きらめく 
\\	際立つ;抜きんでている
\\	(単
\\	/複数なし 精密さ、正確さ 
\\	〔過去〕
\\	〔過分〕
\\	【自】〔完了
\\	眠り込む、眠りに落ちる
\\	(単
\\	/複数なし 変化、変遷、推移 
\\	[派生] 
\\	変わる;…⁴を変える
\\	表題;見出し 
\\	【副】運良く、幸いにも 
\\	〔比較〕-
\\	〔最上〕-
\\	または〔比較〕
\\	〔最上〕
\\	ぬれた、湿った 
\\	【形】(体つきなどが)すらりとした、ほっそりした 
\\	【他】
\\	に…⁴を〕秘密にする、隠す 
\\	(単
\\	/(複1)‐ 勝者、受賞者 (〔女性〕-
\\	(単
\\	/(複1・2・4)-
\\	(複
\\	じゅうたん、カーペット
\\	(単
\\	/(複1)‐
\\	⦅電気⦆コンセント 
\\	(単
\\	/(複1)-
\\	(携帯電話などの)充電器 
\\	風景
\\	…か…か[どちらか一方] 
\\	単
\\	/複
\\	(数詞のあとでは複
\\	ヘクタール(面積の単位、100アール)(〔記号〕
\\	【形】見る価値のある、一見に値する 
\\	1(単
\\	/(複1・2・4)
\\	(複
\\	圧力 
\\	詳しく〈具体的に〉言うと
\\	【他】①〔…⁴を〕展示する、陳列する 
\\	(単
\\	/(複1・2・4)-
\\	(複
\\	刺すこと;刺し傷 4格+
\\	…⁴を見捨てる;自分のもの⁴を放棄する
\\	【形】変わらない、元のままの 
\\	1【他】⦅口語⦆①〔…⁴を〕捕らえる、捕まえる 
\\	泥棒を捕まえる 2【非人称】⦅口語⦆①(…⁴に)(病気・けがなどの)不幸が襲いかかる 
\\	(単
\\	/(複1・2・4)‐(複
\\	放送局
\\	(単
\\	/(複1)-
\\	滞在 
\\	私がドイツに滞在している間に [派生] 
\\	場所》…に滞在する
\\	メニュー
\\	【形】① まじめな、本気の、真剣な 
\\	【形】① 絶え間ない、ひっきりなしの
\\	の過去分詞) 【形】びっくりした、驚いた 
\\	(単2)‐
\\	/(複1・2・3・4)‐ 
\\	タイヤ 
\\	②(樽・桶の)たが
\\	(単
\\	/(複1)-
\\	方言、訛り 
\\	方言を〈で〉しゃべる
\\	1(単2)‐
\\	/(複1・2・3・4)‐
\\	型、タイプ 
\\	その間に、そうこうするうちに
\\	【他】〔…⁴をある種類・体系など
\\	に〕分類する 
\\	の過去分詞)【形】人気のある、好かれている 
\\	【形】歓迎される、喜ばしい 
\\	[ほんとうに]ようこそ!、いらっしゃい! 
\\	【形】魅力的な、チャーミングな 
\\	(単
\\	/複数なし 魅力 
\\	(単
\\	/(複1・2・3・4)‐
\\	名誉、面目 
\\	私は彼の名誉を傷つけてしまった
\\	【形】一日中 
\\	全日〈終日〉の、フルタイムの 
\\	〔現在〕
\\	〔過去〕
\\	〔過分〕
\\	1【自】①(程度が)増す、増加〈増大〉する;強まる 
\\	(単
\\	/(複1)-
\\	イノシシ 
\\	脅し、脅迫;脅威 
\\	1【他】①〔…⁴を〕開業する、開設する、オープンさせる 
\\	(単
\\	/(複1・2・4)-
\\	(複
\\	禁止、禁令 
\\	駐車禁止
\\	【副】① 目下、今のところ
\\	(単
\\	/(複1)..
\\	市議会 
\\	(単
\\	/(複1・2・4)-
\\	(複
\\	(大規模な)計画、プロジェクト
\\	(単
\\	/(複
\\	水曜日 (〔略〕
\\	(単
\\	/(複
\\	ふつう複数で 時間外労働、超過勤務[時間] 
\\	【自】《3格》
\\	に〕賛成する、同意する 
\\	この点に関しては私は君に同意できない (イメージ:
\\	…に対して/適合+
\\	合う,適合する)
\\	(単
\\	/(複1・2・4)..
\\	(複
\\	委員会 
\\	①〔ふつう単数で〕管理 
\\	この地所は国の管理下にある
\\	(単
\\	/(複1・2・4)
\\	(複
\\	(公の)決定、決議 
\\	⦅文語⦆決議〈決定〉する 
\\	…⁴を決定する
\\	(単
\\	/(複
\\	菜食主義者 
\\	(単
\\	/複数なし 聴衆、観客 
\\	(単
\\	/(複1・2・4)- (複
\\	初心者 
\\	/〔女性〕-
\\	…⁴を始める
\\	(単
\\	/(複1・2・4)-
\\	(複
\\	①(気の利いた)小話;冗談、ジョーク 
\\	⦅口語⦆冗談を言うな!
\\	(単
\\	/(複
\\	(丘陵の斜面の)ブドウ畑 
\\	【形】非公式の
\\	Ⅰ【前】〖2格支配〗…の外に、…の外で 
\\	営業時間外 Ⅱ【副】(市などの一定の範囲の)外に、外で、郊外で
\\	平均して;概して
\\	(単
\\	/(複
\\	①(建築現場の)足場 
\\	(単
\\	/(複
\\	⦅ドイツ北部・東部で⦆土曜日 (〔略〕
\\	〔現在〕
\\	〔過去〕
\\	〔過分〕
\\	【自】①(人・郵便物・列車などが)着く、到着する 
\\	列車は時刻どおりにベルリンに着いた 
\\	【他】〔…⁴を〕賃借りする 
\\	【形】① 堅い、固い
\\	〔過去〕
\\	〔過分〕
\\	【自】〔完了
\\	中に入る
\\	(単
\\	/(複
\\	建築〈工事〉現場 
\\	【他】①《4格》〔…⁴に〕気づく 
\\	彼は歩行者に気づくのが遅すぎた [派生] 
\\	…⁴に気づく
\\	(単
\\	/(複
\\	(建築の)現場監督 
\\	【副】① 真ん中で、真ん中へ 
\\	…の最中に 
\\	真夜中に
\\	【他】〔…⁴を〕[再]検査する、点検〈チェック〉する 
\\	検算する 
\\	【形】感謝している 
\\	3格
\\	4格 
\\	に…⁴の事で感謝している
\\	(単
\\	/(複1・2・3・4)‐
\\	カテゴリー、範ちゅう (〔略〕
\\	【他】〔…⁴を〕必要とする 
\\	(単
\\	/(複1・2・4)-
\\	(複
\\	刀、剣 
\\	(単
\\	/(複
\\	魔法使い 
\\	【接】《並列》① または、あるいは 
\\	②【文末で;尻上がりのアクセントで】…でしょう 
\\	彼の言うとおりだ、そうじゃないかい?
\\	2【他】①《4格》〔…⁴に〕別れを告げる
\\	(単
\\	/(複1・2・4)-
\\	(複
\\	平均 
\\	①〔複数なし〕評価 
\\	(単
\\	/(複
\\	②〔ふつう複数で〕箸 
\\	Ⅰ【形】質素な、簡素な 
\\	質素な食事
\\	【他】〔…⁴に〕質問をする、〔…⁴を〕尋問する 
\\	(単
\\	/(複
\\	終着駅、終点 
\\	1【他】①〔…⁴を〕苦しめる、いじめる 
\\	弱変化;(単2・3・4)‐
\\	/(複1・2・3・4)‐
\\	⦅航空⦆パイロット (〔女性〕-
\\	(単
\\	/(複
\\	万引き[する人] [派生] 
\\	泥棒
\\	(単
\\	/複数なし 不正、不当 
\\	彼は私に対して不正を働いた
\\	(単
\\	/(複1・2・4)..
\\	(複
\\	宿、宿泊所 
\\	一夜の宿を探す
\\	(単
\\	/(複1・2・4)..
\\	(複
\\	盗み、窃盗、盗難
\\	Ⅰ【他】〔…⁴を〕調達する、手に入れる 
\\	入手するのが難しい
\\	(単
\\	/(複1・2・4)-
\\	(複
\\	傘
\\	(単
\\	/(複
\\	会計係、レジ係 (〔女性〕-
\\	(単
\\	/(複1・2・4)
\\	(複
\\	塔、タワー 
\\	(単
\\	/(複1・2・3・4)
\\	競技場、スタジアム 
\\	(単
\\	/(複1・2・3・4)‐
\\	教会 
\\	(単
\\	/(複
\\	否定;⦅言語⦆否定[詞] 
\\	【他】〔…⁴を〕否定〈否認〉する 
\\	【形】① 快適でない 
\\	煩わしい、厄介な
\\	1【他】①〔…⁴を〕分け与える、分配する、配る 
\\	2【再】
\\	分散する、広がる
\\	1【他】①〔…⁴を〕回す、回転させる 
\\	②〔…⁴を〕裏返す 2【再】
\\	くるりと向きを変える
\\	【他】
\\	〔…⁴を…⁴に〕浸す、つける 
\\	給油する
\\	(単
\\	/(複1・2・4)-
\\	(複
\\	脳 
\\	(単
\\	/(まれに複
\\	ユーモア;しゃれ
\\	つまり;言い換えれば (〔略〕
\\	【自】〔完了
\\	⦅文語⦆① 育つ、成長する 
\\	植物〈子供たち〉がすくすくと育つ
\\	1【他】①〔…⁴を〕乾かす 
\\	③〔食品⁴を〕乾燥させる
\\	(単
\\	/(複1・2・3・4)‐
\\	①〔複数なし〕愛、愛情;恋、恋愛 
\\	(単
\\	/(複1・2・4)-
\\	(複
\\	テキスト、テクスト(内容上ひとつのまとまりを成す文章)
\\	歌詞
\\	【他】
\\	〔…⁴にプレゼントなど
\\	を〕山ほど与える 
\\	彼はおびただしい量の仕事を背負わされている
\\	【再】
\\	ほれる、夢中になる、恋をする
\\	(単
\\	/(複
\\	演出家;(映画の)監督 
\\	(単
\\	/(複1・2・4)- (複
\\	見物人、観客 (〔女性〕-
\\	【形】すばらしい 
\\	(単
\\	/(複
\\	⦅貝⦆カキ 
\\	【他】①〔…⁴を〕強調する 
\\	②〔語・音節⁴に〕アクセントを置く
\\	Ⅰ【接】《従属》【定動詞後置】…以来、…してから Ⅱ【副】それ以来
\\	(単
\\	/(複1・2・4)-
\\	(複
\\	①⦅言語⦆アクセント 
\\	行 (〔略〕
\\	②〔複数で〕(短い)手紙、便り
\\	【形】①(顔などが)青ざめた、そう白の 
\\	彼女は心配のあまり真っ青になった
\\	(単
\\	/(複
\\	停止;休止 
\\	(単
\\	/(複
\\	花火 
\\	(単
\\	/(複
\\	クレジットカード 
\\	(単
\\	/(複
\\	パジャマ 
\\	ソックス 
\\	(単
\\	/複数なし シャンパン 
\\	(単
\\	/(複1・2・4)
\\	(複
\\	①(雌雄を区別せずに)鶏 
\\	めんどり 
\\	1(単
\\	/(複1・2・4)
\\	(複
\\	①〔複数なし〕穀物 
\\	(単
\\	/(複1・2・3・4)‐
\\	オペラ、歌劇 
\\	(単
\\	/(まさに複
\\	地獄 
\\	(単
\\	/(複1・2・3・4)‐ 家ウサギ、飼いウサギ 
\\	〔過去〕
\\	〔過分〕
\\	臭い、嫌なにおいがする 
\\	3格と】
\\	彼は酒臭い
\\	運良く;幸運にも
\\	1【自】⦅口語⦆(ゲームなどで)いんちき〈ごまかし〉をする 
\\	(単
\\	/(複
\\	不法行為 
\\	〔過去〕
\\	〔過分〕
\\	【自】〔完了
\\	①(液体・気体などが)流れる 
\\	逃げる
\\	豆 
\\	コーヒー豆 
\\	【形】はだしの、素足の
\\	(単
\\	/(複1・2・4)..
\\	(複
\\	あこがれ、切望 
\\	(単
\\	/(複
\\	チップ、心づけ 
\\	(単
\\	(複1・2・4)
\\	(複
\\	ソーセージ 
\\	(単
\\	/(複1・2・4)-
\\	(複
\\	双子 
\\	二人は双子だ 
\\	(単
\\	/(複
\\	主演俳優、主演男優 (〔女性〕-
\\	(単
\\	/(複
\\	戦場 
\\	1【他】①〔…⁴を〕感激させる、熱狂させる 
\\	聴衆〈観衆〉を感激させる
\\	翻訳 
\\	【他】①〔…⁴の〕代わりをする 
\\	彼女の母親の代わりをできる者はいない
\\	とりわけ、特に
\\	(単
\\	/(複1・2・4)..
\\	(複
\\	間違い、誤り 
\\	1【他】①〔…⁴を〕(ほかの場所に)置き換える、移す 
\\	家具の配置を変える ②〔…⁴を〕切り替える 
\\	4格と〗
\\	時計を夏時間に切り替える
\\	(単
\\	/(複
\\	大豆 
\\	(単
\\	/(複1・2・4)- (複
\\	カレンダー、暦 
\\	音無	【形】[その]当時の
\\	1【再】
\\	〔…3・4に〕順応する、適応する
\\	1【自】呼吸する、息をする 
\\	(単
\\	/(複1・2・3・4)‐
\\	または(複1・2・4)-
\\	(複
\\	マフラー
\\	(単
\\	/(複1・2・4)- (複
\\	長靴、ブーツ
\\	(単
\\	/(種類を表わすときのみ複
\\	ゴム 
\\	(単
\\	/(複
\\	ふつう複数で) 宿題 
\\	(単
\\	/(複1・2・4)- (複
\\	孫 
\\	/〔女性〕-
\\	【形】① 卑しい、卑劣〈下劣〉な;下品な 
\\	②⦅口語⦆ひどい
\\	(単
\\	/(複
\\	(平底の)サンダル 
\\	〔現在〕
\\	〔過去〕
\\	〔過分〕
\\	【他】⦅口語⦆〔衣類など⁴を〕身に付けている、着ている 
\\	彼は革のズボンをはいている 
\\	(単
\\	/(複
\\	⦅医学⦆ふくらはぎ 
\\	音無	1【他】〔…⁴を〕引き上げる 
\\	(単
\\	/(複
\\	日焼け 
\\	Ⅰ【前】(空間)…に沿って 
\\	・名詞の後ろに置かれることもある ・名詞の後ろに置かれる場合4格支配、まれに3格支配・名詞の前に置かれる場合3格支配、まれに2格支配・分離前つづりかどうかに注意
\\	(単
\\	/(複
\\	プール[施設] 
\\	プールに行く
\\	【形】① 理想的な
\\	【形】魅力的な、人を引きつける 
\\	一回り 
\\	②(ボクシングの)ラウンド;…回戦;(トランプの)一勝負 
\\	仲間、(親しい人の)グループ 
\\	新しい仲間をグループに入れる 
\\	樹皮
\\	【形】⦅文語⦆非合理[的]な、不合理な 
\\	1〔不変化〕①〖無冠詞で〗西 (〔略〕
\\	(単
\\	/(複1・2・4)-
\\	(複
\\	または(複1・2・3・4)‐
\\	カーニバル(仮装行列・山車(だし)が出て飲めや歌えやのお祭り騒ぎをする)
\\	(単
\\	/(複
\\	断食期間 
\\	⦅口語⦆羽目をはずす〈はずしている〉;はしゃぐ〈はしゃいでいる〉
\\	【副】昼間、日中 
\\	(単
\\	/(複1・2・4)- (複
\\	市長、町長、村長 
\\	/〔女性〕-
\\	皮肉、風刺 
\\	毎回
\\	(単
\\	/(複1・2・4)-
\\	(複
\\	①〔複数なし〕〖無冠詞で〗知らせ
\\	〔過去〕
\\	〔過分〕
\\	1【他】①〔…⁴を〕追い立てる、駆り立てる 
\\	彼女は夫を家から追い出した ④〔仕事・活動など⁴を〕行なう  
\\	くだらないことをする 2【自】①〔完了
\\	または
\\	漂う、漂流する
\\	(単
\\	/(複
\\	サーカス[一座] 
\\	サーカスを見に行く
\\	(単
\\	/(複
\\	(サーカスなどの)クラウン、道化師 
\\	【自】〔完了
\\	鳴り出す
\\	一度に、同時に;突然
\\	【他】〔…⁴を〕笑い物にする、嘲笑する 
\\	(笑い者になるような)ばかなことはするな!
\\	(単
\\	/複数なし 
\\	あざけり、嘲笑 
\\	⦅口語⦆…も同様
\\	(単
\\	/(まれに複
\\	皮肉、風刺 
\\	(単
\\	/(複
\\	ブロンド〈金髪〉の女性 
\\	(単
\\	/(複1・2・3・4)‐ ダース、12個 (〔略〕
\\	(単
\\	/(種類を表わすときのみ複
\\	はちみつ 
\\	(単
\\	/(複1・2・4)
\\	(複
\\	①(殻の硬い)木の実;(特に)クルミ;ナッツ 
\\	(単
\\	/(複
\\	ズボン下、パンツ 
\\	(単
\\	(時に-
\\	/(複1・2・4)
\\	(複
\\	ボール、球 
\\	舞踏会 
\\	人形
\\	(単
\\	/(複
\\	または-
\\	香水 
\\	香水をスプレーする〈つける〉
\\	【形】生態学的な、エコロジーの 
\\	【他】〔…⁴を〕魅惑〈魅了〉する 
\\	人をひきつける〈魅力的な〉人物
\\	①(変化・発達の)段階、時期
\\	スケッチ、挿し絵 
\\	【形】① 完全な 
\\	全部そろっている 
\\	【形】① 論理的な、論理にかなった 
\\	【形】⦅文語⦆愚かな、ばかな 
\\	愚かなふるまい 
\\	【他】①〔罪・過失など⁴を〕犯す 
\\	愚かなことをしでかす
\\	1【他】①〔食料品⁴を〕冷凍する 
\\	(単
\\	/(複1・2・4)- (複
\\	バケツ 
\\	まず最初に
\\	1【他】①〔…⁴を〕予告する 
\\	催し物の予告をする
\\	身ぶり、しぐさ、ジェスチャー 
\\	②(行為による)意思の表示
\\	砂漠 
\\	【他】〔…⁴を〕無視する 
\\	【形】1
\\	完全な;全部そろった 
\\	これで支度は整った
\\	(単
\\	/(複
\\	プラスチック、合成樹脂 
\\	(単
\\	/(複
\\	電気かみそり、シェーバー 
\\	(単
\\	/複数なし 釣り銭;小銭 
\\	1【自】① どん〈ばん〉という音を立てる 
\\	〘北部・中部〙②
\\	(酒場などの)カウンター 
\\	(単
\\	/(複1・2・4)
\\	(複
\\	鋼鉄、鋼(はがね) 
\\	ステンレススチール
\\	(単
\\	/(複
\\	(高速道路などの)ドライブイン 
\\	遮断機 
\\	遮断機を下ろす 
\\	(単
\\	/(複
\\	パーキングビル 
\\	①〔ふつう単数で〕きれいにすること;清掃
\\	Ⅰ【形】自明の、当然の 
\\	Ⅱ【副】もちろん、当然、言うまでもなく 
\\	(単
\\	/複数なし 〖集合的に〗従業員
\\	(または
\\	時間をかける、ゆっくり行なう
\\	1【自】通訳をする 
\\	彼は会議で通訳をした
\\	(単
\\	/(複
\\	[州]知事 
\\	【形】愛情に満ちた;心のこもった 
\\	(単
\\	/(複
\\	(食器・花瓶・置物などの)下敷き、受け皿、コースター 
\\	【再】
\\	感謝する 
\\	3格 
\\	4格 
\\	に…⁴を感謝する
\\	(単
\\	/(複
\\	故郷の町、ふるさとの町 
\\	〔形容詞変化〕青少年;⦅法律⦆少年(14歳以上18歳未満) 
\\	(単
\\	/(複
\\	文字 
\\	【他】〔…⁴を〕選び出す 
\\	【形】デジタル[方式]の 
\\	【形】社会の、社会的な 
\\	〔形容詞変化〕親戚、親族、血族 
\\	②(家畜・作物の)品種
\\	【形】ぎごちない、不器用な 
\\	危なっかしい動作 
\\	(単
\\	/(複1・2・4)- (複
\\	受取人
\\	【副】確かに、きっと 
\\	彼はきっと来るだろう
\\	(単
\\	/(複
\\	教室 
\\	リクエスト中	【他】〔新聞・手紙など⁴を〕折りたたむ 
\\	(または
\\	いまだになお、相変わらず
\\	【形】電気の、電気による 
\\	電子工学の;電子的な
\\	【形】①[圧搾]空気による、空気圧[式]の 
\\	【形】何重もの;何回もの;何倍もの 
\\	幾重にも折り畳まれた紙
\\	【形】③ 古典的な、伝統的な
\\	リクエスト中	Ⅰ【自】
\\	木質化する
\\	(単
\\	/(複
\\	ペンフレンド (〔女性〕-
\\	弱変化;(単2・3・4)‐
\\	/(複1・2・3・4)‐
\\	(新聞・テレビなどの)特派員 
\\	/〔女性〕-
\\	【形】国際的な、国際間の
\\	【自】〔完了
\\	または
\\	ジョギングをする 
\\	(単
\\	/(複
\\	②⦅俗語⦆略歴;(製品などの)簡単なデータ 
\\	(単
\\	/(まれに複
\\	起源、由来 
\\	生まれ 
\\	彼女は生まれはフランスだ。
\\	弱変化;(単2・3・4)‐
\\	/(複
\\	または(単
\\	/(複
\\	⦅鳥⦆オウム 
\\	小屋 
\\	(単
\\	/(複
\\	家主;貸し主 (〔女性〕-
\\	(単
\\	/(複1・2・4)-
\\	(複
\\	①(旅館・飲食店の)主人、マスター (〔女性〕-
\\	(単
\\	/(複1・2・4)-
\\	(複
\\	頂点、ピーク;クライマックス
\\	繰り返し;再三再四
\\	もう一度 
\\	もう一度お願いします
\\	〔過去〕
\\	〔過分〕
\\	【自】〔完了
\\	(物事が)成功する、うまくいく 
\\	私は彼を説得することができなかった
\\	音無	(単
\\	/(複
\\	ワイングラス 
\\	(単
\\	/(複
\\	薬剤師 (〔女性〕-
\\	【副】半日[の間] 
\\	【自】②
\\	〔…⁴に〕当てはまる、該当する 
\\	それは彼〈このケース〉に当てはまることだ (イメージ: 
\\	…に対して/適合〉+
\\	当たる)
\\	4格+
\\	…⁴に関しては、関して言えば
\\	〔過去〕
\\	〔過分〕
\\	【他】①〔犯行など⁴を〕自白する、白状する 
\\	盗みを自白する 
\\	に…⁴を〕打ち明ける、告白する
\\	(単
\\	/(複1・2・4)
\\	(複
\\	泡、あぶく 
\\	(単
\\	/(複
\\	スパークリングワイン 
\\	とりわけ 
\\	最後に、いちばんあとに)
\\	(単
\\	/(複
\\	言い回し [派生] 
\\	話す,しゃべる
\\	【副】① なんとかして、なんらかの方法で 
\\	なんとなく、どことなく
\\	弱変化;(単2・3・4)‐
\\	/(複1)‐
\\	交換留学生 (〔女性〕-
\\	反応する、答える 
\\	(単
\\	/(複1・2・3・4)‐
\\	〘若者語〙①(特に異性との)待ち合わせ〔の約束〕、デート 
\\	会う約束 
\\	会う約束をキャンセルする 
\\	と〕会う約束をする
\\	ケースバイケースで
\\	(単
\\	/(複1・2・4)
\\	(複
\\	場合
\\	【副】いつかある時に 
\\	【形】税の、税制上の 
\\	税負担
\\	リクエスト中	弱変化;(単2・3・4)‐
\\	(まれに単
\\	/(複
\\	(食卓で)隣席の人 
\\	(単
\\	/(複
\\	樹皮 
\\	(単
\\	/(複
\\	⦅魚⦆コイ 
\\	よその;外国の 
\\	〔現在〕
\\	〔過去〕
\\	〔過分〕
\\	1【他】①〔…⁴を〕述べる、挙げる 2【自】①⦅口語⦆自慢する;ほらを吹く 
\\	(単
\\	/(複1・2・4)-
\\	(複
\\	池、沼 
\\	〔現在〕
\\	〔過去〕
\\	〔過分〕
\\	1【他】①〔穴・溝など⁴を〕掘る 
\\	【副】① 最近、近ごろ 
\\	(単
\\	/(複
\\	傾向、動向;流行、トレンド 
\\	(単
\\	/(複1・2・4)
\\	(複
\\	花嫁、新婦 
\\	(単
\\	/(複
\\	封筒 
\\	(単
\\	/(複1・2・4)‐(複
\\	差出人、発送人 (〔略〕
\\	(単
\\	/(複1・2・3・4)‐
\\	(男性の)いとこ、従兄弟 
\\	(単
\\	/(複
\\	(女性の)いとこ 
\\	〔過去〕
\\	〔過分〕
\\	1【自】〔完了
\\	(まれに
\\	(馬などに)乗る 
\\	目下のところ;⦅口語⦆今すぐに
\\	の過去分詞)【形】離婚した 
\\	2【再】
\\	状態》〔…のような〕手触りである 
\\	①〔複数なし〕安全 
\\	②〔複数なし〕信頼性;確実性、確かさ
\\	【形】壮大な、雄大な 
\\	(単
\\	/(複
\\	提供者;プロバイダー 
\\	【形】入手〈購入〉できる 
\\	「あとについて行く」の現在分詞)【形】次の;以下の 
\\	【他】①
\\	〔…⁴は…
\\	の〕お陰である、〔…⁴を…
\\	に〕負うている 
\\	(単
\\	/(まれに複
\\	食欲 
\\	食欲がない
\\	の過去分詞)【形】恋をしている、ほれ込んだ、熱愛している 
\\	彼は彼女に恋をしている 
\\	ほれる,夢中になる,恋をする
\\	(単
\\	/(複1・2・3・4)‐ 
\\	滴 
\\	【形】古風な、時代〈流行〉遅れの 
\\	(単
\\	/(複1・2・3・4)‐
\\	伝統、慣習、しきたり
\\	(単
\\	/(複
\\	結婚指輪 
\\	【再】
\\	と〕婚約する
\\	(単
\\	/(複1・2・3・4)‐
\\	結婚、結婚生活 
\\	【形】夫〈妻〉を亡くした (〔略〕
\\	(単
\\	/(複
\\	順序、順番 
\\	Ⅰ【形】① 楽しげな;上機嫌の、陽気な 
\\	楽しそうな雰囲気
\\	1【再】
\\	精神を集中する 
\\	【形】⦅口語⦆丸みのある、ふくよかな、ふっくらした 
\\	〔比較〕-
\\	〔最上〕-
\\	または〔比較〕
\\	〔最上〕
\\	【形】① 幅の狭い 
\\	狭い道路
\\	【形】① 角(かど)のある、角ばった
\\	【副】その度[ごと]に、そのつど;それぞれ
\\	(単
\\	/(複
\\	婚礼、結婚式 
\\	4格+
\\	…⁴を公表する
\\	生まれる、誕生する
\\	【形】
\\	に〕同意した、了解した、賛成の 
\\	私は彼の提案にまったく同意できない
\\	1【自】《3格》
\\	を〕信用する、信頼する 
\\	2【再】②
\\	方向》勇気を出して〔…へ〕行く、入る3【他】(牧師・戸籍係などが)〔…⁴の〕結婚式を執り行なう 
\\	彼らは教会で結婚式を挙げた
\\	(単
\\	/(複
\\	婚礼の前夜(家の前で陶器などを割って騒ぎたてそれを新郎が片付けると幸福になるという風習) 
\\	弱変化;(単2・3・4)‐
\\	/(複
\\	⦅口語⦆ ばか、あほう 
\\	(単
\\	/(複
\\	夫婦 
\\	【副】① 過ぎ去って、終わって 
\\	危険は去った
\\	〔現在〕
\\	〔過去〕
\\	〔過分〕
\\	1【他】①〔…⁴を〕粉々に壊す、砕く 
\\	(単
\\	/(複
\\	ブライダルブーケ 
\\	(単
\\	/(複
\\	ウェディングドレス、花嫁衣装 
\\	(単
\\	/(複
\\	戸籍役場 
\\	音無	弱変化;(単2・3・4)‐
\\	/(複
\\	結婚立会人 
\\	【形】①(テーマなどが)今日的な、現在問題となっている
\\	【他】①〔…⁴を〕加工する ②〔原稿・本⁴に〕手を加える
\\	【他】①〔郵便など⁴を〕届ける、配達する 
\\	【形】以下に述べる、次の 
\\	1【他】②〔ブローチなど⁴を〕(針・ピンで)留める、つける、〔指輪⁴を〕はめる 
\\	彼を彼女に指輪をはめてやった
\\	(単
\\	/(複1・2・4)- (複
\\	(プロテスタントの)牧師 
\\	①(クリームや果物などを載せた円形の)ケーキ、タルト
\\	Ⅰ【形】① 巨大な;ものすごい 
\\	【形】[およそ]考えつく限りの 
\\	(単
\\	/(複1・2・3・4)‐ 
\\	幸福、幸運 
\\	君が来てくれてありがたい
\\	(遊びに行く人に対して)おおいに楽しんでおいで!
\\	⦅口語⦆(何事かが)起きている、生じている 
\\	いったいどうしたんだ?
\\	(単
\\	/(まれに複
\\	咳(せき) [派生] 
\\	咳をする
\\	(単
\\	/複数なし インフルエンザ 
\\	グループ、集団
\\	【他】〔…⁴に〕感嘆する、感心する 
\\	賞賛,感嘆
\\	【形】垂直の 
\\	【形】水平の 
\\	(単
\\	/(複1・2・4)- (複
\\	ダイニングルーム、食堂 
\\	(単
\\	/(複
\\	事務机、デスク 
\\	(単
\\	/(複
\\	保証金、敷金 
\\	(単
\\	/(複
\\	(比較的大きな)賃貸マンション 
\\	(単
\\	/(複1・2・4)-
\\	(複
\\	①(壁に作りつけの)暖炉 
\\	(単
\\	/(複
\\	市〈町〉の中心部、都心 
\\	(単
\\	/複数なし 大理石 
\\	【形】① 銀製の、銀の 
\\	(単
\\	/(複
\\	または-
\\	(水道の)蛇口 
\\	【形】① 薄い 
\\	1【再】①
\\	と〕(楽しく)話をする、談笑する 
\\	私は彼と話をするのが好きだ 
\\	(単
\\	/(複1・2・4)
\\	(複
\\	膝 
\\	1【他】①〔部屋・家など⁴に〕家具を備えつける、設備を取り付ける
\\	音無	(単
\\	/複数なし 住まい〈部屋〉探し 
\\	【形】住み心地の良い、(家・部屋などが)快適な 
\\	(単
\\	/(複
\\	古い建物;旧館 
\\	音無	(単
\\	/(複
\\	システムキッチン 
\\	(単
\\	/(複
\\	(まれに複
\\	①(交通の)渋滞 
\\	ほこり、ちり
\\	(単
\\	/(複
\\	乗車〈乗船〉券 
\\	(単
\\	/(複
\\	(ホテルなどの)一人部屋、シングルルーム、(病院などの)個室 
\\	【自】① ほほえむ、微笑する 
\\	(単
\\	/(複
\\	非喫煙者 
\\	喫煙者
\\	【形】① 中心の、中央の、中心部の 
\\	(単
\\	/(複
\\	(住宅の)居住面積 
\\	【前】〖2格支配〗⦅商業⦆(別料金の)…抜きで、…プラス (〔略〕
\\	金額は運送料と保険料別料金で250ユーロです
\\	【複】付帯経費〈費用〉、雑費 
\\	(単
\\	/(複
\\	①⦅商業⦆(販売・仲介の)手数料 
\\	10パーセントの手数料をもらう 
\\	(単
\\	/(複
\\	賃貸借契約 
\\	(単
\\	/(複
\\	屋階(おつかい)(屋根の内部に設けた最上階、屋根裏部屋) 
\\	(単
\\	/(複
\\	(まれに複
\\	①〔複数なし〕新築 
\\	(単
\\	/(複
\\	1か月の賃貸料〈家賃〉 
\\	(単
\\	/(複
\\	①(快適な単身者用)アパート、(小さめの)マンション 
\\	〘話〙暖房費こみの家賃 
\\	(単
\\	/(複1・2・4)
\\	(複
\\	①〔複数なし〕サッカー 
\\	(単
\\	/(複1・2・4)- (複
\\	うそつき 
\\	/〔女性〕-
\\	【他】②
\\	に学位・勲章・称号など⁴を〕授ける、授与する 
\\	…にドクターの学位を授ける
\\	音無	Ⅱ【他】
\\	勘定に入れる、加算〈算入〉する 
\\	(単
\\	/(複1・2・4)
\\	(複
\\	①〔複数なし〕電流、電気 
\\	わら、麦わら 
\\	(単
\\	/(複1・2・4)
\\	(複
\\	火事、火災 
\\	火事が発生する 
\\	(単
\\	/(複
\\	象徴、シンボル 
\\	【他】①〔…⁴を〕返してもらう、取り戻す 
\\	3【自】①(住居を)引き払う、引っ越して行く 
\\	私はこの家を引っ越すつもりだ ②(グループで)出かける;(特定の目的をもって)出て行く、出発する
\\	(単
\\	/複数なし ごみ、くず;廃棄物 
\\	保険;保険会社;保険料 
\\	生命保険
\\	(単
\\	/(複1・2・3・4)‐
\\	①(公共の)料金 
\\	1【他】①
\\	に…⁴を〕提供する 
\\	[派生] 
\\	に…⁴を差し出す、勧める
\\	(単
\\	/複数なし 登山 
\\	(単
\\	/(複
\\	旅行案内書、ガイドブック 
\\	(単
\\	/(複
\\	⦅文語⦆ 結婚 
\\	【他】①〔建物など⁴を〕建てる、建設する 
\\	記念碑を建てる
\\	〔過去〕
\\	〔過分〕
\\	1【他】①〔…⁴を〕持ち上げる、上げる 
\\	2【再】①
\\	立ち上がる 
\\	蜂起する
\\	1【他】〔…⁴を〕誓う 
\\	(単
\\	/(複1・2・4)- (複
\\	れんが 
\\	ヤギ 
\\	(単
\\	/(複1・2・4)
\\	(複
\\	印象、感銘 
\\	【形】威厳に満ちた、品位のある 
\\	同行、同伴 
\\	同伴者
\\	(単
\\	/複数なし 世話、看護 
\\	動き;運動 
\\	(単
\\	/(複
\\	①(馬などの)乗り手、騎手 (〔女性〕-
\\	(単
\\	/複数なし 
\\	つきあい、交際;交際相手 
\\	3格 
\\	(または
\\	とつきあいがある
\\	(単
\\	/(複
\\	①(ロケット・大砲などの)発射 [派生] 
\\	射撃,発砲 
\\	(単
\\	/(複
\\	パートナー (〔女性〕-
\\	【他】①〔…⁴を〕征服する、攻略する、占領する 
\\	②〔…⁴を〕獲得する
\\	(単
\\	/(複
\\	(報道などに対する読者・視聴者の)投書 
\\	①〔複数なし〕展示、陳列 
\\	展覧会、展示会 ③〔複数なし〕(証明書類の)発行、交付 
\\	…⁴を展示する
\\	(単
\\	/(複
\\	ふつう複数で 予備知識 
\\	(単
\\	/(複
\\	(新聞・雑誌などの)編集者 
\\	(単
\\	/複数なし 3食付きの宿泊 
\\	(単
\\	/複数なし わら、麦わら 
\\	(単
\\	/(複
\\	①〔複数なし〕(心身の)状態、体調 
\\	挑発;挑戦 
\\	挑戦に応じる 
\\	(単
\\	/(複1・2・4)-
\\	(複
\\	船長 
\\	/〔女性〕-
\\	(単
\\	/(複
\\	日の入り、日没 
\\	(単
\\	/(複
\\	巡航、クルージング 
\\	(単
\\	/(複
\\	高層建築物、高層ビル 
\\	(単
\\	/(複1・2・4)..
\\	(複
\\	(まれに複
\\	記念碑、記念像 
\\	1【他】〔…⁴の〕写真を撮る 
\\	(単
\\	/(まれに複
\\	①(人の)エネルギー 
\\	②(自然界の)エネルギー
\\	(単
\\	/複数なし 換気 
\\	照明
\\	周辺[の地域]、近郊、郊外 
\\	環境;周囲の人々
\\	多面的な、多方面にわたる、幅の広い
\\	【他】③〔…⁴を〕排除する
\\	2【自】練習〈トレーニング〉をする 
\\	義務、責務 
\\	…⁴に…
\\	を義務づける
\\	(単
\\	/複数なし ぜいたく 
\\	(または
\\	ときどき
\\	2【自】休養〈休息〉する 
\\	(単
\\	/(複
\\	または..
\\	サウナ 
\\	サウナに行く 
\\	【他】①〔子供・病人など⁴の〕世話をする、めんどうを見る 
\\	世話,看護
\\	恋愛関係、肉体関係 
\\	(単
\\	/(複
\\	(台所の)流し台[セット] 
\\	(単
\\	/(複
\\	[自動]食器洗い機 
\\	【形】① 目下の、現在の、今のところの 
\\	【形】職業に就いている 
\\	職業を持っている女性たち
\\	(単
\\	/(複1・2・3・4)‐ 収入、所得 
\\	音無	(単
\\	/(複
\\	祝典行列、パレード 
\\	【他】
\\	〔…⁴に…⁴について〕情報を知らせる〈教える〉 〖再帰的に〗
\\	…⁴について情報を入手する、調べる
\\	【形】① 肯定の、肯定的な 
\\	②(肯定的に評価できる)好ましい、有利な
\\	(単
\\	/(複1・2・3・4)‐
\\	年金 ②〔複数なし〕年金生活 
\\	(または
\\	⦅口語⦆年金生活者である 
\\	(または
\\	⦅口語⦆年金生活に入る
\\	(単
\\	/複数なし ふるまい、態度 
\\	【形】① 時間の、時間的な 
\\	【形】② 融通性のある、柔軟な 
\\	【形】①(傾斜が)急な、急勾配の 
\\	あっという間に昇進する 
\\	【形】裕福な、金持ちの 
\\	裕福な家族 
\\	イメージ;想像、考え 
\\	上演、上映
\\	2【再】
\\	力を合わせる、手を組む、協力〈連合〉する 
\\	1【再】
\\	と〕交代する、交替する、入れ替わる 
\\	【他】①〔…⁴を〕尊敬する、敬う 
\\	1【他】②〔…⁴を〕報告する、通報する2【再】①
\\	連絡する、便り〈電話〉をする
\\	(単
\\	/(複
\\	農場、農家の家屋敷 
\\	【他】〔…⁴を〕失望させる、がっかりさせる 
\\	(単
\\	/(まれに複
\\	⦅文語⦆奥様 
\\	(㊟自分の妻には用いない;=
\\	の女性形) 
\\	(あなたの)奥様
\\	【形】〔…に〕居住している 
\\	ハンブルク在住のシュミット氏
\\	【他】①《4格》〔…⁴に〕署名する、サインする 
\\	署名,サイン
\\	〔過去〕
\\	〔過分〕
\\	〔接Ⅱ〕
\\	【他】②〔予想・期待など⁴を〕上回る、越える 
\\	それは私の期待以上であった
\\	(単
\\	/複数なし 外観;顔つき 
\\	【副】私の側では、私としては 
\\	前提条件、必要条件 
\\	(単
\\	/(複1・2・3・4)‐
\\	①〔複数なし〕知り合うこと、面識 
\\	知人
\\	①〔ふつう複数で〕期待、予想、見込み 
\\	【形】財政上の、金融上の;金銭的な 
\\	【他】①〔…⁴を〕取り決める、協定する 
\\	(単
\\	/(複
\\	(比較的長い)時間、期間、時代 
\\	1【他】②〔価格など⁴を〕上げる 2【再】
\\	(価格などが)上がる 
\\	家賃が5パーセント上がった
\\	前払い、前納 
\\	〔現在〕
\\	〔過去〕
\\	〔過分〕
\\	に利益・損害⁴を〕もたらす 
\\	排水、排水設備 
\\	(単
\\	/複数なし ①(エネルギーなどの)供給 
\\	…⁴に必要なもの
\\	を与える、供給する
\\	(単
\\	/複数なし ごみ回収、清掃課〈局〉 
\\	(単
\\	/(複1・2・4)
\\	(複
\\	②(人と人の・組織間の)関係、間柄 
\\	(単
\\	/(複
\\	寝袋 
\\	〔現在〕
\\	〔過去〕
\\	〔過分〕
\\	【他】①〔…⁴を〕(楽しみのために)する、企てる 
\\	旅行〈ドライブ〉をする 
\\	【他】〔部屋など⁴を〕又貸しする 
\\	(単
\\	/(まれに複
\\	利用、使用 
\\	【形】(ホテル・病院などが)定員を超過した、超満員の
\\	【他】①〔…⁴を〕(下から)取り上げる、拾い上げる 
\\	②〔…⁴を〕開始する、〔…⁴に〕着手する 
\\	教授とコンタクトを取る ③〔客など⁴を〕受け入れる;宿泊させる
\\	(単
\\	/(複
\\	取り決め、申し合わせ 
\\	【接】〖定動詞後置〗(前提条件を表して)…である限り、…ならば 
\\	私の知る限りでは
\\	【自】①《3格》
\\	に〕一致する、合致する;相当する 
\\	それは事実に合致していない
\\	(単
\\	/(複1・2・3・4)‐ 
\\	芝生 
\\	芝生を刈る
\\	解雇通告、解約通告 
\\	会社は彼に解雇通告を出した 
\\	に解雇を通告する
\\	(単
\\	/複数なし (約束・期限などを)守ること、遵守 
\\	協定の遵守
\\	1【自】① 退職を申し出る ②《3格》
\\	に〕解雇を通告する2【他】〔…⁴の〕解約を通告する 
\\	賃貸契約を解約する
\\	【形】予定より早い、早すぎる 
\\	予定より早い旅立ち 
\\	【形】そのほかの、それ以外の 
\\	それ以外の彼の態度は立派だった 
\\	そのほかに
\\	(単
\\	/(複
\\	①〔複数なし〕取り決め、協定 
\\	取り決める
\\	(単
\\	/(複
\\	(一つの住居に数人で住む)共同生活〈ルームシェア〉のグループ (〔略〕
\\	1【他】①〔…⁴を〕すすぐ、洗浄する、〔食器類⁴を〕洗う 
\\	食器を洗う
\\	【他】①〔あいさつ・依頼など⁴を〕伝える 
\\	あなたに彼からよろしくとのことです ②〔会議・大会など⁴を〕開催する 
\\	選手権大会を開く
\\	(単
\\	/(複
\\	⦅動物⦆ナマケモノ;⦅口語⦆怠け者 
\\	知らせ、通知、報告 
\\	に…⁴を伝える,知らせる
\\	1【他】①〔…⁴を〕組み立てる、設置する、並べる 
\\	足場を組み立てる 
\\	(単
\\	/(複
\\	楽器 
\\	【他】〔肉・魚など⁴を〕焼き網〈グリル〉で焼く、グリルする 
\\	(単
\\	/(複
\\	煙突掃除人 
\\	【他】①〔…⁴を〕検査する、チェックする ②〔…⁴を〕統制する、支配する;制御〈コントロール〉する 
\\	(単
\\	/(複
\\	ベビーカー、乳母車 
\\	1【他】〔…⁴を〕(趣味・道楽で)組み立てる 
\\	1【他】①
\\	方向》〔…⁴を…に〕貼り付ける、くっつける
\\	1【他】①〔…⁴を〕固定する 
\\	(単
\\	/(複1・2・4)-
\\	(複
\\	羊 
\\	1(単
\\	/(複1・2・3・4)‐
\\	ハト 
\\	⦅話⦆すべてオーケー 
\\	【形】⦅口語⦆ 元気いっぱいの、上機嫌の 
\\	【自】キャンプをする 
\\	弱変化;(単2・3・4)‐
\\	/(複
\\	仲間、伴侶、連れ (〔女性〕..
\\	(単
\\	/(複
\\	ネットワーク、回路網 
\\	【形】① 科学技術の 
\\	技術的な
\\	(単
\\	/複数なし 保護、防護;防御 
\\	自然保護
\\	【形】不吉な 
\\	(単
\\	/(複
\\	指導、手ほどき;手引書、マニュアル 
\\	【他】②〔データ⁴を〕入力する 
\\	…⁴を取り上げる
\\	指示、命令 
\\	(または
\\	指示に従う
\\	(単
\\	/(複
\\	(テレビ・コンピュータなどの)画面、ディスプレー 
\\	(単
\\	/(複1・2・4)..
\\	(複
\\	物、物体 
\\	(単
\\	/(複
\\	(商品・現金との)引換券、商品券 
\\	概要;一覧[表]
\\	【他】①〔鑑定書・計画など⁴を〕作成する 
\\	鑑定書を作成する 
\\	【副】ただで、無料で 
\\	1【他】②〔…⁴を〕打ち切る、中止する 
\\	交渉を打ち切る
\\	確認、証明 
\\	【他】⦅コンピュータ⦆〔…⁴を〕ダウンロードする 
\\	【他】〔…⁴を〕評価する;査定する 
\\	仕事を過大〈過小〉評価する [派生] 
\\	評価,査定
\\	【形】① 1回の;二度とない
\\	(単
\\	/(複
\\	救い主、救助者 
\\	(単
\\	/(複
\\	儀式、式典、セレモニー 
\\	⦅口語⦆ 畜生!
\\	(単
\\	/(複1・2・3・4)‐
\\	①(鋭くとがった)先、先端;頂点 
\\	出発する;出かける
\\	(単
\\	/(複1・2・4)-
\\	(複
\\	(歩いてしか通れない)小道、細道 
\\	(単
\\	/(複
\\	装備;装備品 
\\	1【他】①〔…⁴を〕脱ぐ 
\\	彼はコートと帽子を脱いだ
\\	の過去分詞) 【形】① 反対の、逆の 
\\	(単
\\	/(複1・2・4)-
\\	(複
\\	回り道、迂回 
\\	回り道〈遠回り〉をする
\\	【形】〔格語尾は付かない〕ピンク色の、バラ色の 
\\	【自】①(空中・水中に)浮かぶ、漂う 
\\	一片の白い雲が青空に浮かんでいた
\\	〔現在〕
\\	〔過去〕
\\	〔過分〕
\\	【他】③〔…⁴を〕見落とす、(うっかりしていて)〔…⁴に〕気づかない
\\	【他】〔道⁴を〕歩む、たどる 
\\	新たな道を踏み出す 
\\	(単
\\	/(複1・2・3・4)‐
\\	鐘;ベル、鈴 
\\	4格 
\\	⦅口語⦆ …⁴をいらいらさせる、怒らせる
\\	〔現在〕
\\	〔過去〕
\\	〔過分〕
\\	【他】⦅文語⦆〔…⁴を〕賞賛する、ほめたたえる 
\\	(単
\\	/(複1・2・4)
\\	(複
\\	宝、財宝 
\\	③⦅口語⦆(愛する)大事な人〈子〉 
\\	私の小さな宝(かわいい子供への呼びかけ)
\\	【自】①〖述語と〗(…という)内容である;(…と)書かれている
\\	(単
\\	/(複1・2・4)
\\	(複
\\	頭 
\\	Ⅰ〔現在〕
\\	〔過去〕
\\	〔過分〕
\\	【自】〔完了
\\	生じる、生まれる Ⅱ【形】成人した、大人の
\\	【他】①〔…⁴を〕(傷害・災害などから)守る 
\\	1【他】〔…⁴を〕傾ける 
\\	(単
\\	/複数なし 謙虚、謙そん 
\\	(単
\\	/複数なし 幸せ;健康;繁栄 
\\	乾杯!
\\	(単
\\	/(複1・2・4)-
\\	(複
\\	祈り 
\\	【形】ほんとうの;本物の 
\\	(単
\\	/複数なし 帰還、帰郷;復帰
\\	(単
\\	/(複1・2・3・4)‐
\\	①〔ふつう単数で〕眺め、見晴らし、景色 
\\	庭園を見晴らす部屋 
\\	①〔複数なし〕美、美しさ 
\\	2【再】①
\\	を〕楽しむ、喜ぶ 
\\	【他】①〔…⁴を〕暗唱する 
\\	の過去分詞)【形】確信している 
\\	3格 
\\	を確信している
\\	【形】岩のように強固な、揺るぎない 
\\	【形】① ぞっとするような 
\\	(単
\\	/(複1・2・4)..
\\	(複
\\	思いつき、着想、アイデア 
\\	の念頭にうかぶ,…
\\	に思いつく
\\	【形】① 薄暗い
\\	(単
\\	/(複1・2・3・4)‐
\\	ほら穴、洞窟 
\\	(単
\\	/(複
\\	ボディーガード 
\\	〔過去〕
\\	〔過分〕
\\	【自】〔完了
\\	〔…⁴に〕向かって来る
\\	1【自】戦う、争う 
\\	【他】〔…⁴を〕横断する、横切る 
\\	1【自】賭ける 
\\	35ユーロ〈ビール1本〉賭ける 
\\	4格と〗
\\	ある馬に賭ける
\\	【形】非常に古い、太古〈大昔〉の
\\	①〔複数なし〕永遠、永久 
\\	永遠に ②⦅口語⦆ 非常に長い時間 
\\	(または
\\	ずっと前から
\\	【他】①〔…⁴を〕打ち負かす 
\\	(単
\\	/(複
\\	金貨 
\\	(単
\\	/複数なし 
\\	略奪品、戦利品;獲物
\\	【再】
\\	〔…⁴に〕忍び寄る 
\\	2【他】〔…⁴を〕驚かす、どきっとさせる3【再】
\\	⦅口語⦆ 驚く、どきっとする 
\\	〔過去〕
\\	〔過分〕
\\	〔接Ⅱ〕
\\	【自】〔完了
\\	現われる、姿を見せる 
\\	出版される、発行される 
\\	状態》
\\	に…と〕思われる
\\	(単
\\	/(複1・2・3・4)‐
\\	神経 
\\	②〔ふつう複数で〕ずぶとさ;冷静さ 
\\	冷静さを保つ〈失う〉
\\	(単
\\	/(複1・2・4)-
\\	(複
\\	稲光、いなずま 
\\	【形】②(程度・量などが)ものすごい、ひどい 
\\	ひどい暑さ
\\	段階;等級、グレード
\\	(単
\\	/(複1・2・4)- (複
\\	①〔ふつう単数で〕魔法;呪文 
\\	【他】〔…⁴を〕焦がす 
\\	髪をろうそくの火で焦がしてしまう 
\\	(単
\\	/(複1・2・4)- (複
\\	霧、もや、かすみ 
\\	【他】〔好ましくないこと⁴を〕引き起こす 
\\	(単
\\	/(複1・2・4)
\\	(複
\\	灌木、低木;(灌木の)茂み 
\\	【副】まっすぐに、まっしぐらに;ただちに 
\\	弱変化;(単2・3・4)‐
\\	/(複1・2・3・4)‐
\\	英雄、勇士 
\\	/〔女性〕-
\\	(単
\\	/(複1・2・4)-
\\	(複
\\	地平線、水平線 
\\	【形】無傷の、けがのない 
\\	⦅文語⦆ 元気〈健康〉である
\\	(または
\\	⦅文語⦆ 彼に深い恩義を感じている
\\	(単
\\	/複数なし 
\\	恐怖、恐れ 
\\	【形】(光が)ぎらぎらする 
\\	ぎらぎらする太陽
\\	1【他】①〔病気・苦難など⁴を〕経験する、耐え抜く 
\\	彼はいろいろな目に会ってきた 
\\	の過去分詞)【形】①(事態がどうなるか)期待に満ちた、好奇心のある
\\	1【他】②〔窓ガラスなど⁴を〕たたき壊す 
\\	彼は窓をたたき割った 2【自】①〔完了
\\	または
\\	(雷・爆弾などが)落ちる 
\\	元気のよい、活発な、快活な 
\\	(単
\\	/(複1・2・3・4)‐ ①〔複数なし〕平和、平時 
\\	平和を維持する〈乱す〉
\\	(単
\\	/(複1・2・4)-
\\	(複
\\	時点;時期;日時
\\	(単
\\	/(複1・2・4)
\\	(複
\\	①〔ふつう複数で〕草;薬草、ハーブ、香草 
\\	(単
\\	/(複1・2・4)..
\\	(複
\\	お祝いの言葉、祝詞 
\\	誕生日おめでとう!
\\	興奮、刺激 
\\	興奮する
\\	【自】②
\\	〔…⁴を〕監視する、見張る
\\	【副】常に、いつも 
\\	3格+
\\	は心の重荷が下りる
\\	Ⅰ【形】① 確かな、確実な 
\\	③〖付加語的に〗ある、ある種の 
\\	ベッカーさんとかいう人 ④〖付加語的に〗ある程度の、いくらかのⅡ【副】① きっと 
\\	【副】自宅で;故郷で 
\\	(単
\\	/(複1・2・4)- (複
\\	疑い、疑念、疑惑 
\\	(単
\\	/(複1・2・3・4)‐
\\	①(水の)波 
\\	(単
\\	/複数なし 日光 
\\	やきもきしている、
\\	(不安・あせりなどで)いても立ってもいられない 
\\	(単
\\	/(複
\\	トロフィー 
\\	(単
\\	/(複
\\	種、種子 
\\	1【他】①〔…⁴を〕載せる、置く 
\\	帽子をかぶる〈子供に帽子をかぶらせる〉
\\	(単
\\	/(複
\\	帰り道、帰途、家路 
\\	(単
\\	/複数なし ⦅文語⦆ オーラ、(人・物が発する)独特な雰囲気 
\\	【他】〔…⁴を〕否定する、否認する 
\\	犯行を否認する 
\\	〔現在〕
\\	〔過去〕
\\	〔過分〕
\\	1【他】①〔旅行など⁴を〕始める 
\\	旅行に出る ④〔遺産など⁴を〕引き継ぐ 2【自】〔完了
\\	⦅スポーツ⦆〔…⁴と〕対戦する 
\\	〔現在〕
\\	〔過去〕
\\	〔過分〕
\\	【他】①(壁などが)〔…⁴を〕取り囲む、取り巻く 
\\	【状態受動で】
\\	その湖は高い山に囲まれている
\\	【形】独特の、風変わりな;奇妙な 
\\	変わった男
\\	4格+
\\	…⁴をたくらむ
\\	1【自】①
\\	〔本など
\\	を〕ぱらぱらとめくる
\\	【形】(驚きなどで)口が利けない 
\\	【形】① 誠実な、裏切らない
\\	(単
\\	/(複
\\	⦅口語⦆①⦅ドイツ南部・オーストリア・スイスで⦆ 馬 
\\	1【他】〔…⁴を〕拒否する、拒絶する 
\\	同意〈供述〉を拒否する 
\\	悪事を働く (㊟所有冠詞は主語の人称に応じて変わる)
\\	まったく
\\	3格+
\\	⦅口語⦆ 
\\	にとって足手まといである 
\\	丸太)
\\	(単
\\	/(複
\\	(手の)指先 
\\	岩礁(海面上に出ているものも海面下に隠れているものも指す) 
\\	【形】⦅文語⦆ 奇妙な、風変わりな 
\\	3格 
\\	⦅口語⦆ 
\\	を少しも知らない 
\\	かすかなもの、ごくわずか)
\\	〔現在〕
\\	〔過去〕
\\	〔過分〕
\\	【他】〔…⁴を〕(推測して)[言い]当てる 
\\	(または
\\	⦅口語⦆(驚き・当惑などを表わして)おやおや!;あらあら! 
\\	1【他】①〔…⁴を〕(多数の中から)見つけ出す 
\\	②〔…⁴を〕(調べて)発見する、突き止める
\\	(単
\\	/(複
\\	スカーフ 
\\	〔現在〕
\\	〔過去〕
\\	〔過分〕
\\	【他】①
\\	〔…⁴の…
\\	を〕妨げる 
\\	彼女は彼の仕事を邪魔する ②〔…を〕開催する、挙行する
\\	【自】〔完了
\\	(ドイツ南部・オーストリア・スイスでは
\\	も)〕間近に迫っている 
\\	(単
\\	/複数なし 我が家;故郷 
\\	【他】①〔困難・問題など⁴を〕克服する、切り抜ける 
\\	③〔技術など⁴に〕熟達している、〔…⁴を〕マスターしている
\\	〔現在〕
\\	〔過去〕
\\	〔過分〕
\\	1【他】①〔一定の金額⁴を〕支出する、使う 
\\	ニュース 
\\	(または
\\	(別れのあいさつとして)ご機嫌よう!、さようなら!
\\	(単
\\	/複数なし 好意、親切 
\\	隣人に頼み事をする
\\	【他】①〔…⁴を〕(ある目的のために)訪ねる 
\\	医者にかかる
\\	(単
\\	/(複1・2・3・4)‐
\\	羽毛、羽 
\\	【形】勇敢な、勇気のある 
\\	(単
\\	/複数なし ⦅心理⦆ 高所恐怖症 
\\	報酬、謝礼、賞金 
\\	(単
\\	/複数なし ①〖集合的に〗近所の人々 
\\	近所 
\\	隣人関係、近所づきあい
\\	(単
\\	/(複1・2・4)- (複
\\	①⦅宗教⦆ 祭司 
\\	/〔女性〕-
\\	(単
\\	/(複1・2・4)-
\\	(複
\\	〖ふつう形容詞と〗⦅口語⦆ やつ、男 
\\	【形】②
\\	に…⁴の〕借りがある 
\\	私は彼にまだ5ユーロの借りがある 
\\	に…⁴の〕義務がある
\\	【他】〔贈り物・賞など⁴を〕手渡す、授与する、贈呈する 
\\	4格+
\\	…⁴を見捨てる;自分のもの⁴を放棄する
\\	交代、交替 
\\	と交代する
\\	(単
\\	/(複1・2・4)- (複
\\	支持者、信奉者;ファン (〔女性〕-
\\	ペンダント 
\\	初心者
\\	【他】①〔困難な仕事⁴を〕かたづける 
\\	彼はその困難な課題を処理した 
\\	【形】《3格》
\\	の〕助けになる 
\\	妻は私のこの仕事の手助けをしてくれます
\\	(単
\\	/(複
\\	戦士、兵士 
\\	【副】① 去って、いなくなって 
\\	(単
\\	/(複1・2・4)-
\\	(複
\\	①〔複数なし〕⦅文語⦆ 暗闇、暗黒 
\\	真っ暗な
\\	【形】①(見た目が)りっぱな、堂々とした;体格のいい 
\\	【形】伝説上の;伝説的な 
\\	【副】もしかすると、場合によっては、ことによると 
\\	1【他】①〔仕事など⁴を〕楽にする、軽減する、緩和する
\\	(単
\\	/(複1・2・3・4)‐
\\	荷物 
\\	③(精神的な)重圧、苦労 
\\	【他】①〔目標など⁴を〕とらえ損なう、外す、間違う 
\\	的を外す
\\	【形】② 気高い、高潔な、貴い
\\	⦅口語⦆ ずっとまっすぐに
\\	(単
\\	/(複
\\	お守り、護符、魔よけ 
\\	【形】〔格語尾は付かない〕さまざまな、いろいろな 
\\	さまざまな植物 
\\	【形】役に立つ、有用な、有益な 
\\	(単
\\	/(複
\\	⦅鉱物⦆ エメラルド 
\\	(単
\\	/(複1・2・3・4)‐
\\	海岸、沿岸 
\\	彼女は休暇を海辺で過ごした 
\\	箱;木箱
\\	弱変化;(単2・3・4)‐
\\	/(複1・2・3・4)‐
\\	カラス 
\\	(単
\\	/(複1・2・3・4)‐
\\	キュウリ 
\\	1(単
\\	/(複1・2・4)
\\	(複
\\	①(動物の)角 
\\	【自】〔…で〕キャンプ〈野営〉する 
\\	(単
\\	/(まれに複
\\	遠方 
\\	(単
\\	/(複1・2・4)- (複
\\	革、なめし革 
\\	(単
\\	/(複
\\	ふつう複数で ①(魚・爬虫類などの)鱗 
\\	(単
\\	/(複1・2・3・4)‐
\\	①⦅植物⦆ カブ、テンサイ類 ②⦅口語⦆ 頭
\\	Ⅰ【形】① 活発な、生き生きした、元気な 
\\	(単
\\	/(複1・2・3・4)‐
\\	⦅やや古語⦆① 安宿、宿屋;ユースホステル
\\	(単
\\	/複数なし 絶望、絶望感 
\\	絶望する
\\	早朝に
\\	作用、効果 
\\	(単
\\	/(複1・2・4)
\\	(複
\\	嵐、暴風[雨] 
\\	(単
\\	/(複1・2・4)-
\\	(複
\\	楽器
\\	(単
\\	/(複1・2・3・4)‐
\\	メダル、勲章 
\\	(単
\\	/(複1・2・4)-
\\	(複
\\	瞬間、一瞬 
\\	(相手の言葉を遮って)ちょっと待ってください!
\\	【形】擦りむけた 
\\	(単
\\	/(複
\\	(小さな)酒場、居酒屋、飲み屋 
\\	(単
\\	/(複1・2・4)-
\\	(複
\\	鍛冶屋 
\\	【形】① 信じられない ②⦅口語⦆ とてつもない、ものすごい
\\	〔過去〕
\\	〔過分〕
\\	〔接Ⅱ〕
\\	【自】〔完了
\\	生じる、起こる;(建物などが)建つ 
\\	火事は何が原因で起こったのですか?
\\	(単
\\	/(複1・2・4)-
\\	(複
\\	煙突 
\\	Ⅰ【形】特別な、特殊な 
\\	(単
\\	/複数なし 
\\	怒り、立腹 
\\	いざこざ、トラブル 
\\	3格 
\\	のことで手を焼いている
\\	4格+
\\	⦅口語⦆ …⁴を厳しく叱りつける 
\\	カタツムリ)
\\	(単
\\	/(複1・2・4)- (複
\\	①(中世の)騎士 
\\	【形】① 歴史の;歴史に関する 
\\	(単
\\	/(複
\\	王国 
\\	(単
\\	/(複1・2・4)-
\\	(複
\\	地区、区域;地域 ([略]
\\	住宅区域 
\\	(単
\\	/(複1・2・4)- (複
\\	①⦅数学⦆角(かく)、角度 
\\	隅、片隅 
\\	【形】① 思いきった;大胆な、勇気のある 
\\	涼しい;冷たい
\\	(単
\\	/(複
\\	すり 
\\	【形】以前の、かつての 
\\	旧東ドイツ
\\	防衛、防御 
\\	…⁴を守る
\\	〔現在〕
\\	〔過去〕
\\	〔過分〕
\\	【再】①
\\	場所/方向》〔…に〕腰をおろす;〔…に〕(鳥が)止まる 
\\	場所》〔…に〕居を定める
\\	(単
\\	/複数なし 
\\	悲惨、不幸 
\\	貧困、困窮
\\	(単
\\	/(複1・2・3・4)
\\	(ある行動に伴う)危険、リスク 
\\	1〔過去〕
\\	(①の意味ではまれに
\\	も) 〔過分〕
\\	【他】①⦅口語⦆〔…⁴を〕殴る、たたく 
\\	(単
\\	/(複1・2・3・4)‐ 
\\	犯罪;犯罪的行為 
\\	(単
\\	/(複1・2・3・4)‐
\\	①〔複数なし〕力、勢い 
\\	〔現在〕
\\	〔過去〕
\\	〔過分〕
\\	〔予期しない状態⁴に〕陥る、なる 
\\	危険〈困難〉に陥る
\\	(単
\\	/(複
\\	③〔複数なし〕没落、滅亡、破滅
\\	(単
\\	/(複1・2・4)-
\\	(複
\\	刺激 
\\	【副】いわば、言ってみれば、…も同然
\\	(単
\\	/(複1・2・3・4)‐
\\	①(一人用の)小部屋;(刑務所の)独房 
\\	家具のない小部屋に住む 
\\	【形】自称の、本人〈当事者〉の言による;名目上の 
\\	【自】〔完了
\\	①〔追跡者など
\\	から〕逃れる ②〔…から・へ〕逃げる、逃亡する 
\\	脱獄する 
\\	危険〈追っ手〉をのがれる
\\	1【自】〔完了
\\	③⦅口語⦆〔…に〕着く;〔…に〕(偶然)行き着く 
\\	私たちは無事家に着いた
\\	根本的には、結局のところ
\\	⦅口語⦆ とても倹約している;けちである 
\\	…⁴を回す、裏返す)
\\	【自】①〔…を〕じっと見つめる、凝視する 
\\	私たちは演説者をじっと見つめていた
\\	(単
\\	/複数なし ねたみ、うらやましさ 
\\	【形】① 絶対の、絶対的な 
\\	絶対音感
\\	(単
\\	/(複1・2・3・4)‐
\\	①〔複数なし〕収穫、取り入れ 
\\	(単
\\	/(複
\\	灌漑;灌水、水まき 
\\	1【他】⦅文語⦆①〔…⁴を〕一つにまとめる、統合する、合わせる 
\\	(単
\\	/(複1・2・4)-
\\	(複
\\	帝国、王国 
\\	領域、(ある特定の)世界
\\	【形】① 恐ろしい 
\\	Ⅱ【形】優秀な、すばらしい、極上の 
\\	彼女はすばらしくじょうずにドイツ語を話す 
\\	【形】② 豊富な;ぜいたくな、豪勢な 
\\	豊かな想像力を持っている
\\	(単
\\	/(複1・2・4)-
\\	(複
\\	ロープ、縄、綱 
\\	(単
\\	/(複
\\	パスワード 
\\	(単
\\	/(複1・2・4)-
\\	(複
\\	網 
\\	(単
\\	/(複1・2・4)..
\\	(複
\\	①〔複数なし〕立ち去ること、退去
\\	【形】悲劇的な、悲惨な 
\\	陛下(王・皇室などに対する敬称)
\\	(単
\\	/(複1・2・4)- (複
\\	支配者、統治者 
\\	/〔女性〕-
\\	支配する,統治する
\\	⦅軍事⦆(命令に対して)了解!
\\	Ⅰ【形】① 恥ずべき、下劣な;不名誉な 
\\	有害な
\\	【自】〔完了
\\	①《3格》
\\	の〕あとに続く、あとを追う 
\\	[派生] 
\\	《3格》…
\\	のあとについて行く,あとを追う
\\	(単
\\	/(複1・2・4)- (複
\\	害悪、悪;火災 
\\	(単
\\	/複数なし 
\\	暗闇、暗がり;暗黒 
\\	(単
\\	/複数なし 
\\	食物、食事
\\	(単
\\	/(複1・2・3・4)‐ ①〔複数なし〕本質 
\\	②〔複数なし〕(人の)性質、人となり ③〖特定の形容詞と〗(…な)存在、生き物
\\	(単
\\	/(複1・2・4)..
\\	(複
\\	①〔複数なし〕礼儀 
\\	礼儀正しい
\\	1(単
\\	/複数なし 
\\	まじめ、本気、真剣 
\\	本気なの? [派生] 
\\	まじめな,真剣な,本気の
\\	⦅口語⦆ こっそりと逃げ出す 
\\	ほこり)
\\	(単
\\	/(複1・2・4)..
\\	(複
\\	偶然、偶然の出来事 
\\	(または
\\	⦅口語⦆ 何もしない;ぶらぶらしている
\\	【副】こちらへ、ここへ 
\\	〔形容詞変化〕① 囚人 
\\	捕虜
\\	Ⅰ【形】①(時刻・時間が)早い;初期の 
\\	Ⅱ【副】② 朝に
\\	〔現在〕
\\	〔過去〕
\\	〔過分〕
\\	1【他】〔…⁴に〕(体質的・気質的に)耐えられる 
\\	この植物は寒さに強い 2【再】①
\\	と〕仲良くやっていく 
\\	彼は奥さんの両親と折り合いが悪い 
\\	⦅口語⦆ 苦労して;やっとのことで 
\\	嘆息、
\\	騒音)
\\	【数】《基数》① 1000、千 
\\	②⦅口語⦆ 多数の、無数の
\\	【他】①〔…⁴を〕書き留める、記載する (㊟ふつう状態受動で) 
\\	重要な名前はリストに載っている ②〔…⁴を〕記録に留める 
\\	残念ながら3人の死亡者があった
\\	(単
\\	/(複1・2・4)
\\	(複
\\	跳躍、ジャンプ 
\\	1【他】①〔重いもの⁴を〕苦労して運ぶ 
\\	トランクを駅まで運ぶ
\\	【他】〔…⁴を〕引き起こす、生じさせる 
\\	変化を引き起こす 
\\	【形】驚くべき、驚嘆に値する 
\\	(単
\\	/(複1・2・4)- (複
\\	犯罪者、犯人 
\\	/〔女性〕-
\\	(単
\\	/(複1・2・3・4)‐
\\	修道女 
\\	【自】⦅文語⦆(力などが)作用している、支配している
\\	3格 
\\	をすることができる
\\	【形】① 逃走中の、逃亡中の [派生] 
\\	逃走,逃亡,脱走
\\	1【他】①〔道路など⁴を〕遮断する;〔港など⁴を〕封鎖する、閉鎖する
\\	(単
\\	/(複1・2・3・4)‐
\\	①⦅軍事⦆ 部隊;〔複数で〕軍隊 
\\	1【他】①〔…⁴を〕つかむ ②〔…⁴を〕捕まえる、逮捕する ⑥〔…⁴を〕理解する 
\\	それはとても理解できない!
\\	〔現在〕
\\	〔過去〕
\\	〔過分〕
\\	1【他】②
\\	の過失など⁴を〕許す 【4格なしで】
\\	私を許してくれ! 
\\	私は彼の過失を許してやった 
\\	【文末で】…ですね?
\\	4格+
\\	リクエスト中	…⁴を激怒させる〈おびえさせる〉
\\	〔過去〕
\\	〔過分〕
\\	【他】②〔…⁴を〕示す 
\\	彼は勇気があることを示した
\\	〔過去〕
\\	〔過分〕
\\	【他】〔…を〕避ける;控える 
\\	…⁴を報告する
\\	(単
\\	/(複1・2・3・4)‐
\\	メインストリート、目抜き通り 
\\	4格+
\\	⦅口語⦆ …⁴を非常に良く知っている(㊟所有冠詞は主語の人称に応じて変わる) 
\\	チョッキのポケット)
\\	【副】① それでも、ともかくも;少なくとも;何と言っても 
\\	彼の仕事は確かに不十分ではあるが、ともかくも彼は努力したのだ
\\	①〔複数なし〕注意、関心 
\\	1【他】②〔感情・気持ち⁴を〕引き起こす、呼び起こす 
\\	同情を呼ぶ 
\\	(単
\\	/(複1・2・4)- (複
\\	丘、丘陵 
\\	(単
\\	/(複1・2・3・4)‐
\\	液果、ベリー 
\\	〔現在〕
\\	〔過去〕
\\	〔過分〕
\\	【他】〔高価なもの⁴を〕購入する、獲得する 
\\	【接】《従属》〖定動詞後置〗① …するやいなや、…したらすぐに 
\\	(単
\\	/(複1・2・3・4)‐
\\	①〔複数なし〕悪意、陰険さ 
\\	(または
\\	彼は陰険きわまりない
\\	〔過去〕
\\	〔過分〕
\\	【自】〔完了
\\	①《3格》〔危険・不幸など
\\	を〕逃れる、免れる 
\\	危険を逃れる
\\	〔過去〕
\\	〔過分〕
\\	【他】①〔損害など⁴を〕被る、受ける 
\\	【形】価値の高い、高価な;貴重な 
\\	3格+
\\	の策略などを見抜く 
\\	策略)
\\	【形】買い得な 
\\	お買い得なシャツ
\\	(単
\\	/複数なし 後悔、悔恨;ざんげ 
\\	…⁴を後悔する
\\	【副】(考えた末)結局、最終的に
\\	(単
\\	/(複1・2・4)-
\\	(複
\\	①〔複数なし〕器用;巧みさ、熟練 
\\	(単
\\	/複数なし 汗 
\\	(単
\\	/(複1・2・3・4)‐
\\	刃;かみそりの刃 
\\	1【他】〔…⁴の〕スイッチを切る
\\	…も…も…ない 
\\	(単
\\	/(複1・2・3・4)‐
\\	①(石・れんが・コンクリートなどの)壁、塀 
\\	(単
\\	/(複1・2・4)-
\\	(複
\\	①(人との)接触、連絡、コンタクト 
\\	1(単
\\	/(複1・2・3・4)‐ ①(積み重ねてできた)山、積み重ね、堆積 
\\	港 
\\	(単
\\	/(複1・2・4)
\\	(複
\\	針金、ワイヤー 
\\	〔過去〕
\\	〔過分〕
\\	【自】〔完了
\\	逃走する 
\\	⦅口語⦆ 悪事を働く
\\	(単
\\	/(複
\\	ニンジン 
\\	(単
\\	/複数なし 隠れんぼう、隠しだて[をすること] 
\\	⦅口語⦆ ずらかる
\\	【形】ひどく不快な、嫌な、むかつくような 
\\	嫌なにおい〈やつ〉
\\	(単
\\	/(複1・2・4)- (複
\\	踊り手、ダンサー 
\\	/〔女性〕-
\\	⦅口語⦆ さっさと消えうせろ!
\\	1【他】②〔買物客など⁴に〕応対する 
\\	客に良心的に応対する [派生] 
\\	サービス
\\	【他】④
\\	〔…⁴を…
\\	から〕絞り出す 
\\	(単
\\	/複数なし 
\\	深い悲しみ、悲嘆 
\\	彼は父親を亡くした悲しみに暮れていた
\\	【自】〔完了
\\	絶望する、希望を失う 
\\	絶望した,絶望的な
\\	(単
\\	/(複
\\	①⦅経済⦆(資本の)投資、出資
\\	⦅口語⦆ 大金(たいきん)
\\	(単
\\	/(複1・2・3・4)‐
\\	1①〔ふつう単数で〕やり方、方法、様式 
\\	話し方 2[男・女]〔形容詞変化〕賢者
\\	3格+
\\	⦅俗語⦆ 
\\	に近づく;…
\\	にしつこく迫る 
\\	薄皮)
\\	(単
\\	/(複1・2・3・4)‐
\\	距離、間隔 
\\	遠ざかる,離れる
\\	【他】①〔…⁴を〕(散歩などに)連れ出す、(飲食などに)連れて出る、同伴する ③〔…⁴を〕実行する、遂行する
\\	1【再】
\\	に〕参加する 
\\	彼はその競技〈討議〉に参加した 
\\	【自】きらめく、きらきら輝く 
\\	〔過去〕
\\	〔過分〕
\\	2【再】①
\\	で〕我慢する、
\\	に〕甘んじる
\\	(単
\\	/(複1・2・3・4)‐ ①〔複数なし〕思い出、追憶、記念 
\\	記念品、形見;(旅の)みやげ 
\\	朝早くから夜遅くまで
\\	(単
\\	/複数なし ②〖集合的に〗後継者、若手
\\	Ⅰ【副】部分的に、一部は 
\\	(単
\\	/(複1・2・3・4)‐
\\	タマネギ 
\\	(単
\\	/(まれに複
\\	責任 
\\	4格 
\\	(または
\\	…⁴に対する責任を負っている 
\\	⦅口語⦆(犯行の際)見張りをする
\\	(単
\\	/(複1・2・4)-
\\	(複
\\	(建物の)階 
\\	〔現在〕
\\	〔過去〕
\\	〔過分〕
\\	【他】①〖特定の名詞と〗(…を)する、行なう 
\\	〔…⁴に〕取り組む、取りかかる
\\	〔過去〕
\\	〔過分〕
\\	1【他】①〔…⁴に〕関係する、かかわる 
\\	それはあなたとは関係ありません! 
\\	(単
\\	/複数なし ①⦅口語⦆ 狂気のさた、きわめてばかげた言動 
\\	(または
\\	そんなことをするなんてまったくばかげているよ
\\	(相手の言葉を遮って)ちょっと待ってください!
\\	【他】①〔数量・価値など⁴を〕見積もる ②〔…⁴を〕高く評価する 
\\	彼女は彼の親切のありがたさがわかっている ③⦅口語⦆ 〔…⁴と〕思う、推測する
\\	(または
\\	⦅口語⦆ ひどく不安がっている
\\	(単
\\	/複数なし ⦅口語⦆ ばかげたこと 
\\	こんなくだらないことはやめろ! 
\\	〔過去〕
\\	〔過分〕
\\	(まれに〔過去〕
\\	〔過分〕
\\	〔接Ⅱ〕
\\	【他】〔パンフレット・商品など⁴を〕発送する、送付する 
\\	【自】③⦅口語⦆ うまくいく 
\\	すべてうまくいった
\\	4格+
\\	(または
\\	発音×	…⁴が好きだ 
\\	…⁴を許す、耐える)
\\	4格 
\\	⦅口語⦆ …⁴を避ける
\\	【他】①〔…⁴を〕(調査して)確かめる、突き止める 
\\	死因を確認する ②〔…⁴に〕気づく 
\\	弱変化;(単2・3・4)‐
\\	/(複1・2・3・4)‐
\\	仲間、同級生 (〔女性〕-
\\	憲法 
\\	②〔複数なし〕(心身の)状態、気分 
\\	調子が良い〈悪い〉 
\\	それはとんでもないことだ(←樽の底を打ち抜く)
\\	(単
\\	/(複1・2・3・4)‐
\\	①⦅文語⦆ 天分、才能 ②⦅文語⦆ 贈り物 
\\	(…について)さんざん頭を悩ます
\\	〔現在〕
\\	〔過去〕
\\	〔過分〕
\\	1【他】①〔…⁴を〕買収する、〔…⁴に〕賄賂を贈る 
\\	【形】奇妙な、変な、不思議な 
\\	〔現在〕
\\	〔過去〕
\\	〔過分〕
\\	1【他】①
\\	〔…3・4に…⁴を〕手渡す 
\\	所有者に鍵を手渡す
\\	(単
\\	/(複1・2・4)- (複
\\	①(キリスト教以外の宗教の)神殿、寺院 
\\	(単
\\	/(複1・2・3・4)‐
\\	ハエ 
\\	⦅ことわざ⦆一石二鳥をねらう(←二匹のハエをハエたたき一つで打つ)
\\	【他】〔…⁴を〕保管する、保存する、預かる 
\\	文書を保管する
\\	場所+
\\	…に腰を落ち着ける
\\	【他】①
\\	に希望するもの⁴を〕与える;認める、許可する 
\\	銀行は彼に貸し付けを認めた 
\\	…に保護を与える 
\\	(単
\\	/(複1・2・4)- (複
\\	悪魔 
\\	/〔女性〕-
\\	4格+
\\	…⁴の在庫がある 
\\	倉庫)
\\	(単
\\	/(まれに複
\\	(特にハイキング・ドライブなどの途中の)休息、休憩 
\\	休憩する 
\\	残り;余り
\\	4格+
\\	(または
\\	⦅口語⦆ …⁴を偶然手に入れる
\\	(単
\\	/(複1・2・3・4)..
\\	材料、原料、素材 
\\	用具, 道具
\\	(単
\\	/(複1・2・4)- (複
\\	商人、商売人、小売商 
\\	1【再】
\\	変わる、変化する 
\\	好みが変わる
\\	〔過去〕
\\	〔過分〕
\\	2【再】
\\	〔…であることが〕判明する、明らかになる 
\\	それは間違いだとわかった 
\\	(単
\\	/(複1・2・4)..
\\	(複
\\	決心、決断;決定 
\\	決心する
\\	(単
\\	/(複1・2・3・4)‐
\\	①〔複数なし〕ブロンズ、青銅
\\	【他】①〔鉄など⁴を〕鍛える 
\\	②〔…⁴を〕鍛造する 
\\	の現在分詞)【形】卓越した、すばらしい 
\\	1【他】〔手足など⁴を〕伸ばす 
\\	3格 
\\	に手を伸ばす;⦅比喩⦆…
\\	を手に入れようとする
\\	毎日毎日、明けても暮れても
\\	(単
\\	/(複1・2・3・4)‐
\\	①〔ふつう単数で〕形;姿、格好、体型 
\\	地球は球形をしている。 
\\	彼はほっそりした体つきをしている
\\	(単
\\	/複数なし 
\\	迷惑行為、乱暴 
\\	(または
\\	乱暴をはたらく 
\\	ばかげたこと、ナンセンス 
\\	〔現在〕
\\	〔過去〕
\\	〔過分〕
\\	【他】①〔…⁴を〕許す、許容する 
\\	私は断じてそんなことは許さない 
\\	(イメージ:
\\	〈方向〉+
\\	…させておく)
\\	(単
\\	/(複1・2・4)-
\\	(複
\\	いたずら、悪ふざけ
\\	1【他】①《4格》〔…⁴に〕触れる、接触する、触る 
\\	【4格なしで】
\\	(展示品に)手を触れないでください!
\\	(単
\\	/複数なし 退屈
\\	(単
\\	/(複1・2・4)..
\\	(複
\\	刑務所、監獄、牢獄 
\\	1[男・女]〔形容詞変化〕老人、年寄り 
\\	2[中]〔形容詞変化〕古いもの、古いこと
\\	〔過去〕
\\	〔過分〕
\\	【自】〔完了
\\	起こる、生じる 
\\	またたく間に、あっという間に
\\	【副】その前に、[それ]以前に 
\\	そんなことはこれまでに一度も体験したことがない 
\\	(単
\\	/(複1・2・3・4)‐
\\	調和、融和、一致
\\	実際は、実のところは
\\	【自】〔…へ〕そびえ立つ、〔…から〕突き出る 
\\	塔が空高くそびえている
\\	4格+
\\	ビル・町など⁴を徹底的に破壊する
\\	【形】① 絶望した;絶望的な、どうしようもない 
\\	彼の絶望的な状況
\\	【他】①〔文字など⁴を〕消す、抹消〈削除〉する
\\	(単
\\	/複数なし (将来への)確信、(成功への)確かな見込み 
\\	確信に満ちている,確信しきっている
\\	4格+
\\	⦅口語⦆ …⁴を独り占めしている 
\\	〔農地など⁴を〕賃借りする)
\\	〔現在〕
\\	〔過去〕
\\	〔過分〕
\\	②〔本・ページなど⁴を〕開く
\\	(単
\\	/(複
\\	⦅料理⦆ シチュー、煮込料理(野菜・肉などを同じ鍋に入れて調理したもの)
\\	1【他】②〔…⁴を〕覆う 
\\	食卓の用意をする(テーブルクロスを掛けて食器を並べる) ⑤〔需要・要求など⁴を〕満たす
\\	【副】あまりにも;極端に 
\\	このトランクはあまりにも重すぎる
\\	(単
\\	/(まれに複
\\	旅立ち、(旅への)出発 
\\	【形】高価な、ぜいたくな;貴重な 
\\	〔現在〕
\\	〔過去〕
\\	〔過分〕
\\	1【他】①〔…⁴を〕隠す;〔感情など⁴を〕見せない 
\\	顔を両手で隠す 
\\	2(単
\\	/(複1・2・3・4)‐ 岩壁、岩塊
\\	放心状態である
\\	(単
\\	/(複1・2・3・4)‐
\\	①(小さい)門 ②(守衛のいる)入口
\\	Ⅰ【形】未来の、将来の、今後の 
\\	未来、将来
\\	(単
\\	/(複
\\	カニ 
\\	〔現在〕
\\	〔過去〕
\\	〔過分〕
\\	【他】
\\	に危害・苦痛など⁴を〕与える
\\	(単
\\	/(複1・2・4)- (複
\\	体、肉体 
\\	(単
\\	/(まれに複
\\	または
\\	ほこり、ちり;花粉 
\\	(単
\\	/(複
\\	⦅動物⦆ コウモリ 
\\	(単
\\	/(複1・2・4)-
\\	(複
\\	金属 
\\	2格	
\\	の過程で
\\	(単
\\	/(複1・2・3・4)‐
\\	反応 
\\	【形】大急ぎの、性急な 
\\	4格 
\\	⦅口語⦆ …⁴にねらいをつけている、嫌がらせをする(㊟
\\	は形式目的語) 
\\	予測する、見通す) 
\\	(単
\\	/複数なし 根気、忍耐、頑張り;⦅スポーツ⦆ 持久力
\\	間一髪で、すんでのところで
\\	1【自】〔完了
\\	起こる 
\\	状態》
\\	にとって…のように〕思われる
\\	(単
\\	/(複
\\	溶岩 
\\	〔過去〕
\\	〔過分〕
\\	【自】〔完了
\\	①(自転車・馬などから)降りる 
\\	重要ではない、大したことはない
\\	(単
\\	/(複1・2・3・4)‐
\\	文明;文明化された社会 
\\	1【自】① 汗をかく 
\\	(単
\\	/(複1・2・3・4)‐ 
\\	骨 
\\	(単
\\	/(複1・2・3・4)‐
\\	①〔ふつう単数で〕方法、やり方 
\\	②〔複数で〕作法、マナー 
\\	彼は礼儀作法がなっていない 
\\	(単
\\	/(複
\\	⦅口語⦆ 頑固な人;頑固[な態度] 
\\	(単
\\	/(複1・2・3・4)‐
\\	性質、特質、特性 
\\	(単
\\	/(複1・2・4)- (複
\\	戦車 
\\	甲冑(かっちゅう)、よろい
\\	(単
\\	/(複
\\	オアシス 
\\	(単
\\	/(複1・2・4)
\\	(複
\\	蒸気、湯気 
\\	(単
\\	/(複1・2・4)-
\\	(複
\\	体、肉体;胴体 
\\	【形】屋内で、室内で;その中に 
\\	【他】①
\\	様態》〔…⁴を…に〕取り扱う、遇する 
\\	【形】①(子供が)行儀のよい、おとなしい;おりこうさんの
\\	(単
\\	/(複
\\	②(神道の)神社 
\\	(単
\\	/(複1・2・3・4)‐
\\	仮定、想定 
\\	(または
\\	…と仮定して [派生] 
\\	…⁴と推測する;思う;仮定する
\\	(単
\\	/(複1・2・4)-
\\	(複
\\	手工業
\\	【他】〔…⁴を〕後悔する 
\\	…⁴の世話をする,めんどうを見る
\\	(単
\\	/(複1・2・4)-
\\	(複
\\	火山 
\\	3格	
\\	から遠くない所に
\\	(単
\\	/(複1・2・4)-
\\	(複
\\	①〔複数なし〕心情、心;気質 
\\	居心地のよい
\\	(単
\\	/(複1・2・3・4)‐
\\	①(縁無しの)帽子(ベレー帽・スキー帽など) 
\\	硬貨、コイン 
\\	(単
\\	/複数なし 
\\	暑さ、熱さ;高温 
\\	(単
\\	/(複1・2・4)-
\\	(複
\\	管、パイプ
\\	(単
\\	/(まれに複
\\	争い、けんか 
\\	争う,けんかする
\\	1【他】①〔…⁴を〕(誤って)のみ込む [派生] 
\\	…⁴を飲み込む
\\	【自】①《3格》⦅文語⦆
\\	に〕抵抗する、逆らう、耐える 
\\	誘惑に負けない
\\	3格+
\\	に危害を与えることができない〈与えることができる〉
\\	1【他】①〔…⁴を〕苦しめる 
\\	私は咳に苦しんでいる 
\\	(単
\\	/(複1・2・4)- (複
\\	精通している人、専門家、玄人、通 
\\	/〔女性〕-
\\	…⁴を知っている
\\	3格+
\\	に匹敵する(㊟
\\	は形式目的語)
\\	1【他】〔…⁴を〕きれいにする;清掃する、洗う、浄化する、クリーニングする 
\\	[派生] 
\\	きれいにすること;清掃,洗浄
\\	1【他】①《4格》〔…⁴に〕追いつく 
\\	彼は泥棒に追いついた
\\	【形】一義的な;明白な
\\	なんの不思議もない、当然だ
\\	【形】① 涼しい 
\\	冷たい
\\	【形】① 厚かましい、無遠慮な;[小]生意気な
\\	【他】①〔…⁴を〕催す、開催する ②⦅口語⦆〔…⁴を〕行なう、する [派生] 
\\	開催 
\\	催し物
\\	(または
\\	即座に、すぐに
\\	【副】今日(こんにち)では、近ごろ[では] 
\\	【形】ずうずうしい、厚かましい 
\\	彼はどんどんつけ上がっていった 
\\	(単
\\	/複数なし 
\\	尊敬、敬意、畏敬[の念] 
\\	(単
\\	/(複1・2・3・4)‐
\\	逃げ場、避難所 
\\	3格 
\\	を避ける場所
\\	(単
\\	/(複1・2・3・4)‐
\\	①〔複数なし〕監督、監視 
\\	警察の監視下にある
\\	3格+
\\	に反抗して;危険など
\\	をものともせず
\\	(単
\\	/(複1・2・3・4)‐
\\	①〔複数なし〕探すこと、捜索、追求 
\\	3格 
\\	を探している 
\\	…⁴を探す,捜索する 
\\	Ⅱ【副】① とっくに、とうの昔に、ずっと前に 
\\	それを私はもうとっくに知っている
\\	1【他】①〔…⁴を〕(感覚的に)感じる 
\\	…⁴をすすぐ,洗浄する
\\	【形】① ブロンドの、金髪の 
\\	の現在分詞)【形】魅力的な、チャーミングな 
\\	チャーミングな女の子
\\	(単
\\	/(複1・2・3・4)‐
\\	⦅やや古語⦆ 薬 
\\	薬を飲む 
\\	(単
\\	/(複1・2・4)- (複
\\	頭蓋骨、どくろ 
\\	②⦅口語⦆ 頭 
\\	私は頭がずきずき〈がんがん〉する
\\	(単
\\	/(複1・2・3・4)‐ 
\\	影 
\\	②〔複数なし〕陰、日陰 
\\	⦅口語⦆ 一体何が、一体何を
\\	1【他】①〔…⁴を〕(暴力・脅迫により)奪う、強奪する 
\\	強盗、盗賊
\\	【他】①〔…⁴を〕援助する、支援する、支持する 
\\	若い芸術家を金銭的に援助する
\\	⦅話⦆ 頭がおかしい 
\\	鉢、椀、ボウル)
\\	…すればするほど、それだけいっそう…
\\	【自】① 足りる、十分である 
\\	蓄えは十分ある
\\	(単
\\	/(複1・2・4)-
\\	(複
\\	(同種類の物・動植物の中の)個体 
\\	象はその辺りにはもう二三頭しかいなかった
\\	【他】②〔…⁴を〕確保する;保障する
\\	(単
\\	/(複
\\	怪物、怪獣、モンスター 
\\	〔現在〕
\\	〔過去〕
\\	〔過分〕
\\	【他】①〔…⁴を〕(突然)襲う、襲撃する 
\\	彼は夜に襲撃された
\\	音声
\\	⦅口語⦆ さあ、始めろ!;仕事にかかれ!
\\	(単
\\	/(複1・2・4)
\\	(複
\\	虫(ミミズ・ウジ・寄生虫・ヒルの類)
\\	3格+
\\	⦅口語⦆ 
\\	に大目玉を〈一発〉食らわす
\\	(または
\\	⦅口語⦆(関係などが)壊れる、破綻する
\\	【形】① 朗らかな、明るい;陽気な
\\	1【再】
\\	〔…⁴に〕慣れる 
\\	私はその気候に慣れた
\\	⦅文語⦆ 災いをもたらす 
\\	不幸、災い)
\\	1【再】
\\	思い違いをする、間違う 
\\	私の思い違いでなければ
\\	(単
\\	/(複1・2・4)..
\\	(複
\\	包帯 
\\	③⦅軍事⦆(軍隊の)部隊;編隊
\\	4格+
\\	(または
\\	…⁴にうんざりしている 
\\	うんざりした)
\\	(または
\\	⦅口語⦆ わざわざお尋ねいただきありがとう! 
\\	照会、問い合わせ)
\\	【形】① 疑わしい、嫌疑のある 
\\	疑い,疑念
\\	(または
\\	私個人としては
\\	(または
\\	⦅口語⦆ 大きな口をきく 
\\	⦅話⦆ 大口をたたく)
\\	4格+
\\	⦅口語⦆ …⁴の尻をひっぱたく
\\	(感嘆して)これは驚いた!
\\	⦅話⦆ わが身を救う、(かろうじて)助かる
\\	(単
\\	/(複
\\	熱中症、熱射病 
\\	⦅口語⦆ さじを投げる、あきらめる 
\\	猟銃)
\\	⦅口語⦆ そんなことはありえない(←もしそのとおりなら私はほうきを食ってやる)
\\	2【再】②
\\	に〕加わる 
\\	グループに加わる 
\\	4格+
\\	…⁴に関しては、関していえば
\\	【他】《4格》〔…⁴に〕報いる 
\\	(単
\\	/(複1・2・4)
\\	(複
\\	②〔複数なし〕報い、報酬 
\\	(単
\\	/(複1・2・4)..
\\	(複
\\	富、財産 
\\	〔複数なし〕豊かさ
\\	(単
\\	/(複1・2・3・4)‐
\\	絹、絹糸、絹織物 
\\	(単
\\	/(複1・2・4)
\\	(複
\\	棒;杖
\\	⦅口語⦆ 予想以上に早く;あっという間に
\\	Ⅰ【形】放たれた、外れた 
\\	Ⅱ【副】①(出発・開始などを要求して)さあ 
\\	さあ、行こう!
\\	【形】①(木・葉などが)干からびた、枯れた;(土地などが)不毛の 
\\	⦅口語⦆ 車のスピードを上げる
\\	(単
\\	/(複1・2・4)..
\\	(複
\\	試合、競技 
\\	【形】時宜を得た、ちょうどよい時の、タイミングのよい
\\	遠慮しない;気楽にする
\\	【他】①〔…⁴を〕麻痺させる 【状態受動で】
\\	彼は右足が麻痺している〈半身不随だ〉
\\	【他】①〔…⁴を〕許す、許可する、容認する 
\\	タバコを吸ってもよろしいでしょうか? 
\\	…⁴を…に形づくる、形成する
\\	(単
\\	/複数なし 神経質 
\\	⦅口語⦆ 姿を現わす;顔を出す
\\	【形】① 心からの
\\	2【再】①
\\	すっかり変わる 
\\	〔…⁴に〕変わる
\\	(単
\\	/(複1・2・4)-
\\	(複
\\	②(時代・民族・階級などに特有の)服装 
\\	舞台衣装
\\	(単
\\	/(複1・2・4)- (複
\\	魔法使い 
\\	/〔女性〕
\\	【形】① 残酷な、残忍な 
\\	残酷な刑 
\\	【形】① 真っ暗な、やみの、暗い
\\	【形】今〈これ〉までの、従来の 
\\	今まで[は]、これまで[は]
\\	1【他】①〔…⁴を〕飲み込む 
\\	錠剤を飲み込む ②〔…⁴を〕吸い込む、吸収する
\\	【形】① この上なく幸せな、大喜びの 
\\	彼はこのニュースを聞いて大喜びした 
\\	2【再】①
\\	楽しむ 
\\	私たちはたっぷりと楽しんだ 
\\	とっとと消え失せろ!
\\	の現在分詞)【形】① 激怒している、怒り狂った 
\\	4格(または
\\	4格)
\\	…⁴のことを怒っている 
\\	観客を激怒させる
\\	【形】数年間の、何年もの
\\	【形】崇高な;荘厳な 
\\	崇高な思想〈感情〉
\\	【他】①〔仕事・業績など⁴を〕果たす、成し遂げる 
\\	〔…⁴を〕奮発して買う 
\\	〔…⁴を〕あえてする 
\\	(不適切なこと・ずうずうしいこと、身分不相応なぜいたくなど)をあえてする)
\\	はるかに、群を抜いて(㊟後ろに名詞化された最上級がくる) 
\\	隔たり、間隔) 
\\	はるかに, ずっと)
\\	(単
\\	/(複1・2・4)-
\\	(複
\\	②(料理などの)作り方、調理法、レシピ 
\\	1【他】①
\\	方向》⦅口語⦆〔…⁴を…に〕追いやる ②〔犯人など⁴を〕追いかける、追跡する ③〔…⁴を〕狩る
\\	…を確信している
\\	【副】⦅文語⦆ 今まで、これまで
\\	(単
\\	/(複1・2・4)-
\\	(複
\\	①〔ふつう単数で〕(動物の)毛 
\\	(単
\\	/(複1・2・4)-
\\	(複
\\	①(神の創造した)生物、被造物 
\\	1弱変化;(単2・3・4)‐
\\	/(複1・2・3・4)‐
\\	相続人;継承者、跡継ぎ 
\\	/〔女性〕
\\	法定相続人 [派生] 
\\	…⁴を相続する
\\	(単
\\	/複数なし 賞賛、感嘆 
\\	…⁴に感嘆する、感心する
\\	1【他】〔…⁴を〕つぶやく、ぼそぼそ〈ぶつぶつ〉言う 
\\	意味不明のことをぶつぶつ言う
\\	【副】ただ、単に
\\	(単
\\	/(複1・2・3・4)‐
\\	取り決め、協定 
\\	(単
\\	/(複1・2・4)-
\\	(複
\\	テント 
\\	Ⅰ【形】①(話・行為などが)ばかげた、愚かな;子供じみた 
\\	ばかげたふるまい
\\	【他】〔…⁴を〕可能にする
\\	【形】① 嫌な 
\\	困った
\\	(単
\\	/(複1・2・3・4)‐ 
\\	井戸 
\\	(単
\\	/(複1・2・3・4)‐
\\	城、城塞 
\\	([参考] 
\\	は中世の城郭建築で主に丘の上に築かれた。
\\	「城館」はルネサンス以降、主に平地や経済的中心地に築かれたもので
\\	より居住性が高く大規模。)
\\	〔現在〕
\\	〔過去〕
\\	〔過分〕
\\	【他】①〔…⁴を〕埋葬する 
\\	【状態受動で】
\\	彼はどこに埋葬されているのですか
\\	【副】① 途中で 
\\	旅行〈出張〉中で
\\	(単
\\	/(複1・2・3・4)‐
\\	確信、信念 
\\	【他】〔…⁴を〕予感する、漠然と感じる
\\	【自】① ののしる、悪態をつく ②《3格》⦅文語⦆
\\	を〕呪う
\\	(単
\\	/(複1・2・4)-
\\	(複
\\	手袋、グラブ 
\\	(単
\\	/(複1・2・4)-
\\	(複
\\	登場、出演 
\\	1【他】①〔…⁴を〕他の方向に向ける ②〔注意など⁴を〕そらす
\\	「意味する」の現在分詞)【形】① 重要な、重大な 
\\	(または
\\	⦅口語⦆ 一言もしゃべらない
\\	(単
\\	/(複1・2・4)
\\	(複
\\	①(馬の)鞍 
\\	②(自転車などの)サドル
\\	(単
\\	/(複1・2・3・4)‐
\\	事実 
\\	2格+
\\	の代わりに
\\	〔現在〕
\\	〔過去〕
\\	〔過分〕
\\	1【他】①〔…⁴を〕予定している(㊟ふつう状態受動で)
\\	③〔…⁴を〕(法律などがあらかじめ)定めている 
\\	規定をいくつか変更することが予定されている 2【再】
\\	に〕気をつける、用心する
\\	〔現在〕
\\	〔過去〕
\\	〔過分〕
\\	1【自】〔完了
\\	①(部屋などに)入る 
\\	(単
\\	/(複1・2・3・4)‐
\\	けが、負傷 [派生] 
\\	…⁴にけがをさせる
\\	(単
\\	/(複1・2・3・4)‐
\\	①(時代・階級・職業・地域などに特有な)服装、身なり 
\\	(または
\\	(衣服などが)ぴったりと合う
\\	(単
\\	/(複
\\	または..
\\	疑い、疑念、嫌疑、容疑 
\\	私は…ではないかという疑念をいだいている 
\\	疑わしい
\\	(または
\\	時間ぎりぎりになって
\\	(単
\\	/(複1・2・4)- (複
\\	ふつう複数で 手綱 
\\	馬の手綱を引く
\\	1【他】〔…⁴を〕いたわる、大事にする
\\	【形】才能のある、有能な、能力のある 
\\	(単
\\	/複数なし 歓声、歓呼;歓喜
\\	【他】《4格》〔…⁴に〕教える、わからせる 
\\	彼は私にこの地の交通規則を教えた
\\	(単
\\	/(複1・2・3・4)‐
\\	①〔複数なし〕徳、美徳 
\\	弱変化;(単2・3・4)‐
\\	/(複1・2・3・4)‐
\\	愚か者、ばか 
\\	/〔女性〕
\\	道化
\\	相変わらず、依然として
\\	4格	…⁴を考慮して、…⁴に関して
\\	推察、憶測;考察 
\\	【他】〔…⁴を〕見張る、監視する;警備する 
\\	[派生] 
\\	…⁴を監視する,見張る
\\	③(人との)結び付き、関係 [派生] 
\\	…⁴を…
\\	と結び付ける
\\	【形】混乱した、混とんとした、無秩序な 
\\	(単
\\	/(複1・2・3・4)‐
\\	頼み、願い 
\\	(単
\\	/(複1・2・4)-
\\	(複
\\	王座 
\\	1【他】〔…⁴を〕暖める、温める 
\\	【形】非常に優れた、卓越した、すばらしい 
\\	3格+
\\	に勇敢に立ち向かう 
\\	額、おでこ)
\\	(単
\\	/複数なし 
\\	華やかさ、華麗さ、壮麗さ
\\	⦅文語⦆ こう言っては失礼ですが
\\	【形】① 喜んでいる、うれしい 
\\	[派生語] 
\\	喜び、うれしさ
\\	(単
\\	/(複1・2・3・4)‐
\\	チャンス、好機、機会
\\	(単
\\	/複数なし 感激、熱狂、興奮 
\\	…⁴を感激させる、熱狂させる
\\	1【他】〔…⁴を〕保証する 
\\	【自】② 震える
\\	【副】同時に
\\	4格+
\\	…⁴をたたき壊す〈ずたずたに裂く〉
\\	【他】①〔…⁴を〕おろそかにする、なおざりにする、ほったらかしにする 
\\	家族をほったらかしにする
\\	(単
\\	/(複
\\	一人っ子 
\\	【再】
\\	不定詞句》〔…することを〕拒む、拒絶〈拒否〉する 
\\	彼は私の助言に従うことをかたくなに拒んだ 
\\	…⁴を拒否する、拒絶する
\\	(単
\\	/複数なし 同情、哀れみ 
\\	3格 
\\	(または
\\	を気の毒に思う
\\	犯人を逮捕する
\\	発言〈決定〉権をもつ
\\	3格 
\\	(または
\\	を厳しく処罰する;激しく非難する
\\	〔過去〕
\\	〔過分〕
\\	【他】①
\\	不定詞句》〔…⁴に…することを〕指示する、命令する 
\\	私はだれも中に入れないようにと彼に命じた
\\	【再】①
\\	の〕仲間に加わる 
\\	彼は私たちの仲間に加わった
\\	死ぬ運命にある
\\	【形】①⦅文語⦆(事実などに)忠実な、正確な 
\\	②⦅古語⦆(人が)忠実な、誠実な
\\	【副】①(空間的に)先頭に、先行して 
\\	私たちはいつも先生を先頭に歩いた 
\\	前方へ
\\	【他】
\\	⦅文語⦆
\\	に…⁴を〕与える、授ける 
\\	彼らは彼に全権を与えた。 
\\	(単
\\	/(複
\\	研究者、学者(〔女性〕-
\\	研究する 
\\	(単
\\	/(複1・2・3・4)‐
\\	廃墟、崩れた建物、遺跡 
\\	多すぎる、あまりにも多い
\\	(単
\\	/(複1・2・3・4)‐
\\	①⦅文語⦆ 使命
\\	⦅口語⦆ さっさと話したらどうだ!;いいかげんに白状しろ! 
\\	言語、言葉)
\\	【他】〔…⁴を〕意図する、もくろむ 
\\	(…する)つもりである 
\\	非分離〔過去〕
\\	〔過分〕
\\	1【他】①〔…⁴を〕覆う 
\\	(または
\\	⦅口語⦆(もはやだれの興味も引かない)過去のもの
\\	1【他】〔…⁴を〕退屈させる 
\\	退屈な、つまらない
\\	【形】① 謎に満ちた、神秘的な 
\\	いわくありげな
\\	【形】落ち着きのない、せわしない 
\\	〔過去〕
\\	〔過分〕
\\	2【自】①〔完了
\\	または
\\	(弧・曲線を描くように)揺れる 
\\	振り子が揺れる
\\	【形】① 臆病な、意気地のない 
\\	臆病なやつ
\\	【副】どこにも…ない 
\\	彼の姿はどこにも見当たらなかった
\\	4格 
\\	…⁴に打ち勝つ
\\	非分離〔現在〕
\\	〔過去〕
\\	〔過分〕
\\	【他】①〔…⁴を〕あとに残す 
\\	伝言を残す
\\	ある意味では 
\\	ある、ある種の)
\\	【他】〔…⁴を〕摘む、摘み取る 
\\	〔過去〕
\\	〔過分〕
\\	〔接Ⅱ〕
\\	1【他】①
\\	に…⁴を〕命令する 
\\	3格 
\\	と張り合う(㊟
\\	は形式目的語)
\\	〔過去〕
\\	〔過分〕
\\	【他】②〔…⁴を〕(正しいと)認める
\\	1【他】①〔…⁴を〕犠牲にする 
\\	2【再】①
\\	自分を犠牲にする
\\	1【他】①
\\	に高価なもの・仕事など⁴を〕ゆだねる、任せる 
\\	君は彼にその店を任せるつもりなのかい?
\\	(単
\\	/(複1・2・4)-
\\	(複
\\	劇、演劇、芝居 
\\	(単
\\	/(複1・2・3・4)‐
\\	教訓、教え 
\\	⦅雅⦆ 息をひきとるまで 
\\	呼吸)
\\	①〔ふつう単数で〕援助、支援 
\\	…⁴を援助する,支援する
\\	1【他】①
\\	に…⁴を〕保証する、断言する 
\\	彼女はあしたまた来るとぼくに約束した
\\	【副】疑いもなく、明らかに
\\	Ⅰ【形】① こっけいな、おかしい 
\\	笑い物になる
\\	きわめて…である(㊟
\\	のあとには形容詞がくる)
\\	【自】〔完了
\\	①(あとに)残る 
\\	税引き後7000ユーロが彼の手元に残る ③《場所》〔…に〕とどまる
\\	(単
\\	/複数なし 慰め 
\\	それはたいして慰めにならない
\\	3格+
\\	(または
\\	に役立つ 
\\	…⁴を果たす)
\\	【他】①〔…⁴を〕残念に思う、後悔する ②〔…⁴を〕気の毒に思う 
\\	彼は本当に気の毒だ 
\\	〔現在〕
\\	〔過去〕
\\	〔過分〕
\\	【他】①〔…⁴を〕理解する、(納得して)わかる 
\\	どうして彼に謝らなければならないのか私にはわからない ②〔誤りなど⁴を〕悟る 
\\	最後には彼も自分の誤りを悟った 
\\	認可、許可 
\\	3格+
\\	を苦しめる
\\	(単
\\	/(複1・2・3・4)‐
\\	フェリー、渡し船 
\\	(単
\\	/(複
\\	港町 
\\	1【自】①〔完了
\\	または
\\	(船・人が)帆走する、(帆船・ヨットに乗って)走る 
\\	[派生] 
\\	帆
\\	(または
\\	そのとおりだ
\\	⦅口語⦆ 猫も杓子も、だれも彼も
\\	停泊している
\\	(単
\\	/(複1・2・4)-
\\	(複
\\	大洋、海洋 
\\	弱変化;(単2・3・4)‐
\\	/(複1・2・3・4)‐
\\	船員 
\\	…を船員として雇い入れる
\\	3格 
\\	を待ち受けて見張る 
\\	見張り)
\\	(単
\\	/(複1・2・3・4)‐
\\	情熱 
\\	【自】
\\	〔…⁴に〕同意する、〔…⁴を〕承諾する 
\\	彼は私の提案に同意した
\\	【形】① 意味のある、有意義な
\\	【自】① おしゃべりをする、雑談する 
\\	【自】〔完了
\\	①(ある場所を)通り抜ける、通過する
\\	【副】見たところ;どうやら(…らしい) 
\\	Ⅱ【副】きわめて、非常に
\\	【副】① ちょうど今 
\\	ちょうど今彼女が戸口から入って来るところだ 
\\	たった今 
\\	(単
\\	/(複1・2・4)- (複
\\	①(文書の真正の証しとしての、また封をするための)印、印章 
\\	皇帝の印璽(いんじ) 
\\	勝つ
\\	【形】① 実り豊かな、実りを生む、(土地が)肥沃な
\\	(単
\\	/(複1・2・4)..
\\	(複
\\	①〔ふつう単数で〕(競技・技術などでの)リード、優位 
\\	突出部、張り出し
\\	(単
\\	/(複
\\	灯台 
\\	(単
\\	/複数なし 
\\	立ち入り、入場;入場許可 
\\	正気を失う 
\\	知力;分別)
\\	(単
\\	/(複1・2・4)-
\\	(複
\\	動機 
\\	②⦅美術・文学⦆ 主題、題材
\\	〔現在〕
\\	〔過去〕
\\	〔過分〕
\\	1【自】(程度・機能が)弱まる、衰える、静まる 
\\	(イメージ:
\\	《劣等・低減》)
\\	1(単
\\	/(複1・2・4)
\\	(複
\\	落下、転落、墜落 
\\	[派生] 
\\	転落する,墜落する
\\	【副】⦅文語⦆① かつて、以前、昔 
\\	(単
\\	/(複
\\	お世辞 
\\	…にお世辞を言う 
\\	【他】①〔…⁴を〕全滅させる、根絶する、壊滅させる 
\\	②〔書類・手紙など⁴を〕破棄する、処分する
\\	3格+
\\	発音×	⦅口語⦆ 
\\	をやっつけようとする
\\	4格+
\\	…⁴の扱いを心得ている;…⁴を掌握している
\\	4格+
\\	(または
\\	…⁴を屈服させる
\\	(単
\\	/複数なし 激しい欲望、渇望;貪欲
\\	(単
\\	/(まれに複
\\	または
\\	泥、ぬかるみ 
\\	(単
\\	/(複1・2・4)
\\	(複
\\	沼地、湿地 
\\	(単
\\	/(複
\\	競技会、コンテスト 
\\	1【他】①〔…⁴に〕耳を傾ける;〔…⁴を〕傾聴する 
\\	3と】
\\	私は講演を傾聴した2【再】⦅口語⦆
\\	状態》〔…のように〕聞こえる 
\\	その音はまるで雷のようだった
\\	全身を耳にして聞く
\\	【他】①〔…⁴を〕率いる、指揮する 
\\	方向》〔…⁴を…へ〕導く 
\\	方向》〔…⁴を…へ〕案内する、連れて行く 
\\	指揮,監督
\\	①〔複数なし〕指揮、監督 
\\	会社の指揮をとる 
\\	業績、成績
\\	〔過去〕
\\	〔過分〕
\\	【自】〔完了
\\	よける、道を空ける 
\\	右へよける
\\	【他】
\\	〔…⁴に…
\\	に気をつけるように〕警告する、注意する 
\\	彼は私にすりに注意するようにと言った
\\	1【自】②《3格》
\\	を〕(お世辞などが)うれしがらせる、〔自尊心
\\	を〕くすぐる 【過去分詞で】
\\	私は彼のスピーチを聞いて悪い気がしなかった
\\	(単
\\	/(複1・2・4)- (複
\\	称号;⦅スポーツ⦆ タイトル ([略]
\\	②(本・映画などの)題名、表題
\\	(単
\\	/(複
\\	虹 
\\	1【自】②〔…へ〕ぱちん〈かしゃり〉と音を立てて動く2【他】②⦅口語⦆〔…⁴を〕さっとつかむ、すばやく取る 
\\	彼女はバッグをさっとつかんで行った
\\	(単
\\	/(複1・2・4)
\\	(複
\\	呪い 
\\	逃走、逃亡
\\	⦅口語⦆ しゃべってしまえ
\\	【他】①
\\	〔…⁴を…
\\	で〕覆う 
\\	床にじゅうたんを敷く ③〔…⁴を〕(資料によって)証明する
\\	【形】①⦅宗教⦆ 聖なる 
\\	②⦅文語⦆ 神聖な
\\	(単
\\	/(複1・2・3・4)‐
\\	泉;水源 ③〔ふつう複数で〕出典、原典;文献、資料 
\\	(単
\\	/(複1・2・3・4)‐
\\	泡、水泡、気泡 
\\	(単
\\	/(複1・2・4)-
\\	(複
\\	瞬間、一瞬 
\\	(単
\\	/(複1・2・4)-
\\	(複
\\	①〔ふつう単数で〕毛皮、獣皮 
\\	キノコ 
\\	【他】〔…⁴を〕汲む、すくう 
\\	【他】①〔…⁴を〕引き起こす、もたらす 
\\	その発見が転機をもたらした 
\\	(イメージ:
\\	こちらへ+
\\	…を結果としてもたらす)
\\	抵抗する
\\	(単
\\	/(複1・2・4)..
\\	(複
\\	①〔複数なし〕抵抗、反抗 
\\	(単
\\	/(複1・2・4)..
\\	(複
\\	欲求、必要;需要 
\\	【他】②
\\	に不快なこと⁴を〕免れさせる;
\\	から手間など⁴を〕省く 
\\	私は君に嫌な思いをさせたくなかったんだ 
\\	問題になる 
\\	問題)
\\	【他】①
\\	のために…⁴を〕手に入れる、
\\	に…⁴を〕調達する、世話をする 
\\	【形】③(身体などが)頑健な、タフな
\\	1【自】〔完了
\\	忍び足で歩く 
\\	②(事態の)変化、転換 
\\	(事態などの)悪化
\\	【自】〔完了
\\	①(船・天体などが)沈む、沈没する 
\\	下降する、下りる 
\\	(単
\\	/(複1・2・4)-
\\	(複
\\	毒、毒物、毒薬 
\\	【自】〔完了
\\	①〔…へ・を〕滑る、滑走〈滑空〉する 
\\	(単
\\	/(複1・2・4)-
\\	(複
\\	1〔ふつう単数で〕⦅航空⦆ 機内;⦅海語⦆ 舷側(げんそく)、甲板 
\\	深み、深い所 
\\	(単
\\	/(複1・2・3・4)‐
\\	大砲、カノン砲 
\\	【他】①〔…⁴を〕悩ます、煩わせる 
\\	(単
\\	/(複
\\	驚き、驚愕、恐怖 
\\	驚き〈恐怖〉のあまり 
\\	【形】いわゆる 
\\	(いわゆる温室効果)
\\	(単
\\	/(複1・2・4)- (複
\\	帆 
\\	幸福、幸運 
\\	(単
\\	/(複1・2・4)-
\\	(複
\\	畑、耕地 
\\	②〔複数なし〕野、野原 ⑤(チェス盤などの)升目 ⑧〔複数なし〕⦅やや古語⦆ 戦場 
\\	国家
\\	(単
\\	/複数なし 怒り、立腹 
\\	無力[感]
\\	1【他】〔…⁴の〕償いをする 2【自】
\\	〔…⁴の〕償いをする
\\	【他】①⦅文語⦆〔…⁴を〕見つける、認める 
\\	遠方に船を見つける
\\	〔過去〕
\\	〔過分〕
\\	〔接Ⅱ〕
\\	2【自】③〔完了
\\	〔…⁴に〕出くわす;〔…⁴を〕偶然発見する 
\\	私は町で昔の知人とばったり会った
\\	③〔複数なし〕優れた腕前、名人的技能
\\	【他】〔苦痛・悪状況など⁴を〕和らげる、鎮める;緩和する 
\\	痛みを和らげる
\\	(単
\\	/(複1・2・3・4)‐
\\	(肉体的・精神的な)苦痛、苦しみ 
\\	(または
\\	苦痛に耐える 
\\	(単
\\	/(複1・2・4)- (複
\\	時代、年代 
\\	3格+
\\	(または
\\	⦅口語⦆(罰として)…
\\	を何回も殴る
\\	Ⅰ【形】北の、北方〈北部〉の 
\\	(単
\\	/(まれに複
\\	灰 
\\	【他】②〔…⁴の〕お返しをする
\\	あたり一帯で;いたるところで
\\	【他】①〔行列など⁴を〕先導する;〔部隊など⁴を〕率いる、指揮する [派生] 
\\	…⁴を率いる,統率する
\\	(単
\\	/(複1・2・3・4)‐
\\	嘆き 
\\	息子を失った母親の嘆き
\\	(単
\\	/(複1・2・4)-
\\	(複
\\	10年 
\\	「送る」の過去分詞)【形】① 器用な、熟練した、じょうずな [派生] 
\\	器用;巧みさ、熟練
\\	(単
\\	/(複1・2・3・4)‐
\\	筋肉 
\\	(単
\\	/(複1・2・3・4)‐
\\	①(植物の)とげ 
\\	【形】① 買うことのできる、売り物の 
\\	【他】①〔…⁴を〕(展示・紹介などのために)見せる、披露する 
\\	②〔…⁴を〕上演する、上映する
\\	【形】①(記憶・感覚などが)はっきりしない 
\\	不確かな
\\	(単
\\	/複数なし 瓦礫、石くず
\\	2【他】①〔…⁴を〕統治する、支配する 
\\	政府;統治,支配
\\	(単
\\	/(まれに複
\\	好意、愛情 
\\	彼女は私に好意を示した
\\	①〔複数なし〕支配、統治 
\\	【副】無駄に 
\\	努力はすべて無駄だった
\\	1【他】〔…⁴を〕(棒などで)殴る 
\\	今までずっと
\\	どんな場合でも、必ず
\\	(単
\\	/(複1・2・4)- (複
\\	①(格闘技などの)選手 
\\	/〔女性〕-
\\	〔…で〕名を成す、〔…で〕有名になる
\\	(単
\\	/(複1・2・4)-
\\	(複
\\	目的、ねらい 
\\	目的を達成する
\\	【代】《相互》〔不変化〕互いに
\\	(単
\\	/(複1・2・3・4)‐
\\	仮面、面 
\\	(特定の地域・職に)立候補する
\\	Ⅰ【形】考えられる 
\\	考えられる解決策 Ⅱ【副】非常に、きわめて
\\	(単
\\	/(複1・2・3・4)
\\	(まれに複
\\	原理、原則 
\\	【他】①
\\	に…⁴を〕割り当てる、割りふる
\\	(単
\\	/(複
\\	くじ引き、抽選 
\\	(単
\\	/(複1・2・3・4)‐
\\	平地、平野、台地 
\\	レベル、水準 
\\	1【他】①《4格》〔…⁴に〕参加する、加わる 
\\	2【自】①⦅口語⦆いっしょにやる 
\\	君もいっしょにやるかい? 
\\	本題に入る
\\	(単
\\	/(複1・2・3・4)‐
\\	箱;木箱 
\\	(単
\\	/(複1・2・3・4)‐ ①〔複数なし〕楽しみ、喜び 
\\	お目にかかれてうれしゅうございました 
\\	(または
\\	⦅口語⦆ そばで指をくわえて見ている(何ももらえない) 
\\	管、パイプ)
\\	役割を演じる;重要である
\\	(単
\\	/(複1・2・4)-
\\	(複
\\	くじ;宝くじ 
\\	〔現在〕
\\	〔過去〕
\\	〔過分〕
\\	1【他】③
\\	〔感情など⁴を…
\\	に〕ぶちまける 
\\	(イメージ:
\\	外へ 
\\	出す,放つ)
\\	⦅話⦆ 横になる、寝床に入る
\\	緊張;期待感 
\\	(単
\\	/(複1・2・3・4)‐
\\	10億([略]
\\	(単
\\	/(複1・2・3・4)‐
\\	表面、表層 
\\	〔現在〕
\\	〔過去〕
\\	〔過分〕
\\	Ⅰ【他】②(時を)寝過ごす;
\\	寝過ごして(…を)逸する 
\\	2【他】②⦅口語⦆〔…⁴を〕うっかり忘れる
\\	2【再】①
\\	横になる
\\	⦅卑⦆ …をさんざん打ちのめす 
\\	かゆ;かゆ状のもの
\\	4格+
\\	…⁴に注意している
\\	危険な状態にある
\\	世間の評判になっている
\\	(または
\\	生まれる
\\	(単
\\	/(複1・2・4)- (複
\\	雷、雷鳴 
\\	(単
\\	/複数なし 
\\	恥、不面目、不名誉 
\\	私に恥ずかしい思いをさせないでくれ!
\\	(単
\\	/(複1・2・3・4)‐
\\	王冠、冠 
\\	1【再】
\\	苦情を言う、不平を言う 
\\	【他】①〔…⁴を〕要求する;〔犠牲など⁴を〕払わせる 
\\	その事故は多くの犠牲者を出した
\\	〔過去〕
\\	〔過分〕
\\	【自】〔完了
\\	(ドイツ南部・オーストリア・スイスでは
\\	も)〕《3格》
\\	を〕助ける 
\\	またもや
\\	絶えず;ずっと 
\\	さらに先へと、引き続き)
\\	1【自】②
\\	に〕似合う、ふさわしい 
\\	その靴は君のコートに合っている 
\\	にとって〕都合が良い;好ましい
\\	3格+
\\	に薄気味悪く〈不気味に〉感じられる
\\	ほかならぬ…が
\\	⦅口語⦆ 健康を取り戻している
\\	3格+4格+
\\	に…⁴を自由に使用させる、提供する
\\	弱変化;(単2・3・4)‐
\\	/(複1・2・3・4)‐
\\	または(単
\\	/(複1・2・3・4)‐
\\	スズメ 
\\	驚いた!、なんてことだ!
\\	(単
\\	/(複
\\	ゾンビ(邪悪な力で生き返らされた死者)
\\	(単
\\	/(複
\\	種類;⦅生物⦆ 種(しゅ) 
\\	液体 
\\	の現在分詞)【形】〔1格と4格では格語尾は付かないことがある〕① 十分な、申し分のない 
\\	では
\\	となっていた)
\\	生存している
\\	3格 
\\	かっとなって…
\\	をしてしまう、
\\	(強い感情にかられて)思わず…をする 
\\	…⁴を魅了する、うっとりさせる)
\\	(単
\\	/(複1・2・4)..
\\	(複
\\	①〔複数なし〕終了;完了 
\\	2【再】①
\\	⦅軽蔑して⦆
\\	と〕かかわり合う、つきあう 
\\	あの男とはつきあうな!
\\	〔現在〕
\\	〔過去〕
\\	〔過分〕
\\	1【他】①
\\	⦅文語⦆
\\	に…⁴を〕指し示す、(指し示して)教える、見せる 
\\	外国人に道を教える 
\\	…⁴を示す
\\	1【他】
\\	状態》〔…⁴を…に〕形づくる、形成する 
\\	部屋を華麗に形づくる
\\	⦅口語⦆ 役に立たない
\\	(単
\\	/複数なし 所有物、財産;所有権 
\\	3格 
\\	音無	
\\	の才能がある
\\	1【他】①
\\	方向》〔…⁴を…に〕向ける 
\\	(単
\\	/(複1・2・4)- (複
\\	強盗、盗賊 
\\	4格+
\\	…⁴を瓦礫の山にする;完全に破壊する 
\\	瓦礫、
\\	灰)
\\	1【自】① どなる、わめく ②(動物が)うなる
\\	(単
\\	/(複
\\	署名、(有名人の)サイン 
\\	(単
\\	/(複1・2・4)
\\	(複
\\	こぶし、拳骨 
\\	(単
\\	/(複1・2・4)-
\\	(複
\\	切り傷;切り口
\\	(単
\\	/(複1・2・4)..
\\	(複
\\	滝、瀑布(ばくふ) 
\\	1【他】①《4格》〔…⁴に〕毒を入れる;〔大気など⁴を〕汚染する 
\\	【他】①〔火・字など⁴を〕消す 
\\	②〔…⁴を〕消し去る
\\	放射線 
\\	【形】無口な、寡黙な [派生] 
\\	黙っている,黙る
\\	3格 
\\	と関係がある〈ない〉
\\	2【再】
\\	に〕献身する、専念する、没頭する 
\\	家事に専念する
\\	部分的には、一部は([略]
\\	(単
\\	/(複
\\	または-
\\	トースト[用食パン] 
\\	いつか別な時に
\\	(単
\\	/複数なし 生クリーム、乳脂 
\\	(単
\\	/(複1・2・4)-
\\	(複
\\	生産物、製品 
\\	(単
\\	/複数なし 暗算 
\\	(単
\\	/(複
\\	(スープを鍋からすくうための)ひしゃく、スープレードル 
\\	(単
\\	/(複1・2・4)..
\\	(複
\\	弁護士 
\\	/〔女性〕
\\	〔現在〕
\\	〔過去〕
\\	〔過分〕
\\	3【再】
\\	自制する;控え目な態度をとる 
\\	(または
\\	彼は酒の量を少し控えたほうが良い
\\	(単
\\	/(複
\\	①〔複数なし〕⦅スポーツ⦆ 走り幅跳び 
\\	(単
\\	/(複1・2・3・4)‐(ドイツ南部・オーストリアでは複
\\	も) 
\\	弧、カーブ 
\\	弓 
\\	1【自】〔過去〕
\\	〔過分〕
\\	次第である、
\\	に〕左右される 
\\	2【他】〔過去〕
\\	〔過分〕
\\	①〔…⁴を〕取り外す
\\	形容詞(または副詞)+
\\	できるだけ…
\\	3格+
\\	⦅話⦆ …にうそ八百を並べる、
\\	に一杯食わせる、⦅口語⦆ 
\\	をぺてんにかける 
\\	にうそなど⁴を信じ込ませる)
\\	⦅口語⦆ もう一度
\\	【形】① 多色の;色とりどりの、カラフルな 
\\	②(白・黒・灰色以外の)色のついた 
\\	目に見える 
\\	明白な、明らかな
\\	4格+
\\	⦅口語⦆ …⁴を盗む 
\\	いっしょに行く)
\\	(単
\\	/(複1・2・3・4)‐
\\	真珠 
\\	②(真珠の形をした)小さい玉 
\\	(単
\\	/(複
\\	海底 
\\	細部 
\\	【形】① 不適切な、不都合な;間違った 
\\	場違いである 
\\	(単
\\	/複数なし 海 
\\	それだけでなく、その上さらに
\\	動き出す 
\\	動き)
\\	①〔ふつう複数で〕努力、骨折り 
\\	【形】① よく知られた、よく使われる 【3格と】
\\	この概念は私はよく知らない
\\	(単
\\	/(複
\\	専門用語 
\\	3格+
\\	を訪問する 
\\	…⁴を行なう)
\\	(単
\\	/(複1・2・3・4)‐
\\	円柱、柱 
\\	大理石の円柱 
\\	酸味 
\\	(単
\\	/(複1・2・4)- (複
\\	同行者、連れ、お供 (〔女性〕-
\\	2【他】②《4格》〔…⁴に〕(教科を)教える、授業をする 
\\	彼はこのクラスにもう3年間[ドイツ語を]教えている
\\	〔過去〕
\\	〔過分〕
\\	【自】《3格》
\\	を〕残念がらせる、申し訳なく思わせる 
\\	行けなくて申し訳ありません
\\	【形】怒った、腹を立てている 
\\	[派生] 
\\	怒り
\\	〔現在〕
\\	〔過去〕
\\	〔過分〕
\\	1【他】①〔…⁴を〕止める、停止させる 
\\	2【自】① 止まる、停止する
\\	3格 
\\	に習熟する、慣れる 
\\	親しい、親密な)
\\	Ⅱ【副】⦅文語⦆② 更に、加えて 
\\	デザートを作るのにイチゴと砂糖、更にはワインと生クリームが必要です
\\	【形】詳しい、詳細な 
\\	(単
\\	/(複1・2・4)-
\\	(複
\\	現在地、所在地
\\	【形】② 活動的な、活発な;活動中の
\\	【他】〔…⁴を〕付け加える、添える 
\\	【他】①《4格》〔…⁴に〕火をつける 
\\	〔過去〕
\\	〔過分〕
\\	1【他】①〔…⁴を〕鍵で開ける 
\\	部屋の鍵を開ける
\\	(言うこと・することが)正しい 
\\	君は正しい
\\	(単
\\	/複数なし 下着[類]、肌着[類] 
\\	(単
\\	/(複1・2・4)-
\\	(複
\\	対象、対象物
\\	(単
\\	/(複1・2・3・4)‐
\\	⦅文語⦆(建物の)階、フロアー 
\\	【他】〔…⁴を〕甘やかす 
\\	彼は息子を甘やかしすぎる
\\	(単
\\	/(複1・2・3・4)‐ 
\\	未婚の若い女性、娘、お嬢さん 
\\	…嬢、…さん(未婚の女性に対する敬称で、姓または名の前に添えて用いる)([略]
\\	許し;謝罪 
\\	…⁴を許す;
\\	謝る
\\	【自】〔完了
\\	(短距離を)全力疾走する 
\\	〔過去〕
\\	〔過分〕
\\	1【自】〔完了
\\	〔…⁴に〕向かって行く
\\	(単
\\	/(複1・2・4)..
\\	(複
\\	障害、困難、邪魔;障害物 
\\	〔過去〕
\\	〔過分〕
\\	1【他】①〔困難・感情など⁴を〕克服する、乗り越える 
\\	障害を克服する 
\\	火遊びをする;軽率なことをして危険を冒す
\\	Ⅰ【副】② …する方がよい
\\	1【他】①〔…⁴を〕マスターする、〔技能など⁴に〕習熟する 
\\	楽器を使いこなす、
\\	楽器に習熟する
\\	1【他】①〔…⁴の〕スイッチを切る 
\\	②〔…⁴を〕排除する、締め出す
\\	【接】《従属》〖定動詞後置〗①(手段・方法)…によって、…という方法で 
\\	母親は子供をなでて落ち着かせた
\\	1【他】①〔攻撃・質問など⁴を〕かわす、防ぐ 
\\	(単
\\	/(複1・2・3・4)‐ 
\\	印;目印
\\	(単
\\	/(複1・2・3・4)‐
\\	爆弾 
\\	(単
\\	/(複
\\	悪夢 
\\	(単
\\	/(複
\\	一角獣、ユニコーン(伝説上の動物;貞潔のシンボル) 
\\	困難、難しさ 
\\	【他】①
\\	に…⁴の〕借りがある 
\\	私は彼に10ユーロ借りがある
\\	1【他】①《4格》〔…⁴に〕触れる、触る 
\\	ちょっと私の手を触ってごらん! 
\\	【形】効果的な、有効な 
\\	(単
\\	/(複1・2・4)..
\\	(複
\\	商人、商売人;ビジネスマン (〔女性〕
\\	(単
\\	/(複1・2・4)
\\	(複
\\	購入、買うこと 
\\	(単
\\	/(複1・2・4)
\\	(複
\\	①〔複数なし〕商売、貿易、商業 
\\	商人,商売人
\\	【他】〔…⁴を〕終える、済ます 
\\	仕事を終える
\\	(単
\\	/(複
\\	選択[権]、オプション 
\\	(単
\\	/複数なし 
\\	所有物、財産 
\\	所有 
\\	〔形容詞変化〕① 旅行者;旅客 
\\	Ⅰ【副】徐々に、次第に
\\	(単
\\	/(複
\\	①(スイッチなどの)レバー 
\\	霧 
\\	(単
\\	/(複1・2・3・4)‐
\\	(まれに複
\\	厚紙の箱、ボール箱
\\	始めと終わり(ギリシア語アルファベットの第1字
\\	と最終字
\\	聖書:黙1,8);⦅比⦆ 最重要事、核心、真髄
\\	【形】① 耳の聞こえない、耳の不自由な 
\\	だますこと、詐欺 
\\	〔過去〕
\\	〔過分〕
\\	【自】①《3格》〔誘惑など
\\	に〕屈しない、打ち勝つ 
\\	私はその誘惑に負けた ②《3格》
\\	に〕抵抗する
\\	【自】①
\\	〔人の死など⁴を〕悲しむ、悼む 
\\	彼は父の死を悲しむ
\\	3格	
\\	とは反対に 
\\	反対、逆)
\\	【副】夜に 
\\	1【他】②〔…⁴を〕横切る;(道などが)〔…⁴と〕交差する
\\	とんでもない;まったくその逆ですよ!
\\	3格 
\\	⦅口語⦆ 
\\	を追跡している 
\\	の後をつけ回している 
\\	をほしがっている
\\	【他】②〔欲求など⁴を〕満たす ③〔痛み⁴を〕静める
\\	弱変化;(単2・3・4)‐
\\	/(複1・2・3・4)‐
\\	兵士、軍人 
\\	/〔女性〕
\\	【形】① 中立の、公平な
\\	(単
\\	/(複1・2・4)- (複
\\	(海・川・湖などの)岸、岸辺 
\\	【形】渇望している、貪欲な 
\\	激しい欲望,渇望;貪欲
\\	〔現在〕
\\	〔過去〕
\\	〔過分〕
\\	【他】〔図案など⁴の〕下絵を書く、〔…⁴を〕デザインする 
\\	服をデザインする
\\	【形】極度の、極端な;過激な 
\\	Ⅰ【形】① 密な、(間隔の)詰まった 
\\	Ⅱ【副】密に;(空間的・時間的に)密接して 
\\	【自】〔完了
\\	①〔…へ・から〕(液体・気体が)勢いよく流れる 
\\	だれよりも、人並みはずれて
\\	(単
\\	/(複1・2・4)
\\	(複
\\	火酒、シュナップス(アルコール分の強いブランデー・ウイスキー・焼酎のような蒸留酒の総称)
\\	〔過去〕
\\	〔過分〕
\\	【自】①(手を伸ばして)つかむ、取る 
\\	(食卓で)どうぞ手を伸ばして召し上がってください
\\	(単
\\	/(まれに複
\\	先導者、リーダー (〔女性〕-
\\	(単
\\	/(複1・2・3・4)‐
\\	①〔複数なし〕強さ 
\\	強み、長所
\\	1【他】①〔…⁴を〕強くする、補強する 
\\	〔過去〕
\\	〔過分〕
\\	【他】①〔…⁴を〕使う、使用する 
\\	(単
\\	/(複
\\	入門書、手引き、マニュアル 
\\	(単
\\	/(複
\\	道標、道しるべ 
\\	3格 
\\	と接触する 
\\	接触)
\\	【形】① その土地の、その地方〈地域〉の 
\\	その土地の特殊事情 [派生] 
\\	場所
\\	(単
\\	/(まれに複
\\	交換 
\\	学生〈意見〉の交換
\\	とっとと消えうせろ! 
\\	【再】立ち去る)
\\	(単
\\	/(複1・2・3・4)‐
\\	ヤギ、雌ヤギ 
\\	救助、救出 
\\	【他】①〔…⁴を〕侮辱する、〔…⁴の〕感情を害する 
\\	(単
\\	/(まれに複
\\	支払い 
\\	報酬、賃金 [派生] 
\\	…⁴の代金を支払う;…⁴に報酬を支払う
\\	(単
\\	/複数なし ⦅俗語⦆ 小便
\\	(単
\\	/(複1・2・3・4)‐
\\	①(糸と針のついた)釣りざお
\\	罰、刑罰;制裁 
\\	(単
\\	/(複1・2・3・4)‐
\\	または
\\	型、規準 
\\	図、図解
\\	(単
\\	/(複
\\	ふつう複数で ①(料理の)材料 
\\	(単
\\	/(複1・2・4)- (複
\\	こん棒、棒切れ
\\	(単
\\	/(複1・2・3・4)‐
\\	綱、ロープ、ひも 
\\	(単
\\	/(複1・2・3・4)‐ ①〔複数なし〕鉄、鉄分 ([記号]
\\	(単
\\	/(複1・2・4)-
\\	(複
\\	盾 4格+
\\	…⁴をたくらむ 
\\	(単
\\	/(複1・2・4)- (複
\\	粉、粉末;粉薬 
\\	「(サイズが)合う」の現在分詞)【形】① 適切な、ふさわしい
\\	(単
\\	/(複1・2・3・4)‐
\\	戦い、戦(いくさ) 
\\	戦場
\\	(単
\\	/(複1・2・4)-
\\	(複
\\	原稿、草稿 ([略]
\\	(単
\\	/(複1・2・3・4)‐
\\	①(木・石・金属などの)板、プレート 
\\	(単
\\	/(複1・2・3・4)‐
\\	死体、遺体 
\\	(単
\\	/(複1・2・4)
\\	(複
\\	1①⦅動物⦆ オオカミ 
\\	/〔女性〕
\\	〔過去〕
\\	〔過分〕
\\	【他】①〔…⁴を〕追い払う、追放する 
\\	ハエを追い払う 
\\	動詞の状態にする+
\\	追い立てる、追い出す)
\\	【他】①〔機械など⁴を〕分解する、ばらす ②〔動物など⁴を〕解体する;〔肉など⁴を〕小さく切り分ける
\\	(単
\\	/(複1・2・4)-
\\	(複
\\	①(童話などに登場する)小人(こびと) 
\\	【他】〔服など⁴を〕試着する 
\\	実際に、事実、ほんとうに
\\	(単
\\	/(複1・2・3・4)‐
\\	のど;食道、気管 
\\	【形】① 肉体的な、生理的な 
\\	〔現在〕
\\	〔過去〕
\\	〔過分〕
\\	1【自】①
\\	完了〕
\\	〔…⁴に〕(泥棒などが)押し入る
\\	の過去分詞)【形】①(ドア・家などが)鍵を掛けられた 
\\	【形】無限の、果てしない 
\\	【他】〔…⁴と〕戦う;闘う 
\\	【他】②〔…⁴を〕(危険を冒しても)する、あえてする ③〔…⁴の〕危険を冒す
\\	(単
\\	/(複1・2・4)-
\\	(複
\\	または(複1・2・4)
\\	(複
\\	陸〈空〉軍大将;将軍 
\\	①〔ふつう複数で〕交渉、話し合い 
\\	交渉する
\\	1【自】〔完了
\\	引き返す 
\\	途中で引き返す
\\	【接】《従属》〖定動詞後置〗⦅文語⦆① …する前に、…する以前に 
\\	(単
\\	/複数なし 
\\	寒さ;冷たさ 
\\	(単
\\	/(複
\\	絵本 
\\	(単
\\	/(複1・2・3・4)‐
\\	石炭 
\\	【他】〔…⁴を〕取り違える、混同する 
\\	(単
\\	/(複1・2・4)-
\\	(複
\\	お世辞、お愛想 
\\	〔現在〕
\\	〔過去〕
\\	〔過分〕
\\	1【他】①〔…⁴を〕中継放送する 
\\	に権限・任務など⁴を〕任せる、ゆだねる 
\\	②〔複数なし〕(機械などの)調節、(コンピュータなどの)設定
\\	弱変化;(単2・3・4)‐
\\	/(複
\\	ダイヤモンド 
\\	(単
\\	/(複1・2・4)..
\\	(複
\\	または(複1・2・4)-
\\	(複
\\	公爵(大公
\\	と侯爵
\\	の間の爵位)
\\	/〔女性〕-
\\	〔過去〕
\\	〔過分〕
\\	【自】〔完了
\\	おぼれて死ぬ、溺死(できし)する 
\\	〔過去〕
\\	〔過分〕
\\	【他】②〔…⁴を〕のみ込むように食べる、むさぼり食う
\\	(単
\\	/(複1・2・4)-
\\	(複
\\	矢 
\\	〔形容詞変化〕容疑者 
\\	Ⅰ【形】将来の、未来の 
\\	Ⅱ【副】今後、将来
\\	1【他】〔…⁴を〕完成する、仕上げる 
\\	【形】① 国民の、民族の;国家の
\\	1【自】〔完了
\\	窒息する;息が詰まる
\\	【副】どのみち、いずれにしても、とにかく 
\\	いずれにしても今は時間がないよ
\\	【形】(意志・希望などが)はっきりと表明された、明確な 
\\	それを彼ははっきりと言った
\\	(単
\\	/(複1・2・4)-
\\	(複
\\	①⦅やや古語⦆(一時的な)宿、宿泊所 
\\	(単
\\	/(複1・2・4)-
\\	(複
\\	冗談、ふざけ 
\\	冗談を言う、いたずらをする 
\\	(単
\\	/(まれに複
\\	道徳、モラル 
\\	士気
\\	の過去分詞)【形】① 頭のおかしい、気の狂った
\\	1【他】①〔…⁴を〕養成する 
\\	の現在分詞)【形】一時的な、臨時の 
\\	Ⅰ1【再】
\\	意見が一致する、合意する 
\\	4と〗
\\	私たちは…ということで意見が一致した
\\	(単
\\	/(複1・2・4)-
\\	(複
\\	輸送、運送、移送
\\	(単
\\	/(複1・2・4)-
\\	(複
\\	歩み、一歩;足音 
\\	4格+
\\	狩りで獣⁴をしとめる;⦅口語⦆犯人⁴を逮捕する 
\\	区間;路線)
\\	(単
\\	/複数なし 
\\	消費 
\\	出費、費用、経費 
\\	それは莫大なお金〈労力〉がかかる
\\	弱変化;(単2・3・4)‐
\\	/(複1・2・3・4)‐
\\	専門家、エキスパート (〔女性〕
\\	【他】
\\	〔…⁴に…
\\	を〕委託する、依頼する 
\\	国はその企業にそれの生産を委託した
\\	弱変化;(単2・3・4)‐
\\	/(複1・2・3・4)‐
\\	目撃者;立会人
\\	/〔女性〕
\\	彼はその事故の目撃者だった
\\	(単
\\	/(複1・2・3・4)‐
\\	⦅植物⦆ ユリ 
\\	線
\\	1【自】①〔完了
\\	または
\\	(水中へ)潜る、潜水する 
\\	(単
\\	/(複1・2・3・4)‐
\\	肝臓 
\\	②(食材としての)レバー 
\\	(単
\\	/(複1・2・4)-
\\	(複
\\	幽霊、お化け 
\\	(単
\\	/複数なし ⦅料理⦆ スクランブルエッグ、いり卵 
\\	1【他】①〔部屋など⁴を〕掃く、ほうきで掃除する 
\\	【形】① 有毒な、毒のある 
\\	【他】①⦅文語⦆〔飲食物など⁴を〕用意する 
\\	に精神的・肉体的影響⁴を〕もたらす 
\\	用意のできた
\\	(単
\\	/(複1・2・3・4)‐
\\	ツバメ 
\\	(単
\\	/複数なし 
\\	木綿 
\\	綿 
\\	(単
\\	/(複1・2・4)
\\	(複
\\	(まれに複
\\	ひも 
\\	②⦅口語⦆(電気の)コード 
\\	1【他】①〔…⁴を〕速める、〔…⁴の〕速度を速める
\\	1【他】③〔…⁴を〕中断する 
\\	ストを中断する 
\\	存続する、存在し続ける 
\\	存在する、存続する)
\\	(単
\\	/(複1・2・4)..
\\	(複
\\	勃発(ぼっぱつ) 
\\	【他】〔女性⁴を〕強姦(ごうかん)する、レイプする 
\\	【他】〔動物⁴を〕畜殺する 
\\	豚を食肉処理〈解体〉する
\\	通りすがりに;ついでに
\\	1【他】①〔…⁴を〕貯蔵する 
\\	2【自】②⦅文語⦆〔…に〕野営する、キャンプする 
\\	宿営地、野営地;倉庫
\\	〔現在〕
\\	〔過去〕
\\	〔過分〕
\\	【自】〔完了
\\	①(建物が)荒廃する 
\\	その建物は荒れゆくばかりだった
\\	(単
\\	/(複1・2・4)-
\\	(複
\\	本能 
\\	【自】〔完了
\\	と〕うまくやっていく;
\\	を〕うまくこなす〈処理する〉 
\\	新しい同僚とうまくやっていく
\\	(単
\\	/複数なし ①⦅文語⦆ 幸運、幸福
\\	(単
\\	/(複1・2・4)
\\	(複
\\	くちばし 
\\	【形】① 灰色の、ねずみ色の、グレーの 
\\	1【他】①〔…⁴を〕高める、増大させる、上げる 
\\	砦(とりで)、城塞(じょうさい)、要塞 
\\	まるで…かのように 
\\	まるで…かのように 
\\	(単
\\	/(複1・2・4)
\\	(複
\\	草(個々の、特にイネ科の植物を指す)
\\	あれやこれや;⦅口語⦆ いずれにせよ
\\	(単
\\	/(複1・2・4)- (複
\\	あご
\\	【形】① 左の、左側の 
\\	【自】判断する、判断を下す 
\\	第一印象で判断する
\\	(単
\\	/(複
\\	ブレスレット、腕輪 
\\	〔現在〕
\\	〔過去〕
\\	〔過分〕
\\	【自】〔完了
\\	①(とれて)落ちる 
\\	葉が木から落ちる
\\	(単
\\	/(複1・2・3・4)‐
\\	①(タバコの)パイプ 
\\	【自】出血する 
\\	【形】②(手続きなどが)自動的な、機械的な
\\	(単
\\	/(複
\\	ティーポット、急須 
\\	音無	(単
\\	/複数なし ⦅化学⦆ カルシウム ([記号]
\\	㊟専門用語としては
\\	を用いる)
\\	【形】① 透明な、透けて見える 
\\	(単
\\	/(複1・2・4)- (複
\\	用地、敷地;(スキーの)ゲレンデ
\\	(単
\\	/(種類を表わすときのみ複
\\	または-
\\	コンクリート 
\\	【他】①〔…⁴を〕陳列する、並べる 
\\	(展示する、陳列する)
\\	Ⅱ【接】
\\	不定詞句・
\\	文と〗…する代わりに、…ではなしに 
\\	彼は学校へ行かないで一日中寝ていた
\\	【副】①
\\	の比較級)より早く ③⦅口語⦆ むしろ、どちらかと言えば
\\	(単
\\	/複数なし 堆肥、厩肥(きゅうひ) 
\\	3格+
\\	の邪魔をする 
\\	横、斜め)
\\	4格+
\\	…⁴を考えている;もくろむ 
\\	感覚、知覚)
\\	4格+
\\	…⁴に歓迎のあいさつをする、…⁴を歓迎する
\\	(または
\\	あれやこれや
\\	【副】上へ、上方へ、上って 
\\	Ⅰ【形】明らかな、明白な 
\\	4格+
\\	…⁴の用意〈準備〉をする
\\	(単
\\	/(複1・2・4)- (複
\\	ベール 
\\	1【他】①〔プレゼント・人材など⁴を〕交換する 
\\	2【再】
\\	論じ合う
\\	(単
\\	/(複1・2・4)-
\\	(複
\\	②(文章の)段落;(書物・論文などの)章、節 ([略]
\\	(単
\\	/(複1・2・4)
\\	(複
\\	財産;高価な品 
\\	1【自】① つばを吐く 
\\	【他】⦅文語⦆〔…⁴を〕(あるものが)必要とする、要求する 
\\	1【他】②〔頭脳・感覚など⁴を〕鋭くする、鋭敏にする 
\\	鋭い
\\	【自】〔完了
\\	爆発する、破裂する 
\\	弱変化;(単2・3・4)‐
\\	/(複1・2・3・4)‐
\\	愛国[主義]者(〔女性〕-
\\	(単
\\	/複数なし 
\\	悲しみ、苦しみ 
\\	多くの苦しみに耐えなければならない
\\	1【他】①
\\	方向》〔…⁴を…へ・から〕(力ずくで)引っ張る 
\\	(単
\\	/(複1・2・4)
\\	(複
\\	修道院 
\\	〔現在〕
\\	〔過去〕
\\	〔過分〕
\\	1【自】① 譲歩する、屈服する 
\\	その政治家は決して譲歩しなかった (イメージ:
\\	《劣等》)
\\	(単
\\	/(複1・2・3・4)‐
\\	言い訳、言い逃れ、口実 
\\	…のほかは何も…ない;…しか…ない 
\\	私は真実だけを言っている
\\	【他】〔…⁴を〕抱擁する、抱きしめる 
\\	(単
\\	/(複
\\	果樹園 
\\	の過去分詞)【形】① 親しい、親密な、仲のいい 
\\	私たちはとても親しい間柄だ
\\	Ⅰ【接】〖定動詞後置〗…の限りでは 
\\	私の知る限りでは、彼は今ベルリンにいる
\\	1【自】①
\\	〔苦痛⁴を〕訴える;〔…⁴について〕苦情を言う、不満を述べる 
\\	彼は頭が痛いと言っている 
\\	嘆く、嘆き悲しむ 2【他】
\\	に苦痛など⁴を〕訴える、こぼす
\\	1【再】
\\	ひげをそる 2【他】①〔…⁴の〕ひげをそる 
\\	(単
\\	/(複1・2・4)
\\	(複
\\	①(人間・動物の)ひげ 
\\	の過去分詞)【形】適当な、適した、ふさわしい 
\\	それはプレゼントにぴったりだ 
\\	4格+
\\	…⁴を満足させる
\\	【形】平らな;平坦な 
\\	(単
\\	/(複1・2・4)-
\\	(複
\\	あご、下あご 
\\	〔現在〕
\\	〔過去〕
\\	〔過分〕
\\	1【他】①〔…⁴を〕発音する 
\\	発音
\\	(単
\\	/(複1・2・4)-
\\	(複
\\	芸当;曲芸、手品 
\\	4格.	…⁴が問題である
\\	…だから
\\	【他】〔…⁴を〕処罰する、罰する 
\\	(単
\\	/(複1・2・4)
\\	(複
\\	破損、破壊 
\\	ダム〈堤防〉の決壊 
\\	準備ができている 
\\	(進度が)進んだ)
\\	(単
\\	/(複1・2・4)..
\\	(複
\\	事情;事態、状況
\\	(単
\\	/(複1・2・4)-
\\	(複
\\	ボート 
\\	⦅口語⦆ まさか!;これは驚いた!
\\	3格	(到達地点・到達時間などの限度・上限を表わして)…
\\	まで
\\	1【他】①〔…⁴を〕だます、欺く 
\\	3格+
\\	(または
\\	⦅口語⦆ 
\\	のすぐあとをつけている 
\\	かかと)
\\	【形】① 伝統的な、因襲的な 
\\	伝統的な衣装 
\\	[派生] 
\\	慣習、しきたり
\\	〔過去〕
\\	〔過分〕
\\	【自】〔完了
\\	①(ある所へ)行く
\\	3格 
\\	と連絡〈コンタクト〉をとる
\\	(単
\\	/(複1・2・3・4)..
\\	②〔ふつう複数で〕想像、空想
\\	(単
\\	/複数なし スパイ活動、諜報(ちょうほう)活動 
\\	報告;届け出 
\\	(単
\\	/(複1・2・4)..
\\	(複
\\	(市の立つ)広場、中央広場
\\	爆発、破裂 
\\	(単
\\	/(まれに複1・2・4..
\\	(複
\\	記憶
\\	(単
\\	/(複1・2・4)- (複
\\	大使 
\\	/〔女性〕-
\\	英雄気取りになっている
\\	【自】②(ある状態・雰囲気が)支配している
\\	【自】〔完了
\\	①《3格》
\\	の〕あとについて行く、あとを追う〈たどる〉
\\	②《3格》
\\	を〕追求する、調べる ③《3格》〔仕事など
\\	に〕従事する
\\	(単
\\	/(複1・2・4)- (複
\\	召使、従僕 
\\	/〔女性〕-
\\	(単
\\	/複数なし 
\\	貴族、貴族階級 
\\	血管
\\	(単
\\	/(複1・2・4)-
\\	(複
\\	詩 
\\	【形】原則的な;基本的な、根本的な
\\	【形】相当な、かなりの 
\\	相当な額のお金 
\\	(単
\\	/(複
\\	ひしゃく、お玉 
\\	(単
\\	/(複1・2・4)
\\	(複
\\	リボン、テープ、バンド 
\\	【形】① 無罪の、潔白な 
\\	(単
\\	/(複1・2・4)-
\\	(複
\\	(パンやケーキの)生地、こね粉 
\\	(単
\\	/(複1・2・4)-
\\	(複
\\	布地、生地 
\\	物質;⦅化学⦆元素 
\\	(単
\\	/(複1・2・4)
\\	(複
\\	棺、柩(ひつぎ)
\\	1【他】①〔…⁴を〕なしで済ます 
\\	私はその辞書なしで済ますことはできない(⦅参考⦆ふつう
\\	などと共に用いられる)
\\	1【他】〔家畜・子供など⁴の〕番をする;〔…⁴を〕守る 2【再】①
\\	に〕用心する、注意する
\\	(単
\\	/(複
\\	⦅料理⦆ ショウガ[の根] 
\\	【副】①(未来について)いつか ②(過去について)かつて 
\\	君はかつてこのような体験をしたことがあるか?
\\	1【他】①〔…⁴を〕生産する、製造する 
\\	1【他】①
\\	〔…⁴に…
\\	を〕義務づける 
\\	包装を開けた場合あなたに購入の義務が生じます 
\\	義務,責務
\\	(単
\\	/(複
\\	乳牛 
\\	【形】① 正式な、形式にかなった 
\\	形式ばった、堅苦しい
\\	(単
\\	/複数なし 名声、名誉 
\\	世界的な名声
\\	(単
\\	/(複1・2・3・4)‐
\\	被造物、生物 
\\	【自】歓声をあげる、歓呼する
\\	Ⅰ【形】①(地位・身分が)低い、下級の 
\\	卑しい、下劣な
\\	Ⅰ【形】〔無冠詞の場合格語尾は付かない〕① たっぷりした、十分な 
\\	Ⅱ【副】⦅口語⦆ かなり、相当に
\\	1【他】①
\\	方向》〔…⁴を…へ〕(むりやり)押しやる、押し退ける 
\\	警官はデモ参加者を隅へ追いやった
\\	3格+
\\	⦅口語⦆ 
\\	にとどめを刺す 
\\	残り;余り)
\\	【形】③ 気高い、高潔な
\\	(単
\\	/(複
\\	寄付[金]、寄贈[品] 
\\	…⁴を寄付する
\\	(単
\\	/(複1・2・3・4)‐
\\	①⦅キリスト教⦆ ミサ、ミサ聖祭 
\\	【形】宗教上の;教会の;聖職の 
\\	(単
\\	/(複1・2・4)-
\\	(複
\\	避難民、亡命者 
\\	難民 
\\	【形】おしゃべりな、話し好きの 
\\	(または
\\	永久に
\\	(単
\\	/(複
\\	袋小路;⦅比喩⦆ 行き詰まり 
\\	(単
\\	/複数なし つば、唾液(だえき) 
\\	発音
\\	⦅話⦆ …はすっかりおじけづく
\\	(単
\\	/(複1・2・4)..
\\	(複
\\	きっかけ、動機;原因
\\	(単
\\	/(複
\\	②(舞踏会・カジノなどでの)ドレスコード 
\\	(単
\\	/(複
\\	(客を迎える)主人、ホスト、あるじ (〔女性〕-
\\	(単
\\	/(複1・2・4)-
\\	(複
\\	文体;話し方、書き方 
\\	〔過去〕
\\	〔過分〕
\\	【自】〔完了
\\	①(催し物などが)中止になる;運休する 
\\	会議は中止だ
\\	1【他】①〔…⁴を〕演奏して聞かせる 
\\	【形】① 緩い、緩んだ;たるんだ 
\\	②⦅口語⦆ 厳格でない、気楽な
\\	4格(または
\\	3格)
\\	…4・3にはもったいない 
\\	それは捨てるにはもったいない
\\	いまだかつて…ない
\\	【形】① 根拠のない、理由のない
\\	1〔形容詞変化〕① 見知らぬ人 
\\	父親は子供に知らない人について行かないよう注意した 
\\	よその人;外国人
\\	3格+
\\	に道を譲る;…
\\	を避ける
\\	2【自】〔完了
\\	借金を背負い込む 3【再】
\\	借金をする、負債をつくる
\\	(単
\\	/(複1・2・4)-
\\	(複
\\	奉仕、サービス、尽力 
\\	お客へのサービス
\\	〔現在〕
\\	〔過去〕
\\	〔過分〕
\\	【自】〔完了
\\	と〕(時間的に)重なる、一致する 
\\	今年彼の誕生日は聖霊降臨祭に当たる
\\	【形】①(果実などが)熟した 
\\	成熟した;円熟した ④(ワイン・チーズなどが)熟成した
\\	侮辱 
\\	友人を侮辱する 
\\	…⁴を侮辱する
\\	【形】③ 自信のない 
\\	自信なげなふるまい
\\	【形】① 注意深い 
\\	よく気のつく、思いやりのある
\\	【形】自己中心の、自己中心的な 
\\	〔形容詞変化〕生き残った人、生存者 
\\	(単
\\	/(複1・2・4)-
\\	(複
\\	叫び声、叫び;悲鳴 
\\	(単
\\	/(複1・2・3・4)‐
\\	⦅文語⦆ 細部、細目、詳細 
\\	(または
\\	細部にいたるまで
\\	(単
\\	/(複1・2・4)
\\	(複
\\	①(凍害・霜害などをもたらす)氷点下の寒気、厳寒 
\\	【形】②(体の一部が)堅い、硬直した;かじかんだ
\\	(単
\\	/(複
\\	鍵穴 
\\	積み荷、貨物 
\\	(単
\\	/(複
\\	木炭 
\\	【形】① 抜け目のない、ずる賢い 
\\	抜け目のない若者 
\\	(単
\\	/(複
\\	傷跡 
\\	の過去分詞)【形】酔っ払った 
\\	君はすっかり酔っ払っている
\\	(単
\\	/(複1・2・4)-
\\	(複
\\	男爵(伯爵
\\	と騎士
\\	の間の爵位)
\\	/〔女性〕-
\\	〔過去〕
\\	〔過分〕
\\	【他】②
\\	場所》〔…⁴を…に〕宿泊させる、泊める
\\	【他】①〔席など⁴を〕占める、取っておく 
\\	②〔国・都市など⁴を〕占領する;〔建物など⁴を〕占拠する 
\\	バスの座席を祖母のために確保しておく 
\\	…⁴を所有している
\\	【形】① 見知らぬ、名前のわからない 
\\	名前のわからない犯人
\\	(単
\\	/(複1・2・3・4)‐
\\	魔女、女の魔法使い 
\\	【副】1③[ダー・ルム] だから、それゆえ、それにもかかわらず
\\	〔現在〕
\\	〔過去〕
\\	〔過分〕
\\	1【他】〔皿・ガラスなど⁴を〕割る、壊す 
\\	2【自】〔完了
\\	①(皿・ガラスなどが)割れる、壊れる
\\	Ⅱ【副】主として、とりわけ 
\\	彼は主にゲーテを読む
\\	額、おでこ 
\\	1【他】①〔ほこり・汚れなど⁴を〕ふき取る、ぬぐい取る 
\\	2【自】②〔完了
\\	〔…へ〕さっと動く
\\	(単
\\	/(まれに複
\\	偏愛、特別の愛好心 
\\	私はモーツァルトが大好きです
\\	【複】テーブルマナー 
\\	食事 
\\	⦅比⦆ 前面にのさばり〈しゃしゃり〉出る
\\	Ⅰ【形】① 控え目な;謙虚な 
\\	控え目な願望を述べる
\\	【形】忘れっぽい 
\\	(単
\\	/(複1・2・4)- (複
\\	愛好家、ファン (〔女性〕-
\\	交差点 
\\	②⦅生物⦆[異種]交配;交配種、交雑種 
\\	3【自】〔完了
\\	〔…⁴を・の方へ〕曲がる
\\	【副】(下から向こうの)上へ 
\\	【形】毎週の、1週間ごとの 
\\	【他】①〔…⁴を〕抑圧する、弾圧する 
\\	少数派を弾圧する
\\	(単
\\	/(複1・2・4)-
\\	(複
\\	②〔複数なし〕評判、名声 
\\	評判, 世評)
\\	【副】我が家へ;故郷へ 
\\	1〔現在〕
\\	〔過去〕
\\	〔過分〕
\\	【自】〔完了
\\	腫れる、膨れる 
\\	(単
\\	/(複1・2・3・4)‐
\\	肋骨(ろっこつ)、あばら骨 
\\	〔過去〕
\\	〔過分〕
\\	1【他】①〔…⁴を〕もぎ取る、引きはがす 
\\	カレンダーをめくり取る
\\	2(単
\\	/複数なし 
\\	髄、髄質;骨髄
\\	【他】①〔…⁴を〕要約する、まとめる 
\\	調査の結果をまとめる (イメージ:
\\	集めて・合わせて+
\\	つかむ;理解する)
\\	(単
\\	/(複
\\	パセリ、オランダゼリ 
\\	【他】〔手・腕など⁴に〕クリームを塗る 
\\	(単
\\	/(複1・2・3・4)‐
\\	繊維、筋;繊維質 
\\	【形】寛大な、寛容な 
\\	3格 
\\	に左右される 
\\	…を導く、案内する)
\\	【形】② 客観的な、私情を交えない 
\\	客観的な批判
\\	〔過去〕
\\	〔過分〕
\\	【自】〔完了
\\	様態》
\\	を…に〕扱う 
\\	患者たちを乱暴に扱う 
\\	…⁴を取り囲む
\\	危険を冒す
\\	(単
\\	/(複
\\	乾燥機 
\\	【形】含まれている、加算されている 
\\	(単
\\	/(複1・2・4)..
\\	(複
\\	①(電気・水道・電話などの)接続;通話 
\\	【他】①〔人・活動など⁴を〕援助する、助成する
\\	(単
\\	/(複
\\	(大学第1学期の)新入生 
\\	3格	
\\	に応じて
\\	(単
\\	/(複
\\	バス停[留所] 
\\	(単
\\	/(複
\\	(壁に取り付けた)洗面台;洗面器 
\\	(心身の)障害、ハンディキャップ 
\\	の過去分詞)【形】家具付きの ([略]
\\	少しの
\\	3格+
\\	⦅口語⦆ 
\\	をさんざん殴る、打ちのめす (㊟
\\	は形式目的語)
\\	類似、似ていること;類似点
\\	(単
\\	/複数なし 
\\	眠り、睡眠 
\\	・3格支配 ・名詞のうしろに置かれることもある ・代名詞と共に用いられるときは常にうしろに置かれる Ⅰ【前】①(空間)…の向かい側に ②(相手)…に対して 
\\	私に対して彼女はいつも親切です
\\	(単
\\	/(複1・2・4)-
\\	(複
\\	①〔複数なし〕失うこと、紛失;(人を)亡くすこと 
\\	彼女は真珠のネックレスをなくしたことに気づいた
\\	【形】① 助けてくれるものもない;途方にくれた 
\\	(単
\\	/複数なし 新月 
\\	4格+
\\	(疑い・軽蔑などから)…⁴を横目で見る
\\	(単
\\	/(複1・2・4)-
\\	(複
\\	シカ;雄ジカ 
\\	【形】① 醜い、不格好な 
\\	(単
\\	/(複1・2・3・4)‐
\\	針 
\\	【形】① 大衆的な、大衆に人気のある 
\\	独力で;自分の責任で
\\	(単
\\	/複数なし 
\\	ラジオ放送、ラジオ 
\\	ラジオを聞く
\\	【他】③(中身として)〔…⁴を〕含んでいる、擁している 
\\	この本は400ページある
\\	(単
\\	/(複
\\	専門家、熟練者 
\\	(公共の)機関、施設 
\\	(単
\\	/複数なし 精神医学;⦅俗語⦆精神病院、(病院の)精神科 
\\	Ⅰ【形】① 規則的な、規則正しい 
\\	【形】医学の、医学的な 
\\	[派生] 
\\	医学
\\	【形】利用可能な、役に立つ、有用な
\\	【形】精神的な、心的な 
\\	精神的圧迫
\\	(単
\\	/(複1・2・3・4)‐
\\	統制、支配;制御、コントロール
\\	【形】未成年の(ドイツでは18歳、オーストリアでは19歳、スイスでは20歳未満) 
\\	(単
\\	/(複1・2・4)-
\\	(複
\\	重点、要点、核心
\\	【他】①〔…⁴に〕資金を出す、融資する、出資する 
\\	【副】(前置詞
\\	と
\\	の結合形)いっしょに;相互に
\\	要求;要請 
\\	[派生]
\\	…⁴を要求する
\\	⦅口語⦆ おまえ、気は確かか? 
\\	(単
\\	/(複1・2・4)
\\	(複
\\	斧(おの)、まさかり 
\\	(単
\\	/(複
\\	占い師;予言者 (〔女性〕-
\\	親戚関係、血縁関係 ②〔複数なし〕〖集合的に〗親戚、親族
\\	1(単
\\	/複数なし 岩盤 
\\	弱変化;(単2・3・4)‐
\\	/(複1・2・3・4)‐
\\	⦅文語⦆ 岩壁、岩塊 
\\	(単
\\	/(複
\\	(複数の世代が同居する)大家族 
\\	〔形容詞変化〕シングルファーザー、シングルマザー 
\\	人生経験 
\\	(単
\\	/(複1・2・3・4)‐
\\	苦情、不平 
\\	4格 
\\	…⁴に対して文句を言う [派生] 
\\	苦情を言う,不平を言う
\\	(単
\\	/(複1・2・4)- (複
\\	①〔複数なし〕かび 
\\	(単
\\	/(複
\\	賃借人、借家人 (〔女性〕-
\\	〔過去〕
\\	〔過分〕
\\	再び来る;(病気が)再発する
\\	(新しいものと)取り替えること、交換 
\\	4格+
\\	4格 
\\	…⁴に…⁴を気づかせる 
\\	注意深い;よく気のつく)
\\	弱変化;(単2・3・4)‐
\\	/(複1・2・3・4)‐
\\	専門家、エキスパート 
\\	(単
\\	/(複
\\	転入、引っ越し 
\\	(住居などへの)入居 
\\	3格+
\\	4格.	
\\	にとって…⁴が中心問題である
\\	(単
\\	/(複
\\	⦅ドイツ北部で⦆ 肉屋;食肉解体業者 
\\	(単
\\	/(複
\\	ふつう複数で (動物の、食肉になる)内臓、もつ 
\\	(単
\\	/(複
\\	⦅ドイツ中西部・南部・オーストリア・スイスで⦆ 肉屋、食肉業者 
\\	(または
\\	とんでもない!;何を言うんですか!
\\	の過去分詞)【形】親しい、友好〈友人〉関係にある 
\\	〔過去〕
\\	〔過分〕
\\	【自】〔完了
\\	こちらへ来る 
\\	(単
\\	/(複
\\	くわ;つるはし 
\\	おねしょする
\\	【形】① 2倍の;二重の 
\\	幻想、幻覚、幻影
\\	【形】① どうでもよい、興味がない 
\\	〖3格と〗
\\	誰がそれをやるかは私にはどうでもよい
\\	〔形容詞変化〕①〔ふつう複数で〕家族、親族 
\\	2【他】〔…⁴の〕復讐をする、敵を討つ 
\\	復讐
\\	〔過去〕
\\	〔過分〕
\\	〔接Ⅱ〕
\\	1【他】②〔…⁴を〕引き裂く、裂く 
\\	彼は手紙を細かく引き裂く
\\	1(単
\\	/(複1・2・3・4)‐
\\	処女、乙女 
\\	【形】① 鉄の、鉄製の ②(鉄のように)強固な、ゆるぎない
\\	(単
\\	/(複1・2・3・4)‐
\\	①〔複数なし〕規律、秩序 
\\	(または
\\	規律を守る
\\	(単
\\	/複数なし ①(人の立場などに対する)理解
\\	(単
\\	/複数なし 満月 
\\	〔現在〕
\\	〔過去〕
\\	〔過分〕
\\	1【他】①〔…⁴を〕引き裂く、引きちぎる、破る 
\\	⦅口語⦆ 大丈夫、人に聞かれる心配はない! 
\\	空気がきれいだ;⦅話⦆ ここなら大丈夫だ(のぞかれたり盗み聞きされたりする危険はない)
\\	(単
\\	/(複1・2・4)- (複
\\	オール、櫂(かい) 
\\	オールを出す〈引っ込める〉 
\\	舵(かじ)
\\	【略】等々、など 
\\	弱変化;(単2・3・4)‐
\\	/(複1・2・3・4)‐
\\	①(手工業の)職人 (徒弟の期間を終え、職人試験に合格した者
\\	と
\\	の中間にあたる)(〔女性〕..
\\	(単
\\	/(複1・2・4)
\\	(複
\\	①〔複数なし〕走ること、ランニング 
\\	③〔複数なし〕(時の)経過;(物事の)進行、成り行き 
\\	2格	
\\	のうちに〈間に〉
\\	【副】① 今後[も]、引き続き;相変わらず
\\	【他】〔…⁴を〕作る、作成する;〔編み物・服など⁴を〕製作する 
\\	リスト〈写し〉を作る 
\\	弱変化;(単2・3・4)‐
\\	/(複1・2・3・4)‐
\\	予言者;⦅宗教⦆ 預言者 (〔女性〕-
\\	【形】① 口の利けない 
\\	(単
\\	/(複
\\	⦅植物⦆ キイチゴ、ラズベリー[の実] 
\\	1【自】①(特に教会の鐘が)鳴る 
\\	鐘が鳴る 2【他】〔鐘⁴を〕鳴らす
\\	(単
\\	/(複
\\	流産 
\\	【形】妊娠している 
\\	【他】〔人や動物など⁴を〕虐待する
\\	1【他】①〔…⁴を〕なめる 
\\	1【他】①〔宝物など⁴を〕埋める、埋めて隠す 
\\	宝物を土に埋める
\\	【自】〔完了
\\	⦅文語⦆① 目を覚ます、意識が戻る 
\\	〔過去〕
\\	〔過分〕
\\	【自】②
\\	〔…⁴を〕もくろむ;考え出そうとする
\\	〔形容詞変化〕行方不明者、失踪者 
\\	〔形容詞変化〕妊婦 
\\	〔形容詞変化〕死者、故人 
\\	1【再】
\\	と〕仲直りする、和解する 
\\	彼は妻と仲直りする 2【他】
\\	〔…⁴を…
\\	と〕仲直りさせる、和解させる
\\	【自】《3格》
\\	に〕所属している、
\\	の〕一員である 
\\	彼は私たちのグループの一員だ 
\\	(単
\\	/(複1・2・3・4)‐
\\	①〔複数なし〕好意、愛顧、寵愛(ちょうあい) 
\\	【自】①
\\	〔…⁴を〕待ち伏せする
\\	1【他】①〔…⁴を〕整える;〔問題・用件など⁴を〕解決する 
\\	【形】① この世の、現世の、俗世の 
\\	現世の幸福
\\	(単
\\	/複数なし 
\\	心痛、悲しみ、憂い 
\\	恋の悩み
\\	4格+
\\	…⁴の方に向かって
\\	(単
\\	/(複
\\	対話 
\\	4格+
\\	⦅口語⦆ …⁴に一杯食わす
\\	【自】〔完了
\\	走り〈逃げ〉去る
\\	1【他】〔…⁴を〕怒らせる 
\\	1【他】①〔衣類など⁴を〕縫う
\\	(単
\\	/複数なし 帰宅;帰郷、帰国 
\\	(単
\\	/(複
\\	(物を)失くした人;敗者 
\\	〔過去〕
\\	〔過分〕
\\	1【他】①〔…⁴を〕つねる 
\\	彼女は彼の腕をつねった
\\	【他】〔…⁴を〕落ち着かせる、なだめる 
\\	(または
\\	見たところ、
\\	どうやら
\\	〔現在〕
\\	〔過去〕
\\	〔過分〕
\\	〔接Ⅱ〕
\\	1【他】①〔…⁴を〕(動物が)飲む ②⦅俗語⦆〔酒など⁴を〕がぶ飲みする 
\\	【他】〔…⁴に〕懇願〈嘆願〉する 
\\	〔過去〕
\\	〔過分〕
\\	1【再】①
\\	身をよじる 
\\	痛さのあまり身をよじる
\\	(単
\\	/(複
\\	ウナギ 
\\	⦅口語⦆ (いらだちを込めて)いったい全体 (㊟疑問詞
\\	などと用いる)
\\	(単
\\	/(複
\\	[支]柱 
\\	【他】①〔…⁴を〕ひっくり返す、倒す 
\\	花瓶をひっくり返す
\\	(単
\\	/複数なし 
\\	意識、正気 
\\	意識、自覚
\\	【副】もしかすると、ひょっとすると 
\\	(単
\\	/(複1・2・4)-
\\	(複
\\	眺め、光景 
\\	②〔複数なし〕見ること 
\\	このようなシーンを見てだれもがとてもショックを受けた 
\\	一見して
\\	【副】その代わりに 
\\	(単
\\	/(複1・2・4)..
\\	(複
\\	エレベーター 
\\	埋葬、葬式 
\\	埋葬、葬儀 
\\	厳粛な葬儀
\\	【形】思いやりのない、冷酷な 
\\	薄情な
\\	【他】①〔…⁴を〕軽蔑する、さげすむ 
\\	⦅口語⦆ このワインはなかなかのものだ 
\\	軽蔑,侮蔑
\\	①(内的な)結び付き
\\	1【自】①⦅文語⦆ 休息する、憩う 
\\	ソファーの上で〈木陰で〉休む
\\	1【他】②〔粉・粒状のもの⁴を〕まく、〔調味料など⁴を〕ふりかける 
\\	肥料をまく
\\	(単
\\	/(複1・2・3・4)‐
\\	シャベル、スコップ 
\\	⦅文語⦆ 理解がある〈ない〉 
\\	理解を示す
\\	4格+
\\	(または
\\	…⁴を抱き締める
\\	(単
\\	/複数なし 
\\	静けさ、静寂 
\\	嵐の前の静けさ
\\	【自】〔完了
\\	出血多量で死ぬ 
\\	【形】① 落ち着きのない;不安な 
\\	落ち着きのない子供
\\	4格+
\\	…⁴を試す
\\	(単
\\	/(複1・2・3・4)‐
\\	伝説、言い伝え、説話 
\\	・2格支配 【前】② …以内に、…の内に
\\	問題外である、考慮の余地がない
\\	⦅口語⦆ 際限なく;うんざりするほど
\\	(単
\\	/(複
\\	⦅植物⦆ ラベンダー
\\	(単
\\	/(複
\\	ブレスレット 
\\	【形】非常に大きな、莫大な 
\\	【他】①〔…⁴を〕元気づける、励ます 
\\	(単
\\	/(複1・2・4)
\\	(複
\\	①(鳥・昆虫などの)群れ;(人間の)一団 
\\	ハチの群れ
\\	〔現在〕
\\	〔過去〕
\\	〔過分〕
\\	【自】〔完了
\\	①(火・明かりなどが)消える [派生] 
\\	…⁴を消す
\\	(単
\\	/複数なし 冷や汗 
\\	(単
\\	/複数なし 呼吸、息 
\\	【自】①(仕事・交通などが)滞る;(呼吸などが一時的に)止まる
\\	(単
\\	/(複
\\	納屋、穀物倉 
\\	⦅口語⦆(報告などの終わりに)これで全部です;まあこんなところです 
\\	(アナウンサーが番組の終わりなどに)今日はこれでおしまいです。みなさん、よい晩を
\\	〔現在〕
\\	〔過去〕
\\	〔過分〕
\\	2【自】①
\\	を〕見合わせる、中止する 
\\	訪問をやめる
\\	【他】①
\\	〔…⁴に必要なもの
\\	を〕与える、供給する 
\\	町に飲料水を供給する ③〔…⁴の〕世話をする、めんどうを見る
\\	(単
\\	/(複
\\	軟膏、塗り薬 
\\	【形】ひどく貧しい、赤貧の 
\\	(単
\\	/複数なし 
\\	憎しみ、憎悪 
\\	【他】〔…⁴に〕報いる;〔…⁴の〕お返しをする;〔…⁴の〕仕返し〈報復〉をする 
\\	しっぺ返しをする 
\\	善に報いるに悪をもってする
\\	(単
\\	/(複1・2・3・4)‐
\\	旗 
\\	【副】ところで、ちなみに、それはそうと 
\\	ついこの間から
\\	【形】紫色の、すみれ色の 
\\	こんちくしょうめ
\\	(単
\\	/(複1・2・4)- (複
\\	助力者、協力者;助手 
\\	/〔女性〕-
\\	を手伝う,助ける
\\	(単
\\	/(複1・2・4)-
\\	(複
\\	手足、肢;関節
\\	1【他】①〔…⁴を〕緩める 
\\	(質問に対する強い肯定を表わして)もちろんだとも
\\	(単
\\	/(複
\\	(武器・スポーツ用の)槍 
\\	(または
\\	あちこちで;ときどき
\\	(単
\\	/(複
\\	日の出 
\\	(単
\\	/(複
\\	競馬 
\\	(単
\\	/(複1・2・3・4)‐
\\	賭(かけ) 
\\	⦅話⦆ …の息の根をとめる、…を破滅させる
\\	(単
\\	/(複
\\	裏切り者 
\\	…⁴を裏切る
\\	【副】②(理由などを示して)そのために、それゆえ、だから 
\\	彼は熱があった
\\	そのためパーティには来られなかった 
\\	⦅口語⦆ 現場〈現地〉で
\\	たとえ…でも
\\	〔現在〕
\\	〔過去〕
\\	〔過分〕
\\	【他】①(板を打ち付けて)〔…⁴を〕ふさぐ、〔…⁴を〕くぎ付けする 
\\	板を打ち付けて入口をふさぐ
\\	くつろぐ;楽にする (㊟
\\	は形式目的語)
\\	…の耳に入る
\\	⦅口語⦆ そんなこと知ったことじゃない!、
\\	知るものか、私の知ったことか
\\	(単
\\	/(複1・2・3・4)‐
\\	敗北、敗戦 
\\	(単
\\	/(複
\\	失敗、不成功 
\\	どんな…、どんな種類の…;なんて…
\\	1【再】
\\	扮装する、変装〈仮装〉する
\\	(単
\\	/(複1・2・3・4)‐
\\	①(乗用の)馬車 
\\	馬車に乗って行く
\\	「続く」の現在分詞)【形】長く続く、永続的な;絶え間ない 
\\	長期滞在
\\	の現在分詞)【形】納得のゆく、説得力のある 
\\	偽造;偽造品、偽物 
\\	(単
\\	/(複1・2・4)..
\\	(複
\\	①(川・線路・国境などを)渡る所;橋、陸橋;横断歩道 
\\	【他】①〔…⁴を〕取り除く、かたづける、除去する 
\\	あすの朝ごみをかたづけてくれ 
\\	(単
\\	/(複1・2・4)-
\\	(複
\\	分け前、取り分、割り当て 
\\	お元気で!
\\	(単
\\	/(複1・2・4)-
\\	(複
\\	証明書、証書 
\\	紙幣、札 
\\	3格+
\\	⦅口語⦆ 
\\	にとどめを刺す、
\\	…を殺す〈破滅させる〉
\\	(単
\\	/(複1・2・3・4)
\\	溝、堀 
\\	発音×	待ち伏せする、見張る 
\\	潜伏、待ち伏せ)
\\	【自】値切る 
\\	価格を値切る
\\	(単
\\	/(複
\\	(事故・犯行の)現場、場所 
\\	【形】① 血にまみれた 
\\	血まみれの [派生] 
\\	血、血液
\\	〔過去〕
\\	〔過分〕
\\	【自】〔完了
\\	①(中からこちらの)外へ出て来る;抜け出す 
\\	【形】絶え間ない、不断の、恒常的な
\\	1【自】②
\\	〔…⁴を〕(意のままにできるものとして)持っている 
\\	豊かな経験〈良いコネ〉を持っている
\\	(単
\\	/(複
\\	決闘、果たし合い;(スポーツでの)対決、試合 
\\	【自】〔完了
\\	つまずく 
\\	(単
\\	/(複1・2・4)-
\\	(複
\\	宣誓、誓い 
\\	(単
\\	/複数なし 
\\	軽蔑、侮蔑、さげすみ 
\\	【他】①〔…⁴を〕制限〈限定〉する 
\\	【形】最大限の、最高の、極限の 
\\	【他】①〔…⁴を〕指示する、命じる 
\\	医者は安静を指示した
\\	(単
\\	/(複
\\	スパイ、諜報員 
\\	(単
\\	/(複1・2・3・4)
\\	(国家・州政府の)省、行政官庁 
\\	4格+
\\	(または
\\	⦅口語⦆ 問題など⁴をかたづける
\\	【形】① 完全な、完璧な、欠点のない 
\\	まったくの
\\	(単
\\	/(まれに複
\\	賛成、同意 
\\	に賛成する,同意する
\\	(単
\\	/(複1・2・4)
\\	(複
\\	酔い、酩酊
\\	度を越す
\\	…すらない
\\	…[の場合]は別だが、ただし…の場合を除いて
\\	【形】① 秘密の、内密の;隠された 
\\	(単
\\	/複数なし 
\\	暖かさ、温かさ;温度 
\\	⦅口語⦆ ぽかんと口を開けて見とれている、
\\	(何もせずに)口をぽかんとあけている 
\\	(単
\\	/(複
\\	春 
\\	1【他】②〔…⁴を〕養う、扶養する 
\\	(または
\\	彼の仕事では彼は家族を養うことができない
\\	【形】賢明な 
\\	賢い、賢明な、聡明な
\\	(単
\\	/(複1・2・4)-
\\	(複
\\	①⦅口語⦆ 女、あま
\\	(単
\\	/(複1・2・4)
\\	(複
\\	①(動物の)口 
\\	【自】〔完了
\\	飢え死にする [派生] 
\\	空腹, 飢え
\\	分割;分配 
\\	[派生] 
\\	…⁴を分ける,分割する;…⁴を分け合う,分配する
\\	①(政治的な)革命 
\\	【形】① 軍隊の、軍事上の 
\\	Ⅰ【副】② 横切って
\\	(単
\\	/(複1・2・3・4)‐
\\	チョーク 
\\	4格+
\\	…⁴がまったく好きになれない 
\\	…⁴を我慢する)
\\	(単
\\	/複数なし 田舎の暮らし、田園生活 
\\	【他】①〔…⁴を〕(思いがけないことが)驚かす 
\\	【他】〔…⁴を〕うらやむ、ねたむ 
\\	4格と〗
\\	私は君の冷静さがうらやましい
\\	3格(または
\\	4格)
\\	発音
\\	…3・4は良い〈悪い〉状態である
\\	3【再】
\\	と〕異なる、区別される
\\	(単
\\	/(複1・2・3・4)‐..
\\	トラクター 
\\	(単
\\	/(種類を表わすときのみ複
\\	石灰 
\\	【形】① 心理学の、心理学的な 
\\	心理的な 
\\	心理小説
\\	増大 
\\	(単
\\	/(複1・2・4)-
\\	(複
\\	議席 ③(会社・機関の)本拠、所在地
\\	取り扱い、待遇 
\\	《4格
\\	様態》…⁴を…に取り扱う,遇する
\\	礼儀作法を心得ている
\\	【形】知能の優れた、聡明な;賢明な
\\	(単
\\	/複数なし 保証
\\	(単
\\	/(複1・2・3・4)‐
\\	大使館 
\\	知らせ;[公的な]メッセージ
\\	(単
\\	/(複1・2・4)
\\	(複
\\	くい、支柱 
\\	黄色の 
\\	(単
\\	/(複1・2・3・4)‐
\\	①⦅やや古語⦆(ふつう暖房設備のない)小部屋 
\\	女中部屋 
\\	心痛、悲しみ
\\	(単
\\	/(まれに複
\\	観点、視点
\\	(単
\\	/(複1・2・3・4)‐
\\	割れ目、裂け目 
\\	(単
\\	/(複1・2・4)-
\\	(複
\\	芽 
\\	(単
\\	/(複1・2・4)- (複
\\	①(本などの)索引、インデックス
\\	(単
\\	/(複1・2・3・4)‐
\\	ギター 
\\	心から喜んで、とても喜んで
\\	⦅口語⦆ まず第一に 
\\	まず第一に)
\\	【形】魔法の、魔術〈呪術〉的な 
\\	Ⅰ【形】心からの、心のこもった 
\\	(単
\\	/(複
\\	3分の1
\\	(単
\\	/(複
\\	利用、使用 
\\	炎、火炎 
\\	【形】(一語一句)原文に忠実な 
\\	(単
\\	/(複1・2・4)-
\\	(複
\\	①⦅詩学⦆ 韻 
\\	(単
\\	/(複1・2・4)
\\	(複
\\	カエル 
\\	口を慎む 
\\	…⁴の番をする)
\\	フクロウ 
\\	〔現在〕
\\	〔過去〕
\\	〔過分〕
\\	【他】〔…⁴を〕編む 
\\	(または
\\	髪をお下げに編む
\\	〔形容詞変化〕婚約者、いいなずけ、フィアンセ [派生] 
\\	と婚約する
\\	カタツムリ 
\\	4格+
\\	⦅口語⦆ …⁴を厳しく叱りつける 
\\	⦅昆虫⦆ アリ 
\\	1【他】①〔…⁴を〕おびき寄せる、誘う 
\\	犬をソーセージでおびき寄せる
\\	北半球 
\\	(単
\\	/(複
\\	教科書 
\\	【形】① 公正な;正当な 
\\	(単
\\	/複数なし (商品・製品などの)発送 
\\	【形】芳香のある、香り〈風味〉の良い 
\\	足の指 
\\	【自】① 十分である、足りる 
\\	草地、草原 
\\	初めから
\\	3格+
\\	発音リクエスト中	
\\	の気に障る〈障らない〉
\\	(単
\\	/(複1・2・4)-
\\	(複
\\	①⦅文語⦆ 行為、行動 
\\	違法な行為
\\	(単
\\	/(複1・2・4)-
\\	(複
\\	音、物音 
\\	【自】〔完了
\\	破裂する、裂ける 
\\	起こる 
\\	ここで何が起こっているのですか?
\\	⦅軽蔑して⦆ ばばあ
\\	2【再】
\\	苦情を言う、不平を言う 
\\	3格+
\\	⦅口語⦆ 
\\	をいらだたせる、
\\	…をいらいらさせる
\\	【形】① 哀れな、みすぼらしい 
\\	私の考えでは、
\\	私の意見では
\\	(単
\\	/(複
\\	⦅口語⦆ テレビ[受像機] 
\\	(単
\\	/複数なし 共同生活;同居 
\\	1【他】③〔…⁴を〕調節する;⦅コンピュータ⦆ 設定する [派生] 
\\	調節,設定
\\	(単
\\	/複数なし 昼の休息、昼休み 
\\	(単
\\	/(複
\\	地下駐車場〈ガレージ〉 
\\	(単
\\	/(複
\\	遊び場、遊園地 
\\	【形】①(場所などが)行くことのできる、近づくこと〈立ち入ること〉のできる;(施設などが)使用できる
\\	(単
\\	/(複
\\	車いす 
\\	(単
\\	/(複1・2・4)- (複
\\	雷雨、夕立 
\\	(単
\\	/(複
\\	(草花用の)プランター 
\\	①〔複数なし〕秩序;整頓 
\\	…⁴を整理する,並べる
\\	(単
\\	/(複
\\	(ふつう両刃の)短刀、短剣 
\\	よしわかった、そうしよう!
\\	⦅雅⦆ …を阻止する、…をやめさせる、…を食い止める
\\	4格+
\\	…⁴を開墾する
\\	1(単
\\	/(複1・2・4)
\\	(複
\\	葉 
\\	(または
\\	ずっと以前から
\\	(単
\\	/(複1・2・4)- (複
\\	殺人者、殺人犯 
\\	/〔女性〕-
\\	(単
\\	/複数なし 
\\	バランス、釣り合い;安定 
\\	【自】〔完了
\\	滑る、スリップする、滑って転ぶ 
\\	〔過去〕
\\	〔過分〕
\\	1【他】〔…⁴を〕切り取る、切り離す 
\\	3格+
\\	(または
\\	の道をふさいでいる;⦅口語⦆ 
\\	の邪魔になっている
\\	(単
\\	/(複1・2・3・4)‐
\\	①〔ふつう複数で〕(手足を縛る)鎖、綱
\\	【形】終わりのない、無限の 
\\	【自】②(危険・不幸などが)迫る 
\\	雷雨になりそうだ
\\	1【他】①〔動物⁴を〕飼い慣らす、手なずける 
\\	【形】① 人間の 
\\	人間的な
\\	①〔ふつう単数で〕配慮、気配り;思いやり 
\\	思いやりがない
\\	2【再】①
\\	燃え始める、火がつく
\\	(単
\\	/(複
\\	歩道 
\\	【形】① 厳しい、厳格な 
\\	100万 ([略]
\\	【他】〔…⁴を〕教育する、育てる 
\\	教育
\\	(単
\\	/(複
\\	⦅植物⦆ 桃[の実] 
\\	〔過去〕
\\	〔過分〕
\\	【自】〔完了
\\	あとに残る 
\\	遅れをとる
\\	ほんとうは、実際[は]
\\	【他】①〔子供など⁴を〕誘拐する 
\\	4格+
\\	⦅口語⦆ …⁴をだます
\\	(単
\\	/(複1・2・3・4)‐ 協定、取り決め 
\\	骨折り損である
\\	の喜劇『恋の骨折り損』による)
\\	(単
\\	/(複1・2・4)-
\\	(複
\\	不利、損;短所、欠点 
\\	【他】〔…⁴を〕まく、〔…⁴の〕種をまく 
\\	(単
\\	/(複1・2・4)
\\	(複
\\	子羊 
\\	(単
\\	/(複1・2・4)- (複
\\	やかん;(大型の)鍋 
\\	意味のない、無意味な
\\	(単
\\	/複数なし ①(意味のない)おしゃべり、無駄話
\\	発音リクエスト	
\\	⦅比⦆ …に結婚を申し込む
\\	弱変化;(単2・3・4)‐
\\	/(複1・2・3・4)‐
\\	領主、君主
\\	【前】(というより)むしろ、それどころか、いっそ 
\\	彼は太っているというよりむしろ肥満体というべきだ
\\	Ⅱ1【再】①
\\	〔…⁴を〕気にかける (㊟ふつう否定文で) 
\\	彼女はそのことをまったく気にしなかった
\\	⦅口語⦆ 少しも…ない
\\	(単
\\	(単3・4)-
\\	/(複
\\	ニックネーム、あだ名 
\\	…から
\\	自殺する
\\	【他】②〔…⁴を〕いらいらさせる、怒らせる 
\\	彼は彼女をひどく怒らせた
\\	(単
\\	/複数なし 無関心、むとんじゃく、冷淡 
\\	【形】中毒の、依存症の 
\\	【形】歴史上の、歴史的な 
\\	Ⅰ【副】前に、手前に 
\\	(単
\\	/(複1・2・3・4)‐
\\	①〔複数なし〕チェス、西洋将棋 
\\	1(単
\\	/(複1・2・4)-
\\	(複
\\	正方形 
\\	【形】① 斜めの;傾いた 
\\	傾斜面
\\	【副】① 外に、外で、外側に〈で〉 
\\	西暦 
\\	西暦300年 ([略]
\\	【形】① 敵意のある、敵対した 
\\	敵[軍]の
\\	【副】他方では、その反面
\\	(物に関して)とりわけ
\\	(単
\\	/複数なし ミルクコーヒー 
\\	1【他】〔肉・魚など⁴を〕焼く、ローストする 
\\	【形】自由意志の、自発的な
\\	(物事が)可能だ、容認できる
\\	そいつはおかしくて死にそうだ
\\	①〔複数なし〕賢明さ、知恵 
\\	【副】(先行する発言内容を限定して)ただし、もっとも(…ではあるが)
\\	時間が切迫している
\\	ときおり;ときどき
\\	〔現在〕
\\	〔過去〕
\\	〔過分〕
\\	1【他】①〔長さ・重さなど⁴を〕測る、量る 
\\	(単
\\	/(複
\\	スーパー[マーケット] 
\\	(単
\\	/複数なし ⦅スポーツ⦆ 卓球、ピンポン 
\\	慣習、風習、しきたり 
\\	古い慣習
\\	【形】① 平行の 
\\	3格と】
\\	(または
\\	鉄道は川と平行に走っている
\\	①〔複数なし〕発明、考案 
\\	3格. 発音×	(事態が)…
\\	になる
\\	【形】① 動的な、ダイナミックな;活動的な 
\\	(単
\\	/複数なし 
\\	中世 ([略]
\\	【副】初めて
\\	【形】公式の 
\\	2【再】
\\	に〕従事する、取り組む 
\\	私はちょうど仕事中です
\\	〔過去〕
\\	〔過分〕
\\	【他】①〔…⁴を〕感じる、知覚する 
\\	〔…⁴を…と〕感じる
\\	(単
\\	/複数なし コミュニケーション、情報伝達 
\\	【形】一般的な、全般的な;原則的な
\\	「住んでいる」の過去分詞)【形】① いつもの、ふだんの 
\\	②⦅口語⦆ 漠然とした理解、見当、心当たり (㊟ふつう否定文あるいは疑問文で用いられる) 
\\	君は彼女がどこにいるか知っているか? 
\\	⦅口語⦆(人に何かを聞かれて)全然わかりません!;まさかそんな!
\\	誤って、うっかりして
\\	1【他】①
\\	〔…⁴を…
\\	と〕混ぜる、混ぜ合わせる、ブレンドする 
\\	…から
\\	巻き毛、カールした髪の毛 
\\	(髪の毛が)巻き毛である
\\	どうしていいのかわからない
\\	ひざまづく
\\	【副】同様に、同じく 
\\	楽しい週末を!―ありがとう、あなたもね!
\\	3格+
\\	⦅口語⦆ 
\\	とばったり出会う
\\	…の名で、…を代表して、…の委託を受けて
\\	の過去分詞)【形】① 隠れた、人目につかない 
\\	の過去分詞)【形】② 隠れた、人目につかない 
\\	【他】②〔章・ページなど⁴を〕飛ばす、抜かす 
\\	【形】① 趣味の悪い、無粋な;下品な 
\\	【形】危険を伴った 
\\	(物事が)ひとりでに〈必然的に〉進行する、経過する (㊟所有冠詞は主語の人称に応じて変わる)
\\	危険にさらされている
\\	1(単
\\	/(複1・2・4)-
\\	(複
\\	①(麻・亜麻などの)縄、ロープ、綱 
\\	短い〈長い〉縄 
\\	〔現在〕
\\	〔過去〕
\\	〔過分〕
\\	【他】〔荷物など⁴を〕置いていく、あとに残していく 
\\	1【自】①〔完了
\\	(ドイツ南部で
\\	〔…に〕しゃがんでいる、うずくまっている
\\	⦅口語⦆(しつけ・教育などを)これまでより厳しくする
\\	【形】塩辛い、しょっぱい 
\\	(単
\\	/(複
\\	日光浴 
\\	日光浴をする
\\	(単
\\	/複数なし ⦅口語⦆ のどの痛み 
\\	(単
\\	/(複
\\	下水道[網]
\\	(単
\\	/(複1・2・4)-
\\	(複
\\	泥棒、こそ泥 
\\	/[女性]-
\\	【他】①〔…⁴を〕尊敬する 
\\	(単
\\	/(複1・2・4)- (複
\\	生業、商売 
\\	ブラシ、刷毛 
\\	アレルギー、過敏症
\\	…だけではなく、…も
\\	Ⅰ【形】見かけの、外見上の、うわべの
\\	⦅口語⦆ 決して…ない
\\	さしあたり、当分、当面
\\	かかと (靴・靴下のかかとも指す)
\\	自分の本領を発揮している (㊟所有冠詞は主語の人称に応じて変わる)
\\	(単
\\	/(複1・2・4)- (複
\\	倉庫;穀倉 
\\	前もって、あらかじめ
\\	(単
\\	/(複1・2・4)-
\\	(複
\\	結晶 
\\	1【他】①《4格
\\	場所/方向》〔…⁴を…に〕配置する
\\	(単
\\	/(複1・2・4)
\\	(複
\\	投げること;投てき、投球 
\\	4格+
\\	…⁴をする気にならない
\\	(単
\\	/(複
\\	わな、落とし穴
\\	慈悲、温情
\\	Ⅰ【形】偶然の、思いがけない  Ⅱ【副】① 偶然に、たまたま
\\	⦅文語⦆ 自分の才能を隠す〈隠さない〉
\\	(単
\\	/(複1・2・4)..
\\	(複
\\	長所、利点、メリット
\\	(単
\\	/(複
\\	歴史家、[歴]史学者 (〔女性〕-
\\	(または
\\	ついこの間
\\	(単
\\	/(複
\\	ふつう複数で 不動産 
\\	1【他】①〔…⁴を〕(ある場所に)立てて置く、置く
\\	揺りかご 
\\	揺りかごを揺らす 
\\	(単
\\	/(複1・2・4)-
\\	(複
\\	グリル、焼き網;火格子 
\\	ソーセージを焼き網にのせて焼く
\\	(単
\\	/(複
\\	傑作、名作、秀作
\\	田舎での仕事;畑仕事、農作業
\\	ブドウの栽培地、ワインの産地
\\	(単
\\	/(複
\\	動物のぬいぐるみ
\\	【形】① 人気のある、有名な 
\\	【形】① 無害の;(被害・病気が)軽い 
\\	〔現在〕
\\	〔過去〕
\\	〔過分〕
\\	【再】
\\	様態》〔…に〕ふるまう、〔…の〕態度をとる 
\\	ふるまい,態度
\\	ますます;なおさら 
\\	彼女はそれを私に禁じた、しかしこうなったらなおさらそれをやってやる
\\	【形】②⦅口語⦆ ひどい、ものすごい
\\	【形】① 現実の、実在する 
\\	(単
\\	/(複
\\	⦅言語⦆ 同義〈類義〉語
\\	(単
\\	/(複
\\	⦅言語⦆ 反意〈反義〉語
\\	クモの巣
\\	Ⅰ【形】① ぞっとするような、背筋の凍る
\\	【形】⦅文語⦆ 精神の、精神的な
\\	(激しい)感情、心の動き
\\	〔過去〕
\\	〔過分〕
\\	【他】①〔…⁴を〕生み出す、作り出す
\\	(単
\\	/複数なし 
\\	願望;(性的)欲求 
\\	要求、求め;要請
\\	みなしご、孤児
\\	(単
\\	/(複1・2・4)-
\\	(複
\\	①〔複数なし〕支え、よりどころ 
\\	支えを失う
\\	やっと[のことで]、かろうじて
\\	わずかに…だけ 
\\	私はあと3マルクしか持っていない
\\	【他】①〔…⁴を〕相続する 
\\	(単
\\	/(複
\\	⦅動物⦆ クジラ
\\	(依頼などを受けて)喜んで!
\\	(単
\\	/(複1・2・4)
\\	(複
\\	樽 
\\	⦅口語⦆ それはまったくこっけいだ
\\	計画、企画
\\	(単
\\	/複数なし 
\\	混乱、無秩序 
\\	混とん 
\\	腎臓 
\\	【他】〔困難・苦痛など⁴を〕耐え抜く、しのぐ、克服する 
\\	振動する、震える 
\\	(単
\\	/複数なし (犯人などにかける)懸賞金
\\	殺菌、消毒
\\	1【他】〔…⁴を〕(空間的に)拡張する、拡大する 
\\	1【他】②〔道など⁴を〕(事故車などが)ふさぐ
\\	〔過去〕
\\	〔過分〕
\\	【他】②〔…⁴の〕重さがある
\\	【形】① 日当たりの良い;よく晴れた 
\\	(単
\\	/(複
\\	(国境での)検問所
\\	散弾銃 
\\	〔過去〕
\\	〔過分〕
\\	【自】〔完了
\\	①(ほかの人より)先に行く
\\	懐中電灯
\\	(単
\\	/(複1・2・4)-
\\	(複
\\	①[非常]警報 
\\	警報装置
\\	(単
\\	/複数なし 弾薬
\\	(単
\\	/(複1・2・4)- (複
\\	①(車の)ヘッドライト 
\\	サーチライト、投光器;(劇場の)スポットライト
\\	(ある事柄が)うまくいく
\\	すでに言ったように
\\	(単
\\	/(複
\\	(手工業者の)作業〈仕事〉場
\\	②(乗り物での)旅行、ドライブ 
\\	(単
\\	/複数なし ケロシン(航空機・ロケット用燃料)、灯油
\\	【副】1② その代わりに
\\	①〔ふつう単数で〕それること、そらすこと
\\	(車・仕事などの)スピードをあげる
\\	(単
\\	/(複
\\	発電機
\\	(単
\\	/(複1・2・4)-
\\	(複
\\	銃弾、弾丸
\\	まじめに;本気で
\\	(否定を強めて)まったく…ない
\\	使用;適用、応用
\\	【副】通常は、ふつうは
\\	〔過去〕
\\	〔過分〕
\\	【他】①〔…⁴を〕取り付ける、設置する
\\	〔過去〕
\\	〔過分〕 
\\	【自】〔完了
\\	〔…⁴に〕入り込む 
\\	〔…⁴に〕(軍隊・泥棒などが)侵入する 
\\	家屋に侵入する
\\	(単
\\	/複数なし ⦅物理⦆ 重力
\\	(単
\\	/複数なし 視力
\\	影響、効果、成果
\\	【形】② 内面の、精神的な
\\	【自】〔完了
\\	錆びる、錆びつく
\\	【他】②〔…⁴を〕固定する
\\	(単
\\	/(複
\\	スピーカー、拡声器
\\	【冠】《不定数量》1①【複数名詞と】いくつかの、若干の
\\	(単
\\	/(複1・2・4)-
\\	(複
\\	銃、小銃、鉄砲 
\\	(単
\\	/複数なし ⦅俗語⦆① くそ 
\\	くだらない〈つまらない〉こと〈もの〉 
\\	くそ!;畜生!
\\	(単
\\	/複数なし 長持ちすること、耐久性、持ち
\\	4格+
\\	注意・視線など⁴を引きつける
\\	【形】① 地下〈地中〉の
\\	4格+
\\	…に行かせる;…にしたいようにさせる
\\	4格+
\\	…⁴を見つけ出す
\\	(単
\\	/(複1・2・4)- (複
\\	積み上げられた山 
\\	(単
\\	/(複
\\	⦅数⦆ 体積、容積、容量 
\\	(一般に)かさ、大きさ、[総]量
\\	(単
\\	/(複
\\	線路、レール
\\	(単
\\	/(複
\\	ヘリコプター
\\	【形】① 原始的な、未開の 
\\	【形】夜行性の
\\	(単
\\	/複数なし ⦅動物⦆ 冬眠
\\	失敗する運命にある
\\	(単
\\	/(複1・2・4)
\\	(複
\\	小川 
\\	(単
\\	/複数なし【ふつう無冠詞で】 
\\	無線;無線機 
\\	ニュースを無線で聞く 
\\	ラジオ放送;[ラジオ]放送局 
\\	(単
\\	/(複
\\	ヘッドホン
\\	4格+
\\	…⁴に釈明を求める
\\	(核心を捉えて)手短に話す
\\	⦅話⦆ くたくたに疲れている
\\	【他】①〔…⁴に〕えさを与える 
\\	③〔子供・病人⁴に〕食べ物を与える
\\	3格 
\\	リクエスト中	
\\	とかかわっている;…
\\	を相手にしている (㊟
\\	は形式目的語)
\\	チューリップ 
\\	(単
\\	/(種類を表わすときのみ複
\\	雑草 
\\	【形】外国の、外国人の 
\\	増加、増大 
\\	指示;規定、規則
\\	弱変化;(単2・3・4)‐
\\	/(複1・2・3・4)‐
\\	観光客、旅行者、ツーリスト 
\\	/〔女性〕
\\	(単
\\	/(複1・2・4)- (複
\\	通訳 
\\	/〔女性〕-
\\	1【他】①〔…⁴を〕ほかの場所へ移す、置き換える
\\	放牧場、牧場、牧草地 
\\	紀元前 (略:
\\	栽培植物 
\\	【他】〔…⁴を〕判断する、判定する、評価する 
\\	3格と】
\\	私たちは彼のことを外見で判断している
\\	【形】嫉妬深い、やきもちやきの 
\\	(単
\\	/(複1・2・4)-
\\	(複
\\	(駅・街頭などの)売店、キオスク
\\	(単
\\	/(複
\\	⦅口語⦆ 居眠り、うたた寝 
\\	【形】健康上の;健康に役立つ 
\\	健康の〈に関する〉 
\\	健康によい
\\	【他】④〔制度・製品など⁴を〕導入する、採用する [派生] 
\\	導入, 採用
\\	【他】①
\\	〔…⁴に…
\\	への〕動機を与える
\\	拡大、伸長
\\	労働時間
\\	負担
\\	有給休暇
\\	〔形容詞変化〕上司、上役;上官
\\	②〔複数なし〕雰囲気
\\	(単
\\	/複数なし 大部分、大多数;主要な部分
\\	(単
\\	/(複
\\	嗜好品
\\	【自】〔完了
\\	または
\\	錆びる
\\	(単
\\	/(複1・2・3・4)‐
\\	ふつう複数で 利子、利息、金利 
\\	1【他】①〔…⁴を〕思いきってする、あえてする 
\\	【他】〔…⁴を〕返済する、払い戻す 
\\	発音×	〔自分の和訳〕生食用ブドウ 
\\	本質〈基本〉的に
\\	1【他】②〔物価・経済など⁴を〕安定させる 2【再】
\\	①(物価・経済などが)安定する
\\	【副】⦅文語⦆ そのうえ、それに加えて
\\	【自】① 存在する;実在する、ある 
\\	②〔…⁴を〕接ぎ木する
\\	(単
\\	/(複
\\	②〔複数なし〕(農作物の)作付け、栽培
\\	(単
\\	/(複
\\	対抗措置;対[応]策
\\	【形】① 国家の、国の
\\	【形】文字で書かれた、文書による 
\\	複写、コピー 
\\	【形】誤りのない、欠点〈欠陥〉のない
\\	異議、苦情、クレーム
\\	(単
\\	/(複
\\	初心者
\\	【形】① 音楽の 
\\	【他】①(職業として)〔…⁴を〕営む 
\\	店〈ホテル〉を経営する
\\	【形】① 官庁による、公の;公的な 
\\	生産、産出、製造 
\\	1【他】〔…⁴の〕額になる、〔…⁴に〕(数値が)達する
\\	【他】〔…⁴を〕蒸留する
\\	①⦅地名⦆ ブルゴーニュ(フランス南東部の地方)
\\	(単
\\	/(複1・2・3・4)‐ ④〔複数なし〕範囲、限界;(物事の)枠組み
\\	(単
\\	/(複
\\	工業国
\\	(単
\\	/複数なし 知名度
\\	(単
\\	/(複
\\	納税者
\\	破壊;破滅
\\	チーム、選手団 
\\	弱変化;(単2・3・4)‐
\\	/(複
\\	心理学者 (〔女性〕-
\\	(単
\\	/(複1・2・4)- (複
\\	プレーヤー、選手;ゲーム参加者 
\\	/〔女性〕-
\\	(単
\\	/(複1・2・4)..
\\	(複
\\	(レストランを兼ねた)宿屋、旅館 
\\	1【自】痛む 
\\	【形】① のどが渇いた 
\\	飲食店、レストラン
\\	(単
\\	/(複
\\	建築士
\\	(単
\\	/(まれに複
\\	洪水、氾濫;高潮
\\	(単
\\	/複数なし 長雨
\\	(単
\\	/(複1・2・4)..
\\	(複
\\	決意、意図 
\\	1【他】①〔…⁴を〕支える 
\\	【形】① わずかな;少しの、少ない
\\	消防隊、消防署
\\	【自】(紛争などが)エスカレートする
\\	(単
\\	/(複
\\	インタビュー 
\\	【副】① その間に
\\	(単
\\	/(複
\\	レンタカー
\\	【形】忘れられない
\\	(単
\\	/複数なし 遺産、相続財産
\\	(単
\\	/(複
\\	原始林、原生林 
\\	(単
\\	/(複
\\	(旅行途中での)寄り道 
\\	ベルリンへ寄り道する
\\	(単
\\	/(複1・2・4)-
\\	(複
\\	利益、もうけ;得 
\\	⦅経済⦆ 株、株券
\\	インフレ[-ション];供給過剰
\\	(単
\\	/(複
\\	新興工業国
\\	(単
\\	/複数なし ①⦅鉱物⦆ 金、黄金 ([記号]
\\	【形】黒髪の
\\	ヘアスタイル、髪形 
\\	(単
\\	/(複
\\	ふつう複数で コンタクトレンズ
\\	【形】① スポーツの、運動に関する 
\\	③(服・靴などが)スポーツに適した、スポーツ用の
\\	(単
\\	/(複
\\	家事;宿題
\\	(単
\\	/(複
\\	(自動車などの)レーサー
\\	目下;たった今
\\	(単
\\	/(複1・2・4)-
\\	(複
\\	スキャンダル 
\\	【形】① 均一の、画一的な 
\\	統一のとれた、一貫した
\\	文学、文芸 
\\	②〔複数なし〕【集合的に】書物、書籍、文献
\\	【形】① 受動的な、受け身の;消極的な 
\\	【冠】《定冠詞類》1① そのような、あのような、このような 
\\	〔現在〕
\\	〔過去〕
\\	〔過分〕
\\	1【他】①〔約束・期限など⁴を〕守る
\\	結果、帰結、結論 
\\	【形】具体的な;具象的な 
\\	①〔ふつう単数で〕批判;非難 
\\	批評、評論、論評
\\	(単
\\	/(複1・2・4)
\\	(複
\\	②(学問・職業などの)専門領域、分野;学科、科目
\\	(単
\\	/(複
\\	学歴
\\	(単
\\	/複数なし 数学 ([略]
\\	(単
\\	/(複
\\	開館〈営業〉時間
\\	予約
\\	(単
\\	/(複1・2・4)-
\\	(複
\\	眺め、見晴らし 
\\	(単
\\	/(複
\\	休日;(飲食店の)休業日
\\	(単
\\	/複数なし 
\\	静寂、静けさ;沈黙 
\\	宿泊
\\	(単
\\	/(複
\\	イルカ 
\\	(単
\\	/(複
\\	株価、株式相場
\\	貧乏、貧困 
\\	【他】〔お金⁴を〕払い込む、預け入れる
\\	(単
\\	/(複
\\	貯金通帳
\\	(単
\\	/(複
\\	国債
\\	(単
\\	/(複
\\	貴金属
\\	ちょっとしたもの ②〔ふつう複数で〕ささいなこと、些事
\\	(単
\\	/(複
\\	マッサージ、あんま 
\\	(単
\\	/複数なし 
\\	休養、保養
\\	(単
\\	/(まれに複
\\	消費;消費量〈額〉 
\\	(単
\\	/(複
\\	交通機関 
\\	ハイキング、徒歩旅行、遠足 
\\	危機;⦅経済⦆ 恐慌 
\\	(単
\\	/複数なし 生活水準
\\	【他】〔成功など⁴を〕達成する、成し遂げる
\\	証明書;(学業・勤務の)成績証明書
\\	(単
\\	/(複1・2・4)..
\\	(複
\\	履歴書
\\	(単
\\	/(複1・2・4)-
\\	(複
\\	所得、収入;収益、もうけ
\\	(単
\\	/(複
\\	研究所 
\\	(単
\\	/(複1・2・4)- (複
\\	歩行者 
\\	/〔女性〕-
\\	工業、産業 
\\	①(車・機械などの)故障;(タイヤの)パンク 
\\	(単
\\	/(複
\\	(車の)トランク[ルーム] 
\\	ブレーキ、制動装置 
\\	(単
\\	/(複
\\	ガソリンスタンド 
\\	(単
\\	/(複
\\	(自動車などの)ワイパー
\\	【副】朝に 
\\	(または
\\	朝の6時に
\\	1【自】① ののしる、どなり散らす、悪口を言う 
\\	4格と】
\\	彼は隣人の騒音にひどく文句を言う
\\	その結果…、そのため… 
\\	(または
\\	3格	
\\	と比較して
\\	〔過去〕
\\	〔過分〕
\\	【自】《3格》②
\\	に〕支配される、さらされる
\\	(単
\\	/(複
\\	ふつう複数で (数値などの)揺れ、変動
\\	(単
\\	/複数なし 無秩序、乱雑、混乱
\\	(オーストリア・スイスでは㊚も)(単
\\	/(複
\\	温度計;体温計
\\	(単
\\	/複数なし クイズ
\\	【形】工業の、産業の
\\	(単
\\	/(複
\\	労働力
\\	(単
\\	/(複
\\	原料
\\	(単
\\	/複数なし リサイクリング、資源再利用 
\\	(単
\\	/(複
\\	ごみ集積場 
\\	(単
\\	/複数なし ごみの分別
\\	3格+
\\	⦅口語⦆ 
\\	の神経に障る、…
\\	をいらいらさせる
\\	【他】①〔…⁴に〕予防接種をする 
\\	予防接種を受ける
\\	【形】この前の、先の 
\\	先週に
\\	(単
\\	【定冠詞と】南極[地方]
\\	(単
\\	/(複
\\	歯ブラシ 
\\	【副】一つずつ;少しずつ
\\	(単
\\	/(複
\\	(ビニール製の)大型ごみ袋 
\\	〔現在〕
\\	〔過去〕
\\	〔過分〕
\\	1【自】〔完了
\\	(乗り物が)入ってくる
\\	3格 
\\	リクエスト中	
\\	に貢献する
\\	【形】正確な、精密な 
\\	(単
\\	/(複1・2・4)-
\\	(複
\\	世界大戦
\\	そのときまで;そこまで
\\	(単
\\	/(複
\\	ワイン通
\\	(単
\\	/複数なし 
\\	評判、名声、世評
\\	【他】①〔…⁴を〕収穫する、取り入れる [派生] 
\\	収穫,取り入れ
\\	【形】不利な、損な、不都合な 
\\	現象、現われ 
\\	①⦅商業⦆ 需要 
\\	【形】① 均一な、一様な
\\	【再】
\\	作用する;結果をもたらす 
\\	有利〈不利〉に作用する
\\	【形】気候[上]の;気候に依拠した
\\	(単
\\	/(複
\\	売上[金]、収益[金]
\\	(単
\\	/複数なし 
\\	保存、維持
\\	(単
\\	/複数なし 多様、多彩
\\	【形】① 不都合な、不利な
\\	1【他】〔…⁴に〕化粧する、メーキャップをする 
\\	2【再】
\\	化粧する
\\	いわゆる 
\\	【形】① 否定の;否定的な、拒否的な 
\\	好ましくない、不利な
\\	・2格支配 【前】…に基づいて;(理由を表わして)…のために
\\	実現する;完成する
\\	(単
\\	/(複
\\	消費者 (〔女性〕-
\\	【形】責任のある 
\\	4格 
\\	…⁴に対して責任がある
\\	(単
\\	/(複
\\	登ること;上昇、増加 
\\	(単
\\	/複数なし 停止、静止;停滞
\\	規制、取り締まり、整理
\\	【副】⦅文語⦆ とりわけ、特に
\\	1【他】①〔公園など⁴を〕造る
\\	【他】〔可能性・類似性など⁴を〕示す
\\	(単
\\	/複数なし 抵抗力、耐性
\\	①〔複数なし〕少数;少数派 
\\	(新聞第1面の)大見出し
\\	1【他】①〔…⁴を〕述べる、陳述する
\\	弱変化;(単2・3・4)‐
\\	/(複1・2・3・4)‐
\\	カメラマン、写真家 
\\	/〔女性〕-
\\	(単
\\	/(複1・2・3・4)..
\\	デモクラシー、民主主義 
\\	(単
\\	/(複
\\	老人ホーム、養老院
\\	(単
\\	/複数なし 社会主義
\\	官庁、役所、当局 
\\	(単
\\	/複数なし 
\\	選挙権;被選挙権
\\	(単
\\	/(複1・2・4)-
\\	(複
\\	議会、国会 
\\	[国会]議事堂 
\\	憲法改正
\\	大多数、過半数、大半;多数派 
\\	(単
\\	/複数なし (ドイツの)連邦議会 
\\	〔形容詞変化〕代表;[国会]議員 ([略]
\\	(単
\\	/(複1・2・4)-
\\	(複
\\	家計;家事、家政
\\	【形】① 正しい、正確な、誤りのない 
\\	【形】① 自立した、自主的な
\\	(単
\\	/複数なし 物おじ、恐れ、臆病 
\\	【副】②(税金などを差し引いた)手取りで
\\	(単
\\	/(複
\\	⦅口語⦆ 一時的な仕事、アルバイト;勤め口、職、職業 
\\	ユースホステル
\\	(単
\\	/(複
\\	シーツ、敷布
\\	(単
\\	/(複1・2・4)-
\\	(複
\\	カメラ、写真機 
\\	(単
\\	/(複1・2・4)..
\\	(複
\\	マッチ 
\\	【形】① 保守的な;保守主義の;旧来の、伝統的な 
\\	(単
\\	/(複
\\	農民;(専門教育を受けた)農業経営者
\\	(単
\\	/(複1・2・4)- (複
\\	壁造り職人、左官 
\\	(単
\\	/(複
\\	テレビゲーム
\\	【他】
\\	2(商品として)市場に出す、商品化する
\\	1(単
\\	/(複1・2・4)-
\\	(複
\\	または(複1・2・4)..
\\	(複
\\	スローガン、標語、キャッチフレーズ
\\	3格+
\\	にとって楽しい
\\	(単
\\	/(複
\\	職業安定所 
\\	(単
\\	/複数なし 
\\	心理学 
\\	【形】楽観的な、楽天主義の 
\\	(単
\\	/(複1・2・3・4)‐または..
\\	試験 
\\	(単
\\	/(複
\\	職業[上の]経験
\\	(単
\\	/(複
\\	博士の称号
\\	(単
\\	/複数なし 生命の危険
\\	【前】〖4格支配〗①(手段・方法)…でもって 
\\	⦅国名⦆ ロシア
\\	宗教;宗派
\\	1【再】①
\\	〔…⁴に〕関係する
\end{CJK}
\end{document}