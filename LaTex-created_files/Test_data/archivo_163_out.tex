\documentclass[8pt]{extreport} 
\usepackage{hyperref}
\usepackage{CJKutf8}
\begin{document}
\begin{CJK}{UTF8}{min}
\\	それ	
\\	それはとってもいい話だ。	
\\	それ 
\\	それ は とっても いい はなし だ			
\\	見る	
\\	みる			
\\	私は絵を見るのが好きです。	
\\	見る 
\\	わたし は え を みる の が すき です			
\\	円	
\\	えん			
\\	そこに大きな円を描いて。	
\\	円 
\\	そこ に おおき な えん を かいて			
\\	多い	
\\	おおい			
\\	京都にはお寺が多い。	
\\	多い 
\\	きょうと に は おてら が おおい			
\\	家	
\\	うち			
\\	家に遊びに来てください。	
\\	家 
\\	うち に あそび に きて ください			
\\	これ	
\\	これをください。	
\\	これ 
\\	これ を ください			
\\	新しい	
\\	あたらしい			
\\	私	
\\	わたし			
\\	私が行きましょう。	
\\	私 
\\	わたし が いきましょう			
\\	仕事	
\\	しごと			
\\	3月は仕事が忙しい。	
\\	仕事 
\\	さんがつ は しごと が いそがしい			
\\	始める	
\\	はじめる			
\\	テストを始めてください。	
\\	始める 
\\	てすと を はじめて ください			
\\	なる	
\\	彼は医者になりました。	
\\	なる 
\\	かれ は いしゃ に なりました			
\\	する	
\\	出る	
\\	でる			
\\	今朝は早く家を出ました。	
\\	出る 
\\	けさ は はやく いえ を でました			
\\	使う	
\\	つかう			
\\	このパソコンを使ってください。	
\\	使う 
\\	この ぱそこん を つかって ください			
\\	ところ	
\\	私は友達のところに泊まった。	
\\	ところ 
\\	わたし は ともだち の ところ に とまった			
\\	思う	
\\	おもう			
\\	私もそう思います。	
\\	思う 
\\	わたし も そう おもいます			
\\	持つ	
\\	もつ			
\\	私は車を持っています。	
\\	持つ 
\\	わたし は くるま を もって います			
\\	時間	
\\	じかん			
\\	今は時間がありません。	
\\	時間 
\\	いま は じかん が ありません			
\\	今年	
\\	ことし			
\\	今年はイタリアに旅行したい。	
\\	今年 
\\	ことし は いたりあ に りょこう したい			
\\	知る	
\\	しる			
\\	誰もその話を知らない。	
\\	知る 
\\	だれ も その はなし を しらない			
\\	作る	
\\	つくる			
\\	今、朝ご飯を作っています。	
\\	作る 
\\	いま あさごはん を つくって います			
\\	同じ	
\\	おなじ			
\\	彼の日本語のレベルは私と同じ位だ。	
\\	同じ 
\\	かれ の にほんご の れべる は わたし と おなじ くらい だ			
\\	今	
\\	いま			
\\	彼は今、勉強しています。	
\\	今 
\\	かれ は いま べんきょう して います			
\\	続く	
\\	つづく			
\\	工事は3月まで続きます。	
\\	続く 
\\	こうじ は さんがつ まで つづきます			
\\	国	
\\	くに			
\\	未だ	
\\	まだ			
\\	やる	
\\	一緒に宿題をやろう。	
\\	やる 
\\	いっしょ に しゅくだい を やろう			
\\	言う	
\\	いう			
\\	上司が「一杯、飲もう。」と言った。	
\\	言う 
\\	じょうし が いっぱい のもう と いった			
\\	少ない	
\\	すくない			
\\	今年は雨が少ないです。	
\\	少ない 
\\	ことし は あめ が すくない です 。			
\\	行く	
\\	いく			
\\	決める	
\\	きめる			
\\	帰国することに決めました。	
\\	決める 
\\	きこく する こと に きめました			
\\	高い	
\\	たかい			
\\	これがこの町で一番高いビルです。	
\\	高い 
\\	これ が この まち で いちばん たかい びる です			
\\	考える	
\\	かんがえる			
\\	よく考えてください。	
\\	考える 
\\	よく かんがえて ください			
\\	感じる	
\\	かんじる			
\\	膝に痛みを感じます。	
\\	感じる 
\\	ひざ に いたみ を かんじます			
\\	子供	
\\	こども			
\\	電車で子供が騒いでいた。	
\\	子供 
\\	でんしゃ で こども が さわいで いた			
\\	そう	
\\	私もそう思います。	
\\	そう 
\\	わたし も そう おもいます			
\\	大学	
\\	だいがく			
\\	もう	
\\	彼はもう帰りました。	
\\	もう 
\\	かれ は もう かえりました 。			
\\	近く	
\\	ちかく			
\\	駅の近くで食事をした。	
\\	近く 
\\	えき の ちかく で しょくじ を した			
\\	入れる	
\\	いれる			
\\	彼はかばんに手帳を入れた。	
\\	入れる 
\\	かれ は かばん に てちょう を いれた			
\\	学生	
\\	がくせい			
\\	彼は真面目な学生です。	
\\	学生 
\\	かれ は まじめ な がくせい です			
\\	午後	
\\	ごご			
\\	明日の午後、お客様が来る。	
\\	午後 
\\	あす の ごご おきゃくさま が くる			
\\	長い	
\\	ながい			
\\	彼女の髪はとても長い。	
\\	長い 
\\	かのじょ の かみ は とても ながい			
\\	本	
\\	ほん			
\\	本を1冊買いました。	
\\	本 
\\	ほん を いっさつ かいました			
\\	よく	
\\	彼女はよく旅行に行きます。	
\\	よく 
\\	かのじょ は よく りょこう に いきます			
\\	彼女	
\\	かのじょ			
\\	彼女は
\\	です。	
\\	彼女 
\\	かのじょ は おーえる です			
\\	聞く	
\\	きく			
\\	彼女はラジオを聞いています。	
\\	聞く 
\\	かのじょ は らじお を きいて います			
\\	どう	
\\	あなたはどう思いますか。	
\\	どう 
\\	あなた は どう おもいます か			
\\	生きる	
\\	いきる			
\\	皆一生懸命生きている。	
\\	生きる 
\\	みんな いっしょうけんめい いきて いる			
\\	言葉	
\\	ことば			
\\	顔	
\\	かお			
\\	彼はタオルで顔を拭きました。	
\\	顔 
\\	かれ は たおる で かお を ふきました			
\\	どこ	
\\	あの本をどこに置きましたか。	
\\	どこ 
\\	あの ほん を どこ に おきました か			
\\	一つ	
\\	ひとつ			
\\	それを一つください。	
\\	それ 
\\	一つ 
\\	それ を ひとつ ください			
\\	あげる	
\\	この本、あなたにあげます。	
\\	あげる 
\\	この ほん あなた に あげます			
\\	こう	
\\	こう小さい字は読めない。	
\\	こう 
\\	こう ちいさい じ は よめない			
\\	学校	
\\	がっこう			
\\	学校は8時半に始まります。	
\\	学校 
\\	がっこう は はちじはん に はじまります			
\\	くれる	
\\	友達が誕生日プレゼントをくれた。	
\\	くれる 
\\	ともだち が たんじょうび ぷれぜんと を くれた			
\\	起きる	
\\	おきる			
\\	私は毎朝6時に起きます。	
\\	起きる 
\\	わたし は まいあさ ろくじ に おきます			
\\	春	
\\	はる			
\\	今年の春は暖かいね。	
\\	春 
\\	ことし の はる は あたたかい ね			
\\	午前	
\\	ごぜん			
\\	午前9時のニュースです。	
\\	午前 
\\	ごぜん くじ の にゅーす です			
\\	買う	
\\	かう			
\\	郵便局で切手を買いました。	
\\	買う 
\\	ゆうびんきょく で きって を かいました			
\\	別	
\\	べつ			
\\	終わる	
\\	おわる			
\\	宿題は未だ終わっていません。	
\\	未だ 
\\	終わる 
\\	しゅくだい は まだ おわって いません			
\\	会議は4時に終わります。	
\\	終わる 
\\	かいぎ は よじ に おわります			
\\	若い	
\\	わかい			
\\	彼はまだ若いです。	
\\	若い 
\\	かれ は まだ わかい です			
\\	かなり	
\\	彼はかなり英語が上手です。	
\\	かなり 
\\	かれ は かなり えいご が じょうず です			
\\	置く	
\\	おく			
\\	彼はかばんをいすの上に置きました。	
\\	置く 
\\	かれ は かばん を いす の うえ に おきました			
\\	住む	
\\	すむ			
\\	彼は会社の近くに住んでいる。	
\\	住む 
\\	かれ は かいしゃ の ちかく に すんで いる			
\\	戻る	
\\	もどる			
\\	今、会社に戻ります。	
\\	戻る 
\\	いま かいしゃ に もどります			
\\	働く	
\\	はたらく			
\\	後	
\\	あと			
\\	仕事の後、映画を見た。	
\\	見る 
\\	仕事 
\\	後 
\\	しごと の あと えいが を みた			
\\	先生	
\\	せんせい			
\\	私は日本語の先生になりたいです。	
\\	先生 
\\	わたし は にほんご の せんせい に なりたい です			
\\	立つ	
\\	たつ			
\\	彼はステージに立った。	
\\	立つ 
\\	かれ は すてーじ に たった			
\\	気	
\\	き			
\\	彼は意外に気が小さい。	
\\	気 
\\	かれ は いがい に き が ちいさい			
\\	呼ぶ	
\\	よぶ			
\\	ウェイターを呼びましょう。	
\\	呼ぶ 
\\	うぇいたー を よびましょう			
\\	もっと	
\\	もっと近くに来てください。	
\\	近く 
\\	もっと 
\\	もっと ちかく に きて ください			
\\	大学に行ってもっと勉強したいです。	
\\	大学 
\\	もっと 
\\	だいがく に いって もっと べんきょう したい です			
\\	帰る	
\\	かえる			
\\	家に帰ろう。	
\\	家 
\\	帰る 
\\	うち に かえろう			
\\	気持ち	
\\	きもち			
\\	彼の気持ちが分からない。	
\\	気持ち 
\\	かれ の きもち が わからない			
\\	飛ぶ	
\\	とぶ			
\\	鳥が飛んでいます。	
\\	飛ぶ 
\\	とり が とんで います			
\\	車	
\\	くるま			
\\	彼の車は新しい。	
\\	新しい 
\\	車 
\\	かれ の くるま は あたらしい			
\\	弟が車を買った。	
\\	車 
\\	おとうと が くるま を かった			
\\	変える	
\\	かえる			
\\	旅行の日程を変えました。	
\\	変える 
\\	りょこう の にってい を かえました			
\\	広い	
\\	ひろい			
\\	彼の家はとても広い。	
\\	広い 
\\	かれ の いえ は とても ひろい			
\\	分かる	
\\	わかる			
\\	この言葉の意味が分かりません。	
\\	言葉 
\\	分かる 
\\	この ことば の いみ が わかりません			
\\	数	
\\	かず			
\\	グラスの数が足りません。	
\\	数 
\\	ぐらす の かず が たりません			
\\	近い	
\\	ちかい			
\\	そこ	
\\	走る	
\\	はしる			
\\	彼は毎晩3キロ走っています。	
\\	走る 
\\	かれ は まいばん さんきろ はしって います			
\\	売る	
\\	うる			
\\	彼は家を売った。	
\\	売る 
\\	かれ は いえ を うった			
\\	朝	
\\	あさ			
\\	気持ちのいい朝です。	
\\	気持ち 
\\	朝 
\\	きもち の いい あさ です			
\\	教える	
\\	おしえる			
\\	彼は数学を教えています。	
\\	教える 
\\	かれ は すうがく を おしえて います			
\\	歩く	
\\	あるく			
\\	駅まで歩きましょう。	
\\	歩く 
\\	えき まで あるきましょう			
\\	直ぐ	
\\	すぐ			
\\	直ぐ行きます。	
\\	行く 
\\	直ぐ 
\\	すぐ いきます			
\\	会う	
\\	あう			
\\	また会いましょう。	
\\	会う 
\\	また あいましょう			
\\	部屋	
\\	へや			
\\	私の部屋は2階にあります。	
\\	部屋 
\\	わたし の へや は にかい に あります			
\\	起こる	
\\	おこる			
\\	町で大事件が起こりました。	
\\	起こる 
\\	まち で だいじけん が おこりました			
\\	秋	
\\	あき			
\\	難しい	
\\	むずかしい			
\\	この本は難しいですね。	
\\	本 
\\	難しい 
\\	この ほん は むずかしい です ね			
\\	送る	
\\	おくる			
\\	彼の家に荷物を送りました。	
\\	送る 
\\	かれ の いえ に にもつ を おくりました			
\\	死ぬ	
\\	しぬ			
\\	乗る	
\\	のる			
\\	駅からはタクシーに乗ってください。	
\\	乗る 
\\	えき からは たくしー に のって ください			
\\	安い	
\\	やすい			
\\	この服はとても安かった。	
\\	安い 
\\	この ふく は とても やすかった			
\\	書く	
\\	かく			
\\	彼に手紙を書きました。	
\\	書く 
\\	かれ に てがみ を かきました			
\\	話す	
\\	はなす			
\\	母と電話で話しました。	
\\	話す 
\\	はは と でんわ で はなしました			
\\	頭	
\\	あたま			
\\	今朝から頭が痛い。	
\\	頭 
\\	けさ から あたま が いたい			
\\	体	
\\	からだ			
\\	私は体が丈夫だ。	
\\	体 
\\	わたし は からだ が じょうぶ だ			
\\	ここ	
\\	駅はここから近いです。	
\\	近い 
\\	ここ 
\\	えき は ここ から ちかい です			
\\	ここに本があります。	
\\	ここ 
\\	ここ に ほん が あります			
\\	待つ	
\\	まつ			
\\	あなたが来るのを待っています。	
\\	待つ 
\\	あなた が くる の を まって います			
\\	低い	
\\	ひくい			
\\	彼は背が低い。	
\\	低い 
\\	かれ は せ が ひくい			
\\	もらう	
\\	彼女からプレゼントをもらいました。	
\\	もらう 
\\	かのじょ から ぷれぜんと を もらいました			
\\	食べる	
\\	たべる			
\\	昨日タイカレーを食べました。	
\\	食べる 
\\	きのう たいかれー を たべました			
\\	早い	
\\	はやい			
\\	まだ学校へ行くには早い時間です。	
\\	早い 
\\	まだ がっこう へ いく に は はやい じかん です			
\\	兄	
\\	あに			
\\	兄は水泳が得意です。	
\\	兄 
\\	あに は すいえい が とくい です			
\\	名前	
\\	なまえ			
\\	あなたの名前を教えてください。	
\\	教える 
\\	名前 
\\	あなた の なまえ を おしえて ください			
\\	夫	
\\	おっと			
\\	私の夫はサラリーマンです。	
\\	夫 
\\	わたし の おっと は さらりーまん です			
\\	一	
\\	いち			
\\	一から始めましょう。	
\\	始める 
\\	一 
\\	いち から はじめましょう			
\\	結婚	
\\	けっこん			
\\	彼女は秋に結婚します。	
\\	する 
\\	秋 
\\	結婚 
\\	かのじょ は あき に けっこん します			
\\	彼女は来月結婚します。	
\\	結婚 
\\	かのじょ は らいげつ けっこん します			
\\	親	
\\	おや			
\\	親の愛は有り難い。	
\\	親 
\\	おや の あい は ありがたい			
\\	古い	
\\	ふるい			
\\	私は古い車が好きです。	
\\	古い 
\\	わたし は ふるい くるま が すき です			
\\	美しい	
\\	うつくしい			
\\	この絵は美しいです。	
\\	美しい 
\\	この え は うつくしい です			
\\	足	
\\	あし			
\\	彼は足が長い。	
\\	足 
\\	かれ は あし が ながい			
\\	次	
\\	つぎ			
\\	起こす	
\\	おこす			
\\	明日、6時に起こしてください。	
\\	起こす 
\\	あす ろくじ に おこして ください			
\\	見せる	
\\	みせる			
\\	別の本も見せてください。	
\\	別 
\\	見せる 
\\	べつ の ほん も みせて ください			
\\	その写真を見せてください。	
\\	見せる 
\\	その しゃしん を みせて ください			
\\	娘	
\\	むすめ			
\\	私の娘はアメリカにいます。	
\\	娘 
\\	わたし の むすめ は あめりか に います			
\\	とても	
\\	この本はとてもおもしろい。	
\\	とても 
\\	この ほん は とても おもしろい			
\\	少し	
\\	すこし			
\\	私の国について少しお話しましょう。	
\\	国 
\\	少し 
\\	わたし の くに に ついて すこし おはなし しましょう			
\\	少し疲れました。	
\\	少し 
\\	すこし つかれました			
\\	質問	
\\	しつもん			
\\	質問の意味は分かりましたか。	
\\	分かる 
\\	質問 
\\	しつもん の いみ は わかりました か			
\\	質問のある方はどうぞ。	
\\	質問 
\\	しつもん の ある かた は どうぞ			
\\	楽しむ	
\\	たのしむ			
\\	色	
\\	いろ			
\\	すてきな色のセーターですね。	
\\	色 
\\	すてき な いろ の せーたー です ね			
\\	皆	
\\	みんな			
\\	みんなにお菓子をあげましょう。	
\\	みんな 
\\	みんな に おかし を あげましょう			
\\	取る	
\\	とる			
\\	テストでいい点を取った。	
\\	取る 
\\	てすと で いい てん を とった			
\\	勉強	
\\	べんきょう			
\\	私は日本語を勉強しています。	
\\	勉強 
\\	わたし は にほんご を べんきょう して います			
\\	出来る	
\\	できる			
\\	彼女はイタリア語が出来ます。	
\\	出来る 
\\	かのじょ は いたりあご が できます			
\\	短い	
\\	みじかい			
\\	彼は足が短い。	
\\	短い 
\\	かれ は あし が みじかい			
\\	落ちる	
\\	おちる			
\\	猿も木から落ちる。	
\\	落ちる 
\\	さる も き から おちる			
\\	息子	
\\	むすこ			
\\	うちの息子は大学1年生です。	
\\	息子 
\\	うち の むすこ は だいがく いちねんせい です			
\\	白い	
\\	しろい			
\\	彼は歯が白い。	
\\	白い 
\\	かれ は は が しろい			
\\	誰	
\\	だれ			
\\	誰と会ってみたいですか。	
\\	誰 
\\	だれ と あって みたい です か			
\\	いつも	
\\	飛行機	
\\	ひこうき			
\\	息子は飛行機のおもちゃが好きです。	
\\	息子 
\\	飛行機 
\\	むすこ は ひこうき の おもちゃ が すき です			
\\	勝つ	
\\	かつ			
\\	今日はヤンキースが勝った。	
\\	勝つ 
\\	きょう は やんきーす が かった			
\\	病気	
\\	びょうき			
\\	祖父が病気になった。	
\\	なる 
\\	病気 
\\	そふ が びょうき に なった			
\\	犬が病気で死にました。	
\\	死ぬ 
\\	病気 
\\	いぬ が びょうき で しにました。			
\\	冬	
\\	ふゆ			
\\	カナダの冬はとても寒いです。	
\\	冬 
\\	かなだ の ふゆ は とても さむい です			
\\	年	
\\	とし			
\\	新しい年が始まりました。	
\\	新しい 
\\	年 
\\	あたらしい とし が はじまりました			
\\	重い	
\\	おもい			
\\	このかばんは重いです。	
\\	重い 
\\	この かばん は おもい です			
\\	胸	
\\	むね			
\\	胸に少し痛みがあります。	
\\	胸 
\\	むね に すこし いたみ が あります			
\\	読む	
\\	よむ			
\\	彼女は雑誌を読んでいます。	
\\	読む 
\\	かのじょ は ざっし を よんで います			
\\	払う	
\\	はらう			
\\	私が払いましょう。	
\\	払う 
\\	わたし が はらいましょう			
\\	軽い	
\\	かるい			
\\	この靴はとても軽い。	
\\	軽い 
\\	この くつ は とても かるい			
\\	見つける	
\\	みつける			
\\	新しい仕事を見つけました。	
\\	見つける 
\\	あたらしい しごと を みつけました。			
\\	いつ	
\\	次はいつ会いましょうか。	
\\	会う 
\\	次 
\\	いつ 
\\	つぎ は いつ あいましょう か			
\\	彼女はいつ来ますか。	
\\	いつ 
\\	かのじょ は いつ きます か			
\\	友達	
\\	ともだち			
\\	私には友達がたくさんいます。	
\\	友達 
\\	わたし に は ともだち が たくさん います			
\\	忘れる	
\\	わすれる			
\\	約束を忘れないでください。	
\\	忘れる 
\\	やくそく を わすれない で ください			
\\	酒	
\\	さけ			
\\	彼女は酒に強い。	
\\	酒 
\\	かのじょ は さけ に つよい			
\\	どちら	
\\	肉と魚とどちらが好きですか。	
\\	どちら 
\\	にく と さかな と どちら が すき です か			
\\	姉	
\\	あね			
\\	姉は銀行で働いています。	
\\	働く 
\\	姉 
\\	あね は ぎんこう で はたらいています 。			
\\	姉は大学生です。	
\\	姉 
\\	あね は だいがくせい です			
\\	覚える	
\\	おぼえる			
\\	妹は平仮名を全部覚えました。	
\\	覚える 
\\	いもうと は ひらがな を ぜんぶ おぼえました			
\\	狭い	
\\	せまい			
\\	私の部屋は狭いです。	
\\	部屋 
\\	狭い 
\\	わたし の へや は せまい です			
\\	好き	
\\	すき			
\\	私はワインが好きです。	
\\	好き 
\\	わたし は わいん が すき です			
\\	建てる	
\\	たてる			
\\	私たちは来年、家を建てます。	
\\	建てる 
\\	わたしたち は らいねん いえ を たてます			
\\	どれ	
\\	この中でどれが好きですか。	
\\	どれ 
\\	この なか で どれ が すき です か			
\\	赤い	
\\	あかい			
\\	赤いバラを買いました。	
\\	買う 
\\	赤い 
\\	あかい ばら を かいました			
\\	今日	
\\	きょう			
\\	今日は一人の時間を楽しみたい。	
\\	楽しむ 
\\	今日 
\\	きょう は ひとり の じかん を たのしみたい			
\\	今日は仕事がありません。	
\\	今日 
\\	きょう は しごと が ありません			
\\	着る	
\\	きる			
\\	今日はスーツを着ています。	
\\	着る 
\\	きょう は すーつ を きて います			
\\	笑う	
\\	わらう			
\\	赤ん坊が笑っています。	
\\	笑う 
\\	あかんぼう が わらって います			
\\	一番	
\\	いちばん			
\\	彼はクラスで一番背が高い。	
\\	高い 
\\	一番 
\\	かれ は くらす で いちばん せ が たかい			
\\	授業	
\\	じゅぎょう			
\\	今日は日本語の授業があります。	
\\	授業 
\\	きょう は にほんご の じゅぎょう が あります			
\\	週	
\\	しゅう			
\\	その次の週は空いてますか。	
\\	次 
\\	週 
\\	その つぎ の しゅう は あいてます か			
\\	漢字	
\\	かんじ			
\\	漢字は中国から来ました。	
\\	漢字 
\\	かんじ は ちゅうごく から きました			
\\	自転車	
\\	じてんしゃ			
\\	毎日、駅まで自転車で行きます。	
\\	自転車 
\\	まいにち えき まで じてんしゃ で いきます			
\\	電車	
\\	でんしゃ			
\\	私は電車で通学しています。	
\\	電車 
\\	わたし は でんしゃ で つうがく して います			
\\	探す	
\\	さがす			
\\	彼は郵便局を探していました。	
\\	探す 
\\	かれ は ゆうびんきょく を さがして いました			
\\	紙	
\\	かみ			
\\	紙と鉛筆はありますか。	
\\	紙 
\\	かみ と えんぴつ は あります か			
\\	歌う	
\\	うたう			
\\	私たちは大きな声で歌いました。	
\\	歌う 
\\	わたしたち は おおき な こえ で うたいました			
\\	遅い	
\\	おそい			
\\	前の車はとても遅い。	
\\	遅い 
\\	まえ の くるま は とても おそい			
\\	首	
\\	くび			
\\	きりんの首は長い。	
\\	首 
\\	きりん の くび は ながい			
\\	速い	
\\	はやい			
\\	彼は走るのが速い。	
\\	走る 
\\	速い 
\\	かれ は はしる の が はやい			
\\	一緒に	
\\	いっしょに			
\\	一緒に帰りましょう。	
\\	帰る 
\\	一緒に 
\\	いっしょに かえりましょう			
\\	友達と一緒に宿題をした。	
\\	する 
\\	一緒に 
\\	ともだち と いっしょ に しゅくだい を した			
\\	日記	
\\	にっき			
\\	私は毎日、日記を付けています。	
\\	日記 
\\	わたし は まいにち にっき を つけて います			
\\	今月	
\\	こんげつ			
\\	今月はとても忙しい。	
\\	今月 
\\	こんげつ は とても いそがしい			
\\	遊ぶ	
\\	あそぶ			
\\	子供たちが公園で遊んでいる。	
\\	遊ぶ 
\\	こどもたち が こうえん で あそんで いる			
\\	遠い	
\\	とおい			
\\	家から学校までは遠いです。	
\\	遠い 
\\	いえ から がっこう まで は とおい です			
\\	弱い	
\\	よわい			
\\	その子は体が少し弱い。	
\\	体 
\\	弱い 
\\	その こ は からだ が すこし よわい			
\\	耳	
\\	みみ			
\\	彼女は耳がよく聞こえません。	
\\	耳 
\\	かのじょ は みみ が よく きこえません			
\\	座る	
\\	すわる			
\\	そこに座ってください。	
\\	そこ 
\\	座る 
\\	そこ に すわって ください			
\\	私は窓側の席に座った。	
\\	座る 
\\	わたし は まどがわ の せき に すわった			
\\	右	
\\	みぎ			
\\	右のポケットにハンカチが入っています。	
\\	右 
\\	みぎ の ぽけっと に はんかち が はいって います			
\\	浴びる	
\\	あびる			
\\	私は朝、シャワーを浴びます。	
\\	私 
\\	朝 
\\	浴びる 
\\	わたし は あさ しゃわー を あびます			
\\	肩	
\\	かた			
\\	肩が凝りました。	
\\	肩 
\\	かた が こりました			
\\	飲む	
\\	のむ			
\\	友達とお酒を飲んでいます。	
\\	友達 
\\	酒 
\\	飲む 
\\	ともだち と おさけ を のんで います			
\\	寝る	
\\	ねる			
\\	もう寝よう。	
\\	寝る 
\\	もう ねよう			
\\	消す	
\\	けす			
\\	昼間は電気を消してください。	
\\	消す 
\\	ひるま は でんき を けして ください			
\\	元気	
\\	げんき			
\\	彼女はいつも元気だ。	
\\	いつも 
\\	元気 
\\	かのじょ は いつも げんき だ			
\\	おかげさまで元気です。	
\\	元気 
\\	おかげさま で げんき です			
\\	負ける	
\\	まける			
\\	私たちのチームはその試合で負けた。	
\\	負ける 
\\	わたしたち の ちーむ は その しあい で まけた			
\\	全部	
\\	ぜんぶ			
\\	それ、全部ください。	
\\	全部 
\\	それ ぜんぶ ください			
\\	去年	
\\	きょねん			
\\	私は去年フランスへ行った。	
\\	去年 
\\	わたし は きょねん ふらんす へ いった			
\\	引く	
\\	ひく			
\\	このドアは引いてください。	
\\	引く 
\\	この どあ は ひいて ください			
\\	図書館	
\\	としょかん			
\\	日曜日は図書館に行きます。	
\\	行く 
\\	図書館 
\\	にちようび は としょかん に いきます			
\\	図書館で料理の本を借りた。	
\\	図書館 
\\	としょかん で りょうり の ほん を かりた			
\\	上げる	
\\	あげる			
\\	彼は荷物をあみだなに上げた。	
\\	上げる 
\\	かれ は にもつ を あみだな に あげた			
\\	緑	
\\	みどり			
\\	この町には緑がたくさんあります。	
\\	緑 
\\	この まち に は みどり が たくさん あります			
\\	腕	
\\	うで			
\\	彼の腕は太い。	
\\	腕 
\\	かれ の うで は ふとい			
\\	ドア	
\\	ドアを開けてください。	
\\	ドア 
\\	どあ を あけて ください			
\end{CJK}
\end{document}