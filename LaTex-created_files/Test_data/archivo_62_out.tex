\documentclass[8pt]{extreport} 
\usepackage{hyperref}
\usepackage{CJKutf8}
\begin{document}
\begin{CJK}{UTF8}{min}
\\	「塗る(ぬる)」のアクセントは?1266	塗る 仮名 
\\	ぬる 漢字 
\\	塗る 発音図:	塗る ぬる		塗る ぬる 塗ります ぬります 塗って ぬって 塗った ぬった 塗らない ぬらない 塗らなかった ぬらなかった 塗れば ぬれば 塗らせる ぬらせる 塗られる ぬられる 塗れ ぬれ 塗れる ぬれる 塗ろう ぬろう				動詞-一般 
\\	「一気(いっき)」のアクセントは?1267		一気 いっき		一気 いっき				名詞-普通名詞-一般 
\\	「寄せる(よせる)」のアクセントは?1268	寄せる 仮名 
\\	よせる 漢字 
\\	寄せる 発音図:	寄せる よせる		寄せる よせる 寄せます よせます 寄せて よせて 寄せた よせた 寄せない よせない 寄せなかった よせなかった 寄せれば よせれば 寄せさせる よせさせる 寄せられる よせられる 寄せろ よせろ 寄せられる よせられる 寄せよう よせよう				動詞-一般 
\\	「意識(いしき)」のアクセントは?1269	意識 仮名 
\\	いしき 漢字 
\\	意識 発音図: 助詞: オ	意識 いしき		意識 いしき 意識する いしきする 意識します いしきします 意識して いしきして 意識した いしきした 意識しない いしきしない 意識しなかった いしきしなかった 意識すれば いしきすれば 意識させる いしきさせる 意識される いしきされる 意識しろ いしきしろ 意識できる いしきできる 意識しよう いしきしよう				動詞-一般 
\\	「完成(かんせい)」のアクセントは?1270	完成 仮名 
\\	かんせい 漢字 
\\	完成 発音図: 助詞: オ	完成 かんせい		完成 かんせい 完成する かんせいする 完成します かんせいします 完成して かんせいして 完成した かんせいした 完成しない かんせいしない 完成しなかった かんせいしなかった 完成すれば かんせいすれば 完成させる かんせいさせる 完成される かんせいされる 完成しろ かんせいしろ 完成できる かんせいできる 完成しよう かんせいしよう				動詞-一般 
\\	「高校(こうこう)」のアクセントは?1271	高校 仮名 
\\	こうこう、こーこー 漢字 
\\	高校 発音図: 助詞: オ	高校 こうこう		高校 こうこう				名詞-普通名詞-一般 
\\	「等(とう)」のアクセントは?1272	等 仮名 
\\	とう、とー 漢字 
\\	等 発音図:							接尾辞-名詞的-一般 
\\	「襲う(おそう)」のアクセントは?1273	襲う 仮名 
\\	おそう 漢字 
\\	襲う 発音図: 発音図:	襲う おそう		襲う おそう 襲います おそいます 襲って おそって 襲った おそった 襲わない おそわない 襲わなかった おそわなかった 襲えば おそえば 襲わせる おそわせる 襲われる おそわれる 襲え おそえ 襲える おそえる 襲おう おそおう				動詞-一般 
\\	「自動車(じどうしゃ)」のアクセントは?1274	自動車 仮名 
\\	じどうしゃ、じどーしゃ 漢字 
\\	自動車 発音図: 助詞: オ 発音図: 助詞: オ	自動車 じどうしゃ じどうしゃ		自動車 じどうしゃ じどうしゃ				名詞-普通名詞-一般 
\\	「同じく(おなじく)」のアクセントは?1275	同じく 仮名 
\\	おなじく 漢字 
\\	同じく 発音図:							副詞 
\\	「宜しい(よろしい)」のアクセントは?1276		宜しい よろしい		宜しい よろしい 宜しいです よろしいです 宜しくて よろしくて よろしくて 宜しかった よろしかった よろしかった 宜しくない よろしくない よろしくない 宜しくなかった よろしくなかった よろしくなかった 宜しければ よろしければ よろしければ 宜しい よろしい 宜しく よろしく よろしく				形容詞-一般 
\\	「雲(くも)」のアクセントは?1277	雲 仮名 
\\	くも 漢字 
\\	雲 発音図: 助詞: オ	雲 くも		雲 くも				名詞-普通名詞-一般 
\\	「前項(ぜんこう)」のアクセントは?1278	前項 仮名 
\\	ぜんこう、ぜんこー 漢字 
\\	前項 発音図: 助詞: オ 発音図: 助詞: オ							名詞-普通名詞-一般 
\\	「30日(さんじゅーにち)」のアクセントは?1279	30日 仮名 
\\	さんじゅーにち、さんじゅうにち 漢字 
\\	三十日、30日 発音図: 助詞: オ 発音図: 助詞: オ							名詞-数詞 
\\	「我(われ)」のアクセントは?1280	われ{我} 仮名 
\\	われ 漢字 
\\	我 発音図: 助詞: オ	我 われ		我 われ				代名詞 
\\	「増やす(ふやす)」のアクセントは?1281	増やす 仮名 
\\	ふやす 漢字 
\\	増やす 発音図:	増やす ふやす		増やす ふやす 増やします ふやします 増やして ふやして 増やした ふやした 増やさない ふやさない 増やさなかった ふやさなかった 増やせば ふやせば 増やさせる ふやさせる 増やされる ふやされる 増やせ ふやせ 増やせる ふやせる 増やそう ふやそう				動詞-一般 
\\	「生じる(しょうじる)」のアクセントは?1282	生じる 仮名 
\\	しょうじる、しょーじる 漢字 
\\	生じる 発音図: 発音図:	生じる しょうじる		生じる しょうじる 生じます しょうじます 生じて しょうじて 生じた しょうじた 生じない しょうじない 生じなかった しょうじなかった 生じれば しょうじれば 生じさせる しょうじさせる 生じられる しょうじられる 生じろ しょうじろ 生じよう しょうじよう				動詞-一般 
\\	「締める(しめる)」のアクセントは?1283	締める 仮名 
\\	しめる 漢字 
\\	締める 発音図:	締める しめる		締める しめる 締めます しめます 締めて しめて 締めた しめた 締めない しめない 締めなかった しめなかった 締めれば しめれば 締めさせる しめさせる 締められる しめられる 締めろ しめろ 締められる しめられる 締めよう しめよう				動詞-一般 
\\	「サイズ(さいず)」のアクセントは?1284	サイズ 仮名 
\\	サイズ、さいず 漢字 
\\	発音図: 助詞: オ	サイズ さいず		サイズ さいず				名詞-普通名詞-一般 
\\	「たった(たった)《〜それだけ》」のアクセントは?1285	たった(〜それだけ) 仮名 
\\	たった 漢字 
\\	発音図: 例文: ソれだけ							副詞 
\\	「取り引き(とりひき)」のアクセントは?1286	取り引き 仮名 
\\	とりひき 漢字 
\\	取り引き 発音図: 助詞: オ							名詞-普通名詞-一般 
\\	「植物(しょくぶつ)」のアクセントは?1287	植物 仮名 
\\	しょくぶつ 漢字 
\\	植物 発音図: 助詞: オ	植物 しょくぶつ		植物 しょくぶつ				名詞-普通名詞-一般 
\\	「心配な(しんぱいな)」のアクセントは?1288		心配な しんぱいな		心配な しんぱいな 心配です しんぱいです 心配で しんぱいで 心配だった しんぱいだった 心配じゃない しんぱいじゃない 心配じゃなかった しんぱいじゃなかった 心配ならば しんぱいならば 心配だ しんぱいだ 心配に しんぱいに				形状詞-一般 
\\	「無理な(むりな)」のアクセントは?1289		無理な むりな		無理な むりな 無理です むりです 無理で むりで 無理だった むりだった 無理じゃない むりじゃない 無理じゃなかった むりじゃなかった 無理ならば むりならば 無理だ むりだ 無理に むりに				名詞-普通名詞-一般 
\\	「電車(でんしゃ)」のアクセントは?1290	電車 仮名 
\\	でんしゃ 漢字 
\\	電車 発音図: 助詞: オ 発音図: 助詞: オ	電車 でんしゃ		電車 でんしゃ				名詞-普通名詞-一般 
\\	「任せる(まかせる)」のアクセントは?1291	任せる 仮名 
\\	まかせる 漢字 
\\	任せる 発音図:	任せる まかせる		任せる まかせる 任せます まかせます 任せて まかせて 任せた まかせた 任せない まかせない 任せなかった まかせなかった 任せれば まかせれば 任せさせる まかせさせる 任せられる まかせられる 任せろ まかせろ 任せられる まかせられる 任せよう まかせよう				動詞-一般 
\\	「単なる(たんなる)」のアクセントは?1292	単なる 仮名 
\\	たんなる 漢字 
\\	単なる 発音図:							連体詞 
\\	「靴(くつ)」のアクセントは?1293	靴 仮名 
\\	くつ 漢字 
\\	靴 発音図: 助詞: オ	靴 くつ		靴 くつ				名詞-普通名詞-一般 
\\	「国内(こくない)」のアクセントは?1294	国内 仮名 
\\	こくない 漢字 
\\	国内 発音図: 助詞: オ	国内 こくない		国内 こくない				名詞-普通名詞-一般 
\\	「性質(せいしつ)」のアクセントは?1295	性質 仮名 
\\	せいしつ 漢字 
\\	性質 発音図: 助詞: オ	性質 せいしつ		性質 せいしつ				名詞-普通名詞-一般 
\\	「恋(こい)」のアクセントは?1296	恋 仮名 
\\	こい 漢字 
\\	恋 発音図: 助詞: オ	恋 こい		恋 こい				名詞-普通名詞-一般 
\\	「覆う(おおう)」のアクセントは?1297	覆う 仮名 
\\	おーう、おおう 漢字 
\\	覆う 発音図: 発音図:	覆う おおう		覆う おおう 覆います おおいます 覆って おおって 覆った おおった 覆わない おおわない 覆わなかった おおわなかった 覆えば おおえば 覆わせる おおわせる 覆われる おおわれる 覆え おおえ 覆える おおえる 覆おう おおおう				動詞-一般 
\\	「止む(やむ)」のアクセントは?1298	やむ{止} 仮名 
\\	やむ 漢字 
\\	止む 発音図:	止む やむ		止む やむ 止みます やみます 止んで やんで 止んだ やんだ 止まない やまない 止まなかった やまなかった 止めば やめば 止ませる やませる 止まれる やまれる 止め やめ 止もう やもう				動詞-一般 
\\	「濃い(こい)」のアクセントは?1299	濃い 仮名 
\\	こい 漢字 
\\	濃い 発音図:	濃い こい		濃い こい 濃いです こいです 濃くて こくて 濃かった こかった 濃くない こくない 濃くなかった こくなかった 濃ければ こければ 濃い こい 濃く こく				形容詞-一般 
\\	「恐ろしい(おそろしい)」のアクセントは?1300	恐ろしい 仮名 
\\	おそろしい 漢字 
\\	恐ろしい 発音図:	恐ろしい おそろしい		恐ろしい おそろしい 恐ろしいです おそろしいです 恐ろしくて おそろしくて おそろしくて 恐ろしかった おそろしかった おそろしかった 恐ろしくない おそろしくない おそろしくない 恐ろしくなかった おそろしくなかった おそろしくなかった 恐ろしければ おそろしければ おそろしければ 恐ろしい おそろしい 恐ろしく おそろしく おそろしく				形容詞-一般 
\\	「熱(ねつ)」のアクセントは?1301	熱 仮名 
\\	ねつ 漢字 
\\	熱 発音図: 助詞: オ	熱 ねつ		熱 ねつ				名詞-普通名詞-一般 
\\	「悲しい(かなしい)」のアクセントは?1302	悲しい 仮名 
\\	かなしい 漢字 
\\	悲しい 発音図:	悲しい かなしい		悲しい かなしい 悲しいです かなしいです 悲しくて かなしくて 悲しかった かなしかった 悲しくない かなしくない 悲しくなかった かなしくなかった 悲しければ かなしければ 悲しい かなしい かなしい 悲しく かなしく				形容詞-一般 
\\	「男の子(おとこのこ)」のアクセントは?1303	男の子 仮名 
\\	おとこのこ 漢字 
\\	男の子 発音図: 助詞: オ	男の子 おとこのこ		男の子 おとこのこ				名詞-普通名詞-一般 
\\	「入力(にゅうりょく)」のアクセントは?1304	入力 仮名 
\\	にゅーりょく、にゅうりょく 漢字 
\\	入力 発音図: 助詞: オ 発音図: 助詞: オ	入力 にゅうりょく		入力 にゅうりょく 入力する にゅうりょくする 入力します にゅうりょくします 入力して にゅうりょくして 入力した にゅうりょくした 入力しない にゅうりょくしない 入力しなかった にゅうりょくしなかった 入力すれば にゅうりょくすれば 入力させる にゅうりょくさせる 入力される にゅうりょくされる 入力しろ にゅうりょくしろ 入力できる にゅうりょくできる 入力しよう にゅうりょくしよう				動詞-一般 
\\	「トイレ(といれ)」のアクセントは?1305	トイレ 仮名 
\\	トイレ、といれ 漢字 
\\	発音図: 助詞: オ	トイレ といれ		トイレ といれ				名詞-普通名詞-一般 
\\	「整える(ととのえる)」のアクセントは?1306	整える 仮名 
\\	ととのえる 漢字 
\\	整える 発音図: 発音図:	整える ととのえる ととのえる		整える ととのえる ととのえる 整えます ととのえます 整えて ととのえて 整えた ととのえた 整えない ととのえない 整えなかった ととのえなかった 整えれば ととのえれば ととのえれば 整えさせる ととのえさせる 整えられる ととのえられる 整えろ ととのえろ ととのえろ 整えられる ととのえられる 整えよう ととのえよう				動詞-一般 
\\	「留まる(とどまる)」のアクセントは?1307	とどまる{止;留} 仮名 
\\	とどまる 漢字 
\\	止まる、留まる 発音図:	留まる とどまる		留まる とどまる 留まります とどまります 留まって とどまって 留まった とどまった 留まらない とどまらない 留まらなかった とどまらなかった 留まれば とどまれば 留まらせる とどまらせる 留まられる とどまられる 留まれ とどまれ 留まれる とどまれる 留まろう とどまろう				動詞-一般 
\\	「概念(がいねん)」のアクセントは?1308	概念 仮名 
\\	がいねん 漢字 
\\	概念 発音図: 助詞: オ	概念 がいねん		概念 がいねん				名詞-普通名詞-一般 
\\	「強調(きょうちょう)」のアクセントは?1309	強調 仮名 
\\	きょうちょう、きょーちょー 漢字 
\\	強調 発音図: 助詞: オ	強調 きょうちょう		強調 きょうちょう 強調する きょうちょうする 強調します きょうちょうします 強調して きょうちょうして 強調した きょうちょうした 強調しない きょうちょうしない 強調しなかった きょうちょうしなかった 強調すれば きょうちょうすれば 強調させる きょうちょうさせる 強調される きょうちょうされる 強調しろ きょうちょうしろ 強調できる きょうちょうできる 強調しよう きょうちょうしよう				動詞-一般 
\\	「北海道(ほっかいどう)」のアクセントは?1310	北海道 仮名 
\\	ほっかいどう、ほっかいどー 漢字 
\\	北海道 発音図: 助詞: オ	北海道 ほっかいどう		北海道 ほっかいどう				名詞-固有名詞-地名-一般 
\\	「巨大な(きょだいな)」のアクセントは?1311	巨大 仮名 
\\	きょだい 漢字 
\\	巨大 発音図: 助詞: オ	巨大な きょだいな		巨大な きょだいな 巨大です きょだいです 巨大で きょだいで 巨大だった きょだいだった 巨大じゃない きょだいじゃない 巨大じゃなかった きょだいじゃなかった 巨大ならば きょだいならば 巨大だ きょだいだ 巨大に きょだいに				形状詞-一般 
\\	「間違い(まちがい)」のアクセントは?1312		間違い まちがい		間違い まちがい				名詞-普通名詞-一般 
\\	「複数(ふくすう)」のアクセントは?1313	複数 仮名 
\\	ふくすー、ふくすう 漢字 
\\	複数 発音図: 助詞: オ	複数 ふくすう		複数 ふくすう				名詞-普通名詞-一般 
\\	「昔(むかし)」のアクセントは?1314	昔 仮名 
\\	むかし 漢字 
\\	昔 発音図: 助詞: オ	昔 むかし		昔 むかし				副詞 
\\	「分かれる(わかれる)」のアクセントは?1315	分かれる 仮名 
\\	わかれる 漢字 
\\	分かれる 発音図:	分かれる わかれる		分かれる わかれる 分かれます わかれます 分かれて わかれて 分かれた わかれた 分かれない わかれない 分かれなかった わかれなかった 分かれれば わかれれば 分かれさせる わかれさせる 分かれられる わかれられる 分かれろ わかれろ 分かれられる わかれられる 分かれよう わかれよう				動詞-一般 
\\	「訪ねる(たずねる)」のアクセントは?1316	訪ねる 仮名 
\\	たずねる 漢字 
\\	訪ねる 発音図:	訪ねる たずねる		訪ねる たずねる 訪ねます たずねます 訪ねて たずねて 訪ねた たずねた 訪ねない たずねない 訪ねなかった たずねなかった 訪ねれば たずねれば 訪ねさせる たずねさせる 訪ねられる たずねられる 訪ねろ たずねろ 訪ねられる たずねられる 訪ねよう たずねよう				動詞-一般 
\\	「入り口(いりくち)」のアクセントは?1317	入り口 仮名 
\\	いりくち、いりぐち、はいりぐち 漢字 
\\	入り口 発音図: 助詞: オ 発音図: 助詞: オ 発音図: 助詞: オ							名詞-普通名詞-一般 
\\	「描く・掻く(かく)」のアクセントは?1318	かく{描;斯;掻;舁} 仮名 
\\	かく 漢字 
\\	描く、掻く、舁く、斯 発音図: 例文: ナるうえはやむをえまい	掻く かく		掻く かく 掻きます かきます 掻いて かいて 掻いた かいた 掻かない かかない 掻かなかった かかなかった 掻けば かけば 掻かせる かかせる 掻かれる かかれる 掻け かけ 掻ける かける 掻こう かこう				動詞-一般 
\\	「実行(じっこう)」のアクセントは?1319	実行 仮名 
\\	じっこう、じっこー 漢字 
\\	実行 発音図: 助詞: オ	実行 じっこう		実行 じっこう 実行する じっこうする 実行します じっこうします 実行して じっこうして 実行した じっこうした 実行しない じっこうしない 実行しなかった じっこうしなかった 実行すれば じっこうすれば 実行させる じっこうさせる 実行される じっこうされる 実行しろ じっこうしろ 実行できる じっこうできる 実行しよう じっこうしよう				動詞-一般 
\\	「青い(あおい)」のアクセントは?1320	青い 仮名 
\\	あおい 漢字 
\\	青い 発音図:	青い あおい		青い あおい 青いです あおいです 青くて あおくて あおくて 青かった あおかった あおかった 青くない あおくない あおくない 青くなかった あおくなかった あおくなかった 青ければ あおければ あおければ 青い あおい 青く あおく あおく				形容詞-一般 
\\	「揃う(そろう)」のアクセントは?1321	そろう{揃} 仮名 
\\	そろう 漢字 
\\	揃う 発音図:	揃う そろう		揃う そろう 揃います そろいます 揃って そろって 揃った そろった 揃わない そろわない 揃わなかった そろわなかった 揃えば そろえば 揃わせる そろわせる 揃われる そろわれる 揃え そろえ 揃おう そろおう				動詞-一般 
\\	「地位(ちい)」のアクセントは?1322	地位 仮名 
\\	ちい 漢字 
\\	地位 発音図: 助詞: オ	地位 ちい		地位 ちい				名詞-普通名詞-一般 
\\	「鳥(とり)」のアクセントは?1323	鳥 仮名 
\\	とり 漢字 
\\	鳥 発音図: 例文: 森でガなく	鳥 とり		鳥 とり				名詞-普通名詞-一般 
\\	「皮(かわ)」のアクセントは?1324	皮 仮名 
\\	かわ 漢字 
\\	皮 発音図: 助詞: オ	皮 かわ		皮 かわ				名詞-普通名詞-一般 
\\	「恐れる(おそれる)」のアクセントは?1325	恐れる 仮名 
\\	おそれる 漢字 
\\	恐れる 発音図:	恐れる おそれる		恐れる おそれる 恐れます おそれます 恐れて おそれて 恐れた おそれた 恐れない おそれない 恐れなかった おそれなかった 恐れれば おそれれば 恐れさせる おそれさせる 恐れられる おそれられる 恐れろ おそれろ 恐れよう おそれよう				動詞-一般 
\\	「幅(はば)」のアクセントは?1326	幅 仮名 
\\	はば、ふく 漢字 
\\	幅 発音図: 助詞: オ 例文: この床の間には書のガよい 発音図:	幅 はば		幅 はば				名詞-普通名詞-一般 
\\	「招く(まねく)」のアクセントは?1327	招く 仮名 
\\	まねく 漢字 
\\	招く 発音図:	招く まねく		招く まねく 招きます まねきます 招いて まねいて 招いた まねいた 招かない まねかない 招かなかった まねかなかった 招けば まねけば 招かせる まねかせる 招かれる まねかれる 招け まねけ 招ける まねける 招こう まねこう				動詞-一般 
\\	「ストレス(すとれす)」のアクセントは?1328	ストレス 仮名 
\\	ストレス、すとれす 漢字 
\\	発音図: 助詞: オ	ストレス すとれす		ストレス すとれす				名詞-普通名詞-一般 
\\	「被害者(ひがいしゃ)」のアクセントは?1329	被害者 仮名 
\\	ひがいしゃ 漢字 
\\	被害者 発音図: 助詞: オ							名詞-普通名詞-一般 
\\	「カメラ(かめら)」のアクセントは?1330	カメラ 仮名 
\\	カメラ、かめら 漢字 
\\	発音図: 助詞: オ	カメラ かめら		カメラ かめら				名詞-普通名詞-一般 
\\	「予め(あらかじめ)」のアクセントは?1331	あらかじめ{予} 仮名 
\\	あらかじめ 漢字 
\\	予め 発音図:							副詞 
\\	「仮(かり)《〜に》」のアクセントは?1332	仮(〜に) 仮名 
\\	かり 漢字 
\\	仮 発音図: 助詞: ニ							名詞-普通名詞-一般 
\\	「パリ(ぱり)」のアクセントは?1333		パリ ぱり		パリ ぱり				名詞-固有名詞-地名-一般 
\\	「十一時(じゅういちじ)」のアクセントは?1334		十一時 じゅういちじ		十一時 じゅういちじ				名詞-数詞 
\\	「チャンス(ちゃんす)」のアクセントは?1335	チャンス 仮名 
\\	チャンス、ちゃんす 漢字 
\\	発音図: 助詞: オ	チャンス ちゃんす		チャンス ちゃんす				名詞-普通名詞-一般 
\\	「8日(ようか)」のアクセントは?1336	8日 仮名 
\\	ようか、よーか 漢字 
\\	八日、8日 発音図: 助詞: オ	八日 ようか		八日 ようか				名詞-数詞 
\\	「財産(ざいさん)」のアクセントは?1337	財産 仮名 
\\	ざいさん 漢字 
\\	財産 発音図: 助詞: オ 発音図: 助詞: オ	財産 ざいさん		財産 ざいさん				名詞-普通名詞-一般 
\\	「発展(はってん)」のアクセントは?1338	発展 仮名 
\\	はってん 漢字 
\\	発展 発音図: 助詞: オ	発展 はってん		発展 はってん 発展する はってんする 発展します はってんします 発展して はってんして 発展した はってんした 発展しない はってんしない 発展しなかった はってんしなかった 発展すれば はってんすれば 発展させる はってんさせる 発展される はってんされる 発展しろ はってんしろ 発展できる はってんできる 発展しよう はってんしよう				動詞-一般 
\\	「予算(よさん)」のアクセントは?1339	予算 仮名 
\\	よさん 漢字 
\\	予算 発音図: 助詞: オ 発音図: 助詞: オ	予算 よさん よさん		予算 よさん よさん				名詞-普通名詞-一般 
\\	「線(せん)」のアクセントは?1340	線 仮名 
\\	せん 漢字 
\\	線 発音図: 助詞: オ	線 せん		線 せん				名詞-普通名詞-一般 
\\	「無し(なし)」のアクセントは?1341	無し 仮名 
\\	なし 漢字 
\\	無し 発音図: 例文: 全くモ同然だ	無し なし		無し なし				名詞-普通名詞-一般 
\\	「医者(いしゃ)」のアクセントは?1342	医者 仮名 
\\	いしゃ 漢字 
\\	医者 発音図: 助詞: オ	医者 いしゃ		医者 いしゃ				名詞-普通名詞-一般 
\\	「右手(みぎて)」のアクセントは?1343	右手 仮名 
\\	みぎて 漢字 
\\	右手 発音図: 助詞: オ	右手 みぎて		右手 みぎて				名詞-普通名詞-一般 
\\	「口調(くちょう)」のアクセントは?1344	口調 仮名 
\\	くちょう、くちょー 漢字 
\\	口調 発音図: 助詞: オ							名詞-普通名詞-一般 
\\	「削除(さくじょ)」のアクセントは?1345	削除 仮名 
\\	さくじょ 漢字 
\\	削除 発音図: 助詞: オ	削除 さくじょ		削除 さくじょ 削除する さくじょする 削除します さくじょします 削除して さくじょして 削除した さくじょした 削除しない さくじょしない 削除しなかった さくじょしなかった 削除すれば さくじょすれば 削除させる さくじょさせる 削除される さくじょされる 削除しろ さくじょしろ 削除できる さくじょできる 削除しよう さくじょしよう				動詞-一般 
\\	「巡る(めぐる)」のアクセントは?1346	巡る 仮名 
\\	めぐる 漢字 
\\	巡る 発音図:	巡る めぐる		巡る めぐる 巡ります めぐります 巡って めぐって 巡った めぐった 巡らない めぐらない 巡らなかった めぐらなかった 巡れば めぐれば 巡らせる めぐらせる 巡られる めぐられる 巡れ めぐれ 巡れる めぐれる 巡ろう めぐろう				動詞-一般 
\\	「収入(しゅうにゅう)」のアクセントは?1347	収入 仮名 
\\	しゅーにゅー、しゅうにゅう 漢字 
\\	収入 発音図: 助詞: オ	収入 しゅうにゅう		収入 しゅうにゅう				名詞-普通名詞-一般 
\\	「二十(にじゅう)」のアクセントは?1348	
\\	仮名 
\\	にじゅー、にじゅう 漢字 
\\	二十、20 発音図: 助詞: オ	二十 にじゅう		二十 にじゅう				名詞-数詞 
\\	「暑い(あつい)《お〜》」のアクセントは?1349	暑い(お〜) 仮名 
\\	あつい 漢字 
\\	暑い 発音図:	暑い あつい		暑い あつい 暑いです あついです 暑くて あつくて あつくて 暑かった あつかった あつかった 暑くない あつくない あつくない 暑くなかった あつくなかった あつくなかった 暑ければ あつければ あつければ 暑い あつい 暑く あつく あつく				形容詞-一般 
\\	「売れる(うれる)」のアクセントは?1350	売れる 仮名 
\\	うれる 漢字 
\\	売れる 発音図:	売れる うれる		売れる うれる 売れます うれます 売れて うれて 売れた うれた 売れない うれない 売れなかった うれなかった 売れれば うれれば 売れさせる うれさせる 売れられる うれられる 売れろ うれろ 売れよう うれよう				動詞-一般 
\\	「そろそろ(そろそろ)《〜歩く》」のアクセントは?1351	そろそろ(〜歩く) 仮名 
\\	そろそろ 漢字 
\\	発音図: 例文: アるく							副詞 
\\	「開発(かいはつ)」のアクセントは?1352	開発 仮名 
\\	かいはつ 漢字 
\\	開発 発音図: 助詞: オ	開発 かいはつ		開発 かいはつ 開発する かいはつする 開発します かいはつします 開発して かいはつして 開発した かいはつした 開発しない かいはつしない 開発しなかった かいはつしなかった 開発すれば かいはつすれば 開発させる かいはつさせる 開発される かいはつされる 開発しろ かいはつしろ 開発できる かいはつできる 開発しよう かいはつしよう				動詞-一般 
\\	「旦那(だんな)」のアクセントは?1353	だんな{旦那;檀那} 仮名 
\\	だんな 漢字 
\\	旦那、檀那 発音図: 助詞: オ	旦那 だんな		旦那 だんな				名詞-普通名詞-一般 
\\	「伺う(うかがう)」のアクセントは?1354	伺う 仮名 
\\	うかがう 漢字 
\\	伺う 発音図:	伺う うかがう		伺う うかがう 伺います うかがいます 伺って うかがって 伺った うかがった 伺わない うかがわない 伺わなかった うかがわなかった 伺えば うかがえば 伺わせる うかがわせる 伺われる うかがわれる 伺え うかがえ 伺える うかがえる 伺おう うかがおう				動詞-一般 
\\	「傷(きず)」のアクセントは?1355	傷 仮名 
\\	きず 漢字 
\\	傷 発音図: 助詞: オ	傷 きず		傷 きず				名詞-普通名詞-一般 
\\	「近所(きんじょ)」のアクセントは?1356	近所 仮名 
\\	きんじょ 漢字 
\\	近所 発音図: 助詞: オ	近所 きんじょ		近所 きんじょ				名詞-普通名詞-一般 
\\	「字(じ)」のアクセントは?1357	字 仮名 
\\	あざ、じ 漢字 
\\	字 発音図: 助詞: オ 例文: 上手なオ書く 発音図:	字 じ		字 じ				名詞-普通名詞-一般 
\\	「コーヒー(こーひー)」のアクセントは?1358	コーヒー 仮名 
\\	コーヒー、こーひー 漢字 
\\	発音図: 助詞: オ	コーヒー こーひー		コーヒー こーひー				名詞-普通名詞-一般 
\\	「方針(ほうしん)」のアクセントは?1359	方針 仮名 
\\	ほうしん、ほーしん 漢字 
\\	方針 発音図: 助詞: オ	方針 ほうしん		方針 ほうしん				名詞-普通名詞-一般 
\\	「楽しめる(たのしめる)」のアクセントは?1360	楽しめる 仮名 
\\	たのしめる 漢字 
\\	楽しめる 発音図:							動詞-一般 
\\	「実家(じっか)」のアクセントは?1361	実家 仮名 
\\	じっか 漢字 
\\	実家 発音図: 助詞: オ	実家 じっか		実家 じっか				名詞-普通名詞-一般 
\\	「骨(ほね)」のアクセントは?1362	骨 仮名 
\\	ほね 漢字 
\\	骨 発音図: 助詞: オ	骨 ほね		骨 ほね				名詞-普通名詞-一般 
\\	「1週間(いっしゅうかん)」のアクセントは?1363	1週間 仮名 
\\	いっしゅーかん、いっしゅうかん 漢字 
\\	一週間、1週間 発音図: 助詞: オ	一週間 いっしゅうかん		一週間 いっしゅうかん				名詞-数詞 
\\	「癌(がん)」のアクセントは?1364	がん{雁;癌} 仮名 
\\	がん 漢字 
\\	癌、雁 発音図: 助詞: オ	癌 がん		癌 がん				名詞-普通名詞-一般 
\\	「大いに(おーいに)」のアクセントは?1365	大いに 仮名 
\\	おーいに、おおいに 漢字 
\\	大いに 発音図:							副詞 
\\	「言い方(いーかた)」のアクセントは?1366	言い方 仮名 
\\	いーかた、いいかた 漢字 
\\	言い方 発音図: 助詞: オ							名詞-普通名詞-一般 
\\	「規模(きぼ)」のアクセントは?1367	規模 仮名 
\\	きぼ 漢字 
\\	規模 発音図: 助詞: オ	規模 きぼ		規模 きぼ				名詞-普通名詞-一般 
\\	「必死(ひっし)」のアクセントは?1368	必死 仮名 
\\	ひっし 漢字 
\\	必死 発音図: 助詞: オ	必死 ひっし		必死 ひっし				形状詞-一般 
\\	「必死な(ひっしな)」のアクセントは?1369		必死な ひっしな		必死な ひっしな 必死です ひっしです 必死で ひっしで 必死だった ひっしだった 必死じゃない ひっしじゃない 必死じゃなかった ひっしじゃなかった 必死ならば ひっしならば 必死だ ひっしだ 必死に ひっしに				形状詞-一般 
\\	「光る(ひかる)」のアクセントは?1370	光る 仮名 
\\	ひかる 漢字 
\\	光る 発音図:	光る ひかる		光る ひかる 光ります ひかります 光って ひかって 光った ひかった 光らない ひからない 光らなかった ひからなかった 光れば ひかれば 光らせる ひからせる 光られる ひかられる 光れ ひかれ 光れる ひかれる 光ろう ひかろう				動詞-一般 
\\	「皆様(みなさま)」のアクセントは?1371	皆様 仮名 
\\	みなさま 漢字 
\\	皆様 発音図: 助詞: オ	皆様 みなさま		皆様 みなさま				名詞-普通名詞-一般 
\\	「保存(ほぞん)」のアクセントは?1372	保存 仮名 
\\	ほぞん 漢字 
\\	保存 発音図: 助詞: オ	保存 ほぞん		保存 ほぞん 保存する ほぞんする 保存します ほぞんします 保存して ほぞんして 保存した ほぞんした 保存しない ほぞんしない 保存しなかった ほぞんしなかった 保存すれば ほぞんすれば 保存させる ほぞんさせる 保存される ほぞんされる 保存しろ ほぞんしろ 保存できる ほぞんできる 保存しよう ほぞんしよう				動詞-一般 
\\	「番組(ばんぐみ)」のアクセントは?1373	番組 仮名 
\\	ばんぐみ 漢字 
\\	番組 発音図: 例文: 放送のオ録画する 発音図: 例文: 放送のオ録画する	番組 ばんぐみ		番組 ばんぐみ				名詞-普通名詞-一般 
\\	「要するに(ようするに)」のアクセントは?1374	要するに 仮名 
\\	ようするに、よーするに 漢字 
\\	要するに 発音図:							接続詞 
\\	「十一日(じゅういちにち)」のアクセントは?1375		十一日 じゅういちにち		十一日 じゅういちにち				名詞-数詞 
\\	「食べ物(たべもの)」のアクセントは?1376	食べ物 仮名 
\\	たべもの 漢字 
\\	食べ物 発音図: 助詞: オ 発音図: 助詞: オ	食べ物 たべもの たべもの		食べ物 たべもの たべもの				名詞-普通名詞-一般 
\\	「思想(しそう)」のアクセントは?1377	思想 仮名 
\\	しそう、しそー 漢字 
\\	思想 発音図: 助詞: オ	思想 しそう		思想 しそう				名詞-普通名詞-一般 
\\	「義務(ぎむ)」のアクセントは?1378	義務 仮名 
\\	ぎむ 漢字 
\\	義務 発音図: 助詞: オ	義務 ぎむ		義務 ぎむ				名詞-普通名詞-一般 
\\	「疑う(うたがう)」のアクセントは?1379	疑う 仮名 
\\	うたがう 漢字 
\\	疑う 発音図:	疑う うたがう		疑う うたがう 疑います うたがいます 疑って うたがって 疑った うたがった 疑わない うたがわない 疑わなかった うたがわなかった 疑えば うたがえば 疑わせる うたがわせる 疑われる うたがわれる 疑え うたがえ 疑える うたがえる 疑おう うたがおう				動詞-一般 
\\	「8時(はちじ)」のアクセントは?1380	8時 仮名 
\\	はちじ 漢字 
\\	八時、8時 発音図: 助詞: オ	八時 はちじ		八時 はちじ				名詞-数詞 
\\	「メニュー(めにゅー)」のアクセントは?1381	メニュー 仮名 
\\	メニュー、めにゅー 漢字 
\\	発音図: 助詞: オ	メニュー めにゅー		メニュー めにゅー				名詞-普通名詞-一般 
\\	「関連(かんれん)」のアクセントは?1382	関連 仮名 
\\	かんれん 漢字 
\\	関連 発音図: 助詞: オ	関連 かんれん		関連 かんれん 関連する かんれんする 関連します かんれんします 関連して かんれんして 関連した かんれんした 関連しない かんれんしない 関連しなかった かんれんしなかった 関連すれば かんれんすれば 関連させる かんれんさせる 関連される かんれんされる 関連しろ かんれんしろ 関連できる かんれんできる 関連しよう かんれんしよう				動詞-一般 
\\	「犯す(おかす)」のアクセントは?1383	犯す 仮名 
\\	おかす 漢字 
\\	犯す 発音図: 発音図:	犯す おかす おかす		犯す おかす おかす 犯します おかします 犯して おかして おかして 犯した おかした おかした 犯さない おかさない おかさない 犯さなかった おかさなかった おかさなかった 犯せば おかせば おかせば 犯させる おかさせる おかさせる 犯される おかされる おかされる 犯せ おかせ おかせ 犯せる おかせる おかせる 犯そう おかそう				動詞-一般 
\\	「世(よ)」のアクセントは?1384	世 仮名 
\\	よ 漢字 
\\	世 発音図: 例文: 正直すぎて、ニいれられない	世 よ		世 よ				名詞-普通名詞-一般 
\\	「政策(せいさく)」のアクセントは?1385	政策 仮名 
\\	せいさく 漢字 
\\	政策 発音図: 助詞: オ	政策 せいさく		政策 せいさく				名詞-普通名詞-一般 
\\	「波(なみ)」のアクセントは?1386	波 仮名 
\\	なみ 漢字 
\\	波 発音図: 例文: 高いガ打ちよせる	波 なみ		波 なみ				名詞-普通名詞-一般 
\\	「5時(ごじ)」のアクセントは?1387	5時 仮名 
\\	ごじ 漢字 
\\	五時、5時 発音図: 助詞: オ	五時 ごじ		五時 ごじ				名詞-数詞 
\\	「促す(うながす)」のアクセントは?1388	促す 仮名 
\\	うながす 漢字 
\\	促す 発音図: 発音図:	促す うながす		促す うながす 促します うながします 促して うながして 促した うながした 促さない うながさない 促さなかった うながさなかった 促せば うながせば 促させる うながさせる 促される うながされる 促せ うながせ 促せる うながせる 促そう うながそう				動詞-一般 
\\	「机(つくえ)」のアクセントは?1389	机 仮名 
\\	つくえ 漢字 
\\	机 発音図: 助詞: オ	机 つくえ		机 つくえ				名詞-普通名詞-一般 
\\	「把握(はあくする)」のアクセントは?1390	把握 仮名 
\\	はあく 漢字 
\\	把握 発音図: 助詞: オ	把握する はあくする		把握する はあくする 把握します はあくします 把握して はあくして 把握した はあくした 把握しない はあくしない 把握しなかった はあくしなかった 把握すれば はあくすれば 把握させる はあくさせる 把握される はあくされる 把握しろ はあくしろ 把握できる はあくできる 把握しよう はあくしよう				動詞-一般 
\\	「否定(ひてい)」のアクセントは?1391	否定 仮名 
\\	ひてい 漢字 
\\	否定 発音図: 助詞: オ	否定 ひてい		否定 ひてい 否定する ひていする 否定します ひていします 否定して ひていして 否定した ひていした 否定しない ひていしない 否定しなかった ひていしなかった 否定すれば ひていすれば 否定させる ひていさせる 否定される ひていされる 否定しろ ひていしろ 否定できる ひていできる 否定しよう ひていしよう				動詞-一般 
\\	「予想(よそう)」のアクセントは?1392	予想 仮名 
\\	よそう、よそー 漢字 
\\	予想 発音図: 助詞: オ	予想 よそう		予想 よそう 予想する よそうする 予想します よそうします 予想して よそうして 予想した よそうした 予想しない よそうしない 予想しなかった よそうしなかった 予想すれば よそうすれば 予想させる よそうさせる 予想される よそうされる 予想しろ よそうしろ 予想できる よそうできる 予想しよう よそうしよう				動詞-一般 
\\	「闇(やみ)」のアクセントは?1393	やみ{闇} 仮名 
\\	やみ 漢字 
\\	闇 発音図: 助詞: オ	闇 やみ		闇 やみ				名詞-普通名詞-一般 
\\	「響く(ひびく)」のアクセントは?1394	響く 仮名 
\\	ひびく 漢字 
\\	響く 発音図:	響く ひびく		響く ひびく 響きます ひびきます 響いて ひびいて 響いた ひびいた 響かない ひびかない 響かなかった ひびかなかった 響けば ひびけば 響かせる ひびかせる 響かれる ひびかれる 響け ひびけ 響こう ひびこう				動詞-一般 
\\	「表面(ひょうめん)」のアクセントは?1395	表面 仮名 
\\	ひょうめん、ひょーめん 漢字 
\\	表面 発音図: 助詞: オ	表面 ひょうめん		表面 ひょうめん				名詞-普通名詞-一般 
\\	「二回(にかい)」のアクセントは?1396		二回 にかい		二回 にかい				名詞-数詞 
\\	「運命(うんめい)」のアクセントは?1397	運命 仮名 
\\	うんめい 漢字 
\\	運命 発音図: 助詞: オ	運命 うんめい		運命 うんめい				名詞-普通名詞-一般 
\\	「基礎(きそ)」のアクセントは?1398	基礎 仮名 
\\	きそ 漢字 
\\	基礎 発音図: 助詞: オ 発音図: 助詞: オ	基礎 きそ きそ		基礎 きそ きそ				名詞-普通名詞-一般 
\\	「皆(みな)《〜集まる》」のアクセントは?1399	皆(〜集まる) 仮名 
\\	みな、みんな 漢字 
\\	皆 発音図: 例文: 行く末についてシん配している 発音図: 例文: 定刻前にガ集まりました							名詞-普通名詞-一般 
\\	「皆(みな)《〜が集まる》」のアクセントは?1400	皆(〜が集まる) 仮名 
\\	みな、みんな 漢字 
\\	皆 発音図: 例文: しばらくガそろうまで待ってくれ 発音図: 例文: それはガ思っていることです							名詞-普通名詞-一般 
\\	「乃至(ないし)」のアクセントは?1401	ないし{乃至} 仮名 
\\	ないし 漢字 
\\	乃至 発音図:							接続詞 
\\	「ファン(ふぁん)」のアクセントは?1402	ファン 仮名 
\\	ファン、ふぁん 漢字 
\\	発音図: 助詞: オ	ファン ふぁん		ファン ふぁん				名詞-普通名詞-一般 
\\	「扉(とびら)」のアクセントは?1403	扉 仮名 
\\	とびら 漢字 
\\	扉 発音図: 助詞: オ	扉 とびら		扉 とびら				名詞-普通名詞-一般 
\\	「苦しい(くるしい)」のアクセントは?1404	苦しい 仮名 
\\	くるしい 漢字 
\\	苦しい 発音図:	苦しい くるしい		苦しい くるしい 苦しいです くるしいです 苦しくて くるしくて くるしくて 苦しかった くるしかった くるしかった 苦しくない くるしくない くるしくない 苦しくなかった くるしくなかった くるしくなかった 苦しければ くるしければ くるしければ 苦しい くるしい 苦しく くるしく くるしく				形容詞-一般 
\\	「資格(しかく)」のアクセントは?1405	資格 仮名 
\\	しかく 漢字 
\\	資格 発音図: 例文: 運転するにはガ必要です	資格 しかく		資格 しかく				名詞-普通名詞-一般 
\\	「玄関(げんかん)」のアクセントは?1406	玄関 仮名 
\\	げんかん 漢字 
\\	玄関 発音図: 助詞: オ	玄関 げんかん		玄関 げんかん				名詞-普通名詞-一般 
\\	「成長(せいちょう)」のアクセントは?1407	成長 仮名 
\\	せいちょう、せいちょー 漢字 
\\	成長 発音図: 助詞: オ	成長 せいちょう		成長 せいちょう 成長する せいちょうする 成長します せいちょうします 成長して せいちょうして 成長した せいちょうした 成長しない せいちょうしない 成長しなかった せいちょうしなかった 成長すれば せいちょうすれば 成長させる せいちょうさせる 成長される せいちょうされる 成長しろ せいちょうしろ 成長できる せいちょうできる 成長しよう せいちょうしよう				動詞-一般 
\\	「履く(はく)」のアクセントは?1408	履く 仮名 
\\	はく 漢字 
\\	履く 発音図: 例文: 新しい靴を初めてウれしさ	履く はく		履く はく 履きます はきます 履いて はいて 履いた はいた 履かない はかない 履かなかった はかなかった 履けば はけば 履かせる はかせる 履かれる はかれる 履け はけ 履ける はける 履こう はこう				動詞-一般 
\\	「ビール(びーる)」のアクセントは?1409	ビール 仮名 
\\	ビール、びーる 漢字 
\\	発音図: 助詞: オ	ビール びーる		ビール びーる				名詞-普通名詞-一般 
\\	「上手な(じょうずな)《お〜》」のアクセントは?1410	上手(お〜) 仮名 
\\	じょうず、じょーず 漢字 
\\	上手 発音図: 助詞: オ	上手な じょうずな		上手な じょうずな 上手です じょうずです 上手で じょうずで 上手だった じょうずだった 上手じゃない じょうずじゃない 上手じゃなかった じょうずじゃなかった 上手ならば じょうずならば 上手だ じょうずだ 上手に じょうずに				形状詞-一般 
\\	「喉(のど)」のアクセントは?1411		喉 のど		喉 のど				名詞-普通名詞-一般 
\\	「上回る(うわまわる)」のアクセントは?1412	上回る 仮名 
\\	うわまわる 漢字 
\\	上回る 発音図: 発音図:	上回る うわまわる		上回る うわまわる 上回ります うわまわります 上回って うわまわって 上回った うわまわった 上回らない うわまわらない 上回らなかった うわまわらなかった 上回れば うわまわれば 上回らせる うわまわらせる 上回られる うわまわられる 上回れ うわまわれ 上回れる うわまわれる 上回ろう うわまわろう				動詞-一般 
\\	「空く(あく)」のアクセントは?1414	空く 仮名 
\\	あく 漢字 
\\	空く 発音図:	空く あく		空く あく 空きます あきます 空いて あいて 空いた あいた 空かない あかない 空かなかった あかなかった 空けば あけば 空かせる あかせる 空かされる あかされる 空け あけ 空こう あこう				動詞-一般 
\\	「事例(じれい)」のアクセントは?1415	事例 仮名 
\\	じれい 漢字 
\\	事例 発音図: 助詞: オ							名詞-普通名詞-一般 
\\	「プログラム(ぷろぐらむ)」のアクセントは?1416	プログラム 仮名 
\\	プログラム、ぷろぐらむ 漢字 
\\	発音図: 助詞: オ	プログラム ぷろぐらむ		プログラム ぷろぐらむ				名詞-普通名詞-一般 
\\	「個人的な(こじんてきな)」のアクセントは?1417	個人的 仮名 
\\	こじんてき 漢字 
\\	個人的 発音図:	個人的な こじんてきな		個人的な こじんてきな 個人的です こじんてきです 個人的で こじんてきで 個人的だった こじんてきだった 個人的じゃない こじんてきじゃない 個人的じゃなかった こじんてきじゃなかった 個人的ならば こじんてきならば 個人的だ こじんてきだ 個人的に こじんてきに				形状詞-一般 
\\	「講ずる(こうずる)」のアクセントは?1418	講ずる 仮名 
\\	こうずる、こーずる 漢字 
\\	講ずる 発音図: 発音図:							動詞-一般 
\\	「下る(くだる)」のアクセントは?1419	下る 仮名 
\\	くだる 漢字 
\\	下る 発音図:	下る くだる		下る くだる 下ります くだります 下って くだって 下った くだった 下らない くだらない 下らなかった くだらなかった 下れば くだれば 下らせる くだらせる 下られる くだられる 下れ くだれ 下れる くだれる 下ろう くだろう				動詞-一般 
\\	「メーカー(めーかー)」のアクセントは?1420	メーカー 仮名 
\\	メーカー、めーかー 漢字 
\\	発音図: 助詞: オ 発音図: 助詞: オ	メーカー めーかー めーかー		メーカー めーかー めーかー				名詞-普通名詞-一般 
\\	「6日(むいか)」のアクセントは?1421	6日 仮名 
\\	むいか 漢字 
\\	六日、6日 発音図: 助詞: オ	六日 むいか		六日 むいか				名詞-数詞 
\\	「工場(こうじょう)」のアクセントは?1422	工場 仮名 
\\	こうじょう、こうば、こーじょー、こーば 漢字 
\\	工場 発音図: 助詞: オ 発音図: 助詞: オ	工場 こうじょう		工場 こうじょう				名詞-普通名詞-一般 
\\	「急速な(きゅうそくな)」のアクセントは?1423		急速な きゅうそくな		急速な きゅうそくな 急速です きゅうそくです 急速で きゅうそくで 急速だった きゅうそくだった 急速じゃない きゅうそくじゃない 急速じゃなかった きゅうそくじゃなかった 急速ならば きゅうそくならば 急速だ きゅうそくだ 急速に きゅうそくに				形状詞-一般 
\\	「急速(きゅうそく)」のアクセントは?1424	急速 仮名 
\\	きゅーそく、きゅうそく 漢字 
\\	急速 発音図: 助詞: オ	急速 きゅうそく		急速 きゅうそく				形状詞-一般 
\\	「9日(ここのか)《副》」のアクセントは?1425	9日(副) 仮名 
\\	ここのか 漢字 
\\	九日、9日 発音図: 例文: あとデ期限が来る	九日 ここのか		九日 ここのか				名詞-数詞 
\\	「9日(ここのか)《名》」のアクセントは?1426	9日(名) 仮名 
\\	ここのか 漢字 
\\	九日、9日 発音図: 例文: あとデ期限が来る	九日 ここのか		九日 ここのか				名詞-数詞 
\\	「恥ずかしい(はずかしい)」のアクセントは?1427	恥ずかしい 仮名 
\\	はずかしい 漢字 
\\	恥ずかしい 発音図:	恥ずかしい はずかしい		恥ずかしい はずかしい 恥ずかしいです はずかしいです 恥ずかしくて はずかしくて はずかしくて 恥ずかしかった はずかしかった はずかしかった 恥ずかしくない はずかしくない はずかしくない 恥ずかしくなかった はずかしくなかった はずかしくなかった 恥ずかしければ はずかしければ はずかしければ 恥ずかしい はずかしい 恥ずかしく はずかしく はずかしく				形容詞-一般 
\\	「本書(ほんしょ)」のアクセントは?1428	本書 仮名 
\\	ほんしょ 漢字 
\\	本書 発音図: 助詞: オ							名詞-普通名詞-一般 
\\	「件(けん)」のアクセントは?1429	件 仮名 
\\	けん 漢字 
\\	件 発音図: 助詞: オ	件 けん		件 けん				名詞-普通名詞-一般 
\\	「犯罪(はんざい)」のアクセントは?1430	犯罪 仮名 
\\	はんざい 漢字 
\\	犯罪 発音図: 助詞: オ	犯罪 はんざい		犯罪 はんざい				名詞-普通名詞-一般 
\\	「発揮(はっきする)」のアクセントは?1431	発揮 仮名 
\\	はっき 漢字 
\\	発揮 発音図: 助詞: オ 発音図: 助詞: オ	発揮する はっきする		発揮する はっきする 発揮します はっきします 発揮して はっきして 発揮した はっきした 発揮しない はっきしない 発揮しなかった はっきしなかった 発揮すれば はっきすれば 発揮させる はっきさせる 発揮される はっきされる 発揮しろ はっきしろ 発揮できる はっきできる 発揮しよう はっきしよう				動詞-一般 
\\	「伸び(のび)」のアクセントは?1432	伸び 仮名 
\\	のび 漢字 
\\	伸び 発音図: 助詞: オ							名詞-普通名詞-一般 
\\	「相応しい(ふさわしい)」のアクセントは?1433		相応しい ふさわしい		相応しい ふさわしい 相応しいです ふさわしいです 相応しくて ふさわしくて ふさわしくて 相応しかった ふさわしかった ふさわしかった 相応しくない ふさわしくない ふさわしくない 相応しくなかった ふさわしくなかった ふさわしくなかった 相応しければ ふさわしければ ふさわしければ 相応しい ふさわしい 相応しく ふさわしく ふさわしく				形容詞-一般 
\\	「宗教(しゅうきょう)」のアクセントは?1434	宗教 仮名 
\\	しゅうきょう、しゅーきょー 漢字 
\\	宗教 発音図: 助詞: オ	宗教 しゅうきょう		宗教 しゅうきょう				名詞-普通名詞-一般 
\\	「総合的な(そうごうてきな)」のアクセントは?1435	総合的 仮名 
\\	そうごうてき、そーごーてき 漢字 
\\	総合的 発音図:	総合的な そうごうてきな		総合的な そうごうてきな 総合的です そうごうてきです 総合的で そうごうてきで 総合的だった そうごうてきだった 総合的じゃない そうごうてきじゃない 総合的じゃなかった そうごうてきじゃなかった 総合的ならば そうごうてきならば 総合的だ そうごうてきだ 総合的に そうごうてきに				形状詞-一般 
\\	「集中(しゅうちゅう)」のアクセントは?1436	集中 仮名 
\\	しゅーちゅー、しゅうちゅう 漢字 
\\	集中 発音図: 助詞: オ	集中 しゅうちゅう		集中 しゅうちゅう 集中する しゅうちゅうする 集中します しゅうちゅうします 集中して しゅうちゅうして 集中した しゅうちゅうした 集中しない しゅうちゅうしない 集中しなかった しゅうちゅうしなかった 集中すれば しゅうちゅうすれば 集中させる しゅうちゅうさせる 集中される しゅうちゅうされる 集中しろ しゅうちゅうしろ 集中できる しゅうちゅうできる 集中しよう しゅうちゅうしよう				動詞-一般 
\\	「被る(かぶる)」のアクセントは?1437	かぶる{被} 仮名 
\\	かぶる 漢字 
\\	被る 発音図:	かぶる かぶる		かぶる かぶる かぶります かぶります かぶって かぶって かぶった かぶった かぶらない かぶらない かぶらなかった かぶらなかった かぶれば かぶれば かぶらせる かぶらせる かぶられる かぶられる かぶれ かぶれ かぶれる かぶれる かぶろう かぶろう				動詞-一般 
\\	「溜まる(たまる)」のアクセントは?1438	たまる{溜} 仮名 
\\	たまる 漢字 
\\	溜まる 発音図:	溜まる たまる		溜まる たまる 溜まります たまります 溜まって たまって 溜まった たまった 溜まらない たまらない 溜まらなかった たまらなかった 溜まれば たまれば 溜まらせる たまらせる 溜まられる たまられる 溜まれ たまれ 溜まろう たまろう				動詞-一般 
\\	「法人(ほうじん)」のアクセントは?1439	法人 仮名 
\\	ほうじん、ほーじん 漢字 
\\	法人 発音図: 助詞: オ							名詞-普通名詞-一般 
\\	「社員(しゃいん)」のアクセントは?1440	社員 仮名 
\\	しゃいん 漢字 
\\	社員 発音図: 助詞: オ	社員 しゃいん		社員 しゃいん				名詞-普通名詞-一般 
\\	「両方(りょうほう)」のアクセントは?1441	両方 仮名 
\\	りょうほう、りょーほー 漢字 
\\	両方 発音図: 助詞: オ 発音図: 助詞: オ	両方 りょうほう		両方 りょうほう				名詞-普通名詞-一般 
\\	「世間(せけん)」のアクセントは?1442	世間 仮名 
\\	せけん 漢字 
\\	世間 発音図: 助詞: オ	世間 せけん		世間 せけん				名詞-普通名詞-一般 
\\	「背後(はいご)」のアクセントは?1443	背後 仮名 
\\	はいご 漢字 
\\	背後 発音図: 助詞: オ	背後 はいご		背後 はいご				名詞-普通名詞-一般 
\\	「耐える(たえる)」のアクセントは?1444	耐える 仮名 
\\	たえる 漢字 
\\	耐える 発音図:	耐える たえる		耐える たえる 耐えます たえます 耐えて たえて 耐えた たえた 耐えない たえない 耐えなかった たえなかった 耐えれば たえれば 耐えさせる たえさせる 耐えられる たえられる 耐えろ たえろ 耐えられる たえられる 耐えよう たえよう				動詞-一般 
\\	「具合(ぐあい)」のアクセントは?1445	具合 仮名 
\\	ぐあい 漢字 
\\	具合 発音図: 助詞: オ	具合 ぐあい		具合 ぐあい				名詞-普通名詞-一般 
\\	「都合(つごう)《〜がいい》」のアクセントは?1446	都合(〜がいい) 仮名 
\\	つごう、つごー 漢字 
\\	都合 発音図: 例文: ガいい	都合 つごう		都合 つごう				名詞-普通名詞-一般 
\\	「都合(つごう)《合計》」のアクセントは?1447	都合(合計) 仮名 
\\	つごう、つごー 漢字 
\\	都合 発音図: 助詞: オ	都合 つごう		都合 つごう				名詞-普通名詞-一般 
\\	「公園(こうえん)」のアクセントは?1448	公園 仮名 
\\	こうえん、こーえん 漢字 
\\	公園 発音図: 助詞: オ	公園 こうえん		公園 こうえん				名詞-普通名詞-一般 
\\	「全部(ぜんぶ)」のアクセントは?1449	全部 仮名 
\\	ぜんぶ 漢字 
\\	全部 発音図: 助詞: オ	全部 ぜんぶ		全部 ぜんぶ				名詞-普通名詞-一般 
\\	「直ちに(ただちに)」のアクセントは?1450	直ちに 仮名 
\\	ただちに 漢字 
\\	直ちに 発音図:							副詞 
\\	「接する(せっする)」のアクセントは?1451	接する 仮名 
\\	せっする 漢字 
\\	接する 発音図: 発音図:	接する せっする		接する せっする 接します せっします 接して せっして 接した せっした 接しない せっしない 接しなかった せっしなかった 接すれば せっすれば 接しさせる せっしさせる 接せられる せっせられる 接しろ せっしろ 接せられる せっせられる 接しよう せっしよう				動詞-一般 
\\	「促進(そくしんする)」のアクセントは?1452	促進 仮名 
\\	そくしん 漢字 
\\	促進 発音図: 助詞: オ	促進する そくしんする		促進する そくしんする 促進します そくしんします 促進して そくしんして 促進した そくしんした 促進しない そくしんしない 促進しなかった そくしんしなかった 促進すれば そくしんすれば 促進させる そくしんさせる 促進される そくしんされる 促進しろ そくしんしろ 促進できる そくしんできる 促進しよう そくしんしよう				動詞-一般 
\\	「巻く(まく)」のアクセントは?1453	巻く 仮名 
\\	まく 漢字 
\\	巻く 発音図: 例文: ネジをト計	巻く まく		巻く まく 巻きます まきます 巻いて まいて 巻いた まいた 巻かない まかない 巻かなかった まかなかった 巻けば まけば 巻かせる まかせる 巻かれる まかれる 巻け まけ 巻ける まける 巻こう まこう				動詞-一般 
\\	「安定(あんてい)」のアクセントは?1454	安定 仮名 
\\	あんてい 漢字 
\\	安定 発音図: 助詞: オ	安定 あんてい		安定 あんてい 安定する あんていする 安定します あんていします 安定して あんていして 安定した あんていした 安定しない あんていしない 安定しなかった あんていしなかった 安定すれば あんていすれば 安定させる あんていさせる 安定される あんていされる 安定しろ あんていしろ 安定できる あんていできる 安定しよう あんていしよう				動詞-一般 
\\	「安全な(あんぜんな)」のアクセントは?1455		安全な あんぜんな		安全な あんぜんな 安全です あんぜんです 安全で あんぜんで 安全だった あんぜんだった 安全じゃない あんぜんじゃない 安全じゃなかった あんぜんじゃなかった 安全ならば あんぜんならば 安全だ あんぜんだ 安全に あんぜんに				名詞-普通名詞-一般 
\\	「安全(あんぜん)」のアクセントは?1456	安全 仮名 
\\	あんぜん 漢字 
\\	安全 発音図: 助詞: オ	安全 あんぜん		安全 あんぜん				名詞-普通名詞-一般 
\\	「敢えて(あえて)」のアクセントは?1457	あえて{敢} 仮名 
\\	あえて 漢字 
\\	敢えて 発音図:							副詞 
\\	「微妙な(びみょうな)」のアクセントは?1458		微妙な びみょうな		微妙な びみょうな 微妙です びみょうです 微妙で びみょうで 微妙だった びみょうだった 微妙じゃない びみょうじゃない 微妙じゃなかった びみょうじゃなかった 微妙ならば びみょうならば 微妙だ びみょうだ 微妙に びみょうに				形状詞-一般 
\\	「微妙(びみょう)」のアクセントは?1459	微妙 仮名 
\\	びみょう、びみょー 漢字 
\\	微妙 発音図: 助詞: オ	微妙 びみょう		微妙 びみょう				形状詞-一般 
\\	「確かな(たしかな)」のアクセントは?1460	確か 仮名 
\\	たしか 漢字 
\\	確か 発音図:	確かな たしかな		確かな たしかな 確かです たしかです 確かで たしかで 確かだった たしかだった 確かじゃない たしかじゃない 確かじゃなかった たしかじゃなかった 確かならば たしかならば 確かだ たしかだ 確かに たしかに				副詞 
\\	「アドバイス(あどばいす)」のアクセントは?1461	アドバイス 仮名 
\\	アドバイス、あどばいす 漢字 
\\	発音図: 助詞: オ 発音図: 助詞: オ	アドバイス あどばいす		アドバイス あどばいす				名詞-普通名詞-一般 
\\	「パン(ぱん)《ポ食》」のアクセントは?1462	パン(ポ食) 仮名 
\\	パン、ぱん 漢字 
\\	発音図: 助詞: オ	パン ぱん		パン ぱん				名詞-普通名詞-一般 
\\	「パン(パン)
\\	のアクセントは?1463	パン
\\	仮名 
\\	パン、ぱん 漢字 
\\	発音図: 助詞: オ							名詞-普通名詞-一般 
\\	「却って(かえって)」のアクセントは?1464	かえって{却} 仮名 
\\	かえって 漢字 
\\	却って 発音図:							副詞 
\\	「思い出(おもいで)」のアクセントは?1465	思い出 仮名 
\\	おもいで 漢字 
\\	思い出 発音図: 助詞: オ	思い出 おもいで		思い出 おもいで				名詞-普通名詞-一般 
\\	「計算(けいさん)」のアクセントは?1466	計算 仮名 
\\	けいさん 漢字 
\\	計算 発音図: 助詞: オ	計算 けいさん		計算 けいさん 計算する けいさんする 計算します けいさんします 計算して けいさんして 計算した けいさんした 計算しない けいさんしない 計算しなかった けいさんしなかった 計算すれば けいさんすれば 計算させる けいさんさせる 計算される けいさんされる 計算しろ けいさんしろ 計算できる けいさんできる 計算しよう けいさんしよう				動詞-一般 
\\	「恐怖(きょうふ)」のアクセントは?1467	恐怖 仮名 
\\	きょうふ、きょーふ 漢字 
\\	恐怖 発音図: 助詞: オ 発音図: 助詞: オ							名詞-普通名詞-一般 
\\	「救う(すくう)」のアクセントは?1468	救う 仮名 
\\	すくう 漢字 
\\	救う 発音図:	救う すくう		救う すくう 救います すくいます 救って すくって 救った すくった 救わない すくわない 救わなかった すくわなかった 救えば すくえば 救わせる すくわせる 救われる すくわれる 救え すくえ 救える すくえる 救おう すくおう				動詞-一般 
\\	「マンション(まんしょん)」のアクセントは?1469	マンション 仮名 
\\	マンション、まんしょん 漢字 
\\	発音図: 助詞: オ	マンション まんしょん		マンション まんしょん				名詞-普通名詞-一般 
\\	「観察(かんさつ)」のアクセントは?1470	観察 仮名 
\\	かんさつ 漢字 
\\	観察 発音図: 助詞: オ	観察 かんさつ		観察 かんさつ 観察する かんさつする 観察します かんさつします 観察して かんさつして 観察した かんさつした 観察しない かんさつしない 観察しなかった かんさつしなかった 観察すれば かんさつすれば 観察させる かんさつさせる 観察される かんさつされる 観察しろ かんさつしろ 観察できる かんさつできる 観察しよう かんさつしよう				動詞-一般 
\\	「明かり(あかり)」のアクセントは?1471	明かり{灯} 仮名 
\\	あかり 漢字 
\\	明かり、灯 発音図: 助詞: オ	明かり あかり		明かり あかり				名詞-普通名詞-一般 
\\	「宇宙(うちゅう)」のアクセントは?1472	宇宙 仮名 
\\	うちゅー、うちゅう 漢字 
\\	宇宙 発音図: 助詞: オ	宇宙 うちゅう		宇宙 うちゅう				名詞-普通名詞-一般 
\\	「注ぐ(そそぐ)」のアクセントは?1473	注ぐ 仮名 
\\	そそぐ 漢字 
\\	注ぐ 発音図: 発音図:	注ぐ そそぐ		注ぐ そそぐ 注ぎます そそぎます 注いで そそいで 注いだ そそいだ 注がない そそがない 注がなかった そそがなかった 注げば そそげば 注がせる そそがせる 注がれる そそがれる 注げ そそげ 注げる そそげる 注ごう そそごう				動詞-一般 
\\	「スポーツ(すぽーつ)」のアクセントは?1474	スポーツ 仮名 
\\	スポーツ、すぽーつ 漢字 
\\	発音図: 助詞: オ	スポーツ すぽーつ		スポーツ すぽーつ				名詞-普通名詞-一般 
\\	「整備(せいび)」のアクセントは?1475	整備 仮名 
\\	せいび 漢字 
\\	整備 発音図: 助詞: オ	整備 せいび		整備 せいび 整備する せいびする 整備します せいびします 整備して せいびして 整備した せいびした 整備しない せいびしない 整備しなかった せいびしなかった 整備すれば せいびすれば 整備させる せいびさせる 整備される せいびされる 整備しろ せいびしろ 整備できる せいびできる 整備しよう せいびしよう				動詞-一般 
\\	「内部(ないぶ)」のアクセントは?1476	内部 仮名 
\\	ないぶ 漢字 
\\	内部 発音図: 助詞: オ	内部 ないぶ		内部 ないぶ				名詞-普通名詞-一般 
\\	「重視(じゅうし)」のアクセントは?1477	重視 仮名 
\\	じゅーし、じゅうし 漢字 
\\	重視 発音図: 助詞: オ 発音図: 助詞: オ	重視 じゅうし		重視 じゅうし 重視する じゅうしする 重視します じゅうしします 重視して じゅうしして 重視した じゅうしした 重視しない じゅうししない 重視しなかった じゅうししなかった 重視すれば じゅうしすれば 重視させる じゅうしさせる 重視される じゅうしされる 重視しろ じゅうししろ 重視できる じゅうしできる 重視しよう じゅうししよう				動詞-一般 
\\	「試す(ためす)」のアクセントは?1478	試す 仮名 
\\	ためす 漢字 
\\	試す 発音図:	試す ためす		試す ためす 試します ためします 試して ためして 試した ためした 試さない ためさない 試さなかった ためさなかった 試せば ためせば 試させる ためさせる 試される ためされる 試せ ためせ 試せる ためせる 試そう ためそう				動詞-一般 
\\	「想像(そうぞう)」のアクセントは?1479	想像 仮名 
\\	そうぞう、そーぞー 漢字 
\\	想像 発音図: 助詞: オ	想像 そうぞう		想像 そうぞう 想像する そうぞうする 想像します そうぞうします 想像して そうぞうして 想像した そうぞうした 想像しない そうぞうしない 想像しなかった そうぞうしなかった 想像すれば そうぞうすれば 想像させる そうぞうさせる 想像される そうぞうされる 想像しろ そうぞうしろ 想像できる そうぞうできる 想像しよう そうぞうしよう				動詞-一般 
\\	「取り入れる(とりいれる)」のアクセントは?1480	取り入れる 仮名 
\\	とりいれる 漢字 
\\	取り入れる 発音図: 発音図:	取り入れる とりいれる		取り入れる とりいれる 取り入れます とりいれます 取り入れて とりいれて 取り入れた とりいれた 取り入れない とりいれない 取り入れなかった とりいれなかった 取り入れれば とりいれれば 取り入れさせる とりいれさせる 取り入れられる とりいれられる 取り入れろ とりいれろ 取り入れられる とりいれられる 取り入れよう とりいれよう				動詞-一般 
\\	「赤(あか)」のアクセントは?1481	赤 仮名 
\\	あか 漢字 
\\	赤 発音図: 助詞: オ	赤 あか		赤 あか				名詞-普通名詞-一般 
\\	「及ぼす(およぼす)」のアクセントは?1482	及ぼす 仮名 
\\	およぼす 漢字 
\\	及ぼす 発音図: 発音図:	及ぼす およぼす		及ぼす およぼす 及ぼします およぼします 及ぼして およぼして 及ぼした およぼした 及ぼさない およぼさない 及ぼさなかった およぼさなかった 及ぼせば およぼせば 及ぼさせる およぼさせる 及ぼされる およぼされる 及ぼせ およぼせ 及ぼせる およぼせる 及ぼそう およぼそう				動詞-一般 
\\	「明日(あす)《副》」のアクセントは?1483	あす{明日}(副) 仮名 
\\	あす 漢字 
\\	明日 発音図: 例文: 未来のことを、トも言う							副詞 
\\	「明日(あす)」のアクセントは?1484		明日 あす		明日 あす				副詞 
\\	「明日(あす)《名》」のアクセントは?1485	あす{明日}(名) 仮名 
\\	あす 漢字 
\\	明日 発音図: 例文: あと5分で、日付はニなる							副詞 
\\	「協力(きょうりょく)」のアクセントは?1486	協力 仮名 
\\	きょうりょく、きょーりょく 漢字 
\\	協力 発音図: 助詞: オ	協力 きょうりょく		協力 きょうりょく 協力する きょうりょくする 協力します きょうりょくします 協力して きょうりょくして 協力した きょうりょくした 協力しない きょうりょくしない 協力しなかった きょうりょくしなかった 協力すれば きょうりょくすれば 協力させる きょうりょくさせる 協力される きょうりょくされる 協力しろ きょうりょくしろ 協力できる きょうりょくできる 協力しよう きょうりょくしよう				動詞-一般 
\\	「書類(しょるい)」のアクセントは?1487	書類 仮名 
\\	しょるい 漢字 
\\	書類 発音図: 助詞: オ	書類 しょるい		書類 しょるい				名詞-普通名詞-一般 
\\	「今度(こんど)」のアクセントは?1488	今度 仮名 
\\	こんど 漢字 
\\	今度 発音図: 助詞: オ	今度 こんど		今度 こんど				副詞 
\\	「多様な(たような)」のアクセントは?1489		多様な たような		多様な たような 多様です たようです 多様で たようで 多様だった たようだった 多様じゃない たようじゃない 多様じゃなかった たようじゃなかった 多様ならば たようならば 多様だ たようだ 多様に たように				形状詞-一般 
\\	「多様(たよう)」のアクセントは?1490	多様 仮名 
\\	たよう、たよー 漢字 
\\	多様 発音図: 助詞: オ	多様 たよう		多様 たよう				形状詞-一般 
\\	「真実(しんじつ)」のアクセントは?1491	真実 仮名 
\\	しんじつ 漢字 
\\	真実 発音図: 助詞: オ	真実 しんじつ		真実 しんじつ				名詞-普通名詞-一般 
\\	「江戸(えど)」のアクセントは?1492		江戸 えど		江戸 えど				名詞-固有名詞-地名-一般 
\\	「吹く(ふく)」のアクセントは?1493	吹く 仮名 
\\	ふく 漢字 
\\	吹く 発音図: 例文: 笛をニはよい雰囲気 発音図: 例文: 笛をニはよい雰囲気	吹く ふく		吹く ふく 吹きます ふきます 吹いて ふいて 吹いた ふいた 吹かない ふかない 吹かなかった ふかなかった 吹けば ふけば 吹かせる ふかせる 吹かれる ふかれる 吹け ふけ 吹ける ふける 吹こう ふこう				動詞-一般 
\\	「読者(どくしゃ)」のアクセントは?1494	読者 仮名 
\\	どくしゃ 漢字 
\\	読者 発音図: 助詞: オ	読者 どくしゃ		読者 どくしゃ				名詞-普通名詞-一般 
\\	「廊下(ろうか)」のアクセントは?1495	廊下 仮名 
\\	ろうか、ろーか 漢字 
\\	廊下 発音図: 助詞: オ	廊下 ろうか		廊下 ろうか				名詞-普通名詞-一般 
\\	「メッセージ(めっせーじ)」のアクセントは?1496	メッセージ 仮名 
\\	メッセージ、めっせーじ 漢字 
\\	発音図: 助詞: オ	メッセージ めっせーじ		メッセージ めっせーじ				名詞-普通名詞-一般 
\\	「身近な(みぢかな)」のアクセントは?1497		身近な みぢかな		身近な みぢかな 身近です みぢかです 身近で みぢかで 身近だった みぢかだった 身近じゃない みぢかじゃない 身近じゃなかった みぢかじゃなかった 身近ならば みぢかならば 身近だ みぢかだ 身近に みぢかに				形状詞-一般 
\\	「身近(みぢか)」のアクセントは?1498	身近 仮名 
\\	みじか、みぢか 漢字 
\\	身近 発音図: 助詞: オ	身近 みぢか		身近 みぢか				形状詞-一般 
\\	「合計(ごうけい)」のアクセントは?1499	合計 仮名 
\\	ごうけい、ごーけい 漢字 
\\	合計 発音図: 助詞: オ	合計 ごうけい		合計 ごうけい				名詞-普通名詞-一般 
\\	「趣味(しゅみ)」のアクセントは?1500	趣味 仮名 
\\	しゅみ 漢字 
\\	趣味 発音図: 助詞: オ	趣味 しゅみ		趣味 しゅみ				名詞-普通名詞-一般 
\\	「浮かべる(うかべる)」のアクセントは?1501	浮かべる 仮名 
\\	うかべる 漢字 
\\	浮かべる 発音図:	浮かべる うかべる		浮かべる うかべる 浮かべます うかべます 浮かべて うかべて 浮かべた うかべた 浮かべない うかべない 浮かべなかった うかべなかった 浮かべれば うかべれば 浮かべさせる うかべさせる 浮かべられる うかべられる 浮かべろ うかべろ 浮かべる うかべる 浮かべよう うかべよう				動詞-一般 
\\	「掲載(けいさいする)」のアクセントは?1502	掲載 仮名 
\\	けいさい 漢字 
\\	掲載 発音図: 助詞: オ	掲載する けいさいする		掲載する けいさいする 掲載します けいさいします 掲載して けいさいして 掲載した けいさいした 掲載しない けいさいしない 掲載しなかった けいさいしなかった 掲載すれば けいさいすれば 掲載させる けいさいさせる 掲載される けいさいされる 掲載しろ けいさいしろ 掲載できる けいさいできる 掲載しよう けいさいしよう				動詞-一般 
\\	「寄る(よる)」のアクセントは?1503	寄る 仮名 
\\	よる 漢字 
\\	寄る 発音図:	寄る よる		寄る よる 寄ります よります 寄って よって 寄った よった 寄らない よらない 寄らなかった よらなかった 寄れば よれば 寄らせる よらせる 寄られる よられる 寄れ よれ 寄れる よれる 寄ろう よろう				動詞-一般 
\\	「風景(ふうけい)」のアクセントは?1504	風景 仮名 
\\	ふーけい、ふうけい 漢字 
\\	風景 発音図: 助詞: オ	風景 ふうけい		風景 ふうけい				名詞-普通名詞-一般 
\\	「飛行機(ひこうき)」のアクセントは?1505	飛行機 仮名 
\\	ひこうき、ひこーき 漢字 
\\	飛行機 発音図: 助詞: オ	飛行機 ひこうき		飛行機 ひこうき				名詞-普通名詞-一般 
\\	「作り出す(つくりだす)」のアクセントは?1506	作り出す 仮名 
\\	つくりだす 漢字 
\\	作り出す 発音図: 発音図:	作り出す つくりだす		作り出す つくりだす 作り出します つくりだします 作り出して つくりだして 作り出した つくりだした 作り出さない つくりださない 作り出さなかった つくりださなかった 作り出せば つくりだせば 作り出させる つくりださせる 作り出される つくりだされる 作り出せ つくりだせ 作り出せる つくりだせる 作り出そう つくりだそう				動詞-一般 
\\	「魂(たましい)」のアクセントは?1507	魂 仮名 
\\	たましー、たましい 漢字 
\\	魂 発音図: 助詞: オ	魂 たましい		魂 たましい				名詞-普通名詞-一般 
\\	「担う(になう)」のアクセントは?1508	担う 仮名 
\\	になう 漢字 
\\	担う 発音図:	担う になう		担う になう 担います にないます 担って になって 担った になった 担わない になわない 担わなかった になわなかった 担えば になえば 担わせる になわせる 担われる になわれる 担え になえ 担える になえる 担おう になおう				動詞-一般 
\\	「偉い(えらい)」のアクセントは?1509	偉い 仮名 
\\	えらい 漢字 
\\	偉い 発音図:	偉い えらい		偉い えらい 偉いです えらいです 偉くて えらくて えらくて 偉かった えらかった えらかった 偉くない えらくない えらくない 偉くなかった えらくなかった えらくなかった 偉ければ えらければ えらければ 偉い えらい 偉く えらく えらく				形容詞-一般 
\\	「式(しき)」のアクセントは?1510	式 仮名 
\\	しき 漢字 
\\	式 発音図: 助詞: オ	式 しき		式 しき				名詞-普通名詞-一般 
\\	「遊び(あそび)」のアクセントは?1511	遊び 仮名 
\\	あそび 漢字 
\\	遊び 発音図: 助詞: オ	遊び あそび		遊び あそび				名詞-普通名詞-一般 
\\	「徐々(じょじょ)《〜に》」のアクセントは?1512	徐々(〜に) 仮名 
\\	じょじょ 漢字 
\\	徐々 発音図: 助詞: ニ							名詞-普通名詞-一般 
\\	「上司(じょうし)」のアクセントは?1513	上司 仮名 
\\	じょうし、じょーし 漢字 
\\	上司 発音図: 助詞: オ	上司 じょうし		上司 じょうし				名詞-普通名詞-一般 
\\	「動向(どうこう)」のアクセントは?1514	動向 仮名 
\\	どうこう、どーこー 漢字 
\\	動向 発音図: 助詞: オ	動向 どうこう		動向 どうこう				名詞-普通名詞-一般 
\\	「陥る(おちいる)」のアクセントは?1515	陥る 仮名 
\\	おちいる 漢字 
\\	陥る 発音図: 発音図:	陥る おちいる		陥る おちいる 陥ります おちいります 陥って おちいって 陥った おちいった 陥らない おちいらない 陥らなかった おちいらなかった 陥れば おちいれば 陥らせる おちいらせる 陥られる おちいられる 陥れ おちいれ 陥ろう おちいろう				動詞-一般 
\\	「行政(ぎょうせい)」のアクセントは?1516	行政 仮名 
\\	ぎょうせい、ぎょーせい 漢字 
\\	行政 発音図: 助詞: オ	行政 ぎょうせい		行政 ぎょうせい				名詞-普通名詞-一般 
\\	「ルール(るーる)」のアクセントは?1517	ルール 仮名 
\\	ルール、るーる 漢字 
\\	発音図: 助詞: オ	ルール るーる		ルール るーる				名詞-普通名詞-一般 
\\	「プロ(ぷろ)
\\	のアクセントは?1518	プロ
\\	仮名 
\\	プロ、ぷろ 漢字 
\\	発音図: 助詞: オ	プロ ぷろ		プロ ぷろ				名詞-普通名詞-一般 
\\	「プロ(プロ)
\\	《ド》」のアクセントは?1519	プロ
\\	(ド) 仮名 
\\	プロ、ぷろ 漢字 
\\	発音図: 助詞: オ							名詞-普通名詞-一般 
\\	「プロ(プロ)
\\	のアクセントは?1520	プロ
\\	仮名 
\\	プロ、ぷろ 漢字 
\\	発音図: 助詞: オ							名詞-普通名詞-一般 
\\	「プロ(プロ)
\\	のアクセントは?1521	プロ
\\	仮名 
\\	プロ、ぷろ 漢字 
\\	発音図: 助詞: オ							名詞-普通名詞-一般 
\\	「飛び出す(とびだす)」のアクセントは?1522	飛び出す 仮名 
\\	とびだす 漢字 
\\	飛び出す 発音図:	飛び出す とびだす		飛び出す とびだす 飛び出します とびだします 飛び出して とびだして 飛び出した とびだした 飛び出さない とびださない 飛び出さなかった とびださなかった 飛び出せば とびだせば 飛び出させる とびださせる 飛び出される とびだされる 飛び出せ とびだせ 飛び出せる とびだせる 飛び出そう とびだそう				動詞-一般 
\\	「沖縄(おきなわ)」のアクセントは?1523	沖縄 仮名 
\\	おきなわ 漢字 
\\	沖縄 発音図: 助詞: オ	沖縄 おきなわ		沖縄 おきなわ				名詞-固有名詞-地名-一般 
\\	「収まる(おさまる)」のアクセントは?1524	収まる 仮名 
\\	おさまる 漢字 
\\	収まる 発音図:	収まる おさまる		収まる おさまる 収まります おさまります 収まって おさまって 収まった おさまった 収まらない おさまらない 収まらなかった おさまらなかった 収まれば おさまれば 収まらせる おさまらせる 収まられる おさまられる 収まれ おさまれ 収まれる おさまれる 収まろう おさまろう				動詞-一般 
\\	「今年(ことし)」のアクセントは?1525	ことし{今年} 仮名 
\\	ことし 漢字 
\\	今年 発音図: 助詞: オ	今年 ことし		今年 ことし				副詞 
\\	「今や(いまや)」のアクセントは?1526	今や 仮名 
\\	いまや 漢字 
\\	今や 発音図:							副詞 
\\	「踊る(おどる)」のアクセントは?1527	踊る 仮名 
\\	おどる 漢字 
\\	踊る 発音図:	踊る おどる		踊る おどる 踊ります おどります 踊って おどって 踊った おどった 踊らない おどらない 踊らなかった おどらなかった 踊れば おどれば 踊らせる おどらせる 踊られる おどられる 踊れ おどれ 踊れる おどれる 踊ろう おどろう				動詞-一般 
\\	「都道府県(とどうふけん)」のアクセントは?1528	都道府県 仮名 
\\	とどうふけん、とどーふけん 漢字 
\\	都道府県 発音図: 助詞: オ 発音図: 助詞: オ	都道府県 とどうふけん とどうふけん		都道府県 とどうふけん とどうふけん				名詞-普通名詞-一般 
\\	「天(てん)」のアクセントは?1529	天 仮名 
\\	あめ、てん 漢字 
\\	天 発音図: 助詞: オ 発音図: 助詞: オ	天 てん		天 てん				名詞-普通名詞-一般 
\\	「放つ(はなつ)《異彩を〜》」のアクセントは?1530	放つ(異彩を〜) 仮名 
\\	はなつ 漢字 
\\	放つ 発音図:							動詞-一般 
\\	「触る(さわる)」のアクセントは?1531	触る 仮名 
\\	さわる 漢字 
\\	触る 発音図:	触る さわる		触る さわる 触ります さわります 触って さわって 触った さわった 触らない さわらない 触らなかった さわらなかった 触れば さわれば 触らせる さわらせる 触られる さわられる 触れ さわれ 触れる さわれる 触ろう さわろう				動詞-一般 
\\	「限界(げんかい)」のアクセントは?1532	限界 仮名 
\\	げんかい 漢字 
\\	限界 発音図: 助詞: オ	限界 げんかい		限界 げんかい				名詞-普通名詞-一般 
\\	「朝(あさ)」のアクセントは?1533	朝 仮名 
\\	あさ 漢字 
\\	朝 発音図: 助詞: オ	朝 あさ		朝 あさ				副詞 
\\	「物凄い(ものすごい)」のアクセントは?1534		物凄い ものすごい		物凄い ものすごい 物凄いです ものすごいです 物凄くて ものすごくて ものすごくて 物凄かった ものすごかった ものすごかった 物凄くない ものすごくない ものすごくない 物凄くなかった ものすごくなかった ものすごくなかった 物凄ければ ものすごければ ものすごければ 物凄い ものすごい 物凄く ものすごく ものすごく				形容詞-一般 
\\	「ドラマ(どらま)」のアクセントは?1535	ドラマ 仮名 
\\	ドラマ、どらま 漢字 
\\	発音図: 助詞: オ	ドラマ どらま		ドラマ どらま				名詞-普通名詞-一般 
\\	「農業(のうぎょう)」のアクセントは?1536	農業 仮名 
\\	のうぎょう、のーぎょー 漢字 
\\	農業 発音図: 助詞: オ	農業 のうぎょう		農業 のうぎょう				名詞-普通名詞-一般 
\\	「独自(どくじ)」のアクセントは?1537	独自 仮名 
\\	どくじ 漢字 
\\	独自 発音図: 助詞: オ 発音図: 助詞: オ	独自 どくじ どくじ		独自 どくじ どくじ				形状詞-一般 
\\	「男女(だんじょ)」のアクセントは?1538	男女 仮名 
\\	おとこおんな、だんじょ、なんにょ 漢字 
\\	男女 発音図: 助詞: オ 発音図: 助詞: オ 発音図: 助詞: オ	男女 だんじょ		男女 だんじょ				名詞-普通名詞-一般 
\\	「コメント(こめんと)」のアクセントは?1539	コメント 仮名 
\\	コメント、こめんと 漢字 
\\	発音図: 助詞: オ	コメント こめんと		コメント こめんと				名詞-普通名詞-一般 
\\	「代表(だいひょう)」のアクセントは?1540	代表 仮名 
\\	だいひょう、だいひょー 漢字 
\\	代表 発音図: 助詞: オ	代表 だいひょう		代表 だいひょう 代表する だいひょうする 代表します だいひょうします 代表して だいひょうして 代表した だいひょうした 代表しない だいひょうしない 代表しなかった だいひょうしなかった 代表すれば だいひょうすれば 代表させる だいひょうさせる 代表される だいひょうされる 代表しろ だいひょうしろ 代表できる だいひょうできる 代表しよう だいひょうしよう				動詞-一般 
\\	「カード(かーど)」のアクセントは?1541	カード 仮名 
\\	カード、かーど 漢字 
\\	発音図: 助詞: オ	カード かーど		カード かーど				名詞-普通名詞-一般 
\\	「記録(きろく)」のアクセントは?1542	記録 仮名 
\\	きろく 漢字 
\\	記録 発音図: 助詞: オ	記録 きろく		記録 きろく 記録する きろくする 記録します きろくします 記録して きろくして 記録した きろくした 記録しない きろくしない 記録しなかった きろくしなかった 記録すれば きろくすれば 記録させる きろくさせる 記録される きろくされる 記録しろ きろくしろ 記録できる きろくできる 記録しよう きろくしよう				動詞-一般 
\\	「比率(ひりつ)」のアクセントは?1543	比率 仮名 
\\	ひりつ 漢字 
\\	比率 発音図: 助詞: オ	比率 ひりつ		比率 ひりつ				名詞-普通名詞-一般 
\\	「絶える(たえる)」のアクセントは?1544	絶える 仮名 
\\	たえる 漢字 
\\	絶える 発音図:	絶える たえる		絶える たえる 絶えます たえます 絶えて たえて 絶えた たえた 絶えない たえない 絶えなかった たえなかった 絶えれば たえれば 絶えさせる たえさせる 絶えられる たえられる 絶えろ たえろ 絶えよう たえよう				動詞-一般 
\\	「意志(いし)」のアクセントは?1545	意志 仮名 
\\	いし 漢字 
\\	意志 発音図: 助詞: オ	意志 いし		意志 いし				名詞-普通名詞-一般 
\\	「箱(はこ)」のアクセントは?1546	箱 仮名 
\\	はこ 漢字 
\\	箱 発音図: 助詞: オ	箱 はこ		箱 はこ				名詞-普通名詞-一般 
\\	「改革(かいかく)」のアクセントは?1547	改革 仮名 
\\	かいかく 漢字 
\\	改革 発音図: 助詞: オ	改革 かいかく		改革 かいかく				名詞-普通名詞-一般 
\\	「言語(げんご)」のアクセントは?1548	言語 仮名 
\\	げんご 漢字 
\\	言語 発音図: 助詞: オ	言語 げんご		言語 げんご				名詞-普通名詞-一般 
\\	「果たして(はたして)」のアクセントは?1549	はたして{果} 仮名 
\\	はたして 漢字 
\\	果たして 発音図:							副詞 
\\	「終了(しゅうりょうする)」のアクセントは?1550	終了 仮名 
\\	しゅうりょう、しゅーりょー 漢字 
\\	終了 発音図: 助詞: オ	終了する しゅうりょうする		終了する しゅうりょうする 終了します しゅうりょうします 終了して しゅうりょうして 終了した しゅうりょうした 終了しない しゅうりょうしない 終了しなかった しゅうりょうしなかった 終了すれば しゅうりょうすれば 終了させる しゅうりょうさせる 終了される しゅうりょうされる 終了しろ しゅうりょうしろ 終了できる しゅうりょうできる 終了しよう しゅうりょうしよう				動詞-一般 
\\	「見方(みかた)《物の〜》」のアクセントは?1551	見方(物の〜) 仮名 
\\	みかた 漢字 
\\	見方 発音図: 例文: やはり私のガ正しかった 発音図: 例文: やはり私のガ正しかった	見方 みかた		見方 みかた				名詞-普通名詞-一般 
\\	「何回(なんかい)」のアクセントは?1552	何回 仮名 
\\	なんかい 漢字 
\\	何回 発音図: 助詞: オ							名詞-数詞 
\\	「チェック(ちぇっく)」のアクセントは?1553	チェック 仮名 
\\	チェック、ちぇっく 漢字 
\\	発音図: 助詞: オ	チェック ちぇっく		チェック ちぇっく チェックする ちぇっくする チェックします ちぇっくします チェックして ちぇっくして チェックした ちぇっくした チェックしない ちぇっくしない チェックしなかった ちぇっくしなかった チェックすれば ちぇっくすれば チェックさせる ちぇっくさせる チェックされる ちぇっくされる チェックしろ ちぇっくしろ チェックできる ちぇっくできる チェックしよう ちぇっくしよう				動詞-一般 
\\	「趣旨(しゅし)」のアクセントは?1554	趣旨 仮名 
\\	しゅし 漢字 
\\	趣旨 発音図: 助詞: オ	趣旨 しゅし		趣旨 しゅし				名詞-普通名詞-一般 
\\	「住宅(じゅうたく)」のアクセントは?1555	住宅 仮名 
\\	じゅーたく、じゅうたく 漢字 
\\	住宅 発音図: 助詞: オ	住宅 じゅうたく		住宅 じゅうたく				名詞-普通名詞-一般 
\\	「事前(じぜん)」のアクセントは?1556	事前 仮名 
\\	じぜん 漢字 
\\	事前 発音図: 助詞: オ	事前 じぜん		事前 じぜん				名詞-普通名詞-一般 
\\	「努力(どりょく)」のアクセントは?1557	努力 仮名 
\\	どりょく 漢字 
\\	努力 発音図: 助詞: オ	努力 どりょく		努力 どりょく 努力する どりょくする 努力します どりょくします 努力して どりょくして 努力した どりょくした 努力しない どりょくしない 努力しなかった どりょくしなかった 努力すれば どりょくすれば 努力させる どりょくさせる 努力される どりょくされる 努力しろ どりょくしろ 努力できる どりょくできる 努力しよう どりょくしよう				動詞-一般 
\\	「リスク(りすく)」のアクセントは?1558	リスク 仮名 
\\	リスク、りすく 漢字 
\\	発音図: 助詞: オ	リスク りすく		リスク りすく				名詞-普通名詞-一般 
\\	「エンジン(えんじん)」のアクセントは?1559	エンジン 仮名 
\\	エンジン、えんじん 漢字 
\\	発音図: 助詞: オ	エンジン えんじん		エンジン えんじん				名詞-普通名詞-一般 
\\	「料金(りょうきん)」のアクセントは?1560	料金 仮名 
\\	りょうきん、りょーきん 漢字 
\\	料金 発音図: 助詞: オ 発音図: 助詞: オ	料金 りょうきん		料金 りょうきん				名詞-普通名詞-一般 
\\	「小学生(しょうがくせい)」のアクセントは?1561	小学生 仮名 
\\	しょうがくせい、しょーがくせい 漢字 
\\	小学生 発音図: 助詞: オ	小学生 しょうがくせい		小学生 しょうがくせい				名詞-普通名詞-一般 
\\	「憲法(けんぽう)」のアクセントは?1562	憲法 仮名 
\\	けんぽう、けんぽー 漢字 
\\	憲法 発音図: 助詞: オ	憲法 けんぽう		憲法 けんぽう				名詞-普通名詞-一般 
\\	「登録(とうろく)」のアクセントは?1563	登録 仮名 
\\	とうろく、とーろく 漢字 
\\	登録 発音図: 助詞: オ	登録 とうろく		登録 とうろく 登録する とうろくする 登録します とうろくします 登録して とうろくして 登録した とうろくした 登録しない とうろくしない 登録しなかった とうろくしなかった 登録すれば とうろくすれば 登録させる とうろくさせる 登録される とうろくされる 登録しろ とうろくしろ 登録できる とうろくできる 登録しよう とうろくしよう				動詞-一般 
\\	「施策(しさく)」のアクセントは?1564	施策 仮名 
\\	しさく 漢字 
\\	施策 発音図: 助詞: オ							名詞-普通名詞-一般 
\\	「仕様(しよう)」のアクセントは?1565	仕様 仮名 
\\	しよう、しよー 漢字 
\\	仕様 発音図: 助詞: オ	仕様 しよう		仕様 しよう				名詞-普通名詞-一般 
\\	「確立(かくりつ)」のアクセントは?1566	確立 仮名 
\\	かくりつ 漢字 
\\	確立 発音図: 助詞: オ	確立 かくりつ		確立 かくりつ 確立する かくりつする 確立します かくりつします 確立して かくりつして 確立した かくりつした 確立しない かくりつしない 確立しなかった かくりつしなかった 確立すれば かくりつすれば 確立させる かくりつさせる 確立される かくりつされる 確立しろ かくりつしろ 確立できる かくりつできる 確立しよう かくりつしよう				動詞-一般 
\\	「地図(ちず)」のアクセントは?1567	地図 仮名 
\\	ちず 漢字 
\\	地図 発音図: 助詞: オ	地図 ちず		地図 ちず				名詞-普通名詞-一般 
\\	「スタイル(すたいる)」のアクセントは?1568	スタイル 仮名 
\\	スタイル、すたいる 漢字 
\\	発音図: 助詞: オ	スタイル すたいる		スタイル すたいる				名詞-普通名詞-一般 
\\	「意思(いし)」のアクセントは?1569	意思 仮名 
\\	いし 漢字 
\\	意思 発音図: 助詞: オ	意思 いし		意思 いし				名詞-普通名詞-一般 
\\	「真面目な(まじめな)」のアクセントは?1570		真面目な まじめな		真面目な まじめな 真面目です まじめです 真面目で まじめで 真面目だった まじめだった 真面目じゃない まじめじゃない 真面目じゃなかった まじめじゃなかった 真面目ならば まじめならば 真面目だ まじめだ 真面目に まじめに				形状詞-一般 
\\	「真面目(まじめ)」のアクセントは?1571	まじめ{真面目} 仮名 
\\	まじめ 漢字 
\\	真面目 発音図: 助詞: オ	真面目 まじめ		真面目 まじめ				形状詞-一般 
\\	「知恵(ちえ)」のアクセントは?1572	知恵 仮名 
\\	ちえ 漢字 
\\	知恵 発音図: 助詞: オ	知恵 ちえ		知恵 ちえ				名詞-普通名詞-一般 
\\	「本物(ほんもの)」のアクセントは?1573	本物 仮名 
\\	ほんもの 漢字 
\\	本物 発音図: 助詞: オ	本物 ほんもの		本物 ほんもの				名詞-普通名詞-一般 
\\	「貸す(かす)」のアクセントは?1574	貸す 仮名 
\\	かす 漢字 
\\	貸す 発音図: 助詞: オ	貸す かす		貸す かす 貸します かします 貸して かして 貸した かした 貸さない かさない 貸さなかった かさなかった 貸せば かせば 貸させる かさせる 貸される かされる 貸せ かせ 貸せる かせる 貸そう かそう				動詞-一般 
\\	「両者(りょうしゃ)」のアクセントは?1575	両者 仮名 
\\	りょうしゃ、りょーしゃ 漢字 
\\	両者 発音図: 助詞: オ	両者 りょうしゃ		両者 りょうしゃ				名詞-普通名詞-一般 
\\	「盛んな(さかんな)」のアクセントは?1576	盛ん 仮名 
\\	さかん 漢字 
\\	盛ん 発音図: 助詞: オ	盛んな さかんな		盛んな さかんな 盛んです さかんです 盛んで さかんで 盛んだった さかんだった 盛んじゃない さかんじゃない 盛んじゃなかった さかんじゃなかった 盛んならば さかんならば 盛んだ さかんだ 盛んに さかんに				形状詞-一般 
\\	「女子(じょし)」のアクセントは?1577	女子 仮名 
\\	じょし 漢字 
\\	女子 発音図: 助詞: オ	女子 じょし		女子 じょし				名詞-普通名詞-一般 
\\	「順(じゅん)」のアクセントは?1578	順 仮名 
\\	じゅん 漢字 
\\	順 発音図: 助詞: オ	順 じゅん		順 じゅん				名詞-普通名詞-一般 
\\	「先っき(さっき)」のアクセントは?1579	さっき{先} 仮名 
\\	さっき 漢字 
\\	先っき 発音図: 助詞: オ							名詞-普通名詞-一般 
\\	「話し合う(はなしあう)」のアクセントは?1580	話し合う 仮名 
\\	はなしあう 漢字 
\\	話し合う 発音図: 発音図:	話し合う はなしあう		話し合う はなしあう 話し合います はなしあいます 話し合って はなしあって 話し合った はなしあった 話し合わない はなしあわない 話し合わなかった はなしあわなかった 話し合えば はなしあえば 話し合わせる はなしあわせる 話し合われる はなしあわれる 話し合え はなしあえ 話し合える はなしあえる 話し合おう はなしあおう				動詞-一般 
\\	「裁判所(さいばんしょ)」のアクセントは?1581	裁判所 仮名 
\\	さいばんしょ 漢字 
\\	裁判所 発音図: 例文: あれがデす 発音図: 例文: あれがデす							名詞-普通名詞-一般 
\\	「味わう(あじわう)」のアクセントは?1582	味わう 仮名 
\\	あじわう 漢字 
\\	味わう 発音図: 発音図:	味わう あじわう		味わう あじわう 味わいます あじわいます 味わって あじわって 味わった あじわった 味わわない あじわわない 味わわなかった あじわわなかった 味わえば あじわえば 味わわせる あじわわせる 味わわれる あじわわれる 味わえ あじわえ 味わえる あじわえる 味わおう あじわおう				動詞-一般 
\\	「彼氏(かれし)」のアクセントは?1583	彼氏 仮名 
\\	かれし 漢字 
\\	彼氏 発音図: 助詞: オ							名詞-普通名詞-一般 
\\	「資産(しさん)」のアクセントは?1584	資産 仮名 
\\	しさん 漢字 
\\	資産 発音図: 助詞: オ 発音図: 助詞: オ	資産 しさん		資産 しさん				名詞-普通名詞-一般 
\\	「彼処(あそこ)」のアクセントは?1585	あそこ{彼処} 仮名 
\\	あそこ 漢字 
\\	彼処 発音図: 助詞: オ	あそこ あそこ		あそこ あそこ				代名詞 
\\	「専門家(せんもんか)」のアクセントは?1586	専門家 仮名 
\\	せんもんか 漢字 
\\	専門家 発音図: 助詞: オ	専門家 せんもんか		専門家 せんもんか				名詞-普通名詞-一般 
\\	「文(ぶん)」のアクセントは?1587	文 仮名 
\\	ふみ、ぶん 漢字 
\\	文 発音図: 助詞: オ 発音図: 助詞: オ 発音図: 助詞: オ	文 ぶん		文 ぶん				名詞-普通名詞-一般 
\\	「オークション(オークション)」のアクセントは?1588	オークション 仮名 
\\	オークション、おーくしょん 漢字 
\\	発音図: 助詞: オ							名詞-普通名詞-一般 
\\	「ネット(ネット)」のアクセントは?1589	ネット 仮名 
\\	ネット、ねっと 漢字 
\\	発音図: 助詞: オ							名詞-普通名詞-一般 
\\	「経験(けいけん)」のアクセントは?1590	経験 仮名 
\\	けいけん 漢字 
\\	経験 発音図: 助詞: オ	経験 けいけん		経験 けいけん 経験する けいけんする 経験します けいけんします 経験して けいけんして 経験した けいけんした 経験しない けいけんしない 経験しなかった けいけんしなかった 経験すれば けいけんすれば 経験させる けいけんさせる 経験される けいけんされる 経験しろ けいけんしろ 経験できる けいけんできる 経験しよう けいけんしよう				動詞-一般 
\\	「出来上がる(できあがる)」のアクセントは?1591	出来上がる 仮名 
\\	できあがる 漢字 
\\	出来上がる 発音図: 発音図:	出来上がる できあがる できあがる		出来上がる できあがる できあがる 出来上がります できあがります 出来上がって できあがって できあがって 出来上がった できあがった できあがった 出来上がらない できあがらない できあがらない 出来上がらなかった できあがらなかった できあがらなかった 出来上がれば できあがれば できあがれば 出来上がらせる できあがらせる できあがらせる 出来上がられる できあがられる できあがられる 出来上がれ できあがれ できあがれ 出来上がろう できあがろう				動詞-一般 
\\	「弁護士(べんごし)」のアクセントは?1592	弁護士 仮名 
\\	べんごし 漢字 
\\	弁護士 発音図: 助詞: オ	弁護士 べんごし		弁護士 べんごし				名詞-普通名詞-一般 
\\	「根拠(こんきょ)」のアクセントは?1593	根拠 仮名 
\\	こんきょ 漢字 
\\	根拠 発音図: 助詞: オ	根拠 こんきょ		根拠 こんきょ				名詞-普通名詞-一般 
\\	「2時(にじ)」のアクセントは?1594	2時 仮名 
\\	にじ 漢字 
\\	二時、2時 発音図: 助詞: オ	二時 にじ		二時 にじ				名詞-数詞 
\\	「縁(えん)《えにし;縁側》」のアクセントは?1595	縁(えにし;縁側) 仮名 
\\	えん 漢字 
\\	縁 発音図: 助詞: オ							名詞-普通名詞-一般 
\\	「震える(ふるえる)」のアクセントは?1596	震える 仮名 
\\	ふるえる 漢字 
\\	震える 発音図:	震える ふるえる		震える ふるえる 震えます ふるえます 震えて ふるえて 震えた ふるえた 震えない ふるえない 震えなかった ふるえなかった 震えれば ふるえれば 震えさせる ふるえさせる 震えられる ふるえられる 震えろ ふるえろ 震えよう ふるえよう				動詞-一般 
\\	「縁(えん)」のアクセントは?1597		縁 えん		縁 えん				名詞-普通名詞-一般 
\\	「コンピュータ(こんぴゅーた)」のアクセントは?1598		コンピュータ こんぴゅーた		コンピュータ こんぴゅーた				名詞-普通名詞-一般 
\\	「相変わらず(あいかわらず)」のアクセントは?1599	相変わらず 仮名 
\\	あいかわらず 漢字 
\\	相変わらず 発音図:							副詞 
\\	「如何なる(いかなる)」のアクセントは?1600	いかなる{如何} 仮名 
\\	いかなる 漢字 
\\	如何なる 発音図:							連体詞 
\\	「彼奴(あいつ)」のアクセントは?1601	あいつ{彼奴} 仮名 
\\	あいつ 漢字 
\\	彼奴 発音図: 助詞: オ							代名詞 
\\	「見解(けんかい)」のアクセントは?1602	見解 仮名 
\\	けんかい 漢字 
\\	見解 発音図: 助詞: オ	見解 けんかい		見解 けんかい				名詞-普通名詞-一般 
\\	「好む(このむ)」のアクセントは?1603	好む 仮名 
\\	このむ 漢字 
\\	好む 発音図:	好む このむ		好む このむ 好みます このみます 好んで このんで 好んだ このんだ 好まない このまない 好まなかった このまなかった 好めば このめば 好ませる このませる 好まれる このまれる 好め このめ 好もう このもう				動詞-一般 
\\	「橋(はし)」のアクセントは?1604	橋 仮名 
\\	はし 漢字 
\\	橋 発音図: 例文: 三代の夫婦がオ渡り初めた	橋 はし		橋 はし				名詞-普通名詞-一般 
\\	「道具(どうぐ)」のアクセントは?1605	道具 仮名 
\\	どうぐ、どーぐ 漢字 
\\	道具 発音図: 助詞: オ	道具 どうぐ		道具 どうぐ				名詞-普通名詞-一般 
\\	「世代(せだい)」のアクセントは?1606	世代 仮名 
\\	せだい 漢字 
\\	世代 発音図: 助詞: オ 発音図: 助詞: オ	世代 せだい		世代 せだい				名詞-普通名詞-一般 
\\	「個々(ここ)」のアクセントは?1607	個々 仮名 
\\	ここ 漢字 
\\	個々 発音図: 助詞: オ	個々 ここ		個々 ここ				名詞-普通名詞-一般 
\\	「鋭い(するどい)」のアクセントは?1608	鋭い 仮名 
\\	するどい 漢字 
\\	鋭い 発音図:	鋭い するどい		鋭い するどい 鋭いです するどいです 鋭くて するどくて するどくて 鋭かった するどかった するどかった 鋭くない するどくない するどくない 鋭くなかった するどくなかった するどくなかった 鋭ければ するどければ するどければ 鋭い するどい 鋭く するどく するどく				形容詞-一般 
\\	「トラブル(とらぶる)」のアクセントは?1609	トラブル 仮名 
\\	トラブル、とらぶる 漢字 
\\	発音図: 助詞: オ	トラブル とらぶる		トラブル とらぶる				名詞-普通名詞-一般 
\\	「暮らし(くらし)」のアクセントは?1610	暮らし 仮名 
\\	くらし 漢字 
\\	暮らし 発音図: 助詞: オ	暮らし くらし		暮らし くらし				名詞-普通名詞-一般 
\\	「体重(たいじゅう)」のアクセントは?1611	体重 仮名 
\\	たいじゅー、たいじゅう 漢字 
\\	体重 発音図: 助詞: オ	体重 たいじゅう		体重 たいじゅう				名詞-普通名詞-一般 
\\	「有り難い(ありがたい)」のアクセントは?1612	ありがたい{有難} 仮名 
\\	ありがたい 漢字 
\\	有り難い 発音図:	有り難い ありがたい		有り難い ありがたい 有り難いです ありがたいです 有り難くて ありがたくて ありがたくて 有り難かった ありがたかった ありがたかった 有り難くない ありがたくない ありがたくない 有り難くなかった ありがたくなかった ありがたくなかった 有り難ければ ありがたければ ありがたければ 有り難い ありがたい 有り難く ありがたく ありがたく				形容詞-一般 
\\	「市内(しない)」のアクセントは?1613	市内 仮名 
\\	しない 漢字 
\\	市内 発音図: 助詞: オ	市内 しない		市内 しない				名詞-普通名詞-一般 
\\	「積む(つむ)」のアクセントは?1614	積む 仮名 
\\	つむ 漢字 
\\	積む 発音図:	積む つむ		積む つむ 積みます つみます 積んで つんで 積んだ つんだ 積まない つまない 積まなかった つまなかった 積めば つめば 積ませる つませる 積まれる つまれる 積め つめ 積める つめる 積もう つもう				動詞-一般 
\\	「健康な(けんこうな)」のアクセントは?1615		健康な けんこうな		健康な けんこうな 健康です けんこうです 健康で けんこうで 健康だった けんこうだった 健康じゃない けんこうじゃない 健康じゃなかった けんこうじゃなかった 健康ならば けんこうならば 健康だ けんこうだ 健康に けんこうに				形状詞-一般 
\\	「団体(だんたい)」のアクセントは?1616	団体 仮名 
\\	だんたい 漢字 
\\	団体 発音図: 助詞: オ	団体 だんたい		団体 だんたい				名詞-普通名詞-一般 
\\	「失敗(しっぱい)」のアクセントは?1617	失敗 仮名 
\\	しっぱい 漢字 
\\	失敗 発音図: 助詞: オ	失敗 しっぱい		失敗 しっぱい 失敗する しっぱいする 失敗します しっぱいします 失敗して しっぱいして 失敗した しっぱいした 失敗しない しっぱいしない 失敗しなかった しっぱいしなかった 失敗すれば しっぱいすれば 失敗させる しっぱいさせる 失敗される しっぱいされる 失敗しろ しっぱいしろ 失敗できる しっぱいできる 失敗しよう しっぱいしよう				動詞-一般 
\\	「健康(けんこう)」のアクセントは?1618	健康 仮名 
\\	けんこう、けんこー 漢字 
\\	健康 発音図: 助詞: オ	健康 けんこう		健康 けんこう				形状詞-一般 
\\	「タイトル(たいとる)」のアクセントは?1619	タイトル 仮名 
\\	タイトル、たいとる 漢字 
\\	発音図: 助詞: オ 発音図: 助詞: オ	タイトル たいとる		タイトル たいとる				名詞-普通名詞-一般 
\\	「筋肉(きんにく)」のアクセントは?1620	筋肉 仮名 
\\	きんにく 漢字 
\\	筋肉 発音図: 助詞: オ							名詞-普通名詞-一般 
\\	「鏡(かがみ)」のアクセントは?1621	鏡 仮名 
\\	かがみ 漢字 
\\	鏡 発音図: 助詞: オ	鏡 かがみ		鏡 かがみ				名詞-普通名詞-一般 
\\	「姉(あね)」のアクセントは?1622	姉 仮名 
\\	あね 漢字 
\\	姉 発音図: 助詞: オ	姉 あね		姉 あね				名詞-普通名詞-一般 
\\	「県(けん)」のアクセントは?1623	県 仮名 
\\	けん 漢字 
\\	県 発音図: 助詞: オ	県 けん		県 けん				名詞-普通名詞-一般 
\\	「浴びる(あびる)」のアクセントは?1624	浴びる 仮名 
\\	あびる 漢字 
\\	浴びる 発音図:	浴びる あびる		浴びる あびる 浴びます あびます 浴びて あびて 浴びた あびた 浴びない あびない 浴びなかった あびなかった 浴びれば あびれば 浴びさせる あびさせる 浴びられる あびられる 浴びろ あびろ 浴びられる あびられる 浴びよう あびよう				動詞-一般 
\\	「イタリア(いたりあ)」のアクセントは?1625		イタリア いたりあ		イタリア いたりあ				名詞-固有名詞-地名-国 
\\	「全身(ぜんしん)」のアクセントは?1626	全身 仮名 
\\	ぜんしん 漢字 
\\	全身 発音図: 助詞: オ	全身 ぜんしん		全身 ぜんしん				名詞-普通名詞-一般 
\\	「改善(かいぜん)」のアクセントは?1627	改善 仮名 
\\	かいぜん 漢字 
\\	改善 発音図: 助詞: オ	改善 かいぜん		改善 かいぜん 改善する かいぜんする 改善します かいぜんします 改善して かいぜんして 改善した かいぜんした 改善しない かいぜんしない 改善しなかった かいぜんしなかった 改善すれば かいぜんすれば 改善させる かいぜんさせる 改善される かいぜんされる 改善しろ かいぜんしろ 改善できる かいぜんできる 改善しよう かいぜんしよう				動詞-一般 
\\	「映像(えいぞう)」のアクセントは?1628	映像 仮名 
\\	えいぞう、えいぞー 漢字 
\\	映像 発音図: 助詞: オ	映像 えいぞう		映像 えいぞう				名詞-普通名詞-一般 
\\	「イベント(いべんと)」のアクセントは?1629	イベント 仮名 
\\	イベント、いべんと 漢字 
\\	発音図: 助詞: オ	イベント いべんと		イベント いべんと				名詞-普通名詞-一般 
\\	「好み(このみ)」のアクセントは?1630	好み 仮名 
\\	このみ 漢字 
\\	好み 発音図: 助詞: オ 発音図: 助詞: オ	好み このみ		好み このみ				名詞-普通名詞-一般 
\\	「辿る(たどる)」のアクセントは?1631	たどる{辿} 仮名 
\\	たどる 漢字 
\\	辿る 発音図: 発音図:	辿る たどる		辿る たどる 辿ります たどります 辿って たどって 辿った たどった 辿らない たどらない 辿らなかった たどらなかった 辿れば たどれば 辿らせる たどらせる 辿られる たどられる 辿れ たどれ 辿れる たどれる 辿ろう たどろう				動詞-一般 
\\	「投げる(なげる)」のアクセントは?1632	投げる 仮名 
\\	なげる 漢字 
\\	投げる 発音図:	投げる なげる		投げる なげる 投げます なげます 投げて なげて 投げた なげた 投げない なげない 投げなかった なげなかった 投げれば なげれば 投げさせる なげさせる 投げられる なげられる 投げろ なげろ 投げられる なげられる 投げよう なげよう				動詞-一般 
\\	「貴重(きちょう)」のアクセントは?1633	貴重 仮名 
\\	きちょう、きちょー 漢字 
\\	貴重 発音図: 助詞: オ	貴重 きちょう		貴重 きちょう				形状詞-一般 
\\	「貴重な(きちょうな)」のアクセントは?1634		貴重な きちょうな		貴重な きちょうな 貴重です きちょうです 貴重で きちょうで 貴重だった きちょうだった 貴重じゃない きちょうじゃない 貴重じゃなかった きちょうじゃなかった 貴重ならば きちょうならば 貴重だ きちょうだ 貴重に きちょうに				形状詞-一般 
\\	「周辺(しゅうへん)」のアクセントは?1635	周辺 仮名 
\\	しゅーへん、しゅうへん 漢字 
\\	周辺 発音図: 助詞: オ	周辺 しゅうへん		周辺 しゅうへん				名詞-普通名詞-一般 
\\	「関係(かんけい)」のアクセントは?1636	関係 仮名 
\\	かんけい 漢字 
\\	関係 発音図: 助詞: オ	関係 かんけい		関係 かんけい 関係する かんけいする 関係します かんけいします 関係して かんけいして 関係した かんけいした 関係しない かんけいしない 関係しなかった かんけいしなかった 関係すれば かんけいすれば 関係させる かんけいさせる 関係される かんけいされる 関係しろ かんけいしろ 関係できる かんけいできる 関係しよう かんけいしよう				動詞-一般 
\\	「吐く(はく)」のアクセントは?1637	吐く 仮名 
\\	はく 漢字 
\\	吐く 発音図: 例文: もくもくと煙をエん突	吐く はく		吐く はく 吐きます はきます 吐いて はいて 吐いた はいた 吐かない はかない 吐かなかった はかなかった 吐けば はけば 吐かせる はかせる 吐かれる はかれる 吐け はけ 吐ける はける 吐こう はこう				動詞-一般 
\\	「年金(ねんきん)」のアクセントは?1638	年金 仮名 
\\	ねんきん 漢字 
\\	年金 発音図: 助詞: オ	年金 ねんきん		年金 ねんきん				名詞-普通名詞-一般 
\\	「地面(じめん)」のアクセントは?1639	地面 仮名 
\\	じめん 漢字 
\\	地面 発音図: 助詞: オ	地面 じめん		地面 じめん				名詞-普通名詞-一般 
\\	「ロシア(ろしあ)」のアクセントは?1640		ロシア ろしあ		ロシア ろしあ				名詞-固有名詞-地名-国 
\\	「伯父さん(おじさん)」のアクセントは?1641	伯父さん 仮名 
\\	おじさん 漢字 
\\	伯父さん 発音図: 助詞: オ	伯父さん おじさん		伯父さん おじさん				名詞-普通名詞-一般 
\\	「満ちる(みちる)」のアクセントは?1642	満ちる 仮名 
\\	みちる 漢字 
\\	満ちる 発音図:	満ちる みちる		満ちる みちる 満ちます みちます 満ちて みちて 満ちた みちた 満ちない みちない 満ちなかった みちなかった 満ちれば みちれば 満ちさせる みちさせる 満ちられる みちられる 満ちろ みちろ 満ちよう みちよう				動詞-一般 
\\	「事務所(じむしょ)」のアクセントは?1643	事務所 仮名 
\\	じむしょ 漢字 
\\	事務所 発音図: 助詞: オ	事務所 じむしょ		事務所 じむしょ				名詞-普通名詞-一般 
\\	「スピード(すぴーど)」のアクセントは?1644	スピード 仮名 
\\	スピード、すぴーど 漢字 
\\	発音図: 助詞: オ	スピード すぴーど		スピード すぴーど				名詞-普通名詞-一般 
\\	「撮影(さつえい)」のアクセントは?1645	撮影 仮名 
\\	さつえい 漢字 
\\	撮影 発音図: 助詞: オ	撮影 さつえい		撮影 さつえい 撮影する さつえいする 撮影します さつえいします 撮影して さつえいして 撮影した さつえいした 撮影しない さつえいしない 撮影しなかった さつえいしなかった 撮影すれば さつえいすれば 撮影させる さつえいさせる 撮影される さつえいされる 撮影しろ さつえいしろ 撮影できる さつえいできる 撮影しよう さつえいしよう				動詞-一般 
\\	「無くす(なくす)」のアクセントは?1646	無くす 仮名 
\\	なくす 漢字 
\\	無くす 発音図:	無くす なくす		無くす なくす 無くします なくします 無くして なくして 無くした なくした 無くさない なくさない 無くさなかった なくさなかった 無くせば なくせば 無くさせる なくさせる 無くされる なくされる 無くせ なくせ 無くせる なくせる 無くそう なくそう				動詞-一般 
\\	「鍋(なべ)《お〜》」のアクセントは?1647	なべ{鍋}(お〜) 仮名 
\\	なべ 漢字 
\\	鍋 発音図: 助詞: オ	鍋 なべ		鍋 なべ				名詞-普通名詞-一般 
\\	「気配(けはい)」のアクセントは?1648	気配 仮名 
\\	けはい 漢字 
\\	気配 発音図: 助詞: オ 発音図: 助詞: オ	気配 けはい		気配 けはい				名詞-普通名詞-一般 
\\	「支配(しはい)」のアクセントは?1649	支配 仮名 
\\	しはい 漢字 
\\	支配 発音図: 助詞: オ 発音図: 助詞: オ	支配 しはい		支配 しはい 支配する しはいする 支配します しはいします 支配して しはいして 支配した しはいした 支配しない しはいしない 支配しなかった しはいしなかった 支配すれば しはいすれば 支配させる しはいさせる 支配される しはいされる 支配しろ しはいしろ 支配できる しはいできる 支配しよう しはいしよう				動詞-一般 
\\	「勇気(ゆうき)」のアクセントは?1650	勇気 仮名 
\\	ゆーき、ゆうき 漢字 
\\	勇気 発音図: 助詞: オ	勇気 ゆうき		勇気 ゆうき				名詞-普通名詞-一般 
\\	「内側(うちがわ)」のアクセントは?1651	内側 仮名 
\\	うちがわ 漢字 
\\	内側 発音図: 助詞: オ	内側 うちがわ		内側 うちがわ				名詞-普通名詞-一般 
\\	「需要(じゅよう)」のアクセントは?1652	需要 仮名 
\\	じゅよう、じゅよー 漢字 
\\	需要 発音図: 助詞: オ	需要 じゅよう		需要 じゅよう				名詞-普通名詞-一般 
\\	「丁寧な(ていねいな)」のアクセントは?1653		丁寧な ていねいな		丁寧な ていねいな 丁寧です ていねいです 丁寧で ていねいで 丁寧だった ていねいだった 丁寧じゃない ていねいじゃない 丁寧じゃなかった ていねいじゃなかった 丁寧ならば ていねいならば 丁寧だ ていねいだ 丁寧に ていねいに				形状詞-一般 
\\	「丁寧(ていねい)」のアクセントは?1654	ていねい{丁寧} 仮名 
\\	ていねい 漢字 
\\	丁寧 発音図: 助詞: オ	丁寧 ていねい		丁寧 ていねい				形状詞-一般 
\\	「踏む(ふむ)」のアクセントは?1655	踏む 仮名 
\\	ふむ 漢字 
\\	踏む 発音図:	踏む ふむ		踏む ふむ 踏みます ふみます 踏んで ふんで 踏んだ ふんだ 踏まない ふまない 踏まなかった ふまなかった 踏めば ふめば 踏ませる ふませる 踏まれる ふまれる 踏め ふめ 踏める ふめる 踏もう ふもう				動詞-一般 
\\	「機械(きかい)」のアクセントは?1656	機械 仮名 
\\	きかい 漢字 
\\	機械 発音図: 助詞: オ	機械 きかい		機械 きかい				名詞-普通名詞-一般 
\\	「意(い)《〜に介せず》」のアクセントは?1657	意(〜に介せず) 仮名 
\\	い 漢字 
\\	意 発音図: 例文: 他人のことはニ介さず実行した	意 い		意 い				名詞-普通名詞-一般 
\\	「役立つ(やくだつ)」のアクセントは?1658	役立つ 仮名 
\\	やくだつ 漢字 
\\	役立つ 発音図:	役立つ やくだつ		役立つ やくだつ 役立ちます やくだちます 役立って やくだって 役立った やくだった 役立たない やくだたない 役立たなかった やくだたなかった 役立てば やくだてば 役立たせる やくだたせる 役立たれる やくだたれる 役立て やくだて 役立てる やくだてる 役立とう やくだとう				動詞-一般 
\\	「減らす(へらす)」のアクセントは?1659	減らす 仮名 
\\	へらす 漢字 
\\	減らす 発音図:	減らす へらす		減らす へらす 減らします へらします 減らして へらして 減らした へらした 減らさない へらさない 減らさなかった へらさなかった 減らせば へらせば 減らさせる へらさせる 減らされる へらされる 減らせ へらせ 減らせる へらせる 減らそう へらそう				動詞-一般 
\\	「引き上げる(ひきあげる)」のアクセントは?1660	引き上げる 仮名 
\\	ひきあげる 漢字 
\\	引き上げる 発音図:	引き上げる ひきあげる		引き上げる ひきあげる 引き上げます ひきあげます 引き上げて ひきあげて 引き上げた ひきあげた 引き上げない ひきあげない 引き上げなかった ひきあげなかった 引き上げれば ひきあげれば 引き上げさせる ひきあげさせる 引き上げられる ひきあげられる 引き上げろ ひきあげろ 引き上げられる ひきあげられる 引き上げよう ひきあげよう				動詞-一般 
\\	「荷物(にもつ)」のアクセントは?1661	荷物 仮名 
\\	にもつ 漢字 
\\	荷物 発音図: 助詞: オ	荷物 にもつ		荷物 にもつ				名詞-普通名詞-一般 
\\	「インド(いんど)」のアクセントは?1662		インド いんど		インド いんど				名詞-固有名詞-地名-国 
\\	「彼方此方(あちこち)」のアクセントは?1663	あちこち{彼方此方} 仮名 
\\	あちこち 漢字 
\\	彼方此方 発音図: 助詞: オ 発音図: 助詞: オ	あちこち あちこち		あちこち あちこち				代名詞 
\\	「文書(ぶんしょ)」のアクセントは?1664	文書 仮名 
\\	ぶんしょ、もんじょ 漢字 
\\	文書 発音図: 助詞: オ 発音図: 助詞: オ	文書 ぶんしょ		文書 ぶんしょ				名詞-普通名詞-一般 
\\	「側面(そくめん)《〜を突く》」のアクセントは?1665	側面(〜を突く) 仮名 
\\	そくめん 漢字 
\\	側面 発音図: 例文: オ突く 発音図: 例文: オ突く 発音図: 例文: オ突く	側面 そくめん そくめん		側面 そくめん そくめん				名詞-普通名詞-一般 
\\	「とっても(とっても)」のアクセントは?1666	とっても 仮名 
\\	とっても 漢字 
\\	発音図:							副詞 
\\	「政治家(せいじか)」のアクセントは?1667	政治家 仮名 
\\	せいじか 漢字 
\\	政治家 発音図: 助詞: オ							名詞-普通名詞-一般 
\\	「脇(わき)」のアクセントは?1668	わき{脇;腋} 仮名 
\\	わき 漢字 
\\	脇、腋 発音図: 例文: 相撲ではオ固めるのが大切だ	脇 わき		脇 わき				名詞-普通名詞-一般 
\\	「活躍(かつやく)」のアクセントは?1669	活躍 仮名 
\\	かつやく 漢字 
\\	活躍 発音図: 助詞: オ	活躍 かつやく		活躍 かつやく 活躍する かつやくする 活躍します かつやくします 活躍して かつやくして 活躍した かつやくした 活躍しない かつやくしない 活躍しなかった かつやくしなかった 活躍すれば かつやくすれば 活躍させる かつやくさせる 活躍される かつやくされる 活躍しろ かつやくしろ 活躍できる かつやくできる 活躍しよう かつやくしよう				動詞-一般 
\\	「沸く(わく)」のアクセントは?1670	沸く 仮名 
\\	わく 漢字 
\\	沸く 発音図: 例文: 風呂の湯がマで散歩に出る	沸く わく		沸く わく 沸きます わきます 沸いて わいて 沸いた わいた 沸かない わかない 沸かなかった わかなかった 沸けば わけば 沸かせる わかせる 沸かれる わかれる 沸け わけ 沸こう わこう				動詞-一般 
\\	「クラス(くらす)」のアクセントは?1671	クラス 仮名 
\\	クラス、くらす 漢字 
\\	発音図: 助詞: オ	クラス くらす		クラス くらす				名詞-普通名詞-一般 
\\	「作家(さっか)」のアクセントは?1672	作家 仮名 
\\	さっか 漢字 
\\	作家 発音図: 助詞: オ 発音図: 助詞: オ	作家 さっか		作家 さっか				名詞-普通名詞-一般 
\\	「風邪(かぜ)」のアクセントは?1673	かぜ{風邪} 仮名 
\\	かぜ 漢字 
\\	風邪 発音図: 助詞: オ	風邪 かぜ		風邪 かぜ				名詞-普通名詞-一般 
\\	「本来(ほんらい)」のアクセントは?1674	本来 仮名 
\\	ほんらい 漢字 
\\	本来 発音図: 助詞: オ	本来 ほんらい		本来 ほんらい				副詞 
\\	「隠れる(かくれる)」のアクセントは?1675	隠れる 仮名 
\\	かくれる 漢字 
\\	隠れる 発音図:	隠れる かくれる		隠れる かくれる 隠れます かくれます 隠れて かくれて 隠れた かくれた 隠れない かくれない 隠れなかった かくれなかった 隠れれば かくれれば 隠れさせる かくれさせる 隠れられる かくれられる 隠れろ かくれろ 隠れられる かくれられる 隠れよう かくれよう				動詞-一般 
\\	「帰り(かえり)《お〜》」のアクセントは?1676	帰り(お〜) 仮名 
\\	かえり 漢字 
\\	帰り 発音図: 助詞: オ	帰り かえり		帰り かえり				名詞-普通名詞-一般 
\\	「処理(しょり)」のアクセントは?1677	処理 仮名 
\\	しょり 漢字 
\\	処理 発音図: 助詞: オ	処理 しょり		処理 しょり 処理する しょりする 処理します しょりします 処理して しょりして 処理した しょりした 処理しない しょりしない 処理しなかった しょりしなかった 処理すれば しょりすれば 処理させる しょりさせる 処理される しょりされる 処理しろ しょりしろ 処理できる しょりできる 処理しよう しょりしよう				動詞-一般 
\\	「重大(じゅうだい)」のアクセントは?1678	重大 仮名 
\\	じゅーだい、じゅうだい 漢字 
\\	重大 発音図: 助詞: オ	重大 じゅうだい		重大 じゅうだい				形状詞-一般 
\\	「重大な(じゅうだいな)」のアクセントは?1679		重大な じゅうだいな		重大な じゅうだいな 重大です じゅうだいです 重大で じゅうだいで 重大だった じゅうだいだった 重大じゃない じゅうだいじゃない 重大じゃなかった じゅうだいじゃなかった 重大ならば じゅうだいならば 重大だ じゅうだいだ 重大に じゅうだいに				形状詞-一般 
\\	「税金(ぜいきん)」のアクセントは?1680	税金 仮名 
\\	ぜいきん 漢字 
\\	税金 発音図: 助詞: オ	税金 ぜいきん		税金 ぜいきん				名詞-普通名詞-一般 
\\	「必要性(ひつようせい)」のアクセントは?1681	必要性 仮名 
\\	ひつようせい、ひつよーせい 漢字 
\\	必要性 発音図: 助詞: オ							名詞-普通名詞-一般 
\\	「左手(ひだりて)」のアクセントは?1682	左手 仮名 
\\	ひだりて 漢字 
\\	左手 発音図: 助詞: オ	左手 ひだりて		左手 ひだりて				名詞-普通名詞-一般 
\\	「光景(こうけい)」のアクセントは?1683	光景 仮名 
\\	こうけい、こーけい 漢字 
\\	光景 発音図: 助詞: オ	光景 こうけい		光景 こうけい				名詞-普通名詞-一般 
\\	「本格的(ほんかくてき)」のアクセントは?1684	本格的 仮名 
\\	ほんかくてき 漢字 
\\	本格的 発音図:							形状詞-一般 
\\	「恋人(こいびと)」のアクセントは?1685	恋人 仮名 
\\	こいびと 漢字 
\\	恋人 発音図: 助詞: オ	恋人 こいびと		恋人 こいびと				名詞-普通名詞-一般 
\\	「舌(した)」のアクセントは?1686	舌 仮名 
\\	した 漢字 
\\	舌 発音図: 例文: 口を開けてオ出した	舌 した		舌 した				名詞-普通名詞-一般 
\\	「漂う(ただよう)」のアクセントは?1687	漂う 仮名 
\\	ただよう 漢字 
\\	漂う 発音図:	漂う ただよう		漂う ただよう 漂います ただよいます 漂って ただよって 漂った ただよった 漂わない ただよわない 漂わなかった ただよわなかった 漂えば ただよえば 漂わせる ただよわせる 漂われる ただよわれる 漂え ただよえ 漂える ただよえる 漂おう ただよおう				動詞-一般 
\\	「築く(きずく)」のアクセントは?1688	築く 仮名 
\\	きずく 漢字 
\\	築く 発音図:	築く きずく		築く きずく 築きます きずきます 築いて きずいて 築いた きずいた 築かない きずかない 築かなかった きずかなかった 築けば きずけば 築かせる きずかせる 築かれる きずかれる 築け きずけ 築ける きずける 築こう きずこう				動詞-一般 
\\	「教室(きょうしつ)」のアクセントは?1689	教室 仮名 
\\	きょうしつ、きょーしつ 漢字 
\\	教室 発音図: 助詞: オ	教室 きょうしつ		教室 きょうしつ				名詞-普通名詞-一般 
\\	「油(あぶら)」のアクセントは?1690	油 仮名 
\\	あぶら 漢字 
\\	油 発音図: 助詞: オ	油 あぶら		油 あぶら				名詞-普通名詞-一般 
\\	「見上げる(みあげる)」のアクセントは?1691	見上げる 仮名 
\\	みあげる 漢字 
\\	見上げる 発音図: 発音図:	見上げる みあげる みあげる		見上げる みあげる みあげる 見上げます みあげます 見上げて みあげて みあげて 見上げた みあげた みあげた 見上げない みあげない みあげない 見上げなかった みあげなかった みあげなかった 見上げれば みあげれば みあげれば 見上げさせる みあげさせる みあげさせる 見上げられる みあげられる みあげられる 見上げろ みあげろ みあげろ 見上げられる みあげられる みあげられる 見上げよう みあげよう				動詞-一般 
\\	「人類(じんるい)」のアクセントは?1692	人類 仮名 
\\	じんるい 漢字 
\\	人類 発音図: 助詞: オ	人類 じんるい		人類 じんるい				名詞-普通名詞-一般 
\\	「名称(めいしょう)」のアクセントは?1693	名称 仮名 
\\	めいしょう、めいしょー 漢字 
\\	名称 発音図: 助詞: オ	名称 めいしょう		名称 めいしょう				名詞-普通名詞-一般 
\\	「端(はし)」のアクセントは?1694	端 仮名 
\\	たん、は、はし、はじ、はた 漢字 
\\	端 発音図: 助詞: オ 例文: 古びた池のニこいが寄ってくる 発音図: 例文: 道のワ崩れる心配があった 発音図: 例文: 道のワ崩れる心配があった 発音図: 例文: いろりのニ座る 発音図:	端 はし		端 はし				名詞-普通名詞-一般 
\\	「数値(すーち)」のアクセントは?1695	数値 仮名 
\\	すーち、すうち 漢字 
\\	数値 発音図: 助詞: オ							名詞-普通名詞-一般 
\\	「外国人(がいこくじん)」のアクセントは?1696	外国人 仮名 
\\	がいこくじん 漢字 
\\	外国人 発音図: 助詞: オ	外国人 がいこくじん		外国人 がいこくじん				名詞-普通名詞-一般 
\\	「地震(じしん)」のアクセントは?1697	地震 仮名 
\\	じしん 漢字 
\\	地震 発音図: 助詞: オ	地震 じしん		地震 じしん				名詞-普通名詞-一般 
\\	「災害(さいがい)」のアクセントは?1698	災害 仮名 
\\	さいがい 漢字 
\\	災害 発音図: 助詞: オ	災害 さいがい		災害 さいがい				名詞-普通名詞-一般 
\\	「引っ張る(ひっぱる)」のアクセントは?1699		引っ張る ひっぱる		引っ張る ひっぱる 引っ張ります ひっぱります 引っ張って ひっぱって 引っ張った ひっぱった 引っ張らない ひっぱらない 引っ張らなかった ひっぱらなかった 引っ張れば ひっぱれば 引っ張らせる ひっぱらせる 引っ張られる ひっぱられる 引っ張れ ひっぱれ 引っ張れる ひっぱれる 引っ張ろう ひっぱろう				動詞-一般 
\\	「文句(もんく)」のアクセントは?1700	文句 仮名 
\\	もんく 漢字 
\\	文句 発音図: 助詞: オ	文句 もんく		文句 もんく				名詞-普通名詞-一般 
\\	「鳴る(なる)」のアクセントは?1701	鳴る 仮名 
\\	なる 漢字 
\\	鳴る 発音図:	鳴る なる		鳴る なる 鳴ります なります 鳴って なって 鳴った なった 鳴らない ならない 鳴らなかった ならなかった 鳴れば なれば 鳴らせる ならせる 鳴られる なられる 鳴れ なれ 鳴ろう なろう				動詞-一般 
\\	「強化(きょうか)」のアクセントは?1702	強化 仮名 
\\	きょうか、きょーか 漢字 
\\	強化 発音図: 助詞: オ 発音図: 助詞: オ	強化 きょうか		強化 きょうか				名詞-普通名詞-一般 
\\	「理想(りそう)」のアクセントは?1703	理想 仮名 
\\	りそう、りそー 漢字 
\\	理想 発音図: 助詞: オ	理想 りそう		理想 りそう				名詞-普通名詞-一般 
\\	「担当(たんとうする)」のアクセントは?1704	担当 仮名 
\\	たんとう、たんとー 漢字 
\\	担当 発音図: 助詞: オ	担当する たんとうする		担当する たんとうする 担当します たんとうします 担当して たんとうして 担当した たんとうした 担当しない たんとうしない 担当しなかった たんとうしなかった 担当すれば たんとうすれば 担当させる たんとうさせる 担当される たんとうされる 担当しろ たんとうしろ 担当できる たんとうできる 担当しよう たんとうしよう				動詞-一般 
\\	「物質(ぶっしつ)」のアクセントは?1705	物質 仮名 
\\	ぶっしつ 漢字 
\\	物質 発音図: 助詞: オ	物質 ぶっしつ		物質 ぶっしつ				名詞-普通名詞-一般 
\\	「元気な(げんきな)」のアクセントは?1706		元気な げんきな		元気な げんきな 元気です げんきです 元気で げんきで 元気だった げんきだった 元気じゃない げんきじゃない 元気じゃなかった げんきじゃなかった 元気ならば げんきならば 元気だ げんきだ 元気に げんきに				名詞-普通名詞-一般 
\\	「元気(げんき)」のアクセントは?1707	元気 仮名 
\\	げんき 漢字 
\\	元気 発音図: 助詞: オ	元気 げんき		元気 げんき				名詞-普通名詞-一般 
\\	「主(ぬし)」のアクセントは?1708	主 仮名 
\\	しゅ、ぬし 漢字 
\\	主 発音図: 助詞: オ 例文: 英語をトして勉強している 発音図:	主 ぬし		主 ぬし				名詞-普通名詞-一般 
\\	「録る(とる)」のアクセントは?1709	とる{盗;摂;録;獲} 仮名 
\\	とる 漢字 
\\	録る、盗る、摂る、獲る 発音図:	盗る とる		盗る とる 盗ります とります 盗って とって 盗った とった 盗らない とらない 盗らなかった とらなかった 盗れば とれば 盗らせる とらせる 盗られる とられる 盗れ とれ 盗れる とれる 盗ろう とろう				助動詞 
\\	「死体(したい)」のアクセントは?1710	死体 仮名 
\\	したい 漢字 
\\	死体 発音図: 助詞: オ	死体 したい		死体 したい				名詞-普通名詞-一般 
\\	「餌(えさ)」のアクセントは?1711	えさ{餌} 仮名 
\\	えさ 漢字 
\\	餌 発音図: 例文: 森にはガたくさんある 発音図: 例文: 森にはガたくさんある	餌 えさ		餌 えさ				名詞-普通名詞-一般 
\\	「説(せつ)」のアクセントは?1712	説 仮名 
\\	せつ 漢字 
\\	説 発音図: 助詞: オ 発音図: 助詞: オ	説 せつ		説 せつ				名詞-普通名詞-一般 
\\	「現地(げんち)」のアクセントは?1713	現地 仮名 
\\	げんち 漢字 
\\	現地 発音図: 助詞: オ	現地 げんち		現地 げんち				名詞-普通名詞-一般 
\\	「強い(きつい)」のアクセントは?1714	きつい{強} 仮名 
\\	きつい 漢字 
\\	強い 発音図:	きつい きつい		きつい きつい きついです きついです きつくて きつくて きつかった きつかった きつくない きつくない きつくなかった きつくなかった きつければ きつければ きつい きつい きつい きつく きつく				形容詞-一般 
\\	「集団(しゅうだん)」のアクセントは?1715	集団 仮名 
\\	しゅーだん、しゅうだん 漢字 
\\	集団 発音図: 助詞: オ	集団 しゅうだん		集団 しゅうだん				名詞-普通名詞-一般 
\\	「添える(そえる)」のアクセントは?1716	添える 仮名 
\\	そえる 漢字 
\\	添える 発音図:	添える そえる		添える そえる 添えます そえます 添えて そえて 添えた そえた 添えない そえない 添えなかった そえなかった 添えれば そえれば 添えさせる そえさせる 添えられる そえられる 添えろ そえろ 添えられる そえられる 添えよう そえよう				動詞-一般 
\\	「男子(だんし)」のアクセントは?1717	男子 仮名 
\\	だんし 漢字 
\\	男子 発音図: 助詞: オ	男子 だんし		男子 だんし				名詞-普通名詞-一般 
\\	「使い方(つかいかた)」のアクセントは?1718	使い方 仮名 
\\	つかいかた 漢字 
\\	使い方 発音図: 助詞: オ							名詞-普通名詞-一般 
\\	「根(ね)」のアクセントは?1719	根 仮名 
\\	こん、ね 漢字 
\\	根 発音図: 助詞: オ 例文: ケヤキのワしっかり張っている 発音図:	根 ね		根 ね				名詞-普通名詞-一般 
\\	「序で(ついで)」のアクセントは?1720	ついで{序} 仮名 
\\	ついで 漢字 
\\	序で 発音図: 助詞: オ	序で ついで		序で ついで				名詞-普通名詞-一般 
\\	「電話(でんわ)《お〜》」のアクセントは?1721	電話(お〜) 仮名 
\\	でんわ 漢字 
\\	電話 発音図: 助詞: オ	電話 でんわ		電話 でんわ 電話する でんわする 電話します でんわします 電話して でんわして 電話した でんわした 電話しない でんわしない 電話しなかった でんわしなかった 電話すれば でんわすれば 電話させる でんわさせる 電話される でんわされる 電話しろ でんわしろ 電話できる でんわできる 電話しよう でんわしよう				動詞-一般 
\\	「他方(たほう)」のアクセントは?1722	他方 仮名 
\\	たほう、たほー 漢字 
\\	他方 発音図: 助詞: オ	他方 たほう		他方 たほう				名詞-普通名詞-一般 
\\	「柱(はしら)」のアクセントは?1723	柱 仮名 
\\	はしら 漢字 
\\	柱 発音図: 例文: 丈夫な太い木はトして使う 発音図: 例文: 丈夫な太い木はトして使う	柱 はしら		柱 はしら				名詞-普通名詞-一般 
\\	「株(かぶ)」のアクセントは?1724	株 仮名 
\\	かぶ 漢字 
\\	株 発音図: 助詞: オ	株 かぶ		株 かぶ				名詞-普通名詞-一般 
\\	「習慣(しゅうかん)」のアクセントは?1725	習慣 仮名 
\\	しゅーかん、しゅうかん 漢字 
\\	習慣 発音図: 助詞: オ	習慣 しゅうかん		習慣 しゅうかん				名詞-普通名詞-一般 
\\	「教会(きょうかい)」のアクセントは?1726	教会 仮名 
\\	きょうかい、きょーかい 漢字 
\\	教会 発音図: 助詞: オ	教会 きょうかい		教会 きょうかい				名詞-普通名詞-一般 
\\	「念(ねん)《〜を入れる;〜を押す》」のアクセントは?1727	念(〜を入れる;〜を押す) 仮名 
\\	ねん 漢字 
\\	念 発音図: 例文: オ入れる 発音図: 例文: オ入れる	念 ねん		念 ねん				名詞-普通名詞-一般 
\\	「スタッフ(すたっふ)」のアクセントは?1728	スタッフ 仮名 
\\	スタッフ、すたっふ 漢字 
\\	発音図: 助詞: オ	スタッフ すたっふ		スタッフ すたっふ				名詞-普通名詞-一般 
\\	「大勢(おおぜい)《副》」のアクセントは?1729	大勢(副) 仮名 
\\	おーぜい、おおぜい 漢字 
\\	大勢 発音図: 助詞: オ	大勢 おおぜい		大勢 おおぜい				名詞-普通名詞-一般 
\\	「大勢(おおぜい)《名》」のアクセントは?1730	大勢(名) 仮名 
\\	おーぜい、おおぜい 漢字 
\\	大勢 発音図: 助詞: オ	大勢 おおぜい		大勢 おおぜい				名詞-普通名詞-一般 
\\	「成り立つ(なりたつ)」のアクセントは?1731	成り立つ 仮名 
\\	なりたつ 漢字 
\\	成り立つ 発音図: 発音図:	成り立つ なりたつ		成り立つ なりたつ 成り立ちます なりたちます 成り立って なりたって 成り立った なりたった 成り立たない なりたたない 成り立たなかった なりたたなかった 成り立てば なりたてば 成り立たせる なりたたせる 成り立たれる なりたたれる 成り立て なりたて 成り立とう なりたとう				動詞-一般 
\\	「政令(せいれい)」のアクセントは?1732	政令 仮名 
\\	せいれい 漢字 
\\	政令 発音図: 助詞: オ							名詞-普通名詞-一般 
\\	「足元(あしもと)」のアクセントは?1733		足元 あしもと		足元 あしもと				名詞-普通名詞-一般 
\\	「保険料(ほけんりょう)」のアクセントは?1734	保険料 仮名 
\\	ほけんりょう、ほけんりょー 漢字 
\\	保険料 発音図: 助詞: オ							名詞-普通名詞-一般 
\\	「領域(りょういき)」のアクセントは?1735	領域 仮名 
\\	りょういき、りょーいき 漢字 
\\	領域 発音図: 助詞: オ	領域 りょういき		領域 りょういき				名詞-普通名詞-一般 
\\	「国会(こっかい)」のアクセントは?1736	国会 仮名 
\\	こっかい 漢字 
\\	国会 発音図: 助詞: オ	国会 こっかい		国会 こっかい				名詞-普通名詞-一般 
\\	「項目(こうもく)」のアクセントは?1737	項目 仮名 
\\	こうもく、こーもく 漢字 
\\	項目 発音図: 助詞: オ	項目 こうもく		項目 こうもく				名詞-普通名詞-一般 
\\	「パパ(ぱぱ)」のアクセントは?1738	パパ 仮名 
\\	パパ、ぱぱ 漢字 
\\	発音図: 助詞: オ	パパ ぱぱ		パパ ぱぱ				名詞-普通名詞-一般 
\\	「水準(すいじゅん)」のアクセントは?1739	水準 仮名 
\\	すいじゅん 漢字 
\\	水準 発音図: 助詞: オ	水準 すいじゅん		水準 すいじゅん				名詞-普通名詞-一般 
\\	「心臓(しんぞう)」のアクセントは?1740	心臓 仮名 
\\	しんぞう、しんぞー 漢字 
\\	心臓 発音図: 助詞: オ	心臓 しんぞう		心臓 しんぞう				名詞-普通名詞-一般 
\\	「突く(つく)」のアクセントは?1742	突く 仮名 
\\	つく 漢字 
\\	突く 発音図: 例文: 思い切ってホうがよい	突く つく		突く つく 突きます つきます 突いて ついて 突いた ついた 突かない つかない 突かなかった つかなかった 突けば つけば 突かせる つかせる 突かれる つかれる 突け つけ 突ける つける 突こう つこう				動詞-一般 
\\	「丸い(まるい)」のアクセントは?1743	丸い 仮名 
\\	まるい 漢字 
\\	丸い 発音図:	丸い まるい		丸い まるい 丸いです まるいです 丸くて まるくて 丸かった まるかった 丸くない まるくない 丸くなかった まるくなかった 丸ければ まるければ 丸い まるい まるい 丸く まるく				形容詞-一般 
\\	「悩み(なやみ)」のアクセントは?1744	悩み 仮名 
\\	なやみ 漢字 
\\	悩み 発音図: 助詞: オ	悩み なやみ		悩み なやみ				名詞-普通名詞-一般 
\\	「晒す(さらす)」のアクセントは?1745	さらす{晒} 仮名 
\\	さらす 漢字 
\\	晒す 発音図:	晒す さらす		晒す さらす 晒します さらします 晒して さらして 晒した さらした 晒さない さらさない 晒さなかった さらさなかった 晒せば さらせば 晒させる さらさせる 晒される さらされる 晒せ さらせ 晒せる さらせる 晒そう さらそう				動詞-一般 
\\	「着物(きもの)」のアクセントは?1746	着物 仮名 
\\	きもの 漢字 
\\	着物 発音図: 助詞: オ	着物 きもの		着物 きもの				名詞-普通名詞-一般 
\\	「敷く(しく)」のアクセントは?1747	敷く 仮名 
\\	しく 漢字 
\\	敷く 発音図:	敷く しく		敷く しく 敷きます しきます 敷いて しいて 敷いた しいた 敷かない しかない 敷かなかった しかなかった 敷けば しけば 敷かせる しかせる 敷かれる しかれる 敷け しけ 敷ける しける 敷こう しこう				動詞-一般 
\\	「絞る(しぼる)」のアクセントは?1748	絞る 仮名 
\\	しぼる 漢字 
\\	絞る 発音図:	絞る しぼる		絞る しぼる 絞ります しぼります 絞って しぼって 絞った しぼった 絞らない しぼらない 絞らなかった しぼらなかった 絞れば しぼれば 絞らせる しぼらせる 絞られる しぼられる 絞れ しぼれ 絞れる しぼれる 絞ろう しぼろう				動詞-一般 
\\	「そっと(そっと)《静かに》」のアクセントは?1749	そっと(静かに) 仮名 
\\	そっと 漢字 
\\	発音図:							副詞 
\\	「説く(とく)」のアクセントは?1750	説く 仮名 
\\	とく 漢字 
\\	説く 発音図:	説く とく		説く とく 説きます ときます 説いて といて 説いた といた 説かない とかない 説かなかった とかなかった 説けば とけば 説かせる とかせる 説かれる とかれる 説け とけ 説ける とける 説こう とこう				動詞-一般 
\\	「上記(じょうき)」のアクセントは?1751	上記 仮名 
\\	じょうき、じょーき 漢字 
\\	上記 発音図: 助詞: オ							名詞-普通名詞-一般 
\\	「コース(こーす)」のアクセントは?1752	コース 仮名 
\\	コース、こーす 漢字 
\\	発音図: 助詞: オ	コース こーす		コース こーす				名詞-普通名詞-一般 
\\	「褒める(ほめる)」のアクセントは?1753	褒める{誉} 仮名 
\\	ほめる 漢字 
\\	褒める、誉める 発音図:	褒める ほめる		褒める ほめる 褒めます ほめます 褒めて ほめて 褒めた ほめた 褒めない ほめない 褒めなかった ほめなかった 褒めれば ほめれば 褒めさせる ほめさせる 褒められる ほめられる 褒めろ ほめろ 褒められる ほめられる 褒めよう ほめよう				動詞-一般 
\\	「犠牲(ぎせい)」のアクセントは?1754	犠牲 仮名 
\\	ぎせい 漢字 
\\	犠牲 発音図: 助詞: オ	犠牲 ぎせい		犠牲 ぎせい				名詞-普通名詞-一般 
\\	「導く(みちびく)」のアクセントは?1755	導く 仮名 
\\	みちびく 漢字 
\\	導く 発音図:	導く みちびく		導く みちびく 導きます みちびきます 導いて みちびいて 導いた みちびいた 導かない みちびかない 導かなかった みちびかなかった 導けば みちびけば 導かせる みちびかせる 導かれる みちびかれる 導け みちびけ 導ける みちびける 導こう みちびこう				動詞-一般 
\\	「伝統(でんとう)」のアクセントは?1756	伝統 仮名 
\\	でんとう、でんとー 漢字 
\\	伝統 発音図: 助詞: オ	伝統 でんとう		伝統 でんとう				名詞-普通名詞-一般 
\\	「発行(はっこうする)」のアクセントは?1757	発行 仮名 
\\	はっこう、はっこー 漢字 
\\	発行 発音図: 助詞: オ	発行する はっこうする		発行する はっこうする 発行します はっこうします 発行して はっこうして 発行した はっこうした 発行しない はっこうしない 発行しなかった はっこうしなかった 発行すれば はっこうすれば 発行させる はっこうさせる 発行される はっこうされる 発行しろ はっこうしろ 発行できる はっこうできる 発行しよう はっこうしよう				動詞-一般 
\\	「支払い(しはらい)」のアクセントは?1758	支払い 仮名 
\\	しはらい 漢字 
\\	支払い 発音図: 助詞: オ	支払い しはらい		支払い しはらい				名詞-普通名詞-一般 
\\	「天気(てんき)」のアクセントは?1759	天気 仮名 
\\	てんき 漢字 
\\	天気 発音図: 助詞: オ	天気 てんき		天気 てんき				名詞-普通名詞-一般 
\\	「映る(うつる)」のアクセントは?1760	映る 仮名 
\\	うつる 漢字 
\\	映る 発音図:	映る うつる		映る うつる 映ります うつります 映って うつって 映った うつった 映らない うつらない 映らなかった うつらなかった 映れば うつれば 映らせる うつらせる 映られる うつられる 映れ うつれ 映れる うつれる 映ろう うつろう				動詞-一般 
\\	「参照(さんしょう)」のアクセントは?1761	参照 仮名 
\\	さんしょう、さんしょー 漢字 
\\	参照 発音図: 助詞: オ	参照 さんしょう		参照 さんしょう				名詞-普通名詞-一般 
\\	「反対(はんたい)」のアクセントは?1762	反対 仮名 
\\	はんたい 漢字 
\\	反対 発音図: 助詞: オ	反対 はんたい		反対 はんたい 反対する はんたいする 反対します はんたいします 反対して はんたいして 反対した はんたいした 反対しない はんたいしない 反対しなかった はんたいしなかった 反対すれば はんたいすれば 反対させる はんたいさせる 反対される はんたいされる 反対しろ はんたいしろ 反対できる はんたいできる 反対しよう はんたいしよう				動詞-一般 
\\	「揃える(そろえる)」のアクセントは?1763	そろえる{揃} 仮名 
\\	そろえる 漢字 
\\	揃える 発音図:	揃える そろえる		揃える そろえる 揃えます そろえます 揃えて そろえて 揃えた そろえた 揃えない そろえない 揃えなかった そろえなかった 揃えれば そろえれば 揃えさせる そろえさせる 揃えられる そろえられる 揃えろ そろえろ 揃えられる そろえられる 揃えよう そろえよう				動詞-一般 
\\	「豊富な(ほうふな)」のアクセントは?1764		豊富な ほうふな ほうふな		豊富な ほうふな ほうふな 豊富です ほうふです ほうふです 豊富で ほうふで ほうふで 豊富だった ほうふだった ほうふだった 豊富じゃない ほうふじゃない ほうふじゃない 豊富じゃなかった ほうふじゃなかった ほうふじゃなかった 豊富ならば ほうふならば ほうふならば 豊富だ ほうふだ ほうふだ 豊富に ほうふに ほうふに				形状詞-一般 
\\	「豊富(ほうふ)」のアクセントは?1765	豊富 仮名 
\\	ほうふ、ほーふ 漢字 
\\	豊富 発音図: 助詞: オ 発音図: 助詞: オ	豊富 ほうふ		豊富 ほうふ				形状詞-一般 
\\	「夜(よる)」のアクセントは?1766	夜 仮名 
\\	よ、よる 漢字 
\\	夜 発音図: 助詞: オ 例文: 話が長引きガ更ける 発音図:	夜 よる		夜 よる				副詞 
\\	「何方(どなた)」のアクセントは?1767	どなた{何方} 仮名 
\\	どなた 漢字 
\\	何方 発音図: 助詞: オ	どなた どなた		どなた どなた				代名詞 
\\	「結び付く(むすびつく)」のアクセントは?1768	結び付く 仮名 
\\	むすびつく 漢字 
\\	結び付く 発音図:	結び付く むすびつく		結び付く むすびつく 結び付きます むすびつきます 結び付いて むすびついて 結び付いた むすびついた 結び付かない むすびつかない 結び付かなかった むすびつかなかった 結び付けば むすびつけば 結び付かせる むすびつかせる 結び付かれる むすびつかれる 結び付け むすびつけ 結び付こう むすびつこう				動詞-一般 
\\	「到着(とうちゃく)」のアクセントは?1769	到着 仮名 
\\	とうちゃく、とーちゃく 漢字 
\\	到着 発音図: 助詞: オ	到着 とうちゃく		到着 とうちゃく 到着する とうちゃくする 到着します とうちゃくします 到着して とうちゃくして 到着した とうちゃくした 到着しない とうちゃくしない 到着しなかった とうちゃくしなかった 到着すれば とうちゃくすれば 到着させる とうちゃくさせる 到着される とうちゃくされる 到着しろ とうちゃくしろ 到着できる とうちゃくできる 到着しよう とうちゃくしよう				動詞-一般 
\\	「設立(せつりつする)」のアクセントは?1770	設立 仮名 
\\	せつりつ 漢字 
\\	設立 発音図: 助詞: オ	設立する せつりつする		設立する せつりつする 設立します せつりつします 設立して せつりつして 設立した せつりつした 設立しない せつりつしない 設立しなかった せつりつしなかった 設立すれば せつりつすれば 設立させる せつりつさせる 設立される せつりつされる 設立しろ せつりつしろ 設立できる せつりつできる 設立しよう せつりつしよう				動詞-一般 
\\	「枝(えだ)」のアクセントは?1771	枝 仮名 
\\	え、えだ 漢字 
\\	枝 発音図: 助詞: オ 例文: 庭の梅のオ手折る 発音図:	枝 えだ		枝 えだ				名詞-普通名詞-一般 
\\	「測る(はかる)」のアクセントは?1772	測る 仮名 
\\	はかる 漢字 
\\	測る 発音図:	測る はかる		測る はかる 測ります はかります 測って はかって 測った はかった 測らない はからない 測らなかった はからなかった 測れば はかれば 測らせる はからせる 測られる はかられる 測れ はかれ 測れる はかれる 測ろう はかろう				動詞-一般 
\\	「法螺(ほら)《〜を吹く》」のアクセントは?1773	ほら{法螺}(〜を吹く) 仮名 
\\	ほら 漢字 
\\	法螺 発音図: 例文: オ吹く							感動詞-一般 
\\	「アジア(あじあ)」のアクセントは?1774		アジア あじあ		アジア あじあ				名詞-固有名詞-地名-一般 
\\	「受け止める(うけとめる)」のアクセントは?1775	受け止める 仮名 
\\	うけとめる 漢字 
\\	受け止める 発音図: 発音図:	受け止める うけとめる うけとめる		受け止める うけとめる うけとめる 受け止めます うけとめます 受け止めて うけとめて うけとめて 受け止めた うけとめた うけとめた 受け止めない うけとめない うけとめない 受け止めなかった うけとめなかった うけとめなかった 受け止めれば うけとめれば うけとめれば 受け止めさせる うけとめさせる うけとめさせる 受け止められる うけとめられる うけとめられる 受け止めろ うけとめろ うけとめろ 受け止められる うけとめられる うけとめられる 受け止めよう うけとめよう				動詞-一般 
\\	「深刻(しんこく)」のアクセントは?1776	深刻 仮名 
\\	しんこく 漢字 
\\	深刻 発音図: 助詞: オ	深刻 しんこく		深刻 しんこく				形状詞-一般 
\\	「部下(ぶか)」のアクセントは?1777	部下 仮名 
\\	ぶか 漢字 
\\	部下 発音図: 助詞: オ	部下 ぶか		部下 ぶか				名詞-普通名詞-一般 
\\	「探る(さぐる)」のアクセントは?1778	探る 仮名 
\\	さぐる 漢字 
\\	探る 発音図:	探る さぐる		探る さぐる 探ります さぐります 探って さぐって 探った さぐった 探らない さぐらない 探らなかった さぐらなかった 探れば さぐれば 探らせる さぐらせる 探られる さぐられる 探れ さぐれ 探れる さぐれる 探ろう さぐろう				動詞-一般 
\\	「深刻な(しんこくな)」のアクセントは?1779		深刻な しんこくな		深刻な しんこくな 深刻です しんこくです 深刻で しんこくで 深刻だった しんこくだった 深刻じゃない しんこくじゃない 深刻じゃなかった しんこくじゃなかった 深刻ならば しんこくならば 深刻だ しんこくだ 深刻に しんこくに				形状詞-一般 
\\	「回復(かいふく)」のアクセントは?1780	回復 仮名 
\\	かいふく 漢字 
\\	回復 発音図: 助詞: オ	回復 かいふく		回復 かいふく 回復する かいふくする 回復します かいふくします 回復して かいふくして 回復した かいふくした 回復しない かいふくしない 回復しなかった かいふくしなかった 回復すれば かいふくすれば 回復させる かいふくさせる 回復される かいふくされる 回復しろ かいふくしろ 回復できる かいふくできる 回復しよう かいふくしよう				動詞-一般 
\\	「控える(ひかえる)」のアクセントは?1781	控える 仮名 
\\	ひかえる 漢字 
\\	控える 発音図: 発音図:	控える ひかえる		控える ひかえる 控えます ひかえます 控えて ひかえて 控えた ひかえた 控えない ひかえない 控えなかった ひかえなかった 控えれば ひかえれば 控えさせる ひかえさせる 控えられる ひかえられる 控えろ ひかえろ 控えられる ひかえられる 控えよう ひかえよう				動詞-一般 
\\	「先程(さきほど)」のアクセントは?1782	さきほど{先程} 仮名 
\\	さきほど 漢字 
\\	先程 発音図: 助詞: オ							名詞-普通名詞-一般 
\\	「数える(かぞえる)」のアクセントは?1783	数える 仮名 
\\	かぞえる 漢字 
\\	数える 発音図:	数える かぞえる		数える かぞえる 数えます かぞえます 数えて かぞえて 数えた かぞえた 数えない かぞえない 数えなかった かぞえなかった 数えれば かぞえれば 数えさせる かぞえさせる 数えられる かぞえられる 数えろ かぞえろ 数えられる かぞえられる 数えよう かぞえよう				動詞-一般 
\\	「微かな(かすかな)」のアクセントは?1784	かすか{微} 仮名 
\\	かすか 漢字 
\\	微か 発音図:	微かな かすかな		微かな かすかな 微かです かすかです 微かで かすかで 微かだった かすかだった 微かじゃない かすかじゃない 微かじゃなかった かすかじゃなかった 微かならば かすかならば 微かだ かすかだ 微かに かすかに				形状詞-一般 
\\	「シーン(しーん)」のアクセントは?1785	シーン 仮名 
\\	シーン、しーん 漢字 
\\	発音図: 助詞: オ	シーン しーん		シーン しーん				名詞-普通名詞-一般 
\\	「販売(はんばい)」のアクセントは?1786	販売 仮名 
\\	はんばい 漢字 
\\	販売 発音図: 助詞: オ	販売 はんばい		販売 はんばい 販売する はんばいする 販売します はんばいします 販売して はんばいして 販売した はんばいした 販売しない はんばいしない 販売しなかった はんばいしなかった 販売すれば はんばいすれば 販売させる はんばいさせる 販売される はんばいされる 販売しろ はんばいしろ 販売できる はんばいできる 販売しよう はんばいしよう				動詞-一般 
\\	「職業(しょくぎょう)」のアクセントは?1787	職業 仮名 
\\	しょくぎょう、しょくぎょー 漢字 
\\	職業 発音図: 助詞: オ	職業 しょくぎょう		職業 しょくぎょう				名詞-普通名詞-一般 
\\	「従業員(じゅうぎょういん)」のアクセントは?1788	従業員 仮名 
\\	じゅうぎょういん、じゅーぎょーいん 漢字 
\\	従業員 発音図: 助詞: オ	従業員 じゅうぎょういん		従業員 じゅうぎょういん				名詞-普通名詞-一般 
\\	「形態(けいたい)」のアクセントは?1789	形態 仮名 
\\	けいたい 漢字 
\\	形態 発音図: 助詞: オ	形態 けいたい		形態 けいたい				名詞-普通名詞-一般 
\\	「嫌う(きらう)」のアクセントは?1790	嫌う 仮名 
\\	きらう 漢字 
\\	嫌う 発音図:	嫌う きらう		嫌う きらう 嫌います きらいます 嫌って きらって 嫌った きらった 嫌わない きらわない 嫌わなかった きらわなかった 嫌えば きらえば 嫌わせる きらわせる 嫌われる きらわれる 嫌え きらえ 嫌える きらえる 嫌おう きらおう				動詞-一般 
\\	「対処(たいしょする)」のアクセントは?1791	対処 仮名 
\\	たいしょ 漢字 
\\	対処 発音図: 助詞: オ	対処する たいしょする		対処する たいしょする 対処します たいしょします 対処して たいしょして 対処した たいしょした 対処しない たいしょしない 対処しなかった たいしょしなかった 対処すれば たいしょすれば 対処させる たいしょさせる 対処される たいしょされる 対処しろ たいしょしろ 対処できる たいしょできる 対処しよう たいしょしよう				動詞-一般 
\\	「トップ(とっぷ)」のアクセントは?1792	トップ 仮名 
\\	トップ、とっぷ 漢字 
\\	発音図: 助詞: オ	トップ とっぷ		トップ とっぷ				名詞-普通名詞-一般 
\\	「年間(ねんかん)」のアクセントは?1793	年間 仮名 
\\	ねんかん 漢字 
\\	年間 発音図: 助詞: オ	年間 ねんかん		年間 ねんかん				名詞-普通名詞-一般 
\\	「先輩(せんぱい)」のアクセントは?1794	先輩 仮名 
\\	せんぱい 漢字 
\\	先輩 発音図: 助詞: オ	先輩 せんぱい		先輩 せんぱい				名詞-普通名詞-一般 
\\	「講師(こうし)」のアクセントは?1795	講師 仮名 
\\	こうし、こうじ、こーし、こーじ 漢字 
\\	講師 発音図: 助詞: オ 発音図: 助詞: オ	講師 こうし		講師 こうし				名詞-普通名詞-一般 
\\	「出会い(であい)」のアクセントは?1796	出会い 仮名 
\\	であい 漢字 
\\	出会い 発音図: 助詞: オ	出会い であい であい		出会い であい であい				名詞-普通名詞-一般 
\\	「尽くす(つくす)」のアクセントは?1797	尽くす 仮名 
\\	つくす 漢字 
\\	尽くす 発音図:	尽くす つくす		尽くす つくす 尽くします つくします 尽くして つくして 尽くした つくした 尽くさない つくさない 尽くさなかった つくさなかった 尽くせば つくせば 尽くさせる つくさせる 尽くされる つくされる 尽くせ つくせ 尽くせる つくせる 尽くそう つくそう				動詞-一般 
\\	「週(しゅう)」のアクセントは?1798	週 仮名 
\\	しゅー、しゅう 漢字 
\\	週 発音図: 助詞: オ	週 しゅう		週 しゅう				名詞-普通名詞-一般 
\\	「親父(おやじ)」のアクセントは?1799	おやじ{親父} 仮名 
\\	おやじ 漢字 
\\	親父 発音図: 助詞: オ 発音図: 助詞: オ	親父 おやじ		親父 おやじ				名詞-普通名詞-一般 
\\	「何年(なんねん)」のアクセントは?1800	何年 仮名 
\\	なんねん 漢字 
\\	何年 発音図: 助詞: オ	何年 なんねん		何年 なんねん				名詞-数詞 
\\	「湯(ゆ)《お〜》」のアクセントは?1801	湯(お〜) 仮名 
\\	ゆ 漢字 
\\	湯 発音図: 助詞: オ	湯 ゆ		湯 ゆ				名詞-普通名詞-一般 
\\	「民間(みんかん)」のアクセントは?1802	民間 仮名 
\\	みんかん 漢字 
\\	民間 発音図: 助詞: オ	民間 みんかん		民間 みんかん				名詞-普通名詞-一般 
\\	「穏やかな(おだやかな)」のアクセントは?1803	穏やか 仮名 
\\	おだやか 漢字 
\\	穏やか 発音図: 助詞: ダ	穏やかな おだやかな		穏やかな おだやかな 穏やかです おだやかです 穏やかで おだやかで 穏やかだった おだやかだった 穏やかじゃない おだやかじゃない 穏やかじゃなかった おだやかじゃなかった 穏やかならば おだやかならば 穏やかだ おだやかだ 穏やかに おだやかに				形状詞-一般 
\\	「外れる(はずれる)」のアクセントは?1804	外れる 仮名 
\\	はずれる 漢字 
\\	外れる 発音図:	外れる はずれる		外れる はずれる 外れます はずれます 外れて はずれて 外れた はずれた 外れない はずれない 外れなかった はずれなかった 外れれば はずれれば 外れさせる はずれさせる 外れられる はずれられる 外れろ はずれろ 外れよう はずれよう				動詞-一般 
\\	「ワイン(わいん)」のアクセントは?1805	ワイン 仮名 
\\	ワイン、わいん 漢字 
\\	発音図: 助詞: オ	ワイン わいん		ワイン わいん				名詞-普通名詞-一般 
\\	「城(しろ)」のアクセントは?1806	城 仮名 
\\	しろ 漢字 
\\	城 発音図: 例文: 国宝のワ我が町の誇りだ	城 しろ		城 しろ				名詞-普通名詞-一般 
\\	「経費(けいひ)」のアクセントは?1807	経費 仮名 
\\	けいひ 漢字 
\\	経費 発音図: 助詞: オ	経費 けいひ		経費 けいひ				名詞-普通名詞-一般 
\\	「苦しむ(くるしむ)」のアクセントは?1808	苦しむ 仮名 
\\	くるしむ 漢字 
\\	苦しむ 発音図:	苦しむ くるしむ		苦しむ くるしむ 苦しみます くるしみます 苦しんで くるしんで 苦しんだ くるしんだ 苦しまない くるしまない 苦しまなかった くるしまなかった 苦しめば くるしめば 苦しませる くるしませる 苦しまれる くるしまれる 苦しめ くるしめ 苦しめる くるしめる 苦しもう くるしもう				動詞-一般 
\\	「大した(たいした)《〜ことはない》」のアクセントは?1809	大した(〜ことはない) 仮名 
\\	たいした 漢字 
\\	大した 発音図: 例文: コとはない							連体詞 
\\	「改正(かいせい)」のアクセントは?1810	改正 仮名 
\\	かいせい 漢字 
\\	改正 発音図: 助詞: オ	改正 かいせい		改正 かいせい				名詞-普通名詞-一般 
\\	「泊まる(とまる)」のアクセントは?1811	泊まる 仮名 
\\	とまる 漢字 
\\	泊まる 発音図:	泊まる とまる		泊まる とまる 泊まります とまります 泊まって とまって 泊まった とまった 泊まらない とまらない 泊まらなかった とまらなかった 泊まれば とまれば 泊まらせる とまらせる 泊まられる とまられる 泊まれ とまれ 泊まれる とまれる 泊まろう とまろう				動詞-一般 
\\	「ねえ(ねえ)」のアクセントは?1812	ねえ 仮名 
\\	ねえ、ねい、ねー 漢字 
\\	発音図:							感動詞-一般 
\\	「無論(むろん)」のアクセントは?1813	無論 仮名 
\\	むろん 漢字 
\\	無論 発音図: 助詞: オ							副詞 
\\	「後者(こうしゃ)」のアクセントは?1814	後者 仮名 
\\	こうしゃ、こーしゃ 漢字 
\\	後者 発音図: 助詞: オ	後者 こうしゃ		後者 こうしゃ				名詞-普通名詞-一般 
\\	「天井(てんじょう)」のアクセントは?1815	天井 仮名 
\\	てんじょう、てんじょー 漢字 
\\	天井 発音図: 助詞: オ	天井 てんじょう		天井 てんじょう				名詞-普通名詞-一般 
\\	「壊れる(こわれる)」のアクセントは?1816	壊れる 仮名 
\\	こわれる 漢字 
\\	壊れる 発音図:	壊れる こわれる		壊れる こわれる 壊れます こわれます 壊れて こわれて 壊れた こわれた 壊れない こわれない 壊れなかった こわれなかった 壊れれば こわれれば 壊れさせる こわれさせる 壊れられる こわれられる 壊れろ こわれろ 壊れよう こわれよう				動詞-一般 
\\	「望ましい(のぞましい)」のアクセントは?1817	望ましい 仮名 
\\	のぞましい 漢字 
\\	望ましい 発音図: 発音図:	望ましい のぞましい		望ましい のぞましい 望ましいです のぞましいです 望ましくて のぞましくて のぞましくて 望ましかった のぞましかった のぞましかった 望ましくない のぞましくない のぞましくない 望ましくなかった のぞましくなかった のぞましくなかった 望ましければ のぞましければ のぞましければ 望ましい のぞましい 望ましく のぞましく のぞましく				形容詞-一般 
\\	「三十(さんじゅう)」のアクセントは?1818	
\\	仮名 
\\	さんじゅー、さんじゅう 漢字 
\\	三十、30 発音図: 助詞: オ	三十 さんじゅう		三十 さんじゅう				名詞-数詞 
\\	「大半(たいはん)」のアクセントは?1819	大半 仮名 
\\	たいはん 漢字 
\\	大半 発音図: 助詞: オ 発音図: 助詞: オ	大半 たいはん たいはん		大半 たいはん たいはん				名詞-普通名詞-一般 
\\	「保護者(ほごしゃ)」のアクセントは?1820	保護者 仮名 
\\	ほごしゃ 漢字 
\\	保護者 発音図: 助詞: オ							名詞-普通名詞-一般 
\\	「休み(やすみ)《お〜》」のアクセントは?1821	休み(お〜) 仮名 
\\	やすみ 漢字 
\\	休み 発音図: 助詞: オ	休み やすみ		休み やすみ				名詞-普通名詞-一般 
\\	「依然(いぜん)」のアクセントは?1822	依然 仮名 
\\	いぜん 漢字 
\\	依然 発音図:							副詞 
\\	「引き起こす(ひきおこす)」のアクセントは?1823	引き起こす 仮名 
\\	ひきおこす 漢字 
\\	引き起こす 発音図:	引き起こす ひきおこす		引き起こす ひきおこす 引き起こします ひきおこします 引き起こして ひきおこして 引き起こした ひきおこした 引き起こさない ひきおこさない 引き起こさなかった ひきおこさなかった 引き起こせば ひきおこせば 引き起こさせる ひきおこさせる 引き起こされる ひきおこされる 引き起こせ ひきおこせ 引き起こせる ひきおこせる 引き起こそう ひきおこそう				動詞-一般 
\\	「強力な(きょうりょくな)」のアクセントは?1824	強力 仮名 
\\	きょうりょく、きょーりょく、ごうりき、ごーりき 漢字 
\\	強力 発音図: 助詞: オ 例文: 山小屋でト休憩する 発音図: 例文: 山小屋でト休憩する 発音図:	強力な きょうりょくな		強力な きょうりょくな 強力です きょうりょくです 強力で きょうりょくで 強力だった きょうりょくだった 強力じゃない きょうりょくじゃない 強力じゃなかった きょうりょくじゃなかった 強力ならば きょうりょくならば 強力だ きょうりょくだ 強力に きょうりょくに				形状詞-一般 
\\	「中小企業(ちゅうしょうきぎょう)」のアクセントは?1825		中小企業 ちゅうしょうきぎょう		中小企業 ちゅうしょうきぎょう				名詞-普通名詞-一般 
\\	「幼い(おさない)」のアクセントは?1826	幼い 仮名 
\\	おさない 漢字 
\\	幼い 発音図:	幼い おさない		幼い おさない 幼いです おさないです 幼くて おさなくて おさなくて 幼かった おさなかった おさなかった 幼くない おさなくない おさなくない 幼くなかった おさなくなかった おさなくなかった 幼ければ おさなければ おさなければ 幼い おさない 幼く おさなく おさなく				形容詞-一般 
\\	「生える(はえる)」のアクセントは?1827	生える 仮名 
\\	はえる 漢字 
\\	生える 発音図:	生える はえる		生える はえる 生えます はえます 生えて はえて 生えた はえた 生えない はえない 生えなかった はえなかった 生えれば はえれば 生えさせる はえさせる 生えられる はえられる 生えろ はえろ 生えよう はえよう				動詞-一般 
\\	「納得(なっとく)」のアクセントは?1828	納得 仮名 
\\	なっとく 漢字 
\\	納得 発音図: 助詞: オ	納得 なっとく		納得 なっとく 納得する なっとくする 納得します なっとくします 納得して なっとくして 納得した なっとくした 納得しない なっとくしない 納得しなかった なっとくしなかった 納得すれば なっとくすれば 納得させる なっとくさせる 納得される なっとくされる 納得しろ なっとくしろ 納得できる なっとくできる 納得しよう なっとくしよう				動詞-一般 
\\	「利用者(りようしゃ)」のアクセントは?1829	利用者 仮名 
\\	りようしゃ、りよーしゃ 漢字 
\\	利用者 発音図: 助詞: オ	利用者 りようしゃ		利用者 りようしゃ				名詞-普通名詞-一般 
\\	「結局(けっきょく)」のアクセントは?1830	結局 仮名 
\\	けっきょく 漢字 
\\	結局 発音図: 助詞: オ	結局 けっきょく けっきょく		結局 けっきょく けっきょく				名詞-普通名詞-一般 
\\	「陰(いん)」のアクセントは?1831	陰 仮名 
\\	いん、かげ 漢字 
\\	陰 発音図: 助詞: オ 発音図:							名詞-普通名詞-一般 
\\	「見直し(みなおし)」のアクセントは?1832	見直し 仮名 
\\	みなおし 漢字 
\\	見直し 発音図: 助詞: オ	見直し みなおし		見直し みなおし				名詞-普通名詞-一般 
\\	「6時(ろくじ)」のアクセントは?1833	6時 仮名 
\\	ろくじ 漢字 
\\	六時、6時 発音図: 助詞: オ	六時 ろくじ		六時 ろくじ				名詞-数詞 
\\	「青年(せいねん)」のアクセントは?1834	青年 仮名 
\\	せいねん 漢字 
\\	青年 発音図: 助詞: オ	青年 せいねん		青年 せいねん				名詞-普通名詞-一般 
\\	「株式(かぶしき)」のアクセントは?1835	株式 仮名 
\\	かぶしき 漢字 
\\	株式 発音図: 助詞: オ	株式 かぶしき		株式 かぶしき				名詞-普通名詞-一般 
\\	「大声(おおごえ)」のアクセントは?1836	大声 仮名 
\\	おーごえ、おおごえ、たいせい 漢字 
\\	大声 発音図: 助詞: オ 発音図: 助詞: オ	大声 おおごえ		大声 おおごえ				名詞-普通名詞-一般 
\\	「希望(きぼう)」のアクセントは?1837	希望 仮名 
\\	きぼう、きぼー 漢字 
\\	希望 発音図: 助詞: オ	希望 きぼう		希望 きぼう 希望する きぼうする 希望します きぼうします 希望して きぼうして 希望した きぼうした 希望しない きぼうしない 希望しなかった きぼうしなかった 希望すれば きぼうすれば 希望させる きぼうさせる 希望される きぼうされる 希望しろ きぼうしろ 希望できる きぼうできる 希望しよう きぼうしよう				動詞-一般 
\\	「細胞(さいぼう)」のアクセントは?1838	細胞 仮名 
\\	さいほう、さいぼう、さいほー、さいぼー 漢字 
\\	細胞 発音図: 助詞: オ 発音図: 助詞: オ	細胞 さいぼう		細胞 さいぼう				名詞-普通名詞-一般 
\\	「叱る(しかる)」のアクセントは?1839	しかる{叱} 仮名 
\\	しかる 漢字 
\\	叱る 発音図:	叱る しかる		叱る しかる 叱ります しかります 叱って しかって 叱った しかった 叱らない しからない 叱らなかった しからなかった 叱れば しかれば 叱らせる しからせる 叱られる しかられる 叱れ しかれ 叱れる しかれる 叱ろう しかろう				動詞-一般 
\\	「懐かしい(なつかしい)」のアクセントは?1840	懐かしい 仮名 
\\	なつかしい 漢字 
\\	懐かしい 発音図:	懐かしい なつかしい		懐かしい なつかしい 懐かしいです なつかしいです 懐かしくて なつかしくて なつかしくて 懐かしかった なつかしかった なつかしかった 懐かしくない なつかしくない なつかしくない 懐かしくなかった なつかしくなかった なつかしくなかった 懐かしければ なつかしければ なつかしければ 懐かしい なつかしい 懐かしく なつかしく なつかしく				形容詞-一般 
\\	「祈る(いのる)」のアクセントは?1841	祈る 仮名 
\\	いのる 漢字 
\\	祈る 発音図:	祈る いのる		祈る いのる 祈ります いのります 祈って いのって 祈った いのった 祈らない いのらない 祈らなかった いのらなかった 祈れば いのれば 祈らせる いのらせる 祈られる いのられる 祈れ いのれ 祈れる いのれる 祈ろう いのろう				動詞-一般 
\\	「夕食(ゆうしょく)」のアクセントは?1842	夕食 仮名 
\\	ゆーしょく、ゆうしょく 漢字 
\\	夕食 発音図: 助詞: オ	夕食 ゆうしょく		夕食 ゆうしょく				名詞-普通名詞-一般 
\\	「精神的な(せいしんてきな)」のアクセントは?1843	精神的 仮名 
\\	せいしんてき 漢字 
\\	精神的 発音図:	精神的な せいしんてきな		精神的な せいしんてきな 精神的です せいしんてきです 精神的で せいしんてきで 精神的だった せいしんてきだった 精神的じゃない せいしんてきじゃない 精神的じゃなかった せいしんてきじゃなかった 精神的ならば せいしんてきならば 精神的だ せいしんてきだ 精神的に せいしんてきに				形状詞-一般 
\\	「頼る(たよる)」のアクセントは?1844	頼る 仮名 
\\	たよる 漢字 
\\	頼る 発音図:	頼る たよる		頼る たよる 頼ります たよります 頼って たよって 頼った たよった 頼らない たよらない 頼らなかった たよらなかった 頼れば たよれば 頼らせる たよらせる 頼られる たよられる 頼れ たよれ 頼れる たよれる 頼ろう たよろう				動詞-一般 
\\	「申請(しんせい)」のアクセントは?1845	申請 仮名 
\\	しんせい 漢字 
\\	申請 発音図:	申請 しんせい		申請 しんせい				名詞-普通名詞-一般 
\\	「理論(りろん)」のアクセントは?1846	理論 仮名 
\\	りろん 漢字 
\\	理論 発音図: 助詞: オ	理論 りろん		理論 りろん				名詞-普通名詞-一般 
\\	「神様(かみさま)」のアクセントは?1847	神様 仮名 
\\	かみさま 漢字 
\\	神様 発音図: 助詞: オ	神様 かみさま		神様 かみさま				名詞-普通名詞-一般 
\\	「門(もん)」のアクセントは?1848	門 仮名 
\\	かど、もん 漢字 
\\	門 発音図: 助詞: オ 発音図: 助詞: オ	門 もん		門 もん				名詞-普通名詞-一般 
\\	「権限(けんげん)」のアクセントは?1849	権限 仮名 
\\	けんげん 漢字 
\\	権限 発音図: 助詞: オ	権限 けんげん		権限 けんげん				名詞-普通名詞-一般 
\\	「厚い(あつい)」のアクセントは?1850	厚い 仮名 
\\	あつい 漢字 
\\	厚い 発音図:	厚い あつい		厚い あつい 厚いです あついです 厚くて あつくて 厚かった あつかった 厚くない あつくない 厚くなかった あつくなかった 厚ければ あつければ 厚い あつい あつい 厚く あつく				形容詞-一般 
\\	「揺れる(ゆれる)」のアクセントは?1851	揺れる 仮名 
\\	ゆれる 漢字 
\\	揺れる 発音図:	揺れる ゆれる		揺れる ゆれる 揺れます ゆれます 揺れて ゆれて 揺れた ゆれた 揺れない ゆれない 揺れなかった ゆれなかった 揺れれば ゆれれば 揺れさせる ゆれさせる 揺れられる ゆれられる 揺れろ ゆれろ 揺れられる ゆれられる 揺れよう ゆれよう				動詞-一般 
\\	「溜め息(ためいき)」のアクセントは?1852	ため息{溜} 仮名 
\\	ためいき 漢字 
\\	溜め息、ため息 発音図: 助詞: オ							名詞-普通名詞-一般 
\\	「整理(せいりする)」のアクセントは?1853	整理 仮名 
\\	せいり 漢字 
\\	整理 発音図: 助詞: オ	整理する せいりする		整理する せいりする 整理します せいりします 整理して せいりして 整理した せいりした 整理しない せいりしない 整理しなかった せいりしなかった 整理すれば せいりすれば 整理させる せいりさせる 整理される せいりされる 整理しろ せいりしろ 整理できる せいりできる 整理しよう せいりしよう				動詞-一般 
\\	「駐車場(ちゅうしゃじょう)」のアクセントは?1854	駐車場 仮名 
\\	ちゅうしゃじょう、ちゅーしゃじょー 漢字 
\\	駐車場 発音図: 助詞: オ	駐車場 ちゅうしゃじょう		駐車場 ちゅうしゃじょう				名詞-普通名詞-一般 
\\	「蓋(ふた)」のアクセントは?1855	ふた{蓋} 仮名 
\\	ふた 漢字 
\\	蓋 発音図: 助詞: オ	蓋 ふた		蓋 ふた				名詞-普通名詞-一般 
\\	「タクシー(たくしー)」のアクセントは?1856	タクシー 仮名 
\\	タクシー、たくしー 漢字 
\\	発音図: 助詞: オ	タクシー たくしー		タクシー たくしー				名詞-普通名詞-一般 
\\	「通常(つうじょう)」のアクセントは?1857	通常 仮名 
\\	つうじょう、つーじょー 漢字 
\\	通常 発音図: 助詞: オ	通常 つうじょう		通常 つうじょう				副詞 
\\	「酢(す)《お〜》」のアクセントは?1858	酢(お〜) 仮名 
\\	す 漢字 
\\	酢 発音図: 例文: いつもオ使ったおかずがある	酢 す		酢 す				名詞-普通名詞-一般 
\\	「千円(せんえん)」のアクセントは?1859		千円 せんえん		千円 せんえん				名詞-数詞 
\\	「虫(むし)」のアクセントは?1860	虫 仮名 
\\	むし 漢字 
\\	虫 発音図: 例文: 古文書はかなりニ食われていた	虫 むし		虫 むし				名詞-普通名詞-一般 
\\	「レストラン(れすとらん)」のアクセントは?1861	レストラン 仮名 
\\	レストラン、れすとらん 漢字 
\\	発音図: 助詞: オ	レストラン れすとらん		レストラン れすとらん				名詞-普通名詞-一般 
\\	「忽ち(たちまち)」のアクセントは?1862	たちまち{忽} 仮名 
\\	たちまち 漢字 
\\	忽ち 発音図:							副詞 
\\	「施す(ほどこす)」のアクセントは?1863	施す 仮名 
\\	ほどこす 漢字 
\\	施す 発音図: 発音図:	施す ほどこす		施す ほどこす 施します ほどこします 施して ほどこして 施した ほどこした 施さない ほどこさない 施さなかった ほどこさなかった 施せば ほどこせば 施させる ほどこさせる 施される ほどこされる 施せ ほどこせ 施せる ほどこせる 施そう ほどこそう				動詞-一般 
\\	「見守る(みまもる)」のアクセントは?1864	見守る 仮名 
\\	みまもる 漢字 
\\	見守る 発音図: 発音図:	見守る みまもる みまもる		見守る みまもる みまもる 見守ります みまもります 見守って みまもって みまもって 見守った みまもった みまもった 見守らない みまもらない みまもらない 見守らなかった みまもらなかった みまもらなかった 見守れば みまもれば みまもれば 見守らせる みまもらせる みまもらせる 見守られる みまもられる みまもられる 見守れ みまもれ みまもれ 見守れる みまもれる みまもれる 見守ろう みまもろう				動詞-一般 
\\	「生物(せいぶつ)」のアクセントは?1866	生物 仮名 
\\	せいぶつ 漢字 
\\	生物 発音図: 助詞: オ	生物 せいぶつ		生物 せいぶつ				名詞-普通名詞-一般 
\\	「問題点(もんだいてん)」のアクセントは?1867	問題点 仮名 
\\	もんだいてん 漢字 
\\	問題点 発音図: 助詞: オ	問題点 もんだいてん		問題点 もんだいてん				名詞-普通名詞-一般 
\\	「親しい(したしい)」のアクセントは?1868	親しい 仮名 
\\	したしい 漢字 
\\	親しい 発音図:	親しい したしい		親しい したしい 親しいです したしいです 親しくて したしくて したしくて 親しかった したしかった したしかった 親しくない したしくない したしくない 親しくなかった したしくなかった したしくなかった 親しければ したしければ したしければ 親しい したしい 親しく したしく したしく				形容詞-一般 
\\	「満足(まんぞく)」のアクセントは?1869	満足 仮名 
\\	まんぞく 漢字 
\\	満足 発音図: 助詞: オ	満足 まんぞく		満足 まんぞく 満足する まんぞくする 満足します まんぞくします 満足して まんぞくして 満足した まんぞくした 満足しない まんぞくしない 満足しなかった まんぞくしなかった 満足すれば まんぞくすれば 満足させる まんぞくさせる 満足される まんぞくされる 満足しろ まんぞくしろ 満足できる まんぞくできる 満足しよう まんぞくしよう				動詞-一般 
\\	「屋根(やね)」のアクセントは?1870	屋根 仮名 
\\	やね 漢字 
\\	屋根 発音図: 助詞: オ	屋根 やね		屋根 やね				名詞-普通名詞-一般 
\\	「濡れる(ぬれる)」のアクセントは?1871	ぬれる{濡} 仮名 
\\	ぬれる 漢字 
\\	濡れる 発音図:	濡れる ぬれる		濡れる ぬれる 濡れます ぬれます 濡れて ぬれて 濡れた ぬれた 濡れない ぬれない 濡れなかった ぬれなかった 濡れれば ぬれれば 濡れさせる ぬれさせる 濡れられる ぬれられる 濡れろ ぬれろ 濡れられる ぬれられる 濡れよう ぬれよう				動詞-一般 
\\	「冷蔵庫(れいぞうこ)」のアクセントは?1872	冷蔵庫 仮名 
\\	れいぞうこ、れいぞーこ 漢字 
\\	冷蔵庫 発音図: 助詞: オ	冷蔵庫 れいぞうこ		冷蔵庫 れいぞうこ				名詞-普通名詞-一般 
\\	「噛む(かむ)」のアクセントは?1873	かむ{噛} 仮名 
\\	かむ 漢字 
\\	噛む 発音図:	噛む かむ		噛む かむ 噛みます かみます 噛んで かんで 噛んだ かんだ 噛まない かまない 噛まなかった かまなかった 噛めば かめば 噛ませる かませる 噛まれる かまれる 噛め かめ 噛める かめる 噛もう かもう				動詞-一般 
\\	「医療(いりょう)」のアクセントは?1874	医療 仮名 
\\	いりょう、いりょー 漢字 
\\	医療 発音図: 助詞: オ 発音図: 助詞: オ	医療 いりょう		医療 いりょう				名詞-普通名詞-一般 
\\	「適正(てきせい)」のアクセントは?1875	適正 仮名 
\\	てきせい 漢字 
\\	適正 発音図: 助詞: オ							形状詞-一般 
\\	「取り戻す(とりもどす)」のアクセントは?1876	取り戻す 仮名 
\\	とりもどす 漢字 
\\	取り戻す 発音図: 発音図:	取り戻す とりもどす とりもどす		取り戻す とりもどす とりもどす 取り戻します とりもどします 取り戻して とりもどして とりもどして 取り戻した とりもどした とりもどした 取り戻さない とりもどさない とりもどさない 取り戻さなかった とりもどさなかった とりもどさなかった 取り戻せば とりもどせば とりもどせば 取り戻させる とりもどさせる とりもどさせる 取り戻される とりもどされる とりもどされる 取り戻せ とりもどせ とりもどせ 取り戻せる とりもどせる とりもどせる 取り戻そう とりもどそう				動詞-一般 
\\	「支援(しえん)」のアクセントは?1877	支援 仮名 
\\	しえん 漢字 
\\	支援 発音図: 助詞: オ 発音図: 助詞: オ	支援 しえん しえん		支援 しえん しえん 支援する しえんする 支援します しえんします 支援して しえんして 支援した しえんした 支援しない しえんしない 支援しなかった しえんしなかった 支援すれば しえんすれば 支援させる しえんさせる 支援される しえんされる 支援しろ しえんしろ 支援できる しえんできる 支援しよう しえんしよう				動詞-一般 
\\	「奇妙な(きみょうな)」のアクセントは?1878	奇妙 仮名 
\\	きみょう、きみょー 漢字 
\\	奇妙 発音図: 助詞: オ	奇妙な きみょうな		奇妙な きみょうな 奇妙です きみょうです 奇妙で きみょうで 奇妙だった きみょうだった 奇妙じゃない きみょうじゃない 奇妙じゃなかった きみょうじゃなかった 奇妙ならば きみょうならば 奇妙だ きみょうだ 奇妙に きみょうに				形状詞-一般 
\\	「正面(しょうめん)」のアクセントは?1879	正面 仮名 
\\	しょうめん、しょーめん 漢字 
\\	正面 発音図: 助詞: オ	正面 しょうめん		正面 しょうめん				名詞-普通名詞-一般 
\\	「生き方(いきかた)」のアクセントは?1880	生き方 仮名 
\\	いきかた 漢字 
\\	生き方 発音図: 例文: これが、私のデす 発音図: 例文: これが、私のデす							名詞-普通名詞-一般 
\\	「国際的な(こくさいてきな)」のアクセントは?1881	国際的 仮名 
\\	こくさいてき 漢字 
\\	国際的 発音図:	国際的な こくさいてきな		国際的な こくさいてきな 国際的です こくさいてきです 国際的で こくさいてきで 国際的だった こくさいてきだった 国際的じゃない こくさいてきじゃない 国際的じゃなかった こくさいてきじゃなかった 国際的ならば こくさいてきならば 国際的だ こくさいてきだ 国際的に こくさいてきに				形状詞-一般 
\\	「相当(そうとう)」のアクセントは?1882	相当 仮名 
\\	そうとう、そーとー 漢字 
\\	相当 発音図: 助詞: オ	相当 そうとう		相当 そうとう 相当する そうとうする 相当します そうとうします 相当して そうとうして 相当した そうとうした 相当しない そうとうしない 相当しなかった そうとうしなかった 相当すれば そうとうすれば 相当させる そうとうさせる 相当される そうとうされる 相当しろ そうとうしろ 相当できる そうとうできる 相当しよう そうとうしよう				動詞-一般 
\\	「飼う(かう)」のアクセントは?1883	飼う 仮名 
\\	かう 漢字 
\\	飼う 発音図:	飼う かう		飼う かう 飼います かいます 飼って かって 飼った かった 飼わない かわない 飼わなかった かわなかった 飼えば かえば 飼わせる かわせる 飼われる かわれる 飼え かえ 飼える かえる 飼おう かおう				動詞-一般 
\\	「太い(ふとい)」のアクセントは?1884	太い 仮名 
\\	ふとい 漢字 
\\	太い 発音図:	太い ふとい		太い ふとい 太いです ふといです 太くて ふとくて ふとくて 太かった ふとかった ふとかった 太くない ふとくない ふとくない 太くなかった ふとくなかった ふとくなかった 太ければ ふとければ ふとければ 太い ふとい 太く ふとく ふとく				形容詞-一般 
\\	「連絡(れんらく)」のアクセントは?1885	連絡 仮名 
\\	れんらく 漢字 
\\	連絡 発音図: 助詞: オ	連絡 れんらく		連絡 れんらく 連絡する れんらくする 連絡します れんらくします 連絡して れんらくして 連絡した れんらくした 連絡しない れんらくしない 連絡しなかった れんらくしなかった 連絡すれば れんらくすれば 連絡させる れんらくさせる 連絡される れんらくされる 連絡しろ れんらくしろ 連絡できる れんらくできる 連絡しよう れんらくしよう				動詞-一般 
\\	「常識(じょうしき)」のアクセントは?1886	常識 仮名 
\\	じょうしき、じょーしき 漢字 
\\	常識 発音図: 助詞: オ	常識 じょうしき		常識 じょうしき				名詞-普通名詞-一般 
\\	「各地(かくち)」のアクセントは?1887	各地 仮名 
\\	かくち 漢字 
\\	各地 発音図: 助詞: オ	各地 かくち		各地 かくち				名詞-普通名詞-一般 
\\	「図表(ずひょう)」のアクセントは?1888	図表 仮名 
\\	ずひょう、ずひょー 漢字 
\\	図表 発音図: 助詞: オ	図表 ずひょう		図表 ずひょう				名詞-普通名詞-一般 
\\	「謎(なぞ)」のアクセントは?1889	なぞ{謎} 仮名 
\\	なぞ 漢字 
\\	謎 発音図: 助詞: オ	謎 なぞ		謎 なぞ				名詞-普通名詞-一般 
\\	「頬(ほお)」のアクセントは?1890	ほお{頬} 仮名 
\\	ほお、ほー 漢字 
\\	頬 発音図: 助詞: オ							名詞-普通名詞-一般 
\\	「大匙(おーさじ)」のアクセントは?1891	大さじ{匙} 仮名 
\\	おーさじ、おおさじ 漢字 
\\	大匙、大さじ 発音図: 助詞: オ							名詞-普通名詞-一般 
\\	「時計(とけい)」のアクセントは?1892	時計 仮名 
\\	とけい 漢字 
\\	時計 発音図: 助詞: オ	時計 とけい		時計 とけい				名詞-普通名詞-一般 
\\	「痩せる(やせる)」のアクセントは?1893	やせる{痩} 仮名 
\\	やせる 漢字 
\\	痩せる 発音図:	痩せる やせる		痩せる やせる 痩せます やせます 痩せて やせて 痩せた やせた 痩せない やせない 痩せなかった やせなかった 痩せれば やせれば 痩せさせる やせさせる 痩せられる やせられる 痩せろ やせろ 痩せられる やせられる 痩せよう やせよう				動詞-一般 
\\	「我慢(がまん)」のアクセントは?1894	我慢 仮名 
\\	がまん 漢字 
\\	我慢 発音図: 助詞: オ	我慢 がまん		我慢 がまん 我慢する がまんする 我慢します がまんします 我慢して がまんして 我慢した がまんした 我慢しない がまんしない 我慢しなかった がまんしなかった 我慢すれば がまんすれば 我慢させる がまんさせる 我慢される がまんされる 我慢しろ がまんしろ 我慢できる がまんできる 我慢しよう がまんしよう				動詞-一般 
\\	「権力(けんりょく)」のアクセントは?1895	権力 仮名 
\\	けんりょく 漢字 
\\	権力 発音図: 助詞: オ	権力 けんりょく		権力 けんりょく				名詞-普通名詞-一般 
\\	「砂糖(さとう)」のアクセントは?1896	砂糖 仮名 
\\	さとう、さとー 漢字 
\\	砂糖 発音図: 助詞: オ	砂糖 さとう		砂糖 さとう				名詞-普通名詞-一般 
\\	「充実(じゅうじつ)」のアクセントは?1897	充実 仮名 
\\	じゅーじつ、じゅうじつ 漢字 
\\	充実 発音図: 助詞: オ	充実 じゅうじつ		充実 じゅうじつ 充実する じゅうじつする 充実します じゅうじつします 充実して じゅうじつして 充実した じゅうじつした 充実しない じゅうじつしない 充実しなかった じゅうじつしなかった 充実すれば じゅうじつすれば 充実させる じゅうじつさせる 充実される じゅうじつされる 充実しろ じゅうじつしろ 充実できる じゅうじつできる 充実しよう じゅうじつしよう				動詞-一般 
\\	「大正(たいしょう)」のアクセントは?1898	大正 仮名 
\\	たいしょう、たいしょー 漢字 
\\	大正 発音図: 助詞: オ	大正 たいしょう		大正 たいしょう				名詞-固有名詞-一般 
\\	「永遠(えいえん)」のアクセントは?1899	永遠 仮名 
\\	えいえん 漢字 
\\	永遠 発音図: 助詞: オ	永遠 えいえん		永遠 えいえん				名詞-普通名詞-一般 
\\	「埋める(うめる)」のアクセントは?1901	埋める 仮名 
\\	うめる 漢字 
\\	埋める 発音図:	埋める うめる		埋める うめる 埋めます うめます 埋めて うめて 埋めた うめた 埋めない うめない 埋めなかった うめなかった 埋めれば うめれば 埋めさせる うめさせる 埋められる うめられる 埋めろ うめろ 埋められる うめられる 埋めよう うめよう				動詞-一般 
\\	「感想(かんそう)」のアクセントは?1902	感想 仮名 
\\	かんそう、かんそー 漢字 
\\	感想 発音図: 助詞: オ	感想 かんそう		感想 かんそう				名詞-普通名詞-一般 
\\	「畑(はたけ)」のアクセントは?1903	畑 仮名 
\\	はた、はたけ 漢字 
\\	畑 発音図: 助詞: オ 例文: 山のオ耕す 発音図: 例文: 山のオ耕す 発音図:	畑 はたけ		畑 はたけ				名詞-普通名詞-一般 
\\	「糸(いと)」のアクセントは?1904	糸 仮名 
\\	いと 漢字 
\\	糸 発音図: 助詞: オ	糸 いと		糸 いと				名詞-普通名詞-一般 
\\	「漢字(かんじ)」のアクセントは?1905	漢字 仮名 
\\	かんじ 漢字 
\\	漢字 発音図: 助詞: オ	漢字 かんじ		漢字 かんじ				名詞-普通名詞-一般 
\\	「草(くさ)」のアクセントは?1906	草 仮名 
\\	くさ、そう、そー 漢字 
\\	草 発音図: 助詞: オ 発音図:	草 くさ		草 くさ				名詞-普通名詞-一般 
\\	「脚(あし)」のアクセントは?1907	脚 仮名 
\\	あし、きゃく 漢字 
\\	脚 発音図: 助詞: オ 発音図: 助詞: オ	脚 あし		脚 あし				名詞-普通名詞-一般 
\\	「成績(せいせき)」のアクセントは?1908	成績 仮名 
\\	せいせき 漢字 
\\	成績 発音図: 助詞: オ	成績 せいせき		成績 せいせき				名詞-普通名詞-一般 
\\	「ポケット(ぽけっと)」のアクセントは?1909	ポケット 仮名 
\\	ポケット、ぽけっと 漢字 
\\	発音図: 助詞: オ 発音図: 助詞: オ	ポケット ぽけっと ぽけっと		ポケット ぽけっと ぽけっと				名詞-普通名詞-一般 
\\	「拘泥(こだわる)」のアクセントは?1910	こだわる{拘泥} 仮名 
\\	こだわる 漢字 
\\	拘泥 発音図:	こだわる こだわる		こだわる こだわる こだわります こだわります こだわって こだわって こだわった こだわった こだわらない こだわらない こだわらなかった こだわらなかった こだわれば こだわれば こだわらせる こだわらせる こだわられる こだわられる こだわれ こだわれ こだわれる こだわれる こだわろう こだわろう				動詞-一般 
\\	「パターン(ぱたーん)」のアクセントは?1911	パターン 仮名 
\\	パターン、ぱたーん 漢字 
\\	発音図:	パターン ぱたーん		パターン ぱたーん				名詞-普通名詞-一般 
\\	「素早い(すばやい)」のアクセントは?1912		素早い すばやい		素早い すばやい 素早いです すばやいです 素早くて すばやくて すばやくて 素早かった すばやかった すばやかった 素早くない すばやくない すばやくない 素早くなかった すばやくなかった すばやくなかった 素早ければ すばやければ すばやければ 素早い すばやい 素早く すばやく すばやく				形容詞-一般 
\\	「主人公(しゅじんこう)」のアクセントは?1913	主人公 仮名 
\\	しゅじんこう、しゅじんこー 漢字 
\\	主人公 発音図: 助詞: オ	主人公 しゅじんこう		主人公 しゅじんこう				名詞-普通名詞-一般 
\\	「産業(さんぎょう)」のアクセントは?1914	産業 仮名 
\\	さんぎょう、さんぎょー 漢字 
\\	産業 発音図: 助詞: オ	産業 さんぎょう		産業 さんぎょう				名詞-普通名詞-一般 
\\	「連携(れんけい)」のアクセントは?1915	連携 仮名 
\\	れんけい 漢字 
\\	連携 発音図: 助詞: オ							名詞-普通名詞-一般 
\\	「生活(せいかつ)」のアクセントは?1916	生活 仮名 
\\	せいかつ 漢字 
\\	生活 発音図: 助詞: オ	生活 せいかつ		生活 せいかつ 生活する せいかつする 生活します せいかつします 生活して せいかつして 生活した せいかつした 生活しない せいかつしない 生活しなかった せいかつしなかった 生活すれば せいかつすれば 生活させる せいかつさせる 生活される せいかつされる 生活しろ せいかつしろ 生活できる せいかつできる 生活しよう せいかつしよう				動詞-一般 
\\	「想定(そうていする)」のアクセントは?1917	想定 仮名 
\\	そうてい、そーてい 漢字 
\\	想定 発音図: 助詞: オ	想定する そうていする		想定する そうていする 想定します そうていします 想定して そうていして 想定した そうていした 想定しない そうていしない 想定しなかった そうていしなかった 想定すれば そうていすれば 想定させる そうていさせる 想定される そうていされる 想定しろ そうていしろ 想定できる そうていできる 想定しよう そうていしよう				動詞-一般 
\\	「所得(しょとく)」のアクセントは?1918	所得 仮名 
\\	しょとく 漢字 
\\	所得 発音図: 助詞: オ	所得 しょとく しょとく		所得 しょとく しょとく				名詞-普通名詞-一般 
\\	「突っ込む(つっこむ)」のアクセントは?1919	突っ込む 仮名 
\\	つっこむ 漢字 
\\	突っ込む 発音図:	突っ込む つっこむ		突っ込む つっこむ 突っ込みます つっこみます 突っ込んで つっこんで 突っ込んだ つっこんだ 突っ込まない つっこまない 突っ込まなかった つっこまなかった 突っ込めば つっこめば 突っ込ませる つっこませる 突っ込まれる つっこまれる 突っ込め つっこめ 突っ込める つっこめる 突っ込もう つっこもう				動詞-一般 
\\	「地上(ちじょう)」のアクセントは?1920	地上 仮名 
\\	ちじょう、ちじょー 漢字 
\\	地上 発音図: 助詞: オ	地上 ちじょう		地上 ちじょう				名詞-普通名詞-一般 
\\	「金融機関(きんゆーきかん)」のアクセントは?1921	金融機関 仮名 
\\	きんゆーきかん、きんゆうきかん 漢字 
\\	金融機関 発音図: 助詞: オ 発音図: 助詞: オ							名詞-普通名詞-一般 
\\	「燃える(もえる)」のアクセントは?1922	燃える 仮名 
\\	もえる 漢字 
\\	燃える 発音図:	燃える もえる		燃える もえる 燃えます もえます 燃えて もえて 燃えた もえた 燃えない もえない 燃えなかった もえなかった 燃えれば もえれば 燃えさせる もえさせる 燃えられる もえられる 燃えろ もえろ 燃えよう もえよう				動詞-一般 
\\	「思い切る(おもいきる)」のアクセントは?1923	思い切る 仮名 
\\	おもいきる 漢字 
\\	思い切る 発音図: 発音図: 発音図:	思い切る おもいきる		思い切る おもいきる 思い切ります おもいきります 思い切って おもいきって 思い切った おもいきった 思い切らない おもいきらない 思い切らなかった おもいきらなかった 思い切れば おもいきれば 思い切らせる おもいきらせる 思い切られる おもいきられる 思い切れ おもいきれ 思い切れる おもいきれる 思い切ろう おもいきろう				動詞-一般 
\\	「重なる(かさなる)」のアクセントは?1924	重なる 仮名 
\\	かさなる 漢字 
\\	重なる 発音図:	重なる かさなる		重なる かさなる 重なります かさなります 重なって かさなって 重なった かさなった 重ならない かさならない 重ならなかった かさならなかった 重なれば かさなれば 重ならせる かさならせる 重なられる かさなられる 重なれ かさなれ 重なれる かさなれる 重なろう かさなろう				動詞-一般 
\\	「眠い(ねむい)」のアクセントは?1925	眠い 仮名 
\\	ねむい 漢字 
\\	眠い 発音図:	眠い ねむい		眠い ねむい 眠いです ねむいです 眠くて ねむくて 眠かった ねむかった 眠くない ねむくない 眠くなかった ねむくなかった 眠ければ ねむければ 眠い ねむい ねむい 眠く ねむく				形容詞-一般 
\\	「高校生(こうこうせい)」のアクセントは?1926	高校生 仮名 
\\	こうこうせい、こーこーせい 漢字 
\\	高校生 発音図: 助詞: オ	高校生 こうこうせい		高校生 こうこうせい				名詞-普通名詞-一般 
\\	「携帯電話(けいたいでんわ)」のアクセントは?1927	携帯電話 仮名 
\\	けいたいでんわ 漢字 
\\	携帯電話 発音図: 助詞: オ	携帯電話 けいたいでんわ		携帯電話 けいたいでんわ				名詞-普通名詞-一般 
\\	「此奴(こいつ)」のアクセントは?1928	こいつ{此奴} 仮名 
\\	こいつ 漢字 
\\	此奴 発音図: 助詞: オ							代名詞 
\\	「検索(けんさく)」のアクセントは?1929	検索 仮名 
\\	けんさく 漢字 
\\	検索 発音図: 助詞: オ	検索 けんさく		検索 けんさく 検索する けんさくする 検索します けんさくします 検索して けんさくして 検索した けんさくした 検索しない けんさくしない 検索しなかった けんさくしなかった 検索すれば けんさくすれば 検索させる けんさくさせる 検索される けんさくされる 検索しろ けんさくしろ 検索できる けんさくできる 検索しよう けんさくしよう				動詞-一般 
\\	「深める(ふかめる)」のアクセントは?1930	深める 仮名 
\\	ふかめる 漢字 
\\	深める 発音図:	深める ふかめる		深める ふかめる 深めます ふかめます 深めて ふかめて 深めた ふかめた 深めない ふかめない 深めなかった ふかめなかった 深めれば ふかめれば 深めさせる ふかめさせる 深められる ふかめられる 深めろ ふかめろ 深められる ふかめられる 深めよう ふかめよう				動詞-一般 
\\	「バッグ(ばっぐ)」のアクセントは?1931	バッグ 仮名 
\\	バッグ、ばっぐ 漢字 
\\	発音図: 助詞: オ	バッグ ばっぐ		バッグ ばっぐ				名詞-普通名詞-一般 
\\	「組み合わせ(くみあわせ)」のアクセントは?1932	組み合わせ 仮名 
\\	くみあわせ 漢字 
\\	組み合わせ 発音図: 助詞: オ	組み合わせ くみあわせ		組み合わせ くみあわせ				名詞-普通名詞-一般 
\\	「問い(とい)」のアクセントは?1933	問い 仮名 
\\	とい 漢字 
\\	問い 発音図: 助詞: オ							名詞-普通名詞-一般 
\\	「脱ぐ(ぬぐ)」のアクセントは?1934	脱ぐ 仮名 
\\	ぬぐ 漢字 
\\	脱ぐ 発音図:	脱ぐ ぬぐ		脱ぐ ぬぐ 脱ぎます ぬぎます 脱いで ぬいで 脱いだ ぬいだ 脱がない ぬがない 脱がなかった ぬがなかった 脱げば ぬげば 脱がせる ぬがせる 脱がれる ぬがれる 脱げ ぬげ 脱げる ぬげる 脱ごう ぬごう				動詞-一般 
\\	「何事(なにごと)」のアクセントは?1935	何事 仮名 
\\	なにごと 漢字 
\\	何事 発音図: 助詞: オ							名詞-普通名詞-一般 
\\	「武器(ぶき)」のアクセントは?1936	武器 仮名 
\\	ぶき 漢字 
\\	武器 発音図: 助詞: オ	武器 ぶき		武器 ぶき				名詞-普通名詞-一般 
\\	「アパート(あぱーと)」のアクセントは?1937	アパート 仮名 
\\	アパート、あぱーと 漢字 
\\	発音図: 助詞: オ	アパート あぱーと		アパート あぱーと				名詞-普通名詞-一般 
\\	「割る(わる)」のアクセントは?1938	割る 仮名 
\\	わる 漢字 
\\	割る 発音図:	割る わる		割る わる 割ります わります 割って わって 割った わった 割らない わらない 割らなかった わらなかった 割れば われば 割らせる わらせる 割られる わられる 割れ われ 割れる われる 割ろう わろう				動詞-一般 
\\	「図書館(としょかん)」のアクセントは?1939	図書館 仮名 
\\	としょかん 漢字 
\\	図書館 発音図: 助詞: オ	図書館 としょかん		図書館 としょかん				名詞-普通名詞-一般 
\\	「皮膚(ひふ)」のアクセントは?1940	皮膚 仮名 
\\	ひふ 漢字 
\\	皮膚 発音図: 助詞: オ 発音図: 助詞: オ	皮膚 ひふ		皮膚 ひふ				名詞-普通名詞-一般 
\\	「真ん中(まんなか)」のアクセントは?1941	真ん中 仮名 
\\	まんなか 漢字 
\\	真ん中 発音図: 助詞: オ	真ん中 まんなか		真ん中 まんなか				名詞-普通名詞-一般 
\\	「提案(ていあんする)」のアクセントは?1942	提案 仮名 
\\	ていあん 漢字 
\\	提案 発音図: 助詞: オ	提案する ていあんする		提案する ていあんする 提案します ていあんします 提案して ていあんして 提案した ていあんした 提案しない ていあんしない 提案しなかった ていあんしなかった 提案すれば ていあんすれば 提案させる ていあんさせる 提案される ていあんされる 提案しろ ていあんしろ 提案できる ていあんできる 提案しよう ていあんしよう				動詞-一般 
\\	「有無(うむ)」のアクセントは?1943	有無 仮名 
\\	うむ 漢字 
\\	有無 発音図: 助詞: オ	有無 うむ		有無 うむ				名詞-普通名詞-一般 
\\	「詰める(つめる)」のアクセントは?1944	詰める 仮名 
\\	つめる 漢字 
\\	詰める 発音図:	詰める つめる		詰める つめる 詰めます つめます 詰めて つめて 詰めた つめた 詰めない つめない 詰めなかった つめなかった 詰めれば つめれば 詰めさせる つめさせる 詰められる つめられる 詰めろ つめろ 詰められる つめられる 詰めよう つめよう				動詞-一般 
\\	「意義(いぎ)」のアクセントは?1945	意義 仮名 
\\	いぎ 漢字 
\\	意義 発音図: 助詞: オ	意義 いぎ		意義 いぎ				名詞-普通名詞-一般 
\\	「中学生(ちゅうがくせい)」のアクセントは?1946	中学生 仮名 
\\	ちゅーがくせい、ちゅうがくせい 漢字 
\\	中学生 発音図: 助詞: オ 発音図: 助詞: オ	中学生 ちゅうがくせい ちゅうがくせい		中学生 ちゅうがくせい ちゅうがくせい				名詞-普通名詞-一般 
\\	「証明(しょうめい)」のアクセントは?1947	証明 仮名 
\\	しょうめい、しょーめい 漢字 
\\	証明 発音図: 助詞: オ	証明 しょうめい		証明 しょうめい 証明する しょうめいする 証明します しょうめいします 証明して しょうめいして 証明した しょうめいした 証明しない しょうめいしない 証明しなかった しょうめいしなかった 証明すれば しょうめいすれば 証明させる しょうめいさせる 証明される しょうめいされる 証明しろ しょうめいしろ 証明できる しょうめいできる 証明しよう しょうめいしよう				動詞-一般 
\\	「獲得(かくとくする)」のアクセントは?1948	獲得 仮名 
\\	かくとく 漢字 
\\	獲得 発音図: 助詞: オ	獲得する かくとくする		獲得する かくとくする 獲得します かくとくします 獲得して かくとくして 獲得した かくとくした 獲得しない かくとくしない 獲得しなかった かくとくしなかった 獲得すれば かくとくすれば 獲得させる かくとくさせる 獲得される かくとくされる 獲得しろ かくとくしろ 獲得できる かくとくできる 獲得しよう かくとくしよう				動詞-一般 
\\	「煮る(にる)」のアクセントは?1949	煮る 仮名 
\\	にる 漢字 
\\	煮る 発音図:	煮る にる		煮る にる 煮ます にます 煮て にて 煮た にた 煮ない にない 煮なかった になかった 煮れば にれば 煮させる にさせる 煮られる にられる 煮ろ にろ 煮られる にられる 煮よう によう				動詞-一般 
\\	「要件(ようけん)」のアクセントは?1950	要件 仮名 
\\	ようけん、よーけん 漢字 
\\	要件 発音図: 助詞: オ							名詞-普通名詞-一般 
\\	「営む(いとなむ)」のアクセントは?1951	営む 仮名 
\\	いとなむ 漢字 
\\	営む 発音図:	営む いとなむ		営む いとなむ 営みます いとなみます 営んで いとなんで 営んだ いとなんだ 営まない いとなまない 営まなかった いとなまなかった 営めば いとなめば 営ませる いとなませる 営まれる いとなまれる 営め いとなめ 営める いとなめる 営もう いとなもう				動詞-一般 
\\	「日記(にっき)」のアクセントは?1952	日記 仮名 
\\	にっき 漢字 
\\	日記 発音図: 助詞: オ	日記 にっき		日記 にっき				名詞-普通名詞-一般 
\\	「疑い(うたがい)」のアクセントは?1953	疑い 仮名 
\\	うたがい 漢字 
\\	疑い 発音図: 助詞: オ	疑い うたがい		疑い うたがい				名詞-普通名詞-一般 
\\	「幕府(ばくふ)」のアクセントは?1954	幕府 仮名 
\\	ばくふ 漢字 
\\	幕府 発音図: 助詞: オ 発音図: 助詞: オ							名詞-普通名詞-一般 
\\	「飛び込む(とびこむ)」のアクセントは?1955	飛び込む 仮名 
\\	とびこむ 漢字 
\\	飛び込む 発音図:	飛び込む とびこむ		飛び込む とびこむ 飛び込みます とびこみます 飛び込んで とびこんで 飛び込んだ とびこんだ 飛び込まない とびこまない 飛び込まなかった とびこまなかった 飛び込めば とびこめば 飛び込ませる とびこませる 飛び込まれる とびこまれる 飛び込め とびこめ 飛び込める とびこめる 飛び込もう とびこもう				動詞-一般 
\\	「出発(しゅっぱつ)」のアクセントは?1956	出発 仮名 
\\	しゅっぱつ 漢字 
\\	出発 発音図: 助詞: オ	出発 しゅっぱつ		出発 しゅっぱつ 出発する しゅっぱつする 出発します しゅっぱつします 出発して しゅっぱつして 出発した しゅっぱつした 出発しない しゅっぱつしない 出発しなかった しゅっぱつしなかった 出発すれば しゅっぱつすれば 出発させる しゅっぱつさせる 出発される しゅっぱつされる 出発しろ しゅっぱつしろ 出発できる しゅっぱつできる 出発しよう しゅっぱつしよう				動詞-一般 
\\	「上昇(じょうしょう)」のアクセントは?1957	上昇 仮名 
\\	じょうしょう、じょーしょー 漢字 
\\	上昇 発音図: 助詞: オ	上昇 じょうしょう		上昇 じょうしょう				名詞-普通名詞-一般 
\\	「兄弟(きょうだい)」のアクセントは?1958	兄弟 仮名 
\\	きょうだい、きょーだい、けいてい 漢字 
\\	兄弟 発音図: 助詞: オ 発音図: 助詞: オ	兄弟 きょうだい		兄弟 きょうだい				名詞-普通名詞-一般 
\\	「剣(けん)」のアクセントは?1959	剣 仮名 
\\	けん、つるぎ 漢字 
\\	剣 発音図: 助詞: オ 発音図: 助詞: オ							名詞-普通名詞-一般 
\\	「メディア(めでぃあ)」のアクセントは?1960	メディア 仮名 
\\	メディア、めでぃあ 漢字 
\\	発音図: 助詞: オ	メディア めでぃあ		メディア めでぃあ				名詞-普通名詞-一般 
\\	「関係者(かんけいしゃ)」のアクセントは?1961	関係者 仮名 
\\	かんけいしゃ 漢字 
\\	関係者 発音図: 助詞: オ							名詞-普通名詞-一般 
\\	「ニューヨーク(にゅーよーく)」のアクセントは?1962		ニューヨーク にゅーよーく		ニューヨーク にゅーよーく				名詞-固有名詞-地名-一般 
\\	「大部分(だいぶぶん)」のアクセントは?1963	大部分 仮名 
\\	だいぶぶん 漢字 
\\	大部分 発音図: 助詞: オ	大部分 だいぶぶん		大部分 だいぶぶん				名詞-普通名詞-一般 
\\	「刻む(きざむ)」のアクセントは?1964	刻む 仮名 
\\	きざむ 漢字 
\\	刻む 発音図:	刻む きざむ		刻む きざむ 刻みます きざみます 刻んで きざんで 刻んだ きざんだ 刻まない きざまない 刻まなかった きざまなかった 刻めば きざめば 刻ませる きざませる 刻まれる きざまれる 刻め きざめ 刻める きざめる 刻もう きざもう				動詞-一般 
\\	「炒める(いためる)」のアクセントは?1965	いためる{炒} 仮名 
\\	いためる 漢字 
\\	炒める 発音図:	炒める いためる		炒める いためる 炒めます いためます 炒めて いためて 炒めた いためた 炒めない いためない 炒めなかった いためなかった 炒めれば いためれば 炒めさせる いためさせる 炒められる いためられる 炒めろ いためろ 炒められる いためられる 炒めよう いためよう				動詞-一般 
\\	「一瞬(いっしゅん)」のアクセントは?1966	一瞬 仮名 
\\	いっしゅん 漢字 
\\	一瞬 発音図: 助詞: オ	一瞬 いっしゅん		一瞬 いっしゅん				名詞-普通名詞-一般 
\\	「感謝(かんしゃ)」のアクセントは?1967	感謝 仮名 
\\	かんしゃ 漢字 
\\	感謝 発音図: 助詞: オ	感謝 かんしゃ		感謝 かんしゃ 感謝する かんしゃする 感謝します かんしゃします 感謝して かんしゃして 感謝した かんしゃした 感謝しない かんしゃしない 感謝しなかった かんしゃしなかった 感謝すれば かんしゃすれば 感謝させる かんしゃさせる 感謝される かんしゃされる 感謝しろ かんしゃしろ 感謝できる かんしゃできる 感謝しよう かんしゃしよう				動詞-一般 
\\	「経営者(けいえいしゃ)」のアクセントは?1968	経営者 仮名 
\\	けいえいしゃ 漢字 
\\	経営者 発音図: 助詞: オ	経営者 けいえいしゃ		経営者 けいえいしゃ				名詞-普通名詞-一般 
\\	「コミュニケーション(こみゅにけーしょん)」のアクセントは?1969	コミュニケーション 仮名 
\\	コミュニケーション、こみゅにけーしょん 漢字 
\\	発音図: 助詞: オ	コミュニケーション こみゅにけーしょん		コミュニケーション こみゅにけーしょん				名詞-普通名詞-一般 
\\	「ネットワーク(ねっとわーく)」のアクセントは?1970	ネットワーク 仮名 
\\	ネットワーク、ねっとわーく 漢字 
\\	発音図: 助詞: オ	ネットワーク ねっとわーく		ネットワーク ねっとわーく				名詞-普通名詞-一般 
\\	「拭く(ふく)」のアクセントは?1971	ふく{拭;葺} 仮名 
\\	ふく 漢字 
\\	拭く、葺く 発音図: 例文: 雑巾でニは力が要る	拭く ふく		拭く ふく 拭きます ふきます 拭いて ふいて 拭いた ふいた 拭かない ふかない 拭かなかった ふかなかった 拭けば ふけば 拭かせる ふかせる 拭かれる ふかれる 拭け ふけ 拭ける ふける 拭こう ふこう				動詞-一般 
\\	「親子(おやこ)」のアクセントは?1972	親子 仮名 
\\	おやこ、しんし 漢字 
\\	親子 発音図: 助詞: オ 発音図: 助詞: オ	親子 おやこ		親子 おやこ				名詞-普通名詞-一般 
\\	「交わす(かわす)」のアクセントは?1973	交わす 仮名 
\\	かわす 漢字 
\\	交わす 発音図:	交わす かわす		交わす かわす 交わします かわします 交わして かわして 交わした かわした 交わさない かわさない 交わさなかった かわさなかった 交わせば かわせば 交わさせる かわさせる 交わされる かわされる 交わせ かわせ 交わせる かわせる 交わそう かわそう				動詞-一般 
\\	「愛情(あいじょう)」のアクセントは?1974	愛情 仮名 
\\	あいじょう、あいじょー 漢字 
\\	愛情 発音図: 助詞: オ	愛情 あいじょう		愛情 あいじょう				名詞-普通名詞-一般 
\\	「拙い(まずい)」のアクセントは?1975	まずい{拙;不味} 仮名 
\\	まずい 漢字 
\\	不味い、拙い 発音図:	不味い まずい		不味い まずい 不味いです まずいです 不味くて まずくて まずくて 不味かった まずかった まずかった 不味くない まずくない まずくない 不味くなかった まずくなかった まずくなかった 不味ければ まずければ まずければ 不味い まずい 不味く まずく まずく				形容詞-一般 
\\	「崩れる(くずれる)」のアクセントは?1976	崩れる 仮名 
\\	くずれる 漢字 
\\	崩れる 発音図:	崩れる くずれる		崩れる くずれる 崩れます くずれます 崩れて くずれて 崩れた くずれた 崩れない くずれない 崩れなかった くずれなかった 崩れれば くずれれば 崩れさせる くずれさせる 崩れられる くずれられる 崩れろ くずれろ 崩れられる くずれられる 崩れよう くずれよう				動詞-一般 
\\	「柔らかい(やわらかい)」のアクセントは?1977	柔らかい 仮名 
\\	やわらかい 漢字 
\\	柔らかい 発音図:	柔らかい やわらかい		柔らかい やわらかい 柔らかいです やわらかいです 柔らかくて やわらかくて やわらかくて 柔らかかった やわらかかった やわらかかった 柔らかくない やわらかくない やわらかくない 柔らかくなかった やわらかくなかった やわらかくなかった 柔らかければ やわらかければ やわらかければ 柔らかい やわらかい 柔らかく やわらかく やわらかく				形容詞-一般 
\\	「拾う(ひろう)」のアクセントは?1978	拾う 仮名 
\\	ひろう 漢字 
\\	拾う 発音図:	拾う ひろう		拾う ひろう 拾います ひろいます 拾って ひろって 拾った ひろった 拾わない ひろわない 拾わなかった ひろわなかった 拾えば ひろえば 拾わせる ひろわせる 拾われる ひろわれる 拾え ひろえ 拾える ひろえる 拾おう ひろおう				動詞-一般 
\\	「顧客(こかく)」のアクセントは?1979	顧客 仮名 
\\	こかく、こきゃく 漢字 
\\	顧客 発音図: 助詞: オ 発音図: 助詞: オ							名詞-普通名詞-一般 
\\	「見直す(みなおす)」のアクセントは?1980	見直す 仮名 
\\	みなおす 漢字 
\\	見直す 発音図: 発音図:	見直す みなおす みなおす		見直す みなおす みなおす 見直します みなおします 見直して みなおして みなおして 見直した みなおした みなおした 見直さない みなおさない みなおさない 見直さなかった みなおさなかった みなおさなかった 見直せば みなおせば みなおせば 見直させる みなおさせる みなおさせる 見直される みなおされる みなおされる 見直せ みなおせ みなおせ 見直せる みなおせる みなおせる 見直そう みなおそう				動詞-一般 
\\	「模様(もよう)」のアクセントは?1981	模様 仮名 
\\	もよう、もよー 漢字 
\\	模様 発音図: 助詞: オ	模様 もよう		模様 もよう				名詞-普通名詞-一般 
\\	「反する(はんする)」のアクセントは?1982	反する 仮名 
\\	はんする 漢字 
\\	反する 発音図:	反する はんする		反する はんする 反します はんします 反して はんして 反した はんした 反しない はんしない 反しなかった はんしなかった 反すれば はんすれば 反させる はんさせる 反される はんされる 反しろ はんしろ 反しよう はんしよう				動詞-一般 
\\	「派遣(はけんする)」のアクセントは?1983	派遣 仮名 
\\	はけん 漢字 
\\	派遣 発音図: 助詞: オ	派遣する はけんする		派遣する はけんする 派遣します はけんします 派遣して はけんして 派遣した はけんした 派遣しない はけんしない 派遣しなかった はけんしなかった 派遣すれば はけんすれば 派遣させる はけんさせる 派遣される はけんされる 派遣しろ はけんしろ 派遣できる はけんできる 派遣しよう はけんしよう				動詞-一般 
\\	「前年度(ぜんねんど)」のアクセントは?1984	前年度 仮名 
\\	ぜんねんど 漢字 
\\	前年度 発音図: 助詞: オ							名詞-普通名詞-一般 
\\	「下す(くだす)」のアクセントは?1985	下す 仮名 
\\	くだす 漢字 
\\	下す 発音図:	下す くだす		下す くだす 下します くだします 下して くだして 下した くだした 下さない くださない 下さなかった くださなかった 下せば くだせば 下させる くださせる 下される くだされる 下せ くだせ 下せる くだせる 下そう くだそう				動詞-一般 
\\	「約束(やくそく)」のアクセントは?1986	約束 仮名 
\\	やくそく 漢字 
\\	約束 発音図: 助詞: オ	約束 やくそく		約束 やくそく 約束する やくそくする 約束します やくそくします 約束して やくそくして 約束した やくそくした 約束しない やくそくしない 約束しなかった やくそくしなかった 約束すれば やくそくすれば 約束させる やくそくさせる 約束される やくそくされる 約束しろ やくそくしろ 約束できる やくそくできる 約束しよう やくそくしよう				動詞-一般 
\\	「知り合い(しりあい)」のアクセントは?1987	知り合い 仮名 
\\	しりあい 漢字 
\\	知り合い 発音図: 助詞: オ	知り合い しりあい		知り合い しりあい				名詞-普通名詞-一般 
\\	「レース(れーす)
\\	のアクセントは?1988	レース
\\	仮名 
\\	レース、れーす 漢字 
\\	発音図: 助詞: オ	レース れーす		レース れーす				名詞-普通名詞-一般 
\\	「レース(レース)
\\	のアクセントは?1989	レース
\\	仮名 
\\	レース、れーす 漢字 
\\	発音図: 助詞: オ							名詞-普通名詞-一般 
\\	「配慮(はいりょする)」のアクセントは?1990	配慮 仮名 
\\	はいりょ 漢字 
\\	配慮 発音図: 助詞: オ	配慮する はいりょする		配慮する はいりょする 配慮します はいりょします 配慮して はいりょして 配慮した はいりょした 配慮しない はいりょしない 配慮しなかった はいりょしなかった 配慮すれば はいりょすれば 配慮させる はいりょさせる 配慮される はいりょされる 配慮しろ はいりょしろ 配慮できる はいりょできる 配慮しよう はいりょしよう				動詞-一般 
\\	「マスコミ(ますこみ)」のアクセントは?1991	マスコミ 仮名 
\\	マスコミ、ますこみ 漢字 
\\	発音図: 助詞: オ	マスコミ ますこみ		マスコミ ますこみ				名詞-普通名詞-一般 
\\	「有利(ゆうり)」のアクセントは?1992	有利 仮名 
\\	ゆーり、ゆうり 漢字 
\\	有利 発音図: 助詞: オ	有利 ゆうり		有利 ゆうり				形状詞-一般 
\\	「有利な(ゆうりな)」のアクセントは?1993		有利な ゆうりな		有利な ゆうりな 有利です ゆうりです 有利で ゆうりで 有利だった ゆうりだった 有利じゃない ゆうりじゃない 有利じゃなかった ゆうりじゃなかった 有利ならば ゆうりならば 有利だ ゆうりだ 有利に ゆうりに				形状詞-一般 
\\	「煩い(うるさい)」のアクセントは?1994	うるさい{煩} 仮名 
\\	うるさい 漢字 
\\	煩い 発音図:	うるさい うるさい		うるさい うるさい うるさいです うるさいです うるさくて うるさくて うるさくて うるさかった うるさかった うるさかった うるさくない うるさくない うるさくない うるさくなかった うるさくなかった うるさくなかった うるさければ うるさければ うるさければ うるさい うるさい うるさく うるさく うるさく				形容詞-一般 
\\	「温度(おんど)」のアクセントは?1995	温度 仮名 
\\	おんど 漢字 
\\	温度 発音図: 助詞: オ	温度 おんど		温度 おんど				名詞-普通名詞-一般 
\\	「御免(ごめん)」のアクセントは?1996	ごめん{御免} 仮名 
\\	ごめん 漢字 
\\	御免 発音図: 助詞: オ	御免 ごめん		御免 ごめん				名詞-普通名詞-一般 
\\	「才能(さいのう)」のアクセントは?1997	才能 仮名 
\\	さいのう、さいのー 漢字 
\\	才能 発音図: 助詞: オ	才能 さいのう		才能 さいのう				名詞-普通名詞-一般 
\\	「欠ける(かける)」のアクセントは?1998	欠ける 仮名 
\\	かける 漢字 
\\	欠ける 発音図:	欠ける かける		欠ける かける 欠けます かけます 欠けて かけて 欠けた かけた 欠けない かけない 欠けなかった かけなかった 欠ければ かければ 欠けさせる かけさせる 欠けられる かけられる 欠けろ かけろ 欠けよう かけよう				動詞-一般 
\\	「分析(ぶんせき)」のアクセントは?1999	分析 仮名 
\\	ぶんせき 漢字 
\\	分析 発音図: 助詞: オ	分析 ぶんせき		分析 ぶんせき 分析する ぶんせきする 分析します ぶんせきします 分析して ぶんせきして 分析した ぶんせきした 分析しない ぶんせきしない 分析しなかった ぶんせきしなかった 分析すれば ぶんせきすれば 分析させる ぶんせきさせる 分析される ぶんせきされる 分析しろ ぶんせきしろ 分析できる ぶんせきできる 分析しよう ぶんせきしよう				動詞-一般 
\\	「スーパー(すーぱー)」のアクセントは?2000	スーパー 仮名 
\\	スーパー、すーぱー 漢字 
\\	発音図: 助詞: オ	スーパー すーぱー		スーパー すーぱー				名詞-普通名詞-一般 
\\	「破る(やぶる)」のアクセントは?2001	破る 仮名 
\\	やぶる 漢字 
\\	破る 発音図:	破る やぶる		破る やぶる 破ります やぶります 破って やぶって 破った やぶった 破らない やぶらない 破らなかった やぶらなかった 破れば やぶれば 破らせる やぶらせる 破られる やぶられる 破れ やぶれ 破れる やぶれる 破ろう やぶろう				動詞-一般 
\\	「行動(こうどう)」のアクセントは?2002	行動 仮名 
\\	こうどう、こーどー 漢字 
\\	行動 発音図: 助詞: オ	行動 こうどう		行動 こうどう 行動する こうどうする 行動します こうどうします 行動して こうどうして 行動した こうどうした 行動しない こうどうしない 行動しなかった こうどうしなかった 行動すれば こうどうすれば 行動させる こうどうさせる 行動される こうどうされる 行動しろ こうどうしろ 行動できる こうどうできる 行動しよう こうどうしよう				動詞-一般 
\\	「差し出す(さしだす)」のアクセントは?2003	差し出す 仮名 
\\	さしだす 漢字 
\\	差し出す 発音図: 発音図:	差し出す さしだす		差し出す さしだす 差し出します さしだします 差し出して さしだして 差し出した さしだした 差し出さない さしださない 差し出さなかった さしださなかった 差し出せば さしだせば 差し出させる さしださせる 差し出される さしだされる 差し出せ さしだせ 差し出せる さしだせる 差し出そう さしだそう				動詞-一般 
\\	「離す(はなす)」のアクセントは?2004	離す 仮名 
\\	はなす 漢字 
\\	離す 発音図:	離す はなす		離す はなす 離します はなします 離して はなして 離した はなした 離さない はなさない 離さなかった はなさなかった 離せば はなせば 離させる はなさせる 離される はなされる 離せ はなせ 離せる はなせる 離そう はなそう				動詞-一般 
\\	「持ち込む(もちこむ)」のアクセントは?2005	持ち込む 仮名 
\\	もちこむ 漢字 
\\	持ち込む 発音図: 発音図:	持ち込む もちこむ もちこむ		持ち込む もちこむ もちこむ 持ち込みます もちこみます 持ち込んで もちこんで もちこんで 持ち込んだ もちこんだ もちこんだ 持ち込まない もちこまない もちこまない 持ち込まなかった もちこまなかった もちこまなかった 持ち込めば もちこめば もちこめば 持ち込ませる もちこませる もちこませる 持ち込まれる もちこまれる もちこまれる 持ち込め もちこめ もちこめ 持ち込める もちこめる もちこめる 持ち込もう もちこもう				動詞-一般 
\\	「組み合わせる(くみあわせる)」のアクセントは?2006	組み合わせる 仮名 
\\	くみあわせる 漢字 
\\	組み合わせる 発音図: 発音図:	組み合わせる くみあわせる		組み合わせる くみあわせる 組み合わせます くみあわせます 組み合わせて くみあわせて 組み合わせた くみあわせた 組み合わせない くみあわせない 組み合わせなかった くみあわせなかった 組み合わせれば くみあわせれば 組み合わせさせる くみあわせさせる 組み合わせられる くみあわせられる 組み合わせろ くみあわせろ 組み合わせられる くみあわせられる 組み合わせよう くみあわせよう				動詞-一般 
\\	「危機(きき)」のアクセントは?2007	危機 仮名 
\\	きき 漢字 
\\	危機 発音図: 助詞: オ 発音図: 助詞: オ	危機 きき きき		危機 きき きき				名詞-普通名詞-一般 
\\	「存ずる(ぞんずる)」のアクセントは?2008	存ずる 仮名 
\\	ぞんずる 漢字 
\\	存ずる 発音図: 発音図:	存ずる ぞんずる ぞんずる		存ずる ぞんずる ぞんずる 存じます ぞんじます 存じて ぞんじて ぞんじて 存じた ぞんじた ぞんじた 存じない ぞんじない ぞんじない 存じなかった ぞんじなかった ぞんじなかった 存ずれば ぞんずれば ぞんずれば 存ぜよ ぞんぜよ 存ぜられる ぞんぜられる 存ぜよう ぞんぜよう				動詞-一般 
\\	「試みる(こころみる)」のアクセントは?2009	試みる 仮名 
\\	こころみる 漢字 
\\	試みる 発音図:	試みる こころみる		試みる こころみる 試みます こころみます 試みて こころみて 試みた こころみた 試みない こころみない 試みなかった こころみなかった 試みれば こころみれば 試みさせる こころみさせる 試みられる こころみられる 試みろ こころみろ 試みられる こころみられる 試みよう こころみよう				動詞-一般 
\\	「影響(えいきょう)」のアクセントは?2010	影響 仮名 
\\	えいきょう、えいきょー 漢字 
\\	影響 発音図: 助詞: オ	影響 えいきょう		影響 えいきょう 影響する えいきょうする 影響します えいきょうします 影響して えいきょうして 影響した えいきょうした 影響しない えいきょうしない 影響しなかった えいきょうしなかった 影響すれば えいきょうすれば 影響させる えいきょうさせる 影響される えいきょうされる 影響しろ えいきょうしろ 影響しよう えいきょうしよう				動詞-一般 
\\	「構える(かまえる)」のアクセントは?2011	構える 仮名 
\\	かまえる 漢字 
\\	構える 発音図:	構える かまえる		構える かまえる 構えます かまえます 構えて かまえて 構えた かまえた 構えない かまえない 構えなかった かまえなかった 構えれば かまえれば 構えさせる かまえさせる 構えられる かまえられる 構えろ かまえろ 構えられる かまえられる 構えよう かまえよう				動詞-一般 
\\	「特性(とくせい)」のアクセントは?2012	特性 仮名 
\\	とくせい 漢字 
\\	特性 発音図: 助詞: オ							名詞-普通名詞-一般 
\\	「2日(ふつか)」のアクセントは?2013	2日 仮名 
\\	ふつか 漢字 
\\	二日、2日 発音図: 助詞: オ	二日 ふつか		二日 ふつか				名詞-数詞 
\\	「記入(きにゅう)」のアクセントは?2014	記入 仮名 
\\	きにゅー、きにゅう 漢字 
\\	記入 発音図: 助詞: オ	記入 きにゅう		記入 きにゅう 記入する きにゅうする 記入します きにゅうします 記入して きにゅうして 記入した きにゅうした 記入しない きにゅうしない 記入しなかった きにゅうしなかった 記入すれば きにゅうすれば 記入させる きにゅうさせる 記入される きにゅうされる 記入しろ きにゅうしろ 記入できる きにゅうできる 記入しよう きにゅうしよう				動詞-一般 
\\	「手前(てまえ)」のアクセントは?2015	手前 仮名 
\\	てまえ 漢字 
\\	手前 発音図: 助詞: オ	手前 てまえ		手前 てまえ				名詞-普通名詞-一般 
\\	「微笑む(ほほえむ)」のアクセントは?2016	ほほえむ{微笑} 仮名 
\\	ほほえむ 漢字 
\\	微笑む 発音図:	微笑む ほほえむ		微笑む ほほえむ 微笑みます ほほえみます 微笑んで ほほえんで 微笑んだ ほほえんだ 微笑まない ほほえまない 微笑まなかった ほほえまなかった 微笑めば ほほえめば 微笑ませる ほほえませる 微笑まれる ほほえまれる 微笑め ほほえめ 微笑める ほほえめる 微笑もう ほほえもう				動詞-一般 
\\	「継続(けいぞくする)」のアクセントは?2017	継続 仮名 
\\	けいぞく 漢字 
\\	継続 発音図: 助詞: オ	継続する けいぞくする		継続する けいぞくする 継続します けいぞくします 継続して けいぞくして 継続した けいぞくした 継続しない けいぞくしない 継続しなかった けいぞくしなかった 継続すれば けいぞくすれば 継続させる けいぞくさせる 継続される けいぞくされる 継続しろ けいぞくしろ 継続できる けいぞくできる 継続しよう けいぞくしよう				動詞-一般 
\\	「主体(しゅたい)」のアクセントは?2018	主体 仮名 
\\	しゅたい 漢字 
\\	主体 発音図: 助詞: オ	主体 しゅたい		主体 しゅたい				名詞-普通名詞-一般 
\\	「2人(ふたり)《名》」のアクセントは?2019	2人(名) 仮名 
\\	ふたり 漢字 
\\	二人、2人 発音図: 例文: あの店にガ入った	二人 ふたり		二人 ふたり				副詞 
\\	「クラブ(くらぶ)」のアクセントは?2020	クラブ 仮名 
\\	クラブ、くらぶ 漢字 
\\	発音図: 助詞: オ	クラブ くらぶ くらぶ		クラブ くらぶ くらぶ				名詞-普通名詞-一般 
\\	「2人(ふたり)《副》」のアクセントは?2021	2人(副) 仮名 
\\	ふたり 漢字 
\\	二人、2人 発音図: 例文: あの店にガ入った	二人 ふたり		二人 ふたり				副詞 
\\	「出品(しゅっぴんする)」のアクセントは?2022	出品 仮名 
\\	しゅっぴん 漢字 
\\	出品 発音図: 助詞: オ	出品する しゅっぴんする		出品する しゅっぴんする 出品します しゅっぴんします 出品して しゅっぴんして 出品した しゅっぴんした 出品しない しゅっぴんしない 出品しなかった しゅっぴんしなかった 出品すれば しゅっぴんすれば 出品させる しゅっぴんさせる 出品される しゅっぴんされる 出品しろ しゅっぴんしろ 出品できる しゅっぴんできる 出品しよう しゅっぴんしよう				動詞-一般 
\\	「窺う(うかがう)」のアクセントは?2023	うかがう{窺} 仮名 
\\	うかがう 漢字 
\\	窺う 発音図:							動詞-一般 
\\	「唱える(となえる)」のアクセントは?2024	唱える 仮名 
\\	となえる 漢字 
\\	唱える 発音図: 発音図:	唱える となえる となえる		唱える となえる となえる 唱えます となえます 唱えて となえて 唱えた となえた 唱えない となえない 唱えなかった となえなかった 唱えれば となえれば となえれば 唱えさせる となえさせる 唱えられる となえられる 唱えろ となえろ となえろ 唱えられる となえられる 唱えよう となえよう				動詞-一般 
\\	「特殊(とくしゅ)」のアクセントは?2025	特殊 仮名 
\\	とくしゅ 漢字 
\\	特殊 発音図: 助詞: オ 発音図: 助詞: オ	特殊 とくしゅ		特殊 とくしゅ				形状詞-一般 
\\	「特殊な(とくしゅな)」のアクセントは?2026		特殊な とくしゅな とくしゅな		特殊な とくしゅな とくしゅな 特殊です とくしゅです とくしゅです 特殊で とくしゅで とくしゅで 特殊だった とくしゅだった とくしゅだった 特殊じゃない とくしゅじゃない とくしゅじゃない 特殊じゃなかった とくしゅじゃなかった とくしゅじゃなかった 特殊ならば とくしゅならば とくしゅならば 特殊だ とくしゅだ とくしゅだ 特殊に とくしゅに とくしゅに				形状詞-一般 
\\	「牛(うし)」のアクセントは?2027	牛 仮名 
\\	うし 漢字 
\\	牛 発音図: 助詞: オ	牛 うし		牛 うし				名詞-普通名詞-一般 
\\	「列車(れっしゃ)」のアクセントは?2028	列車 仮名 
\\	れっしゃ 漢字 
\\	列車 発音図: 助詞: オ 発音図: 助詞: オ	列車 れっしゃ		列車 れっしゃ				名詞-普通名詞-一般 
\\	「電話番号(でんわばんごう)」のアクセントは?2029	電話番号 仮名 
\\	でんわばんごう、でんわばんごー 漢字 
\\	電話番号 発音図: 助詞: オ	電話番号 でんわばんごう		電話番号 でんわばんごう				名詞-普通名詞-一般 
\\	「人材(じんざい)」のアクセントは?2030	人材 仮名 
\\	じんざい 漢字 
\\	人材 発音図: 助詞: オ	人材 じんざい		人材 じんざい				名詞-普通名詞-一般 
\\	「卒業(そつぎょう)」のアクセントは?2031	卒業 仮名 
\\	そつぎょう、そつぎょー 漢字 
\\	卒業 発音図: 助詞: オ	卒業 そつぎょう		卒業 そつぎょう 卒業する そつぎょうする 卒業します そつぎょうします 卒業して そつぎょうして 卒業した そつぎょうした 卒業しない そつぎょうしない 卒業しなかった そつぎょうしなかった 卒業すれば そつぎょうすれば 卒業させる そつぎょうさせる 卒業される そつぎょうされる 卒業しろ そつぎょうしろ 卒業できる そつぎょうできる 卒業しよう そつぎょうしよう				動詞-一般 
\\	「肉体(にくたい)」のアクセントは?2032	肉体 仮名 
\\	にくたい 漢字 
\\	肉体 発音図: 助詞: オ	肉体 にくたい		肉体 にくたい				名詞-普通名詞-一般 
\\	「醤油(しょうゆ)」のアクセントは?2033	しょうゆ{醤油} 仮名 
\\	しょうゆ、しょーゆ 漢字 
\\	醤油 発音図: 助詞: オ	醤油 しょうゆ		醤油 しょうゆ				名詞-普通名詞-一般 
\\	「メリット(メリット)」のアクセントは?2034	メリット 仮名 
\\	メリット、めりっと 漢字 
\\	発音図: 助詞: オ							名詞-普通名詞-一般 
\\	「皿(さら)」のアクセントは?2035	皿 仮名 
\\	さら 漢字 
\\	皿 発音図: 助詞: オ	皿 さら		皿 さら				名詞-普通名詞-一般 
\\	「どうせ(どうせ)《結局は》」のアクセントは?2036	どうせ(結局は) 仮名 
\\	どうせ、どーせ 漢字 
\\	発音図:							副詞 
\\	「不可欠(ふかけつ)」のアクセントは?2037	不可欠 仮名 
\\	ふかけつ 漢字 
\\	不可欠 発音図:	不可欠 ふかけつ		不可欠 ふかけつ				形状詞-一般 
\\	「不可欠な(ふかけつな)」のアクセントは?2038		不可欠な ふかけつな		不可欠な ふかけつな 不可欠です ふかけつです 不可欠で ふかけつで 不可欠だった ふかけつだった 不可欠じゃない ふかけつじゃない 不可欠じゃなかった ふかけつじゃなかった 不可欠ならば ふかけつならば 不可欠だ ふかけつだ 不可欠に ふかけつに				形状詞-一般 
\\	「農家(のうか)」のアクセントは?2039	農家 仮名 
\\	のうか、のーか 漢字 
\\	農家 発音図: 助詞: オ	農家 のうか		農家 のうか				名詞-普通名詞-一般 
\\	「いざ(いざ)《〜鎌倉》」のアクセントは?2040	いざ(〜鎌倉) 仮名 
\\	いざ 漢字 
\\	発音図: 例文: カま倉							感動詞-一般 
\\	「悲しみ(かなしみ)」のアクセントは?2041	悲しみ 仮名 
\\	かなしみ 漢字 
\\	悲しみ 発音図: 例文: 主を失ってニくれる 発音図: 例文: 主を失ってニくれる 発音図: 例文: 主を失ってニくれる							名詞-普通名詞-一般 
\\	「祭り(まつり)《お〜》」のアクセントは?2042	祭り(お〜) 仮名 
\\	まつり 漢字 
\\	祭り 発音図: 助詞: オ	祭り まつり まつり		祭り まつり まつり				名詞-普通名詞-一般 
\\	「筆者(ひっしゃ)」のアクセントは?2043	筆者 仮名 
\\	ひっしゃ 漢字 
\\	筆者 発音図: 助詞: オ 発音図: 助詞: オ	筆者 ひっしゃ		筆者 ひっしゃ				名詞-普通名詞-一般 
\\	「恵まれる(めぐまれる)」のアクセントは?2044	恵まれる 仮名 
\\	めぐまれる 漢字 
\\	恵まれる 発音図:	恵まれる めぐまれる		恵まれる めぐまれる 恵まれます めぐまれます 恵まれて めぐまれて 恵まれた めぐまれた 恵まれない めぐまれない 恵まれなかった めぐまれなかった 恵まれれば めぐまれれば 恵まれさせる めぐまれさせる 恵まれられる めぐまれられる 恵まれろ めぐまれろ 恵まれよう めぐまれよう				動詞-一般 
\\	「乾く(かわく)」のアクセントは?2045	乾く 仮名 
\\	かわく 漢字 
\\	乾く 発音図:	乾く かわく		乾く かわく 乾きます かわきます 乾いて かわいて 乾いた かわいた 乾かない かわかない 乾かなかった かわかなかった 乾けば かわけば 乾かせる かわかせる 乾かれる かわかれる 乾け かわけ 乾こう かわこう				動詞-一般 
\\	「煙(けむり)」のアクセントは?2046	煙 仮名 
\\	けむり 漢字 
\\	煙 発音図: 助詞: オ	煙 けむり		煙 けむり				名詞-普通名詞-一般 
\\	「あんな(あんな)《あのような》」のアクセントは?2047	あんな(あのような) 仮名 
\\	あんな 漢字 
\\	発音図:							形状詞-一般 
\\	「主要な(しゅような)」のアクセントは?2048	主要 仮名 
\\	しゅよう、しゅよー 漢字 
\\	主要 発音図: 助詞: オ	主要な しゅような		主要な しゅような 主要です しゅようです 主要で しゅようで 主要だった しゅようだった 主要じゃない しゅようじゃない 主要じゃなかった しゅようじゃなかった 主要ならば しゅようならば 主要だ しゅようだ 主要に しゅように				形状詞-一般 
\\	「振り(ふり)《寝た〜;着物の〜》」のアクセントは?2049	振り(寝た〜;着物の〜) 仮名 
\\	ふり 漢字 
\\	振り 発音図: 例文: たぬきは死んだオする	振り ふり ふり		振り ふり ふり				名詞-普通名詞-一般 
\\	「振り(ふり)《〜をつける》」のアクセントは?2050	振り(〜をつける) 仮名 
\\	ふり 漢字 
\\	振り 発音図: 例文: 踊りのオつける 発音図: 例文: 踊りのオつける	振り ふり ふり		振り ふり ふり				名詞-普通名詞-一般 
\\	「危ない(あぶない)」のアクセントは?2051	危ない 仮名 
\\	あぶない 漢字 
\\	危ない 発音図: 発音図:	危ない あぶない		危ない あぶない 危ないです あぶないです 危なくて あぶなくて あぶなくて 危なかった あぶなかった あぶなかった 危なくない あぶなくない あぶなくない 危なくなかった あぶなくなかった あぶなくなかった 危なければ あぶなければ あぶなければ 危ない あぶない 危なく あぶなく あぶなく				形容詞-一般 
\\	「手伝う(てつだう)」のアクセントは?2052	手伝う 仮名 
\\	てつだう 漢字 
\\	手伝う 発音図:	手伝う てつだう		手伝う てつだう 手伝います てつだいます 手伝って てつだって 手伝った てつだった 手伝わない てつだわない 手伝わなかった てつだわなかった 手伝えば てつだえば 手伝わせる てつだわせる 手伝われる てつだわれる 手伝え てつだえ 手伝える てつだえる 手伝おう てつだおう				動詞-一般 
\\	「曲がる(まがる)」のアクセントは?2053	曲がる 仮名 
\\	まがる 漢字 
\\	曲がる 発音図:	曲がる まがる		曲がる まがる 曲がります まがります 曲がって まがって 曲がった まがった 曲がらない まがらない 曲がらなかった まがらなかった 曲がれば まがれば 曲がらせる まがらせる 曲がられる まがられる 曲がれ まがれ 曲がれる まがれる 曲がろう まがろう				動詞-一般 
\\	「タイミング(たいみんぐ)」のアクセントは?2054	タイミング 仮名 
\\	タイミング、たいみんぐ 漢字 
\\	発音図: 助詞: オ	タイミング たいみんぐ		タイミング たいみんぐ				名詞-普通名詞-一般 
\\	「借金(しゃっきん)」のアクセントは?2055	借金 仮名 
\\	しゃっきん 漢字 
\\	借金 発音図: 助詞: オ	借金 しゃっきん		借金 しゃっきん				名詞-普通名詞-一般 
\\	「批判(ひはん)」のアクセントは?2056	批判 仮名 
\\	ひはん 漢字 
\\	批判 発音図: 助詞: オ	批判 ひはん		批判 ひはん 批判する ひはんする 批判します ひはんします 批判して ひはんして 批判した ひはんした 批判しない ひはんしない 批判しなかった ひはんしなかった 批判すれば ひはんすれば 批判させる ひはんさせる 批判される ひはんされる 批判しろ ひはんしろ 批判できる ひはんできる 批判しよう ひはんしよう				動詞-一般 
\\	「形式(けいしき)」のアクセントは?2057	形式 仮名 
\\	けいしき 漢字 
\\	形式 発音図: 助詞: オ	形式 けいしき		形式 けいしき				名詞-普通名詞-一般 
\\	「引き出す(ひきだす)」のアクセントは?2058	引き出す 仮名 
\\	ひきだす 漢字 
\\	引き出す 発音図:	引き出す ひきだす		引き出す ひきだす 引き出します ひきだします 引き出して ひきだして 引き出した ひきだした 引き出さない ひきださない 引き出さなかった ひきださなかった 引き出せば ひきだせば 引き出させる ひきださせる 引き出される ひきだされる 引き出せ ひきだせ 引き出せる ひきだせる 引き出そう ひきだそう				動詞-一般 
\\	「何しろ(なにしろ)」のアクセントは?2059	何しろ 仮名 
\\	なにしろ 漢字 
\\	何しろ 発音図:							副詞 
\\	「単位(たんい)」のアクセントは?2060	単位 仮名 
\\	たんい 漢字 
\\	単位 発音図: 助詞: オ	単位 たんい		単位 たんい				名詞-普通名詞-一般 
\\	「速度(そくど)」のアクセントは?2061	速度 仮名 
\\	そくど 漢字 
\\	速度 発音図: 助詞: オ	速度 そくど		速度 そくど				名詞-普通名詞-一般 
\\	「江戸時代(えどじだい)」のアクセントは?2062		江戸時代 えどじだい		江戸時代 えどじだい				名詞-普通名詞-一般 
\\	「剥く(むく)」のアクセントは?2063	むく{剥} 仮名 
\\	むく 漢字 
\\	剥く 発音図:	剥く むく		剥く むく 剥きます むきます 剥いて むいて 剥いた むいた 剥かない むかない 剥かなかった むかなかった 剥けば むけば 剥かせる むかせる 剥かれる むかれる 剥け むけ 剥ける むける 剥こう むこう				動詞-一般 
\\	「ゼロ(ぜろ)」のアクセントは?2064	ゼロ 仮名 
\\	ゼロ、ぜろ 漢字 
\\	発音図: 助詞: オ	ゼロ ぜろ		ゼロ ぜろ				名詞-数詞 
\\	「将来(しょうらい)」のアクセントは?2065	将来 仮名 
\\	しょうらい、しょーらい 漢字 
\\	将来 発音図: 助詞: オ	将来 しょうらい		将来 しょうらい				副詞 
\\	「良さ(よさ)」のアクセントは?2066	よさ{良;善;好} 仮名 
\\	よさ 漢字 
\\	良さ、善さ、好さ 発音図: 助詞: オ	良さ よさ		良さ よさ				名詞-普通名詞-一般 
\\	「植える(うえる)」のアクセントは?2067	植える 仮名 
\\	うえる 漢字 
\\	植える 発音図:	植える うえる		植える うえる 植えます うえます 植えて うえて 植えた うえた 植えない うえない 植えなかった うえなかった 植えれば うえれば 植えさせる うえさせる 植えられる うえられる 植えろ うえろ 植えれる うえれる 植えよう うえよう				動詞-一般 
\\	「器(うつわ)」のアクセントは?2068	器 仮名 
\\	うつわ 漢字 
\\	器 発音図: 助詞: オ	器 うつわ		器 うつわ				名詞-普通名詞-一般 
\\	「沈む(しずむ)」のアクセントは?2069	沈む 仮名 
\\	しずむ 漢字 
\\	沈む 発音図:	沈む しずむ		沈む しずむ 沈みます しずみます 沈んで しずんで 沈んだ しずんだ 沈まない しずまない 沈まなかった しずまなかった 沈めば しずめば 沈ませる しずませる 沈まれる しずまれる 沈め しずめ 沈める しずめる 沈もう しずもう				動詞-一般 
\\	「太る(ふとる)」のアクセントは?2070	太る{肥} 仮名 
\\	ふとる 漢字 
\\	太る、肥る 発音図:	太る ふとる		太る ふとる 太ります ふとります 太って ふとって 太った ふとった 太らない ふとらない 太らなかった ふとらなかった 太れば ふとれば 太らせる ふとらせる 太られる ふとられる 太れ ふとれ 太れる ふとれる 太ろう ふとろう				動詞-一般 
\\	「経済的な(けいざいてきな)」のアクセントは?2071	経済的 仮名 
\\	けいざいてき 漢字 
\\	経済的 発音図:	経済的な けいざいてきな		経済的な けいざいてきな 経済的です けいざいてきです 経済的で けいざいてきで 経済的だった けいざいてきだった 経済的じゃない けいざいてきじゃない 経済的じゃなかった けいざいてきじゃなかった 経済的ならば けいざいてきならば 経済的だ けいざいてきだ 経済的に けいざいてきに				形状詞-一般 
\\	「伯母さん(おばさん)」のアクセントは?2072	伯母さん 仮名 
\\	おばさん 漢字 
\\	伯母さん 発音図: 助詞: オ	伯母さん おばさん		伯母さん おばさん				名詞-普通名詞-一般 
\\	「指導(しどう)」のアクセントは?2073	指導 仮名 
\\	しどう、しどー 漢字 
\\	指導 発音図: 助詞: オ	指導 しどう		指導 しどう 指導する しどうする 指導します しどうします 指導して しどうして 指導した しどうした 指導しない しどうしない 指導しなかった しどうしなかった 指導すれば しどうすれば 指導させる しどうさせる 指導される しどうされる 指導しろ しどうしろ 指導できる しどうできる 指導しよう しどうしよう				動詞-一般 
\\	「各国(かっこく)」のアクセントは?2074	各国 仮名 
\\	かっこく 漢字 
\\	各国 発音図: 助詞: オ 発音図: 助詞: オ	各国 かっこく		各国 かっこく				名詞-普通名詞-一般 
\\	「継ぐ(つぐ)」のアクセントは?2075	継ぐ 仮名 
\\	つぐ 漢字 
\\	継ぐ 発音図:	継ぐ つぐ		継ぐ つぐ 継ぎます つぎます 継いで ついで 継いだ ついだ 継がない つがない 継がなかった つがなかった 継げば つげば 継がせる つがせる 継がれる つがれる 継げ つげ 継げる つげる 継ごう つごう				動詞-一般 
\\	「子育て(こそだて)」のアクセントは?2076	子育て 仮名 
\\	こそだて 漢字 
\\	子育て 発音図: 助詞: オ	子育て こそだて		子育て こそだて				名詞-普通名詞-一般 
\\	「解く(とく)」のアクセントは?2078	解く 仮名 
\\	とく 漢字 
\\	解く 発音図:	解く とく		解く とく 解きます ときます 解いて といて 解いた といた 解かない とかない 解かなかった とかなかった 解けば とけば 解かせる とかせる 解かれる とかれる 解け とけ 解ける とける 解こう とこう				動詞-一般 
\\	「漫画(まんが)」のアクセントは?2079	漫画 仮名 
\\	まんが 漢字 
\\	漫画 発音図: 例文: 近ごろはオコミックと呼ぶ	漫画 まんが		漫画 まんが				名詞-普通名詞-一般 
\\	「バイク(ばいく)」のアクセントは?2080	バイク 仮名 
\\	バイク、ばいく 漢字 
\\	発音図: 助詞: オ	バイク ばいく		バイク ばいく				名詞-普通名詞-一般 
\\	「独立(どくりつ)」のアクセントは?2081	独立 仮名 
\\	どくりつ 漢字 
\\	独立 発音図: 助詞: オ	独立 どくりつ		独立 どくりつ 独立する どくりつする 独立します どくりつします 独立して どくりつして 独立した どくりつした 独立しない どくりつしない 独立しなかった どくりつしなかった 独立すれば どくりつすれば 独立させる どくりつさせる 独立される どくりつされる 独立しろ どくりつしろ 独立できる どくりつできる 独立しよう どくりつしよう				動詞-一般 
\\	「背負う(しょう)」のアクセントは?2082	しょう{背負} 仮名 
\\	しょう 漢字 
\\	背負う 発音図:							名詞-普通名詞-一般 
\\	「妾(しょう)」のアクセントは?2083	しょう{妾} 仮名 
\\	しょう、しょー 漢字 
\\	妾 発音図: 助詞: オ							名詞-普通名詞-一般 
\\	「笙(しょう)」のアクセントは?2084	しょう{荘;笙} 仮名 
\\	しょう、しょー 漢字 
\\	笙、荘 発音図: 助詞: オ							名詞-普通名詞-一般 
\\	「到頭(とうとう)」のアクセントは?2085	とうとう{到頭} 仮名 
\\	とうとう、とーとー 漢字 
\\	到頭 発音図:							副詞 
\\	「強さ(つよさ)」のアクセントは?2086	強さ 仮名 
\\	つよさ 漢字 
\\	強さ 発音図: 助詞: オ							名詞-普通名詞-一般 
\\	「早め(はやめ)」のアクセントは?2087	早め 仮名 
\\	はやめ 漢字 
\\	早め 発音図: 例文: 良いことを決めるのはガいい 発音図: 例文: 良いことを決めるのはガいい							形状詞-一般 
\\	「中学(ちゅうがく)」のアクセントは?2088	中学 仮名 
\\	ちゅーがく、ちゅうがく 漢字 
\\	中学 発音図: 助詞: オ	中学 ちゅうがく		中学 ちゅうがく				名詞-普通名詞-一般 
\\	「シーエム(シーエム)
\\	のアクセントは?2089	シーエム
\\	仮名 
\\	シーエム、しーえむ 漢字 
\\	発音図: 助詞: オ							名詞-普通名詞-一般 
\\	「田舎(いなか)」のアクセントは?2090	田舎 仮名 
\\	いなか 漢字 
\\	田舎 発音図: 助詞: オ	田舎 いなか		田舎 いなか				名詞-普通名詞-一般 
\\	「五百円(ごひゃくえん)」のアクセントは?2091		五百円 ごひゃくえん		五百円 ごひゃくえん				名詞-数詞 
\\	「急激な(きゅうげきな)」のアクセントは?2092	急激 仮名 
\\	きゅーげき、きゅうげき 漢字 
\\	急激 発音図: 助詞: オ	急激な きゅうげきな		急激な きゅうげきな 急激です きゅうげきです 急激で きゅうげきで 急激だった きゅうげきだった 急激じゃない きゅうげきじゃない 急激じゃなかった きゅうげきじゃなかった 急激ならば きゅうげきならば 急激だ きゅうげきだ 急激に きゅうげきに				形状詞-一般 
\\	「共通(きょうつう)」のアクセントは?2093	共通 仮名 
\\	きょうつう、きょーつー 漢字 
\\	共通 発音図: 助詞: オ	共通 きょうつう		共通 きょうつう 共通する きょうつうする 共通します きょうつうします 共通して きょうつうして 共通した きょうつうした 共通しない きょうつうしない 共通しなかった きょうつうしなかった 共通すれば きょうつうすれば 共通させる きょうつうさせる 共通される きょうつうされる 共通しろ きょうつうしろ 共通しよう きょうつうしよう				動詞-一般 
\\	「顎(あご)」のアクセントは?2094	あご{顎} 仮名 
\\	あご 漢字 
\\	顎 発音図: 助詞: オ	顎 あご		顎 あご				名詞-普通名詞-一般 
\\	「逮捕(たいほ)」のアクセントは?2095	逮捕 仮名 
\\	たいほ 漢字 
\\	逮捕 発音図: 助詞: オ	逮捕 たいほ		逮捕 たいほ 逮捕する たいほする 逮捕します たいほします 逮捕して たいほして 逮捕した たいほした 逮捕しない たいほしない 逮捕しなかった たいほしなかった 逮捕すれば たいほすれば 逮捕させる たいほさせる 逮捕される たいほされる 逮捕しろ たいほしろ 逮捕できる たいほできる 逮捕しよう たいほしよう				動詞-一般 
\\	「ロンドン(ろんどん)」のアクセントは?2096		ロンドン ろんどん		ロンドン ろんどん				名詞-固有名詞-地名-一般 
\\	「限定(げんていする)」のアクセントは?2097	限定 仮名 
\\	げんてい 漢字 
\\	限定 発音図: 助詞: オ	限定する げんていする		限定する げんていする 限定します げんていします 限定して げんていして 限定した げんていした 限定しない げんていしない 限定しなかった げんていしなかった 限定すれば げんていすれば 限定させる げんていさせる 限定される げんていされる 限定しろ げんていしろ 限定できる げんていできる 限定しよう げんていしよう				動詞-一般 
\\	「布団(ふとん)」のアクセントは?2098	布団 仮名 
\\	ふとん 漢字 
\\	布団 発音図: 助詞: オ	布団 ふとん		布団 ふとん				名詞-普通名詞-一般 
\\	「保険(ほけん)」のアクセントは?2099	保険 仮名 
\\	ほけん 漢字 
\\	保険 発音図: 助詞: オ	保険 ほけん		保険 ほけん				名詞-普通名詞-一般 
\\	「プラス(ぷらす)」のアクセントは?2100	プラス 仮名 
\\	プラス、ぷらす 漢字 
\\	発音図: 助詞: オ 発音図: 助詞: オ	プラス ぷらす ぷらす		プラス ぷらす ぷらす				名詞-普通名詞-一般 
\\	「電気(でんき)」のアクセントは?2101	電気 仮名 
\\	でんき 漢字 
\\	電気 発音図: 助詞: オ	電気 でんき		電気 でんき				名詞-普通名詞-一般 
\\	「引き受ける(ひきうける)」のアクセントは?2102	引き受ける 仮名 
\\	ひきうける 漢字 
\\	引き受ける 発音図:	引き受ける ひきうける		引き受ける ひきうける 引き受けます ひきうけます 引き受けて ひきうけて 引き受けた ひきうけた 引き受けない ひきうけない 引き受けなかった ひきうけなかった 引き受ければ ひきうければ 引き受けさせる ひきうけさせる 引き受けられる ひきうけられる 引き受けろ ひきうけろ 引き受けられる ひきうけられる 引き受けよう ひきうけよう				動詞-一般 
\\	「経過(けいかする)」のアクセントは?2103	経過 仮名 
\\	けいか 漢字 
\\	経過 発音図: 助詞: オ	経過する けいかする		経過する けいかする 経過します けいかします 経過して けいかして 経過した けいかした 経過しない けいかしない 経過しなかった けいかしなかった 経過すれば けいかすれば 経過させる けいかさせる 経過される けいかされる 経過しろ けいかしろ 経過できる けいかできる 経過しよう けいかしよう				動詞-一般 
\\	「社会的な(しゃかいてきな)」のアクセントは?2104	社会的 仮名 
\\	しゃかいてき 漢字 
\\	社会的 発音図:	社会的な しゃかいてきな		社会的な しゃかいてきな 社会的です しゃかいてきです 社会的で しゃかいてきで 社会的だった しゃかいてきだった 社会的じゃない しゃかいてきじゃない 社会的じゃなかった しゃかいてきじゃなかった 社会的ならば しゃかいてきならば 社会的だ しゃかいてきだ 社会的に しゃかいてきに				形状詞-一般 
\\	「熱心な(ねっしんな)」のアクセントは?2105		熱心な ねっしんな ねっしんな		熱心な ねっしんな ねっしんな 熱心です ねっしんです ねっしんです 熱心で ねっしんで ねっしんで 熱心だった ねっしんだった ねっしんだった 熱心じゃない ねっしんじゃない ねっしんじゃない 熱心じゃなかった ねっしんじゃなかった ねっしんじゃなかった 熱心ならば ねっしんならば ねっしんならば 熱心だ ねっしんだ ねっしんだ 熱心に ねっしんに ねっしんに				形状詞-一般 
\\	「ビデオ(びでお)」のアクセントは?2106	ビデオ 仮名 
\\	ビデオ、びでお 漢字 
\\	発音図: 助詞: オ	ビデオ びでお		ビデオ びでお				名詞-普通名詞-一般 
\\	「熱心(ねっしん)」のアクセントは?2107	熱心 仮名 
\\	ねっしん 漢字 
\\	熱心 発音図: 発音図:	熱心 ねっしん		熱心 ねっしん				形状詞-一般 
\\	「近年(きんねん)」のアクセントは?2108	近年 仮名 
\\	きんねん 漢字 
\\	近年 発音図: 助詞: オ	近年 きんねん		近年 きんねん				名詞-普通名詞-一般 
\\	「的確な(てきかくな)」のアクセントは?2109	的確 仮名 
\\	てきかく、てっかく 漢字 
\\	的確 発音図: 助詞: オ 発音図: 助詞: オ	的確な てきかくな		的確な てきかくな 的確です てきかくです 的確で てきかくで 的確だった てきかくだった 的確じゃない てきかくじゃない 的確じゃなかった てきかくじゃなかった 的確ならば てきかくならば 的確だ てきかくだ 的確に てきかくに				形状詞-一般 
\\	「概要(がいよう)」のアクセントは?2110	概要 仮名 
\\	がいよう、がいよー 漢字 
\\	概要 発音図: 助詞: オ	概要 がいよう		概要 がいよう				名詞-普通名詞-一般 
\\	「詳細(しょうさい)」のアクセントは?2111	詳細 仮名 
\\	しょうさい、しょーさい 漢字 
\\	詳細 発音図: 助詞: オ	詳細 しょうさい		詳細 しょうさい				形状詞-一般 
\\	「詳細な(しょうさいな)」のアクセントは?2112		詳細な しょうさいな		詳細な しょうさいな 詳細です しょうさいです 詳細で しょうさいで 詳細だった しょうさいだった 詳細じゃない しょうさいじゃない 詳細じゃなかった しょうさいじゃなかった 詳細ならば しょうさいならば 詳細だ しょうさいだ 詳細に しょうさいに				形状詞-一般 
\\	「改める(あらためる)」のアクセントは?2113	改める 仮名 
\\	あらためる 漢字 
\\	改める 発音図:	改める あらためる		改める あらためる 改めます あらためます 改めて あらためて 改めた あらためた 改めない あらためない 改めなかった あらためなかった 改めれば あらためれば 改めさせる あらためさせる 改められる あらためられる 改めろ あらためろ 改められる あらためられる 改めよう あらためよう				動詞-一般 
\\	「正式な(せいしきな)」のアクセントは?2114	正式 仮名 
\\	せいしき 漢字 
\\	正式 発音図: 助詞: オ	正式な せいしきな		正式な せいしきな 正式です せいしきです 正式で せいしきで 正式だった せいしきだった 正式じゃない せいしきじゃない 正式じゃなかった せいしきじゃなかった 正式ならば せいしきならば 正式だ せいしきだ 正式に せいしきに				形状詞-一般 
\\	「取り付ける(とりつける)」のアクセントは?2115	取り付ける 仮名 
\\	とりつける 漢字 
\\	取り付ける 発音図: 発音図:	取り付ける とりつける		取り付ける とりつける 取り付けます とりつけます 取り付けて とりつけて 取り付けた とりつけた 取り付けない とりつけない 取り付けなかった とりつけなかった 取り付ければ とりつければ 取り付けさせる とりつけさせる 取り付けられる とりつけられる 取り付けろ とりつけろ 取り付けられる とりつけられる 取り付けよう とりつけよう				動詞-一般 
\\	「称する(しょうする)」のアクセントは?2116	称する 仮名 
\\	しょうする、しょーする 漢字 
\\	称する 発音図:	称する しょうする		称する しょうする 称します しょうします 称して しょうして 称した しょうした 称さない しょうさない 称さなかった しょうさなかった 称せば しょうせば 称させる しょうさせる 称される しょうされる 称せ しょうせ 称せる しょうせる 称そう しょうそう				動詞-一般 
\\	「記述(きじゅつ)」のアクセントは?2117	記述 仮名 
\\	きじゅつ 漢字 
\\	記述 発音図: 助詞: オ							名詞-普通名詞-一般 
\\	「幼稚園(ようちえん)」のアクセントは?2118	幼稚園 仮名 
\\	ようちえん、よーちえん 漢字 
\\	幼稚園 発音図: 助詞: オ	幼稚園 ようちえん		幼稚園 ようちえん				名詞-普通名詞-一般 
\\	「幅広い(はばひろい)」のアクセントは?2119	幅広い 仮名 
\\	はばひろい 漢字 
\\	幅広い 発音図:	幅広い はばひろい		幅広い はばひろい 幅広いです はばひろいです 幅広くて はばひろくて はばひろくて 幅広かった はばひろかった はばひろかった 幅広くない はばひろくない はばひろくない 幅広くなかった はばひろくなかった はばひろくなかった 幅広ければ はばひろければ はばひろければ 幅広い はばひろい 幅広く はばひろく はばひろく				形容詞-一般 
\\	「手法(しゅほう)」のアクセントは?2120	手法 仮名 
\\	しゅほう、しゅほー 漢字 
\\	手法 発音図: 助詞: オ	手法 しゅほう		手法 しゅほう				名詞-普通名詞-一般 
\\	「冗談(じょうだん)」のアクセントは?2121	冗談 仮名 
\\	じょうだん、じょーだん 漢字 
\\	冗談 発音図: 助詞: オ	冗談 じょうだん		冗談 じょうだん				名詞-普通名詞-一般 
\\	「活動(かつどう)」のアクセントは?2122	活動 仮名 
\\	かつどう、かつどー 漢字 
\\	活動 発音図: 助詞: オ	活動 かつどう		活動 かつどう 活動する かつどうする 活動します かつどうします 活動して かつどうして 活動した かつどうした 活動しない かつどうしない 活動しなかった かつどうしなかった 活動すれば かつどうすれば 活動させる かつどうさせる 活動される かつどうされる 活動しろ かつどうしろ 活動できる かつどうできる 活動しよう かつどうしよう				動詞-一般 
\\	「児童(じどう)」のアクセントは?2123	児童 仮名 
\\	じどう、じどー 漢字 
\\	児童 発音図: 助詞: オ	児童 じどう		児童 じどう				名詞-普通名詞-一般 
\\	「日常生活(にちじょうせいかつ)」のアクセントは?2124		日常生活 にちじょうせいかつ		日常生活 にちじょうせいかつ				名詞-普通名詞-一般 
\\	「己(おのれ)」のアクセントは?2125	己 仮名 
\\	おのれ 漢字 
\\	己 発音図: 助詞: オ							代名詞 
\\	「管理(かんり)」のアクセントは?2126	管理 仮名 
\\	かんり 漢字 
\\	管理 発音図: 助詞: オ	管理 かんり		管理 かんり 管理する かんりする 管理します かんりします 管理して かんりして 管理した かんりした 管理しない かんりしない 管理しなかった かんりしなかった 管理すれば かんりすれば 管理させる かんりさせる 管理される かんりされる 管理しろ かんりしろ 管理できる かんりできる 管理しよう かんりしよう				動詞-一般 
\\	「個性(こせい)」のアクセントは?2127	個性 仮名 
\\	こせい 漢字 
\\	個性 発音図: 助詞: オ	個性 こせい		個性 こせい				名詞-普通名詞-一般 
\\	「体制(たいせい)」のアクセントは?2128	体制 仮名 
\\	たいせい 漢字 
\\	体制 発音図: 助詞: オ	体制 たいせい		体制 たいせい				名詞-普通名詞-一般 
\\	「重要性(じゅうようせい)」のアクセントは?2129	重要性 仮名 
\\	じゅうようせい、じゅーよーせい 漢字 
\\	重要性 発音図: 助詞: オ							名詞-普通名詞-一般 
\\	「膨らむ(ふくらむ)」のアクセントは?2130	膨らむ 仮名 
\\	ふくらむ 漢字 
\\	膨らむ 発音図:	膨らむ ふくらむ		膨らむ ふくらむ 膨らみます ふくらみます 膨らんで ふくらんで 膨らんだ ふくらんだ 膨らまない ふくらまない 膨らまなかった ふくらまなかった 膨らめば ふくらめば 膨らませる ふくらませる 膨らまれる ふくらまれる 膨らめ ふくらめ 膨らもう ふくらもう				動詞-一般 
\\	「水分(すいぶん)」のアクセントは?2131	水分 仮名 
\\	すいぶん 漢字 
\\	水分 発音図: 助詞: オ	水分 すいぶん		水分 すいぶん				名詞-普通名詞-一般 
\\	「届ける(とどける)」のアクセントは?2132	届ける 仮名 
\\	とどける 漢字 
\\	届ける 発音図:	届ける とどける		届ける とどける 届けます とどけます 届けて とどけて 届けた とどけた 届けない とどけない 届けなかった とどけなかった 届ければ とどければ 届けさせる とどけさせる 届けられる とどけられる 届けろ とどけろ 届けられる とどけられる 届けよう とどけよう				動詞-一般 
\\	「掌(てのひら)」のアクセントは?2133		掌 てのひら てのひら		掌 てのひら てのひら				名詞-普通名詞-一般 
\\	「贈る(おくる)」のアクセントは?2134	贈る 仮名 
\\	おくる 漢字 
\\	贈る 発音図:	贈る おくる		贈る おくる 贈ります おくります 贈って おくって 贈った おくった 贈らない おくらない 贈らなかった おくらなかった 贈れば おくれば 贈らせる おくらせる 贈られる おくられる 贈れ おくれ 贈れる おくれる 贈ろう おくろう				動詞-一般 
\\	「違反(いはん)」のアクセントは?2135	違反 仮名 
\\	いはん 漢字 
\\	違反 発音図: 助詞: オ	違反 いはん		違反 いはん 違反する いはんする 違反します いはんします 違反して いはんして 違反した いはんした 違反しない いはんしない 違反しなかった いはんしなかった 違反すれば いはんすれば 違反させる いはんさせる 違反される いはんされる 違反しろ いはんしろ 違反できる いはんできる 違反しよう いはんしよう				動詞-一般 
\\	「幸せな(しあわせな)」のアクセントは?2136		幸せな しあわせな		幸せな しあわせな 幸せです しあわせです 幸せで しあわせで 幸せだった しあわせだった 幸せじゃない しあわせじゃない 幸せじゃなかった しあわせじゃなかった 幸せならば しあわせならば 幸せだ しあわせだ 幸せに しあわせに				名詞-普通名詞-一般 
\\	「驚き(おどろき)」のアクセントは?2137	驚き 仮名 
\\	おどろき 漢字 
\\	驚き 発音図: 例文: 彼の力はデあった 発音図: 例文: 彼の力はデあった 発音図: 例文: 彼の力はデあった	驚き おどろき		驚き おどろき				名詞-普通名詞-一般 
\\	「照らす(てらす)」のアクセントは?2138	照らす 仮名 
\\	てらす 漢字 
\\	照らす 発音図: 発音図:	照らす てらす		照らす てらす 照らします てらします 照らして てらして 照らした てらした 照らさない てらさない 照らさなかった てらさなかった 照らせば てらせば 照らさせる てらさせる 照らされる てらされる 照らせ てらせ 照らせる てらせる 照らそう てらそう				動詞-一般 
\\	「出現(しゅつげん)」のアクセントは?2139	出現 仮名 
\\	しゅつげん 漢字 
\\	出現 発音図: 助詞: オ	出現 しゅつげん		出現 しゅつげん 出現する しゅつげんする 出現します しゅつげんします 出現して しゅつげんして 出現した しゅつげんした 出現しない しゅつげんしない 出現しなかった しゅつげんしなかった 出現すれば しゅつげんすれば 出現させる しゅつげんさせる 出現される しゅつげんされる 出現しろ しゅつげんしろ 出現できる しゅつげんできる 出現しよう しゅつげんしよう				動詞-一般 
\\	「幸せ(しあわせ)」のアクセントは?2140	幸せ 仮名 
\\	しあわせ 漢字 
\\	幸せ 発音図: 助詞: オ	幸せ しあわせ		幸せ しあわせ				名詞-普通名詞-一般 
\\	「シンプルな(しんぷるな)」のアクセントは?2141	シンプル 仮名 
\\	シンプル、しんぷる 漢字 
\\	発音図: 助詞: オ	シンプルな しんぷるな		シンプルな しんぷるな シンプルです しんぷるです シンプルで しんぷるで シンプルだった しんぷるだった シンプルじゃない しんぷるじゃない シンプルじゃなかった しんぷるじゃなかった シンプルならば しんぷるならば シンプルだ しんぷるだ シンプルに しんぷるに				形状詞-一般 
\\	「食品(しょくひん)」のアクセントは?2142	食品 仮名 
\\	しょくひん 漢字 
\\	食品 発音図: 助詞: オ	食品 しょくひん		食品 しょくひん				名詞-普通名詞-一般 
\\	「本質(ほんしつ)」のアクセントは?2143	本質 仮名 
\\	ほんしつ 漢字 
\\	本質 発音図: 助詞: オ	本質 ほんしつ		本質 ほんしつ				名詞-普通名詞-一般 
\\	「今更(いまさら)」のアクセントは?2144	いまさら{今更} 仮名 
\\	いまさら 漢字 
\\	今更 発音図: 発音図:							副詞 
\\	「騙す(だます)」のアクセントは?2145	だます{騙} 仮名 
\\	だます 漢字 
\\	騙す 発音図:	騙す だます		騙す だます 騙します だまします 騙して だまして 騙した だました 騙さない だまさない 騙さなかった だまさなかった 騙せば だませば 騙させる だまさせる 騙される だまされる 騙せ だませ 騙せる だませる 騙そう だまそう				動詞-一般 
\\	「嵌まる(はまる)」のアクセントは?2146	はまる{填;嵌} 仮名 
\\	はまる 漢字 
\\	嵌まる、填まる 発音図:	はまる はまる		はまる はまる はまります はまります はまって はまって はまった はまった はまらない はまらない はまらなかった はまらなかった はまれば はまれば はまらせる はまらせる はまられる はまられる はまれ はまれ はまれる はまれる はまろう はまろう				動詞-一般 
\\	「論文(ろんぶん)」のアクセントは?2147	論文 仮名 
\\	ろんぶん 漢字 
\\	論文 発音図: 助詞: オ	論文 ろんぶん		論文 ろんぶん				名詞-普通名詞-一般 
\\	「運(うん)」のアクセントは?2148	運 仮名 
\\	うん 漢字 
\\	運 発音図: 助詞: オ	運 うん		運 うん				名詞-普通名詞-一般 
\\	「思い付く(おもいつく)」のアクセントは?2149		思い付く おもいつく		思い付く おもいつく 思い付きます おもいつきます 思い付いて おもいついて 思い付いた おもいついた 思い付かない おもいつかない 思い付かなかった おもいつかなかった 思い付けば おもいつけば 思い付かせる おもいつかせる 思い付かれる おもいつかれる 思い付け おもいつけ 思い付ける おもいつける 思い付こう おもいつこう				動詞-一般 
\\	「ニーズ(ニーズ)」のアクセントは?2150	ニーズ 仮名 
\\	ニーズ、にーず 漢字 
\\	発音図: 助詞: オ							名詞-普通名詞-一般 
\\	「十分(じっぷん)」のアクセントは?2151		十分 じっぷん		十分 じっぷん				名詞-数詞 
\\	「独特な(どくとくな)」のアクセントは?2152		独特な どくとくな		独特な どくとくな 独特です どくとくです 独特で どくとくで 独特だった どくとくだった 独特じゃない どくとくじゃない 独特じゃなかった どくとくじゃなかった 独特ならば どくとくならば 独特だ どくとくだ 独特に どくとくに				形状詞-一般 
\\	「独特(どくとく)」のアクセントは?2153	独特 仮名 
\\	どくとく 漢字 
\\	独特 発音図: 助詞: オ	独特 どくとく		独特 どくとく				形状詞-一般 
\\	「台詞(せりふ)」のアクセントは?2154	せりふ{台詞} 仮名 
\\	せりふ 漢字 
\\	台詞 発音図: 助詞: オ 発音図: 助詞: オ	台詞 せりふ		台詞 せりふ				名詞-普通名詞-一般 
\\	「イン(イン)」のアクセントは?2155	イン 仮名 
\\	イン、いん 漢字 
\\	発音図: 助詞: オ							名詞-普通名詞-一般 
\\	「法案(ほうあん)」のアクセントは?2156	法案 仮名 
\\	ほうあん、ほーあん 漢字 
\\	法案 発音図: 助詞: オ	法案 ほうあん		法案 ほうあん				名詞-普通名詞-一般 
\\	「物事(ものごと)」のアクセントは?2157	物事 仮名 
\\	ものごと 漢字 
\\	物事 発音図: 助詞: オ	物事 ものごと		物事 ものごと				名詞-普通名詞-一般 
\\	「作り上げる(つくりあげる)」のアクセントは?2158	作り上げる 仮名 
\\	つくりあげる 漢字 
\\	作り上げる 発音図:	作り上げる つくりあげる		作り上げる つくりあげる 作り上げます つくりあげます 作り上げて つくりあげて 作り上げた つくりあげた 作り上げない つくりあげない 作り上げなかった つくりあげなかった 作り上げれば つくりあげれば 作り上げさせる つくりあげさせる 作り上げられる つくりあげられる 作り上げろ つくりあげろ 作り上げられる つくりあげられる 作り上げよう つくりあげよう				動詞-一般 
\\	「マイナス(まいなす)」のアクセントは?2159	マイナス 仮名 
\\	マイナス、まいなす 漢字 
\\	発音図: 助詞: オ	マイナス まいなす		マイナス まいなす				名詞-普通名詞-一般 
\\	「株価(かぶか)」のアクセントは?2160	株価 仮名 
\\	かぶか 漢字 
\\	株価 発音図: 助詞: オ 発音図: 助詞: オ	株価 かぶか		株価 かぶか				名詞-普通名詞-一般 
\\	「禁止(きんし)」のアクセントは?2161	禁止 仮名 
\\	きんし 漢字 
\\	禁止 発音図: 助詞: オ	禁止 きんし		禁止 きんし 禁止する きんしする 禁止します きんしします 禁止して きんしして 禁止した きんしした 禁止しない きんししない 禁止しなかった きんししなかった 禁止すれば きんしすれば 禁止させる きんしさせる 禁止される きんしされる 禁止しろ きんししろ 禁止できる きんしできる 禁止しよう きんししよう				動詞-一般 
\\	「法則(ほうそく)」のアクセントは?2162	法則 仮名 
\\	ほうそく、ほーそく 漢字 
\\	法則 発音図: 助詞: オ	法則 ほうそく		法則 ほうそく				名詞-普通名詞-一般 
\\	「氏名(しめい)」のアクセントは?2163	氏名 仮名 
\\	しめい 漢字 
\\	氏名 発音図: 助詞: オ	氏名 しめい		氏名 しめい				名詞-普通名詞-一般 
\\	「出席(しゅっせき)」のアクセントは?2164	出席 仮名 
\\	しゅっせき 漢字 
\\	出席 発音図: 助詞: オ	出席 しゅっせき		出席 しゅっせき 出席する しゅっせきする 出席します しゅっせきします 出席して しゅっせきして 出席した しゅっせきした 出席しない しゅっせきしない 出席しなかった しゅっせきしなかった 出席すれば しゅっせきすれば 出席させる しゅっせきさせる 出席される しゅっせきされる 出席しろ しゅっせきしろ 出席できる しゅっせきできる 出席しよう しゅっせきしよう				動詞-一般 
\\	「等しい(ひとしい)」のアクセントは?2165	等しい 仮名 
\\	ひとしい 漢字 
\\	等しい 発音図:	等しい ひとしい		等しい ひとしい 等しいです ひとしいです 等しくて ひとしくて ひとしくて 等しかった ひとしかった ひとしかった 等しくない ひとしくない ひとしくない 等しくなかった ひとしくなかった ひとしくなかった 等しければ ひとしければ ひとしければ 等しい ひとしい 等しく ひとしく ひとしく				形容詞-一般 
\\	「爪(つめ)」のアクセントは?2166	つめ{爪} 仮名 
\\	つめ 漢字 
\\	爪 発音図: 例文: 夜オ切るのはよくないという	爪 つめ		爪 つめ				名詞-普通名詞-一般 
\\	「給料(きゅうりょう)」のアクセントは?2167	給料 仮名 
\\	きゅうりょう、きゅーりょー 漢字 
\\	給料 発音図: 助詞: オ	給料 きゅうりょう		給料 きゅうりょう				名詞-普通名詞-一般 
\\	「笑み(えみ)」のアクセントは?2168	笑み 仮名 
\\	えみ 漢字 
\\	笑み 発音図: 助詞: オ							名詞-普通名詞-一般 
\\	「野球(やきゅう)」のアクセントは?2169	野球 仮名 
\\	やきゅー、やきゅう 漢字 
\\	野球 発音図: 助詞: オ	野球 やきゅう		野球 やきゅう				名詞-普通名詞-一般 
\\	「体力(たいりょく)」のアクセントは?2170	体力 仮名 
\\	たいりょく 漢字 
\\	体力 発音図: 助詞: オ	体力 たいりょく		体力 たいりょく				名詞-普通名詞-一般 
\\	「業者(ぎょうしゃ)」のアクセントは?2171	業者 仮名 
\\	ぎょうしゃ、ぎょーしゃ 漢字 
\\	業者 発音図: 助詞: オ							名詞-普通名詞-一般 
\\	「ピアノ(ぴあの)」のアクセントは?2172	ピアノ 仮名 
\\	ピアノ、ぴあの 漢字 
\\	発音図: 助詞: オ	ピアノ ぴあの		ピアノ ぴあの				名詞-普通名詞-一般 
\\	「追加(ついかする)」のアクセントは?2173	追加 仮名 
\\	ついか 漢字 
\\	追加 発音図: 助詞: オ	追加する ついかする		追加する ついかする 追加します ついかします 追加して ついかして 追加した ついかした 追加しない ついかしない 追加しなかった ついかしなかった 追加すれば ついかすれば 追加させる ついかさせる 追加される ついかされる 追加しろ ついかしろ 追加できる ついかできる 追加しよう ついかしよう				動詞-一般 
\\	「体験(たいけん)」のアクセントは?2174	体験 仮名 
\\	たいけん 漢字 
\\	体験 発音図: 助詞: オ	体験 たいけん		体験 たいけん 体験する たいけんする 体験します たいけんします 体験して たいけんして 体験した たいけんした 体験しない たいけんしない 体験しなかった たいけんしなかった 体験すれば たいけんすれば 体験させる たいけんさせる 体験される たいけんされる 体験しろ たいけんしろ 体験できる たいけんできる 体験しよう たいけんしよう				動詞-一般 
\\	「制限(せいげん)」のアクセントは?2175	制限 仮名 
\\	せいげん 漢字 
\\	制限 発音図: 助詞: オ	制限 せいげん		制限 せいげん				名詞-普通名詞-一般 
\\	「不要(ふよう)」のアクセントは?2176	不要 仮名 
\\	ふよう、ふよー 漢字 
\\	不要 発音図: 助詞: オ	不要 ふよう		不要 ふよう				形状詞-一般 
\\	「不要な(ふような)」のアクセントは?2177		不要な ふような		不要な ふような 不要です ふようです 不要で ふようで 不要だった ふようだった 不要じゃない ふようじゃない 不要じゃなかった ふようじゃなかった 不要ならば ふようならば 不要だ ふようだ 不要に ふように				形状詞-一般 
\\	「玉葱(たまねぎ)」のアクセントは?2178	たまねぎ{玉葱} 仮名 
\\	たまねぎ 漢字 
\\	玉葱 発音図: 助詞: オ							名詞-普通名詞-一般 
\\	「台湾(たいわん)」のアクセントは?2179		台湾 たいわん		台湾 たいわん				名詞-固有名詞-地名-一般 
\\	「漏らす(もらす)」のアクセントは?2180	漏らす 仮名 
\\	もらす 漢字 
\\	漏らす 発音図:	漏らす もらす		漏らす もらす 漏らします もらします 漏らして もらして 漏らした もらした 漏らさない もらさない 漏らさなかった もらさなかった 漏らせば もらせば 漏らさせる もらさせる 漏らされる もらされる 漏らせ もらせ 漏らせる もらせる 漏らそう もらそう				動詞-一般 
\\	「機関(きかん)」のアクセントは?2181	機関 仮名 
\\	きかん 漢字 
\\	機関 発音図: 助詞: オ 発音図: 助詞: オ	機関 きかん きかん		機関 きかん きかん				名詞-普通名詞-一般 
\\	「破壊(はかい)」のアクセントは?2182	破壊 仮名 
\\	はかい 漢字 
\\	破壊 発音図: 助詞: オ	破壊 はかい		破壊 はかい 破壊する はかいする 破壊します はかいします 破壊して はかいして 破壊した はかいした 破壊しない はかいしない 破壊しなかった はかいしなかった 破壊すれば はかいすれば 破壊させる はかいさせる 破壊される はかいされる 破壊しろ はかいしろ 破壊できる はかいできる 破壊しよう はかいしよう				動詞-一般 
\\	「交換(こうかん)」のアクセントは?2183	交換 仮名 
\\	こうかん、こーかん 漢字 
\\	交換 発音図: 助詞: オ	交換 こうかん		交換 こうかん 交換する こうかんする 交換します こうかんします 交換して こうかんして 交換した こうかんした 交換しない こうかんしない 交換しなかった こうかんしなかった 交換すれば こうかんすれば 交換させる こうかんさせる 交換される こうかんされる 交換しろ こうかんしろ 交換できる こうかんできる 交換しよう こうかんしよう				動詞-一般 
\\	「評判(ひょうばん)」のアクセントは?2184	評判 仮名 
\\	ひょうばん、ひょーばん 漢字 
\\	評判 発音図: 助詞: オ	評判 ひょうばん		評判 ひょうばん				名詞-普通名詞-一般 
\\	「多少(たしょう)」のアクセントは?2185	多少 仮名 
\\	たしょう、たしょー 漢字 
\\	多少 発音図: 助詞: オ	多少 たしょう		多少 たしょう				名詞-普通名詞-一般 
\\	「保護(ほご)」のアクセントは?2186	保護 仮名 
\\	ほご 漢字 
\\	保護 発音図: 助詞: オ	保護 ほご		保護 ほご 保護する ほごする 保護します ほごします 保護して ほごして 保護した ほごした 保護しない ほごしない 保護しなかった ほごしなかった 保護すれば ほごすれば 保護させる ほごさせる 保護される ほごされる 保護しろ ほごしろ 保護できる ほごできる 保護しよう ほごしよう				動詞-一般 
\\	「胃(い)」のアクセントは?2187	胃 仮名 
\\	い 漢字 
\\	胃 発音図: 例文: 最近、ガ痛い	胃 い		胃 い				名詞-普通名詞-一般 
\\	「預ける(あずける)」のアクセントは?2188	預ける 仮名 
\\	あずける 漢字 
\\	預ける 発音図:	預ける あずける		預ける あずける 預けます あずけます 預けて あずけて 預けた あずけた 預けない あずけない 預けなかった あずけなかった 預ければ あずければ 預けさせる あずけさせる 預けられる あずけられる 預けろ あずけろ 預けられる あずけられる 預けよう あずけよう				動詞-一般 
\\	「美しさ(うつくしさ)」のアクセントは?2189		美しさ うつくしさ うつくしさ		美しさ うつくしさ うつくしさ				名詞-普通名詞-一般 
\\	「手順(てじゅん)」のアクセントは?2190	手順 仮名 
\\	てじゅん 漢字 
\\	手順 発音図: 助詞: オ 発音図: 助詞: オ	手順 てじゅん てじゅん		手順 てじゅん てじゅん				名詞-普通名詞-一般 
\\	「結婚式(けっこんしき)」のアクセントは?2191	結婚式 仮名 
\\	けっこんしき 漢字 
\\	結婚式 発音図: 助詞: オ	結婚式 けっこんしき		結婚式 けっこんしき				名詞-普通名詞-一般 
\\	「接続(せつぞく)」のアクセントは?2192	接続 仮名 
\\	せつぞく 漢字 
\\	接続 発音図: 助詞: オ	接続 せつぞく		接続 せつぞく 接続する せつぞくする 接続します せつぞくします 接続して せつぞくして 接続した せつぞくした 接続しない せつぞくしない 接続しなかった せつぞくしなかった 接続すれば せつぞくすれば 接続させる せつぞくさせる 接続される せつぞくされる 接続しろ せつぞくしろ 接続できる せつぞくできる 接続しよう せつぞくしよう				動詞-一般 
\\	「位置(いち)」のアクセントは?2193	位置 仮名 
\\	いち 漢字 
\\	位置 発音図: 助詞: オ	位置 いち		位置 いち 位置する いちする 位置します いちします 位置して いちして 位置した いちした 位置しない いちしない 位置しなかった いちしなかった 位置すれば いちすれば 位置させる いちさせる 位置される いちされる 位置しろ いちしろ 位置できる いちできる 位置しよう いちしよう				動詞-一般 
\\	「血液(けつえき)」のアクセントは?2194	血液 仮名 
\\	けつえき 漢字 
\\	血液 発音図: 助詞: オ	血液 けつえき		血液 けつえき				名詞-普通名詞-一般 
\\	「丘(おか)」のアクセントは?2195	丘{岡} 仮名 
\\	おか 漢字 
\\	丘、岡 発音図: 助詞: オ	丘 おか		丘 おか				名詞-普通名詞-一般 
\\	「スタート(すたーと)」のアクセントは?2196	スタート 仮名 
\\	スタート、すたーと 漢字 
\\	発音図: 助詞: オ 発音図: 助詞: オ	スタート すたーと		スタート すたーと				名詞-普通名詞-一般 
\\	「滑る(すべる)」のアクセントは?2197	滑る 仮名 
\\	すべる 漢字 
\\	滑る 発音図:	滑る すべる		滑る すべる 滑ります すべります 滑って すべって 滑った すべった 滑らない すべらない 滑らなかった すべらなかった 滑れば すべれば 滑らせる すべらせる 滑られる すべられる 滑れ すべれ 滑れる すべれる 滑ろう すべろう				動詞-一般 
\\	「砂(すな)」のアクセントは?2198	砂 仮名 
\\	すな 漢字 
\\	砂 発音図: 助詞: オ	砂 すな		砂 すな				名詞-普通名詞-一般 
\\	「巻き込む(まきこむ)」のアクセントは?2199	巻き込む 仮名 
\\	まきこむ 漢字 
\\	巻き込む 発音図:	巻き込む まきこむ		巻き込む まきこむ 巻き込みます まきこみます 巻き込んで まきこんで 巻き込んだ まきこんだ 巻き込まない まきこまない 巻き込まなかった まきこまなかった 巻き込めば まきこめば 巻き込ませる まきこませる 巻き込まれる まきこまれる 巻き込め まきこめ 巻き込める まきこめる 巻き込もう まきこもう				動詞-一般 
\\	「間に合う(まにあう)」のアクセントは?2200	間に合う 仮名 
\\	まにあう 漢字 
\\	間に合う 発音図:	間に合う まにあう		間に合う まにあう 間に合います まにあいます 間に合って まにあって 間に合った まにあった 間に合わない まにあわない 間に合わなかった まにあわなかった 間に合えば まにあえば 間に合わせる まにあわせる 間に合われる まにあわれる 間に合え まにあえ 間に合える まにあえる 間に合おう まにあおう				動詞-一般 
\\	「ああ(ああ)《〜言う》」のアクセントは?2201	ああ(〜言う) 仮名 
\\	ああ、あー 漢字 
\\	発音図:							副詞 
\\	「嗚呼(ああ)」のアクセントは?2202	ああ{嗚呼} 仮名 
\\	ああ、あー 漢字 
\\	嗚呼 発音図:							副詞 
\\	「言い出す(いいだす)」のアクセントは?2203	言いだす{出} 仮名 
\\	いーだす、いいだす 漢字 
\\	言い出す、言いだす 発音図:	言い出す いいだす		言い出す いいだす 言い出します いいだします 言い出して いいだして 言い出した いいだした 言い出さない いいださない 言い出さなかった いいださなかった 言い出せば いいだせば 言い出させる いいださせる 言い出される いいだされる 言い出せ いいだせ 言い出せる いいだせる 言い出そう いいだそう				動詞-一般 
\\	「ベース(ベース)
\\	のアクセントは?2204	ベース
\\	仮名 
\\	ベース、べーす 漢字 
\\	発音図: 助詞: オ 発音図: 助詞: オ							名詞-普通名詞-一般 
\\	「余程(よほど)」のアクセントは?2205	よほど{余程} 仮名 
\\	よほど 漢字 
\\	余程 発音図:	余程 よほど		余程 よほど				副詞 
\\	「ベース(べーす)
\\	のアクセントは?2206	ベース
\\	仮名 
\\	ベース、べーす 漢字 
\\	発音図: 助詞: オ 発音図: 助詞: オ	ベース べーす べーす		ベース べーす べーす				名詞-普通名詞-一般 
\\	「ピンク(ぴんく)」のアクセントは?2207	ピンク 仮名 
\\	ピンク、ぴんく 漢字 
\\	発音図: 助詞: オ	ピンク ぴんく		ピンク ぴんく				名詞-普通名詞-一般 
\\	「孫(まご)」のアクセントは?2208	孫 仮名 
\\	まご 漢字 
\\	孫 発音図: 助詞: オ	孫 まご		孫 まご				名詞-普通名詞-一般 
\\	「テスト(てすと)」のアクセントは?2209	テスト 仮名 
\\	テスト、てすと 漢字 
\\	発音図: 助詞: オ	テスト てすと		テスト てすと				名詞-普通名詞-一般 
\\	「公表(こうひょう)」のアクセントは?2210	公表 仮名 
\\	こうひょう、こーひょー 漢字 
\\	公表 発音図: 助詞: オ	公表 こうひょう		公表 こうひょう 公表する こうひょうする 公表します こうひょうします 公表して こうひょうして 公表した こうひょうした 公表しない こうひょうしない 公表しなかった こうひょうしなかった 公表すれば こうひょうすれば 公表させる こうひょうさせる 公表される こうひょうされる 公表しろ こうひょうしろ 公表できる こうひょうできる 公表しよう こうひょうしよう				動詞-一般 
\\	「初期(しょき)」のアクセントは?2211	初期 仮名 
\\	しょき 漢字 
\\	初期 発音図: 助詞: オ	初期 しょき		初期 しょき				名詞-普通名詞-一般 
\\	「公開(こうかいする)」のアクセントは?2212	公開 仮名 
\\	こうかい、こーかい 漢字 
\\	公開 発音図: 助詞: オ	公開する こうかいする		公開する こうかいする 公開します こうかいします 公開して こうかいして 公開した こうかいした 公開しない こうかいしない 公開しなかった こうかいしなかった 公開すれば こうかいすれば 公開させる こうかいさせる 公開される こうかいされる 公開しろ こうかいしろ 公開できる こうかいできる 公開しよう こうかいしよう				動詞-一般 
\\	「福祉(ふくし)」のアクセントは?2213	福祉 仮名 
\\	ふくし 漢字 
\\	福祉 発音図: 助詞: オ	福祉 ふくし		福祉 ふくし				名詞-普通名詞-一般 
\\	「圧倒的(あっとうてき)」のアクセントは?2214	圧倒的 仮名 
\\	あっとうてき、あっとーてき 漢字 
\\	圧倒的 発音図:							形状詞-一般 
\\	「墓(はか)《お〜》」のアクセントは?2215	墓(お〜) 仮名 
\\	はか 漢字 
\\	墓 発音図: 助詞: オ	墓 はか		墓 はか				名詞-普通名詞-一般 
\\	「絡む(からむ)」のアクセントは?2216	絡む 仮名 
\\	からむ 漢字 
\\	絡む 発音図:	絡む からむ		絡む からむ 絡みます からみます 絡んで からんで 絡んだ からんだ 絡まない からまない 絡まなかった からまなかった 絡めば からめば 絡ませる からませる 絡まれる からまれる 絡め からめ 絡める からめる 絡もう からもう				動詞-一般 
\\	「円滑(えんかつ)」のアクセントは?2217	円滑 仮名 
\\	えんかつ 漢字 
\\	円滑 発音図: 助詞: オ	円滑 えんかつ		円滑 えんかつ				形状詞-一般 
\\	「主婦(しゅふ)」のアクセントは?2218	主婦 仮名 
\\	しゅふ 漢字 
\\	主婦 発音図: 助詞: オ	主婦 しゅふ		主婦 しゅふ				名詞-普通名詞-一般 
\\	「服装(ふくそう)」のアクセントは?2219	服装 仮名 
\\	ふくそう、ふくそー 漢字 
\\	服装 発音図: 助詞: オ	服装 ふくそう		服装 ふくそう				名詞-普通名詞-一般 
\\	「ラジオ(らじお)」のアクセントは?2220	ラジオ 仮名 
\\	ラジオ、らじお 漢字 
\\	発音図: 助詞: オ	ラジオ らじお		ラジオ らじお				名詞-普通名詞-一般 
\\	「森林(しんりん)」のアクセントは?2221	森林 仮名 
\\	しんりん 漢字 
\\	森林 発音図: 助詞: オ	森林 しんりん		森林 しんりん				名詞-普通名詞-一般 
\\	「加入(かにゅうする)」のアクセントは?2222	加入 仮名 
\\	かにゅー、かにゅう 漢字 
\\	加入 発音図: 助詞: オ	加入する かにゅうする		加入する かにゅうする 加入します かにゅうします 加入して かにゅうして 加入した かにゅうした 加入しない かにゅうしない 加入しなかった かにゅうしなかった 加入すれば かにゅうすれば 加入させる かにゅうさせる 加入される かにゅうされる 加入しろ かにゅうしろ 加入できる かにゅうできる 加入しよう かにゅうしよう				動詞-一般 
\\	「誕生(たんじょう)」のアクセントは?2223	誕生 仮名 
\\	たんじょう、たんじょー 漢字 
\\	誕生 発音図: 助詞: オ	誕生 たんじょう		誕生 たんじょう 誕生する たんじょうする 誕生します たんじょうします 誕生して たんじょうして 誕生した たんじょうした 誕生しない たんじょうしない 誕生しなかった たんじょうしなかった 誕生すれば たんじょうすれば 誕生させる たんじょうさせる 誕生される たんじょうされる 誕生しろ たんじょうしろ 誕生できる たんじょうできる 誕生しよう たんじょうしよう				動詞-一般 
\\	「岩(いわ)」のアクセントは?2224	岩 仮名 
\\	いわ 漢字 
\\	岩 発音図: 助詞: オ	岩 いわ		岩 いわ				名詞-普通名詞-一般 
\\	「譲る(ゆずる)」のアクセントは?2225	譲る 仮名 
\\	ゆずる 漢字 
\\	譲る 発音図:	譲る ゆずる		譲る ゆずる 譲ります ゆずります 譲って ゆずって 譲った ゆずった 譲らない ゆずらない 譲らなかった ゆずらなかった 譲れば ゆずれば 譲らせる ゆずらせる 譲られる ゆずられる 譲れ ゆずれ 譲れる ゆずれる 譲ろう ゆずろう				動詞-一般 
\\	「各種(かくしゅ)」のアクセントは?2226	各種 仮名 
\\	かくしゅ 漢字 
\\	各種 発音図: 助詞: オ	各種 かくしゅ		各種 かくしゅ				名詞-普通名詞-一般 
\\	「心掛ける(こころがける)」のアクセントは?2227		心掛ける こころがける		心掛ける こころがける 心掛けます こころがけます 心掛けて こころがけて 心掛けた こころがけた 心掛けない こころがけない 心掛けなかった こころがけなかった 心掛ければ こころがければ 心掛けさせる こころがけさせる 心掛けられる こころがけられる 心掛けろ こころがけろ 心掛けられる こころがけられる 心掛けよう こころがけよう				動詞-一般 
\\	「炎(ほのお)」のアクセントは?2228	炎 仮名 
\\	ほのお 漢字 
\\	炎 発音図: 助詞: オ 発音図: 助詞: オ	炎 ほのお		炎 ほのお				名詞-普通名詞-一般 
\\	「帽子(ぼうし)」のアクセントは?2229	帽子 仮名 
\\	ぼうし、ぼーし 漢字 
\\	帽子 発音図: 助詞: オ	帽子 ぼうし		帽子 ぼうし				名詞-普通名詞-一般 
\\	「屋敷(やしき)《お〜》」のアクセントは?2230	屋敷(お〜) 仮名 
\\	やしき 漢字 
\\	屋敷 発音図: 助詞: オ	屋敷 やしき		屋敷 やしき				名詞-普通名詞-一般 
\\	「宿(やど)《お〜》」のアクセントは?2231	宿(お〜) 仮名 
\\	やど 漢字 
\\	宿 発音図: 助詞: オ	宿 やど		宿 やど				名詞-普通名詞-一般 
\\	「傾ける(かたむける)」のアクセントは?2232	傾ける 仮名 
\\	かたむける 漢字 
\\	傾ける 発音図:	傾ける かたむける		傾ける かたむける 傾けます かたむけます 傾けて かたむけて 傾けた かたむけた 傾けない かたむけない 傾けなかった かたむけなかった 傾ければ かたむければ 傾けさせる かたむけさせる 傾けられる かたむけられる 傾けろ かたむけろ 傾けられる かたむけられる 傾けよう かたむけよう				動詞-一般 
\\	「窓口(まどぐち)」のアクセントは?2233	窓口 仮名 
\\	まどぐち 漢字 
\\	窓口 発音図: 助詞: オ	窓口 まどぐち		窓口 まどぐち				名詞-普通名詞-一般 
\\	「研究者(けんきゅうしゃ)」のアクセントは?2234	研究者 仮名 
\\	けんきゅーしゃ、けんきゅうしゃ 漢字 
\\	研究者 発音図: 助詞: オ	研究者 けんきゅうしゃ		研究者 けんきゅうしゃ				名詞-普通名詞-一般 
\\	「ぶつかる(ぶつかる)」のアクセントは?2235	ぶつかる 仮名 
\\	ぶつかる 漢字 
\\	発音図:	ぶつかる ぶつかる		ぶつかる ぶつかる ぶつかります ぶつかります ぶつかって ぶつかって ぶつかった ぶつかった ぶつからない ぶつからない ぶつからなかった ぶつからなかった ぶつかれば ぶつかれば ぶつからせる ぶつからせる ぶつかられる ぶつかられる ぶつかれ ぶつかれ ぶつかれる ぶつかれる ぶつかろう ぶつかろう				動詞-一般 
\\	「適する(てきする)」のアクセントは?2236	適する 仮名 
\\	てきする 漢字 
\\	適する 発音図:	適する てきする		適する てきする 適します てきします 適して てきして 適した てきした 適さない てきさない 適さなかった てきさなかった 適せば てきせば 適させる てきさせる 適される てきされる 適しよう てきしよう				動詞-一般 
\\	「衝撃(しょうげき)」のアクセントは?2237	衝撃 仮名 
\\	しょうげき、しょーげき 漢字 
\\	衝撃 発音図: 助詞: オ	衝撃 しょうげき		衝撃 しょうげき				名詞-普通名詞-一般 
\\	「狙い(ねらい)」のアクセントは?2238	狙い 仮名 
\\	ねらい 漢字 
\\	狙い 発音図: 助詞: オ	狙い ねらい		狙い ねらい				名詞-普通名詞-一般 
\\	「温める(あたためる)」のアクセントは?2239	温める 仮名 
\\	あたためる、あっためる 漢字 
\\	温める 発音図: 発音図:	温める あたためる		温める あたためる 温めます あたためます 温めて あたためて 温めた あたためた 温めない あたためない 温めなかった あたためなかった 温めれば あたためれば 温めさせる あたためさせる 温められる あたためられる 温めろ あたためろ 温められる あたためられる 温めよう あたためよう				動詞-一般 
\\	「死亡(しぼう)」のアクセントは?2240	死亡 仮名 
\\	しぼう、しぼー 漢字 
\\	死亡 発音図: 助詞: オ	死亡 しぼう		死亡 しぼう 死亡する しぼうする 死亡します しぼうします 死亡して しぼうして 死亡した しぼうした 死亡しない しぼうしない 死亡しなかった しぼうしなかった 死亡すれば しぼうすれば 死亡させる しぼうさせる 死亡される しぼうされる 死亡しろ しぼうしろ 死亡できる しぼうできる 死亡しよう しぼうしよう				動詞-一般 
\\	「試験(しけん)」のアクセントは?2241	試験 仮名 
\\	しけん 漢字 
\\	試験 発音図: 助詞: オ	試験 しけん		試験 しけん				名詞-普通名詞-一般 
\\	「前者(ぜんしゃ)」のアクセントは?2242	前者 仮名 
\\	ぜんしゃ 漢字 
\\	前者 発音図: 助詞: オ	前者 ぜんしゃ		前者 ぜんしゃ				名詞-普通名詞-一般 
\\	「グラス(ぐらす)」のアクセントは?2243	グラス 仮名 
\\	グラス、ぐらす 漢字 
\\	発音図: 助詞: オ	グラス ぐらす		グラス ぐらす				名詞-普通名詞-一般 
\\	「牛乳(ぎゅうにゅう)」のアクセントは?2244	牛乳 仮名 
\\	ぎゅーにゅー、ぎゅうにゅう 漢字 
\\	牛乳 発音図: 助詞: オ	牛乳 ぎゅうにゅう		牛乳 ぎゅうにゅう				名詞-普通名詞-一般 
\\	「逃れる(のがれる)」のアクセントは?2245	逃れる 仮名 
\\	のがれる 漢字 
\\	逃れる 発音図:	逃れる のがれる		逃れる のがれる 逃れます のがれます 逃れて のがれて 逃れた のがれた 逃れない のがれない 逃れなかった のがれなかった 逃れれば のがれれば 逃れさせる のがれさせる 逃れられる のがれられる 逃れろ のがれろ 逃れられる のがれられる 逃れよう のがれよう				動詞-一般 
\\	「毛(け)」のアクセントは?2246	毛 仮名 
\\	け、もう、もー 漢字 
\\	毛 発音図: 助詞: オ 例文: 心配事が重なりガ抜けた 発音図:	毛 け		毛 け				名詞-普通名詞-一般 
\\	「曖昧(あいまい)」のアクセントは?2247	あいまい{曖昧} 仮名 
\\	あいまい 漢字 
\\	曖昧 発音図:	曖昧 あいまい		曖昧 あいまい				形状詞-一般 
\\	「研究(けんきゅう)」のアクセントは?2248	研究 仮名 
\\	けんきゅー、けんきゅう 漢字 
\\	研究 発音図: 助詞: オ	研究 けんきゅう		研究 けんきゅう 研究する けんきゅうする 研究します けんきゅうします 研究して けんきゅうして 研究した けんきゅうした 研究しない けんきゅうしない 研究しなかった けんきゅうしなかった 研究すれば けんきゅうすれば 研究させる けんきゅうさせる 研究される けんきゅうされる 研究しろ けんきゅうしろ 研究できる けんきゅうできる 研究しよう けんきゅうしよう				動詞-一般 
\\	「曖昧な(あいまいな)」のアクセントは?2249		曖昧な あいまいな		曖昧な あいまいな 曖昧です あいまいです 曖昧で あいまいで 曖昧だった あいまいだった 曖昧じゃない あいまいじゃない 曖昧じゃなかった あいまいじゃなかった 曖昧ならば あいまいならば 曖昧だ あいまいだ 曖昧に あいまいに				形状詞-一般 
\\	「死者(ししゃ)」のアクセントは?2250	死者 仮名 
\\	ししゃ 漢字 
\\	死者 発音図: 助詞: オ	死者 ししゃ		死者 ししゃ				名詞-普通名詞-一般 
\\	「教科書(きょうかしょ)」のアクセントは?2251	教科書 仮名 
\\	きょうかしょ、きょーかしょ 漢字 
\\	教科書 発音図: 助詞: オ	教科書 きょうかしょ		教科書 きょうかしょ				名詞-普通名詞-一般 
\\	「調査(ちょうさ)」のアクセントは?2252	調査 仮名 
\\	ちょうさ、ちょーさ 漢字 
\\	調査 発音図: 助詞: オ	調査 ちょうさ		調査 ちょうさ 調査する ちょうさする 調査します ちょうさします 調査して ちょうさして 調査した ちょうさした 調査しない ちょうさしない 調査しなかった ちょうさしなかった 調査すれば ちょうさすれば 調査させる ちょうささせる 調査される ちょうさされる 調査しろ ちょうさしろ 調査できる ちょうさできる 調査しよう ちょうさしよう				動詞-一般 
\\	「せめて(せめて)《〜これだけでも》」のアクセントは?2253	せめて(〜これだけでも) 仮名 
\\	せめて 漢字 
\\	発音図: 例文: コれだけでも							副詞 
\\	「依頼(いらい)」のアクセントは?2254	依頼 仮名 
\\	いらい 漢字 
\\	依頼 発音図: 助詞: オ	依頼 いらい		依頼 いらい 依頼する いらいする 依頼します いらいします 依頼して いらいして 依頼した いらいした 依頼しない いらいしない 依頼しなかった いらいしなかった 依頼すれば いらいすれば 依頼させる いらいさせる 依頼される いらいされる 依頼しろ いらいしろ 依頼できる いらいできる 依頼しよう いらいしよう				動詞-一般 
\\	「呼び掛ける(よびかける)」のアクセントは?2255		呼び掛ける よびかける		呼び掛ける よびかける 呼び掛けます よびかけます 呼び掛けて よびかけて 呼び掛けた よびかけた 呼び掛けない よびかけない 呼び掛けなかった よびかけなかった 呼び掛ければ よびかければ 呼び掛けさせる よびかけさせる 呼び掛けられる よびかけられる 呼び掛けろ よびかけろ 呼び掛けられる よびかけられる 呼び掛けよう よびかけよう				動詞-一般 
\\	「有り(あり)」のアクセントは?2256	有り 仮名 
\\	あり 漢字 
\\	有り 発音図:							名詞-普通名詞-一般 
\\	「アップ(あっぷ)」のアクセントは?2257	アップ 仮名 
\\	アップ、あっぷ 漢字 
\\	発音図: 助詞: オ	アップ あっぷ		アップ あっぷ アップする あっぷする アップします あっぷします アップして あっぷして アップした あっぷした アップしない あっぷしない アップしなかった あっぷしなかった アップすれば あっぷすれば アップさせる あっぷさせる アップされる あっぷされる アップしろ あっぷしろ アップできる あっぷできる アップしよう あっぷしよう				動詞-一般 
\\	「茹でる(ゆでる)」のアクセントは?2258	ゆでる{茹} 仮名 
\\	ゆでる 漢字 
\\	茹でる 発音図:	茹でる ゆでる		茹でる ゆでる 茹でます ゆでます 茹でて ゆでて 茹でた ゆでた 茹でない ゆでない 茹でなかった ゆでなかった 茹でれば ゆでれば 茹でさせる ゆでさせる 茹でられる ゆでられる 茹でろ ゆでろ 茹でられる ゆでられる 茹でよう ゆでよう				動詞-一般 
\\	「殴る(なぐる)」のアクセントは?2259	殴る 仮名 
\\	なぐる 漢字 
\\	殴る 発音図:	殴る なぐる		殴る なぐる 殴ります なぐります 殴って なぐって 殴った なぐった 殴らない なぐらない 殴らなかった なぐらなかった 殴れば なぐれば 殴らせる なぐらせる 殴られる なぐられる 殴れ なぐれ 殴れる なぐれる 殴ろう なぐろう				動詞-一般 
\\	「ガラス(がらす)」のアクセントは?2260	ガラス 仮名 
\\	ガラス、がらす 漢字 
\\	発音図: 助詞: オ	ガラス がらす		ガラス がらす				名詞-普通名詞-一般 
\\	「平均(へいきん)」のアクセントは?2261	平均 仮名 
\\	へいきん 漢字 
\\	平均 発音図: 助詞: オ	平均 へいきん		平均 へいきん				名詞-普通名詞-一般 
\\	「入院(にゅういん)」のアクセントは?2262	入院 仮名 
\\	にゅーいん、にゅういん 漢字 
\\	入院 発音図: 助詞: オ	入院 にゅういん		入院 にゅういん 入院する にゅういんする 入院します にゅういんします 入院して にゅういんして 入院した にゅういんした 入院しない にゅういんしない 入院しなかった にゅういんしなかった 入院すれば にゅういんすれば 入院させる にゅういんさせる 入院される にゅういんされる 入院しろ にゅういんしろ 入院できる にゅういんできる 入院しよう にゅういんしよう				動詞-一般 
\\	「活発な(かっぱつな)」のアクセントは?2263		活発な かっぱつな		活発な かっぱつな 活発です かっぱつです 活発で かっぱつで 活発だった かっぱつだった 活発じゃない かっぱつじゃない 活発じゃなかった かっぱつじゃなかった 活発ならば かっぱつならば 活発だ かっぱつだ 活発に かっぱつに				形状詞-一般 
\\	「活発(かっぱつ)」のアクセントは?2264	活発 仮名 
\\	かっぱつ 漢字 
\\	活発 発音図: 助詞: オ	活発 かっぱつ		活発 かっぱつ				形状詞-一般 
\\	「嫁(よめ)」のアクセントは?2265	嫁 仮名 
\\	よめ 漢字 
\\	嫁 発音図: 助詞: オ	嫁 よめ		嫁 よめ				名詞-普通名詞-一般 
\\	「間違える(まちがえる)」のアクセントは?2266		間違える まちがえる まちがえる		間違える まちがえる まちがえる 間違えます まちがえます 間違えて まちがえて 間違えた まちがえた 間違えない まちがえない まちがえない 間違えなかった まちがえなかった まちがえなかった 間違えれば まちがえれば まちがえれば 間違えさせる まちがえさせる 間違えられる まちがえられる 間違えろ まちがえろ まちがえろ 間違えよう まちがえよう				動詞-一般 
\\	「頻繁な(ひんぱんな)」のアクセントは?2267		頻繁な ひんぱんな		頻繁な ひんぱんな 頻繁です ひんぱんです 頻繁で ひんぱんで 頻繁だった ひんぱんだった 頻繁じゃない ひんぱんじゃない 頻繁じゃなかった ひんぱんじゃなかった 頻繁ならば ひんぱんならば 頻繁だ ひんぱんだ 頻繁に ひんぱんに				形状詞-一般 
\\	「頻繁(ひんぱん)」のアクセントは?2268	頻繁 仮名 
\\	ひんぱん 漢字 
\\	頻繁 発音図: 助詞: オ	頻繁 ひんぱん		頻繁 ひんぱん				形状詞-一般 
\\	「振るう(ふるう)」のアクセントは?2269	振るう 仮名 
\\	ふるう 漢字 
\\	振るう 発音図:							動詞-一般 
\\	「プレゼント(ぷれぜんと)」のアクセントは?2270	プレゼント 仮名 
\\	プレゼント、ぷれぜんと 漢字 
\\	発音図: 助詞: オ	プレゼント ぷれぜんと		プレゼント ぷれぜんと				名詞-普通名詞-一般 
\\	「経緯(けいい)」のアクセントは?2271	経緯 仮名 
\\	けいい 漢字 
\\	経緯 発音図: 助詞: オ	経緯 けいい		経緯 けいい				名詞-普通名詞-一般 
\\	「茨(ばら)」のアクセントは?2273	ばら{薔薇;茨} 仮名 
\\	ばら 漢字 
\\	薔薇、茨 発音図: 助詞: オ	薔薇 ばら		薔薇 ばら				名詞-普通名詞-一般 
\\	「苦労(くろう)」のアクセントは?2274	苦労 仮名 
\\	くろう、くろー 漢字 
\\	苦労 発音図: 助詞: オ	苦労 くろう		苦労 くろう 苦労する くろうする 苦労します くろうします 苦労して くろうして 苦労した くろうした 苦労しない くろうしない 苦労しなかった くろうしなかった 苦労すれば くろうすれば 苦労させる くろうさせる 苦労される くろうされる 苦労しろ くろうしろ 苦労できる くろうできる 苦労しよう くろうしよう				動詞-一般 
\\	「新鮮(しんせん)」のアクセントは?2275	新鮮 仮名 
\\	しんせん 漢字 
\\	新鮮 発音図: 助詞: オ	新鮮 しんせん		新鮮 しんせん				形状詞-一般 
\\	「壊す(こわす)」のアクセントは?2276	壊す 仮名 
\\	こわす 漢字 
\\	壊す 発音図:	壊す こわす		壊す こわす 壊します こわします 壊して こわして 壊した こわした 壊さない こわさない 壊さなかった こわさなかった 壊せば こわせば 壊させる こわさせる 壊される こわされる 壊せ こわせ 壊せる こわせる 壊そう こわそう				動詞-一般 
\\	「新鮮な(しんせんな)」のアクセントは?2277		新鮮な しんせんな		新鮮な しんせんな 新鮮です しんせんです 新鮮で しんせんで 新鮮だった しんせんだった 新鮮じゃない しんせんじゃない 新鮮じゃなかった しんせんじゃなかった 新鮮ならば しんせんならば 新鮮だ しんせんだ 新鮮に しんせんに				形状詞-一般 
\\	「神経(しんけい)」のアクセントは?2278	神経 仮名 
\\	しんけい 漢字 
\\	神経 発音図: 助詞: オ	神経 しんけい		神経 しんけい				名詞-普通名詞-一般 
\\	「台所(だいどころ)」のアクセントは?2279	台所 仮名 
\\	だいどころ 漢字 
\\	台所 発音図: 助詞: オ	台所 だいどころ		台所 だいどころ				名詞-普通名詞-一般 
\\	「コスト(こすと)」のアクセントは?2280	コスト 仮名 
\\	コスト、こすと 漢字 
\\	発音図: 助詞: オ	コスト こすと		コスト こすと				名詞-普通名詞-一般 
\\	「伝統的な(でんとうてきな)」のアクセントは?2281	伝統的 仮名 
\\	でんとうてき、でんとーてき 漢字 
\\	伝統的 発音図:	伝統的な でんとうてきな		伝統的な でんとうてきな 伝統的です でんとうてきです 伝統的で でんとうてきで 伝統的だった でんとうてきだった 伝統的じゃない でんとうてきじゃない 伝統的じゃなかった でんとうてきじゃなかった 伝統的ならば でんとうてきならば 伝統的だ でんとうてきだ 伝統的に でんとうてきに				形状詞-一般 
\\	「届け出(とどけで)」のアクセントは?2282	届け出 仮名 
\\	とどけいで、とどけで 漢字 
\\	届け出 発音図: 助詞: オ 発音図: 助詞: オ	届け出 とどけで		届け出 とどけで				名詞-普通名詞-一般 
\\	「固める(かためる)」のアクセントは?2283	固める 仮名 
\\	かためる 漢字 
\\	固める 発音図:	固める かためる		固める かためる 固めます かためます 固めて かためて 固めた かためた 固めない かためない 固めなかった かためなかった 固めれば かためれば 固めさせる かためさせる 固められる かためられる 固めろ かためろ 固められる かためられる 固めよう かためよう				動詞-一般 
\\	「判決(はんけつ)」のアクセントは?2284	判決 仮名 
\\	はんけつ 漢字 
\\	判決 発音図: 助詞: オ 発音図: 助詞: オ	判決 はんけつ		判決 はんけつ				名詞-普通名詞-一般 
\\	「浮く(うく)」のアクセントは?2285	浮く 仮名 
\\	うく 漢字 
\\	浮く 発音図:	浮く うく		浮く うく 浮かびます うかびます 浮いて ういて 浮いた ういた 浮かない うかない 浮かなかった うかなかった 浮けば うけば 浮かせる うかせる 浮かれる うかれる 浮け うけ 浮ける うける 浮こう うこう				動詞-一般 
\\	「景気(けいき)《〜がいい》」のアクセントは?2286	景気(〜がいい) 仮名 
\\	けいき 漢字 
\\	景気 発音図: 例文: ガいい	景気 けいき		景気 けいき				名詞-普通名詞-一般 
\\	「稀な(まれな)」のアクセントは?2287	まれ{稀} 仮名 
\\	まれ 漢字 
\\	稀 発音図: 助詞: ダ 発音図: 助詞: ダ	稀な まれな		稀な まれな 稀です まれです 稀で まれで 稀だった まれだった 稀じゃない まれじゃない 稀じゃなかった まれじゃなかった 稀ならば まれならば 稀だ まれだ 稀に まれに				形状詞-一般 
\\	「外部(がいぶ)」のアクセントは?2288	外部 仮名 
\\	がいぶ 漢字 
\\	外部 発音図: 助詞: オ	外部 がいぶ		外部 がいぶ				名詞-普通名詞-一般 
\\	「泳ぐ(およぐ)」のアクセントは?2289	泳ぐ 仮名 
\\	およぐ 漢字 
\\	泳ぐ 発音図:	泳ぐ およぐ		泳ぐ およぐ 泳ぎます およぎます 泳いで およいで 泳いだ およいだ 泳がない およがない 泳がなかった およがなかった 泳げば およげば 泳がせる およがせる 泳がれる およがれる 泳げ およげ 泳げる およげる 泳ごう およごう				動詞-一般 
\\	「磨く(みがく)」のアクセントは?2290	磨く 仮名 
\\	みがく 漢字 
\\	磨く 発音図:	磨く みがく		磨く みがく 磨きます みがきます 磨いて みがいて 磨いた みがいた 磨かない みがかない 磨かなかった みがかなかった 磨けば みがけば 磨かせる みがかせる 磨かれる みがかれる 磨け みがけ 磨ける みがける 磨こう みがこう				動詞-一般 
\\	「規則(きそく)」のアクセントは?2291	規則 仮名 
\\	きそく 漢字 
\\	規則 発音図: 助詞: オ 発音図: 助詞: オ	規則 きそく きそく		規則 きそく きそく				名詞-普通名詞-一般 
\\	「汚れる(よごれる)」のアクセントは?2292	汚れる 仮名 
\\	けがれる、よごれる 漢字 
\\	汚れる 発音図: 発音図:	汚れる よごれる		汚れる よごれる 汚れます よごれます 汚れて よごれて 汚れた よごれた 汚れない よごれない 汚れなかった よごれなかった 汚れれば よごれれば 汚れさせる よごれさせる 汚れられる よごれられる 汚れろ よごれろ 汚れられる よごれられる 汚れよう よごれよう				動詞-一般 
\\	「輪(わ)」のアクセントは?2293	輪 仮名 
\\	わ 漢字 
\\	輪 発音図: 助詞: オ	輪 わ		輪 わ				名詞-普通名詞-一般 
\\	「例外(れいがい)」のアクセントは?2294	例外 仮名 
\\	れいがい 漢字 
\\	例外 発音図: 助詞: オ	例外 れいがい		例外 れいがい				名詞-普通名詞-一般 
\\	「助かる(たすかる)」のアクセントは?2295	助かる 仮名 
\\	たすかる 漢字 
\\	助かる 発音図:	助かる たすかる		助かる たすかる 助かります たすかります 助かって たすかって 助かった たすかった 助からない たすからない 助からなかった たすからなかった 助かれば たすかれば 助からせる たすからせる 助かられる たすかられる 助かれ たすかれ 助かろう たすかろう				動詞-一般 
\\	「雌(めす)」のアクセントは?2296	雌 仮名 
\\	めす 漢字 
\\	雌 発音図: 助詞: オ	雌 めす		雌 めす				名詞-普通名詞-一般 
\\	「論ずる(ろんずる)」のアクセントは?2297	論ずる 仮名 
\\	ろんずる 漢字 
\\	論ずる 発音図: 発音図:							動詞-一般 
\\	「朝食(ちょうしょく)」のアクセントは?2298	朝食 仮名 
\\	ちょうしょく、ちょーしょく 漢字 
\\	朝食 発音図: 助詞: オ	朝食 ちょうしょく		朝食 ちょうしょく				名詞-普通名詞-一般 
\\	「率いる(ひきいる)」のアクセントは?2299	率いる 仮名 
\\	ひきいる 漢字 
\\	率いる 発音図:	率いる ひきいる		率いる ひきいる 率います ひきいます 率いて ひきいて 率いた ひきいた 率いない ひきいない 率いなかった ひきいなかった 率いれば ひきいれば 率いさせる ひきいさせる 率いられる ひきいられる 率いろ ひきいろ 率いられる ひきいられる 率いよう ひきいよう				動詞-一般 
\\	「価値観(かちかん)」のアクセントは?2300	価値観 仮名 
\\	かちかん 漢字 
\\	価値観 発音図: 助詞: オ 発音図: 助詞: オ							名詞-普通名詞-一般 
\\	「片付ける(かたづける)」のアクセントは?2301		片付ける かたづける		片付ける かたづける 片付けます かたづけます 片付けて かたづけて 片付けた かたづけた 片付けない かたづけない 片付けなかった かたづけなかった 片付ければ かたづければ 片付けさせる かたづけさせる 片付けられる かたづけられる 片付けろ かたづけろ 片付けられる かたづけられる 片付けよう かたづけよう				動詞-一般 
\\	「密かな(ひそかな)」のアクセントは?2302	ひそか{密} 仮名 
\\	ひそか 漢字 
\\	密か 発音図: 発音図:	密かな ひそかな ひそかな		密かな ひそかな ひそかな 密かです ひそかです ひそかです 密かで ひそかで ひそかで 密かだった ひそかだった ひそかだった 密かじゃない ひそかじゃない ひそかじゃない 密かじゃなかった ひそかじゃなかった ひそかじゃなかった 密かならば ひそかならば ひそかならば 密かだ ひそかだ ひそかだ 密かに ひそかに ひそかに				形状詞-一般 
\\	「作者(さくしゃ)」のアクセントは?2303	作者 仮名 
\\	さくしゃ 漢字 
\\	作者 発音図: 助詞: オ 発音図: 助詞: オ	作者 さくしゃ		作者 さくしゃ				名詞-普通名詞-一般 
\\	「回答(かいとう)」のアクセントは?2304	回答 仮名 
\\	かいとう、かいとー 漢字 
\\	回答 発音図: 助詞: オ	回答 かいとう		回答 かいとう 回答する かいとうする 回答します かいとうします 回答して かいとうして 回答した かいとうした 回答しない かいとうしない 回答しなかった かいとうしなかった 回答すれば かいとうすれば 回答させる かいとうさせる 回答される かいとうされる 回答しろ かいとうしろ 回答できる かいとうできる 回答しよう かいとうしよう				動詞-一般 
\\	「軍隊(ぐんたい)」のアクセントは?2305	軍隊 仮名 
\\	ぐんたい 漢字 
\\	軍隊 発音図: 助詞: オ	軍隊 ぐんたい		軍隊 ぐんたい				名詞-普通名詞-一般 
\\	「氷(こおり)」のアクセントは?2306	氷 仮名 
\\	こーり、こおり 漢字 
\\	氷 発音図: 助詞: オ	氷 こおり		氷 こおり				名詞-普通名詞-一般 
\\	「角度(かくど)」のアクセントは?2307	角度 仮名 
\\	かくど 漢字 
\\	角度 発音図: 助詞: オ	角度 かくど		角度 かくど				名詞-普通名詞-一般 
\\	「戦後(せんご)」のアクセントは?2308	戦後 仮名 
\\	せんご 漢字 
\\	戦後 発音図: 助詞: オ 発音図: 助詞: オ	戦後 せんご		戦後 せんご				副詞 
\\	「発売(はつばい)」のアクセントは?2309	発売 仮名 
\\	はつばい 漢字 
\\	発売 発音図: 助詞: オ	発売 はつばい		発売 はつばい 発売する はつばいする 発売します はつばいします 発売して はつばいして 発売した はつばいした 発売しない はつばいしない 発売しなかった はつばいしなかった 発売すれば はつばいすれば 発売させる はつばいさせる 発売される はつばいされる 発売しろ はつばいしろ 発売できる はつばいできる 発売しよう はつばいしよう				動詞-一般 
\\	「漏れる(もれる)」のアクセントは?2310	漏れる 仮名 
\\	もれる 漢字 
\\	漏れる 発音図:	漏れる もれる		漏れる もれる 漏れます もれます 漏れて もれて 漏れた もれた 漏れない もれない 漏れなかった もれなかった 漏れれば もれれば 漏れさせる もれさせる 漏れられる もれられる 漏れろ もれろ 漏れよう もれよう				動詞-一般 
\\	「跡(あと)」のアクセントは?2311	跡 仮名 
\\	あと 漢字 
\\	跡 発音図: 助詞: オ	跡 あと		跡 あと				名詞-普通名詞-一般 
\\	「故郷(こきょう)」のアクセントは?2313	故郷 仮名 
\\	こきょう、こきょー 漢字 
\\	故郷 発音図: 助詞: オ	故郷 こきょう		故郷 こきょう				名詞-普通名詞-一般 
\\	「盛り上がる(もりあがる)」のアクセントは?2314	盛り上がる 仮名 
\\	もりあがる 漢字 
\\	盛り上がる 発音図:	盛り上がる もりあがる		盛り上がる もりあがる 盛り上がります もりあがります 盛り上がって もりあがって 盛り上がった もりあがった 盛り上がらない もりあがらない 盛り上がらなかった もりあがらなかった 盛り上がれば もりあがれば 盛り上がらせる もりあがらせる 盛り上がられる もりあがられる 盛り上がれ もりあがれ 盛り上がれる もりあがれる 盛り上がろう もりあがろう				動詞-一般 
\\	「暴力(ぼうりょく)」のアクセントは?2315	暴力 仮名 
\\	ぼうりょく、ぼーりょく 漢字 
\\	暴力 発音図: 助詞: オ	暴力 ぼうりょく		暴力 ぼうりょく				名詞-普通名詞-一般 
\\	「委員会(いいんかい)」のアクセントは?2316	委員会 仮名 
\\	いいんかい 漢字 
\\	委員会 発音図: 助詞: オ	委員会 いいんかい		委員会 いいんかい				名詞-普通名詞-一般 
\\	「あら(あら)」のアクセントは?2317	あら 仮名 
\\	あら 漢字 
\\	発音図:							感動詞-一般 
\\	「粗(あら)《欠点;魚の〜》」のアクセントは?2318	あら{粗}(欠点;魚の〜) 仮名 
\\	あら 漢字 
\\	粗 発音図: 助詞: オ							感動詞-一般 
\\	「支給(しきゅうする)」のアクセントは?2319	支給 仮名 
\\	しきゅー、しきゅう 漢字 
\\	支給 発音図: 助詞: オ	支給する しきゅうする		支給する しきゅうする 支給します しきゅうします 支給して しきゅうして 支給した しきゅうした 支給しない しきゅうしない 支給しなかった しきゅうしなかった 支給すれば しきゅうすれば 支給させる しきゅうさせる 支給される しきゅうされる 支給しろ しきゅうしろ 支給できる しきゅうできる 支給しよう しきゅうしよう				動詞-一般 
\\	「方式(ほうしき)」のアクセントは?2320	方式 仮名 
\\	ほうしき、ほーしき 漢字 
\\	方式 発音図: 助詞: オ	方式 ほうしき		方式 ほうしき				名詞-普通名詞-一般 
\\	「優秀な(ゆうしゅうな)」のアクセントは?2321		優秀な ゆうしゅうな		優秀な ゆうしゅうな 優秀です ゆうしゅうです 優秀で ゆうしゅうで 優秀だった ゆうしゅうだった 優秀じゃない ゆうしゅうじゃない 優秀じゃなかった ゆうしゅうじゃなかった 優秀ならば ゆうしゅうならば 優秀だ ゆうしゅうだ 優秀に ゆうしゅうに				形状詞-一般 
\\	「優秀(ゆうしゅう)」のアクセントは?2322	優秀 仮名 
\\	ゆーしゅー、ゆうしゅう 漢字 
\\	優秀 発音図: 助詞: オ	優秀 ゆうしゅう		優秀 ゆうしゅう				形状詞-一般 
\\	「著者(ちょしゃ)」のアクセントは?2323	著者 仮名 
\\	ちょしゃ 漢字 
\\	著者 発音図: 助詞: オ	著者 ちょしゃ		著者 ちょしゃ				名詞-普通名詞-一般 
\\	「配置(はいち)」のアクセントは?2324	配置 仮名 
\\	はいち 漢字 
\\	配置 発音図: 助詞: オ 発音図: 助詞: オ	配置 はいち		配置 はいち 配置する はいちする 配置します はいちします 配置して はいちして 配置した はいちした 配置しない はいちしない 配置しなかった はいちしなかった 配置すれば はいちすれば 配置させる はいちさせる 配置される はいちされる 配置しろ はいちしろ 配置できる はいちできる 配置しよう はいちしよう				動詞-一般 
\\	「スープ(すーぷ)」のアクセントは?2325	スープ 仮名 
\\	スープ、すーぷ 漢字 
\\	発音図: 助詞: オ	スープ すーぷ		スープ すーぷ				名詞-普通名詞-一般 
\\	「英国(えいこく)」のアクセントは?2326	英国 仮名 
\\	えいこく 漢字 
\\	英国 発音図: 助詞: オ							名詞-固有名詞-地名-国 
\\	「パワー(ぱわー)」のアクセントは?2327	パワー 仮名 
\\	パワー、ぱわー 漢字 
\\	発音図: 助詞: オ	パワー ぱわー		パワー ぱわー				名詞-普通名詞-一般 
\\	「嘗める(なめる)」のアクセントは?2328	なめる{甞;舐} 仮名 
\\	なめる 漢字 
\\	嘗める、舐める 発音図:	嘗める なめる		嘗める なめる 嘗めます なめます 嘗めて なめて 嘗めた なめた 嘗めない なめない 嘗めなかった なめなかった 嘗めれば なめれば 嘗めさせる なめさせる 嘗められる なめられる 嘗めろ なめろ 嘗められる なめられる 嘗めよう なめよう				動詞-一般 
\\	「五つ(いつつ)」のアクセントは?2329	五つ 仮名 
\\	いつつ 漢字 
\\	五つ 発音図: 助詞: オ	五つ いつつ		五つ いつつ				名詞-数詞 
\\	「取り扱い(とりあつかい)」のアクセントは?2330	取り扱い 仮名 
\\	とりあつかい 漢字 
\\	取り扱い 発音図: 助詞: オ							名詞-普通名詞-一般 
\\	「見通し(みとおし)」のアクセントは?2331	見通し 仮名 
\\	みとーし、みとおし 漢字 
\\	見通し 発音図: 助詞: オ	見通し みとおし		見通し みとおし				名詞-普通名詞-一般 
\\	「自衛隊(じえいたい)」のアクセントは?2332	自衛隊 仮名 
\\	じえいたい 漢字 
\\	自衛隊 発音図: 助詞: オ							名詞-普通名詞-一般 
\\	「語(ご)」のアクセントは?2333	語 仮名 
\\	ご 漢字 
\\	語 発音図: 例文: このワどういう意味か							名詞-普通名詞-一般 
\\	「乗り込む(のりこむ)」のアクセントは?2334	乗り込む 仮名 
\\	のりこむ 漢字 
\\	乗り込む 発音図:	乗り込む のりこむ		乗り込む のりこむ 乗り込みます のりこみます 乗り込んで のりこんで 乗り込んだ のりこんだ 乗り込まない のりこまない 乗り込まなかった のりこまなかった 乗り込めば のりこめば 乗り込ませる のりこませる 乗り込まれる のりこまれる 乗り込め のりこめ 乗り込める のりこめる 乗り込もう のりこもう				動詞-一般 
\\	「ナンバー(なんばー)」のアクセントは?2335	ナンバー 仮名 
\\	ナンバー、なんばー 漢字 
\\	発音図: 助詞: オ	ナンバー なんばー		ナンバー なんばー				名詞-普通名詞-一般 
\\	「型(かた)」のアクセントは?2336	型 仮名 
\\	かた 漢字 
\\	型 発音図: 助詞: オ	型 かた		型 かた				名詞-普通名詞-一般 
\\	「一環(いっかん)」のアクセントは?2337	一環 仮名 
\\	いっかん 漢字 
\\	一環 発音図: 助詞: オ 発音図: 助詞: オ							名詞-普通名詞-一般 
\\	「怪しい(あやしい)」のアクセントは?2338	怪しい 仮名 
\\	あやしい 漢字 
\\	怪しい 発音図: 発音図:	怪しい あやしい あやしい		怪しい あやしい あやしい 怪しいです あやしいです 怪しくて あやしくて あやしくて 怪しかった あやしかった あやしかった 怪しくない あやしくない あやしくない あやしくない 怪しくなかった あやしくなかった あやしくなかった あやしくなかった 怪しければ あやしければ あやしければ 怪しい あやしい あやしい 怪しく あやしく あやしく あやしく				形容詞-一般 
\\	「順調な(じゅんちょうな)」のアクセントは?2339		順調な じゅんちょうな		順調な じゅんちょうな 順調です じゅんちょうです 順調で じゅんちょうで 順調だった じゅんちょうだった 順調じゃない じゅんちょうじゃない 順調じゃなかった じゅんちょうじゃなかった 順調ならば じゅんちょうならば 順調だ じゅんちょうだ 順調に じゅんちょうに				形状詞-一般 
\\	「順調(じゅんちょう)」のアクセントは?2340	順調 仮名 
\\	じゅんちょう、じゅんちょー 漢字 
\\	順調 発音図: 助詞: オ	順調 じゅんちょう		順調 じゅんちょう				形状詞-一般 
\\	「演ずる(えんずる)」のアクセントは?2341	演ずる 仮名 
\\	えんずる 漢字 
\\	演ずる 発音図: 発音図:							動詞-一般 
\\	「進行(しんこう)」のアクセントは?2342	進行 仮名 
\\	しんこう、しんこー 漢字 
\\	進行 発音図: 助詞: オ	進行 しんこう		進行 しんこう 進行する しんこうする 進行します しんこうします 進行して しんこうして 進行した しんこうした 進行しない しんこうしない 進行しなかった しんこうしなかった 進行すれば しんこうすれば 進行させる しんこうさせる 進行される しんこうされる 進行しろ しんこうしろ 進行できる しんこうできる 進行しよう しんこうしよう				動詞-一般 
\\	「隙間(すきま)」のアクセントは?2343		隙間 すきま		隙間 すきま				名詞-普通名詞-一般 
\\	「飛ばす(とばす)」のアクセントは?2344	飛ばす 仮名 
\\	とばす 漢字 
\\	飛ばす 発音図:	飛ばす とばす		飛ばす とばす 飛ばします とばします 飛ばして とばして 飛ばした とばした 飛ばさない とばさない 飛ばさなかった とばさなかった 飛ばせば とばせば 飛ばさせる とばさせる 飛ばされる とばされる 飛ばせ とばせ 飛ばせる とばせる 飛ばそう とばそう				動詞-一般 
\\	「林檎(りんご)」のアクセントは?2345	りんご{林檎} 仮名 
\\	りんご 漢字 
\\	林檎 発音図: 助詞: オ	りんご りんご		りんご りんご				名詞-普通名詞-一般 
\\	「発達(はったつ)」のアクセントは?2346	発達 仮名 
\\	はったつ 漢字 
\\	発達 発音図: 助詞: オ	発達 はったつ		発達 はったつ 発達する はったつする 発達します はったつします 発達して はったつして 発達した はったつした 発達しない はったつしない 発達しなかった はったつしなかった 発達すれば はったつすれば 発達させる はったつさせる 発達される はったつされる 発達しろ はったつしろ 発達できる はったつできる 発達しよう はったつしよう				動詞-一般 
\\	「付き合い(つきあい)」のアクセントは?2347	つきあい{付合} 仮名 
\\	つきあい 漢字 
\\	付き合い 発音図: 助詞: オ 発音図: 助詞: オ	付き合い つきあい		付き合い つきあい				名詞-普通名詞-一般 
\\	「追い掛ける(おいかける)」のアクセントは?2348		追い掛ける おいかける		追い掛ける おいかける 追い掛けます おいかけます 追い掛けて おいかけて 追い掛けた おいかけた 追い掛けない おいかけない 追い掛けなかった おいかけなかった 追い掛ければ おいかければ 追い掛けさせる おいかけさせる 追い掛けられる おいかけられる 追い掛けろ おいかけろ 追い掛けられる おいかけられる 追い掛けよう おいかけよう				動詞-一般 
\\	「向上(こうじょう)」のアクセントは?2349	向上 仮名 
\\	こうじょう、こーじょー 漢字 
\\	向上 発音図: 助詞: オ	向上 こうじょう		向上 こうじょう 向上する こうじょうする 向上します こうじょうします 向上して こうじょうして 向上した こうじょうした 向上しない こうじょうしない 向上しなかった こうじょうしなかった 向上すれば こうじょうすれば 向上させる こうじょうさせる 向上される こうじょうされる 向上しろ こうじょうしろ 向上できる こうじょうできる 向上しよう こうじょうしよう				動詞-一般 
\\	「緊張(きんちょう)」のアクセントは?2350	緊張 仮名 
\\	きんちょう、きんちょー 漢字 
\\	緊張 発音図: 助詞: オ	緊張 きんちょう		緊張 きんちょう 緊張する きんちょうする 緊張します きんちょうします 緊張して きんちょうして 緊張した きんちょうした 緊張しない きんちょうしない 緊張しなかった きんちょうしなかった 緊張すれば きんちょうすれば 緊張させる きんちょうさせる 緊張される きんちょうされる 緊張しろ きんちょうしろ 緊張できる きんちょうできる 緊張しよう きんちょうしよう				動詞-一般 
\\	「盗む(ぬすむ)」のアクセントは?2351	盗む 仮名 
\\	ぬすむ 漢字 
\\	盗む 発音図:	盗む ぬすむ		盗む ぬすむ 盗みます ぬすみます 盗んで ぬすんで 盗んだ ぬすんだ 盗まない ぬすまない 盗まなかった ぬすまなかった 盗めば ぬすめば 盗ませる ぬすませる 盗まれる ぬすまれる 盗め ぬすめ 盗める ぬすめる 盗もう ぬすもう				動詞-一般 
\\	「善(よし)」のアクセントは?2352	よし{良;善;好} 仮名 
\\	よし 漢字 
\\	善、良、好 発音図:							感動詞-一般 
\\	「止し(よし)《〜にする》」のアクセントは?2353	よし{止}(〜にする) 仮名 
\\	よし 漢字 
\\	止し 発音図: 例文: ニする 発音図: 例文: ニする							感動詞-一般 
\\	「縦(よし)《〜…であろうとも》」のアクセントは?2354	よし{縦}(〜…であろうとも) 仮名 
\\	よし 漢字 
\\	縦 発音図:							感動詞-一般 
\\	「葦(よし)」のアクセントは?2355	よし{葦} 仮名 
\\	よし 漢字 
\\	葦 発音図: 助詞: オ							感動詞-一般 
\\	「捕まえる(つかまえる)」のアクセントは?2356	捕まえる 仮名 
\\	つかまえる 漢字 
\\	捕まえる 発音図:	捕まえる つかまえる		捕まえる つかまえる 捕まえます つかまえます 捕まえて つかまえて 捕まえた つかまえた 捕まえない つかまえない 捕まえなかった つかまえなかった 捕まえれば つかまえれば 捕まえさせる つかまえさせる 捕まえられる つかまえられる 捕まえろ つかまえろ 捕まえられる つかまえられる 捕まえよう つかまえよう				動詞-一般 
\\	「似合う(にあう)」のアクセントは?2357	似合う 仮名 
\\	にあう 漢字 
\\	似合う 発音図:	似合う にあう		似合う にあう 似合います にあいます 似合って にあって 似合った にあった 似合わない にあわない 似合わなかった にあわなかった 似合えば にあえば 似合わせる にあわせる 似合われる にあわれる 似合え にあえ 似合おう にあおう				動詞-一般 
\\	「分類(ぶんるい)」のアクセントは?2358	分類 仮名 
\\	ぶんるい 漢字 
\\	分類 発音図: 助詞: オ	分類 ぶんるい		分類 ぶんるい 分類する ぶんるいする 分類します ぶんるいします 分類して ぶんるいして 分類した ぶんるいした 分類しない ぶんるいしない 分類しなかった ぶんるいしなかった 分類すれば ぶんるいすれば 分類させる ぶんるいさせる 分類される ぶんるいされる 分類しろ ぶんるいしろ 分類できる ぶんるいできる 分類しよう ぶんるいしよう				動詞-一般 
\\	「プロセス(ぷろせす)」のアクセントは?2359	プロセス 仮名 
\\	プロセス、ぷろせす 漢字 
\\	発音図: 助詞: オ	プロセス ぷろせす		プロセス ぷろせす				名詞-普通名詞-一般 
\\	「付け加える(つけくわえる)」のアクセントは?2360	付け加える 仮名 
\\	つけくわえる 漢字 
\\	付け加える 発音図: 発音図: 発音図:	付け加える つけくわえる つけくわえる つけくわえる		付け加える つけくわえる つけくわえる つけくわえる 付け加えます つけくわえます 付け加えて つけくわえて つけくわえて 付け加えた つけくわえた つけくわえた 付け加えない つけくわえない つけくわえない 付け加えなかった つけくわえなかった つけくわえなかった 付け加えれば つけくわえれば つけくわえれば つけくわえれば 付け加えさせる つけくわえさせる つけくわえさせる 付け加えられる つけくわえられる つけくわえられる 付け加えろ つけくわえろ つけくわえろ つけくわえろ 付け加えられる つけくわえられる つけくわえられる 付け加えよう つけくわえよう				動詞-一般 
\\	「蘇る(よみがえる)」のアクセントは?2361	よみがえる{蘇} 仮名 
\\	よみがえる 漢字 
\\	蘇る 発音図: 発音図:	蘇る よみがえる		蘇る よみがえる 蘇ります よみがえります 蘇って よみがえって 蘇った よみがえった 蘇らない よみがえらない 蘇らなかった よみがえらなかった 蘇れば よみがえれば 蘇らせる よみがえらせる 蘇られる よみがえられる 蘇れ よみがえれ 蘇れる よみがえれる 蘇ろう よみがえろう				動詞-一般 
\\	「同僚(どうりょう)」のアクセントは?2362	同僚 仮名 
\\	どうりょう、どーりょー 漢字 
\\	同僚 発音図: 助詞: オ	同僚 どうりょう		同僚 どうりょう				名詞-普通名詞-一般 
\\	「欲望(よくぼう)」のアクセントは?2363	欲望 仮名 
\\	よくぼう、よくぼー 漢字 
\\	欲望 発音図: 助詞: オ	欲望 よくぼう		欲望 よくぼう				名詞-普通名詞-一般 
\\	「事柄(ことがら)」のアクセントは?2364	事柄 仮名 
\\	ことがら 漢字 
\\	事柄 発音図: 例文: 調べたオ記録する 発音図: 例文: 調べたオ記録する	事柄 ことがら		事柄 ことがら				名詞-普通名詞-一般 
\\	「従事(じゅうじする)」のアクセントは?2365	従事 仮名 
\\	じゅーじ、じゅうじ 漢字 
\\	従事 発音図: 助詞: オ	従事する じゅうじする		従事する じゅうじする 従事します じゅうじします 従事して じゅうじして 従事した じゅうじした 従事しない じゅうじしない 従事しなかった じゅうじしなかった 従事すれば じゅうじすれば 従事させる じゅうじさせる 従事される じゅうじされる 従事しろ じゅうじしろ 従事できる じゅうじできる 従事しよう じゅうじしよう				動詞-一般 
\\	「長男(ちょうなん)」のアクセントは?2366	長男 仮名 
\\	ちょうなん、ちょーなん 漢字 
\\	長男 発音図: 助詞: オ 発音図: 助詞: オ	長男 ちょうなん		長男 ちょうなん				名詞-普通名詞-一般 
\\	「雄(おす)」のアクセントは?2367	雄 仮名 
\\	おす、ゆー 漢字 
\\	雄 発音図: 助詞: オ 例文: ヒヨコのワ雌と区別される 発音図:	雄 おす		雄 おす				名詞-普通名詞-一般 
\\	「配る(くばる)」のアクセントは?2368	配る 仮名 
\\	くばる 漢字 
\\	配る 発音図:	配る くばる		配る くばる 配ります くばります 配って くばって 配った くばった 配らない くばらない 配らなかった くばらなかった 配れば くばれば 配らせる くばらせる 配られる くばられる 配れ くばれ 配れる くばれる 配ろう くばろう				動詞-一般 
\\	「巣(す)」のアクセントは?2369	巣 仮名 
\\	す 漢字 
\\	巣 発音図: 例文: 小枝を集めてオ作る鳥も多い 発音図: 例文: 小枝を集めてオ作る鳥も多い	巣 す す		巣 す す				名詞-普通名詞-一般 
\\	「潰す(つぶす)」のアクセントは?2370	つぶす{潰} 仮名 
\\	つぶす 漢字 
\\	潰す 発音図:	潰す つぶす		潰す つぶす 潰します つぶします 潰して つぶして 潰した つぶした 潰さない つぶさない 潰さなかった つぶさなかった 潰せば つぶせば 潰させる つぶさせる 潰される つぶされる 潰せ つぶせ 潰せる つぶせる 潰そう つぶそう				動詞-一般 
\\	「池(いけ)」のアクセントは?2371	池 仮名 
\\	いけ 漢字 
\\	池 発音図: 助詞: オ	池 いけ		池 いけ				名詞-普通名詞-一般 
\\	「痛む(いたむ)」のアクセントは?2372	痛む 仮名 
\\	いたむ 漢字 
\\	痛む 発音図:	痛む いたむ		痛む いたむ 痛みます いたみます 痛んで いたんで 痛んだ いたんだ 痛まない いたまない 痛まなかった いたまなかった 痛めば いためば 痛ませる いたませる 痛まれる いたまれる 痛め いため 痛もう いたもう				動詞-一般 
\\	「覚める(さめる)」のアクセントは?2373	覚める 仮名 
\\	さめる 漢字 
\\	覚める 発音図:	覚める さめる		覚める さめる 覚めます さめます 覚めて さめて 覚めた さめた 覚めない さめない 覚めなかった さめなかった 覚めれば さめれば 覚めさせる さめさせる 覚められる さめられる 覚めろ さめろ 覚めよう さめよう				動詞-一般 
\\	「誇る(ほこる)」のアクセントは?2374	誇る 仮名 
\\	ほこる 漢字 
\\	誇る 発音図:	誇る ほこる		誇る ほこる 誇ります ほこります 誇って ほこって 誇った ほこった 誇らない ほこらない 誇らなかった ほこらなかった 誇れば ほこれば 誇らせる ほこらせる 誇られる ほこられる 誇れ ほこれ 誇れる ほこれる 誇ろう ほころう				動詞-一般 
\\	「技(わざ)」のアクセントは?2375	技 仮名 
\\	わざ 漢字 
\\	技 発音図: 助詞: オ	技 わざ		技 わざ				名詞-普通名詞-一般 
\\	「滅多な(めったな)《〜に》」のアクセントは?2376	めった{滅多}(〜に) 仮名 
\\	めった 漢字 
\\	滅多 発音図: 助詞: ニ	滅多な めったな		滅多な めったな 滅多に めったに				形状詞-一般 
\\	「袋(ふくろ)」のアクセントは?2377	袋 仮名 
\\	ふくろ 漢字 
\\	袋 発音図: 助詞: オ	袋 ふくろ		袋 ふくろ				名詞-普通名詞-一般 
\\	「調整(ちょうせいする)」のアクセントは?2378	調整 仮名 
\\	ちょうせい、ちょーせい 漢字 
\\	調整 発音図: 助詞: オ	調整する ちょうせいする		調整する ちょうせいする 調整します ちょうせいします 調整して ちょうせいして 調整した ちょうせいした 調整しない ちょうせいしない 調整しなかった ちょうせいしなかった 調整すれば ちょうせいすれば 調整させる ちょうせいさせる 調整される ちょうせいされる 調整しろ ちょうせいしろ 調整できる ちょうせいできる 調整しよう ちょうせいしよう				動詞-一般 
\\	「稼ぐ(かせぐ)」のアクセントは?2379	稼ぐ 仮名 
\\	かせぐ 漢字 
\\	稼ぐ 発音図:	稼ぐ かせぐ		稼ぐ かせぐ 稼ぎます かせぎます 稼いで かせいで 稼いだ かせいだ 稼がない かせがない 稼がなかった かせがなかった 稼げば かせげば 稼がせる かせがせる 稼がれる かせがれる 稼げ かせげ 稼げる かせげる 稼ごう かせごう				動詞-一般 
\\	「取り込む(とりこむ)」のアクセントは?2380	取り込む 仮名 
\\	とりこむ 漢字 
\\	取り込む 発音図: 発音図:							動詞-一般 
\\	「面積(めんせき)」のアクセントは?2381	面積 仮名 
\\	めんせき 漢字 
\\	面積 発音図: 助詞: オ	面積 めんせき		面積 めんせき				名詞-普通名詞-一般 
\\	「人数(にんずう)」のアクセントは?2382	人数 仮名 
\\	にんず、にんずー、にんずう、ひとかず 漢字 
\\	人数 発音図: 助詞: オ 発音図: 助詞: オ 発音図: 助詞: オ	人数 にんずう		人数 にんずう				名詞-普通名詞-一般 
\\	「芸術(げいじゅつ)」のアクセントは?2383	芸術 仮名 
\\	げいじゅつ 漢字 
\\	芸術 発音図: 助詞: オ 発音図: 助詞: オ	芸術 げいじゅつ		芸術 げいじゅつ				名詞-普通名詞-一般 
\\	「倒す(たおす)」のアクセントは?2384	倒す 仮名 
\\	たおす 漢字 
\\	倒す 発音図:	倒す たおす		倒す たおす 倒します たおします 倒して たおして 倒した たおした 倒さない たおさない 倒さなかった たおさなかった 倒せば たおせば 倒させる たおさせる 倒される たおされる 倒せ たおせ 倒せる たおせる 倒そう たおそう				動詞-一般 
\\	「一時的な(いちじてきな)」のアクセントは?2385	一時的 仮名 
\\	いちじてき 漢字 
\\	一時的 発音図:	一時的な いちじてきな		一時的な いちじてきな 一時的です いちじてきです 一時的で いちじてきで 一時的だった いちじてきだった 一時的じゃない いちじてきじゃない 一時的じゃなかった いちじてきじゃなかった 一時的ならば いちじてきならば 一時的だ いちじてきだ 一時的に いちじてきに				形状詞-一般 
\\	「夢中(むちゅう)」のアクセントは?2386	夢中 仮名 
\\	むちゅー、むちゅう 漢字 
\\	夢中 発音図: 助詞: オ	夢中 むちゅう		夢中 むちゅう				形状詞-一般 
\\	「御覧(ごらん)」のアクセントは?2387	ごらん{御覧} 仮名 
\\	ごらん 漢字 
\\	御覧 発音図:	御覧 ごらん		御覧 ごらん				名詞-普通名詞-一般 
\\	「横浜(よこはま)」のアクセントは?2388	横浜 仮名 
\\	よこはま 漢字 
\\	横浜 発音図: 助詞: オ	横浜 よこはま		横浜 よこはま				名詞-固有名詞-地名-一般 
\\	「振り向く(ふりむく)」のアクセントは?2389	振り向く 仮名 
\\	ふりむく 漢字 
\\	振り向く 発音図:	振り向く ふりむく		振り向く ふりむく 振り向きます ふりむきます 振り向いて ふりむいて 振り向いた ふりむいた 振り向かない ふりむかない 振り向かなかった ふりむかなかった 振り向けば ふりむけば 振り向かせる ふりむかせる 振り向かれる ふりむかれる 振り向け ふりむけ 振り向ける ふりむける 振り向こう ふりむこう				動詞-一般 
\\	「取り扱う(とりあつかう)」のアクセントは?2390	取り扱う 仮名 
\\	とりあつかう 漢字 
\\	取り扱う 発音図: 発音図:	取り扱う とりあつかう とりあつかう		取り扱う とりあつかう とりあつかう 取り扱います とりあつかいます 取り扱って とりあつかって とりあつかって 取り扱った とりあつかった とりあつかった 取り扱わない とりあつかわない とりあつかわない 取り扱わなかった とりあつかわなかった とりあつかわなかった 取り扱えば とりあつかえば とりあつかえば 取り扱わせる とりあつかわせる とりあつかわせる 取り扱われる とりあつかわれる とりあつかわれる 取り扱え とりあつかえ とりあつかえ 取り扱える とりあつかえる とりあつかえる 取り扱おう とりあつかおう				動詞-一般 
\\	「次いで(ついで)」のアクセントは?2391	次いで 仮名 
\\	ついで 漢字 
\\	次いで 発音図: 発音図:							副詞 
\\	「固定(こていする)」のアクセントは?2392	固定 仮名 
\\	こてい 漢字 
\\	固定 発音図: 助詞: オ	固定する こていする		固定する こていする 固定します こていします 固定して こていして 固定した こていした 固定しない こていしない 固定しなかった こていしなかった 固定すれば こていすれば 固定させる こていさせる 固定される こていされる 固定しろ こていしろ 固定できる こていできる 固定しよう こていしよう				動詞-一般 
\\	「軍(ぐん)」のアクセントは?2393	軍 仮名 
\\	ぐん 漢字 
\\	軍 発音図: 助詞: オ	軍 ぐん		軍 ぐん				名詞-普通名詞-一般 
\\	「恋愛(れんあい)」のアクセントは?2394	恋愛 仮名 
\\	れんあい 漢字 
\\	恋愛 発音図: 助詞: オ	恋愛 れんあい		恋愛 れんあい				名詞-普通名詞-一般 
\\	「キス(きす)」のアクセントは?2395	キス 仮名 
\\	キス、きす 漢字 
\\	発音図:	キス きす		キス きす				名詞-普通名詞-一般 
\\	「承知(しょうちする)」のアクセントは?2396	承知 仮名 
\\	しょうち、しょーち 漢字 
\\	承知 発音図: 助詞: オ	承知する しょうちする		承知する しょうちする 承知します しょうちします 承知して しょうちして 承知した しょうちした 承知しない しょうちしない 承知しなかった しょうちしなかった 承知すれば しょうちすれば 承知させる しょうちさせる 承知される しょうちされる 承知しろ しょうちしろ 承知できる しょうちできる 承知しよう しょうちしよう				動詞-一般 
\\	「体調(たいちょう)」のアクセントは?2397	体調 仮名 
\\	たいちょう、たいちょー 漢字 
\\	体調 発音図: 助詞: オ	体調 たいちょう		体調 たいちょう				名詞-普通名詞-一般 
\\	「長官(ちょうかん)」のアクセントは?2398	長官 仮名 
\\	ちょうかん、ちょーかん 漢字 
\\	長官 発音図: 助詞: オ	長官 ちょうかん		長官 ちょうかん				名詞-普通名詞-一般 
\\	「生地(きじ)」のアクセントは?2399	生地 仮名 
\\	きじ、せいち 漢字 
\\	生地 発音図: 助詞: オ 発音図: 助詞: オ	生地 きじ		生地 きじ				名詞-普通名詞-一般 
\\	「区別(くべつ)」のアクセントは?2400	区別 仮名 
\\	くべつ 漢字 
\\	区別 発音図: 助詞: オ	区別 くべつ		区別 くべつ 区別する くべつする 区別します くべつします 区別して くべつして 区別した くべつした 区別しない くべつしない 区別しなかった くべつしなかった 区別すれば くべつすれば 区別させる くべつさせる 区別される くべつされる 区別しろ くべつしろ 区別できる くべつできる 区別しよう くべつしよう				動詞-一般 
\\	「動機(どうき)」のアクセントは?2401	動機 仮名 
\\	どうき、どーき 漢字 
\\	動機 発音図: 助詞: オ	動機 どうき		動機 どうき				名詞-普通名詞-一般 
\\	「安全性(あんぜんせい)」のアクセントは?2402	安全性 仮名 
\\	あんぜんせい 漢字 
\\	安全性 発音図: 助詞: オ							名詞-普通名詞-一般 
\\	「九州(きゅうしゅう)」のアクセントは?2403		九州 きゅうしゅう		九州 きゅうしゅう				名詞-固有名詞-地名-一般 
\\	「気温(きおん)」のアクセントは?2404	気温 仮名 
\\	きおん 漢字 
\\	気温 発音図: 助詞: オ	気温 きおん		気温 きおん				名詞-普通名詞-一般 
\\	「半分(はんぶん)」のアクセントは?2405	半分 仮名 
\\	はんぶん 漢字 
\\	半分 発音図: 助詞: オ	半分 はんぶん		半分 はんぶん				副詞 
\\	「反応(はんのう)」のアクセントは?2406	反応 仮名 
\\	はんのう、はんのー 漢字 
\\	反応 発音図: 助詞: オ	反応 はんのう		反応 はんのう 反応する はんのうする 反応します はんのうします 反応して はんのうして 反応した はんのうした 反応しない はんのうしない 反応しなかった はんのうしなかった 反応すれば はんのうすれば 反応させる はんのうさせる 反応される はんのうされる 反応しろ はんのうしろ 反応できる はんのうできる 反応しよう はんのうしよう				動詞-一般 
\\	「半分(はんぶん)《〜残した》」のアクセントは?2407	半分(〜残した) 仮名 
\\	はんぶん 漢字 
\\	半分 発音図: 例文: ノこした	半分 はんぶん		半分 はんぶん				副詞 
\\	「リーダー(りーだー)
\\	のアクセントは?2408	リーダー
\\	仮名 
\\	リーダー、りーだー 漢字 
\\	発音図: 助詞: オ	リーダー りーだー		リーダー りーだー				名詞-普通名詞-一般 
\\	「自治体(じちたい)」のアクセントは?2409	自治体 仮名 
\\	じちたい 漢字 
\\	自治体 発音図: 助詞: オ							名詞-普通名詞-一般 
\\	「リーダー(リーダー)
\\	のアクセントは?2410	リーダー
\\	仮名 
\\	リーダー、りーだー 漢字 
\\	発音図: 助詞: オ							名詞-普通名詞-一般 
\\	「室内(しつない)」のアクセントは?2411	室内 仮名 
\\	しつない 漢字 
\\	室内 発音図: 助詞: オ							名詞-普通名詞-一般 
\\	「謝る(あやまる)」のアクセントは?2412	謝る 仮名 
\\	あやまる 漢字 
\\	謝る 発音図:	謝る あやまる		謝る あやまる 謝ります あやまります 謝って あやまって 謝った あやまった 謝らない あやまらない 謝らなかった あやまらなかった 謝れば あやまれば 謝らせる あやまらせる 謝られる あやまられる 謝れ あやまれ 謝れる あやまれる 謝ろう あやまろう				動詞-一般 
\\	「設備(せつび)」のアクセントは?2413	設備 仮名 
\\	せつび 漢字 
\\	設備 発音図: 助詞: オ	設備 せつび		設備 せつび				名詞-普通名詞-一般 
\\	「名付ける(なづける)」のアクセントは?2414		名付ける なづける		名付ける なづける 名付けます なづけます 名付けて なづけて 名付けた なづけた 名付けない なづけない 名付けなかった なづけなかった 名付ければ なづければ 名付けさせる なづけさせる 名付けられる なづけられる 名付けろ なづけろ 名付けられる なづけられる 名付けよう なづけよう				動詞-一般 
\\	「請求(せいきゅう)」のアクセントは?2415	請求 仮名 
\\	せいきゅー、せいきゅう 漢字 
\\	請求 発音図: 助詞: オ	請求 せいきゅう		請求 せいきゅう 請求する せいきゅうする 請求します せいきゅうします 請求して せいきゅうして 請求した せいきゅうした 請求しない せいきゅうしない 請求しなかった せいきゅうしなかった 請求すれば せいきゅうすれば 請求させる せいきゅうさせる 請求される せいきゅうされる 請求しろ せいきゅうしろ 請求できる せいきゅうできる 請求しよう せいきゅうしよう				動詞-一般 
\\	「民族(みんぞく)」のアクセントは?2416	民族 仮名 
\\	みんぞく 漢字 
\\	民族 発音図: 助詞: オ	民族 みんぞく		民族 みんぞく				名詞-普通名詞-一般 
\\	「外側(そとがわ)」のアクセントは?2417	外側 仮名 
\\	そとがわ 漢字 
\\	外側 発音図: 助詞: オ	外側 そとがわ		外側 そとがわ				名詞-普通名詞-一般 
\\	「トラック(トラック)
\\	のアクセントは?2418	トラック
\\	仮名 
\\	トラック、とらっく 漢字 
\\	発音図: 助詞: オ							名詞-普通名詞-一般 
\\	「トラック(とらっく)
\\	のアクセントは?2419	トラック
\\	仮名 
\\	トラック、とらっく 漢字 
\\	発音図: 助詞: オ	トラック とらっく		トラック とらっく				名詞-普通名詞-一般 
\\	「大人しい(おとなしい)」のアクセントは?2420		大人しい おとなしい		大人しい おとなしい 大人しいです おとなしいです 大人しくて おとなしくて おとなしくて 大人しかった おとなしかった おとなしかった 大人しくない おとなしくない おとなしくない 大人しくなかった おとなしくなかった おとなしくなかった 大人しければ おとなしければ おとなしければ 大人しい おとなしい 大人しく おとなしく おとなしく				形容詞-一般 
\\	「指示(しじ)」のアクセントは?2421	指示 仮名 
\\	しじ 漢字 
\\	指示 発音図: 助詞: オ	指示 しじ		指示 しじ 指示する しじする 指示します しじします 指示して しじして 指示した しじした 指示しない しじしない 指示しなかった しじしなかった 指示すれば しじすれば 指示させる しじさせる 指示される しじされる 指示しろ しじしろ 指示できる しじできる 指示しよう しじしよう				動詞-一般 
\\	「現に(げんに)」のアクセントは?2422	現に 仮名 
\\	げんに 漢字 
\\	現に 発音図:							副詞 
\\	「背負う(せおう)」のアクセントは?2423	背負う 仮名 
\\	せおう 漢字 
\\	背負う 発音図:	背負う せおう		背負う せおう 背負います せおいます 背負って せおって 背負った せおった 背負わない せおわない 背負わなかった せおわなかった 背負えば せおえば 背負わせる せおわせる 背負われる せおわれる 背負え せおえ 背負える せおえる 背負おう せおおう				動詞-一般 
\\	「売り上げ(うりあげ)」のアクセントは?2424	売り上げ 仮名 
\\	うりあげ 漢字 
\\	売り上げ 発音図: 助詞: オ	売り上げ うりあげ		売り上げ うりあげ				名詞-普通名詞-一般 
\\	「眼鏡(めがね)」のアクセントは?2425	眼鏡 仮名 
\\	めがね 漢字 
\\	眼鏡 発音図: 助詞: オ	眼鏡 めがね		眼鏡 めがね				名詞-普通名詞-一般 
\\	「人参(にんじん)」のアクセントは?2426	にんじん{人参} 仮名 
\\	にんじん 漢字 
\\	人参 発音図: 助詞: オ							名詞-普通名詞-一般 
\\	「シャツ(しゃつ)」のアクセントは?2427	シャツ 仮名 
\\	シャツ、しゃつ 漢字 
\\	発音図: 助詞: オ	シャツ しゃつ		シャツ しゃつ				名詞-普通名詞-一般 
\\	「汚れ(よごれ)」のアクセントは?2428	汚れ 仮名 
\\	けがれ、よごれ 漢字 
\\	汚れ 発音図: 助詞: オ 例文: みそぎでオ落とす 発音図: 例文: みそぎでオ落とす 発音図:	汚れ よごれ		汚れ よごれ				名詞-普通名詞-一般 
\\	「リズム(りずむ)」のアクセントは?2429	リズム 仮名 
\\	リズム、りずむ 漢字 
\\	発音図: 助詞: オ	リズム りずむ		リズム りずむ				名詞-普通名詞-一般 
\\	「当事者(とうじしゃ)」のアクセントは?2430	当事者 仮名 
\\	とうじしゃ、とーじしゃ 漢字 
\\	当事者 発音図: 助詞: オ							名詞-普通名詞-一般 
\\	「面倒(めんどう)」のアクセントは?2431	面倒 仮名 
\\	めんどう、めんどー 漢字 
\\	面倒 発音図: 助詞: オ	面倒 めんどう		面倒 めんどう				名詞-普通名詞-一般 
\\	「面倒な(めんどうな)」のアクセントは?2432		面倒な めんどうな		面倒な めんどうな 面倒です めんどうです 面倒で めんどうで 面倒だった めんどうだった 面倒じゃない めんどうじゃない 面倒じゃなかった めんどうじゃなかった 面倒ならば めんどうならば 面倒だ めんどうだ 面倒に めんどうに				名詞-普通名詞-一般 
\\	「刀(かたな)」のアクセントは?2433	刀 仮名 
\\	かたな、とう、とー 漢字 
\\	刀 発音図: 助詞: オ 発音図: 助詞: オ 発音図: 助詞: オ	刀 かたな		刀 かたな				名詞-普通名詞-一般 
\\	「学問(がくもん)」のアクセントは?2434	学問 仮名 
\\	がくもん 漢字 
\\	学問 発音図: 助詞: オ	学問 がくもん		学問 がくもん				名詞-普通名詞-一般 
\\	「悲鳴(ひめい)」のアクセントは?2435	悲鳴 仮名 
\\	ひめい 漢字 
\\	悲鳴 発音図: 助詞: オ	悲鳴 ひめい		悲鳴 ひめい				名詞-普通名詞-一般 
\\	「筋(すじ)」のアクセントは?2436	筋 仮名 
\\	きん、すじ 漢字 
\\	筋 発音図: 助詞: オ 発音図: 助詞: オ	筋 すじ		筋 すじ				名詞-普通名詞-一般 
\\	「負担(ふたん)」のアクセントは?2437	負担 仮名 
\\	ふたん 漢字 
\\	負担 発音図: 助詞: オ	負担 ふたん		負担 ふたん 負担する ふたんする 負担します ふたんします 負担して ふたんして 負担した ふたんした 負担しない ふたんしない 負担しなかった ふたんしなかった 負担すれば ふたんすれば 負担させる ふたんさせる 負担される ふたんされる 負担しろ ふたんしろ 負担できる ふたんできる 負担しよう ふたんしよう				動詞-一般 
\\	「基盤(きばん)」のアクセントは?2438	基盤 仮名 
\\	きばん 漢字 
\\	基盤 発音図: 助詞: オ	基盤 きばん		基盤 きばん				名詞-普通名詞-一般 
\\	「眉(まゆ)」のアクセントは?2439	まゆ{眉} 仮名 
\\	まゆ 漢字 
\\	眉 発音図: 助詞: オ	眉 まゆ		眉 まゆ				名詞-普通名詞-一般 
\\	「鳴らす(ならす)」のアクセントは?2440	鳴らす 仮名 
\\	ならす 漢字 
\\	鳴らす 発音図:	鳴らす ならす		鳴らす ならす 鳴らします ならします 鳴らして ならして 鳴らした ならした 鳴らさない ならさない 鳴らさなかった ならさなかった 鳴らせば ならせば 鳴らさせる ならさせる 鳴らされる ならされる 鳴らせ ならせ 鳴らせる ならせる 鳴らそう ならそう				動詞-一般 
\\	「手元(てもと)」のアクセントは?2441	手もと{元;下;許} 仮名 
\\	てもと 漢字 
\\	手元、手下、手許、手もと 発音図: 助詞: オ	手元 てもと		手元 てもと				名詞-普通名詞-一般 
\\	「仏教(ぶっきょう)」のアクセントは?2442	仏教 仮名 
\\	ぶっきょう、ぶっきょー 漢字 
\\	仏教 発音図: 助詞: オ 発音図: 助詞: オ	仏教 ぶっきょう		仏教 ぶっきょう				名詞-普通名詞-一般 
\\	「形成(けいせい)」のアクセントは?2443	形成 仮名 
\\	けいせい 漢字 
\\	形成 発音図: 助詞: オ	形成 けいせい		形成 けいせい				名詞-普通名詞-一般 
\\	「入り込む(はいりこむ)」のアクセントは?2444	入り込む 仮名 
\\	いりこむ、はいりこむ 漢字 
\\	入り込む 発音図: 発音図: 発音図:	入り込む はいりこむ はいりこむ		入り込む はいりこむ はいりこむ 入り込みます はいりこみます 入り込んで はいりこんで はいりこんで 入り込んだ はいりこんだ はいりこんだ 入り込まない はいりこまない はいりこまない 入り込まなかった はいりこまなかった はいりこまなかった 入り込めば はいりこめば はいりこめば 入り込ませる はいりこませる はいりこませる 入り込まれる はいりこまれる はいりこまれる 入り込め はいりこめ はいりこめ 入り込める はいりこめる はいりこめる 入り込もう はいりこもう				動詞-一般 
\\	「大統領(だいとうりょう)」のアクセントは?2445	大統領 仮名 
\\	だいとうりょう、だいとーりょー 漢字 
\\	大統領 発音図: 助詞: オ	大統領 だいとうりょう		大統領 だいとうりょう				名詞-普通名詞-一般 
\\	「辿り着く(たどりつく)」のアクセントは?2446		辿り着く たどりつく		辿り着く たどりつく 辿り着きます たどりつきます 辿り着いて たどりついて 辿り着いた たどりついた 辿り着かない たどりつかない 辿り着かなかった たどりつかなかった 辿り着けば たどりつけば 辿り着かせる たどりつかせる 辿り着かれる たどりつかれる 辿り着け たどりつけ 辿り着ける たどりつける 辿り着こう たどりつこう				動詞-一般 
\\	「整う(ととのう)」のアクセントは?2447	整う 仮名 
\\	ととのう 漢字 
\\	整う 発音図:	整う ととのう		整う ととのう 整います ととのいます 整って ととのって 整った ととのった 整わない ととのわない 整わなかった ととのわなかった 整えば ととのえば 整わせる ととのわせる 整われる ととのわれる 整え ととのえ 整おう ととのおう				動詞-一般 
\\	「吊る(つる)」のアクセントは?2448	つる{吊;攣} 仮名 
\\	つる 漢字 
\\	吊る、攣る 発音図: 例文: 天井にひもをバ所があります	吊る つる		吊る つる 吊ります つります 吊って つって 吊った つった 吊らない つらない 吊らなかった つらなかった 吊れば つれば 吊らせる つらせる 吊られる つられる 吊れ つれ 吊れる つれる 吊ろう つろう				動詞-一般 
\\	「景色(けしき)」のアクセントは?2449	景色 仮名 
\\	けいしょく、けしき 漢字 
\\	景色 発音図: 助詞: オ 発音図: 助詞: オ 発音図: 助詞: オ	景色 けしき		景色 けしき				名詞-普通名詞-一般 
\\	「其奴(そいつ)」のアクセントは?2450	そいつ{其奴} 仮名 
\\	そいつ 漢字 
\\	其奴 発音図: 助詞: オ							代名詞 
\\	「純粋な(じゅんすいな)」のアクセントは?2451		純粋な じゅんすいな		純粋な じゅんすいな 純粋です じゅんすいです 純粋で じゅんすいで 純粋だった じゅんすいだった 純粋じゃない じゅんすいじゃない 純粋じゃなかった じゅんすいじゃなかった 純粋ならば じゅんすいならば 純粋だ じゅんすいだ 純粋に じゅんすいに				形状詞-一般 
\\	「純粋(じゅんすい)」のアクセントは?2452	純粋 仮名 
\\	じゅんすい 漢字 
\\	純粋 発音図: 助詞: オ	純粋 じゅんすい		純粋 じゅんすい				形状詞-一般 
\\	「バター(ばたー)」のアクセントは?2453	バター 仮名 
\\	バター、ばたー 漢字 
\\	発音図: 助詞: オ	バター ばたー		バター ばたー				名詞-普通名詞-一般 
\\	「毎年(まいとし)」のアクセントは?2454		毎年 まいとし		毎年 まいとし				名詞-普通名詞-一般 
\\	「葉書(はがき)」のアクセントは?2455		葉書 はがき		葉書 はがき				名詞-普通名詞-一般 
\\	「首相(しゅしょう)」のアクセントは?2456	首相 仮名 
\\	しゅしょう、しゅしょー 漢字 
\\	首相 発音図: 助詞: オ	首相 しゅしょう		首相 しゅしょう				名詞-普通名詞-一般 
\\	「女房(にょうぼう)」のアクセントは?2457	女房 仮名 
\\	にょうぼう、にょーぼー 漢字 
\\	女房 発音図: 助詞: オ	女房 にょうぼう		女房 にょうぼう				名詞-普通名詞-一般 
\\	「実績(じっせき)」のアクセントは?2458	実績 仮名 
\\	じっせき 漢字 
\\	実績 発音図: 助詞: オ	実績 じっせき		実績 じっせき				名詞-普通名詞-一般 
\\	「溶ける(とける)」のアクセントは?2459	溶ける 仮名 
\\	とける 漢字 
\\	溶ける 発音図:	溶ける とける		溶ける とける 溶けます とけます 溶けて とけて 溶けた とけた 溶けない とけない 溶けなかった とけなかった 溶ければ とければ 溶けさせる とけさせる 溶けられる とけられる 溶けろ とけろ 溶けよう とけよう				動詞-一般 
\\	「募集(ぼしゅう)」のアクセントは?2460	募集 仮名 
\\	ぼしゅー、ぼしゅう 漢字 
\\	募集 発音図: 助詞: オ	募集 ぼしゅう		募集 ぼしゅう 募集する ぼしゅうする 募集します ぼしゅうします 募集して ぼしゅうして 募集した ぼしゅうした 募集しない ぼしゅうしない 募集しなかった ぼしゅうしなかった 募集すれば ぼしゅうすれば 募集させる ぼしゅうさせる 募集される ぼしゅうされる 募集しろ ぼしゅうしろ 募集できる ぼしゅうできる 募集しよう ぼしゅうしよう				動詞-一般 
\\	「台風(たいふう)」のアクセントは?2461	台風 仮名 
\\	たいふー、たいふう 漢字 
\\	台風 発音図: 助詞: オ	台風 たいふう		台風 たいふう				名詞-普通名詞-一般 
\\	「昼食(ちゅうしょく)」のアクセントは?2462	昼食 仮名 
\\	ちゅーじき、ちゅうじき、ちゅーしょく、ちゅうしょく 漢字 
\\	昼食 発音図: 助詞: オ 発音図: 助詞: オ	昼食 ちゅうしょく		昼食 ちゅうしょく				名詞-普通名詞-一般 
\\	「紐(ひも)」のアクセントは?2463	ひも{紐} 仮名 
\\	ひも 漢字 
\\	紐 発音図: 助詞: オ	紐 ひも		紐 ひも				名詞-普通名詞-一般 
\\	「アイデア(あいであ)」のアクセントは?2464	アイデア 仮名 
\\	アイデア、あいであ 漢字 
\\	発音図: 助詞: オ 発音図: 助詞: オ	アイデア あいであ あいであ		アイデア あいであ あいであ				名詞-普通名詞-一般 
\\	「確率(かくりつ)」のアクセントは?2465	確率 仮名 
\\	かくりつ 漢字 
\\	確率 発音図: 助詞: オ	確率 かくりつ		確率 かくりつ				名詞-普通名詞-一般 
\\	「捕まる(つかまる)」のアクセントは?2466	捕まる 仮名 
\\	つかまる 漢字 
\\	捕まる 発音図:	捕まる つかまる		捕まる つかまる 捕まります つかまります 捕まって つかまって 捕まった つかまった 捕まらない つかまらない 捕まらなかった つかまらなかった 捕まれば つかまれば 捕まらせる つかまらせる 捕まられる つかまられる 捕まれ つかまれ 捕まれる つかまれる 捕まろう つかまろう				動詞-一般 
\\	「文部省(もんぶしょう)」のアクセントは?2467	文部省 仮名 
\\	もんぶしょう、もんぶしょー 漢字 
\\	文部省 発音図: 助詞: オ							名詞-固有名詞-一般 
\\	「解消(かいしょう)」のアクセントは?2468	解消 仮名 
\\	かいしょう、かいしょー 漢字 
\\	解消 発音図: 助詞: オ	解消 かいしょう		解消 かいしょう 解消する かいしょうする 解消します かいしょうします 解消して かいしょうして 解消した かいしょうした 解消ない かいしょしない 解消しなかった かいしょうしなかった 解消すれば かいしょうすれば 解消させる かいしょうさせる 解消される かいしょうされる 解消しろ かいしょうしろ 解消できる かいしょうできる 解消しよう かいしょうしよう				動詞-一般 
\\	「車両(しゃりょう)」のアクセントは?2469	車両 仮名 
\\	しゃりょう、しゃりょー 漢字 
\\	車両 発音図: 助詞: オ							名詞-普通名詞-一般 
\\	「合意(ごうい)《〜に基づく》」のアクセントは?2470	合意(〜に基づく) 仮名 
\\	ごうい、ごーい 漢字 
\\	合意 発音図: 例文: ニ基づく 発音図: 例文: ニ基づく	合意 ごうい		合意 ごうい				名詞-普通名詞-一般 
\\	「解放(かいほう)」のアクセントは?2471	解放 仮名 
\\	かいほう、かいほー 漢字 
\\	解放 発音図: 助詞: オ	解放 かいほう		解放 かいほう 解放する かいほうする 解放します かいほうします 解放して かいほうして 解放した かいほうした 解放しない かいほうしない 解放しなかった かいほうしなかった 解放すれば かいほうすれば 解放させる かいほうさせる 解放される かいほうされる 解放しろ かいほうしろ 解放できる かいほうできる 解放しよう かいほうしよう				動詞-一般 
\\	「実情(じつじょう)」のアクセントは?2472	実情 仮名 
\\	じつじょう、じつじょー 漢字 
\\	実情 発音図: 助詞: オ	実情 じつじょう		実情 じつじょう				名詞-普通名詞-一般 
\\	「放置(ほうちする)」のアクセントは?2473	放置 仮名 
\\	ほうち、ほーち 漢字 
\\	放置 発音図: 助詞: オ 発音図: 助詞: オ	放置する ほうちする		放置する ほうちする 放置します ほうちします 放置して ほうちして 放置した ほうちした 放置しない ほうちしない 放置しなかった ほうちしなかった 放置すれば ほうちすれば 放置させる ほうちさせる 放置される ほうちされる 放置しろ ほうちしろ 放置できる ほうちできる 放置しよう ほうちしよう				動詞-一般 
\\	「固まり(かたまり)」のアクセントは?2474	固まり 仮名 
\\	かたまり 漢字 
\\	固まり 発音図: 助詞: オ	固まり かたまり		固まり かたまり				名詞-普通名詞-一般 
\\	「通過(つうかする)」のアクセントは?2475	通過 仮名 
\\	つーか、つうか 漢字 
\\	通過 発音図: 助詞: オ	通過する つうかする		通過する つうかする 通過します つうかします 通過して つうかして 通過した つうかした 通過しない つうかしない 通過しなかった つうかしなかった 通過すれば つうかすれば 通過させる つうかさせる 通過される つうかされる 通過しろ つうかしろ 通過できる つうかできる 通過しよう つうかしよう				動詞-一般 
\\	「五分(ごふん)」のアクセントは?2476		五分 ごふん		五分 ごふん				名詞-数詞 
\\	「次ぐ(つぐ)」のアクセントは?2477	次ぐ 仮名 
\\	つぐ 漢字 
\\	次ぐ 発音図:	次ぐ つぐ		次ぐ つぐ 次ぎます つぎます 次いで ついで 次いだ ついだ 次がない つがない 次がなかった つがなかった 次げば つげば 次がせる つがせる 次がれる つがれる 次げ つげ 次ごう つごう				動詞-一般 
\\	「切り替える(きりかえる)」のアクセントは?2478	切り替える 仮名 
\\	きりかえる 漢字 
\\	切り替える 発音図: 発音図: 発音図:	切り替える きりかえる きりかえる		切り替える きりかえる きりかえる 切り替えます きりかえます 切り替えて きりかえて きりかえて 切り替えた きりかえた きりかえた 切り替えない きりかえない きりかえない 切り替えなかった きりかえなかった きりかえなかった 切り替えれば きりかえれば きりかえれば 切り替えさせる きりかえさせる きりかえさせる 切り替えられる きりかえられる きりかえられる 切り替えろ きりかえろ きりかえろ 切り替えられる きりかえられる きりかえられる 切り替えよう きりかえよう				動詞-一般 
\\	「貢献(こうけん)」のアクセントは?2479	貢献 仮名 
\\	こうけん、こーけん 漢字 
\\	貢献 発音図: 助詞: オ	貢献 こうけん		貢献 こうけん 貢献する こうけんする 貢献します こうけんします 貢献して こうけんして 貢献した こうけんした 貢献しない こうけんしない 貢献しなかった こうけんしなかった 貢献すれば こうけんすれば 貢献させる こうけんさせる 貢献される こうけんされる 貢献しろ こうけんしろ 貢献できる こうけんできる 貢献しよう こうけんしよう				動詞-一般 
\\	「強烈な(きょうれつな)」のアクセントは?2480	強烈 仮名 
\\	きょうれつ、きょーれつ 漢字 
\\	強烈 発音図: 助詞: オ	強烈な きょうれつな		強烈な きょうれつな 強烈です きょうれつです 強烈で きょうれつで 強烈だった きょうれつだった 強烈じゃない きょうれつじゃない 強烈じゃなかった きょうれつじゃなかった 強烈ならば きょうれつならば 強烈だ きょうれつだ 強烈に きょうれつに				形状詞-一般 
\\	「ブランド(ぶらんど)」のアクセントは?2481	ブランド 仮名 
\\	ブランド、ぶらんど 漢字 
\\	発音図: 助詞: オ	ブランド ぶらんど		ブランド ぶらんど				名詞-普通名詞-一般 
\\	「真っ赤な(まっかな)」のアクセントは?2482	真っ赤 仮名 
\\	まっか 漢字 
\\	真っ赤 発音図: 助詞: ダ	真っ赤な まっかな		真っ赤な まっかな 真っ赤です まっかです 真っ赤で まっかで 真っ赤だった まっかだった 真っ赤じゃない まっかじゃない 真っ赤じゃなかった まっかじゃなかった 真っ赤ならば まっかならば 真っ赤だ まっかだ 真っ赤に まっかに				形状詞-一般 
\\	「尊重(そんちょう)」のアクセントは?2483	尊重 仮名 
\\	そんちょう、そんちょー 漢字 
\\	尊重 発音図: 助詞: オ	尊重 そんちょう		尊重 そんちょう 尊重する そんちょうする 尊重します そんちょうします 尊重して そんちょうして 尊重した そんちょうした 尊重しない そんちょうしない 尊重しなかった そんちょうしなかった 尊重すれば そんちょうすれば 尊重させる そんちょうさせる 尊重される そんちょうされる 尊重しろ そんちょうしろ 尊重できる そんちょうできる 尊重しよう そんちょうしよう				動詞-一般 
\\	「代表的な(だいひょうてきな)」のアクセントは?2484	代表的 仮名 
\\	だいひょうてき、だいひょーてき 漢字 
\\	代表的 発音図: 助詞: オ	代表的な だいひょうてきな		代表的な だいひょうてきな 代表的です だいひょうてきです 代表的で だいひょうてきで 代表的だった だいひょうてきだった 代表的じゃない だいひょうてきじゃない 代表的じゃなかった だいひょうてきじゃなかった 代表的ならば だいひょうてきならば 代表的だ だいひょうてきだ 代表的に だいひょうてきに				形状詞-一般 
\\	「疲れ(つかれ)」のアクセントは?2485	疲れ 仮名 
\\	つかれ 漢字 
\\	疲れ 発音図: 助詞: オ	疲れ つかれ		疲れ つかれ				名詞-普通名詞-一般 
\\	「番号(ばんごう)」のアクセントは?2486	番号 仮名 
\\	ばんごう、ばんごー 漢字 
\\	番号 発音図: 助詞: オ	番号 ばんごう		番号 ばんごう				名詞-普通名詞-一般 
\\	「遺伝子(いでんし)」のアクセントは?2487	遺伝子 仮名 
\\	いでんし 漢字 
\\	遺伝子 発音図: 助詞: オ	遺伝子 いでんし		遺伝子 いでんし				名詞-普通名詞-一般 
\\	「事務(じむ)」のアクセントは?2488	事務 仮名 
\\	じむ 漢字 
\\	事務 発音図: 助詞: オ	事務 じむ		事務 じむ				名詞-普通名詞-一般 
\\	「解釈(かいしゃくする)」のアクセントは?2489	解釈 仮名 
\\	かいしゃく 漢字 
\\	解釈 発音図: 助詞: オ	解釈する かいしゃくする		解釈する かいしゃくする 解釈します かいしゃくします 解釈して かいしゃくして 解釈した かいしゃくした 解釈しない かいしゃくしない 解釈しなかった かいしゃくしなかった 解釈すれば かいしゃくすれば 解釈させる かいしゃくさせる 解釈される かいしゃくされる 解釈しろ かいしゃくしろ 解釈できる かいしゃくできる 解釈しよう かいしゃくしよう				動詞-一般 
\\	「会長(かいちょう)」のアクセントは?2490	会長 仮名 
\\	かいちょう、かいちょー 漢字 
\\	会長 発音図: 助詞: オ	会長 かいちょう		会長 かいちょう				名詞-普通名詞-一般 
\\	「祖父(そふ)」のアクセントは?2491	祖父 仮名 
\\	そふ 漢字 
\\	祖父 発音図: 助詞: オ	祖父 そふ		祖父 そふ				名詞-普通名詞-一般 
\\	「論理(ろんり)」のアクセントは?2492	論理 仮名 
\\	ろんり 漢字 
\\	論理 発音図: 助詞: オ	論理 ろんり		論理 ろんり				名詞-普通名詞-一般 
\\	「オーストラリア(おーすとらりあ)」のアクセントは?2493		オーストラリア おーすとらりあ		オーストラリア おーすとらりあ				名詞-固有名詞-地名-国 
\\	「徹底的な(てっていてきな)」のアクセントは?2494	徹底的 仮名 
\\	てっていてき 漢字 
\\	徹底的 発音図:	徹底的な てっていてきな		徹底的な てっていてきな 徹底的です てっていてきです 徹底的で てっていてきで 徹底的だった てっていてきだった 徹底的じゃない てっていてきじゃない 徹底的じゃなかった てっていてきじゃなかった 徹底的ならば てっていてきならば 徹底的だ てっていてきだ 徹底的に てっていてきに				形状詞-一般 
\\	「弾く(ひく)」のアクセントは?2496	弾く 仮名 
\\	ひく 漢字 
\\	弾く 発音図:	弾く ひく		弾く ひく 弾きます ひきます 弾いて ひいて 弾いた ひいた 弾かない ひかない 弾かなかった ひかなかった 弾けば ひけば 弾かせる ひかせる 弾かれる ひかれる 弾け ひけ 弾ける ひける 弾こう ひこう				動詞-一般 
\\	「ノート(のーと)」のアクセントは?2497	ノート 仮名 
\\	ノート、のーと 漢字 
\\	発音図: 助詞: オ	ノート のーと		ノート のーと				名詞-普通名詞-一般 
\\	「見回す(みまわす)」のアクセントは?2498	見回す 仮名 
\\	みまわす 漢字 
\\	見回す 発音図: 発音図:	見回す みまわす みまわす		見回す みまわす みまわす 見回します みまわします 見回して みまわして みまわして 見回した みまわした みまわした 見回さない みまわさない みまわさない 見回さなかった みまわさなかった みまわさなかった 見回せば みまわせば みまわせば 見回させる みまわさせる みまわさせる 見回される みまわされる みまわされる 見回せ みまわせ みまわせ 見回せる みまわせる みまわせる 見回そう みまわそう				動詞-一般 
\\	「欧米(おうべい)」のアクセントは?2499	欧米 仮名 
\\	おうべい、おーべい 漢字 
\\	欧米 発音図: 助詞: オ	欧米 おうべい		欧米 おうべい				名詞-普通名詞-一般 
\\	「武士(ぶし)」のアクセントは?2500	武士 仮名 
\\	ぶし 漢字 
\\	武士 発音図: 助詞: オ	武士 ぶし		武士 ぶし				名詞-普通名詞-一般 
\\	「籠もる(こもる)」のアクセントは?2501		籠もる こもる		籠もる こもる 籠もります こもります 籠もって こもって 籠もった こもった 籠もらない こもらない 籠もらなかった こもらなかった 籠もれば こもれば 籠もらせる こもらせる 籠もられる こもられる 籠もれ こもれ 籠もれる こもれる 籠もろう こもろう				動詞-一般 
\\	「囁く(ささやく)」のアクセントは?2502	ささやく{囁} 仮名 
\\	ささやく 漢字 
\\	囁く 発音図: 発音図:	囁く ささやく		囁く ささやく 囁きます ささやきます 囁いて ささやいて 囁いた ささやいた 囁かない ささやかない 囁かなかった ささやかなかった 囁けば ささやけば 囁かせる ささやかせる 囁かれる ささやかれる 囁け ささやけ 囁ける ささやける 囁こう ささやこう				動詞-一般 
\\	「隅(すみ)」のアクセントは?2503	隅 仮名 
\\	すみ 漢字 
\\	隅 発音図: 助詞: オ	隅 すみ		隅 すみ				名詞-普通名詞-一般 
\\	「写る(うつる)」のアクセントは?2504	写る 仮名 
\\	うつる 漢字 
\\	写る 発音図:	写る うつる		写る うつる 写ります うつります 写って うつって 写った うつった 写らない うつらない 写らなかった うつらなかった 写れば うつれば 写らせる うつらせる 写られる うつられる 写れ うつれ 写れる うつれる 写ろう うつろう				動詞-一般 
\\	「スペイン(すぺいん)」のアクセントは?2505		スペイン すぺいん		スペイン すぺいん				名詞-固有名詞-地名-国 
\\	「仕上げる(しあげる)」のアクセントは?2506	仕上げる 仮名 
\\	しあげる 漢字 
\\	仕上げる 発音図:	仕上げる しあげる		仕上げる しあげる 仕上げます しあげます 仕上げて しあげて 仕上げた しあげた 仕上げない しあげない 仕上げなかった しあげなかった 仕上げれば しあげれば 仕上げさせる しあげさせる 仕上げられる しあげられる 仕上げろ しあげろ 仕上げられる しあげられる 仕上げよう しあげよう				動詞-一般 
\\	「軟らかい(やわらかい)」のアクセントは?2507	軟らかい 仮名 
\\	やわらかい 漢字 
\\	軟らかい 発音図:							形容詞-一般 
\\	「騒ぐ(さわぐ)」のアクセントは?2508	騒ぐ 仮名 
\\	さわぐ 漢字 
\\	騒ぐ 発音図:	騒ぐ さわぐ		騒ぐ さわぐ 騒ぎます さわぎます 騒いで さわいで 騒いだ さわいだ 騒がない さわがない 騒がなかった さわがなかった 騒げば さわげば 騒がせる さわがせる 騒がれる さわがれる 騒げ さわげ 騒げる さわげる 騒ごう さわごう				動詞-一般 
\\	「廃止(はいし)」のアクセントは?2509	廃止 仮名 
\\	はいし 漢字 
\\	廃止 発音図: 助詞: オ	廃止 はいし		廃止 はいし 廃止する はいしする 廃止します はいしします 廃止して はいしして 廃止した はいしした 廃止しない はいししない 廃止しなかった はいししなかった 廃止すれば はいしすれば 廃止させる はいしさせる 廃止される はいしされる 廃止しろ はいししろ 廃止できる はいしできる 廃止しよう はいししよう				動詞-一般 
\\	「じっくり(じっくり)《〜考える》」のアクセントは?2510	じっくり(〜考える) 仮名 
\\	じっくり 漢字 
\\	発音図: 例文: カんがえる							副詞 
\\	「晴れる(はれる)」のアクセントは?2511	晴れる 仮名 
\\	はれる 漢字 
\\	晴れる 発音図:	晴れる はれる		晴れる はれる 晴れます はれます 晴れて はれて 晴れた はれた 晴れない はれない 晴れなかった はれなかった 晴れれば はれれば 晴れさせる はれさせる 晴れられる はれられる 晴れろ はれろ 晴れよう はれよう				動詞-一般 
\\	「取り除く(とりのぞく)」のアクセントは?2512	取り除く 仮名 
\\	とりのぞく 漢字 
\\	取り除く 発音図: 発音図:	取り除く とりのぞく とりのぞく		取り除く とりのぞく とりのぞく 取り除きます とりのぞきます 取り除いて とりのぞいて とりのぞいて 取り除いた とりのぞいた とりのぞいた 取り除かない とりのぞかない とりのぞかない 取り除かなかった とりのぞかなかった とりのぞかなかった 取り除けば とりのぞけば とりのぞけば 取り除かせる とりのぞかせる とりのぞかせる 取り除かれる とりのぞかれる とりのぞかれる 取り除け とりのぞけ とりのぞけ 取り除ける とりのぞける とりのぞける 取り除こう とりのぞこう				動詞-一般 
\\	「悪化(あっか)」のアクセントは?2513	悪化 仮名 
\\	あっか 漢字 
\\	悪化 発音図: 助詞: オ	悪化 あっか		悪化 あっか 悪化する あっかする 悪化します あっかします 悪化して あっかして 悪化した あっかした 悪化しない あっかしない 悪化しなかった あっかしなかった 悪化すれば あっかすれば 悪化させる あっかさせる 悪化される あっかされる 悪化しろ あっかしろ 悪化しよう あっかしよう				動詞-一般 
\\	「前述(ぜんじゅつ)」のアクセントは?2514	前述 仮名 
\\	ぜんじゅつ 漢字 
\\	前述 発音図: 助詞: オ							名詞-普通名詞-一般 
\\	「サッカー(さっかー)」のアクセントは?2515	サッカー 仮名 
\\	サッカー、さっかー 漢字 
\\	発音図: 助詞: オ	サッカー さっかー		サッカー さっかー				名詞-普通名詞-一般 
\\	「スカート(すかーと)」のアクセントは?2516	スカート 仮名 
\\	スカート、すかーと 漢字 
\\	発音図: 助詞: オ	スカート すかーと		スカート すかーと				名詞-普通名詞-一般 
\\	「普段(ふだん)《〜の服装》」のアクセントは?2517	ふだん{普段}(〜の服装) 仮名 
\\	ふだん 漢字 
\\	普段 発音図: 例文: ノ服装	普段 ふだん		普段 ふだん				副詞 
\\	「貧しい(まずしい)」のアクセントは?2518	貧しい 仮名 
\\	まずしい 漢字 
\\	貧しい 発音図:	貧しい まずしい		貧しい まずしい 貧しいです まずしいです 貧しくて まずしくて まずしくて 貧しかった まずしかった まずしかった 貧しくない まずしくない まずしくない 貧しくなかった まずしくなかった まずしくなかった 貧しければ まずしければ まずしければ 貧しい まずしい 貧しく まずしく まずしく				形容詞-一般 
\\	「案内(あんない)」のアクセントは?2519	案内 仮名 
\\	あんない 漢字 
\\	案内 発音図: 助詞: オ	案内 あんない		案内 あんない 案内する あんないする 案内します あんないします 案内して あんないして 案内した あんないした 案内しない あんないしない 案内しなかった あんないしなかった 案内すれば あんないすれば 案内させる あんないさせる 案内される あんないされる 案内しろ あんないしろ 案内できる あんないできる 案内しよう あんないしよう				動詞-一般 
\\	「任務(にんむ)」のアクセントは?2520	任務 仮名 
\\	にんむ 漢字 
\\	任務 発音図: 助詞: オ	任務 にんむ		任務 にんむ				名詞-普通名詞-一般 
\\	「看板(かんばん)」のアクセントは?2521	看板 仮名 
\\	かんばん 漢字 
\\	看板 発音図: 助詞: オ	看板 かんばん		看板 かんばん				名詞-普通名詞-一般 
\\	「落ち込む(おちこむ)」のアクセントは?2522	落ち込む 仮名 
\\	おちこむ 漢字 
\\	落ち込む 発音図: 発音図:	落ち込む おちこむ おちこむ		落ち込む おちこむ おちこむ 落ち込みます おちこみます 落ち込んで おちこんで おちこんで 落ち込んだ おちこんだ おちこんだ 落ち込まない おちこまない おちこまない 落ち込まなかった おちこまなかった おちこまなかった 落ち込めば おちこめば おちこめば 落ち込ませる おちこませる おちこませる 落ち込まれる おちこまれる おちこまれる 落ち込め おちこめ おちこめ 落ち込もう おちこもう				動詞-一般 
\\	「実質的(じっしつてき)」のアクセントは?2523	実質的 仮名 
\\	じっしつてき 漢字 
\\	実質的 発音図:							形状詞-一般 
\end{CJK}
\end{document}