\documentclass[8pt]{extreport} 
\usepackage{hyperref}
\usepackage{CJKutf8}
\begin{document}
\begin{CJK}{UTF8}{min}
\\	政治家の汚職は国民の信頼を裏切るもので、許すべからざる行為である。	「べからざる」、「べからず」 意味:・・・べきではない。・・・してはならない。 「もので」 意味:・・・ため。・・・だから。・・・ので。	
\\	昔、ある月刊誌で書評を頼まれたことがある。僕は本を書く人間で、批評する人間じゃないから書評ってできればやりたくないんだけど、そのときは事情があって、「まあいいや、やりましょう」、と引き受けた。でも普通どおりにやっても面白くないから、架空の本をでっちあげて、それを詳しく評論することにした。実在しない人の伝記の書評とかね。これはやってみると、なかなか愉快でした。でっちあげをするぶん頭は使うけれど、本を読む時間は節約できる。	「ぶん」、「ぶんには」 意味:だとしたら。・・・のなら。・・・と。	
\\	現在の生産量を維持すべく最大限の努力をしている。	「べく」 意味:・・・ために。・・・には。 補足:「するべく」 
\\	「すべく」。	
\\	来年の博覧会開催に間に合わせるべく,突貫工事が行われている。	「べく」 意味:・・・ために。・・・には。 補足:「するべく 
\\	「すべく」。	
\\	そんなこと気にしていた分には何も出来なくなってしまう。	「ぶん」、「ぶんには」 意味:だとしたら。・・・のなら。・・・と。	
\\	ただ見る分には差し支えないでしょう。	「ぶん」、「ぶんには」 意味:だとしたら。・・・のなら。・・・と。	
\\	忘れているだろうと思いきや,犬は僕を見て嬉しそうに尻尾を振った。	「思いきや」意味:~と思ったところが以外にも。そう感じるだけで「実際にはそうでないかもしれない」と示唆する。	
\\	直ったと思いきやまたすぐに壊れてしまった。	「思いきや」意味:~と思ったところが以外にも。そう感じるだけで「実際にはそうでないかもしれない」と示唆する。	
\\	ネットバブルのさなかとあって、優秀な技術者のヘッドハンティングで提示される年収は、高いが上にもさらなる急上昇を続けていた。	「とあって」 意味:~だから。~なので。	
\\	今日は休日とあって大変な人出だ。	「とあって」 意味:~だから。~なので。	
\\	人気作家A氏の講演会が無料とあって、多くのファンが詰めかけた。	「とあって」 意味:~だから。~なので。	
\\	どんなに狂気じみていても、不適切であっても、ばからしく聞こえたって私は構わない。	「じみる」 意味:~のように見える。~に似ている。~みたい。~らしい。	
\\	ぶしつけなことを申し上げるようですが、それは精神障害者じみています。	「じみる」意味:~のように見える。~に似ている。~みたい。~らしい。	
\\	子供じみたことを言うな。	「じみる」意味:~のように見える。~に似ている。~みたい。~らしい。	
\\	まんまと逃げおおせた。	「~おおせる」 意味:完全に。すっかり終える。成し遂げる。	
\\	いつまでも隠しおおせるものではない。	「~おおせる」 意味:完全に。すっかり終える。成し遂げる。	
\\	聞き損なった。	「~損なう」 意味:1)しくじる。2)危うく・・・ところだった。	
\\	列車に乗り損なった。	「~損なう」 意味:1)しくじる。2)危うく・・・ところだった。	
\\	頃合いを見計らってやらないとやり損なうよ。	「~損なう」 意味:1)しくじる。2)危うく・・・ところだった。	
\\	危うく命を落とし損なった。	「~損なう」 意味:1)しくじる。2)危うく・・・ところだった。	
\\	彼らは水を使い尽くした。	「~尽くす」 意味:全部・・・する。すっかり・・・してしまう。	
\\	その問題はすでに論じ尽くされている。	「~尽くす」 意味:全部・・・する。すっかり・・・してしまう。	
\\	灯油を使い尽くしてしまった。	「~尽くす」 意味:全部・・・する。すっかり・・・してしまう。	
\\	猛火が町を焼き尽くした。	「~尽くす」 意味:全部・・・する。すっかり・・・してしまう。	
\\	1週間でこの本を5冊も読みきれるかしら。	「~切れる」 意味:1)(『切れる』の形で)全部・・・する。すっかり・・・してしまう。2)(『切れない』の形で)できない。	
\\	食べきれない程のごちそうが出た。	「~切れる」 意味:1)(『切れる』の形で)全部・・・する。すっかり・・・してしまう。2)(『切れない』の形で)できない。	
\\	暑くて我慢しきれず上着を脱いだ。	「~切れる」 意味:1)(『切れる』の形で)全部・・・する。すっかり・・・してしまう。2)(『切れない』の形で)できない。	
\\	この仕事は1日ではやり切れない。	「~切れる」 意味:1)(『切れる』の形で)全部・・・する。すっかり・・・してしまう。2)(『切れない』の形で)できない。	
\\	あの臭さにはとてもやり切れない。	「~切れる」 意味:1)(『切れる』の形で)全部・・・する。すっかり・・・してしまう。2)(『切れない』の形で)できない。	
\\	うるさくてやり切れない。	「~切れる」 意味:1)(『切れる』の形で)全部・・・する。すっかり・・・してしまう。2)(『切れない』の形で)できない。	
\\	やり切れないのは給料がちっとも上がらないことだ。	「~切れる」 意味:1)(『切れる』の形で)全部・・・する。すっかり・・・してしまう。2)(『切れない』の形で)できない。	
\\	今夜家に帰るのが待ちきれない。	「~切れる」 意味:1)(『切れる』の形で)全部・・・する。すっかり・・・してしまう。2)(『切れない』の形で)できない。	
\\	結果が待ちきれない。	「~切れる」 意味:1)(『切れる』の形で)全部・・・する。すっかり・・・してしまう。2)(『切れない』の形で)できない。	
\\	靴を履き潰した。	「~潰す」 意味:1)・・・してもうそれ以上使えなくする。・・・して失くす。	
\\	万年筆を五本も書き潰した。	「~潰す」 意味:1)・・・してもうそれ以上使えなくする。・・・して失くす。	
\\	彼女はしゃべりまくった。	「~まくる」 意味:盛んに・・・する。ずっとその動作を続ける。	
\\	彼女は物語を書きまくった。	「~まくる」 意味:盛んに・・・する。ずっとその動作を続ける。	
\\	彼は働きまくった。	「~まくる」 意味:盛んに・・・する。ずっとその動作を続ける。	
\\	私は一歩たりとも退くつもりはない。	「~たりとも」 意味:~であっても。~も。	
\\	1秒たりとも無駄にはできない。	「~たりとも」 意味:~であっても。~も。	
\\	その店からは衣類を一枚たりとも買ったことがない。	「~たりとも」 意味:~であっても。~も。	
\\	お金が仕事をする理由になったことは一度たりともない。それは後から付いてくるものだと常に思ってきた。	「~たりとも」 意味:~であっても。~も。	
\\	彼は芸術家肌である。	「~肌」 意味:気性。気質。ある意味で。	
\\	彼女は多分に外交家肌だ。	「~肌」 意味:気性。気質。ある意味で。	
\\	彼女は詩人肌である。	「~肌」 意味:気性。気質。ある意味で。	
\\	私は疲れ果てている。	「~果てる」 意味:その極に達する。	
\\	呆れ果てて物も言えなかった。	「~果てる」 意味:その極に達する。 補足:「物が言えない」
\\	電車の中で財布を掏られて困り果てた。	「~果てる」 意味:その極に達する。	
\\	お電話またはメールにてできる限りの対応をさせていただきます。	「にて」 意味:
\\	「で」。	
\\	これにて閉廷します。	「にて」 意味:
\\	「で」。	
\\	当選者は抽選にて決定させていただきます。	「にて」 意味:
\\	「で」。	
\\	当イベントの展示場にて、写真撮影を行いたいと思っておりますが、撮影が許可されているのか分かりません。	「にて」 意味:
\\	「で」。	
\\	とある店の前を通りかかった。	「とある」 意味:ある~。	
\\	「あの私立大は入るのすごく難しいって聞いたけど、他にも考えてる学校あるの?」 「私もそう聞いたわ。とある州立大もリストに入れてあるんだ。そっちの方が可能性が高いと思って」	「とある」 意味:ある~。	
\\	私は、日本の、とある村を訪れたいのです。	「とある」 意味:ある~。	
\\	彼らとは家族ぐるみの交際をしている。	「~ぐるみ」 意味:そのものを含んですべて。そのものをひっくるめて全部。	
\\	彼らは家族ぐるみの付き合いを望んでいた。	「~ぐるみ」 意味:そのものを含んですべて。そのものをひっくるめて全部。	
\\	対テロリズム戦を国家ぐるみで決断しなくてはならない。	「~ぐるみ」 意味:そのものを含んですべて。そのものをひっくるめて全部。	
\\	彼女は家族ぐるみの友人です。	「~ぐるみ」 意味:そのものを含んですべて。そのものをひっくるめて全部。	
\\	彼は家族ぐるみで付き合いをしている友人。	「~ぐるみ」 意味:そのものを含んですべて。そのものをひっくるめて全部。	
\\	私はそこで60がらみの男性が30歳くらいの女性と歩いているところを見掛けた。	「~がらみ」 意味:1)大体そのくらい,その前後である。2)それと密接な関係がある。~の関係で。	
\\	オッ、来年のカレンダーじゃん。来年は何回祝日がらみの連休があるかな。	「~ぐるみ」 意味:そのものを含んですべて。そのものをひっくるめて全部。	
\\	その赤ちゃんは涎まみれの顔で笑っている。	「~まみれ」 意味:1)そのものが一面に汚らしい感じでついている。だらけ。2)困った状態である。	
\\	シャツを脱ぎなさい。汗まみれじゃない。	「~まみれ」 意味:1)そのものが一面に汚らしい感じでついている。だらけ。2)困った状態である。	
\\	包丁を油まみれのお湯に入れた。	「~まみれ」 意味:1)そのものが一面に汚らしい感じでついている。だらけ。2)困った状態である。	
\\	自分の収入の範囲内で暮らしていかないと、借金まみれになるよ。	「~まみれ」 意味:1)そのものが一面に汚らしい感じでついている。だらけ。2)困った状態である。	
\\	それはいいことずくめです。	「~ずくめ」 意味:何から何まで、そればかりである。	
\\	彼は全身、黒ずくめでした。	「~ずくめ」 意味:何から何まで、そればかりである。	
\\	彼女の服装は、頭からつま先まで白ずくめでした。	「~ずくめ」 意味:何から何まで、そればかりである。	
\\	良きにつけ悪しきにつけ、金を持っている者に従え。	「~につけ」 意味:1)たびに。2)~うちに。3)(『・・・につけ・・・につけ』の形で)・・・でも・・・でも(いい)。	
\\	ねたみ深い者は、隣人の豊満さを見るにつけ、自らの身が細る。(ことわざ)	「~につけ」 意味:1)たびに。2)~うちに。3)(『・・・につけ・・・につけ』の形で)・・・でも・・・でも(いい)。	
\\	彼女の窮状を聞くにつけ見るにつけ義憤を感じる。	「~につけ」 意味:1)たびに。2)~うちに。3)(『・・・につけ・・・につけ』の形で)・・・でも・・・でも(いい)。	
\\	都会から田舎に移り住んだ人の話を聞くにつけ、「田舎は人を癒す力を持っている」とつくづく感じる。	「~につけ」 意味:1)たびに。2)~うちに。3)(『・・・につけ・・・につけ』の形で)・・・でも・・・でも(いい)。	
\\	この鍋は、いため物に、揚げ物にと何にでも使えて便利です。	「~にと」 意味:~も。~や。	
\\	彼女の口の利き方には意地悪めいたところがある。	「めく」 意味:そのような状態になる。それに似たようす。	
\\	皮肉めくかも知れないが・・・	「めく」 意味:そのような状態になる。それに似たようす。	
\\	だいぶ春めいてきました。	「めく」 意味:そのような状態になる。それに似たようす。	
\\	あの子は大人めいた口をきく。	「めく」 意味:そのような状態になる。それに似たようす。	
\\	農業のかたわら織物もしている。	「かたわら」 意味:
\\	商売のかたわら絵も描く。	「かたわら」 意味:
\\	話を聞くが早いか家を飛び出した。	「が早いか」 意味:・・・するとすぐに。・・・するや否や。・・・するかと思うと。	
\\	見るなり顔色を変えた。	「なり」 意味;1)…したとたんに。…するとすぐ。	
\\	かばんを置くやいなや、外に飛び出した。	「や否や、が否や」 意味:・・・するとすぐに。・・・するが早いか。	
\\	そのミュージカルで主役を演じるやいなや、彼はニューヨーク劇場界の花形になった。	「や否や、が否や」 意味:・・・するとすぐに。・・・するが早いか。	
\\	彼の浮気のニュースが報じられるやいなや、ツイッターはジョークと批判で炎上しました。	「や否や、が否や」 意味:・・・するとすぐに。・・・するが早いか。	
\\	指導者が暗殺されるがいなや、混乱状態に陥った。	「や否や、が否や」 意味:・・・するとすぐに。・・・するが早いか。	
\\	私たちが家に帰るやいなや、彼女は怒りだした。	「や否や、が否や」 意味:・・・するとすぐに。・・・するが早いか。	
\\	そうだ、どうせなら散歩がてらに、林道に行ってプチ森林谷でも・・・。	「がてら」 意味:ある事柄をするときに、それを機会に他の事柄をもする意を表す。…のついでに。	
\\	散歩がてら彼の家に寄った。	「がてら」 意味:ある事柄をするときに、それを機会に他の事柄をもする意を表す。…のついでに。	
\\	見物がてら商用で京都へ行った。	「がてら」 意味:ある事柄をするときに、それを機会に他の事柄をもする意を表す。…のついでに。	
\\	それを皮切りとして全国に暴動が起こった。	「を皮切りに、を皮切りとして」 意味:
\\	彼の発案を皮切りに熱心な討議が始まった。	「を皮切りに、を皮切りとして」 意味:
\\	身をもってその仕事の大変さを知った。	「を以って」 意味:1)
\\	。2)・・・の理由で。3)「を」を強調する。4)よって。それでは。	
\\	彼女の能力をもってすれば,この問題の解決は容易であろう。	"「を以って」 意味:1)
\\	。2)・・・の理由で。3)「を」を強調する。4)よって。それでは。 補足:「すれば」 
\\	スイスは風光の明媚をもって知られる。	「を以って」 意味:1)
\\	。2)・・・の理由で。3)「を」を強調する。4)よって。それでは。	
\\	彼女は秀才をもって知られている。	「を以って」 意味:1)
\\	。2)・・・の理由で。3)「を」を強調する。4)よって。それでは。	
\\	もって彼を最優秀選手と認定する。	「を以って」 意味:1)
\\	。2)・・・の理由で。3)「を」を強調する。4)よって。それでは。	
\\	しかし我々はもし必要とあれば、武力を行使することを躊躇しないであろう。	「とあれば、とあらば」 意味:もしそうであるならば。	
\\	必要とあらば責任が増えても喜んでそれを担うであろう。	「とあれば、とあらば」 意味:もしそうであるならば。	
\\	情報がきちんとあればエイズが怖いことはありません。	「とあれば、とあらば」 意味:もしそうであるならば。	
\\	我が社のサポートエンジニアは、必要とあればお客様のところへ伺ってサポートいたします。	「とあれば、とあらば」 意味:もしそうであるならば。	
\\	お呼びとあればいつでも伺います。	「とあれば、とあらば」 意味:もしそうであるならば。	
\\	愛する人のためとあれば、どんな苦労にも耐えられるはずだ。	「とあれば、とあらば」 意味:もしそうであるならば。	
\\	彼女なくしては会がはじまらない。	「無くしては」 意味:
\\	努力なくしては何物も得られない。	「無くしては」 意味:
\\	困難なくしては、喜びの味も分からない。	「無くしては」 意味:
\\	君の助言なくしてはやっていけない。	「無くしては」 意味:
\\	それは彼ならではの技だ。	「ならではの」 意味:だけの。	
\\	それは日本ならではの料理だ。	「ならではの」 意味:だけの。	
\\	母親ならではの手厚い看護をした。	「ならではの」 意味:だけの。	
\\	いつか皆さんに、わが国または文化ならではの物を幾つか紹介させてください。	「ならではの」 意味:だけの。	
\\	誰でもその人ならではの特別なものを内側に秘めている。	「ならではの」 意味:だけの。	
\\	いやしくも成すに足る事なら立派にやるだけの価値がある。	「に足る」 意味:1)
\\	この農園は私たちの必要を満たすに足るだけの野菜を産出する。	「に足る」 意味:1)
\\	小夜子は印刷業でいい暮らしをするに足るお金をかせいだ。	「に足る」 意味:1)
\\	大衆によって重んじられている人が必ずしもそれを受けるに足る人とは言えない。	「に足る」 意味:1)
\\	遊びかたがた練習をした。	「かたがた」 意味:・・・の序でに。・・・がてらに。	
\\	挨拶かたがた旧交を温めるため彼女を訪ねた。	「かたがた」 意味:1)・・・の序でに。・・・がてらに。	
\\	どんなに後悔したところで、借金の一銭も返せるわけではない。(ことわざ)	「動詞の過去形 
\\	~ところで」 意味:~をしても。~をしたって(何も変わらない)。	
\\	どんなに逃げようとしたところで無駄です。	「動詞の過去形 
\\	~ところで」 意味:~をしても。~をしたって(何も変わらない)。	
\\	それが何であれ私は構わない。	「であれ」 意味:~であっても。	
\\	子供であれ老人であれ関係無い。	「であれ」 意味:~であっても。	
\\	そのレースに勝つ人は誰であれ、その賞をもらえます。	「であれ」 意味:~であっても。	
\\	たとえ天候がどうであれ、私は行きます。	「であれ」 意味:~であっても。	
\\	どこであれ道路を横切る時に右左を見るというのはいいことだ。	「であれ」 意味:~であっても。	
\\	なんであれ物事の本体を捕らえる事が肝心だ。	「であれ」 意味:~であっても。	
\\	偶然であれ故意であれ、彼がそれをしたのは本当だ。	「であれ」 意味:~であっても。	
\\	友人からのファクスで恩師の急逝の報に接し悲しみにたえなかった。	「~に耐えない」 意味:
\\	まこと感謝に耐えぬ。	「~に耐えない」 意味:
\\	それを知って喜びにたえない。	「~に耐えない」 意味:
\\	お忙しいところをお邪魔しました。	「~ところを」 意味:
\\	ちょっと気が緩んだところを相手に付け込まれたんです。	「~ところを」 意味:
\\	彼は溺れかけているところを助けられた。	「~ところを」 意味:
\\	火事が起こる直前に、175センチぐらいで、髭を生やした60代の男性が銀行を立ち去っていくところを見られた。	「~ところを」 意味:
\\	彼女は意識を無くし、呼び掛けても反応しなかったところを発見された。	"「~ところを」 意味:
\\	補足:「呼び掛けても反応しない」
\\	赤ん坊は母親を求めて声を限りに泣き叫んだ。	「を限りに」 意味:
\\	彼女は声を限りに叫んだ。	「を限りに」 意味:
\\	この1回に限ってそれを例外とする。	「に限って」 意味:
\\	この顧客層に限って言えば、使いやすさよりスピードが重視されています。	「に限って」 意味:
\\	うちの子に限ってそんなことはありません。	「に限って」 意味:
\\	他の人はともかく、あなたに限って盗みを働くはずはないと思っていました。	「に限って」 意味:
\\	夏が来るたびに、こんな感じのもの欲しくて探すのよね。でも、欲しい時に限って、割引してないのよ!だからたくさん買って、何枚かは来年の夏用に取っておきたいの。	「に限って」 意味:
\\	彼に限ってうそはつかないはずだ。	「に限って」 意味:
\\	彼に限って、人を裏切るはずがない。	「に限って」 意味:
\\	早く帰りたい時に限って、問題が起こるんだから。	「に限って」 意味:
\\	どうした?今日に限って。	「に限って」 意味:
\\	この事件は、感情ではなく、法律に即して判断すべきだ。	「に即して」 意味:
\\	あなたが賛成しようがするまいが、私は考えを変えるわけにいかない。	「~ようが」「~ようがまいが」 意味:
\\	あなたが彼と一緒に行こうが家に残っていようが、彼は泳ぎに行くだろう。	「~ようが」「~ようがまいが」 意味:
\\	我々は、勝とうが負けようがフェアに戦わなければならない。	「~ようが」「~ようがまいが」 意味:
\\	願っただけで望みが叶うなら貧乏人も金持ちになれようが。	「~が」 意味:
\\	戦車や飛行機は軍隊を打ち破ることはできようが、国民を征服することはできない。	「~ようが」「~ようがまいが」 意味:
\\	彼女が来ようがこないが結果は同じだろう。	「~ようが」「~ようがまいが」 意味:
\\	彼が来ようが来まいが私にはどうでもいいことだ。	「~ようが」「~ようがまいが」 意味:
\\	彼をうそつきということはできようが、悪人とはいえない。	「~ようが」「~ようがまいが」 意味:
\\	すべては準備のいかんにかかっている。	「いかん」 意味:1)
\\	商品開発チームがこのまま開発を続けられるかどうかは、新製品に対する客の反応いかんだ。	「いかん」 意味:1)
\\	この試みが成功するか否かは住民の協力いかんによる。	「いかん」 意味:1)
\\	「~か否か」 意味:~かどうか。	
\\	いくら頼まれても、これ以上はいかんともしがたい。	いかんともしがたい 意味:駄目だ。	
\\	動機のいかんにかかわらず彼は非難されても仕方がない。	「いかん」 意味:1)
\\	天性と努力が相まって彼女は成功した。	「相まって」 意味:
\\	忍耐力と体力が相まって彼は名ピッチャーになった。	「相まって」 意味:
\\	民衆の貧困をよそに、世界の軍事費はほとんど1兆ドルに達しました。	「~を余所に」、「~を余所にして」 意味:・・・にもかかわらず。・・・なのに。	
\\	仕事をよそにして彼は自動車レースに夢中になった。	「~を余所に」、「~を余所にして」 意味:・・・にもかかわらず。・・・なのに。	
\\	皆の騒ぎをよそに,彼は小説に読みふけっていた。	「~を余所に」、「~を余所にして」 意味:・・・にもかかわらず。・・・なのに。	
\\	わずか6ヶ月後に最初の仕事をやめるとしたら、愚かとは言わないまでも賢いとは言えない。	
\\	までも」 意味:
\\	われわれは過ぎし日の事を、必ずしも愛情とは言えないまでも少なくとも一種の憧れを持ってふりかえるのである。	
\\	までも」 意味:
\\	一を聞いて十を知る、とまではいかないまでももうちょっと物分りがよくなってほしいよ。	
\\	までも」 意味:
\\	英語の試験では、彼女はオールAとは言わないまでも、いつも良い点をとる。	
\\	までも」 意味:
\\	君のレポートは、優秀とは言えないまでもかなりよい。	
\\	までも」 意味:
\\	昨晩は、寒いとはいわないまでも涼しかった。	
\\	までも」 意味:
\\	私は敵意を持ったとは言わないまでも冷ややかな応対を受けた。	
\\	までも」 意味:
\\	彼は、無礼だとはいわないまでも、礼儀正しくはなかった。	
\\	までも」 意味:
\\	彼はけちとは言わないまでもとても倹約家だ。	
\\	までも」 意味:
\\	彼は過激とは言わないまでも非常に進歩的だ。	
\\	までも」 意味:
\\	彼は太っていると言えないまでも大柄な人だ。	
\\	までも」 意味:
\\	彼はけちとは言わないまでも、とても倹しい。	
\\	までも」 意味:
\\	自分が利口だと思っても差し支えないが、それだからと言って私を嘲笑うことはできないよ。	「差し支えない」 意味:1)構わない。気にしていない。2)どうぞ。	
\\	下品極まる盛り場に通いつける。	「~きわまる」 意味:
\\	それは残酷極まる行為だ。	「~きわまる」 意味:
\\	女に政治を任せると危険極まりない。	「~きわまりない」 意味:
\\	彼の講義は退屈極まりない。	「~きわまる」 意味:
\\	彼はできるだけ多くの女性と寝ようという淫乱極まりない男です。	「~きわまる」 意味:
\\	彼は見ず知らずの人間に対して無礼極まりない表現を使うことがよくある。	「~きわまる」 意味:
\\	同情の念を禁じ得ない。	「を禁じ得ない」 意味:(ある感情を)抑えることができない。	
\\	感嘆を禁じ得ない。	「を禁じ得ない」意味:(ある感情を)抑えることができない。	
\\	彼の言い方はちょっと恩きせがましい嫌いがあった。	「きらいがある」 意味:気味。また、好ましくない傾向。	
\\	彼は熱心になりすぎる嫌いがあるね。	「きらいがある」 意味:気味。また、好ましくない傾向。	
\\	この説明書は詳しすぎる嫌いがある。	「きらいがある」 意味:気味。また、好ましくない傾向。	
\\	彼が仕事を断ったばかりに、計画は中止を余儀なくされた。	「ばかりに」 意味:・・・だから。・・・なので。 「~を余儀なくされる」 意味:~をしなくてはならない。~をして仕方がない。	
\\	ちょっと無理をしたばかりに、入院する羽目になった。	「ばかりに」 意味:・・・だから。・・・なので。	
\\	この世界に存在するすべてのものは、絶え間なく変化してやまない。	「~てやまない」 意味:
\\	彼女の一日も早い回復を祈って止まない。	「~てやまない」 意味:
\\	「君と話してもしょうがないな」 「そんなこと言うなんて聞き捨てならないわ」	"「~てならない」 意味:1)
\\	~してはならない。 補足:「聞き捨てならない」 
\\	妻が浮気しているという気がしてならないんだ。	「~てならない」 意味:1)
\\	それは聞き捨てならない言葉です。	"「~てならない」 意味:1)
\\	~してはならない。 補足:「聞き捨てならない」 
\\	結果を早く知りたくてたまらない。	「~てたまらない」 意味:
\\	こういう知らせなら待ったかいがあった。	「~かいがある」 意味:1)
\\	「(~かいもない)の形で」
\\	彼女は努力のかいがあって一等になった。	「~かいがある」 意味:1)
\\	「(~かいもない)の形で」
\\	何かやりがいのある仕事につきたい。	"「~かいがある」 意味:1)
\\	「(~かいもない)の形で」
\\	補足:「やりがいがある」
\\	みんなの手厚い看護のかいもなく彼は息を引き取った。	「~かいがある」 意味:1)
\\	「(~かいもない)の形で」
\\	彼には忠告してもかいがなかった。	「~かいがある」 意味:1)
\\	「(~かいもない)の形で」
\\	仲を取り持とうとしたがそのかいがなかった。	「~かいがある」 意味:1)
\\	「(~かいもない)の形で」
\\	そのままではすまないよ。	「~ではすまない」 意味:1)
\\	悪い事に関わる者は、決して悪に染まらずにはすまない。	「~ではすまない」 意味:1)
\\	そこでのご経験を踏まえ、何かアドバイスはありますか?	「~をふまえて」「~をふまえ」「~を踏まえると」 意味:1)
\\	新しい証拠を踏まえて、新たな審理が始まるだろう。	「~をふまえて」「~をふまえ」「~を踏まえると」 意味:1)
\\	筆者は、さまざまな意見を踏まえて書くよう努めている。	「~をふまえて」「~をふまえ」「~を踏まえると」 意味:1)
\\	担当のファイナンシャル・プランナーは、事業を取り巻く現在の状況を踏まえると、事態が好転するには約18~24カ月かかると予測しています。	「~をふまえて」「~をふまえ」「~を踏まえると」 意味:1)
\\	そこを押してお願いします。	「~を押して」 意味:1)
\\	彼女は偏頭痛を押して出掛けた。	「~を押して」 意味:1)
\\	彼は嵐を押して出かけた。	「~を押して」 意味:1)
\\	熱があるのに押して出かけた。	「~を押して」 意味:1)
\\	人身保護法の令状には、囚人にすぐに聴聞会を開くか、さもなければ解放すべしと書かれている。	「~べし」 意味:・・・べきだ。・・・しなくてはならない。 補足:「するべし」
\\	「すべし」。	
\\	生あるうちは楽しむべし。	「~べし」 意味:・・・べきだ。・・・しなくてはならない。 補足:「するべし」
\\	「すべし」。	
\\	後は推して知るべし。	"「~べし」 意味:・・・べきだ。・・・しなくてはならない。 補足:「するべし」
\\	「すべし」。 補足:「推して知るべし」
\\	1か月を経ても音沙汰がない。	「~を経る」 意味:1)
\\	事故以来3年を経た。	「~を経る」 意味:1)
\\	京都を経て大阪へ行った。	「~を経る」 意味:1)
\\	書類は課長を経て重役に渡される。	「~を経る」 意味:1)
\\	許可が下りるまでには複雑な手続きを経なければならない。	「~を経る」 意味:1)
\\	多くの困難を経てようやく成功した。	「~を経る」 意味:1)
\\	悪天候ゆえに旅行は延期された。	「~ゆえ」 意味:~なので。~だから。~のため。	
\\	何ゆえにうそをついたのか?	「~ゆえ」 意味:~なので。~だから。~のため。 補足:「何ゆえ」
\\	「何ゆえに」 
\\	その警官は腕に軽い切り傷を負いながらも、どうにか容疑者をパトカーに乗せることができた。	「ながらも」 意味:
\\	何人もいる友達よりは、僅かながらも自分の財布の中身の方がよほど頼りになる。	「ながらも」 意味:
\\	単語に苦労しながらも、その子はきちんとした文を書いた。	「ながらも」 意味:
\\	可能な限り平和的外交によりながらも、必要なら武力を用いてわれわれは行動する。	「ながらも」 意味:
\\	大統領選で得票数では上回りながらも選挙には敗北した。	「ながらも」 意味:
\\	少量ながらも飲む回数が多過ぎれば、飲み過ぎに変わりない。	「ながらも」 意味:
\\	彼女の所作は大変優雅でありながらも計算されたものでした。	「ながらも」 意味:
\\	憩を夢見ながらも、働き続けよ。(ことわざ)	「ながらも」 意味:
\\	貧乏ながらも幸せな者は、この上もなく豊か。	「ながらも」 意味:
\\	私たち、別に啀み合っている敵同士ではあるまいし。	「~ではあるまいし」意味:・・・ではないし。	
\\	私達は金持ちではあるまいし。	「~ではあるまいし」意味:・・・ではないし。・・・じゃないし。	
\\	技術を習得するには時間がかかり、それを学ぶのに人の一生といえども短い。	「~といえども」意味:1)・・・とは言え。・・・と言うけれども。2)・・・とあれば。	
\\	君の指摘、中らずといえども遠からずだね。	「~といえども」意味:1)・・・とは言え。・・・と言うけれども。2)・・・とあれば。	
\\	親友といえども、その友情を維持する努力が必要である。	「~といえども」意味:1)・・・とは言え。・・・と言うけれども。2)・・・とあれば。	
\\	貧しいといえども彼女は幸せだ。	「~といえども」意味:1)・・・とは言え。・・・と言うけれども。2)・・・とあれば。	
\\	「あのう、車のライト、つけっぱなしですよ」 「あ、ありがとう!いっつもこれやって、バッテリーが上がっちゃうんですよね。どうもありがとう!」	「っぱなし」 意味:その動作をしたままで放置する。	
\\	もう1時間もここに座りっぱなしです。	「っぱなし」 意味:その動作をしたままで放置する。	
\\	ステージの上では緊張しっぱなしだった。	「っぱなし」 意味:その動作をしたままで放置する。	
\\	洋子、汚い靴下を居間に置きっぱなしにしないの! 洗濯室に持っていきなさい!	「っぱなし」 意味:その動作をしたままで放置する。	
\\	分かった、分かった。でも、もう5時間もずっと飲みっぱなしなんだよ。マジやめなって!	「っぱなし」 意味:その動作をしたままで放置する。	
\\	この能力は多くの人々が生まれながらに持っているものだ。	「ながらに」 意味: 
\\	「ながら」, 
\\	「に」
\\	彼女は自分の災難を涙ながらに大げさに訴えた。	「ながらに」 意味: 
\\	「ながら」, 
\\	「に」
\\	子供は生まれながらに創造力を持っている。	「ながらに」 意味: 
\\	「ながら」, 
\\	「に」
\\	涙ながらに彼を罵った。	「ながらに」 意味: 
\\	「ながら」, 
\\	「に」
\\	涙ながらに電話で話した。	「ながらに」 意味: 
\\	「ながら」, 
\\	「に」
\\	家に居ながらにして図書館や美術館の情報を活用できる。	「~ながらにして」 意味:
\\	「ながら」, 
\\	「にして」
\\	日本に居ながらにして世界中の人と話をすることができる。	「~ながらにして」 意味:
\\	「ながら」, 
\\	「にして」
\\	雄一は今や以前にもまして勤勉に働いている。	「~にも増して」 意味:・・・より。	
\\	何物にもまして私は健康になりたい。	「~にも増して」 意味:・・・より。	
\\	彼女は以前にもまして勉強しようと決めたのです。	「~にも増して」 意味:・・・より。	
\\	彼は何物にもまして名誉を重んじる。	「~にも増して」 意味:・・・より。	
\\	彼女がその場で拒絶したことは言うも愚かだ。	「~は愚か」 意味:1)・・・どころか。2)・・・までもない。	
\\	その赤ちゃんは走る事はおろか、歩くことさえできない。	「~は愚か」 意味:1)・・・どころか。2)・・・までもない。	
\\	私の連れは疲れすぎていて走るのはおろか、歩くこともできない、と言った。	「~は愚か」 意味:1)・・・どころか。2)・・・までもない。	
\\	私はバイオリンはおろかギターも弾けない。	「~は愚か」 意味:1)・・・どころか。2)・・・までもない。	
\\	僕はイタリア語を書くのはおろか読めもしない。	「~は愚か」 意味:1)・・・どころか。2)・・・までもない。	
\\	子供の命が助かることを祈りつつ医者に走った。	「~つつ」「~つつある」 意味:1)
\\	母は心配しつつも海外旅行を許してくれた。	「~つつ」「~つつある」 意味:1)
\\	カリキュラムは改善されつつある。	「~つつ」「~つつある」 意味:1)
\\	こうした新しい考え方は、徐々に受け入れられつつある。	「~つつ」「~つつある」 意味:1)
\\	この国では、職場での女性の扱いは次第に平等になりつつある。	「~つつ」「~つつある」 意味:1)
\\	インターネットで音楽をダウンロードする人が増えるにつれ、音楽
\\	はいずれレコードと同じ運命を辿ると考える専門家もいる。	「につれ」、「につれて」 意味:1)~ながら。	
\\	人間の数が増えるにつれ、保護されていない森は破壊される。	「につれ」、「につれて」 意味:1)~ながら。	
\\	年を取るにつれ、人生で得た経験によって賢くなっていくものだ。	「につれ」、「につれて」 意味:1)~ながら。	
\\	お互いを知るようになるにつれて、私たちは親友になった。	「につれ」、「につれて」 意味:1)~ながら。	
\\	パーティーが終わりに近づくにつれて、暗い雰囲気になった。	「につれ」、「につれて」 意味:1)~ながら。	
\\	市民にはこれらの情報をよく知った上で選択をする権利があります。	「過去形+上で」 意味:
\\	補足:「上」
\\	「じょう」。	
\\	子どもは戦い方や憎み方を知った上で、この世に生まれてくるのではない。	「過去形+上で」 意味:
\\	補足:「上」
\\	「じょう」。	
\\	その猫の毛は灰色がかった茶色で濃い縞模様があります。	「~がかった」 意味:~
\\	彼は芝居がかった調子で声を低くした。	「~がかった」 意味:~
\\	彼はその映画で芝居がかった感情表出を見事に行っている。	「~がかった」 意味:~
\\	彼女の髪は背中まである長い、赤みがかったブロンドの縮れた巻き髪です。	「~がかった」 意味:~
\\	新しい言葉を覚えるそばから忘れていっては、いつになっても外国語は上達しない。	「そばから」 意味:
\\	それでは盗んだも同然じゃないか。	「~も同然」 意味:
\\	彼の財産はもうないも同然だ。	「~も同然」 意味:
\\	彼は乞食同然の格好だった。	「~も同然」 意味:
\\	そのピアノは新品同然だった。	「~も同然」 意味:
\\	彼女は知らない人だろうが誰だろうが話し掛けるタイプなんだ。	「~だろうが」 意味:
\\	彼女はピアノだのフルートだのを稽古している。	「~だの」 意味:
\\	忙しいだの病気だのと言ってちっとも約束を守らない。	「~だの」 意味:
\\	彼はなんだのかんだのと文句ばかりを言っている。	「~だの」 意味:
\\	この皿は高熱に耐える。	「~に耐える」「~耐えられる」 意味:1)
\\	この金属は酸には耐えられない。	「~に耐える」「~耐えられる」 意味:1)
\\	彼女ならこの任に堪えられる。	「~に耐える」「~耐えられる」 意味:1)
\\	この花は観賞に堪えない。	「~に耐える」「~耐えられる」 意味:1)
\\	彼女に聞くに堪えない罵詈雑言ばりぞうごんを浴びせた。	「~に耐える」「~耐えられる」 意味:1)
\\	彼の私に対する態度は憤慨に堪えない。	「~にたえない」 意味:
\\	私はテレビなしには過ごせない。	無しに 意味:
\\	私は予告なしに話すように言われた。	無しに 意味:
\end{CJK}
\end{document}