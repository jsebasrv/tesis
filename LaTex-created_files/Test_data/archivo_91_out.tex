\documentclass[8pt]{extreport} 
\usepackage{hyperref}
\usepackage{CJKutf8}
\begin{document}
\begin{CJK}{UTF8}{min}
\\	将来 彼の仕事に将来性がない	しょうらい 
\\	かれのしごとにしょうらいせいがない 
\\	設計 彼がその車を設計した	せっけい 
\\	かれがそのくるまをせっけいした 
\\	集計 私は別のアンケート集計をやってるかな	しゅうけい 
\\	わたしはべつのアンケートしゅうけいをやってるかな 
\\	届ける 家に届けていただけますか	とどける 
\\	いえにとどけていただけますか 
\\	人気 人気があるのですか	にんき 
\\	にんきがあるのですか 
\\	褒める 誰もがいつもトムのことをほめる。	ほめる 
\\	だれもがいつもトムのことをほめる 
\\	結構 いいえ、結構です 結構面白い	けっこう 
\\	いいえ、けっこうです 
\\	けっこうおもしろい 
\\	志望 彼女はデザイナー志望だ	しぼう 
\\	かのじょはデザイナーしぼう 
\\	会う お口に会いますか	あう 
\\	おくちにあいますか 
\\	青 空は青い。	あお 
\\	そらはあおい 
\\	青い 顔が青いよ。	あおい 
\\	かおがあおいよ 
\\	赤 赤ん坊が泣いている	あか 
\\	あかんぼうがないている 
\\	明い 外はまだ明るい	あかるい 
\\	そとはまだあかるい 
\\	秋 秋になりました	あき 
\\	あきになりました 
\\	開く こころを開いて	あく 
\\	こころをひらいて 
\\	上げる 見せて上げる	あげる 
\\	みせてあげる 
\\	遅い 帰るのが遅かったね。	
\\	遅[おそ]い 帰[かえ]るのが 遅[おそ]かったね。
\\	頼む 彼にひとつ頼みごとをしました。	
\\	頼[たの]む 彼[かれ]にひとつ 頼[たの]みごとをしました。
\\	心配 何も心配することはない。	
\\	心配[しんぱい] 何[なに]も 心配[しんぱい]することはない。
\\	構え 返事はしなくても構いませんよ	
\\	構[かま]え 返事[へんじ]はしなくても 構[かま]いませんよ
\\	朝 朝御飯ができています。	
\\	朝[あさ] 朝[あさ] 御飯[ごはん]ができています。
\\	あさって 私はあさってまでこのに滞在します	
\\	滞在 
\\	あさって 私[わたし]はあさってまでこのに 滞在[たいざい]します
\\	足 足が痛い。	
\\	痛 
\\	足[あし] 足[あし]が 痛[いた]い。
\\	明日 明日は明日の風が吹く	
\\	吹 
\\	明日[あした] 明日[あした]は 明日[あした]の 風[かぜ]が 吹[ふ]く
\\	暖かい 部屋は暖かかった。	
\\	暖[あたた]かい 部屋[へや]は 暖[あたた]かかった。
\\	頭 頭が重い。	
\\	頭[あたま] 頭[あたま]が 重[おも]い。
\\	文明 われわれは文明社会に生きている	
\\	社会
\\	文明[ぶんめい] われわれは 文明[ぶんめい] 社会[しゃかい]に 生[い]きている
\\	開化	
\\	開化[かいか]
\\	世間 彼は世間をあっといわせた。	
\\	世-
\\	間-
\\	世間[せけん] 彼[かれ]は 世間[せけん]をあっといわせた。
\\	固い 鉄は固い テニスのラケットを固く握った。	
\\	固[かた]い 鉄[てつ]は 固[かた]い テニスのラケットを 固[かた]く 握[にぎ]った。
\\	前回	
\\	前回[ぜんかい]
\\	新し 私は新しオーフンが欲しい	
\\	新[あたら]し 私[わたし]は 新[あたら]しオーフンが 欲[ほ]しい
\\	厚い これは4
\\	の厚さです 氷はとても厚 く、その上を歩けた	
\\	厚[あつ]い これは 
\\	の 厚[あつ]さです 氷[こおり]はとても 厚く[あつ]く、その 上[うえ]を 歩[ある]けた
\\	姉 私の姉は結構しています。	
\\	姉[あね] 私[わたし]の 姉[あね]は 結構[けっこう]しています。
\\	浴びる シャワーを浴びてもよいでしょうか シャワーを浴びなくちゃ	
\\	浴[あ]びる シャワーを 浴[あ]びてもよいでしょうか シャワーを 浴[あ]びなくちゃ
\\	甘い そのケーキは甘い。 この紅茶は甘すぎる。	
\\	甘[あま]い そのケーキは 甘[あま]い。 この 紅茶[こうちゃ]は 甘[あま]すぎる。
\\	飴 飴がほしい。 のどが痛いんだけど、のど飴持ってない?	
\\	飴[あめ] 飴[あめ]がほしい。 のどが 痛[いた]いんだけど、のど 飴[あめ] 持[も]ってない?
\\	あまり それはあまり価値がない。 あまり希望がない。	
\\	価値 
\\	希望
\\	あまり それはあまり 価値[かち]がない。 あまり 希望[きぼう]がない。
\\	洗う 自分で洗います。 食器を洗いなさい。	
\\	洗[あら]う 自分[じぶん]で 洗[あら]います。 食器[しょっき]を 洗[あら]いなさい。
\\	歩く 歩いてきました。 歩いてすぐですよ。 この辺を歩くなら銃を持って歩け。	
\\	歩[ある]く 歩[ある]いてきました この 辺[あたり]を 歩[ある]くなら 銃[じゅう]を 持[も]って 歩[ある]け。
\\	いくつ いくら	
\\	いくつ いくら
\\	池 池の直径はどのくらいですか? 池には魚がたくさんいる。	
\\	池[いけ] 池[ち]の 直径[ちょっけい]はどのくらいですか? 池[ち]には 魚[さかな]がたくさんいる。
\\	医者 私は医者です。 彼女は医者になるでしょう。 医者は何だって?	
\\	医者[いしゃ] 私[わたし]は 医者[いしゃ]です。 彼女[かのじょ]は 医者[いしゃ]になるでしょう。 医者[いしゃ]は 何[なに]だって?
\\	忙しい 今、忙しの。 忙しくて会社にいます。	
\\	忙[いそが]しい 今[いま]、 忙[いそが]しの。 忙[いそが]しくて 会社[かいしゃ]にいます。
\\	痛い 死ぬというのは痛いのかしら。 右肩が痛いです。	
\\	痛[いた]い 死[し]ぬというのは 痛[いた]いのかしら。 右肩[みぎかた]が 痛[いた]いです。
\\	いつ いつ終わるの? いつ着くの? いつ起こったの?	
\\	いつ いつ 終[お]わるの? いつ 着[つ]くの? いつ 起[お]こったの?
\\	一緒 ご一緒します。 トムと一緒にここにいたいのですが。 我々と一緒にここにいなさい。	
\\	ご 一緒 
\\	一緒[いっしょ] ご 一緒[いっしょ]します。 トムと 一緒[いっしょ]にここにいたいのですが。 我々[われわれ]と 一緒[いっしょ]にここにいなさい。
\\	何時も
\\	いつも いつも笑顔ですね 彼らはいつもそうだ。	
\\	何時[いつ]も
\\	いつも いつも 笑顔[えがお]ですね 彼[かれ]らはいつもそうだ。
\\	妹 私の妹は有名だ。 私の妹はよく泣く。	
\\	妹[いもうと] 私[わたし]の 妹[いもうと]は 有名[ゆうめい]だ。 私[わたし]の 妹[いもうと]はよく 泣[な]く。
\\	嫌 僕が「嫌だ」と言ったらどうする。 私は田舎に住むのは嫌だ	
\\	嫌[いや] 僕[ぼく]が 
\\	嫌[いや]だ」と 言[い]ったらどうする。 私[わたし]は 田舎[いなか]に 住[す]むのは 嫌[いや]だ
\\	入口 ここは入口ではありません。	
\\	入口[いりぐち] ここは 入口[いりぐち]ではありません。
\\	居る 私しか居ないのか? 幽霊は本当にいるのだろうか	
\\	居[い]る 私[わたし]しか 居[い]ないのか? 幽霊[ゆうれい]は 本当[ほんとう]にいるのだろうか
\\	要る 地図が要ります。 説明は要りません。	
\\	要[い]る 地図[ちず]が 要[い]ります。 説明[せつめい]は 要[い]りません。
\\	後ろ 彼の後ろにいる。 彼はドアの後ろに立っていました。	
\\	後[うし]ろ 彼[かれ]の 後[うし]ろにいる。 彼[かれ]はドアの 後[うし]ろに 立[た]っていました。
\\	薄い コーヒーは薄いのが好きです 氷は非常に薄いので君の体重を支えきれないだろう。	
\\	支え 
\\	薄[うす]い コーヒーは 薄[うす]いのが 好[す]きです 氷[こおり]は 非常[ひじょう]に 薄[うす]いので 君[きみ]の 体重[たいじゅう]を 支[ささ]えきれないだろう。
\\	歌う その歌はどんな歌時ですか 彼が歌を歌った。	
\\	歌[うた]う その 歌[うた]はどんな 歌[うた] 時[じ]ですか 彼[かれ]が 歌[うた]を 歌[うた]った。
\\	生まれる ここが父の生まれたところです。 私は京都で生まれた。	
\\	生[う]まれる ここが 父[ちち]の 生[う]まれたところです。 私[わたし]は 京都[きょうと]で 生[う]まれた。
\\	海 その海は青い。 海が静かになった。 海が静かだった。	
\\	海[うみ] その 海[うみ]は 青[あお]い。 海[うみ]が 静[しず]かになった。 海[うみ]が 静[しず]かだった。
\\	売る この本はここでは売られていません 彼の家は売りに出されている。	
\\	売[う]る この 本[ほん]はここでは 売[う]られていません 彼[かれ]の 家[いえ]は 売[う]りに 出[だ]されている。
\\	煩い	
\\	煩[うるさ]い
\\	上着 彼は上着を掛けた 彼はその赤い上着を着た。	
\\	上着[うわぎ] 彼[かれ]は 上着[うわぎ]を 掛[か]けた 彼[かれ]はその 赤[あか]い 上着[うわぎ]を 着[き]た
\\	絵 その絵は素晴らしかったです 彼が絵が好きだからといって絵がうまいということにはならない。	
\\	絵[え] その 絵[え]は 素晴[すば]らしかったです 彼[かれ]が 絵[え]が 好[す]きだからといって 絵[え]がうまいということにはならない。
\\	映画 コメディものの映画です いつその映画を見たのですか。	
\\	映画[えいが] コメディものの 映画[えいが]です いつその 映画[えいが]を 見[み]たのですか。
\\	映画館 映画館にいたんだ。 この市には多くの映画館がある。	
\\	映画[えいが] 館[かん] 映画[えいが] 館[かん]にいたんだ。 この 市[し]には 多[おお]くの 映画[えいが] 館[かん]がある。
\\	駅 駅はどこですか 駅まで供しましょう。	
\\	お供 
\\	駅[えき] 駅[えき]はどこですか 駅[えき]まで 供[きょう]しましょう。
\\	エレベーター エレベーターが故障している。 私たちはエレベーターでしたに降りた。	
\\	故障 
\\	エレベーター エレベーターが 故障[こしょう]している。 私[わたし]たちはエレベーターでしたに 降[お]りた。
\\	鉛筆 これらは私の鉛筆です 鉛筆を落としましたよ。	
\\	鉛筆[えんぴつ] これらは 私[わたし]の 鉛筆[えんぴつ]です 鉛筆[えんぴつ]を 落[お]としましたよ。
\\	多い ここには象が多い インドは人口が多い。	
\\	多[おお]い ここには 象[ぞう]が 多[おお]い インドは 人口[じんこう]が 多[おお]い。
\\	大勢 そこには大勢の人がいた。 コンサートには大勢の聴衆がいた。	
\\	聴衆 
\\	大勢[たいせい] そこには 大勢[たいせい]の 人[ひと]がいた。 コンサートには 大勢[たいせい]の 聴衆[ちょうしゅう]がいた。
\\	お菓子 いろいろなお菓子があった。 お母さんだけがこのお菓子をつくれます。	
\\	お 菓子[かし] いろいろなお 菓子[かし]があった。 お 母[かあ]さんだけがこのお 菓子[かし]をつくれます。
\\	起きる 何も起きなかった。 問題でも起きたのか。 何が起きたのか知しりません。	
\\	起[お]きる 何[なに]も 起[お]きなかった。 問題[もんだい]でも 起[お]きたのか。 何[なに]が 起[お]きたのか 知[し]りません。
\\	置く ここに置いていいですか 彼は銃を下に置いた。	
\\	置[お]く ここに 置[お]いていいですか 彼[かれ]は 銃[じゅう]を 下[した]に 置[お]いた。
\\	奥さん 奥さんがカンカンだぞ。 奥さんによろしく伝えて下さい。	
\\	奥[おく]さん 奥[おく]さんがカンカンだぞ。 奥[おく]さんによろしく 伝[つた]えて 下[くだ]さい。
\\	お酒 彼は酒によっています。 私はさけもたばこもやらない。	
\\	お 酒[さけ] 彼[かれ]は 酒[さけ]によっています。 私[わたし]はさけもたばこもやらない。
\\	お皿 マイケルは皿を割ってしまった。 その皿は、プラスチックで出来ている。	
\\	お 皿[さら] マイケルは 皿[さら]を 割[わ]ってしまった。 その 皿[さら]は、プラスチックで 出来[でき]ている。
\\	伯父/叔父 この家は叔父のものだ。 叔父から砂時計をもらった。	
\\	伯父[おじ]/ 叔父[おじ] この 家[いえ]は 叔父[おじ]のものだ。 叔父[おじ]から 砂時計[すなどけい]をもらった。
\\	教える 教えてよ! 道を教えます。 子供にものを教えるのは、大人に教えるよりは容易だ。	
\\	容易 
\\	教[おし]える 教[おし]えてよ! 道[どう]を 教[おし]えます。 子供[こども]にものを 教[おし]えるのは、 大人[おとな]に 教[おし]えるよりは 容易[ようい]だ。
\\	押す 太鼓判を押す。 彼女はスイッチを押した。 このボタンを決して押してはいけない。	
\\	押[お]す 太鼓判[たいこばん]を 押[お]す。 彼女[かのじょ]はスイッチを 押[お]した。 このボタンを 決[けっ]して 押[お]してはいけない。
\\	お茶 お茶でも飲みませんか? お茶をいかがですか。 お茶が切れています。	
\\	お 茶[ちゃ] お 茶[ちゃ]でも 飲[の]みませんか? お 茶[ちゃ]をいかがですか。 お 茶[ちゃ]が 切[き]れています。
\\	お手洗い 婦人用手洗いはどこですか? 「おしっこしたい」 「ジョニー、そうじゃないだろ。{すみません。お手洗いに行きたいんですが}と言いなさい」	
\\	お 手洗[てあら]い 婦人[ふじん] 用[よう] 手洗[てあら]いはどこですか? 「おしっこしたい」 「ジョニー、そうじゃないだろ。{すみません。お 手洗[てあら]いに 行[い]きたいんですが}と 言[い]いなさい」
\\	弟 弟のことなんかほっときなさい。 私たちは兄弟です。 弟が描きました。	
\\	ほっとく 
\\	描く 
\\	弟[おとうと] 弟[おとうと]のことなんかほっときなさい。 私[わたし]たちは 兄弟[きょうだい]です。 弟[おとうと]が 描[えが]きました。
\\	一昨日 その事故は一昨日起きた。 一昨日私の友人達が遊びに来た。	
\\	一昨日[おととい] その 事故[じこ]は 一昨日[おととい] 起[お]きた。 一昨日[おととい] 私[わたし]の 友人[ゆうじん] 達[たち]が 遊[あそ]びに 来[き]た。
\\	一昨年	
\\	一昨年[おととし]
\\	大人 彼は大人になりつつある その子は大人のような口を利く。	
\\	大人[おとな] 彼[かれ]は 大人[おとな]になりつつある その 子[こ]は 大人[おとな]のような 口[くち]を 利[き]く。
\\	お姉さん あなたのお姉さんは今何をしているの? 彼女とお姉さんを間違えてしまった。	
\\	お 姉[ねえ]さん あなたのお 姉[ねえ]さんは 今[いま] 何[なに]をしているの? 彼女[かのじょ]とお 姉[ねえ]さんを 間違[まちが]えてしまった。
\\	伯母さん/叔母さん 叔母さんが亡くなる前に叔母さんと 和解した。	
\\	和解 
\\	亡くなる 
\\	伯母[おば]さん/ 叔母[おば]さん 叔母[おば]さんが 亡[な]くなる 前[まえ]に 叔母[おば]さんと 和解[わかい]した。
\\	お風呂 お風呂沸いたわよ。 風呂落ち。	
\\	沸い 
\\	お 風呂[ふろ] お 風呂[ふろ] 沸[わ]いたわよ。 風呂[ふろ] 落[お]ち。
\\	お弁当 私たちは学校で早めに弁当を食べた。 私は毎日昼に弁当を食べます。	
\\	お 弁当[べんとう] 私[わたし]たちは 学校[がっこう]で 早[はや]めに 弁当[べんとう]を 食[た]べた。 私[わたし]は 毎日[まいにち] 昼[ひる]に 弁当[べんとう]を 食[た]べます。
\\	覚える すみません。それしか覚えていない。 それらの規則を覚えておきなさい。	
\\	覚[おぼ]える すみません。それしか 覚[おぼ]えていない。 それらの 規則[きそく]を 覚[おぼ]えておきなさい。
\\	お巡りさん お巡りさんに聞け!。	
\\	お 巡[まわ]りさん お 巡[まわ]りさんに 聞[き]け!。
\\	重い 二つのうちどっちが重いの 重いけど、何とか運べます。	
\\	重[おも]い 二[ふた]つのうちどっちが 重[おも]いの 重[おも]いけど、 何[なん]とか 運[はこ]べます。
\\	泳ぐ 泳ぐことはやさしい。 私たちはその湖で泳いだ。	
\\	泳[およ]ぐ 泳[およ]ぐことはやさしい。 私[わたし]たちはその 湖[みずうみ]で 泳[およ]いだ。
\\	降りる ここで降ります。 そこから降りる! タクシーをおりましょう。	
\\	降[お]りる ここで 降[お]ります。 そこから 降[お]りる! タクシーをおりましょう。
\\	終わる 終わったのですか? その問題はこれで終わり。	
\\	終[お]わる 終[お]わったのですか? その 問題[もんだい]はこれで 終[お]わり。
\\	音楽 音楽を聴きます。 音楽家になりたいです。	
\\	音楽[おんがく] 音楽[おんがく]を 聴[き]きます。 音楽家[おんがくか]になりたいです。
\\	外国 外国へいきたい。 外国人って面白いなあ。 彼は外国に長年住んでいた。	
\\	外国[がいこく] 外国[がいこく]へいきたい。 外国[がいこく] 人[じん]って 面白[おもしろ]いなあ。 彼[かれ]は 外国[がいこく]に 長年[ながねん] 住[す]んでいた。
\\	会社 これは、会社払いだ。 彼は会社の仲間です。 あの会社は赤字だ。	
\\	払い 
\\	赤字 
\\	会社[かいしゃ] これは、 会社[かいしゃ] 払[はら]いだ。 彼[かれ]は 会社[かいしゃ]の 仲間[なかま]です。 あの 会社[かいしゃ]は 赤字[あかじ]だ。
\\	階段 彼は階段を転げ落ちた。 彼女は階段を降りてきていた	
\\	転げ 
\\	階段[かいだん] 彼[かれ]は 階段[かいだん]を 転げ落[ころげお]ちた。 彼女[かのじょ]は 階段[かいだん]を 降[お]りてきていた
\\	買い物 買い物に行かなければならない。 私は買い物に来たものです。	
\\	買い物[かいもの] 買い物[かいもの]に 行[い]かなければならない。 私[わたし]は 買い物[かいもの]に 来[き]たものです。
\\	返す ボールを投げ返してください。 彼女は彼にキスを返した。	
\\	返[かえ]す ボールを 投[な]げ 返[かえ]してください。 彼女[かのじょ]は 彼[かれ]にキスを 返[かえ]した。
\\	かかる 何分かかりますか。 どれくらいかかるのかしら。	
\\	かかる 何[なん] 分[ふん]かかりますか。 どれくらいかかるのかしら。
\\	かぎ このかぎは錠の中に入らない。 「カギは錠前に差し込んである」と、彼は付け加えた。	
\\	付け加え 
\\	かぎ このかぎは 錠[じょう]の 中[なか]に 入[はい]らない。 「カギは 錠前[じょうまえ]に 差し込[さしこ]んである」と、 彼[かれ]は 付け加[つけくわ]えた。
\\	かける 電話をかけたい。 まだ後でかけます。	
\\	かける 電話[でんわ]をかけたい。 まだ 後[あと]でかけます。
\\	傘 私は傘をなくした。 私はタクシーに傘を置き忘れた。	
\\	傘[かさ] 私[わたし]は 傘[かさ]をなくした。 私[わたし]はタクシーに 傘[かさ]を 置き忘[おきわす]れた。
\\	貸す 金を貸す。 君に本を貸さなかったかね。たしかに貸したと思うのだが。	
\\	貸[か]す 金[かね]を 貸[か]す。 君[きみ]に 本[ほん]を 貸[か]さなかったかね。たしかに 貸[か]したと 思[おも]うのだが。
\\	風邪 風邪ですね 風邪ひいた。 ただの風邪でしょう。	
\\	風邪[かぜ] 風邪[かぜ]ですね 風邪[かぜ]ひいた。 ただの 風邪[かぜ]でしょう。
\\	家族 ご家族はいますか。 家族みたいなものだしね。 家族全員がよく似ているんだ。	
\\	似て 
\\	家族[かぞく] ご 家族[かぞく]はいますか。 家族[かぞく]みたいなものだしね。 家族[かぞく] 全員[ぜんいん]がよく 似[に]ているんだ。
\\	方 眠った方がいいよ こちらの方がきれいだ。 これはあなた方の勝利だ	
\\	方[かた] 眠[ねむ]った 方[ほう]がいいよ こちらの 方[ほう]がきれいだ。 これはあなた 方[かた]の 勝利[しょうり]だ
\\	カップ このカップは金でできている 彼女はカップにミルクを注いだ	
\\	注いだ 
\\	カップ このカップは 金[かね]でできている 彼女[かのじょ]はカップにミルクを 注[そそ]いだ
\\	家庭 家庭は円満だ。 食事は家庭で食べますか、外食ですか。	
\\	家庭[かてい] 家庭[かてい]は 円満[えんまん]だ。 食事[しょくじ]は 家庭[かてい]で 食[た]べますか、 外食[がいしょく]ですか。
\\	角 すぐその角を曲がったところだ。 彼女は角で左に曲がった。	
\\	角[かく] すぐその 角[かく]を 曲[ま]がったところだ。 彼女[かのじょ]は 角[かく]で 左[ひだり]に 曲[ま]がった。
\\	かばん きみのかばんはここにあるよ どちらのかばんがあなたの物ですか。	
\\	かばん きみのかばんはここにあるよ どちらのかばんがあなたの 物[もの]ですか。
\\	花瓶 彼は花瓶を落とした。 この花瓶は鉄で出来ている。	
\\	花瓶[かびん] 彼[かれ]は 花瓶[かびん]を 落[お]とした。 この 花瓶[かびん]は 鉄[てつ]で 出来[でき]ている。
\\	紙 紙がほしい。 この紙はざらざらしている	
\\	紙[かみ] 紙[し]がほしい。 この 紙[かみ]はざらざらしている
\\	火曜日 火曜日はお暇ですか。 できたら火曜日に来なさい。	
\\	火曜日[かようび] 火曜日[かようび]はお 暇[ひま]ですか。 できたら 火曜日[かようび]に 来[き]なさい。
\\	辛い 誰でも辛いこともあったさ。 トムは辛いカレーが好きだ。	
\\	辛[つら]い 誰[だれ]でも 辛[つら]いこともあったさ。 トムは 辛[つら]いカレーが 好[す]きだ。
\\	体 体中が痛む。 彼はいい体をしている。	
\\	体[からだ] 体[たい] 中[ちゅう]が 痛[いた]む。 彼[かれ]はいい 体[からだ]をしている。
\\	借りる ラケットは借りられますか。 電話をお借りできますか。	
\\	借[か]りる ラケットは 借[か]りられますか。 電話[でんわ]をお 借[か]りできますか。
\\	軽い 症状は軽いですよ。 このいすは軽い。 彼は私の不平を借んじた	
\\	軽[かる]い 症状[しょうじょう]は 軽[かる]いですよ。 このいすは 軽[かる]い。 彼[かれ]は 私[わたし]の 不平[ふへい]を 借[か]んじた
\\	川/河 宇宙にはたくさんの銀河がある これが河? その川は長いです。 彼は川に落ちた。	河 
\\	川 
\\	川[かわ]/ 河[かわ] 宇宙[うちゅう]にはたくさんの 銀河[ぎんが]がある これが 河[かわ]? その 川[かわ]は 長[なが]いです。 彼[かれ]は 川[かわ]に 落[お]ちた。
\\	黄色 バナナは黄色です。 庭の花はすべて黄色だ。 その木の葉は皆黄色になった。	
\\	黄色[きいろ] バナナは 黄色[きいろ]です。 庭[にわ]の 花[はな]はすべて 黄色[きいろ]だ。 その 木の葉[このは]は 皆[みな] 黄色[きいろ]になった。
\\	消える 火が消えた。 明かりは自然に消えた。 彼は跡形もなく消えたんだ。	
\\	跡形 
\\	消[き]える 火[ひ]が 消[き]えた。 明[あ]かりは 自然[しぜん]に 消[き]えた。 彼[かれ]は 跡形[あとかた]もなく 消[き]えたんだ。
\\	聞く アレックスに聞けよ 考えていることを聞くな。やることを聞け。	
\\	聞[き]く アレックスに 聞[き]けよ 考[かんが]えていることを 聞[き]くな。やることを 聞[き]け。
\\	北 風は北から吹いている。 北海道は本州の北にある。	
\\	北[きた] 風[かぜ]は 北[きた]から 吹[ふ]いている。 北海道[ほっかいどう]は 本州[ほんしゅう]の 北[きた]にある。
\\	汚い 私は汚い犬が庭に入ったのを見ました。 あのシャツは汚い。学校に行く前に洗う必要があります。	
\\	汚[きたな]い 私[わたし]は 汚[きたな]い 犬[いぬ]が 庭[にわ]に 入[はい]ったのを 見[み]ました。 あのシャツは 汚[きたな]い。 学校[がっこう]に 行[い]く 前[まえ]に 洗[あら]う 必要[ひつよう]があります。
\\	喫茶店 あそこに喫茶店があります。 ここが、妻と初めて会った喫茶店です。	
\\	喫茶店[きっさてん] あそこに 喫茶店[きっさてん]があります。 ここが、 妻[つま]と 初[はじ]めて 会[あ]った 喫茶店[きっさてん]です。
\\	切手 ここに2枚の切手があります。 ジャックは切手を集めている。	
\\	集めて
\\	切手[きって] ここに 
\\	枚[まい]の 切手[きって]があります。 ジャックは 切手[きって]を 集[あつ]めている。
\\	切符 切符持ってますか? 私は切符をなくしたのです。	
\\	切符[きっぷ] 切符[きっぷ] 持[も]ってますか? 私[わたし]は 切符[きっぷ]をなくしたのです。
\\	昨日 昨日は雪だった。 昨日は木曜日だった。 昨日は曇っていた。	
\\	昨日[きのう] 昨日[きのう]は 雪[ゆき]だった。 昨日[きのう]は 木曜日[もくようび]だった。 昨日[きのう]は 曇[くも]っていた。
\\	牛肉 近頃牛肉は高い。 この牛肉を私に売ってくれた肉屋さんはいつも愛想がよい。	
\\	愛想 
\\	よい 
\\	牛肉[ぎゅうにく] 近頃[ちかごろ] 牛肉[ぎゅうにく]は 高[たか]い。 この 牛肉[ぎゅうにく]を 私[わたし]に 売[う]ってくれた 肉屋[にくや]さんはいつも 愛想[あいそ]がよい。
\\	牛乳 牛乳が腐った。 トムは牛乳をこぼした。	
\\	牛乳[ぎゅうにゅう] 牛乳[ぎゅうにゅう]が 腐[くさ]った。 トムは 牛乳[ぎゅうにゅう]をこぼした。
\\	教室 私は教室にいる。 教室でおしゃべりをしてはいかん。	
\\	教室[きょうしつ] 私[わたし]は 教室[きょうしつ]にいる。 教室[きょうしつ]でおしゃべりをしてはいかん。
\\	兄弟 兄弟はいるか。 貴方の兄弟はどこですか。	
\\	兄弟[きょうだい] 兄弟[きょうだい]はいるか。 貴方[あなた]の 兄弟[きょうだい]はどこですか。
\\	去年 去年は不作だった。 彼は去年がんで亡くなった。	
\\	がん
\\	去年[きょねん] 去年[きょねん]は 不作[ふさく]だった。 彼[かれ]は 去年[きょねん]がんで 亡[な]くなった。
\\	嫌い そんなに嫌われたいの? 彼女は彼を嫌っていた。	
\\	嫌[きら]い そんなに 嫌[きら]われたいの? 彼女[かのじょ]は 彼[かれ]を 嫌[きら]っていた。
\\	切る にんじんを切らせて。 私は息を切らせて走った。	
\\	切[き]る にんじんを 切[き]らせて。 私[わたし]は 息[いき]を 切[き]らせて 走[はし]った。
\\	着る 彼女は姉さんが着るのと同じ種類の服を着る。 このセーターを着なさい。	
\\	着[き]る 彼女[かのじょ]は 姉[ねえ]さんが 着[き]るのと 同[おな]じ 種類[しゅるい]の 服[ふく]を 着[き]る。 このセーターを 着[き]なさい。
\\	きれい きれいな夕日だね。 台所をきれいにしなさい。	
\\	きれい きれいな 夕日[ゆうひ]だね。 台所[だいどころ]をきれいにしなさい。
\\	キロ
\\	キログラム 米はキロ単位で売られる。 学校は2キロ先にある。	
\\	単位
\\	キロ
\\	キログラム 米[まい]はキロ 単位[たんい]で 売[う]られる。 学校[がっこう]は 2キロ 先[さき]にある。
\\	銀行 その銀行はどこにありますか。 角に銀行があります。 その銀行に取引きがある。	
\\	銀行[ぎんこう] その 銀行[ぎんこう]はどこにありますか。 角[かく]に 銀行[ぎんこう]があります。 その 銀行[ぎんこう]に 取引[とりひ]きがある。
\\	金曜日 やっと金曜日だ 金曜日に午後は暇ですか。 金曜日までに必ず電話してね。	
\\	必ず 
\\	金曜日[きんようび] やっと 金曜日[きんようび]だ 金曜日[きんようび]に 午後[ごご]は 暇[ひま]ですか。 金曜日[きんようび]までに 必[かなら]ず 電話[でんわ]してね。
\\	薬 その薬は効いた。 それは毒にも薬にもならない。 薬アレルギーがあります。	
\\	薬[くすり] その 薬[くすり]は 効[き]いた。 それは 毒[どく]にも 薬[くすり]にもならない。 薬[くすり]アレルギーがあります。
\\	果物 果物が好きです。 この果物は不味い。 果物を何かいかがですか	
\\	果物[くだもの] 果物[くだもの]が 好[す]きです。 この 果物[くだもの]は 不味[まず]い。 果物[くだもの]を 何[なに]かいかがですか
\\	靴 靴はどこにありますか。 ここで靴を根がなければなりませんか その靴は時代遅れだ。	
\\	靴[くつ] 靴[くつ]はどこにありますか。 ここで 靴[くつ]を 根[ね]がなければなりませんか その 靴[くつ]は 時代遅[じだいおく]れだ。
\\	靴下 靴下を脱いでください。 靴下はもう乾いたのですか。	
\\	靴下[くつした] 靴下[くつした]を 脱[ぬ]いでください。 靴下[くつした]はもう 乾[かわ]いたのですか。
\\	曇り 曇っています。 明日は曇りになるでしょう。 くもってきた。やがて雨になるかもしれない。	
\\	かもしれない 
\\	曇[くも]り 曇[くも]っています。 明日[あした]は 曇[くも]りになるでしょう。 くもってきた。やがて 雨[あめ]になるかもしれない。
\\	曇る 急に空が曇ってきた。 曇り時々雨でした。 曇りだったので、私は傘を持って行った。	
\\	時々
\\	曇[くも]る 急[きゅう]に 空[そら]が 曇[くも]ってきた。 曇[くも]り 時々[ときどき] 雨[あめ]でした。 曇[くも]りだったので、 私[わたし]は 傘[かさ]を 持[も]って 行[い]った。
\\	暗い ここは暗い。 暗くなってきた。 暗すぎる。	
\\	暗[くら]い ここは 暗[くら]い。 暗[くら]くなってきた。 暗[くら]すぎる。
\\	来る あ、来た来た。 「彼は来ますか」「来ないと思います」 誰も来なかった。	
\\	来[き]る あ、 来[き]た 来[き]た。 
\\	彼[かれ]は 来[き]ますか」
\\	来[こ]ないと 思[おも]います」 誰[だれ]も 来[こ]なかった。
\\	黒 猫は黒いんだ。 黒くないのですか。 黒のものはあなたによく似合います。	
\\	似合います 
\\	黒[くろ] 猫[ねこ]は 黒[くろ]いんだ。 黒[くろ]くないのですか。 黒[くろ]のものはあなたによく 似合[にあ]います。
\\	警官 私は警官だ。 彼女は婦人警官になった。 その警官は泥棒を追っていた。	
\\	婦人警官
\\	泥棒
\\	警官[けいかん] 私[わたし]は 警官[けいかん]だ。 彼女[かのじょ]は 婦人[ふじん] 警官[けいかん]になった。 その 警官[けいかん]は 泥棒[どろぼう]を 追[お]っていた。
\\	今朝 今朝、何をしましたか。 今朝は暖かいですね。	
\\	今朝[けさ] 今朝[けさ]、 何[なに]をしましたか。 今朝[けさ]は 暖[あたた]かいですね。
\\	消す それ消して。 消しておいて。 彼女は明かりを消した。	
\\	消[け]す それ 消[け]して。 消[け]しておいて。 彼女[かのじょ]は 明[あ]かりを 消[け]した。
\\	結構 いえ、結構です。 それだけで結構です。 結構面白い。	
\\	結構[けっこう] いえ、 結構[けっこう]です。 それだけで 結構[けっこう]です。 結構[けっこう] 面白[おもしろ]い。
\\	結婚 結婚しましょう。 私は結婚しています。 彼女と結婚したい。	
\\	結婚[けっこん] 結婚[けっこん]しましょう。 私[わたし]は 結婚[けっこん]しています。 彼女[かのじょ]と 結婚[けっこん]したい。
\\	月曜日 月曜日までにここに来ています。 私は月曜日には働からない。	
\\	月曜日[げつようび] 月曜日[げつようび]までにここに 来[き]ています。 私[わたし]は 月曜日[げつようび]には 働[はたら]からない。
\\	玄関 だれか玄関にいる 玄関のドアは開いていた。 あなたに会いたがっている 人が玄関にいます。	
\\	玄関[げんかん] だれか 玄関[げんかん]にいる 玄関[げんかん]のドアは 開[ひら]いていた。 あなたに 会[あ]いたがっている 人[ひと]が 玄関[げんかん]にいます。
\\	公園 公園に行きなさい。 私たちは公園で走っていた。 私たちは公園のベンチに座った。	
\\	公園[こうえん] 公園[こうえん]に 行[い]きなさい。 私[わたし]たちは 公園[こうえん]で 走[はし]っていた。 私[わたし]たちは 公園[こうえん]のベンチに 座[すわ]った。
\\	交差点 交差点で事故が発生した。 あの交差点で何が起こったのか。 次の交差点を右に曲がって。	
\\	交差点[こうさてん] 交差点[こうさてん]で 事故[じこ]が 発生[はっせい]した。 あの 交差点[こうさてん]で 何[なに]が 起[お]こったのか。 次[つぎ]の 交差点[こうさてん]を 右[みぎ]に 曲[ま]がって。
\\	交番 向こうの交番で聞いてください。 交番はどこにあるかわかりますか。	
\\	交番[こうばん] 向[む]こうの 交番[こうばん]で 聞[き]いてください。 交番[こうばん]はどこにあるかわかりますか。
\\	声 民の声は神の声。 困ったことに、声がほとんどでないのです。 私は声を出して笑った。	
\\	民
\\	困
\\	困ったなあ。
\\	声[こえ] 民[みん]の 声[こえ]は 神[かみ]の 声[こえ]。 困[こま]ったことに、 声[こえ]がほとんどでないのです。 私[わたし]は 声[こえ]を 出[だ]して 笑[わら]った。
\\	コート このコートは暖かい。 コートをお預かりしましょうか。 このコートにはポケットがない。	
\\	預 
\\	コート このコートは 暖[あたた]かい。 コートをお 預[あず]かりしましょうか。 このコートにはポケットがない。
\\	午後 午後、雨になるでしょうか。 ところで午後お暇ですか。	
\\	午後[ごご] 午後[ごご]、 雨[あめ]になるでしょうか。 ところで 午後[ごご]お 暇[ひま]ですか。
\\	九日	
\\	九日[ここのか]
\\	九つ	
\\	九[ここの]つ
\\	午前 彼女は午前7時に起きた。 彼は午前中に本を読んでいた。 母は午前中病院に行きます。	
\\	午前[ごぜん] 彼女[かのじょ]は 午前[ごぜん] 
\\	時[じ]に 起[お]きた。 彼[かれ]は 午前[ごぜん] 中[ちゅう]に 本[ほん]を 読[よ]んでいた。 母[はは]は 午前[ごぜん] 中[ちゅう] 病院[びょういん]に 行[い]きます。
\\	答える あなたの答えは答えになっていない。 きみは答えがわかるって? 君の答えは会っています。	
\\	答[こた]える あなたの 答[こた]えは 答[こた]えになっていない。 きみは 答[こた]えがわかるって? 君[くん]の 答[こた]えは 会[あ]っています。
\\	こちら こちらは初めてですか。 こちらはサービスでございます。 こちらへどうぞ。	
\\	こちら こちらは 初[はじ]めてですか。 こちらはサービスでございます。 こちらへどうぞ。
\\	こっち こっちはいっぱいです。 こっちは雪が降ってる。 こっちの身にもなってよ	
\\	こっち こっちはいっぱいです。 こっちは 雪[ゆき]が 降[ふ]ってる。 こっちの 身[み]にもなってよ
\\	コップ 君のコップ?それとも妹のこっぷ? そのコップを落とすな。	
\\	コップ 君[きみ]のコップ?それとも 妹[いもうと]のこっぷ? そのコップを 落[お]とすな。
\\	今年 今年、夏は寒いですよ 今年は天候不順だった。 今年は雨が多い。	
\\	今年[ことし] 今年[ことし]、 夏[なつ]は 寒[さむ]いですよ 今年[ことし]は 天候[てんこう] 不順[ふじゅん]だった。 今年[ことし]は 雨[あめ]が 多[おお]い。
\\	言葉 この言葉は時代遅れになっている。 この言葉は初耳です。 言葉でそれを表せない。	
\\	言葉[ことば] この 言葉[ことば]は 時代遅[じだいおく]れになっている。 この 言葉[ことば]は 初耳[はつみみ]です。 言葉[ことば]でそれを 表[あらわ]せない。
\\	子供 子供にでもそれはできる。 いくつになっても子供は子供なのである。 彼には子供がいない。	
\\	子供[こども] 子供[こども]にでもそれはできる。 いくつになっても 子供[こども]は 子供[こども]なのである。 彼[かれ]には 子供[こども]がいない。
\\	御飯 朝御飯ができています。 おいしい晩御飯をありがとうございました。 晩御飯を何にしたらいいかわからない。	
\\	御飯[ごはん] 朝[あさ] 御飯[ごはん]ができています。 おいしい 晩[ばん] 御飯[ごはん]をありがとうございました。 晩[ばん] 御飯[ごはん]を 何[なに]にしたらいいかわからない。
\\	コピーする コピーとっとかなきゃ。 違法コピーはいけません 著者の許可なしに本をコピーすることは違法です。	
\\	著者 
\\	許可な 
\\	コピーする コピーとっとかなきゃ。 違法[いほう]コピーはいけません 著者[ちょしゃ]の 許可[きょか]なしに 本[ほん]をコピーすることは 違法[いほう]です。
\\	困る 彼は困っている。 私は金の使い道に困っている。 彼は私が困っていても知らんぷりしている。	
\\	困[こま]る 彼[かれ]は 困[こま]っている。 私[わたし]は 金[きん]の 使い道[つかいみち]に 困[こま]っている。 彼[かれ]は 私[わたし]が 困[こま]っていても 知[し]らんぷりしている。
\\	今月 今月の家賃を払うのを忘れた。 今月は暑かったが、今日は例外だ。	
\\	家賃 
\\	払う 
\\	例外
\\	今月[こんげつ] 今月[こんげつ]の 家賃[やちん]を 払[はら]うのを 忘[わす]れた。 今月[こんげつ]は 暑[あつ]かったが、 今日[きょう]は 例外[れいがい]だ。
\\	今週 今週末は暇? 今週はずっと忙しい。 先生に今週会えますか。	
\\	今週[こんしゅう] 今週[こんしゅう] 末[まつ]は 暇[ひま]? 今週[こんしゅう]はずっと 忙[いそが]しい。 先生[せんせい]に 今週[こんしゅう] 会[あ]えますか。
\\	こんな こんな所出よう! こんなこと初めてだ。 こんな幸せであったことがない。	
\\	こんな こんな 所出[しょしゅつ]よう! こんなこと 初[はじ]めてだ。 こんな 幸[しあわ]せであったことがない。
\\	今晩 今晩は予定がある? 今晩暇かな? 今晩集まりましょう。	
\\	集ま 
\\	今晩[こんばん] 今晩[こんばん]は 予定[よてい]がある? 今晩[こんばん] 暇[ひま]かな? 今晩[こんばん] 集[あつ]まりましょう。
\\	さあ さあね。 さあ、もう一度。 さあさあ、話すのはやめて音楽を聴きなさい。	
\\	さあ さあね。 さあ、もう 一度[いちど]。 さあさあ、 話[はな]すのはやめて 音楽[おんがく]を 聴[き]きなさい。
\\	財布 ほら財布だ。 財布をなくしたの。 私は財布を置き忘れた。	
\\	財布[さいふ] ほら 財布[さいふ]だ。 財布[さいふ]をなくしたの。 私[わたし]は 財布[さいふ]を 置き忘[おきわす]れた。
\\	先 父はついさきほど出かけた。 卵が先か、 二ワトリが先か? 男性より女性が先。	
\\	先[せん] 父[ちち]はついさきほど 出[で]かけた。 卵[たまご]が 先[さき]か、 二ワトリが 先[さき]か? 男性[だんせい]より 女性[じょせい]が 先[さき]。
\\	魚 肉になさいますか、魚になさいますか。 魚をお願いします。 我々は魚を生で食べる。	
\\	我 
\\	生
\\	魚[さかな] 肉[にく]になさいますか、 魚[さかな]になさいますか。 魚[さかな]をお 願[ねが]いします。 我々[われわれ]は 魚[さかな]を 生[なま]で 食[た]べる。
\\	咲く 花は咲く。 花が咲き始めている。 今年はバラが早く咲いている。	
\\	咲[さ]く 花[はな]は 咲[さ]く。 花[はな]が 咲[さ]き 始[はじ]めている。 今年[ことし]はバラが 早[はや]く 咲[さ]いている。
\\	作文 彼は報告書を作文した。 彼女は作文が優秀だ。	
\\	優秀 
\\	作文[さくぶん] 彼[かれ]は 報告[ほうこく] 書[しょ]を 作文[さくぶん]した。 彼女[かのじょ]は 作文[さくぶん]が 優秀[ゆうしゅう]だ。
\\	差す 他人を指差すのは失礼なことです。	
\\	差[さ]す 他人[たにん]を 指[ゆび] 差[さ]すのは 失礼[しつれい]なことです。
\\	雑誌 この雑誌を読みたいですか。 日本の雑誌はありますか。 その雑誌は売り切れだった。	
\\	雑誌[ざっし] この 雑誌[ざっし]を 読[よ]みたいですか。 日本[にっぽん]の 雑誌[ざっし]はありますか。 その 雑誌[ざっし]は 売り切[うりき]れだった。
\\	砂糖 これは砂糖じゃないです。 砂糖がないよ。 砂糖は残っていますか。	
\\	残って 
\\	砂糖[さとう] これは 砂糖[さとう]じゃないです。 砂糖[さとう]がないよ。 砂糖[さとう]は 残[のこ]っていますか。
\\	寒い 家は寒いです。 なんて寒いことでしょう。	
\\	寒[さむ]い 家[いえ]は 寒[さむ]いです。 なんて 寒[さむ]いことでしょう。
\\	さ来年 来年君は何歳になりますか。 この本は来年印刷されるでそう。	
\\	印刷 
\\	さ 来年[らいねん] 来年[らいねん] 君[きみ]は 何[なん] 歳[さい]になりますか。 この 本[ほん]は 来年[らいねん] 印刷[いんさつ]されるでそう。
\\	散歩する 散歩をしましょう。 散歩に出てました。 雨のために散歩ができなかった。	
\\	散歩[さんぽ]する 散歩[さんぽ]をしましょう。 散歩[さんぽ]に 出[で]てました。 雨[あめ]のために 散歩[さんぽ]ができなかった。
\\	塩 もう塩がない。 お塩とって。 塩を入れましたか	
\\	塩[しお] もう 塩[しお]がない。 お 塩[しお]とって。 塩[しお]を 入[い]れましたか
\\	しかし	
\\	しかし
\\	時間 お時間がありますか 残り時間はほとんどない。 出かける時間になった。	
\\	残り 
\\	時間[じかん] お 時間[じかん]がありますか 残[のこ]り 時間[じかん]はほとんどない。 出[で]かける 時間[じかん]になった。
\\	仕事 それが仕事です。 あなたは仕事をしなければならない。 仕事を探している。	
\\	探し 
\\	仕事[しごと] それが 仕事[しごと]です。 あなたは 仕事[しごと]をしなければならない。 仕事[しごと]を 探[さが]している。
\\	辞書 それは誰の辞書ですか。 この辞書は私のものではありません。 辞書にもいろいろある。	
\\	辞書[じしょ] それは 誰[だれ]の 辞書[じしょ]ですか。 この 辞書[じしょ]は 私[わたし]のものではありません。 辞書[じしょ]にもいろいろある。
\\	静か 静かにしていましょう。 お静かに。。。。。 あたりは静かだ。	
\\	静[しず]か 静[しず]かにしていましょう。 お 静[しず]かに。。。。。 あたりは 静[しず]かだ。
\\	下 ベッドの下に何かあるよ。 太陽は地平線下に没した。 我々は彼の指揮下にいる。	
\\	太陽 
\\	地平線 
\\	指揮 
\\	下[した] ベッドの 下[した]に 何[なに]かあるよ。 太陽[たいよう]は 地平線[ちへいせん] 下[か]に 没[ぼっ]した。 我々[われわれ]は 彼[かれ]の 指揮[しき] 下[か]にいる。
\\	質問 質問をどうぞ。 バカな質問があるんだ。 この質問に答えられましか。	
\\	質問[しつもん] 質問[しつもん]をどうぞ。 バカな 質問[しつもん]があるんだ。 この 質問[しつもん]に 答[こた]えられましか。
\\	自転車 それはきみの自転車? 自転車を持っていないのですか。	
\\	自転車[じてんしゃ] それはきみの 自転車[じてんしゃ]? 自転車[じてんしゃ]を 持[も]っていないのですか。
\\	自動車 私は自動車を持っている。 彼は自動車を洗っています。 自動車は速く走っている。	
\\	自動車[じどうしゃ] 私[わたし]は 自動車[じどうしゃ]を 持[も]っている。 彼[かれ]は 自動車[じどうしゃ]を 洗[あら]っています。 自動車[じどうしゃ]は 速[はや]く 走[はし]っている。
\\	死ぬ 死なないで。 私は死ぬのでそうか。 食べないと死にますよ。	
\\	死[し]ぬ 死[し]なないで。 私[わたし]は 死[し]ぬのでそうか。 食[た]べないと 死[し]にますよ。
\\	市引き	
\\	市[し] 引[び]き
\\	自分 自分で洗います。 自分で作りました。 自分としては満足している。	
\\	自分[じぶん] 自分[じぶん]で 洗[あら]います。 自分[じぶん]で 作[つく]りました。 自分[じぶん]としては 満足[まんぞく]している。
\\	閉まる ドアが閉まるうとしている。 このドアはどうにも閉まらない。 店は7時に閉まる。	
\\	閉[し]まる ドアが 閉[し]まるうとしている。 このドアはどうにも 閉[し]まらない。 店[みせ]は 
\\	時[じ]に 閉[し]まる。
\\	閉める 窓を閉めて。 ドアを閉めましょうか。 そこ閉めておいて。	
\\	閉[し]める 窓[まど]を 閉[し]めて。 ドアを 閉[し]めましょうか。 そこ 閉[し]めておいて。
\\	締める シートベルトを締めてください。 商売柄いつもネクタイを締めなければならない。	
\\	締[し]める シートベルトを 締[し]めてください。 商売[しょうばい] 柄[がら]いつもネクタイを 締[し]めなければならない。
\\	じゃ/じゃあ	
\\	じゃ/じゃあ
\\	写真 この写真大好き。 なんてキレイな写真なんでしょう。 ここで写真をとりましょう。	
\\	写真[しゃしん] この 写真[しゃしん] 大好[だいす]き。 なんてキレイな 写真[しゃしん]なんでしょう。 ここで 写真[しゃしん]をとりましょう。
\\	シャツ このシャツはだれのですか。 「あなたはシャツがほしですか」「はい、赤いシャツがほしいです」 僕はシャツを取り替えます。	
\\	シャツ このシャツはだれのですか。 「あなたはシャツがほしですか」「はい、 赤[あか]いシャツがほしいです」 僕[ぼく]はシャツを 取り替[とりか]えます。
\\	シャワー シャワーにするわ。 シャワーが壊れた。 シャワーを浴びなくちゃ。	
\\	シャワー シャワーにするわ。 シャワーが 壊[こわ]れた。 シャワーを 浴[あ]びなくちゃ。
\\	授業 明日授業があります。 それは授業の終わりだった。 受業中は食事しないこと。	
\\	授業[じゅぎょう] 明日[あした] 授業[じゅぎょう]があります。 それは 授業[じゅぎょう]の 終[お]わりだった。 受業[じゅぎょう] 中[ちゅう]は 食事[しょくじ]しないこと。
\\	宿題 宿題がたくさんある。 宿題をやります。 宿題を一緒にしよう。	
\\	宿題[しゅくだい] 宿題[しゅくだい]がたくさんある。 宿題[しゅくだい]をやります。 宿題[しゅくだい]を 一緒[いっしょ]にしよう。
\\	上手 私は上手じゃありません。 料理が上手ですね。 私は上手に泳ぐことができます。	
\\	上手[じょうず] 私[わたし]は 上手[じょうず]じゃありません。 料理[りょうり]が 上手[じょうず]ですね。 私[わたし]は 上手[じょうず]に 泳[およ]ぐことができます。
\\	丈夫 お宅のお子さんは丈夫ですね。 彼にはあなたが寄りかかれる丈夫な肩がある。	
\\	お宅 
\\	肩
\\	丈夫[じょうぶ] お 宅[たく]のお 子[こ]さんは 丈夫[じょうぶ]ですね。 彼[かれ]にはあなたが 寄[よ]りかかれる 丈夫[じょうぶ]な 肩[かた]がある。
\\	食堂 彼かの家の食堂はとても広々としている。 私達の家は食堂を含めて7部屋あります。	
\\	含め 
\\	食堂[しょくどう] 彼[かれ]かの 家[いえ]の 食堂[しょくどう]はとても 広々[ひろびろ]としている。 私[わたし] 達[たち]の 家[いえ]は 食堂[しょくどう]を 含[ふく]めて 
\\	部屋[へや]あります。
\\	知る 知らなかったの。 要するに知らないし知りたくない。 私を知っていますか。	
\\	要する 
\\	知[し]る 知[し]らなかったの。 要[よう]するに 知[し]らないし 知[し]りたくない。 私[わたし]を 知[し]っていますか。
\\	白 「これは白タク?」私は彼に尋ねた。 うちの庭のバラは一部が白で、その他は赤です。	
\\	尋ね 
\\	一部 
\\	白[しろ] 「これは 白[しろ]タク?」 私[わたし]は 彼[かれ]に 尋[たず]ねた。 うちの 庭[にわ]のバラは 一部[いちぶ]が 白[しろ]で、その 他[た]は 赤[あか]です。
\\	白 白いですか。 彼女は白人だ。 すごく面白いかった。	
\\	白[しろ] 白[しろ]いですか。 彼女[かのじょ]は 白人[はくじん]だ。 すごく 面白[おもしろ]いかった。
\\	新聞 新聞どこ? これは日刊新聞です。 この新聞はタダです。	
\\	新聞[しんぶん] 新聞[しんぶん]どこ? これは 日刊[にっかん] 新聞[しんぶん]です。 この 新聞[しんぶん]はタダです。
\\	水曜日 水曜日は空いていますか。	
\\	水曜日[すいようび] 水曜日[すいようび]は 空[あ]いていますか。
\\	吸う 煙草を吸いますか。 タバコは吸いません。 父は以前は煙草を吸っていたけれど今は吸いません。	
\\	以前 
\\	吸[す]う 煙草[たばこ]を 吸[す]いますか。 タバコは 吸[す]いません。 父[ちち]は 以前[いぜん]は 煙草[たばこ]を 吸[す]っていたけれど 今[いま]は 吸[す]いません。
\\	スカート 私のスカートは長すぎます。 彼女は短いスカートが好きだ。 そのスカート、短すぎない?	
\\	スカート 私[わたし]のスカートは 長[なが]すぎます。 彼女[かのじょ]は 短[みじか]いスカートが 好[す]きだ。 そのスカート、 短[みじか]すぎない?
\\	少ない 5は8より少ない。 そう思っている人が少ない。 私の父は口数が少ない。	
\\	少[すく]ない 5は 8より 少[すく]ない。 そう 思[おも]っている 人[ひと]が 少[すく]ない。 私[わたし]の 父[ちち]は 口数[くちかず]が 少[すく]ない。
\\	すぐに すぐに戻ります。 歩いてすぐですよ すぐに出来ます。	
\\	すぐに すぐに 戻[もど]ります。 歩[ある]いてすぐですよ すぐに 出来[でき]ます。
\\	少し 少しなら話せます。 ここで少し休もう。 少し熱が、あります。	
\\	少[すこ]し 少[すこ]しなら 話[はな]せます。 ここで 少[すこ]し 休[やす]もう。 少[すこ]し 熱[ねつ]が、あります。
\\	涼し 涼しくなってきましたね。 それは涼し所に保存しなさい。 朝晩、涼しくなってきましたね。	
\\	涼[すず]し 涼[すず]しくなってきましたね。 それは 涼[すず]し 所[ところ]に 保存[ほぞん]しなさい。 朝晩[あさばん]、 涼[すず]しくなってきましたね。
\\	ストーブ 彼はストーブで体を暖めた。 彼女は寒かったのでストーブを付けた。 その男の子は厚いストーブに触れて手に火傷をした。	
\\	火傷 
\\	ストーブ 彼[かれ]はストーブで 体[からだ]を 暖[あたた]めた。 彼女[かのじょ]は 寒[さむ]かったのでストーブを 付[つ]けた。 その 男の子[おとこのこ]は 厚[あつ]いストーブに 触[ふ]れて 手[て]に 火傷[かしょう]をした。
\\	スプーン スプーンをただけますか。 増夫はスプーンを曲げられる。	
\\	スプーン スプーンをただけますか。 増[ま] 夫[すお]はスプーンを 曲[ま]げられる。
\\	スポーツ スポーツといえば、あなたはどんなスポーツをされますか? スポーツは好き? 何かスポーツをしますか。	
\\	スポーツ スポーツといえば、あなたはどんなスポーツをされますか? スポーツは 好[す]き? 何[なに]かスポーツをしますか。
\\	ズボン それは僕のズボンです。 そのズボンいくら? 私はズボンをはいた。	
\\	ズボン それは 僕[ぼく]のズボンです。 そのズボンいくら? 私[わたし]はズボンをはいた。
\\	住む どこに住んでいますか。 広州に住んでいる。 ボストンに住んでいます。	
\\	住[す]む どこに 住[す]んでいますか。 広州[こうしゅう]に 住[す]んでいる。 ボストンに 住[す]んでいます。
\\	スリッパ	
\\	スリッパ
\\	座る いや、座ってくれ。 ここに座ってもいいですか。 座ってくつろぎませんか。	
\\	座[すわ]る いや、 座[すわ]ってくれ。 ここに 座[すわ]ってもいいですか。 座[すわ]ってくつろぎませんか。
\\	背 私は背が高い。 彼は背が高いですか。 彼は背を曲げた。	
\\	背[せ] 私[わたし]は 背[せ]が 高[たか]い。 彼[かれ]は 背[せ]が 高[たか]いですか。 彼[かれ]は 背[せ]を 曲[ま]げた。
\\	生徒 トムは生徒です。 私は生徒でありません。 英語の好きな生徒もいるし、物理の好きな生徒もいる。	
\\	生徒[せいと] トムは 生徒[せいと]です。 私[わたし]は 生徒[せいと]でありません。 英語[えいご]の 好[す]きな 生徒[せいと]もいるし、 物理[ぶつり]の 好[す]きな 生徒[せいと]もいる。
\\	セーター セーターが欲しんだけど。 このセーターを着なさい。 セーターが裏返しだ。	
\\	裏返 
\\	セーター セーターが 欲[ほ]しんだけど。 このセーターを 着[き]なさい。 セーターが 裏返[うらがえ]しだ。
\\	せっけん	
\\	せっけん
\\	背広 彼はくたくたの背広をきていた。 この背広に似合うネクタイが欲しいのですが。	
\\	くたくた 
\\	背広[せびろ] 彼[かれ]はくたくたの 背広[せびろ]をきていた。 この 背広[せびろ]に 似合[にあ]うネクタイが 欲[ほ]しいのですが。
\\	狭い 視野が狭くなったようです。 私の部屋はとても狭い。 狭き門より入れ。	
\\	狭[せま]い 視野[しや]が 狭[せま]くなったようです。 私[わたし]の 部屋[へや]はとても 狭[せま]い。 狭[せま]き 門[もん]より 入[はい]れ。
\\	ゼロ 一の前にはゼロです。	
\\	ゼロ 一の 前[まえ]にはゼロです。
\\	千 それは全部で三千円です。 総額は三千円になった。 それは二千年にあった。	
\\	千[せん] それは 全部[ぜんぶ]で三 千[せん] 円[えん]です。 総額[そうがく]は三 千[せん] 円[えん]になった。 それは二 千[せん] 年[ねん]にあった。
\\	先月 先月引っ越しました。 彼は先 月アンと結婚した。 彼女は先月アメリカにいました。	
\\	先月[せんげつ] 先月[せんげつ] 引っ越[ひっこ]しました。 彼[かれ]は 先[せん] 月[つき]アンと 結婚[けっこん]した。 彼女[かのじょ]は 先月[せんげつ]アメリカにいました。
\\	先週 貴方は先週ここにいましたか。 私は先週彼女に会った。 先週彼は病気だった。	
\\	先週[せんしゅう] 貴方[あなた]は 先週[せんしゅう]ここにいましたか。 私[わたし]は 先週[せんしゅう] 彼女[かのじょ]に 会[あ]った。 先週[せんしゅう] 彼[かれ]は 病気[びょうき]だった。
\\	先生 先生の名前はグレイ先生でした。 山田先生は物理の先生ですか、地学の先生ですか。 先生がいらっしゃったよ。	
\\	先生[せんせい] 先生[せんせい]の 名前[なまえ]はグレイ 先生[せんせい]でした。 山田[やまだ] 先生[せんせい]は 物理[ぶつり]の 先生[せんせい]ですか、 地学[ちがく]の 先生[せんせい]ですか。 先生[せんせい]がいらっしゃったよ。
\\	洗濯 日曜日には洗濯です。 洗濯場はどこですか? セーターが洗濯で縮んだ。	
\\	縮ん 
\\	洗濯[せんたく] 日曜日[にちようび]には 洗濯[せんたく]です。 洗濯[せんたく] 場[じょう]はどこですか? セーターが 洗濯[せんたく]で 縮[ちぢ]んだ。
\\	全部 全部が美味しかったな~! ほとんど全部分かった。 彼は全部平らげた。	
\\	平ら 
\\	全部[ぜんぶ] 全部[ぜんぶ]が 美味[おい]しかったな~! ほとんど 全部[ぜんぶ] 分[わ]かった。 彼[かれ]は 全[ぜん] 部[ぶ] 平[たい]らげた。
\\	掃除する 彼女は部屋を掃除してくれた。 私は彼に床を掃除させた。 召し使いは床を掃除した。	
\\	床 
\\	掃除[そうじ]する 彼女[かのじょ]は 部屋[へや]を 掃除[そうじ]してくれた。 私[わたし]は 彼[かれ]に 床[ゆか]を 掃除[そうじ]させた。 召[め]し 使[つか]いは 床[ゆか]を 掃除[そうじ]した。
\\	そうして/そして	
\\	そうして/そして
\\	そこ そこにいるのは誰だ。 そこから降りる!	
\\	そこ そこにいるのは 誰[だれ]だ。 そこから 降[お]りる!
\\	そちら そちらは暑いですか そちらには伺えませんが	
\\	そちら そちらは 暑[あつ]いですか そちらには 伺[うかが]えませんが
\\	そっち そっちの天気は? そっちは今何時?	
\\	そっち そっちの 天気[てんき]は? そっちは 今[いま] 何[なん] 時[じ]?
\\	外 外は雪だよ。 外は真っ暗だ。 外はどんよりしている。	
\\	外[そと] 外[がい]は 雪[ゆき]だよ。 外[がい]は 真っ暗[まっくら]だ。 外[がい]はどんよりしている。
\\	その	
\\	その
\\	そば お父さんのそばに座りなさい。 その少女は私のそばにすわった。 君が俺のそばにいてくれたらなあ。	
\\	そば お 父[とう]さんのそばに 座[すわ]りなさい。 その 少女[しょうじょ]は 私[わたし]のそばにすわった。 君[きみ]が 俺[おれ]のそばにいてくれたらなあ。
\\	空 その缶は空だ。 
\\	が空を飛んでいる。 そらは星でいっぱいだ。	
\\	空[そら] その 缶[かん]は 空[そら]だ。 
\\	が 空[そら]を 飛[と]んでいる。 そらは 星[ほし]でいっぱいだ。
\\	それから	
\\	それから
\\	それでは	
\\	それでは
\\	それでも	
\\	それでも
\\	大学 この大学は創成されたのか。 大学で何をするつもりですか 大切なのはどの大学を出たかではなくて、大学で何を学んだかである。	
\\	創成 
\\	大学[だいがく] この 大学[だいがく]は 創成[そうせい]されたのか。 大学[だいがく]で 何[なに]をするつもりですか 大切[たいせつ]なのはどの 大学[だいがく]を 出[で]たかではなくて、 大学[だいがく]で 何[なに]を 学[まな]んだかである。
\\	大使館 日本大使館はどこにありますか。 オランダ大使館はどこですか? 姉はロンドンの合衆国大使館で働いでいます。	
\\	合衆国
\\	大使館[たいしかん] 日本[にっぽん] 大使館[たいしかん]はどこにありますか。 オランダ 大使館[たいしかん]はどこですか? 姉[あね]はロンドンの 合衆国[がっしゅうこく] 大使館[たいしかん]で 働[はたら]いでいます。
\\	大好き 雪が大好きです。 日が大好き。 麻雀が大好きです。 ラザニアが大好きです。	
\\	大好[だいす]き 雪[ゆき]が 大好[だいす]きです。 日[び]が 大好[だいす]き。 麻雀[まーじゃん]が 大好[だいす]きです。 ラザニアが 大好[だいす]きです。
\\	大切 大切にします。 大切なことなんだから。 平和ほど大切なものはない。	
\\	大切[たいせつ] 大切[たいせつ]にします。 大切[たいせつ]なことなんだから。 平和[へいわ]ほど 大切[たいせつ]なものはない。
\\	台所 台所にいます。 台所をよろしく。 台所にはほとんど何もなかった。	
\\	台所[だいどころ] 台所[だいどころ]にいます。 台所[だいどころ]をよろしく。 台所[だいどころ]にはほとんど 何[なに]もなかった。
\\	大変/たいへん 彼はたいへん勇敢だ。 トムはたいへん人気がある。 花嫁はたいへん美しく見えた。	
\\	大変[たいへん]/たいへん 彼[かれ]はたいへん 勇敢[ゆうかん]だ。 トムはたいへん 人気[にんき]がある。 花嫁[はなよめ]はたいへん 美[うつく]しく 見[み]えた。
\\	大変/たいへん かれはたいへん注意深くふるまっている。 その知らせは彼女をたいへん悲しませた。 たいへんお手数をおかけしまして申し訳ございません。	
\\	注意深 
\\	手数 
\\	申し訳 
\\	大変[たいへん]/たいへん かれはたいへん 注意深[ちゅういぶか]くふるまっている。 その 知[し]らせは 彼女[かのじょ]をたいへん 悲[かな]しませた。 たいへんお 手数[てかず]をおかけしまして 申し訳[もうしわけ]ございません。
\\	高い それは高くない。 波が高い。 この時計は高い。	
\\	高[たか]い それは 高[たか]くない。 波[なみ]が 高[たか]い。 この 時計[とけい]は 高[たか]い。
\\	たくさん たくさん欲しい。 友達はたくさんいますか。 宿題がたくさんある。	
\\	たくさん たくさん 欲[ほ]しい。 友達[ともだち]はたくさんいますか。 宿題[しゅくだい]がたくさんある。
\\	タクシー タクシーが到着した。 タクシーに乗ろうよ。 タクシーで行きませんか。	
\\	到着 
\\	乗ろ 
\\	タクシー タクシーが 到着[とうちゃく]した。 タクシーに 乗[の]ろうよ。 タクシーで 行[い]きませんか。
\\	出す 口を出すな。 トムは泣き出した。 此処から出してくれ。	
\\	此処 
\\	出[だ]す 口[ぐち]を 出[だ]すな。 トムは 泣[な]き 出[だ]した。 此処[ここ]から 出[だ]してくれ。
\\	立つ 彼は立った。 彼は立とうとしました。 そこに立っているのは誰ですか。	
\\	立[た]つ 彼[かれ]は 立[た]った。 彼[かれ]は 立[た]とうとしました。 そこに 立[た]っているのは 誰[だれ]ですか。
\\	たて	
\\	たて
\\	建物 建物でエスカレーターがありません。 その建物には非常口がない 兵士たちが建物を占拠した。	
\\	非常口 
\\	占拠 
\\	建物[たてもの] 建物[たてもの]でエスカレーターがありません。 その 建物[たてもの]には 非常口[ひじょうぐち]がない 兵士[へいし]たちが 建物[たてもの]を 占拠[せんきょ]した。
\\	楽しい 旅行は楽しい。 本当に楽しかった。 楽しく過ごしている。	
\\	過ご 
\\	楽[たの]しい 旅行[りょこう]は 楽[たの]しい。 本当[ほんとう]に 楽[たの]しかった。 楽[たの]しく 過[す]ごしている。
\\	頼む 先生に頼もう。 誰は食物をくれと頼んだ。 彼にそうしてくれと頼んだ。	
\\	頼[たの]む 先生[せんせい]に 頼[たの]もう。 誰[だれ]は 食物[しょくもつ]をくれと 頼[たの]んだ。 彼[かれ]にそうしてくれと 頼[たの]んだ。
\\	たばこ/煙草 煙草は吸いません。 私はタバコが吸えません。 ここでタバコを吸ってもかまいません。	
\\	たばこ/ 煙草[たばこ] 煙草[たばこ]は 吸[す]いません。 私[わたし]はタバコが 吸[す]えません。 ここでタバコを 吸[す]ってもかまいません。
\\	多分/たぶん 多分雨になるだろう。 それではたぶん戦争ということになるだろう。 彼は多分有名にならない。	
\\	多分[たぶん]/たぶん 多分[たぶん] 雨[う]になるだろう。 それではたぶん 戦争[せんそう]ということになるだろう。 彼[かれ]は 多分[たぶん] 有名[ゆうめい]にならない。
\\	食べ物 それは食べ物です。 食べ物が冷めます 私は彼に食べ物を与えた。	
\\	冷め 
\\	与え 
\\	食べ物[たべもの] それは 食べ物[たべもの]です。 食べ物[たべもの]が 冷[さ]めます 私[わたし]は 彼[かれ]に 食べ物[たべもの]を 与[あた]えた。
\\	食べる 食べました。 なしを食べています。 食べたいだけ食べなさい。	
\\	食[た]べる 食[た]べました。 なしを 食[た]べています。 食[た]べたいだけ 食[た]べなさい。
\\	卵 卵は嫌いです。 それは卵のように見える。 腐っていない卵もいくつかあったが、残りの卵はみな腐っていた。	
\\	腐 
\\	残り 
\\	卵[たまご] 卵[たまご]は 嫌[きら]いです。 それは 卵[たまご]のように 見[み]える。 腐[くさ]っていない 卵[たまご]もいくつかあったが、 残[のこ]りの 卵[たまご]はみな 腐[くさ]っていた。
\\	誰 では誰ですか。 誰がしたの? 誰でもそれはできる。	
\\	誰[だれ] では 誰[だれ]ですか。 誰[だれ]がしたの? 誰[だれ]でもそれはできる。
\\	誰か 誰かがやってきた。 誰かをお探しですか	
\\	誰[だれ]か 誰[だれ]かがやってきた。 誰[だれ]かをお 探[さが]しですか
\\	誕生日 誕生日いつ? 今日は私の誕生日です。 誕生日はどうでしたか。	
\\	誕生[たんじょう] 日[び] 誕生[たんじょう] 日[び]いつ? 今日[きょう]は 私[わたし]の 誕生[たんじょう] 日[び]です。 誕生[たんじょう] 日[び]はどうでしたか。
\\	だんだん だんだん暗くなりつつある。 外はだんだん明るくなっていく。 ぶどうがだんだん熟してきた。	
\\	熟し 
\\	だんだん だんだん 暗[くら]くなりつつある。 外[そと]はだんだん 明[あか]るくなっていく。 ぶどうがだんだん 熟[じゅく]してきた。
\\	小さい これよりも小さいのはないですか。 小さすぎるわ。 その本は小さい。	
\\	小[ちい]さい これよりも 小[ちい]さいのはないですか。 小[ちい]さすぎるわ。 その 本[ほん]は 小[ちい]さい。
\\	小さな うちには小さな庭しかない。 彼女は小さなパンを持っています。 世界は小さな村である。	
\\	村 
\\	小[ちい]さな うちには 小[ちい]さな 庭[にわ]しかない。 彼女[かのじょ]は 小[ちい]さなパンを 持[も]っています。 世界[せかい]は 小[ちい]さな 村[むら]である。
\\	一日 一日中出掛けていた。 一日は終わろうとしてる。 今日は疲れる一日。	
\\	一 日[にち] 一 日[にち] 中[ちゅう] 出掛[でか]けていた。 一 日[にち]は 終[お]わろうとしてる。 今日[きょう]は 疲[つか]れる一 日[にち]。
\\	使う これ使って。 タクシーが使われていますか? はい、つかわれています。 私はそれを使います。	
\\	使[つか]う これ 使[つか]って。 タクシーが 使[つか]われていますか? はい、つかわれています。 私[わたし]はそれを 使[つか]います。
\\	次 リストは次のとおり。 次のことを説明しなさい。 次の時間は何だっけ?	
\\	次[つぎ] リストは 次[つぎ]のとおり。 次[つぎ]のことを 説明[せつめい]しなさい。 次[つぎ]の 時間[じかん]は 何[なに]だっけ?
\\	着く もう着きましたか。 いつ着くの? かれはいつここに着いたのか?	
\\	着[つ]く もう 着[つ]きましたか。 いつ 着[つ]くの? かれはいつここに 着[つ]いたのか?
\\	机 話しているのはこのペンについてであって、机の上にある方ではない。 これらは私たちの机です。	
\\	机[つくえ] 話[はな]しているのはこのペンについてであって、 机[つくえ]の 上[うえ]にある 方[ほう]ではない。 これらは 私[わたし]たちの 机[つくえ]です。
\\	作る コーヒーを作るわ。 何を作ったの。 ここで小麦を作る。	
\\	作[つく]る コーヒーを 作[つく]るわ。 何[なに]を 作[つく]ったの。 ここで 小麦[こむぎ]を 作[つく]る。
\\	つける テレビをつけて。 電気をつけていただけませんか? 後をつけろ。	
\\	つける テレビをつけて。 電気[でんき]をつけていただけませんか? 後[ご]をつけろ。
\\	勤める 彼はもうここには勤めていない。 パーティーでは誰がホストを勤めるのですか。 彼は広告会社に勤めている。	
\\	広告
\\	勤[つと]める 彼[かれ]はもうここには 勤[つと]めていない。 パーティーでは 誰[だれ]がホストを 勤[つと]めるのですか。 彼[かれ]は 広告[こうこく] 会社[かいしゃ]に 勤[つと]めている。
\\	つまらない このビデオはつまらないよ。 つまらないことで何という騒ぎだ。 外国語を学習するのはつまらないよ。	
\\	騒 
\\	つまらない このビデオはつまらないよ。 つまらないことで 何[なに]という 騒[さわ]ぎだ。 外国[がいこく] 語[ご]を 学習[がくしゅう]するのはつまらないよ。
\\	冷たい レモネードは冷たい。 冷たい水をください。 この壁は触ると冷たい感じがする。	
\\	触る 
\\	冷[つめ]たい レモネードは 冷[つめ]たい。 冷[つめ]たい 水[みず]をください。 この 壁[かべ]は 触[さわ]ると 冷[つめ]たい 感[かん]じがする。
\\	強い 彼は強い。 私は好奇心が強いです。 私は強く信じている。	
\\	好奇心 
\\	強[つよ]い 彼[かれ]は 強[つよ]い。 私[わたし]は 好奇[こうき] 心[しん]が 強[つよ]いです。 私[わたし]は 強[つよ]く 信[しん]じている。
\\	手 手を離せ! 手は尽くした。 そしててが震えていました。 直子さんは泳ぎ手です。	
\\	離 
\\	尽く 
\\	震え 
\\	手[て] 手[て]を 離[はな]せ! 手[て]は 尽[つ]くした。 そしててが 震[ふる]えていました。 直子[なおこ]さんは 泳[およ]ぎ 手[て]です。
\\	テープ そのテープを聞きましょう。 このテープはくっつかない。 テープの両端をつないでください。	
\\	テープ そのテープを 聞[き]きましょう。 このテープはくっつかない。 テープの 両端[りょうたん]をつないでください。
\\	テーブル あれはテーブルです。 このテーブルは木製である。 テーブルを片付けなさい。	
\\	片付け 
\\	テーブル あれはテーブルです。 このテーブルは 木製[もくせい]である。 テーブルを 片付[かたづ]けなさい。
\\	テープレコーダー このテープレコーダーは新しくない。 このテープレコーダーは修理が必要だ。 このテープレコーダーは彼の声を録音した。	
\\	修理 
\\	必要 
\\	録音 
\\	テープレコーダー このテープレコーダーは 新[あたら]しくない。 このテープレコーダーは 修理[しゅうり]が 必要[ひつよう]だ。 このテープレコーダーは 彼[かれ]の 声[こえ]を 録音[ろくおん]した。
\\	出かける。 ちょうど出かけるところだ。 出かける時間になった。 はやく出かけましょう。	
\\	出[で]かける。 ちょうど 出[で]かけるところだ。 出[で]かける 時間[じかん]になった。 はやく 出[で]かけましょう。
\\	手紙 これは誰からの手紙ですか? お手紙を受け取りました。 その手紙を受け取りましたか?	
\\	受け取り 
\\	手紙[てがみ] これは 誰[だれ]からの 手紙[てがみ]ですか? お 手紙[てがみ]を 受け取[うけと]りました。 その 手紙[てがみ]を 受け取[うけと]りましたか?
\\	できる できるだけ速く歩きなさい。 できるだけ速く走りなさい できるだけたくさんの本を読みなさい。	
\\	できる できるだけ 速[はや]く 歩[ある]きなさい。 できるだけ 速[はや]く 走[はし]りなさい できるだけたくさんの 本[ほん]を 読[よ]みなさい。
\\	出口 標識には、出口と書かれている。 火事の時にはこの出口を使ってください。	
\\	標識 
\\	火事 
\\	使 
\\	出口[いでぐち] 標識[ひょうしき]には、 出口[いでぐち]と 書[か]かれている。 火事[かじ]の 時[とき]にはこの 出口[でぐち]を 使[つか]ってください。
\\	近い じゃさよなら、また近いうちにね。 トイレが近いのです。 奇蹟に近いですね。	
\\	近[ちか]い じゃさよなら、また 近[ちか]いうちにね。 トイレが 近[ちか]いのです。 奇蹟[きせき]に 近[ちか]いですね。
\\	違う 違う? これは注文と違います 番号が違っています。	
\\	注文 
\\	違[ちが]う 違[ちが]う? これは 注文[ちゅうもん]と 違[ちが]います 番号[ばんごう]が 違[ちが]っています。
\\	近く この近くにお住まいなんですか? 近くにマクドナルドは? この近くに
\\	はありますか?	
\\	近[ちか]く この 近[ちか]くにお 住[す]まいなんですか? 近[ちか]くにマクドナルドは? この 近[ちか]くに 
\\	はありますか?
\\	地下鉄 地下鉄で行きたいのです。 どこで地下鉄に乗るか知っていますか。 地下鉄の路線図をください。	
\\	路線 
\\	図 
\\	地下鉄[ちかてつ] 地下鉄[ちかてつ]で 行[い]きたいのです。 どこで 地下鉄[ちかてつ]に 乗[の]るか 知[し]っていますか。 地下鉄[ちかてつ]の 路線[ろせん] 図[ず]をください。
\\	茶色 その猫は茶色です。 かれは茶色の目をしている。 その茶色の帽子は古いです。	
\\	茶色[ちゃいろ] その 猫[ねこ]は 茶色[ちゃいろ]です。 かれは 茶色[ちゃいろ]の 目[め]をしている。 その 茶色[ちゃいろ]の 帽子[ぼうし]は 古[ふる]いです。
\\	ちゃわん/茶碗 その茶碗にはひびがある。 誰がその茶碗をわったのか?	
\\	ちゃわん/ 茶碗[ちゃわん] その 茶碗[ちゃわん]にはひびがある。 誰[だれ]がその 茶碗[ちゃわん]をわったのか?
\\	ちょうど/丁度 ちょうど出かけるところだ  ちょうど日曜日にぶつかりますね。 彼女はちょうど今電話した。	
\\	ちょうど ちょうど 出[で]かけるところだ  ちょうど 日曜日[にちようび]にぶつかりますね。 彼女[かのじょ]はちょうど 今[いま] 電話[でんわ]した。
\\	ちょっと もうちょっと。 それはちょっとおかしね。 ちょっと混乱しています。	
\\	混乱 
\\	ちょっと もうちょっと。 それはちょっとおかしね。 ちょっと 混乱[こんらん]しています。
\\	テスト 明日はテストがあります。 テストでカンニングをしたことがありますか。 私はそのテストに合格できた。	
\\	合格 
\\	テスト 明日[あした]はテストがあります。 テストでカンニングをしたことがありますか。 私[わたし]はそのテストに 合格[ごうかく]できた。
\\	では	
\\	では
\\	デパート 一番近いデパートはどこにありますか?	
\\	デパート 一番[いちばん] 近[ちか]いデパートはどこにありますか?
\\	出る 私が出ます。 咳が出ますか。 外へ出たい。	
\\	咳 
\\	出[で]る 私[わたし]が 出[で]ます。 咳[せき]が 出[で]ますか。 外[そと]へ 出[で]たい。
\\	テレビ これはテレビだ。 テレビをつけて。 テレビが故障しています。	
\\	故障 
\\	テレビ これはテレビだ。 テレビをつけて。 テレビが 故障[こしょう]しています。
\\	天気 天気はどうですか。 なんてすばらしいお天気なんだ。 天気が悪くなった。	
\\	天気[てんき] 天気[てんき]はどうですか。 なんてすばらしいお 天気[てんき]なんだ。 天気[てんき]が 悪[わる]くなった。
\\	電気 電気がつきません。 私は電気工です。 この時計は電気で動く。	
\\	動く 
\\	電気[でんき] 電気[でんき]がつきません。 私[わたし]は 電気[でんき] 工[こう]です。 この 時計[とけい]は 電気[でんき]で 動[うご]く。
\\	電車 電車で行こう。 その電車は正確だ 電車が遅れているようだ。	
\\	正確 
\\	電車[でんしゃ] 電車[でんしゃ]で 行[い]こう。 その 電車[でんしゃ]は 正確[せいかく]だ 電車[でんしゃ]が 遅[おく]れているようだ。
\\	電話 君に電話ができなかった。電話が壊れていたんだ。 もう電話するな。 電話が鳴っています。	
\\	壊れ 
\\	鳴って 
\\	電話[でんわ] 君[きみ]に 電話[でんわ]ができなかった。 電話[でんわ]が 壊[こわ]れていたんだ。 もう 電話[でんわ]するな。 電話[でんわ]が 鳴[な]っています。
\\	戸 戸にロックをかけなさい。 戸を押さえてくれ。 戸棚を覗いた。	
\\	押さえ 
\\	戸棚 
\\	戸[と] 戸[と]にロックをかけなさい。 戸[と]を 押[お]さえてくれ。 戸棚[とだな]を 覗[のぞ]いた。
\\	ドア ドアを開けて。 ドアを閉めましょうか。 これは自動ドアです。	
\\	ドア ドアを 開[あ]けて。 ドアを 閉[し]めましょうか。 これは 自動[じどう]ドアです。
\\	どう	
\\	どう
\\	どうして その後どうしていますか。 どうして私なの。 どうして泣いているのですか?	
\\	どうして その 後[ご]どうしていますか。 どうして 私[わたし]なの。 どうして 泣[な]いているのですか?
\\	どうぞ はいどうぞ。 プレゼントをどうぞ。 ケーキをどうぞ。	
\\	どうぞ はいどうぞ。 プレゼントをどうぞ。 ケーキをどうぞ。
\\	動物 動物園にたくさんの動物がいますか。 すべての動物は平等である。しかしある動物は他の動物よりさらに平等である。	
\\	平等 
\\	動物[どうぶつ] 動物[どうぶつ] 園[えん]にたくさんの 動物[どうぶつ]がいますか。 すべての 動物[どうぶつ]は 平等[びょうどう]である。しかしある 動物[どうぶつ]は 他[た]の 動物[どうぶつ]よりさらに 平等[びょうどう]である。
\\	どうも 今日はどうもありがとう。 どうもご親切に。 先日はどうもありがとう。	
\\	親切 
\\	どうも 今日[きょう]はどうもありがとう。 どうもご 親切[しんせつ]に。 先日[せんじつ]はどうもありがとう。
\\	遠い ここから遠いの そんなに遠くありません。 パリまで遠くない。	
\\	遠[とお]い ここから 遠[とお]いの そんなに 遠[とお]くありません。 パリまで 遠[とお]くない。
\\	十日 学校は四月十日から始まります。 一日は二十四時間ある。	
\\	十日[とおか] 学校[がっこう]は 四月[しがつ] 十日[とおか]から 始[はじ]まります。 一 日[にち]は二十四 時間[じかん]ある。
\\	時々 うん、時々はね。 時々ゴルフをします。 私は時々ステートをします。	
\\	時々[ときどき] うん、 時々[ときどき]はね。 時々[ときどき]ゴルフをします。 私[わたし]は 時々[ときどき]ステートをします。
\\	時計 この時計はこの時計ほど高くない。 時計が止まった。 その時計は狂っている	
\\	時計[とけい] この 時計[とけい]はこの 時計[とけい]ほど 高[たか]くない。 時計[とけい]が 止[と]まった。 その 時計[とけい]は 狂[くる]っている
\\	所 危ういところを助かった。 何をしているこんな所で バスの停留所はどこですか。	
\\	停留所 
\\	所[ところ] 危[あや]ういところを 助[たす]かった。 何[なに]をしているこんな 所[ところ]で バスの 停留所[ていりゅうじょ]はどこですか。
\\	年 もう年だよ。 この年は二千十一年です。 彼は年をとり過ぎている。	
\\	年[とし] もう 年[とし]だよ。 この 年[とし]は二 千[せん]十一 年[ねん]です。 彼[かれ]は 年[とし]をとり 過[す]ぎている。
\\	図書館 図書館はどこですか 図書館に彼らしないなかった。 その本は図書館にあるでしょう。	
\\	図書館[としょかん] 図書館[としょかん]はどこですか 図書館[としょかん]に 彼[かれ]らしないなかった。 その 本[ほん]は 図書館[としょかん]にあるでしょう。
\\	どちら どちらが私のものですか どちら様ですか 出口はどちらですか	
\\	どちら どちらが 私[わたし]のものですか どちら 様[さま]ですか 出口[でぐち]はどちらですか
\\	どっち どっち取る どっちが勝ってるの? どっちを使えばいいんだろ?	
\\	どっち どっち 取[と]る どっちが 勝[か]ってるの? どっちを 使[つか]えばいいんだろ?
\\	とても とても感動したわ。 とても暗かった。 とてもキレイでした。	
\\	とても とても 感動[かんどう]したわ。 とても 暗[くら]かった。 とてもキレイでした。
\\	どなた 「どなたですか」 「私です」 「どなたですか」「お母さんよ」 担当の方はどなたですか。	
\\	担当
\\	どなた 「どなたですか」 
\\	私[わたし]です」 「どなたですか」「お 母[かあ]さんよ」 担当[たんとう]の 方[ほう]はどなたですか。
\\	隣 私達は隣同士です かれの隣に座りました。 彼は彼女の隣に座った。	
\\	同士 
\\	隣[となり] 私[わたし] 達[たち]は 隣[となり] 同士[どうし]です かれの 隣[となり]に 座[すわ]りました。 彼[かれ]は 彼女[かのじょ]の 隣[となり]に 座[すわ]った。
\\	どの どのくらい深い? どのくらい広い? どのくらい遠い?	
\\	どの どのくらい 深[ふか]い? どのくらい 広[ひろ]い? どのくらい 遠[とお]い?
\\	飛ぶ 鳥は飛ぶ。 その烏は飛び去った。 
\\	が空を飛んでいる。	
\\	飛び去 
\\	飛[と]ぶ 鳥[とり]は 飛[と]ぶ。 その 烏[からす]は 飛[と]び 去[さ]った。 
\\	が 空[そら]を 飛[と]んでいる。
\\	止まる ここで止まるな。 エンギンが止まった。 時計が止まった。	
\\	止[と]まる ここで 止[と]まるな。 エンギンが 止[と]まった。 時計[とけい]が 止[と]まった。
\\	友達 彼らは友達なんですか。 私は友達がいない。 ずっと友達でいようね。	
\\	友達[ともだち] 彼[かれ]らは 友達[ともだち]なんですか。 私[わたし]は 友達[ともだち]がいない。 ずっと 友達[ともだち]でいようね。
\\	土曜日 今日は土曜日です。 私は土曜日からここにいます。 土曜日に何か計画がありますか。	
\\	計画 
\\	土曜日[どようび] 今日[きょう]は 土曜日[どようび]です。 私[わたし]は 土曜日[どようび]からここにいます。 土曜日[どようび]に 何[なに]か 計画[けいかく]がありますか。
\\	鳥 鳥は鳴く。 鳥には翼がある。 鳥は卵を産む。	
\\	産む 
\\	鳥[とり] 鳥[とり]は 鳴[な]く。 鳥[とり]には 翼[つばさ]がある。 鳥[とり]は 卵[たまご]を 産[う]む。
\\	とり肉 犬は肉を骨からかじりとった	
\\	骨 
\\	とり 肉[にく] 犬[いぬ]は 肉[にく]を 骨[ほね]からかじりとった
\\	取る どっち取る? 好きなものを取れ。 すきなだけ取ってね。	
\\	取[と]る どっち 取[と]る? 好[す]きなものを 取[と]れ。 すきなだけ 取[と]ってね。
\\	撮る ここで写真を撮ってもいいでしょうか。 彼女は写真を撮るのが好きだ。 彼は家族の写真を撮った。	
\\	撮[と]る ここで 写真[しゃしん]を 撮[と]ってもいいでしょうか。 彼女[かのじょ]は 写真[しゃしん]を 撮[と]るのが 好[す]きだ。 彼[かれ]は 家族[かぞく]の 写真[しゃしん]を 撮[と]った。
\\	どち	
\\	どち
\\	ナイフ ナイフがない。 あれはナイフじゃないよ。これがナイフだよ。 ナイフが鋭くない。	
\\	鋭 
\\	ナイフ ナイフがない。 あれはナイフじゃないよ。これがナイフだよ。 ナイフが 鋭[するど]くない。
\\	中 なには気難し人がいる。 合図するまで、なかにいなさい。 食事はオーブンのなかにおいてあります。	
\\	気難し
\\	合図 
\\	中[なか] なには 気難[きむずか]し 人[じん]がいる。 合図[あいず]するまで、なかにいなさい。 食事[しょくじ]はオーブンのなかにおいてあります。
\\	長い 人生は長い、長い道だ。 その川は長いです。 話せば長いことなのです。	
\\	長[なが]い 人生[じんせい]は 長[なが]い、 長[なが]い 道[みち]だ。 その 川[かわ]は 長[なが]いです。 話[はな]せば 長[なが]いことなのです。
\\	鳴く 私の猫は鳴いてます。 猫はにゃーにゃーと鳴く。 木に鳥が鳴いた。	
\\	鳴[な]く 私[わたし]の 猫[ねこ]は 鳴[な]いてます。 猫[ねこ]はにゃーにゃーと 鳴[な]く。 木[き]に 鳥[とり]が 鳴[な]いた。
\\	無く これは文では無い。 悪気は無かったんです。 これは安くは無かったでしょう。	
\\	無[な]く これは 文[ぶん]では 無[な]い。 悪気[わるぎ]は 無[な]かったんです。 これは 安[やす]くは 無[な]かったでしょう。
\\	なぜ なぜ? しかしなぜ。 なぜかれはここにいるのですか。	
\\	なぜ なぜ? しかしなぜ。 なぜかれはここにいるのですか。
\\	夏 夏は終わった。 夏はどうだった? 私は夏には水泳をする。	
\\	水泳 
\\	夏[なつ] 夏[なつ]は 終[お]わった。 夏[なつ]はどうだった? 私[わたし]は 夏[なつ]には 水泳[すいえい]をする。
\\	夏休み 夏休みが終わった。 私は夏休みを楽しみにしている。 この夏休みは働こうと思いました。	
\\	夏休[なつやす]み 夏休[なつやす]みが 終[お]わった。 私[わたし]は 夏休[なつやす]みを 楽[たの]しみにしている。 この 夏休[なつやす]みは 働[はたら]こうと 思[おも]いました。
\\	など/等	
\\	など/ 等[など]
\\	七つ	
\\	七[なな]つ
\\	七日 一週間は七日です。	
\\	七日[なのか] 一週間[いっしゅうかん]は 七日[なのか]です。
\\	名前 彼の名前は変な名前だ。 お嬢さんのお名前は。 それはありふれた名前ですか。	
\\	名前[なまえ] 彼[かれ]の 名前[なまえ]は 変[へん]な 名前[なまえ]だ。 お 嬢[じょう]さんのお 名前[なまえ]は。 それはありふれた 名前[なまえ]ですか。
\\	習う どこでそれを習いましたか。 生け花を習いたい。 私は音楽を習います。	
\\	習[なら]う どこでそれを 習[なら]いましたか。 生け花[いけばな]を 習[なら]いたい。 私[わたし]は 音楽[おんがく]を 習[なら]います。
\\	並ぶ 並んでお待ちください。 兵士が全員並んで立っていた。 ただし、列をつくって並ばなくてはならないが。。。	
\\	列 
\\	並[なら]ぶ 並[なら]んでお 待[ま]ちください。 兵士[へいし]が 全員[ぜんいん] 並[なら]んで 立[た]っていた。 ただし、 列[れつ]をつくって 並[なら]ばなくてはならないが。。。
\\	並べる 我々は大きさの順に並べた。 身長順に男の子を並べる。	
\\	順 
\\	並[なら]べる 我々[われわれ]は 大[おお]きさの 順[じゅん]に 並[なら]べた。 身長[しんちょう] 順[じゅん]に 男の子[おとこのこ]を 並[なら]べる。
\\	なる	
\\	なる
\\	何   なん/なに だから何? ここで何があったんだ? 何もしなかった。	
\\	何[なに]   なん/なに だから 何[なに]? ここで 何[なに]があったんだ? 何[なに]もしなかった。
\\	二 この切符で二人入れる。 二人でどっか行こうか? 私は二時間ここにいます。	
\\	二 この 切符[きっぷ]で 二人[ふたり] 入[い]れる。 二人[ふたり]でどっか 行[い]こうか? 私[わたし]は二 時間[じかん]ここにいます。
\\	賑やか この通りは賑やかだ。	
\\	賑[にぎ]やか この 通[とお]りは 賑[にぎ]やかだ。
\\	肉 その肉は鶏の肉です。 この肉は固いですね。 その肉は冷凍されている。	
\\	肉[にく] その 肉[にく]は 鶏[にわとり]の 肉[にく]です。 この 肉[にく]は 固[かた]いですね。 その 肉[にく]は 冷凍[れいとう]されている。
\\	西 フランスは西ヨーロッパにある。 駅はホテルの西の方にある。 ラクダの長い列が西向かった移動していた。	
\\	ラクダ 
\\	移動 
\\	西[にし] フランスは 西[にし]ヨーロッパにある。 駅[えき]はホテルの 西[にし]の 方[ほう]にある。 ラクダの 長[なが]い 列[れつ]が 西[にし] 向[む]かった 移動[いどう]していた。
\\	日曜日 明日は日曜日です。 日曜日でも開いてますか。 日曜日でも、働きます。	
\\	日曜日[にちようび] 明日[あした]は 日曜日[にちようび]です。 日曜日[にちようび]でも 開[ひら]いてますか。 日曜日[にちようび]でも、 働[はたら]きます。
\\	荷物 お荷物をお持ちしましょうか。 荷物は何個ですか。 私の荷物が壊れています。	
\\	荷物[にもつ] お 荷物[にもつ]をお 持[も]ちしましょうか。 荷物[にもつ]は 何[なん] 個[こ]ですか。 私[わたし]の 荷物[にもつ]が 壊[こわ]れています。
\\	ニュース それは良いニュースなのか? いつそのニュースを聞いたのですか。 あなたはそのニュースを聞きましたか。	
\\	ニュース それは 良[よ]いニュースなのか? いつそのニュースを 聞[き]いたのですか。 あなたはそのニュースを 聞[き]きましたか。
\\	庭 なんと美し庭だこと。 庭を荒らしたのは誰だ。 かれは庭で仕事するのが好きだ。	
\\	庭[にわ] なんと 美[うつく]し 庭[にわ]だこと。 庭[にわ]を 荒[あ]らしたのは 誰[だれ]だ。 かれは 庭[にわ]で 仕事[しごと]するのが 好[す]きだ。
\\	脱ぐ ここで靴を脱がなければなりませんか。 私は服を脱いでいる。 彼はコートを脱いだ。	
\\	脱[ぬ]ぐ ここで 靴[くつ]を 脱[ぬ]がなければなりませんか。 私[わたし]は 服[ふく]を 脱[ぬ]いでいる。 彼[かれ]はコートを 脱[ぬ]いだ。
\\	温い	
\\	温[ぬる]い
\\	ネクタイ このネクタイはいくらですか 君のネクタイはほどけている。 ネクタイが曲がってるよ。	
\\	ネクタイ このネクタイはいくらですか 君[くん]のネクタイはほどけている。 ネクタイが 曲[ま]がってるよ。
\\	猫 猫が大好きです。 その猫は茶色です。 猫は人間ではない。	
\\	猫[ねこ] 猫[ねこ]が 大好[だいす]きです。 その 猫[ねこ]は 茶色[ちゃいろ]です。 猫[ねこ]は 人間[にんげん]ではない。
\\	寝る 寝ないと! 寝るのが好き。 寝てもいいですか。	
\\	寝[ね]る 寝[ね]ないと! 寝[ね]るのが 好[す]き。 寝[ね]てもいいですか。
\\	ノート あのノートは誰のですか? 盗まれたのは彼のノートだった。 私は君にノートをあげよう。	
\\	盗ま 
\\	ノート あのノートは 誰[だれ]のですか? 盗[ぬす]まれたのは 彼[かれ]のノートだった。 私[わたし]は 君[きみ]にノートをあげよう。
\\	登る 塀に登るな。 富士山登ったことある? 私達が高く登れば登るほど、 ますます寒くなった。	
\\	塀 
\\	登[のぼ]る 塀[へい]に 登[のぼ]るな。 富士山[ふじさん] 登[のぼ]ったことある? 私[わたし] 達[たち]が 高[たか]く 登[のぼ]れば 登[のぼ]るほど、 ますます 寒[さむ]くなった。
\\	飲み物 飲み物を自由にお取りください。 何か暖かい飲み物がほしい。 嫌いな飲み物は何ですか。	
\\	自由 
\\	取りく 
\\	飲み物[のみもの] 飲み物[のみもの]を 自由[じゆう]にお 取[と]りください。 何[なに]か 暖[あたた]かい 飲み物[のみもの]がほしい。 嫌[きら]いな 飲み物[のみもの]は 何[なに]ですか。
\\	飲む 飲んだら乗るな、乗るなら飲むな。 コーヒー飲んだよ。 お酒は飲みますか。	
\\	乗る 
\\	飲[の]む 飲[の]んだら 乗[の]るな、 乗[の]るなら 飲[の]むな。 コーヒー 飲[の]んだよ。 お 酒[さけ]は 飲[の]みますか。
\\	乗る 乗りますか? 車に乗って。 タクシーに乗りましょうか。	
\\	乗[の]る 乗[の]りますか? 車[くるま]に 乗[の]って。 タクシーに 乗[の]りましょうか。
\\	歯 ちゃんと歯を磨いていますか。 歯をきれいに磨いなさい。 歯を磨いておけば。。。。 ひどく歯が痛い。	
\\	磨い 
\\	ひどい 
\\	歯[は] ちゃんと 歯[は]を 磨[みが]いていますか。 歯[は]をきれいに 磨[みが]いなさい。 歯[は]を 磨[みが]いておけば。。。。 ひどく 歯[は]が 痛[いた]い。
\\	パーティー パーティーが大好き。 パーティーは終わった。 パーティーは失敗でした	
\\	失敗 
\\	パーティー パーティーが 大好[だいす]き。 パーティーは 終[お]わった。 パーティーは 失敗[しっぱい]でした
\\	灰皿 灰皿を下さい。 灰皿はすぐ手の届く所に置いてください。	
\\	手 
\\	届く 
\\	置 
\\	灰皿[はいざら] 灰皿[はいざら]を 下[くだ]さい。 灰皿[はいざら]はすぐ 手[て]の 届[とど]く 所[ところ]に 置[お]いてください。
\\	葉書 葉書をおくれ。 この店に葉書がたくさんある。 彼女が私に送ってきてくれた葉書には、動物のにおいが大嫌いだと書かれていました。	
\\	葉書[はがき] 葉書[はがき]をおくれ。 この 店[みせ]に 葉書[はがき]がたくさんある。 彼女[かのじょ]が 私[わたし]に 送[おく]ってきてくれた 葉書[はがき]には、 動物[どうぶつ]のにおいが 大嫌[だいきら]いだと 書[か]かれていました。
\\	はく 母は私がミニスカートをはくのを許さなかった。 私は日曜日はたいていジーンズをはいている。	
\\	許 
\\	はく 母[はは]は 私[わたし]がミニスカートをはくのを 許[ゆる]さなかった。 私[わたし]は 日曜日[にちようび]はたいていジーンズをはいている。
\\	箱 その箱はほとんど空です。 その箱はほとんどいっぱいでした。 この箱はあの箱の2倍の大きさだ。	
\\	箱[はこ] その 箱[はこ]はほとんど 空[そら]です。 その 箱[はこ]はほとんどいっぱいでした。 この 箱[はこ]はあの 箱[はこ]の 
\\	倍[ばい]の 大[おお]きさだ。
\\	橋 あの橋は長くはありません。 この橋は木で出来ている。 あの橋は何て長いんだろう!	
\\	橋[はし] あの 橋[はし]は 長[なが]くはありません。 この 橋[はし]は 木[き]で 出来[でき]ている。 あの 橋[はし]は 何[なに]て 長[なが]いんだろう!
\\	はし はしで食べるのは難しですか。	
\\	はし はしで 食[た]べるのは 難[むずか]しですか。
\\	始まる 始まった。 それはいつはじまりますか。 上映はいつから始まりますか。	
\\	始[はじ]まる 始[はじ]まった。 それはいつはじまりますか。 上映[じょうえい]はいつから 始[はじ]まりますか。
\\	始め  初め 初めまして。 私達はそれを初めから知っていた。 この方は、初めに神とともにおられた。	
\\	始[はじ]め   初[はじ]め 初[はじ]めまして。 私[わたし] 達[たち]はそれを 初[はじ]めから 知[し]っていた。 この 方[ほう]は、 初[はじ]めに 神[かみ]とともにおられた。
\\	初めて こちらは初めてですか。 そのとき初めてそれを悟った。 そんなふうに話しけられたのは初めてだった。	
\\	悟 
\\	初[はじ]めて こちらは 初[はじ]めてですか。 そのとき 初[はじ]めてそれを 悟[さと]った。 そんなふうに 話[はな]しけられたのは 初[はじ]めてだった。
\\	走る 走りませんか。 ここでは走るな。 私は走ることができる。	
\\	走[はし]る 走[はし]りませんか。 ここでは 走[はし]るな。 私[わたし]は 走[はし]ることができる。
\\	バス バスで行こう。 バスはどこですか? バスがおくれている。	
\\	バス バスで 行[い]こう。 バスはどこですか? バスがおくれている。
\\	バター バターをもう少し下さい。 バターをこちらに回してください。 バターは何から作られてるの?	
\\	回し 
\\	バター バターをもう 少[すこ]し 下[くだ]さい。 バターをこちらに 回[まわ]してください。 バターは 何[なに]から 作[つく]られてるの?
\\	二十歳 彼女は二十歳になったばりである。 僕はまだ二十歳になっていない。 彼女は二十歳のときに最初の子供を産んだ。	
\\	最初 
\\	産ん 
\\	二十歳[はたち] 彼女[かのじょ]は 二十歳[はたち]になったばりである。 僕[ぼく]はまだ 二十歳[はたち]になっていない。 彼女[かのじょ]は 二十歳[はたち]のときに 最初[さいしょ]の 子供[こども]を 産[う]んだ。
\\	働く 何のために働いているのですか。 なぜここで働いているのか。 人は働かねばならぬ。	
\\	働[はたら]く 何[なに]のために 働[はたら]いているのですか。 なぜここで 働[はたら]いているのか。 人[ひと]は 働[はたら]かねばならぬ。
\\	八 日本では十八歳になると車を運転できます。 コック・レーンの幽霊は英国の十八世紀の幽霊話のひとつです。	
\\	運転 
\\	幽霊 
\\	世紀 
\\	八[はち] 日本[にっぽん]では 十八[じゅうはち] 歳[さい]になると 車[くるま]を 運転[うんてん]できます。 コック・レーンの 幽霊[ゆうれい]は 英国[えいこく]の十八 世紀[せいき]の 幽霊[ゆうれい] 話[はなし]のひとつです。
\\	花 この花はあの花よりも美しい。 蜂は花から花に飛ぶ。 花がしぼんだ。	
\\	蜂 
\\	飛 
\\	しぼむ 
\\	花[はな] この 花[はな]はあの 花[はな]よりも 美[うつく]しい。 蜂[はち]は 花[はな]から 花[はな]に 飛[と]ぶ。 花[はな]がしぼんだ。
\\	鼻 象は鼻が長い。 鼻で息をしていてください。 口髭は鼻の下にはえる。	
\\	口髭 
\\	はえる 
\\	鼻[はな] 象[ぞう]は 鼻[はな]が 長[なが]い。 鼻[はな]で 息[いき]をしていてください。 口髭[くちひげ]は 鼻[はな]の 下[した]にはえる。
\\	話す 彼はそれについて話した。 どこまで話しましたか。 私は話すことができない。	
\\	話[はな]す 彼[かれ]はそれについて 話[はな]した。 どこまで 話[はな]しましたか。 私[わたし]は 話[はな]すことができない。
\\	話 その話は知っています。 その話、話したくない。 人と話ができません。 ちょっと話があるんだけど。	
\\	話[はなし] その 話[はなし]は 知[し]っています。 その 話[はなし]、 話[はな]したくない。 人[ひと]と 話[はなし]ができません。 ちょっと 話[はなし]があるんだけど。
\\	二十日 一日は二十四時間ある。	
\\	二十日[はつか] 一 日[にち]は 二十四[にじゅうよん] 時間[じかん]ある。
\\	早い 早ければ早いほど良い。 早く始めれば早く終わる。 早く用意して。	
\\	用意 
\\	早[はや]い 早[はや]ければ 早[はや]いほど 良[よ]い。 早[はや]く 始[はじ]めれば 早[はや]く 終[お]わる。 早[はや]く 用意[ようい]して。
\\	速い 何て速いんだろう。 リンは走るのが速い。 そんなに速く歩くな。	
\\	速[はや]い 何[なに]て 速[はや]いんだろう。 リンは 走[はし]るのが 速[はや]い。 そんなに 速[はや]く 歩[ある]くな。
\\	春 まだ春じゃない。 春が近づいてる。 冬の後に春が来る。	
\\	春[はる] まだ 春[はる]じゃない。 春[はる]が 近[ちか]づいてる。 冬[ふゆ]の 後[のち]に 春[はる]が 来[く]る。
\\	貼る 地図は壁に貼ってある。 彼は手紙に切手を貼った。 今その絵は壁に貼ってある。	
\\	貼[は]る 地図[ちず]は 壁[かべ]に 貼[は]ってある。 彼[かれ]は 手紙[てがみ]に 切手[きって]を 貼[は]った。 今[いま]その 絵[え]は 壁[かべ]に 貼[は]ってある。
\\	晴れ 晴れだと思うよ。 今日は晴れだ。 明日は晴れだろうか。	
\\	晴[は]れ 晴[は]れだと 思[おも]うよ。 今日[きょう]は 晴[は]れだ。 明日[あした]は 晴[は]れだろうか。
\\	晴れる 晴れないでしょう。 晴れていて暖かでした。 どうやら晴れるようですね。	
\\	晴[は]れる 晴[は]れないでしょう。 晴[は]れていて 暖[あたた]かでした。 どうやら 晴[は]れるようですね。
\\	半 半時間したら彼はここに来るでしょう。 郵便局は半マイル向こうにある。 事故の為私たちは半時間おくれました。	
\\	郵便局 
\\	マイル 
\\	為 
\\	半[はん] 半[はん] 時間[じかん]したら 彼[かれ]はここに 来[く]るでしょう。 郵便[ゆうびん] 局[きょく]は 半[はん]マイル 向[む]こうにある。 事故[じこ]の 為[ため] 私[わたし]たちは 半[はん] 時間[じかん]おくれました。
\\	晩 晩ご飯ですよ、お父さん。 その晩は月が出ていた。 一晩中泣いた。	
\\	晩[ばん] 晩[ばん]ご 飯[はん]ですよ、お 父[とう]さん。 その 晩[ばん]は 月[つき]が 出[で]ていた。 一 晩[ばん] 中[ちゅう] 泣[な]いた。
\\	ハンカチ このハンカチはいくらですか。 あんたは自分のハンカチを落としました。 彼女はハンカチをきちんと畳んだ。	
\\	きちん 
\\	畳 
\\	ハンカチ このハンカチはいくらですか。 あんたは 自分[じぶん]のハンカチを 落[お]としました。 彼女[かのじょ]はハンカチをきちんと 畳[たた]んだ。
\\	番号 番号を間違いのようですか。 7は縁起のいい番号だ。 水素の原子番号は1である。	
\\	水素 
\\	原子 番号 
\\	番号[ばんごう] 番号[ばんごう]を 間違[まちが]いのようですか。 7は 縁起[えんぎ]のいい 番号[ばんごう]だ。 水素[すいそ]の 原子[げんし] 番号[ばんごう]は1である。
\\	晩御飯 おいしい晩御飯をありがとうございました。 晩御飯を何にしたらいいかわからない。 彼女は彼のために特別な晩御飯を作りました。	
\\	晩[ばん] 御飯[ごはん] おいしい 晩[ばん] 御飯[ごはん]をありがとうございました。 晩[ばん] 御飯[ごはん]を 何[なに]にしたらいいかわからない。 彼女[かのじょ]は 彼[かれ]のために 特別[とくべつ]な 晩[ばん] 御飯[ごはん]を 作[つく]りました。
\\	半分 半分ようこせ。 彼女はリンゴを半分にきった。 仕事の半分はかたづいた。	
\\	半分[はんぶん] 半分[はんぶん]ようこせ。 彼女[かのじょ]はリンゴを 半分[はんぶん]にきった。 仕事[しごと]の 半分[はんぶん]はかたづいた。
\\	東 私の部屋は東向きです。 太陽は常に東から昇る。 動物園は町の東にあります。	
\\	常に 
\\	昇 
\\	東[ひがし] 私[わたし]の 部屋[へや]は 東[ひがし] 向[む]きです。 太陽[たいよう]は 常[つね]に 東[ひがし]から 昇[のぼ]る。 動物[どうぶつ] 園[えん]は 町[まち]の 東[ひがし]にあります。
\\	引く 直線を引きなさい。 カートンを引いてもいいですか。 彼女は風邪を引きやすかった。	
\\	直線 
\\	引[ひ]く 直線[ちょくせん]を 引[ひ]きなさい。 カートンを 引[ひ]いてもいいですか。 彼女[かのじょ]は 風邪[かぜ]を 引[ひ]きやすかった。
\\	弾く ところで、バイオリンは弾きますか。 彼はギターを弾き、彼女はピアノを弾いた。	
\\	弾[ひ]く ところで、バイオリンは 弾[ひ]きますか。 彼[かれ]はギターを 弾[ひ]き、 彼女[かのじょ]はピアノを 弾[ひ]いた。
\\	低い 私は背が低すぎる。 私はあなたよりも背が低い。 トムは声が低い。	
\\	低[ひく]い 私[わたし]は 背[せ]が 低[ひく]すぎる。 私[わたし]はあなたよりも 背[せ]が 低[ひく]い。 トムは 声[こえ]が 低[ひく]い。
\\	飛行機 飛行機が見えた。 この飛行機は彼のものです。 この飛行機は巨大だ。	
\\	巨大 
\\	飛行機[ひこうき] 飛行機[ひこうき]が 見[み]えた。 この 飛行機[ひこうき]は 彼[かれ]のものです。 この 飛行機[ひこうき]は 巨大[きょだい]だ。
\\	左 左へ曲がりなさい。 その机を左へ移動してください。 左へ曲がればバス停があります。	
\\	移動 
\\	バス停 
\\	左[ひだり] 左[ひだり]へ 曲[ま]がりなさい。 その 机[つくえ]を 左[ひだり]へ 移動[いどう]してください。 左[ひだり]へ 曲[ま]がれば バス停[ばすてい]があります。
\\	一月 一月は雪が降る。 一月は年の一番目の月です。 一月に彼に会いました。	
\\	一月[ひとつき] 一月[ひとつき]は 雪[ゆき]が 降[ふ]る。 一月[ひとつき]は 年[とし]の一 番目[ばんめ]の 月[げつ]です。 一月[ひとつき]に 彼[かれ]に 会[あ]いました。
\\	一人 選手一人ひとりが最善を尽くした。 私は兄が一人、妹が一人います。 人っ子一人見えなかった。	
\\	選手 
\\	最善を尽くした 
\\	一人[ひとり] 選手[せんしゅ] 一人[ひとり]ひとりが 最善[さいぜん]を 尽[つ]くした。 私[わたし]は 兄[あに]が 一人[ひとり]、 妹[いもうと]が 一人[ひとり]います。 人[ひと]っ 子[こ] 一人[ひとり] 見[み]えなかった。
\\	暇 暇ですよ。 今晩暇かな? ところで午後お暇ですか。	
\\	暇[ひま] 暇[ひま]ですよ。 今晩[こんばん] 暇[ひま]かな? ところで 午後[ごご]お 暇[ひま]ですか。
\\	百 百セントは一ドルになります。 うちの息子は百まで数えられる。 何百という人々がその現場にいた。	
\\	数 
\\	人々 
\\	現場 
\\	百[ひゃく] 百[ひゃく]セントは 一[いち]ドルになります。 うちの 息子[むすこ]は 百[ひゃく]まで 数[かぞ]えられる。 何[なん] 百[ひゃく]という 人々[ひとびと]がその 現場[げんば]にいた。
\\	病院 ここは病院だ。 病院は何処ですか? 私は病院で働いています。	
\\	病院[びょういん] ここは 病院[びょういん]だ。 病院[びょういん]は 何処[どこ]ですか? 私[わたし]は 病院[びょういん]で 働[はたら]いています。
\\	病気 病気だったんだよ。 何の病気ですか。 アンは病気のようだ。	
\\	病気[びょうき] 病気[びょうき]だったんだよ。 何[なに]の 病気[びょうき]ですか。 アンは 病気[びょうき]のようだ。
\\	昼 昼ごはんを食べています。 昼と夜が交互にくる。 今は昼の4時半だよ。	
\\	交互 
\\	昼[ひる] 昼[ひる]ごはんを 食[た]べています。 昼[ひる]と 夜[よる]が 交互[こうご]にくる。 今[いま]は 昼[ひる]の 
\\	時半[じはん]だよ。
\\	昼御飯	
\\	昼[ひる] 御飯[ごはん]
\\	広い どのくらい広い? トムは顔が広い。 その部屋は広くて明るい。	
\\	広[ひろ]い どのくらい 広[ひろ]い? トムは 顔[かお]が 広[ひろ]い。 その 部屋[へや]は 広[ひろ]くて 明[あか]るい。
\\	太い 彼らの家畜は全部太いです。 俺、指が太いから、小さなキーポードじゃだめなんだよ。	
\\	家畜 
\\	太[ふと]い 彼[かれ]らの 家畜[かちく]は 全部[ぜんぶ] 太[ふと]いです。 俺[おれ]、 指[ゆび]が 太[ふと]いから、 小[ちい]さなキーポードじゃだめなんだよ。
\\	冬 冬は去りました。 もうすぐ冬だ。 ある冬の日、彼女と出会った。	
\\	去り 
\\	冬[ふゆ] 冬[ふゆ]は 去[さ]りました。 もうすぐ 冬[ふゆ]だ。 ある 冬[ふゆ]の 日[ひ]、 彼女[かのじょ]と 出会[であ]った。
\\	降る 「雨が降るでしょうか」「降らないでしょう」 雪が降る。 雹が降っています。	
\\	雹 
\\	降[ふ]る 
\\	雨[あめ]が 降[ふ]るでしょうか」
\\	降[ふ]らないでしょう」 雪[ゆき]が 降[ふ]る。 雹[ひょう]が 降[ふ]っています。
\\	古い この本は古い。 ローマは古い街だ。 あれは古いカメラだ。	
\\	街 
\\	古[ふる]い この 本[ほん]は 古[ふる]い。 ローマは 古[ふる]い 街[まち]だ。 あれは 古[ふる]いカメラだ。
\\	ふろ/風呂 私は毎日1回おふろに入ります。	
\\	ふろ/ 風呂[ふろ] 私[わたし]は 毎日[まいにち] 
\\	回[かい]おふろに 入[はい]ります。
\\	文章 父はフランス語の文書を日本語に翻訳した。 注文書にサイズ、色、スタイルを記入して下さい。	
\\	翻訳 
\\	注 
\\	スタイル 
\\	記入 
\\	文章[ぶんしょう] 父[ちち]は フランス語[ふらんすご]の 文書[ぶんしょ]を 日本語[にほんご]に 翻訳[ほんやく]した。 注[ちゅう] 文書[ぶんしょ]にサイズ、 色[いろ]、スタイルを 記入[きにゅう]して 下[くだ]さい。
\\	ページ 1ページ足りない。 ページの下を呼んでごらん。 ページをめくって下さい。	
\\	足り 
\\	""足りない
\\	呼ぶ 
\\	めくる 
\\	ページ 1ページ 足[た]りない。 ページの 下[した]を 呼[よ]んでごらん。 ページをめくって 下[くだ]さい。
\\	下手 彼は字が下手だ。 私の妻は運転が下手だ。 彼は名前を覚えるのが下手だ。	
\\	下手[へた] 彼[かれ]は 字[じ]が 下手[へた]だ。 私[わたし]の 妻[つま]は 運転[うんてん]が 下手[へた]だ。 彼[かれ]は 名前[なまえ]を 覚[おぼ]えるのが 下手[へた]だ。
\\	ベッド 彼女はベッドに横になった。 彼はベッドに腰掛けた 子供がベッドから落ちた。	
\\	腰掛けた 
\\	ベッド 彼女[かのじょ]はベッドに 横[よこ]になった。 彼[かれ]はベッドに 腰掛[こしか]けた 子供[こども]がベッドから 落[お]ちた。
\\	ペット 何がペットは飼ってる? メグはペットに猫を飼っている。 ペットのいない生活なんて考えにくい。	
\\	飼 
\\	生活 
\\	ペット 何[なに]がペットは 飼[か]ってる? メグはペットに 猫[ねこ]を 飼[か]っている。 ペットのいない 生活[せいかつ]なんて 考[かんが]えにくい。
\\	部屋 部屋を出ろ。 これはあなたの部屋ですか。 部屋に誰もいない。	
\\	部屋[へや] 部屋[へや]を 出[しゅつ]ろ。 これはあなたの 部屋[へや]ですか。 部屋[へや]に 誰[だれ]もいない。
\\	辺 この辺臭うんだけど。 その辺にホチキスない? 私はこの辺の地理に明るい。	
\\	臭う 
\\	地理 
\\	辺[あたり] この 辺[あたり] 臭[にお]うんだけど。 その 辺[あたり]にホチキスない? 私[わたし]はこの 辺[あたり]の 地理[ちり]に 明[あか]るい。
\\	ペン これはペンです。 このペンは誰のですか。 例えば、これはペンです。	
\\	例えば 
\\	ペン これはペンです。 このペンは 誰[だれ]のですか。 例[たと]えば、これはペンです。
\\	勉強する 勉強するな。 ここで勉強しなさい。 英語を勉強しましょう。	
\\	勉強[べんきょう]する 勉強[べんきょう]するな。 ここで 勉強[べんきょう]しなさい。 英語[えいご]を 勉強[べんきょう]しましょう。
\\	便利 このナイフは私にはとても便利でした。 その場所は泳ぐのに便利だ。 彼の事務所は駅のすぐ近くで便利だ。	
\\	場所 
\\	事務所 
\\	便利[べんり] このナイフは 私[わたし]にはとても 便利[べんり]でした。 その 場所[ばしょ]は 泳[およ]ぐのに 便利[べんり]だ。 彼[かれ]の 事務所[じむしょ]は 駅[えき]のすぐ 近[ちか]くで 便利[べんり]だ。
\\	帽子 この帽子は私のものです。 帽子がふきとばされた。 あなたの新し帽子、なんて素敵な帽子なんでしょう。	
\\	素敵な 
\\	帽子[ぼうし] この 帽子[ぼうし]は 私[わたし]のものです。 帽子[ぼうし]がふきとばされた。 あなたの 新[あたら]し 帽子[ぼうし]、なんて 素敵[すてき]な 帽子[ぼうし]なんでしょう。
\\	ボールペン 昨日なくしたボールペンは新品だった。 「何か書くものをくれ」「このボールペンでいいですか」	
\\	ボールペン 昨日[きのう]なくしたボールペンは 新品[しんぴん]だった。 
\\	何[なに]か 書[か]くものをくれ」「このボールペンでいいですか」
\\	ほか/他 自分は他人である。 他に何か。 ほかのものがありますか。	
\\	ほか/ 他[た] 自分[じぶん]は 他人[たにん]である。 他[た]に 何[なに]か。 ほかのものがありますか。
\\	ポケット このコートにはポケットがない かれはポケット手を突っ込んだ。 かれはポケットから何かを取り出した。	
\\	突っ込 
\\	取り出 
\\	ポケット このコートにはポケットがない かれはポケット 手[て]を 突っ込[つっこ]んだ。 かれはポケットから 何[なに]かを 取り出[とりだ]した。
\\	欲しい 水が欲しい。 お金が欲しい。 それが是非欲しい。	
\\	是非 
\\	欲[ほ]しい 水[すい]が 欲[ほ]しい。 お 金[かね]が 欲[ほ]しい。 それが 是非[ぜひ] 欲[ほ]しい。
\\	ポスト 私はポストを探してあたりを見回した。 けさ祖母がその手紙をポストに入れてくれました。 私はあなたにこの手紙をポストにいれてもらいたいのです。	
\\	ポスト 私[わたし]はポストを 探[さが]してあたりを 見回[みまわ]した。 けさ 祖母[そぼ]がその 手紙[てがみ]をポストに 入[い]れてくれました。 私[わたし]はあなたにこの 手紙[てがみ]をポストにいれてもらいたいのです。
\\	細い 「痩せている」と言うよりも「細い」と言った方が失礼に当たりません。 光ファイバーケーブルは人間の髪の毛ほどの細さの小さなガラスでできている。	
\\	光ファイバー 
\\	小さな 
\\	ガラス 
\\	細[ほそ]い 
\\	痩[や]せている」と 言[い]うよりも
\\	細[ほそ]い」と 言[い]った 方[ほう]が 失礼[しつれい]に 当[あ]たりません。 光[ひかり]ファイバーケーブルは 人間[にんげん]の 髪の毛[かみのけ]ほどの 細[ほそ]さの 小[ちい]さなガラスでできている。
\\	ボタン ボタンに触れるな。 このボタンを決して押してはいけない。 ブラウスのボタンをはずしてください。	
\\	決して 
\\	押し 
\\	ボタン ボタンに 触[ふ]れるな。 このボタンを 決[けっ]して 押[お]してはいけない。 ブラウスのボタンをはずしてください。
\\	ホテル このホテルはあのホテルより良い。 ホテルに戻ろう。 ホテルが全焼した。	
\\	全焼した 
\\	ホテル このホテルはあのホテルより 良[よ]い。 ホテルに 戻[もど]ろう。 ホテルが 全焼[ぜんしょう]した。
\\	本 この本はとても新しです。 この本を私に買って下さい。 本をたくさん買った。 その本はとても小さい。	
\\	本[ほん] この 本[ほん]はとても 新[あたら]しです。 この 本[ほん]を 私[わたし]に 買[か]って 下[くだ]さい。 本[ほん]をたくさん 買[か]った。 その 本[ほん]はとても 小[ちい]さい。
\\	本棚 彼女は本棚にある数冊の本を彼に見せた。	
\\	本棚[ほんだな] 彼女[かのじょ]は 本棚[ほんだな]にある 数[すう] 冊[さつ]の 本[ほん]を 彼[かれ]に 見[み]せた。
\\	本当 実はそれ、本当なんだ。 「彼の話は本当かな」「本当ではないでしょう」 本当のことを教えて。	
\\	本当[ほんとう] 実[じつ]はそれ、 本当[ほんとう]なんだ。 
\\	彼[かれ]の 話[はなし]は 本当[ほんとう]かな」
\\	本当[ほんとう]ではないでしょう」 本当[ほんとう]のことを 教[おし]えて。
\\	毎朝 毎朝買い物をします。 毎朝、ジョギングすることにした。 わたしは毎朝髭を剃る。	
\\	髭 
\\	剃る 
\\	毎朝[まいあさ] 毎朝[まいあさ] 買い物[かいもの]をします。 毎朝[まいあさ]、ジョギングすることにした。 わたしは 毎朝[まいあさ] 髭[ひげ]を 剃[そ]る。
\\	毎月   まいげつ/まいつき 毎月のがす代はいくらですか。 彼は毎月お金を貯金している。 彼は毎月お金をいくら稼ぎますか。	
\\	稼ぎ 
\\	毎月[まいつき]   まいげつ/まいつき 毎月[まいつき]のがす 代[だい]はいくらですか。 彼[かれ]は 毎月[まいつき]お 金[かね]を 貯金[ちょきん]している。 彼[かれ]は 毎月[まいつき]お 金[かね]をいくら 稼[かせ]ぎますか。
\\	毎週 彼は毎週バイクを洗います。 私は毎週日曜日にテニスをする。 毎週釣りに行く。	
\\	釣り 
\\	毎週[まいしゅう] 彼[かれ]は 毎週[まいしゅう]バイクを 洗[あら]います。 私[わたし]は 毎週[まいしゅう] 日曜日[にちようび]にテニスをする。 毎週[まいしゅう] 釣[つ]りに 行[い]く。
\\	毎日 毎日走っているのですか。 私は毎日泳ぎます。 私は毎日英語を話します。	
\\	毎日[まいにち] 毎日[まいにち] 走[はし]っているのですか。 私[わたし]は 毎日[まいにち] 泳[およ]ぎます。 私[わたし]は 毎日[まいにち] 英語[えいご]を 話[はな]します。
\\	毎年   まいねん/まいとし 毎年行きます。 その会は毎年開かれる。 私たちは毎年夏にキャンプに出掛けます。	
\\	毎年[まいとし]   まいねん/まいとし 毎年[まいとし] 行[い]きます。 その 会[かい]は 毎年[まいとし] 開[ひら]かれる。 私[わたし]たちは 毎年[まいとし] 夏[なつ]にキャンプに 出掛[でか]けます。
\\	毎晩 私は毎晩家にいます。 かれは毎晩彼女に電話する。 私は毎晩ラジオを聞きます。	
\\	毎晩[まいばん] 私[わたし]は 毎晩[まいばん] 家[か]にいます。 かれは 毎晩[まいばん] 彼女[かのじょ]に 電話[でんわ]する。 私[わたし]は 毎晩[まいばん]ラジオを 聞[き]きます。
\\	前 俺の前でもったいをつけるなよ。 1時間前に出たよ。 彼は私の前を走っている。	
\\	前[まえ] 俺[おれ]の 前[まえ]でもったいをつけるなよ。 
\\	時間[じかん] 前[まえ]に 出[で]たよ。 彼[かれ]は 私[わたし]の 前[まえ]を 走[はし]っている。
\\	曲る 左へ曲がりなさい。 その交差点を右に曲がってください。 右に曲がると、博物館に出ますよ。	
\\	博物館 
\\	曲[まが]る 左[ひだり]へ 曲[ま]がりなさい。 その 交差点[こうさてん]を 右[みぎ]に 曲[ま]がってください。 右[みぎ]に 曲[ま]がると、 博物館[はくぶつかん]に 出[で]ますよ。
\\	まずい そのミルクはまずかった。 空腹にまずいものなし。 あ!まずい!ガソリンが切れてきた。	
\\	空腹 
\\	ガソリン 
\\	まずい そのミルクはまずかった。 空腹[くうふく]にまずいものなし。 あ!まずい!ガソリンが 切[き]れてきた。
\\	また	
\\	また
\\	まだ かれはまだここにいる。 まだ分かりません。 まだ眠いの?	
\\	まだ かれはまだここにいる。 まだ 分[わ]かりません。 まだ 眠[ねむ]いの?
\\	町 なんて素敵な町なんだ! 町は眠っていた。 町を案内しましょう。	
\\	案内 
\\	町[まち] なんて 素敵[すてき]な 町[まち]なんだ! 町[まち]は 眠[ねむ]っていた。 町[まち]を 案内[あんない]しましょう。
\\	待つ 待って! お待たせしました。 向こうで待っていてね。 ちょっと待ってね。	
\\	待[ま]つ 待[ま]って! お 待[ま]たせしました。 向[む]こうで 待[ま]っていてね。 ちょっと 待[ま]ってね。
\\	まっすぐ 
\\	真っすぐ まっすぐ前方を見なさい。 マユコはまっすぐ帰宅した。 この通りに沿って、まっすぐ行ってください。	
\\	帰宅 
\\	沿って 
\\	まっすぐ 
\\	真[ま]っすぐ まっすぐ 前方[ぜんぽう]を 見[み]なさい。 マユコはまっすぐ 帰宅[きたく]した。 この 通[とお]りに 沿[そ]って、まっすぐ 行[い]ってください。
\\	マッチ 家々がマッチ箱に見えた。 そのラジオはマッチ箱の大きさしかない。	
\\	マッチ 家[か]々がマッチ 箱[ばこ]に 見[み]えた。 そのラジオはマッチ 箱[ばこ]の 大[おお]きさしかない。
\\	窓 窓を閉めて。 窓を開けて。 窓が左手にあります。 トムは窓を割った。	
\\	割った。
\\	窓[まど] 窓[まど]を 閉[し]めて。 窓[まど]を 開[あ]けて。 窓[まど]が 左手[ひだりて]にあります。 トムは 窓[まど]を 割[わ]った。
\\	丸い/円い 地球は丸い。 私たちは丸くなって座った。 円を描いてくれ。 完全な円にはもう見えないのである。	
\\	地球 
\\	描い 
\\	完全な 
\\	丸[まる]い/ 円[まる]い 地球[ちきゅう]は 丸[まる]い。 私[わたし]たちは 丸[まる]くなって 座[すわ]った。 円[まる]を 描[えが]いてくれ。 完全[かんぜん]な 円[まる]にはもう 見[み]えないのである。
\\	万 これは100万円の価値がある。 1万円でたり? それは1万円ぐらいするだろう。	
\\	価値 
\\	万[まん] これは 
\\	万[まん] 円[えん]の 価値[かち]がある。 
\\	万[まん] 円[えん]でたり? それは1 万[まん] 円[えん]ぐらいするだろう。
\\	万年筆	
\\	万年筆[まんねんひつ]
\\	磨く トムは歯を磨いている。 私は床と家具を磨いた。 君は靴を磨いてしまいましたね?	
\\	床 
\\	家具 
\\	磨[みが]く トムは 歯[は]を 磨[みが]いている。 私[わたし]は 床[ゆか]と 家具[かぐ]を 磨[みが]いた。 君[くん]は 靴[くつ]を 磨[みが]いてしまいましたね?
\\	右 右の袖をあげてください。 右に曲がってください。 向かって右が男子トイレ、左が女子トイレです。	
\\	袖 
\\	右[みぎ] 右[みぎ]の 袖[そで]をあげてください。 右[みぎ]に 曲[ま]がってください。 向[む]かって 右[みぎ]が 男子[だんし]トイレ、 左[ひだり]が 女子[じょし]トイレです。
\\	短い これは短すぎます。 彼女の髪は短い。 私は短い髪が好きです。	
\\	短[みじか]い これは 短[みじか]すぎます。 彼女[かのじょ]の 髪[かみ]は 短[みじか]い。 私[わたし]は 短[みじか]い 髪[かみ]が 好[す]きです。
\\	水 水が欲しい。 ここに水が少しある。 水はほとんど残っていない。 水は液体である	
\\	残って 
\\	液体 
\\	水[みず] 水[みず]が 欲[ほ]しい。 ここに 水[みず]が 少[すこ]しある。 水[みず]はほとんど 残[のこ]っていない。 水[みず]は 液体[えきたい]である
\\	店 この店にはバッグがありますか? その店では砂糖を売っている。 店はたたまれ、それっきりだった。	
\\	砂糖 
\\	店[みせ] この 店[みせ]にはバッグがありますか? その 店[みせ]では 砂糖[さとう]を 売[う]っている。 店[てん]はたたまれ、それっきりだった。
\\	見せる 見せてあげる。 写真見せて。 メニューをみせていただけませんか?	
\\	見[み]せる 見[み]せてあげる。 写真[しゃしん] 見[み]せて。 メニューをみせていただけませんか?
\\	道 道に迷ったんですか。 道を渡ろう。 その道は行き止まりだった。	
\\	迷 
\\	渡 
\\	行き止まり 
\\	道[みち] 道[みち]に 迷[まよ]ったんですか。 道[みち]を 渡[わた]ろう。 その 道[みち]は 行き止[ゆきど]まりだった。
\\	三日 それから三日後に彼は死んだ。 日本人は一日三食です。 彼は三日後に帰ってきた。	
\\	三日[みっか] それから 三日[みっか] 後[ご]に 彼[かれ]は 死[し]んだ。 日本人[にっぽんじん]は 一日[いちにち] 三食[さんしょく]です。 彼[かれ]は 三日[みっか] 後[ご]に 帰[かえ]ってきた。
\\	三つ これを三つください。 彼は三つ食べて、あと三つ食べられるといいました。 私の部屋には窓が三つある。	
\\	三[みっ]つ これを 三[みっ]つください。 彼[かれ]は 三[みっ]つ 食[た]べて、あと 三[みっ]つ 食[た]べられるといいました。 私[わたし]の 部屋[へや]には 窓[まど]が 三[みっ]つある。
\\	緑 テーブルは緑です。 その山は緑です。 私は柵を緑に塗った。	
\\	柵 
\\	塗った 
\\	緑[みどり] テーブルは 緑[みどり]です。 その 山[やま]は 緑[みどり]です。 私[わたし]は 柵[しがらみ]を 緑[みどり]に 塗[ぬ]った。
\\	皆さん。 皆が笑った。 皆が尊敬しています。 皆は幸せになりたい。	
\\	尊敬 
\\	幸せ 
\\	皆[みな]さん。 皆[みな]が 笑[わら]った。 皆[みな]が 尊敬[そんけい]しています。 皆[みな]は 幸[しあわ]せになりたい。
\\	南 鳥は南へ飛んで行った。 南ターミナルはどこですか? フランスは英国の南にある。 その島は日本の南のほうにある。	
\\	南[みなみ] 鳥[とり]は 南[みなみ]へ 飛[と]んで 行[い]った。 南[みなみ]ターミナルはどこですか? フランスは 英国[えいこく]の 南[みなみ]にある。 その 島[しま]は 日本[にっぽん]の 南[みなみ]のほうにある。
\\	耳 壁に耳あり。 耳がかゆいのです。 全身すべてが耳だ。	
\\	かゆい 
\\	全身 
\\	耳[みみ] 壁[かべ]に 耳[みみ]あり。 耳[みみ]がかゆいのです。 全身[ぜんしん]すべてが 耳[みみ]だ。
\\	見る 
\\	観る 実は、この映画を観たことがある。 映画を観るのが好きです。 少なくとも、具合いが悪いということは、家に居て映画を観るための完璧な言い訳になる。	
\\	少なくとも 
\\	具合い 
\\	居 
\\	完璧な 
\\	言い訳 
\\	見[み]る 
\\	観[み]る 実[じつ]は、この 映画[えいが]を 観[み]たことがある。 映画[えいが]を 観[み]るのが 好[す]きです。 少[すく]なくとも、 具合[ぐあ]いが 悪[わる]いということは、 家[いえ]に 居[い]て 映画[えいが]を 観[み]るための 完璧[かんぺき]な 言い訳[いいわけ]になる。
\\	みんな/皆 みんなはどこ? みんな嫌がっているのに。 みんなで祈ろう。	
\\	嫌 
\\	祈ろ 
\\	みんな/ 皆[かい] みんなはどこ? みんな 嫌[いや]がっているのに。 みんなで 祈[いの]ろう。
\\	六日 六日間戦争	
\\	六日[むいか] 六日間[むいかかん] 戦争[せんそう]
\\	向こう 向こうで待っていてね。 あなたのは向こうにあります。 向こうに立っている女の子は誰ですか?	
\\	向[む]こう 向[む]こうで 待[ま]っていてね。 あなたのは 向[む]こうにあります。 向[む]こうに 立[た]っている 女の子[おんなのこ]は 誰[だれ]ですか?
\\	難しい それは君が考えるほど難しくない。 我々が彼を説得することは難しい。 かれにとって人前で話すのは難しいだろう。	
\\	説得 
\\	人前で 
\\	難[むずか]しい それは 君[きみ]が 考[かんが]えるほど 難[むずか]しくない。 我々[われわれ]が 彼[かれ]を 説得[せっとく]することは 難[むずか]しい。 かれにとって 人前[ひとまえ]で 話[はな]すのは 難[むずか]しいだろう。
\\	六つ	
\\	六[むっ]つ
\\	村 世界は小さな村である。 彼は故郷の村に帰った。 教会は村の中心部にある。	
\\	故郷 
\\	教会 
\\	中心部 
\\	村[むら] 世界[せかい]は 小[ちい]さな 村[むら]である。 彼[かれ]は 故郷[こきょう]の 村[むら]に 帰[かえ]った。 教会[きょうかい]は 村[むら]の 中心[ちゅうしん] 部[ぶ]にある。
\\	目 目がかゆいです。 目が疲れました。 コンタクトすると目が乾いた感じになって、目が赤くなります。	
\\	乾い 
\\	目[め] 目[め]がかゆいです。 目[め]が 疲[つか]れました。 コンタクトすると 目[め]が 乾[かわ]いた 感[かん]じになって、 目[め]が 赤[あか]くなります。
\\	メートル その山は海抜2千メートルだ。 駅は100メートル離れたところにある。 湖は約25メートルの深さだ。	
\\	海抜 
\\	離れ 
\\	湖 
\\	約 
\\	深さ 
\\	メートル その 山[やま]は 海抜[かいばつ]2 千[せん]メートルだ。 駅[えき]は100メートル 離[はな]れたところにある。 湖[みずうみ]は 約[やく]25メートルの 深[ふか]さだ。
\\	眼鏡 その眼鏡はいくらだったの。 私の眼鏡はどこですか? トムは眼鏡を拭いた	
\\	拭いた 
\\	眼鏡[めがね] その 眼鏡[めがね]はいくらだったの。 私[わたし]の 眼鏡[めがね]はどこですか? トムは 眼鏡[めがね]を 拭[ふ]いた
\\	もう 彼はもうここに来ている。 彼はもう出たよ。 もう夜中を過ぎた。	
\\	夜中 
\\	過ぎ 
\\	もう 彼[かれ]はもうここに 来[き]ている。 彼[かれ]はもう 出[で]たよ。 もう 夜中[やちゅう]を 過[す]ぎた。
\\	もう一度 さあ、もう一度。 もう一度お目にかかりたいものです。 もう一度それをやってみた。	
\\	もう 一度[いちど] さあ、もう 一度[いちど]。 もう 一度[いちど]お 目[め]にかかりたいものです。 もう 一度[いちど]それをやってみた。
\\	フィルム フィルムがなくなった。 このカメラにはフィルムが入っていない。	
\\	フィルム フィルムがなくなった。 このカメラにはフィルムが 入[はい]っていない。
\\	封筒 封筒がいる。 彼は封筒を切って開けた。 封筒に切手を貼りなさい。	
\\	貼り 
\\	封筒[ふうとう] 封筒[ふうとう]がいる。 彼[かれ]は 封筒[ふうとう]を 切[き]って 開[あ]けた。 封筒[ふうとう]に 切手[きって]を 貼[は]りなさい。
\\	プール 彼は今プールで泳いでいる。 トムはプールに飛び込んだ。 私はプールで泳ぐのは好きではない。	
\\	飛び込ん 
\\	プール 彼[かれ]は 今[こん]プールで 泳[およ]いでいる。 トムはプールに 飛び込[とびこ]んだ。 私[わたし]はプールで 泳[およ]ぐのは 好[す]きではない。
\\	フォーク フォークがありません。 フォークをいただけますか。 フォークがテーブルから落ちた。	
\\	フォーク フォークがありません。 フォークをいただけますか。 フォークがテーブルから 落[お]ちた。
\\	吹く 風が吹く。 強風が吹いていた。 強い風が吹いていた。	
\\	強風 
\\	吹[ふ]く 風[かぜ]が 吹[ふ]く。 強風[きょうふう]が 吹[ふ]いていた。 強[つよ]い 風[かぜ]が 吹[ふ]いていた。
\\	服 この服よりもあの服が好きです。 私はあの服よりむしろこの服の方が好きだ。	
\\	服[ふく] この 服[ふく]よりもあの 服[ふく]が 好[す]きです。 私[わたし]はあの 服[ふく]よりむしろこの 服[ふく]の 方[ほう]が 好[す]きだ。
\\	二つ 猫は、耳が二つある。 火星は二つ衛星がある。 彼女はケーキを二つに切った。	
\\	火星 
\\	衛星
\\	二[ふた]つ 猫[ねこ]は、 耳[みみ]が 二[ふた]つある。 火星[かせい]は 二[ふた]つ 衛星[えいせい]がある。 彼女[かのじょ]はケーキを 二[ふた]つに 切[き]った。
\\	豚肉 この子は豚じゃないわよ。猿よ 木から落ちるリンゴはみな豚に食べられてしまう。	
\\	豚肉[ぶたにく] この 子[こ]は 豚[ぶた]じゃないわよ。 猿[さる]よ 木[き]から 落[お]ちるリンゴはみな 豚[ぶた]に 食[た]べられてしまう。
\\	二人 二人ともありがとう。 ジョンには二人の息子がいます。 彼らには娘が二人います。	
\\	二人[ふたり] 二 人[にん]ともありがとう。 ジョンには二 人[にん]の 息子[むすこ]がいます。 彼[かれ]らには 娘[むすめ]が二 人[にん]います。
\\	二日 このミルクは二日はもつ。 彼は一日か二日で帰ります。 私は二日前にカメラを買った。	
\\	二日[ふつか] このミルクは 二日[ふつか]はもつ。 彼[かれ]は一 日[にち]か 二日[ふつか]で 帰[かえ]ります。 私[わたし]は 二日[ふつか] 前[まえ]にカメラを 買[か]った。
\\	木曜日 昨日は木曜日だった。 先週の木曜日からずっと雨が降っている。 彼女の母親は先週の木曜日から病気です。	
\\	母親 
\\	木曜日[もくようび] 昨日[きのう]は 木曜日[もくようび]だった。 先週[せんしゅう]の 木曜日[もくようび]からずっと 雨[あめ]が 降[ふ]っている。 彼女[かのじょ]の 母親[ははおや]は 先週[せんしゅう]の 木曜日[もくようび]から 病気[びょうき]です。
\\	持つ これは持っていたいのだ。 持てば持つほどもっと欲しくなる。 持てば持つほど、欲張りになる。	
\\	欲張り 
\\	持[も]つ これは 持[も]っていたいのだ。 持[も]てば 持[も]つほどもっと 欲[ほ]しくなる。 持[も]てば 持[も]つほど、 欲張[よくば]りになる。
\\	もっと もっと大きくして。 もっと安くなりますか。 もっと時間があったら、もっとうまくやることが出来たのだが。	
\\	もっと もっと 大[おお]きくして。 もっと 安[やす]くなりますか。 もっと 時間[じかん]があったら、もっとうまくやることが 出来[でき]たのだが。
\\	物 これらの物は私の物ではない。 それに匹敵する物がない。 申告する物はありますか	
\\	匹敵 
\\	申告 
\\	物[もの] これらの 物[もの]は 私[わたし]の 物[もの]ではない。 それに 匹敵[ひってき]する 物[もの]がない。 申告[しんこく]する 物[もの]はありますか
\\	門 自動車を門の外に待ったせてある。 狭き門より入れ。 この門にはペンキを塗る必要がある。	
\\	ペンキ 
\\	塗る 
\\	必要 
\\	門[もん] 自動車[じどうしゃ]を 門[もん]の 外[そと]に 待[ま]ったせてある。 狭[せま]き 門[もん]より 入[い]れ。 この 門[もん]にはペンキを 塗[ぬ]る 必要[ひつよう]がある。
\\	問題 それはほとんど問題にならない。 問題でも起きたのか。 その問題はこれで終わり。	
\\	問題[もんだい] それはほとんど 問題[もんだい]にならない。 問題[もんだい]でも 起[お]きたのか。 その 問題[もんだい]はこれで 終[お]わり。
\\	八百屋	
\\	八百屋[やおや]
\\	野菜 野菜を食べたら? 野菜をたくさん食べなさい。 野菜の値段が下がっている。	
\\	値段 
\\	下がる 
\\	野菜[やさい] 野菜[やさい]を 食[た]べたら? 野菜[やさい]をたくさん 食[た]べなさい。 野菜[やさい]の 値段[ねだん]が 下[さ]がっている。
\\	易しい その本は易しい。 第2課は易しい。 この問題を解くのは私にとって易しい。	
\\	第2 
\\	課 
\\	問題 
\\	解く 
\\	易[やさ]しい その 本[ほん]は 易[やさ]しい。 第[だい]2 課[か]は 易[やさ]しい。 この 問題[もんだい]を 解[と]くのは 私[わたし]にとって 易[やさ]しい。
\\	安い 安いようでした。 とても安かったの。 物価が安くなるだろう。	
\\	安[やす]い 安[やす]いようでした。 とても 安[やす]かったの。 物価[ぶっか]が 安[やす]くなるだろう。
\\	休み 休みらしい休みは久しく取っていない。 明日は休みだ。 休みは終わった。	
\\	久し 
\\	取 
\\	休[やす]み 休[やす]みらしい 休[やす]みは 久[ひさ]しく 取[と]っていない。 明日[あした]は 休[やす]みだ。 休[やす]みは 終[お]わった。
\\	八つ	
\\	八[やっ]つ
\\	山 その山は緑です。 私は山にいました。 その山は人気がある。 ヒマラヤの山はアンデスの山よりも高い。	
\\	山[やま] その 山[やま]は 緑[みどり]です。 私[わたし]は 山[やま]にいました。 その 山[やま]は 人気[にんき]がある。 ヒマラヤの 山[やま]はアンデスの 山[やま]よりも 高[たか]い。
\\	やる やったことあるの? 今日やろうとしたことは全部やった。 それをやっちゃだめだよ。 その仕事をやろう。	
\\	やる やったことあるの? 今日[きょう]やろうとしたことは 全部[ぜんぶ]やった。 それをやっちゃだめだよ。 その 仕事[しごと]をやろう。
\\	夕方 我々は夕方ここに着いた。 今日の夕方にはお届できます。 ハワイの夕方はたいへん美しい。	
\\	着いた 
\\	お届でき 
\\	夕方[ゆうがた] 我々[われわれ]は 夕方[ゆうがた]ここに 着[つ]いた。 今日[きょう]の 夕方[ゆうがた]にはお 届[とどけ]できます。 ハワイの 夕方[ゆうがた]はたいへん 美[うつく]しい。
\\	夕飯 夕飯の時間ですよ。 昨夜は夕飯を食べなかった。 夕飯は家で食べるの?	
\\	昨夜 
\\	夕飯[ゆうはん] 夕飯[ゆうはん]の 時間[じかん]ですよ。 昨夜[さくや]は 夕飯[ゆうはん]を 食[た]べなかった。 夕飯[ゆうはん]は 家[いえ]で 食[た]べるの?
\\	郵便局 郵便局は図書館の隣にある。 郵便局はその茶色の建物だ 駅の前に郵便局があります。	
\\	郵便 
\\	局 
\\	郵便[ゆうびん] 局[きょく] 郵便[ゆうびん] 局[きょく]は 図書館[としょかん]の 隣[となり]にある。 郵便[ゆうびん] 局[きょく]はその 茶色[ちゃいろ]の 建物[たてもの]だ 駅[えき]の 前[まえ]に 郵便[ゆうびん] 局[きょく]があります。
\\	昨夜 昨夜は雨が降っていた。 彼らは昨夜出発した。 私は昨夜ここに着いた。	
\\	出発
\\	着いた 
\\	昨夜[ゆうべ] 昨夜[ゆうべ]は 雨[あめ]が 降[ふ]っていた。 彼[かれ]らは 昨夜[ゆうべ] 出発[しゅっぱつ]した。 私[わたし]は 昨夜[ゆうべ]ここに 着[つ]いた。
\\	有名 彼は有名になった。 彼女は有名になった。 彼は医者として有名だ。	
\\	有名[ゆうめい] 彼[かれ]は 有名[ゆうめい]になった。 彼女[かのじょ]は 有名[ゆうめい]になった。 彼[かれ]は 医者[いしゃ]として 有名[ゆうめい]だ。
\\	雪 雪になるでしょう。 雪が降った。 雪は消えた。	
\\	雪[ゆき] 雪[ゆき]になるでしょう。 雪[ゆき]が 降[ふ]った。 雪[ゆき]は 消[き]えた。
\\	行く 行かないで。 行かねばならないなら行きます。 行くかもしれない行かないかもしれない。	
\\	行[い]く 行[い]かないで。 行[い]かねばならないなら 行[い]きます。 行[い]くかもしれない 行[い]かないかもしれない。
\\	ゆっくりと ゆっくり運転しなさい。 もっとゆっくり歩きなさい。 母はゆっくり話す。	
\\	ゆっくりと ゆっくり 運転[うんてん]しなさい。 もっとゆっくり 歩[ある]きなさい。 母[はは]はゆっくり 話[はな]す。
\\	八日	
\\	八日[ようか]
\\	洋服 この洋服はお買い得です。 とてもすてきな洋服ですね。 どこでその洋服を作ってもらったのですか。	
\\	洋服[ようふく] この 洋服[ようふく]はお 買い得[かいどく]です。 とてもすてきな 洋服[ようふく]ですね。 どこでその 洋服[ようふく]を 作[つく]ってもらったのですか。
\\	よく よく聞こえませんでした。 よく眠れません。 よく外食しますか。	
\\	よく よく 聞[き]こえませんでした。 よく 眠[ねむ]れません。 よく 外食[がいしょく]しますか。
\\	横 私は首を横に振った 彼女はベッドに横になった。 私たちは草の上に横になった。	
\\	振った 
\\	横[よこ] 私[わたし]は 首[くび]を 横[よこ]に 振[ふ]った 彼女[かのじょ]はベッドに 横[よこ]になった。 私[わたし]たちは 草[くさ]の 上[うえ]に 横[よこ]になった。
\\	四日 一日は二十四時間ある。	
\\	四日[よっか] 一 日[にち]は二十四 時間[じかん]ある。
\\	四つ 赤ん坊は四つんばいではった。	
\\	四[よっ]つ 赤ん坊[あかんぼう]は 四[よ]つんばいではった。
\\	六 六時きっかりにここへ来なさい。 その作り方の分量は六人分です。	
\\	作り方 
\\	分量 
\\	六 六 時[じ]きっかりにここへ 来[き]なさい。 その 作り方[つくりかた]の 分量[ぶんりょう]は六 人[にん] 分[ぶん]です。
\\	ワイシャツ このワイシャツは洗濯する必要がある あなたが一番好きなワイシャツを選んでください。	
\\	洗濯する 
\\	必要 
\\	選んでください 
\\	ワイシャツ このワイシャツは 洗濯[せんたく]する 必要[ひつよう]がある あなたが 一番[いちばん] 好[す]きなワイシャツを 選[えら]んでください。
\\	若い 若いなあ 僕は若い。 若かったらようかったのに。	
\\	若[わか]い 若[わか]いなあ 僕[ぼく]は 若[わか]い。 若[わか]かったらようかったのに。
\\	分かる 分かった? 分かるような気がする。 分かってるんでしょ。	
\\	分[わ]かる 分[わ]かった? 分[わ]かるような 気[き]がする。 分[わ]かってるんでしょ。
\\	忘れる これだけは忘れるな。 もう少しで忘れるところだった。 忘れずに電話してね。	
\\	忘[わす]れる これだけは 忘[わす]れるな。 もう 少[すこ]しで 忘[わす]れるところだった。 忘[わす]れずに 電話[でんわ]してね。
\\	私 私はここにいます。 このようにして私はそれをした。 私はここにいたい。	
\\	私[わたし] 私[わたし]はここにいます。 このようにして 私[わたし]はそれをした。 私[わたし]はここにいたい。
\\	渡す 君に渡す物がある。 彼は店員に金を渡した。 手に持っているものを私に渡しなさい。	
\\	渡[わた]す 君[きみ]に 渡[わた]す 物[もの]がある。 彼[かれ]は 店員[てんいん]に 金[かね]を 渡[わた]した。 手[て]に 持[も]っているものを 私[わたし]に 渡[わた]しなさい。
\\	渡る 通りを渡れ。 道を渡ろう。 彼らは川を泳いで渡った。	
\\	渡[わた]る 通[どお]りを 渡[わた]れ。 道[どう]を 渡[わた]ろう。 彼[かれ]らは 川[かわ]を 泳[およ]いで 渡[わた]った。
\\	悪い 悪いようにはしないから。 お前が悪いよ。 悪いのは私です。 視力は悪いです。	
\\	視力 
\\	悪[わる]い 悪[わる]いようにはしないから。 お 前[まえ]が 悪[わる]いよ。 悪[わる]いのは 私[わたし]です。 視力[しりょく]は 悪[わる]いです。
\\	より    ほう	
\\	より    ほう
\\	呼ぶ 警察を呼んで! 彼は私の名前を呼んだ。 だれかが君を呼んでいる	
\\	呼[よ]ぶ 警察[けいさつ]を 呼[よ]んで! 彼[かれ]は 私[わたし]の 名前[なまえ]を 呼[よ]んだ。 だれかが 君[きみ]を 呼[よ]んでいる
\\	読む 私は読みます。 本を読んでいる。 これを私に読んでいただけませんか?	
\\	読[よ]む 私[わたし]は 読[よ]みます。 本[ほん]を 読[よ]んでいる。 これを 私[わたし]に 読[よ]んでいただけませんか?
\\	夜 夜だった。 その夜は曇りだった。 土曜の夜だった。 静かな夜であった。	
\\	夜[よる] 夜[よる]だった。 その 夜[よる]は 曇[くも]りだった。 土曜[どよう]の 夜[よる]だった。 静[しず]かな 夜[よる]であった。
\\	弱い 地理は弱い。 彼女は弱すぎる。 私は船に弱い。 彼女は厚さには弱いんです。	
\\	地理 
\\	船 
\\	弱[よわ]い 地理[ちり]は 弱[よわ]い。 彼女[かのじょ]は 弱[よわ]すぎる。 私[わたし]は 船[ふね]に 弱[よわ]い。 彼女[かのじょ]は 厚[あつ]さには 弱[よわ]いんです。
\\	来月 来月引っ越しするんです。 彼らは来月結婚するだろう。 私は来月16歳になります。	
\\	引っ越し 
\\	来月[らいげつ] 来月[らいげつ] 引っ越[ひっこ]しするんです。 彼[かれ]らは 来月[らいげつ] 結婚[けっこん]するだろう。 私[わたし]は 来月[らいげつ] 
\\	歳[さい]になります。
\\	来週 また来週! 来週また会いたい。 私は来週出発します。	
\\	出発 
\\	来週[らいしゅう] また 来週[らいしゅう]! 来週[らいしゅう]また 会[あ]いたい。 私[わたし]は 来週[らいしゅう] 出発[しゅっぱつ]します。
\\	来年 来年は会えるでしょう。 私は来年外国へ行きます。 この本は来年印刷されるでしょう。	
\\	印刷 
\\	来年[らいねん] 来年[らいねん]は 会[あ]えるでしょう。 私[わたし]は 来年[らいねん] 外国[がいこく]へ 行[い]きます。 この 本[ほん]は 来年[らいねん] 印刷[いんさつ]されるでしょう。
\\	ラジオ ラジオを聞いています。 このラジオはあなたのですか? ラジオをつけてください。 ラジオは壊れている。	
\\	壊れて 
\\	ラジオ ラジオを 聞[き]いています。 このラジオはあなたのですか? ラジオをつけてください。 ラジオは 壊[こわ]れている。
\\	ラジカセ/ラジオカセット	
\\	ラジカセ/ラジオカセット
\\	りっぱ/立派 私の家の近くにりっぱな公園がある。 きみにはりっぱな教育を受けたといえ利点がある	
\\	教育 
\\	受けた 
\\	利点 
\\	りっぱ/ 立派[りっぱ] 私[わたし]の 家[いえ]の 近[ちか]くにりっぱな 公園[こうえん]がある。 きみにはりっぱな 教育[きょういく]を 受[う]けたといえ 利点[りてん]がある
\\	留学生 留学生も入れるのかしら。 彼女ならカナダからの留学生で、僕の伯父のところに滞在しているん。	
\\	滞在 
\\	留学生[りゅうがくせい] 留学生[りゅうがくせい]も 入[い]れるのかしら。 彼女[かのじょ]ならカナダからの 留学生[りゅうがくせい]で、 僕[ぼく]の 伯父[おじ]のところに 滞在[たいざい]しているん。
\\	両親 両親が離婚した。 彼らの両親はうちの両親より年をとっています。	
\\	離婚 
\\	両親[りょうしん] 両親[りょうしん]が 離婚[りこん]した。 彼[かれ]らの 両親[りょうしん]はうちの 両親[りょうしん]より 年[とし]をとっています。
\\	料理 ボブは料理ができる。 料理は得意です。 料理が上手ですね。	
\\	得意 
\\	料理[りょうり] ボブは 料理[りょうり]ができる。 料理[りょうり]は 得意[とくい]です。 料理[りょうり]が 上手[じょうず]ですね。
\\	旅行 旅行に行こう。 よく旅行をしますか。 旅行は楽しい。	
\\	旅行[りょこう] 旅行[りょこう]に 行[い]こう。 よく 旅行[りょこう]をしますか。 旅行[りょこう]は 楽[たの]しい。
\\	零 昨晩は零度以下に下がった。	
\\	零度 
\\	下がる
\\	零[れい] 昨晩[さくばん]は 零[れい] 度[ど] 以下[いか]に 下[さ]がった。
\\	冷蔵庫 うちの冷蔵庫が故障している。 トムは冷蔵庫を直している。 赤い冷蔵庫を見たことがない。	
\\	故障 
\\	直し 
\\	冷蔵庫[れいぞうこ] うちの 冷蔵庫[れいぞうこ]が 故障[こしょう]している。 トムは 冷蔵庫[れいぞうこ]を 直[なお]している。 赤[あか]い 冷蔵庫[れいぞうこ]を 見[み]たことがない。
\\	レコード	
\\	レコード
\\	練習する 彼女はいつもピアノを練習していた。 彼女は朝から、ピアノの練習をしている。 その練習はたくさんの時間がかかるだろう。	
\\	練習[れんしゅう]する 彼女[かのじょ]はいつもピアノを 練習[れんしゅう]していた。 彼女[かのじょ]は 朝[あさ]から、ピアノの 練習[れんしゅう]をしている。 その 練習[れんしゅう]はたくさんの 時間[じかん]がかかるだろう。
\\	廊下	
\\	廊下[ろうか]
\end{CJK}
\end{document}