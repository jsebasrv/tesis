\documentclass[8pt]{extreport} 
\usepackage{hyperref}
\usepackage{CJKutf8}
\begin{document}
\begin{CJK}{UTF8}{min}
\\	閉じる	
\\	とじる			
\\	教科書を閉じてください。	
\\	閉じる 
\\	きょうかしょ を とじて ください			
\\	時	
\\	とき			
\\	時の経つのは早い。	
\\	時 
\\	とき の たつ の は はやい			
\\	米	
\\	こめ			
\\	日本人はお米が大好きです。	
\\	米 
\\	にほんじん は おこめ が だいすき です			
\\	切る	
\\	きる			
\\	この紙を半分に切ってください。	
\\	切る 
\\	この かみ を はんぶん に きって ください			
\\	線	
\\	せん			
\\	赤い線を2本引いて下さい。	
\\	線 
\\	あかい せん を に ほん ひいて ください			
\\	決まる	
\\	きまる			
\\	旅行の日程が決まりました。	
\\	決まる 
\\	りょこう の にってい が きまりました			
\\	楽しい	
\\	たのしい			
\\	彼はとても楽しい人です。	
\\	楽しい 
\\	かれ は とても たのしい ひと です			
\\	服	
\\	ふく			
\\	昨日、新しい服を買った。	
\\	服 
\\	きのう あたらしい ふく を かった			
\\	後ろ	
\\	うしろ			
\\	後ろを向いて。	
\\	後ろ 
\\	うしろ を むいて			
\\	嬉しい	
\\	うれしい			
\\	彼に会えて嬉しかった。	
\\	嬉しい 
\\	かれ に あえて うれしかった			
\\	腰	
\\	こし			
\\	昨日から腰が痛い。	
\\	腰 
\\	きのう から こし が いたい			
\\	日曜日	
\\	にちようび			
\\	日曜日は海に行きました。	
\\	日曜日 
\\	にちようび は うみ に いきました			
\\	開ける	
\\	あける			
\\	窓を開けてください。	
\\	開ける 
\\	まど を あけて ください			
\\	昼	
\\	ひる			
\\	私は昼のドラマを毎日見ます。	
\\	昼 
\\	わたし は ひる の どらま を まいにち みます			
\\	お母さん	
\\	おかあさん			
\\	お母さんによろしくお伝えください。	
\\	お母さん 
\\	おかあさん に よろしく おつたえ ください			
\\	留学生	
\\	りゅうがくせい			
\\	彼は留学生です。	
\\	留学生 
\\	かれ は りゅうがくせい です			
\\	大学生	
\\	だいがくせい			
\\	姉は大学生です。	
\\	大学生 
\\	あね は だいがくせい です			
\\	終わり	
\\	おわり			
\\	夏休みももう終わりだ。	
\\	終わり 
\\	なつやすみ も もう おわり だ			
\\	背	
\\	せ			
\\	彼女は背が高い。	
\\	背 
\\	かのじょ は せ が たかい			
\\	手伝う	
\\	てつだう			
\\	私が手伝いましょう。	
\\	手伝う 
\\	わたし が てつだいましょう			
\\	鼻	
\\	はな			
\\	鼻がかゆいです。	
\\	鼻 
\\	はな が かゆい です			
\\	起きる	
\\	おきる			
\\	近所で盗難事件が起きました。	
\\	起きる 
\\	きんじょ で とうなん じけん が おきました			
\\	熱い	
\\	あつい			
\\	このスープはとても熱い。	
\\	熱い 
\\	この すーぷ は とても あつい			
\\	載せる	
\\	のせる			
\\	皿にケーキを載せました。	
\\	載せる 
\\	さら に ケーキ を のせました 。			
\\	悲しい	
\\	かなしい			
\\	その映画はとても悲しかった。	
\\	悲しい 
\\	その えいが は とても かなしかった			
\\	愛する	
\\	あいする			
\\	私は家族を愛しています。	
\\	愛する 
\\	わたし は かぞく を あいして います			
\\	しゃべる	
\\	彼女はよくしゃべるね。	
\\	しゃべる 
\\	かのじょ は よく しゃべる ね			
\\	近く	
\\	ちかく			
\\	私の家は駅の近くです。	
\\	近く 
\\	わたし の いえ は えき の ちかく です			
\\	甘い	
\\	あまい			
\\	このイチゴは甘い。	
\\	甘い 
\\	この いちご は あまい			
\\	テーブル	
\\	新しいテーブルを買いました。	
\\	テーブル 
\\	あたらしい てーぶる を かいました			
\\	食べ物	
\\	たべもの			
\\	日本の食べ物はとても美味しいです。	
\\	食べ物 
\\	にっぽん の たべもの は とても おいしい です			
\\	始まる	
\\	はじまる			
\\	新しい仕事が始まりました。	
\\	始まる 
\\	あたらしい しごと が はじまりました			
\\	ゲーム	
\\	私たちはビデオゲームをした。	
\\	ゲーム 
\\	わたしたち は びでおげーむ を した			
\\	十	
\\	じゅう			
\\	その子は指で十数えました。	
\\	十 
\\	その こ は ゆび で じゅう かぞえました			
\\	天気	
\\	てんき			
\\	今日はいい天気ですね。	
\\	天気 
\\	きょう は いい てんき です ね			
\\	暑い	
\\	あつい			
\\	今日はとても暑い。	
\\	暑い 
\\	きょう は とても あつい			
\\	太い	
\\	ふとい			
\\	彼女は足が太い。	
\\	太い 
\\	かのじょ は あし が ふとい			
\\	晩	
\\	ばん			
\\	晩ご飯は食べましたか。	
\\	晩 
\\	ばんごはん は たべました か			
\\	土曜日	
\\	どようび			
\\	土曜日の夜はクラブに行きます。	
\\	土曜日 
\\	どようび の よる は くらぶ に いきます			
\\	神	
\\	かみ			
\\	彼は神を信じている。	
\\	神 
\\	かれ は かみ を しんじて いる			
\\	痛い	
\\	いたい			
\\	今日は頭が痛いです。	
\\	痛い 
\\	きょう は あたま が いたい です			
\\	お父さん	
\\	おとうさん			
\\	お父さんは会社員です。	
\\	お父さん 
\\	おとうさん は かいしゃいん です			
\\	多分	
\\	たぶん			
\\	彼女は多分家で寝ています。	
\\	多分 
\\	かのじょ は たぶん いえ で ねて います			
\\	時計	
\\	とけい			
\\	時計を見たらちょうど3時だった。	
\\	時計 
\\	とけい を みたら ちょうど さんじ だった			
\\	泊まる	
\\	とまる			
\\	今日はこのホテルに泊まります。	
\\	泊まる 
\\	きょう は この ほてる に とまります			
\\	どうして	
\\	どうして泣いているの。	
\\	どうして 
\\	どうして ないて いる の			
\\	掛ける	
\\	かける			
\\	夫の服をハンガーに掛けた。	
\\	掛ける 
\\	おっと の ふく を はんがー に かけた			
\\	曲がる	
\\	まがる			
\\	そこを左に曲がってください。	
\\	曲がる 
\\	そこ を ひだり に まがって ください			
\\	おなか	
\\	ミーティング	
\\	朝9時からミーティングが始まった。	
\\	ミーティング 
\\	あさ 
\\	じ から ミーティング が はじまった			
\\	閉める	
\\	しめる			
\\	ちゃんとドアを閉めてよ。	
\\	閉める 
\\	ちゃんと どあ を しめてよ			
\\	嫌い	
\\	きらい			
\\	私はタバコが嫌いです。	
\\	嫌い 
\\	わたし は たばこ が きらい です			
\\	金曜日	
\\	きんようび			
\\	要る	
\\	いる			
\\	予約は要りません。	
\\	要る 
\\	よやく は いりません			
\\	無い	
\\	ない			
\\	ここには何も無い。	
\\	無い 
\\	ここ に は なに も ない			
\\	黄色い	
\\	きいろい			
\\	彼女に黄色いバラを買いました。	
\\	黄色い 
\\	かのじょ に きいろい ばら を かいました			
\\	風邪	
\\	かぜ			
\\	私は風邪を引きました。	
\\	風邪 
\\	わたし は かぜ を ひきました 。			
\\	優しい	
\\	やさしい			
\\	彼はとても優しい人です。	
\\	優しい 
\\	かれ は とても やさしい ひと です			
\\	晴れる	
\\	はれる			
\\	明日は晴れるといいですね。	
\\	晴れる 
\\	あした は はれる と いい です ね			
\\	大好き	
\\	だいすき			
\\	私は犬が大好きだ。	
\\	大好き 
\\	わたし は いぬ が だいすき だ			
\\	茶色	
\\	ちゃいろ			
\\	彼女は茶色の靴を履いています。	
\\	茶色 
\\	かのじょ は ちゃいろ の くつ を はいて います			
\\	汚い	
\\	きたない			
\\	彼の部屋はとても汚い。	
\\	汚い 
\\	かれ の へや は とても きたない			
\\	空く	
\\	すく			
\\	おなかが空きました。	
\\	おなか 
\\	空く 
\\	おなか が すきました			
\\	レストランは空いていました。	
\\	空く 
\\	れすとらん は すいて いました			
\\	上る	
\\	のぼる			
\\	猫が屋根に上っている。	
\\	上る 
\\	ねこ が やね に のぼって いる			
\\	ご飯	
\\	ごはん			
\\	私はパンよりご飯が好きだ。	
\\	ご飯 
\\	わたし は ぱん より ごはん が すき だ			
\\	日	
\\	にち			
\\	私たちは先月11日に結婚しました。	
\\	日 
\\	わたしたち は せんげつ じゅういち にち に けっこん しました			
\\	髪の毛	
\\	かみのけ			
\\	髪の毛が伸びたね。	
\\	髪の毛 
\\	かみのけ が のびた ね			
\\	つける	
\\	電気をつけてください。	
\\	つける 
\\	でんき を つけて ください			
\\	月曜日	
\\	げつようび			
\\	月曜日に会いましょう。	
\\	月曜日 
\\	げつようび に あいましょう			
\\	入る	
\\	はいる			
\\	寒いので中に入ってください。	
\\	入る 
\\	さむい の で なか に はいって ください			
\\	片仮名	
\\	かたかな			
\\	お名前を片仮名で書いてください。	
\\	片仮名 
\\	おなまえ を かたかな で かいて ください			
\\	今週	
\\	こんしゅう			
\\	今週は日本語のテストがあります。	
\\	今週 
\\	こんしゅう は にほんご の てすと が あります			
\\	開く	
\\	ひらく			
\\	32ページを開いてください。	
\\	開く 
\\	さんじゅうにぺーじ を ひらいて ください			
\\	水	
\\	みず			
\\	水を一杯ください。	
\\	水 
\\	みず を いっぱい ください			
\\	あれ	
\\	二	
\\	に			
\\	その人には二回会った。	
\\	二 
\\	その ひと に は に かい あった			
\\	閉まる	
\\	しまる			
\\	お店はもう閉まっていました。	
\\	閉まる 
\\	おみせ は もう しまって いました			
\\	締める	
\\	しめる			
\\	彼はシートベルトを締めた。	
\\	締める 
\\	かれ は しーとべると を しめた			
\\	まずい	
\\	ここの料理はまずい。	
\\	まずい 
\\	ここ の りょうり は まずい			
\\	平仮名	
\\	ひらがな			
\\	私は平仮名を全部読めます。	
\\	平仮名 
\\	わたし は ひらがな を ぜんぶ よめます			
\\	曇る	
\\	くもる			
\\	明日は昼頃から曇るでしょう。	
\\	昼 
\\	曇る 
\\	あす は ひるごろ から くもる でしょう			
\\	触る	
\\	さわる			
\\	絵に触らないでください。	
\\	触る 
\\	え に さわらない で ください			
\\	駄目	
\\	だめ			
\\	書き直す	
\\	かきなおす			
\\	この書類を書き直してください。	
\\	書き直す 
\\	この しょるい を かきなおして ください			
\\	飲み物	
\\	のみもの			
\\	木曜日	
\\	もくようび			
\\	木曜日は仕事が休みです。	
\\	木曜日 
\\	もくようび は しごと が やすみ です			
\\	曜日	
\\	ようび			
\\	曜日を間違えました。	
\\	曜日 
\\	ようび を まちがえました			
\\	そば	
\\	そばにいて下さい。	
\\	そば 
\\	そば に いて ください			
\\	こっち	
\\	こっちに来て下さい。	
\\	こっち 
\\	こっち に きて ください			
\\	火曜日	
\\	かようび			
\\	火曜日に会議があります。	
\\	火曜日 
\\	かようび に かいぎ が あります			
\\	渇く	
\\	かわく			
\\	喉が渇きました。	
\\	渇く 
\\	のど が かわきました			
\\	三	
\\	さん			
\\	彼女は三人の子供の母親だ。	
\\	三 
\\	かのじょ は さんにん の こども の ははおや だ			
\\	水曜日	
\\	すいようび			
\\	水曜日はバイトがあります。	
\\	水曜日 
\\	すいようび は ばいと が あります			
\\	二つ	
\\	ふたつ			
\\	ソフトクリームを二つください。	
\\	二つ 
\\	そふとくりーむ を ふたつ ください			
\\	今晩	
\\	こんばん			
\\	今晩のパーティーは何時からですか。	
\\	今晩 
\\	こんばん の ぱーてぃー は なんじ から です か			
\\	千	
\\	せん			
\\	六つ	
\\	むっつ			
\\	息子は六つになりました。	
\\	六つ 
\\	むすこ は むっつ に なりました			
\\	六日	
\\	むいか			
\\	六日前に日本に帰ってきました。	
\\	六日 
\\	むいかまえ に にほん に かえって きました			
\\	お姉さん	
\\	おねえさん			
\\	昨日、あなたのお姉さんに会ったよ。	
\\	お姉さん 
\\	きのう あなた の おねえさん に あった よ			
\\	直る	
\\	なおる			
\\	クーラーはまだ直りません。	
\\	直る 
\\	くーらー は まだ なおりません			
\\	易しい	
\\	やさしい			
\\	この問題はかなり易しいです。	
\\	易しい 
\\	この もんだい は かなり やさしい です			
\\	辺	
\\	へん			
\\	彼はこの辺に住んでいます。	
\\	辺 
\\	かれ は このへん に すんで います			
\\	よう	
\\	彼女は眠いようです。	
\\	よう 
\\	かのじょ は ねむい よう です			
\\	お兄さん	
\\	おにいさん			
\\	あなたのお兄さんは何歳?	
\\	お兄さん 
\\	あなた の おにいさん は なんさい			
\\	辛い	
\\	からい			
\\	彼は辛いものが好きです。	
\\	辛い 
\\	かれ は からい もの が すき です			
\\	八	
\\	はち			
\\	りんごを八個ください。	
\\	八 
\\	りんご を はちこ ください			
\\	三日	
\\	みっか			
\\	手紙が届くのに三日かかりました。	
\\	三日 
\\	てがみ が とどく の に みっか かかりました			
\\	あそこ	
\\	あそこにバス停があります。	
\\	あそこ 
\\	あそこ に ばすてい が あります			
\\	来る	
\\	くる			
\\	彼は昼過ぎに来ます。	
\\	来る 
\\	かれ は ひるすぎ に きます			
\\	前	
\\	まえ			
\\	その店の前で会いましょう。	
\\	前 
\\	その みせ の まえ で あいましょう			
\\	五日	
\\	いつか			
\\	五月五日は祝日です	
\\	五日 
\\	ごがつ いつか は しゅくじつ です			
\\	一杯	
\\	いっぱい			
\\	プールは人で一杯です。	
\\	一杯 
\\	ぷーる は ひと で いっぱい です			
\\	九	
\\	きゅう			
\\	野球は九人で1チームです。	
\\	九 
\\	やきゅう は きゅうにん で ひとちーむ です			
\\	酸っぱい	
\\	すっぱい			
\\	このぶどうは酸っぱいです。	
\\	酸っぱい 
\\	この ぶどう は すっぱい です			
\\	違う	
\\	ちがう			
\\	答えが違います。	
\\	違う 
\\	こたえ が ちがいます			
\\	細い	
\\	ほそい			
\\	彼女は指が細いですね。	
\\	細い 
\\	かのじょ は ゆび が ほそい です ね			
\\	三つ	
\\	みっつ			
\\	コップを三つ買いました。	
\\	三つ 
\\	こっぷ を みっつ かいました			
\\	八日	
\\	ようか			
\\	八日からイギリスに行きます。	
\\	八日 
\\	ようか から いぎりす に いきます			
\\	高校生	
\\	こうこうせい			
\\	私の弟は高校生です。	
\\	高校生 
\\	わたし の おとうと は こうこうせい です			
\\	上手	
\\	じょうず			
\\	妹は歌が上手です。	
\\	上手 
\\	いもうと は うた が じょうず です			
\\	強い	
\\	つよい			
\\	今日は風が強い。	
\\	強い 
\\	きょう は かぜ が つよい			
\\	七	
\\	なな			
\\	バナナが七本あります。	
\\	七 
\\	ばなな が ななほん あります			
\\	二十日	
\\	はつか			
\\	来月の二十日は弟の誕生日です。	
\\	二十日 
\\	らいげつ の はつか は おとうと の たんじょうび です			
\\	左	
\\	ひだり			
\\	そこを左に曲がってください。	
\\	曲がる 
\\	左 
\\	そこ を ひだり に まがって ください			
\\	二日	
\\	ふつか			
\\	私は二日待った。	
\\	二日 
\\	わたし は ふつか まった			
\\	四つ	
\\	よっつ			
\\	私は腕時計を四つ持っています。	
\\	四つ 
\\	わたし は うでどけい を よっつ もって います			
\\	暖かい	
\\	あたたかい			
\\	このコートはとても暖かい。	
\\	暖かい 
\\	この コート は とても あたたかい。			
\\	有る	
\\	ある			
\\	私の机の上に書類がたくさん有ります。	
\\	有る 
\\	わたし の つくえ の うえ に しょるい が たくさん あります			
\\	良い	
\\	いい			
\\	彼女は良い友達です。	
\\	良い 
\\	かのじょ は いい ともだち です			
\\	五つ	
\\	いつつ			
\\	桃を五つください。	
\\	五つ 
\\	もも を いつつ ください			
\\	妹	
\\	いもうと			
\\	私の妹は小学生です。	
\\	妹 
\\	わたし の いもうと は しょうがくせい です			
\\	上	
\\	うえ			
\\	上を向いて。	
\\	上 
\\	うえ を むいて			
\\	動く	
\\	うごく			
\\	動かないで。	
\\	動く 
\\	うごかない で			
\\	駅	
\\	えき			
\\	駅はどこですか。	
\\	駅 
\\	えき は どこ です か			
\\	おいしい	
\\	このケーキはおいしいね。	
\\	おいしい 
\\	この けーき は おいしい ね			
\\	大きい	
\\	おおきい			
\\	あの大きい建物は何ですか。	
\\	大きい 
\\	あの おおきい たてもの は なん です か			
\\	弟	
\\	おとうと			
\\	弟は野球が好きです。	
\\	弟 
\\	おとうと は やきゅう が すき です			
\\	男	
\\	おとこ			
\\	彼は駄目な男だ。	
\\	駄目 
\\	男 
\\	かれ は だめ な おとこ だ			
\\	男の人が私たちに話しかけた。	
\\	男 
\\	おとこ の ひと が わたしたち に はなしかけた			
\\	大人	
\\	おとな			
\\	お酒は大人になってから。	
\\	大人 
\\	おさけ は おとな に なって から			
\\	お願い	
\\	おねがい			
\\	お願いがあります。	
\\	お願い 
\\	おねがい が あります			
\\	女	
\\	おんな			
\\	店員は若い女の人でした。	
\\	女 
\\	てんいん は わかい おんな の ひと でした			
\\	金	
\\	かね			
\\	これはかなり金がかかった。	
\\	金 
\\	これ は かなり かね が かかった			
\\	彼	
\\	かれ			
\\	彼は私の上司です。	
\\	彼 
\\	かれ は わたし の じょうし です			
\\	昨日	
\\	きのう			
\\	昨日、友達に会った。	
\\	昨日 
\\	きのう ともだち に あった			
\\	奇麗	
\\	きれい			
\\	彼女はとても奇麗だ。	
\\	奇麗 
\\	かのじょ は とても きれい だ			
\\	口	
\\	くち			
\\	口を大きく開けてください。	
\\	口 
\\	くち を おおきく あけて ください			
\\	五	
\\	ご			
\\	五人で旅行に行きました。	
\\	五 
\\	ごにん で りょこう に いきました			
\\	九日	
\\	ここのか			
\\	九日に荷物が届きます。	
\\	九 
\\	九日 
\\	ここのか に にもつ が とどきます			
\\	九つ	
\\	ここのつ			
\\	娘は明日九つになります。	
\\	九つ 
\\	むすめ は あす ここのつ に なります			
\\	答える	
\\	こたえる			
\\	私の質問に答えてください。	
\\	答える 
\\	わたし の しつもん に こたえて ください			
\\	先	
\\	さき			
\\	お先にどうぞ。	
\\	先 
\\	おさきに どうぞ			
\\	寒い	
\\	さむい			
\\	この部屋は寒いです。	
\\	寒い 
\\	この へや は さむい です			
\\	四	
\\	し			
\\	四月に大学に入学しました。	
\\	四 
\\	しがつ に だいがく に にゅうがく しました			
\\	下	
\\	した			
\\	財布は机の下にあった。	
\\	下 
\\	さいふ は つくえ の した に あった			
\\	大丈夫	
\\	だいじょうぶ			
\\	大丈夫ですか。	
\\	大丈夫 
\\	だいじょうぶ です か			
\\	出す	
\\	だす			
\\	彼はかばんから教科書を出した。	
\\	出す 
\\	かれ は かばん から きょうかしょ を だした			
\\	小さい	
\\	ちいさい			
\\	小さい花が咲いています。	
\\	小さい 
\\	ちいさい はな が さいて います			
\\	父	
\\	ちち			
\\	私は父が大好きです。	
\\	父 
\\	わたし は ちち が だいすき です			
\\	一日	
\\	ついたち			
\\	来月の一日は空いていますか。	
\\	一日 
\\	らいげつ の ついたち は あいています か			
\\	月	
\\	つき			
\\	今夜は月がとてもきれいです。	
\\	月 
\\	こんや は つき が とても きれい です			
\\	妻	
\\	つま			
\\	今日は妻の誕生日だ。	
\\	妻 
\\	きょう は つま の たんじょうび だ			
\\	冷たい	
\\	つめたい			
\\	冷たい飲み物をください。	
\\	飲み物 
\\	冷たい 
\\	つめたい のみもの を ください			
\\	手	
\\	て			
\\	分かった人は手を上げてください。	
\\	手 
\\	わかった ひと は て を あげて ください			
\\	十日	
\\	とおか			
\\	十日後に帰ります。	
\\	十日 
\\	とおかご に かえります			
\\	時々	
\\	ときどき			
\\	彼は時々遅刻します。	
\\	時々 
\\	かれ は ときどき ちこく します			
\\	夏	
\\	なつ			
\\	私は夏が大好き。	
\\	大好き 
\\	夏 
\\	わたし は なつ が だいすき			
\\	七つ	
\\	ななつ			
\\	この子は今年七つになります。	
\\	七つ 
\\	この こ は ことし ななつ に なります			
\\	何	
\\	なに			
\\	あれは何ですか。	
\\	あれ 
\\	何 
\\	あれ は なん です か			
\\	夕食には何を食べたいですか。	
\\	何 
\\	ゆうしょく に は なに を たべたい です か			
\\	七日	
\\	なのか			
\\	先月の七日に孫が生まれました。	
\\	七日 
\\	せんげつ の なのか に まご が うまれました			
\\	母	
\\	はは			
\\	昨日、母と話をしました。	
\\	母 
\\	きのう はは と はなし を しました			
\\	人	
\\	ひと			
\\	彼は優しい人です。	
\\	優しい 
\\	人 
\\	かれ は やさしい ひと です			
\\	一人	
\\	ひとり			
\\	そこには私一人しかいなかった。	
\\	一人 
\\	そこ に は わたし ひとり しか いなかった			
\\	百	
\\	ひゃく			
\\	私の祖母は百才です。	
\\	百 
\\	わたし の そぼ は ひゃくさい です			
\\	方	
\\	ほう			
\\	彼は私の方を見ました。	
\\	方 
\\	かれ は わたし の ほう を みました			
\\	他	
\\	ほか			
\\	他に方法がありません。	
\\	他 
\\	ほか に ほうほう が ありません			
\\	僕	
\\	ぼく			
\\	僕は学生です。	
\\	僕 
\\	ぼく は がくせい です			
\\	欲しい	
\\	ほしい			
\\	僕は新しい靴が欲しいです。	
\\	僕 
\\	欲しい 
\\	ぼく は あたらしい くつ が ほしい です			
\\	何か飲み物が欲しいな。	
\\	飲み物 
\\	欲しい 
\\	なに か のみもの が ほしい な 。			
\\	万	
\\	まん			
\\	この靴は1万円です。	
\\	万 
\\	この くつ は いちまんえん です			
\\	見える	
\\	みえる			
\\	ここから富士山がよく見えます。	
\\	見える 
\\	ここ から ふじさん が よく みえます			
\\	店	
\\	みせ			
\\	私はこの店によく来ます。	
\\	店 
\\	わたし は この みせ に よく きます			
\\	道	
\\	みち			
\\	この道を真っ直ぐ行くと駅です。	
\\	道 
\\	この みち を まっすぐ いく と えき です			
\\	目	
\\	め			
\\	彼女は青い目をしています。	
\\	目 
\\	かのじょ は あおい め を して います			
\\	八つ	
\\	やっつ			
\\	あの家には時計が八つあります。	
\\	時計 
\\	八つ 
\\	あの いえ に は とけい が やっつ あります			
\\	止める	
\\	やめる			
\\	話すのを止めてください。	
\\	止める 
\\	はなす の を やめて ください			
\\	四日	
\\	よっか			
\\	新学期は来月の四日からです。	
\\	四日 
\\	しんがっき は らいげつ の よっか から です			
\\	夜	
\\	よる			
\\	金曜日の夜は友達と出かけます。	
\\	金曜日 
\\	夜 
\\	きんようび の よる は ともだち と でかけます			
\\	きのうの夜は家にいました。	
\\	夜 
\\	きのう の よる は いえ に いました			
\\	ハワイは四回目です。	
\\	四 
\\	四 
\\	はわい は よんかいめ です			
\\	来年	
\\	らいねん			
\\	来年一緒に旅行しましょう。	
\\	来年 
\\	らいねん いっしょ に りょこう しましょう			
\\	六	
\\	ろく			
\\	彼には子供が六人います。	
\\	六 
\\	かれ に は こども が ろくにん います			
\\	悪い	
\\	わるい			
\\	たばこは体に悪い。	
\\	悪い 
\\	たばこ は からだ に わるい			
\\	お手洗い	
\\	おてあらい			
\\	お手洗いはどこですか。	
\\	お手洗い 
\\	おてあらい は どこ です か			
\\	ご主人	
\\	ごしゅじん			
\\	ご主人はお元気ですか。	
\\	ご主人 
\\	ごしゅじん は おげんき です か			
\\	本当に	
\\	ほんとうに			
\\	あなたが本当に好きです。	
\\	本当に 
\\	あなた が ほんとうに すき です			
\\	自分	
\\	じぶん			
\\	宿題は自分でやりなさい。	
\\	自分 
\\	しゅくだい は じぶん で やりなさい			
\\	なる	
\\	柿の木に実がたくさんなっています。	
\\	なる 
\\	かきのき に み が たくさん なって います			
\\	ため	
\\	これは医者のためのサイトです。	
\\	ため 
\\	これ は いしゃ の ため の さいと です			
\\	円	
\\	えん			
\\	千円貸してください。	
\\	千 
\\	円 
\\	せんえん かして ください			
\\	カレーライスは700円です。	
\\	円 
\\	かれーらいす は ななひゃくえん です			
\\	場合	
\\	ばあい			
\\	分からない場合は私に聞いてください。	
\\	場合 
\\	わからない ばあい は わたし に きいて ください			
\\	休む	
\\	やすむ			
\\	明日、会社を休みます。	
\\	休む 
\\	あした かいしゃ を やすみます			
\\	見つかる	
\\	みつかる			
\\	メガネが見つかりません。	
\\	見つかる 
\\	めがね が みつかりません			
\end{CJK}
\end{document}