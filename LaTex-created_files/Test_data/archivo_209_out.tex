\documentclass[8pt]{extreport} 
\usepackage{hyperref}
\usepackage{CJKutf8}
\begin{document}
\begin{CJK}{UTF8}{min}
\\	油を売る	
\\	プラプラ油を売っていないで、仕事にもどったらどうなのさ。	あぶらをうる	
\\	むだ話などをして、仕事を怠ける。
\\	あいづちを打つ	
\\	電話で「はい。」と相槌を打たずに、「うん。うん。」と相槌を打つ	あいづらをうつ		
\\	相手の話に調子を合わせてうなずくこと。
\\	青筋を立てる	
\\	二人は青筋を立てて争った	あおすじをたてる	
\\	かんしゃくを起こして怒る。
\\	青菜に塩	
\\	彼は今、株で大損をして青菜に塩の状態だ	あおのにしお	
\\	すっかり元気をなくししおれる。
\\	赤子の手をひねるよう	
\\	この試合は相手が弱すぎて赤子の手をひねるようなものだった	あかこのてをひねるよう	
\\	とても簡単なこと。
\\	明け透けに言う	
\\	あの人はなんでも明け透けに言う	あけすけにいう		
\\	包み隠さず、何でもずけずけという。
\\	挙足をとる・揚げ足を取る	
\\	僕の言うことに必ず揚げ足を取るんだ.	あげあしをとる		
\\	人の言った言葉尻や、ちょっとした言い間違いをとらえて、大袈裟に批評したりする。
\\	顎で人を使う	
\\	「そっか。暇なら掃除しといて。」「なんだそれ。兄貴をアゴで使うなよ。」	あごでひとをつかう	
\\	威張った態度で人を使う。
\\	顎を出す	
\\	彼はレースの後あごを出して, しばらくは口も利けなかった.	あごをだす	
\\	大変疲れてしまう。
\\	味を占める	
\\	一度賭事で味を占めるとなかなかやめられない	あじをしめる	
\\	一度うまく行ったことが忘れられないで、いつもそうだろうと思う。
\\	頭が上がらない	
\\	夫が妻に頭が上がらない人	あたまがあがらない		
\\	従うほかはない。
\\	後の祭り	
\\	今さら嘆いても後の祭りだよ。自分自身の裏切りのせいで、奥さんに三行半を突きつけられたんだろう	あとのまつり	
\\	気がついた時には、時期が遅れてしまっていて、間に合わないこと。
\\	後は野となれ山となれ	
\\	あとはのとなれやまとなれ	
\\	後のことや結果はどうなってもいい、と気にしないこと。
\\	穴があったら入りたい	
\\	漫談に挑戦したのに、誰も笑ってくれない。穴があったら入りたい。	あながあったらはいりたい	
\\	非常に恥ずかしくて、人に顔を見られたくない。
\\	あわを食う	
\\	彼らは泡を食って逃げた	あわをくう		
\\	非常に驚き慌てる。
\\	息を飲む	
\\	ナイヤガラの滝を見たらきっと息を飲みますよ.	いきをのむ		
\\	はっと息を止める。
\\	痛くも痒くもない	
\\	彼が失敗したところで, 僕は痛くもかゆくもない。	いたくもかゆくもない	
\\	少しも苦しみを感じないこと。
\\	痛し痒し	
\\	痛しかゆしでどちらとも決められない。	いたしかゆし	
\\	両方とも同じ程度に良い面と悪い面があって、どちらか一つを取るのに迷うこと。
\\	板に付く	
\\	教壇に立って一年、先生ぶりもやっと板についてきた	いたにつく		
\\	経験を積んだ結果、職業・任務などがその人にぴったり合った感じになる。
\\	一日の長	
\\	料理に関しては、私があなたより腕前がいいということではないのよ。一日の長があるというだけのこと。	いちじつのちょう		
\\	仕事などについて、他の人よりも、少しだけよく知っていて、上手に出来ること。
\\	色を失う	
\\	信じていた人から裏切られたことを知ったときは、さすがに色を失った。	いろをうしなう		
\\	非常に驚くこと。
\\	引導をわたす	
\\	さん、
\\	さん引導を渡してやりなさい。	いんどうをわたす		
\\	最終的な結論を言い渡して、諦めさせる。
\\	浮き足立つ	
\\	敵の猛攻撃を受け、守備軍が浮き足立ってしまった	うきあしだつ	
\\	逃げ腰になって、そわそわしてしまうこと。
\\	鵜の目・鷹の目	
\\	息子がガールフレンドを家に連れてきたとき、妻は鵜の目鷹の目で粗探しをしていた。	うのめ・たかのめ	
\\	何かを探そうとして、鋭い目で辺りを見回す。
\\	後ろ指を指される	
\\	うしろゆびをさされる			陰で悪口をいわれること。
\\	うだつが上がらない	
\\	うだつが上がらない若手社員3つの特徴	うだつがあがらない		
\\	いつも上からおさえられて、よい境遇になれない。
\\	馬が合う	
\\	馬が合う	
\\	相手と気持ちがしっくり合う。
\\	裏をかく	
\\	法律の裏をかく。	うらをかく	
\\	相手の考えていなかったようなことをして、その狙いを外す。
\\	恨みを買う	
\\	嫌われる事と恨みを買う事はべつもの	うらみをかう	
\\	恨まれることをする。
\\	瓜二つ	
\\	兄に瓜二つの弟	うりふたつ	
\\	縦に二つに割った瓜のように、顔つきなどが、非常によく似ている。
\\	売り言葉に買い言葉	
\\	昨日のは売り言葉に買い言葉というやつで、本心じゃなかったんだ。	うりことばにかいことば		
\\	相手の暴言に対して、こちらも暴言で言い返すこと。
\\	うわの空	
\\	うわの空の目	うわのそら	
\\	他のことに心が奪われて、あることに注意が向かないこと。
\\	雲泥の差	
\\	うんでいのさ	
\\	差が大きいこと。
\\	衿を正す(襟を正す)	
\\	その堂に入ると覚えず襟を正す。	えりをただす	
\\	気持ちを引き締めること。
\\	縁起が悪い	
\\	あわせ鏡はなぜ縁起が悪いのですか?	えんぎがわるい		
\\	その物事のせいで、悪い事が起こりそうな感じがすること。
\\	縁起を担ぐ	
\\	上記のような縁起を担ぐ事は行わない。	えんぎをかつぐ		
\\	何をする場合でも、縁起がいい・悪いを気にすること。
\\	大手を振る	
\\	ここでは賄賂(わいろ)が大手を振って通用している	おおでをふる	
\\	両手を大きく振って、堂々と歩くようす。
\\	大きな顔をする	
\\	新米のくせに大きな顔をするな。	おおきなかおをする	
\\	自分一人が偉い者であるようなようすをする。
\\	大船に乗ったような気持ち	
\\	君がやってくれるので、大船に乗ったような気持ちだ	おおぶねにのったようなきもち	
\\	信じ切って安心している気持ち。
\\	大風呂敷を広げる	
\\	威勢よく大風呂敷を広げていたが、結局は実現できなかったそうだ	おおぶろしきをひろげる	
\\	大げさな話を始める。
\\	大目玉を食う	
\\	車をぶつけてしまい、親父から大目玉を食らってしまった。	おおめだまをくう	
\\	ひどく叱られる。
\\	奥の手を出す	
\\	愚者はすぐ奥の手を出し行き詰まる。	おくのてをだす	
\\	取っておきの手段を用いる。
\\	お茶を濁す	
\\	話の途中で急用ができただなんて言って、それでお茶を濁すつもりかい	おちゃをにごす。	
\\	いい加減なことを言ったりして、その場をごまかす。
\\	男一匹	
\\	おとこいっぴき			自覚があり、どこに出しても恥ずかしくない男。
\\	同じ釜の飯を食う	
\\	君と僕は同じ釜の飯を食った仲間だ	おなじかまのめしをくう	
\\	一緒に寝起きして苦楽を共にする。
\\	思う壺	
\\	家に入れてしまったら、セールスマンの思う壷だ	おもうつぼ		
\\	こちらの期待にあてはまる。
\\	お鉢が回る	
\\	一ヶ月に1回当番のお鉢が回ってくる	おはちがまわる		
\\	順番が来る。
\\	折り紙つき	
\\	家庭的な雰囲気、でも味は折り紙つき!	おりがみつき	
\\	これなら確かであるという保証が付いている。
\\	恩に着せる	
\\	同僚には感謝しています。でも、いつまでもそのことを恩に着せられる関係にうんざりしています。	おんにきせる		
\\	ちょっとした事を、ことさらにその人のためにしたとして、有難がらせる。
\\	音頭を取る	
\\	大田さんが音頭を取ってこの企画をスタートさせたのです.	おんどをとる		
\\	人の先頭に立って物事をする。
\end{CJK}
\end{document}