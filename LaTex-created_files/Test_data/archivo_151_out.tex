\documentclass[8pt]{extreport} 
\usepackage{hyperref}
\usepackage{CJKutf8}
\begin{document}
\begin{CJK}{UTF8}{min}
\\	彼の健康状態 
\\	スポーツは無理でしょう。	彼[かれ]の 健康[けんこう] 状態[じょうたい]からしてスポーツは 無理[むり]でしょう。	からして	からして	
\\	彼は時がたつのはなんと早い 
\\	とつくづく考えた。	彼[かれ]は 時[じ]がたつのはなんと 早[はや]いことかとつくづく 考[かんが]えた。	ことか	ことか	
\\	このサイトは検索はおろか、本の内容紹介 
\\	くれるので非常に便利だ。	このサイトは 検索[けんさく]はおろか、 本[ほん]の 内容[ないよう] 紹介[しょうかい]までしてくれるので 非常[ひじょう]に 便利[べんり]だ。	までして	までして	
\\	家族や家庭を犠牲に (する: 
\\	会社のために働くなんてとんでもない。	家族[かぞく]や 家庭[かてい]を 犠牲[ぎせい]にしてまで 会社[かいしゃ]のために 働[はたら]くなんてとんでもない。	してまで	てまで	
\\	南極旅行は、気持ち的には、借金 (する: 
\\	行けるものなら行きたいんだが。	南極[なんきょく] 旅行[りょこう]は、 気持[きも]ち 的[てき]には、 借金[しゃっきん]してでも 行[い]けるものなら 行[い]きたいんだが。	してでも	てでも	
\\	犯人は、だれにも気づかれる 
\\	、その場所を離れた。	犯人[はんにん]は、だれにも 気[き]づかれることなく、その 場所[ばしょ]を 離[はな]れた。	ことなく	ことなく	
\\	大木君、会議だっていうのに、外出しちゃいましたよ。
\\	書類をそろえてくれた 
\\	だよ。	
\\	大木[おおき] 君[くん]、 会議[かいぎ]だっていうのに、 外出[がいしゅつ]しちゃいましたよ。
\\	書類[しょるい]をそろえてくれただけましだよ。	だけまし	だけまし	
\\	勉強する (しかない 
\\	。	勉強[べんきょう]するよりほかない。	(より)ほかない	(より)ほかない	
\\	南極旅行は、気持ち的には、借金してでも行ける 
\\	行きたいんだが。	南極[なんきょく] 旅行[りょこう]は、 気持[きも]ち 的[てき]には、 借金[しゃっきん]してでも 行[い]けるものなら 行[い]きたいんだが。	ものなら	ものなら	
\\	その子供はまるで大人である 
\\	話し方をする。	その 子供[こども]はまるで 大人[おとな]であるかのような 話し方[はなしかた]をする。	かのような	かのよう	
\\	この空模様 (から言うと 
\\	、雨になりそうだ。	この 空模様[そらもよう]からすると、 雨[あめ]になりそうだ。	からすると	からすると	
\\	ずっと本を読んでいて急に立ち上がった 
\\	、めまいがしました。	ずっと 本[ほん]を 読[よ]んでいて 急[きゅう]に 立ち上[たちあ]がったとたん、めまいがしました。	とたん	たとたん(に)	
\\	ショック 
\\	、涙も出ませんでしたよ。	ショックのあまり、 涙[なみだ]も 出[で]ませんでしたよ。ショックのあまり、涙も出ませんでしたよ。	の余り	の余り	
\\	私の子供のころの写真は一枚 (とうえども~ない, 
\\	残ってない。	私[わたし]の 子供[こども]のころの 写真[しゃしん]は一 枚[まい]として 残[のこ]ってない。	として	として~ない	
\\	そういうところが実際にあるかどうか 
\\	、自分の理想とする会社はどんなところかを書いて下さい。	そういうところが 実際[じっさい]にあるかどうかは 別[べつ]として、 自分[じぶん]の 理想[りそう]とする 会社[かいしゃ]はどんなところかを 書[か]いて 下[くだ]さい。	は別として	は別として	
\\	国が違うと、言葉や文化 
\\	、人々の考え方まで違う。	国[くに]が 違[ちが]うと、 言葉[ことば]や 文化[ぶんか]はもちろん、 人々[ひとびと]の 考え方[かんがえかた]まで 違[ちが]う。	はもちろん	はもちろん	
\\	母は今年93歳になった。今は元気だが、 (歳~歳~: 
\\	、病気をすると心配だ。	母[はは]は 今年[ことし]93 歳[さい]になった。 今[いま]は 元気[げんき]だが、 歳[とし]が 歳[とし]だけに、 病気[びょうき]をすると 心配[しんぱい]だ。	歳が歳だけに	が~だけに	
\\	水蒸気 
\\	気体の状態に変わった水のことである。	水蒸気[すいじょうき]とは 気体[きたい]の 状態[じょうたい]に 変[か]わった 水[みず]のことである。	とは	とは	
\\	考えている 
\\	、話しかけられて困った。	考[かんが]えている 最中[さいちゅう]に、 話[はな]しかけられて 困[こま]った。	最中に	最中に	
\\	収入の多寡 
\\	出資額を決める。	収入[しゅうにゅう]の 多寡[たか]に 応[おう]じて 出資[しゅっし] 額[がく]を 決[き]める。	に応じて	に応じて	
\\	毎日水をやった 
\\	、10月になって庭の花が綺麗に咲いた。	毎日[まいにち] 水[すい]をやったかいがあって、 10月[じゅうがつ]になって 庭[にわ]の 花[はな]が 綺麗[きれい]に 咲[さ]いた。	かいがあって	かいがあって~	
\\	彼女は「えい!」
\\	侵入者を投げ飛ばした。	彼女[かのじょ]は「えい!」とばかりに 侵入[しんにゅう] 者[しゃ]を 投げ飛[なげと]ばした。	とばかりに	とばかりに	
\\	課長は余計な質問はするな 
\\	私をにらみつけた。	課長[かちょう]は 余計[よけい]な 質問[しつもん]はするなとばかりに私[わたし]をにらみつけた。	とばかりに	とばかりに	
\\	インド料理やタイ料理 
\\	とか) 南の国の食べ物には辛いものが多い。	インド 料理[りょうり]やタイ 料理[りょうり]といった 南[みなみ]の 国[くに]の 食べ物[たべもの]には 辛[つら]いものが 多[おお]い。	といった	といった	
\\	彼は学生時代は抜群の成績だったが、会社に (入る: 
\\	パッとしない。	彼[かれ]は 学生[がくせい] 時代[じだい]は 抜群[ばつぐん]の 成績[せいせき]だったが、 会社[かいしゃ]に 入[はい]ってからはパッとしない。	入ってからは	てからは	
\\	柔道は若者の健康によい (だけじゃなくて 
\\	、人格形成にもおおいに役立つ。	柔道[じゅうどう]は 若者[わかもの]の 健康[けんこう]によいばかりか、 人格[じんかく] 形成[けいせい]にもおおいに 役立[やくだ]つ。	ばかりか	ばかりか	
\\	彼女はうまく歌えない (だけじゃなくて 
\\	ピアノもよく弾けません。	彼女[かのじょ]はうまく 歌[うた]えないばかりかピアノもよく 弾[ひ]けません。	ばかりか	ばかりか	
\\	さすが毎日トレーニングしている 
\\	すごい体力。	さすが 毎日[まいにち]トレーニングしているだけあってすごい 体力[たいりょく]。	だけあって	だけあって	
\\	何しに来たって?飲みに来た 
\\	じゃないか。	何[なに]しに 来[き]たって? 飲[の]みに 来[き]たに 決[き]まっているじゃないか。	に決まっている	に決まっている	
\\	このままではジムの日本語の成績は下がる 
\\	。	このままではジムの 日本語[にほんご]の 成績[せいせき]は 下[さ]がるばかりだ。	ばかりだ	ばかりだ	
\\	道は通行止めだし。これではそこへ (行く: 
\\	。	道は通行止めだし。これではそこへ行きようがない。	行きようがない	ようがない	
\\	大学を出た 
\\	、必ずしも教養があるわけではない。	まちがえたからといって 彼[かれ]の 事[こと]を 笑[わら]うな。	からといって	からといって	
\\	その運動は急速に (進む: ている, 
\\	。	その 運動[うんどう]は 急速[きゅうそく]に 進[すす]みつつある。	つつある	つつある	
\\	喧嘩 (する: 
\\	どうにもならないだろ。	喧嘩[けんか]したところでどうにもならないだろ。	したところで	たところで	
\\	、アメリカ人は大きな成功を夢見て、危険を冒す傾向が強い。	一方[いっぽう]、アメリカ 人[じん]は 大[おお]きな 成功[せいこう]を 夢見[ゆめみ]て、 危険[きけん]を 冒[おか]す 傾向[けいこう]が 強[つよ]い。	一方・その一方では	一方	
\\	上司に相談 (する: 
\\	お返事できません。	上司[じょうし]に 相談[そうだん]してからでないとお 返事[へんじ]できません。	してからでないと	てからでないと	
\\	彼女は失敗を私
\\	責める)。	彼女[かのじょ]は 失敗[しっぱい]を 私[わたし]のせいにした。	のせいにした	のせいにした	
\\	流言飛語に惑わされるべからず、
\\	、言うは易く行うは難し、と思わない?	流言飛語[りゅうげんひご]に 惑[まど]わされるべからず、とはいうものの、 言[い]うは 易[やす]く 行[おこな]うは 難[かた]し、と 思[おも]わない?	とはいうものの	とはいうものの	
\\	いつもの列車に乗り遅れた 
\\	。	いつもの 列車[れっしゃ]に 乗り遅[のりおく]れたものですから。	ものですから	ものですから	
\\	私の立場 
\\	当然でしょう。	私[わたし]の 立場[たちば]からみると 当然[とうぜん]でしょう。	からみると	からみると	
\\	ホストファミリーは成田空港まで迎えに来てくれた。それ (だけでなく, 
\\	毎日いろいろ面白い所に連れて行ってくれた。	ホストファミリーは 成田空港[なりたくうこう]まで 迎[むか]えに 来[き]てくれた。そればかりでなく 毎日[まいにち]いろいろ 面白[おもしろ]い 所[ところ]に 連[つ]れて 行[い]ってくれた。	ばかりでなく	ばかりでなく	
\\	貧乏 (たとえ~ても, ~でも, 
\\	盗みはいけないよ!	貧乏[びんぼう]にせよ 盗[ぬす]みはいけないよ!	にしても・にしろ
\\	・にせよ
\\	にしても・にしろ・にせよ	
\\	田中 (やっぱり、
\\	、遅れるでしょう.	田中[たなか]、 遅[おく]れるでしょう.	のことだから	のことだから	
\\	母の手紙にはいつも体を大切に (する: 
\\	書いてあります。	母[はは]の 手紙[てがみ]にはいつも 体[からだ]を 大切[たいせつ]にしろと 書[か]いてあります。	しろと	しろと	
\\	本日、退社することになりました社長 
\\	皆様には本当にお世話になりました	本日[ほんじつ]、 退社[たいしゃ]することになりました 社長[しゃちょう]を 初[はじ]め 皆様[みなさま]には 本当[ほんとう]にお 世話[せわ]になりました	を初め	を初め	
\\	外国で病気になる (不安: 
\\	。	外国[がいこく]で 病気[びょうき]になるくらい 不安[ふあん]なことはない。	くらい不安なことはない	くらい~ことはない	
\\	大きな仕事は十分な準備が (ある: 
\\	、成功するのだ。	大[おお]きな 仕事[しごと]は 十分[じゅうぶん]な 準備[じゅんび]があってはじめて、 成功[せいこう]するのだ。	あってはじめて	てはじめて	
\\	このメロンはおいしい。高かった 
\\	。	このメロンはおいしい。 高[たか]かっただけのことはある。	だけのことはある	だけのことはある	
\\	本日は夏 
\\	冷たいお料理をいくつかご紹介します。	本日[ほんじつ]は 夏向[なつむ]きに 冷[つめ]たいお 料理[りょうり]をいくつかご 紹介[しょうかい]します。	向きに	向きに	
\\	試験にパスできなくて、悔しい (非常に...) だ。	試験[しけん]にパスできなくて、 悔[くや]しい 限[かぎ]りだ。	限り	限りだ	
\\	顔 (からすると 
\\	、彼女がミス日本になると思います。	顔[かお]からいうと、 彼女[かのじょ]がミス 日本[にっぽん]になると 思[おも]います。	から言うと・から言えば・から言って	からいうと	
\\	命令のしかたが人によって違うのでは、命令される犬 
\\	困ってしまうだろう。	命令[めいれい]のしかたが 人[ひと]によって 違[ちが]うのでは、 命令[めいれい]される 犬[いぬ]にしたって 困[こま]ってしまうだろう。	にしたって	にしたって	
\\	これからは少しでも人の役に立つことを (考える: 
\\	。	これからは 少[すこ]しでも 人[ひと]の 役に立[やくにた]つことを 考[かんが]えようではないか。	考えようではないか	ようではないか	
\\	携帯電話を使う人の数は増える 
\\	です	携帯[けいたい] 電話[でんわ]を 使[つか]う 人[ひと]の 数[かず]は 増[ふ]える 一方[いっぽう]です	一方	一方	
\\	寒かった 
\\	、風が強かった。	寒[さむ]かった上に、 風[かぜ]が 強[つよ]かった。	上に	上に	
\\	そういう明白な事実があるのに、非を (認める: 
\\	なんてとんでもない奴だ。	そういう 明白[めいはく]な 事実[じじつ]があるのに、 非[ひ]を 認[みと]めようとしないなんてとんでもない 奴[やっこ]だ。	認めようとしない	ようとしない	
\\	父も母ものんびり過ごしています。私 
\\	、毎日ただ忙しく働いています。	父[ちち]も 母[はは]ものんびり 過[す]ごしています。 私[わたし]はというと、 毎日[まいにち]ただ 忙[いそが]しく 働[はたら]いています。	はというと	はというと	
\\	(たとえ)どんなことがあった 
\\	。	たとえどんなことがあったとしても....	たとえ~ても	たとえ~ても	
\\	先生の指導 
\\	試験に受かる事が出来ました。	先生[せんせい]の 指導[しどう]のもとに 試験[しけん]に 受[う]かる 事[こと]が 出来[でき]ました。	のもとに・のもとで	のもとに・~のもとで	
\\	警察 (にしたって 
\\	取り締まりの方法がないだろう。	警察[けいさつ]にしたところで 取り締[とりし]まりの 方法[ほうほう]がないだろう。	にしたところで	にしたところで	
\\	英語は商業 
\\	役に立つ。	英語[えいご]は 商業[しょうぎょう]において 役に立[やくにた]つ。	において	において	
\\	留学について父に相談して (みる: 
\\	、喜んで賛成してくれた。	留学[りゅうがく]について 父[ちち]に 相談[そうだん]して みたところ、 喜[よろこ]んで 賛成[さんせい]してくれた。	みたところ	たところ	
\\	数学 (に関して, 
\\	彼女はクラスでいつも一番だった	数学[すうがく]にかけては 彼女[かのじょ]はクラスでいつも 一番[いちばん]だった	にかけては	にかけては	
\\	買っちゃった!	つい 買[か]っちゃった!	つい	つい~	
\\	よく (考える: 
\\	ご返事いたします。	よく 考[かんが]えた 上[うえ]でご 返事[へんじ]いたします。	考えた上で	上で	
\\	動物を飼う 
\\	責任をもって世話をするべきです。	動物[どうぶつ]を 飼[か]う 上[うえ]は 責任[せきにん]をもって 世話[せわ]をするべきです。	上は	上は	
\\	部屋の電気がまだついている 
\\	、森さんはまだ起きているようだ。	部屋[へや]の 電気[でんき]がまだついているところをみると、 森[もり]さんはまだ 起[お]きているようだ。	ところをみると	ところをみると	
\\	宿題 (や, 
\\	レポート (や, 
\\	で、のんびりする暇もない。	宿題[しゅくだい]やらレポートやらで、のんびりする 暇[ひま]もない。	やら	やら~やら	
\\	大統領が中国に行ったの 
\\	万里の長城を訪問しておりました。	大統領[だいとうりょう]が 中国[ちゅうごく]に 行[い]ったのを 契機[けいき] に万里[ばんり]の 長城[ちょうじょう]を 訪問[ほうもん]しておりました。	を契機に	を契機に	
\\	あの男はどんなばかなことでも (やる: 
\\	。	あの 男[おとこ]はどんなばかなことでもやりかねない。	やりかねない	かねない	
\\	情報を教えて (頂く: 
\\	検討のしようがありません。	情報[じょうほう]を 教[おし]えていただけないことには 検討[けんとう]のしようがありません。	いただけないことには	ないことには	
\\	会を欠席する 
\\	。	会[かい]を 欠席[けっせき]するわけにはいきません。	わけにはいかない	わけにはいかない	
\\	嬉しい 
\\	4月から、給料が1割りほど上がる。	嬉[うれ]しいことに 4月[しがつ]から、 給料[きゅうりょう]が1 割[わ]りほど 上[あ]がる。	ことに	ことに	
\\	(決める: 
\\	、やれるところまでやった方がいい。	一度[いちど] 決[き]めたら、やれるところまでやった 方[ほう]がいい。	一度決めたら	一度~たら~	
\\	日本語は (勉強する: 
\\	面白くなります。	日本語[にほんご]は 勉強[べんきょう]すれば 勉強[べんきょう]するほど 面白[おもしろ]くなります。	勉強すれば勉強するほど	ば~ほど	
\\	どんなこともお金で解決できる 
\\	。	どんなこともお 金[かね]で 解決[かいけつ]できる、というものではありません。	というものではありません	というものではない	
\\	花子は食べる 
\\	、急に元気になる。	花子[はなこ]は 食[た]べることとなると、 急[きゅう]に 元気[げんき]になる。	こととなると	こととなると	
\\	費用の問題 
\\	、旅行の目的地を決める方が先です。	費用[ひよう]の 問題[もんだい]はともかく、 旅行[りょこう]の 目的[もくてき] 地[ち]を 決[き]める 方[ほう]が 先[さき]です。	はともかく(として)	はともかく	
\\	会話 (で, 
\\	日本語覚えてます。	会話[かいわ]を 通[つう]じて 日本語[にほんご] 覚[おぼ]えてます。	を通じて	を通じて	
\\	冗談は 
\\	冗談[じょうだん]は 抜[ぬ]きにして…	抜きにして	抜きにして	
\\	ぼくをおいて (行く: 
\\	ぼくをおいて 行[い]ってやがったんだ!	行ってやがったんだ	てやがる	
\\	悪いと (知る: 
\\	も、ごみを分別せずに捨ててしまう。	悪[わる]いと 知[し]りつつも、ごみを 分別[ふんべつ]せずに 捨[す]ててしまう。	知りつつ	つつ	
\\	5月末までに問題を解決すると約束 (する: 
\\	、どうしても頑張らなければならない。	
\\	月[つき] 末[まつ]までに 問題[もんだい]を 解決[かいけつ]すると 約束[やくそく]した 手前[てまえ]、どうしても 頑張[がんば]らなければならない。	した手前	手前	
\\	この仕事は一時間 
\\	十ドル貰える。	この 仕事[しごと]は一 時間[じかん]につき十ドル 貰[もら]える。	につき	につき	
\\	工事 (のため, 
\\	電車は不通です。	工事[こうじ]につき 電車[でんしゃ]は 不通[ふつう]です。	につき	につき	
\\	我々は決定 
\\	偏見に左右されない。	我々[われわれ]は 決定[けってい]にあたって 偏見[へんけん]に 左右[さゆう]されない。	にあたって	にあたって・にあたり	
\\	決められた時間より早めに着く 
\\	。	決[き]められた 時間[じかん]より 早[はや]めに 着[つ]くにこしたことはない。	にこしたことはない	にこしたことはない	
\\	このドキュメントは間違え 
\\	で、全然役に立たない。	このドキュメントは 間違[まちが]えだらけで、 全然[ぜんぜん] 役に立[やくにた]たない。	だらけ	だらけ	
\\	公園へ行くより (実は, 
\\	動物園へ行きたい。	公園[こうえん]へ 行[い]くよりむしろ 動物[どうぶつ] 園[えん]へ 行[い]きたい。	むしろ	むしろ	
\\	この本には金を払った 
\\	価値がない。	この 本[ほん]には 金[きん]を 払[はら]っただけの 価値[かち]がない。	だけの	だけの	
\\	買い物 (する: 
\\	。	買い物[かいもの]せずにはいられない。	しないでいられない	ないでいられない	
\\	天気予報がいつも当たる 
\\	。	天気[てんき] 予報[よほう]がいつも 当[あ]たるとは 限[かぎ]らない。	とは限らない	とは限らない	
\\	どんな人(でも, 
\\	、この映画は楽しめます。	どんな 人[ひと]であろうと、この 映画[えいが]は 楽[たの]しめます。	であろうと	であろうと	
\\	手紙を貰った (まだそのままの状態)、連絡がない。	手紙[てがみ]を 貰[もら]ったきり、 連絡[れんらく]がない。	きり	きり	
\\	この布は水に強い 
\\	、熱には弱い.	この 布[ぬの]は 水[みず]に 強[つよ]い 反面[はんめん]、 熱[ねつ]には 弱[よわ]い.	反面	反面	
\\	実家に帰った 
\\	昔の学校に行ってみた。	実家[じっか]に 帰[かえ]ったついでに 昔[むかし]の 学校[がっこう]に 行[い]ってみた。	ついでに	ついでに	
\\	彼は年 
\\	老けて見える。	彼[かれ]は 年[とし]のわりに 老[ふ]けて 見[み]える。	のわりに	のわりに	
\\	彼はよく間違いをするが、馬鹿 
\\	。	彼[かれ]はよく 間違[まちが]いをするが、 馬鹿[ばか]どころではない。	どころではない	どころではない	
\\	忙しくて遊ぶ 
\\	。	忙[いそが]しくて 遊[あそ]ぶどころではない。	どころではない	どころではない	
\\	新車を(買う: 
\\	、お金がありません。	新車[しんしゃ]を 買[か]ったばかりに、お 金[かね]がありません。	買ったばかりに	たばかりに	
\\	うそを (つく: 
\\	彼に嫌われた。	うそをついたばかりに 彼[かれ]に 嫌[きら]われた。	ついたばかりに	たばかりに	
\\	ノーベル賞なんて、僕は (貰える: 
\\	。	ノーベル 賞[しょう]なんて、 僕[ぼく]は 貰[もらい]えっこない。	貰えっこない	っこない	
\\	彼がそのゲームを (やる: 
\\	。	彼[かれ]がそのゲームをやりこなしました。	こなしました	こなす	
\\	彼は天才である (に違いない 
\\	。	彼[かれ]は 天才[てんさい]であるに 相違[そうい]ない。	に相違ない	に相違ない	
\\	正月は(食べる: 
\\	寝,(飲む: 
\\	寝の日を過ごしていた。	正月[しょうがつ]は 食[た]べては 寝[ね], 飲[の]んでは 寝[ね]の 日[ひ]を 過[す]ごしていた。	食べては寝, 飲んでは寝	ては	
\\	身を (捨てる: 
\\	浮かぶ、瀬もあれ。	身[み]を 捨[す]ててこそ 浮[う]かぶ、 瀬[せ]もあれ。	捨ててこそ	てこそ	
\\	善人が 
\\	成功するとは限らない。	善人[ぜんにん]が 必[かなら]ずしも 成功[せいこう]するとは 限[かぎ]らない。	必ずしも	必ずしも	
\\	人口の急増 
\\	、いろいろの問題が出てきた。	人口[じんこう]の 急増[きゅうぞう]に 伴[ともな]って、いろいろの 問題[もんだい]が 出[で]てきた。	に伴って	に伴って	
\\	ちょっと話した 
\\	、彼はいつもとまったく変わらないように思えた。	ちょっと 話[はな]したかぎりでは、 彼[かれ]はいつもとまったく 変[か]わらないように 思[おも]えた。	かぎりでは	かぎりでは	
\\	ダイエットしていたのに、痩せる 
\\	、太ってしまいました。	ダイエットしていたのに、 痩[や]せるどころか、 太[ふと]ってしまいました。	どころか	どころか	
\\	金の問題 
\\	、そういう旅行は疲れるよ。	金[きん]の 問題[もんだい]はさておき、そういう 旅行[りょこう]は 疲[つか]れるよ。	はさておき	はさておき	
\\	彼が日本 (だけでなく 
\\	、世界においても有名な物理学者である。	彼[かれ]が 日本[にっぽん]のみならず、 世界[せかい]においても 有名[ゆうめい]な 物理[ぶつり] 学者[がくしゃ]である。	のみならず	のみならず	
\\	この魚は猛毒を持っていて、間違ってそのトゲに (刺される: 
\\	、患部が大きく腫れ上がってしまう。	この 魚[さかな]は 猛毒[もうどく]を 持[も]っていて、 間違[まちが]ってそのトゲに 刺[さ]されようものなら、 患部[かんぶ]が 大[おお]きく 腫れ上[はれあ]がってしまう。	刺されようものなら	ようものなら	
\\	上司 (の代わりに 
\\	会議に出た。	上司[じょうし]に 代[か]わって 会議[かいぎ]に 出[で]た。	に代わって	に代わって	
\\	この条件では拒絶 
\\	。	この 条件[じょうけん]では 拒絶[きょぜつ]に 等[ひと]しい。	に等しい	に等しい	
\\	やると決めた 
\\	最後までやれ	やると 決[き]めたからには 最後[さいご]までやれ	からには・からは	からには	
\\	昨夜は徹夜したので、今日は (眠い: 
\\	。	昨夜[さくや]は 徹夜[てつや]したので、 今日[きょう]は 眠[ねむ]くてしかたがない。	眠くてしかたがない	てしかたがない	
\\	運動をしてもっと健康になる 
\\	。	運動[うんどう]をしてもっと 健康[けんこう]になるようにする。	ようにする	ようにする	
\\	私はあなたを助けてくれる人が居る (一人: 
\\	事を祈ります。	私[わたし]はあなたを 助[たす]けてくれる 人[ひと]が 一人[ひとり]でも 多[おお]く 居[い]る 事[こと]を 祈[いの]ります。	一人でも多く	一人でも多く	
\\	たった1キロを走っただけで、汗 
\\	になるのは情けない。	たった1キロを 走[はし]っただけで、 汗[あせ]まみれになるのは 情[なさ]けない。	まみれ	まみれ	
\\	友達を傷つける 
\\	、 (自分が我慢する: 
\\	。	友達[ともだち]を 傷[きず]つけるくらいなら、 自分[じぶん]が 我慢[がまん]したほうがいい。	くらいなら、自分が我慢したほうがいい	くらいなら~ほうがいい	
\\	山に(登る: ~ながら, 
\\	、人は人生についてさまざまなことを考える。	山[やま]に 登[のぼ]りつつ、 人[ひと]は 人生[じんせい]についてさまざまなことを 考[かんが]える。	登りつつ	つつ	
\\	生徒は学校もともかく (休む: 
\\	だ。	生徒[せいと]は 学校[がっこう]もともかく 休[やす]みがちだ。	休みがち	がち	
\\	彼は料理 (~はもちろん 
\\	食器も洗う。	彼[かれ]は 料理[りょうり]はもとより 食器[しょっき]も 洗[あら]う。	はもとより	はもとより	
\\	散々(考える: 
\\	、 その計画を実行にうつした	散々[さんざん] 考[かんが]えた 挙げ句[あげく]、その 計画[けいかく]を 実行[じっこう]にうつした	考えたあげく	たあげく	
\\	彼の引っ越し先はなんと人口1000人の小さな孤島 
\\	。	彼[かれ]の 引っ越[ひっこ]し 先[さき]はなんと 人口[じんこう] 
\\	人[にん]の 小[ちい]さな 孤島[ことう]ではないか。	ではないか	ではないか	
\\	山田さんの部屋は何か月も掃除していない 
\\	汚い。	山田[やまだ]さんの 部屋[へや]は 何[なん]か 月[げつ]も 掃除[そうじ]していないかのように 汚[きたな]い。	かのように	かのよう	
\\	ドアが (開く: 
\\	が、また閉まってしまった。	ドアが 開[ひら]きかかったが、また 閉[し]まってしまった。	開きかかった	かかる	
\\	両親は年をとっている
\\	、娘としては心配だ。	両親[りょうしん]は 年[とし]をとっているだけに、 娘[むすめ]としては 心配[しんぱい]だ。	だけに	だけに	
\\	電車のドアが (開く: 
\\	乗客が飛び出して来る	電車[でんしゃ]のドアが 開[ひら]くか 開[ひら]かないかのうちに 乗客[じょうきゃく]が 飛び出[とびだ]して 来[く]る	開くか開かないかのうちに	か~ないかのうちに	
\\	「日本」
\\	、何を連想しますか。	
\\	日本[にっぽん]」というと、 何[なに]を 連想[れんそう]しますか。	というと	というと	
\\	子供のことが心配 
\\	。	子供[こども]のことが 心配[しんぱい]でたまらない。	で堪らない	てたまらない	
\\	運動 (する: 
\\	、歩いて行った。	運動[うんどう]したつもりで、 歩[ある]いて 行[い]った。	したつもりで	たつもりで	
\\	昼飯 
\\	仕事をした.	昼飯[ひるめし]ぬきで 仕事[しごと]をした.	ぬきで	ぬきで	
\\	完全を求めると 
\\	失敗する。	完全[かんぜん]を 求[もと]めるとかえって 失敗[しっぱい]する。	かえって	かえって~	
\\	卒業 
\\	、私たちは感謝の印として彼に置き時計を贈りました。	卒業[そつぎょう]に 際[さい]して、 私[わたし]たちは 感謝[かんしゃ]の 印[しるし]として 彼[かれ]に 置き時計[おきどけい]を 贈[おく]りました。	に際して	に際して	
\\	学歴、経験は 
\\	、やる気のある社員を募集します。	学歴[がくれき]、 経験[けいけん]は 問[と]わず、やる 気[き]のある 社員[しゃいん]を 募集[ぼしゅう]します。	問わず	問わず	
\\	人の迷惑 
\\	電車の中で携帯電話で話している人がいる。	人[ひと]の 迷惑[めいわく]もかまわず 電車[でんしゃ]の 中[なか]で 携帯[けいたい] 電話[でんわ]で 話[はな]している 人[ひと]がいる。	も構わず	も構わず	
\\	背が高い (のに 
\\	早く走れない。	背[せ]が 高[たか]いくせに 早[はや]く 走[はし]れない。	くせに	くせに	
\\	弟 
\\	、私のようなうるさい姉はいやでたまらない。	弟[おとうと]にしたら、 私[わたし]のようなうるさい 姉[あね]はいやでたまらない。	にしたら	にしたら	
\\	6月 
\\	寒い朝だ。	6月[ろくがつ]にしては 寒[さむ]い 朝[あさ]だ。	にしては	にしては	
\\	山は高くなる 
\\	、気温が下がる。	山[やま]は 高[たか]くなるにつれて、 気温[きおん]が 下[さ]がる。	につれて	につれて	
\\	あきらめる 
\\	。最後までがんばりぬくぞ。	あきらめるものか。 最後[さいご]までがんばりぬくぞ。	ものか	ものか	
\\	日本経済がこのまま下降線をたどる 
\\	、日本語ブームもやがては終わるかもしれない。	日本[にっぽん] 経済[けいざい]がこのまま 下降[かこう] 線[せん]をたどるとなると、 日本語[にほんご]ブームもやがては 終[お]わるかもしれない。	となると	となると	
\\	これは何かの間違い (ではないか 
\\	。	これは 何[なに]かの 間違[まちが]いではあるまいか。	ではあるまいか	ではあるまいか	
\\	そのような条件には (応じる: 
\\	。	そのような 条件[じょうけん]には 応[おう]じかねる。	応じかねる	かねる	
\\	あの男が犯人だ 
\\	、警察はつかまえるチャンスを逃したことになる。	あの 男[おとこ]が 犯人[はんにん]だとすると、 警察[けいさつ]はつかまえるチャンスを 逃[のが]したことになる。	とすると	とすると	
\\	東京の夏は (暑い: 
\\	。	東京[とうきょう]の 夏[なつ]は 暑[あつ]くてならない。	暑くてならない	てならない	
\\	それ 
\\	、なかなかユニークなキャラクターだ。	それに 加[くわ]えて、なかなかユニークなキャラクターだ。	に加えて	に加えて	
\\	今日は月曜日 
\\	明日は火曜日だ。	今日[きょう]は 月曜日[げつようび]だとしたら, 明日[あした]は 火曜日[かようび]だ。	だとしたら	としたら	
\\	大学に合格して、(嬉しい: 
\\	。	大学[だいがく]に 合格[ごうかく]して、嬉しくてたまらない。	嬉しくてたまらない	たまらない	
\\	試合 
\\	開会式を行います。	試合[しあい]に 先立[さきだ]って 開会[かいかい] 式[しき]を 行[おこな]います。	に先立って	に先立って	
\\	子供のとき、よく川で (遊ぶ: 
\\	。	子供[こども]のとき、よく 川[かわ]で 遊[あそ]んだものだ。	遊んだものだ	たものだ	
\\	彼女がそんなひどいことを言う 
\\	。	彼女[かのじょ]がそんなひどいことを 言[い]うわけがない。	わけがない	わけがない	
\\	帰国するというのは、さんざん (迷う: あげく, 
\\	出した結論です。	帰国[きこく]するというのは、さんざん 迷[まよ]った 末[すえ]に 出[だ]した 結論[けつろん]です。	迷った末に	すえに	
\\	信頼する 
\\	人物だと思っていた、その人にだまされてしまった	信頼[しんらい]するに 足[た]る 人物[じんぶつ]だと 思[おも]っていた、その 人[ひと]にだまされてしまった	に足る	に足る	
\\	雨が (降ってくる: 
\\	もうやんだ。	雨[あめ]が 降[ふ]ってきたかと 思[おも]うともうやんだ。	降ってきたかと思うと	かと思うと~	
\\	最近、忙しい仕事が続いたので少し (疲れる: 
\\	です。	最近[さいきん]、 忙[いそが]しい 仕事[しごと]が 続[つづ]いたので 少[すこ]し 疲[つか]れ 気味[ぎみ]です。	疲れ気味	気味	
\\	1日の仕事を終えた後は、冷えたビール 
\\	よ。	
\\	日[にち]の 仕事[しごと]を 終[お]えた 後[のち]は、 冷[ひ]えたビールに 限[かぎ]るよ。	に限る	に限る	
\\	失礼 (の極まり, 
\\	。	失礼[しつれい] 極[きわ]まりない。	極まりない	極まりない	
\\	支払いは今すぐで (ない: なくても, 
\\	。後でもいいですよ。	支払[しはら]いは 今[いま]すぐでなくてもさしつかえありません。 後[のち]でもいいですよ。	なくてもさしつかえありません	なくてもさしつかえない	
\\	仕事 
\\	、子供達にフランス語を教えるつもりです。	仕事[しごと]のかたわら、 子供[こども] 達[たち]に フランス語[ふらんすご]を 教[おし]えるつもりです。	のかたわら	かたわら	
\\	テーブルの上の (飲む: 
\\	コーヒー、誰の?	テーブルの 上[うえ]の 飲[の]みかけのコーヒー、 誰[だれ]の?	飲みかけの	かけの~	
\\	戦争は死 
\\	。	戦争[せんそう]は 死[し]にほかならない。	にほかならない	にほかならない	
\\	妻の病気 
\\	、家族の結束は再び固まった。	妻[つま]の 病気[びょうき]をきっかけに、 家族[かぞく]の 結束[けっそく]は 再[ふたた]び 固[かた]まった。	をきっかけに	をきっかけに	
\\	(早い: 
\\	ありがたいです。	早[はや]ければ 早[はや]いほどありがたいです	早ければ早いほど	ば~ほど	
\\	白 
\\	団体は去年ニューズになっていた。	白[しろ]ずくめ 団体[だんたい]は 去年[きょねん]ニューズになっていた。	ずくめ	ずくめ	
\\	いいこと 
\\	ですよ。	いいことずくめですよ。	ずくめ	ずくめ	
\\	私が料理すると、(味が濃い、薄い: 
\\	家族から文句がでる。	私[わたし]が 料理[りょうり]すると、 味[あじ]が 濃[こ]いにつけ、 薄[うす]いにつけ 家族[かぞく]から 文句[もんく]がでる。	味が濃いにつけ、薄いにつけ	につけ	
\\	このほうが 
\\	ほど優れている。	このほうが 比[くら]べものにならないほど 優[すぐ]れている。	比べものにならない	比べものにならない	
\\	父は (お酒を飲む: 
\\	タバコも吸うので、健康が心配だ。	父[ちち]はお お酒[さけ]ものめばタバコも 吸[す]うので、 健康[けんこう]が 心配[しんぱい]だ。お	お酒も飲めば	も~ば~も	
\\	(頑張る: 
\\	頑張[がんば]りぬく	頑張りぬく	ぬく	
\\	課長にこう毎晩のように飲みに (誘われる: 
\\	。	課長[かちょう]にこう 毎晩[まいばん]のように 飲[の]みに 誘[さそ]われてはかなわない。	誘われてはかなわない。	てはかなわない。	
\\	このことは今度お目にかかった 
\\	詳しくお話しいたします。	このことは 今度[こんど]お 目[め]にかかった 折[おり]に 詳[くわ]しくお 話[はな]しいたします。	折に	折に	
\\	ここにゴミを捨てる 
\\	。	ここにゴミを 捨[す]てるべからず。	べからず	べからず	
\\	あの映画も終わってしまった。あんなに見たいと思っていたのに、とうとう (見る: 
\\	だった。	あの 映画[えいが]も 終[お]わってしまった。あんなに 見[み]たいと 思[おも]っていたのに、とうとう 見[み]ずじまいだった。	見ずじまい	ずじまい	
\\	雑誌を (読む: 
\\	、そのままうとうと寝てしまった。	雑誌[ざっし]を 読[よ]みかけて、そのままうとうと 寝[ね]てしまった。	読みかけて	かける	
\\	夜遅く雨が降った (らしい, 
\\	、庭が濡れている。	夜[よる] 遅[おそ]く 雨[あめ]が 降[ふ]ったとみえて、 庭[にわ]が 濡[ぬ]れている。	とみえて	とみえて	
\\	(ハンサム: 
\\	恋人を探しやすいです。	ハンサムならばハンサムなほど 恋人[こいびと]を 探[さが]しやすいです。	ハンサムならばハンサムなほど	ば~ほど	
\\	日本へ (行く: 
\\	迷った。	日本へ行こうか行くまいか迷った。	行こうか行くまいか	ようか~まいか	
\\	そんなことは有り得ない	よもやそんなことは 有り得[ありえ]ない	よもや	よもや	
\end{CJK}
\end{document}