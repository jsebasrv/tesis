\documentclass[8pt]{extreport} 
\usepackage{hyperref}
\usepackage{CJKutf8}
\begin{document}
\begin{CJK}{UTF8}{min}
\\	2年	
\\	【名-1】上記[上述・前記]のこと 【名-2】天上、上層部、舞台奥 【前-1】(基準よりも)上側に、~の上に
\\	のように“接触”していない。
\\	のように必ずしも“真上”にあるとは限らない。 【前-2】~より上に[で]、~を越えて、~を超越して、~を超過して◆位置レベル・立場・身分などの上下関係を表す 【前-3】~に勝って 【前-4】~の奥手に、~の後ろに 【形】上記の、上述の、前記の
\\	2年	
\\	【自動-1】(~のように)見える、~と思われる、~らしい 【自動-2】現れる、登場する 【自動-3】出頭する
\\	2年	
\\	【副】注意深く、丁寧に、慎重に、入念に
\\	2年	
\\	【動】
\\	の過去・過去分詞形 
\\	コート
\\	2年	
\\	【名-1】教会◆英では国教会(他の宗派の教会は 
\\	。キリスト教会の構造は入り口から、
\\	(身廊、参拝者のいすが並んでいる場所)、
\\	(内陣、祭壇の周囲・牧師の席)、
\\	(祭壇)、
\\	(後陣、祭壇の後ろ)、
\\	の奥横に 
\\	(袖廊、左右の廊)。 【名-2】(教会での)礼拝 【名-3】聖職 【名-4】すべてのキリスト教徒 【他動】産後礼拝をする◆赤ちゃんが産まれた時の。
\\	2年	
\\	【形-1】利口な、賢い、頭がさえている、思い付きのいい、才気縦横の、上手な、得意な 【形-2】巧妙な、巧みな
\\	2年	
\\	【名-1】単科大学、総合大学の学部、大学、専門学校◆フランス語でも 
\\	【名-2】学会、団体 【名-3】学寮
\\	2年	
\\	【自動】(~に)匹敵する、同等であるとみなす 【他動-1】~を(…と)比べる、比較する、対比する、対照する 【他動-2】~を(…に)なぞらえる、例える
\\	2年	
\\	【自動】一致しない、意見が合わない、意見を異にする、同意しない、異議を唱える、口論する、相違する
\\	2年	
\\	【名】東、東部◆【略】
\\	【形】東にある、東の 【副】東に[へ]
\\	2年	
\\	【形】日常の、毎日の、ありふれた、平凡な◆【注】この単語は形容詞。「毎日」という意味では 
\\	を用いる。
\\	2年	
\\	【副-1】どこでも、どこにも、いずれの場所においても、所かまわず 【副-2】どこで~しても(接続詞的用法)
\\	2年	
\\	【名】事実、現実、真実、実際、真相
\\	2年	
\\	【名】農場、ファーム、農地、飼育場 【自他動】耕す、耕作する
\\	2年	
\\	【1-自動】
\\	の過去形 【2-他動-1】《裁縫》折り縫いする、伏せ縫いする 【2-他動-2】倒す、打ち倒す、投げ倒す 【2-他動-3】(木を)切り倒す、伐採する 【3-形-1】残忍な、凶暴な、どう猛な 【3-形-2】致命的な、破壊的な 【3-形-3】不吉な、不運な 【3-形】残忍な、凶暴な、致命的な 【4-名】皮、獣皮、毛皮 
\\	フェル、【変化】《動》
\\	2年	
\\	【名】十分、充填量 【自動】満ちる、いっぱいになる 【他動-1】~を…に詰める、注入する、満たす、~を…でいっぱいにする 【他動-2】~を補充する 【他動-3】(条件・要求・需要など)を満たす、果たす、~に応じる 【他動-4】(処方せんに従って薬)を調合する 【他動-5】~を塗りつぶす、塗り込む 【他動-6】~に記入する◆【用法】
\\	2年	
\\	【名】飛行 【形】飛んでいる、飛行する、空を飛ぶ
\\	2年	
\\	【名】永遠 【副-1】永遠に、永久に 【副-2】絶えず、ずっと
\\	2年	
\\	【名】《動物》カエル
\\	2年	
\\	【名】完全、十分、全盛、全部 【形-1】いっぱいの、一杯になる、充満する、満ちた、豊かな、満員の、全部の、全面的な、徹底した、完全な、真っ盛りの、最大限の、たっぷりの、満腹で 【形-2】ふっくらした 【副】十分に、まともに
\\	2年	
\\	【名】くず、生ごみ、あら、がらくた、残飯、廃物
\\	2年	
\\	【名-1】案内人、ガイド、案内者 【名-2】案内書、手引書◆【同】
\\	【名-3】指導者 【名-4】指針 【他動-1】~を導く、案内する 【他動-2】~を指導する、相談に乗る
\\	2年	
\\	【名-1】穴、穴部、穴傷、割れ目 【名-2】巣穴 【名-3】欠点、矛盾点 【名-4】苦境 【名-5】落とし穴 【名-6】《野球》2人の内野手の間、三遊間 【名-7】《半導体》空孔、ホール、正孔◆【参考】
\\	◆半導体内で電子が入れる場所。電子が空孔に飛び込むと逆方向の電流になるのでP型半導体になる。 【他動】穴を開ける
\\	2年	
\\	【地名】ハリウッド◆米カリフォルニア州ロサンゼルス北西部の地区で、映画産業の中心地。アメリカ映画(産業)そのものを指すことも多い。◆【略】
\\	【著作】パンク、ハリウッドを行く◆米1989《著》チャールズ・ブコウスキー
\\	【分節】
\\	2年	
\\	【名】ホームステイ、(留学生の)家庭滞在◆日本語から借用された言葉
\\	2年	
\\	【名-1】主人(役)、ホスト◆家に招待された時、招待してもてなしてくれた人をホストまたはホステスと言う。 【名-2】司会者、司会役、案内役 【名-3】(寄生動植物の)宿主 【名-4】多数、大勢 【名-5】軍隊 【名-6】(旅館などの)亭主 【形】主人の 【自動】集団を組む、(敵対目的で)徒党する 【他動-1】~を主催する、~で主人役を務める 【他動-2】主人役として~を接待する 【他動-3】~の司会をする 【他動-4】客を泊める◆【同】
\\	2年	
\\	【名】夫 【他動】~を節約する
\\	2年	
\\	【名-1】アイスクリーム◆クリーム、砂糖、香味料を混ぜて凍らした食品。◆【略】
\\	【名-2】〈米俗〉アヘン、結晶状の麻薬 【名-3】〈英俗〉人、男 【形】
\\	2年	
\\	【自動】想像する、思う、推測する 【他動】~と想像する、思う、心に描く、推測する、仮定する
\\	2年	
\\	【形】工業の、産業の、産業の発達した
\\	2年	
\\	【名-1】通訳者、解釈者、解説者、通訳、通訳官 【名-2】《コ》インタープリター、解釈プログラム
\\	2年	
\\	【他動-1】~を(…に)紹介する、~を引き合わせる 【他動-2】~を発表する、売り出す 【他動-3】~を(…に)導入する、(アイデア)を取り入れる、持ち込む 【他動-4】~を(…に)案内する
\\	2年	
\\	【動】
\\	の過去・過去分詞形 
\\	ケプト
\\	2年	
\\	【名】キロメートル、《単位・長さ》キロメータ
\\	◆【略】
\\	2年	
\\	【名-1】王、君主、国王 【名-2】大立者
\\	2年	
\\	【名】愉快な[笑いになる・気が晴れる]こと 【自動】笑う、あざける、笑って~を示す
\\	2年	
\\	リビングルーム、居間、居室、茶の間◆【略】
\\	◆米国の場合、リビングルームと言うと、居間として使用していなくても玄関から入ってすぐ左手か右手にある部屋を指すことが多い。
\\	2年	
\\	【名】魔法、奇術、魔術、手品、呪術、魔力、不思議な力 【形-1】魔法の、不思議な 【形-2】素晴らしい◆【同】
\\	2年	
\\	【名-1】瞬間、現在、時、時期 【名-2】力のモーメント 【名-3】見せ場 【接】~するとすぐに◆【用法】
\\	2年	
\\	【名】うってつけの人[もの]、ぴったりの人[もの]、適任者、生まれつき才能のある人 【形-1】自然の[な]、自然のままの、自然的な 【形-2】野生の 【形-3】血のつながった、実の 【形-4】普通の、ありのままの 【形-5】性質[性格・性分・気質]に合っている 【形-6】気取らない、飾らない 【形-7】天然の、無添加の、加工しない 【形-8】当然の 【形-9】生まれつきの才能がある、生まれながらの、生まれつきの
\\	2年	
\\	【名】通知、(事前)通告、予告、注意、警告、情報、注目、掲示 【他動-1】~に気が付く、~に注意する、注目する 【他動-2】~に通知する
\\	2年	
\\	【名】パーセント、百分率、割合
\\	の別の言い方
\\	の場合)=
\\	◆【略】
\\	【形】~パーセント(分)の
\\	2年	
\\	【名】ラジオ、無線通信 【形】無線の 【他動】~に無線連絡する、無線で連絡する
\\	2年	
\\	【名-1】金庫 【名-2】食料品を貯蔵するための戸棚 【形】安全な、無事な、別状がない、無難な 【副】安全に、無事に、別状なく
\\	2年	
\\	【自動】
\\	の過去・過去分詞形 
\\	サット
\\	2年	
\\	【形-1】簡単な、簡素な、単純な、容易な 【形-2】気取らない、質素な、素朴な、無邪気な、純真な 【形-3】単一の 【形-4】まったくの
\\	2年	
\\	【名-1】石、石材◆【大きさの順】
\\	【名-2】宝石 【名-3】〈英〉《重量単位》ストーン◆主に人の体重に用いられる。=14ポンド=6.3503キログラム 【名-4】結石 【名-5】核、(桃や梅の実などの)種 【形】石造りの 【自動】(果物の)種を取る 【他動】~に石を投げつける、石をぶつけて殺す
\\	2年	
\\	【副】突然に、急に、すぐに、いきなり、突如として
\\	2年	
\\	【名-1】セーター 【名-2】汗をかく人、汗をかかせるもの[仕事] 【名-3】搾取者、従業員を低賃金で荷重労働させる雇い主
\\	2年	
\\	【名-1】投げること、投げ、投げ技 【名-2】投げられる距離、投げて届く距離[所] 【名-3】一つ、1回、1個 【他動-1】~を投げる、投じる、投入する、投影する、発射する 【他動-2】~を面食らわせる 【他動-3】(パーティなど)を催す、開く 【他動-4】(試合)を投げる、落とす、負ける 【他動-5】(機械のギヤ・レバー・スイッチ)を入れる、(レバーを動かして)~を連結する[切り離す]、(電流)を切る[入れる]
\\	2年	
\\	【名-1】路線、行路、常軌、進路 【名-2】競技用トラック 【名-3】跡、足跡、通った跡 【名-4】証拠 【名-5】抵抗路、飛跡 【名-6】軌道、鉄道路線、輪距 【名-7】《音楽》曲、トラック◆レコード・磁気テープ・コンパクトディスクなどで、データが線状に記録される部分 【自動】決められたコース[道]を進む 【他動】~を通る、探知する、追跡する、追跡記録する、~の後を追う、(進み具合を)見守る
\\	2年	
\\	【名-1】悩み、困難、トラブル、問題、面倒、困惑、心配、厄介、苦しみ 【名-2】骨折り 【名-3】事故、争議、紛争、混乱、騒乱、騒ぎ 【名-4】不具合、不良、故障、支障、異常 【自動】わざわざ~する、骨を折る、心配する 【他動-1】~に面倒[迷惑・厄介]をかける、骨を折らせる、~を煩わせる 【他動-2】~を困らせる、悩ます、乱す、~に心配をかける
\\	2年	
\\	【名-1】声、発声能力 【名-2】音声、歌声、歌手 【名-3】表明 【名-4】音 【名-5】発言、発言権 【名-6】(天・神の)命令 【名-7】《文法》態 【他動】~を声に出す、言い表す、発言する、声に出してはっきり言う、表明する
\\	2年	
\\	【名-1】波、風浪 【名-2】(手・旗を)振ること 【名-3】ウェーブ 【名-4】波動 【名-5】高まり、強まり、(波のような)押し寄せ 【名-6】波状曲線 【自動-1】振られる、手を振って合図する 【自動-2】揺れる、波立つ 【他動-1】~を振る 【他動-2】~をうねらせる
\\	2年	
\\	【名】ウェブサイト、ホームページ◆インターネット上で情報を公開している場所
\\	2年	
\\	【名】西、西部、西部地方、西洋◆【略】
\\	【形】西にある、西の、西部地方の 【副】西に[へ]
\\	2年	
\\	【形】独りで、ただ~だけで、唯一の、離れて◆【注意】主語が 
\\	の場合、「私たちだけ」または「二人きり」と訳すことがある。機械的に 
\\	=「ひとり」としないように。 【副-1】ひとりで 【副-2】単に
\\	2年	
\\	【副】どこでも、どこかに、どこへでも、どこにも
\\	2年	
\\	【名-1】瓶◆【略】
\\	【名-2】〈英俗〉勇気、自信 【他動】~を瓶に詰める
\\	2年	
\\	【形-1】安い、安価な、安物の 【形-2】けちな 【副】廉価で
\\	2年	
\\	【名-1】押さえ具、止める人、止め具、妨げる人 【名-2】抑制、阻止、停止、妨害、防止 【名-3】チェック、点検、検査、調査、監督、管理、観察、査照、試験、照合の印、照合の基準 【名-4】勘定書、おあいそ(う)、請求書、伝票 【名-5】小切手、受取総額 【名-6】預かり札、合札 【名-7】チェック柄の織物、チェック柄の布、格子縞(模様)、市松模様、碁盤模様 【名-8】ひび、割れ、割れ目、切込み、裂け目 【名-9】チップ、王手、数取り 【形】照合用の 【自動-1】(~を)チェックする、調べる、調査する、照査する、照合する、(人に)確認する、(~と)相談する 【自動-2】(~と正しく)一致する 【自動-3】宿泊手続きをする、搭乗手続きをする、出勤する、〈俗〉死ぬ◆【用法】
\\	【自動-4】退職する、退社する、出る、〈俗〉死ぬ◆【用法】
\\	【自動-5】古い資料をあたる、折り返し連絡する、再び会う◆【用法】
\\	【他動-1】~を…と照査する、照合する、~を…で確認する 【他動-2】~をチェックする、調べる、調査する、問い合わせる、調べ上げる、~ということを点検する、確認する、確かめる 【他動-3】チェックマークを付ける、照合印を付ける 【他動-4】(荷物など)を預ける 【他動-5】~を急に止める、中止させる、阻止する 【他動-6】~を抑制する、《野球》~を抑える 【他動-7】小切手で引き出す、~を(給料から)天引きする 【他動-8】(~を)手続きして返す◆【用法】
\\	【他動-9】~を持ち出す、~を受け取る◆【用法】
\\	【他動-10】~の健康診断をする◆【用法】
\\	2年	
\\	【名】事務員、店員、職員、会社員、書記
\\	2年	
\\	【名-1】コミュニケーション、(情報)伝達、通信、連絡、交信 【名-2】伝染
\\	2年	
\\	【形-1】便利な、使いやすい、使い勝手が良い、重宝な 【形-2】都合の良い、うってつけの、あつらえ向きの
\\	2年	
\\	【名】パパ、お父さん、親父
\\	2年	
\\	【代名】どちら 【接】~か~、または 【形-1】(二者のうち)どちらか一方の 【形-2】どちらの~でも、どちらの~も 【副】~も~しない
\\	2年	
\\	【名-1】飛ぶこと、定期航空便、フライト、飛行、飛翔、航空、空の旅、(飛行機の)便、便名 【名-2】逃亡、逃走 【名-3】(時の)経過 【名-4】(想像などの)飛躍、躍動、(才気の)ほとばしり 【名-5】群れ、集団行動、渡り◆飛んでいる鳥・昆虫などの
\\	2年	
\\	【名-1】入り口、門扉、ゲート、門、門に似た狭い通路 【名-2】乗り場、(飛行機の)搭乗ゲート
\\	2年	
\\	【名-1】寄り集まり、収穫量、集めること 【名-2】ギャザー、ひだ 【自動-1】(寄り)集まる 【自動-2】集結する、固まる 【自動-3】増す、次第に募る 【自動-4】膿む、腫れ上がる 【自動-5】皺が寄る 【自動-6】(雲などが)出てくる[垂れ込める] 【他動-1】~を(拾い)集める、かき集める、寄せ集める、収集する 【他動-2】~を収穫する、採取する、摘み取る、~狩りをする、~を取り入れる 【他動-3】(勇気)を奮い起こす、(速力など)を次第に増す、集中する 【他動-4】~から…であると推測する、結論を下す
\\	節を伴う 【他動-5】(スカートなど)にギャザーを寄せる、(眉を)ひそめる、~に皺を寄せる、~を縮める、収縮させる
\\	2年	
\\	【名-1】
\\	ドイツ人◆集合的 【名-2】ドイツ語 【形-1】ドイツ人の 【形-2】ドイツ語の、ドイツ式の 【人名】ジャーマン◆男
\\	2年	
\\	【名-1】贈り物、特別の能力、与えること、プレゼント 【名-2】神からの贈り物、天賦の才、天資、天稟 【他動】~に…を贈呈する
\\	2年	
\\	【名】グラフ、図式 【他動】~をグラフにする、図示する、~をグラフで描く[表す]
\\	2年	
\\	【名-1】ホール、玄関、廊下 【名-2】会館、本部、公会堂、集会所 【名-3】大広間
\\	2年	
\\	【名-1】心臓、胸 【名-2】心、胸の内、気持ち 【名-3】思いやりの心、愛情 【名-4】興味、関心、勇気 【名-5】中心、核心
\\	2年	
\\	【名】家事、家事労働
\\	2年	
\\	【接】どんなふうにでも 【副-1】どんなに~でも、いかに~であろうとも、どれほど~でも 【副-2】けれども、しかしながら、また一方
\\	2年	
\\	【名-1】内部、内側 【名-2】内幕 【名-3】《複数形》内臓、胃腸、おなか、はらわた 【名-4】内角 【前-1】~の内部に、~の内側[内部]で◆【反】
\\	【前-2】~以内に、~足らずで 【形-1】内側の、内部の、内面の 【形-2】秘密の 【副】内側に、中に[で]、内側で
\\	2年	
\\	"【名-1】ジャンプ、跳躍、飛躍、飛越し 【名-2】急上昇、急騰、大幅な伸び、急転 【自動-1】飛び越える、跳ぶ、躍動する 
\\	""と飛び跳ねながらわめいたり、正当な説明もせずに同じことを要求し続けるばかげた状態をこのように表現する。 【自動-2】急増する 【自動-3】急に襲いかかる、飛びかかる 【自動-4】叱る、非難する
\\	2年	
\\	【名】食事
\\	2年	
\\	【名-1】メートル、《単位・長さ》メーター
\\	単位系の長さの基本単位。 【名-2】メーター、計量器、計器、計測器 【名-3】拍子 【他動】(メーターで)~を計る、測定する
\\	2年	
\\	【名】
\\	◆無線用アルファベット 【人名】マイク
\\	の略称
\\	2年	
\\	【名-1】心、心情、精神、気性 【名-2】知性、知力、頭脳、頭 【名-3】意見 【名-4】記憶 【自動-1】嫌だと思う、嫌がる、気にする、気に障る 【自動-2】言うことを聞く、従順である 【他動-1】~を嫌だと思う、嫌がる、気にする、~が気に障る 【他動-2】~に気を付ける、~に注意を払う 【他動-3】~の面倒を見る、~の世話をする
\\	2年	
\\	【名】~さん、~夫人◆既婚女性に対する敬称
\\	2年	
\\	【名】必需品 【形-1】必要な、必須の、欠くことのできない、なくてはならない 【形-2】必然的な
\\	2年	
\\	【国名】ニュージーランド◆【略】
\\	2年	
\\	【名-1】役人、役員、幹部職 【名-2】警官、《呼び掛け》お巡りさん 【名-3】将校 【名-4】船長、高級船員
\\	2年	
\\	【名-1】つるはし◆地面を掘る道具。 【名-2】爪楊枝 【名-3】(株式市場の)ピックアップ◆【参考】
\\	【名-4】候補 【自動-1】つつく、ほじる、つく 【自動-2】ついばむ、(食欲がなさそうに)少しずつ口にする 【自動-3】選ぶ、選定する、選別する、精選する 【自動-4】摘み取る[取れる] 【自動-5】盗む、盗み取る、かすめる 【他動-1】~を摘み取る、採集する、採取する、
\\	に 
\\	を摘んでやる〔用法:
\\	または 
\\	【他動-2】念入りに選ぶ、精選する 【他動-3】(細長いもので)~を突く[つつく・穴を掘る]、(鼻や耳の穴)をほじる 【他動-4】(肉など)をしゃぶる、(毛)をむしり取る、(とげ)を抜く 【他動-5】(鳥が)~をついばむ 【他動-6】(食欲なさそうに)食べる 【他動-7】(欠点やあら)を捜す 【他動-8】~を盗む、盗み取る、かすめる、
\\	から 
\\	を抜き取る〔用法:
\\	【他動-9】(錠)をこじ開ける◆【参考】
\\	【他動-10】(弦楽器)を弾く 【他動-11】(けんか)を仕掛ける 【他動-12】~のきっかけを得る
\\	2年	
\\	【名-1】屋根、屋上 【名-2】てっぺん、最高部、天井 【名-3】天盤、上盤 【名-4】家、家庭
\\	2年	
\\	【名】観光、見物、遊覧 【形】観光の
\\	2年	
\\	【名】柔らかいもの 【形-1】柔らかい、しなやかな、しとやかな、やんわりとした、フニャフニャの 【形-2】穏やかな、優しい 【形-3】滑らかな 【形-4】楽な 【形-5】弱い 【形-6】(市況が)弱気の、軟調の、(価格・景気が)下がり気味の
\\	2年	
\\	【名】南、南部、南部地方◆【略】
\\	【形】南[南部・南部地方]の[にある] 【副】南に[へ・から]
\\	2年	
\\	【形】驚いた
\\	2年	
\\	【名-1】味、味質、風味 【名-2】味覚 【名-3】一口、一飲み、一なめ、(飲食物の)味見、試食 【名-4】ちょっとした経験、気味、一端 【名-5】好み、嗜好、趣味、審美眼、センス 【名-6】趣、風情 【自動】~の味がする、風味がある 【他動-1】~を味わう、味を見る、味見をする、食べる、飲む 【他動-2】~を経験する
\\	2年	
\\	【名-1】試験、テスト、検定、検査 【名-2】ためすもの、試練、試金石 【他動】~をテストする、検査する、試す、検証する、考査する、吟味する
\\	2年	
\\	【形-1】のどの渇いた、喉が乾いた、渇望[切望・熱望]している、強く欲しがる 【形-2】酒好きの 【形-3】土地が乾燥した
\\	2年	
\\	【名】宝、富、宝物、財宝、貴重品、重要なもの、財産◆【語源】ラテン語の 
\\	(倉庫) 【他動】~を大切[大事]に保管[保存]する、大切[大事]にする、取っておく、ためておく、心に銘記する、心にしまっておく
\\	2年	
\\	【名】妻、女房、家内
\\	2年	
\\	【名-1】動物園 【名-2】ごったがえした場所
\\	2年	
\\	【副-1】外国へ[で]、海外に[で] 【副-2】広く
\\	2年	
\\	【前-1】~を横切って、~を横断して、~の向こう側に[へ・で]、~を越えて、~を渡って 【前-2】~と交わるように、交差して 【前-3】~にわたって、~の全域で、~のいたる所に 【形】交差した、十文字の 【副-1】横切って、横断して、向こう側へ 【副-2】直径で
\\	2年	
\\	【副】実は、現実に、実際は[に]、実のところ、本当は[に]、実質的には、現に◆【類】
\\	2年	
\\	【名】アメリカ人、アメリカ英語 【形】アメリカの 【商標】アメリカン 【著作】
\\	アメリカ人◆米1877《著》ヘンリー・ジェイムズ
\\	2年	
\\	【自動】信じる 【他動-1】信じる、真に受ける、確信する、信頼する 【他動-2】思う、考える
\\	2年	
\\	【他動-1】~を…から(無料で)借りる、借用する 【他動-2】~を…から取り入れる、~を失敬する
\\	2年	
\\	【代名】両方、双方 【接】両方[双方]とも
\\	【形】両方の、双方の、ふたつともの 【副】両方[双方]ともに
\\	2年	
\\	【名-1】ビルディング、ビル、建物、建造物◆【略】
\\	【名-2】構築
\\	2年	
\\	【名-1】城、城郭 【名-2】《クリケット》三柱門◆【同】
\\	2年	
\\	【名-1】捕まえること、わな、捕まえたもの、捕獲物、収穫 【名-2】キャッチボール 【名-3】(結婚の)よい相手、つかまえたいい男[女]、興味を引くこと[もの] 【名-4】捕まえるための道具、落とし穴、わな 【名-5】〈俗〉問題点、難点、欠陥、まずいこと 【名-6】引っかけるための道具、留め金 【名-7】(息・声の一時的な)引っかかり、割れ、途切れ 【名-8】《ボート》キャッチ、オールのブレードで水を捕まえること、オールのブレードを水面に落として漕ぎ始めること 【名-9】キャッチ◆古楽時代からイギリス紳士たちが酒場で羽目を外すときなどに歌ってきた輪唱形式のざれ歌 【自動-1】捕まえようとする 【自動-2】(エンジンが)かかる 【自動-3】(声が)詰まる 【他動-1】(動いているもの)を捕まえる、~を捕らえる、~に追いつく 【他動-2】(人)を捕まえる 【他動-3】(乗り物)を捕まえる 【他動-4】~を見つける 【他動-5】(情報)をキャッチする、~を得る 【他動-6】~を聞き取る 【他動-7】~を理解する、~が分かる 【他動-8】(番組・映画)を見る
\\	2年	
\\	【他動】~を選ぶ、~を選択する、~を人選する、~を選り抜く
\\	2年	
\\	【名】掛け時計、置き時計◆【略】
\\	【他動-1】~のタイムを記録[測定]する 【他動-2】(記録)を達成する
\\	2年	
\\	【名-1】同好会、クラブ 【名-2】こん棒、警棒 【名-3】トランプのクラブ 【他動-1】《野球》~を打つ 【他動-2】~を棍棒で殴る 【他動-3】~を出し合う
\\	2年	
\\	【名-1】音楽会、コンサート 【名-2】協力、調和、提携 【自動】協定[協力]する 【他動-1】~と協定[協力]する、~を(…と協議して)調整する 【他動-2】~を考案する、ひねり出す、計画する
\\	2年	
\\	【形-1】かわいらしい、かわいい(顔をしている) 【形-2】気の利いた 【形-3】〈俗〉気取った、生意気な、わざとらしい
\\	2年	
\\	【名】闇、暗黒 【形-1】暗い、闇の、暗黒の 【形-2】浅黒い、皮膚が黒い 【形-3】濃い 【形-4】あいまいな 【形-5】陰気な 【形-6】卑劣な 【形-7】無知の、遅れている 【形-8】閉じている、営業していない
\\	2年	
\\	【形-1】おいしい、美味な 【形-2】実に面白い
\\	2年	
\\	【自動-1】見えなくなる、姿を消す 【自動-2】存在しなくなる、うせる、消滅する、なくなる、消失する、霧散する
\\	2年	
\\	【形】そのほかの 【副-1】別の方法で、ほかに 【副-2】さもなければ、さもないと
\\	2年	
\\	【名】電子メール、
\\	メール、電子郵便◆不可算名詞扱いの場合、
\\	を使用する。◆【語源】
\\	の略。◆【略】
\\	【他動】(人)に電子メール
\\	メール]を送る[送信する・出す]、メールで(人)に~を送る◆【略】
\\	2年	
\\	【代名】十分な量、十分な数、たくさん 【形】十分な、足りる 【副】十分に
\\	2年	
\\	【副】特に、著しく、殊のほか、とりわけ、殊に◆【略】
\\	2年	
\\	【名-1】火災、火、火事、焚き火 【名-2】熱情、激痛、熱気、熱意、やる気 【名-3】きらめき(があるもの)、輝き(を持つもの) 【名-4】炎症 【名-5】砲火、銃火、発砲、発射、射撃 【名-6】(質問などを)集中的に浴びせること 【自動】火がつく、発火する 【他動-1】~に火を付ける、~を発射する、発砲する、射撃する 【他動-2】~を首にする、解雇する 【他動-3】~を奮い立たせる
\\	2年	
\\	【名-1】花火(通常は複数形で用いる)、のろし、爆発物、花火大会、煙火 【名-2】ひらめき、激発、活発な動き、興奮、才気溢れる表現、怒り、妙技
\\	2年	
\\	【名-1】飛ぶこと、飛行 【名-2】飛ぶもの、飛ぶ昆虫、《虫》ハエ、害虫 【名-3】(魚釣りの疑似餌として使う)フライ、毛針◆フライの種類として 
\\	がある。 【名-4】《野球》フライ 【名-5】(衣服の)チャック、ファスナー、ジッパー 【名-6】前立て 【自動-1】飛ぶ 【自動-2】逃げる 【自動-3】飛行する、飛行機で行く 【自動-4】《野球》フライを打つ◆この意味の 
\\	の過去形と過去分詞は 
\\	になる。つまり、
\\	は「飛んだ」だが、
\\	は「フライを打った」、「ボールを飛ばせた」という意味。
\\	【自動-5】はためく 【自動-6】急ぐ 【他動-1】~を飛行機で運ぶ[送る] 【他動-2】(気球)を飛ばす、(凧)を揚げる、(旗)を掲揚する
\\	2年	
\\	【名-1】森林(地帯)、森、樹木林 【名-2】林立◆【用法】
\\	複数名詞 【名-3】《歴史》御狩場 【他動】~に植林する、~に植林[造林]を行う
\\	2年	
\\	【名-1】友好的な人 【名-2】親善試合 【形-1】親切な、友情のこもった、友好的な、好意を持つ、人なつっこい、愛想のいい、当たりがいい、親しげな、親しみやすい、気さくな 【形-2】(機械・ソフトウェアなどが)使いやすい、扱いやすい、(一般にも)分かりやすい
\\	2年	
\\	【名-1】等級、品質等級 【名-2】学年 【名-3】成績の評点、評価 【名-4】勾配、傾斜 【名-5】度、結合度、品位 【自動】等級別になっている 【他動】~を等級分けする、~に等級[ランク]を付ける
\\	2年	
\\	【名-1】地面、地べた、地盤、土地、用地、敷地、地山 【名-2】立場、立脚点、見地、見方 【名-3】領域 【名-4】根拠、原因、理由 【名-5】陸上競技場、運動場、グラウンド 【名-6】アース端子、接地◆【略】
\\	【名-7】地の色 【形】(石・豆などを臼などで)細かくした、(挽いて)摺った、細かくした、(挽いて)粉末の 【自動-1】座礁する 【自動-2】《野球》ゴロを打つ 【自動-3】
\\	の過去・過去分詞形 【他動-1】~を地面に置く 【他動-2】~の基礎を教えこむ 【他動-3】~を接地する、アースする 【他動-4】~を外出禁止にする、自宅謹慎させる、出場禁止にする 【他動-5】
\\	の過去・過去分詞形
\\	2年	
\\	【自動-1】起こる、発生する、降り懸かる 【自動-2】たまたま~する、偶然~する
\\	2年	
\\	【名-1】歴史、歴史学 【名-2】過去のもの、過去のこと 【名-3】経歴、病歴、履歴、前歴、過程 【名-4】過去の人、今では無用の人
\\	2年	
\\	【名-1】ホテル、旅館 【名-2】パブ◆【同】
\\	2年	
\\	【名-1】情報、資料、データ 【名-2】知識 【名-3】通知 【名-4】案内 【名-5】《法律》略式起訴(状)、告訴(状)、告発(状)
\\	2年	
\\	【名-1】朝鮮 【名-2】= 【名-3】= 【分節】
\\	2年	
\\	【名・形】朝鮮人[語](の)、韓国人[語](の)
\\	2年	
\\	【名-1】低いもの 【名-2】ローギア◆自動車の 【名-3】底値、安値 【形-1】低い 【形-2】弱い 【形-3】(声などが)小さい 【形-4】安い 【形-5】低級な、下劣な、卑劣な 【形-6】少ない、乏しい、(燃料などが)不足して 【形-7】粗食で、粗末な 【副】低く
\\	2年	
\\	【名-1】中位、中間 【名-2】媒体、媒介物、媒質、手段、溶剤 【形-1】中間の、中位の 【形-2】《料理》~中_個
\\	2年	
\\	山の名前の前に付く。 例: 富士山 
\\	2年	
\\	【名】ニュース、知らせ、消息、お知らせ、報道、情報、便り
\\	2年	
\\	【名】北、北方、北部、北極地方、北半球、北部地方◆【略】
\\	【形】北にある、北の、北への 【副】北へ、北に[へ]
\\	2年	
\\	【名-1】無、ゼロ 【名-2】無価値なもの、価値のないもの 【名-3】無価値な人、つまらない人、クズ同然の人 【代名】何も~ない、少しも~ない、何物も~ない
\\	2年	
\\	【名】親、両親〔複数形で〕、保護者、守護者 【形】親の、もとの 【自動】親としての役目を果たす 【他動】~の親代わりとなる、親になる、親代わりをする
\\	2年	
\\	【名-1】ペット、愛玩動物 【名-2】お気に入り 【名-3】いい子、おりこうさん◆呼び掛けて 【形-1】お気に入りの、かわいがっている 【形-2】ペットの 【形-3】(言動が)得意の、おはこの 【自動】ペッティングする、愛撫する 【他動-1】~を可愛がる 【他動-2】~をペッティングする、愛撫する
\\	2年	
\\	【名-1】計画、企画、予定、今後の進め方 【名-2】設計図、図面、平面図、伏図 【名-3】意向、つもり、考え、意図◆通常複数形で使う 【他動】~の計画を立てる、~を計画する、~するつもりだ
\\	2年	
\\	【名-1】ひと組 【名-2】セット 【名-3】テレビ 【名-4】集合 【名-5】凝結 【形-1】定められた、規定の、指定の 【形-2】型どおりの、型にはまった 【形-3】(表情が)動かない[堅い・こわばった] 【形-4】《口》用意ができた 【自動】(ゼリーなどが)固まる、(骨が)くっつく[固定する] 【他動-1】~を正しい位置に置く[取り付ける]、整える、定める、配置する、設定する 【他動-2】~を決める 【他動-3】(時間など)を合わせる 【他動-4】~を確定する、(期日)を設ける 【他動-5】卵を抱かせる
\\	2年	
\\	【名-1】空、天、空模様 【名-2】天国 【他動】《ゴルフ》~を高くあげる
\\	2年	
\\	【名】微笑、微笑み、笑顔 【自動】微笑む、微笑する 【他動】笑う◆【用法】修飾語、同族目的語を伴って
\\	2年	
\\	【名-1】空間(的余裕)、スペース、場所 【名-2】宇宙(空間) 【名-3】紙面、空白、記入欄 【名-4】余地 【名-5】自由 【形-1】宇宙の、宇宙空間の 【形-2】空間の 【他動】~に間隔をあける
\\	2年	
\\	【名】競技場、スタジアム、(野)球場
\\	2年	
\\	【名-1】太陽 【名-2】日光 【自動】日光浴する 【他動】日を浴びる、肌を焼く、日干しにする
\\	2年	
\\	【名】小旅行、視察、ツアー、観光旅行、見て歩くこと、見て回ること 【自動】周遊旅行する、視察する、旅行する、見学する、周遊する、巡業する 【他動】~を周遊旅行する、視察する、旅行する、見学する、周遊する、~の巡回公演を行う
\\	2年	
\\	【名-1】町、市、街、都会 【名-2】町民、都民
\\	2年	
\\	【名-1】運行、運動、行程、進行、道程、走行 【名-2】紀行文 【名-3】旅行 【自動-1】(前方に)進む、歩く、走る、動く、移動する、旅する、旅行する 【自動-2】伝わる 【自動-3】売り込みに回る
\\	2年	
\\	【名-1】浪費、無駄、消耗、衰弱 【名-2】廃棄物、廃石、屑 【名-3】荒れ地 【形-1】残った 【形-2】荒れ果てた、荒れた 【自動】衰弱する、やせ衰える、弱る 【他動-1】~を無駄にする、空費する、浪費する、無駄使いする、消耗する、擦り減らす、(スペース)を無駄に消費する 【他動-2】~を(みすみす)逃す 【他動-3】~をやっつける、殺す、完全にぶちのめす 【他動-4】~を壊す、破壊する◆【同】
\\	2年	
\\	【名-1】不安、心配、気がかり、気苦労、懸念、憂慮 【名-2】悩み[心配]の種 【名-3】動物[猟犬]が獲物にくわえて振りまわすこと 【自動-1】心配する、気をもむ、気になる、気がかりである、気に病む、くよくよする 【自動-2】苦労して進む 【他動-1】~を心配させる、悩ます、イライラさせる、~の気をもませる、~に心配[苦労]をかける 【他動-2】(唇)をかむ、~をもてあそぶ、しきりに触る[いじる]
\\	2年	
\\	【自動】(提案などに)同意する、合意する、一致する、我が意を得る、賛成する、賛同する、~を承諾する、承知する、~が性に合う、~に同感である、認める 【他動】(~ということ)を認める
\\	2年	
\\	【副】いつも、以前からずっと、常にいつでも、昔から、前々から
\\	2年	
\\	【代名-1】もう一つ、もう一人 【代名-2】お代わり 【形-1】もう一つ[1個・1杯・一人]の、別の、ほかの 【形-2】違った、新手の、別の 【形-3】~の生まれ変わり、~もまたいつもと同じ~、~もやっぱり~だ、例によって例の如き
\\	2年	
\\	【名-1】答え、回答、返事、応答、《法律》答弁書 【名-2】正解、解答 【名-3】解決策、打開案、対策、対処法 【名-4】応酬、仕返し、反論、返報 【自動-1】答える、返事する 【自動-2】~に対して責任を負う、品質などに責任を持つ 【自動-3】目的[要件]に合う、役立つ 【自動-4】人相書きなどに合う 【他動-1】答えて言う、~に答える、《コ》(モデムなど)が応答する 【他動-2】(ノック・ベル・電話など)に答えて出る 【他動-3】~で報いる、攻撃に応酬する 【他動-4】(願いや要求など)を聞き入れる、~に従う 【他動-5】~に応じる、合致する、(目的や要件など)にかなう 【他動-6】議論や批判などに反論する 【他動-7】応募する
\\	2年	
\\	【代名-1】誰か◆疑問文で 【代名-2】誰も、誰にも◆否定文で 【代名-3】誰でも◆肯定文で
\\	2年	
\\	【代名-1】《肯定文で》何でも、どんなものでも 【代名-2】《否定文で》何も(~ない) 【代名-3】《疑問文で》何か
\\	2年	
\\	【自動-1】尋ねる、質問する、見舞う 【自動-2】求める、要求をする、必要とする 【他動-1】~を尋ねる、質問する、聞く、問う 【他動-2】~を頼む、依頼する、委嘱する、~してほしいと頼む、いう 【他動-3】~を求める
\\	2年	
\\	【名】遠征(試合) 【形-1】離れた所に、遠くに、離れて、遠くにある 【形-2】不在で、いない、留守で、欠席して 【形-3】《スポーツ》遠征の、遠征地での、《野球》相手球場での、ロードの 【形-4】《野球》アウトになった 【副-1】離れて、離れた所で[へ]、あちらへ、向こうへ 【副-2】切り離して、取り去って、取り除いて、追い払って、手放して 【副-3】(次第に)消えて 【副-4】減少して 【副-5】絶え間なく 【副-6】すぐに 【副-7】脱走して、逃げて、逃亡して 【副-8】《野球》アウトになって◆他のスポーツでも用いる場合がある。
\\	2年	
\\	【名-1】赤ちゃん、赤ん坊、乳児、幼児、新生児、乳幼児 【名-2】恋人 【名-3】末っ子、最年少者 【名-4】大事なもの、宝物 【名-5】大変な[厄介な]仕事[任務]、責任、義務、役目 【名-6】弱虫、泣き虫 【名-7】関心のあること、関心事 【形】赤ちゃん[子ども]用の、赤ちゃんのような、小さい、小型の 【他動】(人)を赤ん坊のように扱う、子ども扱いする、大事にする、甘やかす、注意して扱う
\\	2年	
\\	【動】
\\	の過去形 
\\	ビケイム、【分節】
\\	2年	
\\	【接-1】なぜならば、~だから、~なので、~という理由で、なぜなら、なぜかと言うと、(~した)ばかりに◆【略】
\\	【接-2】~したからといって
\\	2年	
\\	【自動】~になる 【他動】~にふさわしい、適する、~と合う、~に似合う
\\	2年	
\\	【動】
\\	の過去形 
\\	ビギャン、【分節】
\\	2年	
\\	【自動-1】始まる、着手する 【自動-2】(戦争・革命などが)勃発する 【他動】~を始める、(話の)口火を切る、~するようになる、~に取りかかる、着手する、参入する、(雑誌など)を創刊する
\\	2年	
\\	【形-1】生まれた 【形-2】生まれながらの、生まれつきの
\\	2年	
\\	【動】
\\	の過去・過去分詞形 
\\	ボート 
\\	金の力になびく
\\	2年	
\\	【他動-1】~を持って来る、(話し相手のところへ)~を持って行く 【他動-2】(話し相手のところに)~を連れて行く[来る] 【他動-3】(~に)~をもたらす 【他動-4】~で売れる 【他動-5】(訴訟など)を起こす 【他動-6】(理由・証拠など)をあげる、提示する 【他動-7】~の状態に至らせる 【他動-8】~するようにさせる 【他動-9】《基本イメージ》主語が何かを持って別の場所へ移動する
\\	2年	
\\	【動】
\\	の過去・過去分詞形 
\\	ブロート
\\	2年	
\\	【名】ビルド、球、体格、体型、組み付け 【自動】徐々に[次第に]高まる[盛り上がる・度を増す] 【他動-1】~を建てる、~を構築する、~を形成する、~を築く、~を建造する、~を架設する 【他動-2】~を確立する、(関係)を作る 【他動-3】~を発足させる 【他動-4】~を集める
\\	2年	
\\	【形】しっかりした造り[構造]の、良い[がっちりした]体格をしている、魅力的な体つきの、体格が~の 【動】
\\	の過去形 
\\	ビルト
\\	2年	
\\	【名-1】買うこと、購入、買い物 【名-2】お買い得品、格安品、特売品、割安の買物、掘り出し物◆【同】
\\	【他動-1】~を買う、~を購入する 【他動-2】~を手なずける、~を抱き込む、~を買収する 【他動-3】(何かを犠牲にして)~を獲得する、~を手に入れる 【他動-4】~におごる 【他動-5】(意見)を受け入れる、~に賛成する、~を(本当と)信じる
\\	2年	
\\	【名-1】持ち運び 【名-2】《ゴルフ》飛距離、キャリー 【名-3】桁上げ 【自動-1】運ぶ 【自動-2】伝わる、届く、(音や声などが)達する 【自動-3】飛ぶ 【他動-1】~を運ぶ、~を持ち運ぶ 【他動-2】~を携行する、~を持っている 【他動-3】(品物)を店に置く、(商品)を扱っている、~を在庫として持っている 【他動-4】~を伝える、~を放送する 【他動-5】~を支える、~を支援する 【他動-6】(選挙区)を制する 【他動-7】~を持ち越す 【他動-8】《ゴルフ》キャリーで~に届く、キャリーで~を越える
\\	2年	
\\	【名-1】サクランボ(色)、サクラの木 【名-2】〈米俗〉新品、初心者 【名-3】〈米俗〉良いもの[こと]、かっこいいもの、いかすもの 【形-1】サクランボ色の 【形-2】〈米俗〉処女の
\\	2年	
\\	【形-1】きれいな、不要なものが取り除かれている、汚れのない、清潔な、清らかな、清廉な、汚染されていない、無菌の 【形-2】偽りのない 【形-3】潔白な、違法な[犯罪にかかわる]ものを所持していない、無罪の、シロの 【形-4】《人の外見が》清潔感[清涼感]のある 【他動-1】~を取り除く、~を削除する、~を駆除する◆不要なものを 【他動-2】~を掃除する、~を清潔にする、~をきれいにする、~を洗濯する、~を消毒する
\\	2年	
\\	【名】コンピュータ、計算機
\\	2年	
\\	【形-1】涼しい、冷たい、冷える 【形-2】さめた、無愛想な、無関心な、冷淡な 【形-3】素晴らしい、すごい、渋い、かっこいい、いけてる 【形-4】冷静な、落ち着いた 【形-5】掛け値なしの、正味の、かっきりの 【他動-1】~を冷やす、~を冷ます、~を冷却する 【他動-2】~を冷え込ませる◆需要を
\\	2年	
\\	"【助動-1】《可能》~することができた
\\	の過去形。過去の一貫した能力を表す。瞬時の能力は 
\\	【助動-2】《推量》(条件が合えば)~ということもありうる、もしかして~できるのではないだろうか、~もないことはない 【助動-3】《否定推量》~のはずがない
\\	(~に違いない)と対比される 【助動-4】《過去の推量》~することもあり得た◆過去における人の態度を表す 【助動-5】《過去の許可》~することが許された 
\\	黒人たちは白人のトイレに入ることすら許されなかった。 【助動-6】《許可》~してよい◆丁寧な響きを伴う 【助動-7】《進行》~していた◆通例、
\\	などの知覚動詞を伴う 【助動-8】《依頼》~していただけませんでしょうか
\\	の形をとり、丁寧な響きを伴う 【助動-9】《軽い提案》~しませんか 【助動-10】《仮定法・過去》~なのに 【助動-11】《仮定法・過去完了》~だったのに 【助動-12】《願望》~したい気がする、~してやりたいくらいだ
\\	の形で、時にいら立ちを伴う 【助動-13】《いら立ち・焦燥》~してもよさそうなものだ
\\	2年	
\\	【名-1】叫び声 【名-2】泣くこと 【名-3】スローガン 【自動】泣く、叫ぶ
\\	2年	
\\	【自動】決定する、決心する、決意する、判決を下す 【他動-1】~を解決する 【他動-2】~を決意する、決心させる、決定する、~に判決を下す、判定を下す
\\	2年	
\\	【形-1】深い、深さがある、奥行きがある◆【略】
\\	【形-2】深い洞察力がある 【形-3】意味の深い、深甚な、深遠な 【形-4】難解な、困難な、難しい、手が付けられない、理解に苦しむ、不可解な 【形-5】強度の、極端な、凄まじい、激しい、切実な、重大な、深刻な 【形-6】(色彩が)濃い 【形-7】(声や音が)太い[低い] 【形-8】大幅な、規模の大きい 【形-9】遠い昔の、遠く離れた 【形-10】ずる賢い、腹黒い 【副】深く、ふけって
\\	2年	
\\	【名-1】金型 【名-2】さい、さいころ、賽の目に切ったもの
\\	の単数形 【自動-1】死ぬ、死亡する、(花が)枯れる◆婉曲表現は 
\\	【自動-2】
\\	の形で》(~で)死にそうだ、(~したくて)たまらない◆【参考】
\\	【自動-3】(バッテリーが)切れる 【自動-4】(風が)静まる 【自動-5】(エンジンが)止まる
\\	2年	
\\	【名-1】違い、差異、相違(点) 【名-2】差額 【名-3】意見の相違 【名-4】不和、争い、紛争 【名-5】《数学》差、階差、差分
\\	2年	
\\	【名-1】飲み物、飲料 【名-2】アルコール飲料、酒類 【名-3】一口、ひと飲み、大酒 【名-4】海 【自動-1】飲み物を飲む 【自動-2】酒を飲む 【自動-3】飲むと~の味がする 【他動-1】~を飲む、飲み干す 【他動-2】~のために乾杯する 【他動-3】飲んで~の状態になる 【他動-4】植物や土地が水分を吸収する
\\	2年	
\\	【前】~の間ずっと、~の期間に、~の間に、~中に
\\	2年	
\\	【代名】すべてのもの
\\	2年	
\\	【名-1】(多くの実例の中から選んだ)例、実例、実施例、用例 【名-2】見本、標本、例題 【名-3】模範、手本 【名-4】見せしめ、戒め
\\	2年	
\\	【形】刺激的な、興奮させる、スリルのある、胸を躍らせる、ワクワクさせる、興奮性の、面白い、心躍る
\\	2年	
\\	【名-1】経験、体験、見聞◆【略】
\\	【名-2】経験から得た知識[技術] 【他動】~を経験する、体験する
\\	2年	
\\	【名-1】目、目玉 【名-2】目つき、まなざし 【名-3】視力、視覚 【名-4】見解、観察の目 【名-5】目に似たもの 【名-6】中心、中央部 【名-7】肉の一番いい部分◆【同】
\\	【名-8】熟成中にチーズにできる穴[気孔] 【他動-1】~をじっと[じろじろ・注意深く]見る[見つめる]、~に注目する、注視する、視野に入れる、観察する 【他動-2】~に色目を使う
\\	2年	
\\	【名-1】顔、顔色、しかめ面 【名-2】面子 【名-3】面、額面、表面、外観、外見、側面 【名-4】化粧 【名-5】時計の文字盤 【他動-1】~の方を向く、~に向かう、~と向かい合う、~を目の前にする、正視する、~に面する 【他動-2】~と顔を合わせる 【他動-3】~に立ち向かう、~に直面する、危険が~の身に迫る 【他動-4】(問題など)に取り組む 【他動-5】~に上塗りする、~の上張りする、(トランプ)の表を出す、~を回れ右させる
\\	2年	
\\	【名】遠方 【形-1】遠い、遠くへの 【形-2】(二者のうちで)遠いほうの 【形-3】(政治的に)極端な 【副-1】遠く離れて、《距離・空間》遠くに、遠くへ 【副-2】《時間》遠く 【副-3】《程度》はるかに、大いに、ずっと
\\	2年	
\\	【名】断食 【形-1】速い、高速の 【形-2】早口の 【形-3】(時計が)進んでいる 【形-4】しっかりとした、ゆるがない、ぐらつかない、固定した、固く締まった 【形-5】不変の 【形-6】変色しない、あせない 【形-7】《ゴルフ》(グリーンが)速い 【副-1】速く 【副-2】固く、しっかりと 【副-3】進んで 【自動】絶食する、断食する、食を控える、精進する
\\	2年	
\\	【名】フェルト◆布 【形】切実な、痛切な、切実に感じられる 【自動】
\\	の過去形 
\\	フェルト
\\	2年	
\\	【名】少数の人、少数の物、選ばれた人々 【形】少数の、あまり~ない、いくつかの、少ない
\\	2年	
\\	【名】発見物、発見、掘り出し物 【自動】《法律》評決を下す 【他動-1】~を見つける、発見する、見いだす、検出する、捜し出す 【他動-2】~に気付く、~と思う、~と考える、~と感じる 【他動-3】~であるところを見つける 【他動-4】~を知る 【他動-5】《数学》(計算して)~を求める
\\	2年	
\\	【名】罰金 【形-1】晴天の、晴れた 【形-2】元気な 【形-3】素晴らしい、すてきな、洗練された、立派な 【形-4】良い、結構な、構わない 【形-5】品質の優れた、上質の、最高級の 【形-6】純度の高い 【形-7】繊細な、細い、細かい 【形-8】巧妙な 【他動】~に罰金を科する、~を科料にする
\\	2年	
\\	【名-1】終わり、仕上げ、仕上がり◆【略】
\\	【名-2】後味、後口 【自動-1】終わる、完成する 【自動-2】終える
\\	の形で 【他動】~を終える、~の締めくくりをつける、片付ける
\\	2年	
\\	【名】追うこと 【自動-1】~について行く 【自動-2】話について行く、理解する、分かる 【自動-3】次に起こる、~ということになる 【他動-1】~に続く、~の次に来る 【他動-2】~の後について行く、後ろからついて行く、~を追跡する、尾行する 【他動-3】(道など)をたどる、~の方へ進む 【他動-4】~に従う、追随する 【他動-5】~に付随する 【他動-6】(流行・最新の動向・関心事など)を追う、~について行く、~に興味を持つ、~のことをよく知っている、~通である 【他動-7】~を目で追う、(事態の進展)を見守る 【他動-8】~の話について行く、(人)の言うことを聞く、~に聞き入る、(相手の話)を理解する
\\	2年	
\\	【自動】忘れる 【他動-1】~を忘れる、見落とす 【他動-2】放っておく、構わない
\\	2年	
\\	【他動-1】~の基礎を築く、~を設立する、創立する、建てる◆【類】
\\	【他動-2】
\\	の過去・過去分詞形
\\	2年	
\\	【名-1】未来、将来(性)、今後、先行き、将来像 【名-2】未来時制 【名-3】先物、先物商品、先物取引◆複数形で 【形】未来の、将来の、今後の、将来に向けた
\\	2年	
\\	【動】
\\	の過去形 
\\	ゲイブ
\\	2年	
\\	"【名】曲がり、たわみ 【自動】~が起こる 【他動-1】~を与える、渡す、施す 
\\	《楽器の音を合わせるのに》
\\	の音を出して。 【他動-2】~を催す 【他動-3】~を発する 【他動-4】~を加える 【他動-5】~に屈する、(仕方なく)~を認める 【他動-6】~を生じさせる
\\	2年	
\\	【自動-1】聞く、聞こえる 【自動-2】伝え聞く、消息を聞く、連絡をもらう 【他動-1】~を聞く、聴く、~が聞こえる、耳にする、うわさに聞いている 【他動-2】(意見)を聴取する、審問する
\\	2年	
\\	【動】
\\	の過去・過去分詞形 
\\	ハード
\\	2年	
\\	【名-1】打撃、一撃、衝突、命中 【名-2】大当たり、大成功、ヒット作 【名-3】《野球》ヒット、安打◆【略】
\\	【名-4】当たり券、当たりくじ 【自動-1】たたく、打つ 【自動-2】ぶつかる、衝突する、的中する 【自動-3】撃つ、襲う、攻撃する、(地震が)起こる 【他動-1】~に撃ち当てる、~に命中させる 【他動-2】~を打つ、ぶつ、たたくぶつかる、衝突[追突]する 【他動-3】(ボタン)を押す 【他動-4】~に達する、~に着く、~に到る 【他動-5】(天災)が襲う、打撃を与える 【他動-6】(政策など)を非難する 【他動-7】(正しい方向性)を見いだす
\\	2年	
\\	【形-1】空腹の、空腹な、腹が減っている、腹ぺこで 【形-2】~に飢えた、ハングリー精神のある、渇望して、切望して
\\	2年	
\\	【名-1】考え、着想、アイデア、案、提案、発想、思い付き、考案、構想、考え方 【名-2】見解、意見 【名-3】知識、心当たり 【名-4】意図、狙い、目的、心 【名-5】思想、理念、観念、概念 【名-6】主題
\\	2年	
\\	【名】仮定、条件 【接】もし~ならば、もし~なら、~という状況だと(仮定)すると、~の場合、~という条件において
\\	2年	
\\	【形-1】重要な、重大な、大切な 【形-2】有力な
\\	2年	
\\	【形-1】関心のある、関心があって、興味があって、興味を持って、興味深げな 【形-2】利害関係のある、関与して、利害関係を持つ 【形-3】私心のある、私利に動かされた
\\	2年	
\\	【名】インターネット◆【略】
\\	インタネット
\\	2年	
\\	【前-1】~の中へ、~の中に 【前-2】~に入れ込んで、のめり込んで、関心を持って、熱中して、夢中になって、没頭して、はまって、打ち込んで、凝って、興味を持って、~が好きで、~という趣味で、~をよく知っている、熱中している 【前-3】~にぶつかって 【前-4】~に変化して
\\	2年	
\\	"【名】加入、結び、結合、接合(個所)、継ぎ目 【自動】加わる、加入する、参加する、交わる、一緒になる 【他動-1】~に加わる、~に加入する、~に参加する、~と交わる、~と一緒になる 【他動-2】~を結び付ける、結合する、連結する、合わせる、つなぐ、接合する、接続する 
\\	このプロジェクトに力を結集しよう。
\\	2年	
\\	【名】生計費 【自動-1】動作を続ける、~し続ける 【自動-2】長持ちする、保存が利く、もつ◆食べ物や飲み物が 【他動-1】~の状態にしておく、(状態)を保つ、~にしておく 【他動-2】~を持ち続ける、保持する 【他動-3】~を借りておく 【他動-4】(動物)を飼う 【他動-5】~の管理をする、~を預かる、守り続ける 【他動-6】~を引き留める 【他動-7】(日記・記録など)をつける 【他動-8】~を守る
\\	2年	
\\	【名-1】種類 【名-2】質、性質、本質 【形】優しい、親切な、いたわる
\\	2年	
\\	【形-1】大きい、広い、《料理》~大_個 【形-2】多い、多数の 【形-3】誇張した
\\	2年	
\\	【形】もっと遅い、より最近の 【副】その後、後ほど
\\	の比較級)、後で
\\	2年	
\\	【自動】学ぶ、知る、分かる 【他動-1】~を学ぶ、暗記する、覚える、習う、身につける 【他動-2】~を知る、分かる、悟る、確認する、突き止める
\\	2年	
\\	【名-1】手紙、書簡、公式文書 【名-2】文字 【他動】~を書き入れる
\\	2年	
\\	【名-1】人生、この世、生涯 【名-2】人命、生命、寿命 【名-3】生活、暮らし 【名-4】生物 【名-5】生き甲斐、活気、元気 【名-6】実物、本物
\\	2年	
\\	【名】少しの距離、少量 【形-1】小さい 【形-2】幼い 【形-3】少量の、少しの 【形-4】ほんの少ししかない、ほとんどない 【形-5】ちょっとした 【副-1】少ししか~ない 【副-2】少し、ちょっと
\\	の形で
\\	2年	
\\	【名】
\\	の複数形 
\\	ライブズ
\\	2年	
\\	【自動-1】負ける◆【反】
\\	【自動-2】損害を受ける、損をする 【他動-1】~を失う、見失う、喪失する、なくす、遺失する 【他動-2】~に迷う 【他動-3】(試合など)に負ける 【他動-4】(赤ん坊)を死産する 【他動-5】~を死なせる、滅ぼす
\\	の形で 【他動-6】(時計)が遅れる 【他動-7】~を浪費する 【他動-8】(体重)を減らす、(血液)を失う 【他動-9】〈俗〉~を取り除く
\\	2年	
\\	【形】~製の、~で作られた 【動】
\\	の過去・過去分詞形 
\\	メイド
\\	2年	
\\	【名】《植物》メイフラワー、さんざし、西洋山査子◆【同】
\\	◆バラ科の落葉低木 【助動-1】《推量・可能性》~かもしれない、~する可能性がある◆【注意】この意味で 
\\	を疑問文に用いることはできない。◆【用法】
\\	は50%の確からしさを示す。アメリカ用法では 
\\	が好まれる傾向にある。 【助動-2】《許可》~することができる、~して(も)よい、~して差し支えない◆【注意】
\\	に対して、
\\	【助動-3】《願望》~でありますように、~を祈る◆改まった場面で用いる。 【助動-4】《目的》~するために 【助動-5】《譲歩》~かもしれませんが
\\	などの接続詞と共起する。
\\	2年	
\\	【名-1】(通例複数形で)手段、方法、手段、手法 【名-2】資力 【名-3】平均、平均値 【名-4】中項 【形-1】劣った、みすぼらしい 【形-2】卑怯な、卑劣な、意地悪な、不親切な 【形-3】けちな、物惜しみする 【形-4】ささいな 【形-5】中間の、普通の、平均の 【他動-1】~を意味する、…とは~を指す[言う] 【他動-2】重要性を持つ、大事である、大切である 【他動-3】~するつもりである 【他動-4】本気で~と言っている、本気である
\\	2年	
\\	【名-1】伝言、メッセージ、通報、連絡事項、通信、通達、伝達内容、声明文、神託◆【略】
\\	【名-2】(ラジオ・テレビの)コマーシャル、お知らせ
\\	2年	
\\	【動】
\\	の過去・過去分詞形 
\\	メット
\\	2年	
\\	【名】ママ、お母さん、おかあちゃん 
\\	マム、【変化】《複》
\\	2年	
\\	【名-1】金銭、通貨、金 【名-2】資金、財産 【名-3】金額 【形】〈俗〉かっこいい◆【同】
\\	2年	
\\	【名】追加の[より多くの・それ以上の・もっと重要な]もの[こと・情報] 【形-1】より大きい・多い、もっと、より多くの、ずっと◆【反】
\\	【形-2】それ以上の、追加の 【副】より大きな程度に
\\	の比較級
\\	2年	
\\	【形-1】最も多くの、最も大きい 【形-2】たいていの、ほとんどの、大部分の 【副】最も(多く)、一番(多く)、何よりも、この上なく
\\	の最上級
\\	2年	
\\	【名-1】山◆【略】
\\	【名-2】山ほどの量、多量◆【参考】
\\	【形】山の、山のような
\\	2年	
\\	【名-1】(ビン・人・袋などの)口、河口、出入り口 【名-2】言葉 【自動】口元で伝える 【他動-1】~を言う、発音する 【他動-2】~を気取って言う、演説調で言う
\\	2年	
\\	【名-1】動き、移動◆【略】
\\	【名-2】引っ越し 【名-3】運動 【名-4】手段、手だて 【名-5】手番 【自動-1】移動する、動く 【自動-2】引っ越す、移転する 【自動-3】異動する 【他動-1】~を移動させる、動かす、移設する 【他動-2】人の心を動かす、感動させる 【他動-3】~を提議する、動議として提案する
\\	2年	
\\	【名-1】映画 【名-2】映画館
\\	2年	
\\	【名】音楽家、作曲家、演奏家
\\	2年	
\\	【名】必要なもの、絶対必要なこと[もの]、なくてはならないもの、必需品、必須◆可算名詞 【助動-1】~しなければならない、~する必要がある◆義務によって 【助動-2】~せざるを得ない、~しないではいられない、~しないと気がすまない、どうしても~してしまう 【助動-3】~に違いない、~のはずだ 【助動-4】《丁寧な勧誘》是非~してください
\\	2年	
\\	【副-1】いつであろうと~ない、少しも~ない、決して~ない、~のはずはない 【副-2】以前に[今までに]~したことがない、いまだかつて~ない 【副-3】まさか~ではあるまい
\\	2年	
\\	【名】次の人[もの] 【形】次の、翌~、隣の 【副】次に
\\	2年	
\\	【名-1】事務所、会社、職場、執務室、事務室 【名-2】任務 【名-3】官職 【名-4】局◆米国政府の庁の下にある 【名-5】省◆英国政府の 【名-6】あっせん、仲介、世話◆この意味では複数形で用いられることが多い。
\\	2年	
\\	【名】1回 【接-1】~するとすぐに 【接-2】いったん~すると、ひとたび~すれば 【副-1】1度、一旦、一度 【副-2】あるとき、かつて、以前
\\	2年	
\\	【名-1】もう一方の人[もの]、相手側 【名-2】別の人[もの]、その他、ほかのもの、他方 【形-1】ほかの、そのほかの、残りの 【形-2】もう一方の、向こうの
\\	2年	
\\	【名-1】外側 【名-2】アウト 【形-1】外に出て、外出して、留守で、離れて、仕事を休んで 【形-2】(世に)出て、公表されて、公になって、出版されて、露見して、ばれて 【形-3】(花が)咲いて 【形-4】不足して、足りない 【形-5】(火が)消えて、(機器が)故障して 【形-6】意識不明で 【形-7】流行遅れになって、廃れて 【形-8】不可能である 【副-1】外へ、外に、外へ出して、外部に 【副-2】取り出して、選び出して 【副-3】機能しなくなって 【副-4】はずれて 【副-5】現れて 【副-6】大声で 【副-7】最後まで、徹底的に、すっかり、完全に 【自動-1】アウトになる 【自動-2】出てくる 【自動-3】ばれる、露見する 【他動-1】~をアウトにする 【他動-2】~を追い出す、突き出す、追放する 【他動-3】~を暴露する
\\	2年	
\\	【名-1】紙 【名-2】新聞 【名-3】文書、書類 【名-4】研究論文、学術論文、論説 【形-1】紙でできた、紙製の 【形-2】紙に書かれた、印刷された 【他動-1】~を紙で包む 【他動-2】~に壁紙を張る
\\	2年	
\\	【名-1】一部、部分 【名-2】部品、パーツ 【名-3】体の一部、器官 【名-4】全体を等分したうちの1個、~分の1◆【用法】助数詞+
\\	の形をとる。例えば、
\\	(5分の1)、
\\	(6分の1)というように。 【名-5】分け前、割合 【名-6】仕事として分け与えられたもの、分担、役、役目、役割 【名-7】側、味方 【名-8】《数学》因数、約数 【副】いくぶん 【自動】分かれる、別れる 【他動】~を分ける、引き離す、切断する
\\	2年	
\\	【名-1】小道 【名-2】パス 【名-3】通過 【名-4】合格 【名-5】手品 【名-6】通行証、許可証 【名-7】定期券、無料乗車券 【名-8】峠、~越え 【名-9】関門 【名-10】形勢、事態、危機 【名-11】色目、くどき 【自動-1】通る、過ぎる、通り過ぎる、通過する、渡る 【自動-2】(時が)経つ 【自動-3】やむ、終わる 【自動-4】亡くなる 【自動-5】合格する 【自動-6】(議案が)通過する 【他動-1】~を通る、通り越す、追い越す 【他動-2】~に合格する、受かる 【他動-3】(議案)を可決する 【他動-4】(判決)を下す 【他動-5】~を手渡す、パスする、パスを送る[出す]、パス出しする、渡す回す 【他動-6】(時)を過ごす
\\	2年	
\\	【名-1】平和、安らぎ、平穏、安心、安定 【名-2】和解、和平 【名-3】静けさ、沈黙 【名-4】治安 【名-5】事なかれ主義
\\	2年	
\\	【名】完了時制 【形-1】完ぺきな、ぴったりの、完全な、まったくの、完全無欠の、欠点がない、至れり尽くせりの、申し分のない、うまい、言うことがない、板についた、格好の、絶好の、最適の、最高の、満点の 【形-2】完了の 【他動】完全にする、仕上げる、完成させる、マスターする、遂行する、腕を磨く
\\	2年	
\\	【名-1】個人 【名-2】人格、人柄 【名-3】人、人物 【名-4】《文法》人称
\\	2年	
\\	"【名-1】場所、個所、住所、席 【名-2】地域、土地 【名-3】広場 【名-4】立場、境遇、環境、~の身 【他動-1】~を置く、設置する、取り付ける、収納する、セットする 
\\	箱の位置に問題がなければ 
\\	キーを押してください。 【他動-2】(注文書)を発行する 【他動-3】~に入賞する 【他動-4】~を認識する、判別する、思い出す 【他動-5】~を掲載する、(広告)を出す
\\	2年	
\\	【名-1】点、得点◆【略】
\\	【名-2】要点 【名-3】目的、効果、意味、意義 【名-4】ヒント 【名-5】小数点 【名-6】先端(部)、先 【名-7】時点 【名-8】斑点 【名-9】程度 【名-10】地点 【名-11】《印刷》ポイント、パイント◆印刷文字の大きさの単位。=約1/72インチ=0.351ミリメートル。コンピュータ・ディスプレイ上では画面設定によっていくらでも変わってしまう。 【名-12】《重量金衡単位》ポイント、パイント◆金などの重量に使う。=0.01カラット=2ミリグラム 【他動-1】~を向ける 【他動-2】~を示す、指差す、指す、指摘する
\\	2年	
\\	【名】貧民、困窮者、貧困層◆【用法】
\\	【形-1】貧しい、乏しい、金がない、困窮して、すかんぴんの、すってんてんの、きゅうきゅうの、ぴいぴいの、一文無しの 【形-2】下手な、得意でない 【形-3】不幸な、哀れな、見すぼらしい、芳しくない 【形-4】質が悪い、(視界などが)悪い、粗末な 【形-5】弱い、貧弱な 【形-6】土地がやせた
\\	2年	
\\	【名】(悪い結果をもたらす)問題、厄介な人、困ったこと、課題、疑問、難問、難題、不具合
\\	2年	
\\	【名】《金融》プット、プットオプション
\\	◆【参考】
\\	◆指定期間内の指定価格での「売り付け」(権利)。反対は 
\\	。 【他動-1】~を置く、《基本イメージ》手に持っているものをどこかに置く、何かを現在の場所から別の場所に置く 【他動-2】ある状態に~を置く、人の管理下に置く、~をさせる 【他動-3】頭の中にあることを言葉に置き換える、~を表現する、言う 【他動-4】~と見なす 【他動-5】~を付ける 【他動-6】~を加える 【他動-7】~に試練を受けさせる 【他動-8】~を記入する
\\	2年	
\\	"【形】雨の、雨模様の、雨降りの、雨の多い、雨に濡れた 
\\	私は梅雨シーズンは嫌いだ。
\\	2年	
\\	【他動-1】~を受ける、受け取る、受領する、入手する 【他動-2】~を受け入れる、(忠告)を聞き入れる 【他動-3】~を経験する 【他動-4】~を支える、~に耐える 【他動-5】~を理解する、信じる 【他動-6】~に接見する、客を迎える、客を招く、歓迎する、応接する 【他動-7】《コ》受信する 【他動-8】(資格など)を得る
\\	2年	
\\	【名-1】川 【名-2】流れ 【名-3】多量の流出
\\	2年	
\\	【名-1】部屋◆【略】
\\	【名-2】場所、空間、余地、スペース 【名-3】機会、余裕 【名-4】集った人 【名-5】下宿 【自動】下宿する、間借りする◆【前置詞】
\\	【他動】(客)を泊める、人に部屋を貸す、~を部屋に案内する
\\	2年	
\\	【形-1】悲しい、悲しげな、悲しむべき、無念な、残念な 【形-2】嘆かわしい、惨めな、不運な 【形-3】くすんだ
\\	2年	
\\	"【形】上述の、前記の、前述の 【動-1】
\\	の過去・過去分詞形 【動-2】続いて、さらに◆「長い発言」の記述の時、分かりやすいように途中に区切りとして入れる。
\\	発言 
\\	発言 
\\	なら、
\\	発言 
\\	発言 
\\	のように。 
\\	セッド
\\	2年	
\\	【動】
\\	の過去形 
\\	サング
\\	2年	
\\	【名-1】言いたいこと、言い分◆【参考】
\\	【名-2】発言の機会 【副-1】約、大体、およそ、~くらい、~といったところ 【副-2】例えば 【間投】ねえ、なあ、教えて、ところで、そういえば◆呼び掛けの言葉 【自動】言う、意見を述べる、口に出す 【他動-1】言う、述べる、発言する 【他動-2】(印刷物に)~と書いてある、(計測器などが)~を示す 【他動-3】仮に~だとしたら
\\	2年	
\\	【名-1】海 【名-2】海辺、海岸 【名-3】波 【形-1】海の、海辺の、海岸の、海に関する 【形-2】船の 【形-3】海軍の
\\	2年	
\\	【自動】便りを出す 【他動-1】~を送る、発送する、届ける、送信する、渡す 【他動-2】~に行かせる、(学校など)に通わせる 【他動-3】(軍隊)を派遣する 【他動-4】~を発散する
\\	2年	
\\	【助動-1】~すべきである、当然~だろう、《契約書・規格書》~しなければならない 【助動-2】~しましょうか 【助動-3】必ず~となるだろう
\\	2年	
\\	【名】買物 【形】買物の、買い物する
\\	2年	
\\	【助動-1】《義務》~すべきである、~べきだ、~しなくてはならない 【助動-2】《過去の単純未来》~だろう◆従属節内の時制の一致 【助動-3】《推量》~のはずだ、~のはずである 【助動-4】~であるよう◆【用法】提案・要求・希望・決定・必要・妥当などを表す主節に続く 
\\	節内で 【助動-5】《後悔》~だったのに◆【用法】
\\	過去分詞の形をとり、過去への推量と後悔を表す。どちらになるかは文脈による。 【助動-6】~するだろうに◆【用法】仮定法・過去 【助動-7】~しただろうに◆【用法】仮定法・過去完了 【助動-8】《控えめ》~したいと思う、~なのですが◆【用法】
\\	などの動詞がよく用いられる 【助動-9】《感情》~であるとは、~するとは◆【用法】
\\	の形で用いられる 【助動-10】(万一)~ならば◆【用法】仮定法で 
\\	節内に用いられ、帰結節は仮定法以外に現在形の文や命令文もよく見られる
\\	2年	
\\	【名-1】ショー、展示会、見世物、展覧会 【名-2】番組 【名-3】見せること 【名-4】見せびらかし、誇示 【名-5】姿を現すこと 【自動-1】見える、あらわになる 【自動-2】(映画・演劇が)上映[上演]される 【他動-1】~を見せる 【他動-2】~を示す、表示する、明らかにする 【他動-3】~を教える、案内する、指示する 【他動-4】姿を現す
\\	2年	
\\	【名】病人◆【用法】
\\	【形-1】病気で、病気の、不健全な、調子が悪い、吐き気がする、むかつく、気持ち悪い 【形-2】うんざりして、こりごりだ 【形-3】悪趣味な、病的な 【形-4】ブラックユーモアの
\\	2年	
\\	【自動-1】歌う 【自動-2】鳴く、さえずる 【自動-3】音を立てる 【自動-4】礼賛する 【他動】~を歌う
\\	2年	
\\	【名】大きさ、サイズ、寸法、規模◆【略】
\\	2年	
\\	【名】睡眠、眠り、眠気 【自動-1】眠る 【自動-2】泊まる 【自動-3】眠っている
\\	2年	
\\	【名-1】雪 【名-2】降雪 【名-3】雪のようなもの 【自動-1】雪が降る 【自動-2】雪のようにドッと来る、殺到する 【他動】~を雪のように降らせる
\\	2年	
\\	【代名】ある人、誰か
\\	2年	
\\	【名】大した[すごい]もの[人]、大切な[大事な・重要な・有り難い・貴重な]もの、偉い人、結構なこと、よい事 【代名】何か、何かかあるもの[こと]
\\	2年	
\\	【名-1】歌、詩歌、歌曲 【名-2】さえずり
\\	2年	
\\	【名-1】音、音声 【名-2】音波 【名-3】海峡、入り江、瀬戸 【形-1】健全な、健康な、正常な 【形-2】しっかりした 【形-3】《食品》いたんでいない 【形-4】正直な、信頼できる、分別の有る 【形-5】十分な◆睡眠などが 【副】ぐっすりと、十分に◆睡眠などが 【自動-2】それとなく探りを入れる、当たってみる、打診する 【自動-2】聞こえる 【自動-3】~のような音がする 【自動-4】~らしく聞こえる、~のように思える、思われる、印象を与える 【自動】《航海》測る、測定◆水深を◆【参考】
\\	【自動】音を出す[たてる]、鳴る、鳴動する 【他動-1】《航海》(水深)を測る、測定する◆【参考】
\\	【他動-2】~を観測する、探る、調査する◆【参考】
\\	【他動-3】~にそれとなく探りを入れる、当たってみる、打診する 【他動】(音)を発する、~を鳴らす
\\	2年	
\\	【名-1】特別の人・物、特別な人[もの] 【名-2】お薦め料理、特別料理 【名-3】臨時列車、臨時バス 【名-4】特売品 【名-5】特別番組 【名-6】特別賞 【形-1】特別な、独特の、特別の、特有の、特設の 【形-2】専門の、専用の
\\	2年	
\\	【名-1】話すこと、話す能力、話し方 【名-2】スピーチ、せりふ、演説 【名-3】話法 【名-4】言語、発語、話し言葉、発話、言語行動
\\	2年	
\\	【名-1】静寂 【名-2】スチール写真 【名-3】蒸留器、蒸留室 【形-1】静止した、じっとした、動かない 【形-2】風のない、静かな、しんとした、ひっそりしている、平静な 【形-3】黙っている 【形-4】音が低い 【副-1】まだ、今[そのとき]でもまだ、いまだに、今もなお、これから先[その後]もなお依然として、やはり、さらに◆同じ動作[状態]が予期したより長く続いている様子 【副-2】それでも、それでもやはり、~にもかかわらず 【他動-1】~を静める、静かにさせる 【他動-2】~を和らげる、鎮静化させる
\\	2年	
\\	【名-1】停止、ストップ、止まること 【名-2】滞在 【名-3】停留所 【名-4】閉鎖音 【名-5】絞り 【自動-1】(動いているものが)止まる、停止する 【自動-2】終わり[おしまい]にする、終了する 【他動-1】(今やっていること)を(一時的に)止める、中断する 【他動-2】《ボクシング》(相手)をノックアウトする 【他動-3】《野球》~を抑える
\\	2年	
\\	【名-1】物語、話 【名-2】ニュース、情報、記事、報道内容、新聞種 【名-3】経歴 【名-4】階、建物の階
\\	2年	
\\	【形-1】そのような、とても~な、それほどの 【形-2】当該の、かかる、前記の、上記の
\\	2年	
\\	【名-1】甘い食べ物[菓子・飲み物]、キャンディー、飴、デザート 【名-2】甘さ、甘美、甘味、芳香 【名-3】いとしい人、愛する人、愛人、恋人 【名-4】快楽、愉快 【形-1】甘い、甘味のある、甘みの強い、甘口の、砂糖を入れた 【形-2】甘ったるい、甘ったるくてうんざりする、センチメンタルな 【形-3】甘美な 【形-4】塩気のない、無塩の 【形-5】硬水でない 【形-6】香りの良い 【形-7】優しい、思いやりのある、親切な、善意の、人柄のいい 【形-8】快い、楽しい、感じの良い、魅惑的な、ほれぼれする、気持ち良い、心地良い、すてきな、素晴らしい、興味をそそる、きれいな、しとやかな 【形-9】好きで、最愛の、大切な、いとしい、かわいらしい 【形-10】〈豪〉申し分ない、言うことない、結構な、調子の良い、万事 
\\	で、万事好調で◆【同】
\\	【形-11】声の良い、声の美しい、美声の 【形-12】扱いやすい、簡単な
\\	2年	
\\	【接】~に比べて、よりむしろ、~よりも
\\	2年	
\\	【名-1】考え、見解、思い付き 【名-2】思想 【名-3】考えること、思考(力)、思索、熟考 【動】
\\	の過去・過去分詞形
\\	2年	
\\	【名】1000、1000個、1000人、1000分の1、1000ドル、1000ポンド 【形】1000の、1000個の、1000人の、1000分の1の、1000ドルの、1000ポンドの
\\	2年	
\\	【前-1】~を通り抜けて、経て、~の中を通って、~を貫いて 【前-2】~の間中、~の至る所に 【前-3】~を通じて、~の手を経て、手を通して、~を経由して 【前-4】~を介して、~に頼んで、~のために 【前-5】~まで(ずっと) 【形】直通の 【副-1】通り抜けて、直通で、最初から最後まで、ずっと 【副-2】やり終えて、終わって、済んで
\\	2年	
\\	【動】
\\	の過去・過去分詞形 
\\	トウルド
\\	2年	
\\	【名】明日、将来、未来 【副】明日は
\\	2年	
\\	【名-1】旅行、外出、遠足、航海、通勤 【名-2】つまずき、過失 【名-3】トリップ、麻薬で幻覚を感じている状態[時間]、麻薬による幻覚症状 【名-4】変なやつ、変わったやつ、おかしなやつ 【自動-1】軽快な足取りで歩む、軽快に動く 【自動-2】踏み外す、つまずく、転ぶ、失敗する、失策をする 【自動-3】(止め金などが)外れる、(回路が)切れる 【自動-4】(機器が)作動する 【自動-5】
\\	を飲む、飛ぶ、トリップしている、麻薬で幻覚症状を起こす、麻薬で幻覚を感じる 【他動-1】~をつまずかせるう、転ばせる、妨害する、失敗させる、(人)の足をすくう、揚げ足を取る 【他動-2】(止め金など)を外す、引き外す、(回路)を切る 【他動-3】(機器)を始動させる
\\	2年	
\\	【形-1】実際どおりの、真の、本当の、本物の、真実の、事実に反しない 【形-2】正しくそろえられた、正確な、忠実な
\\	2年	
\\	【他動-1】~を理解する、了解する、分かる 【他動-2】~を聞いて知っている、~と聞いている
\\	2年	
\\	【形】役立つ、実用的な、有用な、便利な、使いものになる、使い道がある、実益がある、有益な、好都合である
\\	2年	
\\	【名】村、小集落、村落 【形】村の
\\	2年	
\\	【名】訪問、見舞い、参観、視察、巡回、滞在 【他動-1】~を訪れる、訪問する、見物する、見学する、見舞う、見舞いに行く、~に顔を出す 【他動-2】~に滞在する 【他動-3】~にアクセスする
\\	2年	
\\	【名】暖まること、暖めること、暖かい所 【形-1】暖かい、温かい、暖かい感じの、ホカホカする◆【参考】
\\	【形-2】友好的な 【形-3】正解に近い 【形-4】熱心な 【形-5】好色な 【自動】暖かくなる、(~に)心引かれる 【他動】~を暖める
\\	2年	
\\	【動】
\\	の1人称・3人称単数過去形、
\\	動詞の過去形
\\	2年	
\\	【名-1】洗うこと、洗浄 【名-2】洗浄液、洗剤 【名-3】洗濯(物) 【名-4】残飯 【名-5】波のうねり 【名-6】(水彩絵具・壁の塗料・金・銀などの)薄い塗り、淡い色合い 【形】洗える、洗濯がきく 【自動-1】体(の一部)を洗う 【自動-2】打ち寄せる 【他動-1】~を洗う、洗濯する 【他動-2】~を濡らす 【他動-3】~をえぐり取る
\\	2年	
\\	【名-1】水 【名-2】鉱泉水 【名-3】浸水 【名-4】波紋 【名-5】品質◆宝石の 【名-6】分泌液 【名-7】《複数形で》領海、水域◆【同】
\\	【形-1】水の、水上で行う、水力の 【形-2】水中に住む、水棲の 【自動】(水・涙が)流れる、(よだれが)垂れる 【他動】~に水をやる、水を掛ける、給水する、~を水で薄める
\\	2年	
\\	【名-1】衣類、衣服 【名-2】着用 【名-3】擦り切れ、磨耗、損耗 【自動-1】擦り減る、擦り切れる、薄くなる 【自動-2】使用に耐える、長持ちする 【自動-3】(時が)ゆっくり過ぎる、経過する 【他動-1】(衣服・靴・装飾品・香水・ヘアスタイルなど)を身につける、着る、履く、かぶる 【他動-2】(髪・ひげ)を(~の状態に)伸ばしている、生やしている 【他動-3】~を表している 【他動-4】(使い古して)~を擦り減らす、使いつぶす 【他動-5】(人)を疲れさせる
\\	2年	
\\	【動】
\\	の複数・2人称単数過去形、
\\	の仮定法過去形、
\\	の過去形
\\	2年	
\\	【1-助動-1】~するつもりである 【1-助動-2】多分~でしょう 【1-助動-3】どうしても~するものだ、どうしても~したがる◆物の習性をとらえる表現。「太陽は東から昇る」など、実験による試みが不可能なものには用いないのが普通 【1-助動-4】~しなさい
\\	に強勢をおいて発音する◆【用法】主語は通例二人称で、時に掲示板などで用いられる際には三人称 【1-助動-5】(よく)~することがある◆【用法】肯定文で用いられ、頻度の副詞を伴うこともある◆過去の習慣を表す 
\\	の現在時制と考えられる 【1-助動-6】~できる
\\	で置き換えることも可能。 【2-名-1】意志(力) 【2-名-2】遺言(状) 【2-名-3】願望 【2-名-4】意欲 【2-名-5】態度 【2-名-6】命令 【2-自動】望む、欲する 【2-他動】~を望む、意図する、命ずる、決意する 【助動】~だろう、~でしょう
\\	2年	
\\	【前-1】~なしで、~を持たないで、~なしに、~がなければ、~でなければ、~の限外に 【前-2】~しないで、~せずに
\\	◆【略】
\\	2年	
\\	【形】すてきな、不思議な、素晴らしい、見事な
\\	2年	
\\	【名-1】単語、語 【名-2】発言、言い分 【名-3】言葉、言語、一言 【名-4】口論 【名-5】合言葉、神の言葉、聖書、福音 【名-6】うわさ◆【用法】
\\	【名-7】約束、請け合い、保証 【名-8】《コ》ワード、16
\\	長
\\	長)のデータ◆32、64、128
\\	マシンでは、32、64、128
\\	を 
\\	とする場合もある 【他動】~を言葉で表す
\\	2年	
\\	【動】
\\	の過去形 
\\	ロウト
\\	2年	
\\	【形-1】(季節・時間の)まだ早い 【形-2】若い、若々しい 【形-3】年下の、幼い 【形-4】若き日の
\end{CJK}
\end{document}