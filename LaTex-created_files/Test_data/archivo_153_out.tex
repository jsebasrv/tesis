\documentclass[8pt]{extreport} 
\usepackage{hyperref}
\usepackage{CJKutf8}
\begin{document}
\begin{CJK}{UTF8}{min}
\\	農業	
\\	のうぎょう			
\\	私は農業を勉強しています。	
\\	農業 
\\	わたし は のうぎょう を べんきょう して います			
\\	空港	
\\	くうこう			
\\	空港までリムジンバスで行った。	
\\	空港 
\\	くうこう まで りむじん ばす で いった			
\\	珍しい	
\\	めずらしい			
\\	昨日珍しい果物を食べました。	
\\	珍しい 
\\	きのう めずらしい くだもの を たべました			
\\	急ぐ	
\\	いそぐ			
\\	私たちは駅へ急ぎました。	
\\	急ぐ 
\\	わたしたち は えき へ いそぎました			
\\	つもり	
\\	明日からタバコを止めるつもりです。	
\\	つもり 
\\	あした から たばこ を やめる つもり です			
\\	必ず	
\\	かならず			
\\	必ずシートベルトを着けて下さい。	
\\	必ず 
\\	かならず しーとべると を つけて ください			
\\	数字	
\\	すうじ			
\\	数字は苦手です。	
\\	数字 
\\	すうじ は にがて です			
\\	やっと	
\\	やっと仕事が終わりました。	
\\	やっと 
\\	やっと しごと が おわりました			
\\	旅行	
\\	りょこう			
\\	彼女は旅行が好きです。	
\\	旅行 
\\	かのじょ は りょこう が すき です			
\\	音楽	
\\	おんがく			
\\	私は音楽を聞くのが好きだ。	
\\	音楽 
\\	わたし は おんがく を きく の が すき だ			
\\	川	
\\	かわ			
\\	小さな川を渡りました。	
\\	川 
\\	ちいさ な かわ を わたりました			
\\	例えば	
\\	たとえば			
\\	例えば、このソフトで日本語を勉強することができます。	
\\	例えば 
\\	たとえば この そふと で にほんご を べんきょう する こと が できます			
\\	小学校	
\\	しょうがっこう			
\\	家の近くに小学校があります。	
\\	小学校 
\\	いえ の ちかく に しょうがっこう が あります			
\\	壁	
\\	かべ			
\\	壁に絵が掛かっている。	
\\	壁 
\\	かべ に え が かかって いる			
\\	流れる	
\\	ながれる			
\\	ラジオから美しい音楽が流れています。	
\\	音楽 
\\	流れる 
\\	らじお から うつくしい おんがく が ながれて います			
\\	夢	
\\	ゆめ			
\\	私の夢は先生になることです。	
\\	夢 
\\	わたし の ゆめ は せんせい に なる こと です			
\\	逃げる	
\\	にげる			
\\	過ぎる	
\\	すぎる			
\\	時が過ぎるのは速い。	
\\	過ぎる 
\\	とき が すぎる の は はやい 。			
\\	首都	
\\	しゅと			
\\	東京は日本の首都です。	
\\	首都 
\\	とうきょう は にっぽん の しゅと です			
\\	教師	
\\	きょうし			
\\	彼は高校教師だ。	
\\	教師 
\\	かれ は こうこう きょうし だ			
\\	相談	
\\	そうだん			
\\	相談したいことがあります。	
\\	相談 
\\	そうだん したい こと が あります			
\\	用意	
\\	ようい			
\\	食事の用意ができました。	
\\	用意 
\\	しょくじ の ようい が できました			
\\	ニュース	
\\	夜のニュースを見ましたか。	
\\	ニュース 
\\	よる の にゅーす を みました か			
\\	法律	
\\	ほうりつ			
\\	新しい法律ができた。	
\\	法律 
\\	あたらしい ほうりつ が できた			
\\	大変	
\\	たいへん			
\\	大変なことが起こりました。	
\\	大変 
\\	たいへん な こと が おこりました			
\\	戦争	
\\	せんそう			
\\	2003年にイラクで戦争があった。	
\\	戦争 
\\	にせんさんねん に いらく で せんそう が あった			
\\	ずっと	
\\	父は休みの日はずっとテレビを見ている。	
\\	ずっと 
\\	ちち は やすみ の ひ は ずっと てれび を みて いる			
\\	自然	
\\	しぜん			
\\	親が子供を守るのは自然なことだ。	
\\	自然 
\\	おや が こども を まもる の は しぜん な こと だ			
\\	島	
\\	しま			
\\	日本は島国です。	
\\	島 
\\	にっぽん は しまぐに です			
\\	窓	
\\	まど			
\\	鳥が窓から逃げたよ。	
\\	逃げる 
\\	窓 
\\	とり が まど から にげた よ			
\\	窓を開けてください。	
\\	窓 
\\	まど を あけて ください			
\\	考え	
\\	かんがえ			
\\	それは良い考えです。	
\\	考え 
\\	それ は いい かんがえ です			
\\	ビデオ	
\\	私はその番組をビデオに撮った。	
\\	ビデオ 
\\	わたし は その ばんぐみ を びでお に とった			
\\	マンション	
\\	彼はマンションに住んでいます。	
\\	マンション 
\\	かれ は まんしょん に すんで います			
\\	健康	
\\	けんこう			
\\	私は健康な生活を送っています。	
\\	健康 
\\	わたし は けんこう な せいかつ を おくって います			
\\	通り	
\\	とおり			
\\	この通りはにぎやかですね。	
\\	通り 
\\	この とおり は にぎやか です ね			
\\	喜ぶ	
\\	よろこぶ			
\\	彼女はとても喜びました。	
\\	喜ぶ 
\\	かのじょ は とても よろこびました			
\\	入院	
\\	にゅういん			
\\	昨日、母が入院しました。	
\\	入院 
\\	きのう はは が にゅういん しました			
\\	増やす	
\\	ふやす			
\\	あの町は緑を増やしています。	
\\	増やす 
\\	あの まち は みどり を ふやして います			
\\	育てる	
\\	そだてる			
\\	彼女は三人の子を育てました。	
\\	育てる 
\\	かのじょ は さんにん の こ を そだてました			
\\	温度	
\\	おんど			
\\	今、部屋の温度は25度だ。	
\\	温度 
\\	いま へや の おんど は にじゅうごど だ			
\\	雑誌	
\\	ざっし			
\\	この雑誌はよく売れています。	
\\	雑誌 
\\	この ざっし は よく うれて います			
\\	私たち	
\\	わたしたち			
\\	私たちは来月結婚します。	
\\	私たち 
\\	わたしたち は らいげつ けっこん します 。			
\\	食事	
\\	しょくじ			
\\	今日は上司と食事をする。	
\\	食事 
\\	きょう は じょうし と しょくじ を する			
\\	風	
\\	かぜ			
\\	今日は風が強いです。	
\\	風 
\\	きょう は かぜ が つよい です			
\\	早く	
\\	はやく			
\\	なるべく早く来て下さい。	
\\	早く 
\\	なるべく はやく きて ください			
\\	明るい	
\\	あかるい			
\\	彼女は明るい性格です。	
\\	明るい 
\\	かのじょ は あかるい せいかく です			
\\	運転	
\\	うんてん			
\\	父は安全運転だ。	
\\	運転 
\\	ちち は あんぜん うんてん だ			
\\	しばらく	
\\	そこでしばらく休んでいます。	
\\	しばらく 
\\	そこで しばらく やすんで います			
\\	高さ	
\\	たかさ			
\\	富士山の高さは3,776メートルです。	
\\	高さ 
\\	ふじさん の たかさ は さんぜんななひゃく-ななじゅうろくめーとる です			
\\	借りる	
\\	かりる			
\\	彼にビデオを借りました。	
\\	借りる 
\\	かれ に びでお を かりました			
\\	放送	
\\	ほうそう			
\\	就職	
\\	しゅうしょく			
\\	最近、若い人たちの就職が難しくなっています。	
\\	就職 
\\	さいきん わかい ひとたち の しゅうしょく が むずかしく なって います			
\\	大きさ	
\\	おおきさ			
\\	この大きさの封筒が欲しいのですが。	
\\	大きさ 
\\	この おおきさ の ふうとう が ほしい の です が			
\\	困る	
\\	こまる			
\\	ケータイをなくして困っています。	
\\	困る 
\\	けーたい を なくして こまって います			
\\	通る	
\\	とおる			
\\	毎日、この道を通ります。	
\\	通る 
\\	まいにち この みち を とおります			
\\	ガス	
\\	地震でガスが止まった。	
\\	ガス 
\\	じしん で がす が とまった			
\\	夫婦	
\\	ふうふ			
\\	その夫婦はとても仲がいい。	
\\	夫婦 
\\	その ふうふ は とても なか が いい			
\\	立てる	
\\	たてる			
\\	彼はケーキにろうそくを立てた。	
\\	立てる 
\\	かれ は けーき に ろうそく を たてた			
\\	伸びる	
\\	のびる			
\\	髪がだいぶ伸びたね。	
\\	伸びる 
\\	かみ が だいぶ のびた ね			
\\	聞こえる	
\\	きこえる			
\\	殺す	
\\	ころす			
\\	私は生き物を殺すのが嫌いだ。	
\\	殺す 
\\	わたし は いきもの を ころす の が きらい だ			
\\	静か	
\\	しずか			
\\	今年の夏休みには静かな所へ行きたい。	
\\	静か 
\\	ことし の なつやすみ に は しずか な ところ へ いきたい			
\\	分かれる	
\\	わかれる			
\\	グループの中で意見が分かれました。	
\\	分かれる 
\\	ぐるーぷ の なか で いけん が わかれました			
\\	詳しい	
\\	くわしい			
\\	もっと詳しく説明してください。	
\\	詳しい 
\\	もっと くわしく せつめい して ください			
\\	正確	
\\	せいかく			
\\	彼の計算は正確です。	
\\	正確 
\\	かれ の けいさん は せいかく です			
\\	うまい	
\\	彼は野球がうまい。	
\\	うまい 
\\	かれ は やきゅう が うまい			
\\	打つ	
\\	うつ			
\\	転んでひざを打ちました。	
\\	打つ 
\\	ころんで ひざ を うちました			
\\	工業	
\\	こうぎょう			
\\	そこは工業都市だ。	
\\	工業 
\\	そこ は こうぎょう とし だ			
\\	サラリーマン	
\\	父はサラリーマンです。	
\\	サラリーマン 
\\	ちち は さらりーまん です			
\\	有名	
\\	ゆうめい			
\\	彼は有名な歌手です。	
\\	有名 
\\	かれ は ゆうめい な かしゅ です			
\\	捨てる	
\\	すてる			
\\	ゴミを捨ててください。	
\\	捨てる 
\\	ごみ を すてて ください			
\\	移す	
\\	うつす			
\\	机を窓の傍に移しました。	
\\	移す 
\\	つくえ を まど の そば に うつしました			
\\	正しい	
\\	ただしい			
\\	それは正しい答えです。	
\\	正しい 
\\	それ は ただしい こたえ です			
\\	売れる	
\\	うれる			
\\	今年の夏はクーラーがよく売れた。	
\\	売れる 
\\	ことし の なつ は くーらー が よく うれた			
\\	番組	
\\	ばんぐみ			
\\	その番組は来週放送されます。	
\\	放送 
\\	番組 
\\	その ばんぐみ は らいしゅう ほうそう されます			
\\	私はこの番組が好きです。	
\\	番組 
\\	わたし は この ばんぐみ が すき です			
\\	代わる	
\\	かわる			
\\	上司に代わって会議に出た。	
\\	代わる 
\\	じょうし に かわって かいぎ に でた			
\\	着く	
\\	つく			
\\	午後8時に大阪に着きます。	
\\	着く 
\\	ごご はちじ に おおさか に つきます			
\\	バス	
\\	バスで行こう。	
\\	バス 
\\	ばす で いこう			
\\	中学	
\\	ちゅうがく			
\\	息子は中学に通っています。	
\\	中学 
\\	むすこ は ちゅうがく に かよって います			
\\	歌	
\\	うた			
\\	私はその歌を知らなかった。	
\\	歌 
\\	わたし は その うた を しらなかった 。			
\\	薄い	
\\	うすい			
\\	この電子辞書はとても薄い。	
\\	薄い 
\\	この でんし じしょ は とても うすい			
\\	減らす	
\\	へらす			
\\	最近、食事を減らしています。	
\\	減らす 
\\	さいきん しょくじ を へらして います			
\\	科学	
\\	かがく			
\\	科学は常に進歩している。	
\\	科学 
\\	かがく は つねに しんぽ して いる			
\\	約束	
\\	やくそく			
\\	約束は守ります。	
\\	約束 
\\	やくそく は まもります			
\\	出発	
\\	しゅっぱつ			
\\	あと15分で出発です。	
\\	出発 
\\	あと じゅうごふん で しゅっぱつ です			
\\	動かす	
\\	うごかす			
\\	部屋の家具を動かしました。	
\\	動かす 
\\	へや の かぐ を うごかしました			
\\	紹介	
\\	しょうかい			
\\	両親に彼女を紹介した。	
\\	紹介 
\\	りょうしん に かのじょ を しょうかい した			
\\	倒れる	
\\	たおれる			
\\	強風で木が倒れた。	
\\	倒れる 
\\	きょうふう で き が たおれた			
\\	席	
\\	せき			
\\	この席、空いてますか。	
\\	席 
\\	この せき あいてます か			
\\	熱	
\\	ねつ			
\\	昨日の夜、熱が出ました。	
\\	熱 
\\	きのう の よる ねつ が でました			
\\	値段	
\\	ねだん			
\\	このベルトの値段は2500円でした。	
\\	値段 
\\	この べると の ねだん は にせんごひゃくえん でした			
\\	長さ	
\\	ながさ			
\\	このケーブルの長さは1メートルです。	
\\	長さ 
\\	この けーぶる の ながさ は いちめーとる です			
\\	クラス	
\\	この学校は1クラス30人です。	
\\	クラス 
\\	この がっこう は 
\\	クラス 
\\	にん です 。			
\\	隣	
\\	となり			
\\	隣の部屋からテレビの音が聞こえる。	
\\	聞こえる 
\\	隣 
\\	となり の へや から てれび の おと が きこえる			
\\	隣の家には犬がいます。	
\\	隣 
\\	となり の いえ に は いぬ が います			
\\	トラック	
\\	トラックを運転できますか。	
\\	トラック 
\\	とらっく を うんてん できます か			
\\	大使館	
\\	たいしかん			
\\	彼は大使館に勤めています。	
\\	大使館 
\\	かれ は たいしかん に つとめて います			
\\	亡くなる	
\\	なくなる			
\\	おととい、昔の友人が亡くなった。	
\\	亡くなる 
\\	おととい むかし の ゆうじん が なくなった			
\\	パーティー	
\\	明日、うちでパーティーを開きます。	
\\	パーティー 
\\	あした うち で ぱーてぃー を ひらきます			
\\	犬	
\\	いぬ			
\\	この犬はとても賢い。	
\\	犬 
\\	この いぬ は とても かしこい			
\\	うまい寿司屋を見つけたよ。	
\\	うまい 
\\	うまい すしや を みつけた よ			
\\	特別	
\\	とくべつ			
\\	あなたは私にとって特別な人です。
\\	特別 
\\	あなた は わたし に とって とくべつ な ひと です			
\\	教室	
\\	きょうしつ			
\\	私の教室は3階にあります。	
\\	教室 
\\	わたし の きょうしつ は さんがい に あります			
\\	ガラス	
\\	クリスタルガラスの花瓶を買いました。	
\\	ガラス 
\\	くりすたるがらす の かびん を かいました			
\\	振る	
\\	ふる			
\\	犬がしっぽを振っている。	
\\	振る 
\\	いぬ が しっぽ を ふって いる			
\\	落とす	
\\	おとす			
\\	途中で財布を落としました。	
\\	落とす 
\\	とちゅう で さいふ を おとしました			
\\	夕方	
\\	ゆうがた			
\\	夕方、雨が降りました。	
\\	夕方 
\\	ゆうがた あめ が ふりました			
\\	コース	
\\	私は日本語コースを取っています。	
\\	コース 
\\	わたし は にほんご こーす を とって います			
\\	アパート	
\\	彼はアパートに住んでいます。	
\\	アパート 
\\	かれ は あぱーと に すんで います			
\\	小説	
\\	しょうせつ			
\\	私は月に3冊くらい小説を読みます。	
\\	小説 
\\	わたし は つき に さんさつ くらい しょうせつ を よみます			
\\	レコード	
\\	ジャズのレコードをかけました。	
\\	レコード 
\\	じゃず の れこーど を かけました			
\\	どんどん	
\\	ドアをどんどんとたたいた。	
\\	どんどん 
\\	どあ を どんどん と たたいた			
\\	通う	
\\	かよう			
\\	私はジムに通っています。	
\\	通う 
\\	わたし は じむ に かよって います			
\\	秘密	
\\	ひみつ			
\\	これは秘密です。	
\\	秘密 
\\	これ は ひみつ です			
\\	料理	
\\	りょうり			
\\	母は料理が得意です。	
\\	料理 
\\	はは は りょうり が とくい です			
\\	苦しい	
\\	くるしい			
\\	食べ過ぎておなかが苦しい。	
\\	苦しい 
\\	たべすぎて おなか が くるしい			
\\	信じる	
\\	しんじる			
\\	彼はキリストを信じている。	
\\	信じる 
\\	かれ は きりすと を しんじて いる			
\\	何か	
\\	なにか			
\\	道に何か落ちています。	
\\	何か 
\\	みち に なにか おちて います			
\\	カメラ	
\\	カメラが壊れた。	
\\	カメラ 
\\	かめら が こわれた			
\\	下げる	
\\	さげる			
\\	少し音量を下げてください。	
\\	下げる 
\\	すこし おんりょう を さげて ください			
\\	留学	
\\	りゅうがく			
\\	1年間、アメリカに留学しました。	
\\	留学 
\\	いちねんかん あめりか に りゅうがく しました			
\\	無理	
\\	むり			
\\	5時までに家に帰るのは無理です。	
\\	無理 
\\	ごじ まで に いえ に かえる の は むり です			
\\	飾る	
\\	かざる			
\\	テーブルの上に花を飾りました。	
\\	飾る 
\\	てーぶる の うえ に はな を かざりました			
\\	北	
\\	きた			
\\	ロシアは日本の北にあります。	
\\	北 
\\	ろしあ は にっぽん の きた に あります			
\\	訪ねる	
\\	たずねる			
\\	彼はニューヨークの友達を訪ねた。	
\\	訪ねる 
\\	かれ は にゅーよーく の ともだち を たずねた			
\\	テープ	
\\	彼女はその会話をテープに録音した。	
\\	テープ 
\\	かのじょ は その かいわ を てーぷ に ろくおん した			
\\	馬	
\\	うま			
\\	彼は牧場で馬に乗った。	
\\	馬 
\\	かれ は ぼくじょう で うま に のった			
\\	長男	
\\	ちょうなん			
\\	彼の長男は8才です。	
\\	長男 
\\	かれ の ちょうなん は はっさい です			
\\	近所	
\\	きんじょ			
\\	近所にカナダ人が住んでいる。	
\\	近所 
\\	きんじょ に かなだじん が すんで いる			
\\	数学	
\\	すうがく			
\\	兄は数学の先生です。	
\\	数学 
\\	あに は すうがく の せんせい です			
\\	入学	
\\	にゅうがく			
\\	妹は九月にアメリカの大学に入学します。	
\\	入学 
\\	いもうと は くがつ に あめりか の だいがく に にゅうがく します			
\\	黒い	
\\	くろい			
\\	彼女は黒いドレスを着ています。	
\\	黒い 
\\	かのじょ は くろい どれす を きて います			
\\	頼む	
\\	たのむ			
\\	私はハンバーガーを頼みました。	
\\	頼む 
\\	わたし は はんばーがー を たのみました			
\\	並べる	
\\	ならべる			
\\	私は料理をテーブルに並べた。	
\\	並べる 
\\	わたし は りょうり を てーぶる に ならべた			
\\	受け取る	
\\	うけとる			
\\	彼からメールを受け取りました。	
\\	受け取る 
\\	かれ から めーる を うけとりました			
\\	太陽	
\\	たいよう			
\\	太陽が雲に隠れた。	
\\	太陽 
\\	たいよう が くも に かくれた			
\\	年寄り	
\\	としより			
\\	あの村にはお年寄りがたくさん住んでいます。	
\\	年寄り 
\\	あの むら に は おとしより が たくさん すんで います			
\\	味	
\\	あじ			
\\	この料理は味が薄い。	
\\	薄い 
\\	味 
\\	この りょうり は あじ が うすい			
\\	思い出す	
\\	おもいだす			
\\	大切な用事を思い出しました。	
\\	思い出す 
\\	たいせつ な ようじ を おもいだしました			
\\	細かい	
\\	こまかい			
\\	彼女は細かいことにうるさい。	
\\	細かい 
\\	かのじょ は こまかい こと に うるさい			
\\	成績	
\\	せいせき			
\\	成績が上がりました。	
\\	成績 
\\	せいせき が あがりました			
\\	涙	
\\	なみだ			
\\	母は涙を流していた。	
\\	涙 
\\	はは は なみだ を ながして いた			
\\	横	
\\	よこ			
\\	横の長さは1メートルです。	
\\	横 
\\	よこ の ながさ は いちめーとる です			
\\	穴	
\\	あな			
\\	靴下に穴が開いている。	
\\	穴 
\\	くつした に あな が あいて いる			
\\	季節	
\\	きせつ			
\\	私の一番好きな季節は春です。	
\\	季節 
\\	わたし の いちばん すき な きせつ は はる です			
\\	ビール	
\\	夏はビールがとても美味しい。	
\\	ビール 
\\	なつ は びーる が とても おいしい			
\\	故障	
\\	こしょう			
\\	冷蔵庫が故障しました。	
\\	故障 
\\	れいぞうこ が こしょう しました			
\\	裏	
\\	うら			
\\	会社の裏に公園があります。	
\\	裏 
\\	かいしゃ の うら に こうえん が あります			
\\	ページ	
\\	迎える	
\\	むかえる			
\\	姉が空港まで迎えに来てくれます。	
\\	空港 
\\	迎える 
\\	あね が くうこう まで むかえに きて くれます			
\\	野菜	
\\	やさい			
\\	私は毎日たくさん野菜を食べます。	
\\	野菜 
\\	わたし は まいにち たくさん やさい を たべます			
\\	渡す	
\\	わたす			
\\	彼に手紙を渡しました。	
\\	渡す 
\\	かれ に てがみ を わたしました			
\\	贈る	
\\	おくる			
\\	母に花を贈った。	
\\	贈る 
\\	はは に はな を おくった			
\\	合格	
\\	ごうかく			
\\	娘が入学試験に合格しました。	
\\	合格 
\\	むすめ が にゅうがく しけん に ごうかく しました			
\\	グラフ	
\\	彼は売上をグラフにした。	
\\	グラフ 
\\	かれ は うりあげ を ぐらふ に した			
\\	公園	
\\	こうえん			
\\	公園に小さな池があります。	
\\	公園 
\\	こうえん に ちいさ な いけ が あります			
\\	戸	
\\	と			
\\	部屋の戸が開いています。	
\\	戸 
\\	へや の と が あいて います			
\\	中学生	
\\	ちゅうがくせい			
\\	息子は中学生です。	
\\	中学生 
\\	むすこ は ちゅうがくせい です			
\\	玄関	
\\	げんかん			
\\	玄関に花を飾りました。	
\\	飾る 
\\	玄関 
\\	げんかん に はな を かざりました			
\\	撮る	
\\	とる			
\\	写真をたくさん撮りました。	
\\	撮る 
\\	しゃしん を たくさん とりました			
\\	夏休み	
\\	なつやすみ			
\\	夏休みにお祖母ちゃんの家へ行きました。	
\\	夏休み 
\\	なつやすみ に おばあちゃん の うち へ いきました			
\\	毎年	
\\	まいとし			
\\	私は毎年、海外旅行に行きます。	
\\	毎年 
\\	わたし は まいとし かいがい りょこう に いきます			
\\	地図	
\\	ちず			
\\	地図を見て来てください。	
\\	地図 
\\	ちず を みて きて ください			
\\	海岸	
\\	かいがん			
\\	海岸を散歩しましょう。	
\\	海岸 
\\	かいがん を さんぽ しましょう			
\\	急に	
\\	きゅうに			
\\	急に用事を思い出した。	
\\	思い出す 
\\	急に 
\\	きゅうに ようじ を おもいだした			
\\	女の子	
\\	おんなのこ			
\\	あの女の子を知っていますか。	
\\	女の子 
\\	あの おんなのこ を しって います か			
\\	ポスト	
\\	手紙をポストに入れました。	
\\	ポスト 
\\	てがみ を ぽすと に いれました			
\\	暗い	
\\	くらい			
\\	東の空が暗いです。	
\\	暗い 
\\	ひがし の そら が くらい です			
\\	両方	
\\	りょうほう			
\\	チョコレートとケーキを両方ください。	
\\	両方 
\\	ちょこれーと と けーき を りょうほう ください			
\\	向こう	
\\	むこう			
\\	友達は向こうにいます。	
\\	向こう 
\\	ともだち は むこう に います			
\\	忙しい	
\\	いそがしい			
\\	忙しいので手伝ってください。	
\\	忙しい 
\\	いそがしい の で てつだって ください			
\\	給料	
\\	きゅうりょう			
\\	来年から給料が上がります。	
\\	給料 
\\	らいねん から きゅうりょう が あがります			
\\	テスト	
\\	明日は理科のテストです。	
\\	テスト 
\\	あした は りか の てすと です			
\\	あちこち	
\\	私たちは朝からあちこち散歩しました。	
\\	あちこち 
\\	わたしたち は あさ から あちこち さんぽ しました			
\\	切れる	
\\	きれる			
\\	このはさみはよく切れますね。	
\\	切れる 
\\	この はさみ は よく きれます ね			
\\	野球	
\\	やきゅう			
\\	友達と野球をしました。	
\\	野球 
\\	ともだち と やきゅう を しました			
\\	試験	
\\	しけん			
\\	彼は採用試験に合格した。	
\\	試験 
\\	かれ は さいよう しけん に ごうかく した			
\\	台風	
\\	たいふう			
\\	台風が近づいている。	
\\	台風 
\\	たいふう が ちかづいて いる			
\\	規則	
\\	きそく			
\\	あの会社の規則は厳しいです。	
\\	規則 
\\	あの かいしゃ の きそく は きびしい です			
\\	遠く	
\\	とおく			
\\	遠くに船が見えます。	
\\	遠く 
\\	とおく に ふね が みえます			
\\	ゴルフ	
\\	兄はゴルフを始めました。	
\\	ゴルフ 
\\	あに は ごるふ を はじめました			
\\	看護婦	
\\	かんごふ			
\\	母は看護婦です。	
\\	看護婦 
\\	はは は かんごふ です			
\\	ラジオ	
\\	私は毎朝ラジオを聴きます。	
\\	ラジオ 
\\	わたし は まいあさ らじお を ききます			
\\	逢う	
\\	あう			
\\	ついに素晴らしい女性に逢えた。	
\\	逢う 
\\	ついに すばらしい じょせい に あえた 。			
\\	タクシー	
\\	タクシーを呼んでください。	
\\	タクシー 
\\	たくしー を よんで ください			
\\	来月	
\\	らいげつ			
\\	来月から大学生になります。	
\\	来月 
\\	らいげつ から だいがくせい に なります			
\\	教科書	
\\	きょうかしょ			
\\	教科書の36ページを開いてください。	
\\	ページ 
\\	教科書 
\\	きょうかしょ の さんじゅうろくぺーじ を ひらいて ください			
\\	日本語の教科書を忘れた。	
\\	教科書 
\\	にほんご の きょうかしょ を わすれた			
\\	電気	
\\	でんき			
\\	急	
\\	きゅう			
\\	急な坂道を上った。	
\\	急 
\\	きゅう な さかみち を のぼった			
\\	ゆっくり	
\\	もっとゆっくり話してください。	
\\	ゆっくり 
\\	もっと ゆっくり はなして ください			
\\	兄弟	
\\	きょうだい			
\\	彼は3人兄弟です。	
\\	兄弟 
\\	かれ は さんにん きょうだい です			
\\	牛肉	
\\	ぎゅうにく			
\\	夕食に牛肉を買った。	
\\	牛肉 
\\	ゆうしょく に ぎゅうにく を かった			
\\	確か	
\\	たしか			
\\	彼の昇進は確かだ。	
\\	確か 
\\	かれ の しょうしん は たしか だ			
\\	慣れる	
\\	なれる			
\\	新しい家にはもう慣れましたか。	
\\	慣れる 
\\	あたらしい いえ に は もう なれました か			
\\	厚い	
\\	あつい			
\\	その辞書はとても厚い。	
\\	厚い 
\\	その じしょ は とても あつい			
\\	返す	
\\	かえす			
\\	図書館に本を返した。	
\\	返す 
\\	としょかん に ほん を かえした			
\\	中学校	
\\	ちゅうがっこう			
\\	息子の中学校は家から5分です。	
\\	中学校 
\\	むすこ の ちゅうがっこう は いえ から ごふん です			
\\	森	
\\	もり			
\\	私は森を歩くのが好きです。	
\\	森 
\\	わたし は もり を あるく の が すき です			
\\	降りる	
\\	おりる			
\\	次の駅で降ります。	
\\	降りる 
\\	つぎ の えき で おります			
\\	曲	
\\	きょく			
\\	私はこの曲が大好きです。	
\\	曲 
\\	わたし は この きょく が だいすき です			
\\	遊び	
\\	あそび			
\\	お正月には色々な遊びをします。	
\\	遊び 
\\	おしょうがつ に は いろいろ な あそび を します			
\end{CJK}
\end{document}