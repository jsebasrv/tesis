\documentclass[8pt]{extreport} 
\usepackage{hyperref}
\usepackage{CJKutf8}
\begin{document}
\begin{CJK}{UTF8}{min}
\\	危険にさらされ、わらにもすがる思いでいる人々を、私たちは放っておけない。	
\\	~に背を向ける 
\\	(人が)わらにもすがる思いの 
\\	危険にさらされている
\\	議会は発展途上国諸国に財政援助を行うことを決議した。	
\\	援助
\\	英国の君主制は廃止されるべきだと言う人もいるけど、あなたはどう思う?	
\\	君主制
\\	彼らは攻撃を逃れる安全な場所を探したが、隠れる場所はどこにもなかった。	
\\	軍縮において、両国が大きな進展を遂げる見込みはほとんどない。	
\\	で)...な進歩(進展) 
\\	軍備縮小 
\\	可能性、見込み 
\\	軍備縮小
\\	交渉は進行中だ。そろそろ正念場を迎えるだろう。	
\\	進行中である 
\\	極めて重大な
\\	奴隷たちは自分たちの運命を変えたいと切望している。	
\\	貴族たちは思うがままに特権を乱用した。	
\\	〜を乱用する(~を虐待する(音単)) 
\\	思う存分…する
\\	結局、革命それ自体は何の成果ももたらさなかった。	
\\	成果を生む
\\	ボブが「それには反対だ!」と言って会話に割り込んできたけれども、誰も耳を貸さなかった。	
\\	割り込む
\\	私たちは彼の順番をわざと飛ばした。	
\\	←綴りに注意
\\	誰かがいじめられているのを見たのに、彼は見ないふりをした。	
\\	見て見ぬふりをする
\\	ニックが僕とテストでカンニングしたとき、僕は停学処分、ニックの方はまんまと逃れた。	
\\	~で不正行為(カンニング)をする 
\\	~を停学にする 
\\	~の罰をまんまと逃れる 
\\	~から逃れる
\\	彼らはリチャードの富と地位に対する嫉妬から、彼に敵意を抱いている。	
\\	敵対的な 
\\	~をねたんでいる
\\	彼のことは大嫌い!まるで自分が大物であるかのように振舞うのよ。	
\\	すごい人、大物  
\\	すごいこと(もの) *文中のイタリック体(斜体)は強調を表す。
\\	僕たちがほめ言葉をかけたのに、彼は顔をしかめてそっぽを向いた。失礼だ!	
\\	顔をしかめる 
\\	顔をそむける
\\	その莫大な財産は自分が自由にできるものだ、と彼は主張した。	
\\	~の自由にできる
\\	頑固になればなるほど孤立するよ。	
\\	「遠回しな言い方はやめてくれ!僕にどうしろっていうんだ?」 「慌てずに待つしかありません。時がたてばわかります。」	
\\	話をじらす 
\\	慌てずに待つしかないよ(定型表現) 
\\	時がたてばわかる(定型表現)
\end{CJK}
\end{document}