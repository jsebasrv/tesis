\documentclass[8pt]{extreport} 
\usepackage{hyperref}
\usepackage{CJKutf8}
\begin{document}
\begin{CJK}{UTF8}{min}
\\	1年	
\\	【名】俳優、役者、動作主、主体、能動者、関係者
\\	1年	
\\	【名】《動物》ベッキー◆《米》ポトマックの桜並木をかじり倒したビーバー◆【参考】
\\	【人名】ベッキ-
\\	の略称 【分節】
\\	1年	
\\	【名-1】橋、桟橋、主げた 【名-2】歯の矯正治具、ブリッジ、架工義歯 【名-3】鼻梁 【名-4】《トランプ》ブリッジ 【名-5】《コ》ブリッジ
\\	基本参照モデルのデータリンク層において、
\\	セグメントの相互接続に用いられる。 【他動-1】~に橋をかける、橋渡しをする 【他動-2】(空間・すき間・空白)を埋める 【他動-3】乗り越える、~を克服する
\\	1年	
\\	【名-1】ニワトリ、チキン◆【仏語】
\\	【名-2】〈俗〉臆病者、怖がり、弱虫、青二才 【形】臆病な、怖い、ビクビクする
\\	1年	
\\	【名】中華街、中国人街、チャイナタウン 【映画】チャイナ・タウン◆米1974《監督》ロマン・ポランスキー《出演》ジャック・ニコルソン、フェイ・ダナウェイ、ジョン・ヒューストン《受賞》アカデミー脚本賞
\\	1年	
\\	【名】合唱、コーラス、合唱曲 【自他動】合唱する
\\	1年	
\\	【名-1】《菓子》クッキー 【名-2】《インターネット》クッキー
\\	サイトがユーザのコンピュータに記録する小さなサイズの情報。
\\	1年	
\\	【名-1】乳牛、《動物》雌牛、飼い牛、畜牛 【名-2】雌 【他動】~を脅す、おびえさせる
\\	1年	
\\	【名-1】発熱、熱病、熱 【名-2】熱中、興奮
\\	1年	
\\	【名-1】第15(番目)、15日、15番目のもの[人] 【名-2】15分の1 【形-1】第15の、15番目の 【形-2】15分(の1)の
\\	1年	
\\	【名-1】フルート(状のもの)◆【略】
\\	【名-2】溝、段 【名-3】襞飾り、丸溝ひだ、縁模様 【自動】フルートを吹く
\\	1年	
\\	【名-1】アメリカンフットボール、〈米〉フットボール 【名-2】〈英〉サッカー、蹴球 【名-3】〈英〉ラグビー
\\	1年	
\\	【名-1】第14(番目)、14日、14番目のもの[人] 【名-2】14分の1、十四分の一 【形-1】第14の、14番目の 【形-2】14分(の1)の、十四分の一の
\\	1年	
\\	【名】丘、小山、坂
\\	1年	
\\	【間投-1】いやだね、ふん!、エ、エッ、ハァ、フン、ヘッ、ヘン、ン?、何だって? 【間投-2】《付加疑問》~でしょう?、そうではないか? 
\\	ハ
\\	1年	
\\	【名】急ぐこと、急ぐ必要 【自動】急いでする、急ぐ 【他動】~を急がせる
\\	1年	
\\	【名-1】ジャケット、背広の上着、上着 【名-2】本のカバー、カバー 【名-3】被覆物
\\	1年	
\\	【名-1】ジャム 【名-2】混雑 【名-3】窮地、苦境 【名-4】《野球》ピンチ 【自動-1】(機械などが)動かなくなる、故障する、引っかかる、つかえる 【自動-2】一緒に音楽を演奏する、自由にアドリブ演奏を展開する 【他動-1】~を詰め込む、はさむ 【他動-2】~を妨害する
\\	1年	
\\	【名】ジャーナリスト、新聞記者、報道記者
\\	1年	
\\	【人名】ケヴィン◆ファーストネーム◆【愛称】
\\	(ケヴ)◆【語源】ケルト語では、
\\	となる。◆【例】
\\	(俳優)
\\	1年	
\\	【名】台所、調理場、料理室、厨房、調理スペース
\\	1年	
\\	【名】《動物》コアラ
\\	1年	
\\	【名】可能性、不確実性、可能性のあるもの[人] 【副-1】もしかすると、たぶん、おそらく、もしかしたら、たいてい、ことによると 【副-2】おそらくそうだと思います◆質問に対する返答として
\\	1年	
\\	【名-1】《動物》サル◆小さなサル。大きなサル=
\\	【名-2】(サルみたいな)いたずら小僧 【自動】ふざける、いじくる、遊び回る
\\	の形で 【他動】~をまねる
\\	1年	
\\	【名-1】月 【名-2】惑星の衛星 【名-3】月形のもの 【名-4】1カ月◆【同】
\\	【自動-1】夢見心地で見つめる[考える] 【自動-2】うろつく
\\	1年	
\\	【人名】モスコー 【地名】モスクワ、モスコー、第3のローマ
\\	。旧ソ連・ロシアの首都。「第3のローマ」は、ローマ、ビザンティュームに次ぎキリスト教・ローマ文化を継承した事から。
\\	1年	
\\	【名-1】ネズミ、《動物》ハツカネズミ 【名-2】臆病者 【名-3】かわいい子 【名-4】《コ》マウス◆複数形は 
\\	または 
\\	を用いる。
\\	1年	
\\	【人名】ナンシー◆ファーストネーム◆【語源】
\\	の愛称としても。◆【例】
\\	(女優) 【地名】ナンシー◆フランス 【分節】
\\	1年	
\\	【1-名】めん類 【2-名-1】頭 【2-名-2】ばか者、愚か者、あほ、間抜け、能なし、脳たりん、とんま、ぼんくら、鈍いやつ 【3-自動】とりとめもなく即興演奏する、キーボードをでたらめにたたく、楽器をもて遊ぶ、考え抜く、遊ぶ
\\	1年	
\\	【名-1】苦痛、骨折り、痛み、痛覚、疼痛、苦悩 【名-2】悩みの種、おっくうなこと、面倒(なこと)、不快な人[こと・もの] 【自動】痛む 【他動】~に苦痛[痛み]を与える、~を苦しめる
\\	1年	
\\	【名】パンダ、《動物》パンダ
\\	1年	
\\	【名-1】惑星 【名-2】《占星》運星
\\	1年	
\\	【名-1】ラケット、大騒ぎ 【名-2】大騒ぎ、ばか騒ぎ、どんちゃん騒ぎ、騒音 【名-3】不正な商売[仕事]、いかがわしい商売[仕事]、もうかる商売[仕事]、密売、恐喝、悪事
\\	1年	
\\	【形-1】現実の、真に迫った、本物の、実の、実際の、実在する、真実の、本当の、本場の、正統派の 【形-2】実数の 【副】非常に、すごく、とても、極めて、本当に、確かに、実に、まったく
\\	1年	
\\	【1-名】休息、睡眠、眠り、休養、静養 【1-自動-1】休む、休息する、休憩する、休養する、眠る 【1-自動-2】静止する 【1-自動-3】(物がある場所に)ある、置かれている 【1-自動-4】~次第である 【1-他動-1】~を休ませる、休息させる、休憩させる、休養させる 【1-他動-2】(問題など)をそのままの状態にしておく 【1-他動-3】~を置く、~に基づかせる 【2-名-1】残り、残っているもの、残りの部分、残余 【2-名-2】その他、残った人たち 【2-自動】~のままである、依然として~である、相変わらず~である
\\	1年	
\\	【名-1】標示、サイン、標識、看板、貼り紙、プラカード、表示(板) 【名-2】記号、符号、合図 【名-3】表れ、兆し、兆候、印 【名-4】《医》徴候 【名-5】星座 【名-6】痕跡 【自動】〈米〉(署名して)契約する 【他動-1】~に署名する、サインする、自署する、記名する、調印する 【他動-2】~を合図する
\\	1年	
\\	【1-名】(アイス)スケート靴 【2-名】《魚》ガンギエイ 【自動】(アイス)スケートをする
\\	1年	
\\	【名】苦痛、苦悩、痛み 【形-1】利口な、賢明な、賢い、頭が切れる、こしゃくな、さっそうとした 【形-2】気が利く 【形-3】おしゃれな、洗練された、小粋な 【副】厳しく 【自動】うずく、ズキズキ痛む
\\	1年	
\\	【名-1】《料理》スープ、汁、だし◆【仏語】
\\	【名-2】スープ◆波が崩れた後の白い波 【他動】~を興奮させる
\\	1年	
\\	【名】胃痛、腹痛
\\	1年	
\\	【形】ひどい、過酷な、恐ろしい、ものすごい、厳しい、つらい、怖い、猛烈な
\\	1年	
\\	【名-1】《動物》トラ 【名-2】乱暴者、強敵、凶暴な人、残酷な人 【名-3】歓呼、喝さい
\\	1年	
\\	【名-1】トマト、赤なす◆ナス科、一年草。 【名-2】〈俗〉娼婦、女、娘
\\	1年	
\\	【地名】トロント◆カナダ◆【略】
\\	【分節】
\\	1年	
\\	【国名】トルコ共和国◆正式名は 
\\	1年	
\\	【人名】ワイト◆ファミリーネーム◆【語源】「髪や肌が白い」の意。◆【例】
\\	(テレビタレント) 【地名-1】ホワイト川◆米国 【地名-2】ホワイト峠
\\	1年	
\\	【名-1】作家、筆者、書き手、著者、物書き 【名-2】《金融》(手形などの)裏書き人
\\	1年	
\\	【名】ヨーグルト
\\	1年	
\\	【名-1】地域、区域 【名-2】範囲、領域、分野 【名-3】面積 【名-4】地面、空き地
\\	1年	
\\	【同音】
\\	、【名-1】クマ 【名-2】弱気、弱気筋、弱気の売り方、下げ相場◆【語源】
\\	という諺から。「クマを捕まえる前に毛皮を売る」という行為と「有価証券の空売り」(持っていない有価証券の売り付けを行うこと=
\\	が似ている。◆【反】
\\	【名-3】《クマの乱暴さから》乱暴者、無作法者 【名-4】〈俗〉骨の折れる[非常に困難な]仕事[問題]、厄介なこと 【自動-1】実をつける、子を産む 【自動-2】向かう、進む 【自動-3】(~に)もたれる 【自動-4】位置する 【他動-1】重さに耐える、~を支える、~を担う 【他動-2】(重荷)を負う、(義務)を負う、(責任)を持つ、(費用)を負担する 【他動-3】(重さに耐えて)~を運ぶ、持って行く 【他動-4】~を身につける、持つ、有する 【他動-5】~を辛抱する、我慢する、~に耐える 【他動-6】(恨み・悪意・愛情など)を抱く 【他動-7】(子)を産む 【他動-8】(利益や利子を)生む、生み出す、(花実)を生じる、(努力)を実らせる
\\	1年	
\\	【名-1】牛肉 【名-2】筋肉、体力 【名-3】〈俗〉筋肉もりもりの強そうな男 【名-4】〈俗〉不平、苦情、不満◆【類】
\\	【名-5】肉用牛◆【複】
\\	または、ごくまれに 
\\	【名-6】〈俗〉刑事責任、告発 【自動】不平を言う 【他動-1】(牛)を太らせる 【他動-2】~を強化する
\\	の形をとる
\\	1年	
\\	【名】ボート、小舟、汽船、漕艇◆【類】
\\	【他動】~を船で運ぶ
\\	1年	
\\	【人名】ブラウン◆ファミリーネーム◆【語源】「髪や肌が茶色い」の意。◆【例】
\\	(ソウル音楽家)
\\	1年	
\\	【名】カフェテリア、会社や学校などの食堂、セルフサービス式の食堂
\\	1年	
\\	【国名】カナダ 【分節】
\\	1年	
\\	【形-1】注意深い、気を付ける、用心深い、油断しない、用心する、注意する 【形-2】念入りな、入念な、手厚い
\\	1年	
\\	【名-1】喝さい、応援、歓呼、歓声、声援、元気づけ、励まし 【名-2】(お祭り・お祝いの)ごちそう 【自動】喝さい[声援]を送る、喝さいする、応援する、声援する、歓声を上げる、元気づく 【他動】~に喝さい[声援]を送る、喝さいする、歓声を上げる、~を応援する、声援する、~を喜ばせる、元気づける
\\	1年	
\\	【名】おばあちゃん、祖母
\\	1年	
\\	【名】体育館、屋内体操場、体育、体操、スポーツクラブ◆【語源】
\\	の省略形
\\	1年	
\\	【名】ハンバーガー、牛肉のひき肉、ひき肉を使った食べ物、ハンバーグステーキ
\\	1年	
\\	【名-1】縁のある帽子、帽子 【名-2】職業、職、肩書き、仕事、資格
\\	1年	
\\	【名-1】頭痛◆【略】
\\	【名-2】困ったこと、頭痛の元、頭痛の種、悩みの種◆【略】
\\	1年	
\\	【名】休日、祝日、制度上の休暇、休業日、休み 【自動】休暇を過ごす
\\	1年	
\\	【人名】ジェイン◆ファーストネーム◆【語源】
\\	の女性形。同系列の 
\\	なども同様に人気がある。◆【例】
\\	(女優) 【機名】ジェーン◆第二次世界大戦で使用された陸軍97式重爆撃機の連合国コードネーム。後に 
\\	に統一された。型番:キ21。製造会社:三菱。 【雑誌名】ジェーン
\\	1年	
\\	【名-1】子ども 【名-2】子ヤギ、子ヤギ皮、キッド皮 【名-3】若者 【間投】〈米口〉(ねぇ)きみ◆親しい呼び掛けとして 【自動】からかう、冗談を言う、かつぐ
\\	1年	
\\	【名-1】《動物》ライオン 【名-2】勇敢な人 【名-3】人気者、名物男
\\	1年	
\\	【名-1】市場、マーケット、食料品店◆【略】
\\	【名-2】市場(規模)、マーケット 【名-3】市況、相場 【名-4】業界、商圏 【自動】売買を行う 【他動】~を市場で売買する、市場に出す、売り込む
\\	1年	
\\	【人名】メアリー◆ファーストネーム◆【語源】キリストの母マリアから。
\\	や 
\\	も同様。オーストラリアでは 
\\	とも。◆【例】
\\	(女優) 【機名】メアリー◆第二次世界大戦で使用された陸軍98式軽爆撃機の連合国コードネーム。型番:キ32。製造会社:川崎。 【地名】マルイ 【分節】
\\	1年	
\\	【名-1】一員、会員、構成員、顔ぶれ、議員、クラブ員、党員、成員 【名-2】一部分、体の一部 【名-3】《建築》構材、部材
\\	1年	
\\	【地名】ニューヨーク州◆米国◆【略】
\\	1年	
\\	午後
\\	1年	
\\	【名】寛容、許し、恩赦、大赦 【自他動】免除する、放免する 【他動-1】~を許す[赦免する] 【他動-2】失礼する
\\	1年	
\\	【名】ピザ、ピッツァ
\\	1年	
\\	【副】速く、敏速に、手早く、急いで、素早く、直ちに、瞬時に、緊急に、機敏に、俊敏に、瞬く間に
\\	1年	
\\	【名-1】静けさ、静寂、静穏 【名-2】閑静 【名-3】平穏 【形-1】静かな、静粛な 【形-2】無口な、寡黙な 【形-3】物静かな 【形-4】ひっそりとした 【形-5】目立たない、地味な 【形-6】内密の、内に秘めた 【形-7】平穏な、穏やかな、平和な 【他動-1】~を静かにさせる、静める 【他動-2】(気分など)を和らげる、なだめる
\\	1年	
\\	【名】クイズ、小テスト、クイズ番組、簡単な試験 【他動】~に質問する、質問攻めに会わせる
\\	1年	
\\	【名-1】円形、輪 【名-2】一巡 【名-3】周囲 【名-4】一試合 【名-5】(弾薬の)一発分 【名-6】全員に行き渡る量 【前】~を取り巻いて、回って、回転して 【形-1】丸い、円形の 【形-2】一回りする 【形-3】大体の 【形-4】円熟した 【形-5】活発な 【形-6】多数の 【副】~の間中 【自動】丸くなる 【他動-1】~を丸くする、丸める 【他動-2】~を切り上げる、四捨五入する◆【参考】
\\	1年	
\\	【名-1】規則、ルール、規定、法則、規範 【名-2】裁定、命令 【名-3】標準、基準 【名-4】習慣、常習 【名-5】支配、統治(権) 【名-6】《数学》公式 【名-7】定規、物差し 【名-8】罫線 【他動-1】~を統治する、支配する 【他動-2】裁定する、判決する 【他動-3】線を引く 【他動-4】抑制する 【他動-5】指導する、指図する、規定する
\\	1年	
\\	【国名】= 【分節】
\\	1年	
\\	【副-1】心から、誠実に、真心を込めて、真摯に 【副-2】《レター・結び》敬具
\\	1年	
\\	社会科
\\	1年	
\\	【名】《スポーツ》ソフトボール(用の球) 【形】〈米俗〉つまらない 
\\	ソーフトゥボール、ソフトボール、【変化】《複》
\\	、【分節】
\\	1年	
\\	【自動】(金を)使う、費やす、浪費する 【他動-1】(金)を使う、費やす、浪費する 【他動-2】~を過ごす、~して過ごす
\\	1年	
\\	【名】13分の1、第13(番目)、13日、13番目のもの[人] 【形】第13の、13番目の、13分の1の
\\	1年	
\\	【1-名】トースト 【1-他動】(パンなど)を焼く、あぶる 【2-名】祝杯、乾杯、乾杯のあいさつ 【2-自動】祝いの乾杯をする 【2-他動】~に乾杯をする、~の健康を祝して乾杯する
\\	1年	
\\	【名】傘、天蓋 【形-1】傘状の 【形-2】包括的な
\\	1年	
\\	【名-1】制服、軍服、上衣 【名-2】
\\	◆無線用アルファベット 【形】一定の、同じ形の、均一の、同一の、揃った、一様な、同様の、同型の、揃いの、一貫した、一律の 【他動-1】~を均一[同一・同型・一様・同様]にする、そろえる、等しくする 【他動-2】~にユニフォーム[制服]を着せる
\\	1年	
\\	【名】(総合)大学、大学生 【形】大学の
\\	1年	
\\	【名-1】見ること、一見、一覧 【名-2】視界、視野、視力 【名-3】光景、景色、眺望 【名-4】意見、見識、考え、物の見方、見解、見通し 【名-5】概論 【名-6】目的、信念 【名-7】図 【他動-1】~を見る、眺める、視察する 【他動-2】~を(…と)みなす、考える◆【用法】
\\	【他動-3】~を思い巡らす
\\	1年	
\\	【副】午前に
\\	の略
\\	1年	
\\	【国名】アメリカ
\\	1年	
\\	【名-1】玉、ボール、ボールピース、玉石、球 【名-2】ダンスパーティー、(大)舞踏会 【名-3】投げること、投入 【名-4】丸剤 【名-5】〈俗〉根性 【名-6】睾丸
\\	1年	
\\	【名】書店、書籍商、本屋
\\	1年	
\\	【形-1】忙しい、多忙な、せわしない、落ち着きがない、余念がない、~にかまけて(忙しい)、~で手一杯で、手がふさがって、手が放せない、取り込み中で 【形-2】使用中である、(電話が)話し中で 【形-3】混んでいる、通行量が多い 【自他動】忙しくする、せっせと仕事をする
\\	1年	
\\	【名】カメラ、ビデオカメラ、テレビカメラ◆【略】
\\	1年	
\\	【名-1】カード、メッセージカード、はがき 【名-2】トランプの札、トランプのカード 【名-3】キャシュカード、クレジットカード 【形】ボール紙製の 【他動-1】~にカードをつける、~をカードに記入する、~をスコアカードに記入する 【他動-2】毛羽(けば)を立てる、~を梳く
\\	1年	
\\	【名-1】いす 【名-2】講座 【名-3】議長 【名-4】大学教授の職◆通例 
\\	【他動】~の議長を務める
\\	1年	
\\	【名】教室
\\	1年	
\\	【名-1】《証券》引け(値) 【名-2】終わり、終結、結びの句 【形-1】近い、接近した、近接した、緊密な、親しい 【形-2】(空気が)こもっている、ムッとする、人いきれがする 【形-3】ぎりぎりの 【形-4】(織物の)目が詰んでいる 【形-5】周到な 【形-6】原物に非常に近い 【形-7】無口な 【形-8】けちな 【形-9】監禁された 【副-1】近くに、接して、密接して、ぴったりと、すき間なく 【副-2】短く 【自動-1】閉じる、閉まる、締まる、完結する、完了する、終わる 【自動-2】迫る 【他動-1】~を閉じる、~を閉める、~を閉鎖する、~を終える、(契約)をまとめる 【他動-2】~を閉鎖する、~を終える 【他動-3】(契約)をまとめる
\\	1年	
\\	【名-1】雲、《気》雲 【名-2】大群、集団 【自動】曇る 【他動-1】~を雲らせる 【他動-2】~に暗雲を投げかける
\\	1年	
\\	【名】漫画、漫画本、喜劇俳優 【形】喜劇の、こっけいな、おかしい
\\	1年	
\\	【名-1】文化、教養 【名-2】培養(液)、培養物 【他動】~を培養する
\\	1年	
\\	【名-1】コップ、茶碗 【名-2】(料理計量用)カップ1杯◆【略】
\\	◆約250
\\	=1/2パイント 【名-3】酒 【名-4】優勝杯 【名-5】ブラジャーのカップ 【名-6】経験 【名-7】運命 【名-8】《ゴルフ》カップ、ホール 【名-9】《植物》(花の)萼 【他動】~をカップの形にする、へこませる
\\	1年	
\\	【名-1】日付、年月日 【名-2】期日、日取り 【名-3】年代、時代 【名-4】人の生没年、物事の始まりと終わりの年 【名-5】デート 【名-6】〈米口〉デートの相手、付き合っている異性 【名-7】ナツメヤシ(の実)◆【学名】
\\	【自動-1】~から始まる 【自動-2】時代物である、古風になる、時代遅れになる 【自動-3】〈米口〉デートする 【他動-1】~に日付を付ける 【他動-2】年代を定める、~の年齢を示す 【他動-3】~を古く思わせる、時代遅れにする、古臭くする 【他動-4】~とデートする、付き合う
\\	1年	
\\	デパート、百貨店
\\	1年	
\\	【名-1】原動力、駆動(部) 【名-2】意欲、活力、気力、やる気、欲動 【名-3】車道、大通り◆【略】
\\	【名-4】《ゴルフ》ドライバー・ショット、《テニス》ドライブ(ショット)、《バスケ》ドライブ 【名-5】傾向、流れ、動き 【自動】走行する、疾走する、突進する、狙い打ちする 【他動-1】~を運転する 【他動-2】~を駆動する 【他動-3】~を駆り立てる、追いたてる、追いやる 【他動-4】~をこき使う 【他動-5】~を推進する、促進する、後押しする 【他動-6】~を強打する
\\	1年	
\\	【人名】エマ◆ファーストネーム◆【愛称】
\\	(エミィ)◆【語源】王室の名。ゲルマン語で「宇宙」。◆【例】
\\	(女優) 【映画-1】エマ◆米英1972《監督》ジョン・グレニスター《出演》ドラン・ゴッドウィン、ジョン・カーソン
\\	として製作された。 【映画-2】エマ◆米英1996《監督》ダグラス・マクグラス《出演》グウィネス・パルトロウ、ジェレミー・ノーザム、トニー・コレット、グレタ・スカッキ、ジュリエット・スティーヴンソン、アラン・カミング、ポリー・ウォーカー、ユアン・マクレガー 【著作】エマ◆英1815《著》ジェーン・オースティン
\\	【分節】
\\	1年	
\\	【名-1】消しゴム、インク消し、黒板ふき、字消し、石板ふき 【名-2】消す人 【名-3】〈俗〉ノックアウト、ノックアウトパンチ
\\	1年	
\\	【形-1】外国の、対外関係の 【形-2】異質の、無関係の、なじまない、見知らぬ
\\	1年	
\\	【名】果物、果実、フルーツ 【自動】実を結ぶ 【他動】~に実を結ばせる
\\	1年	
\\	【名】別れのあいさつ 【間投】さようなら
\\	1年	
\\	【名-1】緑色、緑 【名-2】《ゴルフ》グリーン 【形-1】緑色の、緑(色)の、青々した 【形-2】(果実が)青い、(果実が)未熟の 【形-3】(人が)未熟の、経験の浅い、世間知らずの 【形-4】青ざめた 【形-5】生き生きした 【形-6】環境に優しい、環境問題意識が高い、地球に優しい 【自動】緑色になる 【他動】~を緑色にする
\\	1年	
\\	【名】推測、推量 【自動】~と推測する 【他動-1】~が…であると推測する、想像する、~を推量する 【他動-2】うまく言い当てる、言い当てる、解きあてる 【他動-3】〈米俗〉~だと思う
\\	1年	
\\	【名-1】髪、毛、髪の毛、毛状のもの、毛髪 【名-2】わずか(の差)
\\	1年	
\\	【名】傷 【形】けがをした 【自動】痛む 【他動-1】~を傷つける 【他動-2】~の感情を損なう、~の感情を害する 【他動-3】~に苦痛を与える 【他動-4】~に損害を与える 【他動-5】~を困らせる
\\	1年	
\\	【名-1】ジュース、汁 【名-2】絞り汁、~を絞ったもの 【名-3】分泌物、分泌液 【名-4】動力源、電気、電力、ガソリン 【名-5】元気、活気、精力、活力 【他動】~の汁をしぼる
\\	1年	
\\	【名-1】湖、湖水 【名-2】《化》レーキ◆【参考】
\\	(媒染剤)◆有機の水溶性色素を無機化合物と化合させ、水に難溶にさせる物。無機化合物の多くは金属水酸化物、ただし、ヘテロポリ酸塩もある。
\\	1年	
\\	【名-1】脚、足◆人、動物、家具の 【名-2】もも肉◆家畜の 【名-3】《服》脚部◆ズボンの足の部分。ストッキングの踵(かかと)を除いた部分。 【名-4】区間、行程、一区間、一行程 【自動】急いで歩く、速く歩く
\\	1年	
\\	【名-1】図書館、書斎 【名-2】標準プログラムの収集、文庫
\\	1年	
\\	【人名】ロンドン 【地名-1】ロンドン◆英国の首都◆【略】
\\	【地名-2】ロンドン◆カナダ・オンタリオ州◆【略】
\\	1年	
\\	【名-1】事柄、状況、事態 【名-2】問題 【名-3】物質、物体、内容 【名-4】困難 【自動-1】重要である、重要になる 【自動-2】問題である、問題となる
\\	1年	
\\	【名-1】牛乳、乳、ミルク◆【仏語】
\\	【名-2】乳汁 【名-3】乳液、樹液 【自動】乳を出す 【他動-1】(牛)乳を搾る 【他動-2】~を絞り取る、~から甘い汁を吸う、~からうまい汁を吸う、食いものにする、搾り上げる、搾取する 【他動-3】~を引き出す
\\	1年	
\\	【名】~先生、~君(男性の姓名・職業名に冠する敬称)、~氏、~様、~さん◆まれにファーストネームの前につけられることがある。手紙などで
\\	と書かれていた場合、その人が男性であることを伝えている。◆【参考】
\\	【他動】(人)をミスター~と呼ぶ
\\	1年	
\\	【名】~さん◆未婚・既婚にかかわらず、女性に対する敬称
\\	1年	
\\	【名-1】正午、真昼 【名-2】最高点、全盛期
\\	1年	
\\	【名-1】ノート、手帳、筆記帳、雑記帳 【名-2】ノート(型)パソコン
\\	1年	
\\	【名】許可 【形-1】体調が良い 【形-2】満足のいく 【他動】~を了承する
\\	1年	
\\	【名-1】《植物》オレンジ、柑橘◆【仏語】
\\	【名-2】オレンジ色
\\	1年	
\\	【形-1】自分自身の、自分のもの、自分の、独自の、血を分けた 【形-2】その物の、特有の 【他動】所有する、認める
\\	1年	
\\	【名-1】パーティー 【名-2】政党、党 【名-3】仲間、メンバー 【名-4】関係者、団体、一行、同行者、グループ 【名-5】相手 【名-6】(契約の)当事者 【形】党内の 【自動-1】パーティーを開く、パーティーに参加する、パーティーを楽しむ、パーティーへ行く、パーティーをする 【自動-2】どんちゃん騒ぎをする、ばか騒ぎをする、酒を飲む、マリファナを吸う
\\	1年	
\\	【名-1】飛行機、航空機 【名-2】平面、面 【名-3】水準 【名-4】スズカケノキ、プラタナス 【形】平らな 【自動-1】飛行機で行く、滑空する 【自動-2】滑走する 【他動】~をかんなで滑らかにする、~にかんなをかける
\\	1年	
\\	郵便局◆【略】
\\	【形】
\\	1年	
\\	【名-1】雨 【名-2】雨季 【自動-1】雨が降る 【自動-2】雨のように落ちる 【自動-3】盛んに浴びせる 【他動-1】雨が降る、~を(雨のように)降らす 【他動-2】~を惜しみなく与える、ふんだんに提供する、雨あられと提供する、注ぐ
\\	1年	
\\	【名-1】構えの姿勢 【名-2】現金 【形-1】用意[準備・支度]ができている、支度が整った、覚悟ができた、用意ができて 【形-2】~しがちな 【形-3】即時の、即座の、素早い、敏速な 【形-4】今にも~しようとする 【形-5】喜んでする、やぶさかでない 【副】あらかじめ、前もって 【他動】~を用意させる、準備する
\\	1年	
\\	【名-1】赤、赤色 【名-2】赤字 【形-1】赤色の、赤くなった、赤の 【形-2】過激な
\\	1年	
\\	【名-1】帰ること、返すこと、元に戻すこと、返却、返品、返納、返上 【名-2】元に戻ること、返ってくること、帰郷、帰国、帰宅、帰還 【名-3】申告、報告 【名-4】収益、報酬、利益 【名-5】(キーボードの)復帰◆【略】
\\	【名-6】開票結果、開票報告 【形】〈英〉往復の 【自動-1】戻る、帰る 【自動-2】(電源が)復帰する 【他動-1】~を返す、戻す、返却する、返品する、返送する、返上する 【他動-2】~に答える、答申する 【他動-3】公表する 【他動-4】選出する 【他動-5】反響する
\\	1年	
\\	【名-1】座席、座、見物席、腰掛け 【名-2】台座、座部 【名-3】(ズボンなどの)しりの部分 【名-4】位置、場所、所在地、本拠地、中枢、中心地 【名-5】巣、病巣、源 【名-6】大庭園[広大な土地]のある郊外の邸宅 【他動】~を着席させる、座らせる(ことができる)、席に案内する
\\	1年	
\\	【名-1】靴 【名-2】蹄鉄 【名-3】状況、立場、苦境 【他動】~に靴を履かせる
\\	1年	
\\	【名-1】短いもの 【名-2】不足 【名-3】短いこと、簡潔 【名-4】《証券》空売り(する人) 【名-5】《電》短絡、ショート、漏電 【形-1】長さが短い、背が低い 【形-2】小柄な 【形-3】時間が短い 【形-4】手短な、簡潔な 【形-5】無愛想な、素っ気ない、素っ気なくあしらう 【形-6】不足の、乏しい、(釣り銭が)足りない、不足して 【形-7】《証券》現物不足の、空売りの 【副-1】短く、短めに、簡潔に 【副-2】急に 【副-3】手前に 【自動】《電》短絡する 【他動-1】~に少ししか与えない 【他動-2】~に釣り銭を少なく渡す 【他動-3】《証券》~を空売りする 【他動-4】《電》~を短絡させる
\\	1年	
\\	【名-1】聖堂、神社 【名-2】聖地 【名-3】聖体容器
\\	1年	
\\	【名-1】スキー、スキー板 【名-2】水上スキー 【自動】スキーで滑走する、スキーをする 【他動】~をスキーで行く
\\	1年	
\\	【名-1】星 【名-2】星形のもの、星印 【名-3】人気者、スター、有名人 【名-4】運命 【形-1】優れた、スターの、第1級の 【形-2】星の 【自動-1】星のように輝く 【自動-2】際立つ 【自動-3】主演する、主役を演じる 【自動-4】見事に演じる、抜群にできる 【他動-1】~に星印を付ける 【他動-2】~をスターにする、呼び物にする
\\	1年	
\\	【形-1】まっすぐな、一直線の 【形-2】連続した 【形-3】きちんとした、生粋の、誠実な、包みかくしのない、率直な、信頼できる 【形-4】決済済みの、貸借のない 【形-5】〈米〉(酒が)ストレートの、(水やソーダで)割ってない 【副-1】一直線に、まっすぐに 【副-2】正しく 【副-3】垂直に 【副-4】連続して、連続的に、続けて、立て続けに、続けざまに、休まずに 【副-5】氷を入れないで
\\	1年	
\\	【代名】彼・彼女らのもの、彼らのもの
\\	1年	
\\	【名】30分の1、第30(番目)、30日、30番目のもの[人] 【形】第30の、330番目の、30分の1の
\\	1年	
\\	【名】〈米俗〉白人にこびへつらう黒人◆【語源】
\\	から 【他動】(黒人が白人に)こびる
\\	とも表記。 【人名】トム
\\	の略称
\\	1年	
\\	【名】
\\	シャツ
\\	1年	
\\	【名】20分の1、第20(番目)、20日、20番目のもの[人] 【形】第20の、20番目の、20分の1の
\\	1年	
\\	【名-1】訪問者、観光客、見舞人、滞在客 【名-2】《スポーツ》ビジター、遠征軍
\\	1年	
\\	【名】バレーボール、バレーボール用のボール
\\	1年	
\\	【名】ボランティア、志願者、志願兵、奉仕活動家、有志 【形】ボランティアの、篤志の、社会事業に熱意のある 【自動】自ら進んでする、志願する、自発的にやる、進んで事に当たる、ボランティアをする 【他動】~を進んでやる、自発的に申し出る、買って出る、自発的に与える[提供する]、進んで~を与える[提供する]、自発的に~を言う
\\	1年	
\\	【名】週末 【形】週末の
\\	1年	
\\	【名-1】窓、窓ガラス、窓口、飾り窓 【名-2】時間枠、時間帯、時期 【名-3】《コ》ウィンドウ、画面、表示 【名-4】知る機会、手段
\\	1年	
\\	【名】悪事、不正、不法行為 【形-1】間違った、誤っている 【形-2】不適切な 【形-3】悪い、邪な、裏の 【形-4】故障した 【副】悪く、不正に、誤って、逆に
\\	1年	
\\	【形】とび切り上等の、第1級の◆【同】
\\	【不-1】(同種類のものの)一つの、1個の、1台の、一人の 【不-2】ある~ 【不-3】~の一片、1回分、一種、一人前、一例 【不-4】~につき、~ごとに、毎~ 【不-5】同じ 【不-6】~のような人、~の作品、~の製品◆【用法】
\\	固有名詞
\\	1年	
\\	【前-1】~について、~に関して 【前-2】~に従事して 【前-3】~頃に 【前-4】〈俗〉~に賛成で、~を支持して 【副-1】約、およそ、大体、ほとんど 【副-2】周囲に 【副-3】動き回って、流行して
\\	1年	
\\	【接】~した後で、~してから 【前-1】~の後で、~の後ろに、~の後に、~の終了後に、~を経て 【前-2】~したにもかかわらず、~したので 【前-3】~の後を追って、~を狙って、~を求めて、~に下心を持って 【前-4】~にちなんで、~の名をとって、~に倣って 【前-5】~の後にまた~が続いて、次々と◆【参考】
\\	【形】後の、次の、後続の 【副】後ろで、後で
\\	1年	
\\	【名】午後
\\	1年	
\\	【副-1】再び、かさねて、この場合もやはり、この場合も先と同様に 【副-2】さらに、また一方
\\	1年	
\\	【副】~前に
\\	1年	
\\	【代名】すべての人[もの・こと] 【形】あらゆる、すべての、最大の、全体の 【副】すっかり、まったく
\\	1年	
\\	【副-1】ほとんど、九分どおり、大体、一種の、匹敵する、いまいち 【副-2】すんでのところで[思わず]~しそうになって、危うく~するところで、~しそうになる、~と言ってもいい
\\	1年	
\\	【副】~もまた、同様に、また、~もやはり(やっぱり)
\\	1年	
\\	【動】
\\	の1人称単数現在形
\\	1年	
\\	【不】一つの◆母音の前で用いる場合。子音の前なら 
\\	を用いる。
\\	1年	
\\	【接-1】~と…、~や…、~も…も、~および…、かつ 【接-2】そして、それから 【接-3】すると、そうすると、また 【接-4】_勝_敗
\\	1年	
\\	【名-1】動物、獣類 【名-2】動物のように行動する人
\\	1年	
\\	【代名-1】《肯定文》どれでも、誰でも 【代名-2】《疑問文》誰か、いくらか 【代名-3】《否定文》どれも、少しも、誰も 【形-1】いくらかの 【形-2】任意の 【形-3】誰も、どれでも、どれも、どんな人[物・事]でも
\\	1年	
\\	【名】4月◆【略】
\\	1年	
\\	【名】《面積単位》アール
\\	02471エーカ=100平方メートル=10メートル平方=30.25坪 【動】
\\	の2人称単数・複数現在形、1人称・3人称複数現在形
\\	1年	
\\	【前-1】~の周りに、~の周囲に 【前-2】約、およそ、~前後の◆【同】
\\	【副-1】囲むように、ぐるりと 【副-2】あちこちに、方々に、動き回って 【副-3】近くに、そばに、この辺に 【副-4】出席(出勤)している、来ている 【副-5】《あちこち動き回っているイメージから》活躍して、活動して、機能して、存在して、現存している、現役で 【副-6】《あちこちを回ってきたというイメージから》経験豊富な、海千山千の、見聞が広い、手慣れた 【副-7】避けて
\\	1年	
\\	【接-1】~のように、~のとおりで[に] 【接-2】~のままで[に] 【接-3】~につれて 【接-4】~している時[最中]の[に]、~しながら、~したままで、~になって 【接-5】~だけれども 【前-1】~のように 【前-2】~の時に 【前-3】~として、~だということで、~であると(して)
\\	などの動詞の目的補語を導く。補語は名詞、形容詞、分詞。
\\	1年	
\\	【前-1】(場所・空間・時間・変動値などの一点を示して)~に、~において、~で、~のときに、~時に、~ページに 【前-2】~を狙って
\\	1年	
\\	【名】8月
\\	1年	
\\	【国名】オーストラリア、濠州、豪州◆国名は、オーストラリア連邦
\\	。◆首都はキャンベラ。面積は日本の20倍。人口約1800万(英国国教徒、カソリック教徒、そのほかのキリスト教徒が1/4ずつ)。濠州は昭和前期以前の漢字表記。◆【参考】
\\	オストレ
\\	イリア、【分節】
\\	1年	
\\	【名-1】背中、後、背、背部、背後部、裏、裏手、奥 【名-2】(本の)綴じてある側、(飛行機の)上面部 【名-3】(ロープで引っ張る形式の)渡し船 【名-4】心の奥底、事の真相 【名-5】(いすの)背もたれ 【名-6】(弦楽器のボディの)裏板◆【反】
\\	【形】後方にある、背後の、裏の、未開の、奥の、逆の、滞納の、遅れた 【副-1】後方へ、後ろに 【副-2】元の位置へ、戻って、(連れ)戻して、帰っている、また流行して 【自動】後ずさりする、後退する、逆流する、逆行する 【他動-1】(車)をバックさせる 【他動-2】~を支援する、応援する、~を支持する、後押しする、バックアップする、推薦する、金銭的援助をする 【他動-3】~に伴奏をつける 【他動-4】(小切手など)に裏書きする
\\	1年	
\\	【形-1】悪い、ひどい 【形-2】苦手で、不得手で、下手な 【形-3】体の具合が悪い、気分がよくない 【形-4】不良な、偽の 【形-5】〈俗〉とても良い、素晴らしい、かっこいい、最高にいかす、超すごい、一流の◆【同】
\\	【形-6】貸し倒れの
\\	1年	
\\	【名-1】袋、紙袋、バッグ、かばん 【名-2】十八番 【名-3】《野球》ベース 【他動】~を袋に入れる、袋に詰める
\\	1年	
\\	【名】野球、ベースボール
\\	1年	
\\	【名-1】バスケットボール 【名-2】バスケットボールで用いられるボール
\\	1年	
\\	【名-1】槽、浴槽、浴場、風呂、湯船 【名-2】入浴、水あび、沐浴 【名-3】溶液
\\	1年	
\\	【自動-1】~として存在する、(人が)いる、(物が)ある 【自動-2】(~の状態)である 【自動-3】(物事が)起きる 【自動-4】《進行》~しているところだ
\\	の形をとる 【自動-5】《不満》いつも~する
\\	の形をとる 【自動-6】《近未来》(近々)~するつもりだ
\\	の形をとる。
\\	は通例、往来・発着を表す動詞 【自動-7】《習慣的行動》(ここのところ)~している 【自動-8】《受動》~される、~されている
\\	過去分詞の形をとる 【自動-9】《予定》~するつもりだ、~する予定だ
\\	の形をとる 【自動-10】《義務》~しなければならない
\\	の形をとる 【自動-11】《運命》~する運命である
\\	の形をとる 【自動-12】《可能》~できる◆通例
\\	過去分詞》の形をとり否定文 【自動-13】《願望》~したい
\\	の形をとる 【自動-14】《仮定法現在》仮に~であるとして 【自動-15】《提案・要求・希望・決定・必要・妥当などを表す主節に続く 
\\	節内で》~されるべきである 【自動-16】《完了》~した
\\	自動詞の過去分詞
\\	など)の形をとる
\\	1年	
\\	【形-1】美しい 【形-2】素晴らしい、見事な、すてきな
\\	1年	
\\	【名-1】ベッド、寝床 【名-2】(トラックなどの)荷台 【名-3】基盤、土台、定盤 【名-4】河床、湖底 【名-5】苗床、花壇、(カキなどの)養殖所 【自動】寝る 【他動-1】~を寝かす、泊める 【他動-2】~を敷き詰める 【他動-3】(人)とベッドを共にする、(人)と寝る、(人)と性的関係を持つ
\\	1年	
\\	【接-1】~する前に、~より前に、~より先に、~した末にやっと 【接-2】~するよりはむしろ 【前】~の前に、~に先立って、~を前にして、面前で、~の出現以前 【副-1】《時間的に》以前に、前に、早く、先に
\\	も 
\\	も「今までに」の意味で使われるが、少しニュアンスが異なる。
\\	は、今目の前で起こっていることに関して言及するときに使い、
\\	は、今目の前で起こっていることとは関係ないことに言及するときに使う。 【副-2】《位置的に》前に、前方に、(人)の面前で、(人)に直接会って
\\	1年	
\\	【名】最も良いもの[こと]、最善、最大、最上 【形】最高の、最も良い、最善の、最大の、この上ない、最上の、誰にも引けを取らない 【副-1】《口》~するのが一番よい、~すべきだ
\\	の強意形 【副-2】最も良く、最高に、一番うまく、一番親しい 【他動】~を凌ぐ、勝る◆【同】
\\	1年	
\\	【形】より良い、気分が前より良い[優れている]、より優れている、優越する、一層多い、上の
\\	の比較級 【副】より良く、もっとうまく[上手に]
\\	の比較級 【自動】よくなる、向上する 【他動-1】~をよくする、~を改善する、~を向上させる 【他動-2】~に勝る、~を凌ぐ
\\	1年	
\\	【前-1】(ふたつの)間で[に] 【前-2】~の合間に、~する傍ら 【前-3】~の間を行き来して、~相互間で[に] 【副】中間に、両者間に、間を隔てて、どっちつかずで
\\	1年	
\\	【形-1】大きい 【形-2】重要な、大事な 【形-3】人気のある 【形-4】偉そうな 【副-1】大きく 【副-2】偉そうに
\\	1年	
\\	【名-1】自転車◆【語源】
\\	の省略形 【名-2】オートバイ、バイク 【自動】自転車に乗る
\\	1年	
\\	【名-1】鳥(類) 【名-2】小禽(類)
\\	1年	
\\	【名】誕生日
\\	1年	
\\	【名-1】黒、黒色 【名-2】黒人 【名-3】黒字 【形-1】黒い 【形-2】黒人の 【形-3】(コーヒーを)ブラックで 【形-4】真っ暗の、暗黒の 【形-5】悪い、不正な、邪な、やみ値の 【形-6】(機嫌が)険悪な 【形-7】不吉な 【自他動】黒くなる、黒くする
\\	1年	
\\	【名-1】青、青色 【名-2】〈英〉正選手 【名-3】〈英〉青章◆英国、オックスフォード
\\	大学、ケンブリッジ
\\	大学の対抗競技に代表選手として出場し活躍した人に与えられる 【形-1】青い 【形-2】青ざめた 【形-3】憂うつな、陰気な 【形-4】堅苦しい 【形-5】見込みのない 【自動】青くなる 【他動】~を青くする
\\	1年	
\\	【名-1】本、書籍、書物、図書、単行本、作品 【名-2】帳簿、会計簿 【自動-1】予約する 【自動-2】〈俗〉とても速く走る[走り去る]、早く行く 【他動-1】~に記入する、(帳簿)に記載する、(予算など)を計上する 【他動-2】~を予約する
\\	1年	
\\	【名-1】ボックス、箱、貯金箱 【名-2】囲み記事 【名-3】(劇場の)ボックス席◆【同】
\\	【名-4】《野球》ボックス◆バッター、コーチ、キャッチャーが入る場所 【名-5】〈英俗〉テレビ 【名-6】〈米俗〉コンピュータ 【名-7】保護預かり箱◆金融会社が顧客の保護預り有価証券を収納するケース 【名-8】苦境、窮地◆【同】
\\	【名-9】《植物》ツゲ◆【同】
\\	【名-10】詰め所 【他動-1】箱に入れる[詰める・詰め込む] 【他動-2】~を取り囲む 【他動-3】~を殴る、~をこぶしで殴る 【他動-4】~とボクシングをする、ボクシングの試合に出る
\\	1年	
\\	【名-1】少年、男の子、小僧 【名-2】息子 【間投】わあ、いやはや、おや、へえー、まったく、やあ、しめしめ◆歓喜・驚き・落胆などを表す。
\\	1年	
\\	【名】朝食 【自動】朝食を食べる[とる]
\\	1年	
\\	【名-1】兄弟、兄、弟 【名-2】仲間、同僚 【名-3】同胞、同国人 【名-4】同宗者、同一宗教の人、信者仲間 【間投】《いら立ち・驚き・嫌悪・落胆・失望などを表して》おいおい、おやまあ、へえ、なんだって、いまいましい、やだなあ、くそっ
\\	1年	
\\	【名-1】バス 【名-2】《コ》バス(型)、母線
\\	トポロジーの一つ。他にスター型
\\	、リング型
\\	がある。 【自動-1】バスで行く 【自動-2】〈米口〉ウェイター[ウェイトレス]の助手として働く◆【参考】
\\	【他動】~をバスで輸送する
\\	1年	
\\	【名】「しかし」という言葉、異議 【接-1】しかし、ただし◆逆接 【接-2】そうではなくて、~ではなくて 【前】~以外の、~を除いて、~をおいて 【副】ほんの
\\	1年	
\\	【前-1】~によって 【前-2】~の近く[付近・そば・辺り・わき・手元]に
\\	よりも近い 【前-3】~の近く[付近・そば・辺り]を通って[通過して・通り越して]、~を経路として、~を経由して、~経由で 【前-4】(関係性が非常に近い)~の力を借りて、~を用いて、~を使って、~によって、~から 【前-5】~までに◆終了時刻・期限を表す 【前-6】~に従って、~に準拠して、~に基づいて 【前-7】~の差で、~の差をつけて 【前-8】~を単位として、~個ずつ、~人ずつ 【前-9】~別に 【前-10】~の点については、~に関しては 【副-1】そばに、ほとりに、傍らに 【副-2】そばを通り過ぎて、通り過ぎて 【副-3】立ち寄って
\\	1年	
\\	"【名】(トーナメントで)不戦勝 【形】= 
\\	【間投】さようなら、
\\	の略 
\\	!""別れるときに、彼はただ「バイ!」とだけ言った。
\\	1年	
\\	"【名-1】電話 【名-2】訪問 【名-3】呼び出し、召集、就任依頼[要請] 【名-4】使命、天職 【名-5】審判の判定、決定 【名-6】必要、要求 【名-7】履行請求◆特に 
\\	で 【名-8】《金融》コールオプション
\\	◆【参考】
\\	◆指定期間内の指定価格での「買付け」(権利)。反対は 
\\	。 【自動-1】電話をかける 【自動-2】訪れる 【他動-1】~に電話する 【他動-2】~を呼ぶ、~を呼び起こす、~を呼び出す、~に連絡する、~を召集する 【他動-3】~を喚起する 【他動-4】~を招く 【他動-5】~を…と考える[思う・みなす]
\\	1年	
\\	【名】桟 
\\	ケイム
\\	1年	
\\	【1-助動-1】《可能》~できる、~しようと思えばできる、~する能力がある 【1-助動-2】《推量》~する可能性がある、~することがありうる、~する[になる]ことがある、~しない[にならない]とも限らない 【1-助動-3】《否定推量》~のはずがない
\\	(~に違いない)と対比される 【1-助動-4】《許可》~して(も)よい、~しても差し支えない
\\	より口語的な響きを伴う 【1-助動-5】《依頼》~してくれないか◆【用法】
\\	の形をとり、口語的な響きを伴う 【1-助動-6】《提案・命令》~してはどうか◆【注意】時に皮肉めいた響きを伴う◆【用法】通例、二人称を主語として 【1-助動-7】《小言》~してもいいではないか◆小言 【2-名-1】缶、缶詰 【2-名-2】バケツ 【2-名-3】ジョッキ 【2-他動-1】~を缶詰めにする 【2-他動-2】〈俗〉~を首にする、~を解雇する、~を終わりにする、~をキャンセルする 【2-他動-3】〈俗〉テープやフィルムに録音[録画・記録]する 【2-他動-4】《ゴルフ》ホールに入れる
\\	1年	
\\	キャノット、キャンノット、【分節】
\\	の否定形◆口語では短縮形の 
\\	が用いられることが多い。
\\	1年	
\\	【名-1】縁なし帽子、野球帽 【名-2】キノコのかさ 【名-3】蓋、キャップ 【名-4】雷管 【名-5】上限、最高限度、天井 【名-6】口金 【他動-1】~に帽子をかぶせる、~に蓋をかぶせる、~の頂上を覆う、~を冠する 【他動-2】~の上限を定める、~を制限する 【他動-3】~を代表に選ぶ 【他動-4】~を凌駕する、~を凌ぐ 【他動-5】《野球》駄目押し点を入れる 【他動-6】~を仕上げる、総仕上げをする、最後を締めくくる
\\	1年	
\\	【名】自動車、車、乗用車
\\	1年	
\\	【名-1】心配、苦労 【名-2】世話、介護、手入れ 【名-3】配慮、注意 【自動-1】大事[重要]だと思う、(重要なことなので~に)関心がある、(~を)気にかける、心配する、気遣う 【自動-2】(~かどうか)気にする、構う◆【用法】通例、否定文・疑問文・条件文 【自動-3】世話をする、看護する、面倒を見る 【他動】~を気にする、かまう◆【用法】通例、否定文・疑問文で
\\	1年	
\\	【名-1】ネコ、ネコ科の動物 【名-2】ナマズ 【名-3】キャタピラー 【自動】こっそりと動く、人目を盗むように動く
\\	1年	
\\	【名】400、四百◆ローマ数字。=-100
\\	。例えば、147=100
\\	(4と9だけは、-1
\\	5および 
\\	10で表す)。◆【参考】
\\	1年	
\\	【名-1】変化、変更、移行、交換◆【略】
\\	【名-2】変更点、修正点 【名-3】おつり、つり銭 【名-4】細かいお金、小銭 【自動】変わる、変化する、変遷する 【他動-1】~を変える、~を変更する、~を変換する 【他動-2】~を取り替える、~を交換する 【他動-3】~を乗り換える 【他動-4】~を両替する、~をくずす◆お金を
\\	1年	
\\	【名】子ども、子、小児
\\	1年	
\\	【名】子どもたち
\\	の複数形)
\\	1年	
\\	【国名-1】中華人民共和国、中国◆【同】
\\	【国名-2】中華民国◆【同】
\\	【分節】
\\	1年	
\\	【名】中国人、中国語、中華料理(店) 【形】中国の、中国語の、中国人の、中国製の、中国人[語]の
\\	1年	
\\	【名-1】都市、市◆【参考】
\\	【名-2】全市民 【形】都会の
\\	1年	
\\	【名-1】授業、学級、クラス 【名-2】分野、類、部類、種類 【名-3】階級 【名-4】気品、上品さ、礼儀正しさと教養が備わっていること 【名-5】〈米〉同期生 【名-6】《生物分類》綱◆生物分類については 
\\	または 
\\	を参照。 【他動】~を分類する、~の等級を定める
\\	1年	
\\	【形-1】曇った、雲の 【形-2】はっきりしない、ぼやけた、あいまいな、濁った
\\	1年	
\\	【名】コーヒー
\\	1年	
\\	【名-1】寒さ、冷たさ、低温度、冷気 【名-2】風邪、感冒 【形-1】冷たい、きんきんに冷えた[冷やした] 【形-2】寒い、寒々とした 【形-3】冷酷な、無情な、非情な、残酷な、つれない◆【同】
\\	【形-4】冷静な 【形-5】面白くない 【形-6】正解から遠い、目的物から遠い 【副-1】冷やして、冷えてから 【副-2】完全に、完ぺきに、まったく、確実に 【副-3】急に、いきなり 【副-4】きっぱりと、あっさり 【副-5】〈俗〉現行犯で
\\	1年	
\\	【名-1】色、色合い、色彩、彩色、色度、色味 【名-2】顔色、血色、皮膚の色 【名-3】人柄、個性、味 【名-4】象徴、軍旗 【自動-1】色付く、変色する 【自動-2】ほおを染める、顔を赤らめる 【他動-1】~に色をつける、~を着色する、~を彩色する、~を彩る、~を染める 【他動-2】~を潤色する、~を間違って伝える 【他動-3】~に特色を与える 【他動-4】~に影響を与える
\\	1年	
\\	【自動-1】目的地[中心]に向かって進む、来る、(人が)出現する 【自動-2】生じる 【自動-3】~という状態になる
\\	補語」の形を取り、補語には話し手にとって望ましい状態を示す形容詞が来ることが多い。◆【参考】
\\	【自動-4】~という状態で売られている 【自動-5】《仮定法現在形で》(時が)来ると 【他動-1】~歳になる[近づく] 【他動-2】~を行う、~を演じる 【他動-3】~ぶる、~風を吹かせる◆【用法】
\\	名詞(または形容詞)◆非難の意を込めて
\\	1年	
\\	【名】コック、料理をする人、料理人、板前 【自動-1】料理を作る 【自動-2】煮える、焼ける、火が通る、火通りが良い 【他動-1】(熱を用いて)(目的の状態になるまで)~を料理する、~を調理する、~を加熱(調理)する、~を煮る、~をたく、~を焼く、~を揚げる、~を蒸す 【他動-2】~をごまかす
\\	1年	
\\	【名-1】国、国土、国家 【名-2】田舎、地方 【名-3】土地
\\	1年	
\\	【名-1】方向、コース、進路、路線 【名-2】経過、進行、成り行き 【名-3】課程、講座、科目、単位 【名-4】針路 【名-5】《医》治療単位 【名-6】(石・レンガなどの)段 【名-7】月経
\\	1年	
\\	【名-1】1日、日 【名-2】日中 【名-3】時代 【名-4】全盛期
\\	1年	
\\	"【名】いとしい人、親切な人 【形-1】親愛な、いとしい、かわいい、敬愛する、大切な 
\\	""彼女の手紙は「親愛なるジョン」で始まっていた。 【形-2】《レター》拝啓 【形-3】〈英〉高価な、値段が高い
\\	1年	
\\	【名】12月
\\	1年	
\\	【名-1】机 【名-2】受付 【名-3】編集主任 【形】卓上の
\\	1年	
\\	【動】
\\	の過去形
\\	1年	
\\	【形-1】違った、別の、異なった、相違する、違う、趣を異にする、間違った 【形-2】いろいろな、種々の、さまざまの
\\	1年	
\\	【形】難しい、困難な、難解な、扱いにくい、~しにくい、気難しい、厳しい◆【反】
\\	1年	
\\	【名】夕食、ディナー、正餐、晩餐会、祝宴◆昼または夜に取る一日のうちの主要な食事のこと◆【略】
\\	1年	
\\	【助動-1】《疑問文》~しますか? 【助動-2】《否定文》~しない 【助動-3】《強調》ぜひ、どうぞ、本当に、やはり、実に 【自動-1】する 【自動-2】振る舞う 【自動-3】役に立つ、間に合う、とりあえず用が足りる 【自動-4】(慣習・社会通念・規則などに)適する、好ましい、許される◆否定形で用いられる 【自動-5】《現状》(人の健康状態が)~である、(国・法人・事柄の状態が)~である
\\	の形をとる 【他動-1】~をする 【他動-2】~の役に立つ、間に合う、とりあえず用が足りる 【他動-3】(役)を演じる 【他動-4】~の物まねをする 【他動-5】(場所)を見物する 【他動-6】(距離)を進む 【他動-7】《口》(刑期)を務める
\\	1年	
\\	【名-1】医者、医師、内科医 【名-2】博士、博士の資格、博士号 【名-3】修理屋、修繕屋 【他動-1】~に治療を施す、~を治療する 【他動-2】~に薬を混ぜる、~に混ぜ物を入れる 【他動-3】(機械など)を修理する、~を改ざんする、ごまかす、~に勝手に手を加える、手を加えてごまかす、~を不正に変更する
\\	1年	
\\	【動】
\\	の3人称単数現在形
\\	1年	
\\	【名】犬 【他動】~に付きまとう、~を尾行する、~にたかる、悪さをする
\\	1年	
\\	【名】扉、ドア、戸、扉口
\\	1年	
\\	【名-1】(鳥の)綿毛、(鳥の)産毛 【名-2】《植物》軟毛 【名-3】(小高い)草原 【前】~の下へ 【形-1】下にある、降りて来る 【形-2】(コンピュータなどが)ダウンして、停止して、故障して、(機能などが)うまく働かない 【形-3】倒れている、病気にかかってい 【形-4】(精神的に)落ち込んでいる、暗い気分で、気がふさぐ、元気がない、憂うつな、陰気なる 【形-5】(売り上げなどが)落ち込んで 【形-6】落ち着いた、穏やかな 【形-7】ダウンの、羽毛入りの 【副-1】下に、下方へ、下って 【副-2】離れた場所に、そっちへ、向こうへ 【副-3】本気で、身を入れて 【他動-1】~を降ろす、下に置く 【他動-2】~を撃墜する、《野球》破る 【他動-3】~をグイッと飲み干す、ゴクゴク飲む
\\	1年	
\\	【名】夢、夢現象、夢うつつの状態、理想 【形】夢のように素晴らしい、夢の中の 【自動】夢を見る 【他動】~の夢を見る、~を夢見る
\\	1年	
\\	【代名】各々、それぞれ、めいめい 【形】それぞれの、一つ一つの、めいめいの 【副】各々、それぞれ、一つにつき、めいめい◆【略】
\\	1年	
\\	【形-1】(定刻より)早い、早期の、初期の 【形-2】早朝の 【形-3】上旬の 【副-1】早い時期に、(定刻より)早く、早期に、初期に 【副-2】朝早く、早朝に
\\	1年	
\\	【副-1】容易に、たやすく 【副-2】余裕を持って 【副-3】軽く、優に
\\	1年	
\\	【形-1】たやすい、やさしい、容易な、簡単な、分かりやすい、対処しやすい、扱いやすい、楽勝の◆【略】
\\	【形-2】気楽な、安楽な、着心地の楽な、ゆったりとした 【副-1】たやすく、容易に、簡単に、楽に、やすやすと 【副-2】気楽に、のんびりと、ゆっくりと 【副-3】落ち着いて、慎重に、気を付けて 【副-4】少なくとも、優に
\\	1年	
\\	【自動-1】食べる 【自動-2】(人を精神的に)破壊する、むしばむ 【他動-1】~を食べる 【他動-2】~を心配させる、~の気をもませる、~を困らせる、イライラさせる
\\	1年	
\\	【名-1】卵 【名-2】卵子、卵細胞 【名-3】繭、さなぎ 【名-4】卵の形をしたもの、ゼロ、頭 【自動】卵を集める 【他動-1】~の上に卵をかける、~に卵を混ぜ合わせる 【他動-2】~に(生)卵を投げ(つけ)る 【他動-3】~を扇動する
\\	1年	
\\	【名】8、8個、8人、8歳、8時 【形】8の、8個の、8人の、8歳の
\\	1年	
\\	【名】18、18個、18人、18歳、18時 【形】18の、18個の、18人の、18歳の
\\	1年	
\\	【名-1】8分の1 【名-2】第8(番目)、8番目のもの[人] 【形-1】第8の、8番目の 【形-2】8分(の1)の 【副】8番目に
\\	1年	
\\	【名】80、80個、80人、80歳 【形】80の、80個の、80人の、80歳の
\\	1年	
\\	【名】11、11個、11人、11歳、11時 【形】11の、11個の、11人の、11歳の
\\	1年	
\\	【名-1】11分の1 【名-2】第11 【形-1】第11の、第11番目の 【形-2】11分(の1)の
\\	1年	
\\	【名-1】英語 【名-2】英国人◆【注意】ウェールズおよびスコットランド出身者は、英国人という意味で 
\\	を使うことを嫌うので、彼らに対しては 
\\	を使ったほうが良い。 【形】イギリスの 【人名】イングリッシュ 【地名】イングリッシュ、イングリッシュ川
\\	1年	
\\	【自動】楽しむ 【他動-1】~を楽しむ 【他動-2】~を味わう 【他動-3】~に恵まれている、(好ましいもの)を持っている 【他動-4】(利益など)を享受する
\\	1年	
\\	【名-1】夕方、晩◆日没または夕食の時間から就寝時間まで 【名-2】衰退期、晩年 【名-3】夜会、夕べ
\\	1年	
\\	【形-1】あらゆる、すべての、どれも皆、いかなる 【形-2】~ごとに、~おきに、毎~
\\	1年	
\\	【代名】あらゆる人、すべての人、一人残らず
\\	1年	
\\	【名】弁解、言い訳、口実、弁明、釈明、理由 【他動-1】(人)を許す、容赦する、罪から免れさせる 【他動-2】言い訳をする、弁解をする 【他動-3】退出[中座]を許す 【他動-4】~を免除する
\\	1年	
\\	【名-1】落下、墜落、撃墜 【名-2】減少 【名-3】降雨量、降雪量 【名-4】〈米〉秋◆【語源】木の葉が“落ちる”季節だから◆【参考】
\\	【名-5】崩壊、破綻、陥落、愛顧を失うこと、受けが悪くなること、挫折 【自動-1】(バランスを失ったため・重力のせいで)高い位置から低い位置へ移動する、落ちる、落下する、(雨・雪などが)降る 【自動-2】倒れる、滅びる、壊れる、砕ける、崩壊する、陥落する、破綻する 【自動-3】(質・数量が)減少する 【自動-4】減益となる、(利益が)落ちる、(株価などが)下落する 【自動-5】(範囲に)収まる 【自動-6】~が起こる 【自動-7】(風が)凪ぐ
\\	1年	
\\	【名-1】家族 【名-2】家系 【名-3】~の仲間、~族
\\	の形で 【名-4】科、群、族◆生物の分類 【名-5】車種
\\	1年	
\\	【形】有名な
\\	1年	
\\	【名-1】ファン 【名-2】扇風機、送風機、うちわ、扇 【自動-1】扇形に広がる 【自動-2】《野球》三振する 【他動-1】風を送る、~をあおぐ、かき立てる、扇動する 【他動-2】《野球》(打者)を三振に打ちとる
\\	1年	
\\	【名-1】父、父親 【名-2】神父 【自動】父親としての責任を果たす 【他動-1】(自分の精子で)~の父親になる 【他動-2】~を生み出す、~を作り出した人である
\\	1年	
\\	【名-1】お気に入り、大好きなもの 【名-2】本命、本命馬、優勝候補◆【用法】
\\	【名-3】人気株◆【用法】
\\	【形】お気に入りの、大好きな、ひいきの
\\	1年	
\\	【名】2月
\\	1年	
\\	【名-1】感触、手触り 【名-2】勘 【名-3】(場所や状況の)特色[特徴]、雰囲気 【自動-1】感じがする、感じる 【自動-2】~したい気がする 【自動-3】~のような気がする 【自動-4】まるで~のようだ
\\	の形をとる 【自動-5】強い興味を抱く、しっかりとした意見を持っている 【自動-6】探る、探す 【自動-7】同情する 【他動-1】~を感じる、感知する 【他動-2】~に触ってみる、~に触る、~を感じてみる 【他動-3】(何となく)~ではないかと思う 【他動-4】(精神的感情や被害など)を受ける[感じる] 【他動-5】~するのを感じる
\\	1年	
\\	【名】祝祭、祭、定期的催し物、行事 【形】祝祭の、祭日の
\\	1年	
\\	【名】15、15個、15人、15歳、15時 【形】15の、15個の、15人の、15歳の
\\	1年	
\\	【名-1】第5(番目)、5番目のもの[人] 【名-2】5分の1 【形-1】第5の、5番目の 【形-2】5分(の1)の 【副】5番目に
\\	1年	
\\	【名】50、50個、50人、50歳 【形】50の、50個の、50人の、50歳の
\\	1年	
\\	【名】一番、トップ、第一、一番手、一流、一位、第一の人[もの]、一級品 【形】一番の、トップの、第一の、一番手の、一流の、一位の、トップを切る、先駆けをする、首位を占める◆【反】
\\	◆【略】1
\\	【副】最初に、初めて、一番に、第一に、まず、何よりも
\\	1年	
\\	【名】魚、魚料理、鮮魚◆【仏語】
\\	【自動】捜す、探す、探る 【他動】魚を捕る、~を釣る
\\	1年	
\\	【名】5、5個、5人、5歳、5時 【形】5の、5個の、5人の、5歳の
\\	1年	
\\	【名-1】食物、食品、食料 【名-2】餌、肥料、飼料 【名-3】(精神および日々の生活に)必要なもの、(心の)糧
\\	1年	
\\	【接】~だから、その理由は~だから 【前-1】~のために、~に対して 【前-2】~を目的として、目指した、~のための、~に使われて、~にふさわしい 【前-3】(目的地に)向かって(進んで)、~に行こうとして 【前-4】~にとって、~に関して 【前-5】~に賛成で、賛成して、~を支援して、後援して 【前-6】~を気に入って 【前-7】~の代わりに 【前-8】~にしては
\\	1年	
\\	【名】40、40個、40人、40歳 【形】40の、40個の、40人の、40歳の
\\	1年	
\\	【名】4、4個、4人、4歳、4時 【形】4の、個の、4人の、4歳の
\\	1年	
\\	【名】14、14個、14人、14歳、14時 【形】14の、14個の、14人の、14歳の
\\	1年	
\\	【名-1】4分の1、第4(番目)、四分の一、4日、4番目のもの[人] 【名-2】4分の1、第4(番目)、四分の一、4日、4番目のもの[人] 【形-1】第4の、4番目の 【形-2】4分(の1)の、四分の一の 【副】4番目に
\\	1年	
\\	【形-1】自由な 【形-2】無料の、ただの、無償の、免除された、サービスの 【形-3】暇で、手があいて[すいて]いる、用がない 【形-4】~のない 【形-5】遊離型の 【形-6】接続されていない、つないでいない 【副-1】自由に 【副-2】ただで、無料で、料金を取らずに 【他動-1】~を自由にする、解放する 【他動-2】~を取り除く 【他動-3】(市場)を自由化する
\\	1年	
\\	【名】金曜日◆【略】
\\	【副】金曜日に
\\	の形で
\\	1年	
\\	【名-1】友人、仲間、友達、相棒 【名-2】支持[共鳴・後援]者、味方、力[助け]になるもの 【名-3】フレンド教会の信者◆通例大文字で始める。
\\	(クエーカー教徒)と同義
\\	1年	
\\	【前-1】~から◆【略】
\\	【前-2】~が提供している
\\	1年	
\\	【名】楽しみ、面白いこと 【形】楽しい、ゆかいな 【他動】~をからかう
\\	1年	
\\	【名-1】ゲーム、試合、遊び、競争 【名-2】仕事、職業 【名-3】獲物、猟の獲物、(獲物の)肉 【名-4】目標、的、標的、狙い、意図、もくろみ 【名-5】冗談 【名-6】最も得意とすること 【形】闘志[勇気・元気]のある、負けん気の強い、頑張り屋の
\\	1年	
\\	【自動-1】着く、達する 【自動-2】(~の状態に)なる、(~するに)至る 【他動-1】~に至る、~に到着する 【他動-2】(何かの作用・影響により何か)を得る、自分のものにする、買う、手に入れる、入手する、確保する、調達する 【他動-3】~の状態にする 【他動-4】(相手の言ったことなど)を頭の中に得る、理解する、分かる◆【同】
\\	【他動-5】~を連れてくる、呼んでくる、迎える 【他動-6】(人)に~を取ってくる、(人)に~を買ってやる 【他動-7】~をやっつける、殺す、~に仕返しする、報復する 【他動-8】《野球》~をアウトにする 【他動-9】~を苦しめる、悩ませる
\\	1年	
\\	【名】少女、女の子、女児
\\	1年	
\\	"【名-1】進むこと、進行 【名-2】順番 【名-3】試すこと、試み 【名-4】成功 【名-5】成立した取引 【名-6】許可 【自動-1】前方に進む、行く、及ぶ 【自動-2】続ける、継続する 【自動-3】作動する、進展する、達する、~の状態になる 【自動-4】物が売れる、流通している 【自動-5】(金などが)費やされる 【自動-6】現在地から離れて行く、去る、消えていく 【自動-7】~と書いてある 【自動-8】(声援して)~頑張れ 【自動-9】
\\	動詞》思い切って、ぜひ 【他動-1】~と言う、~と言った
\\	、(動物が)~と鳴く 
\\	猫はニャーニャーと鳴く 【他動-2】~を我慢する、~に耐える◆【用法】通例、否定文で用いる。 【他動-3】(金銭)を賭ける 【他動-4】~を産出する 【他動-5】重さが~である 【他動-6】~を味わう◆【用法】通例、
\\	飲食物」の形をとる。
\\	1年	
\\	【名-1】よいこと[もの]、役に立つこと[もの]、価値、効用、利益、幸福、ためになること 【名-2】親切、善、(美)徳、善良な人々、善人 【形-1】良い、優良な、上等な、結構な、優れた、望ましい、立派な、見事な、ためになる 【形-2】元気な、健康な、好調な 【形-3】善良な、親切な、ちゃんとした 【形-4】おいしい 【形-5】上手な、有能な、器用な、適した、得意で 【形-6】十分な、たっぷりの、豊富な、かなりの、申し分のない、完全な 【形-7】有効な◆【同】
\\	【形-8】まるまる~の、少なくとも~の◆数量を表す語を修飾して 【副】かなり、うんと、じっくり、立派に、うまく
\\	1年	
\\	【動】
\\	の過去・過去分詞形 
\\	ガット、ゴット
\\	1年	
\\	【名】祖父
\\	1年	
\\	【名】祖母
\\	1年	
\\	【名】偉大な人、要人、名工◆【用法】
\\	【形-1】大きい、大きな、大の、巨大な 【形-2】偉大な、卓越した 【形-3】すてきな、素晴らしい 【形-4】おいしい 【形-5】重大な 【形-6】多い
\\	1年	
\\	【名】ギター
\\	1年	
\\	【動】
\\	の過去・過去分詞形 
\\	ハッド、ハド 
\\	一杯食わされる、ひどい目に遭う、だまされる◆【同】
\\	1年	
\\	【形-1】幸福な、満足な、幸せな 【形-2】幸運な 【形-3】うれしい 【形-4】満足して 【形-5】適切な 【形-6】〈俗〉酔って◆【類】
\\	1年	
\\	【形-1】(物が)硬い、堅い、硬質の、堅固な、強力な、硬くてなかなか壊れない、じょうぶな、硬音の 【形-2】難しい、困難な、つらい 【形-3】熱心な、精を出して 【形-4】(事実・情報・証拠などが)否定できない、確かな、信頼できる 【形-5】(表情が)険しい 【形-6】(言葉などが)きつい 【形-7】強情な、頑固な 【副-1】熱心に、一生懸命 【副-2】激しく、ひどく 【副-3】かたく 【副-4】なかなか~しない、やっと◆【参考】
\\	1年	
\\	【動】
\\	の3人称単数現在形
\\	1年	
\\	【他動-1】~を(自分の領域内に)持っている、所有している、~の持ち主である、自分の領域内に入れる、自分の所有物にする 【他動-2】~を飼う 【他動-3】~を経験として持っている、経験する 【他動-4】(時間)がある 【他動-5】~を~の状態にする 【他動-6】(人)に~させる[してもらう] 【他動-7】~が分かる 【他動-8】~を食べる 【他動-9】(人)を迎える、招待する 【他動-10】~をだます
\\	1年	
\\	【名】彼、男性、オスの動物 【代名】彼は、彼が
\\	1年	
\\	【名-1】頭、頭部、頭状のもの 【名-2】頭脳 【名-3】先端、先頭、上枠 【名-4】リーダー、代表、長、局長、所長、社長、学部長、理事長、委員長、団長、主任、元首、党首、世帯主、長官、ボス 【名-5】(硬貨の)表 【名-6】事態の最高点、クライマックス、山場、危機 【名-7】《文法》主要語 【名-8】水頭 【自動】頭を向ける、向かって進む 【他動-1】~の先頭に立つ 【他動-2】~を代表する、率いる
\\	1年	
\\	【間投】やあ、こんにちは、(電話で)もしもし、あのう
\\	1年	
\\	【名-1】助力、助け、援助、支援、力添え 【名-2】役に立つもの 【名-3】救済(法) 【名-4】従業員、雇人 【名-5】《コ》ヘルプ(画面) 【自動】役立つ、役に立つ、助けになる 【他動-1】~を助ける、手助けする、手伝う、救う、救済する◆「~するのを手伝う」と言う場合、
\\	の 
\\	は省略されることが多い。 【他動-2】~に役立つ、~するのに一役買う 【他動-3】~を促進する 【他動-4】~を避ける、抑える、制する、禁じる
\\	または 
\\	とともに使う
\\	1年	
\\	【代名】彼女を[に・の]
\\	1年	
\\	【副-1】ここで、ここへ、この世で 【副-2】ほら◆人の注意を引くとき
\\	1年	
\\	【代名】彼女のもの
\\	1年	
\\	【間投】やあ、こんにちは、ご機嫌よう、ねえ◆あいさつ、呼び掛け
\\	1年	
\\	【名-1】高い所、高所 【名-2】高値 【形-1】高い、高いところにある、高地の 【形-2】高さが~の 【形-3】たけなわの 【形-4】高級な、高貴な 【副-1】高く 【副-2】高額に
\\	1年	
\\	【代名】彼に、彼を
\\	1年	
\\	【代名】彼の、彼のもの、その
\\	1年	
\\	【名-1】家庭、自宅、家 【名-2】故郷、本国 【名-3】生息地 【名-4】療養所 【名-5】基地、本拠地 【形】家庭の、自国の 【副】自宅へ、家に、家に向かって、家に帰って、帰宅して、故郷に、本国へ、胸にこたえるほど、十分に
\\	は単に「家にいる」という意味
\\	1年	
\\	【名-1】宿題◆【反】
\\	【名-2】下調べ、準備
\\	1年	
\\	【名】希望、希望を与えるもの、見込み、期待 【自動】望む、希望する 【他動】~を願う、希望する、期待する、望む
\\	1年	
\\	【名】病院◆病院の正式名は 
\\	ではなくて 
\\	だが、日常会話では 
\\	の方が多く使われる。
\\	1年	
\\	【名】(~に対する)激しい性欲、偏愛、強い好み、性的に興奮していること 【形-1】暑い、熱い 【形-2】激しい、怒った、ハラハラする 【形-3】辛い、ヒリヒリする、ピリピリする 【形-4】はつらつとした、エネルギッシュな、活気に満ちた、好調な 【形-5】うまい、上手な 【形-6】できたての、最新の、新しい 【形-7】流行している、人気のある 【形-8】正解に極めて近い、目的物に極めて近い
\\	1年	
\\	【名】時間、時刻、60分◆数を12単位で表すバビロニアの習慣から(12はいろいろな数、2、3、4で等分できる)。夜の長さを表すのに12時間とした。
\\	は古代ギリシャ語 
\\	から。◆【略】
\\	1年	
\\	【名-1】家、住宅、家屋 【名-2】家族、家庭、所帯 【名-3】一家、家系、一族 【名-4】巣、小屋 【名-5】建物 【名-6】劇場 【名-7】議事堂 【名-8】修道院 【名-9】バー、レストラン、店 【名-10】カジノ、賭博場 【名-11】議会、下院◆【用法】
\\	【名-12】聴衆 【名-13】《音楽》ハウス 【名-14】〈英〉(パブリックスクールなどの)学生寮、寄宿舎[寮]、(大学の)学部 【名-15】企業 【他動-1】~を収容する、~に住居を与える、家をあてがう 【他動-2】~を収納する、格納する、内蔵する、~が保管場所となる、保管する
\\	1年	
\\	【名-1】方法、やり方◆通例 
\\	を付けて。 【名-2】「どうして」という質問[問いかけ] 【副-1】どのようにして、どのような方法で、いかにして 【副-2】どんな状態で 【副-3】どうして 【副-4】どのくらいの、どれほど、どの程度の、どの程度 【副-5】なんと
\\	1年	
\\	【名】100、100個、100人、100歳◆【略】
\\	【形】100の、100個の、100人の、100歳の
\\	1年	
\\	【名-1】アルファベットのアイ 【名-2】《ローマ数字》1、一◆【参考】
\\	(ローマ数字) 【代名】私は、私が
\\	1年	
\\	【名】人気のある人[もの] 【前-1】~の中に、~の中で、(大きな範囲・空間・領域の)中で[に] 【前-2】(時間の範囲の)中で[に]、~のとき(に)、~年に、~月に、~後に 【前-3】~に包まれて、~にくるまれて 【前-4】身体(の一部)を~の中に入れて、(服を)着て、(服を)まとって、(靴を)履いて、(帽子を)かぶって 【前-5】~の分野[範囲・範疇・世界]の中で[に] 【前-6】集団[グループ]の中で[に]、~に参加して 【前-7】~の状態で[に] 【形-1】今流行中の、流行している、はやっている、ナウい、受けている 【形-2】~に属する 【副】中へ、家[職場]の中で[に]、在宅して、在室して、家[職場]に帰って[戻って]
\\	1年	
\\	【形】興味を起こさせる、関心をひく、興味深い、興味のある、人の関心を引く、注意を引き付ける、面白い
\\	1年	
\\	【自動】
\\	の3人称単数現在形
\\	1年	
\\	【名】(鬼ごっこの)鬼 【代名】それ◆何かを指す言葉
\\	1年	
\\	【代名】それの、あれの
\\	1年	
\\	【名】1月
\\	1年	
\\	日本◆【略】
\\	1年	
\\	【名】日本人[語] 【形】日本(製)の、日本人[語]の、日本的な
\\	1年	
\\	【名-1】仕事、職、勤め口◆代名詞で受ける際、女性名詞扱いすることがある 【名-2】やるべきこと、任務、義務、責務 【名-3】大変なこと 【他動】~の仲買をする、~の仲立ちをする
\\	1年	
\\	【名】7月
\\	1年	
\\	【名】6月
\\	1年	
\\	【名-1】後輩 【名-2】(大学)3年生 【形】年下の、年少の、若手の
\\	1年	
\\	【形】正しい、公正な、当然の、もっともな 【副-1】ちょうど、~だけ、今しがた 【副-2】実に、まさに、まったく 【副-3】単に、ちょっと、ただ~だけ、たった~だけ 【副-4】ようやく
\\	1年	
\\	【自動】知っている、理解している 【他動-1】~を知る、知っている、承知している 【他動-2】~と知り合いである 【他動-3】~が分かる 【他動-4】~と見分けがつく、~を識別する
\\	1年	
\\	【名-1】言語、ことば 【名-2】言葉遣い 【名-3】専門用語
\\	1年	
\\	【名-1】最後、最下位 【名-2】長続きすること、持続力、耐久力、根気 【名-3】靴型 【形-1】終わりの、最後の◆【反】
\\	【形-2】最下位の、最も不適当な 【形-3】最近の、この前の、すぐ前の、先の◆現在に一番近いという意味で使用される 【形-4】とっておきの 【副-1】最後に、一番終わりに 【副-2】この前に 【自動-1】続く、存続する、持続する、耐える、持ちこたえる 【自動-2】必要を満たす、足りる、間に合う、十分である
\\	1年	
\\	【形-1】遅い、遅刻の、時間[定刻]に遅れる、遅れた 【形-2】下旬の、末期の、終わりごろの 【形-3】最近の 【形-4】(最近)死んだ、亡くなった、今は亡き、故人となった、故~ 【副】遅れて、遅く
\\	1年	
\\	【名-1】休暇、許可 【名-2】告別 【自動-1】去る、退く 【自動-2】出発する 【自動-3】葉を出す 【他動-1】~から離れる、~を[から]出る[出発する・発つ・出立する・後にする・去る] 【他動-2】~を退く、~から撤退する、(会社)を辞める、中止する 【他動-3】(人)と別れる 【他動-4】~を残す、置きっぱなしにする、置き忘れる 【他動-5】~をそのままにしておく、(人)をそっとしておく、ほうっておく、放任しておく、なすがままにさせておく、放置する、ある状態のままにしておく 【他動-6】~を見捨てる 【他動-7】~を預ける 【他動-8】~を任せておく、任せる、頼む、委ねる
\\	1年	
\\	【名】左、左側 【形】左の、左側の 【副】左へ 【自動】
\\	の過去・過去分詞形
\\	1年	
\\	レッツ 
\\	の縮約形
\\	1年	
\\	【名-1】光、太陽光 【名-2】明かり、照明 【名-3】発火物、火 【名-4】観点、見方 【形-1】明るい 【形-2】軽い 【形-3】消化の良い、(味が)あっさりした 【形-4】(濃さ・濃度が)薄い、淡い 【形-5】有害物質の含有量が少ない、カロリー[アルコール分・ニコチン]の少ない、低カロリー[アルコール]の 【形-6】軽妙な 【副】軽く、容易に 【自動】点灯する、火がつく 【他動】~を明るくする、~に火を付ける、点火する、灯を灯す
\\	1年	
\\	"【名-1】好み、趣味 【名-2】似たもの、同類、同等のもの、~のようなもの[人] 【前】~に似た、~のような、~のように、~と同じように 
\\	ほら、よく言うじゃないか、娘は母親に似るって。 【形】同じの、似ている、類似の、類似した、同様の、同種の 【間投】その…、まあ、~という感じで、例えば、何か◆つなぎ言葉として、会話の中でよく使われる。時々 
\\	のスラング的言い方)の代わりに使われる。
\\	があまりにも頻繁に使われるので、これは間合いをとるための音として
\\	の代わりに使われているように思われる。使い過ぎると教養が低いと思われるので注意。 【他動-1】~を望む、欲する 【他動-2】~を気に入る、好む 【他動-3】~に適する[適合する・合う・似合う]
\\	1年	
\\	【名-1】線、直線 【名-2】電話線、(電話)回線、通信網、通信線、電話 【名-3】境界線 【名-4】走査線 【名-5】描線、輪郭線 【名-6】手相の線、筋、縞線、皺 【名-7】糸、釣り糸、ひも、綱、物干し綱
\\	、縄 【名-8】電力線 【名-9】列、行列、連続したものの線、並び 【名-10】(文字が並んだ)行、詩の1行、短詩 【名-11】一筆、短信 【名-12】役者のせりふ、ひとくだり、一節、口癖 【名-13】系列、血統、家系、種族 【名-14】輪郭、外形 【名-15】ドレスなどのスタイル、シルエット 【名-16】道筋、進路、方向、道 【名-17】路線、方向、方針、政策、主義、傾向 【名-18】専門 【名-19】商売、職業 【名-20】取扱商品、仕入れ品、在庫品 【名-21】(~についての)情報 【名-22】《軍事》戦列、陣形、布陣 【名-23】赤道 【名-24】〈英〉結婚証明書 【自動-1】並ぶ 【自動-2】《野球》ライナーを打つ 【他動-1】~に線を引く、罫線を引く、~を線で描く 【他動-2】~に皺を寄せる 【他動-3】~を一列に並べる 【他動-4】~に裏地を付ける、~で裏打ちする、~の内側を覆う 【他動-5】(胃袋など)を満たす
\\	1年	
\\	【自動】耳を傾ける、傾聴する、聴く、聞く、耳を貸す
\\	1年	
\\	【形-1】生きている、本物の 【形-2】生き生きしている、活気のある、精力的な、元気な 【形-3】生放送の、生の、実況中継の◆テレビやラジオなどの番組が 【形-4】当面の、目下の 【形-5】(火が)燃えている、電流が通じている、弾がこめられた、まだ使える 【副】生放送で、生で 【自動-1】住む、居住する、暮らす、生活する、所在する 【自動-2】生きる、生存する、生きている 【自動-3】生き長らえる、生き延びる、残る、存続する、続く、持ちこたえる 【他動】~な生活を送る、~の生活をする◆【参考】
\\	1年	
\\	【名-1】長い間、長時間、長いもの 【名-2】〈英〉長期休暇 【名-3】長期公債 【名-4】長音 【名-5】長ズボン 【形-1】(細)長い◆【反】
\\	【形-2】~の長さの 【形-3】遠くまで達する 【形-4】強気である、強気買いに出る 【副-1】長く、長い間 【副-2】長々と続く、時間がかかって 【副-3】ずっと後に、久しく 【副-4】遠くへ、遠くまで 【自動】待ち望む、思いこがれる、待ちこがれる、あこがれる、切望する、熱望する
\\	1年	
\\	【名-1】見ること、眺めること、調査 【名-2】様子、外観、見た目 【名-3】目つき 【自動-1】~に目を向ける、見る、眺める 【自動-2】~の傾向にある 【自動-3】~の目つきをする、~にふさわしい様子である 【自動-4】捜す、調べる 【自動-5】注意を向ける 【自動-6】~のように見える、~と思われる 【他動-1】見る 【他動-2】~のように見える 【他動-3】注意する、気を付ける、確かめる、確認する
\\	1年	
\\	【名-1】たくさん、1組、多量、多額 【名-2】くじ(引き) 【名-3】分け前 【名-4】抽選 【名-5】運命、宿命 【名-6】一区画、用地 【名-7】ロット、製造の一単位、商品発注時の数量単位 【名-8】人の群れ、組 【名-9】《口》連中、やつ 【副】おおいに、とても、たいそう 【自動】くじを引く 【他動-1】くじを引く 【他動-2】~を分け与える、区分けする
\\	1年	
\\	【名-1】愛、愛情、愛着、好意、恋、恋愛(関係) 【名-2】《恋人・夫婦間の呼び掛け》あなた、おまえ 【名-3】好きなもの、愛人、恋人、(自分が)愛する人 【名-4】ゼロ、0点 【他動-1】~を愛する、かわいがる、大事にする、~に恋する 【他動-2】~が大好きだ
\\	1年	
\\	【名】昼食、ランチ 【自動】昼食を取る[食べる] 【他動】~に昼食を出す
\\	1年	
\\	【名】型、製造元 【自動-1】急いで進む、素早く動く[移動する]、(道が)走る[伸びる]、向かう◆【用法】
\\	【自動-2】(潮が)満ちる◆【用法】通例進行形 【自動-3】~しようとする◆【用法】
\\	【自動-4】~するふりをする◆【用法】
\\	直説法・仮定法 【自動-5】物まね[形態模写]をする◆【用法】
\\	【自動-6】~の状態になる◆【用法】
\\	形容詞 【他動-1】~を作る、構成する 【他動-2】~の状態を作り出す、~にする、~になる 【他動-3】(人)を~の状態にする、(人)に~させる◆やりたくないことを強制的にさせる場合が多い 【他動-4】~を引き起こす 【他動-5】~を行う 【他動-6】~と思う 【他動-7】~に達する 【他動-8】~を作り上げる、~に成功する、~がうまくいく、もたらす 【他動-9】~に間に合う
\\	1年	
\\	【名-1】人間、人、人類 【名-2】男(性)、男子、男らしい男 【名-3】部下 【他動-1】~に要員を配置する 【他動-2】~を受け持つ
\\	1年	
\\	【名】多数の人[もの]、多くのもの 【形】多くの、たくさんの、多様な
\\	1年	
\\	【名】3月
\\	1年	
\\	【名】数学、計算 
\\	マス
\\	1年	
\\	【名】5月
\\	1年	
\\	【代名】私に、私を 【形】自己中心の、自分本位の、わがままな
\\	1年	
\\	【名】会、大会、会合、試合 【自動-1】会う 【自動-2】触れる、接触する、交わる 【自動-3】(会が)開かれる 【自動-4】集まる 【自動-5】対戦する 【自動-6】兼ね備わる 【他動-1】~に会う、~と会合する 【他動-2】~を出迎える 【他動-3】~と接触する、~と知り合いになる 【他動-4】(要求・条件など)に合う、~を満足させる、かなえる、満たす、~にかなう、(願望など)を達成する 【他動-5】~に応じる、~に対処する、~に対応する 【他動-6】~を直視する
\\	1年	
\\	【名-1】鉱山、採掘坑、鉱業場 【名-2】宝庫 【名-3】機雷、地雷◆【同】
\\	◆対人、対車両の地下爆弾。特に対人地雷は、相手を殺さず足だけを吹っ飛ばし、敵に救助・治療させ、結果的に敵の戦力を削ぐのを目的とする。プラスティック製地雷は検出が困難。対車両地雷の中には数度の感知後、爆発するのもある。 【代名】私のもの 【他動】~を掘る、採掘する、採鉱する
\\	1年	
\\	【名-1】分◆【語源】ラテン語の 
\\	(極めて小さなもの)で、時間の“小さな”単位という意味で「分」となった。 【名-2】わずかな時間、寸刻 【名-3】瞬間 【名-4】メモ、覚書、控え◆【語源説-1】“数分”でかける手短な/簡潔なものだから◆【語源説-2】“小さな”手書き文字で口述筆記されたものだから 【名-5】議事録 【形-1】極めて小さい、微小な、微細な、細かい 【形-2】詳細な、綿密な 【形-3】ささいな、取るに足りない、くだらない
\\	1年	
\\	【名-1】ミス、失敗、間違い、見逃し 【名-2】~嬢、~さん◆独身女性の名字の前に付ける敬称 【名-3】《呼び掛け》お嬢さん 【自動-1】的を外す 【自動-2】失敗する 【自動-3】着火しない、点火しない 【他動-1】~がなくて不自由する、物足りない気持ちがする、~がない[いない]のを寂しく思う、~が恋しい 【他動-2】~し損なう、~に当たらない、~を免れる、(鉄砲の弾)が外れる 【他動-3】逃す、見逃す、見損ねる、~を抜かす、聞き漏らす 【他動-4】~を欠席する、《野球》欠場する 【他動-5】~に乗り遅れる、~に間に合わない
\\	1年	
\\	【名】月曜日◆【略】
\\	1年	
\\	【名-1】(暦の)月、月◆【略】
\\	【名-2】《複数形》乳児月齢
\\	1年	
\\	【名-1】朝、午前、夜明け 【名-2】初期、始まり、早期
\\	1年	
\\	【名-1】母親、母(性) 【名-2】義母、養母 【名-3】母性愛 【名-4】根源 【形】母の、母国の、母としての 【他動】~の母になる、~を産む、生み出す
\\	1年	
\\	【名】多量、たくさん 【形】多い、多量の、たくさんの 【副-1】ずっと、非常に、大変、とても 【副-2】大体
\\	1年	
\\	【名-1】音楽、楽曲、音曲、美しい調べ 【名-2】音感 【名-3】楽譜
\\	1年	
\\	【代名】私の 【間投】(驚きを表すときなど)おやまあ、まあ!
\\	1年	
\\	【名-1】名前、名称、氏名 【名-2】名目、名義 【名-3】評判 【名-4】悪口 【名-5】有名な人 【他動-1】~に名をつける、~に名前をつける、~を命名する 【他動-2】~を指名する、任命する 【他動-3】~を告発する 【他動-4】(日時・場所・人)を指定する 【他動-5】~の名前を挙げる
\\	1年	
\\	【前】~しそうで 【前】~の近くに、~の近くで、身近に、接近して 【形】近い、近くにある、接近した、近接した 【形】近親の、密接な 【副-1】近い場所へ、近くへ、近くに 【副-2】密接に 【自他動】近づく、近寄る、接近する
\\	1年	
\\	【名-1】必要性 【名-2】必要なもの、必要物、生理的[心理的]要求 【他動】~する必要がある、~を必要とする、~に用[用事]がある
\\	1年	
\\	【形-1】新しい、新発見の、新型の 【形-2】新入りの、新参の 【形-3】慣れない 【形-4】一変した 【形-5】現代的な 【副-1】最近 【副-2】新しく
\\	1年	
\\	【形-1】よい、素晴らしい、すてきな、快い 【形-2】立派な、上品な、行儀の良い、洗練された 【形-3】微細な 【形-4】親切な 【形-5】気難しい 【形-6】風味の良い
\\	1年	
\\	【名】夜、暗闇、暗黒、晩 【形】夜の、夜間勤務の 【間投】おやすみ
\\	の省略形
\\	1年	
\\	【名】9、9個、9人、9歳、9時 【形】9の、9個の、9人の、9歳の
\\	1年	
\\	【名】19、19個、19人、19歳、19時 【形】19の、19個の、19人の、19歳の
\\	1年	
\\	【名】90、90個、90人、90歳 【形】90の、90個の、90人の、90歳の
\\	1年	
\\	【名】9分の1、第9(番目)、九分の一、1/9、9番目のもの[人] 【形】第9の、9分(の1)の、9番目の、9分の1の、九分の一の
\\	1年	
\\	【名】ノーという返事、
\\	という言葉、否定、否認、拒否、不同意 【形-1】少しの~もない、ゼロの、少しも~ない 【形-2】決して~ない 【副】いいえ、そうではない、それは違う
\\	1年	
\\	【副】《否定を示して》~でない
\\	1年	
\\	【名-1】11月◆【略】
\\	【名-2】
\\	◆無線用アルファベット
\\	1年	
\\	【接】~だから、~である以上は 【副-1】今、現在、たった今、直ちに、すぐに 【副-2】これから 【副-3】さて、ところで◆話を切り出すときに
\\	1年	
\\	【名-1】数、人数、個数◆【略】
\\	【名-2】番号、電話番号 【名-3】(出版物の)号、(書物の)巻◆【略】
\\	【名-4】構成員、会員、社員、組員、クラブ員 【名-5】商品 【名-6】多数、総数 【自動-1】数える 【自動-2】(数量が~に)達する、合計~になる 【他動-1】~に番号をつける 【他動-2】~を数える 【他動-3】~の数を定める、~を制限する 【他動-4】~歳になる
\\	1年	
\\	【名-1】看護人、看護師◆性別で異なる医療従事者の名称(看護婦・看護士)は「看護師」に統一された(2002年3月1日施行) 【名-2】養成する人 【他動-1】~を看病する、看護する 【他動-2】(思想など)を心に抱く 【他動-3】~を養育する、大事に育てる 【他動-4】~に授乳する、乳を飲ませる 【他動-5】(子どもが乳を飲むように飲み物)をチビチビ飲む、ゆっくりと大切そうに飲む、(喫茶店・バーなどで1杯だけ注文して)長時間ねばる 【他動-6】(病気)の治療につとめる、養生する、(怪我をした部分)をかばう
\\	1年	
\\	【副】~時、時計では、~時の位置◆【語源】
\\	の短縮形
\\	1年	
\\	【名】10月
\\	1年	
\\	【前-1】~の 【前-2】~から成る、~の入った、~で作った 【前-3】~について 【前-4】~から(取ってきた[取り出した])、~のうちの 【前-5】~のための 【前-6】~から起算して
\\	1年	
\\	"【前-1】~から離れて[それて・外れて] 【前-2】~より下回って[差し引いて・割り引いて] 【前-3】~沖に 【形-1】基準[標準]から離れた[はずれた]、本道から分かれた、間違った 【形-2】季節はずれの 【形-3】不況の 【形-4】休みの 【形-5】調子が悪い 【形-6】古くなった、腐った 【形-7】不満足な 【形-8】失礼な 【形-9】去って 【形-10】取れて 【形-11】割り引いて、差し引いて 【副-1】離れて、それて、外れて 【副-2】(仕事を)休んで 【副-3】(スイッチが)切れて、(機能が)停止して、(時計が)狂って、テンポを乱して 
\\	その機能を使えないようにすることができる。 【副-4】最後まで 【副-5】~沖に[で] 【自動】船が陸を離れる 【他動-1】約束をうち切る 【他動-2】取り去る、脱ぐ
\\	1年	
\\	【副-1】しばしば、頻繁に、よく 【副-2】多くの場合、たいてい
\\	1年	
\\	【間投】おお、おや、まあ、ああ、おー!◆驚きや感動を表す
\\	1年	
\\	【形-1】古い、年月を経た、老朽(化)した 【形-2】年とった、年長の、老いた 【形-3】~歳の 【形-4】かつての、前の 【形-5】昔からの、昔なじみの
\\	1年	
\\	【前-1】~の上に 【前-2】~に接して、~に接触して 【前-3】(人)のおごりで 【前-4】~に向かって◆出発点から目的地までを結ぶ線の上にいるイメージ 【前-5】~の状態で 【前-6】~に従事して 【前-7】~に関して、~についての 【前-8】~によって 【形-1】(電気の接点が)接触して、(機械が)作動して、動作が継続して、動作が続いて、動作中で、進行中で、始まって、起こって、放送中で、上演中で 【形-2】相手の意見[提案・申し出]と自分の気持ちの接点が合って、賛成である、乗り気である 【形-3】出番が来て 【形-4】予定どおりで 【形-5】調子が良い 【副-1】上に 【副-2】(スイッチが)入って、(機能が)働いて 【副-3】ずっと、どんどん
\\	1年	
\\	【名】1、1個、1人、1歳、1時 【代名】人、物 【形-1】1の、1個の、1人の、1歳の 【形-2】~という人◆【用法】
\\	人名
\\	1年	
\\	【接】ただし、だがしかし、しかし一方では 【形-1】唯一の、たったの、ただ一つの 【形-2】二つとない、最良の、最適の 【副-1】ただ[たった]~だけ、~だけである、~しか…しない 【副-2】つい~、たった~ 【副-3】~になって初めて[やっと・ようやく]
\\	1年	
\\	【形-1】開かれた、開いた、(店が)開いている、(誰でも自由に)出入りできる、営業中で、開会中で、開館して、開催中で、開場中で 【形-2】門戸を開放した、解禁の、公開の 【形-3】(花が)咲いて 【形-4】むき出しの、蓋なしの、上のない 【形-5】屋外の 【形-6】率直な、あからさまな、公然とした、公明正大な、隠し立てしない 【形-7】偏見のない 【形-8】空席の 【形-9】決まっていない、確定していない、確かになっていない 【形-10】(性格が)オープンな 【形-11】すき間のある 【形-12】無防備の 【形-13】広々とした、見通しが利いて 【形-14】温暖な 【自動-1】開く 【自動-2】始まる 【他動-1】~を開く 【他動-2】~を公開する 【他動-3】~を広げる 【他動-4】~を始める、開始する 【他動-5】《野球》突破口を開く 【他動-6】~を切断する
\\	1年	
\\	【接-1】~か、~ないし~、または、もしくは 【接-2】そうしなければ、さもなければ 【接-3】すなわち、つまり
\\	1年	
\\	【代名-1】われわれの、私たちの 【代名-2】例の、話題になっている
\\	1年	
\\	【代名】私たち[われわれ]のもの
\\	1年	
\\	【前-1】~の真上に、~の上空に 【前-2】~の全体を覆って、一面に 【前-3】~を越えて、~の向こう側に、物越しに、~を越えて向こう側へ 【前-4】~を超えて、~より多くて、~以上、~以降 【前-5】~を支配して 【前-6】~の間、始めから終わりまで 【前-7】移って 【前-8】~しながら 【形-1】上の 【形-2】優れた 【形-3】余分の 【形-4】(卵の)両面を良く焼いた 【副-1】上方に 【副-2】一面に 【副-3】終わって、終了して、完了して、おしまいになって 【副-4】もう一度、繰り返して 【副-5】(話し手の)家に、やって来る 【副-6】《無線》応答どうぞ 【他動】~を越える
\\	1年	
\\	【名-1】公園 【名-2】競技場 【名-3】遊園地 【名-4】大庭園 【名-5】駐車場 【自動】駐車する 【他動-1】~を駐車させる、駐車する 【他動-2】~を置いておく
\\	1年	
\\	【名-1】ペン 【名-2】文筆(業)、作家 【名-3】おり、囲い、家畜の檻 【他動-1】~を執筆する、書く◆【同】
\\	【他動-2】~を囲いに入れる、閉じ込める
\\	1年	
\\	【名-1】鉛筆 【名-2】眉墨、口紅 【他動】~を鉛筆で書く[記す・描く・下書きをする]
\\	1年	
\\	【名-1】人々、世間 【名-2】国民、民族、人民、臣民 【他動-1】(場所に)~を住まわせる 【他動-2】~を満たす、供給する
\\	1年	
\\	【名-1】電話(機) 【名-2】音 【他動】~に電話する、電話をかける[する・入れる]
\\	1年	
\\	【名】ピアノ、ピアノ曲 【形】ピアノの、弱い◆【略】
\\	1年	
\\	【名-1】絵、像、絵画 【名-2】写真 【名-3】映画 【名-4】絵柄、図面、図式 【名-5】光景、見物 【名-6】生き写し、そっくり、うり二つ 【名-7】実態、事実、状況、事態◆【用法】
\\	◆【類】
\\	【他動-1】~を描写する、絵に描く 【他動-2】~を頭[心]の中に描く、想像する、心に描く◆【同】
\\	1年	
\\	【名-1】劇、演劇、芝居 【名-2】しゃれ 【名-3】《スポーツ》プレー 【名-4】(機構的な)遊び 【自動-1】遊ぶ、楽しむ 【自動-2】試合をする、競技を行う 【自動-3】演奏する、(音などが)出る[鳴る] 【自動-4】芝居を行う、(映画や劇などが)上映[放映]される、(脚本が)上演できる[可能である] 【自動-5】振る舞う、行動をとる◆【用法】
\\	形容詞
\\	など)、
\\	副詞 【自動-6】(~の)ふりをする◆【用法】
\\	形容詞
\\	など) 【自動-7】(~の)状態にある◆【用法】
\\	形容詞
\\	など) 【自動-8】ちらつく、漂う、飛び跳ねる 【自動-9】(水や光を)ふきだす、(ホースが)水をふきだす、(銃が)撃たれる[発射される] 【自動-10】(機械などが)自由に動く 【自動-11】ばくちをする 【他動-1】(球技)を行う、試合[対戦]をする、(遊び)をする、競技をする 【他動-2】~ごっこ(遊び)をする 【他動-3】(楽器や曲)を弾く[演奏する]、(ラジオや 
\\	など)をかける◆【用法】
\\	楽器の形をとるが最近では 
\\	の省略も多く見られる 【他動-4】(役など)を演じる、(役割)を果たす、~のように振る舞う 【他動-5】(ポジション)につく 【他動-6】(場所で)~を公演[上演]する 【他動-7】(選手)を使う[起用する] 【他動-8】(冗談)をする 【他動-9】(金)を賭ける、投資を行う 【他動-10】(いたずら)をする 【他動-11】《囲碁》対局する
\\	1年	
\\	【名-1】選手、演奏者、遊ぶ人、競技者、俳優 【名-2】同時に複数の女性と付き合う男 【名-3】遊撃手 【名-4】プレーヤー◆政治的、商業的に影響力のある役割を持った人、団体、企業のこと 【名-5】プレーヤー◆録音を再生する装置
\\	1年	
\\	【間投-1】どうぞ、どうか◆【略】
\\	【間投-2】どうぞ! 【他動-1】~を喜ばせる、人を満足させる、楽しませる 【他動-2】~の気に入る
\\	1年	
\\	【形-1】評判が良い、好かれている、人気のある[高い]、よく知られている、顔が売れている、評判の良い、好評な、好かれる、好評である、受けが良い、有名である、衆望がある、信望がある、評判を呼ぶ、流行する 【形-2】よくある、多く出回っている 【形-3】一般人の、民衆の、大衆の、庶民の、通俗の
\\	1年	
\\	【名-1】練習、訓練、演習 【名-2】実践、実行、履行 【名-3】いつもすること、しきたり、慣習、慣例、(社会的)慣行、習慣、習わし 【名-4】実務、(医者・弁護士の)業務、診療、営業、開業(場所)、仕事 【名-5】熟練、手腕 【名-6】訴訟手続き 【名-7】患者、事件依頼人 【自動】(医師や弁護士などが)開業する 【他動-1】~を練習する 【他動-2】~を実行する、順守する、実践する、実施する 【他動-3】~をいつも行う、習慣的に行う 【他動-4】~を営む
\\	1年	
\\	【名-1】今、現在 【名-2】《法律》文書類、本証書 【名-3】《文法》現在時制 【名-4】プレゼント、贈り物 【形-1】存在している、そこにある 【形-2】出席している、出勤している、参列する、末席を汚す、居合わせている 【形-3】現在の、今の、目下の、当面の 【形-4】《文法》現在時制の 【他動-1】~を贈る、進呈する 【他動-2】~を与える、渡す、差し出す、提出する、提示する、提起する 【他動-3】~を紹介する 【他動-4】~を公開する、見せる、上演する、示す 【他動-5】~を姿を現す
\\	1年	
\\	【名-1】質問、問題、疑問、問い、質疑 【名-2】論題、議題、懸案 【名-3】論点 【名-4】事柄、事情 【名-5】《文法》疑問文 【他動-1】~に質問する、尋ねる、問い合わせる、試問する、職務質問する、尋問する 【他動-2】~を疑う、異議を申し立てる、疑問に思う、疑問を呈する[投げかける]、問題にする[として取り上げる]
\\	1年	
\\	【名-1】《コ》読み込み、読み取り 【名-2】読み物 【自動-1】読む 【自動-2】~と書いてある、~と読める 【他動-1】~を読む 【他動-2】~を校閲する 【他動-3】《コ》~を読み取る、読み出す、読み込む 【他動-4】(表示装置が値)を示す 【他動-5】(議案書)を読み上げる、読会にかける 【他動-6】〈英〉(大学である科目)を専攻する◆〈米〉
\\	【他動-7】《無線》~を聞き取る
\\	1年	
\\	【副】実際には、ほんとうは、確かに、本当に、真に、心から、本気で、絶対に、実際に[は]、現実に[は]、実は、いかにも
\\	1年	
\\	【名】レストラン、料理店、飲食店
\\	1年	
\\	【名-1】米、ライス◆【仏語】
\\	【名-2】稲
\\	1年	
\\	【名-1】車に乗せること、乗車、ドライブ、乗ること 【名-2】乗る時間 【名-3】車 【自動-1】(乗り物に)乗る、乗馬する、馬で進む、~にまたがる 【自動-2】浮かぶ 【自動-3】重なる 【他動-1】~に乗る、~を乗せる、自転車で~を乗り越える 【他動-2】~にかかる 【他動-3】責める
\\	1年	
\\	【名-1】右、右側にあるもの、右手、右足、(制御装置などの)右ボタン、右回り 【名-2】正しさ、正当性、正義、正しい行い、正しい考え方、道理、道理に合ったこと、善、公正さ 【名-3】真相 【名-4】正確さ、的確さ 【名-5】正しい要求、正当な要求、当然の権利 【名-6】権益、所有権、版権 【名-7】新株引受権 【名-8】所有権を持っているもの、財産 【名-9】右翼(の人) 【形-1】右の、右手の、右側の、右方の 【形-2】正しい、正当な◆【語源】人間は心臓の反対側にある手をよく使うところから、その手を使うことが正しいとされた。 【形-3】ちょうど良い、適した、適切な、適当な、妥当な、適合した、正確な、ぴったりの、手ごろな、ふさわしい、好都合の、合致した 【形-4】《文末に付けて》~ですよね?、~でしょ?◆自分の言ったことが正しいかどうかを確認するため、あるいはコメントを求めるために付ける 【形-5】正しい側の、表の、表側の、表面の、上面の、正面の 【形-6】まっすぐな 【形-7】直角の、垂直の 【形-8】人間的に正しい、信用できる、正義の、善良な 【形-9】《野球》右翼の、ライトの 【形-10】政治的に右側の、右翼の、右派の 【形-11】本当の◆【同】
\\	【形-12】だいじょうぶで◆【同】
\\	【副-1】右に 【副-2】すぐに、直ちに 【副-3】完全に、すっかり 【副-4】うまい具合に、都合良く 【副-5】本当に◆【同】
\\	【自動-1】正しい位置に戻る 【自動-2】直立する 【他動-1】~を正しい位置に戻す 【他動-2】~を直立させる 【他動-3】~を立て直す、改善する 【他動-4】~を正しく取り扱う 【他動-5】(不正)を正す
\\	1年	
\\	【名-1】走ること、走行 【名-2】競走 【名-3】トランプの一続き 【名-4】取り付け 【形】(~によって)運営[経営]されている◆【参考】
\\	【自動-1】走る 【自動-2】急ぐ 【自動-3】(プログラムなどが)起動する、作動する 【自動-4】競争に参加する、立候補する、選挙戦に出馬する 【自動-5】通う、運行する 【自動-6】流れる 【自動-7】鼻が出る 【自動-8】続く 【自動-9】伝わる、広まる 【自動-10】(ストッキングが)伝線する 【自動-11】(色が)にじむ、溶解する、落ちる◆色・化粧・染料・塗料などが 【自動-12】転がる 【他動-1】~を動かす 【他動-2】~を運営する、経営する、管理する 【他動-3】(実験・テストなど)を行う 【他動-4】~を立候補させる、選挙戦に出馬させる 【他動-5】(費用)がかかる 【他動-6】~を放映する、放送する、(ビデオ)を再生する 【他動-7】~を掲載する、印刷する 【他動-8】(プログラム)を実行する 【他動-9】追う 【他動-10】ぶつける 【他動-11】水を流す
\\	1年	
\\	【名】《料理》サラダ◆緑菜、果物にドレッシングをかけたもの。酢が入っている時にはワインは添えない。ワインにはリンゴ酢が入っているので。◆〈フランス語〉
\\	1年	
\\	【名】土曜日◆【略】
\\	【副】土曜日に
\\	1年	
\\	【名】のこぎり 【他動】~をのこぎりで切る
\\	1年	
\\	【名-1】学校、教習所◆【略】
\\	【名-2】学部 【名-3】授業、学校教育 【名-4】校舎 【名-5】全校生徒 【名-6】学派、流派 【名-7】群れ 【他動】~に教え込む
\\	1年	
\\	【名-1】(自然)科学 【名-2】科学的知識 【名-3】技(術)、術
\\	1年	
\\	【名-1】季節 【名-2】時期 【名-3】良い時期、適当な時期、旬、出盛り、最高潮、シーズン 【名-4】(スポーツの)シーズン戦[時期・活動期] 【他動】~に風味をつける、味付けをする、加味する◆【同】
\\	1年	
\\	"【名-1】第2番目の人[もの]、第2日、第2の人[もの] 【名-2】介添人 【名-3】瞬間、ちょっとの間、一刻◆【略】
\\	【名-4】《単位・時間》秒◆【略】
\\	【名-5】《単位・角度》秒◆【同】1/60 
\\	◆【略】
\\	◆角度の何度、何分、何秒の「秒」。
\\	31では何分・何秒を使わずに、度だけの使用を推奨(例:5゜30'45""を使わず 
\\	5125゜と表示)。 【名-6】《単位・天文》秒◆【同】
\\	◆【略】
\\	【形-1】第2の、2番目の◆【略】2
\\	【形-2】もう一つの 【副】第2に、第二に 【他動-1】~に賛成する、~を支持する 【他動-2】~を後援する、援助する、助ける
\\	1年	
\\	【名】会うこと、見ること 【自動-1】見える 【自動-2】(相手の考え・意図・説明が)見える、理解する 【他動-1】~を見る、~が見える、目につく、目撃する、見物する 【他動-2】~に会う 【他動-3】(相手の考え・意図・説明)が見える、~を理解する 【他動-4】~と考える 【他動-5】~を予想する、見込む 【他動-6】~を調べる 【他動-7】~と交際する、付き合う 【他動-8】~に気を付ける、配慮する 【他動-9】~を経験する 【他動-10】読む
\\	1年	
\\	【名】9月
\\	1年	
\\	【名】7、7個、7人、7歳、7時 【形】7の、7個の、7人の、7歳の
\\	1年	
\\	【名】17、17個、17人、17歳、17時 【形】17の、17個の、17人の、17歳の
\\	1年	
\\	【名】7分の1、第7(番目)、7番目のもの[人] 【形】第7の、7番目の、7分の1の
\\	1年	
\\	【名】70、70個、70人、70歳 【形】70の、70個の、70人の、70歳の
\\	1年	
\\	【名】雌、女性、メスの動物 【代名】彼女は、彼女が 【形】雌の、女性の、女性的な
\\	1年	
\\	【名-1】店、売店、商店、売り場 【名-2】仕事場 【自動-1】買い物をする 【自動-2】探す、検討する 【他動】~へ買い物に行く、~で商品を見て回る、買い物をする
\\	1年	
\\	【名-1】歌手、シンガー、歌う人、声楽家、歌姫 【名-2】詩人 【名-3】鳴き鳥 【名-4】焦がす人、毛焼きをする人
\\	1年	
\\	【名-1】姉妹、姉、妹 【名-2】シスター◆【略】
\\	【名-3】親しい女の人
\\	1年	
\\	【自動-1】座る 【自動-2】巣につく 【自動-3】(物がある場所に)ある、使われずに置かれている 【自動-4】動かずに休んでいる、(物事の動きを)そのままにしておく 【自動-5】(法廷・議会・裁判所などが)開会[開廷]する、議事を行う 【他動-1】~に座らせる 【他動-2】(馬)を乗りこなす 【他動-3】(試験)を受ける
\\	1年	
\\	【名】6、6個、6人、6歳、6時 【形】6の、6個の、6人の、6歳の
\\	1年	
\\	【名】16、16個、16人、16歳、16時 【形】16の、16個の、16人の、16歳の
\\	1年	
\\	【名】6分の1、第6(番目)、6番目のもの[人] 【形】第6の、6番目の、6分の1の
\\	1年	
\\	【名】60、60個、60人、60歳 【形】60の、60個の、60人の、60歳の
\\	1年	
\\	【形-1】サイズ[粒・なり]が小さい、小規模の、小作りの 【形-2】小粒[小型](種)の 【形-3】つまらない、取るに足りない 【形-4】狭量な 【形-5】《料理》~小_個 【形-6】少ない 【副】小さく
\\	1年	
\\	【接】~するように 【副-1】そのように、そう、そのようにして 【副-2】それほど 【副-3】それで、だから、従って 【副-4】とても
\\	1年	
\\	【名】《スポーツ》サッカー、蹴球
\\	1年	
\\	【代名-1】いくらか、多少 【代名-2】一部(の人々) 【形-1】いくらかの、いくつかの、いくばくかの 【形-2】ある 【形-3】大した、なかなかの 【形-4】かなりの(量の)、相当な 【副-1】およそ、約 【副-2】いくぶん、多少、いくらか 【副-3】なかなか、ずいぶん
\\	1年	
\\	【副】時々、たまに(は)
\\	1年	
\\	【副-1】もうすぐ、間もなく、程なく 【副-2】早く 【副-3】すぐに
\\	1年	
\\	【形-1】すまないと思って、すまない(と思う)、すまなく思って、心苦しく思う 【形-2】後悔する 【形-3】気の毒で、遺憾とする、残念に思う、気の毒な、残念に思って 【形-4】哀れな、惨めな、嘆かわしい、情けない、悲しい
\\	1年	
\\	【名】言語、話すこと 【自動-1】話す、口を開く 【自動-2】演説する、講演する 【自動-3】通常、話し手が意見あるいは考えを述べるときに用いる動詞。 【他動】~を話す、~について口を開く◆通常、話し手が意見あるいは考えを述べるときに用いる動詞。
\\	1年	
\\	【名-1】スポーツ、運動競技 【名-2】娯楽 【名-3】戯れ 【名-4】付き合いのいい人、勝負にこだわらない人 【名-5】《男性間の呼び掛け》あんた、君 【形】スポーツの 【自動-1】スポーツをする 【自動-2】楽しく過ごす、遊ぶ 【他動-1】スポーツをして(時)を過ごす 【他動-2】~を見せびらかす、誇らしげに示す
\\	1年	
\\	【名-1】春 【名-2】泉、湧水 【名-3】スプリング、バネ、ぜんまい 【自動-1】跳ねる、跳ぶ、はじく 【自動-2】(~から)生じる 【自動-3】(~から)湧き出る 【他動-1】スプリング[バネ]を付ける 【他動-2】~を飛び立たせる 【他動-3】~を裂く、割る
\\	1年	
\\	【名-1】スタンド 【名-2】立場、態度、見解、姿勢 【名-3】証人席
\\	【名-4】立つ位置[場所] 【名-5】見物席 【名-6】根拠 【名-7】(立ち木などの)木立 【自動-1】立っている 【自動-2】立ち上がる、立つ 【自動-3】~の身長である、身長は~である 【自動-4】有効である 【自動-5】(記録が)破られない 【自動-6】現時点で~である 【他動-1】~を立たせる 【他動-2】~を我慢する 【他動-3】~をおごる
\\	1年	
\\	【名-1】出発、開始、きっかけ、スタート、始まり、最初の部分、出だし、出足、出発点、発進、着手、発端、先頭、起動、立ち上げ、機先 【名-2】便益、有利 【名-3】発作、衝動的な動き 【名-4】びっくり、驚いてハッとすること 【自動-1】出発する、歩き出す 【自動-2】起こる、始まる、始める、着手する、事を起こす 【自動-3】ビクッ[ギクリ・ギョッ・ハッ]とする、ビクッと動く、跳び上がる、素早く逃げ去る◆驚き・恐怖などで 【自動-4】飛び出す 【自動-5】動き始める、運動を始める、機械が動く 【他動-1】~を~し始める、開始する、始める、(事業)を起こす 【他動-2】(獲物)を飛び立たせる、狩り出す 【他動-3】《野球》~に先発する、登板する
\\	1年	
\\	【名-1】駅、ステーション 【名-2】地域本部 【名-3】局、署 【名-4】身分、地位 【他動】~を配置する、駐在させる、配属する
\\	1年	
\\	【名-1】滞在 【名-2】(物理的な)支え 【名-3】(精神的な)支え、頼り 【名-4】《法律》(執行などの)延期、停止、中止 【自動-1】ある状態が継続する、~のままでいる 【自動-2】とどまる 【自動-3】滞在する 【他動-1】~を止まらせる、食い止める 【他動-2】~を延期する 【他動-3】~の終わりまでいる、持ちこたえる
\\	1年	
\\	【名】通り、街路、道端、街頭、町◆【略】
\\	【形】通りの
\\	1年	
\\	【名-1】学生 【名-2】研究家、研究者
\\	1年	
\\	【名-1】勉強、研究 【名-2】調査、検討 【名-3】書斎 【名-4】練習曲 【自動-1】勉強する、研究する、学ぶ 【自動-2】観察する、調査する 【他動-1】~を詳しく調べる、観察する、検討する 【他動-2】勉強する、研究する、学ぶ
\\	1年	
\\	【名-1】主題、題目、テーマ、話題、題材 【名-2】教科、科目、学科 【名-3】主語、主部 【名-4】対象、被写体、被検者、被験者、被害届の対象となる人 【名-5】臣下、家来 【名-6】臣民、非統治者◆王、支配者に対しての 【形-1】支配下にある、支配を受ける、支配を受けている、~に従属している 【形-2】~にかかりやすい 【他動-1】~を支配する、支配下に置く、服従させる 【他動-2】~を受けさせる
\\	1年	
\\	【名】夏
\\	1年	
\\	【名】日曜日◆【略】
\\	【副】日曜日に
\\	1年	
\\	【形-1】日のよく照る、明るく日が照る、日当たりの良い 【形-2】明るい、快活な
\\	1年	
\\	【形-1】確信して、確信している、固く信じている、自信がある、確かな 【形-2】きっと~する 【形-3】信頼できる、当てになる 【副-1】確かに 【副-2】《返答》承知しました
\\	1年	
\\	【名】ひと泳ぎ 【自動-1】泳ぐ 【自動-2】浮く、浮かぶ、浮遊する 【自動-3】浸る、溢れる 【自動-4】フラフラする、目まいがする
\\	1年	
\\	【人名】シド、シドニー 【地名】シドニー◆オーストラリア。◆ニューサウスウェールズ州 
\\	の州都。◆【略】
\\	【分節】
\\	1年	
\\	【名-1】テーブル、台、仕事台 【名-2】表 【名-3】委員会、会議 【形】テーブルの、台状の、テーブルの上に置ける、卓上の、食卓用の、食事の 【他動-1】~をテーブルの上に置く 【他動-2】~を表に記入する 【他動-3】~を棚上げする、保留する、延期する、後回しにする、見送る、延ばす、持ち越す◆【同】
\\	【他動-4】~を提出する 【他動-5】(議案として法律)を審議する
\\	1年	
\\	【名-1】取ること、取得、獲得、取ったもの、取られたもの、利益、収穫(高)、取り分、所得 【名-2】(映画撮影の)テーク、撮影ショット 【名-3】見解、見方、解釈 【自動-1】取る、捕える 【自動-2】(植物が)根づく 【自動-3】(薬などが)効く 【自動-4】(火が)つく 【自動-5】人気を博す 【他動-1】~を(自分の意志で手に)取る、(自分の領域・縄張り・体内に)取り込む 【他動-2】~を買う、選ぶ 【他動-3】~を(場所へ)連れていく、搬送する、持ち込む 【他動-4】(人目・関心)を引く、~をうっとりさせる、魅了する 【他動-5】~を解釈する、理解する、受け取る、みなす 【他動-6】~を要する、必要とする 【他動-7】(薬)を飲む、服用する 【他動-8】(乗り物)に乗る、(乗り物)を利用する 【他動-9】~を例として挙げる 【他動-10】(人)をやっつける、(人)の命を奪う 【他動-11】~を食べる 【他動-12】~を受け入れる
\\	1年	
\\	【名-1】話すこと、話、話題 【名-2】《複数形で》会談、会議、協議、交渉、話し合い 【自動-1】話す、口を開く、論じる 【自動-2】ものを言う、効力を発する
\\	1年	
\\	【形-1】背の高い、丈が高い、長い、背の高い人のための 【形-2】法外な、大げさな
\\	1年	
\\	【名-1】お茶、茶、茶葉 【名-2】午後のお茶◆【同】
\\	【名-3】〈英・豪〉夕食◆紅茶とサンドイッチ軽食の午後食
\\	。 【名-4】茶会◆紅茶、茶などでもてなすパーティ。
\\	1年	
\\	【自動】教える、教師をする◆恒久的な知識の伝達、つまり学習の場合にのみ用いる動詞。「駅までの道を教えてください」という場合の「教える」は 
\\	を使う
\\	【他動】~を教える、指導する
\\	1年	
\\	【名】先生、教師
\\	1年	
\\	【名-1】チーム、仲間、グループ、組、団、班 【名-2】一組の馬[牛・犬]◆2頭以上の牽引動物を表す 【形】チームワークに専念する 【自動】チームを構成する、協力し合う 【他動-1】~をつなげる、結び付ける、組にする、組み合わせる 【他動-2】~を下請けさせる
\\	1年	
\\	【他動-1】~に話す、言う、告げる、教える、伝える◆相手からの返答を期待せずに一方的に話す。従って、「彼に黙るように言った」という場合は、一方的な命令なので 
\\	となり、
\\	とはならない。 【他動-2】~にきっぱり言う、命じる 【他動-3】~を見分ける、分かる
\\	1年	
\\	【名-1】10、10個、10人、10歳、10時 【名-2】最高のもの[人・美人]、10点満点 【形】10の、10個の、10人の、10歳の
\\	1年	
\\	【名】テニス、庭球
\\	1年	
\\	【名】10分の1、第10(番目)、1割、10番目のもの[人] 【形】第10の、10番目の、10分の1の
\\	1年	
\\	【名】感謝、謝意、謝辞 【他動-1】~に感謝する、~に礼を言う 【他動-2】~にお願いする
\\	1年	
\\	【代名-1】それ、あれ 【代名-2】そんなこと、そのようなこと 【接】~ということ 【形】その、あの 【副】それくらいに、それほど
\\	1年	
\\	"【副-1】かえって、ますます◆最上級や比較級の形容詞、序数、一部の形容詞
\\	など)の前に付ける 【副-2】~すればますます…
\\	比較級、
\\	比較級 【定-1】その、例の◆文脈や状況から、話し手聞き手双方で認知され得る名詞に付ける 
\\	""私たちはどんなときに 
\\	を使うのか知らなかった。◆この例文では、それぞれの冠詞がかっこでくくられることによって一種の名詞のように扱われている。 【定-2】◆限定を受ける名詞の前に付ける 【定-3】◆一つしか存在しないものの前に付ける 【定-4】◆船舶・建物・鉄道・山脈・群島・半島・砂漠・河川・海・海峡・新聞・雑誌・連邦国家・連合[連邦]などに付ける。◆【注意】駅名には 
\\	を付けない 【定-5】一番素晴らしい、最高の 【定-6】~な人々
\\	形容詞(または現在分詞・過去分詞)の形で、関連する集団を表し、通例、複数扱い◆【注意】
\\	(故人)、
\\	(任命された人)など、時に単数扱いになるものもある 【定-7】~なもの
\\	形容詞(または分詞)の形で、関連する抽象名詞の集まりを表し、通例単数扱い 【定-8】~というもの
\\	単数名詞で種族全体を表す 【定-9】~家、~一家
\\	姓の複数形 【定-10】◆楽器やダンスの前に慣用的に付ける。口語で、楽器の 
\\	はしばしば省略される 【定-11】◆年代などの前に付く 【定-12】◆所有格の代わりに用いる。直接の関心は目的語の名詞にあり、
\\	の付いた名詞には二次的な関心が向けられている 【定-13】
\\	単位》で計量単位を表す 【定-14】◆時に 
\\	が付いた名詞がイメージする動作や行為を表す 【定-15】よく知られた、有名な◆人名の前に付ける 【定-16】◆爵位・称号に付ける 【定-17】◆(歴史上の)人物名で用いられる 【定-18】十分な◆否定文で用いられる
\\	1年	
\\	【代名】彼・彼女らの、それらの
\\	1年	
\\	【代名】彼らを(に)、彼女らを(に)
\\	1年	
\\	【形】そのときの、当時の 【副-1】あのとき、そのとき 【副-2】その次に、それから、次に、その後、間もなく 【副-3】その結果、ひいては
\\	1年	
\\	【名】そこ、あそこ 【副】そこで、そこに、そこへ
\\	1年	
\\	【代名】これら(の)、これら
\\	の複数形 【形】これらの、こうした、このような、こういった、この
\\	1年	
\\	【代名】彼・彼女らは、それら、彼ら
\\	1年	
\\	【名-1】もの、物体 【名-2】衣服、衣類、道具、財産、所有物 【名-3】こと、行為、仕事 【名-4】物事、事物、風物 【名-5】《複数形》状況、事情、事態 【名-6】考え、意見、特別な感情 【名-7】大切なこと、要点、ポイント 【名-8】〈米俗〉好きな[得意な・気に入った]こと[もの]
\\	1年	
\\	【自動】考える、思う、熟考する 【他動】~を考える[思う・熟考する]
\\	1年	
\\	【名】第3のもの[人]、3分の1、第3(番目)、3番目のもの[人]、3等級 【形】第3の、3分の1の、3番目の、3等級の 【副】第3に、3番目に
\\	1年	
\\	【名】13、13個、13人、13歳、13時 【形】13の、13個の、13人の、13歳の
\\	1年	
\\	【名】30、30個、30人、30歳 【形】30の、30個の、30人の、30歳の
\\	1年	
\\	【代名】これ、この 【副】これくらいに、こんなに、これほどに
\\	1年	
\\	【代名】それら・あれら(の人[もの])、それら
\\	の複数形 【形】それらの、あの
\\	1年	
\\	【名】3、3個、3人、3歳、3時 【形】3の、3個の、3人の、3歳の
\\	1年	
\\	【名】木曜日◆【略】
\\	【副】木曜日に
\\	1年	
\\	【名-1】時間、時、時間帯、歳月 【名-2】時代 【名-3】期間、時期、当時、頃 【名-4】刑期 【名-5】時刻、時点 【名-6】~回、~度 【名-7】~倍 【名-8】適切な時機、ふさわしい時、潮時、好機、チャンス 【名-9】速度、テンポ、拍子 【他動-1】~の時刻を決める、~の時期を選ぶ、~の時間を計る 【他動-2】
\\	を~するように調節する
\\	1年	
\\	【形-1】疲れた 【形-2】うんざりした、飽きた 【形-3】古くさい、使い尽くした
\\	1年	
\\	【前-1】(物事の進行・人の気持ち・意志・考え・状態・運命・目的などが)~ヘ(の)、~の方向に向かって、~を目的(地)として 【前-2】~すること◆不定詞の名詞用法を導く 【前-3】~する、~すべき、~するための◆不定詞の形容詞用法を導く 【前-4】~するために、~して結果◆不定詞の副詞用法を導く
\\	1年	
\\	【名-1】今日、本日 【名-2】現今、現代、目下 【副】今日は、現在は
\\	1年	
\\	【副-1】ともに、一緒に、一堂に集まって、連れ立って 【副-2】全体的に、総合して 【副-3】同時に、続けて 【副-4】きちんと
\\	1年	
\\	【副-1】~もまた、加えてさらに 【副-2】標準[基準・平均的]レベルを超えて、過度に、あまりに、~すぎる
\\	1年	
\\	【動】
\\	の過去形 
\\	トゥック
\\	1年	
\\	【名-1】列車、汽車 【名-2】列、行列 【名-3】結果、余波 【自動-1】トレーニングする、練習する、訓練する、訓練を受ける 【自動-2】列車で旅行する、列車で行く 【他動-1】~をしつける、教え込む、教育する、訓練する、養成する、~の身体を慣らす 【他動-2】(カメラ・銃・ライトなど)を向ける
\\	1年	
\\	【名-1】木 【名-2】系図、階層
\\	1年	
\\	【名】試しにやってみること、試み 【自動】努める、やってみる、試みる、しようと努力する
\\	【他動-1】~を試す、やってみる、試みる、試験する、~に当たってみる 【他動-2】~を裁判する、裁判にかける、審理する 【他動-3】~の忍耐力を試す、~を試練にさらす、~に苦難を与える、~をつらい目に遭わす、~を悩ます
\\	1年	
\\	【名】火曜日
\\	1年	
\\	【名-1】回転、回旋、旋回 【名-2】順番、番 【名-3】ショック、衝撃、発作 【名-4】(競技場の)コーナー 【自動-1】曲がる、回転する、回動する、向きを変える、振り返る 【自動-2】変化する、(変わって~に)なる、転身する 【他動-1】~の方向を変える 【他動-2】~を回転させる 【他動-3】~を曲がる 【他動-4】(年齢が~才)になる 【他動-5】~を変化させる、訳す 【他動-6】~を熟考する 【他動-7】~をろくろで作る
\\	1年	
\\	【名】テレビ◆【語源】
\\	の略。◆【用法】口語では
\\	よりも
\\	の方が圧倒的に多く使われる。「テレビを見る」の“見る”に相当する動詞は、
\\	よりも 
\\	の方が圧倒的に多く使われる。
\\	というように 
\\	の前に 
\\	を付けることはまれである。
\\	1年	
\\	【名】12分の1、第12(番目)、12番目のもの[人] 【形】第12の、12番目の、12分の1の
\\	1年	
\\	【名】12、12個、12人、12歳、12時 【形】12の、12個の、12人の、12歳の
\\	1年	
\\	【名】20、20個、20人、20歳、20時 【形】20の、20個の、20人の、20歳の
\\	1年	
\\	【名】2、2個、2人、2歳、2時 【形】2の、2個の、2人の、2歳の
\\	1年	
\\	【前-1】~の(真)下に、(基準としているものの)真下に、~下部に 【前-2】~未満 【前-3】~に基づいて 【前-4】~の下で 【前-5】~を受けて 【前-6】(書物の)~の欄[項目]に 【副】~未満・以下の
\\	1年	
\\	【名-1】上り、上昇 【名-2】元気、幸せ 【前】~の上方へ 【形-1】上に向かう、上への 【形-2】時間が過ぎて、時間が切れて、終わりで 【形-3】起きている、作動している、(機器の)電源がオンになってる 【形-4】《野球》打席に立って 【副-1】上方へ、上って、上へ、上に、上昇して 【副-2】直立して、起きて 【副-3】北へ 【副-4】近づいて、近くに、こちらに 【副-5】《話》起こって、進行中で 【自動】上昇する 【他動】~を上げる、上昇させる
\\	1年	
\\	【代名】われわれを(に)
\\	の目的格
\\	1年	
\\	【名-1】使うこと、利用(法)、使用 【名-2】用途、使途 【名-3】役立つこと、効用 【他動-1】~を使う、利用する、生かす、働かせる 【他動-2】(利己的な目的に人)を利用する◆【同】
\\	【他動-3】
\\	の形で》~が手に入るとありがたい、~が必要である、~が欲しい 【他動-4】~を飲む 【他動-5】~を消費する
\\	1年	
\\	【副】たいてい、いつもは、普通、通例、通常◆【略】
\\	1年	
\\	【名-1】休暇、休廷期間、休日 【名-2】辞職、立ち退き 【自動】休暇を取る、休暇を過ごす
\\	1年	
\\	【形】《強調》まさに、まさしく、~でも、~だけでも 【副-1】とても、まさに、極めて、大変、非常に 【副-2】
\\	あまり~ない、大して~ない、さほど~ない、ただ~だけ
\\	1年	
\\	【名】ビデオ、テレビ、映像、ビデオテープ 【形】ビデオの、テレビの、映像の
\\	1年	
\\	【名】待機、待つこと、待ち時間 【自動-1】(行動に移る前に)じっとしている、待つ 【自動-2】給仕する 【他動-1】~を待つ 【他動-2】~を遅らせる、延期する◆【類】
\\	1年	
\\	【名-1】歩行、歩くこと、徒歩、散歩 【名-2】歩き方 【名-3】歩道 【名-4】職業、身分、部門、分野 【名-5】暮らしぶり、世渡り 【名-6】たやすいこと、簡単なこと 【名-7】《野球》四球(による出塁)、フォアボール 【名-8】家畜の飼育場、囲い 【自動-1】歩く、歩行する、散歩する 【自動-2】うろつく 【自動-3】世を渡る 【自動-4】《野球》四球で歩く、四球を選ぶ 【他動-1】~を歩かせる、歩いて送る、散歩させる 【他動-2】~を四球で1塁に歩かせる
\\	1年	
\\	【名-1】必要なもの 【名-2】欲望 【名-3】欠乏、不足 【名-4】困窮、貧困 【名-5】必要 【他動-1】~が欲しい、~を望んでいる、必要とする、~したいと思う
\\	【他動-2】~に用がある 【他動-3】~した方が良い、~すべきだ
\\	1年	
\\	【名-1】腕時計 【名-2】観察 【名-3】見張り、警戒、警備員 【他動-1】(意識して)~をじっと見る、観察する 【他動-2】~に注意する、気を付ける 【他動-3】~を見張る、見守る、~の世話をする 【他動-4】~を期待して待つ、待機する
\\	1年	
\\	【1-名-1】道、進路、通り道、道のり、道路、方向、回路 【1-名-2】方法、やり方、手段、方途、様式 【1-名-3】現状、状態 【1-名-4】習慣、風習、しきたり、習わし、慣習、癖、世の常 【2-副】はるかに、ずっと、かなり
\\	1年	
\\	【代名】われわれは、私たちは、人は、朕は
\\	1年	
\\	【名-1】天気、気候、天候 【名-2】浮き沈み 【名-3】風雨 【形】風上の、風上に向かった 【自動-1】外気で変色する、外気で変化する、色あせる、風化する 【自動-2】外気に耐える、風雨に耐える、悪天候に耐える 【他動-1】(嵐・困難など)を切り抜ける、乗り切る、乗り越える 【他動-2】~の風上の方に船を進める、風上を通る 【他動-3】~を風雨・外気にさらす、変色させる、脱色させる、乾かす、風化させる、干す 【他動-4】~に傾斜をつける
\\	1年	
\\	【名】水曜日◆ドイツ語の
\\	の意味。
\\	は北欧神話の最高神(北欧語のオーディン
\\	。◆【略】
\\	【副】水曜日に
\\	1年	
\\	【名】週、1週間
\\	1年	
\\	【名】歓迎、歓迎の言葉、歓待 【形-1】ありがたい、うれしい、歓迎される、願ってもない、耳よりな 【形-2】自由に使ってよい 【間投】ようこそ 【他動】~を歓迎する、喜んで迎える[受け取る]、歓待する、もてなす
\\	1年	
\\	【1-形】(体の)調子が良い、健康である、健在である、壮健である 【1-副-1】上手に、うまく、よく 【1-副-2】親密に 【1-副-3】はっきりと、たぶん 【1-副-4】満足に、落ち着いて 【1-副-5】健康に 【1-副-6】かなり、ずいぶん、十分に 【1-間投-1】やれやれ、よろしい 【1-間投-2】ええと、うーん、そうですねー、さて、おや、それで、なるほど、そうなのです◆話のつなぎ 【2-名】井戸、鉱泉、泉、源泉 【2-自動】沸き出る、噴出する 【2-他動】~を沸き出させる
\\	1年	
\\	【動】
\\	の過去形◆【語源】古語 
\\	(行く)の過去形より 
\\	ウェント
\\	1年	
\\	【代名-1】何 【代名-2】~するもの[こと]◆関係代名詞 【形-1】何の、どれほど、どんな、何という 【形-2】全部の◆関係形容詞
\\	1年	
\\	【代名】いつ 【接】~するときに、~するときは、~する場合は、~ならば 【副-1】いつ 【副-2】(~する)時、(そして)その時◆関係副詞
\\	1年	
\\	【代名】どこ 【接】~する所に、~する所はどこへでも、~する場合には、~するところの 【副】どこに、どこで、どんな点で
\\	1年	
\\	【代名-1】どちら(の)、どちらか 【代名-2】~するところの、そしてそれ[そのこと]は◆関係代名詞
\\	1年	
\\	【名-1】白い運動着、白衣、白布、白い服地 【名-2】ホワイトソース 【名-3】白、白色、白絵の具、白色染料、白色顔料 【名-4】(卵の)白身、卵白、精白小麦粉、白パン 【名-5】白目 【名-6】白人 【形-1】白い、白の 【形-2】白髪の 【形-3】空白の、白紙の 【形-4】白衣の 【形-5】雪の降る、雪の積もった 【形-6】激怒して蒼白で、青ざめた、青白い 【形-7】白色人種の、白人の、白人支配の、白人専用の 【形-8】ミルク[クリーム]を入れた 【形-9】ワインが白の、透明な、無色の 【形-10】〈米俗〉正直な、誠実な、公明正大な、信用の置ける 【形-11】善意の、無害な、罪のない 【形-12】白熱の、激しい 【形-13】反革命派の
\\	1年	
\\	【代名-1】誰、誰が、誰を 【代名-2】~する人、そしてその人は◆関係代名詞
\\	1年	
\\	【代名-1】誰の、誰のもの 【代名-2】その(~が)~する(ところの)、そしてその人の◆関係代名詞
\\	1年	
\\	【名】理由 【副】なぜ、どのような理由で、何のために、どうして 【間投-1】おや、まあ、あら◆驚きを示す 【間投-2】もちろん◆承諾を示す
\\	1年	
\\	【名】冬、末期 【自動】冬を過ごす
\\	1年	
\\	【名】願い、望みの物、願望、希望、希望の点、祈り、祈願、請い、望みのもの、要請、祝福の言葉 【他動-1】(可能性があまりないこと)を望む 【他動-2】~を祈る
\\	1年	
\\	【前-1】~と一緒に、~と一体になって、~とともに、~を相手にして、~と一緒に暮らして 【前-2】(組織・団体など)と一体で、~に雇われて、~に勤務して、~の一員で 【前-3】~と考え方が一体となって、~と同意見で、~に賛成して、(人の話を)理解して、~に味方して◆【反】
\\	【前-4】~と時間的に一体となって、~と同時に、~につれて、~が同時に起こって、~しながら、~した状態で 【前-5】~と自分の体が一体になって、~を身につけて、~を備えて、(現在)~を持っている、手元に~がある、~を持って、~があれば、(道具などを)使って 【前-6】~と物理的に一体となって、~が付いて、~を含んで◆【反】
\\	【前-7】~が原因で、~のために、~のせいで、その上
\\	1年	
\\	【名】女、女性、婦人 【他動】~をウーマン[女]と呼ぶ
\\	1年	
\\	【名-1】労働、仕事、研究、作業、労力 【名-2】職業、勤め口 【名-3】業績、作業結果 【名-4】作品、著作物 【名-5】仕掛け◆複数形で 【名-6】仕業 【自動-1】働く、作業する、勉強する、取り組む 【自動-2】勤めている 【自動-3】徐々に動く、努力して進む 【自動-4】正常に[正しく・目的どおりに]機能[動作]する、(テレビ・ラジオなどが)つく、効く、効き目がある、功を奏する、役に立つ、使える、うまくいく 【他動-1】~を動かす 【他動-2】~をもたらす 【他動-3】~を働かせる、働かせて…(の状態)にする◆【用法】
\\	~…の形を取る。…には形容詞または副詞(句)が来る。 【他動-4】~を生む、引き起こす、生じさせる、もたらす 【他動-5】~を経営する、管理する、~の係[担当]である 【他動-6】~を徐々[次第]に…の状態にする◆【用法】
\\	…の形を取る 【他動-7】~に(…するよう)働きかける[仕向ける]◆【用法】
\\	の形を取る 【他動-8】~を(苦心して)解く、解決する、じっくりと考える 【他動-9】細かな手作業で~を行う、(細工)を行う
\\	1年	
\\	【名-1】世界 【名-2】世間の人々、世界の人々、世の中、世間 【名-3】~界 【名-4】宇宙
\\	1年	
\\	【名】(ワオ!と叫ぶほどの)大成功 【他動】あっと言わせる、大喜び[感動・熱狂・エキサイト]させる、大向こうをうならせる◆【類】
\\	【間投】ワオ!、わあ!◆喜び・驚きなどを表す 
\\	ワウ
\\	1年	
\\	【自動】手紙で知らせる、手紙を書いて送る、手紙を書く、便りをする、書き送る 【他動-1】~を書く、執筆する 【他動-2】《コ》(プログラム)を書く[作成する] 【他動-3】《コ》~を書き込む◆記録媒体にデータを 【他動-4】~を評する
\\	1年	
\\	【名-1】年、1カ年◆【略】
\\	【名-2】年齢、歳 【名-3】長い年月 【名-4】年度、学年 【名-5】時代◆【用法】通例、複数形
\\	1年	
\\	【名-1】黄色◆【略】
\\	【名-2】臆病 【形-1】黄色の 【形-2】臆病な 【形-3】嫉妬深い 【他動】~を黄色にする
\\	1年	
\\	【名】イエスという返事、「はい」という返答[返事・言葉・承諾]、肯定の返答[返事・言葉・承諾] 【副】はい、そうです、そのとおり
\\	1年	
\\	【名】昨日 【形】昨日の 【副】昨日は
\\	1年	
\\	【代名-1】あなた(たち)、貴方、貴殿 【代名-2】お客様◆店員が客に向かって 
\\	と言う場合
\\	1年	
\\	【代名】あなたの、あなたたちの
\\	の所有格
\\	1年	
\\	【代名】あなた(たち)のもの
\\	1年	
\\	【名-1】ゼロ、0、零、零度、零点、空、無 【名-2】何の役にも立たないもの、虫けら 【名-3】銃の照準 【名-4】ゼロ戦 【他動】~をゼロにする、ゼロに合わせる◆【参考】
\end{CJK}
\end{document}