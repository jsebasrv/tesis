\documentclass[8pt]{extreport} 
\usepackage{hyperref}
\usepackage{CJKutf8}
\begin{document}
\begin{CJK}{UTF8}{min}
\\	左	
\\	サ	ひだり	サ	左下	ひだりした	
\\	左手	ひだりて	
\\	左足	ひだりあし	
\\	左方	さほう	
\\	右	
\\	ウ ・ ユウ	みぎ	ウ	右上	みぎうえ	
\\	左右	さゆう	
\\	右から左へ	みぎからひだりへ	
\\	右方	うほう	
\\	石	
\\	セキ ・ シャク ・ コク	いし	セキ	小石	こいし	
\\	宝石	ほうせき	
\\	一石二鳥	いっせきにちょう	
\\	石	いし	
\\	若	
\\	ジャク ・ ニャク	わか.い ・ わか- ・ も.しくわ ・ も.し
\\	ジャク	若い	わかい	
\\	若年	じゃくねん	
\\	若者	わかもの	
\\	苦	
\\	ク	くる.しい ・ -ぐる.しい ・ くる.しむ ・ くる.しめる ・ にが.い ・ にが.る	ク	苦しい	くるしい	
\\	苦味	にがみ	
\\	苦手	にがて	
\\	在	
\\	ザイ	あ.る	ザイ	不在	ふざい	
\\	在校生	ざいこうせい	
\\	駐在	ちゅうざい	
\\	在日	ざいにち	
\\	存	
\\	ソン ・ ゾン		ソン	存在	そんざい	
\\	存じる	ぞんじる	
\\	共存	きょうぞん	
\\	干	
\\	カン	ほ.す ・ ほ.し- ・ -ぼ.し ・ ひ.る	カン	干潮	かんちょう	
\\	干す	ほす	
\\	若干	じゃっかん	
\\	潮の干満	しおのかんまん	
\\	刊	
\\	カン		カン	刊行	かんこう	
\\	週刊誌	しゅうかんし	
\\	夕刊	ゆうかん	
\\	汗	
\\	カン	あせ	カン	発汗する	はっかんする	
\\	汗をかく	あせをかく	
\\	発汗する	はっかんする	
\\	汗をかく	あせをかく	
\\	竿	
\\	カン	さお	カン	竿竹	さおだけ	
\\	物干し竿	ものほしざお	
\\	利	
\\	リ	き.く	リ	左利き	ひだりきき	
\\	利子	りし	
\\	有利な	ゆうりな	
\\	利く	きく	
\\	利用する	りようする	
\\	害	
\\	ガイ		ガイ	利害	りがい	
\\	公害	こうがい	
\\	有害な	ゆうがいな	
\\	害する	がいする	
\\	益	
\\	エキ ・ ヤク	ま.す
\\	エキ	利益	りえき	
\\	益々	ますます	
\\	公益	こうえき	
\\	有益な	ゆうえきな	
\\	溢	
\\	イツ	あふ.れる ・ こぼ.れる ・ こぼ.す	イツ	
\\	溢れる 
\\	あふれる; こぼれる
\\	(あふれる) 
\\	(こぼれる) 
\\	溢れる 
\\	こぼれる	
\\	溢す	こぼす	
\\	割	
\\	カツ	わ.る ・ わり ・ わ.り ・ わ.れる ・ さ.く	カツ	割く	さく	
\\	割れる	われる	
\\	分割する	ぶんかつする	
\\	コップを割る	コップをわる	
\\	三割五分	さんわりごぶ	
\\	憲	
\\	ケン		ケン	憲法	けんぽう	
\\	合憲性	ごうけんせい	
\\	合憲的	ごうけんてき	
\\	羽	
\\	ウ	は ・ は ・ はね	ウ	羽化する	うかする	
\\	一羽	いちわ	
\\	羽音	はおと	
\\	羽根	はね	
\\	翌	
\\	ヨク		ヨク	翌日	よくじつ	
\\	翌朝	よくあさ	
\\	習	
\\	シュウ	なら.う ・ なら.い	シュウ	習得	しゅうとく	
\\	習う	ならう	
\\	習い事	ならいごと	
\\	学習	がくしゅう	
\\	弓	
\\	キュウ	ゆみ	キュウ	弓道	きゅうどう	
\\	弓	ゆみ	
\\	引	
\\	イン	ひ.く ・ ひ.き ・ ひ.き- ・ -び.き ・ ひ.ける	イン	引く	ひく	
\\	引力	いんりょく	
\\	弓を引く	ゆみをひく	
\\	引き出し	ひきだし	
\\	割り引き	わりびき	
\\	強	
\\	キョウ ・ ゴウ	つよ.い ・ つよ.まる ・ つよ.める ・ し.いる	キョウ	根強い	ねづよい	
\\	強いる	しいる	
\\	強調する	きょうちょうする	
\\	強い	つよい	
\\	強大な	きょうだいな	
\\	弱	
\\	ジャク	よわ.い ・ よわ.る ・ よわ.まる ・ よわ.める	ジャク	弱気な	よわきな	
\\	弱点	じゃくてん	
\\	弱い	よわい	
\\	200人弱	にひゃくにんじゃく	
\\	弱肉強食	じゃくにくきょうしょく	
\\	火を弱める	ひをよわめる	
\\	風	
\\	フウ ・ フ	かぜ ・ かざ- ・ -かぜ	フウ	風が弱まった	かぜがよわまった	
\\	和風	わふう	
\\	家風	かふう	
\\	己	
\\	コ ・ キ	おのれ	コ ・ キ	己を信じる	おのれをしんじる	
\\	自己	じこ	
\\	利己	りこ	
\\	自己中心	じこちゅうしん	
\\	記	
\\	キ	しる.す	キ	記入する	きにゅうする	
\\	記す	しるす	
\\	記者	きしゃ	
\\	伝記	でんき	
\\	記事	きじ	
\\	紀	
\\	キ		キ	風紀	ふうき	
\\	紀元	きげん	
\\	紀元前	きげんぜん	
\\	改	
\\	カイ	あらた.める ・ あらた.まる	カイ	改正	かいせい	
\\	改新	かいしん	
\\	改めて	あらためて	
\\	改める	あらためる	
\\	改憲	かいけん	
\\	起	
\\	キ	お.きる ・ お.こる ・ お.こす ・	キ	起こる	おこる	
\\	起源(=起原)	きげん	
\\	起きる	おきる	
\\	起立する	きりつする	
\\	起点	きてん	
\\	姓	
\\	セイ ・ ショウ		セイ	姓名	せいめい	
\\	姓名	せいめい	
\\	由	
\\	ユ ・ ユウ ・ ユイ	よし	ユ	自由	じゆう	
\\	由来	ゆらい	
\\	不自由	ふじゆう	
\\	油	
\\	ユ	あぶら	ユ	石油	せきゆ	
\\	油田	ゆでん	
\\	油	あぶら	
\\	オリーブ油	オリーブゆ	
\\	原油	げんゆ	
\\	井	
\\	セイ ・ ショウ	い	セイ	井戸	いど	
\\	天井	てんじょう	
\\	油井	ゆせい	
\\	井戸	いど	
\\	天井	てんじょう	
\\	油井	ゆせい	
\\	囲	
\\	イ	かこ.む ・ かこ.う ・ かこ.い	イ	囲む	かこむ	
\\	周囲	しゅうい	
\\	取り囲む	とりかこむ	
\\	囲いに入れる	かこいにいれる	
\\	囲まれる	かこまれる	
\\	丼	
\\	ドン	どんぶり	ドン													牛丼	ぎゅうどん	
\\	天丼	てんどん	
\\	丁	
\\	チョウ ・ テイ		チョウ													ラーメン三丁	ラーメンさんちょう	
\\	四丁目	よんちょうめ	
\\	丁度	ちょうど	
\\	寧	
\\	ネイ		ネイ	丁寧な	ていねいな	
\\	丁寧語	ていねいご	
\\	町	
\\	チョウ	まち	チョウ	下町	したまち	
\\	町人	ちょうにん	
\\	室町時代	むろまちじだい	
\\	灯	
\\	トウ	ひ	トウ	電灯	でんとう	
\\	灯光	とうこう	
\\	灯	ひ	
\\	庁	
\\	チョウ		チョウ	官庁	かんちょう	
\\	都庁	とちょう	
\\	貯	
\\	チョ		チョ	貯金	ちょきん	
\\	貯水池	ちょすいち	
\\	易	
\\	エキ ・ イ	やさ.しい ・ やす.い
\\	エキ	安易な	あんいな	
\\	分かり易い	わかりやすい	
\\	易者	えきしゃ	
\\	易しい	やさしい	
\\	賜	
\\	シ	たまわ.る	シ	賜る	たまわる	
\\	賜金	しきん	
\\	賜る	たまわる	
\\	賜金	しきん	
\\	場	
\\	ジョウ	ば	ジョウ	場所	ばしょ	
\\	工場	こうじょう	
\\	場合	ばあい	
\\	駐車場	ちゅうしゃじょう	
\\	湯	
\\	トウ	ゆ	トウ	お湯	おゆ	
\\	温湯	おんとう	
\\	お湯	おゆ	
\\	門	
\\	モン	かど	モン	門口	かどぐち	
\\	正門	せいもん	
\\	部門	ぶもん	
\\	門	かど	
\\	名門	めいもん	
\\	間	
\\	カン ・ ケン	あいだ ・ ま	カン	間に合う	まにあう	
\\	時間	じかん	
\\	長い間	ながいあいだ	
\\	間 
\\	あいだ; ま
\\	あいだ: 
\\	ま: 
\\	の間に	のあいだに	
\\	空間	くうかん	
\\	閉	
\\	ヘイ	と.じる ・ と.ざす ・ し.める ・ し.まる	ヘイ													閉会	へいかい	
\\	閉まる	しまる	
\\	閉店時間	へいてんじかん	
\\	閉じる	とじる	
\\	開	
\\	カイ	ひら.く ・ ひら.き ・ -び.らき ・ ひら.ける ・ あ.く ・ あ.ける	カイ	開会	かいかい	
\\	開く 
\\	ひらく 
\\	あく	
\\	開発する	かいはつする	
\\	関	
\\	カン	せき ・ -ぜき ・ かか.わる	カン	関わる	かかわる	
\\	大関	おおぜき	
\\	関節	かんせつ	
\\	関する	かんする	
\\	問	
\\	モン	と.う ・ と.い ・ とん	モン	問いかける	といかける	
\\	問う	とう	
\\	質問	しつもん	
\\	学問	がくもん	
\\	問い合わせる	といあわせる	
\\	聞	
\\	ブン ・ モン	き.く ・ き.こえる	ブン	新聞	しんぶん	
\\	聞く	きく	
\\	聞き手	ききて	
\\	見聞	けんぶん	
\\	道を聞く	みちをきく	
\\	訪	
\\	ホウ	おとず.れる ・ たず.ねる	ホウ	訪問	ほうもん	
\\	訪ねる	たずねる	
\\	訪日	ほうにち	
\\	送	
\\	ソウ	おく.る	ソウ	送信	そうしん	
\\	送る	おくる	
\\	見送る	みおくる	
\\	送別会	そうべつかい	
\\	券	
\\	ケン		ケン	入場券	にゅうじょうけん	
\\	回数券	かいすうけん	
\\	日銀券	にちぎんけん	
\\	包	
\\	ホウ	つつ.む	ホウ	包囲する	ほういする	
\\	包む	つつむ	
\\	小包	こづつみ	
\\	包丁	ほうちょう	
\\	巻	
\\	カン	ま.く ・ まき ・ ま.き	カン	巻く	まく	
\\	上中下巻	じょうちゅうげかん	
\\	別巻	べっかん	
\\	巻き込む	まきこむ	
\\	圏	
\\	ケン		ケン	(通信)圏外	(つうしん)けんがい	
\\	首都圏	しゅとけん	
\\	生物圏	せいぶつけん	
\\	勝	
\\	ショウ	か.つ ・ -が.ち ・ まさ.る	ショウ	勝利	しょうり	
\\	勝つ	かつ	
\\	勝る	まさる	
\\	早い者勝ち	はやいものがち	
\\	戦	
\\	セン	いくさ ・ たたか.う	セン	戦う	たたかう	
\\	作戦	さくせん	
\\	戦線	せんせん	
\\	決勝戦	けっしょうせん	
\\	舌戦	ぜっせん	
\\	単	
\\	タン		タン	単に	たんに	
\\	単なる	たんなる	
\\	単数	たんすう	
\\	簡	
\\	カン		カン	簡単な	かんたんな	
\\	簡易な	かんいな	
\\	過	
\\	カ	す.ぎる ・ -す.ぎる ・ -す.ぎ ・ す.ごす ・ あやま.つ ・ あやま.ち	カ	過つ	あやまつ	
\\	過去	かこ	
\\	過ごす	すごす	
\\	過半数	かはんすう	
\\	近過ぎる	ちかすぎる	
\\	過ぎる	すぎる	
\\	骨	
\\	コツ	ほね	コツ	骨組み	ほねぐみ	
\\	人骨	じんこつ	
\\	背骨	せぼね	
\\	骨	ほね	
\\	昼	
\\	チュウ	ひる	チュウ	昼間	ひるま	
\\	昼食	ちゅうしょく	
\\	昼休み	ひるやすみ	
\\	白昼	はくちゅう	
\\	昼	ひる	
\\	夜	
\\	ヤ	よ ・ よる	ヤ	夜間	やかん	
\\	夜中	よなか	
\\	夜	よる	
\\	今夜	こんや	
\\	液	
\\	エキ		エキ	液体	えきたい	
\\	血液	けつえき	
\\	乳液	にゅうえき	
\\	戒	
\\	カイ	いまし.める	カイ	十戒	じゅっかい	
\\	戒める	いましめる	
\\	十戒	じゅっかい	
\\	戒める	いましめる	
\\	幾	
\\	キ	いく- ・ いく.つ ・ いく.ら	キ	幾つ	いくつ	
\\	幾分	いくぶん	
\\	畿	
\\	キ		キ	近畿地方	きんきちほう	
\\	近畿地方	きんきちほう	
\\	磯	
\\	キ	いそ	キ	磯辺	いそべ	
\\	磯辺	いそべ	
\\	機	
\\	キ	はた	キ	機会	きかい	
\\	機関	きかん	
\\	好機	こうき	
\\	機体	きたい	
\\	械	
\\	カイ		カイ	機械	きかい	
\\	機械化	きかいか	
\\	器械	きかい	
\\	飛	
\\	ヒ	と.ぶ ・ と.ばす ・ -と.ばす	ヒ	飛行	ひこう	
\\	飛ぶ	とぶ	
\\	飛行機	ひこうき	
\\	飛び上がる	とびあがる	
\\	飛ばす	とばす	
\\	氏	
\\	シ	うじ ・ -うじ	シ	故川田氏	こかわだし	
\\	氏神	うじがみ	
\\	氏名	しめい	
\\	故川田氏	こかわだし	
\\	氏神	うじがみ	
\\	氏名	しめい	
\\	民	
\\	ミン	たみ	ミン	国民	こくみん	
\\	自由の民	じゆうのたみ	
\\	民芸	みんげい	
\\	市民	しみん	
\\	民間	みんかん	
\\	紙	
\\	シ	かみ	シ	手紙	てがみ	
\\	用紙	ようし	
\\	紙	かみ	
\\	和紙	わし	
\\	日刊紙	にっかんし	
\\	低	
\\	テイ	ひく.い ・ ひく.める ・ ひく.まる	テイ	最低の	さいていの	
\\	低い	ひくい	
\\	低温	ていおん	
\\	低下する	ていかする	
\\	低める	ひくめる	
\\	抵	
\\	テイ		テイ	大抵	たいてい	
\\	抵当	ていとう	
\\	邸	
\\	テイ		テイ													公邸	こうてい	
\\	底	
\\	テイ	そこ	テイ	川底	かわぞこ	
\\	根底	こんてい	
\\	海底	かいてい	
\\	心の底から	こころのそこから	
\\	巨	
\\	キョ		キョ	巨大な	きょだいな	
\\	巨人	きょじん	
\\	臣	
\\	シン		シン	大臣	だいじん	
\\	臣民	しんみん	
\\	大臣	だいじん	
\\	臣民	しんみん	
\\	基	
\\	キ	もと ・ もとい	キ	基本	きほん	
\\	基づく	もとづく	
\\	データを基にする	データをもとにする	
\\	基地	きち	
\\	期	
\\	キ ・ ゴ		キ	期間	きかん	
\\	期待する	きたいする	
\\	期限	きげん	
\\	時期	じき	
\\	学期	がっき	
\\	毛	
\\	モウ	け	モウ	毛布	もうふ	
\\	毛糸	けいと	
\\	羽毛	うもう	
\\	毛	け	
\\	尾	
\\	ビ	お	ビ	尾を引く	おをひく	
\\	語尾	ごび	
\\	尾	お	
\\	育	
\\	イク	そだ.つ ・ そだ.ち ・ そだ.てる ・ はぐく.む	イク	体育	たいいく	
\\	育てる	そだてる	
\\	発育する	はついくする	
\\	育つ	そだつ	
\\	育む	はぐくむ	
\\	羊	
\\	ヨウ	ひつじ	ヨウ	羊毛	ようもう	
\\	子羊	こひつじ	
\\	羊肉	ようにく	
\\	羊	ひつじ	
\\	洋	
\\	ヨウ		ヨウ	大洋	たいよう	
\\	洋風	ようふう	
\\	洋式	ようしき	
\\	魚	
\\	ギョ	うお ・ さかな ・ -ざかな	ギョ	魚屋	さかなや	
\\	魚市場	うおいちば	
\\	魚類	ぎょるい	
\\	金魚	きんぎょ	
\\	魚 
\\	さかな; うお
\\	さなか: 
\\	うお: 
\\	魚市場)																
\\	鮮	
\\	セン	あざ.やか	セン	新鮮な	しんせんな	
\\	北朝鮮	きたちょうせん	
\\	韓 
\\	鮮やかな	あざやかな	
\\	蘇	
\\	ソ ・ ス	よみがえ.る	ソ	蘇生する	そせいする	
\\	蘇生する	そせいする	
\\	蘇る	よみがえる	
\\	詳	
\\	ショウ	くわ.しい	ショウ	詳しい	くわしい	
\\	詳細	しょうさい	
\\	詳しい	くわしい	
\\	詳細	しょうさい	
\\	祥	
\\	ショウ		ショウ	不祥事	ふしょうじ	
\\	不祥事	ふしょうじ	
\\	美	
\\	ビ	うつく.しい	ビ	美人	びじん	
\\	美しい	うつくしい	
\\	美味しい	おいしい	
\\	業	
\\	ギョウ ・ ゴウ	わざ	ギョウ	人間業	にんげんわざ	
\\	産業	さんぎょう	
\\	事業	じぎょう	
\\	作業	さぎょう	
\\	工業	こうぎょう	
\\	実	
\\	ジツ	み ・ みの.る	ジツ	実る	みのる	
\\	事実	じじつ	
\\	実に美しい	じつにうつくしい	
\\	養	
\\	ヨウ	やしな.う	ヨウ	養う	やしなう	
\\	養育する	よういくする	
\\	養子	ようし	
\\	養成する	ようせいする	
\end{CJK}
\end{document}