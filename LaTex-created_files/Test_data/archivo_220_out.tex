\documentclass[8pt]{extreport} 
\usepackage{hyperref}
\usepackage{CJKutf8}
\begin{document}
\begin{CJK}{UTF8}{min}
\\	きょうし	
\\	せんせい	
\\	がくせい	
\\	せいと	
\\	わたしはがくせいです	
\\	げいじゅつか	
\\	がか (画家)	
\\	にほん (日本)	
\\	にほんご (日本語)	
\\	にほんじん (日本人)	
\\	フランス	
\\	ちゅごく	
\\	かんこく	
\\	イギリス	
\\	アメリカ	
\\	えいご (英語)	
\\	どぞうよろしく	
\\	はじめまして	
\\	わたしはマットです	
\\	こちらは	
\\	さん	
\\	べんごし	
\\	はい	
\\	いいえ	
\\	これ	
\\	それ	
\\	あれ	
\\	これはほんですか	
\\	ペン	
\\	かみ	
\\	ほん (本)	
\\	かぎ	
\\	いす	
\\	しんぶん	
\\	とけい	
\\	けいたい	
\\	くつ	
\\	かさ	
\\	みす	
\\	おはよう	
\\	こんばんは	
\\	こんにちは	
\\	おやすみなさい	
\\	あなた	
\\	なに	
\\	くるま	
\\	かいしゃ	
\\	なまえ	
\\	イス	
\\	せかい (世界)	
\\	あき (秋)	
\\	おーさか	
\\	えき	
\\	でんわ	
\\	ばんごう	
\\	あなたのなまえはですか	
\\	にちようび (日曜日)	
\\	げつようび (月曜日)	
\\	かようび (火曜日)	
\\	すいようび (水曜日)	
\\	もくようび (木曜日)	
\\	きんようび(金曜日)	
\\	どようび (土曜日)	
\\	なにようび (何曜日)	
\\	いちがつ (一月)	
\\	にがつ (二月)	
\\	さんがつ (三月)	
\\	よんがつ (四月)	
\\	ごがつ (五月)	
\\	ろくがつ (六月)	
\\	しちがつ (七月)	
\\	はちがつ (八月)	
\\	くがつ (九月)	
\\	じゅうがつ (十月)	
\\	じゅういちがつ (十一月)	
\\	じゅうにがつ (十二月)	
\\	ともだち (友達)	
\\	ともだち (友達) と	
\\	だれと (誰と)	
\\	きの (昨日)	
\\	きょう (今日)	
\\	あした (明日)	
\\	せんしゅう (先週)	
\\	こんしゅう (今週))	
\\	らいしゅう (来週)	
\\	せんげつ (先月)	
\\	こんげつ (今月)	
\\	らいげつ (来月)	
\\	ちかてつ (地下鉄)	
\\	でんしゃ (電車)	
\\	くるま (車)	
\\	タクシー	
\\	しんかんせん (新幹線)	
\\	ふね (船)	
\\	バイク	
\\	じてんしゃ (自転車)	
\\	ひこうき (飛行機)	
\\	くうこう (空港)	
\\	えき (駅)	
\\	ししゃ (支社)	
\\	こうえん (公園)	
\\	うち (家)	
\\	かぞく (家族)	
\\	くる (来る)	
\\	かえる (帰る)	
\\	いく (行く)	
\\	ごご (午後)	
\\	ごぜん (午前)	
\\	めいし (名刺)	
\\	ひとつ (一つ)	
\\	ふたつ (二つ)	
\\	みっつ (三つ)	
\\	よっつ (四つ)	
\\	いつつ (五つ)	
\\	むっつ (六つ)	
\\	ななつ (七つ)	
\\	やっつ (八つ)	
\\	ここのつ (九つ)	
\\	とお	
\\	ついたち (一日)	
\\	ふつか (二日)	
\\	みっか (三日)	
\\	よっか (四日)	
\\	いつか (五日)	
\\	むいか (六日)	
\\	なのか (七日)	
\\	ようか (八日)	
\\	ここのか (九日)	
\\	とおか (十日)	
\\	じゅうななにち (十七
\\	日)	
\\	じゅうななこ	
\\	いっぽん (一本)	
\\	にほん (二本)	
\\	さんぼん (三本)	
\\	よんほん (四本)	
\\	ごほん (五本)	
\\	ろっぽん (六本)	
\\	ななほん (七本)	
\\	はっぽん (八本)	
\\	きゅうほん (九本)	
\\	じゅうっぽん (十っぽん)	
\\	じゅうななほん (十七本)	
\\	いちまい (一枚)	
\\	にまい (二枚)	
\\	さんまい (三枚)	
\\	よんまい (四枚)	
\\	ごまい (五枚)	
\\	ろくまい (六枚)	
\\	ななまい (七枚)	
\\	はちまい(八枚)	
\\	きゅうまい (九枚)	
\\	じゅうまい (十枚)	
\\	じゅうななまい (十七枚)	
\\	いっさつ (一冊)	
\\	にさつ (二冊)	
\\	さんさつ (三冊)	
\\	よんさつ (四冊)	
\\	ごさつ (五冊)	
\\	ろくさつ (六冊)	
\\	ななさつ (七さつ)	
\\	はさつ (八冊)	
\\	きゅうさつ (九冊)	
\\	じゅうさつ (十冊)	
\\	じゅうななさつ (十七冊)	
\\	ひとり (一人)	
\\	ふたり (二人)	
\\	ごにん (五人)	
\\	きゅうじゅうきゅうにん (九十九人)	
\\	いっぴき (一匹)	
\\	にひき (二匹)	
\\	さんびき (三匹)	
\\	よんびき (四匹)	
\\	たんじょうび (誕生日)	
\\	がんばってね (頑張ってね)	
\\	はん (半)	
\\	あかい (赤い)	
\\	あおい (青い)	
\\	どれ	
\\	くろい (黒い)	
\\	しろい (白い)	
\\	おおきい (大きい)	
\\	ちいさい (小さい)	
\\	どこ	
\\	あさごはん (朝ご飯)	
\\	ひるやすみ (昼休み)	
\\	ひるごはん (ひるごん華)	
\\	ばんごはん (晩御飯)	
\\	ゆうびんきょく (郵便局)	
\\	いっぱい (一杯)	
\\	にはい (二杯)	
\\	さんばい (三杯)	
\\	よんはい (四杯)	
\\	ごはい (五杯)	
\\	ろっぱい (六杯)	
\\	ななはい (七杯)	
\\	はっぱい (八杯)	
\\	きゅうはい (九杯)	
\\	じゅうぱい (十杯)	
\\	じゅうななはい(十七杯)	
\\	いちだい (一台)	
\\	ななだい (七台)	
\\	にせんきゅうひゃくにじゅうさんだい (二千九百二十三台)	
\\	から 
\\	まで です	
\\	それをみせてください	
\\	そのぺんはいくらですか	
\\	しんぶん (新聞)	
\\	どれ (何れ)	
\\	どのしんぶん (何の新聞)	
\\	ざっし (雑誌)	
\\	けっこです	
\\	やすい (安い)	
\\	たかい(高い)	
\\	ふうとう (封筒)	
\\	はさみ (鋏)	
\\	ホチキス	
\\	りんご (林檎)	
\\	しろいかみをじゅうまいください (白い紙を十枚ください)	
\\	こんや (今夜)	
\\	わたしはろくじにぎんこうにいきます (私はろくじに銀行に行きます)	
\\	あなたはどこにいきますか	
\\	わたしはあしたトバさんとしごとにいきます (私は明日トバさんと仕事に行きます)	
\\	いく (行く)	
\\	くる (来る)	
\\	かえる (帰る)	
\\	あなたはだれとカナダにいきますか (あなたは誰とカナダに行きますか)	
\\	いきます	
\\	いきません	
\\	いきました	
\\	いきませんでした	
\\	きます	
\\	きません	
\\	きました	
\\	きませんでした	
\\	かえります	
\\	かえりません	
\\	かえりました	
\\	かえりませんでした	
\\	いつ (何時)	
\\	じゃあ	
\\	すみません	
\\	いただきます	
\\	ごちそうさまでした (ご馳走様でした)	
\\	だいがく (大学)	
\\	じゅうしょ (住所)	
\\	たいしかん (大使館)	
\\	てちょう (手帳)	
\\	おねがいします (お願いします)	
\\	おんなのひと (女の人)	
\\	いま (今)	
\\	はがき (葉書)	
\\	きって (切手)	
\\	ひとりで (一人で)	
\\	もしもし	
\\	そちら (其方)	
\\	ことし (今年)	
\\	きょねん (去年)	
\\	うんてんしゅ(運転手)	
\\	わたしはちかてつでサンフランシスコにいきます (私は地下鉄でサンフランシスコに行きます)	
\\	なんで	
\\	りょこう (旅行)	
\\	おんせん (温泉)	
\\	あります (有ります)	
\\	いちご (苺)	
\\	えいがかん (映画館)	
\\	ことし (今年)	
\\	きょねん (去年)	
\\	らいねん (来年)	
\\	ぞう (像)	
\\	けいかく (計画)	
\\	おしずし (押し寿司)	
\\	にぎり (握り)	
\\	かぜ(風)	
\\	かがく (科学)	
\\	すいか (西瓜)	
\\	としょかん (図書館)	
\\	いっかい (一階)	
\\	何階	
\\	きょうはなんようびですか (今日は何曜日ですか)	
\\	きょうはなんにちですか (今日は何日ですか)	
\\	あしたはどようびです (明日は土曜日です)	
\\	せつぶん (節分)	
\\	いる (居る)	
\\	ある (有る)	
\\	ボクはなんかいにありますか (ボクは何階に有りますか)	
\\	ボクはにかいにあります (ボクは二階に有ります)	
\\	ちゅうしゃじょう (駐車場)	
\\	うけつけ (受付)	
\\	いえ (家)	
\\	まえ (前)	
\\	うしろ (後ろ)	
\\	となり (隣)	
\\	ちかく (近く)	
\\	ここ (此処)	
\\	きょうはにがつここのかです (今日は二月九日です)	
\\	わたしのけいたいはテブルのうえにあります (私の携帯はテブルの上に有ります)	
\\	かいだん (階段)	
\\	だんろ (暖炉)	
\\	まど (窓)	
\\	しょくぶつ (植物)	
\\	どうぶつ (動物)	
\\	こうじょう (工場)	
\\	き (木)	
\\	すぎのき(杉の木)	
\\	のこりもの (残り物)	
\\	くずかご (屑篭)	
\\	つくえ (机)	
\\	しんしつ (寝室)	
\\	くつした (靴下)	
\\	ゆか (床)	
\\	てんじょう (天井)	
\\	かべ (壁)	
\\	ねずみ (鼠)	
\\	ひきだし (引き出し)	
\\	そく(促
\\	はな(花)	
\\	はな (鼻)	
\\	かお (顔)	
\\	たんす (箪笥)	
\\	となりのへや (隣の部屋)	
\\	だれもいません (誰も
\\	居ません)	
\\	だれもいます(誰も居ます)	
\\	けしゴム(消しゴム)	
\\	たくさん (沢山)	
\\	なにも (何も)	
\\	なにもありません (何も有りません)	
\\	いくつ (幾つ)	
\\	おとこのひと (男の人)	
\\	おんなのこ (女の子)	
\\	おとこのこ (男の子)	
\\	ほんだな(本棚)	
\\	れいぞうこ (冷蔵庫)	
\\	かいが (絵画)	
\\	え (絵)	
\\	しょくたく (食卓)	
\\	にわとり (鶏)	
\\	かも (鴨)	
\\	げいじゅつさくひん (芸術作品)	
\\	だいどころ (台所)	
\\	いま (居間)	
\\	しょさい (書斎)	
\\	わたしのほんだなにほんがひゃくさつあります(私の本棚に本が百冊有ります)	
\\	しちめんちょう (七面鳥)	
\\	ベッドのうえにねこがいます (ベッドの上に猫が居ます)	
\\	にんじん (人参)	
\\	サンダル	
\\	ヨット	
\\	ごましお(胡麻塩)	
\\	ロケット	
\\	はち (蜂)	
\\	くじら (鯨)	
\\	ジュース	
\\	にそく (二足)	
\\	さんそう (三艘)	
\\	よんつぶ (四粒)	
\\	ろくわ(六羽)	
\\	はっとう(八頭)	
\\	じゅっこ (十個)	
\end{CJK}
\end{document}