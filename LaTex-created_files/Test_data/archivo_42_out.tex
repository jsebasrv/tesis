\documentclass[8pt]{extreport} 
\usepackage{hyperref}
\usepackage{CJKutf8}
\begin{document}
\begin{CJK}{UTF8}{min}
\\	百	百[ひゃく]	ひゃく	
\\	私の祖母は百才です。	私[わたし]の 祖母[そぼ]は 百[ひゃく] 才[さい]です。	わたし の そぼ は ひゃくさい です	
\\	私[わたし]の 祖母[そぼ]は
\\	才[さい]です。			
\\	金	金[かね]	かね	
\\	これはかなり金がかかった。	これはかなり 金[かね]がかかった。	これ は かなり かね が かかった	
\\	これはかなり
\\	がかかった。			
\\	金曜日	金曜日[きんようび]	きんようび	
\\	金曜日の夜は友達と出かけます。	金曜日[きんようび]の 夜[よる]は 友達[ともだち]と 出[で]かけます。	きんようび の よる は ともだち と でかけます	
\\	の 夜[よる]は 友達[ともだち]と 出[で]かけます。			
\\	火曜日	火曜日[かようび]	かようび	
\\	火曜日に会議があります。	火曜日[かようび]に 会議[かいぎ]があります。	かようび に かいぎ が あります	
\\	に 会議[かいぎ]があります。			
\\	週	週[しゅう]	しゅう	
\\	その次の週は空いてますか。	その 次[つぎ]の 週[しゅう]は 空[あ]いてますか。	その つぎ の しゅう は あいてます か	
\\	その 次[つぎ]の
\\	は 空[あ]いてますか。			
\\	行く	行[い]く	いく	
\\	日曜日は図書館に行きます。	日曜日[にちようび]は 図書館[としょかん]に 行[い]きます。	にちようび は としょかん に いきます	
\\	日曜日[にちようび]は 図書館[としょかん]に
\\	目	目[め]	め	
\\	彼女は青い目をしています。	彼女[かのじょ]は 青[あお]い 目[め]をしています。	かのじょ は あおい め を して います	
\\	彼女[かのじょ]は 青[あお]い
\\	をしています。			
\\	耳	耳[みみ]	みみ	
\\	彼女は耳がよく聞こえません。	彼女[かのじょ]は 耳[みみ]がよく 聞[き]こえません。	かのじょ は みみ が よく きこえません	
\\	彼女[かのじょ]は
\\	がよく 聞[き]こえません。			
\\	足	足[あし]	あし	
\\	彼は足が長い。	彼[かれ]は 足[あし]が 長[なが]い。	かれ は あし が ながい	
\\	彼[かれ]は
\\	が 長[なが]い。			
\\	空く	空[す]く	すく	
\\	レストランは空いていました。	レストランは 空[す]いていました。	れすとらん は すいて いました	
\\	レストランは
\\	男	男[おとこ]	おとこ	
\\	男の人が私たちに話しかけた。	男[おとこ]の 人[ひと]が 私[わたし]たちに 話[はな]しかけた。	おとこ の ひと が わたしたち に はなしかけた	
\\	の 人[ひと]が 私[わたし]たちに 話[はな]しかけた。			
\\	私	私[わたし]	わたし	
\\	私が行きましょう。	私[わたし]が 行[い]きましょう。	わたし が いきましょう	
\\	が 行[い]きましょう。			
\\	開ける	開[あ]ける	あける	
\\	窓を開けてください。	窓[まど]を 開[あ]けてください。	まど を あけて ください	
\\	窓[まど]を
\\	ください。			
\\	開く	開[ひら]く	ひらく	
\\	32ページを開いてください。	32[さんじゅうに]ページを 開[ひら]いてください。	さんじゅうにぺーじ を ひらいて ください	
\\	32[さんじゅうに]ページを
\\	ください。			
\\	閉じる	閉[と]じる	とじる	
\\	教科書を閉じてください。	教科書[きょうかしょ]を 閉[と]じてください。	きょうかしょ を とじて ください	
\\	教科書[きょうかしょ]を
\\	ください。			
\\	閉める	閉[し]める	しめる	
\\	ちゃんとドアを閉めてよ。	ちゃんとドアを 閉[し]めてよ。	ちゃんと どあ を しめてよ	
\\	ちゃんとドアを
\\	よ。			
\\	閉まる	閉[し]まる	しまる	
\\	お店はもう閉まっていました。	お 店[みせ]はもう 閉[し]まっていました。	おみせ は もう しまって いました	
\\	お 店[みせ]はもう
\\	聞く	聞[き]く	きく	
\\	彼女はラジオを聞いています。	彼女[かのじょ]はラジオを 聞[き]いています。	かのじょ は らじお を きいて います	
\\	彼女[かのじょ]はラジオを
\\	高い	高[たか]い	たかい	
\\	これがこの町で一番高いビルです。	これがこの 町[まち]で 一番[いちばん] 高[たか]いビルです。	これ が この まち で いちばん たかい びる です	
\\	これがこの 町[まち]で 一番[いちばん]
\\	ビルです。			
\\	食べる	食[た]べる	たべる	
\\	昨日タイカレーを食べました。	昨日[きのう]タイカレーを 食[た]べました。	きのう たいかれー を たべました	
\\	昨日[きのう]タイカレーを
\\	飲む	飲[の]む	のむ	
\\	友達とお酒を飲んでいます。	友達[ともだち]とお 酒[さけ]を 飲[の]んでいます。	ともだち と おさけ を のんで います	
\\	友達[ともだち]とお 酒[さけ]を
\\	買う	買[か]う	かう	
\\	郵便局で切手を買いました。	郵便局[ゆうびんきょく]で 切手[きって]を 買[か]いました。	ゆうびんきょく で きって を かいました	
\\	郵便局[ゆうびんきょく]で 切手[きって]を
\\	見る	見[み]る	みる	
\\	私は絵を見るのが好きです。	私[わたし]は 絵[え]を 見[み]るのが 好[す]きです。	わたし は え を みる の が すき です	
\\	私[わたし]は 絵[え]を
\\	のが 好[す]きです。			
\\	見せる	見[み]せる	みせる	
\\	その写真を見せてください。	その 写真[しゃしん]を 見[み]せてください。	その しゃしん を みせて ください	
\\	その 写真[しゃしん]を
\\	ください。			
\\	見つける	見[み]つける	みつける	
\\	新しい仕事を見つけました。	新[あたら]しい 仕事[しごと]を 見[み]つけました。	あたらしい しごと を みつけました。	
\\	新[あたら]しい 仕事[しごと]を
\\	見える	見[み]える	みえる	
\\	ここから富士山がよく見えます。	ここから 富士山[ふじさん]がよく 見[み]えます。	ここ から ふじさん が よく みえます	
\\	ここから 富士山[ふじさん]がよく
\\	見つかる	見[み]つかる	みつかる	
\\	メガネが見つかりません。	メガネが 見[み]つかりません。	めがね が みつかりません	
\\	メガネが
\\	言う	言[い]う	いう	
\\	上司が「一杯、飲もう。」と言った。	上司[じょうし]が
\\	一杯[いっぱい]、 飲[の]もう。」と 言[い]った。	じょうし が いっぱい のもう と いった	
\\	上司[じょうし]が
\\	一杯[いっぱい]、 飲[の]もう。」と
\\	話す	話[はな]す	はなす	
\\	母と電話で話しました。	母[はは]と 電話[でんわ]で 話[はな]しました。	はは と でんわ で はなしました	
\\	母[はは]と 電話[でんわ]で
\\	読む	読[よ]む	よむ	
\\	彼女は雑誌を読んでいます。	彼女[かのじょ]は 雑誌[ざっし]を 読[よ]んでいます。	かのじょ は ざっし を よんで います	
\\	彼女[かのじょ]は 雑誌[ざっし]を
\\	漢字	漢字[かんじ]	かんじ	
\\	漢字は中国から来ました。	漢字[かんじ]は 中国[ちゅうごく]から 来[き]ました。	かんじ は ちゅうごく から きました	
\\	は 中国[ちゅうごく]から 来[き]ました。			
\\	覚える	覚[おぼ]える	おぼえる	
\\	妹は平仮名を全部覚えました。	妹[いもうと]は 平仮名[ひらがな]を 全部[ぜんぶ] 覚[おぼ]えました。	いもうと は ひらがな を ぜんぶ おぼえました	
\\	妹[いもうと]は 平仮名[ひらがな]を 全部[ぜんぶ]
\\	車	車[くるま]	くるま	
\\	弟が車を買った。	弟[おとうと]が 車[くるま]を 買[か]った。	おとうと が くるま を かった	
\\	弟[おとうと]が
\\	を 買[か]った。			
\\	電車	電車[でんしゃ]	でんしゃ	
\\	私は電車で通学しています。	私[わたし]は 電車[でんしゃ]で 通学[つうがく]しています。	わたし は でんしゃ で つうがく して います	
\\	私[わたし]は
\\	で 通学[つうがく]しています。			
\\	駅	駅[えき]	えき	
\\	駅はどこですか。	駅[えき]はどこですか。	えき は どこ です か	
\\	はどこですか。			
\\	道	道[みち]	みち	
\\	この道を真っ直ぐ行くと駅です。	この 道[みち]を 真[ま]っ 直[す]ぐ 行[い]くと 駅[えき]です。	この みち を まっすぐ いく と えき です	
\\	この
\\	を 真[ま]っ 直[す]ぐ 行[い]くと 駅[えき]です。			
\\	走る	走[はし]る	はしる	
\\	彼は毎晩3キロ走っています。	彼[かれ]は 毎晩3[まいばん さん]キロ 走[はし]っています。	かれ は まいばん さんきろ はしって います	
\\	彼[かれ]は 毎晩3[まいばん さん]キロ
\\	近く	近[ちか]く	ちかく	
\\	駅の近くで食事をした。	駅[えき]の 近[ちか]くで 食事[しょくじ]をした。	えき の ちかく で しょくじ を した	
\\	駅[えき]の
\\	で 食事[しょくじ]をした。			
\\	近い	近[ちか]い	ちかい	
\\	駅はここから近いです。	駅[えき]はここから 近[ちか]いです。	えき は ここ から ちかい です	
\\	駅[えき]はここから
\\	です。			
\\	近く	近[ちか]く	ちかく	
\\	私の家は駅の近くです。	私[わたし]の 家[いえ]は 駅[えき]の 近[ちか]くです。	わたし の いえ は えき の ちかく です	
\\	私[わたし]の 家[いえ]は 駅[えき]の
\\	です。			
\\	遠い	遠[とお]い	とおい	
\\	家から学校までは遠いです。	家[いえ]から 学校[がっこう]までは 遠[とお]いです。	いえ から がっこう まで は とおい です	
\\	家[いえ]から 学校[がっこう]までは
\\	です。			
\\	長い	長[なが]い	ながい	
\\	彼女の髪はとても長い。	彼女[かのじょ]の 髪[かみ]はとても 長[なが]い。	かのじょ の かみ は とても ながい	
\\	彼女[かのじょ]の 髪[かみ]はとても
\\	短い	短[みじか]い	みじかい	
\\	彼は足が短い。	彼[かれ]は 足[あし]が 短[みじか]い。	かれ は あし が みじかい	
\\	彼[かれ]は 足[あし]が
\\	白い	白[しろ]い	しろい	
\\	彼は歯が白い。	彼[かれ]は 歯[は]が 白[しろ]い。	かれ は は が しろい	
\\	彼[かれ]は 歯[は]が
\\	赤い	赤[あか]い	あかい	
\\	赤いバラを買いました。	赤[あか]いバラを 買[か]いました。	あかい ばら を かいました	
\\	バラを 買[か]いました。			
\\	部屋	部屋[へや]	へや	
\\	私の部屋は2階にあります。	私[わたし]の 部屋[へや]は 2階[にかい]にあります。	わたし の へや は にかい に あります	
\\	私[わたし]の
\\	は 2階[にかい]にあります。			
\\	米	米[こめ]	こめ	
\\	日本人はお米が大好きです。	日本人[にほんじん]はお 米[こめ]が 大好[だいす]きです。	にほんじん は おこめ が だいすき です	
\\	日本人[にほんじん]はお
\\	が 大好[だいす]きです。			
\\	無い	無[な]い	ない	
\\	ここには何も無い。	ここには 何[なに]も 無[な]い。	ここ に は なに も ない	
\\	ここには 何[なに]も
\\	秋	秋[あき]	あき	
\\	彼女は秋に結婚します。	彼女[かのじょ]は 秋[あき]に 結婚[けっこん]します。	かのじょ は あき に けっこん します	
\\	彼女[かのじょ]は
\\	に 結婚[けっこん]します。			
\\	熱い	熱[あつ]い	あつい	
\\	このスープはとても熱い。	このスープはとても 熱[あつ]い。	この すーぷ は とても あつい	
\\	このスープはとても
\\	良い	良[い]い	いい	
\\	彼女は良い友達です。	彼女[かのじょ]は 良[い]い 友達[ともだち]です。	かのじょ は いい ともだち です	
\\	彼女[かのじょ]は
\\	友達[ともだち]です。			
\\	考える	考[かんが]える	かんがえる	
\\	よく考えてください。	よく 考[かんが]えてください。	よく かんがえて ください	
\\	よく
\\	ください。			
\\	知る	知[し]る	しる	
\\	誰もその話を知らない。	誰[だれ]もその 話[はなし]を 知[し]らない。	だれ も その はなし を しらない	
\\	誰[だれ]もその 話[はなし]を
\\	食べ物	食[た]べ 物[もの]	たべもの	
\\	日本の食べ物はとても美味しいです。	日本[にっぽん]の 食[た]べ 物[もの]はとても 美味[おい]しいです。	にっぽん の たべもの は とても おいしい です	
\\	日本[にっぽん]の
\\	はとても 美味[おい]しいです。			
\\	飲み物	飲[の]み 物[もの]	のみもの	
\\	何か飲み物が欲しいな。	何[なに]か 飲[の]み 物[もの]が 欲[ほ]しいな。	なに か のみもの が ほしい な 。	
\\	何[なに]か
\\	が 欲[ほ]しいな。			
\\	重い	重[おも]い	おもい	
\\	このかばんは重いです。	このかばんは 重[おも]いです。	この かばん は おもい です	
\\	このかばんは
\\	です。			
\\	軽い	軽[かる]い	かるい	
\\	この靴はとても軽い。	この 靴[くつ]はとても 軽[かる]い。	この くつ は とても かるい	
\\	この 靴[くつ]はとても
\\	送る	送[おく]る	おくる	
\\	彼の家に荷物を送りました。	彼[かれ]の 家[いえ]に 荷物[にもつ]を 送[おく]りました。	かれ の いえ に にもつ を おくりました	
\\	彼[かれ]の 家[いえ]に 荷物[にもつ]を
\\	生きる	生[い]きる	いきる	
\\	皆一生懸命生きている。	皆一生懸命[みんな いっしょうけんめい] 生[い]きている。	みんな いっしょうけんめい いきて いる	
\\	皆一生懸命[みんな いっしょうけんめい]
\\	高校生	高校生[こうこうせい]	こうこうせい	
\\	私の弟は高校生です。	私[わたし]の 弟[おとうと]は 高校生[こうこうせい]です。	わたし の おとうと は こうこうせい です	
\\	私[わたし]の 弟[おとうと]は
\\	です。			
\\	質問	質問[しつもん]	しつもん	
\\	質問のある方はどうぞ。	質問[しつもん]のある 方[かた]はどうぞ。	しつもん の ある かた は どうぞ	
\\	のある 方[かた]はどうぞ。			
\\	難しい	難[むずか]しい	むずかしい	
\\	この本は難しいですね。	この 本[ほん]は 難[むずか]しいですね。	この ほん は むずかしい です ね	
\\	この 本[ほん]は
\\	ですね。			
\\	負ける	負[ま]ける	まける	
\\	私たちのチームはその試合で負けた。	私[わたし]たちのチームはその 試合[しあい]で 負[ま]けた。	わたしたち の ちーむ は その しあい で まけた	
\\	私[わたし]たちのチームはその 試合[しあい]で
\\	要る	要[い]る	いる	
\\	予約は要りません。	予約[よやく]は 要[い]りません。	よやく は いりません	
\\	予約[よやく]は
\\	着る	着[き]る	きる	
\\	今日はスーツを着ています。	今日[きょう]はスーツを 着[き]ています。	きょう は すーつ を きて います	
\\	今日[きょう]はスーツを
\\	立つ	立[た]つ	たつ	
\\	彼はステージに立った。	彼[かれ]はステージに 立[た]った。	かれ は すてーじ に たった	
\\	彼[かれ]はステージに
\\	速い	速[はや]い	はやい	
\\	彼は走るのが速い。	彼[かれ]は 走[はし]るのが 速[はや]い。	かれ は はしる の が はやい	
\\	彼[かれ]は 走[はし]るのが
\\	遅い	遅[おそ]い	おそい	
\\	前の車はとても遅い。	前[まえ]の 車[くるま]はとても 遅[おそ]い。	まえ の くるま は とても おそい	
\\	前[まえ]の 車[くるま]はとても
\\	終わる	終[お]わる	おわる	
\\	会議は4時に終わります。	会議[かいぎ]は 4時[よじ]に 終[お]わります。	かいぎ は よじ に おわります	
\\	会議[かいぎ]は 4時[よじ]に
\\	終わり	終[お]わり	おわり	
\\	夏休みももう終わりだ。	夏休[なつやす]みももう 終[お]わりだ。	なつやすみ も もう おわり だ	
\\	夏休[なつやす]みももう
\\	だ。			
\\	紙	紙[かみ]	かみ	
\\	紙と鉛筆はありますか。	紙[かみ]と 鉛筆[えんぴつ]はありますか。	かみ と えんぴつ は あります か	
\\	と 鉛筆[えんぴつ]はありますか。			
\\	色	色[いろ]	いろ	
\\	すてきな色のセーターですね。	すてきな 色[いろ]のセーターですね。	すてき な いろ の せーたー です ね	
\\	すてきな
\\	のセーターですね。			
\\	茶色	茶色[ちゃいろ]	ちゃいろ	
\\	彼女は茶色の靴を履いています。	彼女[かのじょ]は 茶色[ちゃいろ]の 靴[くつ]を 履[は]いています。	かのじょ は ちゃいろ の くつ を はいて います	
\\	彼女[かのじょ]は
\\	の 靴[くつ]を 履[は]いています。			
\\	直ぐ	直[す]ぐ	すぐ	
\\	直ぐ行きます。	直[す]ぐ 行[い]きます。	すぐ いきます	
\\	行[い]きます。			
\\	直る	直[なお]る	なおる	
\\	クーラーはまだ直りません。	クーラーはまだ 直[なお]りません。	くーらー は まだ なおりません	
\\	クーラーはまだ
\\	線	線[せん]	せん	
\\	赤い線を2本引いて下さい。	赤[あか]い 線[せん]を 2本引[に ほん ひ]いて 下[くだ]さい。	あかい せん を に ほん ひいて ください	
\\	赤[あか]い
\\	を 2本引[に ほん ひ]いて 下[くだ]さい。			
\\	違う	違[ちが]う	ちがう	
\\	答えが違います。	答[こた]えが 違[ちが]います。	こたえ が ちがいます	
\\	答[こた]えが
\\	遊ぶ	遊[あそ]ぶ	あそぶ	
\\	子供たちが公園で遊んでいる。	子供[こども]たちが 公園[こうえん]で 遊[あそ]んでいる。	こどもたち が こうえん で あそんで いる	
\\	子供[こども]たちが 公園[こうえん]で
\\	父	父[ちち]	ちち	
\\	私は父が大好きです。	私[わたし]は 父[ちち]が 大好[だいす]きです。	わたし は ちち が だいすき です	
\\	私[わたし]は
\\	が 大好[だいす]きです。			
\\	親	親[おや]	おや	
\\	親の愛は有り難い。	親[おや]の 愛[あい]は 有[あ]り 難[がた]い。	おや の あい は ありがたい	
\\	の 愛[あい]は 有[あ]り 難[がた]い。			
\\	若い	若[わか]い	わかい	
\\	彼はまだ若いです。	彼[かれ]はまだ 若[わか]いです。	かれ は まだ わかい です	
\\	彼[かれ]はまだ
\\	です。			
\\	結婚	結婚[けっこん]	けっこん	
\\	彼女は来月結婚します。	彼女[かのじょ]は 来月[らいげつ] 結婚[けっこん]します。	かのじょ は らいげつ けっこん します	
\\	彼女[かのじょ]は 来月[らいげつ]
\\	します。			
\\	自転車	自転車[じてんしゃ]	じてんしゃ	
\\	毎日、駅まで自転車で行きます。	毎日[まいにち]、 駅[えき]まで 自転車[じてんしゃ]で 行[い]きます。	まいにち えき まで じてんしゃ で いきます	
\\	毎日[まいにち]、 駅[えき]まで
\\	で 行[い]きます。			
\\	自分	自分[じぶん]	じぶん	
\\	宿題は自分でやりなさい。	宿題[しゅくだい]は 自分[じぶん]でやりなさい。	しゅくだい は じぶん で やりなさい	
\\	宿題[しゅくだい]は
\\	でやりなさい。			
\\	答える	答[こた]える	こたえる	
\\	私の質問に答えてください。	私[わたし]の 質問[しつもん]に 答[こた]えてください。	わたし の しつもん に こたえて ください	
\\	私[わたし]の 質問[しつもん]に
\\	ください。			
\\	病気	病気[びょうき]	びょうき	
\\	祖父が病気になった。	祖父[そふ]が 病気[びょうき]になった。	そふ が びょうき に なった	
\\	祖父[そふ]が
\\	になった。			
\\	痛い	痛[いた]い	いたい	
\\	今日は頭が痛いです。	今日[きょう]は 頭[あたま]が 痛[いた]いです。	きょう は あたま が いたい です	
\\	今日[きょう]は 頭[あたま]が
\\	です。			
\\	酒	酒[さけ]	さけ	
\\	彼女は酒に強い。	彼女[かのじょ]は 酒[さけ]に 強[つよ]い。	かのじょ は さけ に つよい	
\\	彼女[かのじょ]は
\\	に 強[つよ]い。			
\\	飛ぶ	飛[と]ぶ	とぶ	
\\	鳥が飛んでいます。	鳥[とり]が 飛[と]んでいます。	とり が とんで います	
\\	鳥[とり]が
\\	飛行機	飛行機[ひこうき]	ひこうき	
\\	息子は飛行機のおもちゃが好きです。	息子[むすこ]は 飛行機[ひこうき]のおもちゃが 好[す]きです。	むすこ は ひこうき の おもちゃ が すき です	
\\	息子[むすこ]は
\\	のおもちゃが 好[す]きです。			
\\	続く	続[つづ]く	つづく	
\\	工事は3月まで続きます。	工事[こうじ]は 3月[さんがつ]まで 続[つづ]きます。	こうじ は さんがつ まで つづきます	
\\	工事[こうじ]は 3月[さんがつ]まで
\\	首	首[くび]	くび	
\\	きりんの首は長い。	きりんの 首[くび]は 長[なが]い。	きりん の くび は ながい	
\\	きりんの
\\	は 長[なが]い。			
\\	頭	頭[あたま]	あたま	
\\	今朝から頭が痛い。	今朝[けさ]から 頭[あたま]が 痛[いた]い。	けさ から あたま が いたい	
\\	今朝[けさ]から
\\	が 痛[いた]い。			
\\	顔	顔[かお]	かお	
\\	彼はタオルで顔を拭きました。	彼[かれ]はタオルで 顔[かお]を 拭[ふ]きました。	かれ は たおる で かお を ふきました	
\\	彼[かれ]はタオルで
\\	を 拭[ふ]きました。			
\\	落ちる	落[お]ちる	おちる	
\\	猿も木から落ちる。	猿[さる]も 木[き]から 落[お]ちる。	さる も き から おちる	
\\	猿[さる]も 木[き]から
\\	神	神[かみ]	かみ	
\\	彼は神を信じている。	彼[かれ]は 神[かみ]を 信[しん]じている。	かれ は かみ を しんじて いる	
\\	彼[かれ]は
\\	を 信[しん]じている。			
\\	置く	置[お]く	おく	
\\	彼はかばんをいすの上に置きました。	彼[かれ]はかばんをいすの 上[うえ]に 置[お]きました。	かれ は かばん を いす の うえ に おきました	
\\	彼[かれ]はかばんをいすの 上[うえ]に
\\	辺	辺[へん]	へん	
\\	彼はこの辺に住んでいます。	彼[かれ]はこの 辺[へん]に 住[す]んでいます。	かれ は このへん に すんで います	
\\	彼[かれ]はこの
\\	に 住[す]んでいます。			
\\	黄色い	黄色[きいろ]い	きいろい	
\\	彼女に黄色いバラを買いました。	彼女[かのじょ]に 黄色[きいろ]いバラを 買[か]いました。	かのじょ に きいろい ばら を かいました	
\\	彼女[かのじょ]に
\\	バラを 買[か]いました。			
\\	緑	緑[みどり]	みどり	
\\	この町には緑がたくさんあります。	この 町[まち]には 緑[みどり]がたくさんあります。	この まち に は みどり が たくさん あります	
\\	この 町[まち]には
\\	がたくさんあります。			
\\	留学生	留学生[りゅうがくせい]	りゅうがくせい	
\\	彼は留学生です。	彼[かれ]は 留学生[りゅうがくせい]です。	かれ は りゅうがくせい です	
\\	彼[かれ]は
\\	です。			
\\	起きる	起[お]きる	おきる	
\\	私は毎朝6時に起きます。	私[わたし]は 毎朝6時[まいあさ ろくじ]に 起[お]きます。	わたし は まいあさ ろくじ に おきます	
\\	私[わたし]は 毎朝6時[まいあさ ろくじ]に
\\	起こる	起[お]こる	おこる	
\\	町で大事件が起こりました。	町[まち]で 大事件[だいじけん]が 起[お]こりました。	まち で だいじけん が おこりました	
\\	町[まち]で 大事件[だいじけん]が
\\	起こす	起[お]こす	おこす	
\\	明日、6時に起こしてください。	明日[あす]、 6時[ろくじ]に 起[お]こしてください。	あす ろくじ に おこして ください	
\\	明日[あす]、 6時[ろくじ]に
\\	ください。			
\\	起きる	起[お]きる	おきる	
\\	近所で盗難事件が起きました。	近所[きんじょ]で 盗難事件[とうなん じけん]が 起[お]きました。	きんじょ で とうなん じけん が おきました	
\\	近所[きんじょ]で 盗難事件[とうなん じけん]が
\\	細い	細[ほそ]い	ほそい	
\\	彼女は指が細いですね。	彼女[かのじょ]は 指[ゆび]が 細[ほそ]いですね。	かのじょ は ゆび が ほそい です ね	
\\	彼女[かのじょ]は 指[ゆび]が
\\	ですね。			
\\	載せる	載[の]せる	のせる	
\\	皿にケーキを載せました。	皿[さら]にケーキを 載[の]せました。	さら に けーき を のせました。	
\\	皿[さら]にケーキを
\\	締める	締[し]める	しめる	
\\	彼はシートベルトを締めた。	彼[かれ]はシートベルトを 締[し]めた。	かれ は しーとべると を しめた	
\\	彼[かれ]はシートベルトを
\\	甘い	甘[あま]い	あまい	
\\	このイチゴは甘い。	このイチゴは 甘[あま]い。	この いちご は あまい	
\\	このイチゴは
\\	辛い	辛[から]い	からい	
\\	彼は辛いものが好きです。	彼[かれ]は 辛[から]いものが 好[す]きです。	かれ は からい もの が すき です	
\\	彼[かれ]は
\\	ものが 好[す]きです。			
\\	誰	誰[だれ]	だれ	
\\	誰と会ってみたいですか。	誰[だれ]と 会[あ]ってみたいですか。	だれ と あって みたい です か	
\\	と 会[あ]ってみたいですか。			
\\	笑う	笑[わら]う	わらう	
\\	赤ん坊が笑っています。	赤[あか]ん 坊[ぼう]が 笑[わら]っています。	あかんぼう が わらって います	
\\	赤[あか]ん 坊[ぼう]が
\\	酸っぱい	酸[す]っぱい	すっぱい	
\\	このぶどうは酸っぱいです。	このぶどうは 酸[す]っぱいです。	この ぶどう は すっぱい です	
\\	このぶどうは
\\	です。			
\\	言葉	言葉[ことば]	ことば	
\\	この言葉の意味が分かりません。	この 言葉[ことば]の 意味[いみ]が 分[わ]かりません。	この ことば の いみ が わかりません	
\\	この
\\	の 意味[いみ]が 分[わ]かりません。			
\\	胸	胸[むね]	むね	
\\	胸に少し痛みがあります。	胸[むね]に 少[すこ]し 痛[いた]みがあります。	むね に すこし いたみ が あります	
\\	に 少[すこ]し 痛[いた]みがあります。			
\\	腰	腰[こし]	こし	
\\	昨日から腰が痛い。	昨日[きのう]から 腰[こし]が 痛[いた]い。	きのう から こし が いたい	
\\	昨日[きのう]から
\\	が 痛[いた]い。			
\\	背	背[せ]	せ	
\\	彼女は背が高い。	彼女[かのじょ]は 背[せ]が 高[たか]い。	かのじょ は せ が たかい	
\\	彼女[かのじょ]は
\\	が 高[たか]い。			
\\	片仮名	片仮名[かたかな]	かたかな	
\\	お名前を片仮名で書いてください。	お 名前[なまえ]を 片仮名[かたかな]で 書[か]いてください。	おなまえ を かたかな で かいて ください	
\\	お 名前[なまえ]を
\\	で 書[か]いてください。			
\\	美しい	美[うつく]しい	うつくしい	
\\	この絵は美しいです。	この 絵[え]は 美[うつく]しいです。	この え は うつくしい です	
\\	この 絵[え]は
\\	です。			
\\	狭い	狭[せま]い	せまい	
\\	私の部屋は狭いです。	私[わたし]の 部屋[へや]は 狭[せま]いです。	わたし の へや は せまい です	
\\	私[わたし]の 部屋[へや]は
\\	です。			
\\	触る	触[さわ]る	さわる	
\\	絵に触らないでください。	絵[え]に 触[さわ]らないでください。	え に さわらない で ください	
\\	絵[え]に
\\	ください。			
\\	渇く	渇[かわ]く	かわく	
\\	喉が渇きました。	喉[のど]が 渇[かわ]きました。	のど が かわきました	
\\	喉[のど]が
\\	髪の毛	髪[かみ]の 毛[け]	かみのけ	
\\	髪の毛が伸びたね。	髪[かみ]の 毛[け]が 伸[の]びたね。	かみのけ が のびた ね	
\\	が 伸[の]びたね。			
\\	肩	肩[かた]	かた	
\\	肩が凝りました。	肩[かた]が 凝[こ]りました。	かた が こりました	
\\	が 凝[こ]りました。			
\\	鼻	鼻[はな]	はな	
\\	鼻がかゆいです。	鼻[はな]がかゆいです。	はな が かゆい です	
\\	がかゆいです。			
\\	腕	腕[うで]	うで	
\\	彼の腕は太い。	彼[かれ]の 腕[うで]は 太[ふと]い。	かれ の うで は ふとい	
\\	彼[かれ]の
\\	は 太[ふと]い。			
\\	駄目	駄目[だめ]	だめ	
\\	彼は駄目な男だ。	彼[かれ]は 駄目[だめ]な 男[おとこ]だ。	かれ は だめ な おとこ だ	
\\	彼[かれ]は
\\	な 男[おとこ]だ。			
\\	風邪	風邪[かぜ]	かぜ	
\\	私は風邪を引きました。	私[わたし]は 風邪[かぜ]を 引[ひ]きました。	わたし は かぜ を ひきました 。	
\\	私[わたし]は
\\	を 引[ひ]きました。			
\\	火	火[ひ]	ひ	
\\	火を消して。	火[ひ]を 消[け]して。	ひ を けして	
\\	を 消[け]して。			
\\	金	金[きん]	きん	
\\	彼女は金のネックレスをしています。	彼女[かのじょ]は 金[きん]のネックレスをしています。	かのじょ は きん の ねっくれす を して います	
\\	彼女[かのじょ]は
\\	のネックレスをしています。			
\\	行う	行[おこな]う	おこなう	
\\	その会社は来月、キャンペーンを行う。	その 会社[かいしゃ]は 来月[らいげつ]、キャンペーンを 行[おこな]う。	その かいしゃ は らいげつ きゃんぺーん を おこなう	
\\	その 会社[かいしゃ]は 来月[らいげつ]、キャンペーンを
\\	行き	行[い]き	いき	
\\	行きは新幹線で行った。	行[い]きは 新幹線[しんかんせん]で 行[い]った。	いき は しんかんせん で いった	
\\	は 新幹線[しんかんせん]で 行[い]った。			
\\	行き	行[ゆ]き	ゆき	
\\	東京行きの列車に乗った。	東京[とうきょう] 行[ゆ]きの 列車[れっしゃ]に 乗[の]った。	とうきょうゆき の れっしゃ に のった	
\\	東京[とうきょう]
\\	の 列車[れっしゃ]に 乗[の]った。			
\\	足りる	足[た]りる	たりる	
\\	お金が足りなくて買えなかった。	お 金[かね]が 足[た]りなくて 買[か]えなかった。	おかね が たりなくて かえなかった	
\\	お 金[かね]が
\\	買[か]えなかった。			
\\	足す	足[た]す	たす	
\\	母は味噌汁に水を足した。	母[はは]は 味噌汁[みそしる]に 水[みず]を 足[た]した。	はは は みそしる に みず を たした	
\\	母[はは]は 味噌汁[みそしる]に 水[みず]を
\\	空く	空[あ]く	あく	
\\	後ろの席が空いています。	後[うし]ろの 席[せき]が 空[あ]いています。	うしろ の せき が あいて います	
\\	後[うし]ろの 席[せき]が
\\	空手	空手[からて]	からて	
\\	彼は空手を習っています。	彼[かれ]は 空手[からて]を 習[なら]っています。	かれ は からて を ならって います	
\\	彼[かれ]は
\\	を 習[なら]っています。			
\\	空	空[そら]	そら	
\\	空が真っ青です。	空[そら]が 真[ま]っ 青[さお]です。	そら が まっさお です	
\\	が 真[ま]っ 青[さお]です。			
\\	石	石[いし]	いし	
\\	私は石につまづいた。	私[わたし]は 石[いし]につまづいた。	わたし は いし に つまづいた	
\\	私[わたし]は
\\	につまづいた。			
\\	田んぼ	田[た]んぼ	たんぼ	
\\	この辺は田んぼがたくさんあります。	この 辺[へん]は 田[た]んぼがたくさんあります。	このへん は たんぼ が たくさん あります	
\\	この 辺[へん]は
\\	がたくさんあります。			
\\	花	花[はな]	はな	
\\	きれいな花が咲きました。	きれいな 花[はな]が 咲[さ]きました。	きれい な はな が さきました	
\\	きれいな
\\	が 咲[さ]きました。			
\\	男の子	男[おとこ]の 子[こ]	おとこのこ	
\\	男の子たちがサッカーをしている。	男[おとこ]の 子[こ]たちがサッカーをしている。	おとこのこたち が さっかー を して いる	
\\	たちがサッカーをしている。			
\\	私たち	私[わたし]たち	わたしたち	
\\	私たちは来月結婚します。	私[わたし]たちは 来月結婚[らいげつ けっこん]します。	わたしたち は らいげつ けっこん します 。	
\\	は 来月結婚[らいげつ けっこん]します。			
\\	達する	達[たっ]する	たっする	
\\	気温は35度に達した。	気温[きおん]は 35度[さんじゅうごど]に 達[たっ]した。	きおん は さんじゅうごど に たっした	
\\	気温[きおん]は 35度[さんじゅうごど]に
\\	空気	空気[くうき]	くうき	
\\	ここは空気がきれいです。	ここは 空気[くうき]がきれいです。	ここ は くうき が きれい です	
\\	ここは
\\	がきれいです。			
\\	雨	雨[あめ]	あめ	
\\	雨が降っています。	雨[あめ]が 降[ふ]っています。	あめ が ふって います	
\\	が 降[ふ]っています。			
\\	雪	雪[ゆき]	ゆき	
\\	クリスマスに雪が降りました。	クリスマスに 雪[ゆき]が 降[ふ]りました。	くりすます に ゆき が ふりました	
\\	クリスマスに
\\	が 降[ふ]りました。			
\\	青い	青[あお]い	あおい	
\\	ここの海はとても青い。	ここの 海[うみ]はとても 青[あお]い。	ここ の うみ は とても あおい	
\\	ここの 海[うみ]はとても
\\	青	青[あお]	あお	
\\	私の好きな色は青です。	私[わたし]の 好[す]きな 色[いろ]は 青[あお]です。	わたし の すき な いろ は あお です	
\\	私[わたし]の 好[す]きな 色[いろ]は
\\	です。			
\\	西	西[にし]	にし	
\\	太陽は西に沈みます。	太陽[たいよう]は 西[にし]に 沈[しず]みます。	たいよう は にし に しずみます	
\\	太陽[たいよう]は
\\	に 沈[しず]みます。			
\\	開く	開[あ]く	あく	
\\	電車のドアが開きました。	電車[でんしゃ]のドアが 開[あ]きました。	でんしゃ の どあ が あきました	
\\	電車[でんしゃ]のドアが
\\	聞こえる	聞[き]こえる	きこえる	
\\	隣の部屋からテレビの音が聞こえる。	隣[となり]の 部屋[へや]からテレビの 音[おと]が 聞[き]こえる。	となり の へや から てれび の おと が きこえる	
\\	隣[となり]の 部屋[へや]からテレビの 音[おと]が
\\	間	間[あいだ]	あいだ	
\\	雲の間から月が出た。	雲[くも]の 間[あいだ]から 月[つき]が 出[で]た。	くも の あいだ から つき が でた	
\\	雲[くも]の
\\	から 月[つき]が 出[で]た。			
\\	高さ	高[たか]さ	たかさ	
\\	富士山の高さは3,776メートルです。	富士山[ふじさん]の 高[たか]さは 
\\	776[さんぜんななひゃく-ななじゅうろく]メートルです。	ふじさん の たかさ は さんぜんななひゃく-ななじゅうろくめーとる です	
\\	富士山[ふじさん]の
\\	は 
\\	776[さんぜんななひゃく-ななじゅうろく]メートルです。			
\\	牛	牛[うし]	うし	
\\	牛が草を食べています。	牛[うし]が 草[くさ]を 食[た]べています。	うし が くさ を たべて います	
\\	が 草[くさ]を 食[た]べています。			
\\	見方	見方[みかた]	みかた	
\\	彼に対する見方が変わりました。	彼[かれ]に 対[たい]する 見方[みかた]が 変[か]わりました。	かれ に たいする みかた が かわりました	
\\	彼[かれ]に 対[たい]する
\\	が 変[か]わりました。			
\\	花見	花見[はなみ]	はなみ	
\\	友達と花見をしました。	友達[ともだち]と 花見[はなみ]をしました。	ともだち と はなみ を しました	
\\	友達[ともだち]と
\\	をしました。			
\\	言い方	言[い]い 方[かた]	いいかた	
\\	そんな言い方をしてはいけません。	そんな 言[い]い 方[かた]をしてはいけません。	そんな いいかた を して は いけません	
\\	そんな
\\	をしてはいけません。			
\\	話	話[はなし]	はなし	
\\	あなたの話は面白いね。	あなたの 話[はなし]は 面白[おもしろ]いね。	あなた の はなし は おもしろい ね	
\\	あなたの
\\	は 面白[おもしろ]いね。			
\\	読み	読[よ]み	よみ	
\\	母に読み書きを習いました。	母[はは]に 読[よ]み 書[か]きを 習[なら]いました。	はは に よみかき を ならいました	
\\	母[はは]に
\\	書[か]きを 習[なら]いました。			
\\	読み方	読[よ]み 方[かた]	よみかた	
\\	この漢字の読み方を教えてください。	この 漢字[かんじ]の 読[よ]み 方[かた]を 教[おし]えてください。	この かんじ の よみかた を おしえて ください	
\\	この 漢字[かんじ]の
\\	を 教[おし]えてください。			
\\	語る	語[かた]る	かたる	
\\	今夜は大いに語りましょう。	今夜[こんや]は 大[おお]いに 語[かた]りましょう。	こんや は おおいに かたりましょう	
\\	今夜[こんや]は 大[おお]いに
\\	言語	言語[げんご]	げんご	
\\	彼はアジアの言語を研究している。	彼[かれ]はアジアの 言語[げんご]を 研究[けんきゅう]している。	かれ は あじあ の げんご を けんきゅう して いる	
\\	彼[かれ]はアジアの
\\	を 研究[けんきゅう]している。			
\\	英語	英語[えいご]	えいご	
\\	あなたは英語が話せますか。	あなたは 英語[えいご]が 話[はな]せますか。	あなた は えいご が はなせます か	
\\	あなたは
\\	が 話[はな]せますか。			
\\	覚める	覚[さ]める	さめる	
\\	今朝は6時に目が覚めた。	今朝[けさ]は 6時[ろくじ]に 目[め]が 覚[さ]めた。	けさ は ろくじ に め が さめた	
\\	今朝[けさ]は 6時[ろくじ]に 目[め]が
\\	覚ます	覚[さ]ます	さます	
\\	子供が目を覚ました。	子供[こども]が 目[め]を 覚[さ]ました。	こども が め を さました	
\\	子供[こども]が 目[め]を
\\	話し合う	話[はな]し 合[あ]う	はなしあう	
\\	私たちはよく話し合いました。	私[わたし]たちはよく 話[はな]し 合[あ]いました。	わたしたち は よく はなしあいました	
\\	私[わたし]たちはよく
\\	間に合う	間[ま]に 合[あ]う	まにあう	
\\	授業に間に合いました。	授業[じゅぎょう]に 間[ま]に 合[あ]いました。	じゅぎょう に まにあいました	
\\	授業[じゅぎょう]に
\\	社会	社会[しゃかい]	しゃかい	
\\	これは大きな社会問題になっている。	これは 大[おお]きな 社会[しゃかい] 問題[もんだい]になっている。	これ は おおき な しゃかい もんだい に なって いる	
\\	これは 大[おお]きな
\\	問題[もんだい]になっている。			
\\	社員	社員[しゃいん]	しゃいん	
\\	彼は優秀な社員です。	彼[かれ]は 優秀[ゆうしゅう]な 社員[しゃいん]です。	かれ は ゆうしゅう な しゃいん です	
\\	彼[かれ]は 優秀[ゆうしゅう]な
\\	です。			
\\	食事	食事[しょくじ]	しょくじ	
\\	今日は上司と食事をする。	今日[きょう]は 上司[じょうし]と 食事[しょくじ]をする。	きょう は じょうし と しょくじ を する	
\\	今日[きょう]は 上司[じょうし]と
\\	をする。			
\\	火事	火事[かじ]	かじ	
\\	火事です。119番に電話してください。	火事[かじ]です。 119番[ひゃくじゅうきゅうばん]に 電話[でんわ]してください。	かじ です ひゃくじゅうきゅうばん に でんわ して ください	
\\	です。 119番[ひゃくじゅうきゅうばん]に 電話[でんわ]してください。			
\\	電話	電話[でんわ]	でんわ	
\\	あとで電話します。	あとで 電話[でんわ]します。	あとで でんわ します	
\\	あとで
\\	します。			
\\	電気	電気[でんき]	でんき	
\\	電気をつけてください。	電気[でんき]をつけてください。	でんき をつけてください	
\\	をつけてください。			
\\	車	車[くるま]	くるま	
\\	この道は車が多い。	この 道[みち]は 車[くるま]が 多[おお]い。	この みち は くるま が おおい	
\\	この 道[みち]は
\\	が 多[おお]い。			
\\	駅員	駅員[えきいん]	えきいん	
\\	駅員に聞きましょう。	駅員[えきいん]に 聞[き]きましょう。	えきいん に ききましょう	
\\	に 聞[き]きましょう。			
\\	通り	通[とお]り	とおり	
\\	この通りはにぎやかですね。	この 通[とお]りはにぎやかですね。	この とおり は にぎやか です ね	
\\	この
\\	はにぎやかですね。			
\\	通る	通[とお]る	とおる	
\\	毎日、この道を通ります。	毎日[まいにち]、この 道[みち]を 通[とお]ります。	まいにち この みち を とおります	
\\	毎日[まいにち]、この 道[みち]を
\\	通う	通[かよ]う	かよう	
\\	私はジムに通っています。	私[わたし]はジムに 通[かよ]っています。	わたし は じむ に かよって います	
\\	私[わたし]はジムに
\\	車道	車道[しゃどう]	しゃどう	
\\	車道の工事が始まりました。	車道[しゃどう]の 工事[こうじ]が 始[はじ]まりました。	しゃどう の こうじ が はじまりました	
\\	の 工事[こうじ]が 始[はじ]まりました。			
\\	道路	道路[どうろ]	どうろ	
\\	この道路は3年前にできました。	この 道路[どうろ]は 3年前[さんねんまえ]にできました。	この どうろ は さんねんまえ に できました	
\\	この
\\	は 3年前[さんねんまえ]にできました。			
\\	渡す	渡[わた]す	わたす	
\\	彼に手紙を渡しました。	彼[かれ]に 手紙[てがみ]を 渡[わた]しました。	かれ に てがみ を わたしました	
\\	彼[かれ]に 手紙[てがみ]を
\\	渡る	渡[わた]る	わたる	
\\	私たちは歩いて橋を渡った。	私[わたし]たちは 歩[ある]いて 橋[はし]を 渡[わた]った。	わたしたち は あるいて はし を わたった	
\\	私[わたし]たちは 歩[ある]いて 橋[はし]を
\\	遠く	遠[とお]く	とおく	
\\	遠くに船が見えます。	遠[とお]くに 船[ふね]が 見[み]えます。	とおく に ふね が みえます	
\\	に 船[ふね]が 見[み]えます。			
\\	社長	社長[しゃちょう]	しゃちょう	
\\	おばは小さな会社の社長です。	おばは 小[ちい]さな 会社[かいしゃ]の 社長[しゃちょう]です。	おば は ちいさ な かいしゃ の しゃちょう です	
\\	おばは 小[ちい]さな 会社[かいしゃ]の
\\	です。			
\\	長さ	長[なが]さ	ながさ	
\\	このケーブルの長さは1メートルです。	このケーブルの 長[なが]さは 1[いち]メートルです。	この けーぶる の ながさ は いちめーとる です	
\\	このケーブルの
\\	は 1[いち]メートルです。			
\\	長男	長男[ちょうなん]	ちょうなん	
\\	彼の長男は8才です。	彼[かれ]の 長男[ちょうなん]は 8才[はっさい]です。	かれ の ちょうなん は はっさい です	
\\	彼[かれ]の
\\	は 8才[はっさい]です。			
\\	長女	長女[ちょうじょ]	ちょうじょ	
\\	うちの長女は10歳です。	うちの 長女[ちょうじょ]は 10歳[じゅっさい]です。	うち の ちょうじょ は じゅっさい です	
\\	うちの
\\	は 10歳[じゅっさい]です。			
\\	部分	部分[ぶぶん]	ぶぶん	
\\	この部分は問題ないです。	この 部分[ぶぶん]は 問題[もんだい]ないです。	この ぶぶん は もんだい ない です	
\\	この
\\	は 問題[もんだい]ないです。			
\\	白	白[しろ]	しろ	
\\	白は雪の色です。	白[しろ]は 雪[ゆき]の 色[いろ]です。	しろ は ゆき の いろ です	
\\	は 雪[ゆき]の 色[いろ]です。			
\\	黒い	黒[くろ]い	くろい	
\\	彼女は黒いドレスを着ています。	彼女[かのじょ]は 黒[くろ]いドレスを 着[き]ています。	かのじょ は くろい どれす を きて います	
\\	彼女[かのじょ]は
\\	ドレスを 着[き]ています。			
\\	黒	黒[くろ]	くろ	
\\	黒のボールペンはありますか。	黒[くろ]のボールペンはありますか。	くろ の ぼーるぺん は あります か	
\\	のボールペンはありますか。			
\\	赤ちゃん	赤[あか]ちゃん	あかちゃん	
\\	ベッドで赤ちゃんが眠っています。	ベッドで 赤[あか]ちゃんが 眠[ねむ]っています。	べっど で あかちゃん が ねむって います	
\\	ベッドで
\\	が 眠[ねむ]っています。			
\\	赤	赤[あか]	あか	
\\	信号が赤に変わりました。	信号[しんごう]が 赤[あか]に 変[か]わりました。	しんごう が あか に かわりました	
\\	信号[しんごう]が
\\	に 変[か]わりました。			
\\	銀行	銀行[ぎんこう]	ぎんこう	
\\	銀行は3時まで開いています。	銀行[ぎんこう]は 3時[さんじ]まで 開[あ]いています。	ぎんこう は さんじ まで あいて います	
\\	は 3時[さんじ]まで 開[あ]いています。			
\\	銀	銀[ぎん]	ぎん	
\\	彼は銀メダルを取った。	彼[かれ]は 銀[ぎん]メダルを 取[と]った。	かれ は ぎんめだる を とった	
\\	彼[かれ]は
\\	メダルを 取[と]った。			
\\	牛肉	牛肉[ぎゅうにく]	ぎゅうにく	
\\	夕食に牛肉を買った。	夕食[ゆうしょく]に 牛肉[ぎゅうにく]を 買[か]った。	ゆうしょく に ぎゅうにく を かった	
\\	夕食[ゆうしょく]に
\\	を 買[か]った。			
\\	肉	肉[にく]	にく	
\\	肉が焼けました。	肉[にく]が 焼[や]けました。	にく が やけました	
\\	が 焼[や]けました。			
\\	魚	魚[さかな]	さかな	
\\	肉と魚とどちらが好きですか。	肉[にく]と 魚[さかな]とどちらが 好[す]きですか。	にく と さかな と どちら が すき です か	
\\	肉[にく]と
\\	とどちらが 好[す]きですか。			
\\	野菜	野菜[やさい]	やさい	
\\	私は毎日たくさん野菜を食べます。	私[わたし]は 毎日[まいにち]たくさん 野菜[やさい]を 食[た]べます。	わたし は まいにち たくさん やさい を たべます	
\\	私[わたし]は 毎日[まいにち]たくさん
\\	を 食[た]べます。			
\\	茶	茶[ちゃ]	ちゃ	
\\	私たちは毎日お茶を飲みます。	私[わたし]たちは 毎日[まいにち]お 茶[ちゃ]を 飲[の]みます。	わたしたち は まいにち おちゃ を のみます	
\\	私[わたし]たちは 毎日[まいにち]お
\\	を 飲[の]みます。			
\\	茶わん	茶[ちゃ]わん	ちゃわん	
\\	茶わんにご飯をよそいました。	茶[ちゃ]わんにご 飯[はん]をよそいました。	ちゃわん に ごはん を よそいました	
\\	にご 飯[はん]をよそいました。			
\\	週末	週末[しゅうまつ]	しゅうまつ	
\\	週末は家でゆっくりします。	週末[しゅうまつ]は 家[うち]でゆっくりします。	しゅうまつ は うち で ゆっくり します	
\\	は 家[うち]でゆっくりします。			
\\	無理	無理[むり]	むり	
\\	5時までに家に帰るのは無理です。	5時[ごじ]までに 家[いえ]に 帰[かえ]るのは 無理[むり]です。	ごじ まで に いえ に かえる の は むり です	
\\	5時[ごじ]までに 家[いえ]に 帰[かえ]るのは
\\	です。			
\\	無くす	無[な]くす	なくす	
\\	今日、鍵を無くしました。	今日[きょう]、 鍵[かぎ]を 無[な]くしました。	きょう かぎ を なくしました	
\\	今日[きょう]、 鍵[かぎ]を
\\	無くなる	無[な]くなる	なくなる	
\\	もうお金が無くなりました。	もうお 金[かね]が 無[な]くなりました。	もう おかね が なくなりました	
\\	もうお 金[かね]が
\\	用いる	用[もち]いる	もちいる	
\\	彼はその詩を用いて自分の気持ちを伝えた。	彼[かれ]はその 詩[し]を 用[もち]いて 自分[じぶん]の 気持[きも]ちを 伝[つた]えた。	かれ は その し を もちいて じぶん の きもち を つたえた	
\\	彼[かれ]はその 詩[し]を
\\	自分[じぶん]の 気持[きも]ちを 伝[つた]えた。			
\\	用事	用事[ようじ]	ようじ	
\\	父は用事で出掛けています。	父[ちち]は 用事[ようじ]で 出掛[でか]けています。	ちち は ようじ で でかけて います	
\\	父[ちち]は
\\	で 出掛[でか]けています。			
\\	販売	販売[はんばい]	はんばい	
\\	前売券は窓口で販売しています。	前売券[まえうりけん]は 窓口[まどぐち]で 販売[はんばい]しています。	まえうりけん は まどぐち で はんばい して います	
\\	前売券[まえうりけん]は 窓口[まどぐち]で
\\	しています。			
\\	階段	階段[かいだん]	かいだん	
\\	私たちは駅の階段をかけ上がった。	私[わたし]たちは 駅[えき]の 階段[かいだん]をかけ 上[あ]がった。	わたしたち は えき の かいだん を かけあがった	
\\	私[わたし]たちは 駅[えき]の
\\	をかけ 上[あ]がった。			
\\	熱	熱[ねつ]	ねつ	
\\	昨日の夜、熱が出ました。	昨日[きのう]の 夜[よる]、 熱[ねつ]が 出[で]ました。	きのう の よる ねつ が でました	
\\	昨日[きのう]の 夜[よる]、
\\	が 出[で]ました。			
\\	温度	温度[おんど]	おんど	
\\	今、部屋の温度は25度だ。	今[いま]、 部屋[へや]の 温度[おんど]は 25度[にじゅうごど]だ。	いま へや の おんど は にじゅうごど だ	
\\	今[いま]、 部屋[へや]の
\\	は 25度[にじゅうごど]だ。			
\\	風	風[かぜ]	かぜ	
\\	今日は風が強いです。	今日[きょう]は 風[かぜ]が 強[つよ]いです。	きょう は かぜ が つよい です	
\\	今日[きょう]は
\\	が 強[つよ]いです。			
\\	良い	良[よ]い	よい	
\\	彼は良いところだけを強調した。	彼[かれ]は 良[よ]いところだけを 強調[きょうちょう]した。	かれ は よい ところ だけ を きょうちょう した	
\\	彼[かれ]は
\\	ところだけを 強調[きょうちょう]した。			
\\	考え	考[かんが]え	かんがえ	
\\	それは良い考えです。	それは 良[い]い 考[かんが]えです。	それ は いい かんがえ です	
\\	それは 良[い]い
\\	です。			
\\	解決	解決[かいけつ]	かいけつ	
\\	トラブルがやっと解決した。	トラブルがやっと 解決[かいけつ]した。	とらぶる が やっと かいけつ した	
\\	トラブルがやっと
\\	した。			
\\	知らせる	知[し]らせる	しらせる	
\\	皆に会議の日にちを知らせた。	皆[みんな]に 会議[かいぎ]の 日[ひ]にちを 知[し]らせた。	みんな に かいぎ の ひにち を しらせた	
\\	皆[みんな]に 会議[かいぎ]の 日[ひ]にちを
\\	能力	能力[のうりょく]	のうりょく	
\\	彼は能力のある社員です。	彼[かれ]は 能力[のうりょく]のある 社員[しゃいん]です。	かれ は のうりょく の ある しゃいん です	
\\	彼[かれ]は
\\	のある 社員[しゃいん]です。			
\\	郵便	郵便[ゆうびん]	ゆうびん	
\\	さっき郵便が届きました。	さっき 郵便[ゆうびん]が 届[とど]きました。	さっき ゆうびん が とどきました	
\\	さっき
\\	が 届[とど]きました。			
\\	郵便屋さん	郵便屋[ゆうびんや]さん	ゆうびんやさん	
\\	郵便屋さんはもう来ましたか。	郵便屋[ゆうびんや]さんはもう 来[き]ましたか。	ゆうびんやさん は もう きました か	
\\	はもう 来[き]ましたか。			
\\	郵便局	郵便局[ゆうびんきょく]	ゆうびんきょく	
\\	郵便局で切手を買いました。	郵便局[ゆうびんきょく]で 切手[きって]を 買[か]いました。	ゆうびんきょくで きって を かいました	
\\	で 切手[きって]を 買[か]いました。			
\\	番地	番地[ばんち]	ばんち	
\\	その建物の番地は分かりますか。	その 建物[たてもの]の 番地[ばんち]は 分[わ]かりますか。	その たてもの の ばんち は わかります か	
\\	その 建物[たてもの]の
\\	は 分[わ]かりますか。			
\\	番号	番号[ばんごう]	ばんごう	
\\	この番号に電話してください。	この 番号[ばんごう]に 電話[でんわ]してください。	この ばんごう に でんわ して ください	
\\	この
\\	に 電話[でんわ]してください。			
\\	近所	近所[きんじょ]	きんじょ	
\\	近所にカナダ人が住んでいる。	近所[きんじょ]にカナダ 人[じん]が 住[す]んでいる。	きんじょ に かなだじん が すんで いる	
\\	にカナダ 人[じん]が 住[す]んでいる。			
\\	町	町[まち]	まち	
\\	あの町は緑を増やしています。	あの 町[まち]は 緑[みどり]を 増[ふ]やしています。	あの まち は みどり を ふやして います	
\\	あの
\\	は 緑[みどり]を 増[ふ]やしています。			
\\	都市	都市[とし]	とし	
\\	東京は日本一大きな都市です。	東京[とうきょう]は 日本一大[にほんいち おお]きな 都市[とし]です。	とうきょう は にほんいち おおき な とし です	
\\	東京[とうきょう]は 日本一大[にほんいち おお]きな
\\	です。			
\\	都合	都合[つごう]	つごう	
\\	今日は都合が悪くて行けません。	今日[きょう]は 都合[つごう]が 悪[わる]くて 行[い]けません。	きょう は つごう が わるくて いけません	
\\	今日[きょう]は
\\	が 悪[わる]くて 行[い]けません。			
\\	買い物	買[か]い 物[もの]	かいもの	
\\	母は買い物に出かけています。	母[はは]は 買[か]い 物[もの]に 出[で]かけています。	はは は かいもの に でかけて います	
\\	母[はは]は
\\	に 出[で]かけています。			
\\	荷物	荷物[にもつ]	にもつ	
\\	彼の家に荷物を送りました。	彼[かれ]の 家[いえ]に 荷物[にもつ]を 送[おく]りました。	かれ の いえ に にもつ を おくりました	
\\	彼[かれ]の 家[いえ]に
\\	を 送[おく]りました。			
\\	見物	見物[けんぶつ]	けんぶつ	
\\	皆で東京見物をした。	皆[みんな]で 東京[とうきょう] 見物[けんぶつ]をした。	みんな で とうきょう けんぶつ を した	
\\	皆[みんな]で 東京[とうきょう]
\\	をした。			
\\	物	物[もの]	もの	
\\	彼女の家には物がたくさんあります。	彼女[かのじょ]の 家[いえ]には 物[もの]がたくさんあります。	かのじょ の いえ に は もの が たくさん あります	
\\	彼女[かのじょ]の 家[いえ]には
\\	がたくさんあります。			
\\	重さ	重[おも]さ	おもさ	
\\	この荷物の重さを計ってください。	この 荷物[にもつ]の 重[おも]さを 計[はか]ってください。	この にもつ の おもさ を はかって ください	
\\	この 荷物[にもつ]の
\\	を 計[はか]ってください。			
\\	配る	配[くば]る	くばる	
\\	会議で書類を配った。	会議[かいぎ]で 書類[しょるい]を 配[くば]った。	かいぎ で しょるい を くばった	
\\	会議[かいぎ]で 書類[しょるい]を
\\	配達	配達[はいたつ]	はいたつ	
\\	彼は新聞配達をしている。	彼[かれ]は 新聞[しんぶん] 配達[はいたつ]をしている。	かれ は しんぶん はいたつ を して いる	
\\	彼[かれ]は 新聞[しんぶん]
\\	をしている。			
\\	見送る	見送[みおく]る	みおくる	
\\	彼が外国に行くのを見送りました。	彼[かれ]が 外国[がいこく]に 行[い]くのを 見送[みおく]りました。	かれ が がいこく に いく の を みおくりました	
\\	彼[かれ]が 外国[がいこく]に 行[い]くのを
\\	見送り	見送[みおく]り	みおくり	
\\	駅に友人の見送りに行きました。	駅[えき]に 友人[ゆうじん]の 見送[みおく]りに 行[い]きました。	えき に ゆうじん の みおくり に いきました	
\\	駅[えき]に 友人[ゆうじん]の
\\	に 行[い]きました。			
\\	金持ち	金持[かねも]ち	かねもち	
\\	彼は金持ちです。	彼[かれ]は 金持[かねも]ちです。	かれ は かねもち です	
\\	彼[かれ]は
\\	です。			
\\	生まれる	生[う]まれる	うまれる	
\\	姉夫婦に男の子が生まれました。	姉夫婦[あね ふうふ]に 男[おとこ]の 子[こ]が 生[う]まれました。	あね ふうふ に おとこのこ が うまれました	
\\	姉夫婦[あね ふうふ]に 男[おとこ]の 子[こ]が
\\	生む	生[う]む	うむ	
\\	うちのネコが子猫を生みました。	うちのネコが 子猫[こねこ]を 生[う]みました。	うち の ねこ が こねこ を うみました	
\\	うちのネコが 子猫[こねこ]を
\\	生産	生産[せいさん]	せいさん	
\\	ボルドーはワインの生産で有名だ。	ボルドーはワインの 生産[せいさん]で 有名[ゆうめい]だ。	ぼるどー は わいん の せいさん で ゆうめい だ	
\\	ボルドーはワインの
\\	で 有名[ゆうめい]だ。			
\\	生活	生活[せいかつ]	せいかつ	
\\	日本での生活は楽しいです。	日本[にほん]での 生活[せいかつ]は 楽[たの]しいです。	にほん で の せいかつ は たのしい です	
\\	日本[にほん]での
\\	は 楽[たの]しいです。			
\\	生徒	生徒[せいと]	せいと	
\\	このクラスの生徒は30人です。	このクラスの 生徒[せいと]は 30人[さんじゅうにん]です。	この くらす の せいと は さんじゅうにん です	
\\	このクラスの
\\	は 30人[さんじゅうにん]です。			
\\	見学	見学[けんがく]	けんがく	
\\	今日、工場の見学に行きました。	今日[きょう]、 工場[こうじょう]の 見学[けんがく]に 行[い]きました。	きょう こうじょう の けんがく に いきました	
\\	今日[きょう]、 工場[こうじょう]の
\\	に 行[い]きました。			
\\	通学	通学[つうがく]	つうがく	
\\	毎朝、通学に1時間かかる。	毎朝[まいあさ]、 通学[つうがく]に 1時間[いちじかん]かかる。	まいあさ つうがく に いちじかん かかる	
\\	毎朝[まいあさ]、
\\	に 1時間[いちじかん]かかる。			
\\	高校	高校[こうこう]	こうこう	
\\	妹は高校に通っています。	妹[いもうと]は 高校[こうこう]に 通[かよ]っています。	いもうと は こうこう に かよって います	
\\	妹[いもうと]は
\\	に 通[かよ]っています。			
\\	育てる	育[そだ]てる	そだてる	
\\	彼女は三人の子を育てました。	彼女[かのじょ]は 三人[さんにん]の 子[こ]を 育[そだ]てました。	かのじょ は さんにん の こ を そだてました	
\\	彼女[かのじょ]は 三人[さんにん]の 子[こ]を
\\	育つ	育[そだ]つ	そだつ	
\\	野菜がよく育っている。	野菜[やさい]がよく 育[そだ]っている。	やさい が よく そだって いる	
\\	野菜[やさい]がよく
\\	練習	練習[れんしゅう]	れんしゅう	
\\	娘は今、バイオリンを練習しています。	娘[むすめ]は 今[いま]、バイオリンを 練習[れんしゅう]しています。	むすめ は いま ばいおりん を れんしゅう して います	
\\	娘[むすめ]は 今[いま]、バイオリンを
\\	しています。			
\\	習う	習[なら]う	ならう	
\\	彼は空手を習っています。	彼[かれ]は 空手[からて]を 習[なら]っています。	かれ は からて を ならって います	
\\	彼[かれ]は 空手[からて]を
\\	習慣	習慣[しゅうかん]	しゅうかん	
\\	毎朝コーヒーを飲むのが習慣です。	毎朝[まいあさ]コーヒーを 飲[の]むのが 習慣[しゅうかん]です。	まいあさ こーひー を のむ の が しゅうかん です	
\\	毎朝[まいあさ]コーヒーを 飲[の]むのが
\\	です。			
\\	研究	研究[けんきゅう]	けんきゅう	
\\	彼は何年も地震の研究をしている。	彼[かれ]は 何年[なんねん]も 地震[じしん]の 研究[けんきゅう]をしている。	かれ は なんねん も じしん の けんきゅう を して いる	
\\	彼[かれ]は 何年[なんねん]も 地震[じしん]の
\\	をしている。			
\\	試験	試験[しけん]	しけん	
\\	明日の試験、頑張ってね。	明日[あした]の 試験[しけん]、 頑張[がんば]ってね。	あした の しけん がんばって ね	
\\	明日[あした]の
\\	、 頑張[がんば]ってね。			
\\	簡単	簡単[かんたん]	かんたん	
\\	この料理はとても簡単です。	この 料理[りょうり]はとても 簡単[かんたん]です。	この りょうり は とても かんたん です	
\\	この 料理[りょうり]はとても
\\	です。			
\\	複雑	複雑[ふくざつ]	ふくざつ	
\\	このプログラムはとても複雑です。	このプログラムはとても 複雑[ふくざつ]です。	この ぷろぐらむ は とても ふくざつ です	
\\	このプログラムはとても
\\	です。			
\\	規則	規則[きそく]	きそく	
\\	あの会社の規則は厳しいです。	あの 会社[かいしゃ]の 規則[きそく]は 厳[きび]しいです。	あの かいしゃ の きそく は きびしい です	
\\	あの 会社[かいしゃ]の
\\	は 厳[きび]しいです。			
\\	経験	経験[けいけん]	けいけん	
\\	今日の試合はいい経験になりました。	今日[きょう]の 試合[しあい]はいい 経験[けいけん]になりました。	きょう の しあい は いい けいけん に なりました	
\\	今日[きょう]の 試合[しあい]はいい
\\	になりました。			
\\	経つ	経[た]つ	たつ	
\\	あれから14年が経ちました。	あれから 14年[じゅうよねん]が 経[た]ちました。	あれ から じゅうよねん が たちました	
\\	あれから 14年[じゅうよねん]が
\\	経済	経済[けいざい]	けいざい	
\\	大学で経済を勉強しました。	大学[だいがく]で 経済[けいざい]を 勉強[べんきょう]しました。	だいがく で けいざい を べんきょう しました	
\\	大学[だいがく]で
\\	を 勉強[べんきょう]しました。			
\\	経営	経営[けいえい]	けいえい	
\\	我が社の経営はうまくいっています。	我[わ]が 社[しゃ]の 経営[けいえい]はうまくいっています。	わがしゃ の けいえい は うまく いって います	
\\	我[わ]が 社[しゃ]の
\\	はうまくいっています。			
\\	産業	産業[さんぎょう]	さんぎょう	
\\	日本の自動車産業は世界的に有名だ。	日本[にほん]の 自動車[じどうしゃ] 産業[さんぎょう]は 世界的[せかいてき]に 有名[ゆうめい]だ。	にほん の じどうしゃ さんぎょう は せかいてき に ゆうめい だ	
\\	日本[にほん]の 自動車[じどうしゃ]
\\	は 世界的[せかいてき]に 有名[ゆうめい]だ。			
\\	製造	製造[せいぞう]	せいぞう	
\\	この工場ではエアコンを製造しています。	この 工場[こうじょう]ではエアコンを 製造[せいぞう]しています。	この こうじょう で は えあこん を せいぞう して います	
\\	この 工場[こうじょう]ではエアコンを
\\	しています。			
\\	重要	重要[じゅうよう]	じゅうよう	
\\	これは重要な書類です。	これは 重要[じゅうよう]な 書類[しょるい]です。	これ は じゅうよう な しょるい です	
\\	これは
\\	な 書類[しょるい]です。			
\\	要求	要求[ようきゅう]	ようきゅう	
\\	彼は私の要求にこたえた。	彼[かれ]は 私[わたし]の 要求[ようきゅう]にこたえた。	かれ は わたし の ようきゅう に こたえた	
\\	彼[かれ]は 私[わたし]の
\\	にこたえた。			
\\	目覚まし時計	目覚[めざ]まし 時計[どけい]	めざましどけい	
\\	7時に目覚まし時計が鳴りました。	7時[しちじ]に 目覚[めざ]まし 時計[どけい]が 鳴[な]りました。	しちじ に めざましどけい が なりました	
\\	7時[しちじ]に
\\	が 鳴[な]りました。			
\\	計算	計算[けいさん]	けいさん	
\\	その計算は間違っている。	その 計算[けいさん]は 間違[まちが]っている。	その けいさん は まちがって いる	
\\	その
\\	は 間違[まちが]っている。			
\\	足し算	足[た]し 算[ざん]	たしざん	
\\	娘は学校で足し算を習った。	娘[むすめ]は 学校[がっこう]で 足[た]し 算[ざん]を 習[なら]った。	むすめ は がっこう で たしざん を ならった	
\\	娘[むすめ]は 学校[がっこう]で
\\	を 習[なら]った。			
\\	返す	返[かえ]す	かえす	
\\	図書館に本を返した。	図書館[としょかん]に 本[ほん]を 返[かえ]した。	としょかん に ほん を かえした	
\\	図書館[としょかん]に 本[ほん]を
\\	返事	返事[へんじ]	へんじ	
\\	手紙の返事を出しました。	手紙[てがみ]の 返事[へんじ]を 出[だ]しました。	てがみ の へんじ を だしました	
\\	手紙[てがみ]の
\\	を 出[だ]しました。			
\\	貸す	貸[か]す	かす	
\\	私は彼に本を貸しています。	私[わたし]は 彼[かれ]に 本[ほん]を 貸[か]しています。	わたし は かれ に ほん を かして います	
\\	私[わたし]は 彼[かれ]に 本[ほん]を
\\	貸し出す	貸[か]し 出[だ]す	かしだす	
\\	その本は貸し出し中です。	その 本[ほん]は 貸[か]し 出[だ]し 中[ちゅう]です。	その ほん は かしだしちゅう です	
\\	その 本[ほん]は
\\	中[ちゅう]です。			
\\	申し込む	申[もう]し 込[こ]む	もうしこむ	
\\	彼女はそのセミナーに申し込んだ。	彼女[かのじょ]はそのセミナーに 申[もう]し 込[こ]んだ。	かのじょ は その せみなー に もうしこんだ	
\\	彼女[かのじょ]はそのセミナーに
\\	限る	限[かぎ]る	かぎる	
\\	このサービスは週末に限ります	このサービスは 週末[しゅうまつ]に 限[かぎ]ります	この さーびす は しゅうまつ に かぎります	
\\	このサービスは 週末[しゅうまつ]に
\\	約束	約束[やくそく]	やくそく	
\\	約束は守ります。	約束[やくそく]は 守[まも]ります。	やくそく は まもります	
\\	は 守[まも]ります。			
\\	減る	減[へ]る	へる	
\\	体重がかなり減りました。	体重[たいじゅう]がかなり 減[へ]りました。	たいじゅう が かなり へりました	
\\	体重[たいじゅう]がかなり
\\	減らす	減[へ]らす	へらす	
\\	最近、食事を減らしています。	最近[さいきん]、 食事[しょくじ]を 減[へ]らしています。	さいきん しょくじ を へらして います	
\\	最近[さいきん]、 食事[しょくじ]を
\\	降りる	降[お]りる	おりる	
\\	次の駅で降ります。	次[つぎ]の 駅[えき]で 降[お]ります。	つぎ の えき で おります。	
\\	次[つぎ]の 駅[えき]で
\\	降ろす	降[お]ろす	おろす	
\\	彼は車から荷物を降ろした。	彼[かれ]は 車[くるま]から 荷物[にもつ]を 降[お]ろした。	かれ は くるま から にもつ を おろした	
\\	彼[かれ]は 車[くるま]から 荷物[にもつ]を
\\	降る	降[ふ]る	ふる	
\\	激しい雨が降っています。	激[はげ]しい 雨[あめ]が 降[ふ]っています。	はげしい あめ が ふって います	
\\	激[はげ]しい 雨[あめ]が
\\	着く	着[つ]く	つく	
\\	午後8時に大阪に着きます。	午後8時[ごご はちじ]に 大阪[おおさか]に 着[つ]きます。	ごご はちじ に おおさか に つきます	
\\	午後8時[ごご はちじ]に 大阪[おおさか]に
\\	着物	着物[きもの]	きもの	
\\	彼女は着物がよく似合います。	彼女[かのじょ]は 着物[きもの]がよく 似合[にあ]います。	かのじょ は きもの が よく にあいます	
\\	彼女[かのじょ]は
\\	がよく 似合[にあ]います。			
\\	着せる	着[き]せる	きせる	
\\	娘に可愛いドレスを着せた。	娘[むすめ]に 可愛[かわい]いドレスを 着[き]せた。	むすめ に かわいい どれす を きせた	
\\	娘[むすめ]に 可愛[かわい]いドレスを
\\	脱ぐ	脱[ぬ]ぐ	ぬぐ	
\\	靴を脱いでください。	靴[くつ]を 脱[ぬ]いでください。	くつ を ぬいで ください	
\\	靴[くつ]を
\\	ください。			
\\	立場	立場[たちば]	たちば	
\\	彼女は自分の立場を分かっていない。	彼女[かのじょ]は 自分[じぶん]の 立場[たちば]を 分[わ]かっていない。	かのじょ は じぶん の たちば を わかって いない	
\\	彼女[かのじょ]は 自分[じぶん]の
\\	を 分[わ]かっていない。			
\\	目立つ	目立[めだ]つ	めだつ	
\\	彼女の大きな帽子はとても目立つ。	彼女[かのじょ]の 大[おお]きな 帽子[ぼうし]はとても 目立[めだ]つ。	かのじょ の おおき な ぼうし は とても めだつ	
\\	彼女[かのじょ]の 大[おお]きな 帽子[ぼうし]はとても
\\	立てる	立[た]てる	たてる	
\\	彼はケーキにろうそくを立てた。	彼[かれ]はケーキにろうそくを 立[た]てた。	かれ は けーき に ろうそく を たてた	
\\	彼[かれ]はケーキにろうそくを
\\	運ぶ	運[はこ]ぶ	はこぶ	
\\	いすを二階に運んでください。	いすを 二階[にかい]に 運[はこ]んでください。	いす を にかい に はこんで ください	
\\	いすを 二階[にかい]に
\\	ください。			
\\	運転	運転[うんてん]	うんてん	
\\	父は安全運転だ。	父[ちち]は 安全[あんぜん] 運転[うんてん]だ。	ちち は あんぜん うんてん だ	
\\	父[ちち]は 安全[あんぜん]
\\	だ。			
\\	運転手	運転手[うんてんしゅ]	うんてんしゅ	
\\	彼はタクシーの運転手です。	彼[かれ]はタクシーの 運転手[うんてんしゅ]です。	かれ は たくしー の うんてんしゅ です	
\\	彼[かれ]はタクシーの
\\	です。			
\\	転ぶ	転[ころ]ぶ	ころぶ	
\\	お祖母ちゃんがお風呂場で転んだ。	お 祖母[ばあ]ちゃんがお 風呂場[ふろば]で 転[ころ]んだ。	おばあちゃん が おふろば で ころんだ	
\\	お 祖母[ばあ]ちゃんがお 風呂場[ふろば]で
\\	移る	移[うつ]る	うつる	
\\	彼女は新しい会社に移った。	彼女[かのじょ]は 新[あたら]しい 会社[かいしゃ]に 移[うつ]った。	かのじょ は あたらしい かいしゃ に うつった	
\\	彼女[かのじょ]は 新[あたら]しい 会社[かいしゃ]に
\\	移す	移[うつ]す	うつす	
\\	机を窓の傍に移しました。	机[つくえ]を 窓[まど]の 傍[そば]に 移[うつ]しました。	つくえ を まど の そば に うつしました	
\\	机[つくえ]を 窓[まど]の 傍[そば]に
\\	運動	運動[うんどう]	うんどう	
\\	彼はもっと運動した方がいい。	彼[かれ]はもっと 運動[うんどう]した 方[ほう]がいい。	かれ は もっと うんどう した ほう が いい	
\\	彼[かれ]はもっと
\\	した 方[ほう]がいい。			
\\	速さ	速[はや]さ	はやさ	
\\	新幹線の速さはどれ位ですか。	新幹線[しんかんせん]の 速[はや]さはどれ 位[くらい]ですか。	しんかんせん の はやさ は どれ くらい です か	
\\	新幹線[しんかんせん]の
\\	はどれ 位[くらい]ですか。			
\\	速達	速達[そくたつ]	そくたつ	
\\	これを速達で送りたいのですが。	これを 速達[そくたつ]で 送[おく]りたいのですが。	これ を そくたつ で おくりたい の です が	
\\	これを
\\	で 送[おく]りたいのですが。			
\\	遅れる	遅[おく]れる	おくれる	
\\	今朝彼女は学校に遅れました。	今朝彼女[けさ かのじょ]は 学校[がっこう]に 遅[おく]れました。	けさ かのじょ は がっこう に おくれました	
\\	今朝彼女[けさ かのじょ]は 学校[がっこう]に
\\	開始	開始[かいし]	かいし	
\\	運動会は9時開始です。	運動会[うんどうかい]は 9時[くじ] 開始[かいし]です。	うんどうかい は くじ かいし です	
\\	運動会[うんどうかい]は 9時[くじ]
\\	です。			
\\	現在	現在[げんざい]	げんざい	
\\	現在の気温は30度です。	現在[げんざい]の 気温[きおん]は 30度[さんじゅうど]です。	げんざい の きおん は さんじゅうど です	
\\	の 気温[きおん]は 30度[さんじゅうど]です。			
\\	通過	通過[つうか]	つうか	
\\	次の駅は通過します。	次[つぎ]の 駅[えき]は 通過[つうか]します。	つぎ の えき は つうか します	
\\	次[つぎ]の 駅[えき]は
\\	します。			
\\	過ぎる	過[す]ぎる	すぎる	
\\	時が過ぎるのは速い。	時[とき]が 過[す]ぎるのは 速[はや]い。	とき が すぎる の は はやい 。	
\\	時[とき]が
\\	のは 速[はや]い。			
\\	過去	過去[かこ]	かこ	
\\	それは過去の話だ。	それは 過去[かこ]の 話[はなし]だ。	それ は かこ の はなし だ	
\\	それは
\\	の 話[はなし]だ。			
\\	開発	開発[かいはつ]	かいはつ	
\\	ダムの開発に住民は反対しています。	ダムの 開発[かいはつ]に 住民[じゅうみん]は 反対[はんたい]しています。	だむ の かいはつ に じゅうみん は はんたい して います	
\\	ダムの
\\	に 住民[じゅうみん]は 反対[はんたい]しています。			
\\	発生	発生[はっせい]	はっせい	
\\	交差点で事故が発生した。	交差点[こうさてん]で 事故[じこ]が 発生[はっせい]した。	こうさてん で じこ が はっせい した	
\\	交差点[こうさてん]で 事故[じこ]が
\\	した。			
\\	発言	発言[はつげん]	はつげん	
\\	会議で全員が発言した。	会議[かいぎ]で 全員[ぜんいん]が 発言[はつげん]した。	かいぎ で ぜんいん が はつげん した	
\\	会議[かいぎ]で 全員[ぜんいん]が
\\	した。			
\\	発見	発見[はっけん]	はっけん	
\\	新しい星が発見された。	新[あたら]しい 星[ほし]が 発見[はっけん]された。	あたらしい ほし が はっけん された	
\\	新[あたら]しい 星[ほし]が
\\	された。			
\\	発車	発車[はっしゃ]	はっしゃ	
\\	バスが発車します。	バスが 発車[はっしゃ]します。	ばす が はっしゃ します	
\\	バスが
\\	します。			
\\	発表	発表[はっぴょう]	はっぴょう	
\\	合格者が発表された。	合格者[ごうかくしゃ]が 発表[はっぴょう]された。	ごうかくしゃ が はっぴょう された	
\\	合格者[ごうかくしゃ]が
\\	された。			
\\	表現	表現[ひょうげん]	ひょうげん	
\\	彼は歌で自分の気持ちを表現した。	彼[かれ]は 歌[うた]で 自分[じぶん]の 気持[きも]ちを 表現[ひょうげん]した。	かれ は うた で じぶん の きもち を ひょうげん した	
\\	彼[かれ]は 歌[うた]で 自分[じぶん]の 気持[きも]ちを
\\	した。			
\\	表	表[おもて]	おもて	
\\	表に人が来ています。	表[おもて]に 人[ひと]が 来[き]ています。	おもてに ひと が きて います	
\\	に 人[ひと]が 来[き]ています。			
\\	表	表[ひょう]	ひょう	
\\	この表を見てください。	この 表[ひょう]を 見[み]てください。	この ひょう を みて ください	
\\	この
\\	を 見[み]てください。			
\\	絵	絵[え]	え	
\\	これは有名な画家の絵です。	これは 有名[ゆうめい]な 画家[がか]の 絵[え]です。	これ は ゆうめい な がか の え です	
\\	これは 有名[ゆうめい]な 画家[がか]の
\\	です。			
\\	雑誌	雑誌[ざっし]	ざっし	
\\	この雑誌はよく売れています。	この 雑誌[ざっし]はよく 売[う]れています。	この ざっし は よく うれて います	
\\	この
\\	はよく 売[う]れています。			
\\	音	音[おと]	おと	
\\	雨の音が聞こえる。	雨[あめ]の 音[おと]が 聞[き]こえる。	あめ の おと が きこえる	
\\	雨[あめ]の
\\	が 聞[き]こえる。			
\\	音	音[おと]	おと	
\\	このピアノは音があまり良くない。	このピアノは 音[おと]があまり 良[よ]くない。	この ぴあの は おと が あまり よく ない	
\\	このピアノは
\\	があまり 良[よ]くない。			
\\	音楽	音楽[おんがく]	おんがく	
\\	私は音楽を聞くのが好きだ。	私[わたし]は 音楽[おんがく]を 聞[き]くのが 好[す]きだ。	わたし は おんがく を きく の が すき だ	
\\	私[わたし]は
\\	を 聞[き]くのが 好[す]きだ。			
\\	薬	薬[くすり]	くすり	
\\	この薬を必ず飲んでください。	この 薬[くすり]を 必[かなら]ず 飲[の]んでください。	この くすり を かならず のんで ください	
\\	この
\\	を 必[かなら]ず 飲[の]んでください。			
\\	計画	計画[けいかく]	けいかく	
\\	彼は一人旅の計画を立てた。	彼[かれ]は 一人旅[ひとりたび]の 計画[けいかく]を 立[た]てた。	かれ は ひとりたび の けいかく を たてた	
\\	彼[かれ]は 一人旅[ひとりたび]の
\\	を 立[た]てた。			
\\	面白い	面白[おもしろ]い	おもしろい	
\\	この本は全然面白くなかった。	この 本[ほん]は全然[ぜんぜん] 面白[おもしろ]くなかった。	この ほん は ぜんぜん おもしろく なかった	
\\	この 本[ほん]は 全然[ぜんぜん]
\\	真っ赤	真[ま]っ 赤[か]	まっか	
\\	彼の顔は真っ赤でした。	彼[かれ]の 顔[かお]は 真[ま]っ 赤[か]でした。	かれ の かお は まっか でした	
\\	彼[かれ]の 顔[かお]は
\\	でした。			
\\	真面目	真面目[まじめ]	まじめ	
\\	彼は真面目な人です。	彼[かれ]は 真面目[まじめ]な 人[ひと]です。	かれ は まじめ な ひと です	
\\	彼[かれ]は
\\	な 人[ひと]です。			
\\	真ん中	真[ま]ん 中[なか]	まんなか	
\\	道の真ん中に人が立っている。	道[みち]の 真[ま]ん 中[なか]に 人[ひと]が 立[た]っている。	みち の まんなか に ひと が たって いる	
\\	道[みち]の
\\	に 人[ひと]が 立[た]っている。			
\\	真っ白	真[ま]っ 白[しろ]	まっしろ	
\\	外は雪で真っ白だった。	外[そと]は 雪[ゆき]で 真[ま]っ 白[しろ]だった。	そと は ゆき で まっしろ だった	
\\	外[そと]は 雪[ゆき]で
\\	だった。			
\\	真っ暗	真[ま]っ 暗[くら]	まっくら	
\\	外は真っ暗です。	外[そと]は 真[ま]っ 暗[くら]です。	そと は まっくら です	
\\	外[そと]は
\\	です。			
\\	真っ黒	真[ま]っ 黒[くろ]	まっくろ	
\\	インクで手が真っ黒になった	インクで 手[て]が 真[ま]っ 黒[くろ]になった	いんく で て が まっくろ に なった	
\\	インクで 手[て]が
\\	になった			
\\	真っ青	真[ま]っ 青[さお]	まっさお	
\\	空が真っ青です。	空[そら]が 真[ま]っ 青[さお]です。	そら が まっさお です	
\\	空[そら]が
\\	です。			
\\	色々	色々[いろいろ]	いろいろ	
\\	彼は色々なことを知っている。	彼[かれ]は 色々[いろいろ]なことを 知[し]っている。	かれ は いろいろ な こと を しって いる	
\\	彼[かれ]は
\\	なことを 知[し]っている。			
\\	種類	種類[しゅるい]	しゅるい	
\\	バラには色々な種類があります。	バラには 色々[いろいろ]な 種類[しゅるい]があります。	ばら に は いろいろ な しゅるい が あります	
\\	バラには 色々[いろいろ]な
\\	があります。			
\\	直す	直[なお]す	なおす	
\\	私がそれを直しました。	私[わたし]がそれを 直[なお]しました。	わたし が それ を なおしました	
\\	私[わたし]がそれを
\\	真っ直ぐ	真[ま]っ 直[す]ぐ	まっすぐ	
\\	この道を真っ直ぐ行ってください。	この 道[みち]を 真[ま]っ 直[す]ぐ 行[い]ってください。	この みち を まっすぐ いって ください	
\\	この 道[みち]を
\\	行[い]ってください。			
\\	直接	直接[ちょくせつ]	ちょくせつ	
\\	彼に直接お願いしなさい。	彼[かれ]に 直接[ちょくせつ]お 願[ねが]いしなさい。	かれ に ちょくせつ おねがい しなさい	
\\	彼[かれ]に
\\	お 願[ねが]いしなさい。			
\\	角	角[かど]	かど	
\\	次の角で左に曲がってください。	次[つぎ]の 角[かど]で 左[ひだり]に 曲[ま]がってください。	つぎ の かど で ひだり に まがって ください	
\\	次[つぎ]の
\\	で 左[ひだり]に 曲[ま]がってください。			
\\	間違い	間違[まちが]い	まちがい	
\\	この文には間違いがあります。	この 文[ぶん]には 間違[まちが]いがあります。	この ぶん に は まちがい が あります	
\\	この 文[ぶん]には
\\	があります。			
\\	間違える	間違[まちが]える	まちがえる	
\\	電話番号を間違えました。	電話番号[でんわ ばんごう]を 間違[まちが]えました。	でんわ ばんごう を まちがえました	
\\	電話番号[でんわ ばんごう]を
\\	間違う	間違[まちが]う	まちがう	
\\	あなたは間違っている。	あなたは 間違[まちが]っている。	あなた は まちがって いる	
\\	あなたは
\\	遊び	遊[あそ]び	あそび	
\\	お正月には色々な遊びをします。	お 正月[しょうがつ]には 色々[いろいろ]な 遊[あそ]びをします。	おしょうがつ に は いろいろ な あそび を します	
\\	お 正月[しょうがつ]には 色々[いろいろ]な
\\	をします。			
\\	窓	窓[まど]	まど	
\\	窓を開けてください。	窓[まど]を 開[あ]けてください。	まど を あけて ください	
\\	を 開[あ]けてください。			
\\	親しい	親[した]しい	したしい	
\\	週末、親しい友達を家に呼んだ。	週末[しゅうまつ]、 親[した]しい 友達[ともだち]を 家[いえ]に 呼[よ]んだ。	しゅうまつ したしい ともだち を いえ に よんだ	
\\	週末[しゅうまつ]、
\\	友達[ともだち]を 家[いえ]に 呼[よ]んだ。			
\\	親切	親切[しんせつ]	しんせつ	
\\	親切にしてくださってどうもありがとうございます。	親切[しんせつ]にしてくださってどうもありがとうございます。	しんせつ に して くださって どうも ありがとう ございます	
\\	にしてくださってどうもありがとうございます。			
\\	若者	若者[わかもの]	わかもの	
\\	最近の若者は本を読まない。	最近[さいきん]の 若者[わかもの]は 本[ほん]を 読[よ]まない。	さいきん の わかもの は ほん を よまない	
\\	最近[さいきん]の
\\	は 本[ほん]を 読[よ]まない。			
\\	結合	結合[けつごう]	けつごう	
\\	ファイルを結合して1つにしました。	ファイルを 結合[けつごう]して 1[ひと]つにしました。	ふぁいる を けつごう して ひとつ に しました	
\\	ファイルを
\\	して 1[ひと]つにしました。			
\\	結果	結果[けっか]	けっか	
\\	試合の結果を早く知りたい。	試合[しあい]の 結果[けっか]を 早[はや]く 知[し]りたい。	しあい の けっか を はやく しりたい	
\\	試合[しあい]の
\\	を 早[はや]く 知[し]りたい。			
\\	課題	課題[かだい]	かだい	
\\	夏休みの課題は何ですか。	夏休[なつやす]みの 課題[かだい]は 何[なん]ですか。	なつやすみ の かだい は なん です か	
\\	夏休[なつやす]みの
\\	は 何[なん]ですか。			
\\	課	課[か]	か	
\\	今から課のミーティングがある。	今[いま]から 課[か]のミーティングがある。	いま から か の みーてぃんぐ が ある	
\\	今[いま]から
\\	のミーティングがある。			
\\	自動車	自動車[じどうしゃ]	じどうしゃ	
\\	彼は自動車会社に就職した。	彼[かれ]は 自動車[じどうしゃ] 会社[がいしゃ]に 就職[しゅうしょく]した。	かれ は じどうしゃ がいしゃ に しゅうしょく した	
\\	彼[かれ]は
\\	会社[がいしゃ]に 就職[しゅうしょく]した。			
\\	自然	自然[しぜん]	しぜん	
\\	みんなで自然を守りましょう。	みんなで 自然[しぜん]を 守[まも]りましょう。	みんな で しぜん を まもりましょう	
\\	みんなで
\\	を 守[まも]りましょう。			
\\	自然	自然[しぜん]	しぜん	
\\	親が子供を守るのは自然なことだ。	親[おや]が 子供[こども]を 守[まも]るのは 自然[しぜん]なことだ。	おや が こども を まもる の は しぜん な こと だ	
\\	親[おや]が 子供[こども]を 守[まも]るのは
\\	なことだ。			
\\	自習	自習[じしゅう]	じしゅう	
\\	明日は自習の時間がある。	明日[あす]は 自習[じしゅう]の 時間[じかん]がある。	あす は じしゅう の じかん が ある	
\\	明日[あす]は
\\	の 時間[じかん]がある。			
\\	理由	理由[りゆう]	りゆう	
\\	遅れた理由を教えてください。	遅[おく]れた 理由[りゆう]を 教[おし]えてください。	おくれた りゆう を おしえて ください	
\\	遅[おく]れた
\\	を 教[おし]えてください。			
\\	自由	自由[じゆう]	じゆう	
\\	今日は自由な時間が多い。	今日[きょう]は 自由[じゆう]な 時間[じかん]が 多[おお]い。	きょう は じゆう な じかん が おおい	
\\	今日[きょう]は
\\	な 時間[じかん]が 多[おお]い。			
\\	頼む	頼[たの]む	たのむ	
\\	私はハンバーガーを頼みました。	私[わたし]はハンバーガーを 頼[たの]みました。	わたし は はんばーがー を たのみました	
\\	私[わたし]はハンバーガーを
\\	議員	議員[ぎいん]	ぎいん	
\\	彼は国会議員です。	彼[かれ]は 国会[こっかい] 議員[ぎいん]です。	かれ は こっかい ぎいん です	
\\	彼[かれ]は 国会[こっかい]
\\	です。			
\\	答え	答[こた]え	こたえ	
\\	彼はその問題の答えが分からない。	彼[かれ]はその 問題[もんだい]の 答[こた]えが 分[わ]からない。	かれ は その もんだい の こたえ が わからない	
\\	彼[かれ]はその 問題[もんだい]の
\\	が 分[わ]からない。			
\\	特に	特[とく]に	とくに	
\\	特に質問はありません。	特[とく]に 質問[しつもん]はありません。	とくに しつもん は ありません	
\\	質問[しつもん]はありません。			
\\	特急	特急[とっきゅう]	とっきゅう	
\\	東京まで特急で3時間かかります。	東京[とうきょう]まで 特急[とっきゅう]で 3時間[さんじかん]かかります。	とうきょう まで とっきゅう で さんじかん かかります	
\\	東京[とうきょう]まで
\\	で 3時間[さんじかん]かかります。			
\\	特別	特別[とくべつ]	とくべつ	
\\	あなたは私にとって特別な人です。	あなたは 私[わたし]にとって 特別[とくべつ]な 人[ひと]です。	あなた は わたし に とって とくべつ な ひと です	
\\	あなたは 私[わたし]にとって
\\	な 人[ひと]です。			
\\	目的	目的[もくてき]	もくてき	
\\	彼が来た目的が分かりません。	彼[かれ]が 来[き]た 目的[もくてき]が 分[わ]かりません。	かれ が きた もくてき が わかりません	
\\	彼[かれ]が 来[き]た
\\	が 分[わ]かりません。			
\\	高等学校	高等学校[こうとうがっこう]	こうとうがっこう	
\\	弟が高等学校を卒業しました。	弟[おとうと]が 高等学校[こうとうがっこう]を 卒業[そつぎょう]しました。	おとうと が こうとうがっこう を そつぎょう しました	
\\	弟[おとうと]が
\\	を 卒業[そつぎょう]しました。			
\\	病院	病院[びょういん]	びょういん	
\\	病院はどこですか。	病院[びょういん]はどこですか。	びょういん は どこ です か	
\\	はどこですか。			
\\	科学	科学[かがく]	かがく	
\\	科学は常に進歩している。	科学[かがく]は 常[つね]に 進歩[しんぽ]している。	かがく は つねに しんぽ して いる	
\\	は 常[つね]に 進歩[しんぽ]している。			
\\	理科	理科[りか]	りか	
\\	私は理科が得意です。	私[わたし]は 理科[りか]が 得意[とくい]です。	わたし は りか が とくい です	
\\	私[わたし]は
\\	が 得意[とくい]です。			
\\	疲れる	疲[つか]れる	つかれる	
\\	私は疲れてふらふらです。	私[わたし]は 疲[つか]れてふらふらです。	わたし は つかれて ふらふら です	
\\	私[わたし]は
\\	ふらふらです。			
\\	禁煙	禁煙[きんえん]	きんえん	
\\	彼は今、禁煙しています。	彼[かれ]は 今[いま]、 禁煙[きんえん]しています。	かれ は いま きんえん して います	
\\	彼[かれ]は 今[いま]、
\\	しています。			
\\	酔っ払い	酔[よ]っ 払[ぱら]い	よっぱらい	
\\	彼はただの酔っ払いです。	彼[かれ]はただの 酔[よ]っ 払[ぱら]いです。	かれ は ただ の よっぱらい です	
\\	彼[かれ]はただの
\\	です。			
\\	酔っ払う	酔[よ]っ 払[ぱら]う	よっぱらう	
\\	昨夜は酔っ払いました。	昨夜[ゆうべ]は 酔[よ]っ 払[ぱら]いました。	ゆうべ は よっぱらいました	
\\	昨夜[ゆうべ]は
\\	用意	用意[ようい]	ようい	
\\	食事の用意ができました。	食事[しょくじ]の 用意[ようい]ができました。	しょくじ の ようい が できました	
\\	食事[しょくじ]の
\\	ができました。			
\\	確か	確[たし]か	たしか	
\\	彼の昇進は確かだ。	彼[かれ]の 昇進[しょうしん]は 確[たし]かだ。	かれ の しょうしん は たしか だ	
\\	彼[かれ]の 昇進[しょうしん]は
\\	だ。			
\\	確認	確認[かくにん]	かくにん	
\\	もう一度、予約を確認した。	もう 一度[いちど]、 予約[よやく]を 確認[かくにん]した。	もう いちど よやく を かくにん した	
\\	もう 一度[いちど]、 予約[よやく]を
\\	した。			
\\	準備	準備[じゅんび]	じゅんび	
\\	明日の会議の準備をした。	明日[あす]の 会議[かいぎ]の 準備[じゅんび]をした。	あす の かいぎ の じゅんび を した	
\\	明日[あす]の 会議[かいぎ]の
\\	をした。			
\\	設計	設計[せっけい]	せっけい	
\\	兄は船の設計をしています。	兄[あに]は 船[ふね]の 設計[せっけい]をしています。	あに は ふね の せっけい を して います	
\\	兄[あに]は 船[ふね]の
\\	をしています。			
\\	設ける	設[もう]ける	もうける	
\\	授業の終わりに復習の時間を設けた。	授業[じゅぎょう]の 終[お]わりに 復習[ふくしゅう]の 時間[じかん]を 設[もう]けた。	じゅぎょう の おわり に ふくしゅう の じかん を もうけた	
\\	授業[じゅぎょう]の 終[お]わりに 復習[ふくしゅう]の 時間[じかん]を
\\	説明	説明[せつめい]	せつめい	
\\	この単語の意味を説明してください。	この 単語[たんご]の 意味[いみ]を 説明[せつめい]してください。	この たんご の いみ を せつめい して ください	
\\	この 単語[たんご]の 意味[いみ]を
\\	してください。			
\\	祭り	祭[まつ]り	まつり	
\\	彼女は祭りが大好きです。	彼女[かのじょ]は 祭[まつ]りが 大好[だいす]きです。	かのじょ は まつり が だいすき です 。	
\\	彼女[かのじょ]は
\\	が 大好[だいす]きです。			
\\	飛行場	飛行場[ひこうじょう]	ひこうじょう	
\\	バスが飛行場に着きました。	バスが 飛行場[ひこうじょう]に 着[つ]きました。	ばす が ひこうじょう に つきました	
\\	バスが
\\	に 着[つ]きました。			
\\	航空便	航空便[こうくうびん]	こうくうびん	
\\	航空便で書類が届きました。	航空便[こうくうびん]で 書類[しょるい]が 届[とど]きました。	こうくうびん で しょるい が とどきました	
\\	で 書類[しょるい]が 届[とど]きました。			
\\	船便	船便[ふなびん]	ふなびん	
\\	アメリカの友人から船便が届いた。	アメリカの 友人[ゆうじん]から 船便[ふなびん]が 届[とど]いた。	あめりか の ゆうじん から ふなびん が とどいた	
\\	アメリカの 友人[ゆうじん]から
\\	が 届[とど]いた。			
\\	船	船[ふね]	ふね	
\\	私たちは船に乗った。	私[わたし]たちは 船[ふね]に 乗[の]った。	わたしたち は ふね に のった	
\\	私[わたし]たちは
\\	に 乗[の]った。			
\\	空港	空港[くうこう]	くうこう	
\\	空港までリムジンバスで行った。	空港[くうこう]までリムジンバスで 行[い]った。	くうこう まで りむじん ばす で いった	
\\	までリムジンバスで 行[い]った。			
\\	港	港[みなと]	みなと	
\\	港に船が着きました。	港[みなと]に 船[ふね]が 着[つ]きました。	みなと に ふね が つきました	
\\	に 船[ふね]が 着[つ]きました。			
\\	資金	資金[しきん]	しきん	
\\	私たちは今、結婚資金を貯めています。	私[わたし]たちは 今[いま]、 結婚[けっこん] 資金[しきん]を 貯[た]めています。	わたしたち は いま けっこん しきん を ためて います	
\\	私[わたし]たちは 今[いま]、 結婚[けっこん]
\\	を 貯[た]めています。			
\\	願う	願[ねが]う	ねがう	
\\	彼が元気になるよう願っています。	彼[かれ]が 元気[げんき]になるよう 願[ねが]っています。	かれ が げんき に なる よう ねがって います	
\\	彼[かれ]が 元気[げんき]になるよう
\\	異なる	異[こと]なる	ことなる	
\\	彼と私はいつも意見が異なる。	彼[かれ]と 私[わたし]はいつも 意見[いけん]が 異[こと]なる。	かれ と わたし は いつも いけん が ことなる	
\\	彼[かれ]と 私[わたし]はいつも 意見[いけん]が
\\	通常	通常[つうじょう]	つうじょう	
\\	通常は夜8時まで営業しています。	通常[つうじょう]は 夜8時[よる はちじ]まで 営業[えいぎょう]しています。	つうじょう は よる はちじ まで えいぎょう して います	
\\	は 夜8時[よる はちじ]まで 営業[えいぎょう]しています。			
\\	非常に	非常[ひじょう]に	ひじょうに	
\\	これは非常に重要です。	これは 非常[ひじょう]に 重要[じゅうよう]です。	これ は ひじょう に じゅうよう です	
\\	これは
\\	重要[じゅうよう]です。			
\\	調べる	調[しら]べる	しらべる	
\\	この単語の意味を辞書で調べましょう。	この 単語[たんご]の 意味[いみ]を 辞書[じしょ]で 調[しら]べましょう。	この たんご の いみ を じしょ で しらべましょう	
\\	この 単語[たんご]の 意味[いみ]を 辞書[じしょ]で
\\	調査	調査[ちょうさ]	ちょうさ	
\\	私たちがその問題を調査しています。	私[わたし]たちがその 問題[もんだい]を 調査[ちょうさ]しています。	わたしたち が その もんだい を ちょうさ して います	
\\	私[わたし]たちがその 問題[もんだい]を
\\	しています。			
\\	示す	示[しめ]す	しめす	
\\	彼は新製品に興味を示している。	彼[かれ]は 新製品[しんせいひん]に 興味[きょうみ]を 示[しめ]している。	かれ は しんせいひん に きょうみ を しめして いる	
\\	彼[かれ]は 新製品[しんせいひん]に 興味[きょうみ]を
\\	連れて行く	連[つ]れて 行[い]く	つれていく	
\\	私も連れて行ってください。	私[わたし]も 連[つ]れて 行[い]ってください。	わたし も つれて いって ください	
\\	私[わたし]も
\\	ください。			
\\	連れて来る	連[つ]れて 来[く]る	つれてくる	
\\	息子が友達を連れて来ました。	息子[むすこ]が 友達[ともだち]を 連[つ]れて 来[き]ました。	むすこ が ともだち を つれて きました	
\\	息子[むすこ]が 友達[ともだち]を
\\	続ける	続[つづ]ける	つづける	
\\	仕事を続けてください。	仕事[しごと]を 続[つづ]けてください。	しごと を つづけて ください	
\\	仕事[しごと]を
\\	ください。			
\\	相手	相手[あいて]	あいて	
\\	試合の相手は誰ですか。	試合[しあい]の 相手[あいて]は 誰[だれ]ですか。	しあい の あいて は だれ です か	
\\	試合[しあい]の
\\	は 誰[だれ]ですか。			
\\	相談	相談[そうだん]	そうだん	
\\	相談したいことがあります。	相談[そうだん]したいことがあります。	そうだん したい こと が あります	
\\	したいことがあります。			
\\	記者	記者[きしゃ]	きしゃ	
\\	彼は新聞記者です。	彼[かれ]は 新聞[しんぶん] 記者[きしゃ]です。	かれ は しんぶん きしゃ です	
\\	彼[かれ]は 新聞[しんぶん]
\\	です。			
\\	記録	記録[きろく]	きろく	
\\	マラソンで世界記録が出た。	マラソンで 世界[せかい] 記録[きろく]が 出[で]た。	まらそん で せかい きろく が でた	
\\	マラソンで 世界[せかい]
\\	が 出[で]た。			
\\	録音	録音[ろくおん]	ろくおん	
\\	そのラジオ番組はもう録音しました。	そのラジオ 番組[ばんぐみ]はもう 録音[ろくおん]しました。	その らじお ばんぐみ は もう ろくおん しました	
\\	そのラジオ 番組[ばんぐみ]はもう
\\	しました。			
\\	登る	登[のぼ]る	のぼる	
\\	私たちは昨年、富士山に登りました。	私[わたし]たちは 昨年[さくねん]、 富士山[ふじさん]に 登[のぼ]りました。	わたしたち は さくねん ふじさん に のぼりました	
\\	私[わたし]たちは 昨年[さくねん]、 富士山[ふじさん]に
\\	関する	関[かん]する	かんする	
\\	その問題に関する記事を読みました。	その 問題[もんだい]に 関[かん]する 記事[きじ]を 読[よ]みました。	その もんだい に かんする きじ を よみました	
\\	その 問題[もんだい]に
\\	記事[きじ]を 読[よ]みました。			
\\	関連	関連[かんれん]	かんれん	
\\	関連のセクションへ連絡した。	関連[かんれん]のセクションへ 連絡[れんらく]した。	かんれん の せくしょん へ れんらく した	
\\	のセクションへ 連絡[れんらく]した。			
\\	関係	関係[かんけい]	かんけい	
\\	彼はその事件に関係がない。	彼[かれ]はその 事件[じけん]に 関係[かんけい]がない。	かれ は その じけん に かんけい が ない	
\\	彼[かれ]はその 事件[じけん]に
\\	がない。			
\\	状況	状況[じょうきょう]	じょうきょう	
\\	この状況では出発は難しいです。	この 状況[じょうきょう]では 出発[しゅっぱつ]は 難[むずか]しいです。	この じょうきょう で は しゅっぱつ は むずかしい です	
\\	この
\\	では 出発[しゅっぱつ]は 難[むずか]しいです。			
\\	状態	状態[じょうたい]	じょうたい	
\\	ここは道の状態がとても悪いです。	ここは 道[みち]の 状態[じょうたい]がとても 悪[わる]いです。	ここ は みち の じょうたい が とても わるい です	
\\	ここは 道[みち]の
\\	がとても 悪[わる]いです。			
\\	選ぶ	選[えら]ぶ	えらぶ	
\\	良い家を選ぶのは難しい。	良[よ]い 家[いえ]を 選[えら]ぶのは 難[むずか]しい。	よい いえ を えらぶ の は むずかしい	
\\	良[よ]い 家[いえ]を
\\	のは 難[むずか]しい。			
\\	選手	選手[せんしゅ]	せんしゅ	
\\	彼はプロのサッカー選手だ。	彼[かれ]はプロのサッカー 選手[せんしゅ]だ。	かれ は ぷろ の さっかー せんしゅ だ	
\\	彼[かれ]はプロのサッカー
\\	だ。			
\\	選挙	選挙[せんきょ]	せんきょ	
\\	彼は選挙に出るつもりだ。	彼[かれ]は 選挙[せんきょ]に 出[で]るつもりだ。	かれ は せんきょ に でる つもり だ	
\\	彼[かれ]は
\\	に 出[で]るつもりだ。			
\\	首相	首相[しゅしょう]	しゅしょう	
\\	今の首相はあまり力がない。	今[いま]の 首相[しゅしょう]はあまり 力[ちから]がない。	いま の しゅしょう は あまり ちから が ない	
\\	今[いま]の
\\	はあまり 力[ちから]がない。			
\\	首都	首都[しゅと]	しゅと	
\\	東京は日本の首都です。	東京[とうきょう]は 日本[にっぽん]の 首都[しゅと]です。	とうきょう は にっぽん の しゅと です	
\\	東京[とうきょう]は 日本[にっぽん]の
\\	です。			
\\	革	革[かわ]	かわ	
\\	革のベルトを買いました。	革[かわ]のベルトを 買[か]いました。	かわ の べると を かいました	
\\	のベルトを 買[か]いました。			
\\	番組	番組[ばんぐみ]	ばんぐみ	
\\	私はこの番組が好きです。	私[わたし]はこの 番組[ばんぐみ]が 好[す]きです。	わたし は この ばんぐみ が すき です	
\\	私[わたし]はこの
\\	が 好[す]きです。			
\\	組み立てる	組[く]み 立[た]てる	くみたてる	
\\	日曜日に本棚を組み立てます。	日曜日[にちようび]に 本棚[ほんだな]を 組[く]み 立[た]てます。	にちようび に ほんだな を くみたてます	
\\	日曜日[にちようび]に 本棚[ほんだな]を
\\	組織	組織[そしき]	そしき	
\\	彼はある組織のリーダーだ。	彼[かれ]はある 組織[そしき]のリーダーだ。	かれ は ある そしき の りーだー だ	
\\	彼[かれ]はある
\\	のリーダーだ。			
\\	進める	進[すす]める	すすめる	
\\	早く授業を進めましょう。	早[はや]く 授業[じゅぎょう]を 進[すす]めましょう。	はやく じゅぎょう を すすめましょう	
\\	早[はや]く 授業[じゅぎょう]を
\\	進む	進[すす]む	すすむ	
\\	前に進んでください。	前[まえ]に 進[すす]んでください。	まえ に すすんで ください	
\\	前[まえ]に
\\	ください。			
\\	進学	進学[しんがく]	しんがく	
\\	彼女は来年、大学に進学する。	彼女[かのじょ]は 来年[らいねん]、 大学[だいがく]に 進学[しんがく]する。	かのじょ は らいねん だいがく に しんがく する	
\\	彼女[かのじょ]は 来年[らいねん]、 大学[だいがく]に
\\	する。			
\\	責任	責任[せきにん]	せきにん	
\\	彼は失敗の責任を取って、会社を辞めた。	彼[かれ]は 失敗[しっぱい]の 責任[せきにん]を 取[と]って、 会社[かいしゃ]を 辞[や]めた。	かれ は しっぱい の せきにん を とって かいしゃ を やめた	
\\	彼[かれ]は 失敗[しっぱい]の
\\	を 取[と]って、 会社[かいしゃ]を 辞[や]めた。			
\\	辞める	辞[や]める	やめる	
\\	彼は会社を辞めます。	彼[かれ]は 会社[かいしゃ]を 辞[や]めます。	かれ は かいしゃ を やめます	
\\	彼[かれ]は 会社[かいしゃ]を
\\	辞書	辞書[じしょ]	じしょ	
\\	彼女は辞書をよく使います。	彼女[かのじょ]は 辞書[じしょ]をよく 使[つか]います。	かのじょ は じしょ を よく つかいます	
\\	彼女[かのじょ]は
\\	をよく 使[つか]います。			
\\	通勤	通勤[つうきん]	つうきん	
\\	毎朝、通勤に30分かかります。	毎朝[まいあさ]、 通勤[つうきん]に 30分[さんじゅっぷん]かかります。	まいあさ つうきん に さんじゅっぷん かかります	
\\	毎朝[まいあさ]、
\\	に 30分[さんじゅっぷん]かかります。			
\\	集める	集[あつ]める	あつめる	
\\	弟は切手を集めています。	弟[おとうと]は 切手[きって]を 集[あつ]めています。	おとうと は きって を あつめて います	
\\	弟[おとうと]は 切手[きって]を
\\	集まる	集[あつ]まる	あつまる	
\\	駅前に人が集まっています。	駅前[えきまえ]に 人[ひと]が 集[あつ]まっています。	えきまえ に ひと が あつまって います	
\\	駅前[えきまえ]に 人[ひと]が
\\	給料	給料[きゅうりょう]	きゅうりょう	
\\	来年から給料が上がります。	来年[らいねん]から 給料[きゅうりょう]が 上[あ]がります。	らいねん から きゅうりょう が あがります	
\\	来年[らいねん]から
\\	が 上[あ]がります。			
\\	退院	退院[たいいん]	たいいん	
\\	母が今日退院します。	母[はは]が 今日[きょう] 退院[たいいん]します。	はは が きょう たいいん します	
\\	母[はは]が 今日[きょう]
\\	します。			
\\	評価	評価[ひょうか]	ひょうか	
\\	最近、彼の評価が上がった。	最近[さいきん]、 彼[かれ]の 評価[ひょうか]が 上[あ]がった。	さいきん かれ の ひょうか が あがった	
\\	最近[さいきん]、 彼[かれ]の
\\	が 上[あ]がった。			
\\	着替える	着替[きが]える	きがえる	
\\	彼はきれいな服に着替えた。	彼[かれ]はきれいな 服[ふく]に 着替[きが]えた。	かれ は きれい な ふく に きがえた	
\\	彼[かれ]はきれいな 服[ふく]に
\\	被る	被[かぶ]る	かぶる	
\\	帽子を被って外出した。	帽子[ぼうし]を 被[かぶ]って 外出[がいしゅつ]した。	ぼうし を かぶって がいしゅつ した	
\\	帽子[ぼうし]を
\\	外出[がいしゅつ]した。			
\\	破る	破[やぶ]る	やぶる	
\\	彼は約束を破った。	彼[かれ]は 約束[やくそく]を 破[やぶ]った。	かれ は やくそく を やぶった	
\\	彼[かれ]は 約束[やくそく]を
\\	破れる	破[やぶ]れる	やぶれる	
\\	シャツが破れている。	シャツが 破[やぶ]れている。	しゃつ が やぶれて いる	
\\	シャツが
\\	立派	立派[りっぱ]	りっぱ	
\\	彼は立派な人です。	彼[かれ]は 立派[りっぱ]な 人[ひと]です。	かれ は りっぱ な ひと です	
\\	彼[かれ]は
\\	な 人[ひと]です。			
\\	警察	警察[けいさつ]	けいさつ	
\\	警察を呼んでください。	警察[けいさつ]を 呼[よ]んでください。	けいさつ を よんで ください	
\\	を 呼[よ]んでください。			
\\	管理	管理[かんり]	かんり	
\\	彼女が私のスケジュールを管理している。	彼女[かのじょ]が 私[わたし]のスケジュールを 管理[かんり]している。	かのじょ が わたし の すけじゅーる を かんり して いる	
\\	彼女[かのじょ]が 私[わたし]のスケジュールを
\\	している。			
\\	盗む	盗[ぬす]む	ぬすむ	
\\	誰かが私のカバンを盗みました。	誰[だれ]かが 私[わたし]のカバンを 盗[ぬす]みました。	だれか が わたし の かばん を ぬすみました	
\\	誰[だれ]かが 私[わたし]のカバンを
\\	逮捕	逮捕[たいほ]	たいほ	
\\	逃げていた犯人が逮捕されました。	逃[に]げていた 犯人[はんにん]が 逮捕[たいほ]されました。	にげて いた はんにん が たいほ されました	
\\	逃[に]げていた 犯人[はんにん]が
\\	されました。			
\\	逃げる	逃[に]げる	にげる	
\\	鳥が窓から逃げたよ。	鳥[とり]が 窓[まど]から 逃[に]げたよ。	とり が まど から にげた よ	
\\	鳥[とり]が 窓[まど]から
\\	よ。			
\\	競争	競争[きょうそう]	きょうそう	
\\	2社は互いに競争している。	2社[にしゃ]は 互[たが]いに 競争[きょうそう]している。	にしゃ は たがい に きょうそう して いる	
\\	2社[にしゃ]は 互[たが]いに
\\	している。			
\\	混む・込む	混[こ]む・ 込[こ]む	こむ・こむ	
\\	電車が込んでいる。	電車[でんしゃ]が 込[こ]んでいる。	でんしゃ が こんで いる	
\\	電車[でんしゃ]が
\\	電池	電池[でんち]	でんち	
\\	新しい電池を入れましょう。	新[あたら]しい 電池[でんち]を 入[い]れましょう。	あたらしい でんち を いれましょう	
\\	新[あたら]しい
\\	を 入[い]れましょう。			
\\	深い	深[ふか]い	ふかい	
\\	あの池はとても深い。	あの 池[いけ]はとても 深[ふか]い。	あの いけ は とても ふかい	
\\	あの 池[いけ]はとても
\\	深さ	深[ふか]さ	ふかさ	
\\	このプールの深さは2メートルです。	このプールの 深[ふか]さは 2[に]メートルです。	この ぷーる の ふかさ は にめーとる です	
\\	このプールの
\\	は 2[に]メートルです。			
\\	落とす	落[お]とす	おとす	
\\	途中で財布を落としました。	途中[とちゅう]で 財布[さいふ]を 落[お]としました。	とちゅう で さいふ を おとしました	
\\	途中[とちゅう]で 財布[さいふ]を
\\	涼しい	涼[すず]しい	すずしい	
\\	夕方は涼しくなりますよ。	夕方[ゆうがた]は 涼[すず]しくなりますよ。	ゆうがた は すずしく なります よ	
\\	夕方[ゆうがた]は
\\	なりますよ。			
\\	野球	野球[やきゅう]	やきゅう	
\\	友達と野球をしました。	友達[ともだち]と 野球[やきゅう]をしました。	ともだち と やきゅう を しました	
\\	友達[ともだち]と
\\	をしました。			
\\	雲	雲[くも]	くも	
\\	今日は雲が多い。	今日[きょう]は 雲[くも]が 多[おお]い。	きょう は くも が おおい	
\\	今日[きょう]は
\\	が 多[おお]い。			
\\	震える	震[ふる]える	ふるえる	
\\	彼は寒くて震えていました。	彼[かれ]は 寒[さむ]くて 震[ふる]えていました。	かれ は さむくて ふるえて いました	
\\	彼[かれ]は 寒[さむ]くて
\\	神社	神社[じんじゃ]	じんじゃ	
\\	京都には神社がたくさんある。	京都[きょうと]には 神社[じんじゃ]がたくさんある。	きょうと に は じんじゃ が たくさん ある	
\\	京都[きょうと]には
\\	がたくさんある。			
\\	秘密	秘密[ひみつ]	ひみつ	
\\	これは秘密です。	これは 秘密[ひみつ]です。	これ は ひみつ です	
\\	これは
\\	です。			
\\	離婚	離婚[りこん]	りこん	
\\	友人が離婚しました。	友人[ゆうじん]が 離婚[りこん]しました。	ゆうじん が りこん しました	
\\	友人[ゆうじん]が
\\	しました。			
\\	隣	隣[となり]	となり	
\\	隣の家には犬がいます。	隣[となり]の 家[いえ]には 犬[いぬ]がいます。	となり の いえ に は いぬ が います	
\\	の 家[いえ]には 犬[いぬ]がいます。			
\\	黄色	黄色[きいろ]	きいろ	
\\	信号は黄色でした。	信号[しんごう]は 黄色[きいろ]でした。	しんごう は きいろ でした	
\\	信号[しんごう]は
\\	でした。			
\\	訪れる	訪[おとず]れる	おとずれる	
\\	私は夏に京都を訪れました。	私[わたし]は 夏[なつ]に 京都[きょうと]を 訪[おとず]れました。	わたし は なつ に きょうと を おとずれました	
\\	私[わたし]は 夏[なつ]に 京都[きょうと]を
\\	訪ねる	訪[たず]ねる	たずねる	
\\	彼はニューヨークの友達を訪ねた。	彼[かれ]はニューヨークの 友達[ともだち]を 訪[たず]ねた。	かれ は にゅーよーく の ともだち を たずねた	
\\	彼[かれ]はニューヨークの 友達[ともだち]を
\\	諸国	諸国[しょこく]	しょこく	
\\	彼はヨーロッパ諸国を旅行した。	彼[かれ]はヨーロッパ 諸国[しょこく]を 旅行[りょこう]した。	かれ は よーろっぱ しょこく を りょこう した	
\\	彼[かれ]はヨーロッパ
\\	を 旅行[りょこう]した。			
\\	緑色	緑色[みどりいろ]	みどりいろ	
\\	彼は緑色のジャケットを着ています。	彼[かれ]は 緑色[みどりいろ]のジャケットを 着[き]ています。	かれ は みどりいろ の じゃけっと を きて います	
\\	彼[かれ]は
\\	のジャケットを 着[き]ています。			
\\	貿易	貿易[ぼうえき]	ぼうえき	
\\	父は貿易の仕事をしています。	父[ちち]は 貿易[ぼうえき]の 仕事[しごと]をしています。	ちち は ぼうえき の しごと を して います	
\\	父[ちち]は
\\	の 仕事[しごと]をしています。			
\\	輸入	輸入[ゆにゅう]	ゆにゅう	
\\	これはイタリアから輸入した服です。	これはイタリアから 輸入[ゆにゅう]した 服[ふく]です。	これ は いたりあ から ゆにゅう した ふく です	
\\	これはイタリアから
\\	した 服[ふく]です。			
\\	輸出	輸出[ゆしゅつ]	ゆしゅつ	
\\	彼の会社は車を輸出しています。	彼[かれ]の 会社[かいしゃ]は 車[くるま]を 輸出[ゆしゅつ]しています。	かれ の かいしゃ は くるま を ゆしゅつ して います	
\\	彼[かれ]の 会社[かいしゃ]は 車[くるま]を
\\	しています。			
\\	繰り返す	繰[く]り 返[かえ]す	くりかえす	
\\	彼女は同じ間違いを繰り返した。	彼女[かのじょ]は 同[おな]じ 間違[まちが]いを 繰[く]り 返[かえ]した。	かのじょ は おなじ まちがい を くりかえした	
\\	彼女[かのじょ]は 同[おな]じ 間違[まちが]いを
\\	留学	留学[りゅうがく]	りゅうがく	
\\	1年間、アメリカに留学しました。	1年間[いちねんかん]、アメリカに 留学[りゅうがく]しました。	いちねんかん あめりか に りゅうがく しました	
\\	1年間[いちねんかん]、アメリカに
\\	しました。			
\\	留守	留守[るす]	るす	
\\	父は今、留守です。	父[ちち]は 今[いま]、 留守[るす]です。	ちち は いま るす です	
\\	父[ちち]は 今[いま]、
\\	です。			
\\	自宅	自宅[じたく]	じたく	
\\	自宅に電話を下さい。	自宅[じたく]に 電話[でんわ]を 下[くだ]さい。	じたく に でんわ を ください	
\\	に 電話[でんわ]を 下[くだ]さい。			
\\	静か	静[しず]か	しずか	
\\	今年の夏休みには静かな所へ行きたい。	今年[ことし]の 夏休[なつやす]みには 静[しず]かな 所[ところ]へ 行[い]きたい。	ことし の なつやすみ に は しずか な ところ へ いきたい	
\\	今年[ことし]の 夏休[なつやす]みには
\\	な 所[ところ]へ 行[い]きたい。			
\\	趣味	趣味[しゅみ]	しゅみ	
\\	私の趣味は映画とテニスです。	私[わたし]の 趣味[しゅみ]は 映画[えいが]とテニスです。	わたし の しゅみ は えいが と てにす です	
\\	私[わたし]の
\\	は 映画[えいが]とテニスです。			
\\	片道	片道[かたみち]	かたみち	
\\	東京まで片道切符を買った。	東京[とうきょう]まで 片道[かたみち] 切符[きっぷ]を 買[か]った。	とうきょう まで かたみち きっぷ を かった	
\\	東京[とうきょう]まで
\\	切符[きっぷ]を 買[か]った。			
\\	裏	裏[うら]	うら	
\\	会社の裏に公園があります。	会社[かいしゃ]の 裏[うら]に 公園[こうえん]があります。	かいしゃ の うら に こうえん が あります	
\\	会社[かいしゃ]の
\\	に 公園[こうえん]があります。			
\\	裏返す	裏返[うらがえ]す	うらがえす	
\\	彼は紙を裏返しました。	彼[かれ]は 紙[かみ]を 裏返[うらがえ]しました。	かれ は かみ を うらがえしました	
\\	彼[かれ]は 紙[かみ]を
\\	週刊誌	週刊誌[しゅうかんし]	しゅうかんし	
\\	その週刊誌はゴシップばかりだ。	その 週刊誌[しゅうかんし]はゴシップばかりだ。	その しゅうかんし は ごしっぷ ばかり だ	
\\	その
\\	はゴシップばかりだ。			
\\	詳しい	詳[くわ]しい	くわしい	
\\	もっと詳しく説明してください。	もっと 詳[くわ]しく 説明[せつめい]してください。	もっと くわしく せつめい して ください	
\\	もっと
\\	説明[せつめい]してください。			
\\	細かい	細[こま]かい	こまかい	
\\	彼女は細かいことにうるさい。	彼女[かのじょ]は 細[こま]かいことにうるさい。	かのじょ は こまかい こと に うるさい	
\\	彼女[かのじょ]は
\\	ことにうるさい。			
\\	細か	細[こま]か	こまか	
\\	それは細かな問題です。	それは 細[こま]かな 問題[もんだい]です。	それ は こまか な もんだい です	
\\	それは
\\	な 問題[もんだい]です。			
\\	積もる	積[つ]もる	つもる	
\\	雪が積もっています。	雪[ゆき]が 積[つ]もっています。	ゆき が つもって います	
\\	雪[ゆき]が
\\	訳	訳[やく]	やく	
\\	その文の訳を読みました。	その 文[ぶん]の 訳[やく]を 読[よ]みました。	その ぶん の やく を よみました	
\\	その 文[ぶん]の
\\	を 読[よ]みました。			
\\	訳す	訳[やく]す	やくす	
\\	この文を訳してください。	この 文[ぶん]を 訳[やく]してください。	この ぶん を やくして ください	
\\	この 文[ぶん]を
\\	ください。			
\\	片付ける	片付[かたづ]ける	かたづける	
\\	早く部屋を片付けなさい。	早[はや]く 部屋[へや]を 片付[かたづ]けなさい。	はやく へや を かたづけなさい	
\\	早[はや]く 部屋[へや]を
\\	近付く	近付[ちかづ]く	ちかづく	
\\	女性が近付いてきました。	女性[じょせい]が 近付[ちかづ]いてきました。	じょせい が ちかづいて きました	
\\	女性[じょせい]が
\\	きました。			
\\	片付く	片付[かたづ]く	かたづく	
\\	仕事が大分片付いた。	仕事[しごと]が 大分[だいぶ] 片付[かたづ]いた。	しごと が だいぶ かたづいた	
\\	仕事[しごと]が 大分[だいぶ]
\\	貼る	貼[は]る	はる	
\\	机にシールを貼った。	机[つくえ]にシールを 貼[は]った。	つくえ に しーる を はった	
\\	机[つくえ]にシールを
\\	記念日	記念日[きねんび]	きねんび	
\\	今日は両親の結婚記念日です。	今日[きょう]は 両親[りょうしん]の 結婚[けっこん] 記念日[きねんび]です。	きょう は りょうしん の けっこん きねんび です	
\\	今日[きょう]は 両親[りょうしん]の 結婚[けっこん]
\\	です。			
\\	除く	除[のぞ]く	のぞく	
\\	部長を除く全員が土曜日も働いた。	部長[ぶちょう]を 除[のぞ]く 全員[ぜんいん]が 土曜日[どようび]も 働[はたら]いた。	ぶちょう を のぞく ぜんいん が どようび も はたらいた	
\\	部長[ぶちょう]を
\\	全員[ぜんいん]が 土曜日[どようび]も 働[はたら]いた。			
\\	遅刻	遅刻[ちこく]	ちこく	
\\	遅刻しないでください。	遅刻[ちこく]しないでください。	ちこく しない で ください	
\\	しないでください。			
\\	緩い	緩[ゆる]い	ゆるい	
\\	このズボンは緩いです。	このズボンは 緩[ゆる]いです。	この ずぼん は ゆるい です	
\\	このズボンは
\\	です。			
\\	苦しい	苦[くる]しい	くるしい	
\\	食べ過ぎておなかが苦しい。	食[た]べ 過[す]ぎておなかが 苦[くる]しい。	たべすぎて おなか が くるしい	
\\	食[た]べ 過[す]ぎておなかが
\\	苦い	苦[にが]い	にがい	
\\	私は苦いコーヒーが好きです。	私[わたし]は 苦[にが]いコーヒーが 好[す]きです。	わたし は にがい こーひー が すき です	
\\	私[わたし]は
\\	コーヒーが 好[す]きです。			
\\	苦手	苦手[にがて]	にがて	
\\	私は料理が苦手です。	私[わたし]は 料理[りょうり]が 苦手[にがて]です。	わたし は りょうり が にがて です	
\\	私[わたし]は 料理[りょうり]が
\\	です。			
\\	貧乏	貧乏[びんぼう]	びんぼう	
\\	彼は昔は貧乏だった。	彼[かれ]は 昔[むかし]は 貧乏[びんぼう]だった。	かれ は むかし は びんぼう だった	
\\	彼[かれ]は 昔[むかし]は
\\	だった。			
\\	砂糖	砂糖[さとう]	さとう	
\\	コーヒーに砂糖は入れますか。	コーヒーに 砂糖[さとう]は 入[い]れますか。	こーひー に さとう は いれます か	
\\	コーヒーに
\\	は 入[い]れますか。			
\\	規模	規模[きぼ]	きぼ	
\\	この動物園は日本一の規模です。	この 動物園[どうぶつえん]は 日本一[にっぽんいち]の 規模[きぼ]です。	この どうぶつえん は にっぽんいち の きぼ です	
\\	この 動物園[どうぶつえん]は 日本一[にっぽんいち]の
\\	です。			
\\	農業	農業[のうぎょう]	のうぎょう	
\\	私は農業を勉強しています。	私[わたし]は 農業[のうぎょう]を 勉強[べんきょう]しています。	わたし は のうぎょう を べんきょう して います	
\\	私[わたし]は
\\	を 勉強[べんきょう]しています。			
\\	濃い	濃[こ]い	こい	
\\	私は濃い味が好きだ。	私[わたし]は 濃[こ]い 味[あじ]が 好[す]きだ。	わたし は こい あじ が すき だ	
\\	私[わたし]は
\\	味[あじ]が 好[す]きだ。			
\\	薄い	薄[うす]い	うすい	
\\	この電子辞書はとても薄い。	この 電子辞書[でんし じしょ]はとても 薄[うす]い。	この でんし じしょ は とても うすい	
\\	この 電子辞書[でんし じしょ]はとても
\\	迫る	迫[せま]る	せまる	
\\	締め切りが迫っています。	締[し]め 切[き]りが 迫[せま]っています。	しめきり が せまって います	
\\	締[し]め 切[き]りが
\\	越える	越[こ]える	こえる	
\\	私たちは高い山を越えました。	私[わたし]たちは 高[たか]い 山[やま]を 越[こ]えました。	わたしたち は たかい やま を こえました	
\\	私[わたし]たちは 高[たか]い 山[やま]を
\\	追い越す	追[お]い 越[こ]す	おいこす	
\\	大きなトラックが私たちを追い越した。	大[おお]きなトラックが 私[わたし]たちを 追[お]い 越[こ]した。	おおき な とらっく が わたしたち を おいこした	
\\	大[おお]きなトラックが 私[わたし]たちを
\\	謝る	謝[あやま]る	あやまる	
\\	彼は直ぐに謝りました。	彼[かれ]は 直[す]ぐに 謝[あやま]りました。	かれ は すぐ に あやまりました	
\\	彼[かれ]は 直[す]ぐに
\\	程度	程度[ていど]	ていど	
\\	この程度の怪我なら大丈夫です。	この 程度[ていど]の 怪我[けが]なら 大丈夫[だいじょうぶ]です。	この ていど の けが なら だいじょうぶ です	
\\	この
\\	の 怪我[けが]なら 大丈夫[だいじょうぶ]です。			
\\	誘う	誘[さそ]う	さそう	
\\	彼女をデートに誘った。	彼女[かのじょ]をデートに 誘[さそ]った。	かのじょ を でーと に さそった	
\\	彼女[かのじょ]をデートに
\\	独身	独身[どくしん]	どくしん	
\\	私の兄はまだ独身です。	私[わたし]の 兄[あに]はまだ 独身[どくしん]です。	わたし の あに は まだ どくしん です	
\\	私[わたし]の 兄[あに]はまだ
\\	です。			
\\	紹介	紹介[しょうかい]	しょうかい	
\\	両親に彼女を紹介した。	両親[りょうしん]に 彼女[かのじょ]を 紹介[しょうかい]した。	りょうしん に かのじょ を しょうかい した	
\\	両親[りょうしん]に 彼女[かのじょ]を
\\	した。			
\\	皆さん	皆[みな]さん	みなさん	
\\	皆さんにお話があります。	皆[みな]さんにお 話[はなし]があります。	みなさん に おはなし が あります	
\\	にお 話[はなし]があります。			
\\	皆様	皆様[みなさま]	みなさま	
\\	皆様、こんにちは。	皆様[みなさま]、こんにちは。	みなさま こんにちは	
\\	、こんにちは。			
\\	誰か	誰[だれ]か	だれか	
\\	誰かに聞いてみてください。	誰[だれ]かに 聞[き]いてみてください。	だれか に きいて みて ください	
\\	に 聞[き]いてみてください。			
\\	誕生日	誕生日[たんじょうび]	たんじょうび	
\\	彼女の誕生日は7月16日です。	彼女[かのじょ]の 誕生日[たんじょうび]は 7月16日[しちがつ じゅうろくにち]です。	かのじょ の たんじょうび は しちがつ じゅうろくにち です	
\\	彼女[かのじょ]の
\\	は 7月16日[しちがつ じゅうろくにち]です。			
\\	祝日	祝日[しゅくじつ]	しゅくじつ	
\\	5月3日は祝日だ。	5月3日[ごがつ みっか]は 祝日[しゅくじつ]だ。	ごがつ みっか は しゅくじつ だ	
\\	5月3日[ごがつ みっか]は
\\	だ。			
\\	涙	涙[なみだ]	なみだ	
\\	彼女の目から涙がこぼれた。	彼女[かのじょ]の 目[め]から 涙[なみだ]がこぼれた。	かのじょ の め から なみだ が こぼれた	
\\	彼女[かのじょ]の 目[め]から
\\	がこぼれた。			
\\	看護婦	看護婦[かんごふ]	かんごふ	
\\	母は看護婦です。	母[はは]は 看護婦[かんごふ]です。	はは は かんごふ です	
\\	母[はは]は
\\	です。			
\\	述べる	述[の]べる	のべる	
\\	上司が意見を述べた。	上司[じょうし]が 意見[いけん]を 述[の]べた。	じょうし が いけん を のべた	
\\	上司[じょうし]が 意見[いけん]を
\\	訴える	訴[うった]える	うったえる	
\\	彼女は会社を訴えた。	彼女[かのじょ]は 会社[かいしゃ]を 訴[うった]えた。	かのじょ は かいしゃ を うったえた	
\\	彼女[かのじょ]は 会社[かいしゃ]を
\\	迷う	迷[まよ]う	まよう	
\\	道に迷いました。	道[みち]に 迷[まよ]いました。	みち に まよいました	
\\	道[みち]に
\\	迷惑	迷惑[めいわく]	めいわく	
\\	人に迷惑をかけてはいけません。	人[ひと]に 迷惑[めいわく]をかけてはいけません。	ひと に めいわく を かけて は いけません	
\\	人[ひと]に
\\	をかけてはいけません。			
\\	贈る	贈[おく]る	おくる	
\\	母に花を贈った。	母[はは]に 花[はな]を 贈[おく]った。	はは に はな を おくった	
\\	母[はは]に 花[はな]を
\\	贈り物	贈[おく]り 物[もの]	おくりもの	
\\	すてきな贈り物をどうもありがとう。	すてきな 贈[おく]り 物[もの]をどうもありがとう。	すてき な おくりもの を どうも ありがとう	
\\	すてきな
\\	をどうもありがとう。			
\\	貯金	貯金[ちょきん]	ちょきん	
\\	貯金は十分にあります。	貯金[ちょきん]は 十分[じゅうぶん]にあります。	ちょきん は じゅうぶん に あります	
\\	は 十分[じゅうぶん]にあります。			
\\	預ける	預[あず]ける	あずける	
\\	鍵を彼に預けた。	鍵[かぎ]を 彼[かれ]に 預[あず]けた。	かぎ を かれ に あずけた	
\\	鍵[かぎ]を 彼[かれ]に
\\	預かる	預[あず]かる	あずかる	
\\	荷物を預かってください。	荷物[にもつ]を 預[あず]かってください。	にもつ を あずかって ください	
\\	荷物[にもつ]を
\\	ください。			
\\	燃える	燃[も]える	もえる	
\\	山が燃えています。	山[やま]が 燃[も]えています。	やま が もえて います	
\\	山[やま]が
\\	焼ける	焼[や]ける	やける	
\\	肉が焼けました。	肉[にく]が 焼[や]けました。	にく が やけました	
\\	肉[にく]が
\\	焼く	焼[や]く	やく	
\\	今、魚を焼いています。	今[いま]、 魚[さかな]を 焼[や]いています。	いま さかな を やいて います	
\\	今[いま]、 魚[さかな]を
\\	素晴らしい	素晴[すば]らしい	すばらしい	
\\	素晴らしい景色ですね。	素晴[すば]らしい 景色[けしき]ですね。	すばらしい けしき です ね	
\\	景色[けしき]ですね。			
\\	黒板	黒板[こくばん]	こくばん	
\\	答えを黒板に書いてください。	答[こた]えを 黒板[こくばん]に 書[か]いてください。	こたえ を こくばん に かいて ください	
\\	答[こた]えを
\\	に 書[か]いてください。			
\\	草	草[くさ]	くさ	
\\	庭に草が生えている。	庭[にわ]に 草[くさ]が 生[は]えている。	にわ に くさ が はえて いる	
\\	庭[にわ]に
\\	が 生[は]えている。			
\\	葉	葉[は]	は	
\\	これは桜の葉です。	これは 桜[さくら]の 葉[は]です。	これ は さくら の は です	
\\	これは 桜[さくら]の
\\	です。			
\\	絵葉書	絵葉書[えはがき]	えはがき	
\\	友達から絵葉書が届きました。	友達[ともだち]から 絵葉書[えはがき]が 届[とど]きました。	ともだち から えはがき が とどきました	
\\	友達[ともだち]から
\\	が 届[とど]きました。			
\\	葉書	葉書[はがき]	はがき	
\\	家族に葉書を書いています。	家族[かぞく]に 葉書[はがき]を 書[か]いています。	かぞく に はがき を かいています 。	
\\	家族[かぞく]に
\\	を 書[か]いています。			
\\	胃	胃[い]	い	
\\	昨日から胃が痛い。	昨日[きのう]から 胃[い]が 痛[いた]い。	きのう から い が いたい	
\\	昨日[きのう]から
\\	が 痛[いた]い。			
\\	皿	皿[さら]	さら	
\\	皿にケーキを載せました。	皿[さら]にケーキを 載[の]せました。	さら に けーき を のせました。	
\\	にケーキを 載[の]せました。			
\\	血	血[ち]	ち	
\\	血が出ていますよ。	血[ち]が 出[で]ていますよ。	ち が でて います よ	
\\	が 出[で]ていますよ。			
\\	背景	背景[はいけい]	はいけい	
\\	事件の背景に何があったのだろう。	事件[じけん]の 背景[はいけい]に 何[なに]があったのだろう。	じけん の はいけい に なに が あった の だろう	
\\	事件[じけん]の
\\	に 何[なに]があったのだろう。			
\\	骨	骨[ほね]	ほね	
\\	彼は足の骨を折りました。	彼[かれ]は 足[あし]の 骨[ほね]を 折[お]りました。	かれ は あし の ほね を おりました	
\\	彼[かれ]は 足[あし]の
\\	を 折[お]りました。			
\\	滑る	滑[すべ]る	すべる	
\\	彼は雪の上を滑った。	彼[かれ]は 雪[ゆき]の 上[うえ]を 滑[すべ]った。	かれ は ゆき の うえ を すべった	
\\	彼[かれ]は 雪[ゆき]の 上[うえ]を
\\	珍しい	珍[めずら]しい	めずらしい	
\\	昨日珍しい果物を食べました。	昨日[きのう] 珍[めずら]しい 果物[くだもの]を 食[た]べました。	きのう めずらしい くだもの を たべました	
\\	昨日[きのう]
\\	果物[くだもの]を 食[た]べました。			
\\	装置	装置[そうち]	そうち	
\\	実験にはこの装置を使います。	実験[じっけん]にはこの 装置[そうち]を 使[つか]います。	じっけん に は この そうち を つかいます	
\\	実験[じっけん]にはこの
\\	を 使[つか]います。			
\\	送り仮名	送[おく]り 仮名[がな]	おくりがな	
\\	その送り仮名は間違っています。	その 送[おく]り 仮名[がな]は 間違[まちが]っています。	その おくりがな は まちがって います	
\\	その
\\	は 間違[まちが]っています。			
\\	鏡	鏡[かがみ]	かがみ	
\\	彼女は鏡を見て髪を直した。	彼女[かのじょ]は 鏡[かがみ]を 見[み]て 髪[かみ]を 直[なお]した。	かのじょ は かがみ を みて かみ を なおした	
\\	彼女[かのじょ]は
\\	を 見[み]て 髪[かみ]を 直[なお]した。			
\\	美術館	美術館[びじゅつかん]	びじゅつかん	
\\	昨日、美術館に行きました。	昨日[きのう]、 美術館[びじゅつかん]に 行[い]きました。	きのう びじゅつかん に いきました	
\\	昨日[きのう]、
\\	に 行[い]きました。			
\\	美人	美人[びじん]	びじん	
\\	彼のお母さんは美人です。	彼[かれ]のお 母[かあ]さんは 美人[びじん]です。	かれ の おかあさん は びじん です	
\\	彼[かれ]のお 母[かあ]さんは
\\	です。			
\\	美容院	美容院[びよういん]	びよういん	
\\	父は美容院で髪を切ります。	父[ちち]は 美容院[びよういん]で 髪[かみ]を 切[き]ります。	ちち は びよういん で かみ を きります	
\\	父[ちち]は
\\	で 髪[かみ]を 切[き]ります。			
\\	開催	開催[かいさい]	かいさい	
\\	京都で国際会議が開催された。	京都[きょうと]で 国際会議[こくさいかいぎ]が 開催[かいさい]された。	きょうと で こくさいかいぎ が かいさい された	
\\	京都[きょうと]で 国際会議[こくさいかいぎ]が
\\	された。			
\\	特徴	特徴[とくちょう]	とくちょう	
\\	この曲線がこの車の特徴です。	この 曲線[きょくせん]がこの 車[くるま]の 特徴[とくちょう]です。	この きょくせん が この くるま の とくちょう です	
\\	この 曲線[きょくせん]がこの 車[くるま]の
\\	です。			
\\	許す	許[ゆる]す	ゆるす	
\\	私は彼を許しました。	私[わたし]は 彼[かれ]を 許[ゆる]しました。	わたし は かれ を ゆるしました	
\\	私[わたし]は 彼[かれ]を
\\	鳥	鳥[とり]	とり	
\\	鳥が飛んでいます。	鳥[とり]が 飛[と]んでいます。	とり が とんで います	
\\	が 飛[と]んでいます。			
\\	鳴る	鳴[な]る	なる	
\\	今朝、5時に電話が鳴った。	今朝[けさ]、 5時[ごじ]に 電話[でんわ]が 鳴[な]った。	けさ ごじ に でんわ が なった	
\\	今朝[けさ]、 5時[ごじ]に 電話[でんわ]が
\\	鳴く	鳴[な]く	なく	
\\	どこかでネコが鳴いている。	どこかでネコが 鳴[な]いている。	どこか で ねこ が ないて いる	
\\	どこかでネコが
\\	犬	犬[いぬ]	いぬ	
\\	この犬はとても賢い。	この 犬[いぬ]はとても 賢[かしこ]い。	この いぬ は とても かしこい	
\\	この
\\	はとても 賢[かしこ]い。			
\\	馬	馬[うま]	うま	
\\	彼は牧場で馬に乗った。	彼[かれ]は 牧場[ぼくじょう]で 馬[うま]に 乗[の]った。	かれ は ぼくじょう で うま に のった	
\\	彼[かれ]は 牧場[ぼくじょう]で
\\	に 乗[の]った。			
\\	駐車場	駐車場[ちゅうしゃじょう]	ちゅうしゃじょう	
\\	車は駐車場に止めてください。	車[くるま]は 駐車場[ちゅうしゃじょう]に 止[と]めてください。	くるま は ちゅうしゃじょう に とめて ください	
\\	車[くるま]は
\\	に 止[と]めてください。			
\\	騒ぐ	騒[さわ]ぐ	さわぐ	
\\	電車の中で騒がないでください。	電車[でんしゃ]の 中[なか]で 騒[さわ]がないでください。	でんしゃ の なか で さわがない で ください	
\\	電車[でんしゃ]の 中[なか]で
\\	ください。			
\\	激しい	激[はげ]しい	はげしい	
\\	激しい雨が降っています。	激[はげ]しい 雨[あめ]が 降[ふ]っています。	はげしい あめ が ふって います	
\\	雨[あめ]が 降[ふ]っています。			
\\	驚く	驚[おどろ]く	おどろく	
\\	彼は血を見て驚いた。	彼[かれ]は 血[ち]を 見[み]て 驚[おどろ]いた。	かれ は ち を みて おどろいた	
\\	彼[かれ]は 血[ち]を 見[み]て
\\	突き当たる	突[つ]き 当[あ]たる	つきあたる	
\\	突き当たったら右に曲がってください。	突[つ]き 当[あ]たったら 右[みぎ]に 曲[ま]がってください。	つきあたったら みぎ に まがって ください	
\\	右[みぎ]に 曲[ま]がってください。			
\\	突き当たり	突[つ]き 当[あ]たり	つきあたり	
\\	この先は突き当たりです。	この 先[さき]は 突[つ]き 当[あ]たりです。	この さき は つきあたり です	
\\	この 先[さき]は
\\	です。			
\\	絶対に	絶対[ぜったい]に	ぜったいに	
\\	絶対に駄目です。	絶対[ぜったい]に 駄目[だめ]です。	ぜったいに だめ です	
\\	駄目[だめ]です。			
\\	消防車	消防車[しょうぼうしゃ]	しょうぼうしゃ	
\\	消防車が4台も来た。	消防車[しょうぼうしゃ]が 4台[よんだい]も 来[き]た。	しょうぼうしゃ が よんだい も きた	
\\	が 4台[よんだい]も 来[き]た。			
\\	糸	糸[いと]	いと	
\\	母は糸を針に通した。	母[はは]は 糸[いと]を 針[はり]に 通[とお]した。	はは は いと を はり に とおした	
\\	母[はは]は
\\	を 針[はり]に 通[とお]した。			
\\	爪	爪[つめ]	つめ	
\\	爪が伸びています。	爪[つめ]が 伸[の]びています。	つめ が のびて います	
\\	が 伸[の]びています。			
\\	釣る	釣[つ]る	つる	
\\	昨日大きな魚を釣りました。	昨日大[きのう おお]きな 魚[さかな]を 釣[つ]りました。	きのう おおき な さかな を つりました	
\\	昨日大[きのう おお]きな 魚[さかな]を
\\	鍵	鍵[かぎ]	かぎ	
\\	出かける時は鍵を掛けてください。	出[で]かける 時[とき]は 鍵[かぎ]を 掛[か]けてください。	でかける とき は かぎ を かけて ください	
\\	出[で]かける 時[とき]は
\\	を 掛[か]けてください。			
\\	鍋	鍋[なべ]	なべ	
\\	鍋にスープが入っているよ。	鍋[なべ]にスープが 入[はい]っているよ。	なべ に すーぷ が はいって いる よ	
\\	にスープが 入[はい]っているよ。			
\\	湯	湯[ゆ]	ゆ	
\\	お風呂のお湯が一杯です。	お 風呂[ふろ]のお 湯[ゆ]が 一杯[いっぱい]です。	おふろ の おゆ が いっぱい です	
\\	お 風呂[ふろ]のお
\\	が 一杯[いっぱい]です。			
\\	濡れる	濡[ぬ]れる	ぬれる	
\\	雨で濡れてしまった。	雨[あめ]で 濡[ぬ]れてしまった。	あめ で ぬれて しまった	
\\	雨[あめ]で
\\	納豆	納豆[なっとう]	なっとう	
\\	私は納豆をよく食べます。	私[わたし]は 納豆[なっとう]をよく 食[た]べます。	わたし は なっとう を よく たべます	
\\	私[わたし]は
\\	をよく 食[た]べます。			
\\	畑	畑[はたけ]	はたけ	
\\	母は畑仕事が好きだ。	母[はは]は 畑[はたけ] 仕事[しごと]が 好[す]きだ。	はは は はたけ しごと が すき だ	
\\	母[はは]は
\\	仕事[しごと]が 好[す]きだ。			
\\	炊く	炊[た]く	たく	
\\	母は毎朝ご飯を炊く。	母[はは]は 毎朝[まいあさ]ご 飯[はん]を 炊[た]く。	はは は まいあさ ごはん を たく	
\\	母[はは]は 毎朝[まいあさ]ご 飯[はん]を
\\	自炊	自炊[じすい]	じすい	
\\	私は毎日、自炊している。	私[わたし]は 毎日[まいにち]、 自炊[じすい]している。	わたし は まいにち じすい して いる	
\\	私[わたし]は 毎日[まいにち]、
\\	している。			
\\	灰皿	灰皿[はいざら]	はいざら	
\\	灰皿をください。	灰皿[はいざら]をください。	はいざら を ください	
\\	をください。			
\\	灰	灰[はい]	はい	
\\	火事ですべて灰になった。	火事[かじ]ですべて 灰[はい]になった。	かじ で すべて はい に なった	
\\	火事[かじ]ですべて
\\	になった。			
\\	灰色	灰色[はいいろ]	はいいろ	
\\	今日の空は灰色だ。	今日[きょう]の 空[そら]は 灰色[はいいろ]だ。	きょう の そら は はいいろ だ	
\\	今日[きょう]の 空[そら]は
\\	だ。			
\\	西暦	西暦[せいれき]	せいれき	
\\	西暦1964年に東京でオリンピックがあった。	西暦[せいれき] 1964年[せんきゅうひゃくろくじゅうよねん]に 東京[とうきょう]でオリンピックがあった。	せいれき せんきゅうひゃくろくじゅうよねん に とうきょう で おりんぴっく が あった	
\\	1964年[せんきゅうひゃくろくじゅうよねん]に 東京[とうきょう]でオリンピックがあった。			
\\	髪	髪[かみ]	かみ	
\\	昨日、髪を切りました。	昨日[きのう]、 髪[かみ]を 切[き]りました。	きのう かみ を きりました	
\\	昨日[きのう]、
\\	を 切[き]りました。			
\\	舌	舌[した]	した	
\\	舌を出してください。	舌[した]を 出[だ]してください。	した を だして ください	
\\	を 出[だ]してください。			
\\	臭い	臭[くさ]い	くさい	
\\	納豆は臭い。	納豆[なっとう]は 臭[くさ]い。	なっとう は くさい	
\\	納豆[なっとう]は
\\	豚	豚[ぶた]	ぶた	
\\	豚はすごく鼻が良い。	豚[ぶた]はすごく 鼻[はな]が 良[い]い。	ぶた は すごく はな が いい	
\\	はすごく 鼻[はな]が 良[い]い。			
\\	豚肉	豚肉[ぶたにく]	ぶたにく	
\\	豚肉は美味しいです。	豚肉[ぶたにく]は 美味[おい]しいです。	ぶたにく は おいしい です	
\\	は 美味[おい]しいです。			
\\	鶏	鶏[にわとり]	にわとり	
\\	その鶏は毎朝鳴きます。	その 鶏[にわとり]は 毎朝鳴[まいあさ な]きます。	その にわとり は まいあさ なきます	
\\	その
\\	は 毎朝鳴[まいあさ な]きます。			
\\	腕時計	腕時計[うでどけい]	うでどけい	
\\	私の腕時計は遅れている。	私[わたし]の 腕時計[うでどけい]は 遅[おく]れている。	わたし の うでどけい は おくれて いる	
\\	私[わたし]の
\\	は 遅[おく]れている。			
\\	鉛筆	鉛筆[えんぴつ]	えんぴつ	
\\	鉛筆を貸して下さい。	鉛筆[えんぴつ]を 貸[か]して 下[くだ]さい。	えんぴつ を かして ください	
\\	を 貸[か]して 下[くだ]さい。			
\\	箱	箱[はこ]	はこ	
\\	この箱は重い。	この 箱[はこ]は 重[おも]い。	この はこ は おもい	
\\	この
\\	は 重[おも]い。			
\\	袋	袋[ふくろ]	ふくろ	
\\	袋はいりません。	袋[ふくろ]はいりません。	ふくろ は いりません	
\\	はいりません。			
\\	財布	財布[さいふ]	さいふ	
\\	会社に財布を忘れた。	会社[かいしゃ]に 財布[さいふ]を 忘[わす]れた。	かいしゃ に さいふ を わすれた	
\\	会社[かいしゃ]に
\\	を 忘[わす]れた。			
\\	飾る	飾[かざ]る	かざる	
\\	テーブルの上に花を飾りました。	テーブルの 上[うえ]に 花[はな]を 飾[かざ]りました。	てーぶる の うえ に はな を かざりました	
\\	テーブルの 上[うえ]に 花[はな]を
\\	電話帳	電話帳[でんわちょう]	でんわちょう	
\\	電話帳を見せてください。	電話帳[でんわちょう]を 見[み]せてください。	でんわちょう を みせて ください	
\\	を 見[み]せてください。			
\\	牛乳	牛乳[ぎゅうにゅう]	ぎゅうにゅう	
\\	私は毎朝、牛乳を飲む。	私[わたし]は 毎朝[まいあさ]、 牛乳[ぎゅうにゅう]を 飲[の]む。	わたし は まいあさ ぎゅうにゅう を のむ	
\\	私[わたし]は 毎朝[まいあさ]、
\\	を 飲[の]む。			
\\	玄関	玄関[げんかん]	げんかん	
\\	玄関に花を飾りました。	玄関[げんかん]に 花[はな]を 飾[かざ]りました。	げんかん に はな を かざりました	
\\	に 花[はな]を 飾[かざ]りました。			
\\	眼鏡	眼鏡[めがね]	めがね	
\\	彼は眼鏡をかけています。	彼[かれ]は 眼鏡[めがね]をかけています。	かれ は めがね を かけて います	
\\	彼[かれ]は
\\	をかけています。			
\\	眠い	眠[ねむ]い	ねむい	
\\	今日はとても眠いです。	今日[きょう]はとても 眠[ねむ]いです。	きょう は とても ねむい です	
\\	今日[きょう]はとても
\\	です。			
\\	眠る	眠[ねむ]る	ねむる	
\\	昨日は8時間眠りました。	昨日[きのう]は 8時間[はちじかん] 眠[ねむ]りました。	きのう は はちじかん ねむりました	
\\	昨日[きのう]は 8時間[はちじかん]
\\	迎える	迎[むか]える	むかえる	
\\	姉が空港まで迎えに来てくれます。	姉[あね]が 空港[くうこう]まで 迎[むか]えに 来[き]てくれます。	あね が くうこう まで むかえに きて くれます	
\\	姉[あね]が 空港[くうこう]まで
\\	に 来[き]てくれます。			
\\	穴	穴[あな]	あな	
\\	靴下に穴が開いている。	靴下[くつした]に 穴[あな]が 開[あ]いている。	くつした に あな が あいて いる	
\\	靴下[くつした]に
\\	が 開[あ]いている。			
\\	隠れる	隠[かく]れる	かくれる	
\\	太陽が雲に隠れた。	太陽[たいよう]が 雲[くも]に 隠[かく]れた。	たいよう が くも に かくれた	
\\	太陽[たいよう]が 雲[くも]に
\\	随分	随分[ずいぶん]	ずいぶん	
\\	随分早く終わったね。	随分[ずいぶん] 早[はや]く 終[お]わったね。	ずいぶん はやく おわった ね	
\\	早[はや]く 終[お]わったね。			
\\	頑張る	頑張[がんば]る	がんばる	
\\	明日の試験、頑張ってね。	明日[あした]の 試験[しけん]、 頑張[がんば]ってね。	あした の しけん がんばって ね	
\\	明日[あした]の 試験[しけん]、
\\	ね。			
\\	頃	頃[ころ]	ころ	
\\	子供の頃、よくその公園で遊んだ。	子供[こども]の 頃[ころ]、よくその 公園[こうえん]で 遊[あそ]んだ。	こども の ころ よく その こうえん で あそんだ	
\\	子供[こども]の
\\	、よくその 公園[こうえん]で 遊[あそ]んだ。			
\\	消防署	消防署[しょうぼうしょ]	しょうぼうしょ	
\\	この町には消防署が1つしかありません。	この 町[まち]には 消防署[しょうぼうしょ]が 1[ひと]つしかありません。	この まち に は しょうぼうしょ が ひとつ しか ありません	
\\	この 町[まち]には
\\	が 1[ひと]つしかありません。			
\\	缶詰	缶詰[かんづめ]	かんづめ	
\\	缶詰のフルーツはとても甘い。	缶詰[かんづめ]のフルーツはとても 甘[あま]い。	かんづめ の ふるーつ は とても あまい	
\\	のフルーツはとても 甘[あま]い。			
\\	缶	缶[かん]	かん	
\\	コーヒーはその缶に入っています。	コーヒーはその 缶[かん]に 入[はい]っています。	こーひー は その かん に はいって います	
\\	コーヒーはその
\\	に 入[はい]っています。			
\\	腐る	腐[くさ]る	くさる	
\\	リンゴが腐っている。	リンゴが 腐[くさ]っている。	りんご が くさって いる	
\\	リンゴが
\\	豆腐	豆腐[とうふ]	とうふ	
\\	私は毎日豆腐を食べます。	私[わたし]は 毎日[まいにち] 豆腐[とうふ]を 食[た]べます。	わたし は まいにち とうふ を たべます	
\\	私[わたし]は 毎日[まいにち]
\\	を 食[た]べます。			
\\	畳	畳[たたみ]	たたみ	
\\	そのホテルには畳の部屋がある。	そのホテルには 畳[たたみ]の 部屋[へや]がある。	その ほてる に は たたみ の へや が ある	
\\	そのホテルには
\\	の 部屋[へや]がある。			
\\	畳む	畳[たた]む	たたむ	
\\	布団を畳んでください。	布団[ふとん]を 畳[たた]んでください。	ふとん を たたんで ください	
\\	布団[ふとん]を
\\	ください。			
\\	雷	雷[かみなり]	かみなり	
\\	雷が鳴っています。	雷[かみなり]が 鳴[な]っています。	かみなり が なって います	
\\	が 鳴[な]っています。			
\\	零	零[れい]	れい	
\\	今ちょうど零時です。	今[いま]ちょうど 零[れい] 時[じ]です。	いま ちょうど れいじ です	
\\	今[いま]ちょうど
\\	時[じ]です。			
\\	踊る	踊[おど]る	おどる	
\\	彼女はクラブで踊るのが好きです。	彼女[かのじょ]はクラブで 踊[おど]るのが 好[す]きです。	かのじょ は くらぶ で おどる の が すき です	
\\	彼女[かのじょ]はクラブで
\\	のが 好[す]きです。			
\\	踊り	踊[おど]り	おどり	
\\	この踊りは易しいですよ。	この 踊[おど]りは 易[やさ]しいですよ。	この おどり は やさしい です よ	
\\	この
\\	は 易[やさ]しいですよ。			
\\	踏む	踏[ふ]む	ふむ	
\\	運転手がブレーキを踏んだ。	運転手[うんてんしゅ]がブレーキを 踏[ふ]んだ。	うんてんしゅ が ぶれーき を ふんだ	
\\	運転手[うんてんしゅ]がブレーキを
\\	踏切	踏切[ふみきり]	ふみきり	
\\	その踏切は長い。	その 踏切[ふみきり]は 長[なが]い。	その ふみきり は ながい	
\\	その
\\	は 長[なが]い。			
\\	蹴る	蹴[け]る	ける	
\\	ゴールキーパーがボールを蹴った。	ゴールキーパーがボールを 蹴[け]った。	ごーるきーぱー が ぼーる を けった。	
\\	ゴールキーパーがボールを
\\	食堂	食堂[しょくどう]	しょくどう	
\\	大学の食堂は安い。	大学[だいがく]の 食堂[しょくどう]は 安[やす]い。	だいがく の しょくどう は やすい	
\\	大学[だいがく]の
\\	は 安[やす]い。			
\\	猫	猫[ねこ]	ねこ	
\\	私は猫が大好きです。	私[わたし]は 猫[ねこ]が 大好[だいす]きです。	わたし は ねこ が だいすき です	
\\	私[わたし]は
\\	が 大好[だいす]きです。			
\\	祖父	祖父[そふ]	そふ	
\\	祖父は元気です。	祖父[そふ]は 元気[げんき]です。	そふ は げんき です	
\\	は 元気[げんき]です。			
\\	祖母	祖母[そぼ]	そぼ	
\\	祖母は京都で生まれました。	祖母[そぼ]は 京都[きょうと]で 生[う]まれました。	そぼ は きょうと で うまれました	
\\	は 京都[きょうと]で 生[う]まれました。			
\\	邪魔	邪魔[じゃま]	じゃま	
\\	邪魔です、どいてください。	邪魔[じゃま]です、どいてください。	じゃま です どいて ください	
\\	です、どいてください。			
\\	風邪薬	風邪薬[かぜぐすり]	かぜぐすり	
\\	この風邪薬を飲みなさい。	この 風邪薬[かぜぐすり]を 飲[の]みなさい。	この かぜぐすり を のみなさい	
\\	この
\\	を 飲[の]みなさい。			
\\	風呂屋	風呂屋[ふろや]	ふろや	
\\	昨日、友達とお風呂屋さんに行った。	昨日[きのう]、 友達[ともだち]とお 風呂屋[ふろや]さんに 行[い]った。	きのう ともだち と おふろやさん に いった	
\\	昨日[きのう]、 友達[ともだち]とお
\\	さんに 行[い]った。			
\\	風呂	風呂[ふろ]	ふろ	
\\	父は今お風呂に入っています。	父[ちち]は 今[いま]お 風呂[ふろ]に 入[はい]っています。	ちち は いま お ふろ に はいって います	
\\	父[ちち]は 今[いま]お
\\	に 入[はい]っています。			
\\	紫	紫[むらさき]	むらさき	
\\	彼女は紫のドレスを着ていた。	彼女[かのじょ]は 紫[むらさき]のドレスを 着[き]ていた。	かのじょ は むらさき の どれす を きて いた	
\\	彼女[かのじょ]は
\\	のドレスを 着[き]ていた。			
\\	紅茶	紅茶[こうちゃ]	こうちゃ	
\\	温かい紅茶が飲みたい。	温[あたた]かい 紅茶[こうちゃ]が 飲[の]みたい。	あたたかい こうちゃ が のみたい	
\\	温[あたた]かい
\\	が 飲[の]みたい。			
\\	遭う	遭[あ]う	あう	
\\	彼は交通事故に遭った。	彼[かれ]は 交通事故[こうつう じこ]に 遭[あ]った。	かれ は こうつう じこ に あった	
\\	彼[かれ]は 交通事故[こうつう じこ]に
\\	遠慮	遠慮[えんりょ]	えんりょ	
\\	私は遠慮します。	私[わたし]は 遠慮[えんりょ]します。	わたし は えんりょ します	
\\	私[わたし]は
\\	します。			
\\	賑やか	賑[にぎ]やか	にぎやか	
\\	浅草は賑やかな街です。	浅草[あさくさ]は 賑[にぎ]やかな 街[まち]です。	あさくさ は にぎやか な まち です	
\\	浅草[あさくさ]は
\\	な 街[まち]です。			
\\	蚊	蚊[か]	か	
\\	蚊に足を刺された。	蚊[か]に 足[あし]を 刺[さ]された。	か に あし を さされた	
\\	に 足[あし]を 刺[さ]された。			
\\	靴	靴[くつ]	くつ	
\\	靴が汚れた。	靴[くつ]が 汚[よご]れた。	くつ が よごれた	
\\	が 汚[よご]れた。			
\\	靴下	靴下[くつした]	くつした	
\\	この靴下は3足で1000円です。	この 靴下[くつした]は 3足[さんそく]で 1000円[せんえん]です。	この くつした は さんそく で せんえん です	
\\	この
\\	は 3足[さんそく]で 1000円[せんえん]です。			
\\	磨く	磨[みが]く	みがく	
\\	靴を磨いてください。	靴[くつ]を 磨[みが]いてください。	くつ を みがいて ください	
\\	靴[くつ]を
\\	ください。			
\\	瓶	瓶[びん]	びん	
\\	瓶ビールを注文した。	瓶[びん]ビールを 注文[ちゅうもん]した。	びんびーる を ちゅうもん した	
\\	ビールを 注文[ちゅうもん]した。			
\\	褒める	褒[ほ]める	ほめる	
\\	頑張ったので褒められました。	頑張[がんば]ったので 褒[ほ]められました。	がんばった の で ほめられました	
\\	頑張[がんば]ったので
\\	袖	袖[そで]	そで	
\\	このシャツは袖が短い。	このシャツは 袖[そで]が 短[みじか]い。	この しゃつ は そで が みじかい	
\\	このシャツは
\\	が 短[みじか]い。			
\\	長袖	長袖[ながそで]	ながそで	
\\	今日は寒いので長袖を着ました。	今日[きょう]は 寒[さむ]いので 長袖[ながそで]を 着[き]ました。	きょう は さむい ので ながそで を きました 。	
\\	今日[きょう]は 寒[さむ]いので
\\	を 着[き]ました。			
\\	馬鹿	馬鹿[ばか]	ばか	
\\	私は馬鹿だった。	私[わたし]は 馬鹿[ばか]だった。	わたし は ばか だった	
\\	私[わたし]は
\\	だった。			
\\	苺	苺[いちご]	いちご	
\\	この苺はとても甘い。	この 苺[いちご]はとても 甘[あま]い。	この いちご は とても あまい	
\\	この
\\	はとても 甘[あま]い。			
\\	茄子	茄子[なす]	なす	
\\	夕飯に茄子の天ぷらを食べました。	夕飯[ゆうはん]に 茄子[なす]の 天[てん]ぷらを 食[た]べました。	ゆうはん に なす の てんぷら を たべました 。	
\\	夕飯[ゆうはん]に
\\	の 天[てん]ぷらを 食[た]べました。			
\\	逢う	逢[あ]う	あう	
\\	ついに素晴らしい女性に逢えた。	ついに 素晴[すば]らしい 女性[じょせい]に 逢[あ]えた。	ついに すばらしい じょせい に あえた 。	
\\	ついに 素晴[すば]らしい 女性[じょせい]に
\\	痩せる	痩[や]せる	やせる	
\\	私は少し痩せました。	私[わたし]は 少[すこ]し 痩[や]せました。	わたし は すこし やせました	
\\	私[わたし]は 少[すこ]し
\\	箸	箸[はし]	はし	
\\	箸を上手に使えるよ。	箸[はし]を 上手[じょうず]に 使[つか]えるよ。	はし を じょうず に つかえる よ	
\\	を 上手[じょうず]に 使[つか]えるよ。			
\\	糊	糊[のり]	のり	
\\	糊で2枚の紙を貼り合わせた。	糊[のり]で 2枚[にまい]の 紙[かみ]を 貼[は]り 合[あ]わせた。	のり で にまい の かみ を はり あわせた	
\\	で 2枚[にまい]の 紙[かみ]を 貼[は]り 合[あ]わせた。			
\\	醤油	醤油[しょうゆ]	しょうゆ	
\\	もう少し醤油を足してください。	もう 少[すこ]し 醤油[しょうゆ]を 足[た]してください。	もうすこし しょうゆ を たして ください 。	
\\	もう 少[すこ]し
\\	を 足[た]してください。			
\\	鋏	鋏[はさみ]	はさみ	
\\	この鋏はよく切れる。	この 鋏[はさみ]はよく 切[き]れる。	この はさみ は よく きれる	
\\	この
\\	はよく 切[き]れる。			
\\	鞄	鞄[かばん]	かばん	
\\	その黒い鞄は僕のです。	その 黒[くろ]い 鞄[かばん]は 僕[ぼく]のです。	その くろい かばん は ぼく の です	
\\	その 黒[くろ]い
\\	は 僕[ぼく]のです。			
\\	顎	顎[あご]	あご	
\\	顎が痛い。	顎[あご]が 痛[いた]い。	あご が いたい	
\\	が 痛[いた]い。			
\\	飴	飴[あめ]	あめ	
\\	缶に飴が入っています。	缶[かん]に 飴[あめ]が 入[はい]っています。	かん に あめ が はいって います	
\\	缶[かん]に
\\	が 入[はい]っています。			
\\	石鹸	石鹸[せっけん]	せっけん	
\\	石鹸で手を洗ってください。	石鹸[せっけん]で 手[て]を 洗[あら]ってください。	せっけん で て を あらって ください	
\\	で 手[て]を 洗[あら]ってください。			
\\	金もうけ	金[かね]もうけ	かねもうけ	
\\	金もうけが彼の趣味だ。	金[かね]もうけが 彼[かれ]の 趣味[しゅみ]だ。	かねもうけ が かれ の しゅみ だ	
\\	が 彼[かれ]の 趣味[しゅみ]だ。			
\\	行き来	行[い]き 来[き]	いきき	
\\	最近彼のところに行き来してないの。	最近彼[さいきん かれ]のところに 行[い]き 来[き]してないの。	さいきん かれ の ところ に いきき して ない の	
\\	最近彼[さいきん かれ]のところに
\\	してないの。			
\\	行	行[ぎょう]	ぎょう	
\\	5行以内で答えを書いてください。	5[ご] 行[ぎょう] 以内[いない]で 答[こた]えを 書[か]いてください。	ごぎょう いない で こたえ を かいて ください	
\\	5[ご]
\\	以内[いない]で 答[こた]えを 書[か]いてください。			
\\	週休	週休[しゅうきゅう]	しゅうきゅう	
\\	うちの会社は週休2日です。	うちの 会社[かいしゃ]は 週休[しゅうきゅう] 2日[ふつか]です。	うち の かいしゃ は しゅうきゅう ふつか です	
\\	うちの 会社[かいしゃ]は
\\	2日[ふつか]です。			
\\	火力	火力[かりょく]	かりょく	
\\	火力を弱めないと肉がこげるよ。	火力[かりょく]を 弱[よわ]めないと 肉[にく]がこげるよ。	かりょく を よわめない と にく が こげる よ	
\\	を 弱[よわ]めないと 肉[にく]がこげるよ。			
\\	足下	足下[あしもと]	あしもと	
\\	暗いので足下に気を付けてください。	暗[くら]いので 足下[あしもと]に 気[き]を 付[つ]けてください。	くらい の で あしもと に き を つけて ください	
\\	暗[くら]いので
\\	に 気[き]を 付[つ]けてください。			
\\	火山	火山[かざん]	かざん	
\\	島で火山が噴火したよ。	島[しま]で 火山[かざん]が 噴火[ふんか]したよ。	しま で かざん が ふんか した よ	
\\	島[しま]で
\\	が 噴火[ふんか]したよ。			
\\	空中	空中[くうちゅう]	くうちゅう	
\\	蝶々が空中を舞っています。	蝶々[ちょうちょう]が 空中[くうちゅう]を 舞[ま]っています。	ちょうちょう が くうちゅう を まって います	
\\	蝶々[ちょうちょう]が
\\	を 舞[ま]っています。			
\\	空ける	空[あ]ける	あける	
\\	彼女はお年寄りのために席を空けたんだ。	彼女[かのじょ]はお 年寄[としよ]りのために 席[せき]を 空[あ]けたんだ。	かのじょ は おとしより の ため に せき を あけた ん だ	
\\	彼女[かのじょ]はお 年寄[としよ]りのために 席[せき]を
\\	んだ。			
\\	空っぽ	空[から]っぽ	からっぽ	
\\	僕の財布は空っぽだよ。	僕[ぼく]の 財布[さいふ]は 空[から]っぽだよ。	ぼく の さいふ は からっぽ だ よ	
\\	僕[ぼく]の 財布[さいふ]は
\\	だよ。			
\\	空き	空[あ]き	あき	
\\	部屋の空きはありますか。	部屋[へや]の 空[あ]きはありますか。	へや の あき は あります か	
\\	部屋[へや]の
\\	はありますか。			
\\	空	空[から]	から	
\\	この瓶はもう空ね。	この 瓶[びん]はもう 空[から]ね。	この びん は もう から ね	
\\	この 瓶[びん]はもう
\\	ね。			
\\	男らしい	男[おとこ]らしい	おとこらしい	
\\	彼は男らしさを心掛けているね。	彼[かれ]は 男[おとこ]らしさを 心掛[こころが]けているね。	かれ は おとこらしさ を こころがけて いる ね	
\\	彼[かれ]は
\\	を 心掛[こころが]けているね。			
\\	雨天	雨天[うてん]	うてん	
\\	試合は雨天中止です。	試合[しあい]は 雨天[うてん] 中止[ちゅうし]です。	しあい は うてん ちゅうし です	
\\	試合[しあい]は
\\	中止[ちゅうし]です。			
\\	青年	青年[せいねん]	せいねん	
\\	町の青年たちはボランティア活動をしています。	町[まち]の 青年[せいねん]たちはボランティア 活動[かつどう]をしています。	まち の せいねんたち は ぼらんてぃあ かつどう を して います	
\\	町[まち]の
\\	たちはボランティア 活動[かつどう]をしています。			
\\	青空	青空[あおぞら]	あおぞら	
\\	雲一つない青空ですね。	雲一[くも ひと]つない 青空[あおぞら]ですね。	くも ひとつ ない あおぞら です ね	
\\	雲一[くも ひと]つない
\\	ですね。			
\\	開き	開[あ]き	あき	
\\	このブラウスは後ろ開きです。	このブラウスは 後[うし]ろ 開[あ]きです。	この ぶらうす は うしろあき です	
\\	このブラウスは 後[うし]ろ
\\	です。			
\\	聞かす	聞[き]かす	きかす	
\\	子供に昔話を聞かせてあげたの。	子供[こども]に 昔話[むかしばなし]を 聞[き]かせてあげたの。	こども に むかしばなし を きかせてあげた の 。	
\\	子供[こども]に 昔話[むかしばなし]を
\\	あげたの。			
\\	聞き手	聞[き]き 手[て]	ききて	
\\	彼女はいつも聞き手にまわるね。	彼女[かのじょ]はいつも 聞[き]き 手[て]にまわるね。	かのじょ は いつも ききて に まわる ね	
\\	彼女[かのじょ]はいつも
\\	にまわるね。			
\\	間	間[かん]	かん	
\\	その間に彼は居なくなっていました。	その 間[かん]に 彼[かれ]は 居[い]なくなっていました。	その かん に かれ は いなく なって いました	
\\	その
\\	に 彼[かれ]は 居[い]なくなっていました。			
\\	空間	空間[くうかん]	くうかん	
\\	狭い空間に物がたくさん置いてあるね。	狭[せま]い 空間[くうかん]に 物[もの]がたくさん 置[お]いてあるね。	せまい くうかん に もの が たくさん おいて ある ね	
\\	狭[せま]い
\\	に 物[もの]がたくさん 置[お]いてあるね。			
\\	食う	食[く]う	くう	
\\	腹一杯食った。	腹一杯[はらいっぱい] 食[く]った。	はらいっぱい くった	
\\	腹一杯[はらいっぱい]
\\	食パン	食[しょく]パン	しょくぱん	
\\	朝食に食パンを2枚食べました。	朝食[ちょうしょく]に 食[しょく]パンを 2枚食[にまい た]べました。	ちょうしょく に しょくぱん を にまい たべました	
\\	朝食[ちょうしょく]に
\\	を 2枚食[にまい た]べました。			
\\	言い出す	言[い]い 出[だ]す	いいだす	
\\	突然何を言い出すのかと思った。	突然何[とつぜん なに]を 言[い]い 出[だ]すのかと 思[おも]った。	とつぜん なに を いいだす の か と おもった	
\\	突然何[とつぜん なに]を
\\	のかと 思[おも]った。			
\\	英文	英文[えいぶん]	えいぶん	
\\	彼女は英文の手紙を書きました。	彼女[かのじょ]は 英文[えいぶん]の 手紙[てがみ]を 書[か]きました。	かのじょ は えいぶん の てがみ を かきました	
\\	彼女[かのじょ]は
\\	の 手紙[てがみ]を 書[か]きました。			
\\	漢語	漢語[かんご]	かんご	
\\	漢語はもともと外来語です。	漢語[かんご]はもともと 外来語[がいらいご]です。	かんご は もともと がいらいご です	
\\	はもともと 外来語[がいらいご]です。			
\\	覚え	覚[おぼ]え	おぼえ	
\\	この子は覚えが早いね。	この 子[こ]は 覚[おぼ]えが 早[はや]いね。	この こ は おぼえ が はやい ね	
\\	この 子[こ]は
\\	が 早[はや]いね。			
\\	開会	開会[かいかい]	かいかい	
\\	運動会は9時に開会します。	運動会[うんどうかい]は 9時[くじ]に 開会[かいかい]します。	うんどうかい は くじ に かいかい します	
\\	運動会[うんどうかい]は 9時[くじ]に
\\	します。			
\\	社会人	社会人[しゃかいじん]	しゃかいじん	
\\	この春に彼は社会人になったよ。	この 春[はる]に 彼[かれ]は 社会人[しゃかいじん]になったよ。	この はる に かれ は しゃかいじん に なった よ	
\\	この 春[はる]に 彼[かれ]は
\\	になったよ。			
\\	行事	行事[ぎょうじ]	ぎょうじ	
\\	今日は学校で行事がありました。	今日[きょう]は 学校[がっこう]で 行事[ぎょうじ]がありました。	きょう は がっこう で ぎょうじ が ありました	
\\	今日[きょう]は 学校[がっこう]で
\\	がありました。			
\\	車内	車内[しゃない]	しゃない	
\\	車内に忘れ物があったよ。	車内[しゃない]に 忘[わす]れ 物[もの]があったよ。	しゃない に わすれもの が あった よ	
\\	に 忘[わす]れ 物[もの]があったよ。			
\\	駅前	駅前[えきまえ]	えきまえ	
\\	駅前に交番が有ります。	駅前[えきまえ]に 交番[こうばん]が 有[あ]ります。	えきまえ に こうばん が あります	
\\	に 交番[こうばん]が 有[あ]ります。			
\\	空き地	空[あ]き 地[ち]	あきち	
\\	空き地で工事が始まった。	空[あ]き 地[ち]で 工事[こうじ]が 始[はじ]まった。	あきち で こうじ が はじまった	
\\	で 工事[こうじ]が 始[はじ]まった。			
\\	行き止まり	行[い]き 止[ど]まり	いきどまり	
\\	ここから先は行き止まりです。	ここから 先[さき]は 行[い]き 止[ど]まりです。	ここ から さき は いきどまり です	
\\	ここから 先[さき]は
\\	です。			
\\	高度	高度[こうど]	こうど	
\\	高度な技術を持つ人材を集めました。	高度[こうど]な 技術[ぎじゅつ]を 持[も]つ 人材[じんざい]を 集[あつ]めました。	こうど な ぎじゅつ を もつ じんざい を あつめました	
\\	な 技術[ぎじゅつ]を 持[も]つ 人材[じんざい]を 集[あつ]めました。			
\\	遠足	遠足[えんそく]	えんそく	
\\	明日は遠足で動物園に行きます。	明日[あす]は 遠足[えんそく]で 動物園[どうぶつえん]に 行[い]きます。	あす は えんそく で どうぶつえん に いきます	
\\	明日[あす]は
\\	で 動物園[どうぶつえん]に 行[い]きます。			
\\	駅長	駅長[えきちょう]	えきちょう	
\\	彼は駅長です。	彼[かれ]は 駅長[えきちょう]です。	かれ は えきちょう です	
\\	彼[かれ]は
\\	です。			
\\	青白い	青白[あおじろ]い	あおじろい	
\\	彼女は青白い顔をしているね。	彼女[かのじょ]は 青白[あおじろ]い 顔[かお]をしているね。	かのじょ は あおじろい かお を して いる ね	
\\	彼女[かのじょ]は
\\	顔[かお]をしているね。			
\\	黒字	黒字[くろじ]	くろじ	
\\	わが社は今年度、黒字となりました。	わが 社[しゃ]は 今年度[こんねんど]、 黒字[くろじ]となりました。	わがしゃ は こんねんど くろじ と なりました	
\\	わが 社[しゃ]は 今年度[こんねんど]、
\\	となりました。			
\\	赤道	赤道[せきどう]	せきどう	
\\	赤道に沿って旅をしました。	赤道[せきどう]に 沿[そ]って 旅[たび]をしました。	せきどう に そって たび を しました	
\\	に 沿[そ]って 旅[たび]をしました。			
\\	赤字	赤字[あかじ]	あかじ	
\\	私たちの会社は今月は赤字よ。	私[わたし]たちの 会社[かいしゃ]は 今月[こんげつ]は 赤字[あかじ]よ。	わたしたち の かいしゃ は こんげつ は あかじ よ	
\\	私[わたし]たちの 会社[かいしゃ]は 今月[こんげつ]は
\\	よ。			
\\	私鉄	私鉄[してつ]	してつ	
\\	私は私鉄の職員です。	私[わたし]は 私鉄[してつ]の 職員[しょくいん]です。	わたし は してつ の しょくいん です	
\\	私[わたし]は
\\	の 職員[しょくいん]です。			
\\	飲料水	飲料水[いんりょうすい]	いんりょうすい	
\\	被災地では飲料水が不足しているの。	被災地[ひさいち]では 飲料水[いんりょうすい]が 不足[ふそく]しているの。	ひさいち で は いんりょうすい が ふそく して いる の	
\\	被災地[ひさいち]では
\\	が 不足[ふそく]しているの。			
\\	食料	食料[しょくりょう]	しょくりょう	
\\	食料はこの箱に入っています。	食料[しょくりょう]はこの 箱[はこ]に 入[はい]っています。	しょくりょう は この はこ に はいって います	
\\	はこの 箱[はこ]に 入[はい]っています。			
\\	見解	見解[けんかい]	けんかい	
\\	あなたの見解を聞かせてください。	あなたの 見解[けんかい]を 聞[き]かせてください。	あなた の けんかい を きかせて ください	
\\	あなたの
\\	を 聞[き]かせてください。			
\\	私用	私用[しよう]	しよう	
\\	私用で出かけなければなりません。	私用[しよう]で 出[で]かけなければなりません。	しよう で でかけなければ なりません	
\\	で 出[で]かけなければなりません。			
\\	食費	食費[しょくひ]	しょくひ	
\\	男の子が3人もいるので食費がかさみます。	男[おとこ]の 子[こ]が 3人[さんにん]もいるので 食費[しょくひ]がかさみます。	おとこ の こ が さんにん も いる の で しょくひ が かさみます	
\\	男[おとこ]の 子[こ]が 3人[さんにん]もいるので
\\	がかさみます。			
\\	私費	私費[しひ]	しひ	
\\	彼は私費で留学したんだ。	彼[かれ]は 私費[しひ]で 留学[りゅうがく]したんだ。	かれ は しひ で りゅうがく した ん だ	
\\	彼[かれ]は
\\	で 留学[りゅうがく]したんだ。			
\\	消費	消費[しょうひ]	しょうひ	
\\	日本は消費大国といわれています。	日本[にっぽん]は 消費[しょうひ] 大国[たいこく]といわれています。	にっぽん は しょうひ たいこく と いわれて います	
\\	日本[にっぽん]は
\\	大国[たいこく]といわれています。			
\\	消火	消火[しょうか]	しょうか	
\\	火事は無事消火されました。	火事[かじ]は 無事[ぶじ] 消火[しょうか]されました。	かじ は ぶじ しょうか されました	
\\	火事[かじ]は 無事[ぶじ]
\\	されました。			
\\	開店	開店[かいてん]	かいてん	
\\	デパートは10時に開店しますよ。	デパートは 10時[じゅうじ]に 開店[かいてん]しますよ。	でぱーと は じゅうじ に かいてん します よ	
\\	デパートは 10時[じゅうじ]に
\\	しますよ。			
\\	食品	食品[しょくひん]	しょくひん	
\\	食品は日曜日にまとめて買います。	食品[しょくひん]は 日曜日[にちようび]にまとめて 買[か]います。	しょくひん は にちようび に まとめて かいます	
\\	は 日曜日[にちようび]にまとめて 買[か]います。			
\\	高価	高価[こうか]	こうか	
\\	彼女は高価な宝石を持っているわ。	彼女[かのじょ]は 高価[こうか]な 宝石[ほうせき]を 持[も]っているわ。	かのじょ は こうか な ほうせき を もって いる わ	
\\	彼女[かのじょ]は
\\	な 宝石[ほうせき]を 持[も]っているわ。			
\\	青春	青春[せいしゅん]	せいしゅん	
\\	この曲を聞くと青春の頃を思い出します。	この 曲[きょく]を 聞[き]くと 青春[せいしゅん]の 頃[ころ]を 思[おも]い 出[だ]します。	この きょく を きく と せいしゅん の ころ を おもいだします	
\\	この 曲[きょく]を 聞[き]くと
\\	の 頃[ころ]を 思[おも]い 出[だ]します。			
\\	秋分	秋分[しゅうぶん]	しゅうぶん	
\\	秋分の日は毎年9月23日頃です。	秋分[しゅうぶん]の 日[ひ]は 毎年9月23日頃[まいとし くがつ にじゅうさんにちごろ]です。	しゅうぶん の ひ は まいとし くがつ にじゅうさんにちごろ です	
\\	の 日[ひ]は 毎年9月23日頃[まいとし くがつ にじゅうさんにちごろ]です。			
\\	高温	高温[こうおん]	こうおん	
\\	金属は高温で溶かします。	金属[きんぞく]は 高温[こうおん]で 溶[と]かします。	きんぞく は こうおん で とかします	
\\	金属[きんぞく]は
\\	で 溶[と]かします。			
\\	温暖	温暖[おんだん]	おんだん	
\\	この地方は温暖で暮らしやすいな。	この 地方[ちほう]は 温暖[おんだん]で 暮[く]らしやすいな。	この ちほう は おんだん で くらし やすい な	
\\	この 地方[ちほう]は
\\	で 暮[く]らしやすいな。			
\\	秋風	秋風[あきかぜ]	あきかぜ	
\\	秋風が気持ちいいね。	秋風[あきかぜ]が 気持[きも]ちいいね。	あきかぜ が きもち いい ね	
\\	が 気持[きも]ちいいね。			
\\	知り合う	知[し]り 合[あ]う	しりあう	
\\	お二人はどこで知り合ったのですか。	お 二人[ふたり]はどこで 知[し]り 合[あ]ったのですか。	おふたり は どこ で しりあった の です か	
\\	お 二人[ふたり]はどこで
\\	のですか。			
\\	知れる	知[し]れる	しれる	
\\	他人に知れるとまずいことになるわね。	他人[たにん]に 知[し]れるとまずいことになるわね。	たにん に しれる と まずい こと に なる わ ね	
\\	他人[たにん]に
\\	とまずいことになるわね。			
\\	知り合い	知[し]り 合[あ]い	しりあい	
\\	街で知り合いを見かけたよ。	街[まち]で 知[し]り 合[あ]いを 見[み]かけたよ。	まち で しりあい を みかけた よ	
\\	街[まち]で
\\	を 見[み]かけたよ。			
\\	知らせ	知[し]らせ	しらせ	
\\	今日、合格の知らせをもらいました。	今日[きょう]、 合格[ごうかく]の 知[し]らせをもらいました。	きょう ごうかく の しらせ を もらいました	
\\	今日[きょう]、 合格[ごうかく]の
\\	をもらいました。			
\\	知らず知らず	知[し]らず 知[し]らず	しらずしらず	
\\	知らず知らずのうちに疲れがたまっていたよ。	知[し]らず 知[し]らずのうちに 疲[つか]れがたまっていたよ。	しらずしらず の うち に つかれ が たまって いた よ	
\\	のうちに 疲[つか]れがたまっていたよ。			
\\	県	県[けん]	けん	
\\	県の代表は2名です。	県[けん]の 代表[だいひょう]は 2名[にめい]です。	けん の だいひょう は にめい です	
\\	の 代表[だいひょう]は 2名[にめい]です。			
\\	見物人	見物人[けんぶつにん]	けんぶつにん	
\\	見物人が大勢集まっているね。	見物人[けんぶつにん]が 大勢集[おおぜい あつ]まっているね。	けんぶつにん が おおぜい あつまって いる ね	
\\	が 大勢集[おおぜい あつ]まっているね。			
\\	食物	食物[しょくもつ]	しょくもつ	
\\	人間にとって水は食物より大切なの。	人間[にんげん]にとって 水[みず]は 食物[しょくもつ]より 大切[たいせつ]なの。	にんげん に とって みず は しょくもつ より たいせつ なの	
\\	人間[にんげん]にとって 水[みず]は
\\	より 大切[たいせつ]なの。			
\\	重ねる	重[かさ]ねる	かさねる	
\\	荷物はここに重ねてください。	荷物[にもつ]はここに 重[かさ]ねてください。	にもつ は ここ に かさねて ください	
\\	荷物[にもつ]はここに
\\	ください。			
\\	重なる	重[かさ]なる	かさなる	
\\	高速道路で事故が重なった。	高速道路[こうそく どうろ]で 事故[じこ]が 重[かさ]なった。	こうそく どうろ で じこ が かさなった	
\\	高速道路[こうそく どうろ]で 事故[じこ]が
\\	重大	重大[じゅうだい]	じゅうだい	
\\	重大な発表があります。	重大[じゅうだい]な 発表[はっぴょう]があります。	じゅうだい な はっぴょう が あります	
\\	な 発表[はっぴょう]があります。			
\\	重み	重[おも]み	おもみ	
\\	雪の重みで枝が折れそうだ。	雪[ゆき]の 重[おも]みで 枝[えだ]が 折[お]れそうだ。	ゆき の おもみ で えだ が おれ そう だ	
\\	雪[ゆき]の
\\	で 枝[えだ]が 折[お]れそうだ。			
\\	重たい	重[おも]たい	おもたい	
\\	この鞄は重たいです。	この 鞄[かばん]は 重[おも]たいです。	この かばん は おもたい です	
\\	この 鞄[かばん]は
\\	です。			
\\	重量	重量[じゅうりょう]	じゅうりょう	
\\	この荷物はかなりの重量ですね。	この 荷物[にもつ]はかなりの 重量[じゅうりょう]ですね。	この にもつ は かなり の じゅうりょう です ね	
\\	この 荷物[にもつ]はかなりの
\\	ですね。			
\\	聞き取り	聞[き]き 取[と]り	ききとり	
\\	英語の聞き取り試験を受けたよ。	英語[えいご]の 聞[き]き 取[と]り 試験[しけん]を 受[う]けたよ。	えいご の ききとり しけん を うけた よ	
\\	英語[えいご]の
\\	試験[しけん]を 受[う]けたよ。			
\\	重役	重役[じゅうやく]	じゅうやく	
\\	明日、重役会議が開かれます。	明日[あす]、 重役[じゅうやく] 会議[かいぎ]が 開[ひら]かれます。	あす じゅうやくかいぎ が ひらかれます	
\\	明日[あす]、
\\	会議[かいぎ]が 開[ひら]かれます。			
\\	生じる	生[しょう]じる	しょうじる	
\\	両者の間に摩擦が生じています。	両者[りょうしゃ]の 間[あいだ]に 摩擦[まさつ]が 生[しょう]じています。	りょうしゃ の あいだ に まさつ が しょうじて います	
\\	両者[りょうしゃ]の 間[あいだ]に 摩擦[まさつ]が
\\	生まれ	生[う]まれ	うまれ	
\\	彼女は京都の生まれです。	彼女[かのじょ]は 京都[きょうと]の 生[う]まれです。	かのじょ は きょうと の うまれ です	
\\	彼女[かのじょ]は 京都[きょうと]の
\\	です。			
\\	生み出す	生[う]み 出[だ]す	うみだす	
\\	彼は数々の名作を生み出した。	彼[かれ]は 数々[かずかず]の 名作[めいさく]を 生[う]み 出[だ]した。	かれ は かずかず の めいさく を うみだした	
\\	彼[かれ]は 数々[かずかず]の 名作[めいさく]を
\\	生き方	生[い]き 方[かた]	いきかた	
\\	自分らしい生き方をしなさい。	自分[じぶん]らしい 生[い]き 方[かた]をしなさい。	じぶん らしい いきかた を しなさい	
\\	自分[じぶん]らしい
\\	をしなさい。			
\\	生かす	生[い]かす	いかす	
\\	彼女は語学力を仕事に生かしているね。	彼女[かのじょ]は 語学力[ごがくりょく]を 仕事[しごと]に 生[い]かしているね。	かのじょ は ごがくりょく を しごと に いかして いる ね	
\\	彼女[かのじょ]は 語学力[ごがくりょく]を 仕事[しごと]に
\\	ね。			
\\	生き物	生[い]き 物[もの]	いきもの	
\\	生き物を大切にしましょう。	生[い]き 物[もの]を 大切[たいせつ]にしましょう。	いきもの を たいせつ に しましょう	
\\	を 大切[たいせつ]にしましょう。			
\\	生け花	生[い]け 花[ばな]	いけばな	
\\	彼女は生け花の先生です。	彼女[かのじょ]は 生[い]け 花[ばな]の 先生[せんせい]です。	かのじょ は いけばな の せんせい です	
\\	彼女[かのじょ]は
\\	の 先生[せんせい]です。			
\\	生年月日	生年月日[せいねんがっぴ]	せいねんがっぴ	
\\	ここに生年月日を記入してください。	ここに 生年[せいねん] 月日[がっぴ]を 記入[きにゅう]してください。	ここ に せいねんがっぴ を きにゅう して ください	
\\	ここに
\\	を 記入[きにゅう]してください。			
\\	生理	生理[せいり]	せいり	
\\	昨日、生理が始まったの。	昨日[きのう]、 生理[せいり]が 始[はじ]まったの。	きのう せいり が はじまった の	
\\	昨日[きのう]、
\\	が 始[はじ]まったの。			
\\	生まれつき	生[う]まれつき	うまれつき	
\\	彼女は生まれつき丈夫です。	彼女[かのじょ]は 生[う]まれつき 丈夫[じょうぶ]です。	かのじょ は うまれつき じょうぶ です	
\\	彼女[かのじょ]は
\\	丈夫[じょうぶ]です。			
\\	生	生[せい]	せい	
\\	私がこの世に生を受けて80年が過ぎたわ。	私[わたし]がこの 世[よ]に 生[せい]を 受[う]けて 80年[はちじゅうねん]が 過[す]ぎたわ。	わたし が このよ に せい を うけて はちじゅうねん が すぎた わ	
\\	私[わたし]がこの 世[よ]に
\\	を 受[う]けて 80年[はちじゅうねん]が 過[す]ぎたわ。			
\\	生物	生物[せいぶつ]	せいぶつ	
\\	海の底には不思議な生物がたくさんいるよ。	海[うみ]の 底[そこ]には 不思議[ふしぎ]な 生物[せいぶつ]がたくさんいるよ。	うみ の そこ に は ふしぎ な せいぶつ が たくさん いる よ	
\\	海[うみ]の 底[そこ]には 不思議[ふしぎ]な
\\	がたくさんいるよ。			
\\	生き生きと	生[い]き 生[い]きと	いきいきと	
\\	子供たちが生き生きと遊んでいるね。	子供[こども]たちが 生[い]き 生[い]きと 遊[あそ]んでいるね。	こどもたち が いきいきと あそんで いる ね	
\\	子供[こども]たちが
\\	遊[あそ]んでいるね。			
\\	産地	産地[さんち]	さんち	
\\	その地方はお茶の産地です。	その 地方[ちほう]はお 茶[ちゃ]の 産地[さんち]です。	その ちほう は おちゃ の さんち です	
\\	その 地方[ちほう]はお 茶[ちゃ]の
\\	です。			
\\	語学	語学[ごがく]	ごがく	
\\	海外で語学の勉強をします。	海外[かいがい]で 語学[ごがく]の 勉強[べんきょう]をします。	かいがい で ごがく の べんきょう を します	
\\	海外[かいがい]で
\\	の 勉強[べんきょう]をします。			
\\	試合	試合[しあい]	しあい	
\\	試合の結果を早く知りたい。	試合[しあい]の 結果[けっか]を 早[はや]く 知[し]りたい。	しあい の けっか を はやく しりたい	
\\	の 結果[けっか]を 早[はや]く 知[し]りたい。			
\\	試みる	試[こころ]みる	こころみる	
\\	彼は実験を試みたのよ。	彼[かれ]は 実験[じっけん]を 試[こころ]みたのよ。	かれ は じっけん を こころみた の よ	
\\	彼[かれ]は 実験[じっけん]を
\\	のよ。			
\\	質	質[しつ]	しつ	
\\	量より質の方が大事です。	量[りょう]より 質[しつ]の 方[ほう]が 大事[だいじ]です。	りょう より しつ の ほう が だいじ です	
\\	量[りょう]より
\\	の 方[ほう]が 大事[だいじ]です。			
\\	重点	重点[じゅうてん]	じゅうてん	
\\	子供の自主性に重点を置いています。	子供[こども]の 自主性[じしゅせい]に 重点[じゅうてん]を 置[お]いています。	こども の じしゅせい に じゅうてん を おいて います	
\\	子供[こども]の 自主性[じしゅせい]に
\\	を 置[お]いています。			
\\	見当	見当[けんとう]	けんとう	
\\	この仕事には何日必要か見当もつかないね。	この 仕事[しごと]には 何日必要[なんにち ひつよう]か 見当[けんとう]もつかないね。	この しごと に は なんにち ひつよう か けんとう も つかない ね	
\\	この 仕事[しごと]には 何日必要[なんにち ひつよう]か
\\	もつかないね。			
\\	規制	規制[きせい]	きせい	
\\	牛肉の輸入が厳しく規制されているわね。	牛肉[ぎゅうにく]の 輸入[ゆにゅう]が 厳[きび]しく 規制[きせい]されているわね。	ぎゅうにく の ゆにゅう が きびしく きせい されて いる わ ね	
\\	牛肉[ぎゅうにく]の 輸入[ゆにゅう]が 厳[きび]しく
\\	されているわね。			
\\	経費	経費[けいひ]	けいひ	
\\	これからは経費を節約しましょう。	これからは 経費[けいひ]を 節約[せつやく]しましょう。	これ から は けいひ を せつやく しましょう	
\\	これからは
\\	を 節約[せつやく]しましょう。			
\\	済む	済[す]む	すむ	
\\	父の手術が無事に済みました。	父[ちち]の 手術[しゅじゅつ]が 無事[ぶじ]に 済[す]みました。	ちち の しゅじゅつ が ぶじ に すみました	
\\	父[ちち]の 手術[しゅじゅつ]が 無事[ぶじ]に
\\	済ませる	済[す]ませる	すませる	
\\	昼ごはんは簡単に済ませましょう。	昼[ひる]ごはんは 簡単[かんたん]に 済[す]ませましょう。	ひるごはん は かんたん に すませましょう	
\\	昼[ひる]ごはんは 簡単[かんたん]に
\\	済ます	済[す]ます	すます	
\\	宿題を済ませてから遊びなさい。	宿題[しゅくだい]を 済[す]ませてから 遊[あそ]びなさい。	しゅくだい を すませて から あそびなさい 。	
\\	宿題[しゅくだい]を
\\	から 遊[あそ]びなさい。			
\\	金利	金利[きんり]	きんり	
\\	最近、銀行の金利が少し上がった。	最近[さいきん]、 銀行[ぎんこう]の 金利[きんり]が 少[すこ]し 上[あ]がった。	さいきん ぎんこう の きんり が すこし あがった	
\\	最近[さいきん]、 銀行[ぎんこう]の
\\	が 少[すこ]し 上[あ]がった。			
\\	芸術	芸術[げいじゅつ]	げいじゅつ	
\\	ここは芸術を愛する国です。	ここは 芸術[げいじゅつ]を 愛[あい]する 国[くに]です。	ここ は げいじゅつ を あいする くに です	
\\	ここは
\\	を 愛[あい]する 国[くに]です。			
\\	製品	製品[せいひん]	せいひん	
\\	当社の製品は3年間の保障つきです。	当社[とうしゃ]の 製品[せいひん]は 3年間[さんねんかん]の 保障[ほしょう]つきです。	とうしゃ の せいひん は さんねんかん の ほしょうつき です	
\\	当社[とうしゃ]の
\\	は 3年間[さんねんかん]の 保障[ほしょう]つきです。			
\\	製作	製作[せいさく]	せいさく	
\\	この映画は香港で製作されました。	この 映画[えいが]は 香港[ほんこん]で 製作[せいさく]されました。	この えいが は ほんこん で せいさく されました	
\\	この 映画[えいが]は 香港[ほんこん]で
\\	されました。			
\\	額	額[がく]	がく	
\\	写真を額に入れて飾ったんだ。	写真[しゃしん]を 額[がく]に 入[い]れて 飾[かざ]ったんだ。	しゃしん を がく に いれて かざった ん だ	
\\	写真[しゃしん]を
\\	に 入[い]れて 飾[かざ]ったんだ。			
\\	金額	金額[きんがく]	きんがく	
\\	レシートで買い物の金額を確かめたよ。	レシートで 買[か]い 物[もの]の 金額[きんがく]を 確[たし]かめたよ。	れしーと で かいもの の きんがく を たしかめた よ	
\\	レシートで 買[か]い 物[もの]の
\\	を 確[たし]かめたよ。			
\\	計	計[けい]	けい	
\\	3人分の代金は計6000円です。	3人分[さんにんぶん]の 代金[だいきん]は 計[けい] 6000円[ろくせんえん]です。	さんにんぶん の だいきん は けい ろくせんえん です	
\\	3人分[さんにんぶん]の 代金[だいきん]は
\\	6000円[ろくせんえん]です。			
\\	言い返す	言[い]い 返[かえ]す	いいかえす	
\\	彼女は負けずに言い返したの。	彼女[かのじょ]は 負[ま]けずに 言[い]い 返[かえ]したの。	かのじょ は まけず に いいかえした の	
\\	彼女[かのじょ]は 負[ま]けずに
\\	の。			
\\	返る	返[かえ]る	かえる	
\\	もう一度原点に返って考えましょう。	もう 一度原点[いちど げんてん]に 返[かえ]って 考[かんが]えましょう。	もういちど げんてん に かえって かんがえましょう	
\\	もう 一度原点[いちど げんてん]に
\\	考[かんが]えましょう。			
\\	貸し出し	貸[か]し 出[だ]し	かしだし	
\\	その本は貸し出ししていません。	その 本[ほん]は 貸[か]し 出[だ]ししていません。	その ほん は かしだし して いません	
\\	その 本[ほん]は
\\	していません。			
\\	貸し	貸[か]し	かし	
\\	彼には貸しがあるの。	彼[かれ]には 貸[か]しがあるの。	かれ に は かし が ある の	
\\	彼[かれ]には
\\	があるの。			
\\	申請	申請[しんせい]	しんせい	
\\	これからパスポートの申請に行きます。	これからパスポートの 申請[しんせい]に 行[い]きます。	これから ぱすぽーと の しんせい に いきます	
\\	これからパスポートの
\\	に 行[い]きます。			
\\	込める	込[こ]める	こめる	
\\	彼は感情を込めてその歌を歌ったの。	彼[かれ]は 感情[かんじょう]を 込[こ]めてその 歌[うた]を 歌[うた]ったの。	かれ は かんじょう を こめて その うた を うたった の	
\\	彼[かれ]は 感情[かんじょう]を
\\	その 歌[うた]を 歌[うた]ったの。			
\\	限り	限[かぎ]り	かぎり	
\\	限りある資源を大切にしよう。	限[かぎ]りある 資源[しげん]を 大切[たいせつ]にしよう。	かぎり ある しげん を たいせつ に しよう	
\\	ある 資源[しげん]を 大切[たいせつ]にしよう。			
\\	限界	限界[げんかい]	げんかい	
\\	もう我慢の限界です。	もう 我慢[がまん]の 限界[げんかい]です。	もう がまん の げんかい です	
\\	もう 我慢[がまん]の
\\	です。			
\\	限度	限度[げんど]	げんど	
\\	物事には限度があります。	物事[ものごと]には 限度[げんど]があります。	ものごと に は げんど が あります	
\\	物事[ものごと]には
\\	があります。			
\\	近代	近代[きんだい]	きんだい	
\\	近代の技術の発展はものすごいです。	近代[きんだい]の 技術[ぎじゅつ]の 発展[はってん]はものすごいです。	きんだい の ぎじゅつ の はってん は ものすごい です	
\\	の 技術[ぎじゅつ]の 発展[はってん]はものすごいです。			
\\	規定	規定[きてい]	きてい	
\\	代金には規定の手数料が含まれます。	代金[だいきん]には 規定[きてい]の 手数料[てすうりょう]が 含[ふく]まれます。	だいきん に は きてい の てすうりょう が ふくまれます	
\\	代金[だいきん]には
\\	の 手数料[てすうりょう]が 含[ふく]まれます。			
\\	減少	減少[げんしょう]	げんしょう	
\\	この国は人口が減少しているわね。	この 国[くに]は 人口[じんこう]が 減少[げんしょう]しているわね。	この くに は じんこう が げんしょう して いる わ ね	
\\	この 国[くに]は 人口[じんこう]が
\\	しているわね。			
\\	雨降り	雨降[あめふ]り	あめふり	
\\	雨降りで月が見えなかったね。	雨降[あめふ]りで 月[つき]が 見[み]えなかったね。	あめふり で つき が みえなかった ね	
\\	で 月[つき]が 見[み]えなかったね。			
\\	税金	税金[ぜいきん]	ぜいきん	
\\	これは税金の無駄遣いだね。	これは 税金[ぜいきん]の 無駄遣[むだづか]いだね。	これ は ぜいきん の むだづかい だ ね	
\\	これは
\\	の 無駄遣[むだづか]いだね。			
\\	税	税[ぜい]	ぜい	
\\	これ以上税が上がると生活できないね。	これ 以上[いじょう] 税[ぜい]が 上[あ]がると 生活[せいかつ]できないね。	これ いじょう ぜい が あがる と せいかつ できない ね	
\\	これ 以上[いじょう]
\\	が 上[あ]がると 生活[せいかつ]できないね。			
\\	私立	私立[しりつ]	しりつ	
\\	私立の大学は学費が高い。	私立[しりつ]の 大学[だいがく]は 学費[がくひ]が 高[たか]い。	しりつ の だいがく は がくひ が たかい	
\\	の 大学[だいがく]は 学費[がくひ]が 高[たか]い。			
\\	県立	県立[けんりつ]	けんりつ	
\\	彼は県立の高校に通っているよ。	彼[かれ]は 県立[けんりつ]の 高校[こうこう]に 通[かよ]っているよ。	かれ は けんりつ の こうこう に かよって いる よ	
\\	彼[かれ]は
\\	の 高校[こうこう]に 通[かよ]っているよ。			
\\	運用	運用[うんよう]	うんよう	
\\	会計士に資金運用について相談したよ。	会計士[かいけいし]に 資金[しきん] 運用[うんよう]について 相談[そうだん]したよ。	かいけいし に しきん うんよう に ついて そうだん した よ	
\\	会計士[かいけいし]に 資金[しきん]
\\	について 相談[そうだん]したよ。			
\\	運営	運営[うんえい]	うんえい	
\\	その事業は国が運営しています。	その 事業[じぎょう]は 国[くに]が 運営[うんえい]しています。	その じぎょう は くに が うんえい して います	
\\	その 事業[じぎょう]は 国[くに]が
\\	しています。			
\\	運賃	運賃[うんちん]	うんちん	
\\	鉄道の運賃が値上げされたね。	鉄道[てつどう]の 運賃[うんちん]が 値上[ねあ]げされたね。	てつどう の うんちん が ねあげ された ね	
\\	鉄道[てつどう]の
\\	が 値上[ねあ]げされたね。			
\\	運	運[うん]	うん	
\\	彼は運のいい男です。	彼[かれ]は 運[うん]のいい 男[おとこ]です。	かれ は うん の いい おとこ です	
\\	彼[かれ]は
\\	のいい 男[おとこ]です。			
\\	運送	運送[うんそう]	うんそう	
\\	彼は運送会社に勤めているの。	彼[かれ]は 運送[うんそう] 会社[がいしゃ]に 勤[つと]めているの。	かれ は うんそう がいしゃ に つとめて いる の	
\\	彼[かれ]は
\\	会社[がいしゃ]に 勤[つと]めているの。			
\\	転がる	転[ころ]がる	ころがる	
\\	猫は砂の上で転がったんだ。	猫[ねこ]は 砂[すな]の 上[うえ]で 転[ころ]がったんだ。	ねこ は すな の うえ で ころがった ん だ	
\\	猫[ねこ]は 砂[すな]の 上[うえ]で
\\	んだ。			
\\	転がす	転[ころ]がす	ころがす	
\\	まず、肉をパン粉の上で転がします。	まず、 肉[にく]をパン 粉[こ]の 上[うえ]で 転[ころ]がします。	まず にく を ぱんこ の うえ で ころがします	
\\	まず、 肉[にく]をパン 粉[こ]の 上[うえ]で
\\	移転	移転[いてん]	いてん	
\\	彼の会社は移転したよ。	彼[かれ]の 会社[かいしゃ]は 移転[いてん]したよ。	かれ の かいしゃ は いてん した よ	
\\	彼[かれ]の 会社[かいしゃ]は
\\	したよ。			
\\	移動	移動[いどう]	いどう	
\\	私たちはレンタカーで移動しました。	私[わたし]たちはレンタカーで 移動[いどう]しました。	わたしたち は れんたかー で いどう しました	
\\	私[わたし]たちはレンタカーで
\\	しました。			
\\	行動	行動[こうどう]	こうどう	
\\	彼の行動は理解できない。	彼[かれ]の 行動[こうどう]は 理解[りかい]できない。	かれ の こうどう は りかい できない	
\\	彼[かれ]の
\\	は 理解[りかい]できない。			
\\	運動会	運動会[うんどうかい]	うんどうかい	
\\	運動会で一位になったよ。	運動会[うんどうかい]で 一位[いちい]になったよ。	うんどうかい で いちい に なった よ	
\\	で 一位[いちい]になったよ。			
\\	運動場	運動場[うんどうじょう]	うんどうじょう	
\\	運動場でサッカーをしよう。	運動場[うんどうじょう]でサッカーをしよう。	うんどうじょう で さっかー を しよう	
\\	でサッカーをしよう。			
\\	運動場	運動場[うんどうじょう]	うんどうじょう	
\\	運動場でサッカーをしよう。	運動場[うんどうじょう]でサッカーをしよう。	うんどうじょう で さっかー を しよう	
\\	でサッカーをしよう。			
\\	高速	高速[こうそく]	こうそく	
\\	高速バスで東京に行ったんだ。	高速[こうそく]バスで 東京[とうきょう]に 行[い]ったんだ。	こうそくばす で とうきょう に いった ん だ	
\\	バスで 東京[とうきょう]に 行[い]ったんだ。			
\\	高速道路	高速道路[こうそくどうろ]	こうそくどうろ	
\\	高速道路は混んでたよ。	高速道路[こうそく どうろ]は 混[こ]んでたよ。	こうそく どうろ は こん でた よ 。	
\\	は 混[こ]んでたよ。			
\\	遅れ	遅[おく]れ	おくれ	
\\	電車が10分遅れで到着したんだ。	電車[でんしゃ]が 10分[じゅっ ぷん] 遅[おく]れで 到着[とうちゃく]したんだ。	でんしゃ が じゅっ ぷん おくれ で とうちゃく した ん だ	
\\	電車[でんしゃ]が 10分[じゅっ ぷん]
\\	で 到着[とうちゃく]したんだ。			
\\	遅らす	遅[おく]らす	おくらす	
\\	私たちは出発を1日遅らしたの。	私[わたし]たちは 出発[しゅっぱつ]を 1日[いちにち] 遅[おく]らしたの。	わたしたち は しゅっぱつ を いちにち おくらした の	
\\	私[わたし]たちは 出発[しゅっぱつ]を 1日[いちにち]
\\	の。			
\\	終える	終[お]える	おえる	
\\	やっと宿題を終えた。	やっと 宿題[しゅくだい]を 終[お]えた。	やっと しゅくだい を おえた	
\\	やっと 宿題[しゅくだい]を
\\	終点	終点[しゅうてん]	しゅうてん	
\\	電車の終点で降りました。	電車[でんしゃ]の 終点[しゅうてん]で 降[お]りました。	でんしゃ の しゅうてん で おりました	
\\	電車[でんしゃ]の
\\	で 降[お]りました。			
\\	終電	終電[しゅうでん]	しゅうでん	
\\	終電にやっと間に合った。	終電[しゅうでん]にやっと 間[ま]に 合[あ]った。	しゅうでん に やっと まにあった	
\\	にやっと 間[ま]に 合[あ]った。			
\\	現地	現地[げんち]	げんち	
\\	現地の天気は雨です。	現地[げんち]の 天気[てんき]は 雨[あめ]です。	げんち の てんき は あめ です	
\\	の 天気[てんき]は 雨[あめ]です。			
\\	現代	現代[げんだい]	げんだい	
\\	現代の科学の進歩には驚くよ。	現代[げんだい]の 科学[かがく]の 進歩[しんぽ]には 驚[おどろ]くよ。	げんだい の かがく の しんぽ に は おどろく よ	
\\	の 科学[かがく]の 進歩[しんぽ]には 驚[おどろ]くよ。			
\\	現金	現金[げんきん]	げんきん	
\\	代金は現金でお願いします。	代金[だいきん]は 現金[げんきん]でお 願[ねが]いします。	だいきん は げんきん で おねがい します	
\\	代金[だいきん]は
\\	でお 願[ねが]いします。			
\\	現れる	現[あらわ]れる	あらわれる	
\\	雲の間から太陽が現れました。	雲[くも]の 間[あいだ]から 太陽[たいよう]が 現[あらわ]れました。	くも の あいだ から たいよう が あらわれました	
\\	雲[くも]の 間[あいだ]から 太陽[たいよう]が
\\	現場	現場[げんば]	げんば	
\\	事故現場には入れません。	事故[じこ] 現場[げんば]には 入[はい]れません。	じこ げんば に は はいれません	
\\	事故[じこ]
\\	には 入[はい]れません。			
\\	現住所	現住所[げんじゅうしょ]	げんじゅうしょ	
\\	ここには現住所を書いてください。	ここには 現住所[げんじゅうしょ]を 書[か]いてください。	ここ に は げんじゅうしょ を かいて ください	
\\	ここには
\\	を 書[か]いてください。			
\\	現す	現[あらわ]す	あらわす	
\\	彼はようやく姿を現しましたね。	彼[かれ]はようやく 姿[すがた]を 現[あらわ]しましたね。	かれ は ようやく すがた を あらわしました ね	
\\	彼[かれ]はようやく 姿[すがた]を
\\	ね。			
\\	現れ	現[あらわ]れ	あらわれ	
\\	それは彼女の期待の現れです。	それは 彼女[かのじょ]の 期待[きたい]の 現[あらわ]れです。	それ は かのじょ の きたい の あらわれ です	
\\	それは 彼女[かのじょ]の 期待[きたい]の
\\	です。			
\\	現実	現実[げんじつ]	げんじつ	
\\	現実は予想より厳しかった。	現実[げんじつ]は 予想[よそう]より 厳[きび]しかった。	げんじつ は よそう より きびしかった	
\\	は 予想[よそう]より 厳[きび]しかった。			
\\	過ごす	過[す]ごす	すごす	
\\	夏休みを高原で過ごしました。	夏休[なつやす]みを 高原[こうげん]で 過[す]ごしました。	なつやすみ を こうげん で すごしました	
\\	夏休[なつやす]みを 高原[こうげん]で
\\	経過	経過[けいか]	けいか	
\\	手術後の経過は順調です。	手術後[しゅじゅつ ご]の 経過[けいか]は 順調[じゅんちょう]です。	しゅじゅつ ご の けいか は じゅんちょう です	
\\	手術後[しゅじゅつ ご]の
\\	は 順調[じゅんちょう]です。			
\\	過ち	過[あやま]ち	あやまち	
\\	同じ過ちを繰り返さないことだ。	同[おな]じ 過[あやま]ちを 繰[く]り 返[かえ]さないことだ。	おなじ あやまち を くりかえさない こと だ	
\\	同[おな]じ
\\	を 繰[く]り 返[かえ]さないことだ。			
\\	表す	表[あらわ]す	あらわす	
\\	私たちは万歳をして喜びを表しました。	私[わたし]たちは 万歳[ばんざい]をして 喜[よろこ]びを 表[あらわ]しました。	わたしたち は ばんざい を して よろこび を あらわしました	
\\	私[わたし]たちは 万歳[ばんざい]をして 喜[よろこ]びを
\\	言い表わす	言[い]い 表[あら]わす	いいあらわす	
\\	この気持ちをうまく言い表わせません。	この 気持[きも]ちをうまく 言[い]い 表[あら]わせません。	この きもち を うまく いいあらわせません	
\\	この 気持[きも]ちをうまく
\\	絵本	絵本[えほん]	えほん	
\\	その子は絵本が大好きです。	その 子[こ]は 絵本[えほん]が 大好[だいす]きです。	その こ は えほん が だいすき です	
\\	その 子[こ]は
\\	が 大好[だいす]きです。			
\\	雑音	雑音[ざつおん]	ざつおん	
\\	雑音がひどくて声が聞こえません。	雑音[ざつおん]がひどくて 声[こえ]が 聞[き]こえません。	ざつおん が ひどくて こえ が きこえません	
\\	がひどくて 声[こえ]が 聞[き]こえません。			
\\	足音	足音[あしおと]	あしおと	
\\	廊下から大きな足音が聞こえた。	廊下[ろうか]から 大[おお]きな 足音[あしおと]が 聞[き]こえた。	ろうか から おおき な あしおと が きこえた	
\\	廊下[ろうか]から 大[おお]きな
\\	が 聞[き]こえた。			
\\	音読み	音読[おんよ]み	おんよみ	
\\	この漢字の音読みは何ですか。	この 漢字[かんじ]の 音読[おんよ]みは 何[なん]ですか。	この かんじ の おんよみ は なん です か	
\\	この 漢字[かんじ]の
\\	は 何[なん]ですか。			
\\	行楽	行楽[こうらく]	こうらく	
\\	秋は行楽にいい季節ですね。	秋[あき]は 行楽[こうらく]にいい 季節[きせつ]ですね。	あき は こうらく に いい きせつ です ね	
\\	秋[あき]は
\\	にいい 季節[きせつ]ですね。			
\\	薬指	薬指[くすりゆび]	くすりゆび	
\\	彼女は薬指に指輪をはめています。	彼女[かのじょ]は 薬指[くすりゆび]に 指輪[ゆびわ]をはめています。	かのじょ は くすりゆび に ゆびわ を はめて います	
\\	彼女[かのじょ]は
\\	に 指輪[ゆびわ]をはめています。			
\\	食欲	食欲[しょくよく]	しょくよく	
\\	今日は食欲がありません。	今日[きょう]は 食欲[しょくよく]がありません。	きょう は しょくよく が ありません	
\\	今日[きょう]は
\\	がありません。			
\\	画家	画家[がか]	がか	
\\	この絵はスペインの画家が描きました。	この 絵[え]はスペインの 画家[がか]が 描[か]きました。	この え は すぺいん の がか が かきました	
\\	この 絵[え]はスペインの
\\	が 描[か]きました。			
\\	画面	画面[がめん]	がめん	
\\	テレビの画面が明るすぎる。	テレビの 画面[がめん]が 明[あか]るすぎる。	てれび の がめん が あかる すぎる	
\\	テレビの
\\	が 明[あか]るすぎる。			
\\	真実	真実[しんじつ]	しんじつ	
\\	真実は一つです。	真実[しんじつ]は 一[ひと]つです。	しんじつ は ひとつ です	
\\	は 一[ひと]つです。			
\\	真理	真理[しんり]	しんり	
\\	彼は人生の真理を求めて旅に出た。	彼[かれ]は 人生[じんせい]の 真理[しんり]を 求[もと]めて 旅[たび]に 出[で]た。	かれ は じんせい の しんり を もとめて たび に でた	
\\	彼[かれ]は 人生[じんせい]の
\\	を 求[もと]めて 旅[たび]に 出[で]た。			
\\	金色	金色[きんいろ]	きんいろ	
\\	あの寺の屋根は金色ですね。	あの 寺[てら]の 屋根[やね]は 金色[きんいろ]ですね。	あの てら の やね は きんいろ です ね	
\\	あの 寺[てら]の 屋根[やね]は
\\	ですね。			
\\	種目	種目[しゅもく]	しゅもく	
\\	その選手は2種目でメダルを取ったわ。	その 選手[せんしゅ]は 2[に] 種目[しゅもく]でメダルを 取[と]ったわ。	その せんしゅ は にしゅもく で めだる を とった わ	
\\	その 選手[せんしゅ]は 2[に]
\\	でメダルを 取[と]ったわ。			
\\	間接	間接[かんせつ]	かんせつ	
\\	それは間接的な原因の一つです。	それは 間接[かんせつ] 的[てき]な 原因[げんいん]の 一[ひと]つです。	それ は かんせつてき な げんいん の ひとつ です	
\\	それは
\\	的[てき]な 原因[げんいん]の 一[ひと]つです。			
\\	角度	角度[かくど]	かくど	
\\	この角度からは画面が見えにくいですね。	この 角度[かくど]からは 画面[がめん]が 見[み]えにくいですね。	この かくど からは がめん が みえ にくい です ね	
\\	この
\\	からは 画面[がめん]が 見[み]えにくいですね。			
\\	漢和	漢和[かんわ]	かんわ	
\\	漢和辞典で漢字の意味を調べたの。	漢和[かんわ] 辞典[じてん]で 漢字[かんじ]の 意味[いみ]を 調[しら]べたの。	かんわ じてん で かんじ の いみ を しらべた の	
\\	辞典[じてん]で 漢字[かんじ]の 意味[いみ]を 調[しら]べたの。			
\\	英和	英和[えいわ]	えいわ	
\\	英和辞書をよく使います。	英和[えいわ] 辞書[じしょ]をよく 使[つか]います。	えいわ じしょ を よく つかいます	
\\	辞書[じしょ]をよく 使[つか]います。			
\\	西洋	西洋[せいよう]	せいよう	
\\	私は西洋の建築に興味があります。	私[わたし]は 西洋[せいよう]の 建築[けんちく]に 興味[きょうみ]があります。	わたし は せいよう の けんちく に きょうみ が あります	
\\	私[わたし]は
\\	の 建築[けんちく]に 興味[きょうみ]があります。			
\\	西洋人	西洋人[せいようじん]	せいようじん	
\\	西洋人は正座が苦手よ。	西洋人[せいようじん]は 正座[せいざ]が 苦手[にがて]よ。	せいようじん は せいざ が にがて よ	
\\	は 正座[せいざ]が 苦手[にがて]よ。			
\\	親子	親子[おやこ]	おやこ	
\\	池に鴨の親子がいます。	池[いけ]に 鴨[かも]の 親子[おやこ]がいます。	いけ に かも の おやこ が います	
\\	池[いけ]に 鴨[かも]の
\\	がいます。			
\\	親類	親類[しんるい]	しんるい	
\\	彼は幼い時に親類に預けられたんだ。	彼[かれ]は 幼[おさな]い 時[とき]に 親類[しんるい]に 預[あず]けられたんだ。	かれ は おさない とき に しんるい に あずけられた ん だ	
\\	彼[かれ]は 幼[おさな]い 時[とき]に
\\	に 預[あず]けられたんだ。			
\\	親しむ	親[した]しむ	したしむ	
\\	ハイキングは自然に親しむ良い機会ですよ。	ハイキングは 自然[しぜん]に 親[した]しむ 良[い]い 機会[きかい]ですよ。	はいきんぐ は しぜん に したしむ いい きかい です よ	
\\	ハイキングは 自然[しぜん]に
\\	良[い]い 機会[きかい]ですよ。			
\\	親切	親切[しんせつ]	しんせつ	
\\	ご親切は決して忘れません。	ご 親切[しんせつ]は 決[けっ]して 忘[わす]れません。	ごしんせつ は けっして わすれません	
\\	ご
\\	は 決[けっ]して 忘[わす]れません。			
\\	親友	親友[しんゆう]	しんゆう	
\\	彼は私の親友です。	彼[かれ]は 私[わたし]の 親友[しんゆう]です。	かれ は わたし の しんゆう です	
\\	彼[かれ]は 私[わたし]の
\\	です。			
\\	親指	親指[おやゆび]	おやゆび	
\\	親指を怪我しました。	親指[おやゆび]を 怪我[けが]しました。	おやゆび を けがしました	
\\	を 怪我[けが]しました。			
\\	親しみ	親[した]しみ	したしみ	
\\	彼には誰もが親しみを感じます。	彼[かれ]には 誰[だれ]もが 親[した]しみを 感[かん]じます。	かれ に は だれ も が したしみ を かんじます	
\\	彼[かれ]には 誰[だれ]もが
\\	を 感[かん]じます。			
\\	消費者	消費者[しょうひしゃ]	しょうひしゃ	
\\	この法律は消費者を守るためのものです。	この 法律[ほうりつ]は 消費者[しょうひしゃ]を 守[まも]るためのものです。	この ほうりつ は しょうひしゃ を まもる ため の もの です	
\\	この 法律[ほうりつ]は
\\	を 守[まも]るためのものです。			
\\	結局	結局[けっきょく]	けっきょく	
\\	彼は結局何を言いたかったのだろう。	彼[かれ]は 結局[けっきょく] 何[なに]を 言[い]いたかったのだろう。	かれ は けっきょく なに を いいたかった の だろう	
\\	彼[かれ]は
\\	何[なに]を 言[い]いたかったのだろう。			
\\	課長	課長[かちょう]	かちょう	
\\	彼は課長に昇進しました。	彼[かれ]は 課長[かちょう]に 昇進[しょうしん]しました。	かれ は かちょう に しょうしん しました	
\\	彼[かれ]は
\\	に 昇進[しょうしん]しました。			
\\	自動	自動[じどう]	じどう	
\\	このドアは自動よ。	このドアは 自動[じどう]よ。	この どあ は じどう よ	
\\	このドアは
\\	よ。			
\\	自国	自国[じこく]	じこく	
\\	自国の文化を大切にしましょう。	自国[じこく]の 文化[ぶんか]を 大切[たいせつ]にしましょう。	じこく の ぶんか を たいせつ に しましょう	
\\	の 文化[ぶんか]を 大切[たいせつ]にしましょう。			
\\	自然に	自然[しぜん]に	しぜんに	
\\	硬くならないで、自然に話して下さい。	硬[かた]くならないで、 自然[しぜん]に 話[はな]して 下[くだ]さい。	かたく ならない で しぜんに はなして ください	
\\	硬[かた]くならないで、
\\	話[はな]して 下[くだ]さい。			
\\	経由	経由[けいゆ]	けいゆ	
\\	メールを経由して広がるウィルスもあります。	メールを 経由[けいゆ]して 広[ひろ]がるウィルスもあります。	めーる を けいゆ して ひろがる うぃるす も あります	
\\	メールを
\\	して 広[ひろ]がるウィルスもあります。			
\\	自信	自信[じしん]	じしん	
\\	あの人の顔には自信があふれているね。	あの 人[ひと]の 顔[かお]には 自信[じしん]があふれているね。	あの ひと の かお に は じしん が あふれて いる ね	
\\	あの 人[ひと]の 顔[かお]には
\\	があふれているね。			
\\	自主	自主[じしゅ]	じしゅ	
\\	その選手は自主トレーニングを始めたの。	その 選手[せんしゅ]は 自主[じしゅ]トレーニングを 始[はじ]めたの。	その せんしゅ は じしゅとれーにんぐ を はじめた の	
\\	その 選手[せんしゅ]は
\\	トレーニングを 始[はじ]めたの。			
\\	義理	義理[ぎり]	ぎり	
\\	彼は義理を大切にしています。	彼[かれ]は 義理[ぎり]を 大切[たいせつ]にしています。	かれ は ぎり を たいせつ に して います	
\\	彼[かれ]は
\\	を 大切[たいせつ]にしています。			
\\	議会	議会[ぎかい]	ぎかい	
\\	議会で区の来年の予定が話し合われたんだ。	議会[ぎかい]で 区[く]の 来年[らいねん]の 予定[よてい]が 話[はな]し 合[あ]われたんだ。	ぎかい で く の らいねん の よてい が はなしあわれた ん だ 。	
\\	で 区[く]の 来年[らいねん]の 予定[よてい]が 話[はな]し 合[あ]われたんだ。			
\\	議長	議長[ぎちょう]	ぎちょう	
\\	議長が開会の挨拶をしました。	議長[ぎちょう]が 開会[かいかい]の 挨拶[あいさつ]をしました。	ぎちょう が かいかい の あいさつ を しました	
\\	が 開会[かいかい]の 挨拶[あいさつ]をしました。			
\\	議題	議題[ぎだい]	ぎだい	
\\	今日の議題は面白そうです。	今日[きょう]の 議題[ぎだい]は 面白[おもしろ]そうです。	きょう の ぎだい は おもしろ そう です	
\\	今日[きょう]の
\\	は 面白[おもしろ]そうです。			
\\	議論	議論[ぎろん]	ぎろん	
\\	父は議論好きです。	父[ちち]は 議論[ぎろん] 好[ず]きです。	ちち は ぎろんずき です	
\\	父[ちち]は
\\	好[ず]きです。			
\\	結論	結論[けつろん]	けつろん	
\\	今日の話し合いでは結論が出なかった。	今日[きょう]の 話[はな]し 合[あ]いでは 結論[けつろん]が 出[で]なかった。	きょう の はなしあい で は けつろん が でなかった	
\\	今日[きょう]の 話[はな]し 合[あ]いでは
\\	が 出[で]なかった。			
\\	言論	言論[げんろん]	げんろん	
\\	言論の自由は国民の権利です。	言論[げんろん]の 自由[じゆう]は 国民[こくみん]の 権利[けんり]です。	げんろん の じゆう は こくみん の けんり です	
\\	の 自由[じゆう]は 国民[こくみん]の 権利[けんり]です。			
\\	違反	違反[いはん]	いはん	
\\	彼は校則に違反しました。	彼[かれ]は 校則[こうそく]に 違反[いはん]しました。	かれ は こうそく に いはん しました	
\\	彼[かれ]は 校則[こうそく]に
\\	しました。			
\\	解答	解答[かいとう]	かいとう	
\\	インターネットで試験の解答を確認したんだ。	インターネットで 試験[しけん]の 解答[かいとう]を 確認[かくにん]したんだ。	いんたーねっと で しけん の かいとう を かくにん した ん だ	
\\	インターネットで 試験[しけん]の
\\	を 確認[かくにん]したんだ。			
\\	経済的	経済的[けいざいてき]	けいざいてき	
\\	経済的な効果は計り知れません。	経済的[けいざいてき]な 効果[こうか]は 計[はか]り 知[し]れません。	けいざいてき な こうか は はかり しれません	
\\	な 効果[こうか]は 計[はか]り 知[し]れません。			
\\	自主的	自主的[じしゅてき]	じしゅてき	
\\	学生たちは自主的に勉強会を開いたわよ。	学生[がくせい]たちは 自主的[じしゅてき]に 勉強会[べんきょうかい]を 開[ひら]いたわよ。	がくせいたち は じしゅてき に べんきょうかい を ひらいた わ よ	
\\	学生[がくせい]たちは
\\	に 勉強会[べんきょうかい]を 開[ひら]いたわよ。			
\\	高等	高等[こうとう]	こうとう	
\\	人間は高等な生物と言われている。	人間[にんげん]は 高等[こうとう]な 生物[せいぶつ]と 言[い]われている。	にんげん は こうとう な せいぶつ と いわれて いる	
\\	人間[にんげん]は
\\	な 生物[せいぶつ]と 言[い]われている。			
\\	科学者	科学者[かがくしゃ]	かがくしゃ	
\\	彼はとても有名な科学者だ。	彼[かれ]はとても 有名[ゆうめい]な 科学者[かがくしゃ]だ。	かれ は とても ゆうめい な かがくしゃ だ	
\\	彼[かれ]はとても 有名[ゆうめい]な
\\	だ。			
\\	科目	科目[かもく]	かもく	
\\	今日は3科目のテストを受けたよ。	今日[きょう]は 3[さん] 科目[かもく]のテストを 受[う]けたよ。	きょう は さんかもく の てすと を うけた よ	
\\	今日[きょう]は 3[さん]
\\	のテストを 受[う]けたよ。			
\\	自然科学	自然科学[しぜんかがく]	しぜんかがく	
\\	姉は大学で自然科学を学んでいます。	姉[あね]は 大学[だいがく]で 自然科学[しぜんかがく]を 学[まな]んでいます。	あね は だいがく で しぜんかがく を まなんで います	
\\	姉[あね]は 大学[だいがく]で
\\	を 学[まな]んでいます。			
\\	社会科学	社会科学[しゃかいかがく]	しゃかいかがく	
\\	社会科学の分野に関心があります。	社会科学[しゃかいかがく]の 分野[ぶんや]に 関心[かんしん]があります。	しゃかいかがく の ぶんや に かんしん が あります	
\\	の 分野[ぶんや]に 関心[かんしん]があります。			
\\	生死	生死[せいし]	せいし	
\\	これは人の生死にかかわる問題です。	これは 人[ひと]の 生死[せいし]にかかわる 問題[もんだい]です。	これ は ひと の せいし に かかわる もんだい で す	
\\	これは 人[ひと]の
\\	にかかわる 問題[もんだい]です。			
\\	痛み	痛[いた]み	いたみ	
\\	背中に痛みがあります。	背中[せなか]に 痛[いた]みがあります。	せなか に いたみ が あります	
\\	背中[せなか]に
\\	があります。			
\\	痛む	痛[いた]む	いたむ	
\\	虫歯がずきずき痛みます。	虫歯[むしば]がずきずき 痛[いた]みます。	むしば が ずきずき いたみます	
\\	虫歯[むしば]がずきずき
\\	禁止	禁止[きんし]	きんし	
\\	交差点付近は駐車禁止です。	交差点付近[こうさてん ふきん]は 駐車[ちゅうしゃ] 禁止[きんし]です。	こうさてん ふきん は ちゅうしゃ きんし です	
\\	交差点付近[こうさてん ふきん]は 駐車[ちゅうしゃ]
\\	です。			
\\	禁じる	禁[きん]じる	きんじる	
\\	市は昨年から歩きタバコを禁じているの。	市[し]は 昨年[さくねん]から 歩[ある]きタバコを 禁[きん]じているの。	し は さくねん から あるきたばこ を きんじて いる の	
\\	市[し]は 昨年[さくねん]から 歩[ある]きタバコを
\\	の。			
\\	煙	煙[けむり]	けむり	
\\	煙突から煙がまっすぐ上っていたの。	煙突[えんとつ]から 煙[けむり]がまっすぐ 上[あが]っていたの。	えんとつ から けむり が まっすぐ あがって いた の	
\\	煙突[えんとつ]から
\\	がまっすぐ 上[あが]っていたの。			
\\	酒屋	酒屋[さかや]	さかや	
\\	そこの酒屋さんでビールを買って来て。	そこの 酒屋[さかや]さんでビールを 買[か]って 来[き]て。	そこ の さかやさん で びーる を かって きて	
\\	そこの
\\	さんでビールを 買[か]って 来[き]て。			
\\	禁酒	禁酒[きんしゅ]	きんしゅ	
\\	彼は禁酒しています。	彼[かれ]は 禁酒[きんしゅ]しています。	かれ は きんしゅ して います	
\\	彼[かれ]は
\\	しています。			
\\	険しい	険[けわ]しい	けわしい	
\\	父の表情が険しくなったの。	父[ちち]の 表情[ひょうじょう]が 険[けわ]しくなったの。	ちち の ひょうじょう が けわしく なった の	
\\	父[ちち]の 表情[ひょうじょう]が
\\	の。			
\\	証券	証券[しょうけん]	しょうけん	
\\	証券を売って資金にしようと思うの。	証券[しょうけん]を 売[う]って 資金[しきん]にしようと 思[おも]うの。	しょうけん を うって しきん に しよう と おもう の	
\\	を 売[う]って 資金[しきん]にしようと 思[おも]うの。			
\\	証明	証明[しょうめい]	しょうめい	
\\	容疑者のアリバイが証明されました。	容疑者[ようぎしゃ]のアリバイが 証明[しょうめい]されました。	ようぎしゃ の ありばい が しょうめい されました	
\\	容疑者[ようぎしゃ]のアリバイが
\\	されました。			
\\	生存	生存[せいぞん]	せいぞん	
\\	乗客は全員生存しています。	乗客[じょうきゃく]は 全員[ぜんいん] 生存[せいぞん]しています。	じょうきゃく は ぜんいん せいぞん して います	
\\	乗客[じょうきゃく]は 全員[ぜんいん]
\\	しています。			
\\	確保	確保[かくほ]	かくほ	
\\	優れた人材の確保は重要です。	優[すぐ]れた 人材[じんざい]の 確保[かくほ]は 重要[じゅうよう]です。	すぐれた じんざい の かくほ は じゅうよう です	
\\	優[すぐ]れた 人材[じんざい]の
\\	は 重要[じゅうよう]です。			
\\	確立	確立[かくりつ]	かくりつ	
\\	北欧では社会保障が確立されています。	北欧[ほくおう]では 社会保障[しゃかい ほしょう]が 確立[かくりつ]されています。	ほくおう で は しゃかい ほしょう が かくりつ されて います	
\\	北欧[ほくおう]では 社会保障[しゃかい ほしょう]が
\\	されています。			
\\	確定	確定[かくてい]	かくてい	
\\	試合の代表メンバーが確定したの。	試合[しあい]の 代表[だいひょう]メンバーが 確定[かくてい]したの。	しあい の だいひょう めんばー が かくてい した の	
\\	試合[しあい]の 代表[だいひょう]メンバーが
\\	したの。			
\\	確実	確実[かくじつ]	かくじつ	
\\	彼女が将来、大統領になることは確実です。	彼女[かのじょ]が 将来[しょうらい]、 大統領[だいとうりょう]になることは 確実[かくじつ]です。	かのじょ が しょうらい だいとうりょう に なる こと は かくじつ です	
\\	彼女[かのじょ]が 将来[しょうらい]、 大統領[だいとうりょう]になることは
\\	です。			
\\	確信	確信[かくしん]	かくしん	
\\	私は彼の成功を確信しています。	私[わたし]は 彼[かれ]の 成功[せいこう]を 確信[かくしん]しています。	わたし は かれ の せいこう を かくしん して います	
\\	私[わたし]は 彼[かれ]の 成功[せいこう]を
\\	しています。			
\\	確率	確率[かくりつ]	かくりつ	
\\	天気予報は当たる確率が高くなりましたね。	天気予報[てんき よほう]は 当[あ]たる 確率[かくりつ]が 高[たか]くなりましたね。	てんき よほう は あたる かくりつ が たかく なりました ね	
\\	天気予報[てんき よほう]は 当[あ]たる
\\	が 高[たか]くなりましたね。			
\\	軽率	軽率[けいそつ]	けいそつ	
\\	軽率な発言はしないよう気をつけなさい。	軽率[けいそつ]な 発言[はつげん]はしないよう 気[き]をつけなさい。	けいそつ な はつげん は しない よう き を つけなさい	
\\	な 発言[はつげん]はしないよう 気[き]をつけなさい。			
\\	食器	食器[しょっき]	しょっき	
\\	食器を全部新しくしました。	食器[しょっき]を 全部新[ぜんぶ あたら]しくしました。	しょっき を ぜんぶ あたらしく しました	
\\	を 全部新[ぜんぶ あたら]しくしました。			
\\	消火器	消火器[しょうかき]	しょうかき	
\\	消火器を交換しました。	消火器[しょうかき]を 交換[こうかん]しました。	しょうかき を こうかん しました	
\\	を 交換[こうかん]しました。			
\\	雨具	雨具[あまぐ]	あまぐ	
\\	雨具の用意を忘れないように。	雨具[あまぐ]の 用意[ようい]を 忘[わす]れないように。	あまぐ の ようい を わすれない よう に	
\\	の 用意[ようい]を 忘[わす]れないように。			
\\	準急	準急[じゅんきゅう]	じゅんきゅう	
\\	新宿駅まで準急で行ったのよ。	新宿駅[しんじゅくえき]まで 準急[じゅんきゅう]で 行[い]ったのよ。	しんじゅくえき まで じゅんきゅう で いった の よ	
\\	新宿駅[しんじゅくえき]まで
\\	で 行[い]ったのよ。			
\\	設立	設立[せつりつ]	せつりつ	
\\	当社は10年前に設立されたのよ。	当社[とうしゃ]は 10年前[じゅうねんまえ]に 設立[せつりつ]されたのよ。	とうしゃ は じゅうねんまえ に せつりつ された の よ	
\\	当社[とうしゃ]は 10年前[じゅうねんまえ]に
\\	されたのよ。			
\\	設備	設備[せつび]	せつび	
\\	この研究所には最新の設備が揃っています。	この 研究所[けんきゅうじょ]には 最新[さいしん]の 設備[せつび]が 揃[そろ]っています。	この けんきゅうじょ に は さいしん の せつび が そろって います	
\\	この 研究所[けんきゅうじょ]には 最新[さいしん]の
\\	が 揃[そろ]っています。			
\\	説	説[せつ]	せつ	
\\	私は彼の説が正しいと思う。	私[わたし]は 彼[かれ]の 説[せつ]が 正[ただ]しいと 思[おも]う。	わたし は かれ の せつ が ただしい と おもう	
\\	私[わたし]は 彼[かれ]の
\\	が 正[ただ]しいと 思[おも]う。			
\\	解説	解説[かいせつ]	かいせつ	
\\	新聞の解説欄を読みました。	新聞[しんぶん]の 解説[かいせつ] 欄[らん]を 読[よ]みました。	しんぶん の かいせつ らん を よみました	
\\	新聞[しんぶん]の
\\	欄[らん]を 読[よ]みました。			
\\	社説	社説[しゃせつ]	しゃせつ	
\\	この新聞の社説は面白いね。	この 新聞[しんぶん]の 社説[しゃせつ]は 面白[おもしろ]いね。	この しんぶん の しゃせつ は おもしろい ね	
\\	この 新聞[しんぶん]の
\\	は 面白[おもしろ]いね。			
\\	祭日	祭日[さいじつ]	さいじつ	
\\	うちの会社は祭日は休みです。	うちの 会社[かいしゃ]は 祭日[さいじつ]は 休[やす]みです。	うち の かいしゃ は さいじつ は やすみ です	
\\	うちの 会社[かいしゃ]は
\\	は 休[やす]みです。			
\\	際	際[さい]	さい	
\\	この際はっきり言っておきます。	この 際[さい]はっきり 言[い]っておきます。	この さい はっきり いって おきます	
\\	この
\\	はっきり 言[い]っておきます。			
\\	航空	航空[こうくう]	こうくう	
\\	会議で航空の安全について話し合ったの。	会議[かいぎ]で 航空[こうくう]の 安全[あんぜん]について 話[はな]し 合[あ]ったの。	かいぎ で こうくう の あんぜん に ついて はなしあった の	
\\	会議[かいぎ]で
\\	の 安全[あんぜん]について 話[はな]し 合[あ]ったの。			
\\	航空機	航空機[こうくうき]	こうくうき	
\\	去年は航空機の事故が多かった。	去年[きょねん]は 航空機[こうくうき]の 事故[じこ]が 多[おお]かった。	きょねん は こうくうき の じこ が おおかった	
\\	去年[きょねん]は
\\	の 事故[じこ]が 多[おお]かった。			
\\	航空券	航空券[こうくうけん]	こうくうけん	
\\	電話で航空券を予約しました。	電話[でんわ]で 航空券[こうくうけん]を 予約[よやく]しました。	でんわ で こうくうけん を よやく しました	
\\	電話[でんわ]で
\\	を 予約[よやく]しました。			
\\	結成	結成[けっせい]	けっせい	
\\	新しい代表チームが結成されました。	新[あたら]しい 代表[だいひょう]チームが 結成[けっせい]されました。	あたらしい だいひょう ちーむ が けっせい されました	
\\	新[あたら]しい 代表[だいひょう]チームが
\\	されました。			
\\	賛成	賛成[さんせい]	さんせい	
\\	私は彼の意見に賛成だ。	私[わたし]は 彼[かれ]の 意見[いけん]に 賛成[さんせい]だ。	わたし は かれ の いけん に さんせい だ	
\\	私[わたし]は 彼[かれ]の 意見[いけん]に
\\	だ。			
\\	高原	高原[こうげん]	こうげん	
\\	高原には気持ちのいい風が吹いていたよ。	高原[こうげん]には 気持[きも]ちのいい 風[かぜ]が 吹[ふ]いていたよ。	こうげん に は きもち の いい かぜ が ふいていた よ 。	
\\	には 気持[きも]ちのいい 風[かぜ]が 吹[ふ]いていたよ。			
\\	資料	資料[しりょう]	しりょう	
\\	図書館で資料を借りて来ました。	図書館[としょかん]で 資料[しりょう]を 借[か]りて 来[き]ました。	としょかん で しりょう を かりて きました	
\\	図書館[としょかん]で
\\	を 借[か]りて 来[き]ました。			
\\	資産	資産[しさん]	しさん	
\\	大臣の資産が公表されたね。	大臣[だいじん]の 資産[しさん]が 公表[こうひょう]されたね。	だいじん の しさん が こうひょう された ね	
\\	大臣[だいじん]の
\\	が 公表[こうひょう]されたね。			
\\	資格	資格[しかく]	しかく	
\\	日本語を教える資格を取りたいです。	日本語[にほんご]を 教[おし]える 資格[しかく]を 取[と]りたいです。	にほんご を おしえる しかく を とりたい です	
\\	日本語[にほんご]を 教[おし]える
\\	を 取[と]りたいです。			
\\	資本主義	資本主義[しほんしゅぎ]	しほんしゅぎ	
\\	その国は次第に資本主義になったわね。	その 国[くに]は 次第[しだい]に 資本主義[しほんしゅぎ]になったわね。	その くに は しだいに しほんしゅぎ に なった わ ね	
\\	その 国[くに]は 次第[しだい]に
\\	になったわね。			
\\	資源	資源[しげん]	しげん	
\\	地球の資源を守りましょう。	地球[ちきゅう]の 資源[しげん]を 守[まも]りましょう。	ちきゅう の しげん を まもりましょう	
\\	地球[ちきゅう]の
\\	を 守[まも]りましょう。			
\\	願書	願書[がんしょ]	がんしょ	
\\	今日大学に願書を送りました。	今日大学[きょう だいがく]に 願書[がんしょ]を 送[おく]りました。	きょう だいがく に がんしょ を おくりました	
\\	今日大学[きょう だいがく]に
\\	を 送[おく]りました。			
\\	異性	異性[いせい]	いせい	
\\	息子はもう異性を意識している。	息子[むすこ]はもう 異性[いせい]を 意識[いしき]している。	むすこ は もう いせい を いしき して いる	
\\	息子[むすこ]はもう
\\	を 意識[いしき]している。			
\\	異常	異常[いじょう]	いじょう	
\\	今年の夏は異常な暑さですね。	今年[ことし]の 夏[なつ]は 異常[いじょう]な 暑[あつ]さですね。	ことし の なつ は いじょう な あつさ です ね	
\\	今年[ことし]の 夏[なつ]は
\\	な 暑[あつ]さですね。			
\\	調べ	調[しら]べ	しらべ	
\\	警察の調べで女性の身元がわかりました。	警察[けいさつ]の 調[しら]べで 女性[じょせい]の 身元[みもと]がわかりました。	けいさつ の しらべ で じょせい の みもと が わかりました	
\\	警察[けいさつ]の
\\	で 女性[じょせい]の 身元[みもと]がわかりました。			
\\	節約	節約[せつやく]	せつやく	
\\	電気や水を節約しましょう。	電気[でんき]や 水[みず]を 節約[せつやく]しましょう。	でんき や みず を せつやく しましょう	
\\	電気[でんき]や 水[みず]を
\\	しましょう。			
\\	真相	真相[しんそう]	しんそう	
\\	最近、事件の真相が明らかになったよ。	最近[さいきん]、 事件[じけん]の 真相[しんそう]が 明[あき]らかになったよ。	さいきん じけん の しんそう が あきらか に なった よ	
\\	最近[さいきん]、 事件[じけん]の
\\	が 明[あき]らかになったよ。			
\\	相変わらず	相変[あいか]わらず	あいかわらず	
\\	彼は相変わらず忙しいですね。	彼[かれ]は 相変[あいか]わらず 忙[いそが]しいですね。	かれ は あいかわらず いそがしい です ね	
\\	彼[かれ]は
\\	忙[いそが]しいですね。			
\\	雑談	雑談[ざつだん]	ざつだん	
\\	その先生はいつも授業の前に雑談をするの。	その 先生[せんせい]はいつも 授業[じゅぎょう]の 前[まえ]に 雑談[ざつだん]をするの。	その せんせい は いつも じゅぎょう の まえ に ざつだん を する の	
\\	その 先生[せんせい]はいつも 授業[じゅぎょう]の 前[まえ]に
\\	をするの。			
\\	記事	記事[きじ]	きじ	
\\	環境問題に関する記事を読んだの。	環境問題[かんきょう もんだい]に 関[かん]する 記事[きじ]を 読[よ]んだの。	かんきょう もんだい に かんする きじ を よんだ の	
\\	環境問題[かんきょう もんだい]に 関[かん]する
\\	を 読[よ]んだの。			
\\	記号	記号[きごう]	きごう	
\\	地図にはいろいろな記号が使われているのね。	地図[ちず]にはいろいろな 記号[きごう]が 使[つか]われているのね。	ちず に は いろいろ な きごう が つかわれて いる の ね	
\\	地図[ちず]にはいろいろな
\\	が 使[つか]われているのね。			
\\	記入	記入[きにゅう]	きにゅう	
\\	こちらにお名前をご記入ください。	こちらにお 名前[なまえ]をご 記入[きにゅう]ください。	こちら に おなまえ を ご きにゅう ください	
\\	こちらにお 名前[なまえ]をご
\\	ください。			
\\	記憶	記憶[きおく]	きおく	
\\	当時のことはしっかり記憶しているよ。	当時[とうじ]のことはしっかり 記憶[きおく]しているよ。	とうじ の こと は しっかり きおく して いる よ	
\\	当時[とうじ]のことはしっかり
\\	しているよ。			
\\	関心	関心[かんしん]	かんしん	
\\	彼は政治に関心が強いね。	彼[かれ]は 政治[せいじ]に 関心[かんしん]が 強[つよ]いね。	かれ は せいじ に かんしん が つよい ね	
\\	彼[かれ]は 政治[せいじ]に
\\	が 強[つよ]いね。			
\\	税関	税関[ぜいかん]	ぜいかん	
\\	毛皮を税関で没収されたんだ。	毛皮[けがわ]を 税関[ぜいかん]で 没収[ぼっしゅう]されたんだ。	けがわ を ぜいかん で ぼっしゅう された ん だ	
\\	毛皮[けがわ]を
\\	で 没収[ぼっしゅう]されたんだ。			
\\	関節	関節[かんせつ]	かんせつ	
\\	手首の関節をひねっちゃった。	手首[てくび]の 関節[かんせつ]をひねっちゃった。	てくび の かんせつ を ひねっちゃった	
\\	手首[てくび]の
\\	をひねっちゃった。			
\\	関わる	関[かか]わる	かかわる	
\\	医師は人の命に関わる大切な職業だよ。	医師[いし]は 人[ひと]の 命[いのち]に 関[かか]わる 大切[たいせつ]な 職業[しょくぎょう]だよ。	いし は ひと の いのち に かかわる たいせつ な しょくぎょう だ よ	
\\	医師[いし]は 人[ひと]の 命[いのち]に
\\	大切[たいせつ]な 職業[しょくぎょう]だよ。			
\\	現状	現状[げんじょう]	げんじょう	
\\	問題を解決できないのが現状です。	問題[もんだい]を 解決[かいけつ]できないのが 現状[げんじょう]です。	もんだい を かいけつ できない の が げんじょう です	
\\	問題[もんだい]を 解決[かいけつ]できないのが
\\	です。			
\\	行政	行政[ぎょうせい]	ぎょうせい	
\\	年金問題は行政の最大の課題のひとつよ。	年金問題[ねんきん もんだい]は 行政[ぎょうせい]の 最大[さいだい]の 課題[かだい]のひとつよ。	ねんきん もんだい は ぎょうせい の さいだい の かだい の ひとつ よ	
\\	年金問題[ねんきん もんだい]は
\\	の 最大[さいだい]の 課題[かだい]のひとつよ。			
\\	策	策[さく]	さく	
\\	その問題に対する策を皆で考えたの。	その 問題[もんだい]に 対[たい]する 策[さく]を 皆[みんな]で 考[かんが]えたの。	その もんだい に たいする さく を みんな で かんがえた の	
\\	その 問題[もんだい]に 対[たい]する
\\	を 皆[みんな]で 考[かんが]えたの。			
\\	補う	補[おぎな]う	おぎなう	
\\	夏は水分を十分に補いましょう。	夏[なつ]は 水分[すいぶん]を 十分[じゅうぶん]に 補[おぎな]いましょう。	なつ は すいぶん を じゅうぶん に おぎないましょう	
\\	夏[なつ]は 水分[すいぶん]を 十分[じゅうぶん]に
\\	足首	足首[あしくび]	あしくび	
\\	彼は足首を痛めています。	彼[かれ]は 足首[あしくび]を 痛[いた]めています。	かれ は あしくび を いためて います	
\\	彼[かれ]は
\\	を 痛[いた]めています。			
\\	首脳	首脳[しゅのう]	しゅのう	
\\	東京で五カ国の首脳会談が開かれています。	東京[とうきょう]で 五[ご]カ 国[こく]の 首脳[しゅのう] 会談[かいだん]が 開[ひら]かれています。	とうきょう で ごかこく の しゅのうかいだん が ひらかれて います	
\\	東京[とうきょう]で 五[ご]カ 国[こく]の
\\	会談[かいだん]が 開[ひら]かれています。			
\\	頭	頭[かしら]	かしら	
\\	彼は一家のお頭だったの。	彼[かれ]は 一家[いっか]のお 頭[かしら]だったの。	かれ は いっか の おかしら だった の	
\\	彼[かれ]は 一家[いっか]のお
\\	だったの。			
\\	頭痛	頭痛[ずつう]	ずつう	
\\	今日は頭痛がします。	今日[きょう]は 頭痛[ずつう]がします。	きょう は ずつう が します	
\\	今日[きょう]は
\\	がします。			
\\	顔色	顔色[かおいろ]	かおいろ	
\\	彼は顔色がよくありませんね。	彼[かれ]は 顔色[かおいろ]がよくありませんね。	かれ は かおいろ が よく ありません ね 。	
\\	彼[かれ]は
\\	がよくありませんね。			
\\	革命	革命[かくめい]	かくめい	
\\	それは歴史上の大きな革命です。	それは 歴史上[れきしじょう]の 大[おお]きな 革命[かくめい]です。	それ は れきしじょう の おおき な かくめい です	
\\	それは 歴史上[れきしじょう]の 大[おお]きな
\\	です。			
\\	生命	生命[せいめい]	せいめい	
\\	生命は海から始まったと言われている。	生命[せいめい]は 海[うみ]から 始[はじ]まったと 言[い]われている。	せいめい は うみ から はじまった と いわれて いる	
\\	は 海[うみ]から 始[はじ]まったと 言[い]われている。			
\\	運命	運命[うんめい]	うんめい	
\\	運命には逆らえないよ。	運命[うんめい]には 逆[さか]らえないよ。	うんめい に は さからえない よ	
\\	には 逆[さか]らえないよ。			
\\	組合	組合[くみあい]	くみあい	
\\	今日は組合の集まりがあるわ。	今日[きょう]は 組合[くみあい]の 集[あつ]まりがあるわ。	きょう は くみあい の あつまり が ある わ	
\\	今日[きょう]は
\\	の 集[あつ]まりがあるわ。			
\\	組	組[くみ]	くみ	
\\	彼は1年2組の生徒です。	彼[かれ]は 1年2[いちねん に] 組[くみ]の 生徒[せいと]です。	かれ は いちねん にくみ の せいと です	
\\	彼[かれ]は 1年2[いちねん に]
\\	の 生徒[せいと]です。			
\\	組む	組[く]む	くむ	
\\	このプロジェクトで私は彼と組んでいるんだ。	このプロジェクトで 私[わたし]は 彼[かれ]と 組[く]んでいるんだ。	この ぷろじぇくと で わたし は かれ と くんで いる ん だ	
\\	このプロジェクトで 私[わたし]は 彼[かれ]と
\\	んだ。			
\\	組み合わせる	組[く]み 合[あ]わせる	くみあわせる	
\\	いろいろな花を組み合わせ花束を作りました。	いろいろな 花[はな]を 組[く]み 合[あ]わせ 花束[はなたば]を 作[つく]りました。	いろいろ な はな を くみあわせ はなたば を つくりました	
\\	いろいろな 花[はな]を
\\	花束[はなたば]を 作[つく]りました。			
\\	組み込む	組[く]み 込[こ]む	くみこむ	
\\	キャンペーンに新しいイベントを組み込みました。	キャンペーンに 新[あたら]しいイベントを 組[く]み 込[こ]みました。	きゃんぺーん に あたらしい いべんと を くみこみました	
\\	キャンペーンに 新[あたら]しいイベントを
\\	組み合わせ	組[く]み 合[あ]わせ	くみあわせ	
\\	色の組み合わせで印象が変わりますよ。	色[いろ]の 組[く]み 合[あ]わせで 印象[いんしょう]が 変[か]わりますよ。	いろ の くみあわせ で いんしょう が かわります よ	
\\	色[いろ]の
\\	で 印象[いんしょう]が 変[か]わりますよ。			
\\	織物	織物[おりもの]	おりもの	
\\	その町は織物業で有名です。	その 町[まち]は 織物[おりもの] 業[ぎょう]で 有名[ゆうめい]です。	その まち は おりものぎょう で ゆうめい です	
\\	その 町[まち]は
\\	業[ぎょう]で 有名[ゆうめい]です。			
\\	進出	進出[しんしゅつ]	しんしゅつ	
\\	大手チェーン店が進出している。	大手[おおて]チェーン 店[てん]が 進出[しんしゅつ]している。	おおて ちぇーんてん が しんしゅつ して いる	
\\	大手[おおて]チェーン 店[てん]が
\\	している。			
\\	進行	進行[しんこう]	しんこう	
\\	学会は予定通りに進行しています。	学会[がっかい]は 予定通[よてい どお]りに 進行[しんこう]しています。	がっかい は よてい どおり に しんこう して います	
\\	学会[がっかい]は 予定通[よてい どお]りに
\\	しています。			
\\	進歩	進歩[しんぽ]	しんぽ	
\\	科学技術は目覚しく進歩しているの。	科学技術[かがく ぎじゅつ]は 目覚[めざま]しく 進歩[しんぽ]しているの。	かがく ぎじゅつ は めざましく しんぽ して いる の	
\\	科学技術[かがく ぎじゅつ]は 目覚[めざま]しく
\\	しているの。			
\\	進路	進路[しんろ]	しんろ	
\\	卒業後の進路を迷っています。	卒業後[そつぎょう ご]の 進路[しんろ]を 迷[まよ]っています。	そつぎょう ご の しんろ を まよって います	
\\	卒業後[そつぎょう ご]の
\\	を 迷[まよ]っています。			
\\	行進	行進[こうしん]	こうしん	
\\	開会式で選手たちが行進しました。	開会式[かいかいしき]で 選手[せんしゅ]たちが 行進[こうしん]しました。	かいかいしき で せんしゅたち が こうしん しました	
\\	開会式[かいかいしき]で 選手[せんしゅ]たちが
\\	しました。			
\\	辞任	辞任[じにん]	じにん	
\\	社長の辞任が決まりました。	社長[しゃちょう]の 辞任[じにん]が 決[き]まりました。	しゃちょう の じにん が きまりました	
\\	社長[しゃちょう]の
\\	が 決[き]まりました。			
\\	義務	義務[ぎむ]	ぎむ	
\\	労働は国民の義務です。	労働[ろうどう]は 国民[こくみん]の 義務[ぎむ]です。	ろうどう は こくみん の ぎむ です	
\\	労働[ろうどう]は 国民[こくみん]の
\\	です。			
\\	雇用	雇用[こよう]	こよう	
\\	彼はその会社と雇用契約を結んだの。	彼[かれ]はその 会社[かいしゃ]と 雇用[こよう] 契約[けいやく]を 結[むす]んだの。	かれ は その かいしゃ と こよう けいやく を むすんだ の	
\\	彼[かれ]はその 会社[かいしゃ]と
\\	契約[けいやく]を 結[むす]んだの。			
\\	集中	集中[しゅうちゅう]	しゅうちゅう	
\\	勉強に集中しなさい。	勉強[べんきょう]に 集中[しゅうちゅう]しなさい。	べんきょう に しゅうちゅう しなさい	
\\	勉強[べんきょう]に
\\	しなさい。			
\\	集合	集合[しゅうごう]	しゅうごう	
\\	7時5分に駅で集合しましょう。	7時5分[しち じ ご ふん]に 駅[えき]で 集合[しゅうごう]しましょう。	しち じ ご ふん に えき で しゅうごう しましょう	
\\	7時5分[しち じ ご ふん]に 駅[えき]で
\\	しましょう。			
\\	集まり	集[あつ]まり	あつまり	
\\	雨で集まりが悪いですね。	雨[あめ]で 集[あつ]まりが 悪[わる]いですね。	あめ で あつまり が わるい です ね	
\\	雨[あめ]で
\\	が 悪[わる]いですね。			
\\	需要	需要[じゅよう]	じゅよう	
\\	需要が多過ぎて生産が追い付きません。	需要[じゅよう]が 多過[おおす]ぎて 生産[せいさん]が 追[お]い 付[つ]きません。	じゅよう が おおすぎ て せいさん が おいつきません	
\\	が 多過[おおす]ぎて 生産[せいさん]が 追[お]い 付[つ]きません。			
\\	職員	職員[しょくいん]	しょくいん	
\\	ここは職員専用の出入り口です。	ここは 職員[しょくいん] 専用[せんよう]の 出入[でい]り 口[ぐち]です。	ここ は しょくいん せんよう の でいりぐち です	
\\	ここは
\\	専用[せんよう]の 出入[でい]り 口[ぐち]です。			
\\	職場	職場[しょくば]	しょくば	
\\	自宅から職場まで1時間かかります。	自宅[じたく]から 職場[しょくば]まで 1時間[いちじかん]かかります。	じたく から しょくば まで いちじかん かかります	
\\	自宅[じたく]から
\\	まで 1時間[いちじかん]かかります。			
\\	職業	職業[しょくぎょう]	しょくぎょう	
\\	あなたの職業を教えてください。	あなたの 職業[しょくぎょう]を 教[おし]えてください。	あなた の しょくぎょう を おしえて ください	
\\	あなたの
\\	を 教[おし]えてください。			
\\	追う	追[お]う	おう	
\\	警官は怪しい男の後を追ったよ。	警官[けいかん]は 怪[あや]しい 男[おとこ]の 後[あと]を 追[お]ったよ。	けいかん は あやしい おとこ の あと を おった よ	
\\	警官[けいかん]は 怪[あや]しい 男[おとこ]の 後[あと]を
\\	よ。			
\\	追い出す	追[お]い 出[だ]す	おいだす	
\\	彼は家から追い出された。	彼[かれ]は 家[いえ]から 追[お]い 出[だ]された。	かれ は いえ から おいだされた	
\\	彼[かれ]は 家[いえ]から
\\	空想	空想[くうそう]	くうそう	
\\	彼は空想にふけっているの。	彼[かれ]は 空想[くうそう]にふけっているの。	かれ は くうそう に ふけって いる の	
\\	彼[かれ]は
\\	にふけっているの。			
\\	現像	現像[げんぞう]	げんぞう	
\\	このフィルムを現像してください。	このフィルムを 現像[げんぞう]してください。	この ふぃるむ を げんぞう して ください	
\\	このフィルムを
\\	してください。			
\\	現象	現象[げんしょう]	げんしょう	
\\	村では最近、不思議な現象が起きています。	村[むら]では 最近[さいきん]、 不思議[ふしぎ]な 現象[げんしょう]が 起[お]きています。	むら で は さいきん ふしぎ な げんしょう が おきて います	
\\	村[むら]では 最近[さいきん]、 不思議[ふしぎ]な
\\	が 起[お]きています。			
\\	障子	障子[しょうじ]	しょうじ	
\\	猫が障子を破った。	猫[ねこ]が 障子[しょうじ]を 破[やぶ]った。	ねこ が しょうじ を やぶった	
\\	猫[ねこ]が
\\	を 破[やぶ]った。			
\\	言い換える	言[い]い 換[か]える	いいかえる	
\\	彼は易しい言葉に言い換えたんだ。	彼[かれ]は 易[やさ]しい 言葉[ことば]に 言[い]い 換[か]えたんだ。	かれ は やさしい ことば に いいかえた ん だ	
\\	彼[かれ]は 易[やさ]しい 言葉[ことば]に
\\	んだ。			
\\	着替え	着替[きが]え	きがえ	
\\	着替えを旅行カバンに詰めたよ。	着替[きが]えを 旅行[りょこう]カバンに 詰[つ]めたよ。	きがえ を りょこう かばん に つめた よ	
\\	を 旅行[りょこう]カバンに 詰[つ]めたよ。			
\\	火災	火災[かさい]	かさい	
\\	火災の原因は放火だそうです。	火災[かさい]の 原因[げんいん]は 放火[ほうか]だそうです。	かさい の げんいん は ほうか だ そう です	
\\	の 原因[げんいん]は 放火[ほうか]だそうです。			
\\	災難	災難[さいなん]	さいなん	
\\	旅先で思いがけない災難にあいました。	旅先[たびさき]で 思[おも]いがけない 災難[さいなん]にあいました。	たびさき で おもいがけない さいなん に あいました	
\\	旅先[たびさき]で 思[おも]いがけない
\\	にあいました。			
\\	障害	障害[しょうがい]	しょうがい	
\\	まずは障害を取り除いてから計画を進めよう。	まずは 障害[しょうがい]を 取[と]り 除[のぞ]いてから 計画[けいかく]を 進[すす]めよう。	まず は しょうがい を とりのぞいて から けいかく を すすめよう	
\\	まずは
\\	を 取[と]り 除[のぞ]いてから 計画[けいかく]を 進[すす]めよう。			
\\	災害	災害[さいがい]	さいがい	
\\	地震は自然災害のひとつです。	地震[じしん]は 自然[しぜん] 災害[さいがい]のひとつです。	じしん は しぜん さいがい の ひとつ です	
\\	地震[じしん]は 自然[しぜん]
\\	のひとつです。			
\\	被せる	被[かぶ]せる	かぶせる	
\\	本にカバーを被せたの。	本[ほん]にカバーを 被[かぶ]せたの。	ほん に かばー を かぶせた の 。	
\\	本[ほん]にカバーを
\\	の。			
\\	警告	警告[けいこく]	けいこく	
\\	車に駐車違反の警告を貼られました。	車[くるま]に 駐車違反[ちゅうしゃ いはん]の 警告[けいこく]を 貼[は]られました。	くるま に ちゅうしゃ いはん の けいこく を はられました	
\\	車[くるま]に 駐車違反[ちゅうしゃ いはん]の
\\	を 貼[は]られました。			
\\	警官	警官[けいかん]	けいかん	
\\	道で警官に呼び止められた。	道[みち]で 警官[けいかん]に 呼[よ]び 止[と]められた。	みち で けいかん に よびとめられた	
\\	道[みち]で
\\	に 呼[よ]び 止[と]められた。			
\\	管	管[かん]	かん	
\\	ガス管が爆発しました。	ガス 管[かん]が 爆発[ばくはつ]しました。	がすかん が ばくはつ しました	
\\	ガス
\\	が 爆発[ばくはつ]しました。			
\\	犯す	犯[おか]す	おかす	
\\	彼は大きな過ちを犯している。	彼[かれ]は 大[おお]きな 過[あやま]ちを 犯[おか]している。	かれ は おおき な あやまち を おかして いる	
\\	彼[かれ]は 大[おお]きな 過[あやま]ちを
\\	自殺	自殺[じさつ]	じさつ	
\\	犯人は警察に捕まる前に自殺しました。	犯人[はんにん]は 警察[けいさつ]に 捕[つか]まる 前[まえ]に 自殺[じさつ]しました。	はんにん は けいさつ に つかまる まえ に じさつ しました	
\\	犯人[はんにん]は 警察[けいさつ]に 捕[つか]まる 前[まえ]に
\\	しました。			
\\	競技	競技[きょうぎ]	きょうぎ	
\\	彼は個人競技のスポーツが好きです。	彼[かれ]は 個人[こじん] 競技[きょうぎ]のスポーツが 好[す]きです。	かれ は こじん きょうぎ の すぽーつ が すき です	
\\	彼[かれ]は 個人[こじん]
\\	のスポーツが 好[す]きです。			
\\	混雑	混雑[こんざつ]	こんざつ	
\\	今日はデパートが混雑していたよ。	今日[きょう]はデパートが 混雑[こんざつ]していたよ。	きょう は でぱーと が こんざつ して いた よ	
\\	今日[きょう]はデパートが
\\	していたよ。			
\\	混乱	混乱[こんらん]	こんらん	
\\	地震の後、町は大混乱だったよ。	地震[じしん]の 後[あと]、 町[まち]は 大[だい] 混乱[こんらん]だったよ。	じしん の あと まち は だいこんらん だった よ	
\\	地震[じしん]の 後[あと]、 町[まち]は 大[だい]
\\	だったよ。			
\\	索引	索引[さくいん]	さくいん	
\\	索引はたいてい本の後ろについていますよ。	索引[さくいん]はたいてい 本[ほん]の 後[うし]ろについていますよ。	さくいん は たいてい ほん の うしろ に ついて います よ	
\\	はたいてい 本[ほん]の 後[うし]ろについていますよ。			
\\	落ち着く	落[お]ち 着[つ]く	おちつく	
\\	私の話を落ち着いて聞いてください。	私[わたし]の 話[はなし]を 落[お]ち 着[つ]いて 聞[き]いてください。	わたし の はなし を おちついて きいて ください	
\\	私[わたし]の 話[はなし]を
\\	聞[き]いてください。			
\\	落ち着き	落[お]ち 着[つ]き	おちつき	
\\	弟は落ち着きが足りません。	弟[おとうと]は 落[お]ち 着[つ]きが 足[た]りません。	おとうと は おちつき が たりません	
\\	弟[おとうと]は
\\	が 足[た]りません。			
\\	落とし物	落[お]とし 物[もの]	おとしもの	
\\	財布の落とし物を拾いました。	財布[さいふ]の 落[お]とし 物[もの]を 拾[ひろ]いました。	さいふ の おとしもの を ひろいました	
\\	財布[さいふ]の
\\	を 拾[ひろ]いました。			
\\	石油	石油[せきゆ]	せきゆ	
\\	日本は石油のほとんどを輸入しています。	日本[にっぽん]は 石油[せきゆ]のほとんどを 輸入[ゆにゅう]しています。	にっぽん は せきゆ の ほとんど を ゆにゅう して います	
\\	日本[にっぽん]は
\\	のほとんどを 輸入[ゆにゅう]しています。			
\\	車庫	車庫[しゃこ]	しゃこ	
\\	車を車庫に入れておきました。	車[くるま]を 車庫[しゃこ]に 入[い]れておきました。	くるま を しゃこ に いれて おきました	
\\	車[くるま]を
\\	に 入[い]れておきました。			
\\	金庫	金庫[きんこ]	きんこ	
\\	ホテルの金庫に貴重品を入れたよ。	ホテルの 金庫[きんこ]に 貴重品[きちょうひん]を 入[い]れたよ。	ほてる の きんこ に きちょうひん を いれた よ	
\\	ホテルの
\\	に 貴重品[きちょうひん]を 入[い]れたよ。			
\\	軽快	軽快[けいかい]	けいかい	
\\	彼らは軽快なステップで踊り出したの。	彼[かれ]らは 軽快[けいかい]なステップで 踊[おど]り 出[だ]したの。	かれら は けいかい な すてっぷ で おどりだした の	
\\	彼[かれ]らは
\\	なステップで 踊[おど]り 出[だ]したの。			
\\	湿度	湿度[しつど]	しつど	
\\	日本の夏は湿度が高いです。	日本[にほん]の 夏[なつ]は 湿度[しつど]が 高[たか]いです。	にほん の なつ は しつど が たかい です	
\\	日本[にほん]の 夏[なつ]は
\\	が 高[たか]いです。			
\\	湿っぽい	湿[しめ]っぽい	しめっぽい	
\\	雨の日が続いて家の中が湿っぽいな。	雨[あめ]の 日[ひ]が 続[つづ]いて 家[いえ]の 中[なか]が 湿[しめ]っぽいな。	あめ の ひ が つづいて いえ の なか が しめっぽい な	
\\	雨[あめ]の 日[ひ]が 続[つづ]いて 家[いえ]の 中[なか]が
\\	な。			
\\	湿気	湿気[しっけ]	しっけ	
\\	この部屋は暗くて湿気が多いね。	この 部屋[へや]は 暗[くら]くて 湿気[しっけ]が 多[おお]いね。	この へや は くらくて しっけ が おおい ね	
\\	この 部屋[へや]は 暗[くら]くて
\\	が 多[おお]いね。			
\\	湿る	湿[しめ]る	しめる	
\\	洗濯物がまだ湿っています。	洗濯物[せんたくもの]がまだ 湿[しめ]っています。	せんたくもの が まだ しめって います	
\\	洗濯物[せんたくもの]がまだ
\\	環境	環境[かんきょう]	かんきょう	
\\	引っ越して環境が変わりました。	引[ひ]っ 越[こ]して 環境[かんきょう]が 変[か]わりました。	ひっこして かんきょう が かわりました	
\\	引[ひ]っ 越[こ]して
\\	が 変[か]わりました。			
\\	観察	観察[かんさつ]	かんさつ	
\\	花の成長を観察して日記を書きなさい。	花[はな]の 成長[せいちょう]を 観察[かんさつ]して 日記[にっき]を 書[か]きなさい。	はな の せいちょう を かんさつ して にっき を かきなさい	
\\	花[はな]の 成長[せいちょう]を
\\	して 日記[にっき]を 書[か]きなさい。			
\\	観客	観客[かんきゃく]	かんきゃく	
\\	観客は興奮していたよ。	観客[かんきゃく]は 興奮[こうふん]していたよ。	かんきゃく は こうふん して いた よ	
\\	は 興奮[こうふん]していたよ。			
\\	観光	観光[かんこう]	かんこう	
\\	京都で3日間観光しました。	京都[きょうと]で 3日間[みっかかん] 観光[かんこう]しました。	きょうと で みっかかん かんこう しました	
\\	京都[きょうと]で 3日間[みっかかん]
\\	しました。			
\\	観測	観測[かんそく]	かんそく	
\\	先生が星の観測に連れて行ってくれました。	先生[せんせい]が 星[ほし]の 観測[かんそく]に 連[つ]れて 行[い]ってくれました。	せんせい が ほし の かんそく に つれて いって くれました	
\\	先生[せんせい]が 星[ほし]の
\\	に 連[つ]れて 行[い]ってくれました。			
\\	衛星	衛星[えいせい]	えいせい	
\\	衛星が打ち上げられましたね。	衛星[えいせい]が 打[う]ち 上[あ]げられましたね。	えいせい が うちあげられました ね	
\\	が 打[う]ち 上[あ]げられましたね。			
\\	衛生	衛生[えいせい]	えいせい	
\\	衛生には十分気を付けましょう。	衛生[えいせい]には 十分気[じゅうぶん き]を 付[つ]けましょう。	えいせい に は じゅうぶん き を つけましょう	
\\	には 十分気[じゅうぶん き]を 付[つ]けましょう。			
\\	衛生的	衛生的[えいせいてき]	えいせいてき	
\\	このレストランは衛生的で安心です。	このレストランは 衛生的[えいせいてき]で 安心[あんしん]です。	この れすとらん は えいせいてき で あんしん です	
\\	このレストランは
\\	で 安心[あんしん]です。			
\\	球	球[きゅう]	きゅう	
\\	彼は球技が大好きです。	彼[かれ]は 球[きゅう] 技[ぎ]が 大好[だいす]きです。	かれ は きゅうぎ が だいすき です	
\\	彼[かれ]は
\\	技[ぎ]が 大好[だいす]きです。			
\\	震度	震度[しんど]	しんど	
\\	今朝の地震は震度3でしたよ。	今朝[けさ]の 地震[じしん]は 震度[しんど] 3[さん]でしたよ。	けさ の じしん は しんど さん でした よ	
\\	今朝[けさ]の 地震[じしん]は
\\	3[さん]でしたよ。			
\\	神経	神経[しんけい]	しんけい	
\\	彼は神経が細かいですね。	彼[かれ]は 神経[しんけい]が 細[こま]かいですね。	かれ は しんけい が こまかい です ね	
\\	彼[かれ]は
\\	が 細[こま]かいですね。			
\\	神様	神様[かみさま]	かみさま	
\\	神様にお願いしました。	神様[かみさま]にお 願[ねが]いしました。	かみさま に おねがい しました	
\\	にお 願[ねが]いしました。			
\\	神道	神道[しんとう]	しんとう	
\\	彼女の家は神道です。	彼女[かのじょ]の 家[いえ]は 神道[しんとう]です。	かのじょ の いえ は しんとう です	
\\	彼女[かのじょ]の 家[いえ]は
\\	です。			
\\	精神	精神[せいしん]	せいしん	
\\	彼女は今、精神が不安定だ。	彼女[かのじょ]は 今[いま]、 精神[せいしん]が 不安定[ふあんてい]だ。	かのじょ は いま せいしん が ふあんてい だ	
\\	彼女[かのじょ]は 今[いま]、
\\	が 不安定[ふあんてい]だ。			
\\	精算	精算[せいさん]	せいさん	
\\	降りる駅で料金を精算して下さい。	降[お]りる 駅[えき]で 料金[りょうきん]を 精算[せいさん]して 下[くだ]さい。	おりる えき で りょうきん を せいさん して ください	
\\	降[お]りる 駅[えき]で 料金[りょうきん]を
\\	して 下[くだ]さい。			
\\	貴重	貴重[きちょう]	きちょう	
\\	彼は貴重な体験をしたわね。	彼[かれ]は 貴重[きちょう]な 体験[たいけん]をしたわね。	かれ は きちょう な たいけん を した わ ね	
\\	彼[かれ]は
\\	な 体験[たいけん]をしたわね。			
\\	貴重品	貴重品[きちょうひん]	きちょうひん	
\\	貴重品は自分で持っていてください。	貴重品[きちょうひん]は 自分[じぶん]で 持[も]っていてください。	きちょうひん は じぶん で もって いて ください	
\\	は 自分[じぶん]で 持[も]っていてください。			
\\	跡	跡[あと]	あと	
\\	ここにタイヤの跡があるわ。	ここにタイヤの 跡[あと]があるわ。	ここ に たいや の あと が ある わ	
\\	ここにタイヤの
\\	があるわ。			
\\	足跡	足跡[あしあと]	あしあと	
\\	雪の上にうさぎの足跡があった。	雪[ゆき]の 上[うえ]にうさぎの 足跡[あしあと]があった。	ゆき の うえ に うさぎ の あしあと が あった	
\\	雪[ゆき]の 上[うえ]にうさぎの
\\	があった。			
\\	経歴	経歴[けいれき]	けいれき	
\\	あなたの経歴をメールで送ってください。	あなたの 経歴[けいれき]をメールで 送[おく]ってください。	あなた の けいれき を めーる で おくって ください	
\\	あなたの
\\	をメールで 送[おく]ってください。			
\\	王	王[おう]	おう	
\\	王の墓を見学したよ。	王[おう]の 墓[はか]を 見学[けんがく]したよ。	おう の はか を けんがく した よ	
\\	の 墓[はか]を 見学[けんがく]したよ。			
\\	王様	王様[おうさま]	おうさま	
\\	その国の王様はとても賢い。	その 国[くに]の 王様[おうさま]はとても 賢[かしこ]い。	その くに の おうさま は とても かしこい	
\\	その 国[くに]の
\\	はとても 賢[かしこ]い。			
\\	結構	結構[けっこう]	けっこう	
\\	お腹がいっぱいなのでお代わりは結構です。	お 腹[なか]がいっぱいなのでお 代[か]わりは 結構[けっこう]です。	おなか が いっぱい な の で おかわり は けっこう です	
\\	お 腹[なか]がいっぱいなのでお 代[か]わりは
\\	です。			
\\	結構	結構[けっこう]	けっこう	
\\	彼女は結構めがねが似合うね。	彼女[かのじょ]は 結構[けっこう]めがねが 似合[にあ]うね。	かのじょ は けっこう めがね が にあう ね	
\\	彼女[かのじょ]は
\\	めがねが 似合[にあ]うね。			
\\	設置	設置[せっち]	せっち	
\\	お店に防犯カメラが設置されたわね。	お 店[みせ]に 防犯[ぼうはん]カメラが 設置[せっち]されたわね。	お みせ に ぼうはん かめら が せっち された わ ね	
\\	お 店[みせ]に 防犯[ぼうはん]カメラが
\\	されたわね。			
\\	距離	距離[きょり]	きょり	
\\	彼らは長い距離を歩き続けたんだ。	彼[かれ]らは 長[なが]い 距離[きょり]を 歩[ある]き 続[つづ]けたんだ。	かれら は ながい きょり を あるきつづけた ん だ	
\\	彼[かれ]らは 長[なが]い
\\	を 歩[ある]き 続[つづ]けたんだ。			
\\	辺り	辺[あた]り	あたり	
\\	辺りを見回したの。	辺[あた]りを 見回[みまわ]したの。	あたり を みまわした の	
\\	を 見回[みまわ]したの。			
\\	継続	継続[けいぞく]	けいぞく	
\\	ものごとは根気よく継続することが大切です。	ものごとは 根気[こんき]よく 継続[けいぞく]することが 大切[たいせつ]です。	もの ごと は こんき よく けいぞく する こと が たいせつ です	
\\	ものごとは 根気[こんき]よく
\\	することが 大切[たいせつ]です。			
\\	陸	陸[おか]	おか	
\\	私たちは舟を降りて、陸に上がったんだ。	私[わたし]たちは 舟[ふね]を 降[お]りて、 陸[おか]に 上[あ]がったんだ。	わたしたち は ふね を おりて おか に あがった ん だ	
\\	私[わたし]たちは 舟[ふね]を 降[お]りて、
\\	に 上[あ]がったんだ。			
\\	貨物	貨物[かもつ]	かもつ	
\\	このトランクは貨物で送ろう。	このトランクは 貨物[かもつ]で 送[おく]ろう。	この とらんく は かもつ で おくろう	
\\	このトランクは
\\	で 送[おく]ろう。			
\\	車輪	車輪[しゃりん]	しゃりん	
\\	このトラックの車輪は頑丈そうですね。	このトラックの 車輪[しゃりん]は 頑丈[がんじょう]そうですね。	この とらっく の しゃりん は がんじょう そう です ね	
\\	このトラックの
\\	は 頑丈[がんじょう]そうですね。			
\\	渋い	渋[しぶ]い	しぶい	
\\	この柿は渋いね。	この 柿[かき]は 渋[しぶ]いね。	この かき は しぶい ね	
\\	この 柿[かき]は
\\	ね。			
\\	渋滞	渋滞[じゅうたい]	じゅうたい	
\\	この国道はよく渋滞します。	この 国道[こくどう]はよく 渋滞[じゅうたい]します。	この こくどう は よく じゅうたい します	
\\	この 国道[こくどう]はよく
\\	します。			
\\	温帯	温帯[おんたい]	おんたい	
\\	日本は温帯にあります。	日本[にっぽん]は 温帯[おんたい]にあります。	にっぽん は おんたい に あります	
\\	日本[にっぽん]は
\\	にあります。			
\\	起こす	起[お]こす	おこす	
\\	彼は暴力事件を起こしたんだよ。	彼[かれ]は 暴力事件[ぼうりょく じけん]を 起[お]こしたんだよ。	かれ は ぼうりょく じけん を おこした ん だ よ	
\\	彼[かれ]は 暴力事件[ぼうりょく じけん]を
\\	んだよ。			
\\	起源	起源[きげん]	きげん	
\\	言葉の起源に大変興味があります。	言葉[ことば]の 起源[きげん]に 大変興味[たいへん きょうみ]があります。	ことば の きげん に たいへん きょうみ が あります	
\\	言葉[ことば]の
\\	に 大変興味[たいへん きょうみ]があります。			
\\	起き上がる	起[お]き 上[あ]がる	おきあがる	
\\	弟はようやくベッドから起き上がった。	弟[おとうと]はようやくベッドから 起[お]き 上[あ]がった。	おとうと は ようやく べっど から おきあがった。	
\\	弟[おとうと]はようやくベッドから
\\	静まる	静[しず]まる	しずまる	
\\	台風が去って風が静まったね。	台風[たいふう]が 去[さ]って 風[かぜ]が 静[しず]まったね。	たいふう が さって かぜ が しずまった ね	
\\	台風[たいふう]が 去[さ]って 風[かぜ]が
\\	ね。			
\\	片手	片手[かたて]	かたて	
\\	片手運転は危ないよ。	片手[かたて] 運転[うんてん]は 危[あぶ]ないよ。	かたて うんてん は あぶないよ	
\\	運転[うんてん]は 危[あぶ]ないよ。			
\\	片方	片方[かたほう]	かたほう	
\\	片方の目がかゆいです。	片方[かたほう]の 目[め]がかゆいです。	かたほう の め が かゆい です	
\\	の 目[め]がかゆいです。			
\\	片側	片側[かたがわ]	かたがわ	
\\	この道は片側通行です。	この 道[みち]は 片側[かたがわ] 通行[つうこう]です。	この みち は かたがわ つうこう です	
\\	この 道[みち]は
\\	通行[つうこう]です。			
\\	裏切る	裏切[うらぎ]る	うらぎる	
\\	彼は仲間を裏切ったの。	彼[かれ]は 仲間[なかま]を 裏切[うらぎ]ったの。	かれ は なかま を うらぎった の	
\\	彼[かれ]は 仲間[なかま]を
\\	の。			
\\	裏口	裏口[うらぐち]	うらぐち	
\\	裏口へお回りください。	裏口[うらぐち]へお 回[まわ]りください。	うらぐち へ おまわり ください	
\\	へお 回[まわ]りください。			
\\	裏返し	裏返[うらがえ]し	うらがえし	
\\	シャツを裏返しに着ているよ。	シャツを 裏返[うらがえ]しに 着[き]ているよ。	しゃつ を うらがえし に きて いる よ	
\\	シャツを
\\	に 着[き]ているよ。			
\\	裏門	裏門[うらもん]	うらもん	
\\	奴は裏門から出てきたぜ。	奴[やつ]は 裏門[うらもん]から 出[で]てきたぜ。	やつ は うらもん から でて きた ぜ	
\\	奴[やつ]は
\\	から 出[で]てきたぜ。			
\\	裏表	裏表[うらおもて]	うらおもて	
\\	シャツを裏表に着ていますよ。	シャツを 裏表[うらおもて]に 着[き]ていますよ。	しゃつ を うらおもて に きて います よ	
\\	シャツを
\\	に 着[き]ていますよ。			
\\	項目	項目[こうもく]	こうもく	
\\	論文の項目を分かりやすく整理したわ。	論文[ろんぶん]の 項目[こうもく]を 分[わ]かりやすく 整理[せいり]したわ。	ろんぶん の こうもく を わかりやすく せいり した わ	
\\	論文[ろんぶん]の
\\	を 分[わ]かりやすく 整理[せいり]したわ。			
\\	週刊	週刊[しゅうかん]	しゅうかん	
\\	この週刊誌を毎週買っているの。	この 週刊[しゅうかん] 誌[し]を 毎週買[まいしゅう か]っているの。	この しゅうかんし を まいしゅう かって いる の	
\\	この
\\	誌[し]を 毎週買[まいしゅう か]っているの。			
\\	詳細	詳細[しょうさい]	しょうさい	
\\	詳細はお気軽にお問い合わせ下さい。	詳細[しょうさい]はお 気軽[きがる]にお 問[と]い 合[あ]わせ 下[くだ]さい。	しょうさい は おきがる に おといあわせ ください	
\\	はお 気軽[きがる]にお 問[と]い 合[あ]わせ 下[くだ]さい。			
\\	積極的	積極的[せっきょくてき]	せっきょくてき	
\\	姉は何に対しても積極的です。	姉[あね]は 何[なに]に 対[たい]しても 積極的[せっきょくてき]です。	あね は なに に たいして も せっきょくてき です	
\\	姉[あね]は 何[なに]に 対[たい]しても
\\	です。			
\\	言い訳	言[い]い 訳[わけ]	いいわけ	
\\	あなたは言い訳が多すぎます。	あなたは 言[い]い 訳[わけ]が 多[おお]すぎます。	あなた は いいわけ が おおすぎます	
\\	あなたは
\\	が 多[おお]すぎます。			
\\	誤り	誤[あやま]り	あやまり	
\\	解答に誤りが3つ有りますよ。	解答[かいとう]に 誤[あやま]りが 3[みっ]つ 有[あ]りますよ。	かいとう に あやまり が みっつ あります よ	
\\	解答[かいとう]に
\\	が 3[みっ]つ 有[あ]りますよ。			
\\	誤る	誤[あやま]る	あやまる	
\\	彼は機械の操作を誤った。	彼[かれ]は 機械[きかい]の 操作[そうさ]を 誤[あやま]った。	かれ は きかい の そうさ を あやまった	
\\	彼[かれ]は 機械[きかい]の 操作[そうさ]を
\\	誤解	誤解[ごかい]	ごかい	
\\	私の気持ちを誤解しているんじゃない!	私[わたし]の 気持[きも]ちを 誤解[ごかい]しているんじゃない!	わたし の きもち を ごかい して いる ん じゃ ない	
\\	私[わたし]の 気持[きも]ちを
\\	しているんじゃない!			
\\	追い付く	追[お]い 付[つ]く	おいつく	
\\	駅でようやく彼に追い付きました。	駅[えき]でようやく 彼[かれ]に 追[お]い 付[つ]きました。	えき で ようやく かれ に おいつきました	
\\	駅[えき]でようやく 彼[かれ]に
\\	顔付き	顔付[かおつ]き	かおつき	
\\	彼は恐い顔付きで話したの。	彼[かれ]は 恐[こわ]い 顔付[かおつ]きで 話[はな]したの。	かれ は こわい かおつき で はなした の	
\\	彼[かれ]は 恐[こわ]い
\\	で 話[はな]したの。			
\\	言付ける	言付[ことづ]ける	ことづける	
\\	彼女への伝言を言付けたの。	彼女[かのじょ]への 伝言[でんごん]を 言付[ことづ]けたの。	かのじょ へ の でんごん を ことづけた の	
\\	彼女[かのじょ]への 伝言[でんごん]を
\\	の。			
\\	金属	金属[きんぞく]	きんぞく	
\\	アルミニウムは金属の一種です。	アルミニウムは 金属[きんぞく]の 一種[いっしゅ]です。	あるみにうむ は きんぞく の いっしゅ です	
\\	アルミニウムは
\\	の 一種[いっしゅ]です。			
\\	省略	省略[しょうりゃく]	しょうりゃく	
\\	地図内の小さい建物は省略してあります。	地図内[ちずない]の 小[ちい]さい 建物[たてもの]は 省略[しょうりゃく]してあります。	ちずない の ちいさい たてもの は しょうりゃく して あります	
\\	地図内[ちずない]の 小[ちい]さい 建物[たてもの]は
\\	してあります。			
\\	記念	記念[きねん]	きねん	
\\	卒業の記念にみんなで旅行したよ。	卒業[そつぎょう]の 記念[きねん]にみんなで 旅行[りょこう]したよ。	そつぎょう の きねん に みんな で りょこう した よ	
\\	卒業[そつぎょう]の
\\	にみんなで 旅行[りょこう]したよ。			
\\	順調	順調[じゅんちょう]	じゅんちょう	
\\	進み具合は全て順調です。	進[すす]み 具合[ぐあい]は 全[すべ]て 順調[じゅんちょう]です。	すすみ ぐあい は すべて じゅんちょう です	
\\	進[すす]み 具合[ぐあい]は 全[すべ]て
\\	です。			
\\	順	順[じゅん]	じゅん	
\\	あいうえお順に並べて下さい。	あいうえお 順[じゅん]に 並[なら]べて 下[くだ]さい。	あいうえお じゅん に ならべて ください	
\\	あいうえお
\\	に 並[なら]べて 下[くだ]さい。			
\\	順番	順番[じゅんばん]	じゅんばん	
\\	これを順番通りに並べ替えて下さい。	これを 順番[じゅんばん] 通[どお]りに 並[なら]べ 替[か]えて 下[くだ]さい。	これ を じゅんばん どおり に ならべ かえて ください	
\\	これを
\\	通[どお]りに 並[なら]べ 替[か]えて 下[くだ]さい。			
\\	順位	順位[じゅんい]	じゅんい	
\\	今年は去年よりも順位が上がりました。	今年[ことし]は 去年[きょねん]よりも 順位[じゅんい]が 上[あ]がりました。	ことし は きょねん より も じゅんい が あがりました	
\\	今年[ことし]は 去年[きょねん]よりも
\\	が 上[あ]がりました。			
\\	順々に	順々[じゅんじゅん]に	じゅんじゅんに	
\\	子供たちは順々にお菓子を受け取ったよ。	子供[こども]たちは 順々[じゅんじゅん]にお 菓子[かし]を 受[う]け 取[と]ったよ。	こどもたち は じゅんじゅんに おかし を うけとった よ	
\\	子供[こども]たちは
\\	お 菓子[かし]を 受[う]け 取[と]ったよ。			
\\	順序	順序[じゅんじょ]	じゅんじょ	
\\	正しい順序で操作してください。	正[ただ]しい 順序[じゅんじょ]で 操作[そうさ]してください。	ただしい じゅんじょ で そうさ して ください	
\\	正[ただ]しい
\\	で 操作[そうさ]してください。			
\\	逆	逆[ぎゃく]	ぎゃく	
\\	駅は逆方向です。	駅[えき]は 逆[ぎゃく] 方向[ほうこう]です。	えき は ぎゃくほうこう です	
\\	駅[えき]は
\\	方向[ほうこう]です。			
\\	逆らう	逆[さか]らう	さからう	
\\	親に逆らうのは良くない事です。	親[おや]に 逆[さか]らうのは 良[よ]くない 事[こと]です。	おや に さからう の は よくない こと です	
\\	親[おや]に
\\	のは 良[よ]くない 事[こと]です。			
\\	逆さ	逆[さか]さ	さかさ	
\\	絵を逆さにしてみてごらん。	絵[え]を 逆[さか]さにしてみてごらん。	え を さかさ に して みて ごらん	
\\	絵[え]を
\\	にしてみてごらん。			
\\	逆さま	逆[さか]さま	さかさま	
\\	それじゃあ上と下が逆さまだよ。	それじゃあ 上[うえ]と 下[した]が 逆[さか]さまだよ。	それじゃあ うえ と した が さかさま だ よ	
\\	それじゃあ 上[うえ]と 下[した]が
\\	だよ。			
\\	行列	行列[ぎょうれつ]	ぎょうれつ	
\\	店の前に長い行列ができていますよ。	店[みせ]の 前[まえ]に 長[なが]い 行列[ぎょうれつ]ができていますよ。	みせ の まえ に ながい ぎょうれつ が できて います よ	
\\	店[みせ]の 前[まえ]に 長[なが]い
\\	ができていますよ。			
\\	深刻	深刻[しんこく]	しんこく	
\\	これは深刻な問題です。	これは 深刻[しんこく]な 問題[もんだい]です。	これ は しんこく な もんだい で す	
\\	これは
\\	な 問題[もんだい]です。			
\\	締め切り	締[し]め 切[き]り	しめきり	
\\	申し込みの締め切りはいつですか。	申[もう]し 込[こ]みの 締[し]め 切[き]りはいつですか。	もうしこみ の しめきり は いつ です か	
\\	申[もう]し 込[こ]みの
\\	はいつですか。			
\\	締め切る	締[し]め 切[き]る	しめきる	
\\	応募受付は締め切りました。	応募受付[おうぼ うけつけ]は 締[し]め 切[き]りました。	おうぼ うけつけ は しめきりました	
\\	応募受付[おうぼ うけつけ]は
\\	締まる	締[し]まる	しまる	
\\	レバーを右に回すと締まります。	レバーを 右[みぎ]に 回[まわ]すと 締[し]まります。	ればー を みぎ に まわす と しまります	
\\	レバーを 右[みぎ]に 回[まわ]すと
\\	苦しむ	苦[くる]しむ	くるしむ	
\\	彼女はアレルギーに苦しんでいます。	彼女[かのじょ]はアレルギーに 苦[くる]しんでいます。	かのじょ は あれるぎー に くるしんで います	
\\	彼女[かのじょ]はアレルギーに
\\	苦情	苦情[くじょう]	くじょう	
\\	店に苦情の電話をかけました。	店[みせ]に 苦情[くじょう]の 電話[でんわ]をかけました。	みせ に くじょう の でんわ を かけました	
\\	店[みせ]に
\\	の 電話[でんわ]をかけました。			
\\	苦心	苦心[くしん]	くしん	
\\	苦心して絵を描き上げました。	苦心[くしん]して 絵[え]を 描[か]き 上[あ]げました。	くしん して え を かきあげました	
\\	して 絵[え]を 描[か]き 上[あ]げました。			
\\	苦しみ	苦[くる]しみ	くるしみ	
\\	誰も彼の苦しみを理解していなかったな。	誰[だれ]も 彼[かれ]の 苦[くる]しみを 理解[りかい]していなかったな。	だれ も かれ の くるしみ を りかい して いなかった な	
\\	誰[だれ]も 彼[かれ]の
\\	を 理解[りかい]していなかったな。			
\\	苦痛	苦痛[くつう]	くつう	
\\	私は人前で話すことが苦痛です。	私[わたし]は 人前[ひとまえ]で 話[はな]すことが 苦痛[くつう]です。	わたし は ひとまえ で はなす こと が くつう です	
\\	私[わたし]は 人前[ひとまえ]で 話[はな]すことが
\\	です。			
\\	苦しめる	苦[くる]しめる	くるしめる	
\\	強い日差しが選手たちを苦しめたの。	強[つよ]い 日差[ひざ]しが 選手[せんしゅ]たちを 苦[くる]しめたの。	つよい ひざし が せんしゅたち を くるしめた の	
\\	強[つよ]い 日差[ひざ]しが 選手[せんしゅ]たちを
\\	の。			
\\	苦労	苦労[くろう]	くろう	
\\	母は苦労して私たちを育てたの。	母[はは]は 苦労[くろう]して 私[わたし]たちを 育[そだ]てたの。	はは は くろう して わたしたち を そだてた の	
\\	母[はは]は
\\	して 私[わたし]たちを 育[そだ]てたの。			
\\	過労	過労[かろう]	かろう	
\\	彼は過労のために倒れたの。	彼[かれ]は 過労[かろう]のために 倒[たお]れたの。	かれ は かろう の ため に たおれた の	
\\	彼[かれ]は
\\	のために 倒[たお]れたの。			
\\	甘える	甘[あま]える	あまえる	
\\	彼女の親切に甘えました。	彼女[かのじょ]の 親切[しんせつ]に 甘[あま]えました。	かのじょ の しんせつ に あまえました	
\\	彼女[かのじょ]の 親切[しんせつ]に
\\	甘やかす	甘[あま]やかす	あまやかす	
\\	子供を甘やかしてはいけないの。	子供[こども]を 甘[あま]やかしてはいけないの。	こども を あまやかしては いけない の	
\\	子供[こども]を
\\	はいけないの。			
\\	辛い	辛[から]い	からい	
\\	部長は僕の仕事に辛い評価を出したんだ。	部長[ぶちょう]は 僕[ぼく]の 仕事[しごと]に 辛[から]い 評価[ひょうか]を 出[だ]したんだ。	ぶちょう は ぼく の しごと に からい ひょうか を だした ん だ	
\\	部長[ぶちょう]は 僕[ぼく]の 仕事[しごと]に
\\	評価[ひょうか]を 出[だ]したんだ。			
\\	砂	砂[すな]	すな	
\\	靴に砂が入ってしまった。	靴[くつ]に 砂[すな]が 入[はい]ってしまった。	くつ に すな が はいって しまった	
\\	靴[くつ]に
\\	が 入[はい]ってしまった。			
\\	砂漠	砂漠[さばく]	さばく	
\\	砂漠ではほとんど雨が降らないのよ。	砂漠[さばく]ではほとんど 雨[あめ]が 降[ふ]らないのよ。	さばく で は ほとんど あめ が ふらない の よ	
\\	ではほとんど 雨[あめ]が 降[ふ]らないのよ。			
\\	漁業	漁業[ぎょぎょう]	ぎょぎょう	
\\	彼らは漁業を営んでいるんだ。	彼[かれ]らは 漁業[ぎょぎょう]を 営[いとな]んでいるんだ。	かれら は ぎょぎょう を いとなん でいる ん だ 。	
\\	彼[かれ]らは
\\	を 営[いとな]んでいるんだ。			
\\	薄暗い	薄暗[うすぐら]い	うすぐらい	
\\	外はもう薄暗くなりましたよ。	外[そと]はもう 薄暗[うすぐら]くなりましたよ。	そと は もう うすぐらく なりました よ	
\\	外[そと]はもう
\\	よ。			
\\	薄める	薄[うす]める	うすめる	
\\	スープを少し薄めましょうか。	スープを 少[すこ]し 薄[うす]めましょうか。	すーぷ を すこし うすめましょう か	
\\	スープを 少[すこ]し
\\	か。			
\\	薄着	薄着[うすぎ]	うすぎ	
\\	子供たちは冬でも薄着で通します。	子供[こども]たちは 冬[ふゆ]でも 薄着[うすぎ]で 通[とお]します。	こどもたち は ふゆ で も うすぎ で とおします	
\\	子供[こども]たちは 冬[ふゆ]でも
\\	で 通[とお]します。			
\\	高気圧	高気圧[こうきあつ]	こうきあつ	
\\	この暑さは高気圧のせいです。	この 暑[あつ]さは 高気圧[こうきあつ]のせいです。	この あつさ は こうきあつ の せい です	
\\	この 暑[あつ]さは
\\	のせいです。			
\\	縮小	縮小[しゅくしょう]	しゅくしょう	
\\	これを縮小して印刷してください。	これを 縮小[しゅくしょう]して 印刷[いんさつ]してください。	これ を しゅくしょう して いんさつ して ください	
\\	これを
\\	して 印刷[いんさつ]してください。			
\\	越す	越[こ]す	こす	
\\	暑さは峠を越したね。	暑[あつ]さは 峠[とうげ]を 越[こ]したね。	あつさ は とうげ を こした ね	
\\	暑[あつ]さは 峠[とうげ]を
\\	ね。			
\\	追い抜く	追[お]い 抜[ぬ]く	おいぬく	
\\	リレーで彼は2人追い抜いたね。	リレーで 彼[かれ]は 2人[ふたり] 追[お]い 抜[ぬ]いたね。	りれー で かれ は ふたり おいぬいた ね	
\\	リレーで 彼[かれ]は 2人[ふたり]
\\	ね。			
\\	過程	過程[かてい]	かてい	
\\	プログラムの過程に問題があるの。	プログラムの 過程[かてい]に 問題[もんだい]があるの。	ぷろぐらむ の かてい に もんだい が ある の	
\\	プログラムの
\\	に 問題[もんだい]があるの。			
\\	課程	課程[かてい]	かてい	
\\	一年生の課程を修了したの。	一年生[いちねんせい]の 課程[かてい]を 修了[しゅうりょう]したの。	いちねんせい の かてい を しゅうりょう した の	
\\	一年生[いちねんせい]の
\\	を 修了[しゅうりょう]したの。			
\\	行程	行程[こうてい]	こうてい	
\\	全部で6時間の行程です。	全部[ぜんぶ]で 6時間[ろくじかん]の 行程[こうてい]です。	ぜんぶ で ろくじかん の こうてい です	
\\	全部[ぜんぶ]で 6時間[ろくじかん]の
\\	です。			
\\	透き通る	透[す]き 通[とお]る	すきとおる	
\\	湖の水は透き通っていたよ。	湖[みずうみ]の 水[みず]は 透[す]き 通[とお]っていたよ。	みずうみ の みず は すきとおって いた よ	
\\	湖[みずうみ]の 水[みず]は
\\	よ。			
\\	自身	自身[じしん]	じしん	
\\	自分自身を疑ってはいけません。	自分[じぶん] 自身[じしん]を 疑[うたが]ってはいけません。	じぶん じしん を うたがって は いけません	
\\	自分[じぶん]
\\	を 疑[うたが]ってはいけません。			
\\	身長	身長[しんちょう]	しんちょう	
\\	身長はどれくらいありますか。	身長[しんちょう]はどれくらいありますか。	しんちょう は どれ くらい あります か	
\\	はどれくらいありますか。			
\\	証拠	証拠[しょうこ]	しょうこ	
\\	証拠を見つけるまで逮捕はできません。	証拠[しょうこ]を 見[み]つけるまで 逮捕[たいほ]はできません。	しょうこ を みつける まで たいほ は できません	
\\	を 見[み]つけるまで 逮捕[たいほ]はできません。			
\\	照明	照明[しょうめい]	しょうめい	
\\	もっと照明を明るくしてください。	もっと 照明[しょうめい]を 明[あか]るくしてください。	もっと しょうめい を あかるく して ください	
\\	もっと
\\	を 明[あか]るくしてください。			
\\	産婦人科	産婦人科[さんふじんか]	さんふじんか	
\\	最近、産婦人科の不足が問題になっているね。	最近[さいきん]、 産婦人科[さんふじんか]の 不足[ふそく]が 問題[もんだい]になっているね。	さいきん さんふじんか の ふそく が もんだい に なって いる ね	
\\	最近[さいきん]、
\\	の 不足[ふそく]が 問題[もんだい]になっているね。			
\\	高齢	高齢[こうれい]	こうれい	
\\	彼は高齢を理由に社長を辞めたよ。	彼[かれ]は 高齢[こうれい]を 理由[りゆう]に 社長[しゃちょう]を 辞[や]めたよ。	かれ は こうれい を りゆう に しゃちょう を やめた よ	
\\	彼[かれ]は
\\	を 理由[りゆう]に 社長[しゃちょう]を 辞[や]めたよ。			
\\	祝う	祝[いわ]う	いわう	
\\	家族で彼の合格を祝ったよ。	家族[かぞく]で 彼[かれ]の 合格[ごうかく]を 祝[いわ]ったよ。	かぞく で かれ の ごうかく を いわった よ	
\\	家族[かぞく]で 彼[かれ]の 合格[ごうかく]を
\\	よ。			
\\	祝い	祝[いわ]い	いわい	
\\	結婚のお祝いに食器を頂きました。	結婚[けっこん]のお 祝[いわ]いに 食器[しょっき]を 頂[いただ]きました。	けっこん の おいわい に しょっき を いただきました	
\\	結婚[けっこん]のお
\\	に 食器[しょっき]を 頂[いただ]きました。			
\\	純粋	純粋[じゅんすい]	じゅんすい	
\\	子供の純粋な心を傷付けてはいけません。	子供[こども]の 純粋[じゅんすい]な 心[こころ]を 傷付[きずつ]けてはいけません。	こども の じゅんすい な こころ を きずつけて は いけません	
\\	子供[こども]の
\\	な 心[こころ]を 傷付[きずつ]けてはいけません。			
\\	笑顔	笑顔[えがお]	えがお	
\\	赤ちゃんの笑顔が可愛いい。	赤[あか]ちゃんの 笑顔[えがお]が 可愛[かわ]いい。	あかちゃん の えがお が かわいい	
\\	赤[あか]ちゃんの
\\	が 可愛[かわ]いい。			
\\	看護	看護[かんご]	かんご	
\\	この病院は24時間看護です。	この 病院[びょういん]は 24時間[にじゅうよじかん] 看護[かんご]です。	この びょういん は にじゅうよじかん かんご です	
\\	この 病院[びょういん]は 24時間[にじゅうよじかん]
\\	です。			
\\	看病	看病[かんびょう]	かんびょう	
\\	彼女は一晩中彼を看病したの。	彼女[かのじょ]は 一晩中彼[ひとばんじゅう かれ]を 看病[かんびょう]したの。	かのじょ は ひとばんじゅう かれ を かんびょう した の	
\\	彼女[かのじょ]は 一晩中彼[ひとばんじゅう かれ]を
\\	したの。			
\\	肯定	肯定[こうてい]	こうてい	
\\	僕は彼のしたことは肯定できないな。	僕[ぼく]は 彼[かれ]のしたことは 肯定[こうてい]できないな。	ぼく は かれ の した こと は こうてい できない な	
\\	僕[ぼく]は 彼[かれ]のしたことは
\\	できないな。			
\\	記述	記述[きじゅつ]	きじゅつ	
\\	日本の古い料理法についての記述を読んだんだ。	日本[にほん]の 古[ふる]い 料理法[りょうりほう]についての 記述[きじゅつ]を 読[よ]んだんだ。	にほん の ふるい りょうりほう に ついて の きじゅつ を よんだ ん だ	
\\	日本[にほん]の 古[ふる]い 料理法[りょうりほう]についての
\\	を 読[よ]んだんだ。			
\\	裁判	裁判[さいばん]	さいばん	
\\	裁判の様子はテレビで中継されたわよ。	裁判[さいばん]の 様子[ようす]はテレビで 中継[ちゅうけい]されたわよ。	さいばん の ようす は てれび で ちゅうけい された わ よ	
\\	の 様子[ようす]はテレビで 中継[ちゅうけい]されたわよ。			
\\	裁判所	裁判所[さいばんしょ]	さいばんしょ	
\\	彼らは裁判所の前で知らせを待っています。	彼[かれ]らは 裁判所[さいばんしょ]の 前[まえ]で 知[し]らせを 待[ま]っています。	かれら は さいばんしょ の まえ で しらせ を まって います	
\\	彼[かれ]らは
\\	の 前[まえ]で 知[し]らせを 待[ま]っています。			
\\	訴え	訴[うった]え	うったえ	
\\	彼女の訴えは認められたわ。	彼女[かのじょ]の 訴[うった]えは 認[みと]められたわ。	かのじょ の うったえ は みとめられた わ	
\\	彼女[かのじょ]の
\\	は 認[みと]められたわ。			
\\	疑い	疑[うたが]い	うたがい	
\\	彼は盗みの疑いをかけられたの。	彼[かれ]は 盗[ぬす]みの 疑[うたが]いをかけられたの。	かれ は ぬすみ の うたがい を かけられた の	
\\	彼[かれ]は 盗[ぬす]みの
\\	をかけられたの。			
\\	疑問	疑問[ぎもん]	ぎもん	
\\	彼の言葉が本当かは疑問です。	彼[かれ]の 言葉[ことば]が 本当[ほんとう]かは 疑問[ぎもん]です。	かれ の ことば が ほんとう か は ぎもん です	
\\	彼[かれ]の 言葉[ことば]が 本当[ほんとう]かは
\\	です。			
\\	疑う	疑[うたが]う	うたがう	
\\	なぜあなたは私を疑うのですか。	なぜあなたは 私[わたし]を 疑[うたが]うのですか。	なぜ あなた は わたし を うたがう の です か	
\\	なぜあなたは 私[わたし]を
\\	のですか。			
\\	疑わしい	疑[うたが]わしい	うたがわしい	
\\	疑わしい場所は全部調べよう。	疑[うたが]わしい 場所[ばしょ]は 全部調[ぜんぶ しら]べよう。	うたがわしい ばしょ は ぜんぶ しらべよう	
\\	場所[ばしょ]は 全部調[ぜんぶ しら]べよう。			
\\	著しい	著[いちじる]しい	いちじるしい	
\\	彼の成長は著しいです。	彼[かれ]の 成長[せいちょう]は 著[いちじる]しいです。	かれ の せいちょう は いちじるしい です	
\\	彼[かれ]の 成長[せいちょう]は
\\	です。			
\\	著す	著[あらわ]す	あらわす	
\\	この本では自然の大切さがよく著されているよ。	この 本[ほん]では 自然[しぜん]の 大切[たいせつ]さがよく 著[あらわ]されているよ。	この ほん で は しぜん の たいせつさ が よく あらわされて いる よ	
\\	この 本[ほん]では 自然[しぜん]の 大切[たいせつ]さがよく
\\	よ。			
\\	行為	行為[こうい]	こうい	
\\	彼の行為はみんなの誤解を招いたわ。	彼[かれ]の 行為[こうい]はみんなの 誤解[ごかい]を 招[まね]いたわ。	かれ の こうい は みんな の ごかい を まねいた わ	
\\	彼[かれ]の
\\	はみんなの 誤解[ごかい]を 招[まね]いたわ。			
\\	賞	賞[しょう]	しょう	
\\	このデザインは数々の賞を受けています。	このデザインは 数々[かずかず]の 賞[しょう]を 受[う]けています。	この でざいん は かずかず の しょう を うけて います	
\\	このデザインは 数々[かずかず]の
\\	を 受[う]けています。			
\\	賞品	賞品[しょうひん]	しょうひん	
\\	パーティーのビンゴの賞品は何がいいだろう。	パーティーのビンゴの 賞品[しょうひん]は 何[なに]がいいだろう。	ぱーてぃー の びんご の しょうひん は なに が いいだろう	
\\	パーティーのビンゴの
\\	は 何[なに]がいいだろう。			
\\	観賞	観賞[かんしょう]	かんしょう	
\\	この鉢植えは観賞用です。	この 鉢植[はちう]えは 観賞[かんしょう] 用[よう]です。	この はちうえ は かんしょうよう です	
\\	この 鉢植[はちう]えは
\\	用[よう]です。			
\\	財政	財政[ざいせい]	ざいせい	
\\	国の財政はとても苦しい状態だな。	国[くに]の 財政[ざいせい]はとても 苦[くる]しい 状態[じょうたい]だな。	くに の ざいせい は とても くるしい じょうたい だ な	
\\	国[くに]の
\\	はとても 苦[くる]しい 状態[じょうたい]だな。			
\\	財産	財産[ざいさん]	ざいさん	
\\	彼は株で財産の半分を失ったの。	彼[かれ]は 株[かぶ]で 財産[ざいさん]の 半分[はんぶん]を 失[うしな]ったの。	かれ は かぶ で ざいさん の はんぶん を うしなった の	
\\	彼[かれ]は 株[かぶ]で
\\	の 半分[はんぶん]を 失[うしな]ったの。			
\\	金融	金融[きんゆう]	きんゆう	
\\	彼は金融関係の会社で働いています。	彼[かれ]は 金融[きんゆう] 関係[かんけい]の 会社[かいしゃ]で 働[はたら]いています。	かれ は きんゆう かんけい の かいしゃ で はたらいて います	
\\	彼[かれ]は
\\	関係[かんけい]の 会社[かいしゃ]で 働[はたら]いています。			
\\	間隔	間隔[かんかく]	かんかく	
\\	電車は5分間隔で来ますよ。	電車[でんしゃ]は 5分[ごふん] 間隔[かんかく]で 来[き]ますよ。	でんしゃ は ごふん かんかく で きます よ	
\\	電車[でんしゃ]は 5分[ごふん]
\\	で 来[き]ますよ。			
\\	終了	終了[しゅうりょう]	しゅうりょう	
\\	コンサートは夜7時5分に終了しました。	コンサートは 夜7時5分[よる しち じ ご ふん]に 終了[しゅうりょう]しました。	こんさーと は よる しち じ ご ふん に しゅうりょう しました	
\\	コンサートは 夜7時5分[よる しち じ ご ふん]に
\\	しました。			
\\	納める	納[おさ]める	おさめる	
\\	自動車税を納めたよ。	自動車税[じどうしゃぜい]を 納[おさ]めたよ。	じどうしゃぜい を おさめた よ	
\\	自動車税[じどうしゃぜい]を
\\	よ。			
\\	説得	説得[せっとく]	せっとく	
\\	なんとか親を説得してみるよ。	なんとか 親[おや]を 説得[せっとく]してみるよ。	なんとか おや を せっとく して みる よ	
\\	なんとか 親[おや]を
\\	してみるよ。			
\\	酸素	酸素[さんそ]	さんそ	
\\	私たちは酸素無しでは生きていけない。	私[わたし]たちは 酸素[さんそ] 無[な]しでは 生[い]きていけない。	わたしたち は さんそ なし で は いきて いけない	
\\	私[わたし]たちは
\\	無[な]しでは 生[い]きていけない。			
\\	素直	素直[すなお]	すなお	
\\	彼女はとても素直で可愛いですね。	彼女[かのじょ]はとても 素直[すなお]で 可愛[かわい]いですね。	かのじょ は とても すなお で かわいい です ね	
\\	彼女[かのじょ]はとても
\\	で 可愛[かわい]いですね。			
\\	素早い	素早[すばや]い	すばやい	
\\	彼は素早くあたりを見回したの。	彼[かれ]は 素早[すばや]くあたりを 見回[みまわ]したの。	かれ は すばやく あたり を みまわした の	
\\	彼[かれ]は
\\	あたりを 見回[みまわ]したの。			
\\	石炭	石炭[せきたん]	せきたん	
\\	小屋に石炭の山があります。	小屋[こや]に 石炭[せきたん]の 山[やま]があります。	こや に せきたん の やま が あります	
\\	小屋[こや]に
\\	の 山[やま]があります。			
\\	解散	解散[かいさん]	かいさん	
\\	来年、衆議院が解散されるだろう。	来年[らいねん]、 衆議院[しゅうぎいん]が 解散[かいさん]されるだろう。	らいねん しゅうぎいん が かいさん される だろう	
\\	来年[らいねん]、 衆議院[しゅうぎいん]が
\\	されるだろう。			
\\	看板	看板[かんばん]	かんばん	
\\	店の看板を塗り替えました。	店[みせ]の 看板[かんばん]を 塗[ぬ]り 替[か]えました。	みせ の かんばん を ぬりかえました	
\\	店[みせ]の
\\	を 塗[ぬ]り 替[か]えました。			
\\	草花	草花[くさばな]	くさばな	
\\	草花を大切にしましょう。	草花[くさばな]を 大切[たいせつ]にしましょう。	くさばな を たいせつ に しましょう	
\\	を 大切[たいせつ]にしましょう。			
\\	草木	草木[くさき]	くさき	
\\	この庭は草木が枯れているね。	この 庭[にわ]は 草木[くさき]が 枯[か]れているね。	この にわ は くさき が かれて いる ね	
\\	この 庭[にわ]は
\\	が 枯[か]れているね。			
\\	言葉遣い	言葉遣[ことばづか]い	ことばづかい	
\\	先生にそんな言葉遣いをしてはいけません。	先生[せんせい]にそんな 言葉遣[ことばづか]いをしてはいけません。	せんせい に そんな ことばづかい を して は いけません	
\\	先生[せんせい]にそんな
\\	をしてはいけません。			
\\	落ち葉	落[お]ち 葉[ば]	おちば	
\\	落ち葉の季節になりましたね。	落[お]ち 葉[ば]の 季節[きせつ]になりましたね。	おちば の きせつ に なりました ね	
\\	の 季節[きせつ]になりましたね。			
\\	高級	高級[こうきゅう]	こうきゅう	
\\	私たちは高級ホテルに泊まったの。	私[わたし]たちは 高級[こうきゅう]ホテルに 泊[と]まったの。	わたしたち は こうきゅう ほてる に とまった の	
\\	私[わたし]たちは
\\	ホテルに 泊[と]まったの。			
\\	級	級[きゅう]	きゅう	
\\	彼は書道3級です。	彼[かれ]は 書道3[しょどう さん] 級[きゅう]です。	かれ は しょどう さんきゅう です	
\\	彼[かれ]は 書道3[しょどう さん]
\\	です。			
\\	血管	血管[けっかん]	けっかん	
\\	年をとると血管が硬くなります。	年[とし]をとると 血管[けっかん]が 硬[かた]くなります。	とし を とる と けっかん が かたく なります	
\\	年[とし]をとると
\\	が 硬[かた]くなります。			
\\	液体	液体[えきたい]	えきたい	
\\	洗濯に液体の洗剤を使っています。	洗濯[せんたく]に 液体[えきたい]の 洗剤[せんざい]を 使[つか]っています。	せんたく に えきたい の せんざい を つかって います	
\\	洗濯[せんたく]に
\\	の 洗剤[せんざい]を 使[つか]っています。			
\\	血液	血液[けつえき]	けつえき	
\\	心臓は全身に血液を送り出している。	心臓[しんぞう]は 全身[ぜんしん]に 血液[けつえき]を 送[おく]り 出[だ]している。	しんぞう は ぜんしん に けつえき を おくりだして いる	
\\	心臓[しんぞう]は 全身[ぜんしん]に
\\	を 送[おく]り 出[だ]している。			
\\	液	液[えき]	えき	
\\	容器から液がこぼれていますよ。	容器[ようき]から 液[えき]がこぼれていますよ。	ようき から えき が こぼれて います よ	
\\	容器[ようき]から
\\	がこぼれていますよ。			
\\	背中	背中[せなか]	せなか	
\\	背中がかゆいです。	背中[せなか]がかゆいです。	せなか が かゆい です	
\\	がかゆいです。			
\\	背広	背広[せびろ]	せびろ	
\\	背広をクリーニングに出しました。	背広[せびろ]をクリーニングに 出[だ]しました。	せびろ を くりーにんぐ に だしました	
\\	をクリーニングに 出[だ]しました。			
\\	骨折	骨折[こっせつ]	こっせつ	
\\	彼女はスキーで足を骨折したの。	彼女[かのじょ]はスキーで 足[あし]を 骨折[こっせつ]したの。	かのじょ は すきー で あし を こっせつ した の	
\\	彼女[かのじょ]はスキーで 足[あし]を
\\	したの。			
\\	診断	診断[しんだん]	しんだん	
\\	医師は異常なしと診断したんだ。	医師[いし]は 異常[いじょう]なしと 診断[しんだん]したんだ。	いし は いじょう なし と しんだん した ん だ	
\\	医師[いし]は 異常[いじょう]なしと
\\	したんだ。			
\\	診察	診察[しんさつ]	しんさつ	
\\	今日、病院で診察してもらったの。	今日[きょう]、 病院[びょういん]で 診察[しんさつ]してもらったの。	きょう びょういん で しんさつ して もらった の	
\\	今日[きょう]、 病院[びょういん]で
\\	してもらったの。			
\\	消毒	消毒[しょうどく]	しょうどく	
\\	足の傷を消毒したわ。	足[あし]の 傷[きず]を 消毒[しょうどく]したわ。	あし の きず を しょうどく した わ	
\\	足[あし]の 傷[きず]を
\\	したわ。			
\\	症状	症状[しょうじょう]	しょうじょう	
\\	医者に症状を説明したよ。	医者[いしゃ]に 症状[しょうじょう]を 説明[せつめい]したよ。	いしゃ に しょうじょう を せつめい した よ	
\\	医者[いしゃ]に
\\	を 説明[せつめい]したよ。			
\\	清書	清書[せいしょ]	せいしょ	
\\	この手紙を清書してください。	この 手紙[てがみ]を 清書[せいしょ]してください。	この てがみ を せいしょ して ください	
\\	この 手紙[てがみ]を
\\	してください。			
\\	清潔	清潔[せいけつ]	せいけつ	
\\	レストランは清潔が第一です。	レストランは 清潔[せいけつ]が 第一[だいいち]です。	れすとらん は せいけつ が だいいち です	
\\	レストランは
\\	が 第一[だいいち]です。			
\\	監督	監督[かんとく]	かんとく	
\\	その監督はアカデミー賞を受賞したよな。	その 監督[かんとく]はアカデミー 賞[しょう]を 受賞[じゅしょう]したよな。	その かんとく は あかでみーしょう を じゅしょう した よ な	
\\	その
\\	はアカデミー 賞[しょう]を 受賞[じゅしょう]したよな。			
\\	編む	編[あ]む	あむ	
\\	妹はマフラーを編みました。	妹[いもうと]はマフラーを 編[あ]みました。	いもうと は まふらー を あみました	
\\	妹[いもうと]はマフラーを
\\	解放	解放[かいほう]	かいほう	
\\	人質が解放されてよかった。	人質[ひとじち]が 解放[かいほう]されてよかった。	ひとじち が かいほう されて よかった	
\\	人質[ひとじち]が
\\	されてよかった。			
\\	開放	開放[かいほう]	かいほう	
\\	その国は市場の開放を求められているの。	その 国[くに]は 市場[しじょう]の 開放[かいほう]を 求[もと]められているの。	その くに は しじょう の かいほう を もとめられて いる の	
\\	その 国[くに]は 市場[しじょう]の
\\	を 求[もと]められているの。			
\\	重視	重視[じゅうし]	じゅうし	
\\	あの企業では学歴が重視されるよ。	あの 企業[きぎょう]では 学歴[がくれき]が 重視[じゅうし]されるよ。	あの きぎょう で は がくれき が じゅうし される よ	
\\	あの 企業[きぎょう]では 学歴[がくれき]が
\\	されるよ。			
\\	視点	視点[してん]	してん	
\\	視点を変えて見てみましょう。	視点[してん]を 変[か]えて 見[み]てみましょう。	してん を かえて みて みましょう	
\\	を 変[か]えて 見[み]てみましょう。			
\\	近視	近視[きんし]	きんし	
\\	彼は軽い近視です。	彼[かれ]は 軽[かる]い 近視[きんし]です。	かれ は かるい きんし です	
\\	彼[かれ]は 軽[かる]い
\\	です。			
\\	衣類	衣類[いるい]	いるい	
\\	衣類の整理をしたよ。	衣類[いるい]の 整理[せいり]をしたよ。	いるい の せいり を した よ	
\\	の 整理[せいり]をしたよ。			
\\	衣服	衣服[いふく]	いふく	
\\	彼の会社は衣服を扱っているよ。	彼[かれ]の 会社[かいしゃ]は 衣服[いふく]を 扱[あつか]っているよ。	かれ の かいしゃ は いふく を あつかって いる よ	
\\	彼[かれ]の 会社[かいしゃ]は
\\	を 扱[あつか]っているよ。			
\\	衣料	衣料[いりょう]	いりょう	
\\	その会社は衣料を扱っているんだ。	その 会社[かいしゃ]は 衣料[いりょう]を 扱[あつか]っているんだ。	その かいしゃ は いりょう を あつかって いる ん だ	
\\	その 会社[かいしゃ]は
\\	を 扱[あつか]っているんだ。			
\\	衣食住	衣食住[いしょくじゅう]	いしょくじゅう	
\\	土地によって衣食住は変化する。	土地[とち]によって 衣食住[いしょくじゅう]は 変化[へんか]する。	とち によって いしょくじゅう は へんか する	
\\	土地[とち]によって
\\	は 変化[へんか]する。			
\\	演説	演説[えんぜつ]	えんぜつ	
\\	彼は地方で演説したのよ。	彼[かれ]は 地方[ちほう]で 演説[えんぜつ]したのよ。	かれ は ちほう で えんぜつ した の よ	
\\	彼[かれ]は 地方[ちほう]で
\\	したのよ。			
\\	演じる	演[えん]じる	えんじる	
\\	彼はよく刑事の役を演じるね。	彼[かれ]はよく 刑事[けいじ]の 役[やく]を 演[えん]じるね。	かれ は よく けいじ の やく を えんじる ね	
\\	彼[かれ]はよく 刑事[けいじ]の 役[やく]を
\\	ね。			
\\	演劇	演劇[えんげき]	えんげき	
\\	彼女は演劇を学んでいます。	彼女[かのじょ]は 演劇[えんげき]を 学[まな]んでいます。	かのじょ は えんげき を まなんで います	
\\	彼女[かのじょ]は
\\	を 学[まな]んでいます。			
\\	集団	集団[しゅうだん]	しゅうだん	
\\	テロ集団が警察に捕まりました。	テロ 集団[しゅうだん]が 警察[けいさつ]に 捕[つか]まりました。	てろ しゅうだん が けいさつ に つかまりました	
\\	テロ
\\	が 警察[けいさつ]に 捕[つか]まりました。			
\\	辞典	辞典[じてん]	じてん	
\\	知らない言葉を辞典で調べました。	知[し]らない 言葉[ことば]を 辞典[じてん]で 調[しら]べました。	しらない ことば を じてん で しらべました	
\\	知[し]らない 言葉[ことば]を
\\	で 調[しら]べました。			
\\	象徴	象徴[しょうちょう]	しょうちょう	
\\	天皇は日本国の象徴です。	天皇[てんのう]は 日本国[にほんこく]の 象徴[しょうちょう]です。	てんのう は にほんこく の しょうちょう です	
\\	天皇[てんのう]は 日本国[にほんこく]の
\\	です。			
\\	許可	許可[きょか]	きょか	
\\	先生に許可をもらって早退しました。	先生[せんせい]に 許可[きょか]をもらって 早退[そうたい]しました。	せんせい に きょか を もらって そうたい しました	
\\	先生[せんせい]に
\\	をもらって 早退[そうたい]しました。			
\\	訓練	訓練[くんれん]	くんれん	
\\	学校で避難訓練がありました。	学校[がっこう]で 避難[ひなん] 訓練[くんれん]がありました。	がっこう で ひなん くんれん が ありました	
\\	学校[がっこう]で 避難[ひなん]
\\	がありました。			
\\	訓読み	訓読[くんよ]み	くんよみ	
\\	漢字には音読みと訓読みがあります。	漢字[かんじ]には 音読[おんよ]みと 訓読[くんよ]みがあります。	かんじ に は おんよみ と くんよみ が あります	
\\	漢字[かんじ]には 音読[おんよ]みと
\\	があります。			
\\	訓	訓[くん]	くん	
\\	漢字の読み方には、音と訓の2通りがあるの。	漢字[かんじ]の 読[よ]み 方[かた]には、 音[おん]と 訓[くん]の 2通[ふたとお]りがあるの。	かんじ の よみかた に は おん と くん の ふたとおり が ある の	
\\	漢字[かんじ]の 読[よ]み 方[かた]には、 音[おん]と
\\	の 2通[ふたとお]りがあるの。			
\\	購入	購入[こうにゅう]	こうにゅう	
\\	入学式の後、教科書を購入してください。	入学式[にゅうがくしき]の 後[あと]、 教科書[きょうかしょ]を 購入[こうにゅう]してください。	にゅうがくしき の あと きょうかしょ を こうにゅう して ください	
\\	入学式[にゅうがくしき]の 後[あと]、 教科書[きょうかしょ]を
\\	してください。			
\\	講演	講演[こうえん]	こうえん	
\\	彼の講演は評判がいい。	彼[かれ]の 講演[こうえん]は 評判[ひょうばん]がいい。	かれ の こうえん は ひょうばん が いい	
\\	彼[かれ]の
\\	は 評判[ひょうばん]がいい。			
\\	講義	講義[こうぎ]	こうぎ	
\\	彼の講義はとても分かりやすいですね。	彼[かれ]の 講義[こうぎ]はとても 分[わ]かりやすいですね。	かれ の こうぎ は とても わかり やすい です ね	
\\	彼[かれ]の
\\	はとても 分[わ]かりやすいですね。			
\\	講師	講師[こうし]	こうし	
\\	専門家を講師に招いた。	専門家[せんもんか]を 講師[こうし]に 招[まね]いた。	せんもんか を こうし に まねいた	
\\	専門家[せんもんか]を
\\	に 招[まね]いた。			
\\	騒ぎ	騒[さわ]ぎ	さわぎ	
\\	一体何の騒ぎですか。	一体何[いったい なん]の 騒[さわ]ぎですか。	いったい なん の さわぎ です か	
\\	一体何[いったい なん]の
\\	ですか。			
\\	騒がしい	騒[さわ]がしい	さわがしい	
\\	何だか表が騒がしい。	何[なん]だか 表[おもて]が 騒[さわ]がしい。	なんだか おもて が さわがしい	
\\	何[なん]だか 表[おもて]が
\\	飼う	飼[か]う	かう	
\\	ペットを飼ったことはありますか。	ペットを 飼[か]ったことはありますか。	ぺっと を かった こと は あります か	
\\	ペットを
\\	ことはありますか。			
\\	興味	興味[きょうみ]	きょうみ	
\\	私は歴史に興味があります。	私[わたし]は 歴史[れきし]に 興味[きょうみ]があります。	わたし は れきし に きょうみ が あります	
\\	私[わたし]は 歴史[れきし]に
\\	があります。			
\\	興奮	興奮[こうふん]	こうふん	
\\	彼女の優勝に感激し興奮しました。	彼女[かのじょ]の 優勝[ゆうしょう]に 感激[かんげき]し 興奮[こうふん]しました。	かのじょ の ゆうしょう に かんげき し こうふん しました	
\\	彼女[かのじょ]の 優勝[ゆうしょう]に 感激[かんげき]し
\\	しました。			
\\	驚き	驚[おどろ]き	おどろき	
\\	彼女は驚きを隠せませんでしたよ。	彼女[かのじょ]は 驚[おどろ]きを 隠[かく]せませんでしたよ。	かのじょ は おどろき を かくせませんでした よ	
\\	彼女[かのじょ]は
\\	を 隠[かく]せませんでしたよ。			
\\	驚かす	驚[おどろ]かす	おどろかす	
\\	あなたを驚かす話があります。	あなたを 驚[おどろ]かす 話[はなし]があります。	あなた を おどろかす はなし が あります	
\\	あなたを
\\	話[はなし]があります。			
\\	至る	至[いた]る	いたる	
\\	ようやく結論に至ったようね	ようやく 結論[けつろん]に 至[いた]ったようね	ようやく けつろん に いたった よう ね	
\\	ようやく 結論[けつろん]に
\\	ようね			
\\	至る所	至[いた]る 所[ところ]	いたるところ	
\\	コンビニは至る所にあります。	コンビニは 至[いた]る 所[ところ]にあります。	こんびに は いたるところ に あります	
\\	コンビニは
\\	にあります。			
\\	至急	至急[しきゅう]	しきゅう	
\\	会社から「至急」との連絡があったんだ。	会社[かいしゃ]から
\\	至急[しきゅう]」との 連絡[れんらく]があったんだ。	かいしゃ から しきゅう と の れんらく が あった ん だ	
\\	会社[かいしゃ]から
\\	との 連絡[れんらく]があったんだ。			
\\	致す	致[いた]す	いたす	
\\	私からご連絡致します。	私[わたし]からご 連絡[れんらく] 致[いた]します。	わたし から ごれんらく いたします	
\\	私[わたし]からご 連絡[れんらく]
\\	狭まる	狭[せば]まる	せばまる	
\\	ここから道の幅が狭まっています。	ここから 道[みち]の 幅[はば]が 狭[せば]まっています。	ここ から みち の はば が せばまって います	
\\	ここから 道[みち]の 幅[はば]が
\\	狭める	狭[せば]める	せばめる	
\\	もっと範囲を狭めて探しましょう	もっと 範囲[はんい]を 狭[せば]めて 探[さが]しましょう	もっと はんい を せばめて さがしましょう	
\\	もっと 範囲[はんい]を
\\	探[さが]しましょう			
\\	硬さ	硬[かた]さ	かたさ	
\\	理科の時間に石の硬さを比べたよ。	理科[りか]の 時間[じかん]に 石[いし]の 硬[かた]さを 比[くら]べたよ。	りか の じかん に いし の かたさ を くらべた よ	
\\	理科[りか]の 時間[じかん]に 石[いし]の
\\	を 比[くら]べたよ。			
\\	緊急	緊急[きんきゅう]	きんきゅう	
\\	緊急事態が発生した。	緊急[きんきゅう] 事態[じたい]が 発生[はっせい]した。	きんきゅう じたい が はっせい した	
\\	事態[じたい]が 発生[はっせい]した。			
\\	緊張	緊張[きんちょう]	きんちょう	
\\	彼は緊張していたな。	彼[かれ]は 緊張[きんちょう]していたな。	かれ は きんちょう して いた な	
\\	彼[かれ]は
\\	していたな。			
\\	衝突	衝突[しょうとつ]	しょうとつ	
\\	そこの角で車同士が衝突したのよ。	そこの 角[かど]で 車同士[くるま どうし]が 衝突[しょうとつ]したのよ。	そこ の かど で くるま どうし が しょうとつ した の よ	
\\	そこの 角[かど]で 車同士[くるま どうし]が
\\	したのよ。			
\\	煙突	煙突[えんとつ]	えんとつ	
\\	工場の煙突から煙が上がっている。	工場[こうじょう]の 煙突[えんとつ]から 煙[けむり]が 上[あ]がっている。	こうじょう の えんとつ から けむり が あがって いる	
\\	工場[こうじょう]の
\\	から 煙[けむり]が 上[あ]がっている。			
\\	避ける	避[さ]ける	さける	
\\	彼女は昨日から私のことを避けているようなんだ。	彼女[かのじょ]は 昨日[きのう]から 私[わたし]のことを 避[さ]けているようなんだ。	かのじょ は きのう から わたし の こと を さけて いる よう なん だ	
\\	彼女[かのじょ]は 昨日[きのう]から 私[わたし]のことを
\\	ようなんだ。			
\\	軍事	軍事[ぐんじ]	ぐんじ	
\\	彼は軍事に関わる仕事を続けてきたんだ。	彼[かれ]は 軍事[ぐんじ]に 関[かか]わる 仕事[しごと]を 続[つづ]けてきたんだ。	かれ は ぐんじ に かかわる しごと を つづけて きた ん だ	
\\	彼[かれ]は
\\	に 関[かか]わる 仕事[しごと]を 続[つづ]けてきたんだ。			
\\	軍	軍[ぐん]	ぐん	
\\	怪我人は軍の病院に運ばれたよ。	怪我人[けがにん]は 軍[ぐん]の 病院[びょういん]に 運[はこ]ばれたよ。	けがにん は ぐん の びょういん に はこばれた よ	
\\	怪我人[けがにん]は
\\	の 病院[びょういん]に 運[はこ]ばれたよ。			
\\	自衛隊	自衛隊[じえいたい]	じえいたい	
\\	自衛隊がイラクに派遣されたわ。	自衛隊[じえいたい]がイラクに 派遣[はけん]されたわ。	じえいたい が いらく に はけん された わ	
\\	がイラクに 派遣[はけん]されたわ。			
\\	軍隊	軍隊[ぐんたい]	ぐんたい	
\\	彼は軍隊に入ったよ。	彼[かれ]は 軍隊[ぐんたい]に 入[はい]ったよ。	かれ は ぐんたい に はいった よ	
\\	彼[かれ]は
\\	に 入[はい]ったよ。			
\\	結核	結核[けっかく]	けっかく	
\\	昔は結核でたくさんの人が亡くなった。	昔[むかし]は 結核[けっかく]でたくさんの 人[ひと]が 亡[な]くなった。	むかし は けっかく で たくさん の ひと が なくなった	
\\	昔[むかし]は
\\	でたくさんの 人[ひと]が 亡[な]くなった。			
\\	襲う	襲[おそ]う	おそう	
\\	山で登山客が熊に襲われたよ。	山[やま]で 登山客[とざん きゃく]が 熊[くま]に 襲[おそ]われたよ。	やま で とざん きゃく が くま に おそわれた よ	
\\	山[やま]で 登山客[とざん きゃく]が 熊[くま]に
\\	よ。			
\\	絶対	絶対[ぜったい]	ぜったい	
\\	そんなことは絶対できないよ。	そんなことは 絶対[ぜったい]できないよ。	そんな こと は ぜったい できない よ	
\\	そんなことは
\\	できないよ。			
\\	消防	消防[しょうぼう]	しょうぼう	
\\	消防士は勇敢でなければならない。	消防[しょうぼう] 士[し]は 勇敢[ゆうかん]でなければならない。	しょうぼうし は ゆうかん で なけれ ば ならない	
\\	士[し]は 勇敢[ゆうかん]でなければならない。			
\\	素敵	素敵[すてき]	すてき	
\\	素敵なプレゼントをありがとう。	素敵[すてき]なプレゼントをありがとう。	すてき な ぷれぜんと を ありがとう	
\\	なプレゼントをありがとう。			
\\	脅かす	脅[おど]かす	おどかす	
\\	脅かさないでよ。	脅[おど]かさないでよ。	おどかさない で よ	
\\	よ。			
\\	系列	系列[けいれつ]	けいれつ	
\\	高校卒業後、系列の大学に進みました。	高校卒業後[こうこう そつぎょう ご]、 系列[けいれつ]の 大学[だいがく]に 進[すす]みました。	こうこう そつぎょう ご けいれつ の だいがく に すすみました	
\\	高校卒業後[こうこう そつぎょう ご]、
\\	の 大学[だいがく]に 進[すす]みました。			
\\	系統	系統[けいとう]	けいとう	
\\	電気系統を図面で確かめました。	電気[でんき] 系統[けいとう]を 図面[ずめん]で 確[たし]かめました。	でんき けいとう を ずめん で たしかめました	
\\	電気[でんき]
\\	を 図面[ずめん]で 確[たし]かめました。			
\\	絹	絹[きぬ]	きぬ	
\\	このシャツは絹でできています。	このシャツは 絹[きぬ]でできています。	この しゃつ は きぬ で できて います	
\\	このシャツは
\\	でできています。			
\\	維持	維持[いじ]	いじ	
\\	彼女は何とか健康を維持しているわね。	彼女[かのじょ]は 何[なん]とか 健康[けんこう]を 維持[いじ]しているわね。	かのじょ は なんとか けんこう を いじ して いる わ ね	
\\	彼女[かのじょ]は 何[なん]とか 健康[けんこう]を
\\	しているわね。			
\\	紳士	紳士[しんし]	しんし	
\\	身なりのいい紳士が話しかけてきたの。	身[み]なりのいい 紳士[しんし]が 話[はな]しかけてきたの。	みなり の いい しんし が はなしかけて きた の	
\\	身[み]なりのいい
\\	が 話[はな]しかけてきたの。			
\\	詩	詩[し]	し	
\\	彼女の詩は世界中で有名になったんだ。	彼女[かのじょ]の 詩[し]は 世界中[せかいじゅう]で 有名[ゆうめい]になったんだ。	かのじょ の し は せかいじゅう で ゆうめい に なった ん だ	
\\	彼女[かのじょ]の
\\	は 世界中[せかいじゅう]で 有名[ゆうめい]になったんだ。			
\\	詩人	詩人[しじん]	しじん	
\\	彼は優れた詩人です。	彼[かれ]は 優[すぐ]れた 詩人[しじん]です。	かれ は すぐれた しじん です	
\\	彼[かれ]は 優[すぐ]れた
\\	です。			
\\	誠実	誠実[せいじつ]	せいじつ	
\\	彼はとても誠実な人です。	彼[かれ]はとても 誠実[せいじつ]な 人[ひと]です。	かれ は とても せいじつ な ひと です	
\\	彼[かれ]はとても
\\	な 人[ひと]です。			
\\	盛ん	盛[さか]ん	さかん	
\\	その都市は商業が盛んよ。	その 都市[とし]は 商業[しょうぎょう]が 盛[さか]んよ。	その とし は しょうぎょう が さかん よ	
\\	その 都市[とし]は 商業[しょうぎょう]が
\\	よ。			
\\	謙遜	謙遜[けんそん]	けんそん	
\\	そんなに謙遜しなくてもいい。	そんなに 謙遜[けんそん]しなくてもいい。	そんな に けんそん しなくて も いい	
\\	そんなに
\\	しなくてもいい。			
\\	鋭い	鋭[するど]い	するどい	
\\	鋭い刃物で指を怪我した。	鋭[するど]い 刃物[はもの]で 指[ゆび]を 怪我[けが]した。	するどい はもの で ゆび を けが した	
\\	刃物[はもの]で 指[ゆび]を 怪我[けが]した。			
\\	鎖	鎖[くさり]	くさり	
\\	犬を鎖でつなぎました。	犬[いぬ]を 鎖[くさり]でつなぎました。	いぬ を くさり で つなぎました	
\\	犬[いぬ]を
\\	でつなぎました。			
\\	鑑賞	鑑賞[かんしょう]	かんしょう	
\\	今夜はオペラ鑑賞に行きます。	今夜[こんや]はオペラ 鑑賞[かんしょう]に 行[い]きます。	こんや は おぺら かんしょう に いきます	
\\	今夜[こんや]はオペラ
\\	に 行[い]きます。			
\\	鐘	鐘[かね]	かね	
\\	この鐘の音はとてもきれいですね。	この 鐘[かね]の 音[ね]はとてもきれいですね。	この かね の ね は とても きれい です ね	
\\	この
\\	の 音[ね]はとてもきれいですね。			
\\	砂浜	砂浜[すなはま]	すなはま	
\\	朝早く砂浜を散歩しました。	朝早[あさ はや]く 砂浜[すなはま]を 散歩[さんぽ]しました。	あさ はやく すなはま を さんぽ しました	
\\	朝早[あさ はや]く
\\	を 散歩[さんぽ]しました。			
\\	温泉	温泉[おんせん]	おんせん	
\\	ここの温泉はよく効くそうです。	ここの 温泉[おんせん]はよく 効[き]くそうです。	ここ の おんせん は よく きく そう です	
\\	ここの
\\	はよく 効[き]くそうです。			
\\	澄む	澄[す]む	すむ	
\\	山の空気は澄んでいますね。	山[やま]の 空気[くうき]は 澄[す]んでいますね。	やま の くうき は すんで います ね	
\\	山[やま]の 空気[くうき]は
\\	ね。			
\\	粉	粉[こな]	こな	
\\	その白い粉は赤ちゃんのミルクです。	その 白[しろ]い 粉[こな]は 赤[あか]ちゃんのミルクです。	その しろい こな は あかちゃん の みるく です	
\\	その 白[しろ]い
\\	は 赤[あか]ちゃんのミルクです。			
\\	炊事	炊事[すいじ]	すいじ	
\\	うちでは夫も炊事をします。	うちでは 夫[おっと]も 炊事[すいじ]をします。	うち で は おっと も すいじ を します	
\\	うちでは 夫[おっと]も
\\	をします。			
\\	皮	皮[かわ]	かわ	
\\	りんごの皮をむきましたよ。	りんごの 皮[かわ]をむきましたよ。	りんご の かわ を むきました よ	
\\	りんごの
\\	をむきましたよ。			
\\	白髪	白髪[しらが]	しらが	
\\	祖父は白髪が少ないほうです。	祖父[そふ]は 白髪[しらが]が 少[すく]ないほうです。	そふ は しらが が すくない ほう です	
\\	祖父[そふ]は
\\	が 少[すく]ないほうです。			
\\	耳鼻科	耳鼻科[じびか]	じびか	
\\	今、耳鼻科にかかっています。	今[いま]、 耳鼻科[じびか]にかかっています。	いま じびか に かかって います	
\\	今[いま]、
\\	にかかっています。			
\\	解釈	解釈[かいしゃく]	かいしゃく	
\\	この詩を解釈してみましょう。	この 詩[し]を 解釈[かいしゃく]してみましょう。	この し を かいしゃく して みましょう	
\\	この 詩[し]を
\\	してみましょう。			
\\	高層	高層[こうそう]	こうそう	
\\	このあたりは高層ビルが増えましたね。	このあたりは 高層[こうそう]ビルが 増[ふ]えましたね。	この あたり は こうそうびる が ふえました ね	
\\	このあたりは
\\	ビルが 増[ふ]えましたね。			
\\	覆う	覆[おお]う	おおう	
\\	雲が空を覆っているね。	雲[くも]が 空[そら]を 覆[おお]っているね。	くも が そら を おおって いる ね	
\\	雲[くも]が 空[そら]を
\\	ね。			
\\	鮮やか	鮮[あざ]やか	あざやか	
\\	山は鮮やかな緑でした。	山[やま]は 鮮[あざ]やかな 緑[みどり]でした。	やま は あざやか な みどり でした	
\\	山[やま]は
\\	な 緑[みどり]でした。			
\\	群集	群集[ぐんしゅう]	ぐんしゅう	
\\	数百人の群集が集まっているの。	数百人[すうひゃくにん]の 群集[ぐんしゅう]が 集[あつ]まっているの。	すうひゃくにん の ぐんしゅう が あつまって いる の	
\\	数百人[すうひゃくにん]の
\\	が 集[あつ]まっているの。			
\\	郡	郡[ぐん]	ぐん	
\\	この郡は来年、市になりますよ。	この 郡[ぐん]は 来年[らいねん]、 市[し]になりますよ。	この ぐん は らいねん し に なります よ	
\\	この
\\	は 来年[らいねん]、 市[し]になりますよ。			
\\	脂	脂[あぶら]	あぶら	
\\	この肉は脂が多い。	この 肉[にく]は 脂[あぶら]が 多[おお]い。	この にく は あぶら が おおい	
\\	この 肉[にく]は
\\	が 多[おお]い。			
\\	脂肪	脂肪[しぼう]	しぼう	
\\	お腹の脂肪を取りたい。	お 腹[なか]の 脂肪[しぼう]を 取[と]りたい。	おなか の しぼう を とりたい	
\\	お 腹[なか]の
\\	を 取[と]りたい。			
\\	胃腸	胃腸[いちょう]	いちょう	
\\	彼は胃腸が強くありません。	彼[かれ]は 胃腸[いちょう]が 強[つよ]くありません。	かれ は いちょう が つよく ありません	
\\	彼[かれ]は
\\	が 強[つよ]くありません。			
\\	肝心	肝心[かんじん]	かんじん	
\\	何事も最初が肝心だ。	何事[なにごと]も 最初[さいしょ]が 肝心[かんじん]だ。	なにごと も さいしょ が かんじん だ	
\\	何事[なにごと]も 最初[さいしょ]が
\\	だ。			
\\	芝生	芝生[しばふ]	しばふ	
\\	芝生がよく手入れされていますね。	芝生[しばふ]がよく 手入[てい]れされていますね。	しばふ が よく ていれ されて います ね	
\\	がよく 手入[てい]れされていますね。			
\\	茎	茎[くき]	くき	
\\	チューリップは茎が長いです。	チューリップは 茎[くき]が 長[なが]いです。	ちゅーりっぷ は くき が ながい です	
\\	チューリップは
\\	が 長[なが]いです。			
\\	荒れる	荒[あ]れる	あれる	
\\	台風で山も海も荒れています。	台風[たいふう]で 山[やま]も 海[うみ]も 荒[あ]れています。	たいふう で やま も うみ も あれて います	
\\	台風[たいふう]で 山[やま]も 海[うみ]も
\\	荒い	荒[あら]い	あらい	
\\	彼は金遣いが荒いな。	彼[かれ]は 金遣[かねづか]いが 荒[あら]いな。	かれ は かねづかい が あらい な	
\\	彼[かれ]は 金遣[かねづか]いが
\\	な。			
\\	荒い	荒[あら]い	あらい	
\\	うちの犬は気が荒い。	うちの 犬[いぬ]は 気[き]が 荒[あら]い。	うち の いぬ は き が あらい	
\\	うちの 犬[いぬ]は 気[き]が
\\	荒す	荒[あら]す	あらす	
\\	彼は強盗に家の中を荒されたの。	彼[かれ]は 強盗[ごうとう]に 家[いえ]の 中[なか]を 荒[あら]されたの。	かれ は ごうとう に いえ の なか を あらされた の	
\\	彼[かれ]は 強盗[ごうとう]に 家[いえ]の 中[なか]を
\\	の。			
\\	行儀	行儀[ぎょうぎ]	ぎょうぎ	
\\	この子は本当に行儀の良い子です。	この 子[こ]は 本当[ほんとう]に 行儀[ぎょうぎ]の 良[い]い 子[こ]です。	この こ は ほんとう に ぎょうぎ の いい こ です	
\\	この 子[こ]は 本当[ほんとう]に
\\	の 良[い]い 子[こ]です。			
\\	犠牲	犠牲[ぎせい]	ぎせい	
\\	多くの人が災害の犠牲になったの。	多[おお]くの 人[ひと]が 災害[さいがい]の 犠牲[ぎせい]になったの。	おおく の ひと が さいがい の ぎせい に なった の	
\\	多[おお]くの 人[ひと]が 災害[さいがい]の
\\	になったの。			
\\	赤ん坊	赤[あか]ん 坊[ぼう]	あかんぼう	
\\	その頃彼はまだ赤ん坊だったよね。	その 頃彼[ころ かれ]はまだ 赤[あか]ん 坊[ぼう]だったよね。	その ころ かれ は まだ あかんぼう だった よ ね	
\\	その 頃彼[ころ かれ]はまだ
\\	だったよね。			
\\	筋肉	筋肉[きんにく]	きんにく	
\\	彼は最近筋肉を鍛えているよ。	彼[かれ]は 最近[さいきん] 筋肉[きんにく]を 鍛[きた]えているよ。	かれ は さいきん きんにく を きたえて いる よ	
\\	彼[かれ]は 最近[さいきん]
\\	を 鍛[きた]えているよ。			
\\	筋道	筋道[すじみち]	すじみち	
\\	筋道を立てて考えなさい。	筋道[すじみち]を 立[た]てて 考[かんが]えなさい。	すじみち を たてて かんがえなさい	
\\	を 立[た]てて 考[かんが]えなさい。			
\\	筋	筋[すじ]	すじ	
\\	転んで筋を痛めてしまいました。	転[ころ]んで 筋[すじ]を 痛[いた]めてしまいました。	ころんで すじ を いためて しまいました	
\\	転[ころ]んで
\\	を 痛[いた]めてしまいました。			
\\	裂ける	裂[さ]ける	さける	
\\	釘に引っかけて袖が裂けてしまったな。	釘[くぎ]に 引[ひ]っかけて 袖[そで]が 裂[さ]けてしまったな。	くぎ に ひっかけて そで が さけて しまった な	
\\	釘[くぎ]に 引[ひ]っかけて 袖[そで]が
\\	な。			
\\	自己	自己[じこ]	じこ	
\\	これからは自己の判断で行動してください。	これからは 自己[じこ]の 判断[はんだん]で 行動[こうどう]してください。	これ から は じこ の はんだん で こうどう して ください	
\\	これからは
\\	の 判断[はんだん]で 行動[こうどう]してください。			
\\	飽きる	飽[あ]きる	あきる	
\\	その子はおもちゃに飽きたようだね。	その 子[こ]はおもちゃに 飽[あ]きたようだね。	その こ は おもちゃ に あきた よう だ ね	
\\	その 子[こ]はおもちゃに
\\	ようだね。			
\\	飾り	飾[かざ]り	かざり	
\\	この部屋は飾りが多すぎるな。	この 部屋[へや]は 飾[かざ]りが 多[おお]すぎるな。	この へや は かざり が おおすぎる な	
\\	この 部屋[へや]は
\\	が 多[おお]すぎるな。			
\\	餌	餌[えさ]	えさ	
\\	毎日金魚に餌をやります。	毎日金魚[まいにち きんぎょ]に 餌[えさ]をやります。	まいにち きんぎょ に えさ を やります	
\\	毎日金魚[まいにち きんぎょ]に
\\	をやります。			
\\	育児	育児[いくじ]	いくじ	
\\	彼女は育児で忙しいよ。	彼女[かのじょ]は 育児[いくじ]で 忙[いそが]しいよ。	かのじょ は いくじ で いそがしい よ	
\\	彼女[かのじょ]は
\\	で 忙[いそが]しいよ。			
\\	眼科	眼科[がんか]	がんか	
\\	眼科で視力検査をしました。	眼科[がんか]で 視力検査[しりょく けんさ]をしました。	がんか で しりょく けんさ を しました	
\\	で 視力検査[しりょく けんさ]をしました。			
\\	近眼	近眼[きんがん]	きんがん	
\\	彼女は近眼です。	彼女[かのじょ]は 近眼[きんがん]です。	かのじょ は きんがん です	
\\	彼女[かのじょ]は
\\	です。			
\\	瞬間	瞬間[しゅんかん]	しゅんかん	
\\	その瞬間、猫が跳び出しました。	その 瞬間[しゅんかん]、 猫[ねこ]が 跳[と]び 出[だ]しました。	その しゅんかん ねこ が とびだしました	
\\	その
\\	、 猫[ねこ]が 跳[と]び 出[だ]しました。			
\\	睡眠	睡眠[すいみん]	すいみん	
\\	1日7時間は睡眠を取るようにしています。	1日7時間[いちにち しちじかん]は 睡眠[すいみん]を 取[と]るようにしています。	いちにち しちじかん は すいみん を とる よう に して います	
\\	1日7時間[いちにち しちじかん]は
\\	を 取[と]るようにしています。			
\\	腰掛ける	腰掛[こしか]ける	こしかける	
\\	私はベンチに腰掛けたの。	私[わたし]はベンチに 腰掛[こしか]けたの。	わたし は べんち に こしかけた の	
\\	私[わたし]はベンチに
\\	の。			
\\	追い掛ける	追[お]い 掛[か]ける	おいかける	
\\	パトカーがバイクを追いかけてるぞ。	パトカーがバイクを 追[お]いかけてるぞ。	ぱとかー が ばいく を おいかけて る ぞ	
\\	パトカーがバイクを
\\	ぞ。			
\\	腰掛け	腰掛[こしか]け	こしかけ	
\\	腰掛けはありませんか。	腰掛[こしか]けはありませんか。	こしかけ は ありません か	
\\	はありませんか。			
\\	輝く	輝[かがや]く	かがやく	
\\	彼女の瞳は喜びで輝いていますね。	彼女[かのじょ]の 瞳[ひとみ]は 喜[よろこ]びで 輝[かがや]いていますね。	かのじょ の ひとみ は よろこび で かがやいて います ね	
\\	彼女[かのじょ]の 瞳[ひとみ]は 喜[よろこ]びで
\\	ね。			
\\	隅	隅[すみ]	すみ	
\\	部屋の隅にいすが積んであったの。	部屋[へや]の 隅[すみ]にいすが 積[つ]んであったの。	へや の すみ に いす が つんで あった の	
\\	部屋[へや]の
\\	にいすが 積[つ]んであったの。			
\\	警戒	警戒[けいかい]	けいかい	
\\	地震のあとは津波に警戒してください。	地震[じしん]のあとは 津波[つなみ]に 警戒[けいかい]してください。	じしん の あと は つなみ に けいかい して ください	
\\	地震[じしん]のあとは 津波[つなみ]に
\\	してください。			
\\	相撲	相撲[すもう]	すもう	
\\	お相撲さんは、みんな力持ちだ。	お 相撲[すもう]さんは、みんな 力[ちから] 持[も]ちだ。	おすもうさん は みんな ちからもち だ	
\\	お
\\	さんは、みんな 力[ちから] 持[も]ちだ。			
\\	衰える	衰[おとろ]える	おとろえる	
\\	年と共に体力が衰えています。	年[とし]と 共[とも]に 体力[たいりょく]が 衰[おとろ]えています。	とし と とも に たいりょく が おとろえて います	
\\	年[とし]と 共[とも]に 体力[たいりょく]が
\\	覚悟	覚悟[かくご]	かくご	
\\	覚悟はできています。	覚悟[かくご]はできています。	かくご は できて います	
\\	はできています。			
\\	自慢	自慢[じまん]	じまん	
\\	彼女はよく自分の成績を自慢するよね。	彼女[かのじょ]はよく 自分[じぶん]の 成績[せいせき]を 自慢[じまん]するよね。	かのじょ は よく じぶん の せいせき を じまん する よ ね	
\\	彼女[かのじょ]はよく 自分[じぶん]の 成績[せいせき]を
\\	するよね。			
\\	食卓	食卓[しょくたく]	しょくたく	
\\	食卓に花を飾りましょう。	食卓[しょくたく]に 花[はな]を 飾[かざ]りましょう。	しょくたく に はな を かざりましょう	
\\	に 花[はな]を 飾[かざ]りましょう。			
\\	焦点	焦点[しょうてん]	しょうてん	
\\	焦点を絞って話しましょう。	焦点[しょうてん]を 絞[しぼ]って 話[はな]しましょう。	しょうてん を しぼって はなしましょう	
\\	を 絞[しぼ]って 話[はな]しましょう。			
\\	焦る	焦[あせ]る	あせる	
\\	何をそんなに焦っているのですか。	何[なに]をそんなに 焦[あせ]っているのですか。	なに を そんなに あせって いる の です か	
\\	何[なに]をそんなに
\\	のですか。			
\\	焦げる	焦[こ]げる	こげる	
\\	シチューが焦げたよー。	シチューが 焦[こ]げたよー。	しちゅー が こげた よー	
\\	シチューが
\\	よー。			
\\	英雄	英雄[えいゆう]	えいゆう	
\\	彼は国の英雄ですね。	彼[かれ]は 国[くに]の 英雄[えいゆう]ですね。	かれ は くに の えいゆう です ね	
\\	彼[かれ]は 国[くに]の
\\	ですね。			
\\	雄	雄[おす]	おす	
\\	うちの猫は雄です。	うちの 猫[ねこ]は 雄[おす]です。	うち の ねこ は おす です	
\\	うちの 猫[ねこ]は
\\	です。			
\\	獲得	獲得[かくとく]	かくとく	
\\	その選手の獲得賞金は2億円だって。	その 選手[せんしゅ]の 獲得[かくとく] 賞金[しょうきん]は 2億円[におくえん]だって。	その せんしゅ の かくとく しょうきん は におくえん だって	
\\	その 選手[せんしゅ]の
\\	賞金[しょうきん]は 2億円[におくえん]だって。			
\\	稲	稲[いね]	いね	
\\	日本は稲の品種が多いよ。	日本[にっぽん]は 稲[いね]の 品種[ひんしゅ]が 多[おお]いよ。	にっぽん は いね の ひんしゅ が おおい よ	
\\	日本[にっぽん]は
\\	の 品種[ひんしゅ]が 多[おお]いよ。			
\\	稼ぐ	稼[かせ]ぐ	かせぐ	
\\	彼はお金をだいぶ稼いだらしいよ。	彼[かれ]はお 金[かね]をだいぶ 稼[かせ]いだらしいよ。	かれ は おかね を だいぶ かせいだ らしい よ	
\\	彼[かれ]はお 金[かね]をだいぶ
\\	らしいよ。			
\\	穏やか	穏[おだ]やか	おだやか	
\\	彼らは穏やかな暮らしをしているの。	彼[かれ]らは 穏[おだ]やかな 暮[く]らしをしているの。	かれら は おだやか な くらし を して い の	
\\	彼[かれ]らは
\\	な 暮[く]らしをしているの。			
\\	隠す	隠[かく]す	かくす	
\\	僕に隠していることは無いですか。	僕[ぼく]に 隠[かく]していることは 無[な]いですか。	ぼく に かくして いる こと は ない です か	
\\	僕[ぼく]に
\\	ことは 無[な]いですか。			
\\	陰	陰[かげ]	かげ	
\\	陰で少し休憩しましょう。	陰[かげ]で 少[すこ]し 休憩[きゅうけい]しましょう。	かげ で すこし きゅうけい しましょう	
\\	で 少[すこ]し 休憩[きゅうけい]しましょう。			
\\	砕く	砕[くだ]く	くだく	
\\	氷を細かく砕いてください。	氷[こおり]を 細[こま]かく 砕[くだ]いてください。	こおり を こまかく くだいて ください	
\\	氷[こおり]を 細[こま]かく
\\	ください。			
\\	砕ける	砕[くだ]ける	くだける	
\\	石が粉々に砕けましたね。	石[いし]が 粉々[こなごな]に 砕[くだ]けましたね。	いし が こなごな に くだけました ね	
\\	石[いし]が 粉々[こなごな]に
\\	ね。			
\\	貝	貝[かい]	かい	
\\	浜辺できれいな貝を拾いました。	浜辺[はまべ]できれいな 貝[かい]を 拾[ひろ]いました。	はまべ で きれい な かい を ひろいました	
\\	浜辺[はまべ]できれいな
\\	を 拾[ひろ]いました。			
\\	賢い	賢[かしこ]い	かしこい	
\\	彼は冷静で賢い男だね。	彼[かれ]は 冷静[れいせい]で 賢[かしこ]い 男[おとこ]だね。	かれ は れいせい で かしこい おとこ だ ね	
\\	彼[かれ]は 冷静[れいせい]で
\\	男[おとこ]だね。			
\\	頻りに	頻[しき]りに	しきりに	
\\	彼は頻りに時計を気にしていた。	彼[かれ]は 頻[しき]りに 時計[とけい]を 気[き]にしていた。	かれ は しきりに とけい を き に して いた	
\\	彼[かれ]は
\\	時計[とけい]を 気[き]にしていた。			
\\	頑固	頑固[がんこ]	がんこ	
\\	なんて頑固な子だ。	なんて 頑固[がんこ]な 子[こ]だ。	なんて がんこ な こ だ	
\\	なんて
\\	な 子[こ]だ。			
\\	頂く	頂[いただ]く	いただく	
\\	お客様にお菓子を頂いた。	お 客様[きゃくさま]にお 菓子[かし]を 頂[いただ]いた。	おきゃくさま に おかし を いただいた	
\\	お 客様[きゃくさま]にお 菓子[かし]を
\\	真剣	真剣[しんけん]	しんけん	
\\	彼は真剣に話を聞いていました。	彼[かれ]は 真剣[しんけん]に 話[はなし]を 聞[き]いていました。	かれ は しんけん に はなし を きいて いました	
\\	彼[かれ]は
\\	に 話[はなし]を 聞[き]いていました。			
\\	獣	獣[けもの]	けもの	
\\	彼は獣のような目をしていたな。	彼[かれ]は 獣[けもの]のような 目[め]をしていたな。	かれ は けもの の よう な め を して いた な	
\\	彼[かれ]は
\\	のような 目[め]をしていたな。			
\\	駆ける	駆[か]ける	かける	
\\	子供たちは広場に駆けて行ったわよ。	子供[こども]たちは 広場[ひろば]に 駆[か]けて 行[い]ったわよ。	こどもたち は ひろば に かけて いった わ よ	
\\	子供[こども]たちは 広場[ひろば]に
\\	行[い]ったわよ。			
\\	駆け足	駆[か]け 足[あし]	かけあし	
\\	駅まで駆け足で行ったよ。	駅[えき]まで 駆[か]け 足[あし]で 行[い]ったよ。	えき まで かけあし で いった よ	
\\	駅[えき]まで
\\	で 行[い]ったよ。			
\\	県庁	県庁[けんちょう]	けんちょう	
\\	あの白い建物が県庁です。	あの 白[しろ]い 建物[たてもの]が 県庁[けんちょう]です。	あの しろい たてもの が けんちょう です	
\\	あの 白[しろ]い 建物[たてもの]が
\\	です。			
\\	軸	軸[じく]	じく	
\\	この線を軸にして図形を回転してください。	この 線[せん]を 軸[じく]にして 図形[ずけい]を 回転[かいてん]してください。	この せん を じく に して ずけい を かいてん して ください	
\\	この 線[せん]を
\\	にして 図形[ずけい]を 回転[かいてん]してください。			
\\	香り	香[かお]り	かおり	
\\	この花はいい香りがしますね。	この 花[はな]はいい 香[かお]りがしますね。	この はな は いい かおり が します ね	
\\	この 花[はな]はいい
\\	がしますね。			
\\	香水	香水[こうすい]	こうすい	
\\	香水のいい香りがした。	香水[こうすい]のいい 香[かお]りがした。	こうすい の いい かおり が した 。	
\\	のいい 香[かお]りがした。			
\\	結晶	結晶[けっしょう]	けっしょう	
\\	雪の結晶にはいろいろな形があります。	雪[ゆき]の 結晶[けっしょう]にはいろいろな 形[かたち]があります。	ゆき の けっしょう に は いろいろ な かたち が あります	
\\	雪[ゆき]の
\\	にはいろいろな 形[かたち]があります。			
\\	霧	霧[きり]	きり	
\\	霧の深い夜のことでした。	霧[きり]の 深[ふか]い 夜[よる]のことでした。	きり の ふかい よる の こと でした	
\\	の 深[ふか]い 夜[よる]のことでした。			
\\	霜	霜[しも]	しも	
\\	今朝は庭の草に霜が降りていたの。	今朝[けさ]は 庭[にわ]の 草[くさ]に 霜[しも]が 降[お]りていたの。	けさ は にわ の くさ に しも が おりて いた の	
\\	今朝[けさ]は 庭[にわ]の 草[くさ]に
\\	が 降[お]りていたの。			
\\	衆議院	衆議院[しゅうぎいん]	しゅうぎいん	
\\	予算案が衆議院を通過したな。	予算案[よさんあん]が 衆議院[しゅうぎいん]を 通過[つうか]したな。	よさんあん が しゅうぎいん を つうか した な	
\\	予算案[よさんあん]が
\\	を 通過[つうか]したな。			
\\	観衆	観衆[かんしゅう]	かんしゅう	
\\	スタジアムは観衆で満員だったよ。	スタジアムは 観衆[かんしゅう]で 満員[まんいん]だったよ。	すたじあむ は かんしゅう で まんいん だった よ	
\\	スタジアムは
\\	で 満員[まんいん]だったよ。			
\\	群衆	群衆[ぐんしゅう]	ぐんしゅう	
\\	彼はマイクで群衆に話しかけたんだ。	彼[かれ]はマイクで 群衆[ぐんしゅう]に 話[はな]しかけたんだ。	かれ は まいく で ぐんしゅう に はなしかけた ん だ	
\\	彼[かれ]はマイクで
\\	に 話[はな]しかけたんだ。			
\\	金銭	金銭[きんせん]	きんせん	
\\	金銭のトラブルには関わりたくありません。	金銭[きんせん]のトラブルには 関[かか]わりたくありません。	きんせん の とらぶる に は かかわりたく ありません	
\\	のトラブルには 関[かか]わりたくありません。			
\\	講堂	講堂[こうどう]	こうどう	
\\	全員、講堂に集まってください。	全員[ぜんいん]、 講堂[こうどう]に 集[あつ]まってください。	ぜんいん こうどう に あつまって ください	
\\	全員[ぜんいん]、
\\	に 集[あつ]まってください。			
\\	[じぇいあーる]	じぇいあーる	
\\	私は通勤に
\\	を使うんだ。	私[わたし]は 通勤[つうきん]に 
\\	[じぇいあーる]を 使[つか]うんだ。	わたし は つうきん に じぇいあーる を つかう ん だ	
\\	私[わたし]は 通勤[つうきん]に
\\	を 使[つか]うんだ。			
\\	狂う	狂[くる]う	くるう	
\\	この時計はすぐ狂うの。	この 時計[とけい]はすぐ 狂[くる]うの。	この とけい は すぐ くるう の	
\\	この 時計[とけい]はすぐ
\\	の。			
\\	猿	猿[さる]	さる	
\\	山で猿の親子を見ました。	山[やま]で 猿[さる]の 親子[おやこ]を 見[み]ました。	やま で さる の おやこ を みました	
\\	山[やま]で
\\	の 親子[おやこ]を 見[み]ました。			
\\	章	章[しょう]	しょう	
\\	この本の第6章が特に好きです。	この 本[ほん]の 第6[だいろく] 章[しょう]が 特[とく]に 好[す]きです。	この ほん の だいろくしょう が とくに すき です	
\\	この 本[ほん]の 第6[だいろく]
\\	が 特[とく]に 好[す]きです。			
\\	頑丈	頑丈[がんじょう]	がんじょう	
\\	錠を頑丈なものに替えました。	錠[じょう]を 頑丈[がんじょう]なものに 替[か]えました。	じょう を がんじょう な もの に かえました	
\\	錠[じょう]を
\\	なものに 替[か]えました。			
\\	親戚	親戚[しんせき]	しんせき	
\\	祖父の家に親戚が集まったんだ。	祖父[そふ]の 家[いえ]に 親戚[しんせき]が 集[あつ]まったんだ。	そふ の いえ に しんせき が あつまった ん だ	
\\	祖父[そふ]の 家[いえ]に
\\	が 集[あつ]まったんだ。			
\\	祈る	祈[いの]る	いのる	
\\	皆が人質の無事を祈っているわよ。	皆[みんな]が 人質[ひとじち]の 無事[ぶじ]を 祈[いの]っているわよ。	みんな が ひとじち の ぶじ を いのって いる わ よ	
\\	皆[みんな]が 人質[ひとじち]の 無事[ぶじ]を
\\	わよ。			
\\	祈り	祈[いの]り	いのり	
\\	その日、国民は平和への祈りを捧げるの。	その 日[ひ]、 国民[こくみん]は 平和[へいわ]への 祈[いの]りを 捧[ささ]げるの。	その ひ こくみん は へいわ へ の いのり を ささげる の	
\\	その 日[ひ]、 国民[こくみん]は 平和[へいわ]への
\\	を 捧[ささ]げるの。			
\\	襟	襟[えり]	えり	
\\	襟の大きいコートを買いました。	襟[えり]の 大[おお]きいコートを 買[か]いました。	えり の おおきい こーと を かいました	
\\	の 大[おお]きいコートを 買[か]いました。			
\\	粗筋	粗筋[あらすじ]	あらすじ	
\\	その映画はどんな粗筋ですか。	その 映画[えいが]はどんな 粗筋[あらすじ]ですか。	その えいが は どんな あらすじ です か	
\\	その 映画[えいが]はどんな
\\	ですか。			
\\	酢	酢[す]	す	
\\	お酢を入れすぎて酸っぱい。	お 酢[す]を 入[い]れすぎて 酸[す]っぱい。	お す を いれすぎて すっぱい	
\\	お
\\	を 入[い]れすぎて 酸[す]っぱい。			
\\	郊外	郊外[こうがい]	こうがい	
\\	私は郊外に家を買いました。	私[わたし]は 郊外[こうがい]に 家[いえ]を 買[か]いました。	わたし は こうがい に いえ を かいました	
\\	私[わたし]は
\\	に 家[いえ]を 買[か]いました。			
\\	近郊	近郊[きんこう]	きんこう	
\\	彼らは東京近郊に住んでいる。	彼[かれ]らは 東京[とうきょう] 近郊[きんこう]に 住[す]んでいる。	かれら は とうきょう きんこう に すんで いる	
\\	彼[かれ]らは 東京[とうきょう]
\\	に 住[す]んでいる。			
\\	邪魔	邪魔[じゃま]	じゃま	
\\	邪魔です、どいてください。	邪魔[じゃま]です、どいてください。	じゃま です どいて ください	
\\	です、どいてください。			
\\	親善	親善[しんぜん]	しんぜん	
\\	これからも両国の親善を深めましょう。	これからも 両国[りょうこく]の 親善[しんぜん]を 深[ふか]めましょう。	これから も りょうこく の しんぜん を ふかめましょう	
\\	これからも 両国[りょうこく]の
\\	を 深[ふか]めましょう。			
\\	網	網[あみ]	あみ	
\\	少年は網でその蝶を捕まえた。	少年[しょうねん]は 網[あみ]でその 蝶[ちょう]を 捕[つか]まえた。	しょうねん は あみ で その ちょう を つかまえた	
\\	少年[しょうねん]は
\\	でその 蝶[ちょう]を 捕[つか]まえた。			
\\	縛る	縛[しば]る	しばる	
\\	古新聞をひもで縛ったよ。	古新聞[ふるしんぶん]をひもで 縛[しば]ったよ。	ふるしんぶん を ひも で しばった よ 。	
\\	古新聞[ふるしんぶん]をひもで
\\	よ。			
\\	絞る	絞[しぼ]る	しぼる	
\\	布をもっと固く絞りなさい。	布[ぬの]をもっと 固[かた]く 絞[しぼ]りなさい。	ぬの を もっと かたく しぼりなさい	
\\	布[ぬの]をもっと 固[かた]く
\\	紺	紺[こん]	こん	
\\	日本の制服は紺が多いね。	日本[にほん]の 制服[せいふく]は 紺[こん]が 多[おお]いね。	にほん の せいふく は こん が おおい ね	
\\	日本[にほん]の 制服[せいふく]は
\\	が 多[おお]いね。			
\\	紅葉	紅葉[こうよう]	こうよう	
\\	この山は紅葉がとても美しい。	この 山[やま]は 紅葉[こうよう]がとても 美[うつく]しい。	この やま は こうよう が とても うつくしい	
\\	この 山[やま]は
\\	がとても 美[うつく]しい。			
\\	考慮	考慮[こうりょ]	こうりょ	
\\	あなたの事情を考慮して予定をたてました。	あなたの 事情[じじょう]を 考慮[こうりょ]して 予定[よてい]をたてました。	あなた の じじょう を こうりょ して よてい を たてました	
\\	あなたの 事情[じじょう]を
\\	して 予定[よてい]をたてました。			
\\	癖	癖[くせ]	くせ	
\\	爪をかむ癖は直した方がいい。	爪[つめ]をかむ 癖[くせ]は 直[なお]した 方[ほう]がいい。	つめ を かむ くせ は なおした ほう が いい	
\\	爪[つめ]をかむ
\\	は 直[なお]した 方[ほう]がいい。			
\\	田舎	田舎[いなか]	いなか	
\\	私は毎年夏に田舎に帰ります。	私[わたし]は 毎年夏[まいとし なつ]に 田舎[いなか]に 帰[かえ]ります。	わたし は まいとし なつ に いなか に かえります	
\\	私[わたし]は 毎年夏[まいとし なつ]に
\\	に 帰[かえ]ります。			
\\	生涯	生涯[しょうがい]	しょうがい	
\\	彼は80年の生涯を閉じました。	彼[かれ]は 80年[はちじゅうねん]の 生涯[しょうがい]を 閉[と]じました。	かれ は はちじゅうねん の しょうがい を とじました	
\\	彼[かれ]は 80年[はちじゅうねん]の
\\	を 閉[と]じました。			
\\	賭ける	賭[か]ける	かける	
\\	彼は新しい仕事に人生を賭けている。	彼[かれ]は 新[あたら]しい 仕事[しごと]に 人生[じんせい]を 賭[か]けている。	かれ は あたらしい しごと に じんせい を かけて いる	
\\	彼[かれ]は 新[あたら]しい 仕事[しごと]に 人生[じんせい]を
\\	蛍光灯	蛍光灯[けいこうとう]	けいこうとう	
\\	古い蛍光灯を取り替えてください。	古[ふる]い 蛍光灯[けいこうとう]を 取[と]り 替[か]えてください。	ふるい けいこうとう を とりかえて ください	
\\	古[ふる]い
\\	を 取[と]り 替[か]えてください。			
\\	車掌	車掌[しゃしょう]	しゃしょう	
\\	車掌さんが車内を回って来たよ。	車掌[しゃしょう]さんが 車内[しゃない]を 回[まわ]って 来[き]たよ。	しゃしょう さん が しゃない を まわって きた よ	
\\	さんが 車内[しゃない]を 回[まわ]って 来[き]たよ。			
\\	芋	芋[いも]	いも	
\\	お芋の料理はお好きですか。	お 芋[いも]の 料理[りょうり]はお 好[す]きですか。	お いも の りょうり は お すき です か	
\\	お
\\	の 料理[りょうり]はお 好[す]きですか。			
\\	菊	菊[きく]	きく	
\\	菊の花を買ってきました。	菊[きく]の 花[はな]を 買[か]ってきました。	きく の はな を かって きました	
\\	の 花[はな]を 買[か]ってきました。			
\\	軽蔑	軽蔑[けいべつ]	けいべつ	
\\	彼女は彼を軽蔑していたの。	彼女[かのじょ]は 彼[かれ]を 軽蔑[けいべつ]していたの。	かのじょ は かれ を けいべつ して いた の	
\\	彼女[かのじょ]は 彼[かれ]を
\\	していたの。			
\\	諦める	諦[あきら]める	あきらめる	
\\	彼は留学の夢を諦めていないよ。	彼[かれ]は 留学[りゅうがく]の 夢[ゆめ]を 諦[あきら]めていないよ。	かれ は りゅうがく の ゆめ を あきらめて いない よ	
\\	彼[かれ]は 留学[りゅうがく]の 夢[ゆめ]を
\\	よ。			
\\	演奏	演奏[えんそう]	えんそう	
\\	彼女のピアノの演奏は素晴らしいね。	彼女[かのじょ]のピアノの 演奏[えんそう]は 素晴[すば]らしいね。	かのじょ の ぴあの の えんそう は すばらしい ね	
\\	彼女[かのじょ]のピアノの
\\	は 素晴[すば]らしいね。			
\\	窮屈	窮屈[きゅうくつ]	きゅうくつ	
\\	この服は窮屈になったな。	この 服[ふく]は 窮屈[きゅうくつ]になったな。	この ふく は きゅうくつ に なった な	
\\	この 服[ふく]は
\\	になったな。			
\\	貨幣	貨幣[かへい]	かへい	
\\	博物館で昔の貨幣を見ました。	博物館[はくぶつかん]で 昔[むかし]の 貨幣[かへい]を 見[み]ました。	はくぶつかん で むかし の かへい を みました	
\\	博物館[はくぶつかん]で 昔[むかし]の
\\	を 見[み]ました。			
\\	真珠	真珠[しんじゅ]	しんじゅ	
\\	母に真珠のネックレスをもらいました。	母[はは]に 真珠[しんじゅ]のネックレスをもらいました。	はは に しんじゅ の ねっくれす を もらいました	
\\	母[はは]に
\\	のネックレスをもらいました。			
\\	花瓶	花瓶[かびん]	かびん	
\\	クリスタルガラスの花瓶を買いました。	クリスタルガラスの 花瓶[かびん]を 買[か]いました。	くりすたるがらす の かびん を かいました	
\\	クリスタルガラスの
\\	を 買[か]いました。			
\\	羨ましい	羨[うらや]ましい	うらやましい	
\\	彼の才能は羨ましい程だね。	彼[かれ]の 才能[さいのう]は 羨[うらや]ましい 程[ほど]だね。	かれ の さいのう は うらやましい ほど だ ね	
\\	彼[かれ]の 才能[さいのう]は
\\	程[ほど]だね。			
\\	漕ぐ	漕[こ]ぐ	こぐ	
\\	彼らは交代でボートを漕いだんだ。	彼[かれ]らは 交代[こうたい]でボートを 漕[こ]いだんだ。	かれら は こうたい で ぼーと を こいだ ん だ	
\\	彼[かれ]らは 交代[こうたい]でボートを
\\	んだ。			
\\	稽古	稽古[けいこ]	けいこ	
\\	相撲の稽古はとても厳しいよ。	相撲[すもう]の 稽古[けいこ]はとても 厳[きび]しいよ。	すもう の けいこ は とても きびしい よ	
\\	相撲[すもう]の
\\	はとても 厳[きび]しいよ。			
\\	紙屑	紙屑[かみくず]	かみくず	
\\	紙屑は屑入れに入れなさい。	紙屑[かみくず]は 屑入[くず い]れに 入[い]れなさい。	かみくず は くず いれ に いれなさい 。	
\\	は 屑入[くず い]れに 入[い]れなさい。			
\\	溺れる	溺[おぼ]れる	おぼれる	
\\	彼は溺れている子を助けたんだ。	彼[かれ]は 溺[おぼ]れている 子[こ]を 助[たす]けたんだ。	かれ は おぼれて いる こ を たすけた ん だ	
\\	彼[かれ]は
\\	子[こ]を 助[たす]けたんだ。			
\\	苛め	苛[いじ]め	いじめ	
\\	学校での苛めはなかなか減らない。	学校[がっこう]での 苛[いじ]めはなかなか 減[へ]らない。	がっこう で の いじめ は なかなか へらない 。	
\\	学校[がっこう]での
\\	はなかなか 減[へ]らない。			
\\	苛める	苛[いじ]める	いじめる	
\\	動物を苛めてはいけません。	動物[どうぶつ]を 苛[いじ]めてはいけません。	どうぶつ を いじめて は いけません	
\\	動物[どうぶつ]を
\\	はいけません。			
\\	臆病	臆病[おくびょう]	おくびょう	
\\	弟は小さい頃は臆病だったんだ。	弟[おとうと]は 小[ちい]さい 頃[ころ]は 臆病[おくびょう]だったんだ。	おとうと は ちいさい ころ は おくびょう だった ん だ	
\\	弟[おとうと]は 小[ちい]さい 頃[ころ]は
\\	だったんだ。			
\\	車椅子	車椅子[くるまいす]	くるまいす	
\\	彼女は車椅子に乗っています。	彼女[かのじょ]は 車椅子[くるまいす]に 乗[の]っています。	かのじょ は くるまいす に のっています 。	
\\	彼女[かのじょ]は
\\	に 乗[の]っています。			
\\	釘	釘[くぎ]	くぎ	
\\	釘を踏んで怪我をしました。	釘[くぎ]を 踏[ふ]んで 怪我[けが]をしました。	くぎ を ふんで けが を しました	
\\	を 踏[ふ]んで 怪我[けが]をしました。			
\\	錆びる	錆[さ]びる	さびる	
\\	包丁が錆びてしまった。	包丁[ほうちょう]が 錆[さ]びてしまった。	ほうちょう が さびて しまった	
\\	包丁[ほうちょう]が
\\	爽やか	爽[さわ]やか	さわやか	
\\	レモンの爽やかな香りがした。	レモンの 爽[さわ]やかな 香[かお]りがした。	れもん の さわやか な かおり が した	
\\	レモンの
\\	な 香[かお]りがした。			
\\	雀	雀[すずめ]	すずめ	
\\	朝は雀の声で目が覚めます。	朝[あさ]は 雀[すずめ]の 声[こえ]で 目[め]が 覚[さ]めます。	あさ は すずめ の こえ で め が さめます	
\\	朝[あさ]は
\\	の 声[こえ]で 目[め]が 覚[さ]めます。			
\\	甥	甥[おい]	おい	
\\	私の甥は3才です。	私[わたし]の 甥[おい]は 3才[さんさい]です。	わたし の おい は さんさい です	
\\	私[わたし]の
\\	は 3才[さんさい]です。			
\\	炒める	炒[いた]める	いためる	
\\	次に、野菜を炒めてください。	次[つぎ]に、 野菜[やさい]を 炒[いた]めてください。	つぎ に やさい を いためて ください	
\\	次[つぎ]に、 野菜[やさい]を
\\	ください。			
\\	行方	行方[ゆくえ]	ゆくえ	
\\	彼の行方が分かりません。	彼[かれ]の 行方[ゆくえ]が 分[わ]かりません。	かれ の ゆくえ が わかりません	
\\	彼[かれ]の
\\	が 分[わ]かりません。			
\\	目上	目上[めうえ]	めうえ	
\\	彼は目上の人にとても気を使います。	彼[かれ]は 目上[めうえ]の 人[ひと]にとても 気[き]を 使[つか]います。	かれ は めうえ の ひと に とても き を つかいます	
\\	彼[かれ]は
\\	の 人[ひと]にとても 気[き]を 使[つか]います。			
\\	目方	目方[めかた]	めかた	
\\	この荷の目方は70キロってとこかね。	この 荷[に]の 目方[めかた]は70キロってとこかね。	この に の めかた は 
\\	きろ って とこ か ね	
\\	この 荷[に]の
\\	は70キロってとこかね。			
\\	目下	目下[めした]	めした	
\\	彼は目下の者にも優しいんだ。	彼[かれ]は 目下[めした]の 者[もの]にも 優[やさ]しいんだ。	かれ は めした の もの に も やさしい ん だ	
\\	彼[かれ]は
\\	の 者[もの]にも 優[やさ]しいんだ。			
\\	目下	目下[もっか]	もっか	
\\	息子は目下勉強中です。	息子[むすこ]は 目下[もっか] 勉強中[べんきょうちゅう]です。	むすこ は もっか べんきょうちゅう です	
\\	息子[むすこ]は
\\	勉強中[べんきょうちゅう]です。			
\\	田	田[た]	た	
\\	今は田に水を入れる季節です。	今[いま]は 田[た]に 水[みず]を 入[い]れる 季節[きせつ]です。	いま は た に みず を いれる きせつ です	
\\	今[いま]は
\\	に 水[みず]を 入[い]れる 季節[きせつ]です。			
\\	花火	花火[はなび]	はなび	
\\	昨日、花火を見に行ったよ。	昨日[きのう]、 花火[はなび]を 見[み]に 行[い]ったよ。	きのう はなび を み に いった よ	
\\	昨日[きのう]、
\\	を 見[み]に 行[い]ったよ。			
\\	男女	男女[だんじょ]	だんじょ	
\\	そのスポーツは男女一緒にします。	そのスポーツは 男女[だんじょ] 一緒[いっしょ]にします。	その すぽーつ は だんじょ いっしょ に します	
\\	そのスポーツは
\\	一緒[いっしょ]にします。			
\\	男子	男子[だんし]	だんし	
\\	男子はこっちに並んで。	男子[だんし]はこっちに 並[なら]んで。	だんし は こっち に ならんで	
\\	はこっちに 並[なら]んで。			
\\	私	私[わたくし]	わたくし	
\\	私は北海道の出身です。	私[わたくし]は 北海道[ほっかいどう]の 出身[しゅっしん]です。	わたくし は ほっかいどう の しゅっしん です	
\\	は 北海道[ほっかいどう]の 出身[しゅっしん]です。			
\\	門	門[もん]	もん	
\\	8時に学校の門が開きます。	
\\	時[じ]に 学校[がっこう]の 門[もん]が 開[ひら]きます。	
\\	じ に がっこう の もん が ひらきます	
\\	時[じ]に 学校[がっこう]の
\\	が 開[ひら]きます。			
\\	開ける	開[ひら]ける	ひらける	
\\	霧が晴れて視界が開けたね。	霧[きり]が 晴[は]れて 視界[しかい]が 開[ひら]けたね。	きり が はれて しかい が ひらけた ね	
\\	霧[きり]が 晴[は]れて 視界[しかい]が
\\	ね。			
\\	間	間[ま]	ま	
\\	彼は少し間を置いてから返事した。	彼[かれ]は 少[すこ]し 間[ま]を 置[お]いてから 返事[へんじ]した。	かれ は すこし ま を おいて から へんじ した	
\\	彼[かれ]は 少[すこ]し
\\	を 置[お]いてから 返事[へんじ]した。			
\\	高まる	高[たか]まる	たかまる	
\\	その計画に対する反対の声が高まっているわ。	その 計画[けいかく]に 対[たい]する 反対[はんたい]の 声[こえ]が 高[たか]まっているわ。	その けいかく に たいする はんたい の こえ が たかまって いる わ	
\\	その 計画[けいかく]に 対[たい]する 反対[はんたい]の 声[こえ]が
\\	わ。			
\\	高める	高[たか]める	たかめる	
\\	自分を高めるのは大切なことです。	自分[じぶん]を 高[たか]めるのは 大切[たいせつ]なことです。	じぶん を たかめる の は たいせつ な こと です	
\\	自分[じぶん]を
\\	のは 大切[たいせつ]なことです。			
\\	飲み水	飲[の]み 水[みず]	のみみず	
\\	昔は川の水が飲み水でした。	昔[むかし]は 川[かわ]の 水[みず]が 飲[の]み 水[みず]でした。	むかし は かわ の みず が のみみず でした	
\\	昔[むかし]は 川[かわ]の 水[みず]が
\\	でした。			
\\	飯	飯[めし]	めし	
\\	今朝は飯を食べたかい。	今朝[けさ]は 飯[めし]を 食[た]べたかい。	けさ は めし を たべた かい	
\\	今朝[けさ]は
\\	を 食[た]べたかい。			
\\	見上げる	見上[みあ]げる	みあげる	
\\	私は空を見上げたの。	私[わたし]は 空[そら]を 見上[みあ]げたの。	わたし は そら を みあげた の	
\\	私[わたし]は 空[そら]を
\\	の。			
\\	見下ろす	見下[みお]ろす	みおろす	
\\	山の頂上からふもとを見下ろしたの。	山[やま]の 頂上[ちょうじょう]からふもとを 見下[みお]ろしたの。	やま の ちょうじょう から ふもと を みおろした の	
\\	山[やま]の 頂上[ちょうじょう]からふもとを
\\	の。			
\\	見出し	見出[みだ]し	みだし	
\\	新聞の見出しが彼の目を引いたんだ。	新聞[しんぶん]の 見出[みだ]しが 彼[かれ]の 目[め]を 引[ひ]いたんだ。	しんぶん の みだし が かれ の め を ひいた ん だ	
\\	新聞[しんぶん]の
\\	が 彼[かれ]の 目[め]を 引[ひ]いたんだ。			
\\	見本	見本[みほん]	みほん	
\\	見本はこちらにございます。	見本[みほん]はこちらにございます。	みほん は こちら に ございます	
\\	はこちらにございます。			
\\	見つめる	見[み]つめる	みつめる	
\\	彼はその絵をずっと見つめていたの。	彼[かれ]はその 絵[え]をずっと 見[み]つめていたの。	かれ は その え を ずっと みつめていた の 。	
\\	彼[かれ]はその 絵[え]をずっと
\\	の。			
\\	見晴らし	見晴[みは]らし	みはらし	
\\	このベランダは見晴らしがいい。	このベランダは 見晴[みは]らしがいい。	この べらんだ は みはらし が いい	
\\	このベランダは
\\	がいい。			
\\	見かけ	見[み]かけ	みかけ	
\\	彼は見かけによらず優しいね。	彼[かれ]は 見[み]かけによらず 優[やさ]しいね。	かれ は みかけ に よらず やさしい ね	
\\	彼[かれ]は
\\	によらず 優[やさ]しいね。			
\\	読書	読書[どくしょ]	どくしょ	
\\	私の趣味は読書です。	私[わたし]の 趣味[しゅみ]は 読書[どくしょ]です。	わたし の しゅみ は どくしょ です	
\\	私[わたし]の 趣味[しゅみ]は
\\	です。			
\\	読み書き	読[よ]み 書[か]き	よみかき	
\\	息子は学校で読み書きを勉強しています。	息子[むすこ]は 学校[がっこう]で 読[よ]み 書[か]きを 勉強[べんきょう]しています。	むすこ は がっこう で よみかき を べんきょう して います	
\\	息子[むすこ]は 学校[がっこう]で
\\	を 勉強[べんきょう]しています。			
\\	目覚ましい	目覚[めざ]ましい	めざましい	
\\	あの国は目覚ましい発展を遂げたの。	あの 国[くに]は 目覚[めざ]ましい 発展[はってん]を 遂[と]げたの。	あの くに は めざましい はってん を とげた の	
\\	あの 国[くに]は
\\	発展[はってん]を 遂[と]げたの。			
\\	見覚え	見覚[みおぼ]え	みおぼえ	
\\	この人に見覚えがありますか。	この 人[ひと]に 見覚[みおぼ]えがありますか。	この ひと に みおぼえ が あります か	
\\	この 人[ひと]に
\\	がありますか。			
\\	目覚まし	目覚[めざ]まし	めざまし	
\\	目覚ましにコーヒーを飲んだの。	目覚[めざ]ましにコーヒーを 飲[の]んだの。	めざまし に こーひー を のんだ の	
\\	にコーヒーを 飲[の]んだの。			
\\	閉会	閉会[へいかい]	へいかい	
\\	会長が閉会の挨拶をしました。	会長[かいちょう]が 閉会[へいかい]の 挨拶[あいさつ]をしました。	かいちょう が へいかい の あいさつ を しました	
\\	会長[かいちょう]が
\\	の 挨拶[あいさつ]をしました。			
\\	話し合い	話[はな]し 合[あ]い	はなしあい	
\\	プロジェクトメンバーと話し合いをしたの。	プロジェクトメンバーと 話[はな]し 合[あ]いをしたの。	ぷろじぇくと めんばー と はなしあい を した の	
\\	プロジェクトメンバーと
\\	をしたの。			
\\	見合い	見合[みあ]い	みあい	
\\	うちの両親はお見合い結婚でした。	うちの 両親[りょうしん]はお 見合[みあ]い 結婚[けっこん]でした。	うち の りょうしん は おみあい けっこん でした	
\\	うちの 両親[りょうしん]はお
\\	結婚[けっこん]でした。			
\\	間に合わせる	間[ま]に 合[あ]わせる	まにあわせる	
\\	昼食はクッキーで間に合わせたの。	昼食[ちゅうしょく]はクッキーで 間[ま]に 合[あ]わせたの。	ちゅうしょく は くっきー で まにあわせた の	
\\	昼食[ちゅうしょく]はクッキーで
\\	の。			
\\	満足	満足[まんぞく]	まんぞく	
\\	彼は結果に満足したようです。	彼[かれ]は 結果[けっか]に 満足[まんぞく]したようです。	かれ は けっか に まんぞく した よう です	
\\	彼[かれ]は 結果[けっか]に
\\	したようです。			
\\	満たす	満[み]たす	みたす	
\\	彼は応募の条件を満たしていない。	彼[かれ]は 応募[おうぼ]の 条件[じょうけん]を 満[み]たしていない。	かれ は おうぼ の じょうけん を みたして いない	
\\	彼[かれ]は 応募[おうぼ]の 条件[じょうけん]を
\\	いない。			
\\	満ちる	満[み]ちる	みちる	
\\	月が満ちてきましたね。	月[つき]が 満[み]ちてきましたね。	つき が みちて きました ね	
\\	月[つき]が
\\	ね。			
\\	満員	満員[まんいん]	まんいん	
\\	このバスは満員です。	このバスは 満員[まんいん]です。	この ばす は まんいん です	
\\	このバスは
\\	です。			
\\	満月	満月[まんげつ]	まんげつ	
\\	今日は満月ですね。	今日[きょう]は 満月[まんげつ]ですね。	きょう は まんげつ です ね	
\\	今日[きょう]は
\\	ですね。			
\\	見事	見事[みごと]	みごと	
\\	彼は見事なジャンプを見せたね。	彼[かれ]は 見事[みごと]なジャンプを 見[み]せたね。	かれ は みごと な じゃんぷ を みせた ね	
\\	彼[かれ]は
\\	なジャンプを 見[み]せたね。			
\\	電力	電力[でんりょく]	でんりょく	
\\	このエアコンはあまり電力を使いません。	このエアコンはあまり 電力[でんりょく]を 使[つか]いません。	この えあこん は あまり でんりょく を つかいません	
\\	このエアコンはあまり
\\	を 使[つか]いません。			
\\	電子	電子[でんし]	でんし	
\\	電子辞書はとても便利です。	電子[でんし] 辞書[じしょ]はとても 便利[べんり]です。	でんし じしょ は とても べんり です	
\\	辞書[じしょ]はとても 便利[べんり]です。			
\\	見通し	見通[みとお]し	みとおし	
\\	仕事の見通しがたたない。	仕事[しごと]の 見通[みとお]しがたたない。	しごと の みとおし が たたない	
\\	仕事[しごと]の
\\	がたたない。			
\\	通じる	通[つう]じる	つうじる	
\\	その国では英語は通じますか。	その 国[くに]では 英語[えいご]は 通[つう]じますか。	その くに で は えいご は つうじます か	
\\	その 国[くに]では 英語[えいご]は
\\	か。			
\\	通す	通[とお]す	とおす	
\\	針に糸を通してください。	針[はり]に 糸[いと]を 通[とお]してください。	はり に いと を とおして ください	
\\	針[はり]に 糸[いと]を
\\	ください。			
\\	通行	通行[つうこう]	つうこう	
\\	この道は通行できません。	この 道[みち]は 通行[つうこう]できません。	この みち は つうこう できません	
\\	この 道[みち]は
\\	できません。			
\\	通路	通路[つうろ]	つうろ	
\\	通路の右側にトイレがありますよ。	通路[つうろ]の 右側[みぎがわ]にトイレがありますよ。	つうろ の みぎがわ に といれ が あります よ	
\\	の 右側[みぎがわ]にトイレがありますよ。			
\\	通行止め	通行止[つうこうど]め	つうこうどめ	
\\	あの道路は通行止めだそうです。	あの 道路[どうろ]は 通行止[つうこうど]めだそうです。	あの どうろ は つうこうどめ だ そう です	
\\	あの 道路[どうろ]は
\\	だそうです。			
\\	見渡す	見渡[みわた]す	みわたす	
\\	丘の上から草原を見渡したの。	丘[おか]の 上[うえ]から 草原[そうげん]を 見渡[みわた]したの。	おか の うえ から そうげん を みわたした の	
\\	丘[おか]の 上[うえ]から 草原[そうげん]を
\\	の。			
\\	間近	間近[まぢか]	まぢか	
\\	有名人を間近で見たよ。	有名人[ゆうめいじん]を 間近[まぢか]で 見[み]たよ。	ゆうめいじん を まぢか で みた よ	
\\	有名人[ゆうめいじん]を
\\	で 見[み]たよ。			
\\	近道	近道[ちかみち]	ちかみち	
\\	こっちが近道です。	こっちが 近道[ちかみち]です。	こっち が ちかみち です	
\\	こっちが
\\	です。			
\\	長年	長年[ながねん]	ながねん	
\\	彼は長年の友人です。	彼[かれ]は 長年[ながねん]の 友人[ゆうじん]です。	かれ は ながねん の ゆうじん です	
\\	彼[かれ]は
\\	の 友人[ゆうじん]です。			
\\	長らく	長[なが]らく	ながらく	
\\	長らくお待たせしました。	長[なが]らくお 待[ま]たせしました。	ながらく お またせ しました 。	
\\	お 待[ま]たせしました。			
\\	長話	長話[ながばなし]	ながばなし	
\\	母が電話で長話をしているんだ。	母[はは]が 電話[でんわ]で 長話[ながばなし]をしているんだ。	はは が でんわ で ながばなし を して いる ん だ	
\\	母[はは]が 電話[でんわ]で
\\	をしているんだ。			
\\	短大	短大[たんだい]	たんだい	
\\	妹は短大を卒業しました。	妹[いもうと]は 短大[たんだい]を 卒業[そつぎょう]しました。	いもうと は たんだい を そつぎょう しました	
\\	妹[いもうと]は
\\	を 卒業[そつぎょう]しました。			
\\	短気	短気[たんき]	たんき	
\\	彼は短気な人です。	彼[かれ]は 短気[たんき]な 人[ひと]です。	かれ は たんき な ひと です	
\\	彼[かれ]は
\\	な 人[ひと]です。			
\\	部門	部門[ぶもん]	ぶもん	
\\	あの歌手は3部門で賞を取ったの。	あの 歌手[かしゅ]は3 部門[ぶもん]で 賞[しょう]を 取[と]ったの。	あの かしゅ は 
\\	ぶもん で しょう を とった の	
\\	あの 歌手[かしゅ]は3
\\	で 賞[しょう]を 取[と]ったの。			
\\	部長	部長[ぶちょう]	ぶちょう	
\\	部長に仕事の相談をしたんだ。	部長[ぶちょう]に 仕事[しごと]の 相談[そうだん]をしたんだ。	ぶちょう に しごと の そうだん を した ん だ	
\\	に 仕事[しごと]の 相談[そうだん]をしたんだ。			
\\	部下	部下[ぶか]	ぶか	
\\	彼は優秀な部下を持っているわね。	彼[かれ]は 優秀[ゆうしゅう]な 部下[ぶか]を 持[も]っているわね。	かれ は ゆうしゅう な ぶか を もって いる わ ね	
\\	彼[かれ]は 優秀[ゆうしゅう]な
\\	を 持[も]っているわね。			
\\	鉄道	鉄道[てつどう]	てつどう	
\\	日本は鉄道がとても発達しています。	日本[にっぽん]は 鉄道[てつどう]がとても 発達[はったつ]しています。	にっぽん は てつどう が とても はったつ して います	
\\	日本[にっぽん]は
\\	がとても 発達[はったつ]しています。			
\\	鉄	鉄[てつ]	てつ	
\\	この鍋は鉄でできています。	この 鍋[なべ]は 鉄[てつ]でできています。	この なべ は てつ で できて います	
\\	この 鍋[なべ]は
\\	でできています。			
\\	肉体	肉体[にくたい]	にくたい	
\\	肉体はいつか滅びます。	肉体[にくたい]はいつか 滅[ほろ]びます。	にくたい は いつか ほろびます	
\\	はいつか 滅[ほろ]びます。			
\\	白菜	白菜[はくさい]	はくさい	
\\	彼女は白菜の漬物が好きです。	彼女[かのじょ]は 白菜[はくさい]の 漬物[つけもの]が 好[す]きです。	かのじょ は はくさい の つけもの が すき です	
\\	彼女[かのじょ]は
\\	の 漬物[つけもの]が 好[す]きです。			
\\	理解	理解[りかい]	りかい	
\\	理解してくれてありがとう。	理解[りかい]してくれてありがとう。	りかい して くれて ありがとう	
\\	してくれてありがとう。			
\\	解く	解[と]く	とく	
\\	この問題を解けますか。	この 問題[もんだい]を 解[と]けますか。	この もんだい を とけます か	
\\	この 問題[もんだい]を
\\	か。			
\\	解ける	解[と]ける	とける	
\\	やっと難しい問題が解けました。	やっと 難[むずか]しい 問題[もんだい]が 解[と]けました。	やっと むずかしい もんだい が とけました	
\\	やっと 難[むずか]しい 問題[もんだい]が
\\	無理	無理[むり]	むり	
\\	無理はしないでください。	無理[むり]はしないでください。	むり は しない で ください	
\\	はしないでください。			
\\	無料	無料[むりょう]	むりょう	
\\	お飲み物は無料でございます。	お 飲[の]み 物[もの]は 無料[むりょう]でございます。	お のみもの は むりょう で ございます	
\\	お 飲[の]み 物[もの]は
\\	でございます。			
\\	無事	無事[ぶじ]	ぶじ	
\\	無事、家に着きました。	無事[ぶじ]、 家[いえ]に 着[つ]きました。	ぶじ いえ に つきました	
\\	、 家[いえ]に 着[つ]きました。			
\\	無口	無口[むくち]	むくち	
\\	彼女は無口な人ですね。	彼女[かのじょ]は 無口[むくち]な 人[ひと]ですね。	かのじょ は むくち な ひと です ね	
\\	彼女[かのじょ]は
\\	な 人[ひと]ですね。			
\\	間も無く	間[ま]も 無[な]く	まもなく	
\\	あの飛行機は間も無く着陸しますね。	あの 飛行機[ひこうき]は 間[ま]も 無[な]く 着陸[ちゃくりく]しますね。	あの ひこうき は まもなく ちゃくりく します ね	
\\	あの 飛行機[ひこうき]は
\\	着陸[ちゃくりく]しますね。			
\\	無言	無言[むごん]	むごん	
\\	彼は一日中無言だったな。	彼[かれ]は 一日中[いちにちじゅう] 無言[むごん]だったな。	かれ は いちにちじゅう むごん だった な	
\\	彼[かれ]は 一日中[いちにちじゅう]
\\	だったな。			
\\	無茶	無茶[むちゃ]	むちゃ	
\\	無茶をしないでくださいね。	無茶[むちゃ]をしないでくださいね。	むちゃ を しない で ください ね	
\\	をしないでくださいね。			
\\	無理やり	無理[むり]やり	むりやり	
\\	荷物を無理やりかばんに詰めたよ。	荷物[にもつ]を 無理[むり]やりかばんに 詰[つ]めたよ。	にもつ を むりやり かばん に つめた よ	
\\	荷物[にもつ]を
\\	かばんに 詰[つ]めたよ。			
\\	用	用[よう]	よう	
\\	私に何か用ですか。	私[わたし]に 何[なに]か 用[よう]ですか。	わたし に なにか よう です か	
\\	私[わたし]に 何[なに]か
\\	ですか。			
\\	無用	無用[むよう]	むよう	
\\	心配は無用です。	心配[しんぱい]は 無用[むよう]です。	しんぱい は むよう です	
\\	心配[しんぱい]は
\\	です。			
\\	費用	費用[ひよう]	ひよう	
\\	イタリア旅行の費用は30万円です。	イタリア 旅行[りょこう]の 費用[ひよう]は30 万円[まんえん]です。	いたりあ りょこう の ひよう は 
\\	まんえん です	
\\	イタリア 旅行[りょこう]の
\\	は30 万円[まんえん]です。			
\\	費やす	費[つい]やす	ついやす	
\\	私は語学の勉強にかなりの時間を費やしています。	私[わたし]は 語学[ごがく]の 勉強[べんきょう]にかなりの 時間[じかん]を 費[つい]やしています。	わたし は ごがく の べんきょう に かなり の じかん を ついやして います	
\\	私[わたし]は 語学[ごがく]の 勉強[べんきょう]にかなりの 時間[じかん]を
\\	閉店	閉店[へいてん]	へいてん	
\\	この店は8時に閉店します。	この 店[みせ]は8 時[じ]に 閉店[へいてん]します。	この みせ は 
\\	じ に へいてん します	
\\	この 店[みせ]は8 時[じ]に
\\	します。			
\\	部品	部品[ぶひん]	ぶひん	
\\	車の部品を取り替えたんだ。	車[くるま]の 部品[ぶひん]を 取[と]り 替[か]えたんだ。	くるま の ぶひん を とりかえた ん だ	
\\	車[くるま]の
\\	を 取[と]り 替[か]えたんだ。			
\\	熱する	熱[ねっ]する	ねっする	
\\	鉄は熱すると曲がるんだ。	鉄[てつ]は 熱[ねっ]すると 曲[ま]がるんだ。	てつ は ねっする と まがる ん だ	
\\	鉄[てつ]は
\\	と 曲[ま]がるんだ。			
\\	電報	電報[でんぽう]	でんぽう	
\\	実家の母から電報が来ました。	実家[じっか]の 母[はは]から 電報[でんぽう]が 来[き]ました。	じっか の はは から でんぽう が きました	
\\	実家[じっか]の 母[はは]から
\\	が 来[き]ました。			
\\	熱心	熱心[ねっしん]	ねっしん	
\\	彼女は4年間熱心に勉強したわ。	彼女[かのじょ]は4 年間[ねんかん] 熱心[ねっしん]に 勉強[べんきょう]したわ。	かのじょ は 
\\	ねんかん ねっしん に べんきょう した わ	
\\	彼女[かのじょ]は4 年間[ねんかん]
\\	に 勉強[べんきょう]したわ。			
\\	良心	良心[りょうしん]	りょうしん	
\\	私は良心に従って行動します。	私[わたし]は 良心[りょうしん]に 従[したが]って 行動[こうどう]します。	わたし は りょうしん に したがって こうどう します	
\\	私[わたし]は
\\	に 従[したが]って 行動[こうどう]します。			
\\	用心	用心[ようじん]	ようじん	
\\	夜道は用心して歩きましょう。	夜道[よみち]は 用心[ようじん]して 歩[ある]きましょう。	よみち は ようじん して あるきましょう	
\\	夜道[よみち]は
\\	して 歩[ある]きましょう。			
\\	知事	知事[ちじ]	ちじ	
\\	彼は有能な知事ね。	彼[かれ]は 有能[ゆうのう]な 知事[ちじ]ね。	かれ は ゆうのう な ちじ ね	
\\	彼[かれ]は 有能[ゆうのう]な
\\	ね。			
\\	通知	通知[つうち]	つうち	
\\	明日、詳細を通知します。	明日[あす]、 詳細[しょうさい]を 通知[つうち]します。	あす、 しょうさい を つうち します	
\\	明日[あす]、 詳細[しょうさい]を
\\	します。			
\\	無知	無知[むち]	むち	
\\	彼女は政治について無知でした。	彼女[かのじょ]は 政治[せいじ]について 無知[むち]でした。	かのじょ は せいじ に ついて むち でした	
\\	彼女[かのじょ]は 政治[せいじ]について
\\	でした。			
\\	知人	知人[ちじん]	ちじん	
\\	彼は昔からの知人です。	彼[かれ]は 昔[むかし]からの 知人[ちじん]です。	かれ は むかし から の ちじん です	
\\	彼[かれ]は 昔[むかし]からの
\\	です。			
\\	知能	知能[ちのう]	ちのう	
\\	あの子の知能はとても高いそうです。	あの 子[こ]の 知能[ちのう]はとても 高[たか]いそうです。	あの こ の ちのう は とても たかい そう です	
\\	あの 子[こ]の
\\	はとても 高[たか]いそうです。			
\\	無能	無能[むのう]	むのう	
\\	彼は無能だ。	彼[かれ]は 無能[むのう]だ。	かれ は むのう だ	
\\	彼[かれ]は
\\	だ。			
\\	能	能[のう]	のう	
\\	先日、初めて能を見に行きました。	先日[せんじつ]、 初[はじ]めて 能[のう]を 見[み]に 行[い]きました。	せんじつ はじめて のう を み に いきました	
\\	先日[せんじつ]、 初[はじ]めて
\\	を 見[み]に 行[い]きました。			
\\	能	能[のう]	のう	
\\	彼は勉強するしか能のない人間だね。	彼[かれ]は 勉強[べんきょう]するしか 能[のう]のない 人間[にんげん]だね。	かれ は べんきょう する しか のう の ない にんげん だ ね	
\\	彼[かれ]は 勉強[べんきょう]するしか
\\	のない 人間[にんげん]だね。			
\\	番	番[ばん]	ばん	
\\	今日は私が皿を洗う番ですね。	今日[きょう]は 私[わたし]が 皿[さら]を 洗[あら]う 番[ばん]ですね。	きょう は わたし が さら を あらう ばん です ね	
\\	今日[きょう]は 私[わたし]が 皿[さら]を 洗[あら]う
\\	ですね。			
\\	長所	長所[ちょうしょ]	ちょうしょ	
\\	君の長所は明るいところだね。	君[きみ]の 長所[ちょうしょ]は 明[あか]るいところだね。	きみ の ちょうしょ は あかるい ところ だ ね	
\\	君[きみ]の
\\	は 明[あか]るいところだね。			
\\	短所	短所[たんしょ]	たんしょ	
\\	すぐにあきらめてしまうのが彼の短所です。	すぐにあきらめてしまうのが 彼[かれ]の 短所[たんしょ]です。	すぐ に あきらめて しまう の が かれ の たんしょ です	
\\	すぐにあきらめてしまうのが 彼[かれ]の
\\	です。			
\\	町外れ	町外[まちはず]れ	まちはずれ	
\\	彼女は町外れの工場で働いているよ。	彼女[かのじょ]は 町外[まちはず]れの 工場[こうじょう]で 働[はたら]いているよ。	かのじょ は まちはずれ の こうじょう で はたらいて いる よ	
\\	彼女[かのじょ]は
\\	の 工場[こうじょう]で 働[はたら]いているよ。			
\\	町中	町中[まちなか]	まちなか	
\\	町中で彼女に出会ったよ。	町中[まちなか]で 彼女[かのじょ]に 出会[であ]ったよ。	まちなか で かのじょ に であった よ	
\\	で 彼女[かのじょ]に 出会[であ]ったよ。			
\\	都内	都内[とない]	とない	
\\	彼の職場は都内にあります。	彼[かれ]の 職場[しょくば]は 都内[とない]にあります。	かれ の しょくば は とない に あります	
\\	彼[かれ]の 職場[しょくば]は
\\	にあります。			
\\	都心	都心[としん]	としん	
\\	彼は都心に住んでいます。	彼[かれ]は 都心[としん]に 住[す]んでいます。	かれ は としん に すんで います	
\\	彼[かれ]は
\\	に 住[す]んでいます。			
\\	都	都[と]	と	
\\	彼女は都の職員です。	彼女[かのじょ]は 都[と]の 職員[しょくいん]です。	かのじょ は と の しょくいん です	
\\	彼女[かのじょ]は
\\	の 職員[しょくいん]です。			
\\	都会	都会[とかい]	とかい	
\\	彼は都会での生活を楽しんでいるの。	彼[かれ]は 都会[とかい]での 生活[せいかつ]を 楽[たの]しんでいるの。	かれ は とかい で の せいかつ を たのしんで いる の	
\\	彼[かれ]は
\\	での 生活[せいかつ]を 楽[たの]しんでいるの。			
\\	都	都[みやこ]	みやこ	
\\	ミラノはファッションの都です。	ミラノはファッションの 都[みやこ]です。	みらの は ふぁっしょん の みやこ です	
\\	ミラノはファッションの
\\	です。			
\\	物価	物価[ぶっか]	ぶっか	
\\	この国の物価はここ10年でだいぶ上がったね。	この 国[くに]の 物価[ぶっか]はここ10 年[ねん]でだいぶ 上[あ]がったね。	この くに の ぶっか は ここ 
\\	ねん で だいぶ あがった ね	
\\	この 国[くに]の
\\	はここ10 年[ねん]でだいぶ 上[あ]がったね。			
\\	物体	物体[ぶったい]	ぶったい	
\\	謎の物体が空を飛んでいます。	謎[なぞ]の 物体[ぶったい]が 空[そら]を 飛[と]んでいます。	なぞ の ぶったい が そら を とんで います	
\\	謎[なぞ]の
\\	が 空[そら]を 飛[と]んでいます。			
\\	物理	物理[ぶつり]	ぶつり	
\\	彼は物理を専攻している。	彼[かれ]は 物理[ぶつり]を 専攻[せんこう]している。	かれ は ぶつり を せんこう して いる	
\\	彼[かれ]は
\\	を 専攻[せんこう]している。			
\\	物事	物事[ものごと]	ものごと	
\\	彼はいつも物事を深く考えるの。	彼[かれ]はいつも 物事[ものごと]を 深[ふか]く 考[かんが]えるの。	かれ は いつも ものごと を ふかく かんがえる の	
\\	彼[かれ]はいつも
\\	を 深[ふか]く 考[かんが]えるの。			
\\	物知り	物知[ものし]り	ものしり	
\\	あの老人はとても物知りだね。	あの 老人[ろうじん]はとても 物知[ものし]りだね。	あの ろうじん は とても ものしり だ ね	
\\	あの 老人[ろうじん]はとても
\\	だね。			
\\	物覚え	物覚[ものおぼ]え	ものおぼえ	
\\	彼女は物覚えがいい。	彼女[かのじょ]は 物覚[ものおぼ]えがいい。	かのじょ は ものおぼえ が いい	
\\	彼女[かのじょ]は
\\	がいい。			
\\	量る	量[はか]る	はかる	
\\	はかりで小麦粉の分量を量ったよ。	はかりで 小麦粉[こむぎこ]の 分量[ぶんりょう]を 量[はか]ったよ。	はかり で こむぎこ の ぶんりょう を はかった よ	
\\	はかりで 小麦粉[こむぎこ]の 分量[ぶんりょう]を
\\	よ。			
\\	量	量[りょう]	りょう	
\\	最近、食事の量を減らしています。	最近[さいきん]、 食事[しょくじ]の 量[りょう]を 減[へ]らしています。	さいきん しょくじ の りょう を へらして います	
\\	最近[さいきん]、 食事[しょくじ]の
\\	を 減[へ]らしています。			
\\	送金	送金[そうきん]	そうきん	
\\	取引先への送金を済ませました。	取引先[とりひきさき]への 送金[そうきん]を 済[す]ませました。	とりひきさき へ の そうきん を すませました	
\\	取引先[とりひきさき]への
\\	を 済[す]ませました。			
\\	郵送	郵送[ゆうそう]	ゆうそう	
\\	書類を郵送してください。	書類[しょるい]を 郵送[ゆうそう]してください。	しょるい を ゆうそう して ください	
\\	書類[しょるい]を
\\	してください。			
\\	長持ち	長持[ながも]ち	ながもち	
\\	このコートは長持ちしている。	このコートは 長持[ながも]ちしている。	この こーと は ながもち して いる	
\\	このコートは
\\	している。			
\\	生える	生[は]える	はえる	
\\	息子に新しい歯が生えました。	息子[むすこ]に 新[あたら]しい 歯[は]が 生[は]えました。	むすこ に あたらしい は が はえました	
\\	息子[むすこ]に 新[あたら]しい 歯[は]が
\\	長生き	長生[ながい]き	ながいき	
\\	この村の人たちは長生きです。	この 村[むら]の 人[ひと]たちは 長生[ながい]きです。	この むら の ひとたち は ながいき です	
\\	この 村[むら]の 人[ひと]たちは
\\	です。			
\\	生やす	生[は]やす	はやす	
\\	彼はヒゲを生やしています。	彼[かれ]はヒゲを 生[は]やしています。	かれ は ひげ を はやして います	
\\	彼[かれ]はヒゲを
\\	生	生[なま]	なま	
\\	彼は生の魚が食べられません。	彼[かれ]は 生[なま]の 魚[さかな]が 食[た]べられません。	かれ は なま の さかな が たべられません	
\\	彼[かれ]は
\\	の 魚[さかな]が 食[た]べられません。			
\\	無生物	無生物[むせいぶつ]	むせいぶつ	
\\	岩は無生物に分類される。	岩[いわ]は 無生物[むせいぶつ]に 分類[ぶんるい]される。	いわ は むせいぶつ に ぶんるい される	
\\	岩[いわ]は
\\	に 分類[ぶんるい]される。			
\\	男性	男性[だんせい]	だんせい	
\\	彼は素敵な男性です。	彼[かれ]は 素敵[すてき]な 男性[だんせい]です。	かれ は すてき な だんせい です	
\\	彼[かれ]は 素敵[すてき]な
\\	です。			
\\	理性	理性[りせい]	りせい	
\\	人間は理性を持つ動物です。	人間[にんげん]は 理性[りせい]を 持[も]つ 動物[どうぶつ]です。	にんげん は りせい を もつ どうぶつ です	
\\	人間[にんげん]は
\\	を 持[も]つ 動物[どうぶつ]です。			
\\	長引く	長引[ながび]く	ながびく	
\\	会議が長引いているようです。	会議[かいぎ]が 長引[ながび]いているようです。	かいぎ が ながびいて いる よう です	
\\	会議[かいぎ]が
\\	いるようです。			
\\	見習う	見習[みなら]う	みならう	
\\	彼を見習ってもっと勉強します。	彼[かれ]を 見習[みなら]ってもっと 勉強[べんきょう]します。	かれ を みならって もっと べんきょう します	
\\	彼[かれ]を
\\	もっと 勉強[べんきょう]します。			
\\	試す	試[ため]す	ためす	
\\	彼はそのソフトウェアを試したの。	彼[かれ]はそのソフトウェアを 試[ため]したの。	かれ は その そふとうぇあ を ためした の	
\\	彼[かれ]はそのソフトウェアを
\\	の。			
\\	試し	試[ため]し	ためし	
\\	試しにこの曲を弾いてみてください。	試[ため]しにこの 曲[きょく]を 弾[ひ]いてみてください。	ためし に この きょく を ひいて みて ください	
\\	にこの 曲[きょく]を 弾[ひ]いてみてください。			
\\	物質	物質[ぶっしつ]	ぶっしつ	
\\	この製品は有害な化学物質を含んでいるわよ。	この 製品[せいひん]は 有害[ゆうがい]な 化学[かがく] 物質[ぶっしつ]を 含[ふく]んでいるわよ。	この せいひん は ゆうがい な かがくぶっしつ を ふくんで いる わ よ	
\\	この 製品[せいひん]は 有害[ゆうがい]な 化学[かがく]
\\	を 含[ふく]んでいるわよ。			
\\	話題	話題[わだい]	わだい	
\\	ここが話題のレストランです。	ここが 話題[わだい]のレストランです。	ここ が わだい の れすとらん です	
\\	ここが
\\	のレストランです。			
\\	題名	題名[だいめい]	だいめい	
\\	この本の題名は「坊ちゃん」です。	この 本[ほん]の 題名[だいめい]は
\\	坊[ぼっ]ちゃん」です。	この ほん の だいめい は ぼっちゃん です	
\\	この 本[ほん]の
\\	は
\\	坊[ぼっ]ちゃん」です。			
\\	題	題[だい]	だい	
\\	その曲の題を思い出せません。	その 曲[きょく]の 題[だい]を 思[おも]い 出[だ]せません。	その きょく の だい を おもいだせません	
\\	その 曲[きょく]の
\\	を 思[おも]い 出[だ]せません。			
\\	満点	満点[まんてん]	まんてん	
\\	国語のテストは満点でした。	国語[こくご]のテストは 満点[まんてん]でした。	こくご の てすと は まんてん でした	
\\	国語[こくご]のテストは
\\	でした。			
\\	点	点[てん]	てん	
\\	その点は心配ありません。	その 点[てん]は 心配[しんぱい]ありません。	その てん は しんぱい ありません	
\\	その
\\	は 心配[しんぱい]ありません。			
\\	複数	複数[ふくすう]	ふくすう	
\\	複数のレポートをひとつにまとめています。	複数[ふくすう]のレポートをひとつにまとめています。	ふくすう の れぽーと を ひとつ に まとめて います	
\\	のレポートをひとつにまとめています。			
\\	無数	無数[むすう]	むすう	
\\	宇宙には無数の星があります。	宇宙[うちゅう]には 無数[むすう]の 星[ほし]があります。	うちゅう に は むすう の ほし が あります	
\\	宇宙[うちゅう]には
\\	の 星[ほし]があります。			
\\	点数	点数[てんすう]	てんすう	
\\	今回はテストの点数が悪かった。	今回[こんかい]はテストの 点数[てんすう]が 悪[わる]かった。	こんかい は てすと の てんすう が わるかった	
\\	今回[こんかい]はテストの
\\	が 悪[わる]かった。			
\\	負け	負[ま]け	まけ	
\\	この勝負はあなたの負けです。	この 勝負[しょうぶ]はあなたの 負[ま]けです。	この しょうぶ は あなた の まけ です	
\\	この 勝負[しょうぶ]はあなたの
\\	です。			
\\	負担	負担[ふたん]	ふたん	
\\	交通費は個人負担です。	交通費[こうつうひ]は 個人[こじん] 負担[ふたん]です。	こうつうひ は こじん ふたん です	
\\	交通費[こうつうひ]は 個人[こじん]
\\	です。			
\\	経る	経[へ]る	へる	
\\	彼は新聞記者を経て作家になりました。	彼[かれ]は 新聞記者[しんぶん きしゃ]を 経[へ]て 作家[さっか]になりました。	かれ は しんぶん きしゃ を へて さっか に なりました	
\\	彼[かれ]は 新聞記者[しんぶん きしゃ]を
\\	作家[さっか]になりました。			
\\	買収	買収[ばいしゅう]	ばいしゅう	
\\	彼は買収されたらしいわ。	彼[かれ]は 買収[ばいしゅう]されたらしいわ。	かれ は ばいしゅう された らしい わ	
\\	彼[かれ]は
\\	されたらしいわ。			
\\	要する	要[よう]する	ようする	
\\	このビルは完成までに2年を要した。	このビルは 完成[かんせい]までに2 年[ねん]を 要[よう]した。	この びる は かんせい まで に 
\\	ねん を ようした	
\\	このビルは 完成[かんせい]までに2 年[ねん]を
\\	要するに	要[よう]するに	ようするに	
\\	要するに時機を待つべきだ。	要[よう]するに 時機[じき]を 待[ま]つべきだ。	ようするに じき を まつべき だ	
\\	時機[じき]を 待[ま]つべきだ。			
\\	要点	要点[ようてん]	ようてん	
\\	話の要点だけ教えてください。	話[はなし]の 要点[ようてん]だけ 教[おし]えてください。	はなし の ようてん だけ おしえて ください	
\\	話[はなし]の
\\	だけ 教[おし]えてください。			
\\	要請	要請[ようせい]	ようせい	
\\	その国の政府は各国に支援を要請したのよ。	その 国[くに]の 政府[せいふ]は 各国[かっこく]に 支援[しえん]を 要請[ようせい]したのよ。	その くに の せいふ は かっこく に しえん を ようせい した の よ	
\\	その 国[くに]の 政府[せいふ]は 各国[かっこく]に 支援[しえん]を
\\	したのよ。			
\\	額	額[ひたい]	ひたい	
\\	額に汗をかいたよ。	額[ひたい]に 汗[あせ]をかいたよ。	ひたい に あせ を かいた よ	
\\	に 汗[あせ]をかいたよ。			
\\	計る	計[はか]る	はかる	
\\	100メートル走のタイムを計ったんだ。	100メートル 走[そう]のタイムを 計[はか]ったんだ。	
\\	めーとるそう の たいむ を はかった ん だ	
\\	100メートル 走[そう]のタイムを
\\	んだ。			
\\	物差し	物差[ものさ]し	ものさし	
\\	30センチの物差しをください。	30センチの 物差[ものさ]しをください。	
\\	せんち の ものさし を ください	
\\	30センチの
\\	をください。			
\\	申し出る	申[もう]し 出[で]る	もうしでる	
\\	彼はプロジェクトへの参加を申し出たよ。	彼[かれ]はプロジェクトへの 参加[さんか]を 申[もう]し 出[で]たよ。	かれ は ぷろじぇくと へ の さんか を もうしでた よ	
\\	彼[かれ]はプロジェクトへの 参加[さんか]を
\\	よ。			
\\	申し上げる	申[もう]し 上[あ]げる	もうしあげる	
\\	結果を申し上げます。	結果[けっか]を 申[もう]し 上[あ]げます。	けっか を もうしあげます	
\\	結果[けっか]を
\\	申す	申[もう]す	もうす	
\\	私は鈴木と申します。	私[わたくし]は 鈴木[すずき]と 申[もう]します。	わたくし は すずき と もうします	
\\	私[わたくし]は 鈴木[すずき]と
\\	見込み	見込[みこ]み	みこみ	
\\	3月に大学を卒業の見込みです。	
\\	月[がつ]に 大学[だいがく]を 卒業[そつぎょう]の 見込[みこ]みです。	
\\	がつ に だいがく を そつぎょう の みこみ です	
\\	月[がつ]に 大学[だいがく]を 卒業[そつぎょう]の
\\	です。			
\\	申し込み	申[もう]し 込[こ]み	もうしこみ	
\\	今日、スポーツジムの申し込みをしました。	今日[きょう]、スポーツジムの 申[もう]し 込[こ]みをしました。	きょう すぽーつじむ の もうしこみ を しました	
\\	今日[きょう]、スポーツジムの
\\	をしました。			
\\	飲み込む	飲[の]み 込[こ]む	のみこむ	
\\	彼は薬を一気に飲み込んだ。	彼[かれ]は 薬[くすり]を 一気[いっき]に 飲[の]み 込[こ]んだ。	かれ は くすり を いっきに のみこんだ	
\\	彼[かれ]は 薬[くすり]を 一気[いっき]に
\\	長期	長期[ちょうき]	ちょうき	
\\	今回は長期の滞在です。	今回[こんかい]は 長期[ちょうき]の 滞在[たいざい]です。	こんかい は ちょうき の たいざい です	
\\	今回[こんかい]は
\\	の 滞在[たいざい]です。			
\\	短期	短期[たんき]	たんき	
\\	明日から2週間、短期のアルバイトをします。	明日[あした]から2 週間[しゅうかん]、 短期[たんき]のアルバイトをします。	あした から 
\\	しゅうかん たんき の あるばいと を します	
\\	明日[あした]から2 週間[しゅうかん]、
\\	のアルバイトをします。			
\\	短期大学	短期大学[たんきだいがく]	たんきだいがく	
\\	彼女は短期大学で日本文学を勉強しました。	彼女[かのじょ]は 短期大学[たんきだいがく]で 日本文学[にほん ぶんがく]を 勉強[べんきょう]しました。	かのじょ は たんきだいがく で にほん ぶんがく を べんきょう しました	
\\	彼女[かのじょ]は
\\	で 日本文学[にほん ぶんがく]を 勉強[べんきょう]しました。			
\\	無限	無限[むげん]	むげん	
\\	資源は無限ではありません。	資源[しげん]は 無限[むげん]ではありません。	しげん は むげん で は ありません	
\\	資源[しげん]は
\\	ではありません。			
\\	無制限	無制限[むせいげん]	むせいげん	
\\	このサイトでは音楽を無制限でダウンロードできる。	このサイトでは 音楽[おんがく]を 無制限[むせいげん]でダウンロードできる。	この さいと で は おんがく を むせいげん で だうんろーど できる	
\\	このサイトでは 音楽[おんがく]を
\\	でダウンロードできる。			
\\	目指す	目指[めざ]す	めざす	
\\	私は料理人を目指しています。	私[わたし]は 料理人[りょうりにん]を 目指[めざ]しています。	わたし は りょうりにん を めざして います	
\\	私[わたし]は 料理人[りょうりにん]を
\\	花束	花束[はなたば]	はなたば	
\\	卒業式に花束をもらいました。	卒業式[そつぎょうしき]に 花束[はなたば]をもらいました。	そつぎょうしき に はなたば を もらいました	
\\	卒業式[そつぎょうしき]に
\\	をもらいました。			
\\	着ける	着[つ]ける	つける	
\\	玄関に車を着けます。	玄関[げんかん]に 車[くるま]を 着[つ]けます。	げんかん に くるま を つけます	
\\	玄関[げんかん]に 車[くるま]を
\\	立ち上がる	立[た]ち 上[あ]がる	たちあがる	
\\	彼は急に立ち上がったの。	彼[かれ]は 急[きゅう]に 立[た]ち 上[あ]がったの。	かれ は きゅう に たちあがった の	
\\	彼[かれ]は 急[きゅう]に
\\	の。			
\\	都立	都立[とりつ]	とりつ	
\\	ここは都立の病院です。	ここは 都立[とりつ]の 病院[びょういん]です。	ここ は とりつ の びょういん です	
\\	ここは
\\	の 病院[びょういん]です。			
\\	立て込む	立[た]て 込[こ]む	たてこむ	
\\	この辺は住宅が立て込んで います。	この 辺[へん]は 住宅[じゅうたく]が 立[た]て 込[こ]んでいます。	このへん は じゅうたく が たてこんで います	
\\	この 辺[へん]は 住宅[じゅうたく]が
\\	います。			
\\	立ち止まる	立[た]ち 止[ど]まる	たちどまる	
\\	人々は立ち止まって上を見上げたの。	人々[ひとびと]は 立[た]ち 止[ど]まって 上[うえ]を 見上[みあ]げたの。	ひとびと は たちどまって うえ を みあげた の	
\\	人々[ひとびと]は
\\	上[うえ]を 見上[みあ]げたの。			
\\	着席	着席[ちゃくせき]	ちゃくせき	
\\	みなさん、着席してください。	みなさん、 着席[ちゃくせき]してください。	みなさん ちゃくせき して ください	
\\	みなさん、
\\	してください。			
\\	目次	目次[もくじ]	もくじ	
\\	読みたい章を目次で探しました。	読[よ]みたい 章[しょう]を 目次[もくじ]で 探[さが]しました。	よみたい しょう を もくじ で さがしました	
\\	読[よ]みたい 章[しょう]を
\\	で 探[さが]しました。			
\\	速度	速度[そくど]	そくど	
\\	新幹線の速度はどれくらいですか。	新幹線[しんかんせん]の 速度[そくど]はどれくらいですか。	しんかんせん の そくど は どれ くらい です か	
\\	新幹線[しんかんせん]の
\\	はどれくらいですか。			
\\	速達便	速達便[そくたつびん]	そくたつびん	
\\	速達便なら明日の午前中に届きます。	速達便[そくたつびん]なら 明日[あした]の 午前中[ごぜんちゅう]に 届[とど]きます。	そくたつびん なら あした の ごぜんちゅう に とどきます	
\\	なら 明日[あした]の 午前中[ごぜんちゅう]に 届[とど]きます。			
\\	通り過ぎる	通[とお]り 過[す]ぎる	とおりすぎる	
\\	うっかり目的地を通り過ぎた。	うっかり 目的地[もくてきち]を 通[とお]り 過[す]ぎた。	うっかり もくてきち を とおりすぎた	
\\	うっかり 目的地[もくてきち]を
\\	発売	発売[はつばい]	はつばい	
\\	新しい車が発売された。	新[あたら]しい 車[くるま]が 発売[はつばい]された。	あたらしい くるま が はつばい された	
\\	新[あたら]しい 車[くるま]が
\\	された。			
\\	発つ	発[た]つ	たつ	
\\	彼は明日メキシコへ発ちます。	彼[かれ]は 明日[あした]メキシコへ 発[た]ちます。	かれ は あした めきしこ へ たちます	
\\	彼[かれ]は 明日[あした]メキシコへ
\\	発行	発行[はっこう]	はっこう	
\\	その雑誌は年に4回発行されているんだ。	その 雑誌[ざっし]は 年[ねん]に4 回[かい] 発行[はっこう]されているんだ。	その ざっし は ねん に 
\\	かい はっこう されて いる ん だ	
\\	その 雑誌[ざっし]は 年[ねん]に4 回[かい]
\\	されているんだ。			
\\	発達	発達[はったつ]	はったつ	
\\	通信技術の発達は目覚ましいな。	通信技術[つうしん ぎじゅつ]の 発達[はったつ]は 目覚[めざ]ましいな。	つうしん ぎじゅつ の はったつ は めざましい な	
\\	通信技術[つうしん ぎじゅつ]の
\\	は 目覚[めざ]ましいな。			
\\	発明	発明[はつめい]	はつめい	
\\	彼は偉大な発明王です。	彼[かれ]は 偉大[いだい]な 発明[はつめい] 王[おう]です。	かれ は いだい な はつめいおう です	
\\	彼[かれ]は 偉大[いだい]な
\\	王[おう]です。			
\\	発電	発電[はつでん]	はつでん	
\\	ここでは太陽エネルギーを使って発電しています。	ここでは 太陽[たいよう]エネルギーを 使[つか]って 発電[はつでん]しています。	ここ で は たいよう えねるぎー を つかって はつでん して います	
\\	ここでは 太陽[たいよう]エネルギーを 使[つか]って
\\	しています。			
\\	発電所	発電所[はつでんしょ]	はつでんしょ	
\\	すぐそこに発電所があります。	すぐそこに 発電所[はつでんしょ]があります。	すぐ そこ に はつでんしょ が あります	
\\	すぐそこに
\\	があります。			
\\	発熱	発熱[はつねつ]	はつねつ	
\\	娘が突然、発熱したんだ。	娘[むすめ]が 突然[とつぜん]、 発熱[はつねつ]したんだ。	むすめ が とつぜん はつねつ した ん だ	
\\	娘[むすめ]が 突然[とつぜん]、
\\	したんだ。			
\\	表情	表情[ひょうじょう]	ひょうじょう	
\\	彼はとても表情が豊かですね。	彼[かれ]はとても 表情[ひょうじょう]が 豊[ゆた]かですね。	かれ は とても ひょうじょう が ゆたか です ね	
\\	彼[かれ]はとても
\\	が 豊[ゆた]かですね。			
\\	用紙	用紙[ようし]	ようし	
\\	この用紙に名前を書いてください。	この 用紙[ようし]に 名前[なまえ]を 書[か]いてください。	この ようし に なまえ を かいて ください	
\\	この
\\	に 名前[なまえ]を 書[か]いてください。			
\\	表紙	表紙[ひょうし]	ひょうし	
\\	その人気アイドルが今月の表紙だよ。	その 人気[にんき]アイドルが 今月[こんげつ]の 表紙[ひょうし]だよ。	その にんき あいどる が こんげつ の ひょうし だ よ	
\\	その 人気[にんき]アイドルが 今月[こんげつ]の
\\	だよ。			
\\	白紙	白紙[はくし]	はくし	
\\	テストを白紙で出したの。	テストを 白紙[はくし]で 出[だ]したの。	てすと を はくし で だした の	
\\	テストを
\\	で 出[だ]したの。			
\\	音	音[ね]	ね	
\\	秋は虫の音が心地良いです。	秋[あき]は 虫[むし]の 音[ね]が 心地良[ここちい]いです。	あき は むし の ね が ここちいい です	
\\	秋[あき]は 虫[むし]の
\\	が 心地良[ここちい]いです。			
\\	発音	発音[はつおん]	はつおん	
\\	この単語を発音してください。	この 単語[たんご]を 発音[はつおん]してください。	この たんご を はつおん して ください	
\\	この 単語[たんご]を
\\	してください。			
\\	物音	物音[ものおと]	ものおと	
\\	物音がしたので見に行った。	物音[ものおと]がしたので 見[み]に 行[い]った。	ものおと が した の で み に いった	
\\	がしたので 見[み]に 行[い]った。			
\\	薬品	薬品[やくひん]	やくひん	
\\	彼女は薬品を戸棚から出したんだ。	彼女[かのじょ]は 薬品[やくひん]を 戸棚[とだな]から 出[だ]したんだ。	かのじょ は やくひん を とだな から だした ん だ	
\\	彼女[かのじょ]は
\\	を 戸棚[とだな]から 出[だ]したんだ。			
\\	薬局	薬局[やっきょく]	やっきょく	
\\	薬局で目薬を買いました。	薬局[やっきょく]で 目薬[めぐすり]を 買[か]いました。	やっきょく で めぐすり を かいました	
\\	で 目薬[めぐすり]を 買[か]いました。			
\\	目薬	目薬[めぐすり]	めぐすり	
\\	目が疲れたので目薬をさしたよ。	目[め]が 疲[つか]れたので 目薬[めぐすり]をさしたよ。	め が つかれた の で めぐすり を さした よ	
\\	目[め]が 疲[つか]れたので
\\	をさしたよ。			
\\	表面	表面[ひょうめん]	ひょうめん	
\\	月の表面にはクレーターがたくさんあるね。	月[つき]の 表面[ひょうめん]にはクレーターがたくさんあるね。	つき の ひょうめん に は くれーたー が たくさん ある ね	
\\	月[つき]の
\\	にはクレーターがたくさんあるね。			
\\	面する	面[めん]する	めんする	
\\	私の家は川に面しています。	私[わたし]の 家[いえ]は 川[かわ]に 面[めん]しています。	わたし の いえ は かわ に めんして います	
\\	私[わたし]の 家[いえ]は 川[かわ]に
\\	面	面[めん]	めん	
\\	彼の意外な面を見た。	彼[かれ]の 意外[いがい]な 面[めん]を 見[み]た。	かれ の いがい な めん を みた	
\\	彼[かれ]の 意外[いがい]な
\\	を 見[み]た。			
\\	真っ先	真[ま]っ 先[さき]	まっさき	
\\	合格を真っ先に両親に伝えたの。	合格[ごうかく]を 真[ま]っ 先[さき]に 両親[りょうしん]に 伝[つた]えたの。	ごうかく を まっさき に りょうしん に つたえた の	
\\	合格[ごうかく]を
\\	に 両親[りょうしん]に 伝[つた]えたの。			
\\	真夏	真夏[まなつ]	まなつ	
\\	真夏のビールはおいしい。	真夏[まなつ]のビールはおいしい。	まなつ の びーる は おいしい	
\\	のビールはおいしい。			
\\	真夜中	真夜中[まよなか]	まよなか	
\\	彼は真夜中に帰ってきたの。	彼[かれ]は 真夜中[まよなか]に 帰[かえ]ってきたの。	かれ は まよなか に かえって きた の	
\\	彼[かれ]は
\\	に 帰[かえ]ってきたの。			
\\	真っ最中	真[ま]っ 最中[さいちゅう]	まっさいちゅう	
\\	今は試験の真っ最中です。	今[いま]は 試験[しけん]の 真[ま]っ 最中[さいちゅう]です。	いま は しけん の まっさいちゅう です	
\\	今[いま]は 試験[しけん]の
\\	です。			
\\	真心	真心[まごころ]	まごころ	
\\	彼女は真心をこめて彼にケーキを作ったよ。	彼女[かのじょ]は 真心[まごころ]をこめて 彼[かれ]にケーキを 作[つく]ったよ。	かのじょ は まごころ を こめて かれ に けーき を つくった よ	
\\	彼女[かのじょ]は
\\	をこめて 彼[かれ]にケーキを 作[つく]ったよ。			
\\	無色	無色[むしょく]	むしょく	
\\	アルコールは無色です。	アルコールは 無色[むしょく]です。	あるこーる は むしょく です	
\\	アルコールは
\\	です。			
\\	長方形	長方形[ちょうほうけい]	ちょうほうけい	
\\	私の部屋は長方形です。	私[わたし]の 部屋[へや]は 長方形[ちょうほうけい]です。	わたし の へや は ちょうほうけい です	
\\	私[わたし]の 部屋[へや]は
\\	です。			
\\	種	種[たね]	たね	
\\	プランターにトマトの種をまきました。	プランターにトマトの 種[たね]をまきました。	ぷらんたーに とまと の たね を まきました 。	
\\	プランターにトマトの
\\	をまきました。			
\\	直前	直前[ちょくぜん]	ちょくぜん	
\\	彼は旅行の直前に熱が出たの。	彼[かれ]は 旅行[りょこう]の 直前[ちょくぜん]に 熱[ねつ]が 出[で]たの。	かれ は りょこう の ちょくぜん に ねつ が でた の	
\\	彼[かれ]は 旅行[りょこう]の
\\	に 熱[ねつ]が 出[で]たの。			
\\	直ちに	直[ただ]ちに	ただちに	
\\	直ちに会社に戻ってください。	直[ただ]ちに 会社[かいしゃ]に 戻[もど]ってください。	ただちに かいしゃ に もどって ください	
\\	会社[かいしゃ]に 戻[もど]ってください。			
\\	見直す	見直[みなお]す	みなおす	
\\	彼のことを見直しました。	彼[かれ]のことを 見直[みなお]しました。	かれ の こと を みなおしました	
\\	彼[かれ]のことを
\\	直後	直後[ちょくご]	ちょくご	
\\	食べた直後に寝ないほうがいいよ。	食[た]べた 直後[ちょくご]に 寝[ね]ないほうがいいよ。	たべた ちょくご に ねない ほう が いい よ	
\\	食[た]べた
\\	に 寝[ね]ないほうがいいよ。			
\\	立ち直る	立[た]ち 直[なお]る	たちなおる	
\\	彼女は失敗から立ち直ったよ。	彼女[かのじょ]は 失敗[しっぱい]から 立[た]ち 直[なお]ったよ。	かのじょ は しっぱい から たちなおった よ	
\\	彼女[かのじょ]は 失敗[しっぱい]から
\\	よ。			
\\	直通	直通[ちょくつう]	ちょくつう	
\\	これは私の直通の電話番号です。	これは 私[わたし]の 直通[ちょくつう]の 電話番号[でんわ ばんごう]です。	これ は わたし の ちょくつう の でんわ ばんごう です 。	
\\	これは 私[わたし]の
\\	の 電話番号[でんわ ばんごう]です。			
\\	面接	面接[めんせつ]	めんせつ	
\\	新しい仕事の面接に行ってきました。	新[あたら]しい 仕事[しごと]の 面接[めんせつ]に 行[い]ってきました。	あたらしい しごと の めんせつ に いって きました	
\\	新[あたら]しい 仕事[しごと]の
\\	に 行[い]ってきました。			
\\	無線	無線[むせん]	むせん	
\\	運転手が無線で話していたの。	運転手[うんてんしゅ]が 無線[むせん]で 話[はな]していたの。	うんてんしゅ が むせん で はなして いた の	
\\	運転手[うんてんしゅ]が
\\	で 話[はな]していたの。			
\\	線路	線路[せんろ]	せんろ	
\\	子猫が線路に迷い込んだぞ。	子猫[こねこ]が 線路[せんろ]に 迷[まよ]い 込[こ]んだぞ。	こねこ が せんろ に まよいこんだ ぞ	
\\	子猫[こねこ]が
\\	に 迷[まよ]い 込[こ]んだぞ。			
\\	直線	直線[ちょくせん]	ちょくせん	
\\	直線を描いてください。	直線[ちょくせん]を 描[か]いてください。	ちょくせん を かいて ください	
\\	を 描[か]いてください。			
\\	電線	電線[でんせん]	でんせん	
\\	風で電線が揺れていますね。	風[かぜ]で 電線[でんせん]が 揺[ゆ]れていますね。	かぜ で でんせん が ゆれて います ね	
\\	風[かぜ]で
\\	が 揺[ゆ]れていますね。			
\\	脱線	脱線[だっせん]	だっせん	
\\	地震で電車が脱線したよ。	地震[じしん]で 電車[でんしゃ]が 脱線[だっせん]したよ。	じしん で でんしゃ が だっせん した よ	
\\	地震[じしん]で 電車[でんしゃ]が
\\	したよ。			
\\	白線	白線[はくせん]	はくせん	
\\	白線の内側に下がってお待ちください。	白線[はくせん]の 内側[うちがわ]に 下[さ]がってお 待[ま]ちください。	はくせん の うちがわ に さがって おまち ください	
\\	の 内側[うちがわ]に 下[さ]がってお 待[ま]ちください。			
\\	直角	直角[ちょっかく]	ちょっかく	
\\	ここで直角に曲がってください。	ここで 直角[ちょっかく]に 曲[ま]がってください。	ここ で ちょっかく に まがって ください	
\\	ここで
\\	に 曲[ま]がってください。			
\\	町角	町角[まちかど]	まちかど	
\\	町角にカフェがありました。	町角[まちかど]にカフェがありました。	まちかど に かふぇ が ありました	
\\	にカフェがありました。			
\\	真四角	真四角[ましかく]	ましかく	
\\	彼の部屋には真四角な窓があったんだ。	彼[かれ]の 部屋[へや]には 真四角[ましかく]な 窓[まど]があったんだ。	かれ の へや に は ましかく な まど が あった ん だ	
\\	彼[かれ]の 部屋[へや]には
\\	な 窓[まど]があったんだ。			
\\	違いない	違[ちが]いない	ちがいない	
\\	彼は成功するに違いないわ。	彼[かれ]は 成功[せいこう]するに 違[ちが]いないわ。	かれ は せいこう する に ちがいない わ	
\\	彼[かれ]は 成功[せいこう]するに
\\	わ。			
\\	見違える	見違[みちが]える	みちがえる	
\\	掃除をしたら部屋が見違えるようになったの。	掃除[そうじ]をしたら 部屋[へや]が 見違[みちが]えるようになったの。	そうじ を したら へや が みちがえる よう に なった の	
\\	掃除[そうじ]をしたら 部屋[へや]が
\\	ようになったの。			
\\	違い	違[ちが]い	ちがい	
\\	この二つには大きな違いがある。	この 二[ふた]つには 大[おお]きな 違[ちが]いがある。	この ふたつ に は おおき な ちがい が ある	
\\	この 二[ふた]つには 大[おお]きな
\\	がある。			
\\	窓口	窓口[まどぐち]	まどぐち	
\\	お振込みは3番の窓口です。	お 振込[ふりこ]みは3 番[ばん]の 窓口[まどぐち]です。	おふりこみ は 
\\	ばん の まどぐち です	
\\	お 振込[ふりこ]みは3 番[ばん]の
\\	です。			
\\	父母	父母[ふぼ]	ふぼ	
\\	学校から父母に連絡があったんだ。	学校[がっこう]から 父母[ふぼ]に 連絡[れんらく]があったんだ。	がっこう から ふぼ に れんらく が あった ん だ	
\\	学校[がっこう]から
\\	に 連絡[れんらく]があったんだ。			
\\	父親	父親[ちちおや]	ちちおや	
\\	彼の父親は先生です。	彼[かれ]の 父親[ちちおや]は 先生[せんせい]です。	かれ の ちちおや は せんせい です 。	
\\	彼[かれ]の
\\	は 先生[せんせい]です。			
\\	父兄	父兄[ふけい]	ふけい	
\\	父兄の方々はこちらへどうぞ。	父兄[ふけい]の 方々[かたがた]はこちらへどうぞ。	ふけい の かたがた は こちら へ どうぞ	
\\	の 方々[かたがた]はこちらへどうぞ。			
\\	第一	第一[だいいち]	だいいち	
\\	私には仕事が第一です。	私[わたし]には 仕事[しごと]が 第一[だいいち]です。	わたし に は しごと が だいいち です	
\\	私[わたし]には 仕事[しごと]が
\\	です。			
\\	若々しい	若々[わかわか]しい	わかわかしい	
\\	彼女はとても若々しい人です。	彼女[かのじょ]はとても 若々[わかわか]しい 人[ひと]です。	かのじょ は とても わかわかしい ひと です	
\\	彼女[かのじょ]はとても
\\	人[ひと]です。			
\\	読者	読者[どくしゃ]	どくしゃ	
\\	新聞の読者の投書欄は好きです。	新聞[しんぶん]の 読者[どくしゃ]の 投書欄[とうしょらん]は 好[す]きです。	しんぶん の どくしゃ の とうしょらん は すき です	
\\	新聞[しんぶん]の
\\	の 投書欄[とうしょらん]は 好[す]きです。			
\\	者	者[もの]	もの	
\\	私は先ほど電話した者です。	私[わたし]は 先[さき]ほど 電話[でんわ]した 者[もの]です。	わたし は さきほど でんわ した もの です	
\\	私[わたし]は 先[さき]ほど 電話[でんわ]した
\\	です。			
\\	結ぶ	結[むす]ぶ	むすぶ	
\\	彼女は髪にリボンを結んだね。	彼女[かのじょ]は 髪[かみ]にリボンを 結[むす]んだね。	かのじょ は かみ に りぼん を むすんだ ね	
\\	彼女[かのじょ]は 髪[かみ]にリボンを
\\	ね。			
\\	結び	結[むす]び	むすび	
\\	彼は会の結びとしてスピーチをしたよ。	彼[かれ]は 会[かい]の 結[むす]びとしてスピーチをしたよ。	かれ は かい の むすび と して すぴーち を した よ	
\\	彼[かれ]は 会[かい]の
\\	としてスピーチをしたよ。			
\\	無効	無効[むこう]	むこう	
\\	このクーポンはもう無効です。	このクーポンはもう 無効[むこう]です。	この くーぽん は もう むこう です	
\\	このクーポンはもう
\\	です。			
\\	自ら	自[みずか]ら	みずから	
\\	社長自らがその会社と交渉したよ。	社長[しゃちょう] 自[みずか]らがその 会社[かいしゃ]と 交渉[こうしょう]したよ。	しゃちょう みずから が その かいしゃ と こうしょう した よ	
\\	社長[しゃちょう]
\\	がその 会社[かいしゃ]と 交渉[こうしょう]したよ。			
\\	通信	通信[つうしん]	つうしん	
\\	通信教育はとても便利です。	通信[つうしん] 教育[きょういく]はとても 便利[べんり]です。	つうしん きょういく は とても べんり です	
\\	教育[きょういく]はとても 便利[べんり]です。			
\\	頼る	頼[たよ]る	たよる	
\\	人に頼らないで、自分でやってごらん。	人[ひと]に 頼[たよ]らないで、 自分[じぶん]でやってごらん。	ひと に たよらない で じぶん で やって ごらん	
\\	人[ひと]に
\\	、 自分[じぶん]でやってごらん。			
\\	頼み	頼[たの]み	たのみ	
\\	あなたに頼みがあるんですけど。	あなたに 頼[たの]みがあるんですけど。	あなた に たのみ が ある ん です けど	
\\	あなたに
\\	があるんですけど。			
\\	頼もしい	頼[たの]もしい	たのもしい	
\\	彼は頼もしい人です。	彼[かれ]は 頼[たの]もしい 人[ひと]です。	かれ は たのもしい ひと です	
\\	彼[かれ]は
\\	人[ひと]です。			
\\	理論	理論[りろん]	りろん	
\\	彼は大学で音楽の理論を勉強したんだよ。	彼[かれ]は 大学[だいがく]で 音楽[おんがく]の 理論[りろん]を 勉強[べんきょう]したんだよ。	かれ は だいがく で おんがく の りろん を べんきょう した ん だ よ	
\\	彼[かれ]は 大学[だいがく]で 音楽[おんがく]の
\\	を 勉強[べんきょう]したんだよ。			
\\	論理	論理[ろんり]	ろんり	
\\	数学は論理に基づく学問です。	数学[すうがく]は 論理[ろんり]に 基[もと]づく 学問[がくもん]です。	すうがく は ろんり に もとづく がくもん です	
\\	数学[すうがく]は
\\	に 基[もと]づく 学問[がくもん]です。			
\\	論文	論文[ろんぶん]	ろんぶん	
\\	論文を仕上げないと卒業できませんよ。	論文[ろんぶん]を 仕上[しあ]げないと 卒業[そつぎょう]できませんよ。	ろんぶん を しあげない と そつぎょう できません よ	
\\	を 仕上[しあ]げないと 卒業[そつぎょう]できませんよ。			
\\	論じる	論[ろん]じる	ろんじる	
\\	仲間と地球の未来について論じたの。	仲間[なかま]と 地球[ちきゅう]の 未来[みらい]について 論[ろん]じたの。	なかま と ちきゅう の みらい に ついて ろんじた の	
\\	仲間[なかま]と 地球[ちきゅう]の 未来[みらい]について
\\	の。			
\\	無論	無論[むろん]	むろん	
\\	ぼくは無論きみに賛成だ。	ぼくは 無論[むろん]きみに 賛成[さんせい]だ。	ぼく は むろん きみ に さんせい だ	
\\	ぼくは
\\	きみに 賛成[さんせい]だ。			
\\	論	論[ろん]	ろん	
\\	彼の人生論は独特だな。	彼[かれ]の 人生[じんせい] 論[ろん]は 独特[どくとく]だな。	かれ の じんせいろん は どくとく だ な	
\\	彼[かれ]の 人生[じんせい]
\\	は 独特[どくとく]だな。			
\\	特定	特定[とくてい]	とくてい	
\\	警察は犯人を特定したらしいよ。	警察[けいさつ]は 犯人[はんにん]を 特定[とくてい]したらしいよ。	けいさつ は はんにん を とくてい した らしい よ	
\\	警察[けいさつ]は 犯人[はんにん]を
\\	したらしいよ。			
\\	特色	特色[とくしょく]	とくしょく	
\\	その学校の教育は特色がありますね。	その 学校[がっこう]の 教育[きょういく]は 特色[とくしょく]がありますね。	その がっこう の きょういく は とくしょく が あります ね	
\\	その 学校[がっこう]の 教育[きょういく]は
\\	がありますね。			
\\	特有	特有[とくゆう]	とくゆう	
\\	これは子供に特有の病気です。	これは 子供[こども]に 特有[とくゆう]の 病気[びょうき]です。	これ は こども に とくゆう の びょうき です	
\\	これは 子供[こども]に
\\	の 病気[びょうき]です。			
\\	送別	送別[そうべつ]	そうべつ	
\\	送別の辞は誰に頼みましょうか。	送別[そうべつ]の 辞[じ]は 誰[だれ]に 頼[たの]みましょうか。	そうべつ の じ は だれ に たのみましょう か	
\\	の 辞[じ]は 誰[だれ]に 頼[たの]みましょうか。			
\\	送別会	送別会[そうべつかい]	そうべつかい	
\\	来週、課長の送別会を開きます。	来週[らいしゅう]、 課長[かちょう]の 送別会[そうべつかい]を 開[ひら]きます。	らいしゅう かちょう の そうべつかい を ひらきます	
\\	来週[らいしゅう]、 課長[かちょう]の
\\	を 開[ひら]きます。			
\\	論理的	論理的[ろんりてき]	ろんりてき	
\\	彼は論理的な人です。	彼[かれ]は 論理的[ろんりてき]な 人[ひと]です。	かれ は ろんりてき な ひと です	
\\	彼[かれ]は
\\	な 人[ひと]です。			
\\	知的	知的[ちてき]	ちてき	
\\	彼はすごく知的な人です。	彼[かれ]はすごく 知的[ちてき]な 人[ひと]です。	かれ は すごく ちてき な ひと です	
\\	彼[かれ]はすごく
\\	な 人[ひと]です。			
\\	目的地	目的地[もくてきち]	もくてきち	
\\	やっと目的地に着いたよ。	やっと 目的地[もくてきち]に 着[つ]いたよ。	やっと もくてきち に ついた よ	
\\	やっと
\\	に 着[つ]いたよ。			
\\	男性的	男性的[だんせいてき]	だんせいてき	
\\	彼はとても男性的な人です。	彼[かれ]はとても 男性的[だんせいてき]な 人[ひと]です。	かれ は とても だんせいてき な ひと です	
\\	彼[かれ]はとても
\\	な 人[ひと]です。			
\\	等しい	等[ひと]しい	ひとしい	
\\	私は株の知識がないに等しいです。	私[わたし]は 株[かぶ]の 知識[ちしき]がないに 等[ひと]しいです。	わたし は かぶ の ちしき が ない に ひとしい です	
\\	私[わたし]は 株[かぶ]の 知識[ちしき]がないに
\\	です。			
\\	病室	病室[びょうしつ]	びょうしつ	
\\	その病室はとても広くてきれいだった。	その 病室[びょうしつ]はとても 広[ひろ]くてきれいだった。	その びょうしつ は とても ひろくて きれい だった	
\\	その
\\	はとても 広[ひろ]くてきれいだった。			
\\	病人	病人[びょうにん]	びょうにん	
\\	病人が出たので電車が少し止まったんだ。	病人[びょうにん]が 出[で]たので 電車[でんしゃ]が 少[すこ]し 止[と]まったんだ。	びょうにん が でた の で でんしゃ が すこし とまった ん だ	
\\	が 出[で]たので 電車[でんしゃ]が 少[すこ]し 止[と]まったんだ。			
\\	病死	病死[びょうし]	びょうし	
\\	その作家は35歳という若さで病死した。	その 作家[さっか]は35 歳[さい]という 若[わか]さで 病死[びょうし]した。	その さっか は 
\\	さい と いう わかさ で びょうし した	
\\	その 作家[さっか]は35 歳[さい]という 若[わか]さで
\\	した。			
\\	疲れ	疲[つか]れ	つかれ	
\\	最近疲れがたまっています。	最近[さいきん] 疲[つか]れがたまっています。	さいきん つかれ が たまって います	
\\	最近[さいきん]
\\	がたまっています。			
\\	立ち入り禁止	立[た]ち 入[い]り 禁止[きんし]	たちいりきんし	
\\	ここは立ち入り禁止です。	ここは 立[た]ち 入[い]り 禁止[きんし]です。	ここ は たちいりきんし です	
\\	ここは
\\	です。			
\\	酔う	酔[よ]う	よう	
\\	みんなかなり酔っていたの。	みんなかなり 酔[よ]っていたの。	みんな かなり よって いた の	
\\	みんなかなり
\\	の。			
\\	無意味	無意味[むいみ]	むいみ	
\\	そんなことをしても無意味よ。	そんなことをしても 無意味[むいみ]よ。	そんな こと を して も むいみ よ	
\\	そんなことをしても
\\	よ。			
\\	生意気	生意気[なまいき]	なまいき	
\\	あの子は生意気だと思います。	あの 子[こ]は 生意気[なまいき]だと 思[おも]います。	あの こ は なまいき だ と おもいます	
\\	あの 子[こ]は
\\	だと 思[おも]います。			
\\	確かめる	確[たし]かめる	たしかめる	
\\	母はお釣りを確かめたの。	母[はは]はお 釣[つ]りを 確[たし]かめたの。	はは は おつり を たしかめた の	
\\	母[はは]はお 釣[つ]りを
\\	の。			
\\	認可	認可[にんか]	にんか	
\\	この薬はまだ認可されていません。	この 薬[くすり]はまだ 認可[にんか]されていません。	この くすり は まだ にんか されて いません	
\\	この 薬[くすり]はまだ
\\	されていません。			
\\	認める	認[みと]める	みとめる	
\\	父が彼女との結婚を認めてくれました。	父[ちち]が 彼女[かのじょ]との 結婚[けっこん]を 認[みと]めてくれました。	ちち が かのじょ と の けっこん を みとめて くれました	
\\	父[ちち]が 彼女[かのじょ]との 結婚[けっこん]を
\\	率	率[りつ]	りつ	
\\	その手術の成功率は90
\\	だそうです。	その 手術[しゅじゅつ]の 成功[せいこう] 率[りつ]は90
\\	だそうです。	その しゅじゅつ の せいこうりつ は 
\\	だ そう です	
\\	その 手術[しゅじゅつ]の 成功[せいこう]
\\	は90
\\	だそうです。			
\\	率直	率直[そっちょく]	そっちょく	
\\	率直なご意見ありがとうございました。	率直[そっちょく]なご 意見[いけん]ありがとうございました。	そっちょく な ごいけん ありがとう ございました	
\\	なご 意見[いけん]ありがとうございました。			
\\	能率	能率[のうりつ]	のうりつ	
\\	この方法だと能率がいいですね。	この 方法[ほうほう]だと 能率[のうりつ]がいいですね。	この ほうほう だ と のうりつ が いい です ね	
\\	この 方法[ほうほう]だと
\\	がいいですね。			
\\	道具	道具[どうぐ]	どうぐ	
\\	道具は全部揃っていますか。	道具[どうぐ]は 全部揃[ぜんぶ そろ]っていますか。	どうぐ は ぜんぶ そろって います か	
\\	は 全部揃[ぜんぶ そろ]っていますか。			
\\	説く	説[と]く	とく	
\\	彼は非暴力を説きました。	彼[かれ]は 非暴力[ひぼうりょく]を 説[と]きました。	かれ は ひぼうりょく を ときました	
\\	彼[かれ]は 非暴力[ひぼうりょく]を
\\	遊園地	遊園地[ゆうえんち]	ゆうえんち	
\\	この遊園地のチケットは3000円です。	この 遊園地[ゆうえんち]のチケットは 3000円[さんぜんえん]です。	この ゆうえんち の ちけっと は さんぜんえん です	
\\	この
\\	のチケットは 3000円[さんぜんえん]です。			
\\	飛び出す	飛[と]び 出[だ]す	とびだす	
\\	彼は道路に飛び出したの。	彼[かれ]は 道路[どうろ]に 飛[と]び 出[だ]したの。	かれ は どうろ に とびだした の	
\\	彼[かれ]は 道路[どうろ]に
\\	の。			
\\	飛ばす	飛[と]ばす	とばす	
\\	子供が紙飛行機を飛ばしています。	子供[こども]が 紙飛行機[かみひこうき]を 飛[と]ばしています。	こども が かみひこうき を とばして います	
\\	子供[こども]が 紙飛行機[かみひこうき]を
\\	飛び上がる	飛[と]び 上[あ]がる	とびあがる	
\\	大きな音にびっくりして飛び上がりました。	大[おお]きな 音[おと]にびっくりして 飛[と]び 上[あ]がりました。	おおき な おと に びっくり して とびあがりました	
\\	大[おお]きな 音[おと]にびっくりして
\\	飛び下りる	飛[と]び 下[お]りる	とびおりる	
\\	猫が屋根から飛び下りました。	猫[ねこ]が 屋根[やね]から 飛[と]び 下[お]りました。	ねこ が やね から とびおりました	
\\	猫[ねこ]が 屋根[やね]から
\\	飛び込む	飛[と]び 込[こ]む	とびこむ	
\\	カエルが池に飛び込んだね。	カエルが 池[いけ]に 飛[と]び 込[こ]んだね。	かえる が いけ に とびこんだ ね	
\\	カエルが 池[いけ]に
\\	ね。			
\\	飛行	飛行[ひこう]	ひこう	
\\	このフライトの飛行時間は約3時間です。	このフライトの 飛行[ひこう] 時間[じかん]は 約3時間[やく 
\\	じかん]です。	この ふらいと の ひこう じかん は やく 
\\	じかん です	
\\	このフライトの
\\	時間[じかん]は 約3時間[やく 
\\	じかん]です。			
\\	船長	船長[せんちょう]	せんちょう	
\\	私がこの船の船長です。	私[わたし]がこの 船[ふね]の 船長[せんちょう]です。	わたし が この ふね の せんちょう です	
\\	私[わたし]がこの 船[ふね]の
\\	です。			
\\	風船	風船[ふうせん]	ふうせん	
\\	子供が風船を膨らませているね。	子供[こども]が 風船[ふうせん]を 膨[ふく]らませているね。	こども が ふうせん を ふくらませて いる ね	
\\	子供[こども]が
\\	を 膨[ふく]らませているね。			
\\	達成	達成[たっせい]	たっせい	
\\	彼は今月の売上目標を達成した。	彼[かれ]は 今月[こんげつ]の 売上目標[うりあげ もくひょう]を 達成[たっせい]した。	かれ は こんげつ の うりあげ もくひょう を たっせい した	
\\	彼[かれ]は 今月[こんげつ]の 売上目標[うりあげ もくひょう]を
\\	した。			
\\	野原	野原[のはら]	のはら	
\\	私たちは野原で花をつんだの。	私[わたし]たちは 野原[のはら]で 花[はな]をつんだの。	わたしたち は のはら で はな を つんだ の	
\\	私[わたし]たちは
\\	で 花[はな]をつんだの。			
\\	要因	要因[よういん]	よういん	
\\	私たちはがんの要因を研究しています。	私[わたし]たちはがんの 要因[よういん]を 研究[けんきゅう]しています。	わたしたち は がん の よういん を けんきゅう して います	
\\	私[わたし]たちはがんの
\\	を 研究[けんきゅう]しています。			
\\	物資	物資[ぶっし]	ぶっし	
\\	被災地に物資を送ったよ。	被災地[ひさいち]に 物資[ぶっし]を 送[おく]ったよ。	ひさいち に ぶっし を おくった よ	
\\	被災地[ひさいち]に
\\	を 送[おく]ったよ。			
\\	願い	願[ねが]い	ねがい	
\\	世界の平和が私たちの願いです。	世界[せかい]の 平和[へいわ]が 私[わたし]たちの 願[ねが]いです。	せかい の へいわ が わたし たち の ねがい です	
\\	世界[せかい]の 平和[へいわ]が 私[わたし]たちの
\\	です。			
\\	知識	知識[ちしき]	ちしき	
\\	私は旅行から多くの知識を得た。	私[わたし]は 旅行[りょこう]から 多[おお]くの 知識[ちしき]を 得[え]た。	わたし は りょこう から おおく の ちしき を えた	
\\	私[わたし]は 旅行[りょこう]から 多[おお]くの
\\	を 得[え]た。			
\\	認識	認識[にんしき]	にんしき	
\\	その件は終わったと認識しています。	その 件[けん]は 終[お]わったと 認識[にんしき]しています。	その けん は おわった と にんしき して います	
\\	その 件[けん]は 終[お]わったと
\\	しています。			
\\	無意識	無意識[むいしき]	むいしき	
\\	私は無意識に彼を傷つけてしまった。	私[わたし]は 無意識[むいしき]に 彼[かれ]を 傷[きず]つけてしまった。	わたし は むいしき に かれ を きずつけて しまった	
\\	私[わたし]は
\\	に 彼[かれ]を 傷[きず]つけてしまった。			
\\	非難	非難[ひなん]	ひなん	
\\	国民は総理大臣を非難しているね。	国民[こくみん]は 総理大臣[そうり だいじん]を 非難[ひなん]しているね。	こくみん は そうり だいじん を ひなん して いる ね	
\\	国民[こくみん]は 総理大臣[そうり だいじん]を
\\	しているね。			
\\	非常	非常[ひじょう]	ひじょう	
\\	非常事態です。	非常[ひじょう] 事態[じたい]です。	ひじょう じたい です	
\\	事態[じたい]です。			
\\	非常識	非常識[ひじょうしき]	ひじょうしき	
\\	彼は非常識な時間に電話してきたの。	彼[かれ]は 非常識[ひじょうしき]な 時間[じかん]に 電話[でんわ]してきたの。	かれ は ひじょうしき な じかん に でんわ して きた の	
\\	彼[かれ]は
\\	な 時間[じかん]に 電話[でんわ]してきたの。			
\\	非常口	非常口[ひじょうぐち]	ひじょうぐち	
\\	ビルの非常口を確認したの。	ビルの 非常口[ひじょうぐち]を 確認[かくにん]したの。	びる の ひじょうぐち を かくにん した の	
\\	ビルの
\\	を 確認[かくにん]したの。			
\\	調子	調子[ちょうし]	ちょうし	
\\	体の調子がとても良いです。	体[からだ]の 調子[ちょうし]がとても 良[い]いです。	からだ の ちょうし が とても いい です	
\\	体[からだ]の
\\	がとても 良[い]いです。			
\\	調和	調和[ちょうわ]	ちょうわ	
\\	彼女は自然と調和した暮らし方をしているの。	彼女[かのじょ]は 自然[しぜん]と 調和[ちょうわ]した 暮[く]らし 方[かた]をしているの。	かのじょ は しぜん と ちょうわ した くらし かた を して いる の	
\\	彼女[かのじょ]は 自然[しぜん]と
\\	した 暮[く]らし 方[かた]をしているの。			
\\	調味料	調味料[ちょうみりょう]	ちょうみりょう	
\\	塩、コショウはよく使われる調味料です。	塩[しお]、コショウはよく 使[つか]われる 調味料[ちょうみりょう]です。	しお こしょう は よく つかわれる ちょうみりょう です	
\\	塩[しお]、コショウはよく 使[つか]われる
\\	です。			
\\	調整	調整[ちょうせい]	ちょうせい	
\\	今、スケジュールの調整をしています。	今[いま]、スケジュールの 調整[ちょうせい]をしています。	いま すけじゅーる の ちょうせい を して います	
\\	今[いま]、スケジュールの
\\	をしています。			
\\	調節	調節[ちょうせつ]	ちょうせつ	
\\	ここで部屋の温度が調節できます。	ここで 部屋[へや]の 温度[おんど]が 調節[ちょうせつ]できます。	ここ で へや の おんど が ちょうせつ できます	
\\	ここで 部屋[へや]の 温度[おんど]が
\\	できます。			
\\	答案	答案[とうあん]	とうあん	
\\	今から答案を集めます。	今[いま]から 答案[とうあん]を 集[あつ]めます。	いま から とうあん を あつめます	
\\	今[いま]から
\\	を 集[あつ]めます。			
\\	目標	目標[もくひょう]	もくひょう	
\\	私は父を目標にしています。	私[わたし]は 父[ちち]を 目標[もくひょう]にしています。	わたし は ちち を もくひょう に して います	
\\	私[わたし]は 父[ちち]を
\\	にしています。			
\\	連れる	連[つ]れる	つれる	
\\	親が君を一度連れて来いって言うんだ。	親[おや]が 君[きみ]を 一度[いちど] 連[つ]れて 来[こ]いって 言[い]うんだ。	おや が きみ を いちど つれて こい って いう ん だ	
\\	親[おや]が 君[きみ]を 一度[いちど]
\\	来[こ]いって 言[い]うんだ。			
\\	連日	連日[れんじつ]	れんじつ	
\\	展覧会は連日賑わいました。	展覧会[てんらんかい]は 連日[れんじつ] 賑[にぎ]わいました。	てんらんかい は れんじつ にぎわいました	
\\	展覧会[てんらんかい]は
\\	賑[にぎ]わいました。			
\\	連休	連休[れんきゅう]	れんきゅう	
\\	今度の連休は実家に帰ります。	今度[こんど]の 連休[れんきゅう]は 実家[じっか]に 帰[かえ]ります。	こんど の れんきゅう は じっか に かえります	
\\	今度[こんど]の
\\	は 実家[じっか]に 帰[かえ]ります。			
\\	連れ	連[つ]れ	つれ	
\\	彼女は私の連れです。	彼女[かのじょ]は 私[わたし]の 連[つ]れです。	かのじょ は わたし の つれ です	
\\	彼女[かのじょ]は 私[わたし]の
\\	です。			
\\	連絡	連絡[れんらく]	れんらく	
\\	仕事が終わったら連絡します。	仕事[しごと]が 終[お]わったら 連絡[れんらく]します。	しごと が おわったら れんらく します	
\\	仕事[しごと]が 終[お]わったら
\\	します。			
\\	連続	連続[れんぞく]	れんぞく	
\\	彼女の人生は苦労の連続でした。	彼女[かのじょ]の 人生[じんせい]は 苦労[くろう]の 連続[れんぞく]でした。	かのじょ の じんせい は くろう の れんぞく でした	
\\	彼女[かのじょ]の 人生[じんせい]は 苦労[くろう]の
\\	でした。			
\\	続々	続々[ぞくぞく]	ぞくぞく	
\\	お客さんが続々とやって来ましたよ。	お 客[きゃく]さんが 続々[ぞくぞく]とやって 来[き]ましたよ。	おきゃくさん が ぞくぞく と やってきました よ	
\\	お 客[きゃく]さんが
\\	とやって 来[き]ましたよ。			
\\	続き	続[つづ]き	つづき	
\\	話の続きは電話でしましょう。	話[はなし]の 続[つづ]きは 電話[でんわ]でしましょう。	はなし の つづき は でんわ で しましょう	
\\	話[はなし]の
\\	は 電話[でんわ]でしましょう。			
\\	長続き	長続[ながつづ]き	ながつづき	
\\	彼は仕事が長続きしません。	彼[かれ]は 仕事[しごと]が 長続[ながつづ]きしません。	かれ は しごと が ながつづき しません	
\\	彼[かれ]は 仕事[しごと]が
\\	しません。			
\\	相場	相場[そうば]	そうば	
\\	この辺りの家賃の相場はいくらですか。	この 辺[あた]りの 家賃[やちん]の 相場[そうば]はいくらですか。	この あたり の やちん の そうば は いくら です か	
\\	この 辺[あた]りの 家賃[やちん]の
\\	はいくらですか。			
\\	相当	相当[そうとう]	そうとう	
\\	1ポンドは454グラムに相当します。	1[いち]ポンドは454グラムに 相当[そうとう]します。	いちぽんど は 
\\	ぐらむ に そうとう します	
\\	1[いち]ポンドは454グラムに
\\	します。			
\\	相応しい	相応[ふさわ]しい	ふさわしい	
\\	その場に相応しい服装で来てください。	その 場[ば]に 相応[ふさわ]しい 服装[ふくそう]で 来[き]てください。	その ば に ふさわしい ふくそう で きて ください	
\\	その 場[ば]に
\\	服装[ふくそう]で 来[き]てください。			
\\	録画	録画[ろくが]	ろくが	
\\	好きな番組を録画したの。	好[す]きな 番組[ばんぐみ]を 録画[ろくが]したの。	すき な ばんぐみ を ろくが した の	
\\	好[す]きな 番組[ばんぐみ]を
\\	したの。			
\\	録音テープ	録音[ろくおん]テープ	ろくおんテープ	
\\	インタビューはこの録音テープに入っています。	インタビューはこの 録音[ろくおん]テープに 入[はい]っています。	いんたびゅー は この ろくおんてーぷ に はいって います	
\\	インタビューはこの
\\	に 入[はい]っています。			
\\	登場	登場[とうじょう]	とうじょう	
\\	彼女の登場で会場は盛り上がったね。	彼女[かのじょ]の 登場[とうじょう]で 会場[かいじょう]は 盛[も]り 上[あ]がったね。	かのじょ の とうじょう で かいじょう は もりあがった ね	
\\	彼女[かのじょ]の
\\	で 会場[かいじょう]は 盛[も]り 上[あ]がったね。			
\\	登録	登録[とうろく]	とうろく	
\\	心理学の授業に登録しましたか。	心理学[しんりがく]の 授業[じゅぎょう]に 登録[とうろく]しましたか。	しんりがく の じゅぎょう に とうろく しました か	
\\	心理学[しんりがく]の 授業[じゅぎょう]に
\\	しましたか。			
\\	登山	登山[とざん]	とざん	
\\	夏休みには家族で登山をします。	夏休[なつやす]みには 家族[かぞく]で 登山[とざん]をします。	なつやすみ に は かぞく で とざん を します	
\\	夏休[なつやす]みには 家族[かぞく]で
\\	をします。			
\\	登校	登校[とうこう]	とうこう	
\\	生徒たちは朝8時ごろ登校します。	生徒[せいと]たちは 朝8時[あさ 
\\	じ]ごろ 登校[とうこう]します。	せいとたち は あさ 
\\	じごろ とうこう します	
\\	生徒[せいと]たちは 朝8時[あさ 
\\	じ]ごろ
\\	します。			
\\	無関心	無関心[むかんしん]	むかんしん	
\\	彼女は政治に無関心です。	彼女[かのじょ]は 政治[せいじ]に 無関心[むかんしん]です。	かのじょ は せいじ に むかんしん です	
\\	彼女[かのじょ]は 政治[せいじ]に
\\	です。			
\\	無関係	無関係[むかんけい]	むかんけい	
\\	彼はこの事件と無関係です。	彼[かれ]はこの 事件[じけん]と 無関係[むかんけい]です。	かれ は この じけん と むかんけい です	
\\	彼[かれ]はこの 事件[じけん]と
\\	です。			
\\	野党	野党[やとう]	やとう	
\\	野党が与党を厳しく非難していましたね。	野党[やとう]が 与党[よとう]を 厳[きび]しく 非難[ひなん]していましたね。	やとう が よとう を きびしく ひなん して いました ね	
\\	が 与党[よとう]を 厳[きび]しく 非難[ひなん]していましたね。			
\\	選択	選択[せんたく]	せんたく	
\\	この5種類から選択できます。	この5 種類[しゅるい]から 選択[せんたく]できます。	この 
\\	しゅるい から せんたく できます	
\\	この5 種類[しゅるい]から
\\	できます。			
\\	脳	脳[のう]	のう	
\\	そのクイズ番組は脳を刺激するね。	そのクイズ 番組[ばんぐみ]は 脳[のう]を 刺激[しげき]するね。	その くいずばんぐみ は のう を しげき する ね	
\\	そのクイズ 番組[ばんぐみ]は
\\	を 刺激[しげき]するね。			
\\	領土	領土[りょうど]	りょうど	
\\	ここから先は隣の国の領土です。	ここから 先[さき]は 隣[となり]の 国[くに]の 領土[りょうど]です。	ここ から さき は となり の くに の りょうど です	
\\	ここから 先[さき]は 隣[となり]の 国[くに]の
\\	です。			
\\	領事館	領事館[りょうじかん]	りょうじかん	
\\	彼はアメリカ領事館に出かけたよ。	彼[かれ]はアメリカ 領事館[りょうじかん]に 出[で]かけたよ。	かれ は あめりか りょうじかん に でかけた よ	
\\	彼[かれ]はアメリカ
\\	に 出[で]かけたよ。			
\\	領収書	領収書[りょうしゅうしょ]	りょうしゅうしょ	
\\	領収書をください。	領収書[りょうしゅうしょ]をください。	りょうしゅうしょ を ください	
\\	をください。			
\\	無責任	無責任[むせきにん]	むせきにん	
\\	無責任な行動は許されません。	無責任[むせきにん]な 行動[こうどう]は 許[ゆる]されません。	むせきにん な こうどう は ゆるされません	
\\	な 行動[こうどう]は 許[ゆる]されません。			
\\	転勤	転勤[てんきん]	てんきん	
\\	彼は大阪に転勤しました。	彼[かれ]は 大阪[おおさか]に 転勤[てんきん]しました。	かれ は おおさか に てんきん しました	
\\	彼[かれ]は 大阪[おおさか]に
\\	しました。			
\\	雇う	雇[やと]う	やとう	
\\	新しい社員を雇いました。	新[あたら]しい 社員[しゃいん]を 雇[やと]いました。	あたらしい しゃいん を やといました	
\\	新[あたら]しい 社員[しゃいん]を
\\	退職	退職[たいしょく]	たいしょく	
\\	私は今月一杯で退職します。	私[わたし]は 今月一杯[こんげつ いっぱい]で 退職[たいしょく]します。	わたし は こんげつ いっぱい で たいしょく します	
\\	私[わたし]は 今月一杯[こんげつ いっぱい]で
\\	します。			
\\	退学	退学[たいがく]	たいがく	
\\	弟は高校を退学しました。	弟[おとうと]は 高校[こうこう]を 退学[たいがく]しました。	おとうと は こうこう を たいがく しました	
\\	弟[おとうと]は 高校[こうこう]を
\\	しました。			
\\	理屈	理屈[りくつ]	りくつ	
\\	彼には理屈が通じない。	彼[かれ]には 理屈[りくつ]が 通[つう]じない。	かれ に は りくつ が つうじない	
\\	彼[かれ]には
\\	が 通[つう]じない。			
\\	退屈	退屈[たいくつ]	たいくつ	
\\	校長の退屈な話が続いたんだよ。	校長[こうちょう]の 退屈[たいくつ]な 話[はなし]が 続[つづ]いたんだよ。	こうちょう の たいくつ な はなし が つづいた ん だ よ	
\\	校長[こうちょう]の
\\	な 話[はなし]が 続[つづ]いたんだよ。			
\\	用件	用件[ようけん]	ようけん	
\\	用件をメモしておきました。	用件[ようけん]をメモしておきました。	ようけん を めも して おきました	
\\	をメモしておきました。			
\\	追加	追加[ついか]	ついか	
\\	オーダーを追加しました。	オーダーを 追加[ついか]しました。	おーだー を ついか しました	
\\	オーダーを
\\	しました。			
\\	見比べる	見比[みくら]べる	みくらべる	
\\	彼女は二人の顔を見比べたの。	彼女[かのじょ]は 二人[ふたり]の 顔[かお]を 見比[みくら]べたの。	かのじょ は ふたり の かお を みくらべた の	
\\	彼女[かのじょ]は 二人[ふたり]の 顔[かお]を
\\	の。			
\\	評判	評判[ひょうばん]	ひょうばん	
\\	評判の良いレストランに行きました。	評判[ひょうばん]の 良[い]いレストランに 行[い]きました。	ひょうばん の いい れすとらん に いきました	
\\	の 良[い]いレストランに 行[い]きました。			
\\	理想	理想[りそう]	りそう	
\\	彼は高い理想を持っているの。	彼[かれ]は 高[たか]い 理想[りそう]を 持[も]っているの。	かれ は たかい りそう を もって いる の	
\\	彼[かれ]は 高[たか]い
\\	を 持[も]っているの。			
\\	象	象[ぞう]	ぞう	
\\	象に乗ってみたいです。	象[ぞう]に 乗[の]ってみたいです。	ぞう に のって みたい です	
\\	に 乗[の]ってみたいです。			
\\	負傷	負傷[ふしょう]	ふしょう	
\\	その事故で多くの人が負傷したの。	その 事故[じこ]で 多[おお]くの 人[ひと]が 負傷[ふしょう]したの。	その じこ で おおく の ひと が ふしょう した の	
\\	その 事故[じこ]で 多[おお]くの 人[ひと]が
\\	したの。			
\\	立て替える	立[た]て 替[か]える	たてかえる	
\\	お金がないなら私が立て替えておきます。	お 金[かね]がないなら 私[わたし]が 立[た]て 替[か]えておきます。	おかね が ない なら わたし が たてかえて おきます	
\\	お 金[かね]がないなら 私[わたし]が
\\	無害	無害[むがい]	むがい	
\\	この農薬は人には無害です。	この 農薬[のうやく]は 人[ひと]には 無害[むがい]です。	この のうやく は ひと に は むがい です	
\\	この 農薬[のうやく]は 人[ひと]には
\\	です。			
\\	被害	被害[ひがい]	ひがい	
\\	彼の家は台風の被害にあったの。	彼[かれ]の 家[いえ]は 台風[たいふう]の 被害[ひがい]にあったの。	かれ の いえ は たいふう の ひがい に あった の	
\\	彼[かれ]の 家[いえ]は 台風[たいふう]の
\\	にあったの。			
\\	破る	破[やぶ]る	やぶる	
\\	彼は強敵を見事に破ったね。	彼[かれ]は 強敵[きょうてき]を 見事[みごと]に 破[やぶ]ったね。	かれ は きょうてき を みごと に やぶった ね	
\\	彼[かれ]は 強敵[きょうてき]を 見事[みごと]に
\\	ね。			
\\	破壊	破壊[はかい]	はかい	
\\	自然の破壊が進んでいるのよ。	自然[しぜん]の 破壊[はかい]が 進[すす]んでいるのよ。	しぜん の はかい が すすんで いる の よ	
\\	自然[しぜん]の
\\	が 進[すす]んでいるのよ。			
\\	犯人	犯人[はんにん]	はんにん	
\\	あいつが犯人です。	あいつが 犯人[はんにん]です。	あいつ が はんにん です	
\\	あいつが
\\	です。			
\\	犯罪	犯罪[はんざい]	はんざい	
\\	最近犯罪が減っています。	最近[さいきん] 犯罪[はんざい]が 減[へ]っています。	さいきん はんざい が へって います	
\\	最近[さいきん]
\\	が 減[へ]っています。			
\\	罪	罪[つみ]	つみ	
\\	彼の罪は重いな。	彼[かれ]の 罪[つみ]は 重[おも]いな。	かれ の つみ は おもい な	
\\	彼[かれ]の
\\	は 重[おも]いな。			
\\	盗難	盗難[とうなん]	とうなん	
\\	昨夜、自転車の盗難にあったよ。	昨夜[さくや]、 自転車[じてんしゃ]の 盗難[とうなん]にあったよ。	さくや じてんしゃ の とうなん に あった よ	
\\	昨夜[さくや]、 自転車[じてんしゃ]の
\\	にあったよ。			
\\	逃げ出す	逃[に]げ 出[だ]す	にげだす	
\\	トラがおりから逃げ出しました。	トラがおりから 逃[に]げ 出[だ]しました。	とら が おり から にげだしました	
\\	トラがおりから
\\	逃がす	逃[に]がす	にがす	
\\	釣った魚を逃がしました。	釣[つ]った 魚[さかな]を 逃[に]がしました。	つった さかな を にがしました	
\\	釣[つ]った 魚[さかな]を
\\	逃げ道	逃[に]げ 道[みち]	にげみち	
\\	失敗した時の逃げ道を考えたの。	失敗[しっぱい]した 時[とき]の 逃[に]げ 道[みち]を 考[かんが]えたの。	しっぱい した とき の にげみち を かんがえた の	
\\	失敗[しっぱい]した 時[とき]の
\\	を 考[かんが]えたの。			
\\	混ぜる	混[ま]ぜる	まぜる	
\\	カレーにヨーグルトを入れて、よく混ぜてください。	カレーにヨーグルトを 入[い]れて、よく 混[ま]ぜてください。	かれー に よーぐると を いれて よく まぜて ください	
\\	カレーにヨーグルトを 入[い]れて、よく
\\	ください。			
\\	混ざる	混[ま]ざる	まざる	
\\	水と油は混ざりません。	水[みず]と 油[あぶら]は 混[ま]ざりません。	みず と あぶら は まざりません	
\\	水[みず]と 油[あぶら]は
\\	混じる	混[ま]じる	まじる	
\\	電話の声に雑音が混じっていたの。	電話[でんわ]の 声[こえ]に 雑音[ざつおん]が 混[ま]じっていたの。	でんわ の こえ に ざつおん が まじって いた の	
\\	電話[でんわ]の 声[こえ]に 雑音[ざつおん]が
\\	の。			
\\	統合	統合[とうごう]	とうごう	
\\	三つの町が統合されて新しい市が誕生したのよ。	三[みっ]つの 町[まち]が 統合[とうごう]されて 新[あたら]しい 市[し]が 誕生[たんじょう]したのよ。	みっつ の まち が とうごう されて あたらしい し が たんじょう した の よ	
\\	三[みっ]つの 町[まち]が
\\	されて 新[あたら]しい 市[し]が 誕生[たんじょう]したのよ。			
\\	統一	統一[とういつ]	とういつ	
\\	文字の大きさは統一してください。	文字[もじ]の 大[おお]きさは 統一[とういつ]してください。	もじ の おおきさ は とういつ して ください	
\\	文字[もじ]の 大[おお]きさは
\\	してください。			
\\	統計	統計[とうけい]	とうけい	
\\	これは昨年の売り上げの統計です。	これは 昨年[さくねん]の 売[う]り 上[あ]げの 統計[とうけい]です。	これ は さくねん の うりあげ の とうけい です	
\\	これは 昨年[さくねん]の 売[う]り 上[あ]げの
\\	です。			
\\	総会	総会[そうかい]	そうかい	
\\	来月の総会は東京で行われます。	来月[らいげつ]の 総会[そうかい]は 東京[とうきょう]で 行[おこな]われます。	らいげつ の そうかい は とうきょう で おこなわれます	
\\	来月[らいげつ]の
\\	は 東京[とうきょう]で 行[おこな]われます。			
\\	総合	総合[そうごう]	そうごう	
\\	全員の意見を総合してみましょう。	全員[ぜんいん]の 意見[いけん]を 総合[そうごう]してみましょう。	ぜんいん の いけん を そうごう して みましょう	
\\	全員[ぜんいん]の 意見[いけん]を
\\	してみましょう。			
\\	総理	総理[そうり]	そうり	
\\	総理はヨーロッパを訪問中です。	総理[そうり]はヨーロッパを 訪問中[ほうもんちゅう]です。	そうり は よーろっぱ を ほうもんちゅう です	
\\	はヨーロッパを 訪問中[ほうもんちゅう]です。			
\\	総数	総数[そうすう]	そうすう	
\\	参加者の総数は705人でした。	参加者[さんかしゃ]の 総数[そうすう]は705 人[にん]でした。	さんかしゃ の そうすう は 
\\	にん でした	
\\	参加者[さんかしゃ]の
\\	は705 人[にん]でした。			
\\	湖	湖[みずうみ]	みずうみ	
\\	湖でスケートをした。	湖[みずうみ]でスケートをした。	みずうみ で すけーと を した	
\\	でスケートをした。			
\\	深める	深[ふか]める	ふかめる	
\\	彼女は異文化に対する理解を深めたね。	彼女[かのじょ]は 異文化[いぶんか]に 対[たい]する 理解[りかい]を 深[ふか]めたね。	かのじょ は いぶんか に たいする りかい を ふかめた ね	
\\	彼女[かのじょ]は 異文化[いぶんか]に 対[たい]する 理解[りかい]を
\\	ね。			
\\	深まる	深[ふか]まる	ふかまる	
\\	二人の愛はますます深まっているね。	二人[ふたり]の 愛[あい]はますます 深[ふか]まっているね。	ふたり の あい は ますます ふかまっている ね	
\\	二人[ふたり]の 愛[あい]はますます
\\	ね。			
\\	見落とす	見落[みお]とす	みおとす	
\\	間違いを見落としたの。	間違[まちが]いを 見落[みお]としたの。	まちがい を みおとした の	
\\	間違[まちが]いを
\\	の。			
\\	落第	落第[らくだい]	らくだい	
\\	このテストに失敗したら落第です。	このテストに 失敗[しっぱい]したら 落第[らくだい]です。	この てすと に しっぱい したら らくだい です	
\\	このテストに 失敗[しっぱい]したら
\\	です。			
\\	適用	適用[てきよう]	てきよう	
\\	この場合、保険が適用されますか。	この 場合[ばあい]、 保険[ほけん]が 適用[てきよう]されますか。	この ばあい ほけん が てきよう されます か	
\\	この 場合[ばあい]、 保険[ほけん]が
\\	されますか。			
\\	適切	適切[てきせつ]	てきせつ	
\\	彼は適切な言葉で説明してくれました。	彼[かれ]は 適切[てきせつ]な 言葉[ことば]で 説明[せつめい]してくれました。	かれ は てきせつ な ことば で せつめい して くれました	
\\	彼[かれ]は
\\	な 言葉[ことば]で 説明[せつめい]してくれました。			
\\	適応	適応[てきおう]	てきおう	
\\	彼はどんな環境にも適応できます。	彼[かれ]はどんな 環境[かんきょう]にも 適応[てきおう]できます。	かれ は どんな かんきょう に も てきおう できます	
\\	彼[かれ]はどんな 環境[かんきょう]にも
\\	できます。			
\\	適する	適[てき]する	てきする	
\\	彼はこの仕事に適しています。	彼[かれ]はこの 仕事[しごと]に 適[てき]しています。	かれ は この しごと に てきしています	
\\	彼[かれ]はこの 仕事[しごと]に
\\	適当	適当[てきとう]	てきとう	
\\	その質問の適当な答えが見つかりません。	その 質問[しつもん]の 適当[てきとう]な 答[こた]えが 見[み]つかりません。	その しつもん の てきとう な こたえ が みつかりません	
\\	その 質問[しつもん]の
\\	な 答[こた]えが 見[み]つかりません。			
\\	適度	適度[てきど]	てきど	
\\	健康のため、適度な運動が必要です。	健康[けんこう]のため、 適度[てきど]な 運動[うんどう]が 必要[ひつよう]です。	けんこう の ため てきど な うんどう が ひつよう です	
\\	健康[けんこう]のため、
\\	な 運動[うんどう]が 必要[ひつよう]です。			
\\	風景	風景[ふうけい]	ふうけい	
\\	私は山の風景が好きです。	私[わたし]は 山[やま]の 風景[ふうけい]が 好[す]きです。	わたし は やま の ふうけい が すき です	
\\	私[わたし]は 山[やま]の
\\	が 好[す]きです。			
\\	響く	響[ひび]く	ひびく	
\\	彼の声はよく響きます。	彼[かれ]の 声[こえ]はよく 響[ひび]きます。	かれ の こえ は よく ひびきます	
\\	彼[かれ]の 声[こえ]はよく
\\	測定	測定[そくてい]	そくてい	
\\	これから身長と体重を測定します。	これから 身長[しんちょう]と 体重[たいじゅう]を 測定[そくてい]します。	これから しんちょう と たいじゅう を そくてい します	
\\	これから 身長[しんちょう]と 体重[たいじゅう]を
\\	します。			
\\	測る	測[はか]る	はかる	
\\	このドアの高さを測ってください。	このドアの 高[たか]さを 測[はか]ってください。	この どあ の たかさ を はかって ください	
\\	このドアの 高[たか]さを
\\	ください。			
\\	陽気	陽気[ようき]	ようき	
\\	彼女はとても陽気です。	彼女[かのじょ]はとても 陽気[ようき]です。	かのじょ は とても ようき です	
\\	彼女[かのじょ]はとても
\\	です。			
\\	電球	電球[でんきゅう]	でんきゅう	
\\	電球が切れたので交換しましょう。	電球[でんきゅう]が 切[き]れたので 交換[こうかん]しましょう。	でんきゅう が きれた の で こうかん しましょう	
\\	が 切[き]れたので 交換[こうかん]しましょう。			
\\	秘書	秘書[ひしょ]	ひしょ	
\\	私の秘書はとても優秀です。	私[わたし]の 秘書[ひしょ]はとても 優秀[ゆうしゅう]です。	わたし の ひしょ は とても ゆうしゅう です	
\\	私[わたし]の
\\	はとても 優秀[ゆうしゅう]です。			
\\	貴い	貴[とうと]い	とうとい	
\\	その事故で貴い命が失われました。	その 事故[じこ]で 貴[とうと]い 命[いのち]が 失[うしな]われました。	その じこ で とうとい いのち が うしなわれました	
\\	その 事故[じこ]で
\\	命[いのち]が 失[うしな]われました。			
\\	追跡	追跡[ついせき]	ついせき	
\\	パトカーが車を追跡しているわ。	パトカーが 車[くるま]を 追跡[ついせき]しているわ。	ぱとかー が くるま を ついせき して いる わ	
\\	パトカーが 車[くるま]を
\\	しているわ。			
\\	近寄る	近寄[ちかよ]る	ちかよる	
\\	彼に近寄らないで。	彼[かれ]に 近寄[ちかよ]らないで。	かれ に ちかよらない で	
\\	彼[かれ]に
\\	離れる	離[はな]れる	はなれる	
\\	実家を4年間離れていました。	実家[じっか]を4 年間[ねんかん] 離[はな]れていました。	じっか を 
\\	ねんかん はなれて いました	
\\	実家[じっか]を4 年間[ねんかん]
\\	離す	離[はな]す	はなす	
\\	子供の手を離さないでください。	子供[こども]の 手[て]を 離[はな]さないでください。	こども の て を はなさないで ください	
\\	子供[こども]の 手[て]を
\\	ください。			
\\	範囲	範囲[はんい]	はんい	
\\	知っている範囲で教えてください。	知[し]っている 範囲[はんい]で 教[おし]えてください。	しって いる はんい で おしえて ください	
\\	知[し]っている
\\	で 教[おし]えてください。			
\\	隣り合う	隣[とな]り 合[あ]う	となりあう	
\\	この町は山と海が隣り合っています。	この 町[まち]は 山[やま]と 海[うみ]が 隣[とな]り 合[あ]っています。	この まち は やま と うみ が となりあって います	
\\	この 町[まち]は 山[やま]と 海[うみ]が
\\	継ぐ	継[つ]ぐ	つぐ	
\\	彼は父親の店を継いだの。	彼[かれ]は 父親[ちちおや]の 店[みせ]を 継[つ]いだの。	かれ は ちちおや の みせ を ついだ の	
\\	彼[かれ]は 父親[ちちおや]の 店[みせ]を
\\	の。			
\\	着陸	着陸[ちゃくりく]	ちゃくりく	
\\	飛行機が無事着陸したわね。	飛行機[ひこうき]が 無事[ぶじ] 着陸[ちゃくりく]したわね。	ひこうき が ぶじ ちゃくりく した わ ね	
\\	飛行機[ひこうき]が 無事[ぶじ]
\\	したわね。			
\\	離陸	離陸[りりく]	りりく	
\\	まもなく飛行機が離陸します。	まもなく 飛行機[ひこうき]が 離陸[りりく]します。	まもなく ひこうき が りりく します	
\\	まもなく 飛行機[ひこうき]が
\\	します。			
\\	陸上	陸上[りくじょう]	りくじょう	
\\	あの動物は陸上で生活しています。	あの 動物[どうぶつ]は 陸上[りくじょう]で 生活[せいかつ]しています。	あの どうぶつ は りくじょう で せいかつ して います	
\\	あの 動物[どうぶつ]は
\\	で 生活[せいかつ]しています。			
\\	陸	陸[りく]	りく	
\\	ウミガメが陸に上がってきたんだ。	ウミガメが 陸[りく]に 上[あ]がってきたんだ。	うみがめ が りく に あがって きた ん だ	
\\	ウミガメが
\\	に 上[あ]がってきたんだ。			
\\	端	端[はし]	はし	
\\	辞書は本棚の端にあります。	辞書[じしょ]は 本棚[ほんだな]の 端[はし]にあります。	じしょ は ほんだな の はし に あります	
\\	辞書[じしょ]は 本棚[ほんだな]の
\\	にあります。			
\\	訪問	訪問[ほうもん]	ほうもん	
\\	来週、妻の実家を訪問します。	来週[らいしゅう]、 妻[つま]の 実家[じっか]を 訪問[ほうもん]します。	らいしゅう つま の じっか を ほうもん します	
\\	来週[らいしゅう]、 妻[つま]の 実家[じっか]を
\\	します。			
\\	百貨店	百貨店[ひゃっかてん]	ひゃっかてん	
\\	友達と百貨店で買い物をしました。	友達[ともだち]と 百貨店[ひゃっかてん]で 買[か]い 物[もの]をしました。	ともだち と ひゃっかてん で かいもの を しました	
\\	友達[ともだち]と
\\	で 買[か]い 物[もの]をしました。			
\\	輸送	輸送[ゆそう]	ゆそう	
\\	この商品はトラックで輸送します。	この 商品[しょうひん]はトラックで 輸送[ゆそう]します。	この しょうひん は とらっく で ゆそう します	
\\	この 商品[しょうひん]はトラックで
\\	します。			
\\	縦	縦[たて]	たて	
\\	縦2列に並んでください。	縦[たて] 
\\	列[れつ]に 並[なら]んでください。	たて 
\\	れつ に ならんで ください	
\\	列[れつ]に 並[なら]んでください。			
\\	縦書き	縦書[たてが]き	たてがき	
\\	作文は縦書きで書いてください。	作文[さくぶん]は 縦書[たてが]きで 書[か]いてください。	さくぶん は たてがき で かいてください	
\\	作文[さくぶん]は
\\	で 書[か]いてください。			
\\	滞在	滞在[たいざい]	たいざい	
\\	日本では北海道に滞在しました。	日本[にっぽん]では 北海道[ほっかいどう]に 滞在[たいざい]しました。	にっぽん で は ほっかいどう に たいざい しました	
\\	日本[にっぽん]では 北海道[ほっかいどう]に
\\	しました。			
\\	熱帯	熱帯[ねったい]	ねったい	
\\	店に熱帯の果物が並んでいますよ。	店[みせ]に 熱帯[ねったい]の 果物[くだもの]が 並[なら]んでいますよ。	みせ に ねったい の くだもの が ならんで います よ	
\\	店[みせ]に
\\	の 果物[くだもの]が 並[なら]んでいますよ。			
\\	見守る	見守[みまも]る	みまもる	
\\	皆で暖かく見守りましょう。	皆[みんな]で 暖[あたた]かく 見守[みまも]りましょう。	みんな で あたたかく みまもりましょう	
\\	皆[みんな]で 暖[あたた]かく
\\	留守番	留守番[るすばん]	るすばん	
\\	私は留守番の者です。	私[わたし]は 留守番[るすばん]の 者[もの]です。	わたし は るすばん の もの です	
\\	私[わたし]は
\\	の 者[もの]です。			
\\	要項	要項[ようこう]	ようこう	
\\	募集要項をよくお読みください。	募集[ぼしゅう] 要項[ようこう]をよくお 読[よ]みください。	ぼしゅう ようこう を よく お よみ ください	
\\	募集[ぼしゅう]
\\	をよくお 読[よ]みください。			
\\	目印	目印[めじるし]	めじるし	
\\	私の家は赤い屋根が目印です。	私[わたし]の 家[いえ]は 赤[あか]い 屋根[やね]が 目印[めじるし]です。	わたし の いえ は あかい やね が めじるし です	
\\	私[わたし]の 家[いえ]は 赤[あか]い 屋根[やね]が
\\	です。			
\\	細長い	細長[ほそなが]い	ほそながい	
\\	その細長い棒を取ってください。	その 細長[ほそなが]い 棒[ぼう]を 取[と]ってください。	その ほそながい ぼう を とって ください	
\\	その
\\	棒[ぼう]を 取[と]ってください。			
\\	載る	載[の]る	のる	
\\	先生の論文が雑誌に載りましたよ。	先生[せんせい]の 論文[ろんぶん]が 雑誌[ざっし]に 載[の]りましたよ。	せんせい の ろんぶん が ざっし に のりました よ	
\\	先生[せんせい]の 論文[ろんぶん]が 雑誌[ざっし]に
\\	よ。			
\\	積む	積[つ]む	つむ	
\\	車に荷物を積んでください。	車[くるま]に 荷物[にもつ]を 積[つ]んでください。	くるま に にもつ を つんで ください	
\\	車[くるま]に 荷物[にもつ]を
\\	ください。			
\\	面積	面積[めんせき]	めんせき	
\\	この土地の面積はどれ位ですか。	この 土地[とち]の 面積[めんせき]はどれ 位[くらい]ですか。	この とち の めんせき は どれ くらい です か	
\\	この 土地[とち]の
\\	はどれ 位[くらい]ですか。			
\\	翻訳	翻訳[ほんやく]	ほんやく	
\\	彼の翻訳は分かりやすいですよ。	彼[かれ]の 翻訳[ほんやく]は 分[わ]かりやすいですよ。	かれ の ほんやく は わかりやすい です よ	
\\	彼[かれ]の
\\	は 分[わ]かりやすいですよ。			
\\	通訳	通訳[つうやく]	つうやく	
\\	私は通訳です。	私[わたし]は 通訳[つうやく]です。	わたし は つうやく です	
\\	私[わたし]は
\\	です。			
\\	申し訳ない	申[もう]し 訳[わけ]ない	もうしわけない	
\\	彼には申し訳ないことをした。	彼[かれ]には 申[もう]し 訳[わけ]ないことをした。	かれ に は もうしわけない こと を した	
\\	彼[かれ]には
\\	ことをした。			
\\	訳	訳[わけ]	わけ	
\\	そんな訳で、私は仕事を辞めました。	そんな 訳[わけ]で、 私[わたし]は 仕事[しごと]を 辞[や]めました。	そんな わけ で わたし は しごと を やめました	
\\	そんな
\\	で、 私[わたし]は 仕事[しごと]を 辞[や]めました。			
\\	申し訳	申[もう]し 訳[わけ]	もうしわけ	
\\	申し訳ございません。	申[もう]し 訳[わけ]ございません。	もうしわけ ございません	
\\	ございません。			
\\	訂正	訂正[ていせい]	ていせい	
\\	間違いを訂正しました。	間違[まちが]いを 訂正[ていせい]しました。	まちがい を ていせい しました	
\\	間違[まちが]いを
\\	しました。			
\\	討論	討論[とうろん]	とうろん	
\\	その問題についてクラスで討論しました。	その 問題[もんだい]についてクラスで 討論[とうろん]しました。	その もんだい に ついて くらす で とうろん しました	
\\	その 問題[もんだい]についてクラスで
\\	しました。			
\\	添える	添[そ]える	そえる	
\\	贈り物に手書きのカードを添えました。	贈[おく]り 物[もの]に 手書[てが]きのカードを 添[そ]えました。	おくりもの に てがき の かーど を そえました	
\\	贈[おく]り 物[もの]に 手書[てが]きのカードを
\\	添う	添[そ]う	そう	
\\	あなたの期待に添えなくてすみません。	あなたの 期待[きたい]に 添[そ]えなくてすみません。	あなた の きたい に そえなくて すみません	
\\	あなたの 期待[きたい]に
\\	すみません。			
\\	近付ける	近付[ちかづ]ける	ちかづける	
\\	私は顔を窓に近付けたんだ。	私[わたし]は 顔[かお]を 窓[まど]に 近付[ちかづ]けたんだ。	わたし は かお を まど に ちかづけた ん だ	
\\	私[わたし]は 顔[かお]を 窓[まど]に
\\	んだ。			
\\	省く	省[はぶ]く	はぶく	
\\	時間がないので詳細は省きます。	時間[じかん]がないので 詳細[しょうさい]は 省[はぶ]きます。	じかん が ない の で しょうさい は はぶきます	
\\	時間[じかん]がないので 詳細[しょうさい]は
\\	略す	略[りゃく]す	りゃくす	
\\	国際連合を略して国連といいます。	国際連合[こくさいれんごう]を 略[りゃく]して 国連[こくれん]といいます。	こくさいれんごう を りゃく して こくれん と いいます	
\\	国際連合[こくさいれんごう]を
\\	国連[こくれん]といいます。			
\\	略	略[りゃく]	りゃく	
\\	は何の略か知っていますか。	
\\	[あいてぃー]は 何[なん]の 略[りゃく]か 知[し]っていますか。	あいてぃー は なん の りゃく か しって います か	
\\	[あいてぃー]は 何[なん]の
\\	か 知[し]っていますか。			
\\	道順	道順[みちじゅん]	みちじゅん	
\\	家から学校までの道順を教えてください。	家[いえ]から 学校[がっこう]までの 道順[みちじゅん]を 教[おし]えてください。	いえ から がっこう まで の みちじゅん を おしえて ください	
\\	家[いえ]から 学校[がっこう]までの
\\	を 教[おし]えてください。			
\\	用例	用例[ようれい]	ようれい	
\\	用例を使って説明してください。	用例[ようれい]を 使[つか]って 説明[せつめい]してください。	ようれい を つかって せつめい して ください	
\\	を 使[つか]って 説明[せつめい]してください。			
\\	途中	途中[とちゅう]	とちゅう	
\\	途中で30分ほど休みましょう。	途中[とちゅう]で30 分[ぷん]ほど 休[やす]みましょう。	とちゅう で 
\\	ぷん ほど やすみましょう	
\\	で30 分[ぷん]ほど 休[やす]みましょう。			
\\	用途	用途[ようと]	ようと	
\\	この道具の用途を説明します。	この 道具[どうぐ]の 用途[ようと]を 説明[せつめい]します。	この どうぐ の ようと を せつめい します	
\\	この 道具[どうぐ]の
\\	を 説明[せつめい]します。			
\\	緩める	緩[ゆる]める	ゆるめる	
\\	食べ過ぎたのでベルトを緩めたの。	食[た]べ 過[す]ぎたのでベルトを 緩[ゆる]めたの。	たべすぎた の で べると を ゆるめた の	
\\	食[た]べ 過[す]ぎたのでベルトを
\\	の。			
\\	緩やか	緩[ゆる]やか	ゆるやか	
\\	ここから先は緩やかな坂になっています。	ここから 先[さき]は 緩[ゆる]やかな 坂[さか]になっています。	ここ から さき は ゆるやか な さか に なって います	
\\	ここから 先[さき]は
\\	な 坂[さか]になっています。			
\\	緩む	緩[ゆる]む	ゆるむ	
\\	彼は気が緩んでいます。	彼[かれ]は 気[き]が 緩[ゆる]んでいます。	かれ は き が ゆるんで います	
\\	彼[かれ]は 気[き]が
\\	養う	養[やしな]う	やしなう	
\\	私は3人の子供を養っています。	私[わたし]は3 人[にん]の 子供[こども]を 養[やしな]っています。	わたし は 
\\	にん の こども を やしなって います	
\\	私[わたし]は3 人[にん]の 子供[こども]を
\\	います。			
\\	豊か	豊[ゆた]か	ゆたか	
\\	彼は心が豊かな人です。	彼[かれ]は 心[こころ]が 豊[ゆた]かな 人[ひと]です。	かれ は こころ が ゆたか な ひと です	
\\	彼[かれ]は 心[こころ]が
\\	な 人[ひと]です。			
\\	豊富	豊富[ほうふ]	ほうふ	
\\	この果物はビタミンが豊富です。	この 果物[くだもの]はビタミンが 豊富[ほうふ]です。	この くだもの は びたみん が ほうふ です	
\\	この 果物[くだもの]はビタミンが
\\	です。			
\\	目茶苦茶	目茶苦茶[めちゃくちゃ]	めちゃくちゃ	
\\	彼の運転は目茶苦茶です。	彼[かれ]の 運転[うんてん]は 目茶苦茶[めちゃくちゃ]です。	かれ の うんてん は めちゃくちゃ です	
\\	彼[かれ]の 運転[うんてん]は
\\	です。			
\\	疲労	疲労[ひろう]	ひろう	
\\	部下が疲労で倒れたの。	部下[ぶか]が 疲労[ひろう]で 倒[たお]れたの。	ぶか が ひろう で たおれた の	
\\	部下[ぶか]が
\\	で 倒[たお]れたの。			
\\	貧しい	貧[まず]しい	まずしい	
\\	彼は貧しい家庭で育ったんだ。	彼[かれ]は 貧[まず]しい 家庭[かてい]で 育[そだ]ったんだ。	かれ は まずしい かてい で そだった ん だ	
\\	彼[かれ]は
\\	家庭[かてい]で 育[そだ]ったんだ。			
\\	貧乏人	貧乏人[びんぼうにん]	びんぼうにん	
\\	貧乏人が大金持ちになることもあるさ。	貧乏人[びんぼうにん]が 大金持[おおがねも]ちになることもあるさ。	びんぼうにん が おおがねもち に なる こと も ある さ	
\\	が 大金持[おおがねも]ちになることもあるさ。			
\\	辛い	辛[つら]い	つらい	
\\	この仕事は辛いです。	この 仕事[しごと]は 辛[つら]いです。	この しごと は つらい です	
\\	この 仕事[しごと]は
\\	です。			
\\	福祉	福祉[ふくし]	ふくし	
\\	姉は福祉の勉強をしています。	姉[あね]は 福祉[ふくし]の 勉強[べんきょう]をしています。	あね は ふくし の べんきょう を して います	
\\	姉[あね]は
\\	の 勉強[べんきょう]をしています。			
\\	農民	農民[のうみん]	のうみん	
\\	中世の農民は貧しかった。	中世[ちゅうせい]の 農民[のうみん]は 貧[まず]しかった。	ちゅうせい の のうみん は まずしかった	
\\	中世[ちゅうせい]の
\\	は 貧[まず]しかった。			
\\	農家	農家[のうか]	のうか	
\\	彼は農家に生まれました。	彼[かれ]は 農家[のうか]に 生[う]まれました。	かれ は のうか に うまれました	
\\	彼[かれ]は
\\	に 生[う]まれました。			
\\	農村	農村[のうそん]	のうそん	
\\	私は農村で育ちました。	私[わたし]は 農村[のうそん]で 育[そだ]ちました。	わたし は のうそん で そだちました	
\\	私[わたし]は
\\	で 育[そだ]ちました。			
\\	短縮	短縮[たんしゅく]	たんしゅく	
\\	今日は授業を1時間に短縮します。	今日[きょう]は 授業[じゅぎょう]を1 時間[じかん]に 短縮[たんしゅく]します。	きょう は じゅぎょう を 
\\	じかん に たんしゅく します	
\\	今日[きょう]は 授業[じゅぎょう]を1 時間[じかん]に
\\	します。			
\\	縮める	縮[ちぢ]める	ちぢめる	
\\	彼はタイムを1秒縮めたの。	彼[かれ]はタイムを1 秒[びょう] 縮[ちぢ]めたの。	かれ は たいむ を 
\\	びょう ちぢめた の	
\\	彼[かれ]はタイムを1 秒[びょう]
\\	の。			
\\	縮む	縮[ちぢ]む	ちぢむ	
\\	この服の縮み具合はひどい。	この 服[ふく]の 縮[ちぢ]み 具合[ぐあい]はひどい。	この ふく の ちぢみ ぐあい は ひどい	
\\	この 服[ふく]の
\\	具合[ぐあい]はひどい。			
\\	縮み	縮[ちぢ]み	ちぢみ	
\\	この服の縮み具合はひどい。	この 服[ふく]の 縮[ちぢ]み 具合[ぐあい]はひどい。	この ふく の ちぢみ ぐあい は ひどい	
\\	この 服[ふく]の
\\	具合[ぐあい]はひどい。			
\\	縮まる	縮[ちぢ]まる	ちぢまる	
\\	兄との身長の差が縮まった。	兄[あに]との 身長[しんちょう]の 差[さ]が 縮[ちぢ]まった。	あに と の しんちょう の さ が ちぢまった	
\\	兄[あに]との 身長[しんちょう]の 差[さ]が
\\	追伸	追伸[ついしん]	ついしん	
\\	追伸、お兄さんはお元気ですか。	追伸[ついしん]、お 兄[にい]さんはお 元気[げんき]ですか。	ついしん おにいさん は おげんき です か	
\\	、お 兄[にい]さんはお 元気[げんき]ですか。			
\\	超過	超過[ちょうか]	ちょうか	
\\	この荷物は重量超過です。	この 荷物[にもつ]は 重量[じゅうりょう] 超過[ちょうか]です。	この にもつ は じゅうりょう ちょうか です	
\\	この 荷物[にもつ]は 重量[じゅうりょう]
\\	です。			
\\	膨らむ	膨[ふく]らむ	ふくらむ	
\\	桜のつぼみが膨らんだのね。	桜[さくら]のつぼみが 膨[ふく]らんだのね。	さくら の つぼみ が ふくらんだ の ね	
\\	桜[さくら]のつぼみが
\\	のね。			
\\	膨れる	膨[ふく]れる	ふくれる	
\\	お腹が膨れたら眠くなった。	お 腹[なか]が 膨[ふく]れたら 眠[ねむ]くなった。	おなか が ふくれたら ねむく なった	
\\	お 腹[なか]が
\\	眠[ねむ]くなった。			
\\	礼	礼[れい]	れい	
\\	先生にお礼の手紙を書きました。	先生[せんせい]にお 礼[れい]の 手紙[てがみ]を 書[か]きました。	せんせい に おれい の てがみ を かきました	
\\	先生[せんせい]にお
\\	の 手紙[てがみ]を 書[か]きました。			
\\	程	程[ほど]	ほど	
\\	冗談にも程がある。	冗談[じょうだん]にも 程[ほど]がある。	じょうだん に も ほど が ある	
\\	冗談[じょうだん]にも
\\	がある。			
\\	透明	透明[とうめい]	とうめい	
\\	ゴミは透明な袋に入れて出してください。	ゴミは 透明[とうめい]な 袋[ふくろ]に 入[い]れて 出[だ]してください。	ごみ は とうめい な ふくろ に いれて だして ください	
\\	ゴミは
\\	な 袋[ふくろ]に 入[い]れて 出[だ]してください。			
\\	要望	要望[ようぼう]	ようぼう	
\\	お客様の要望を聞かせてください。	お 客様[きゃくさま]の 要望[ようぼう]を 聞[き]かせてください。	おきゃくさま の ようぼう を きかせて ください	
\\	お 客様[きゃくさま]の
\\	を 聞[き]かせてください。			
\\	独立	独立[どくりつ]	どくりつ	
\\	先日会社から独立しました。	先日会社[せんじつ かいしゃ]から 独立[どくりつ]しました。	せんじつ かいしゃ から どくりつ しました	
\\	先日会社[せんじつ かいしゃ]から
\\	しました。			
\\	独特	独特[どくとく]	どくとく	
\\	彼の服のセンスは独特よね。	彼[かれ]の 服[ふく]のセンスは 独特[どくとく]よね。	かれ の ふく の せんす は どくとく よ ね	
\\	彼[かれ]の 服[ふく]のセンスは
\\	よね。			
\\	独り言	独[ひと]り 言[ごと]	ひとりごと	
\\	彼女はいつも独り言を言うんだ。	彼女[かのじょ]はいつも 独[ひと]り 言[ごと]を 言[い]うんだ。	かのじょ は いつも ひとりごと を いう ん だ	
\\	彼女[かのじょ]はいつも
\\	を 言[い]うんだ。			
\\	身	身[み]	み	
\\	旅行中は身の安全が第一です。	旅行中[りょこうちゅう]は 身[み]の 安全[あんぜん]が 第一[だいいち]です。	りょこうちゅう は み の あんぜん が だいいち です	
\\	旅行中[りょこうちゅう]は
\\	の 安全[あんぜん]が 第一[だいいち]です。			
\\	身近	身近[みぢか]	みぢか	
\\	私にとって動物は身近な存在です。	私[わたし]にとって 動物[どうぶつ]は 身近[みぢか]な 存在[そんざい]です。	わたし に とって どうぶつ は みぢか な そんざい です	
\\	私[わたし]にとって 動物[どうぶつ]は
\\	な 存在[そんざい]です。			
\\	身の回り	身[み]の 回[まわ]り	みのまわり	
\\	身の回りのお世話は私がします。	身[み]の 回[まわ]りのお 世話[せわ]は 私[わたし]がします。	みのまわり の おせわ は わたし が します 。	
\\	のお 世話[せわ]は 私[わたし]がします。			
\\	身なり	身[み]なり	みなり	
\\	身なりのいい人が入ってきた。	身[み]なりのいい 人[ひと]が 入[はい]ってきた。	みなり の いい ひと が はいって きた	
\\	のいい 人[ひと]が 入[はい]ってきた。			
\\	身振り	身振[みぶ]り	みぶり	
\\	彼は身振りを交えて説明してくれたの。	彼[かれ]は 身振[みぶ]りを 交[まじ]えて 説明[せつめい]してくれたの。	かれ は みぶり を まじえて せつめい して くれた の	
\\	彼[かれ]は
\\	を 交[まじ]えて 説明[せつめい]してくれたの。			
\\	独占	独占[どくせん]	どくせん	
\\	この2社が市場を独占しています。	この2 社[しゃ]が 市場[しじょう]を 独占[どくせん]しています。	この 
\\	しゃ が しじょう を どくせん して います	
\\	この2 社[しゃ]が 市場[しじょう]を
\\	しています。			
\\	照らす	照[て]らす	てらす	
\\	月が庭を照らしている。	月[つき]が 庭[にわ]を 照[て]らしている。	つき が にわ を てらして いる	
\\	月[つき]が 庭[にわ]を
\\	照る	照[て]る	てる	
\\	日差しが強く照りつけますね。	日差[ひざ]しが 強[つよ]く 照[て]りつけますね。	ひざし が つよく てりつけます ね	
\\	日差[ひざ]しが 強[つよ]く
\\	ね。			
\\	相互	相互[そうご]	そうご	
\\	チームの中では相互の信頼が大切です。	チームの 中[なか]では 相互[そうご]の 信頼[しんらい]が 大切[たいせつ]です。	ちーむ の なか で は そうご の しんらい が たいせつ です	
\\	チームの 中[なか]では
\\	の 信頼[しんらい]が 大切[たいせつ]です。			
\\	皆	皆[みな]	みな	
\\	私の昇進を皆が喜んでくれたの。	私[わたし]の 昇進[しょうしん]を 皆[みな]が 喜[よろこ]んでくれたの。	わたし の しょうしん を みな が よろこんで くれた の	
\\	私[わたし]の 昇進[しょうしん]を
\\	が 喜[よろこ]んでくれたの。			
\\	誕生	誕生[たんじょう]	たんじょう	
\\	先月、娘が誕生しました。	先月[せんげつ]、 娘[むすめ]が 誕生[たんじょう]しました。	せんげつ むすめ が たんじょう しました	
\\	先月[せんげつ]、 娘[むすめ]が
\\	しました。			
\\	笑い	笑[わら]い	わらい	
\\	あの家は笑いが絶えないね。	あの 家[いえ]は 笑[わら]いが 絶[た]えないね。	あの いえ は わらい が たえない ね	
\\	あの 家[いえ]は
\\	が 絶[た]えないね。			
\\	総裁	総裁[そうさい]	そうさい	
\\	あの人は日本銀行の総裁です。	あの 人[ひと]は 日本銀行[にっぽんぎんこう]の 総裁[そうさい]です。	あの ひと は にっぽんぎんこう の そうさい です	
\\	あの 人[ひと]は 日本銀行[にっぽんぎんこう]の
\\	です。			
\\	迷信	迷信[めいしん]	めいしん	
\\	村の人たちは迷信を信じています。	村[むら]の 人[ひと]たちは 迷信[めいしん]を 信[しん]じています。	むら の ひとたち は めいしん を しんじて います	
\\	村[むら]の 人[ひと]たちは
\\	を 信[しん]じています。			
\\	迷子	迷子[まいご]	まいご	
\\	うちの子が迷子になりました。	うちの 子[こ]が 迷子[まいご]になりました。	うち の こ が まいご に なりました	
\\	うちの 子[こ]が
\\	になりました。			
\\	領域	領域[りょういき]	りょういき	
\\	彼女は料理の腕が素人の領域を超えているわ。	彼女[かのじょ]は 料理[りょうり]の 腕[うで]が 素人[しろうと]の 領域[りょういき]を 超[こ]えているわ。	かのじょ は りょうり の うで が しろうと の りょういき を こえて いる わ	
\\	彼女[かのじょ]は 料理[りょうり]の 腕[うで]が 素人[しろうと]の
\\	を 超[こ]えているわ。			
\\	謎	謎[なぞ]	なぞ	
\\	ピラミッドには謎が多いんだ。	ピラミッドには 謎[なぞ]が 多[おお]いんだ。	ぴらみっど に は なぞ が おおい ん だ	
\\	ピラミッドには
\\	が 多[おお]いんだ。			
\\	著者	著者[ちょしゃ]	ちょしゃ	
\\	この本の著者はイギリス人です。	この 本[ほん]の 著者[ちょしゃ]はイギリス 人[じん]です。	この ほん の ちょしゃ は いぎりすじん です	
\\	この 本[ほん]の
\\	はイギリス 人[じん]です。			
\\	著書	著書[ちょしょ]	ちょしょ	
\\	彼の新しい著書が出版されたね。	彼[かれ]の 新[あたら]しい 著書[ちょしょ]が 出版[しゅっぱん]されたね。	かれ の あたらしい ちょしょ が しゅっぱん された ね	
\\	彼[かれ]の 新[あたら]しい
\\	が 出版[しゅっぱん]されたね。			
\\	預金	預金[よきん]	よきん	
\\	私はこの銀行に預金しています。	私[わたし]はこの 銀行[ぎんこう]に 預金[よきん]しています。	わたし は この ぎんこう に よきん して います	
\\	私[わたし]はこの 銀行[ぎんこう]に
\\	しています。			
\\	納得	納得[なっとく]	なっとく	
\\	彼の説明で納得できました。	彼[かれ]の 説明[せつめい]で 納得[なっとく]できました。	かれ の せつめい で なっとく できました	
\\	彼[かれ]の 説明[せつめい]で
\\	できました。			
\\	燃料	燃料[ねんりょう]	ねんりょう	
\\	車から燃料がもれていますよ。	車[くるま]から 燃料[ねんりょう]がもれていますよ。	くるま から ねんりょう が もれて います よ	
\\	車[くるま]から
\\	がもれていますよ。			
\\	燃やす	燃[も]やす	もやす	
\\	古い手紙を燃やしたんだ。	古[ふる]い 手紙[てがみ]を 燃[も]やしたんだ。	ふるい てがみ を もやした ん だ	
\\	古[ふる]い 手紙[てがみ]を
\\	んだ。			
\\	焼き肉	焼[や]き 肉[にく]	やきにく	
\\	彼は焼き肉が大好きです。	彼[かれ]は 焼[や]き 肉[にく]が 大好[だいす]きです。	かれ は やきにく が だいすき です	
\\	彼[かれ]は
\\	が 大好[だいす]きです。			
\\	焼きそば	焼[や]きそば	やきそば	
\\	お祭りで焼きそばを食べたよ。	お 祭[まつ]りで 焼[や]きそばを 食[た]べたよ。	おまつり で やきそば を たべた よ	
\\	お 祭[まつ]りで
\\	を 食[た]べたよ。			
\\	田植え	田植[たう]え	たうえ	
\\	5月は田植えの季節です。	
\\	月[がつ]は 田植[たう]えの 季節[きせつ]です。	
\\	がつ は たうえ の きせつ です	
\\	月[がつ]は
\\	の 季節[きせつ]です。			
\\	話し言葉	話[はな]し 言葉[ことば]	はなしことば	
\\	話し言葉と書き言葉は少し違いますね。	話[はな]し 言葉[ことば]と 書[か]き 言葉[ことば]は 少[すこ]し 違[ちが]いますね。	はなしことば と かきことば は すこし ちがいます ね	
\\	と 書[か]き 言葉[ことば]は 少[すこ]し 違[ちが]いますね。			
\\	葉っぱ	葉[は]っぱ	はっぱ	
\\	もみじの葉っぱが赤くなりましたね。	もみじの 葉[は]っぱが 赤[あか]くなりましたね。	もみじ の はっぱ が あかく なりました ね	
\\	もみじの
\\	が 赤[あか]くなりましたね。			
\\	腹	腹[はら]	はら	
\\	腹が減って動けない。	腹[はら]が 減[へ]って 動[うご]けない。	はら が へって うごけない	
\\	が 減[へ]って 動[うご]けない。			
\\	腹一杯	腹一杯[はらいっぱい]	はらいっぱい	
\\	夕食を腹一杯食べた。	夕食[ゆうしょく]を 腹一杯[はらいっぱい] 食[た]べた。	ゆうしょく を はらいっぱい たべた	
\\	夕食[ゆうしょく]を
\\	食[た]べた。			
\\	肺	肺[はい]	はい	
\\	彼は肺の病気にかかったのよ。	彼[かれ]は 肺[はい]の 病気[びょうき]にかかったのよ。	かれ は はい の びょうき に かかった の よ	
\\	彼[かれ]は
\\	の 病気[びょうき]にかかったのよ。			
\\	溶ける	溶[と]ける	とける	
\\	暑さでアイスクリームが溶けてしまった。	暑[あつ]さでアイスクリームが 溶[と]けてしまった。	あつさ で あいすくりーむ が とけて しまった	
\\	暑[あつ]さでアイスクリームが
\\	溶かす	溶[と]かす	とかす	
\\	春の太陽が雪を溶かしました。	春[はる]の 太陽[たいよう]が 雪[ゆき]を 溶[と]かしました。	はる の たいよう が ゆき を とかしました	
\\	春[はる]の 太陽[たいよう]が 雪[ゆき]を
\\	滑らか	滑[なめ]らか	なめらか	
\\	この生地は滑らかな手触りが特長です。	この 生地[きじ]は 滑[なめ]らかな 手触[てざわ]りが 特長[とくちょう]です。	この きじ は なめらか な てざわり が とくちょう です	
\\	この 生地[きじ]は
\\	な 手触[てざわ]りが 特長[とくちょう]です。			
\\	診る	診[み]る	みる	
\\	今日、医者に診てもらいました。	今日[きょう]、 医者[いしゃ]に 診[み]てもらいました。	きょう 、 いしゃ に みてもらいました。	
\\	今日[きょう]、 医者[いしゃ]に
\\	編集	編集[へんしゅう]	へんしゅう	
\\	私は雑誌の編集の仕事をしています。	私[わたし]は 雑誌[ざっし]の 編集[へんしゅう]の 仕事[しごと]をしています。	わたし は ざっし の へんしゅう の しごと を して います	
\\	私[わたし]は 雑誌[ざっし]の
\\	の 仕事[しごと]をしています。			
\\	追放	追放[ついほう]	ついほう	
\\	彼は国外に追放されました。	彼[かれ]は 国外[こくがい]に 追放[ついほう]されました。	かれ は こくがい に ついほう されました	
\\	彼[かれ]は 国外[こくがい]に
\\	されました。			
\\	無視	無視[むし]	むし	
\\	彼の意見は無視されたよ。	彼[かれ]の 意見[いけん]は 無視[むし]されたよ。	かれ の いけん は むし された よ	
\\	彼[かれ]の 意見[いけん]は
\\	されたよ。			
\\	銅	銅[どう]	どう	
\\	10円玉は銅でできています。	
\\	円玉[えんだま]は 銅[どう]でできています。	
\\	えんだま は どう で できて います	
\\	円玉[えんだま]は
\\	でできています。			
\\	美術	美術[びじゅつ]	びじゅつ	
\\	弟は美術を専攻しています。	弟[おとうと]は 美術[びじゅつ]を 専攻[せんこう]しています。	おとうと は びじゅつ を せんこう して います	
\\	弟[おとうと]は
\\	を 専攻[せんこう]しています。			
\\	発展	発展[はってん]	はってん	
\\	あの会社は目覚しく発展しているね。	あの 会社[かいしゃ]は 目覚[めざま]しく 発展[はってん]しているね。	あの かいしゃ は めざましく はってん して いる ね	
\\	あの 会社[かいしゃ]は 目覚[めざま]しく
\\	しているね。			
\\	百科事典	百科事典[ひゃっかじてん]	ひゃっかじてん	
\\	めずらしい猫について百科事典で調べたの。	めずらしい 猫[ねこ]について 百科事典[ひゃっかじてん]で 調[しら]べたの。	めずらしい ねこ に ついて ひゃっかじてん で しらべた の	
\\	めずらしい 猫[ねこ]について
\\	で 調[しら]べたの。			
\\	特殊	特殊[とくしゅ]	とくしゅ	
\\	彼は特殊な能力を持っています。	彼[かれ]は 特殊[とくしゅ]な 能力[のうりょく]を 持[も]っています。	かれ は とくしゅ な のうりょく を もって います	
\\	彼[かれ]は
\\	な 能力[のうりょく]を 持[も]っています。			
\\	焼き鳥	焼[や]き 鳥[とり]	やきとり	
\\	焼き鳥はビールに合うな。	焼[や]き 鳥[とり]はビールに 合[あ]うな。	やきとり は びーる に あう な	
\\	はビールに 合[あ]うな。			
\\	鳴らす	鳴[な]らす	ならす	
\\	お坊さんが鐘を鳴らしていますね。	お 坊[ぼう]さんが 鐘[かね]を 鳴[な]らしていますね。	おぼうさん が かね を ならして います ね	
\\	お 坊[ぼう]さんが 鐘[かね]を
\\	ね。			
\\	笑い声	笑[わら]い 声[ごえ]	わらいごえ	
\\	部屋の中から笑い声が聞こえた。	部屋[へや]の 中[なか]から 笑[わら]い 声[ごえ]が 聞[き]こえた。	へや の なか から わらいごえ が きこえた	
\\	部屋[へや]の 中[なか]から
\\	が 聞[き]こえた。			
\\	話し声	話[はな]し 声[ごえ]	はなしごえ	
\\	隣の部屋から話し声が聞こえます。	隣[となり]の 部屋[へや]から 話[はな]し 声[ごえ]が 聞[き]こえます。	となり の へや から はなしごえ が きこえます	
\\	隣[となり]の 部屋[へや]から
\\	が 聞[き]こえます。			
\\	駐車	駐車[ちゅうしゃ]	ちゅうしゃ	
\\	車はここに駐車してください。	車[くるま]はここに 駐車[ちゅうしゃ]してください。	くるま は ここ に ちゅうしゃ して ください	
\\	車[くるま]はここに
\\	してください。			
\\	騒音	騒音[そうおん]	そうおん	
\\	窓から車の騒音が入って来ます。	窓[まど]から 車[くるま]の 騒音[そうおん]が 入[はい]って 来[き]ます。	まど から くるま の そうおん が はいって きます	
\\	窓[まど]から 車[くるま]の
\\	が 入[はい]って 来[き]ます。			
\\	騒動	騒動[そうどう]	そうどう	
\\	警察がやって来て騒動を静めたんだ。	警察[けいさつ]がやって 来[き]て 騒動[そうどう]を 静[しず]めたんだ。	けいさつ が やってきて そうどう を しずめた ん だ	
\\	警察[けいさつ]がやって 来[き]て
\\	を 静[しず]めたんだ。			
\\	騒々しい	騒々[そうぞう]しい	そうぞうしい	
\\	隣の家はいつも騒々しい。	隣[となり]の 家[いえ]はいつも 騒々[そうぞう]しい。	となり の いえ は いつも そうぞうしい	
\\	隣[となり]の 家[いえ]はいつも
\\	面倒	面倒[めんどう]	めんどう	
\\	会社で面倒なことが起こったの。	会社[かいしゃ]で 面倒[めんどう]なことが 起[お]こったの。	かいしゃ で めんどう な こと が おこった の	
\\	会社[かいしゃ]で
\\	なことが 起[お]こったの。			
\\	突然	突然[とつぜん]	とつぜん	
\\	彼は突然走り出したのよ。	彼[かれ]は 突然[とつぜん] 走[はし]り 出[だ]したのよ。	かれ は とつぜん はしりだした の よ	
\\	彼[かれ]は
\\	走[はし]り 出[だ]したのよ。			
\\	突っ込む	突[つ]っ 込[こ]む	つっこむ	
\\	ポケットに財布を突っ込んじゃった。	ポケットに 財布[さいふ]を 突[つ]っ 込[こ]んじゃった。	ぽけっと に さいふ を つっこん じゃった	
\\	ポケットに 財布[さいふ]を
\\	突く	突[つ]く	つく	
\\	彼はビリヤードの球を上手に突くね。	彼[かれ]はビリヤードの 球[たま]を 上手[じょうず]に 突[つ]くね。	かれ は びりやーど の たま を じょうず に つく ね	
\\	彼[かれ]はビリヤードの 球[たま]を 上手[じょうず]に
\\	ね。			
\\	追突	追突[ついとつ]	ついとつ	
\\	バスが乗用車に追突したよ。	バスが 乗用車[じょうようしゃ]に 追突[ついとつ]したよ。	ばす が じょうようしゃ に ついとつ した よ	
\\	バスが 乗用車[じょうようしゃ]に
\\	したよ。			
\\	触れる	触[ふ]れる	ふれる	
\\	手と手が触れてどきどきしたよ。	手[て]と 手[て]が 触[ふ]れてどきどきしたよ。	て と て が ふれて どきどき した よ	
\\	手[て]と 手[て]が
\\	どきどきしたよ。			
\\	避難	避難[ひなん]	ひなん	
\\	危ないので避難してください。	危[あぶ]ないので 避難[ひなん]してください。	あぶない の で ひなん して ください	
\\	危[あぶ]ないので
\\	してください。			
\\	真ん丸	真[ま]ん 丸[まる]	まんまる	
\\	今日は月が真ん丸です。	今日[きょう]は 月[つき]が 真[ま]ん 丸[まる]です。	きょう は つき が まんまる です	
\\	今日[きょう]は 月[つき]が
\\	です。			
\\	爆発	爆発[ばくはつ]	ばくはつ	
\\	ダイナマイトが爆発したんだ。	ダイナマイトが 爆発[ばくはつ]したんだ。	だいなまいと が ばくはつ した ん だ	
\\	ダイナマイトが
\\	したんだ。			
\\	爆弾	爆弾[ばくだん]	ばくだん	
\\	その町に爆弾が落とされたの。	その 町[まち]に 爆弾[ばくだん]が 落[お]とされたの。	その まち に ばくだん が おとされた の	
\\	その 町[まち]に
\\	が 落[お]とされたの。			
\\	絶えず	絶[た]えず	たえず	
\\	今日は絶えず電話が鳴りました。	今日[きょう]は 絶[た]えず 電話[でんわ]が 鳴[な]りました。	きょう は たえず でんわ が なりました	
\\	今日[きょう]は
\\	電話[でんわ]が 鳴[な]りました。			
\\	滅びる	滅[ほろ]びる	ほろびる	
\\	その文明は何千年も前に滅びてしまったの。	その 文明[ぶんめい]は 何千年[なんぜんねん]も 前[まえ]に 滅[ほろ]びてしまったの。	その ぶんめい は なんぜんねん も まえ に ほろびて しまった の	
\\	その 文明[ぶんめい]は 何千年[なんぜんねん]も 前[まえ]に
\\	しまったの。			
\\	防ぐ	防[ふせ]ぐ	ふせぐ	
\\	怪我を防ぐためによくストレッチをしてください。	怪我[けが]を 防[ふせ]ぐためによくストレッチをしてください。	けが を ふせぐ ため に よく すとれっち を して ください	
\\	怪我[けが]を
\\	ためによくストレッチをしてください。			
\\	防衛	防衛[ぼうえい]	ぼうえい	
\\	チャンピオンがタイトルを防衛した。	チャンピオンがタイトルを 防衛[ぼうえい]した。	ちゃんぴおん が たいとる を ぼうえい した	
\\	チャンピオンがタイトルを
\\	した。			
\\	防止	防止[ぼうし]	ぼうし	
\\	警察は犯罪の防止に努めています。	警察[けいさつ]は 犯罪[はんざい]の 防止[ぼうし]に 努[つと]めています。	けいさつ は はんざい の ぼうし に つとめて います	
\\	警察[けいさつ]は 犯罪[はんざい]の
\\	に 努[つと]めています。			
\\	防火	防火[ぼうか]	ぼうか	
\\	学校で防火訓練がありました。	学校[がっこう]で 防火[ぼうか] 訓練[くんれん]がありました。	がっこう で ぼうかくんれん が ありました	
\\	学校[がっこう]で
\\	訓練[くんれん]がありました。			
\\	理系	理系[りけい]	りけい	
\\	彼は理系の学生です。	彼[かれ]は 理系[りけい]の 学生[がくせい]です。	かれ は りけい の がくせい です	
\\	彼[かれ]は
\\	の 学生[がくせい]です。			
\\	綿	綿[めん]	めん	
\\	私は綿のシャツをよく着ます。	私[わたし]は 綿[めん]のシャツをよく 着[き]ます。	わたし は めん の しゃつ を よく きます	
\\	私[わたし]は
\\	のシャツをよく 着[き]ます。			
\\	繁栄	繁栄[はんえい]	はんえい	
\\	あの国は経済的に繁栄しているわね。	あの 国[くに]は 経済的[けいざいてき]に 繁栄[はんえい]しているわね。	あの くに は けいざいてき に はんえい して いる わ ね	
\\	あの 国[くに]は 経済的[けいざいてき]に
\\	しているわね。			
\\	耐える	耐[た]える	たえる	
\\	このビルは大地震に耐えられるでしょうか。	このビルは 大地震[おおじしん]に 耐[た]えられるでしょうか。	この びる は おおじしん に たえられる でしょう か	
\\	このビルは 大地震[おおじしん]に
\\	でしょうか。			
\\	矢印	矢印[やじるし]	やじるし	
\\	矢印にそって歩いてください。	矢印[やじるし]にそって 歩[ある]いてください。	やじるし に そって あるいて ください	
\\	にそって 歩[ある]いてください。			
\\	爪切り	爪切[つめき]り	つめきり	
\\	爪切りを取ってください。	爪切[つめき]りを 取[と]ってください。	つめきり を とって ください	
\\	を 取[と]ってください。			
\\	詰める	詰[つ]める	つめる	
\\	かばんに荷物を詰めました。	かばんに 荷物[にもつ]を 詰[つ]めました。	かばん に にもつ を つめました	
\\	かばんに 荷物[にもつ]を
\\	詰まる	詰[つ]まる	つまる	
\\	プリンタのインクが詰まってしまいました。	プリンタのインクが 詰[つ]まってしまいました。	ぷりんた の いんく が つまって しまいました	
\\	プリンタのインクが
\\	しまいました。			
\\	譲る	譲[ゆず]る	ゆずる	
\\	上司が車を譲ってくれたの。	上司[じょうし]が 車[くるま]を 譲[ゆず]ってくれたの。	じょうし が くるま を ゆずって くれた の	
\\	上司[じょうし]が 車[くるま]を
\\	の。			
\\	誠に	誠[まこと]に	まことに	
\\	誠にありがとうございます。	誠[まこと]にありがとうございます。	まことに ありがとう ございます	
\\	ありがとうございます。			
\\	盛り上げる	盛[も]り 上[あ]げる	もりあげる	
\\	応援団が試合を盛り上げたんだ。	応援団[おうえんだん]が 試合[しあい]を 盛[も]り 上[あ]げたんだ。	おうえんだん が しあい を もりあげた ん だ	
\\	応援団[おうえんだん]が 試合[しあい]を
\\	んだ。			
\\	盛り上がる	盛[も]り 上[あ]がる	もりあがる	
\\	昨夜のパーティはとても盛り上がったな。	昨夜[さくや]のパーティはとても 盛[も]り 上[あ]がったな。	さくや の ぱーてぃ は とても もりあがった な	
\\	昨夜[さくや]のパーティはとても
\\	な。			
\\	盛る	盛[も]る	もる	
\\	ご飯を盛ってください。	ご 飯[はん]を 盛[も]ってください。	ごはん を もって ください	
\\	ご 飯[はん]を
\\	ください。			
\\	針	針[はり]	はり	
\\	母は糸を針に通した。	母[はは]は 糸[いと]を 針[はり]に 通[とお]した。	はは は いと を はり に とおした	
\\	母[はは]は 糸[いと]を
\\	に 通[とお]した。			
\\	釣り	釣[つ]り	つり	
\\	父は釣りが大好きです。	父[ちち]は 釣[つ]りが 大好[だいす]きです。	ちち は つり が だいすき です	
\\	父[ちち]は
\\	が 大好[だいす]きです。			
\\	釣り合い	釣[つ]り 合[あ]い	つりあい	
\\	この紙飛行機は左右の釣り合いがとれていないな。	この 紙飛行機[かみひこうき]は 左右[さゆう]の 釣[つ]り 合[あ]いがとれていないな。	この かみひこうき は さゆう の つりあい が とれて いない な	
\\	この 紙飛行機[かみひこうき]は 左右[さゆう]の
\\	がとれていないな。			
\\	釣り合う	釣[つ]り 合[あ]う	つりあう	
\\	収入と支出が釣り合っていない。	収入[しゅうにゅう]と 支出[ししゅつ]が 釣[つ]り 合[あ]っていない。	しゅうにゅう と ししゅつ が つりあって いない	
\\	収入[しゅうにゅう]と 支出[ししゅつ]が
\\	鈍い	鈍[にぶ]い	にぶい	
\\	今日は頭の回転が鈍いです。	今日[きょう]は 頭[あたま]の 回転[かいてん]が 鈍[にぶ]いです。	きょう は あたま の かいてん が にぶい です	
\\	今日[きょう]は 頭[あたま]の 回転[かいてん]が
\\	です。			
\\	鈍感	鈍感[どんかん]	どんかん	
\\	あの人は少し鈍感だと思います。	あの 人[ひと]は 少[すこ]し 鈍感[どんかん]だと 思[おも]います。	あの ひと は すこし どんかん だ と おもいます	
\\	あの 人[ひと]は 少[すこ]し
\\	だと 思[おも]います。			
\\	谷	谷[たに]	たに	
\\	その村は深い谷にあるの。	その 村[むら]は 深[ふか]い 谷[たに]にあるの。	その むら は ふかい たに に ある の	
\\	その 村[むら]は 深[ふか]い
\\	にあるの。			
\\	湾	湾[わん]	わん	
\\	東京湾は重要な役割を果たしている。	東京[とうきょう] 湾[わん]は 重要[じゅうよう]な 役割[やくわり]を 果[は]たしている。	とうきょうわん は じゅうよう な やくわり を はたして いる	
\\	東京[とうきょう]
\\	は 重要[じゅうよう]な 役割[やくわり]を 果[は]たしている。			
\\	潜る	潜[もぐ]る	もぐる	
\\	彼は長い間海に潜っていたの。	彼[かれ]は 長[なが]い 間海[あいだ うみ]に 潜[もぐ]っていたの。	かれ は ながい あいだ うみ に もぐって いた の	
\\	彼[かれ]は 長[なが]い 間海[あいだ うみ]に
\\	の。			
\\	濁る	濁[にご]る	にごる	
\\	その池の水は濁っているね。	その 池[いけ]の 水[みず]は 濁[にご]っているね。	その いけ の みず は にごって いる ね	
\\	その 池[いけ]の 水[みず]は
\\	ね。			
\\	湯飲み	湯飲[ゆの]み	ゆのみ	
\\	湯飲みにはお茶を入れてお茶碗にはご飯をよそいます。	湯飲[ゆの]みにはお 茶[ちゃ]を 入[い]れてお 茶碗[ちゃわん]にはご 飯[はん]をよそいます。	ゆのみ に は おちゃ を いれて お ちゃわん に は ごはん を よそいます	
\\	にはお 茶[ちゃ]を 入[い]れてお 茶碗[ちゃわん]にはご 飯[はん]をよそいます。			
\\	湯気	湯気[ゆげ]	ゆげ	
\\	やかんから湯気が出ていますよ。	やかんから 湯気[ゆげ]が 出[で]ていますよ。	やかん から ゆげ が でて います よ	
\\	やかんから
\\	が 出[で]ていますよ。			
\\	漏れる	漏[も]れる	もれる	
\\	彼のヘッドフォンから音が漏れているね。	彼[かれ]のヘッドフォンから 音[おと]が 漏[も]れているね。	かれ の へっどふぉん から おと が もれて いる ね	
\\	彼[かれ]のヘッドフォンから 音[おと]が
\\	いるね。			
\\	漏らす	漏[も]らす	もらす	
\\	秘密を漏らしたのは彼です。	秘密[ひみつ]を 漏[も]らしたのは 彼[かれ]です。	ひみつ を もらした の は かれ です	
\\	秘密[ひみつ]を
\\	のは 彼[かれ]です。			
\\	漏る	漏[も]る	もる	
\\	天井から雨が漏るの。	天井[てんじょう]から 雨[あめ]が 漏[も]るの。	てんじょう から あめ が もる の	
\\	天井[てんじょう]から 雨[あめ]が
\\	の。			
\\	濡らす	濡[ぬ]らす	ぬらす	
\\	携帯電話を水で濡らしてしまったんだ。	携帯電話[けいたい でんわ]を 水[みず]で 濡[ぬ]らしてしまったんだ。	けいたい でんわ を みず で ぬらして しまった ん だ	
\\	携帯電話[けいたい でんわ]を 水[みず]で
\\	んだ。			
\\	溝	溝[みぞ]	みぞ	
\\	車が溝にはまったの。	車[くるま]が 溝[みぞ]にはまったの。	くるま が みぞ に はまった の	
\\	車[くるま]が
\\	にはまったの。			
\\	虫歯	虫歯[むしば]	むしば	
\\	虫歯は予防できます。	虫歯[むしば]は 予防[よぼう]できます。	むしば は よぼう できます	
\\	は 予防[よぼう]できます。			
\\	虫	虫[むし]	むし	
\\	庭で秋の虫が鳴いているね。	庭[にわ]で 秋[あき]の 虫[むし]が 鳴[な]いているね。	にわ で あき の むし が ないて いる ね	
\\	庭[にわ]で 秋[あき]の
\\	が 鳴[な]いているね。			
\\	道徳	道徳[どうとく]	どうとく	
\\	子供には道徳をきちんと教えていきたいと思います。	子供[こども]には 道徳[どうとく]をきちんと 教[おし]えていきたいと 思[おも]います。	こども に は どうとく を きちんと おしえて いきたいと おもいます	
\\	子供[こども]には
\\	をきちんと 教[おし]えていきたいと 思[おも]います。			
\\	直径	直径[ちょっけい]	ちょっけい	
\\	この円は直径6センチです。	この 円[えん]は 直径[ちょっけい]6センチです。	この えん は ちょっけい 
\\	せんち です	
\\	この 円[えん]は
\\	6センチです。			
\\	豆	豆[まめ]	まめ	
\\	今、豆を煮ています。	今[いま]、 豆[まめ]を 煮[に]ています。	いま まめ を にて います	
\\	今[いま]、
\\	を 煮[に]ています。			
\\	粒	粒[つぶ]	つぶ	
\\	その子は大粒の涙を浮かべていたの。	その 子[こ]は 大[おお] 粒[つぶ]の 涙[なみだ]を 浮[う]かべていたの。	その こ は おおつぶ の なみだ を うかべていた の	
\\	その 子[こ]は 大[おお]
\\	の 涙[なみだ]を 浮[う]かべていたの。			
\\	麦	麦[むぎ]	むぎ	
\\	この畑では麦を作っています。	この 畑[はたけ]では 麦[むぎ]を 作[つく]っています。	この はたけ で は むぎ を つくって います	
\\	この 畑[はたけ]では
\\	を 作[つく]っています。			
\\	田畑	田畑[たはた]	たはた	
\\	この村には田畑がたくさんあります。	この 村[むら]には 田畑[たはた]がたくさんあります。	この むら に は たはた が たくさん あります	
\\	この 村[むら]には
\\	がたくさんあります。			
\\	灯油	灯油[とうゆ]	とうゆ	
\\	ストーブの灯油がなくなったわよ。	ストーブの 灯油[とうゆ]がなくなったわよ。	すとーぶ の とうゆ が なくなった わ よ	
\\	ストーブの
\\	がなくなったわよ。			
\\	電灯	電灯[でんとう]	でんとう	
\\	午後5時には電灯がつきます。	午後5時[ごご 
\\	じ]には 電灯[でんとう]がつきます。	ごご 
\\	じ に は でんとう が つきます	
\\	午後5時[ごご 
\\	じ]には
\\	がつきます。			
\\	炎	炎[ほのお]	ほのお	
\\	ろうそくの炎が部屋を照らしたんだ。	ろうそくの 炎[ほのお]が 部屋[へや]を 照[て]らしたんだ。	ろうそく の ほのお が へや を てらした ん だ	
\\	ろうそくの
\\	が 部屋[へや]を 照[て]らしたんだ。			
\\	皮肉	皮肉[ひにく]	ひにく	
\\	彼は皮肉ばかり言います。	彼[かれ]は 皮肉[ひにく]ばかり 言[い]います。	かれ は ひにく ばかり いいます	
\\	彼[かれ]は
\\	ばかり 言[い]います。			
\\	皮膚	皮膚[ひふ]	ひふ	
\\	冬は皮膚が乾燥しますね。	冬[ふゆ]は 皮膚[ひふ]が 乾燥[かんそう]しますね。	ふゆ は ひふ が かんそう します ね	
\\	冬[ふゆ]は
\\	が 乾燥[かんそう]しますね。			
\\	臭い	臭[にお]い	におい	
\\	ここはいやな臭いがする。	ここはいやな 臭[にお]いがする。	ここ は いや な におい が する 。	
\\	ここはいやな
\\	がする。			
\\	生臭い	生臭[なまぐさ]い	なまぐさい	
\\	まな板が生臭いです。	まな 板[いた]が 生臭[なまぐさ]いです。	まないた が なまぐさい です	
\\	まな 板[いた]が
\\	です。			
\\	臭う	臭[にお]う	におう	
\\	流しが臭うのできれいにしたよ。	流[なが]しが 臭[にお]うのできれいにしたよ。	ながし が におう の で きれい に した よ	
\\	流[なが]しが
\\	のできれいにしたよ。			
\\	面倒臭い	面倒臭[めんどうくさ]い	めんどうくさい	
\\	この計算は面倒臭いなあ。	この 計算[けいさん]は 面倒臭[めんどうくさ]いなあ。	この けいさん は めんどうくさい なあ	
\\	この 計算[けいさん]は
\\	なあ。			
\\	羽根	羽根[はね]	はね	
\\	このペンは鳥の羽根で作られています。	このペンは 鳥[とり]の 羽根[はね]で 作[つく]られています。	この ぺん は とり の はね で つくられて います	
\\	このペンは 鳥[とり]の
\\	で 作[つく]られています。			
\\	翌日	翌日[よくじつ]	よくじつ	
\\	翌日、彼に会いに行きました。	翌日[よくじつ]、 彼[かれ]に 会[あ]いに 行[い]きました。	よくじつ かれ に あい に いきました	
\\	、 彼[かれ]に 会[あ]いに 行[い]きました。			
\\	翌朝	翌朝[よくあさ]	よくあさ	
\\	翌朝は快晴だったね。	翌朝[よくあさ]は 快晴[かいせい]だったね。	よくあさ は かいせい だった ね	
\\	は 快晴[かいせい]だったね。			
\\	翌年	翌年[よくねん]	よくねん	
\\	彼女は大学卒業の翌年に結婚したの。	彼女[かのじょ]は 大学卒業[だいがく そつぎょう]の 翌年[よくねん]に 結婚[けっこん]したの。	かのじょ は だいがく そつぎょう の よくねん に けっこん した の	
\\	彼女[かのじょ]は 大学卒業[だいがく そつぎょう]の
\\	に 結婚[けっこん]したの。			
\\	群れ	群[む]れ	むれ	
\\	湖に鳥の群れがいたよ。	湖[みずうみ]に 鳥[とり]の 群[む]れがいたよ。	みずうみ に とり の むれ が いた よ	
\\	湖[みずうみ]に 鳥[とり]の
\\	がいたよ。			
\\	要旨	要旨[ようし]	ようし	
\\	会議の要旨は次の通りです。	会議[かいぎ]の 要旨[ようし]は 次[つぎ]の 通[とお]りです。	かいぎ の ようし は つぎ の とおり です	
\\	会議[かいぎ]の
\\	は 次[つぎ]の 通[とお]りです。			
\\	肌	肌[はだ]	はだ	
\\	彼女は肌がとても白いですね。	彼女[かのじょ]は 肌[はだ]がとても 白[しろ]いですね。	かのじょ は はだ が とても しろい です ね	
\\	彼女[かのじょ]は
\\	がとても 白[しろ]いですね。			
\\	肌着	肌着[はだぎ]	はだぎ	
\\	寒いので暖かい肌着を着ました。	寒[さむ]いので 暖[あたた]かい 肌着[はだぎ]を 着[き]ました。	さむい の で あたたかい はだぎ を きました	
\\	寒[さむ]いので 暖[あたた]かい
\\	を 着[き]ました。			
\\	肌色	肌色[はだいろ]	はだいろ	
\\	彼女は肌色のシャツを着ています。	彼女[かのじょ]は 肌色[はだいろ]のシャツを 着[き]ています。	かのじょ は はだいろ の しゃつ を きて います	
\\	彼女[かのじょ]は
\\	のシャツを 着[き]ています。			
\\	腸	腸[ちょう]	ちょう	
\\	彼は腸の手術を受けたの。	彼[かれ]は 腸[ちょう]の 手術[しゅじゅつ]を 受[う]けたの。	かれ は ちょう の しゅじゅつ を うけた の	
\\	彼[かれ]は
\\	の 手術[しゅじゅつ]を 受[う]けたの。			
\\	芽	芽[め]	め	
\\	チューリップの芽が出ました。	チューリップの 芽[め]が 出[で]ました。	ちゅーりっぷ の め が でました	
\\	チューリップの
\\	が 出[で]ました。			
\\	葬式	葬式[そうしき]	そうしき	
\\	彼の葬式は明日です。	彼[かれ]の 葬式[そうしき]は 明日[あした]です。	かれ の そうしき は あした です	
\\	彼[かれ]の
\\	は 明日[あした]です。			
\\	礼儀	礼儀[れいぎ]	れいぎ	
\\	彼女は礼儀が身に付いています。	彼女[かのじょ]は 礼儀[れいぎ]が 身[み]に 付[つ]いています。	かのじょ は れいぎ が み に ついて います	
\\	彼女[かのじょ]は
\\	が 身[み]に 付[つ]いています。			
\\	竹	竹[たけ]	たけ	
\\	この笛は竹でできている。	この 笛[ふえ]は 竹[たけ]でできている。	この ふえ は たけ で できて いる	
\\	この 笛[ふえ]は
\\	でできている。			
\\	筆者	筆者[ひっしゃ]	ひっしゃ	
\\	筆者の趣旨を考えてください。	筆者[ひっしゃ]の 趣旨[しゅし]を 考[かんが]えてください。	ひっしゃ の しゅし を かんがえて ください	
\\	の 趣旨[しゅし]を 考[かんが]えてください。			
\\	筆記試験	筆記試験[ひっきしけん]	ひっきしけん	
\\	その会社の筆記試験は難しかったよ。	その 会社[かいしゃ]の 筆記試験[ひっきしけん]は 難[むずか]しかったよ。	その かいしゃ の ひっきしけん は むずかしかった よ	
\\	その 会社[かいしゃ]の
\\	は 難[むずか]しかったよ。			
\\	筆記用具	筆記用具[ひっきようぐ]	ひっきようぐ	
\\	今日は筆記用具を忘れました。	今日[きょう]は 筆記用具[ひっきようぐ]を 忘[わす]れました。	きょう は ひっきようぐ を わすれました	
\\	今日[きょう]は
\\	を 忘[わす]れました。			
\\	筆	筆[ふで]	ふで	
\\	彼は筆で手紙を書いたの。	彼[かれ]は 筆[ふで]で 手紙[てがみ]を 書[か]いたの。	かれ は ふで で てがみ を かいた の	
\\	彼[かれ]は
\\	で 手紙[てがみ]を 書[か]いたの。			
\\	笛	笛[ふえ]	ふえ	
\\	彼は笛を吹くのが上手いね。	彼[かれ]は 笛[ふえ]を 吹[ふ]くのが 上手[うま]いね。	かれ は ふえ を ふく の が うまい ね	
\\	彼[かれ]は
\\	を 吹[ふ]くのが 上手[うま]いね。			
\\	通帳	通帳[つうちょう]	つうちょう	
\\	銀行の通帳を持ってきてください。	銀行[ぎんこう]の 通帳[つうちょう]を 持[も]ってきてください。	ぎんこう の つうちょう を もって きて ください	
\\	銀行[ぎんこう]の
\\	を 持[も]ってきてください。			
\\	花嫁	花嫁[はなよめ]	はなよめ	
\\	花嫁が父親と一緒に入ってきましたね。	花嫁[はなよめ]が 父親[ちちおや]と 一緒[いっしょ]に 入[はい]ってきましたね。	はなよめ が ちちおや と いっしょ に はいって きました ね	
\\	が 父親[ちちおや]と 一緒[いっしょ]に 入[はい]ってきましたね。			
\\	花婿	花婿[はなむこ]	はなむこ	
\\	花婿と花嫁が並んで登場したんだ。	花婿[はなむこ]と 花嫁[はなよめ]が 並[なら]んで 登場[とうじょう]したんだ。	はなむこ と はなよめ が ならんで とうじょう した ん だ	
\\	と 花嫁[はなよめ]が 並[なら]んで 登場[とうじょう]したんだ。			
\\	眠たい	眠[ねむ]たい	ねむたい	
\\	まだ8時なのに、もう眠たいです。	まだ8 時[じ]なのに、もう 眠[ねむ]たいです。	まだ 
\\	じ な の に もう ねむたい です	
\\	まだ8 時[じ]なのに、もう
\\	です。			
\\	眠り	眠[ねむ]り	ねむり	
\\	彼女はいつもより早く眠りに着いたよ。	彼女[かのじょ]はいつもより 早[はや]く 眠[ねむ]りに 着[つ]いたよ。	かのじょ は いつも より はやく ねむり に ついた よ	
\\	彼女[かのじょ]はいつもより 早[はや]く
\\	に 着[つ]いたよ。			
\\	眺める	眺[なが]める	ながめる	
\\	猫が窓から外を眺めているよ。	猫[ねこ]が 窓[まど]から 外[そと]を 眺[なが]めているよ。	ねこ が まど から そと を ながめて いる よ	
\\	猫[ねこ]が 窓[まど]から 外[そと]を
\\	いるよ。			
\\	眺め	眺[なが]め	ながめ	
\\	ここからの眺めは最高です。	ここからの 眺[なが]めは 最高[さいこう]です。	ここ からの ながめ は さいこう です	
\\	ここからの
\\	は 最高[さいこう]です。			
\\	話し掛ける	話[はな]し 掛[か]ける	はなしかける	
\\	知らない人が話し掛けてきた。	知[し]らない 人[ひと]が 話[はな]し 掛[か]けてきた。	しらない ひと が はなしかけて きた	
\\	知[し]らない 人[ひと]が
\\	通り掛かる	通[とお]り 掛[か]かる	とおりかかる	
\\	その店を通り掛かったらバーゲンをやっていた。	その 店[みせ]を 通[とお]り 掛[か]かったらバーゲンをやっていた。	その みせ を とおりかかったら ばーげん を やっていた 。	
\\	その 店[みせ]を
\\	バーゲンをやっていた。			
\\	迎え	迎[むか]え	むかえ	
\\	駅まで迎えに来て下さい。	駅[えき]まで 迎[むか]えに 来[き]て 下[くだ]さい。	えき まで むかえ に きて ください	
\\	駅[えき]まで
\\	に 来[き]て 下[くだ]さい。			
\\	遂げる	遂[と]げる	とげる	
\\	この会社は急成長を遂げました。	この 会社[かいしゃ]は 急成長[きゅうせいちょう]を 遂[と]げました。	この かいしゃ は きゅうせいちょう を とげました	
\\	この 会社[かいしゃ]は 急成長[きゅうせいちょう]を
\\	総理大臣	総理大臣[そうりだいじん]	そうりだいじん	
\\	国民の6割が総理大臣を支持しています。	国民[こくみん]の6 割[わり]が 総理大臣[そうりだいじん]を 支持[しじ]しています。	こくみん の 
\\	わり が そうりだいじん を しじ して います	
\\	国民[こくみん]の6 割[わり]が
\\	を 支持[しじ]しています。			
\\	臨む	臨[のぞ]む	のぞむ	
\\	選手は最高の状態で大会に臨んだわ。	選手[せんしゅ]は 最高[さいこう]の 状態[じょうたい]で 大会[たいかい]に 臨[のぞ]んだわ。	せんしゅ は さいこう の じょうたい で たいかい に のぞんだ わ	
\\	選手[せんしゅ]は 最高[さいこう]の 状態[じょうたい]で 大会[たいかい]に
\\	わ。			
\\	臨時	臨時[りんじ]	りんじ	
\\	彼女は臨時の店員です。	彼女[かのじょ]は 臨時[りんじ]の 店員[てんいん]です。	かのじょ は りんじ の てんいん です	
\\	彼女[かのじょ]は
\\	の 店員[てんいん]です。			
\\	闘う	闘[たたか]う	たたかう	
\\	彼は病気と闘ったの。	彼[かれ]は 病気[びょうき]と 闘[たたか]ったの。	かれ は びょうき と たたかった の	
\\	彼[かれ]は 病気[びょうき]と
\\	の。			
\\	漫画	漫画[まんが]	まんが	
\\	妹は漫画が好きです。	妹[いもうと]は 漫画[まんが]が 好[す]きです。	いもうと は まんが が すき です	
\\	妹[いもうと]は
\\	が 好[す]きです。			
\\	電卓	電卓[でんたく]	でんたく	
\\	この電卓はポケットに入ります。	この 電卓[でんたく]はポケットに 入[はい]ります。	この でんたく は ぽけっと に はいります	
\\	この
\\	はポケットに 入[はい]ります。			
\\	雄大	雄大[ゆうだい]	ゆうだい	
\\	雄大な景色に感動したの。	雄大[ゆうだい]な 景色[けしき]に 感動[かんどう]したの。	ゆうだい な けしき に かんどう した の	
\\	な 景色[けしき]に 感動[かんどう]したの。			
\\	雌	雌[めす]	めす	
\\	この猫は雌です。	この 猫[ねこ]は 雌[めす]です。	この ねこ は めす です	
\\	この 猫[ねこ]は
\\	です。			
\\	秩序	秩序[ちつじょ]	ちつじょ	
\\	その国の社会秩序は乱れているわ。	その 国[くに]の 社会[しゃかい] 秩序[ちつじょ]は 乱[みだ]れているわ。	その くに の しゃかい ちつじょ は みだれて いる わ	
\\	その 国[くに]の 社会[しゃかい]
\\	は 乱[みだ]れているわ。			
\\	物陰	物陰[ものかげ]	ものかげ	
\\	私たちは物陰に隠れたんだ。	私[わたし]たちは 物陰[ものかげ]に 隠[かく]れたんだ。	わたしたち は ものかげ に かくれた ん だ	
\\	私[わたし]たちは
\\	に 隠[かく]れたんだ。			
\\	魅力	魅力[みりょく]	みりょく	
\\	彼女の明るさに魅力を感じました。	彼女[かのじょ]の 明[あか]るさに 魅力[みりょく]を 感[かん]じました。	かのじょ の あかるさ に みりょく を かんじました	
\\	彼女[かのじょ]の 明[あか]るさに
\\	を 感[かん]じました。			
\\	醜い	醜[みにく]い	みにくい	
\\	醜い争いはやめましょう。	醜[みにく]い 争[あらそ]いはやめましょう。	みにくい あらそい は やめましょう	
\\	争[あらそ]いはやめましょう。			
\\	頂上	頂上[ちょうじょう]	ちょうじょう	
\\	あと少しで山の頂上です。	あと 少[すこ]しで 山[やま]の 頂上[ちょうじょう]です。	あと すこし で やま の ちょうじょう です	
\\	あと 少[すこ]しで 山[やま]の
\\	です。			
\\	近頃	近頃[ちかごろ]	ちかごろ	
\\	近頃、彼女の様子がおかしい。	近頃[ちかごろ]、 彼女[かのじょ]の 様子[ようす]がおかしい。	ちかごろ かのじょ の ようす が おかしい	
\\	、 彼女[かのじょ]の 様子[ようす]がおかしい。			
\\	矛盾	矛盾[むじゅん]	むじゅん	
\\	彼の言っていることは矛盾しています。	彼[かれ]の 言[い]っていることは 矛盾[むじゅん]しています。	かれ の いって いる こと は むじゅん して います	
\\	彼[かれ]の 言[い]っていることは
\\	しています。			
\\	罰金	罰金[ばっきん]	ばっきん	
\\	罰金を3万円もとられたよ。	罰金[ばっきん]を3 万円[まんえん]もとられたよ。	ばっきん を 
\\	まんえん も とられた よ	
\\	を3 万円[まんえん]もとられたよ。			
\\	罰する	罰[ばっ]する	ばっする	
\\	彼は違法駐車で罰せられたんだ。	彼[かれ]は 違法駐車[いほう ちゅうしゃ]で 罰[ばっ]せられたんだ。	かれ は いほう ちゅうしゃ で ばっせられた ん だ	
\\	彼[かれ]は 違法駐車[いほう ちゅうしゃ]で
\\	んだ。			
\\	罰	罰[ばつ]	ばつ	
\\	彼は悪いことをしたので罰を受けたんだ。	彼[かれ]は 悪[わる]いことをしたので 罰[ばつ]を 受[う]けたんだ。	かれ は わるい こと を した ので ばつ を うけた ん だ	
\\	彼[かれ]は 悪[わる]いことをしたので
\\	を 受[う]けたんだ。			
\\	老人	老人[ろうじん]	ろうじん	
\\	子供が老人の手を引いて歩いていたの。	子供[こども]が 老人[ろうじん]の 手[て]を 引[ひ]いて 歩[ある]いていたの。	こども が ろうじん の て を ひいて あるいて いた の	
\\	子供[こども]が
\\	の 手[て]を 引[ひ]いて 歩[ある]いていたの。			
\\	煮る	煮[に]る	にる	
\\	今、豆を煮ています。	今[いま]、 豆[まめ]を 煮[に]ています。	いま まめ を にて います	
\\	今[いま]、 豆[まめ]を
\\	煮える	煮[に]える	にえる	
\\	もうすぐじゃがいもが煮えます。	もうすぐじゃがいもが 煮[に]えます。	もうすぐ じゃがいも が にえます	
\\	もうすぐじゃがいもが
\\	蒸し暑い	蒸[む]し 暑[あつ]い	むしあつい	
\\	今日は蒸し暑いね。	今日[きょう]は 蒸[む]し 暑[あつ]いね。	きょう は むしあつい ね	
\\	今日[きょう]は
\\	ね。			
\\	黙る	黙[だま]る	だまる	
\\	彼はいつも黙って仕事をします。	彼[かれ]はいつも 黙[だま]って 仕事[しごと]をします。	かれ は いつも だまって しごと を します	
\\	彼[かれ]はいつも
\\	仕事[しごと]をします。			
\\	無駄	無駄[むだ]	むだ	
\\	それは時間の無駄です。	それは 時間[じかん]の 無駄[むだ]です。	それ は じかん の むだ です	
\\	それは 時間[じかん]の
\\	です。			
\\	無駄遣い	無駄遣[むだづか]い	むだづかい	
\\	これは税金の無駄遣いだね。	これは 税金[ぜいきん]の 無駄遣[むだづか]いだね。	これ は ぜいきん の むだづかい だ ね	
\\	これは 税金[ぜいきん]の
\\	だね。			
\\	返却	返却[へんきゃく]	へんきゃく	
\\	図書館に本を返却したよ。	図書館[としょかん]に 本[ほん]を 返却[へんきゃく]したよ。	としょかん に ほん を へんきゃく した よ	
\\	図書館[としょかん]に 本[ほん]を
\\	したよ。			
\\	都庁	都庁[とちょう]	とちょう	
\\	あの高い建物が都庁です。	あの 高[たか]い 建物[たてもの]が 都庁[とちょう]です。	あの たかい たてもの が とちょう です	
\\	あの 高[たか]い 建物[たてもの]が
\\	です。			
\\	牧場	牧場[ぼくじょう]	ぼくじょう	
\\	叔父は牧場を持っているんだ。	叔父[おじ]は 牧場[ぼくじょう]を 持[も]っているんだ。	おじ は ぼくじょう を もって いる ん だ	
\\	叔父[おじ]は
\\	を 持[も]っているんだ。			
\\	露	露[つゆ]	つゆ	
\\	花に露がついていました。	花[はな]に 露[つゆ]がついていました。	はな に つゆ が ついて いました	
\\	花[はな]に
\\	がついていました。			
\\	零下	零下[れいか]	れいか	
\\	今日は零下の寒さでしたね。	今日[きょう]は 零下[れいか]の 寒[さむ]さでしたね。	きょう は れいか の さむさ でした ね	
\\	今日[きょう]は
\\	の 寒[さむ]さでしたね。			
\\	雰囲気	雰囲気[ふんいき]	ふんいき	
\\	とても雰囲気のいいお店ですね。	とても 雰囲気[ふんいき]のいいお 店[みせ]ですね。	とても ふんいき の いい おみせ です ね	
\\	とても
\\	のいいお 店[みせ]ですね。			
\\	盆地	盆地[ぼんち]	ぼんち	
\\	盆地は夏、とても暑いんだ。	盆地[ぼんち]は 夏[なつ]、とても 暑[あつ]いんだ。	ぼんち は なつ、とても あつい ん だ	
\\	は 夏[なつ]、とても 暑[あつ]いんだ。			
\\	盆	盆[ぼん]	ぼん	
\\	お盆は実家に帰りました。	お 盆[ぼん]は 実家[じっか]に 帰[かえ]りました。	お ぼん は じっか に かえりました	
\\	お
\\	は 実家[じっか]に 帰[かえ]りました。			
\\	盆	盆[ぼん]	ぼん	
\\	母が料理をお盆に乗せたの。	母[はは]が 料理[りょうり]をお 盆[ぼん]に 乗[の]せたの。	はは が りょうり を おぼん に のせた の	
\\	母[はは]が 料理[りょうり]をお
\\	に 乗[の]せたの。			
\\	舞台	舞台[ぶたい]	ぶたい	
\\	その物語は京都が舞台だ。	その 物語[ものがたり]は 京都[きょうと]が 舞台[ぶたい]だ。	その ものがたり は きょうと が ぶたい だ	
\\	その 物語[ものがたり]は 京都[きょうと]が
\\	だ。			
\\	見舞う	見舞[みま]う	みまう	
\\	昨日友人を見舞ったの。	昨日友人[きのう ゆうじん]を 見舞[みま]ったの。	きのう ゆうじん を みまった の	
\\	昨日友人[きのう ゆうじん]を
\\	の。			
\\	見舞い	見舞[みま]い	みまい	
\\	友達が見舞いに来てくれたよ。	友達[ともだち]が 見舞[みま]いに 来[き]てくれたよ。	ともだち が みまい に きて くれた よ	
\\	友達[ともだち]が
\\	に 来[き]てくれたよ。			
\\	盆踊り	盆踊[ぼんおど]り	ぼんおどり	
\\	みんなで盆踊りに行きました。	みんなで 盆踊[ぼんおど]りに 行[い]きました。	みんな で ぼんおどり に いきました	
\\	みんなで
\\	に 行[い]きました。			
\\	跳ねる	跳[は]ねる	はねる	
\\	ウサギが雪の上を跳ねていったよ。	ウサギが 雪[ゆき]の 上[うえ]を 跳[は]ねていったよ。	うさぎ が ゆき の うえ を はねて いった よ	
\\	ウサギが 雪[ゆき]の 上[うえ]を
\\	よ。			
\\	銭湯	銭湯[せんとう]	せんとう	
\\	銭湯は昔より少なくなったわね。	銭湯[せんとう]は 昔[むかし]より 少[すく]なくなったわね。	せんとう は むかし より すくなく なった わ ね	
\\	は 昔[むかし]より 少[すく]なくなったわね。			
\\	狙う	狙[ねら]う	ねらう	
\\	来年は優勝を狙います。	来年[らいねん]は 優勝[ゆうしょう]を 狙[ねら]います。	らいねん は ゆうしょう を ねらいます	
\\	来年[らいねん]は 優勝[ゆうしょう]を
\\	裸	裸[はだか]	はだか	
\\	彼らは裸のつきあいをしているよ。	彼[かれ]らは 裸[はだか]のつきあいをしているよ。	かれら は はだか の つきあい を して いる よ	
\\	彼[かれ]らは
\\	のつきあいをしているよ。			
\\	祖先	祖先[そせん]	そせん	
\\	犬も猫も祖先は同じ動物らしいよ。	犬[いぬ]も 猫[ねこ]も 祖先[そせん]は 同[おな]じ 動物[どうぶつ]らしいよ。	いぬ も ねこ も そせん は おなじ どうぶつ らしい よ	
\\	犬[いぬ]も 猫[ねこ]も
\\	は 同[おな]じ 動物[どうぶつ]らしいよ。			
\\	無邪気	無邪気[むじゃき]	むじゃき	
\\	子供たちが無邪気に遊んでいるね。	子供[こども]たちが 無邪気[むじゃき]に 遊[あそ]んでいるね。	こどもたち が むじゃき に あそんで いる ね	
\\	子供[こども]たちが
\\	に 遊[あそ]んでいるね。			
\\	玉ねぎ	玉[たま]ねぎ	たまねぎ	
\\	私は玉ねぎが嫌いです。	私[わたし]は 玉[たま]ねぎが 嫌[きら]いです。	わたし は たまねぎ が きらい です 。	
\\	私[わたし]は
\\	が 嫌[きら]いです。			
\\	玉	玉[たま]	たま	
\\	彼の顔に玉のような汗が流れていたの。	彼[かれ]の 顔[かお]に 玉[たま]のような 汗[あせ]が 流[なが]れていたの。	かれ の かお に たま の よう な あせ が ながれて いた の	
\\	彼[かれ]の 顔[かお]に
\\	のような 汗[あせ]が 流[なが]れていたの。			
\\	風呂場	風呂場[ふろば]	ふろば	
\\	私は風呂場で歯を磨きます。	私[わたし]は 風呂場[ふろば]で 歯[は]を 磨[みが]きます。	わたし は ふろば で は を みがきます	
\\	私[わたし]は
\\	で 歯[は]を 磨[みが]きます。			
\\	班	班[はん]	はん	
\\	班のメンバーは5人です。	班[はん]のメンバーは5 人[にん]です。	はん の めんばー は 
\\	にん です	
\\	のメンバーは5 人[にん]です。			
\\	綱	綱[つな]	つな	
\\	この綱は直径20
\\	あるそうです。	この 綱[つな]は 直径20
\\	[ちょっけい 
\\	せんち]あるそうです。	この つな は ちょっけい 
\\	せんち ある そう です	
\\	この
\\	は 直径20
\\	[ちょっけい 
\\	せんち]あるそうです。			
\\	縄	縄[なわ]	なわ	
\\	枝を縄でしばったよ。	枝[えだ]を 縄[なわ]でしばったよ。	えだ を なわ で しばった よ	
\\	枝[えだ]を
\\	でしばったよ。			
\\	縫う	縫[ぬ]う	ぬう	
\\	彼女は子供のスカートを縫ったんだ。	彼女[かのじょ]は 子供[こども]のスカートを 縫[ぬ]ったんだ。	かのじょ は こども の すかーと を ぬった ん だ	
\\	彼女[かのじょ]は 子供[こども]のスカートを
\\	んだ。			
\\	虹	虹[にじ]	にじ	
\\	雨が上がって美しい虹が出たね。	雨[あめ]が 上[あ]がって 美[うつく]しい 虹[にじ]が 出[で]たね。	あめ が あがって うつくしい にじ が でた ね 。	
\\	雨[あめ]が 上[あ]がって 美[うつく]しい
\\	が 出[で]たね。			
\\	漬ける	漬[つ]ける	つける	
\\	魚を味噌に漬けました。	魚[さかな]を 味噌[みそ]に 漬[つ]けました。	さかな を みそ に つけました	
\\	魚[さかな]を 味噌[みそ]に
\\	漬け物	漬[つ]け 物[もの]	つけもの	
\\	祖母は漬け物を自分で作ります。	祖母[そぼ]は 漬[つ]け 物[もの]を 自分[じぶん]で 作[つく]ります。	そぼ は つけもの を じぶん で つくります	
\\	祖母[そぼ]は
\\	を 自分[じぶん]で 作[つく]ります。			
\\	潰す	潰[つぶ]す	つぶす	
\\	ペットボトルは潰して捨てましょう。	ペットボトルは 潰[つぶ]して 捨[す]てましょう。	ぺっとぼとる は つぶして すてましょう	
\\	ペットボトルは
\\	捨[す]てましょう。			
\\	潰れる	潰[つぶ]れる	つぶれる	
\\	箱の角が潰れていますよ。	箱[はこ]の 角[かど]が 潰[つぶ]れていますよ。	はこ の かど が つぶれて います よ	
\\	箱[はこ]の 角[かど]が
\\	いますよ。			
\\	賑わう	賑[にぎ]わう	にぎわう	
\\	ここは正月はたくさんの人で賑わいます。	ここは 正月[しょうがつ]はたくさんの 人[ひと]で 賑[にぎ]わいます。	ここ は しょうがつ は たくさん の ひと で にぎわいます	
\\	ここは 正月[しょうがつ]はたくさんの 人[ひと]で
\\	蛇	蛇[へび]	へび	
\\	蛇がカエルを捕まえたの。	蛇[へび]がカエルを 捕[つか]まえたの。	へび が かえる を つかまえた の	
\\	がカエルを 捕[つか]まえたの。			
\\	華やか	華[はな]やか	はなやか	
\\	彼女は華やかな女性ですね。	彼女[かのじょ]は 華[はな]やかな 女性[じょせい]ですね。	かのじょ は はなやか な じょせい です ね	
\\	彼女[かのじょ]は
\\	な 女性[じょせい]ですね。			
\\	誓う	誓[ちか]う	ちかう	
\\	二人は一生を共にすることを誓ったの。	二人[ふたり]は 一生[いっしょう]を 共[とも]にすることを 誓[ちか]ったの。	ふたり は いっしょう を ともに する こと を ちかった の	
\\	二人[ふたり]は 一生[いっしょう]を 共[とも]にすることを
\\	の。			
\\	蝶蝶	蝶蝶[ちょうちょう]	ちょうちょう	
\\	見て、きれいな蝶蝶が飛んでいる。	見[み]て、きれいな 蝶蝶[ちょうちょう]が 飛[と]んでいる。	みて 、 きれい な ちょうちょう が とん でいる 。	
\\	見[み]て、きれいな
\\	が 飛[と]んでいる。			
\\	鳩	鳩[はと]	はと	
\\	公園の鳩にえさをやったの。	公園[こうえん]の 鳩[はと]にえさをやったの。	こうえん の はと に えさ を やった の	
\\	公園[こうえん]の
\\	にえさをやったの。			
\\	雛祭	雛祭[ひなまつり]	ひなまつり	
\\	3月3日は雛祭りです。	
\\	月[がつ]3 日[か]は 雛祭[ひなまつ]りです。	
\\	がつ 
\\	か は ひなまつり です	
\\	月[がつ]3 日[か]は
\\	です。			
\\	物凄い	物凄[ものすご]い	ものすごい	
\\	夕方、物凄い雨が降ったね。	夕方[ゆうがた]、 物凄[ものすご]い 雨[あめ]が 降[ふ]ったね。	ゆうがた、 ものすごい あめ が ふった ね	
\\	夕方[ゆうがた]、
\\	雨[あめ]が 降[ふ]ったね。			
\\	溜め息	溜[た]め 息[いき]	ためいき	
\\	母は溜め息をついたの。	母[はは]は 溜[た]め 息[いき]をついたの。	はは は ためいき を ついた の	
\\	母[はは]は
\\	をついたの。			
\\	溜まる	溜[た]まる	たまる	
\\	彼はストレスが溜まっているの。	彼[かれ]はストレスが 溜[た]まっているの。	かれ は すとれす が たまって いる の	
\\	彼[かれ]はストレスが
\\	の。			
\\	溜める	溜[た]める	ためる	
\\	お風呂に水を溜めておいてください。	お 風呂[ふろ]に 水[みず]を 溜[た]めておいてください。	おふろ に みず を ためて おいて ください	
\\	お 風呂[ふろ]に 水[みず]を
\\	ください。			
\\	遥か	遥[はる]か	はるか	
\\	遥か向こうに目的地が見えてきた。	遥[はる]か 向[む]こうに 目的地[もくてきち]が 見[み]えてきた。	はるか むこう に もくてきち が みえて きた	
\\	向[む]こうに 目的地[もくてきち]が 見[み]えてきた。			
\\	股	股[また]	また	
\\	自転車の乗り過ぎで股が痛いです。	自転車[じてんしゃ]の 乗[の]り 過[す]ぎで 股[また]が 痛[いた]いです。	じてんしゃ の のりすぎ で また が いたい です	
\\	自転車[じてんしゃ]の 乗[の]り 過[す]ぎで
\\	が 痛[いた]いです。			
\\	焚く	焚[た]く	たく	
\\	薪を集めて火を焚いたよ。	薪[まき]を 集[あつ]めて 火[ひ]を 焚[た]いたよ。	まき を あつめて ひ を たいた よ	
\\	薪[まき]を 集[あつ]めて 火[ひ]を
\\	よ。			
\\	眉	眉[まゆ]	まゆ	
\\	彼は眉が濃いね。	彼[かれ]は 眉[まゆ]が 濃[こ]いね。	かれ は まゆ が こい ね	
\\	彼[かれ]は
\\	が 濃[こ]いね。			
\\	眉毛	眉毛[まゆげ]	まゆげ	
\\	彼女の眉毛は太いな。	彼女[かのじょ]の 眉毛[まゆげ]は 太[ふと]いな。	かのじょ の まゆげ は ふとい な	
\\	彼女[かのじょ]の
\\	は 太[ふと]いな。			
\\	蛋白質	蛋白質[たんぱくしつ]	たんぱくしつ	
\\	蛋白質は大切な栄養です。	蛋白質[たんぱくしつ]は 大切[たいせつ]な 栄養[えいよう]です。	たんぱくしつ は たいせつ な えいよう です	
\\	は 大切[たいせつ]な 栄養[えいよう]です。			
\\	蜂	蜂[はち]	はち	
\\	友達が蜂に刺されたんだ。	友達[ともだち]が 蜂[はち]に 刺[さ]されたんだ。	ともだち が はち に さされた ん だ	
\\	友達[ともだち]が
\\	に 刺[さ]されたんだ。			
\\	頂戴	頂戴[ちょうだい]	ちょうだい	
\\	それ、ひとつ頂戴。	それ、ひとつ 頂戴[ちょうだい]。	それ ひとつ ちょうだい	
\\	それ、ひとつ
\\	詫びる	詫[わ]びる	わびる	
\\	彼は彼女に心から詫びたんだ。	彼[かれ]は 彼女[かのじょ]に 心[こころ]から 詫[わ]びたんだ。	かれ は かのじょ に こころ から わびた ん だ	
\\	彼[かれ]は 彼女[かのじょ]に 心[こころ]から
\\	んだ。			
\\	頬	頬[ほお]	ほお	
\\	彼女はほおを赤く染めたわ。	彼女[かのじょ]はほおを 赤[あか]く 染[そ]めたわ。	かのじょ は ほお を あかく そめた わ	
\\	彼女[かのじょ]は
\\	を 赤[あか]く 染[そ]めたわ。			
\\	餅	餅[もち]	もち	
\\	正月には餅を食べますよ。	正月[しょうがつ]には 餅[もち]を 食[た]べますよ。	しょうがつ に は もち を たべます よ	
\\	正月[しょうがつ]には
\\	を 食[た]べますよ。			
\\	騙す	騙[だま]す	だます	
\\	人を騙してはいけません。	人[ひと]を 騙[だま]してはいけません。	ひと を だまして は いけません	
\\	人[ひと]を
\\	はいけません。			
\end{CJK}
\end{document}