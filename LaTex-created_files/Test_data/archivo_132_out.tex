\documentclass[8pt]{extreport} 
\usepackage{hyperref}
\usepackage{CJKutf8}
\begin{document}
\begin{CJK}{UTF8}{min}
\\	私はその晩興奮のあまり寝られなかった。	私[わたし]はその 晩[ばん] 興奮[こうふん]のあまり 寝[ね]られなかった。	
\\	あまり	
\\	この酒は飲めば飲むほど美味しくなります。	この 酒[さけ]は 飲[の]めば 飲[の]むほど 美味[おい]しくなります。	
\\	~ば~ほど	
\\	日本では子供ばかりか大人さえ漫画を読んでいる。	日本[にっぽん]では 子供[こども]ばかりか 大人[おとな]さえ 漫画[まんが]を 読[よ]んでいる。	
\\	~ばかりか~(さえ)	
\\	そんなことを人に言うべき{では/じゃ}ありません。	そんなことを 人[ひと]に 言[い]うべき{では/じゃ}ありません。	
\\	べきだ	
\\	私は今日三日分の仕事を片付けた。	私[わたし]は 今日[きょう]三 日[にち] 分[ぶん]の 仕事[しごと]を 片付[かたづ]けた。	
\\	-分	
\\	私は彼女とは初めて会った。だが、前から知っていたような親しみを感じた。	私[わたし]は 彼女[かのじょ]とは 初[はじ]めて 会[あ]った。だが、 前[まえ]から 知[し]っていたような 親[した]しみを 感[かん]じた。	
\\	だが	
\\	ジョンは奥さんのことをちょっとも褒めない。時々口を聞かないこともある。だからと言って、奥さんを愛していないわけではない。	ジョンは 奥[おく]さんのことをちょっとも 褒[ほ]めない。 時々[ときどき] 口[くち]を 聞[き]かないこともある。だからと 言[い]って、 奥[おく]さんを 愛[あい]していないわけではない。	
\\	だからと言って	
\\	その会議では私はただ座っているだけですかった。	その 会議[かいぎ]では 私[わたし]はただ 座[すわ]っているだけですかった。	
\\	だけで	
\\	病院に担ぎ込んだ時、その男の顔は血だらけだった。	病院[びょういん]に 担[かつ]ぎ 込[こ]んだ 時[とき]、その 男[おとこ]の 顔[かお]は 血[ち]だらけだった。	
\\	-だらけ	
\\	食べて飲んで、五人で、七万くらいでした。	食[た]べて 飲[の]んで、五 人[にん]で、七 万[まん]くらいでした。	
\\	で	
\\	その交渉は極めて困難であろう。	その 交渉[こうしょう]は 極[きわ]めて 困難[こんなん]であろう。	
\\	であろう	
\\	大型車に一人で乗るのは不経済である。	大型車[おおがたしゃ]に一 人[にん]で 乗[の]るのは 不経済[ふけいざい]である。	
\\	である	
\\	あの人は本を読むどころか、 新間も読まない。	あの 人[ひと]は 本[ほん]を 読[よ]むどころか、  新間[しんま]も 読[よ]まない。	
\\	どころか	
\\	あの先生の授業はどうも面白くない。	あの 先生[せんせい]の 授業[じゅぎょう]はどうも 面白[おもしろ]くない。	
\\	どうも	
\\	英夫は由美子に結婚を断られた時にどんなに残念に思ったことか。	英夫[ひでお]は 由美子[ゆみこ]に 結婚[けっこん]を 断[ことわ]られた 時[とき]にどんなに 残念[ざんねん]に 思[おも]ったことか。	
\\	どんなに-(こと)か	
\\	人間はどうせ死ぬんだから、 あくせく働いても仕方がない。	人間[にんげん]はどうせ 死[し]ぬんだから、 あくせく 働[はたら]いても 仕方[しかた]がない。	
\\	どうせ	
\\	このカレーライスはインド風に、 とても辛くしてあります。	このカレーライスはインド 風[ふう]に、 とても 辛[つら]くしてあります。	
\\	風に	
\\	私は小さい時病気がちでした。	私[わたし]は 小[ちい]さい 時[とき] 病気[びょうき]がちでした。	
\\	-がち	
\\	彼の行為は理解しがたい。	彼[かれ]の 行為[こうい]は 理解[りかい]しがたい。	
\\	-がたい	
\\	叱られると思ったのに、逆にほめられた。	叱[しか]られると 思[おも]ったのに、 逆[ぎゃく]にほめられた。	
\\	逆に	
\\	子供は小言を言うほど反発するものだ。	子供[こども]は 小言[こごと]を 言[い]うほど 反発[はんぱつ]するものだ。	
\\	ほど	
\\	私は日本酒以外の酒は飲まない。	私[わたし]は 日本酒[にほんしゅ] 以外[いがい]の 酒[さけ]は 飲[の]まない。	
\\	以外	
\\	体をよく動かしている以上は、人間の体は衰えたいらしい。	体[からだ]をよく 動[うご]かしている 以上[いじょう]は、 人間[にんげん]の 体[からだ]は 衰[おとろ]えたいらしい。	
\\	以上(は)	
\\	彼はいかにも全部分かっているかのように話しているが、その実何も分かっていない。	彼[かれ]はいかにも 全部[ぜんぶ] 分[わ]かっているかのように 話[はな]しているが、その 実[じつ] 何[なに]も 分[わ]かっていない。	
\\	いかにも	
\\	乗るなら飲むな。飲んだら乗るな。	乗[の]るなら 飲[の]むな。 飲[の]んだら 乗[の]るな。	
\\	ストレスは、一方では害になることもあるが、他方では生活のいい刺激にもなるそうだ。	ストレスは、 一方[いっぽう]では 害[がい]になることもあるが、 他方[たほう]では 生活[せいかつ]のいい 刺激[しげき]にもなるそうだ。	
\\	一方で(は) 
\\	他方で(は)	
\\	この映画は教育上よくない	この 映画[えいが]は 教育[きょういく] 上[じょう]よくない	
\\	上	
\\	薬を飲んだら、かえって病気がひどくなった。	薬[くすり]を 飲[の]んだら、かえって 病気[びょうき]がひどくなった。	
\\	かえって	
\\	私がここにいる限り心配は無用です。	私[わたし]がここにいる 限[かぎ]り 心配[しんぱい]は 無用[むよう]です。	
\\	限り	
\\	貸し出しは一回三冊限りです。	貸し出[かしだ]しは一 回[かい]三 冊[さつ] 限[かぎ]りです。	
\\	限り 
\\	傘を駅に置き忘れたのか事務所に忘れたのか、はっきり覚えていません。	傘[かさ]を 駅[えき]に 置き忘[おきわす]れたのか 事務所[じむしょ]に 忘[わす]れたのか、はっきり 覚[おぼ]えていません。	
\\	-か-か	
\\	今週末にはなにをしようかな。	今週末[こんしゅうまつ]にはなにをしようかな。	
\\	かな	
\\	運動をよくする人が必ずしも長生きするとは限らない。	運動[うんどう]をよくする 人[ひと]が 必[かなら]ずしも 長生[ながい]きするとは 限[かぎ]らない。	
\\	必ずしも	
\\	暴動すら起りかねないような緊迫した状況だった。	暴動[ぼうどう]すら 起[おこ]りかねないような 緊迫[きんぱく]した 状況[じょうきょう]だった。	
\\	-かねる	
\\	その女の人は私に家族のことから自分の悩みに至るまで、細かに話した。	その 女の人[おんなのひと]は 私[わたし]に 家族[かぞく]のことから 自分[じぶん]の 悩[なや]みに 至[いた]るまで、 細[こま]かに 話[はな]した。	
\\	-から-に至るまで	
\\	日本は六月から七月にかけて梅雨が続く。	日本[にっぽん]は 六月[ろくがつ]から 七月[しちがつ]にかけて 梅雨[つゆ]が 続[つづ]く。	
\\	から~にかけて	
\\	女だからと言って侮ってはいけない。	女[おんな]だからと 言[い]って 侮[あなど]ってはいけない。	
\\	からと言って	
\\	十年後の自分を予想するのは難しかろう。	十 年[ねん] 後[ご]の 自分[じぶん]を 予想[よそう]するのは 難[むずか]しかろう。	
\\	かろう	
\\	誰にでも分かるような書き方をして下さい。	誰[だれ]にでも 分[わ]かるような 書き方[かきかた]をして 下[くだ]さい。	
\\	方をする	
\\	毎日運動をすれば長生きをするかと言うと、そうでもなさそうだ。	毎[まい] 日運動[にちうんどう]をすれば 長生[ながい]きをするかと 言[い]うと、そうでもなさそうだ。	
\\	かと言うと	
\\	前のアパートは設備が悪かった代わりに家賃が安かった。	前[まえ]のアパートは 設備[せつび]が 悪[わる]かった 代[か]わりに 家賃[やちん]が 安[やす]かった。	
\\	(の)代わりに	
\\	妻と相談した結果、家を買うことにした。	妻[つま]と 相談[そうだん]した 結果[けっか]、 家[いえ]を 買[か]うことにした。	
\\	結果	
\\	あの人は間抜けに見えるでしょう。でも、結構頭がいいんです。	あの 人[ひと]は 間抜[まぬ]けに 見[み]えるでしょう。でも、 結構[けっこう] 頭[あたま]がいいんです。	
\\	結構	
\\	このたびこの会の会員に加えていただきました。	このたびこの 会[かい]の 会員[かいいん]に 加[くわ]えていただきました。	
\\	この	
\\	これこそ我々が探し求めていたものだ。	これこそ 我々[われわれ]が 探し求[さがしもと]めていたものだ。	
\\	こそ	
\\	私はこうした話しには耳をかさないことにしている。	私[わたし]はこうした 話[はな]しには 耳[みみ]をかさないことにしている。	
\\	こうした	
\\	私語は慎むこと。	私語[しご]は 慎[つつし]むこと。	
\\	こと	
\\	彼のパーティーに行かなかったことで彼の気持ちを害したのでなければよいが。	彼[かれ]のパーティーに 行[い]かなかったことで 彼[かれ]の 気持[きも]ちを 害[がい]したのでなければよいが。	
\\	ことで	
\\	今これをしておかなかったら後で大変なことになる。	今[いま]これをしておかなかったら 後[のち]で 大変[たいへん]なことになる。	
\\	ことになる	
\\	彼が負けたのは勝ちを焦り過ぎたことによる。	彼[かれ]が 負[ま]けたのは 勝[か]ちを 焦[あせ]り 過[す]ぎたことによる。	
\\	ことによる	
\\	何もそんなに慌てることはない。	何[なに]もそんなに 慌[あわ]てることはない。	
\\	ことはない	
\\	洋子は一人でいるのが寂しく、最近猫を飼い始めた。	洋子[ようこ]は一 人[にん]でいるのが 寂[さび]しく、 最近[さいきん] 猫[ねこ]を 飼[か]い 始[はじ]めた。	
\\	-く	
\\	私達は一歩も歩けないくらい疲れていた。	私[わたし] 達[たち]は一 歩[ほ]も 歩[ある]けないくらい 疲[つか]れていた。	
\\	くらい	
\\	彼は大学のくせに漫画ばかり読んでいる。	彼[かれ]は 大学[だいがく]のくせに 漫画[まんが]ばかり 読[よ]んでいる。	
\\	くせに	
\\	彼がみんなに尊敬されたのは言うまでもない。	彼[かれ]がみんなに 尊敬[そんけい]されたのは 言[い]うまでもない。	
\\	までもない	
\\	清のパーティーにはもう行くまい。	清[きよし]のパーティーにはもう 行[い]くまい。	
\\	まい	
\\	まさかあんな良い人を殺すなんてあるはずがない。	まさかあんな 良[よ]い 人[ひと]を 殺[ころ]すなんてあるはずがない。	
\\	まさか	
\\	こんな苦しい生活をするくらいなら、死んだ方がましだ。	こんな 苦[くる]しい 生活[せいかつ]をするくらいなら、 死[し]んだ 方[ほう]がましだ。	
\\	ましだ	
\\	現金(か)、または小切手でお払い下さい。	現金[げんきん](か)、または 小切手[こぎって]でお 払[はら]い 下[くだ]さい。	
\\	または	
\\	私の車右から二台目です。	私[わたし]の 車[くるま] 右[みぎ]から二 台[だい] 目[め]です。	
\\	-目	
\\	この食堂は衛生面によく気を配っている。	この 食堂[しょくどう]は 衛生[えいせい] 面[めん]によく 気[き]を 配[くば]っている。	
\\	-面	
\\	三百ページの本を一時間で読んでみせるぞ。	三 百[ひゃく]ページの 本[ほん]を 一時間[いちじかん]で 読[よ]んでみせるぞ。	
\\	みせる	
\\	安くもないのに、そうしてそんなのもを買うんですか。	安[やす]くもないのに、そうしてそんなのもを 買[か]うんですか。	
\\	も	
\\	ビールは二ダースも買っておけば大丈夫だ。	ビールは二ダースも 買[か]っておけば 大丈夫[だいじょうぶ]だ。	
\\	-も-ば	
\\	この物語は面白くもなければおかしくもない。	この 物語[ものがたり]は 面白[おもしろ]くもなければおかしくもない。	
\\	-も-も	
\\	これは革命的発見とも言うべきものだ。	これは 革命的[かくめいてき] 発見[はっけん]とも 言[い]うべきものだ。	
\\	もの(だ)	
\\	よく飲んだなあ。	よく 飲[の]んだなあ。	
\\	なあ	
\\	毎日運動しろなどとは言いませんから、週に二、三度運動したらどうですか。	毎[まい] 日運動[にちうんどう]しろなどとは 言[い]いませんから、 週[しゅう]に二、三 度[ど] 運動[うんどう]したらどうですか。	
\\	などと	
\\	この部屋は狭いながら(も)居心地がよい。	この 部屋[へや]は 狭[せま]いながら(も) 居心地[いごこち]がよい。	
\\	ながら(も)	
\\	日本語の新聞は読まないんですか。B:いいえ、読まないこと{も/は}ないんですが、時たまですね。	
\\	[えい]: 日本語[にほんご]の 新聞[しんぶん]は 読[よ]まないんですか。 
\\	[びー]:いいえ、 読[よ]まないこと{も/は}ないんですが、 時[とき]たまですね。	
\\	ないことも 
\\	はない	
\\	今年は桜の花がなかなか咲きません。	今年[ことし]は 桜[さくら]の 花[はな]がなかなか 咲[さ]きません。	
\\	なかなか	
\\	昨日見た映画は実に面白くなく、途中で寝てしまった。	昨日[きのう] 見[み]た 映画[えいが]は 実[じつ]に 面白[おもしろ]くなく、 途中[とちゅう]で 寝[ね]てしまった。	
\\	-なく	
\\	何でもこの辺は物価が非常に高いそうですよ。	何[なん]でもこの 辺[へん]は 物価[ぶっか]が 非常[ひじょう]に 高[たか]いそうですよ。	
\\	何でも	
\\	一日中ボスにがなり立てられるんで、何しろ、ストレスが多いんだ。	一 日[にち] 中ボス[ちゅうぼす]にがなり 立[た]てられるんで、 何[なに]しろ、ストレスが 多[おお]いんだ。	
\\	何しろ	
\\	夫が単身赴任しているので、寂しくてなりません。	夫[おっと]が 単身赴任[たんしんふにん]しているので、 寂[さび]しくてなりません。	
\\	ならない	
\\	日本を知る為には一度は日本へ行かねばならない。	日本[にっぽん]を 知[し]る 為[ため]には一 度[ど]は 日本[にっぽん]へ 行[い]かねばならない。	
\\	-ねばならない	
\\	散歩するなり、泳ぐなり、なにか運動をした方がいいですよ。	散歩[さんぽ]するなり、 泳[およ]ぐなり、なにか 運動[うんどう]をした 方[ほう]がいいですよ。	
\\	-なり-なり	
\\	私は私なりに、人生観を持っています。	私[わたし]は 私[わたし]なりに、 人生観[じんせいかん]を 持[も]っています。	
\\	なりに	
\\	あなたなしでは生きていけない。	あなたなしでは 生[い]きていけない。	
\\	なしでは	
\\	飛行機で行けば簡単に行ける所でしょうに。	飛行機[ひこうき]で 行[い]けば 簡単[かんたん]に 行[い]ける 所[ところ]でしょうに。	
\\	に	
\\	私の留学に当たり、父は自分の経験を話してくれた。	私[わたし]の 留学[りゅうがく]に 当[あ]たり、 父[ちち]は 自分[じぶん]の 経験[けいけん]を 話[はな]してくれた。	
\\	に当たって/当たり	
\\	浩は両親の期待に反して高校を出てからコックになった。	浩[ひろし]は 両親[りょうしん]の 期待[きたい]に 反[はん]して 高校[こうこう]を 出[で]てからコックになった。	
\\	に反して/反する	
\\	結婚は人生の墓場にほかならない。	結婚[けっこん]は 人生[じんせい]の 墓場[はかば]にほかならない。	
\\	にほかならない	
\\	このバーは男性だけに限らず女性の間にも人気がある。	このバーは 男性[だんせい]だけに 限[かぎ]らず 女性[じょせい]の 間[ま]にも 人気[にんき]がある。	
\\	に限らず	
\\	傘を持って来ない日に限って雨が降るんですよ。	傘[かさ]を 持[も]って 来[こ]ない 日[ひ]に 限[かぎ]って 雨[あめ]が 降[ふ]るんですよ。	
\\	に限って	
\\	人間の言語習得に関してはまだまだ不明のことが多い。	人間[にんげん]の 言語[げんご] 習得[しゅうとく]に 関[かん]してはまだまだ 不明[ふめい]のことが 多[おお]い。	
\\	に関して/関する	
\\	私が父に代わって、空港までお客さんを迎えに行った。	私[わたし]が 父[ちち]に 代[か]わって、 空港[くうこう]までお 客[きゃく]さんを 迎[むか]えに 行[い]った。	
\\	に代わって	
\\	今年は去年に比べて雨の日が多いようだ。	今年[ことし]は 去年[きょねん]に 比[くら]べて 雨[あめ]の 日[ひ]が 多[おお]いようだ。	
\\	に比べると/比べて	
\\	激しい雨(だった)にもかかわらず、サッカーの試合は続いた。	激[はげ]しい 雨[あめ](だった)にもかかわらず、サッカーの 試合[しあい]は 続[つづ]いた。	
\\	にもかかわらず	
\\	事実に基づいてお話しします。	事実[じじつ]に 基[もと]づいてお 話[はな]しします。	
\\	にも基づいて/基づく	
\\	四月半ばになると、このへんは桜が綺麗に咲きます。	四月[しがつ] 半[なか]ばになると、このへんは 桜[さくら]が 綺麗[きれい]に 咲[さ]きます。	
\\	になると	
\\	コンピュータは近い将来においてほとんどの家庭に行き渡るだろう。	コンピュータは 近い将来[ちかいしょうらい]においてほとんどの 家庭[かてい]に 行き渡[いきわた]るだろう。	
\\	において/おける	
\\	年をとるに従い、体力が衰える。	年[とし]をとるに 従[したが]い、 体力[たいりょく]が 衰[おとろ]える。	
\\	に従って/従い	
\\	これは私の私見に過ぎない。	これは 私[わたし]の 私見[しけん]に 過[す]ぎない。	
\\	に過ぎない	
\\	私は政治に対して強い関心がある。	私[わたし]は 政治[せいじ]に 対[たい]して 強[つよ]い 関心[かんしん]がある。	
\\	に対して/対し	
\\	これは我々にとって無視出来ない問題だ。	これは 我々[われわれ]にとって 無視[むし] 出来[でき]ない 問題[もんだい]だ。	
\\	にとって	
\\	毎日の生活について聞いてもいいですか。	毎日[まいにち]の 生活[せいかつ]について 聞[き]いてもいいですか。	
\\	について	
\\	切符は一人につき三枚までにして下さい。	切符[きっぷ]は一 人[にん]につき三 枚[まい]までにして 下[くだ]さい。	
\\	につき	
\\	病気が治ってくるにつれて、食欲が出てきた。	病気[びょうき]が 治[なお]ってくるにつれて、 食欲[しょくよく]が 出[で]てきた。	
\\	につれて/つれ	
\\	あの人を説得するには時間がかかる。	あの 人[ひと]を 説得[せっとく]するには 時間[じかん]がかかる。	
\\	には	
\\	戦争によって父を亡くした。	戦争[せんそう]によって 父[ちち]を 亡[な]くした。	
\\	にとって/より	
\\	時間の関係で、残念ですが、詳しくお話し出来ません。	時間[じかん]の 関係[かんけい]で、 残念[ざんねん]ですが、 詳[くわ]しくお 話[はな]し 出来[でき]ません。	
\\	の関係で	
\\	子供のことは私が責任を持ちます。	子供[こども]のことは 私[わたし]が 責任[せきにん]を 持[も]ちます。	
\\	のこと	
\\	傷みやすい豆腐のことだから、冷蔵庫に入れておいた方がいいですよ。	傷[いた]みやすい 豆腐[とうふ]のことだから、 冷蔵庫[れいぞうこ]に 入[い]れておいた 方[ほう]がいいですよ。	
\\	のことだから	
\\	要点のみ(を)話して下さい。	要点[ようてん]のみ(を) 話[はな]して 下[くだ]さい。	
\\	のみ	
\\	このインフレの下では金を貯めても意味がない。	このインフレの 下[した]では 金[きん]を 貯[た]めても 意味[いみ]がない。	
\\	の下で	
\\	この計画は書類の上では問題なさそうだ。	この 計画[けいかく]は 書類[しょるい]の 上[うえ]では 問題[もんだい]なさそうだ。	
\\	の上では	
\\	本格的なロボットが現れたのは1960年代後半のことである。	本格的[ほんかくてき]なロボットが 現[あらわ]れたのは1960 年代[ねんだい] 後半[こうはん]のことである。	
\\	-のは-のことだ	
\\	知らぬことを知らぬと言うには勇気が要る。	知[し]らぬことを 知[し]らぬと 言[い]うには 勇気[ゆうき]が 要[い]る。	
\\	-ぬ	
\\	お客様はまだお帰りじゃありません。	お 客様[きゃくさま]はまだお 帰[かえ]りじゃありません。	
\\	お-だ	
\\	キャシーは、寿司をはじめとして、日本食なら何でも食べる。	キャシーは、 寿司[すし]をはじめとして、 日本食[にほんしょく]なら 何[なん]でも 食[た]べる。	
\\	をはじめ(として)	
\\	少々お待ち下さい。	少々[しょうしょう]お 待[ま]ち 下[くだ]さい。	
\\	お-下さい	
\\	子供の非行を犯罪映画に結び付けるのは短絡的なように思われる。	子供[こども]の 非行[ひこう]を 犯罪[はんざい] 映画[えいが]に 結び付[むすびつ]けるのは 短絡的[たんらくてき]なように 思[おも]われる。	
\\	思われる	
\\	この高校は全寮制になっており、全学生が三つの寮で生活している。	この 高校[こうこう]は 全寮制[ぜんりょうせい]になっており、 全[ぜん] 学生[がくせい]が 三[みっ]つの 寮[りょう]で 生活[せいかつ]している。	
\\	おり	
\\	お書きになった御本を通して、先生のことは存じ上げておりました。	お 書[が]きになった 御本[ごほん]を 通[とお]して、 先生[せんせい]のことは 存じ上[ぞんじあ]げておりました。	
\\	を通して	
\\	友達にまだお金を借りっぱなしだ。	友達[ともだち]にまだお 金[かね]を 借[か]りっぱなしだ。	
\\	っぱなし	
\\	サングラスをかけたやくざっぽい男が街角に立っていた。	サングラスをかけたやくざっぽい 男[おとこ]が 街角[まちかど]に 立[た]っていた。	
\\	っぽい	
\\	今年の冬は五十年来の寒さだ。	今年[ことし]の 冬[ふゆ]は五十 年来[ねんらい]の 寒[さむ]さだ。	
\\	来	
\\	今年の忘年会はまた例の中華料理屋だそうだ。	今年[ことし]の 忘年会[ぼうねんかい]はまた 例[れい]の 中華料理[ちゅうかりょうり] 屋[や]だそうだ。	
\\	例の	
\\	頭がよくなる薬はない。	頭[あたま]がよくなる 薬[くすり]はない。	
\\	そんな恥ずかしいことが出来ますか。	そんな 恥[は]ずかしいことが 出来[でき]ますか。	
\\	僕の犬はろくに芸もしない。	僕[ぼく]の 犬[いぬ]はろくに 芸[げい]もしない。	
\\	ろくに-ない	
\\	君は日本語、読めないよね。B:いや、読めるさ。	
\\	[えい]: 君[くん]は 日本語[にほんご]、 読[よ]めないよね。 
\\	[びー]:いや、 読[よ]めるさ。	
\\	さ	
\\	その歌は子供でさえ知っている。	その 歌[うた]は 子供[こども]でさえ 知[し]っている。	
\\	さえ	
\\	旅行の際には必ずカメラを持って行きます。	旅行[りょこう]の 際[さい]には 必[かなら]ずカメラを 持[も]って 行[い]きます。	
\\	際(に)	
\\	さすが元ボートの選手だけ{あって/に}、体格がいい。	さすが 元[もと]ボートの 選手[せんしゅ]だけ{あって/に}、 体格[たいかく]がいい。	
\\	さすが	
\\	私が失敗したのは彼のせいだ。	私[わたし]が 失敗[しっぱい]したのは 彼[かれ]のせいだ。	
\\	せい	
\\	せめて両親には分かってもらいたいんですが。	せめて 両親[りょうしん]には 分[わ]かってもらいたいんですが。	
\\	せめて	
\\	大阪に着き次第、お電話を差し上げます。	大阪[おおさか]に 着[つ]き 次第[しだい]、お 電話[でんわ]を 差し上[さしあ]げます。	
\\	次第	
\\	彼はいつも夜遅くまで勉強し、しかも朝は誰よりも早く起きる。	彼[かれ]はいつも 夜遅[よるおそ]くまで 勉強[べんきょう]し、しかも 朝[あさ]は 誰[だれ]よりも 早[はや]く 起[お]きる。	
\\	しかも	
\\	日本の車は故障が少ない。したがってよく売れる。	日本[にっぽん]の 車[くるま]は 故障[こしょう]が 少[すく]ない。したがってよく 売[う]れる。	
\\	したがって	
\\	魚が体にいいそうですよ。B:そうかと言って、毎日魚ばかりを食べているわけにもいかないでしょう。	
\\	[えい]: 魚[さかな]が 体[からだ]にいいそうですよ。 
\\	[びー]:そうかと 言[い]って、 毎日[まいにち] 魚[ぎょ]ばかりを 食[た]べているわけにもいかないでしょう。	
\\	そうかと言って	
\\	友達が夕食簡単なものでいいと言った。そこで、ピザを注文した。	友達[ともだち]が 夕食[ゆうしょく] 簡単[かんたん]なものでいいと 言[い]った。そこで、ピザを 注文[ちゅうもん]した。	
\\	そこで	
\\	彼は飲み物を勧めるかもしれません。そこで、遠慮をしてはいけません。	彼[かれ]は 飲み物[のみもの]を 勧[すす]めるかもしれません。そこで、 遠慮[えんりょ]をしてはいけません。	
\\	そこで 
\\	トムはフットボールの選手としては小さいが、そこを脚力で補っている。	トムはフットボールの 選手[せんしゅ]としては 小[ちい]さいが、そこを 脚力[きゃくりょく]で 補[おぎな]っている。	
\\	そこを	
\\	私は子供の時、プールで危うく溺れそうになったことがある。	私[わたし]は 子供[こども]の 時[とき]、プールで 危[あや]うく 溺[おぼ]れそうになったことがある。	
\\	そうになる	
\\	バナナは安い。その上、栄養も豊富だ。	バナナは 安[やす]い。その 上[うえ]、 栄養[えいよう]も 豊富[ほうふ]だ。	
\\	その上	
\\	両親は二人の結婚に反対している。それでも二人は結婚するつもりらしい。	両親[りょうしん]は二 人[にん]の 結婚[けっこん]に 反対[はんたい]している。それでも二 人[にん]は 結婚[けっこん]するつもりらしい。	
\\	それでも	
\\	今晩パーティーに行かない。B:それどころ{か、/じゃないよ。}親父が入院したんだ。	
\\	[えい]: 今晩[こんばん]パーティーに 行[い]かない。 
\\	[びー]:それどころ{か、/じゃないよ。} 親父[おやじ]が 入院[にゅういん]したんだ。	
\\	それどころか	
\\	来年は日本で働くんですね。
\\	いや、それが行けなくなってしまったんです。	
\\	来年[らいねん]は 日本[にっぽん]で 働[はたら]くんですね。 
\\	いや、それが 行[い]けなくなってしまったんです。	
\\	それが	
\\	彼はその場で、それも現金で払ってくれた。	彼[かれ]はその 場[ば]で、それも 現金[げんきん]で 払[はら]ってくれた。	
\\	それも	
\\	このアパートは明るくて清潔だ。それに家賃が安い。	このアパートは 明[あか]るくて 清潔[せいけつ]だ。それに 家賃[やちん]が 安[やす]い。	
\\	それに	
\\	スポーツはテニスとゴルフをします。それと、学生時代はアイスホッケーをしていました。	スポーツはテニスとゴルフをします。それと、 学生時代[がくせいじだい]はアイスホッケーをしていました。	
\\	それと	
\\	勝ちたいでしょうね。B:(ええ)それは勝ちたいですよ。	
\\	[えい]: 勝[か]ちたいでしょうね。 
\\	[びー]:(ええ)それは 勝[か]ちたいですよ。	
\\	それは	
\\	読んだ小説はそれぞれ面白かった。	読[よ]んだ 小説[しょうせつ]はそれぞれ 面白[おもしろ]かった。	
\\	それぞれ	
\\	家に帰ったらすぐ、寝てしまいました。	家[いえ]に 帰[かえ]ったらすぐ、 寝[ね]てしまいました。	
\\	すぐ	
\\	日本へ行くたびに新しいことを学んで帰ります。	日本[にっぽん]へ 行[い]くたびに 新[あたら]しいことを 学[まな]んで 帰[かえ]ります。	
\\	たびに	
\\	私達が困っていた時、助けてくれた者はただの一人もいなかった。	私[わたし] 達[たち]が 困[こま]っていた 時[とき]、 助[たす]けてくれた 者[もの]はただの一 人[にん]もいなかった。	
\\	ただ	
\\	ただの冗談ですよ。	ただの 冗談[じょうだん]ですよ。	
\\	ただの	
\\	そのパーティーのことは確かに聞いたが、いつだったか思い出せない。	そのパーティーのことは 確[たし]かに 聞[き]いたが、いつだったか 思い出[おもいだ]せない。	
\\	確かに-が	
\\	一週間に一度ぐらい運動したところで、あまり効果はないでしょう。	一週間[いっしゅうかん]に一 度[ど]ぐらい 運動[うんどう]したところで、あまり 効果[こうか]はないでしょう。	
\\	たところで	
\\	こちらにいらっしゃってもう何年になりますか。	こちらにいらっしゃってもう 何[なん] 年[ねん]になりますか。	
\\	て	
\\	考えは文字にしてみて初めてはっきりすることが多い。	考[かんが]えは 文字[もじ]にしてみて 初[はじ]めてはっきりすることが 多[おお]い。	
\\	て初めて	
\\	住みやすさの点では私はこの町の方が前に住んだ町より気に入っている。	住[す]みやすさの 点[てん]では 私[わたし]はこの 町[まち]の 方[ほう]が 前[まえ]に 住[す]んだ 町[まち]より 気に入[きにい]っている。	
\\	点(で)	
\\	こんなに部屋が汚くては、お客さんが見えた時に恥ずかしい。	こんなに 部屋[へや]が 汚[きたな]くては、お 客[きゃく]さんが 見[み]えた 時[とき]に 恥[は]ずかしい。	
\\	ては	
\\	幸夫は来年は家が買えると大変喜んでいる。	幸夫[ゆきお]は 来年[らいねん]は 家[いえ]が 買[か]えると 大変[たいへん] 喜[よろこ]んでいる。	
\\	と	
\\	銃声と同時に警官の一人が倒れた。	銃声[じゅうせい]と 同時[どうじ]に 警官[けいかん]の一 人[にん]が 倒[たお]れた。	
\\	と同時に	
\\	仕事が見つかりました。と言ってもパートなんですが。	仕事[しごと]が 見[み]つかりました。と 言[い]ってもパートなんですが。	
\\	と言っても	
\\	彼はもう我慢が出来ないという風に荒々しく立ち上がった。	彼[かれ]はもう 我慢[がまん]が 出来[でき]ないという 風[かぜ]に 荒々[あらあら]しく 立ち上[たちあ]がった。	
\\	という風に	
\\	彼がこの集まりに来ないということは考えられない。	彼[かれ]がこの 集[あつ]まりに 来[こ]ないということは 考[かんが]えられない。	
\\	ということは	
\\	学生は貧乏だというのに、結構いい車を乗り回しているね。	学生[がくせい]は 貧乏[びんぼう]だというのに、 結構[けっこう]いい 車[くるま]を 乗り回[のりまわ]しているね。	
\\	というのに	
\\	パソコンというのはパーソナル・コンピュータのことだ。	パソコンというのはパーソナル・コンピュータのことだ。	
\\	というのは-ことだ	
\\	柔道と言うと、毎日練習した学生時代を思い出す。	柔道[じゅうどう]と 言[い]うと、 毎日[まいにち] 練習[れんしゅう]した 学生時代[がくせいじだい]を 思い出[おもいだ]す。	
\\	と言うと	
\\	カーラは臆病というより(は)むしろ用心深い。	カーラは 臆病[おくびょう]というより(は)むしろ 用心深[ようじんぶか]い。	
\\	というより(は)	
\\	高木さんは大阪に出張するとかでパーティーに来なかった。	高木[たかぎ]さんは 大阪[おおさか]に 出張[しゅっちょう]するとかでパーティーに 来[こ]なかった。	
\\	とかで	
\\	部長に頼んでみたところ、喜んで引き受けてくれた。	部長[ぶちょう]に 頼[たの]んでみたところ、 喜[よろこ]んで 引き受[ひきう]けてくれた。	
\\	ところ	
\\	彼女にプロポーズしたんだろう?どうだった?B:ところが、駄目だったんだ。	
\\	[えい]: 彼女[かのじょ]にプロポーズしたんだろう?どうだった? 
\\	[びー]:ところが、 駄目[だめ]だったんだ。	
\\	ところが	
\\	私は世界のどこに住むうとも一向に構わない。	私[わたし]は 世界[せかい]のどこに 住[す]むうとも 一向[いっこう]に 構[かま]わない。	
\\	とも	
\\	戦後義務教育は九年間となった。	戦後[せんご] 義務教育[ぎむきょういく]は九 年間[ねんかん]となった。	
\\	となる	
\\	秘密が漏れたとなるとこの計画は変更しなければならない。	秘密[ひみつ]が 漏[も]れたとなるとこの 計画[けいかく]は 変更[へんこう]しなければならない。	
\\	となると	
\\	私はただ言われた通り(に)やっているだけです。	私[わたし]はただ 言[い]われた 通[とお]り(に)やっているだけです。	
\\	通り(に)	
\\	温度は一定(だ)とする。	温度[おんど]は 一定[いってい](だ)とする。	
\\	とする 
\\	姉は足がすらっとしている。	姉[あね]は 足[あし]がすらっとしている。	
\\	とする 
\\	ドアを開けたら、途端にカレーの匂いがした。	ドアを 開[あ]けたら、 途端[とたん]にカレーの 匂[にお]いがした。	
\\	途端(に)	
\\	朝から降りそうだった雨がとうとう降り降りだした。	朝[あさ]から 降[お]りそうだった 雨[あめ]がとうとう 降[ふ]り 降[お]りだした。	
\\	とうとう	
\\	武士階級は封建制を共に滅びだ。	武士階級[ぶしかいきゅう]は 封建制[ほうけんせい]を 共[とも]に 滅[ほろ]びだ。	
\\	と共に	
\\	漫画が子供の教育にとって有害だとは限らない。	漫画[まんが]が 子供[こども]の 教育[きょういく]にとって 有害[ゆうがい]だとは 限[かぎ]らない。	
\\	とは限らない	
\\	運動をした後は気分がよくなりませんね。つまり、運動でストレスがなくなるということなんですよ。	運動[うんどう]をした 後[のち]は 気分[きぶん]がよくなりませんね。つまり、 運動[うんどう]でストレスがなくなるということなんですよ。	
\\	つまり	
\\	彼女は夫の無罪を信じつつ息を引き取った。	彼女[かのじょ]は 夫[おっと]の 無罪[むざい]を 信[しん]じつつ 息[いき]を 引き取[ひきと]った。	
\\	つつ	
\\	日本の生活は高くつくと知った上で、日本へ留学した。	日本[にっぽん]の 生活[せいかつ]は 高[たか]くつくと 知[し]った 上[うえ]で、 日本[にっぽん]へ 留学[りゅうがく]した。	
\\	(の)上で	
\\	この映画は話の筋が面白い上(に)配役がいい。	この 映画[えいが]は 話[はなし]の 筋[すじ]が 面白[おもしろ]い 上[うえ](に) 配役[はいやく]がいい。	
\\	上(に)	
\\	日本語の基礎はだいたい二年間で学び得る。	日本語[にほんご]の 基礎[きそ]はだいたい二 年間[ねんかん]で 学[まな]び 得[え]る。	
\\	得る (うる・える)	
\\	ジョンは日本で生まれ、十歳まで日本の学校で勉強した。	ジョンは 日本[にっぽん]で 生[う]まれ、十 歳[さい]まで 日本[にっぽん]の 学校[がっこう]で 勉強[べんきょう]した。	
\\	彼はまだ若いから、考えが浅い。	彼[かれ]はまだ 若[わか]いから、 考[かんが]えが 浅[あさ]い。	
\\	疲れてはいるが、まだ休みたくない。	疲[つか]れてはいるが、まだ 休[やす]みたくない。	
\\	は	
\\	日本は土地は言うまでもなく、家も非常に高い。	日本[にっぽん]は 土地[とち]は 言[い]うまでもなく、 家[いえ]も 非常[ひじょう]に 高[たか]い。	
\\	は言うまでもなく	
\\	赤ん坊は自分ではものを言わないが、話しかけられている時に言葉を学習しているわけだ。	赤ん坊[あかんぼう]は 自分[じぶん]ではものを 言[い]わないが、 話[はな]しかけられている 時[とき]に 言葉[ことば]を 学習[がくしゅう]しているわけだ。	
\\	わけだ	
\\	結果さえよければいいというわけではない。	結果[けっか]さえよければいいというわけではない。	
\\	わけではない	
\\	そんなことをしていいわけがない。	そんなことをしていいわけがない。	
\\	わけがない	
\\	これから運転しなければなりませんから、酒を飲むわけにはいきません。	これから 運転[うんてん]しなければなりませんから、 酒[さけ]を 飲[の]むわけにはいきません。	
\\	わけにはいかない	
\\	僕は弟に自転車をやる詰まりだ。	僕[ぼく]は 弟[おとうと]に 自転車[じてんしゃ]をやる 詰[つ]まりだ。	
\\	やる 
\\	私は正をからかってやった。	私[わたし]は 正[せい]をからかってやった。	
\\	やる 
\\	やっとのことで、家が買えた。	やっとのことで、 家[いえ]が 買[か]えた。	
\\	やっと	
\\	ほかにやりようはないのだろうか。	ほかにやりようはないのだろうか。	
\\	-よう 
\\	あのタワーに上れば、町全体がよく見えよう。	あのタワーに 上[のぼ]れば、 町[まち] 全体[ぜんたい]がよく 見[み]えよう。	
\\	-よう 
\\	バズで通勤する方がより経済です。	バズで 通勤[つうきん]する 方[ほう]がより 経済[けいざい]です。	
\\	より	
\\	これだけの人が拳銃で殺されているのだから、研修所持をより寂しくコントロールせざるを得ないと思う。	これだけの 人[ひと]が 拳銃[けんじゅう]で 殺[ころ]されているのだから、 研修所[けんしゅうじょ] 持[じ]をより 寂[さび]しくコントロールせざるを 得[え]ないと 思[おも]う。	
\\	ざるを得ない	
\\	急がないとバスに乗り遅れるぞ。	急[いそ]がないとバスに 乗り遅[のりおく]れるぞ。	
\\	ぞ	
\end{CJK}
\end{document}