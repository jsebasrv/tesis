\documentclass[8pt]{extreport} 
\usepackage{hyperref}
\usepackage{CJKutf8}
\begin{document}
\begin{CJK}{UTF8}{min}
\\	たばこ屋	
\\	たばこや			
\\	私はたばこ屋でライターを買った。	
\\	たばこ屋 
\\	わたし は たばこや で らいたー を かった			
\\	庭	
\\	にわ			
\\	庭に白い猫がいます。	
\\	庭 
\\	にわ に しろい ねこ が います			
\\	流行る	
\\	はやる			
\\	去年はスニーカーが流行りました。	
\\	流行る 
\\	きょねん は すにーかー が はやりました			
\\	左側	
\\	ひだりがわ			
\\	画面の左側を見てください。	
\\	左側 
\\	がめん の ひだりがわ を みて ください			
\\	袋	
\\	ふくろ			
\\	袋はいりません。	
\\	袋 
\\	ふくろ は いりません			
\\	風呂屋	
\\	ふろや			
\\	昨日、友達とお風呂屋さんに行った。	
\\	風呂屋 
\\	きのう ともだち と おふろやさん に いった			
\\	骨	
\\	ほね			
\\	彼は足の骨を折りました。	
\\	骨 
\\	かれ は あし の ほね を おりました			
\\	右側	
\\	みぎがわ			
\\	彼女はいつも私の右側を歩きます。	
\\	右側 
\\	かのじょ は いつも わたし の みぎがわ を あるきます			
\\	眼鏡	
\\	めがね			
\\	彼は眼鏡をかけています。	
\\	眼鏡 
\\	かれ は めがね を かけて います			
\\	やけど	
\\	彼は手にやけどをしました。	
\\	やけど 
\\	かれ は て に やけど を しました			
\\	汚れ	
\\	よごれ			
\\	靴の汚れを落としました。	
\\	汚れ 
\\	くつ の よごれ を おとしました			
\\	割り算	
\\	わりざん			
\\	割り算は小学校で習います。	
\\	割り算 
\\	わりざん は しょうがっこう で ならいます			
\\	アイスコーヒー	
\\	アイスコーヒーをください。	
\\	アイスコーヒー 
\\	あいすこーひー を ください			
\\	間	
\\	あいだ			
\\	雲の間から月が出た。	
\\	間 
\\	くも の あいだ から つき が でた			
\\	顎	
\\	あご			
\\	顎が痛い。	
\\	顎 
\\	あご が いたい			
\\	朝御飯	
\\	あさごはん			
\\	七時に朝御飯を食べました。	
\\	朝御飯 
\\	しちじ に あさごはん を たべました			
\\	暖まる	
\\	あたたまる			
\\	まだ部屋が暖まらない。	
\\	暖まる 
\\	まだ へや が あたたまらない。			
\\	あっち	
\\	郵便局はあっちです。	
\\	あっち 
\\	ゆうびんきょく は あっち です			
\\	あんなに	
\\	あんなにいい人はいません。	
\\	あんなに 
\\	あんなに いい ひと は いません			
\\	行き	
\\	いき			
\\	行きは新幹線で行った。	
\\	行き 
\\	いき は しんかんせん で いった			
\\	いたずら	
\\	その子はいたずらが大好きだ。	
\\	いたずら 
\\	その こ は いたずら が だいすき だ			
\\	ウェートレス	
\\	ウェートレスを呼んだ。	
\\	ウェートレス 
\\	うぇーとれす を よんだ			
\\	お祝い	
\\	おいわい			
\\	彼の就職のお祝いをしよう。	
\\	お祝い 
\\	かれ の しゅうしょく の おいわい を しよう			
\\	お菓子	
\\	おかし			
\\	お母さんと一緒にお菓子を作りました。	
\\	お菓子 
\\	おかあさん と いっしょ に おかし を つくりました			
\\	お釣り	
\\	おつり			
\\	母はお釣りを確かめた。	
\\	お釣り 
\\	はは は おつり を たしかめた			
\\	大人しい	
\\	おとなしい			
\\	私の彼女はとても大人しいです。	
\\	大人しい 
\\	わたし の かのじょ は とても おとなしい です			
\\	お祖母さん	
\\	おばあさん			
\\	彼女はお祖母さんと住んでいる。	
\\	お祖母さん 
\\	かのじょ は おばあさん と すん でいる 。			
\\	おばさん	
\\	おばさん一家が遊びに来ました。	
\\	おばさん 
\\	おばさん いっか が あそび に きました			
\\	お土産	
\\	おみやげ			
\\	父はお土産にお菓子を買ってきた。	
\\	お土産 
\\	ちち は おみやげ に おかし を かって きた			
\\	表	
\\	おもて			
\\	表に人が来ています。	
\\	表 
\\	おもて に ひと が きて います			
\\	書き取る	
\\	かきとる			
\\	話しのポイントを書き取った。	
\\	書き取る 
\\	はなし の ぽいんと を かきとった			
\\	泊める	
\\	とめる			
\\	友達をうちに泊めてあげました。	
\\	泊める 
\\	ともだち を うち に とめて あげました			
\\	歌手	
\\	かしゅ			
\\	風邪薬	
\\	かぜぐすり			
\\	この風邪薬を飲みなさい。	
\\	風邪薬 
\\	この かぜぐすり を のみなさい			
\\	方	
\\	かた			
\\	次の方、どうぞ。	
\\	方 
\\	つぎ の かた どうぞ			
\\	形	
\\	かたち			
\\	その椅子は変わった形をしている。	
\\	形 
\\	その いす は かわった かたち を して いる			
\\	鞄	
\\	かばん			
\\	その黒い鞄は僕のです。	
\\	鞄 
\\	その くろい かばん は ぼく の です			
\\	被る	
\\	かぶる			
\\	帽子を被って外出した。	
\\	被る 
\\	ぼうし を かぶって がいしゅつ した			
\\	雷	
\\	かみなり			
\\	雷が鳴っています。	
\\	雷 
\\	かみなり が なって います			
\\	上がる	
\\	あがる			
\\	彼は人前だと上がってしまう。	
\\	上がる 
\\	かれ は ひとまえ だ と あがって しまう			
\\	空手	
\\	からて			
\\	彼は空手を習っています。	
\\	空手 
\\	かれ は からて を ならって います			
\\	缶	
\\	かん			
\\	コーヒーはその缶に入っています。	
\\	缶 
\\	こーひー は その かん に はいって います			
\\	元旦	
\\	がんたん			
\\	元旦に彼から年賀状が来た。	
\\	元旦 
\\	がんたん に かれ から ねんがじょう が きた 。			
\\	着替える	
\\	きがえる			
\\	彼はきれいな服に着替えた。	
\\	着替える 
\\	かれ は きれい な ふく に きがえた			
\\	厳しい	
\\	きびしい			
\\	私の上司はとても厳しい。	
\\	厳しい 
\\	わたし の じょうし は とても きびしい			
\\	客	
\\	きゃく			
\\	その店は若い客が多いです。	
\\	客 
\\	その みせ は わかい きゃく が おおい です			
\\	金	
\\	きん			
\\	彼女は金のネックレスをしています。	
\\	金 
\\	かのじょ は きん の ねっくれす を して います			
\\	くし	
\\	彼はくしで髪をとかした。	
\\	くし 
\\	かれ は くし で かみ を とかした			
\\	雲	
\\	くも			
\\	今日は雲が多い。	
\\	雲 
\\	きょう は くも が おおい			
\\	クリーニング	
\\	スーツをクリーニングに出しました。	
\\	クリーニング 
\\	すーつ を くりーにんぐ に だしました			
\\	景色	
\\	けしき			
\\	ここは景色がきれいですね。	
\\	景色 
\\	ここ は けしき が きれい です ね			
\\	喧嘩	
\\	けんか			
\\	喧嘩はやめて。	
\\	喧嘩 
\\	けんか は やめて			
\\	答え	
\\	こたえ			
\\	彼はその問題の答えが分からない。	
\\	答え 
\\	かれ は その もんだい の こたえ が わからない			
\\	ごちそう	
\\	テーブルの上にごちそうが並んでいる。	
\\	ごちそう 
\\	てーぶる の うえ に ごちそう が ならんで いる			
\\	ごちそうする	
\\	今日の夕食は私がごちそうします。	
\\	ごちそう 
\\	ごちそうする 
\\	きょう の ゆうしょく は わたし が ごちそう します			
\\	この頃	
\\	このごろ			
\\	この頃、よく肩が凝る。	
\\	この頃 
\\	このごろ よく かた が こる			
\\	こぼれる	
\\	彼女の目から涙がこぼれた。	
\\	こぼれる 
\\	かのじょ の め から なみだ が こぼれた			
\\	壊れる	
\\	こわれる			
\\	会社のパソコンが壊れた。	
\\	壊れる 
\\	かいしゃ の ぱそこん が こわれた			
\\	こんなに	
\\	こんなに面白い本は初めて読んだ。	
\\	こんなに 
\\	こんなに おもしろい ほん は はじめて よんだ			
\\	ざあざあ	
\\	雨がざあざあ降っている。	
\\	ざあざあ 
\\	あめ が ざあざあ ふって いる			
\\	再来月	
\\	さらいげつ			
\\	再来月まで予約で一杯です。	
\\	再来月 
\\	さらいげつ まで よやく で いっぱい です			
\\	再来週	
\\	さらいしゅう			
\\	再来週は忙しいです。	
\\	再来週 
\\	さらいしゅう は いそがしい です			
\\	騒ぐ	
\\	さわぐ			
\\	電車の中で騒がないでください。	
\\	騒ぐ 
\\	でんしゃ の なか で さわがない で ください			
\\	残念	
\\	ざんねん			
\\	その試合は残念な結果になった。	
\\	残念 
\\	その しあい は ざんねん な けっか に なった			
\\	叱る	
\\	しかる			
\\	父親が子供を叱っている。	
\\	叱る 
\\	ちちおや が こども を しかって いる			
\\	仕舞う	
\\	しまう			
\\	彼女は大切な書類を机に仕舞った。	
\\	仕舞う 
\\	かのじょ は たいせつ な しょるい を つくえ に しまった。			
\\	氏名	
\\	しめい			
\\	ここに住所と氏名を書いてください。	
\\	氏名 
\\	ここ に じゅうしょ と しめい を かいて ください			
\\	ジャガイモ	
\\	ポテトサラダを作るから、ジャガイモを買ってきて。	
\\	ジャガイモ 
\\	ぽてと さらだ を つくる から じゃがいも を かって きて			
\\	車道	
\\	しゃどう			
\\	車道の工事が始まりました。	
\\	車道 
\\	しゃどう の こうじ が はじまりました			
\\	丈夫	
\\	じょうぶ			
\\	祖母は身体が丈夫だ。	
\\	丈夫 
\\	そぼ は からだ が じょうぶ だ			
\\	醤油	
\\	しょうゆ			
\\	もう少し醤油を足してください。	
\\	醤油 
\\	もうすこし しょうゆ を たして ください 。			
\\	白	
\\	しろ			
\\	白は雪の色です。	
\\	白 
\\	しろ は ゆき の いろ です			
\\	心配	
\\	しんぱい			
\\	明日のプレゼンテーションが心配だ。	
\\	心配 
\\	あす の ぷれぜんてーしょん が しんぱい だ			
\\	石鹸	
\\	せっけん			
\\	石鹸で手を洗ってください。	
\\	石鹸 
\\	せっけん で て を あらって ください			
\\	洗濯	
\\	せんたく			
\\	一週間、洗濯をしていない。	
\\	洗濯 
\\	いっしゅうかん せんたく を して いない			
\\	掃除機	
\\	そうじき			
\\	掃除機が壊れた。	
\\	掃除機 
\\	そうじき が こわれた			
\\	袖	
\\	そで			
\\	このシャツは袖が短い。	
\\	袖 
\\	この しゃつ は そで が みじかい			
\\	外	
\\	そと			
\\	外は暑いよ。	
\\	外 
\\	そと は あついよ			
\\	祖父	
\\	そふ			
\\	祖父は元気です。	
\\	祖父 
\\	そふ は げんき です			
\\	ソファー	
\\	このソファーは気持ちがいい。	
\\	ソファー 
\\	この そふぁー は きもち が いい			
\\	祖母	
\\	そぼ			
\\	祖母は京都で生まれました。	
\\	祖母 
\\	そぼ は きょうと で うまれました			
\\	空	
\\	そら			
\\	空の写真を撮るのが好きです。	
\\	空 
\\	そら の しゃしん を とる の が すき です			
\\	剃る	
\\	そる			
\\	父は毎日髭を剃ります。	
\\	剃る 
\\	ちち は まいにち ひげ を そります			
\\	台	
\\	だい			
\\	そこにちょうど良い台がある。	
\\	台 
\\	そこ に ちょうど いい だい が ある			
\\	大事	
\\	だいじ			
\\	お体をお大事に。	
\\	大事 
\\	おからだ を おだいじ に			
\\	大切	
\\	たいせつ			
\\	これは母が大切にしていた指輪です。	
\\	大切 
\\	これ は はは が たいせつ に して いた ゆびわ です			
\\	倒す	
\\	たおす			
\\	そのスキーヤーはポールを倒した。	
\\	倒す 
\\	その すきーやー は ぽーる を たおした			
\\	助ける	
\\	たすける			
\\	彼女は病気の犬を助けた。	
\\	助ける 
\\	かのじょ は びょうき の いぬ を たすけた			
\\	たばこ	
\\	彼はたばこを吸いません。	
\\	たばこ 
\\	かれ は たばこ を すいません			
\\	誰か	
\\	だれか			
\\	誰かに聞いてみてください。	
\\	誰か 
\\	だれか に きいて みて ください			
\\	血	
\\	ち			
\\	血が出ていますよ。	
\\	血 
\\	ち が でて います よ			
\\	茶わん	
\\	ちゃわん			
\\	茶わんにご飯をよそいました。	
\\	茶わん 
\\	ちゃわん に ごはん を よそいました			
\\	昼食	
\\	ちゅうしょく			
\\	昼食に寿司を食べました。	
\\	昼食 
\\	ちゅうしょく に すし を たべました			
\\	長女	
\\	ちょうじょ			
\\	うちの長女は10歳です。	
\\	長女 
\\	うち の ちょうじょ は じゅっさい です			
\\	朝食	
\\	ちょうしょく			
\\	朝食に納豆を食べました。	
\\	朝食 
\\	ちょうしょく に なっとう を たべました			
\\	ちょうど	
\\	値段はちょうど3万円です。	
\\	ちょうど 
\\	ねだん は ちょうど さんまんえん です			
\\	ちょっと	
\\	ちょっとだけ待っていてください。	
\\	ちょっと 
\\	ちょっと だけ まって いて ください			
\\	つく	
\\	部屋の電気がついています。	
\\	つく 
\\	へや の でんき が ついて います			
\\	机	
\\	つくえ			
\\	新しい机を買ってもらいました。	
\\	机 
\\	あたらしい つくえ を かって もらいました			
\\	勤める	
\\	つとめる			
\\	私は銀行に勤めています。	
\\	勤める 
\\	わたし は ぎんこう に つとめて います			
\\	爪	
\\	つめ			
\\	爪が伸びています。	
\\	爪 
\\	つめ が のびて います			
\\	寺	
\\	てら			
\\	あそこに古いお寺があります。	
\\	寺 
\\	あそこ に ふるい おてら が あります			
\\	てんぷら	
\\	私はえびのてんぷらが好きです。	
\\	てんぷら 
\\	わたし は えび の てんぷら が すき です			
\\	年上	
\\	としうえ			
\\	彼は私より年上です。	
\\	年上 
\\	かれ は わたし より としうえ です			
\\	鳥	
\\	とり			
\\	あそこにきれいな鳥がいます。	
\\	鳥 
\\	あそこ に きれい な とり が います			
\\	とんとん	
\\	母の肩をとんとんたたいた。	
\\	とんとん 
\\	はは の かた を とんとん たたいた			
\\	長袖	
\\	ながそで			
\\	今日は寒いので長袖を着ました。	
\\	長袖 
\\	きょう は さむい ので ながそで を きました 。			
\\	茄子	
\\	なす			
\\	夕飯に茄子の天ぷらを食べました。	
\\	茄子 
\\	ゆうはん に なすび の てんぷら を たべました 。			
\\	何	
\\	なん			
\\	質問は何ですか。	
\\	何 
\\	しつもん は なん です か			
\\	兄さん	
\\	にいさん			
\\	兄さん、おめでとう。	
\\	兄さん 
\\	にいさん おめでとう			
\\	賑やか	
\\	にぎやか			
\\	浅草は賑やかな街です。	
\\	賑やか 
\\	あさくさ は にぎやか な まち です			
\\	肉	
\\	にく			
\\	昨日の夜は肉をたくさん食べました。	
\\	肉 
\\	きのう の よる は にく を たくさん たべました			
\\	鶏	
\\	にわとり			
\\	その鶏は毎朝鳴きます。	
\\	鶏 
\\	その にわとり は まいあさ なきます			
\\	人気	
\\	にんき			
\\	このバンドはとても人気があるよ。	
\\	人気 
\\	この ばんど は とても にんき が ある よ			
\\	人間	
\\	にんげん			
\\	人間の心は複雑です。	
\\	人間 
\\	にんげん の こころ は ふくざつ です			
\\	にんじん	
\\	私はにんじんが嫌いだ。	
\\	にんじん 
\\	わたし は にんじん が きらい だ			
\\	ぬるい	
\\	濡れる	
\\	ぬれる			
\\	雨で濡れてしまった。	
\\	濡れる 
\\	あめ で ぬれて しまった			
\\	姉さん	
\\	ねえさん			
\\	姉さん、ごめんね。	
\\	姉さん 
\\	ねえさん ごめん ね			
\\	眠い	
\\	ねむい			
\\	今日はとても眠いです。	
\\	眠い 
\\	きょう は とても ねむい です			
\\	眠る	
\\	ねむる			
\\	昨日は8時間眠りました。	
\\	眠る 
\\	きのう は はちじかん ねむりました			
\\	上り	
\\	のぼり			
\\	これは上り電車です。	
\\	上り 
\\	これ は のぼり でんしゃ です			
\\	パーセント	
\\	この村の80パーセントは老人です。	
\\	パーセント 
\\	この むら の はちじゅっぱーせんと は ろうじん です			
\\	灰	
\\	はい			
\\	火事ですべて灰になった。	
\\	灰 
\\	かじ で すべて はい に なった			
\\	灰色	
\\	はいいろ			
\\	今日の空は灰色だ。	
\\	灰色 
\\	きょう の そら は はいいろ だ			
\\	拍手	
\\	はくしゅ			
\\	大きな拍手が上がった。	
\\	拍手 
\\	おおき な はくしゅ が あがった			
\\	橋	
\\	はし			
\\	あの橋は日本で一番長い。	
\\	橋 
\\	あの はし は にほん で いちばん ながい			
\\	初め	
\\	はじめ			
\\	初めは上手くできませんでした。	
\\	初め 
\\	はじめ は うまく できません でした 。			
\\	柱	
\\	はしら			
\\	この家の柱は太い。	
\\	柱 
\\	この いえ の はしら は ふとい			
\\	旗	
\\	はた			
\\	旗が風に揺れている。	
\\	旗 
\\	はた が かぜ に ゆれて いる			
\\	貼る	
\\	はる			
\\	机にシールを貼った。	
\\	机 
\\	貼る 
\\	つくえ に しーる を はった			
\\	半袖	
\\	はんそで			
\\	今日は、半袖のシャツを着よう。	
\\	半袖 
\\	きょう は 、 はんそで の シャツ を きよう 。			
\\	半月	
\\	はんつき			
\\	半月前に日本に来ました。	
\\	半月 
\\	はんつきまえ に にほん に きました			
\\	半日	
\\	はんにち			
\\	今日は半日だけ仕事だ。	
\\	半日 
\\	きょう は はんにち だけ しごと だ			
\\	火	
\\	ひ			
\\	火を消して。	
\\	火 
\\	ひ を けして			
\\	弾く	
\\	ひく			
\\	彼はギターを弾きます。	
\\	弾く 
\\	かれ は ぎたー を ひきます			
\\	ひざ	
\\	スキーでひざを怪我しました。	
\\	ひざ 
\\	すきー で ひざ を けがしました			
\\	ひじ	
\\	彼はひじに怪我をした。	
\\	ひじ 
\\	かれ は ひじ に けが を した			
\\	人々	
\\	ひとびと			
\\	あの村の人々はとても親切です。	
\\	人々 
\\	あの むら の ひとびと は とても しんせつ です			
\\	ひも	
\\	靴のひもが切れた。	
\\	ひも 
\\	くつ の ひも が きれた			
\\	この表を見てください。	
\\	表 
\\	この ひょう を みて ください			
\\	昼過ぎ	
\\	ひるすぎ			
\\	今日は昼過ぎまで寝ていた。	
\\	昼過ぎ 
\\	きょう は ひるすぎ まで ねて いた			
\\	瓶	
\\	びん			
\\	瓶ビールを注文した。	
\\	瓶 
\\	びんびーる を ちゅうもん した			
\\	拭く	
\\	ふく			
\\	タオルで体を拭きました。	
\\	拭く 
\\	たおる で からだ を ふきました			
\\	復習	
\\	ふくしゅう			
\\	昨日の復習をしましたか。	
\\	復習 
\\	きのう の ふくしゅう を しました か			
\\	豚肉	
\\	ぶたにく			
\\	豚肉は美味しいです。	
\\	豚肉 
\\	ぶたにく は おいしい です			
\\	船便	
\\	ふなびん			
\\	アメリカの友人から船便が届いた。	
\\	船便 
\\	あめりか の ゆうじん から ふなびん が とどいた			
\\	船	
\\	ふね			
\\	私たちは船に乗った。	
\\	船 
\\	わたしたち は ふね に のった			
\\	降る	
\\	ふる			
\\	明日は雨が降るでしょう。	
\\	降る 
\\	あす は あめ が ふる でしょう			
\\	風呂	
\\	ふろ			
\\	風呂がぬるかった。	
\\	ぬるい 
\\	風呂 
\\	ふろ が ぬるかった			
\\	父は今お風呂に入っています。	
\\	風呂 
\\	ちち は いま お ふろ に はいって います			
\\	平成	
\\	へいせい			
\\	彼女は平成3年生まれです。	
\\	平成 
\\	かのじょ は へいせい さんねん うまれ です			
\\	下手	
\\	へた			
\\	その歌手は歌が下手だ。	
\\	歌手 
\\	下手 
\\	その かしゅ は うた が へた だ			
\\	私は歌が下手だ。	
\\	下手 
\\	わたし は うた が へた だ			
\\	返事	
\\	へんじ			
\\	手紙の返事を出しました。	
\\	返事 
\\	てがみ の へんじ を だしました			
\\	便所	
\\	べんじょ			
\\	便所はそこです。	
\\	便所 
\\	べんじょ は そこ です			
\\	星	
\\	ほし			
\\	今夜は星がよく見えます。	
\\	星 
\\	こんや は ほし が よく みえます			
\\	本当	
\\	ほんとう			
\\	その話は本当ですか。	
\\	本当 
\\	その はなし は ほんとう です か			
\\	毎月	
\\	まいつき			
\\	私は毎月貯金をしています。	
\\	毎月 
\\	わたし は まいつき ちょきん を して います			
\\	まく	
\\	日本では、二月に豆をまく行事がある。	
\\	まく 
\\	にほん で は にがつ に まめ を まく ぎょうじ が ある			
\\	枕	
\\	まくら			
\\	私は低い枕が好きです。	
\\	枕 
\\	わたし は ひくい まくら が すき です			
\\	孫	
\\	まご			
\\	昨日、孫が生まれました。	
\\	孫 
\\	きのう まご が うまれました			
\\	又	
\\	また			
\\	明日、また来ます。	
\\	又 
\\	あした また きます			
\\	町	
\\	まち			
\\	ここは小さな町です。	
\\	町 
\\	ここ は ちいさ な まち です			
\\	丸	
\\	まる			
\\	紙に大きな丸を書きました。	
\\	丸 
\\	かみ に おおき な まる を かきました			
\\	港	
\\	みなと			
\\	港に船が着きました。	
\\	港 
\\	みなと に ふね が つきました			
\\	南	
\\	みなみ			
\\	私の家は町の南にあります。	
\\	南 
\\	わたし の いえ は まち の みなみ に あります			
\\	紫	
\\	むらさき			
\\	彼女は紫のドレスを着ていた。	
\\	紫 
\\	かのじょ は むらさき の どれす を きて いた			
\\	姪	
\\	めい			
\\	私の姪は3才です。	
\\	姪 
\\	わたし の めい は さんさい です			
\\	木綿	
\\	もめん			
\\	彼女は木綿のシャツを着ています。	
\\	木綿 
\\	かのじょ は もめん の しゃつ を きて います			
\\	やかん	
\\	やかんでお湯を沸かしました。	
\\	やかん 
\\	やかん で おゆ を わかしました			
\\	焼く	
\\	やく			
\\	今、魚を焼いています。	
\\	焼く 
\\	いま さかな を やいて います			
\\	痩せる	
\\	やせる			
\\	私は少し痩せました。	
\\	痩せる 
\\	わたし は すこし やせました			
\\	山	
\\	やま			
\\	山の空気はきれいだ。	
\\	山 
\\	やま の くうき は きれい だ			
\\	湯	
\\	ゆ			
\\	お風呂のお湯が一杯です。	
\\	湯 
\\	おふろ の おゆ が いっぱい です			
\\	夕食	
\\	ゆうしょく			
\\	夕食は7時です。	
\\	夕食 
\\	ゆうしょく は しちじ です			
\\	昨夜	
\\	ゆうべ			
\\	昨夜、流れ星を見ました。	
\\	昨夜 
\\	ゆうべ ながれぼし を みました			
\\	東京行きの列車に乗った。	
\\	行き 
\\	とうきょうゆき の れっしゃ に のった			
\\	指	
\\	ゆび			
\\	彼は指が太い。	
\\	指 
\\	かれ は ゆび が ふとい			
\\	緩い	
\\	ゆるい			
\\	このズボンは緩いです。	
\\	緩い 
\\	この ずぼん は ゆるい です			
\\	良い	
\\	よい			
\\	彼は良い人です。	
\\	良い 
\\	かれ は よい ひと です			
\\	ようこそ	
\\	日本へようこそ。	
\\	ようこそ 
\\	にっぽん へ ようこそ			
\\	汚れる	
\\	よごれる			
\\	エプロンをしないと服が汚れます。	
\\	汚れる 
\\	えぷろん を しない と ふく が よごれます			
\\	四つ角	
\\	よつかど			
\\	あそこの四つ角を左に曲がってください。	
\\	四つ角 
\\	あそこ の よつかど を ひだり に まがって ください			
\\	夜中	
\\	よなか			
\\	夜中に電話がありました。	
\\	夜中 
\\	よなか に でんわ が ありました			
\\	両親	
\\	りょうしん			
\\	私の両親は大阪に住んでいます。	
\\	両親 
\\	わたし の りょうしん は おおさか に すんで います			
\\	零	
\\	れい			
\\	今ちょうど零時です。	
\\	ちょうど 
\\	零 
\\	いま ちょうど れいじ です			
\\	曲がり角	
\\	まがりかど			
\\	ポストはそこの曲がり角にあります。	
\\	曲がり角 
\\	ぽすと は そこ の まがりかど に あります			
\\	ウェーター	
\\	ウェーターが水を運んできました。	
\\	ウェーター 
\\	うぇーたー が みず を はこんで きました			
\\	お医者さん	
\\	おいしゃさん			
\\	熱があるのでお医者さんに行った。	
\\	お医者さん 
\\	ねつ が ある の で おいしゃさん に いった			
\\	細か	
\\	こまか			
\\	それは細かな問題です。	
\\	細か 
\\	それ は こまか な もんだい です			
\\	絶対に	
\\	ぜったいに			
\\	絶対に駄目です。	
\\	絶対に 
\\	ぜったいに だめ です			
\\	そんなに	
\\	一度にそんなにたくさんはできない。	
\\	そんなに 
\\	いちど に そんなに たくさん は できない			
\\	連れて行く	
\\	つれていく			
\\	私も連れて行ってください。	
\\	連れて行く 
\\	わたし も つれて いって ください			
\\	連れて来る	
\\	つれてくる			
\\	息子が友達を連れて来ました。	
\\	連れて来る 
\\	むすこ が ともだち を つれて きました			
\\	似ている	
\\	にている			
\\	私は母に似ています。	
\\	似ている 
\\	わたし は はは に にて います			
\\	音	
\\	おと			
\\	このピアノは音があまり良くない。	
\\	音 
\\	この ぴあの は おと が あまり よく ない			
\\	始めに	
\\	はじめに			
\\	始めにスープが出ます。	
\\	始めに 
\\	はじめに すーぷ が でます			
\\	一人で	
\\	ひとりで			
\\	今日は一人で映画を見ます。	
\\	一人で 
\\	きょう は ひとりで えいが を みます			
\\	不動産屋	
\\	ふどうさんや			
\\	私の父は不動産屋です。	
\\	不動産屋 
\\	わたし の ちち は ふどうさんや です			
\\	郵便屋さん	
\\	ゆうびんやさん			
\\	郵便屋さんはもう来ましたか。	
\\	郵便屋さん 
\\	ゆうびんやさん は もう きました か			
\\	可	
\\	か			
\\	このアルバイトは「学生可」ですね。	
\\	可 
\\	この あるばいと は がくせい か です ね			
\end{CJK}
\end{document}