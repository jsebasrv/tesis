\documentclass[8pt]{extreport} 
\usepackage{hyperref}
\usepackage{CJKutf8}
\begin{document}
\begin{CJK}{UTF8}{min}
\\	経験	
\\	けいけん			
\\	今日の試合はいい経験になりました。	
\\	経験 
\\	きょう の しあい は いい けいけん に なりました			
\\	能力	
\\	のうりょく			
\\	彼は能力のある社員です。	
\\	能力 
\\	かれ は のうりょく の ある しゃいん です			
\\	最初	
\\	さいしょ			
\\	5ページの最初を見てください。	
\\	最初 
\\	ごぺーじ の さいしょ を みて ください			
\\	国際	
\\	こくさい			
\\	ここで国際会議が開かれます。	
\\	国際 
\\	ここ で こくさい かいぎ が ひらかれます			
\\	議員	
\\	ぎいん			
\\	彼は国会議員です。	
\\	議員 
\\	かれ は こっかい ぎいん です			
\\	記者	
\\	きしゃ			
\\	彼は新聞記者です。	
\\	記者 
\\	かれ は しんぶん きしゃ です			
\\	場所	
\\	ばしょ			
\\	会社の場所を教えてください。	
\\	場所 
\\	かいしゃ の ばしょ を おしえて ください			
\\	運動	
\\	うんどう			
\\	彼はもっと運動した方がいい。	
\\	運動 
\\	かれ は もっと うんどう した ほう が いい			
\\	サービス	
\\	サービス			
\\	この店はサービスがいい。	
\\	サービス 
\\	この みせ は サービス が いい 。			
\\	集まる	
\\	あつまる			
\\	駅前に人が集まっています。	
\\	集まる 
\\	えきまえ に ひと が あつまって います			
\\	政策	
\\	せいさく			
\\	新しい政策はあまり良いとは思えません。	
\\	政策 
\\	あたらしい せいさく は あまり いい と は おもえません			
\\	車	
\\	くるま			
\\	この道は車が多い。	
\\	車 
\\	この みち は くるま が おおい			
\\	拡大	
\\	かくだい			
\\	この図を拡大コピーしてください。	
\\	拡大 
\\	この ず を かくだい こぴー して ください			
\\	進む	
\\	すすむ			
\\	前に進んでください。	
\\	進む 
\\	まえ に すすんで ください			
\\	達する	
\\	たっする			
\\	気温は35度に達した。	
\\	達する 
\\	きおん は さんじゅうごど に たっした			
\\	会議	
\\	かいぎ			
\\	今日の午後、大事な会議があります。	
\\	会議 
\\	きょう の ごご だいじ な かいぎ が あります			
\\	装置	
\\	そうち			
\\	実験にはこの装置を使います。	
\\	装置 
\\	じっけん に は この そうち を つかいます			
\\	主張	
\\	しゅちょう			
\\	上司は私の主張を受け入れた。	
\\	主張 
\\	じょうし は わたし の しゅちょう を うけいれた			
\\	記録	
\\	きろく			
\\	マラソンで世界記録が出た。	
\\	記録 
\\	まらそん で せかい きろく が でた			
\\	年度	
\\	ねんど			
\\	売り上げは年度によって違います。	
\\	年度 
\\	うりあげ は ねんど に よって ちがいます			
\\	産業	
\\	さんぎょう			
\\	背景	
\\	はいけい			
\\	事件の背景に何があったのだろう。	
\\	背景 
\\	じけん の はいけい に なに が あった の だろう			
\\	材料	
\\	ざいりょう			
\\	サラダの材料をそろえました。	
\\	材料 
\\	さらだ の ざいりょう を そろえました			
\\	要求	
\\	ようきゅう			
\\	彼は私の要求にこたえた。	
\\	要求 
\\	かれ は わたし の ようきゅう に こたえた			
\\	市民	
\\	しみん			
\\	世紀	
\\	せいき			
\\	新しい世紀の始まりです。	
\\	世紀 
\\	あたらしい せいき の はじまり です			
\\	大幅	
\\	おおはば			
\\	計画を大幅に変更した。	
\\	大幅 
\\	けいかく を おおはば に へんこう した			
\\	加工	
\\	かこう			
\\	この工場では魚を加工している。	
\\	加工 
\\	この こうじょう で は さかな を かこう して いる			
\\	実行	
\\	じっこう			
\\	彼はその計画を実行した。	
\\	実行 
\\	かれ は その けいかく を じっこう した			
\\	判断	
\\	はんだん			
\\	彼の判断は正しい。	
\\	判断 
\\	かれ の はんだん は ただしい			
\\	組織	
\\	そしき			
\\	彼はある組織のリーダーだ。	
\\	組織 
\\	かれ は ある そしき の りーだー だ			
\\	輸出	
\\	ゆしゅつ			
\\	彼の会社は車を輸出しています。	
\\	輸出 
\\	かれ の かいしゃ は くるま を ゆしゅつ して います			
\\	警察	
\\	けいさつ			
\\	警察を呼んでください。	
\\	警察 
\\	けいさつ を よんで ください			
\\	伝える	
\\	つたえる			
\\	みんなにこのことを伝えてください。	
\\	伝える 
\\	みんな に この こと を つたえて ください			
\\	公開	
\\	こうかい			
\\	その映画は今日、公開されます。	
\\	公開 
\\	その えいが は きょう こうかい されます			
\\	向かう	
\\	むかう			
\\	今、会社に向かっています。	
\\	向かう 
\\	いま かいしゃ に むかって います			
\\	直す	
\\	なおす			
\\	私がそれを直しました。	
\\	直す 
\\	わたし が それ を なおしました			
\\	具体的	
\\	ぐたいてき			
\\	具体的な例をいくつか見せてください。	
\\	具体的 
\\	ぐたいてき な れい を いくつか みせて ください			
\\	願う	
\\	ねがう			
\\	彼が元気になるよう願っています。	
\\	願う 
\\	かれ が げんき に なる よう ねがって います			
\\	グループ	
\\	グループ			
\\	店に学生のグループが来た。	
\\	グループ 
\\	みせ に がくせい の ぐるーぷ が きた			
\\	個人	
\\	こじん			
\\	これは私個人の意見です。	
\\	個人 
\\	これ は わたし こじん の いけん です			
\\	基づく	
\\	もとづく			
\\	この話は真実に基づいています。	
\\	基づく 
\\	この はなし は しんじつ に もとづいて います			
\\	全く	
\\	まったく			
\\	私には全く分かりません。	
\\	全く 
\\	わたし に は まったく わかりません			
\\	見方	
\\	みかた			
\\	彼に対する見方が変わりました。	
\\	見方 
\\	かれ に たいする みかた が かわりました			
\\	姿勢	
\\	しせい			
\\	あの子はいつも姿勢が悪い。	
\\	姿勢 
\\	あの こ は いつも しせい が わるい			
\\	協力	
\\	きょうりょく			
\\	このプロジェクトにはみんなの協力が必要です。	
\\	協力 
\\	この ぷろじぇくと に は みんな の きょうりょく が ひつよう です			
\\	担当	
\\	たんとう			
\\	私はセールスを担当しています。	
\\	担当 
\\	わたし は せーるす を たんとう して います			
\\	提案	
\\	ていあん			
\\	そのアイデアは彼の提案です。	
\\	提案 
\\	その あいであ は かれ の ていあん です			
\\	最後	
\\	さいご			
\\	今日が夏休み最後の日だ。	
\\	最後 
\\	きょう が なつやすみ さいご の ひ だ			
\\	招く	
\\	まねく			
\\	両親を食事に招いた。	
\\	招く 
\\	りょうしん を しょくじ に まねいた			
\\	ホテル	
\\	今ホテルに着きました。	
\\	ホテル 
\\	いま ほてる に つきました			
\\	再び	
\\	ふたたび			
\\	彼は再びここに戻ってきた。	
\\	再び 
\\	かれ は ふたたび ここ に もどって きた			
\\	上昇	
\\	じょうしょう			
\\	地球の気温は上昇している。	
\\	上昇 
\\	ちきゅう の きおん は じょうしょう して いる			
\\	機械	
\\	きかい			
\\	新しい機械が壊れた。	
\\	機械 
\\	あたらしい きかい が こわれた			
\\	注目	
\\	ちゅうもく			
\\	私たちはその会社に注目している。	
\\	注目 
\\	わたしたち は その かいしゃ に ちゅうもく して いる			
\\	過去	
\\	かこ			
\\	それは過去の話だ。	
\\	過去 
\\	それ は かこ の はなし だ			
\\	作品	
\\	さくひん			
\\	この絵はゴッホの作品だ。	
\\	作品 
\\	この え は ごっほ の さくひん だ			
\\	直接	
\\	ちょくせつ			
\\	彼に直接お願いしなさい。	
\\	直接 
\\	かれ に ちょくせつ おねがい しなさい			
\\	批判	
\\	ひはん			
\\	彼は同僚を批判した。	
\\	批判 
\\	かれ は どうりょう を ひはん した			
\\	文化	
\\	ぶんか			
\\	私はこの国の文化を勉強しています。	
\\	文化 
\\	わたし は この くに の ぶんか を べんきょう して います			
\\	施設	
\\	しせつ			
\\	そのホテルにはレジャー施設がたくさんある。	
\\	施設 
\\	その ほてる に は れじゃー しせつ が たくさん ある			
\\	成功	
\\	せいこう			
\\	実際	
\\	じっさい			
\\	彼は実際にはあまり背が高くない。	
\\	実際 
\\	かれ は じっさい に は あまり せ が たかく ない			
\\	当然	
\\	とうぜん			
\\	彼女が怒るのも当然だ。	
\\	当然 
\\	かのじょ が おこる の も とうぜん だ			
\\	年間	
\\	ねんかん			
\\	簡単	
\\	かんたん			
\\	この料理はとても簡単です。	
\\	簡単 
\\	この りょうり は とても かんたん です			
\\	迫る	
\\	せまる			
\\	締め切りが迫っています。	
\\	迫る 
\\	しめきり が せまって います			
\\	まとめる	
\\	彼女は荷物をまとめて出て行った。	
\\	まとめる 
\\	かのじょ は にもつ を まとめて でて いった			
\\	占める	
\\	しめる			
\\	私に届くメールのうち、迷惑メールが7割を占めている。	
\\	占める 
\\	わたし に とどく めーる の うち めいわく めーる が ななわり を しめて いる			
\\	残す	
\\	のこす			
\\	彼女はメッセージを残しました。	
\\	残す 
\\	かのじょ は めっせーじ を のこしました			
\\	言語	
\\	げんご			
\\	彼はアジアの言語を研究している。	
\\	言語 
\\	かれ は あじあ の げんご を けんきゅう して いる			
\\	教育	
\\	きょういく			
\\	彼は海外で教育を受けました。	
\\	教育 
\\	かれ は かいがい で きょういく を うけました			
\\	他	
\\	た			
\\	留学生はアジア人が多く、その他は3割です。	
\\	他 
\\	りゅうがくせい は あじあじん が おおく そのた は さんわり です			
\\	取引	
\\	とりひき			
\\	私たちは中国の会社と取引しています。	
\\	取引 
\\	わたしたち は ちゅうごく の かいしゃ と とりひき して います			
\\	自動車	
\\	じどうしゃ			
\\	日本の自動車産業は世界的に有名だ。	
\\	産業 
\\	自動車 
\\	にほん の じどうしゃ さんぎょう は せかいてき に ゆうめい だ			
\\	彼は自動車会社に就職した。	
\\	自動車 
\\	かれ は じどうしゃ がいしゃ に しゅうしょく した			
\\	反対	
\\	はんたい			
\\	私は反対です。	
\\	反対 
\\	わたし は はんたい です			
\\	若者	
\\	わかもの			
\\	最近の若者は本を読まない。	
\\	若者 
\\	さいきん の わかもの は ほん を よまない			
\\	分	
\\	ぶん			
\\	このケーキはあなたの分です。	
\\	分 
\\	この けーき は あなた の ぶん です			
\\	課題	
\\	かだい			
\\	夏休みの課題は何ですか。	
\\	課題 
\\	なつやすみ の かだい は なん です か			
\\	通常	
\\	つうじょう			
\\	通常は夜8時まで営業しています。	
\\	通常 
\\	つうじょう は よる はちじ まで えいぎょう して います			
\\	同様	
\\	どうよう			
\\	私たちは彼を家族同様に思っている。	
\\	同様 
\\	わたしたち は かれ を かぞく どうよう に おもって いる			
\\	程度	
\\	ていど			
\\	この程度の怪我なら大丈夫です。	
\\	程度 
\\	この ていど の けが なら だいじょうぶ です			
\\	当たる	
\\	あたる			
\\	ボールが彼の頭に当たった。	
\\	当たる 
\\	ぼーる が かれ の あたま に あたった			
\\	訴える	
\\	うったえる			
\\	彼女は会社を訴えた。	
\\	訴える 
\\	かのじょ は かいしゃ を うったえた			
\\	事情	
\\	じじょう			
\\	あなたの事情はよく分かりました。	
\\	事情 
\\	あなた の じじょう は よく わかりました			
\\	発言	
\\	はつげん			
\\	会議で全員が発言した。	
\\	発言 
\\	かいぎ で ぜんいん が はつげん した			
\\	傾向	
\\	けいこう			
\\	最近の若者は難しい本を読まない傾向がある。	
\\	若者 
\\	傾向 
\\	さいきん の わかもの は むずかしい ほん を よまない けいこう が ある			
\\	開催	
\\	かいさい			
\\	京都で国際会議が開催された。	
\\	国際 
\\	開催 
\\	きょうと で こくさいかいぎ が かいさい された			
\\	文字	
\\	もじ			
\\	壁に文字が書いてあった。	
\\	文字 
\\	かべ に もじ が かいて あった			
\\	子	
\\	こ			
\\	その子は日本語が分からない。	
\\	子 
\\	その こ は にほんご が わからない			
\\	述べる	
\\	のべる			
\\	上司が意見を述べた。	
\\	述べる 
\\	じょうし が いけん を のべた			
\\	活動	
\\	かつどう			
\\	彼は地域の活動に参加した。	
\\	活動 
\\	かれ は ちいき の かつどう に さんか した			
\\	期間	
\\	きかん			
\\	テスト期間は10日から15日までだ。	
\\	期間 
\\	てすと きかん は とおか から じゅうごにち まで だ			
\\	一定	
\\	いってい			
\\	申し込むには一定の資格が要る。	
\\	一定 
\\	もうしこむ に は いってい の しかく が いる			
\\	住民	
\\	じゅうみん			
\\	地域の住民が集まって話合いをした。	
\\	住民 
\\	ちいき の じゅうみん が あつまって はなしあい を した			
\\	届く	
\\	とどく			
\\	昨日、父から手紙が届いた。	
\\	届く 
\\	きのう ちち から てがみ が とどいた			
\\	やはり	
\\	彼はやはり遅刻しました。	
\\	やはり 
\\	かれ は やはり ちこく しました			
\\	効果	
\\	こうか			
\\	この薬には胃を守る効果がある。	
\\	効果 
\\	この くすり に は い を まもる こうか が ある			
\\	状況	
\\	じょうきょう			
\\	この状況では出発は難しいです。	
\\	状況 
\\	この じょうきょう で は しゅっぱつ は むずかしい です			
\\	確認	
\\	かくにん			
\\	もう一度、予約を確認した。	
\\	確認 
\\	もう いちど よやく を かくにん した			
\\	位置	
\\	いち			
\\	私の町は東京の北に位置します。	
\\	位置 
\\	わたし の まち は とうきょう の きた に いち します			
\\	はっきり	
\\	はっきり			
\\	今日は山がはっきり見える。	
\\	はっきり 
\\	きょう は やま が はっきり みえる			
\\	つまり	
\\	つまり、あなたは何も知らないのですね。	
\\	つまり 
\\	つまり あなた は なに も しらない の です ね			
\\	通過	
\\	つうか			
\\	次の駅は通過します。	
\\	通過 
\\	つぎ の えき は つうか します			
\\	深い	
\\	ふかい			
\\	あの池はとても深い。	
\\	深い 
\\	あの いけ は とても ふかい			
\\	候補	
\\	こうほ			
\\	会長の候補は3人います。	
\\	候補 
\\	かいちょう の こうほ は さんにん います			
\\	選手	
\\	せんしゅ			
\\	彼はプロのサッカー選手だ。	
\\	選手 
\\	かれ は ぷろ の さっかー せんしゅ だ			
\\	音	
\\	おと			
\\	雨の音が聞こえる。	
\\	音 
\\	あめ の おと が きこえる			
\\	大会	
\\	たいかい			
\\	夏には川辺で花火大会があります。	
\\	大会 
\\	なつ に は かわべ で はなびたいかい が あります			
\\	基本	
\\	きほん			
\\	今、ジャズダンスの基本を習っています。	
\\	基本 
\\	いま じゃず だんす の きほん を ならって います			
\\	契約	
\\	けいやく			
\\	その選手は新しいチームと契約した。	
\\	契約 
\\	その せんしゅ は あたらしい ちーむ と けいやく した			
\\	発見	
\\	はっけん			
\\	新しい星が発見された。	
\\	発見 
\\	あたらしい ほし が はっけん された			
\\	当時	
\\	とうじ			
\\	彼女は当時、まだ3才だった。	
\\	当時 
\\	かのじょ は とうじ まだ さんさい だった			
\\	自宅	
\\	じたく			
\\	自宅に電話を下さい。	
\\	自宅 
\\	じたく に でんわ を ください			
\\	空気	
\\	くうき			
\\	ここは空気がきれいです。	
\\	空気 
\\	ここ は くうき が きれい です			
\\	設ける	
\\	もうける			
\\	授業の終わりに復習の時間を設けた。	
\\	設ける 
\\	じゅぎょう の おわり に ふくしゅう の じかん を もうけた			
\\	歴史	
\\	れきし			
\\	私は歴史に興味があります。	
\\	歴史 
\\	わたし は れきし に きょうみ が あります			
\\	各国	
\\	かっこく			
\\	各国の代表がニューヨークに集まった。	
\\	各国 
\\	かっこく の だいひょう が にゅーよーく に あつまった			
\\	特に	
\\	とくに			
\\	特に質問はありません。	
\\	特に 
\\	とくに しつもん は ありません			
\\	実験	
\\	じっけん			
\\	ついに実験が成功した。	
\\	成功 
\\	実験 
\\	ついに じっけん が せいこう した			
\\	科学の授業で実験をした。	
\\	実験 
\\	かがく の じゅぎょう で じっけん を した			
\\	訪れる	
\\	おとずれる			
\\	年間5万人がここを訪れます。	
\\	年間 
\\	訪れる 
\\	ねんかん ごまんにん が ここ を おとずれます			
\\	私は夏に京都を訪れました。	
\\	訪れる 
\\	わたし は なつ に きょうと を おとずれました			
\\	建物	
\\	たてもの			
\\	これは日本一古い建物です。	
\\	建物 
\\	これ は にほんいち ふるい たてもの です			
\\	ビル	
\\	私の会社はあのビルの8階です。	
\\	ビル 
\\	わたし の かいしゃ は あの びる の はちかい です			
\\	彼ら	
\\	かれら			
\\	彼らはバスケットの選手です。	
\\	彼ら 
\\	かれら は ばすけっと の せんしゅ です			
\\	少年	
\\	しょうねん			
\\	少年たちがサッカーをしている。	
\\	少年 
\\	しょうねんたち が さっかー を して いる			
\\	諸国	
\\	しょこく			
\\	彼はヨーロッパ諸国を旅行した。	
\\	諸国 
\\	かれ は よーろっぱ しょこく を りょこう した			
\\	もちろん	
\\	もちろん一緒に行きます。	
\\	もちろん 
\\	もちろん いっしょ に いきます			
\\	事務所	
\\	じむしょ			
\\	後で事務所に来てください。	
\\	事務所 
\\	あとで じむしょ に きて ください			
\\	責任	
\\	せきにん			
\\	彼は失敗の責任を取って、会社を辞めた。	
\\	責任 
\\	かれ は しっぱい の せきにん を とって かいしゃ を やめた			
\\	かつて	
\\	かつて			
\\	かつて私が学生だった頃のことです。	
\\	かつて 
\\	かつて わたし が がくせい だった ころ の こと です			
\\	限る	
\\	かぎる			
\\	このサービスは週末に限ります	
\\	限る 
\\	この さーびす は しゅうまつ に かぎります			
\\	海外	
\\	かいがい			
\\	彼は海外での生活が長いです。	
\\	海外 
\\	かれ は かいがい で の せいかつ が ながい です			
\\	条件	
\\	じょうけん			
\\	この条件では厳し過ぎます。	
\\	条件 
\\	この じょうけん で は きびし すぎます			
\\	家庭	
\\	かてい			
\\	彼は家庭を大切にしている。	
\\	家庭 
\\	かれ は かてい を たいせつ に して いる			
\\	教授	
\\	きょうじゅ			
\\	彼は化学の教授です。	
\\	教授 
\\	かれ は かがく の きょうじゅ です			
\\	人口	
\\	じんこう			
\\	その国の人口はどのくらいですか。	
\\	人口 
\\	その くに の じんこう は どの くらい です か			
\\	海	
\\	うみ			
\\	海は広くて大きい。	
\\	海 
\\	うみ は ひろく て おおきい			
\\	基準	
\\	きじゅん			
\\	判断の基準が示された。	
\\	基準 
\\	はんだん の きじゅん が しめされた			
\\	攻撃	
\\	こうげき			
\\	2003年にアメリカはイラクを攻撃した。	
\\	攻撃 
\\	にせんさんねん に あめりか は いらく を こうげき した			
\\	時代	
\\	じだい			
\\	今は便利さとスピードの時代だ。	
\\	時代 
\\	いま は べんりさ と すぴーど の じだい だ			
\\	時期	
\\	じき			
\\	今はあなたにとって大事な時期です。	
\\	時期 
\\	いま は あなた に とって だいじ な じき です			
\\	製造	
\\	せいぞう			
\\	この工場ではエアコンを製造しています。	
\\	製造 
\\	この こうじょう で は えあこん を せいぞう して います			
\\	大型	
\\	おおがた			
\\	大型のテレビを買った。	
\\	大型 
\\	おおがた の てれび を かった			
\\	周辺	
\\	しゅうへん			
\\	この周辺には大学が多い。	
\\	周辺 
\\	この しゅうへん に は だいがく が おおい			
\\	異なる	
\\	ことなる			
\\	彼と私はいつも意見が異なる。	
\\	異なる 
\\	かれ と わたし は いつも いけん が ことなる			
\\	逮捕	
\\	たいほ			
\\	逃げていた犯人が逮捕されました。	
\\	逮捕 
\\	にげて いた はんにん が たいほ されました			
\\	関連	
\\	かんれん			
\\	関連のセクションへ連絡した。	
\\	関連 
\\	かんれん の せくしょん へ れんらく した			
\\	貿易	
\\	ぼうえき			
\\	父は貿易の仕事をしています。	
\\	貿易 
\\	ちち は ぼうえき の しごと を して います			
\\	社員	
\\	しゃいん			
\\	彼は優秀な社員です。	
\\	社員 
\\	かれ は ゆうしゅう な しゃいん です			
\\	結合	
\\	けつごう			
\\	ファイルを結合して1つにしました。	
\\	結合 
\\	ふぁいる を けつごう して ひとつ に しました			
\\	自然	
\\	しぜん			
\\	スポーツ	
\\	あなたは何かスポーツをしていますか。	
\\	スポーツ 
\\	あなた は なにか すぽーつ を して います か			
\\	患者	
\\	かんじゃ			
\\	患者は眠っています。	
\\	患者 
\\	かんじゃ は ねむって います			
\\	なぜ	
\\	なぜ来なかったの。	
\\	なぜ 
\\	なぜ こなかった の			
\\	映画	
\\	えいが			
\\	彼はよく映画を見ます。	
\\	映画 
\\	かれ は よく えいが を みます			
\\	並ぶ	
\\	ならぶ			
\\	ここに並んでください。	
\\	並ぶ 
\\	ここ に ならんで ください			
\\	生徒	
\\	せいと			
\\	このクラスの生徒は30人です。	
\\	生徒 
\\	この くらす の せいと は さんじゅうにん です			
\\	そのまま	
\\	そのままお待ちください。	
\\	そのまま 
\\	そのまま おまち ください			
\\	今度	
\\	こんど			
\\	今度はどこに行きたいですか。	
\\	今度 
\\	こんど は どこ に いきたい です か			
\\	分ける	
\\	わける			
\\	お菓子をみんなで分けました。	
\\	分ける 
\\	おかし を みんな で わけました			
\\	運ぶ	
\\	はこぶ			
\\	いすを二階に運んでください。	
\\	運ぶ 
\\	いす を にかい に はこんで ください			
\\	複雑	
\\	ふくざつ			
\\	このプログラムはとても複雑です。	
\\	複雑 
\\	この ぷろぐらむ は とても ふくざつ です			
\\	回る	
\\	まわる			
\\	高校	
\\	こうこう			
\\	妹は高校に通っています。	
\\	高校 
\\	いもうと は こうこう に かよって います			
\\	工事	
\\	こうじ			
\\	工事の音がうるさい。	
\\	工事 
\\	こうじ の おと が うるさい			
\\	毎日	
\\	まいにち			
\\	私たちは毎日散歩をします。	
\\	毎日 
\\	わたしたち は まいにち さんぽ を します			
\\	帰国	
\\	きこく			
\\	彼は帰国しました。	
\\	帰国 
\\	かれ は きこく しました			
\\	雨	
\\	あめ			
\\	雨が降っています。	
\\	雨 
\\	あめ が ふって います			
\\	経つ	
\\	たつ			
\\	あれから14年が経ちました。	
\\	経つ 
\\	あれ から じゅうよねん が たちました			
\\	花	
\\	はな			
\\	きれいな花が咲きました。	
\\	花 
\\	きれい な はな が さきました			
\\	出席	
\\	しゅっせき			
\\	午後は会議に出席します。	
\\	出席 
\\	ごご は かいぎ に しゅっせき します			
\\	外国人	
\\	がいこくじん			
\\	日本に住む外国人が増えています。	
\\	外国人 
\\	にほん に すむ がいこくじん が ふえて います			
\\	道路	
\\	どうろ			
\\	この道路は3年前にできました。	
\\	道路 
\\	この どうろ は さんねんまえ に できました			
\\	地球	
\\	ちきゅう			
\\	月は地球のまわりを回っています。	
\\	回る 
\\	地球 
\\	つき は ちきゅう の まわり を まわって います			
\\	地球は丸い。	
\\	地球 
\\	ちきゅう は まるい			
\\	英語	
\\	えいご			
\\	あなたは英語が話せますか。	
\\	英語 
\\	あなた は えいご が はなせます か			
\\	危険	
\\	きけん			
\\	その地域は今、危険だ。	
\\	危険 
\\	その ちいき は いま きけん だ			
\\	普通	
\\	ふつう			
\\	彼女は普通の女の子だ。	
\\	普通 
\\	かのじょ は ふつう の おんな の こ だ			
\\	説明	
\\	せつめい			
\\	この単語の意味を説明してください。	
\\	説明 
\\	この たんご の いみ を せつめい して ください			
\\	半分	
\\	はんぶん			
\\	お菓子を友達に半分あげた。	
\\	半分 
\\	おかし を ともだち に はんぶん あげた			
\\	注意	
\\	ちゅうい			
\\	車に注意してください。	
\\	注意 
\\	くるま に ちゅうい して ください			
\\	安全	
\\	あんぜん			
\\	市民の安全は大切だ。	
\\	市民 
\\	安全 
\\	しみん の あんぜん は たいせつ だ			
\\	安全が第一です。	
\\	安全 
\\	あんぜん が だいいち です			
\\	手紙	
\\	てがみ			
\\	友人から手紙をもらいました。	
\\	手紙 
\\	ゆうじん から てがみ を もらいました			
\\	努力	
\\	どりょく			
\\	もっと努力しよう。	
\\	努力 
\\	もっと どりょく しよう			
\\	手術	
\\	しゅじゅつ			
\\	父は胸の手術をした。	
\\	手術 
\\	ちち は むね の しゅじゅつ を した			
\\	準備	
\\	じゅんび			
\\	明日の会議の準備をした。	
\\	準備 
\\	あす の かいぎ の じゅんび を した			
\\	守る	
\\	まもる			
\\	みんなで自然を守りましょう。	
\\	自然 
\\	守る 
\\	みんな で しぜん を まもりましょう			
\\	彼は約束を守る人です。	
\\	守る 
\\	かれ は やくそく を まもる ひと です			
\\	驚く	
\\	おどろく			
\\	彼は血を見て驚いた。	
\\	驚く 
\\	かれ は ち を みて おどろいた			
\\	地震	
\\	じしん			
\\	日本は地震が多いです。	
\\	地震 
\\	にほん は じしん が おおい です			
\\	話し合う	
\\	はなしあう			
\\	私たちはよく話し合いました。	
\\	話し合う 
\\	わたしたち は よく はなしあいました			
\\	木	
\\	き			
\\	台風で木が倒れた。	
\\	木 
\\	たいふう で き が たおれた			
\\	許す	
\\	ゆるす			
\\	私は彼を許しました。	
\\	許す 
\\	わたし は かれ を ゆるしました			
\\	雪	
\\	ゆき			
\\	クリスマスに雪が降りました。	
\\	雪 
\\	くりすます に ゆき が ふりました			
\\	動物	
\\	どうぶつ			
\\	私は動物が大好きです。	
\\	動物 
\\	わたし は どうぶつ が だいすき です			
\\	合う	
\\	あう			
\\	この靴は私の足に合っている。	
\\	合う 
\\	この くつ は わたし の あし に あって いる			
\\	平和	
\\	へいわ			
\\	この国は平和です。	
\\	平和 
\\	この くに は へいわ です			
\\	非常に	
\\	ひじょうに			
\\	これは非常に重要です。	
\\	非常に 
\\	これ は ひじょう に じゅうよう です			
\\	遅れる	
\\	おくれる			
\\	今朝彼女は学校に遅れました。	
\\	遅れる 
\\	けさ かのじょ は がっこう に おくれました			
\\	もし	
\\	もし			
\\	もし雨が降ったら、行きません。	
\\	もし 
\\	もし あめ が ふったら いきません			
\\	卒業	
\\	そつぎょう			
\\	私は去年、大学を卒業した。	
\\	卒業 
\\	わたし は きょねん だいがく を そつぎょう した			
\\	移る	
\\	うつる			
\\	彼女は新しい会社に移った。	
\\	移る 
\\	かのじょ は あたらしい かいしゃ に うつった			
\end{CJK}
\end{document}