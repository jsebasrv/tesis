\documentclass[8pt]{extreport} 
\usepackage{hyperref}
\usepackage{CJKutf8}
\begin{document}
\begin{CJK}{UTF8}{min}
\\	乾杯	
\\	乾杯[かんぱい]
\\	戴きます	
\\	戴[いただ]きます
\\	美味しい	
\\	美味[おい]しい
\\	まあまあ	
\\	まあまあ
\\	ご馳走様でした	
\\	ご 馳走様[ちそうさま]でした
\\	納豆	
\\	納豆[なっとう]
\\	始めまして	
\\	始[はじ]めまして
\\	どうぞ宜しくお願いします	
\\	どうぞ 宜[よろ]しくお 願[ねが]いします
\\	私	
\\	私[わたし]
\\	元気	
\\	元気[げんき]
\\	凄い	
\\	凄[すご]い
\\	絶好調	
\\	絶好調[ぜっこうちょう]
\\	ばっちり	
\\	ばっちり
\\	下さい /お願いします	
\\	下[くだ]さい/お 願[ねが]いします
\\	どうも	
\\	どうも
\\	はい	
\\	はい
\\	いいえ	
\\	いいえ
\\	済みません	
\\	済[す]みません
\\	アメリカ	
\\	アメリカ
\\	カナダ	
\\	カナダ
\\	イギリス	
\\	イギリス
\\	オーストラリア	
\\	オーストラリア
\\	インド	
\\	インド
\\	日本	
\\	日本[にっぽん]
\\	~人[じん]
\\	左様なら	
\\	左様[さよう]なら
\\	またね	
\\	またね
\\	じゃあね	
\\	じゃあね
\\	名前	
\\	名前[なまえ]
\\	国籍	
\\	国籍[こくせき]
\\	こちらこそ	
\\	こちらこそ
\\	ゆっくり	
\\	ゆっくり
\\	あなた	
\\	あなた
\\	彼	
\\	彼[かれ]
\\	彼女	
\\	彼女[かのじょ]
\\	あなた達	
\\	あなた達[たち]
\\	彼ら	
\\	彼[かれ]ら
\\	彼女ら	
\\	彼女[かのじょ]ら
\\	です	
\\	です
\\	私達	
\\	私[わたし]達[たち]
\\	ノルウェー	
\\	ノルウェー
\\	も
\\	楽しい	
\\	楽[たの]しい
\\	忙しい	
\\	忙[いそが]しい
\\	寒い	
\\	寒[さむ]い
\\	誰	
\\	誰[だれ]
\\	学生	
\\	学生[がくせい]
\\	どこ	
\\	どこ
\\	~さん	
\\	~さん
\\	サンタさん	
\\	サンタさん
\\	メリークリスマス	
\\	メリークリスマス
\\	天皇	
\\	天皇[てんのう]
\\	誕生日	
\\	誕生[たんじょう] 日[び]
\\	天皇誕生日	
\\	天皇誕生日[てんのうたんじょうび]
\\	サンタクロース	
\\	サンタクロース
\\	大丈夫	
\\	大丈夫[だいじょうぶ]
\\	だめ	
\\	だめ
\\	蛸	
\\	蛸[たこ]
\\	鮪	
\\	鮪[まぐろ]
\\	海栗	
\\	海栗[うに]
\\	お勧めお願いします	
\\	お 勧[すす]めお 願[ねが]いします
\\	いらっしゃいませ	
\\	いらっしゃいませ
\\	いらっしゃい	
\\	いらっしゃい
\\	お腹いっぱい	
\\	お 腹[なか]いっぱい
\\	お会計お願いします	
\\	お会計[かいけい]お願[ねが]いします
\\	お早う	
\\	お早[はよ]う
\\	今晩は	
\\	今晩[こんばん]は
\\	こんにちは	
\\	こんにちは
\\	でございます	
\\	です	でございます
\\	また、明日ね	
\\	また、 明日[あした]ね
\\	素敵	
\\	素敵[すてき]
\\	ありがとう	
\\	ありがとう
\\	とても	
\\	とても
\\	美しい	
\\	美[うつく]しい
\\	煩い	
\\	煩[うるさ]い
\\	明けましておめでとうございます	
\\	明[あ]けましておめでとうございます
\\	今年も宜しくお願いします。	
\\	今年[ことし]も宜[よろ]しくお願[ねが]いします。
\\	あけおめ。ことよろ。	
\\	あけましておめでとうございます。 ことしもよろしくおねがいします。	あけおめ。ことよろ。
\\	おめでとう	
\\	おめでとう
\\	今年	
\\	今年[ことし]
\\	紅白歌合戦	
\\	紅白[こうはく]歌合戦[うたがっせん]
\\	よいお年を	
\\	よいお年[とし]を
\\	お年玉	
\\	お年玉[としだま]
\\	年越しそば	
\\	年越[としこ]しそば
\\	歌	
\\	歌[うた]
\\	年越し	
\\	年越[としこ]し
\\	イタリア	
\\	イタリア
\\	モスバーガー	
\\	モスバーガー
\\	料理	
\\	料理[りょうり]
\\	日本料理	
\\	日本[にっぽん]料理[りょうり]
\\	タイ	
\\	タイ
\\	フランス	
\\	フランス
\\	何	
\\	何[なに]
\\	好き	
\\	好[す]き
\\	豆腐	
\\	豆腐[とうふ]
\\	大好き	
\\	大好[だいす]き
\\	物凄い	
\\	物凄[ものすご]い
\\	よ
\\	これ	
\\	これ
\\	それ	
\\	それ
\\	あれ	
\\	あれ
\\	ご出身	
\\	ご出身[しゅっしん]
\\	歌舞伎町	
\\	歌舞伎町[かぶきちょう]
\\	何処から来ましたか。	
\\	何処[どこ]から 来[き]ましたか。
\\	の
\\	フィッシュアンドチップス	
\\	フィッシュアンドチップス
\\	相撲	
\\	相撲[すもう]
\\	カレー	
\\	カレー
\\	野球	
\\	野球[やきゅう]
\\	ハンバーガー	
\\	ハンバーガー
\\	サッカー	
\\	サッカー
\\	アイススケート	
\\	アイススケート
\\	もう一度お願いします。	
\\	もう一度[いちど]お願[ねが]いします。
\\	お手洗い	
\\	お手洗[てあら]い
\\	トイレ	
\\	トイレ
\\	駅弁	
\\	駅弁[えきべん]
\\	新幹線	
\\	新幹線[しんかんせん]
\\	何度ですか。	
\\	何[なん]度[ど]ですか。
\\	でしょう	
\\	でしょう
\\	何て~	
\\	何[なに]て~
\\	風邪気味	
\\	風邪[かぜ]気味[ぎみ]
\\	度	
\\	度[ど]
\\	今日	
\\	今日[きょう]
\\	三度でしょう	
\\	三度[さんど]でしょう
\\	元気一杯	
\\	元気[げんき]一杯[はい]
\\	あまり	
\\	あまり
\\	鞄	
\\	鞄[かばん]
\\	新しい	
\\	新[あたら]しい
\\	スーパー	
\\	スーパー
\\	もしもし	
\\	もしもし
\\	~ですけど。 
\\	はい、~です。	
\\	~ですけど。 
\\	はい、~です。
\\	それだけです	
\\	それだけです
\\	だけ	
\\	だけ
\\	今	
\\	今[いま]
\\	携帯電話 (形態)	
\\	携帯[けいたい]電話[でんわ] (形態[けいたい])
\\	電話	
\\	電話[でんわ]
\\	タクシー	
\\	タクシー
\\	タクシー乗り場	
\\	タクシー 乗り場[のりば]
\\	~まで	
\\	~まで
\\	ここでいい	
\\	ここでいい
\\	領収書	
\\	領収[りょうしゅう]書[しょ]
\\	調子はどうですか。	
\\	げんきですか)	調子[ちょうし]はどうですか。
\\	お蔭様で、~	
\\	お蔭様[かげさま]で、~
\\	悪い	
\\	悪[わる]い
\\	いい	
\\	いい
\\	目茶目茶	
\\	目茶目茶[めちゃめちゃ]
\\	英語で大丈夫ですか。	
\\	英語[えいご]で 大丈夫[だいじょうぶ]ですか。
\\	(地名)行きですか。	
\\	(地名[ちめい]) 行[ゆ]きですか。
\\	違います	
\\	違[ちが]います
\\	朝飯前	
\\	朝飯前[あさめしまえ]
\\	前	
\\	前[まえ]
\\	父	
\\	父[ちち]
\\	母	
\\	母[はは]
\\	妹	
\\	妹[いもうと]
\\	弟	
\\	弟[おとうと]
\\	兄	
\\	兄[あに]
\\	姉	
\\	姉[あね]
\\	多い	
\\	多[おお]い
\\	一人	
\\	一人[ひとり]
\\	二人	
\\	二人[ふたり]
\\	ご家族	
\\	ご 家族[かぞく]
\\	ご家族は何人ですか。	
\\	ご 家族[かぞく]は 何[なん] 人[にん]ですか。
\\	三人	
\\	三[さん] 人[にん]
\\	四人	
\\	四[よ] 人[にん]
\\	五人	
\\	五[ご] 人[にん]
\\	六人	
\\	六[ろく] 人[にん]
\\	七人	
\\	七[しち] 人[にん]
\\	八人	
\\	八[はち] 人[にん]
\\	九人	
\\	九[きゅう] 人[にん]
\\	十人	
\\	十[じゅう] 人[にん]
\\	何人	
\\	何[なん] 人[にん]
\\	一石二鳥	
\\	一石二鳥[いっせきにちょう]
\\	易しい	
\\	易[やさ]しい
\\	可愛い	
\\	可愛[かわい]い
\\	格好いい	
\\	格好[かっこう]いい
\\	面白い	
\\	面白[おもしろ]い
\\	面白くて可愛い	
\\	面白[おもしろ]くて 可愛[かわい]い
\\	い-
\\	いい.	い 
\\	くて いい 
\\	よくて
\\	家族	
\\	家族[かぞく]
\\	写真	
\\	写真[しゃしん]
\\	お父さん	
\\	お 父[とう]さん
\\	お母さん	
\\	お 母[かあ]さん
\\	お兄さん	
\\	お 兄[にい]さん
\\	お姉さん	
\\	お 姉[ねえ]さん
\\	弟さん	
\\	弟[おとうと]さん
\\	妹さん	
\\	妹[いもうと]さん
\\	~さい いっさい 
\\	はたち 
\\	百	
\\	百[ひゃく]
\\	本当	
\\	本当[ほんとう]
\\	相変わらず	
\\	相変[あいか]わらず
\\	何歳	
\\	何[なん] 歳[さい]
\\	友達	
\\	友達[ともだち]
\\	ここ	
\\	ここ
\\	そこ	
\\	そこ
\\	あそこ	
\\	あそこ
\\	家 
\\	家[いえ]
\\	家 
\\	家[うち]
\\	近い	
\\	近[ちか]い
\\	隣	
\\	隣[となり]
\\	可也	
\\	可也[かなり]
\\	遠い	
\\	遠[とお]い
\\	可也遠い	
\\	可也[かなり] 遠[とお]い
\\	成程	
\\	成[なる] 程[ほど]
\\	です 
\\	でしょう
\\	ベルト	
\\	ベルト
\\	ブランド	
\\	ブランド
\\	防止	
\\	防止[ぼうし]
\\	この~	
\\	この~
\\	その~	
\\	その~
\\	あの~	
\\	あの~
\\	どの~	
\\	どの~
\\	お昼	
\\	お 昼[ひる]
\\	店	
\\	店[みせ]
\\	けっこう	
\\	けっこう
\\	豚カツ	
\\	豚カツ[とんかつ]
\\	オッケー	
\\	オッケー
\\	問題	
\\	問題[もんだい]
\\	問題ない	
\\	問題[もんだい]ない
\\	お昼の時間	
\\	お 昼[ひる]の 時間[じかん]
\\	(ここ)でいい	
\\	(ここ)でいい
\\	結構おいしい	
\\	結構[けっこう]おいしい
\\	刺身でいいですね。	
\\	刺身[さしみ]でいいですね。
\\	この店はおいしいです。	
\\	この 店[みせ]はおいしいです。
\\	デパ地下	
\\	デパ 地下[ちか]
\\	焼き鳥	
\\	焼き鳥[やきとり]
\\	ねぎま	
\\	ねぎま
\\	たれ	
\\	やきとり 
\\	たれ
\\	塩	
\\	塩[しお]
\\	二百	
\\	二[に] 百[ひゃく]
\\	三百	
\\	三[さん] 百[びゃく]
\\	四百	
\\	四[よん] 百[ひゃく]
\\	五百	
\\	五[ご] 百[ひゃく]
\\	六百	
\\	六[ろっ] 百[ぴゃく]
\\	七百	
\\	七[なな] 百[ひゃく]
\\	八百	
\\	八[はっ] 百[ぴゃく]
\\	九百	
\\	九[きゅう] 百[ひゃく]
\\	千	
\\	千[せん]
\\	一番	
\\	一番[いちばん]
\\	一番悪い	
\\	一番[いちばん] 悪[わる]い
\\	一本ずつ	
\\	一本[いっぽん]ずつ
\\	焼き鳥はいかがですか。	
\\	焼き鳥[やきとり]はいかがですか。
\\	すみません、これは幾らですか。	
\\	すみません、これは 幾[いく]らですか。
\\	円	
\\	円[えん]
\\	それは一本百円です。	
\\	それは 一本[いっぽん] 百[ひゃく] 円[えん]です。
\\	はい。六百円でございます。	
\\	はい。 六百[ろっぴゃく] 円[えん]でございます。
\\	本[ほん]
\\	一本	
\\	一本[いっぽん]
\\	二本	
\\	二本[にほん]
\\	三本	
\\	三本[さんぼん]
\\	四本	
\\	四本[よんほん]
\\	五本	
\\	五本[ごほん]
\\	六本	
\\	六本[ろっぽん]
\\	七本	
\\	七本[ななほん]
\\	八本	
\\	八本[はっぽん]
\\	九本	
\\	九本[きゅうほん]
\\	十本	
\\	十本[じゅっぽん]
\\	何本	
\\	何[なん] 本[ほん]
\\	か
\\	と
\\	言う	
\\	言[い]う
\\	は
\\	で
\\	読む	
\\	読[よ]む
\\	これは英語で何と言いますか。	
\\	これは 英語[えいご]で 何[なん]と 言[い]いますか。
\\	これは日本語で何と言いますか。	
\\	これは 日本語[にほんご]で 何[なん]と 言[い]いますか。
\\	どういたしまして	
\\	どういたしまして
\\	これはたこわさびです。	
\\	これはたこわさびです。
\\	予約	
\\	予約[よやく]
\\	貸切	
\\	貸切[かしきり]
\\	空車	
\\	空車[くうしゃ]
\\	支払い	
\\	支払[しはら]い
\\	どこまでですか。	
\\	どこまでですか。
\end{CJK}
\end{document}