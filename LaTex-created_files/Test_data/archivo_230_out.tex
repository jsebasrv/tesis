\documentclass[8pt]{extreport} 
\usepackage{hyperref}
\usepackage{CJKutf8}
\begin{document}
\begin{CJK}{UTF8}{min}
\\	趣味の修養
\\	會津八一
\\	【テキスト中に現れる記号について】
\\	入力者注 主に外字の説明や、傍点の位置の指定
\\	(数字は、
\\	0213の面区点番号または
\\	、底本のページと行数)
\\	(例)※[#二の字点、1-2-22]
\\	二倍の踊り字(「く」を縦に長くしたような形の繰り返し記号)
\\	(例)何處までも/\
\\	何處までも/\芋畑や雜木林ばかりで退屈な汽車の窓に、小ぢんまりとした木立が見えて、それが近づくにつれて庭には草花が綺麗に咲かせてあつて、その中に白い鷄が遊んで居る、家の造りも面白い、こんな時に、飛ぶやうに通り過ぎて行く旅人の目にも、先づ床しいものは其家の主人である。また裏長屋の軒竝を歩いて居るうちに、不圖ある家の窓から床の間の一軸、それが名も無い畫家の作であるかも知れぬ、その前に活けてある花瓶が市價の乏しいものであつても、無暗に其家の主人を懷しがらせることがある。吾々が人を懷かしく思ふやうに人がまた吾々を懷かしく思ふこともあるかもしれぬ。私はこれが面白いことだと思ふ。しかし世の中には、誰に見せても少しも床しくも懷しくも思はれぬ人もあり、また誰を見ても床しくも感じない人もある。
\\	一體或る種類の人々が吾々の目に床しく見えるのは第一に其人にそれだけの趣味が備つてゐるからだ。しかしそれにしても吾々自身に、人を床しく思ふだけの趣味が無ければならぬ。磁石でなければ鐵片を引きつけない。鐵片でなければ磁石に吸ひつかない。たゞの石ころのやうな人間にはなりたくないものである。
\\	吾々が世の中を行くのに、必ずしも名利を一生の目的としなくとも、路は名利の中をうねる。その間に起るいろ/\の問題にぶつかつてそれを切り拔けて進むだけの覺悟が無くてはならぬ。人間がたゞ蒸氣汽罐のやうに強健で砲彈のやうに勇氣があつても、それは羨むべきではない。道徳も、藝術も、宗教も戀愛も此一面が備つてこそ生れ出るのだ。しかしまた、人間として此一面を備へただけでは、所謂文弱に傾いて仕舞ふ。自分の枝に咲いた美しい大きい花を支へるだけの力がなくて、泥に曳きづる蔓草のやうな生活も決して羨ましいものではない。全體として完全な人格者には、この兩面ともに大切である。青年の頃は修養の時代だ、趣味の修養を忘れてはならぬ。
\\	日本人を全體として見ると、昔から外國人からは懷かしく思はれて來た。またこちらからも外國人を懷かしがつて來た。決して殺風景な國民ではない。君子の國、美術の國、愛らしい國と、いろ/\に外國人から呼びなされて居ると同時に、道徳も美術も文學も宗教も、殆ど全的の影響を外國から受けるのが日本人の習はしであつた。つまり趣味的の修養にかけては、世界の一方で稀な發達を遂げて居たといつてもよからう。
\\	しかし今日は新らしい文明の利器や設備が急速にどし/\輸入せられ、發明せられ、社會の新らしい組織、制度さへ要望せられて來る世の中である。先づ以て滔々たる物質主義の弊を救ふ爲めに趣味の修養が大切である。ところが時勢の變轉が急速な爲めに、老人と中年、中年と青年、殆ど世代の差とともに甚だしい趣味の相違を來して仕舞つたかの風があつて、茶の湯、生花に固執する老人とダンス、洋畫に狂奔する青年の間には隔絶した距離が出來て居る。それは止むを得ぬことでもある。しかしその間隔の甚だしくなるにつれて相互の理解がなくなつて、遂には敵味方とわかれて仕舞ふのは嗟かはしき事と言はねばならぬ。之はお互に其趣味の長所を知りあつて融和の出來ぬものであらうか。これが出來たら老人も青年も趣味の境地がずつと擴大せられて、したがつて共に豐富な趣味の所有者となるであらう。そして其趣味は一層こまやかに、一層生々としたものになるであらう。今は隨分烈しい世の中である。その世の中で日本人ほど複雜な境涯に在る國民はない。その複雜な生活の内面には複雜な趣が無ければならぬ。今日の或る種類の老人どもが考へるやうに單純な支那趣味位で今の活きた日本人を滿足させる事は出來ない。しかし又、考の淺い青年等のやうに單純な佛蘭西趣味や米國趣味で、活きたまゝの日本人を滿足させることは同樣に不可能である。套習も直譯もいけない。それは共に生きた人間の生きた趣味には滿足を與へるものではない。紅茶も珈琲も知らなかつた時代の茶の湯の形式は萬古不滅のものではないと同時に、佛蘭西人露西亞人の鑑賞眼にも普遍的の價値は無い。今の日本には現代といふことを忘れた老人と、日本といふ國を忘れた青年とが別々なことを考へて居るのだ。
\\	私は昨年或る富豪の邸へ招かれて、支那の歴代の名畫を見せられたことがある。唐宋から清朝までの凡そ名の高い人々の傑作が順々に目の前に展開せらるゝのを見て、私の心は一種の陶醉を感じて居た。すると其席上に來合せたTといふ名の一人の老人が、突然私をかへりみて、『失禮ながら此ういふ畫を御覽になるには御年が未だ御若過ぎるではありませんか』と云ふのであつた。それに答へて私は大いに論じた。
\\	『年が若くて支那の名畫を鑑賞が出來ぬといふことは不思議な御説である。先刻、明の文徴明の畫幅がくり展げられた時に、あなたは聲を放つて感歎されたではありませんか。所があの畫は彼が二十四歳の時の作品であります。若し繪畫の鑑賞に年齡が關係あるものとすれば、私こそあなたよりもあの畫の味が解る筈ではありませんか。一體幽玄な東洋藝術の趣味は、六十にも七十にもなつて枯木寒巖の如き老人にならなければ味ひ得ぬやうに考へるのは今日の老人達の通弊ではありませんか。私などは其反對に、あの畫のやうに強い色と軟かい線で自由に描かれた葵の圖の味は老人には解るまいかと思つて居ります。古來、支那でも日本でも、私ほどの年齡にもならぬうちに立派な詩文や書畫を作つた人は數ふるにも勝へないほどであります。それを一概に老人の專有物のやうになさるにも及びますまい。』
\\	私の返事は大體かうであつた。實際、この自ら獨り高しとして他人の窺ひ知ることを許さぬ態度、ことに青年の趣味を全然否定し無視しやうとする態度、時代は常に生新なる趣味を要求して其趣味の鑑賞者は老人よりも寧ろ常に青年であることを忘るゝが如き態度、かういふ態度には老人が殊に陷りやすい。そして青年の趣味から故らに遠かつて行く。
\\	青年の方ではてんで、例へば油畫や西洋音樂などは、迚も老人に解るべきものではないやうに思つて、展覽會や演奏會に、白髮の老人を見出すことをさも不思議な事に考へるものが多いが、一體今日の青年の口にするゲーテの詩とか、レオナード・ダ・ヴィンチやミケランヂェロの畫とかいふものは、みな一代の大天才が幾十年の深酷な生涯の閲歴からの收獲物であつて、これを青年の心でなければ解らぬと考へるのは如何いふものであらう。今では西洋の文藝といへば青年の獨占のやうに考へられて居る。これは不思議なことである。私はゲーテやレオナードの或る種類の作物は、とてもたゞの青年には解りさうも無いと思ふ。そして、それよりも彼等にもつと味ひよく解り易いものが東洋にあると思ふ。しかしそれを老人連が與へやうとせぬ、又青年も與へられやうとはせぬ。かうして老人と青年とが益※[#二の字点、1-2-22]遠かつて行く。
\\	新らしい洋畫を見て玉葱や、大根や、林檎や、まるで八百屋の店先の樣だと貶す老人連は、南畫などに好んで描かれる水仙や、蓮根や、佛手柑が、どれだけもそれ等と相違の無いことに氣がつかない。カンナといふ花が青年の油繪にはよく描かれる。老人連はこれをしつこい[#「しつこい」に傍点]とかあくどい[#「あくどい」に傍点]とかいつて嫌ふが、焉ぞ知らん、カンナの漢名は紅蕉といつて、それが南畫の中にでも現はれると至極上品のものゝやうに老人はいふ、さうなると青年は最早それを好まない。また竹は熱帶の植物であるからには、其美しさは勿論いやが上にも密生繁茂した其姿にあらねばならぬ。ところが老人達は二三竿の疎らな枝葉の上に宿る一片の涼味のほかに趣味を見出すまいとする。或は竹とともに南畫などに四君子として描かれる蘭の畫、それを見て多くの青年は何の興味をも起さない。それが熱帶植物として温室で培養されてゐる時にのみ青年は蘭を我がものゝ如く云ふ。ところが其の温室へ老人を連れ込めば、老人はしきりに其の強い香氣を苦にする。その癖、一輪の蘭が幽谷に咲けば、一國にその香を傳へるとか、萬國に其の香を送るとかいつて書物や文章の上でのみ珍重して居る。かういふ例は一々擧げることは出來ぬ。つまり老人のやうに青年も偏狹で、青年の如く老人も淺薄で、そして其の偏狹と淺薄との爲めに彼等の趣味がます/\貧弱空疎に流れ行くのである。皮相を貫いて眞底を見ることゝ、枝葉を離れて根幹に歸すること、これを今の世の老人にも青年にも望まねばならぬ。
\\	偉大なる國民は偉大なる個人の集合であらねばならぬ。その偉大なる個人はいふまでもなく圓滿な人格の所有者でなければならぬ。その圓滿なる人格には豐富な趣味を缺くことは許さない。老人も青年も、すべてもつと/\謙遜なれ、誠實なれ、活眼を開け、もつと眞劒なれ、もつと徹底的なれ、そして吾々の生活をして更に高尚ならしめよ。更に豐富ならしめよ、更に意味深からしめよ。しかし吾々は、前途の短い老人よりも、むしろ人生の修養時代にあるところの青年諸君の奮發を希望するのである。
\\	(これ早稻田中學講議録記者の爲めに語りたるところなり。今日たま/\其筆記を點檢しつゝある時、代議士S翁來訪、手にせる扇を予に示していふ、此扇はT老人の書して贈らるゝところ、たま/\僕の趣味に適せず、君乞ふ僕の爲めに更に一面を揮毫せんことをと。予筆下に一句を書して贈る。他日S翁去つてこれを老人に示さば、老人も自ら説あるべし、これ予が最も聞かんと欲するところなり。)[#地から2字上げ][#1段階小さな文字](大正十一年六月十八日)[#小さな文字終わり]
\\	底本:「會津八一全集 第七卷」中央公論社
\\	1982(昭和57)年4月25日初版發行
\\	初出:「興風」
\\	1922(大正11)年8月
\\	入力:フクポー
\\	校正:杉浦鳥見
\\	2018年10月24日作成
\\	青空文庫作成ファイル:
\\	このファイルは、インターネットの図書館、青空文庫
\\	で作られました。入力、校正、制作にあたったのは、ボランティアの皆さんです。
\end{CJK}
\end{document}