\documentclass[8pt]{extreport} 
\usepackage{hyperref}
\usepackage{CJKutf8}
\begin{document}
\begin{CJK}{UTF8}{min}
\\	毎日同じ物を食
\\	こと]は、面白くない。	毎日同じ物を食べるのは、面白くない。	まいにちおなじものをたべるのは、おもしろくない。	
\\	「の」 
\\	毎日[まいにち] 同[おな]じ 物[もの]を 食[た]べるのは、 面白[おもしろ]くない。		
\\	電気が点
\\	。	電気が点いた。	でんきがついた。	
\\	電気[でんき]が 点[つ]いた。		
\\	ボブは、魚が好きなんだ
\\	。/そうだね。	ボブは、魚が好きなんだよね。/そうだね。	ボブは、さかながすきなんだよね。/そうだね。	
\\	「よね」	ボブは、 魚[さかな]が 好[す]きなんだよね。/そうだね。		
\\	魚
\\	。	魚だ。	さかなだ。	
\\	「だ」	魚[さかな]だ。		
\\	猫は
\\	。	猫はいない。	ねこはいない。	
\\	猫[ねこ]はいない。		
\\	晩ご飯を食
\\	人は、映画で見
\\	銀行に行った。	晩ご飯を食べなかった人は、映画で見た銀行に行った。	ばんごはんをたべなかったひとは、えいがでみたぎんこうにいった。	
\\	晩[ばん]ご 飯[はん]を 食[た]べなかった 人[ひと]は、 映画[えいが]で 見[み]た 銀行[ぎんこう]に 行[い]った。		
\\	学生
\\	。	学生じゃない。	がくせいじゃない。	
\\	学生[がくせい]じゃない。		
\\	いい天気だ
\\	。/そう
\\	。	いい天気だね。/そうね。	いいてんきだね。/そうね。	
\\	「ね」 
\\	いい 天気[てんき]だね。/そうね。		
\\	高速道路
\\	走る。	高速道路を走る。	こうそくどうろをはしる。	
\\	「を」 
\\	高速[こうそく] 道路[どうろ]を 走[はし]る。		
\\	朝ご飯を食
\\	。/どうして?	朝ご飯を食べるんじゃなかった。/どうして?	あさごはんをたべるんじゃなかった。/どうして?	
\\	「の」 
\\	朝[あさ]ご 飯[はん]を 食[た]べるんじゃなかった。/どうして?		
\\	魚が好
\\	タイプ。	魚が好きなタイプ。	さかながすきなタイプ。	
\\	魚[さかな]が 好[す]きなタイプ。		
\\	静
\\	が、アリス
\\	部屋だ。	静かなのが、アリスの部屋だ。	しずかなのが、アリスのへやだ。	
\\	「の」 
\\	静[しず]かなのが、アリスの 部屋[へや]だ。		
\\	私は買
\\	。	私は買わない。	わたしはかわない。	
\\	私[わたし]は 買[か]わない。		
\\	今、授業がない
\\	/ううん、ある。	今、授業がないんじゃない?/ううん、ある。	いま、じゅぎょうがないんじゃない?/ううん、ある。	
\\	「の」 
\\	今[いま]、 授業[じゅぎょう]がないんじゃない?/ううん、ある。		
\\	勝ち
\\	向かう。	勝ちへ向かう。	かちへむかう。	
\\	「へ」 
\\	勝[か]ちへ 向[む]かう。		
\\	部屋を出
\\	。	部屋を出た。	へやをでた。	
\\	部屋[へや]を 出[で]た。		
\\	お金が
\\	。	お金がない。	おかねがない。	
\\	お 金[かね]がない。		
\\	本
\\	雑誌
\\	葉書を買った。	本と雑誌と葉書を買った。	ほんとざっしとはがきをかった。	
\\	「と」 
\\	本[ほん]と 雑誌[ざっし]と 葉書[はがき]を 買[か]った。		
\\	誰が窓を開
\\	誰が窓を開けた?	だれがまどをあけた?	
\\	誰[だれ]が 窓[まど]を 開[あ]けた?		
\\	私も遊
\\	。	私も遊んだ。	わたしもあそんだ。	
\\	私[わたし]も 遊[あそ]んだ。		
\\	今日はいい天気だ
\\	。/うん。でも、明日雨が降る
\\	。	今日はいい天気だね。/うん。でも、明日雨が降るよ。	きょうはいいてんきだね。/うん。でも、あしたあめがふるよ。	
\\	「よ」 
\\	今日[きょう]はいい 天気[てんき]だね。/うん。でも、 明日[あした] 雨[あめ]が 降[ふ]るよ。		
\\	今は忙しい
\\	。	今は忙しいの。	いまはいそがしいの。	
\\	「の」 
\\	今[いま]は 忙[いそが]しいの。		
\\	今日
\\	試験だ。/ジム
\\	/ジム
\\	明日。	今日は試験だ。/ジムは?/ジムは明日。	きょうはしけんだ。/ジムは?/ジムはあした。	
\\	「は」 
\\	今日[きょう]は 試験[しけん]だ。/ジムは?/ジムは 明日[あした]。		
\\	魚が好
\\	タイプ。	魚が好きだったタイプ。	さかながすきだったタイプ。	
\\	魚[さかな]が 好[す]きだったタイプ。		
\\	友達は、来年、日本
\\	行く。	友達は、来年、日本に行く。	ともだちは、らいねん、にほんにいく。	
\\	「に」 
\\	友達[ともだち]は、 来年[らいねん]、 日本[にっぽん]に 行[い]く。		
\\	街
\\	ぶらぶら歩く。	街をぶらぶら歩く。	まちをぶらぶらあるく。	
\\	「を」 
\\	街[まち]をぶらぶら 歩[ある]く。		
\\	毎日、日本語
\\	勉強する。	毎日、日本語を勉強する。	まいにち、にほんごをべんきょうする。	
\\	「を」 
\\	毎日[まいにち]、 日本語[にほんご]を 勉強[べんきょう]する。		
\\	ジュース
\\	飲んだ。	ジュースを飲んだ。	ジュースをのんだ。	
\\	「を」 
\\	ジュースを 飲[の]んだ。		
\\	アリスは学生?/うん、トム
\\	学生。	アリスは学生?/うん、トムも学生。	アリスはがくせい?/うん、トムもがくせい。	
\\	「も」 
\\	アリスは 学生[がくせい]?/うん、トムも 学生[がくせい]。		
\\	お金が
\\	。	お金がなかった。	おかねがなかった。	
\\	お 金[かね]がなかった。		
\\	その人が買
\\	/ううん、先生が買
\\	。	その人が買うんじゃなかったの?/ううん、先生が買うんだ。	そのひとがかうんじゃなかったの?/ううん、せんせいがかうんだ。	
\\	「の」 
\\	その 人[ひと]が 買[か]うんじゃなかったの?/ううん、 先生[せんせい]が 買[か]うんだ。		
\\	ボブの声は、
\\	大きい。	ボブの声は、結構大きい。	ボブのこえは、けっこうおおきい。	
\\	ボブの 声[こえ]は、 結構[けっこう] 大[おお]きい。		
\\	ボブは魚が好
\\	。	ボブは魚が好きじゃない。	ボブはさかながすきじゃない。	
\\	ボブは 魚[さかな]が 好[す]きじゃない。		
\\	高
\\	ビル。	高くなかったビル。	たかくなかったビル。	
\\	高[たか]くなかったビル。		
\\	何できた?/バスできた。
\\	何で?	何できた?/バスできた。
\\	何で?	なにできた?/バスできた。	
\\	「で」 
\\	「何」	何[なん]できた?/バスできた。 
\\	何[なに]で?		
\\	ボブは魚が好
\\	。	ボブは魚が好きだった。	ボブはさかながすきだった。	
\\	ボブは 魚[さかな]が 好[す]きだった。		
\\	お金が
\\	。	お金がある。	おかねがある。	
\\	お 金[かね]がある。		
\\	ジムが遊
\\	。	ジムが遊ばない。	ジムはあそばない。	
\\	ジムが 遊[あそ]ばない。		
\\	時間がない
\\	。/大丈夫だ
\\	。	時間がないよ。/大丈夫だよ。	じかんがないよ。/だいじょうぶだよ。	
\\	「よ」 
\\	時間[じかん]がないよ。/ 大丈夫[だいじょうぶ]だよ。		
\\	猫は
\\	。	猫はいなかった。	ねこはいなかった。	
\\	猫[ねこ]はいなかった。		
\\	ボブも
\\	。	ボブもしない。	ボブもしない。	
\\	ボブもしない。		
\\	友達は、来年
\\	日本
\\	行く。	友達は、来年に日本に行く。	ともだちは、らいねんににほんにいく。	
\\	「に」 
\\	友達[ともだち]は、 来年[らいねん]に 日本[にっぽん]に 行[い]く。		
\\	面白い映画だった
\\	全然面白くなかった。	面白い映画だったね。/え?全然面白くなかった。	おもしろいえいがだったね。/え?ぜんぜんおもしろくなかった。	
\\	「ね」 
\\	面白[おもしろ]い 映画[えいが]だったね。/え? 全然[ぜんぜん] 面白[おもしろ]くなかった。		
\\	友達
\\	話した。	友達と話した。	ともだちとはなした。	
\\	「と」 
\\	友達[ともだち]と 話[はな]した。		
\\	値段が高
\\	レストランはあまり好きじゃない。	値段が高いレストランはあまり好きじゃない。	ねだんがたかいレストランはあまりすきじゃない。	
\\	値段[ねだん]が 高[たか]いレストランはあまり 好[す]きじゃない。		
\\	毎日勉強
\\	こと
\\	は大変。	毎日勉強するのは大変。	まいにちべんきょうのはたいへん。	
\\	「の」 
\\	毎日[まいにち] 勉強[べんきょう]するのは 大変[たいへん]。		
\\	アリスは学生?/うん、
\\	トム
\\	学生じゃない。	アリスは学生?/うん、でもトムは学生じゃない。	アリスはがくせい?/うん、でもトムはがくせいじゃない。	
\\	「も」 
\\	アリスは 学生[がくせい]?/うん、でもトムは 学生[がくせい]じゃない。		
\\	今日は、走
\\	。	今日は、走った。	きょうは、はしった。	
\\	今日[きょう]は、 走[はし]った。		
\\	静かな高
\\	ビル。	静かな高いビル。	すずかなたかいビル。	
\\	静[しず]かな 高[たか]いビル。		
\\	。	食べる。	たべる。	
\\	食[た]べる。		
\\	誰
\\	学生?	誰が学生?	だれががくせい?	
\\	「が」 
\\	誰[だれ]が 学生[がくせい]?		
\\	魚
\\	。	魚じゃない。	さかなじゃない。	
\\	魚[さかな]じゃない。		
\\	バス
\\	帰る。	バスで帰る。	バスでかえる。	
\\	「で」 
\\	バスで 帰[かえ]る。		
\\	ジム
\\	。	ジムなのだ。	ジムなのだ。	
\\	「の」 
\\	ジムなのだ。		
\\	魚が好
\\	タイプは、肉が好
\\	。	魚が好きじゃないタイプは、肉が好きだ。	さかながすきじゃないタイプは、にくがすきだ。	
\\	魚[さかな]が 好[す]きじゃないタイプは、 肉[にく]が 好[す]きだ。		
\\	先生
\\	会った。	先生と会った。	せんせいとあった。	
\\	「と」 
\\	先生[せんせい]と 会[あ]った。		
\\	電気を消
\\	。	電気を消す。	でんきをけす。	
\\	電気[でんき]を 消[け]す。		
\\	靴
\\	シャツを買う。	靴やシャツを買う。	くつやシャツをかう。	
\\	「や」 
\\	「とか」 
\\	靴[くつ]やシャツを 買[か]う。		
\\	魚が好
\\	タイプ。	魚が好きじゃないタイプ。	さかながすきじゃないタイプ。	
\\	魚[さかな]が 好[す]きじゃないタイプ。		
\\	日本語
\\	、習う。	日本語は、習う。	にほんごは、ならう。	
\\	日本語[にほんご]は、 習[なら]う。		
\\	静か
\\	。	静かじゃない。	しずかじゃない。	
\\	静[しず]かじゃない。		
\\	学校に行った?/行かなかった。/図書館
\\	は? /図書館
\\	行かなかった。	学校に行った?/行かなかった。/図書館には? /図書館にも行かなかった。	「がっこうにいった?/いかなかった。/としょかんには?/としょかんにもいかなかった。	
\\	学校[がっこう]に 行[い]った?/ 行[い]かなかった。/ 図書館[としょかん]には?/ 図書館[としょかん]にも 行[い]かなかった。		
\\	家
\\	帰らない。	家に帰らない。	{いえ
\\	うち}にかえらない。	
\\	「に」 
\\	家[いえ]に 帰[かえ]らない。		
\\	日本語
\\	習う。	日本語を習う。	にほんごをならう。	
\\	日本語[にほんご]を 習[なら]う。		
\\	魚
\\	食べる。	魚を食べる。	さかなをたべる。	
\\	「を」 
\\	魚[さかな]を 食[た]べる。		
\\	ジムが
\\	。	ジムがしなかった。	ジムがしなかった。	
\\	ジムがしなかった。		
\\	勉強は、
\\	。	勉強は、した。	べんきょうは、した。	
\\	勉強[べんきょう]は、した。		
\\	友達は親切
\\	人。	友達は親切な人。	ともだちはしんせつなひと。	
\\	友達[ともだち]は 親切[しんせつ]な 人[ひと]。		
\\	アリスは食
\\	。	アリスは食べなかった。	アリスはたべなかった。	
\\	アリスは 食[た]べなかった。		
\\	高
\\	ビル。	高かったビル。	たかかったビル。	
\\	高[たか]かったビル。		
\\	飲み物
\\	カップ
\\	ナプキンは、いらない?	飲み物やカップやナプキンは、いらない?	のみものやカップやナプキンは、いらない?	
\\	「や」 
\\	「とか」 
\\	飲み物[のみもの]やカップやナプキンは、いらない?		
\\	魚
\\	。	魚だった。	さかなだった。	
\\	魚[さかな]だった。		
\\	この町は、最近大
\\	変わった。	この町は、最近大きく変わった。	このまちは、さいきんおおきくかわった。	
\\	この 町[まち]は、 最近[さいきん] 大[おお]きく 変[か]わった。		
\\	授業に行く
\\	こと]を忘れた。	授業に行くのを忘れた。	じゅぎょうにいくのをわすれた。	
\\	「の」 
\\	授業[じゅぎょう]に 行[い]くのを 忘[わす]れた。		
\\	先週に図書館
\\	行った。	先週に図書館に行った。	せんしゅうにとしょかんにいった。	
\\	「に」 
\\	先週[せんしゅう]に 図書館[としょかん]に 行[い]った。		
\\	どこに行
\\	/授業に行
\\	。	どこに行くの?/授業に行くんだ。	どこにいくの?/じゅぎょうにいくんだ。	
\\	「の」 
\\	どこに 行[い]くの?/ 授業[じゅぎょう]に 行[い]くんだ。		
\\	高
\\	ビル。	高いビル。	たかいビル。	
\\	高[たか]いビル。		
\\	部屋
\\	来る。	部屋へ来る。	へやへくる。	
\\	「へ」 
\\	部屋[へや]へ 来[く]る。		
\\	映画は、全部見
\\	。	映画は、全部見た。	えいがは、ぜんぶみた。	
\\	映画[えいが]は、 全部[ぜんぶ] 見[み]た。		
\\	部屋
\\	来る。	部屋に来る。	へやにくる。	
\\	「に」 
\\	部屋[へや]に 来[く]る。		
\\	ジムは医者
\\	なる。	ジムは医者になる。	ジムはいしゃになる。	
\\	「に」 
\\	ジムは 医者[いしゃ]になる。		
\\	ジム
\\	明日?/明日じゃない。	ジムは明日?/明日じゃない。	ジムはあした?/あしたじゃない。	
\\	「は」 
\\	ジムは 明日[あした]?/ 明日[あした]じゃない。		
\\	何できた?/暇だから。
\\	何で?	何できた?/暇だから。
\\	何で?	なんできた?/ひまだから。	
\\	「で」 
\\	「何」	何[なん]できた?/ 暇[ひま]だから。 
\\	何[なに]で?		
\\	値段があんまりよ
\\	。	値段があんまりよくない。	ねだんがあんまりよくない。	
\\	値段[ねだん]があんまりよくない。		
\\	家
\\	帰らない。	家へ帰らない。	{いえ
\\	うち}へかえらない。	
\\	「へ」 
\\	家[いえ]へ 帰[かえ]らない。		
\\	子供
\\	アリスが立派な大人になった。	子供だったアリスが立派な大人になった。	こどもだったアリスがりっぱなおとなになった。	
\\	子供[こども]だったアリスが 立派[りっぱ]な 大人[おとな]になった。		
\\	今は忙しい
\\	今は忙しいの?	いまはいそがしいの?	
\\	「の」 
\\	今[いま]は 忙[いそが]しいの?		
\\	魚が好
\\	タイプ。	魚が好きじゃなかったタイプ。	さかながすきじゃなかったタイプ。	
\\	魚[さかな]が 好[す]きじゃなかったタイプ。		
\\	私が電気を点
\\	。	私が電気を点けた。	わたしがでんきをつけた。	
\\	私[わたし]が 電気[でんき]を 点[つ]けた。		
\\	レストラン
\\	昼ご飯を食べた。	レストランで昼ご飯を食べた。	レストランでひるごはんをたべた。	
\\	「で」 
\\	レストランで 昼[ひる]ご 飯[はん]を 食[た]べた。		
\\	私は買
\\	。	私は買わなかった。	わたしはかわなかった。	
\\	私[わたし]は 買[か]わなかった。		
\\	学生
\\	人は、学校に行かない。	学生じゃない人は、学校に行かない。	がくせいじゃないひとは、がっこうにいかない。	
\\	学生[がくせい]じゃない 人[ひと]は、 学校[がっこう]に 行[い]かない。		
\\	ボブ
\\	本。	ボブの本。	ボブのほん。	
\\	「の」 
\\	ボブの 本[ほん]。		
\\	静
\\	人。	静かな人。	しずかなひと。	
\\	静[しず]かな 人[ひと]。		
\\	ボブも行
\\	。	ボブも行かなかった。	ボブもいかなかった。	
\\	ボブも 行[い]かなかった。		
\\	白
\\	こと]は、かわいい。	白いのは、かわいい。	しろいのは、かわいい。	
\\	「の」 
\\	白[しろ]いのは、かわいい。		
\\	ジムが
\\	。	ジムが遊ぶ。	ジムはあそぶ。	
\\	ジムが 遊[あそ]ぶ。		
\\	猫は部屋
\\	いる。	猫は部屋にいる。	ねこはへやにいる。	
\\	「に」 
\\	猫[ねこ]は 部屋[へや]にいる。		
\\	アリスは食
\\	。	アリスは食べない。	アリスはたべない。	
\\	アリスは 食[た]べない。		
\\	アリスは、アメリカ
\\	来た。	アリスは、アメリカから来た。	アリスは、アメリカからきた。	
\\	「に」 
\\	アリスは、アメリカから 来[き]た。		
\\	映画を
\\	見た。	映画をたくさん見た。	えいがをたくさんみた。	
\\	映画[えいが]をたくさん 見[み]た。		
\\	そのシャツは誰
\\	/ボブ
\\	だ。	そのシャツは誰の?/ボブのだ。	そのシャツはだれの?/ボブのだ。	
\\	「の」 
\\	そのシャツは 誰[だれ]の?/ボブのだ。		
\\	。	魚。	さかな。	
\\	「だ」	魚[さかな]。		
\\	今は忙しい
\\	。	今は忙しいのだ。	いまはいそがしいのだ。	
\\	「の」 
\\	今[いま]は 忙[いそが]しいのだ。		
\\	最近、
\\	食べない。	最近、全然食べない。	さいきん、ぜんぜんたべない。	
\\	最近[さいきん]、 全然[ぜんぜん] 食[た]べない。		
\\	ボブは、アメリカ
\\	大学
\\	学生だ。	ボブは、アメリカの大学の学生だ。	ボブは、アメリカのだいがくのがくせいだ。	
\\	「の」 
\\	ボブは、アメリカの 大学[だいがく]の 学生[がくせい]だ。		
\\	映画館
\\	見た。	映画館で見た。	えいがかんでみた。	
\\	「で」 
\\	映画[えいが] 館[かん]で 見[み]た。		
\\	赤いズボンを買
\\	友達はボブだ。	赤いズボンを買う友達はボブだ。	あかいズボンをかうともだちはボブだ。	
\\	赤[あか]いズボンを 買[か]う 友達[ともだち]はボブだ。		
\\	ボブは日本
\\	行った。	ボブは日本へ行った。	ボブはにほんへいった。	
\\	「へ」 
\\	ボブは 日本[にっぽん]へ 行[い]った。		
\\	ボブも
\\	。	ボブもする。	ボブもする。	
\\	ボブもする。		
\\	静か
\\	。	静かじゃなかった。	しぞかじゃなかった。	
\\	静[しず]かじゃなかった。		
\\	椅子が台所
\\	あった。	椅子が台所にあった。	いすがだいどころにあった。	
\\	「に」 
\\	椅子[いす]が 台所[だいどころ]にあった。		
\\	白
\\	は、かわいい。	白い物は、かわいい。	しろいものは、かわいい。	
\\	「の」 
\\	もの 
\\	白[しろ]い 物[もの]は、かわいい。		
\\	靴
\\	や)]シャツを買う。	靴とかシャツを買う。	くつとかシャツをかう。	
\\	「や」 
\\	「とか」 
\\	靴[くつ]とかシャツを 買[か]う。		
\\	ボブは、いつも勉強
\\	人だ。	ボブは、いつも勉強する人だ。	ボブは、いつもべんきょうするひとだ。	
\\	ボブは、いつも 勉強[べんきょう]する 人[ひと]だ。		
\\	窓がどうして開
\\	窓がどうして開いた?	まどがどうしてあいた?	
\\	窓[まど]がどうして 開[ひら]いた?		
\\	私は
\\	。	私は買う。	わたしはかう。	
\\	私[わたし]は 買[か]う。		
\\	いい友達
\\	会った。	いい友達に会った。	いいともだちにあった。	
\\	「に」 
\\	いい 友達[ともだち]に 会[あ]った。		
\\	ボブは朝ご飯を早
\\	食べた。	ボブは朝ご飯を早く食べた。	ボブはあさごはんをはやくたべた。	
\\	ボブは 朝[あさ]ご 飯[はん]を 早[はや]く 食[た]べた。		
\\	飲み物
\\	や)]カップ
\\	や)]ナプキンは、いらない?	飲み物とかカップとかナプキンは、いらない?	のみものとかカップとかナプキンは、いらない?	
\\	「や」 
\\	「とか」 
\\	飲み物[のみもの]とかカップとかナプキンは、いらない?		
\\	アリス
\\	学生?/うん、学生。	アリスは学生?/うん、学生。	アリスはがくせい。/うん、がくせい	
\\	「は」 
\\	アリスは 学生[がくせい]?/うん、 学生[がくせい]。		
\\	先週に映画を見
\\	人は誰?	先週に映画を見た人は誰?	せんしゅうにえいがをみたひとはだれ?	
\\	先週[せんしゅう]に 映画[えいが]を 見[み]た 人[ひと]は 誰[だれ]?		
\\	どこで食べる? /イタリアレストラン
\\	はどう?	どこで食べる? /イタリアレストランではどう?	どこでだべる?/イタリアレストランではどう?	
\\	どこで 食[た]べる?/イタリアレストランではどう?		
\\	学生
\\	誰?	学生は誰?	がくせいはだれ?	
\\	「が」 
\\	学生[がくせい]は 誰[だれ]?		
\\	スプーン
\\	フォークで魚を食べた。	スプーンとフォークで魚を食べた。	スプーンとフォークでさかなをたべた。	
\\	「と」 
\\	スプーンとフォークで 魚[さかな]を 食[た]べた。		
\\	図書館の中では、静
\\	する。	図書館の中では、静かにする。	としょかんのなかでは、しずかにする。	
\\	図書館[としょかん]の 中[なか]では、 静[しず]かにする。		
\\	アリスは学生?/ううん、トム
\\	学生じゃない。	アリスは学生?/ううん、トムも学生じゃない。	アリスはがくせい?/ううん、トムもがくせいじゃない。	
\\	「も」 
\\	アリスは 学生[がくせい]?/ううん、トムも 学生[がくせい]じゃない。		
\\	彼はかっこよ
\\	彼はかっこよかった!	かれはかっこよかった。	
\\	彼[かれ]はかっこよかった!		
\\	今、授業がある
\\	/今は、ないんだ。	今、授業があるんじゃない?/今は、ないんだ。	いま、じゅぎょうがあるんじゃない?/いまは、ないんだ。	
\\	「の」 
\\	今[いま]、 授業[じゅぎょう]があるんじゃない?/ 今[いま]は、ないんだ。		
\\	宿題を今日
\\	明日までする。	宿題を今日から明日までする。	しゅくだいをきょうからあしたまでする。	
\\	「に」 
\\	宿題[しゅくだい]を 今日[きょう]から 明日[あした]までする。		
\\	誰
\\	学生?/ジム
\\	学生。	誰が学生?/ジムが学生。	だれががくせい?/ジムはがくせい。	
\\	「が」 
\\	誰[だれ]が 学生[がくせい]?/ジムが 学生[がくせい]。		
\\	アリスは
\\	。	アリスは食べる。	アリスはたべる。	
\\	アリスは 食[た]べる。		
\\	ボブは日本
\\	行った。	ボブは日本に行った。	ボブはにほんにいった。	
\\	「に」 
\\	ボブは 日本[にっぽん]に 行[い]った。		
\\	ボブは魚が
\\	。	ボブは魚が好きだ。	ボブはさかながすきだ。	
\\	ボブは 魚[さかな]が 好[す]きだ。		
\\	学生
\\	。	学生じゃなかった。	がくせいじゃなかった。	
\\	学生[がくせい]じゃなかった。		
\\	友達が来
\\	。	友達が来た。	ともだちがきた。	
\\	友達[ともだち]が 来[き]た。		
\\	ジム
\\	。	ジムのだ。	ジムのだ。	
\\	「の」 
\\	ジムのだ。		
\\	高
\\	ビル。	高くないビル。	たかくないビル。	
\\	高[たか]くないビル。		
\\	友達
\\	アリスは、いい友達になった。	友達じゃなかったアリスは、いい友達になった。	ともだちじゃなかったアリスは、いいともだちになった。	
\\	友達[ともだち]じゃなかったアリスは、いい 友達[ともだち]になった。		
\\	友達は
\\	。	友達は親切。	ともだちはしんせつ。	
\\	友達[ともだち]は 親切[しんせつ]。		
\\	アリスは自分の部屋をきれい
\\	した。	アリスは自分の部屋をきれいにした。	アリスはじぶんのへやをきれいにした。	
\\	アリスは 自分[じぶん]の 部屋[へや]をきれいにした。		
\\	授業に行く
\\	の]を忘れた。	授業に行くことを忘れた。	じゅぎょうにいくことをわすれた。	
\\	「の」 
\\	こと 
\\	授業[じゅぎょう]に 行[い]くことを 忘[わす]れた。		
\\	猫は
\\	。	猫はいる。	ねこはいる。	
\\	猫[ねこ]はいる。		
\\	ボブは魚が好
\\	。	ボブは魚が好きじゃなかった。	ボブはさかながすきじゃなかった。	
\\	ボブは 魚[さかな]が 好[す]きじゃなかった。		
\\	メールアドレス
\\	登録した。	メールアドレスを登録した。	メールアドレスをとうろくした。	
\\	「を」 
\\	メールアドレスを 登録[とうろく]した。		
\\	高
\\	静かなビル。	高くない静かなビル。	たかくないしずかなビル。	
\\	高[たか]くない 静[しず]かなビル。		
\\	ご飯は、食
\\	。	ご飯は、食べた。	ごはんは、たべた。	
\\	ご 飯[はん]は、 食[た]べた。		
\\	先週に医者
\\	ボブは、仕事を辞めた。	先週に医者だったボブは、仕事を辞めた。	せんしゅうにいしゃだったボブは、しごとをやめた。	
\\	先週[せんしゅう]に 医者[いしゃ]だったボブは、 仕事[しごと]を 辞[や]めた。		
\\	電気が消
\\	。	電気が消える。	でんきがきえる。	
\\	電気[でんき]が 消[き]える。		
\end{CJK}
\end{document}