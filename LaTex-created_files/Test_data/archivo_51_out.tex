\documentclass[8pt]{extreport} 
\usepackage{hyperref}
\usepackage{CJKutf8}
\begin{document}
\begin{CJK}{UTF8}{min}
\\	雑煮
\\	おせち料理
\\	お年玉
\\	袋
\\	二年参り
\\	元旦
\\	ごちそうさま。
\\	乾杯
\\	美味しい
\\	いただきます。
\\	頂く
\\	乾杯
\\	納豆
\\	まあまあ
\\	初めまして
\\	私
\\	宜しく
\\	お願いします。
\\	山口
\\	渡辺
\\	初めまして。	
\\	初めまして。 
\\	初めまして。	
\\	初めまして。 
\\	宜しくお願いします。	
\\	宜しくお願いします。 
\\	元気
\\	凄い
\\	絶好調
\\	ばっちり
\\	お元気ですか。	
\\	お元気ですか。 
\\	はい、元気ですよ。	
\\	はい、元気ですよ。 
\\	お元気ですか。	
\\	お元気ですか。 
\\	はい、絶好調です!	
\\	はい、絶好調です! 
\\	草履
\\	下駄
\\	袴
\\	足袋
\\	成人の日
\\	成人式
\\	着物
\\	振袖
\\	帯
\\	勉強
\\	迷信
\\	霊柩車
\\	親指
\\	夜の蜘蛛
\\	招き猫
\\	お守り
\\	大学入試センター試験
\\	歴史
\\	国語
\\	-語
\\	日本語
\\	英語
\\	数学
\\	予備校
\\	浪人生
\\	地理
\\	東京大学
\\	大学
\\	入試
\\	センター
\\	試験
\\	青山学院
\\	青
\\	山
\\	慶応大学
\\	留年
\\	ゴールデンウイーク
\\	新幹線
\\	温泉地
\\	国内旅行
\\	海外旅行
\\	憲法記念日
\\	緑の日
\\	なるほど
\\	ゴールデンタイム
\\	黄金週間
\\	伊豆
\\	方言
\\	福岡
\\	鹿児島
\\	熊本
\\	宮崎
\\	長崎
\\	大分
\\	佐賀
\\	博多
\\	ラーメン
\\	沖縄
\\	九州
\\	四国
\\	弁
\\	標準語
\\	東北弁
\\	東京弁
\\	関西弁
\\	九州弁
\\	長崎弁
\\	北海道
\\	本州
\\	豚骨
\\	おねえちゃん、この席取っとっと。	
\\	おねえちゃん、この席取っとっと。 
\\	ううん、空いとるよ。	
\\	ううん、空いとるよ。 
\\	あー、何ば食べよっと。	
\\	あー、何ば食べよっと。 
\\	見て分からんと。	
\\	見て分からんと。 
\\	そのラーメンうまかと。	
\\	そのラーメンうまかと。 
\\	おいしかよ。	
\\	おいしかよ。 
\\	じゃあ、おじさん、このラーメン一つ!	
\\	じゃあ、おじさん、このラーメン一つ! 
\\	よし、ゴールデンウィークは何しよったと。	
\\	よし、ゴールデンウィークは何しよったと。 
\\	温泉に行ったばい。	
\\	温泉に行ったばい。 
\\	良かねー。どこの温泉に行ったと。	
\\	良かねー。どこの温泉に行ったと。 
\\	雲仙の温泉に行ったとよ。	
\\	雲仙の温泉に行ったとよ。 
\\	多かったやろう。	
\\	多かったやろう。 
\\	多かったけど、良かったばい。ひろこは何ばしたと。	
\\	多かったけど、良かったばい。ひろこは何ばしたと。 
\\	ずっと働きよったけん、どこも行かんかった。	
\\	ずっと働きよったけん、どこも行かんかった。 
\\	大変やったねー。	
\\	大変やったねー。 
\\	うん、かなり大変やったー。	
\\	うん、かなり大変やったー。 
\\	ばってん、頑張らんばばい!	
\\	ばってん、頑張らんばばい! 
\\	動物
\\	亀
\\	土佐犬
\\	豆柴
\\	柴犬
\\	秋田犬
\\	鶯
\\	海老
\\	金魚すくい
\\	金魚
\\	鯉
\\	尾頭付
\\	鯛
\\	犬
\\	猫
\\	狸寝入り
\\	蛙
\\	お犬様
\\	生類憐みの令
\\	蜻蛉
\\	蛍
\\	狐
\\	狸
\\	犬
\\	蝉
\\	羽根つき
\\	絵札
\\	読み札
\\	羽子板
\\	罰ゲーム
\\	スイカ割り
\\	鬼ごっこ
\\	遊び
\\	墨
\\	トランプ
\\	カルタ
\\	凧揚げ
\\	短歌
\\	福笑い
\\	折り紙
\\	千羽鶴
\\	綾取り
\\	剣玉
\\	隠れん坊
\\	じゃん拳
\\	お中元
\\	吹き流し
\\	短冊
\\	七夕
\\	残暑見舞い
\\	暑中見舞い
\\	お返し
\\	お歳暮
\\	山芋
\\	お素麺
\\	巾着
\\	見合い
\\	恋愛結婚
\\	家庭教師
\\	結婚
\\	二次会
\\	お見合い結婚
\\	できちゃった結婚
\\	プロポーズ
\\	提灯
\\	赤提灯
\\	大文字焼き
\\	灯篭流し
\\	盆踊り
\\	五山の送り火
\\	お盆
\\	結納
\\	新婚旅行
\\	仲人
\\	結納返し
\\	教会
\\	神前
\\	白無垢
\\	角隠し
\\	人前
\\	結婚届
\\	地味婚
\\	六次会
\\	五次会
\\	結婚式
\\	三三九度
\\	神道
\\	仏教
\\	キリスト教
\\	儀式
\\	披露宴
\\	二次会
\\	三次会
\\	四次会
\\	綿帽子
\\	防災
\\	強風
\\	避難所
\\	地震
\\	水害
\\	消火器
\\	消火
\\	火災
\\	関東大震災
\\	訓練
\\	豪雨
\\	ケーキカット
\\	キャンドルサービス
\\	披露宴
\\	敬老の日
\\	健康
\\	祝う
\\	長寿国
\\	長寿
\\	敬う
\\	老人
\\	祝日
\\	国民
\\	福祉
\\	体育の日
\\	ハッピーマンデー制度
\\	運動会
\\	お弁当
\\	騎馬戦
\\	綱引き
\\	皇太子
\\	皇太子妃
\\	お化け
\\	文化勲章
\\	文化
\\	勲章
\\	七五三
\\	着物
\\	袴
\\	羽織
\\	帯
\\	千歳飴
\\	勤労感謝の日
\\	新嘗祭
\\	勤労
\\	感謝
\\	天皇誕生日
\\	天皇
\\	天皇陛下
\\	万歳
\\	クリスマス
\\	メリー・クリスマス!
\\	クリスマスケーキ
\\	イルミネイション
\\	福袋
\\	季節の食べ物
\\	はんぺん
\\	大根
\\	白菜
\\	水菜
\\	石狩なべ
\\	掘り炬燵
\\	炬燵
\\	鋤焼き
\\	鍋
\\	牡蠣
\\	屋台
\\	大掃除
\\	忘年会
\\	新年会
\\	忘年会
\\	御節料理
\\	居酒屋
\\	相撲
\\	力士
\\	勝ち越し
\\	負け越し
\\	土俵
\\	行司
\\	横綱
\\	大関
\\	関脇
\\	親方
\\	部屋
\\	幕内
\\	十両
\\	幕下
\\	三段目
\\	序二段
\\	序ノ口
\\	ちゃんこ鍋
\\	日本相撲協会
\\	小結
\\	わらし
\\	へば
\\	入る
\\	あんずましい (あずましい)
\\	んだはんで
\\	んだ
\\	しばれる
\\	たげ
\\	津軽漬け
\\	行ぐ、 行く
\\	おめ)	な
\\	わ
\\	なも
\\	こった
\\	おめ
\\	まんず
\\	おめ、こった時間にどさ?	
\\	おめ、こった時間にどさ? 
\\	なも、わはわらしっこむが迎えに行ぐじゃ。なは?	
\\	なも、わはわらしっこむが迎えに行ぐじゃ。なは? 
\\	わー津軽漬け買いさ行くじゃ。	
\\	わー津軽漬け買いさ行くじゃ。 
\\	んだがー。今夜もたげしばれるな。	
\\	んだがー。今夜もたげしばれるな。 
\\	んだ。んだはんでわは早ぐ帰ってあんずましい湯さ入るじゃ。へば、まんず!	
\\	んだ。んだはんでわは早ぐ帰ってあんずましい湯さ入るじゃ。へば、まんず! 
\\	なんぼ
\\	決める
\\	かちゃくちゃ!
\\	へば
\\	たけぇ
\\	どんだ?
\\	せば
\\	んだが
\\	(お母ちゃん)	かっちゃ
\\	どしちゃー?
\\	けっぱる
\\	お、どしちゃー?何しちゅんず?	
\\	お、どしちゃー?何しちゅんず? 
\\	わ、かっちゃの誕生日プレゼント買わねばまいね。	
\\	わ、かっちゃの誕生日プレゼント買わねばまいね。 
\\	んだが。せば一緒に探してけら。これどんだ?	
\\	んだが。せば一緒に探してけら。これどんだ? 
\\	なんぼ?3000円が。たけえな、どすべ?	
\\	なんぼ?3000円が。たけえな、どすべ? 
\\	へばこれは?	
\\	へばこれは? 
\\	そいはいぐねな。あ、どすべ、どすべ・・・	
\\	そいはいぐねな。あ、どすべ、どすべ・・・ 
\\	わいー!かちゃくちゃね!おめはやぐ決めろじゃ!どんだんず!わ、もう行ぐはんで。おめ一人でけっぱれ。	
\\	わいー!かちゃくちゃね!おめはやぐ決めろじゃ!どんだんず!わ、もう行ぐはんで。おめ一人でけっぱれ。 
\\	んだが。	
\\	んだが。 
\\	やばい
\\	弘前公園
\\	にんにく
\\	林檎
\\	(そうだね・
\\	んだ
\\	どせば
\\	どすべ
\\	叱られる
\\	さ
\\	(お母ちゃん・
\\	かっちゃ
\\	めぐせ
\\	なげる
\\	(駄目・
\\	まいね
\\	そった
\\	なして
\\	点数
\\	弘前城
\\	やべー。まだテストの点数悪がったじゃー。どすべ?	
\\	やべー。まだテストの点数悪がったじゃー。どすべ? 
\\	なしてそったにまいね?	
\\	なしてそったにまいね? 
\\	めぐせべよー。かっちゃさも叱られるんだね。どせば?	
\\	めぐせべよー。かっちゃさも叱られるんだね。どせば? 
\\	なげてまえば?	
\\	なげてまえば? 
\\	んだ。なげるわ。	
\\	んだ。なげるわ。 
\\	(どうしてる)	どしちゃ?
\\	(駄目)	まいね
\\	べ
\\	でかい
\\	(から)	はんで
\\	食う
\\	まま
\\	(いっぱい)	がっぱど
\\	かっちゃく
\\	(そうだね)	んだ
\\	(だろう・でっしょう)	んだべ
\\	(そうか)	んだが
\\	(している)	しちゃ
\\	散歩
\\	(私)	わ
\\	(何してるの?)	何しちゅんず?
\\	(そうなんだ?)	んだのか
\\	どしちゃ?何しちゅんず?	
\\	どしちゃ?何しちゅんず? 
\\	わ、今、犬の散歩しちゃのよ。	
\\	わ、今、犬の散歩しちゃのよ。 
\\	んだが、なんぼめんこい犬っこだな。でけえ。	
\\	んだが、なんぼめんこい犬っこだな。でけえ。 
\\	んだべ。がっぱどまま食うはんででけえべ?	
\\	んだべ。がっぱどまま食うはんででけえべ? 
\\	んだ。うちの犬はすぐかっちゃくはんでまいねよ。	
\\	んだ。うちの犬はすぐかっちゃくはんでまいねよ。 
\\	んだのかー。危ねな!	
\\	んだのかー。危ねな! 
\\	がっぱど
\\	(そうだな)	んだな
\\	しばれる
\\	滑る
\\	(でしょう)	だべ
\\	(私たち)	わんど
\\	冷 
\\	しゃっこい
\\	今年も雪ががっぱど降ったな~	
\\	今年も雪ががっぱど降ったな~ 
\\	んだな。しばれるじゃ。	
\\	んだな。しばれるじゃ。 
\\	木の上もすげえ雪だじゃ。	
\\	木の上もすげえ雪だじゃ。 
\\	この雪は滑るから気つけねばまいね。	
\\	この雪は滑るから気つけねばまいね。 
\\	大丈夫だべ。わんど猿だじゃな?	
\\	大丈夫だべ。わんど猿だじゃな? 
\\	いや、気い抜けばあぶねよ?	
\\	いや、気い抜けばあぶねよ? 
\\	いてえーーしゃっけーーー!!	
\\	(笑い声)だがら、言ったべな	
\\	(笑い声)だがら、言ったべな 
\\	はりつけ
\\	間もなく
\\	めっちゃ
\\	ひかえ
\\	香ばしい
\\	かぐわしい
\\	あやー
\\	しょうしい
\\	だろも
\\	しょ
\\	ばか
\\	なーち
\\	ちんこい
\\	こんげ
\\	おめさん
\\	しかも
\\	あちさん
\\	あやー、勇樹じゃねっけ。	
\\	あやー、勇樹じゃねっけ。 
\\	あやー、しかもおーってねかったろも、おめさん、こんげところでなにしてん?	
\\	あやー、しかもおーってねかったろも、おめさん、こんげところでなにしてん? 
\\	いま犬の散歩してる所だてば。	
\\	いま犬の散歩してる所だてば。 
\\	そういのー、ほーんね、ちんこい犬らのー。そういえばさ、なーちの隣げに、ばかかっけぇ東京のしょが引っ越してきたらしいねっけ。	
\\	そういのー、ほーんね、ちんこい犬らのー。そういえばさ、なーちの隣げに、ばかかっけぇ東京のしょが引っ越してきたらしいねっけ。 
\\	ほーんね、そういんよー。だろも、しょうしで話すなんていやらいや。	
\\	ほーんね、そういんよー。だろも、しょうしで話すなんていやらいや。 
\\	ほんねもう、、まごまごしてっと、あちさんに取られるこてさ!	
\\	ほんねもう、、まごまごしてっと、あちさんに取られるこてさ! 
\\	いとしげ
\\	おなご
\\	おーここ
\\	なにね
\\	あんま
\\	かんべ
\\	はよ
\\	そっでは
\\	やぃや、いとぉーしげらねぇ。おなごのこらかね?	
\\	やぃや、いとぉーしげらねぇ。おなごのこらかね? 
\\	男ん子らよ。	
\\	男ん子らよ。 
\\	おーここ、なにね、そうらけ?あんまいとしげらすけ、おなごのこらかと思たてぇ。かんべしてくんなせや。なんて名前らいや?	
\\	おーここ、なにね、そうらけ?あんまいとしげらすけ、おなごのこらかと思たてぇ。かんべしてくんなせや。なんて名前らいや? 
\\	浩介らよ。	
\\	浩介らよ。 
\\	こんがさぁべ所にいねぇて、はよ帰らせ!ぼぼ、風邪こと引くこてね。	
\\	こんがさぁべ所にいねぇて、はよ帰らせ!ぼぼ、風邪こと引くこてね。 
\\	おめえ
\\	がめる
\\	いんや
\\	~がん
\\	だすけ
\\	こく
\\	しゃっつける
\\	なぁや俺の犬がめたろー。	
\\	なぁや俺の犬がめたろー。 
\\	おめ、何こきやがるんで?おらがめてねいや。	
\\	おめ、何こきやがるんで?おらがめてねいや。 
\\	いんや、なぁが、がめんがん見てたれー。	
\\	いんや、なぁが、がめんがん見てたれー。 
\\	証拠でもあんかや。	
\\	証拠でもあんかや。 
\\	だすけ、俺が見てたて言うてっろぉ。	
\\	だすけ、俺が見てたて言うてっろぉ。 
\\	てんぽこくでねーて。	
\\	てんぽこくでねーて。 
\\	てんぽでねぃっや。この目ではっきりと見たて。	
\\	てんぽでねぃっや。この目ではっきりと見たて。 
\\	裏声
\\	のぉ、わしのブランデー知らんか?	
\\	のぉ、わしのブランデー知らんか? 
\\	昨日買うてここに置いとったんじゃけど、ねぇんじゃ。	
\\	昨日買うてここに置いとったんじゃけど、ねぇんじゃ。 
\\	ああ、それな。ごめん。私全部飲んでしもうた。	
\\	ああ、それな。ごめん。私全部飲んでしもうた。  
\\	はぁ?ちばけな。あれ高かったんで。	
\\	はぁ?ちばけな。あれ高かったんで。 
\\	でもまあ、お互い様いうことかの。 わしもあんたのウイスキー飲んでしもうたけぇ。	
\\	でもまあ、お互い様いうことかの。 わしもあんたのウイスキー飲んでしもうたけぇ。   
\\	なんじゃ、それ?	
\\	なんじゃ、それ? 
\\	あんたこそ気がわりぃわぁ。どうしてくれるん?	
\\	あんたこそ気がわりぃわぁ。どうしてくれるん? 
\\	よう言うわ。人のチョコレートこっそり食ようったの、知らんとでも思よんか?	
\\	よう言うわ。人のチョコレートこっそり食ようったの、知らんとでも思よんか?  
\\	男がな、女子供の食べるチョコレートなんか食べられな、気持ちわりぃ。	
\\	男がな、女子供の食べるチョコレートなんか食べられな、気持ちわりぃ。 
\\	あんただって気持ちわりぃがな。	
\\	あんただって気持ちわりぃがな。 
\\	確定申告
\\	来週いっぱい
\\	税金
\\	今後
\\	市役所
\\	宴会
\\	花柄
\\	(かさを)さす
\\	まし
\\	わし、確定申告行かんといけん。いつまでじゃったかのぉ?	
\\	わし、確定申告行かんといけん。いつまでじゃったかのぉ? 
\\	来週一杯じゃった思うわ。早う行かれえ。	
\\	来週一杯じゃった思うわ。早う行かれえ。 
\\	でも何であんなに税金持って行かれるんで?	
\\	でも何であんなに税金持って行かれるんで? 
\\	もうかなわんわ。ほんま、今後の日本はどうなるん?おえりゃあせなぁ。	
\\	もうかなわんわ。ほんま、今後の日本はどうなるん?おえりゃあせなぁ。 
\\	おえ、変なことしたんじゃねんか?	
\\	おえ、変なことしたんじゃねんか? 
\\	いいや。	
\\	いいや。 
\\	この前市役所に行って、私らの税金が宴会に使われとらんか確かめてきただけ。	
\\	この前市役所に行って、私らの税金が宴会に使われとらんか確かめてきただけ。 
\\	温暖化
\\	白桃
\\	直前
\\	氷水
\\	冷蔵庫
\\	甘さ
\\	落ちる
\\	あちいのー。かなわんのー。どねーかならんかのー。	
\\	あちいのー。かなわんのー。どねーかならんかのー。 
\\	しゃあないわな、温暖化が進みょうるし。	
\\	しゃあないわな、温暖化が進みょうるし。 
\\	でも白桃は暑いほうが甘うて美味しいよ。	
\\	でも白桃は暑いほうが甘うて美味しいよ。 
\\	確かにのー。よその桃は食えんのー。	
\\	確かにのー。よその桃は食えんのー。 
\\	そういやあ、岡山からいつ届くんな?	
\\	そういやあ、岡山からいつ届くんな? 
\\	さっき届いたよ。	
\\	さっき届いたよ。 
\\	食べる直前に氷水で冷やして食びょうな。	
\\	食べる直前に氷水で冷やして食びょうな。 
\\	正月
\\	初詣
\\	年賀状
\\	おはよう
\\	こんばんは。
\\	こんにちは。
\\	おはようございます
\\	おはよう。	
\\	おはよう。 
\\	おはようございます。	
\\	おはようございます。 
\\	こんにちは。	
\\	こんにちは。 
\\	こんにちは。	
\\	こんにちは。 
\\	こんばんは!	
\\	こんばんは! 
\\	こんばんは。	
\\	こんばんは。 
\\	です
\\	はじめまして
\\	よろしくおねがいします
\\	はじめまして。テイラーです。	
\\	はじめまして。テイラーです。 
\\	はじめまして。かおりです。	
\\	はじめまして。かおりです。 
\\	よろしくおねがいします。	
\\	よろしくおねがいします。 
\\	よろしくおねがいします。	
\\	よろしくおねがいします。 
\\	ありがとうございます。
\\	おみやげ
\\	どういたしまして
\\	おみやげです。	
\\	おみやげです。 
\\	ありがとうございます!	
\\	ありがとうございます! 
\\	どういたしまして。	
\\	どういたしまして。 
\\	あの・・・
\\	すみません
\\	あし
\\	はい?
\\	イタタタ	
\\	イタタタ 
\\	あの…すみません。	
\\	あの…すみません。 
\\	はい?	
\\	はい? 
\\	あし…。	
\\	あし…。 
\\	あ、すみません!	
\\	あ、すみません! 
\\	お客さん
\\	注文
\\	メニュー
\\	はい
\\	水
\\	お会計
\\	お願いします
\\	すみません!メニュー、おねがいします。	
\\	すみません!メニュー、おねがいします。 
\\	はい。	
\\	はい。 
\\	すみません、みず、おねがいします。	
\\	すみません、みず、おねがいします。 
\\	はい。	
\\	はい。 
\\	すみません、ちゅうもん、おねがいします。	
\\	すみません、ちゅうもん、おねがいします。 
\\	はーい。	
\\	はーい。 
\\	お久しぶりです
\\	久しぶり。
\\	さん
\\	あ、テイラーさん?	
\\	あ、テイラーさん? 
\\	まさとさん!おひさしぶりです!	
\\	まさとさん!おひさしぶりです! 
\\	おひさしぶりです!	
\\	おひさしぶりです! 
\\	私
\\	よろしくお願いします
\\	さん
\\	です
\\	ぼく
\\	はじめまして。わたしはかおりです。	
\\	はじめまして。わたしはかおりです。 
\\	かおりさん?はじめまして。ぼくはテイラーです。	
\\	かおりさん?はじめまして。ぼくはテイラーです。 
\\	よろしくおねがいします。	
\\	よろしくおねがいします。 
\\	よろしくおねがいします。	
\\	よろしくおねがいします。 
\\	これ
\\	おみやげ
\\	ありがとうございます
\\	いえいえ
\\	どうぞ
\\	テイラーさん、これはおみやげです。どうぞ。	
\\	テイラーさん、これはおみやげです。どうぞ。 
\\	ありがとうございます!	
\\	ありがとうございます! 
\\	いえいえ。	
\\	いえいえ。 
\\	肉
\\	これ
\\	はい、そうです。
\\	日本
\\	お菓子
\\	ようかん
\\	野菜
\\	魚
\\	日本の
\\	これは…おかしですか。	
\\	これは…おかしですか。 
\\	はい、そうです。日本のおかしです。	
\\	はい、そうです。日本のおかしです。 
\\	そうです。ようかんです。	
\\	"そうです。ようかんです。 
\\	よ・う・か・ん?	
\\	よ・う・か・ん? 
\\	家
\\	部屋
\\	犬
\\	おじゃまします
\\	そうです
\\	どうぞ
\\	車
\\	猫
\\	お上がりください
\\	これはテイラーさんのいえですか?	
\\	これはテイラーさんのいえですか? 
\\	そうです。どうぞ。	
\\	そうです。どうぞ。 
\\	おじゃまします。	
\\	おじゃまします。 
\\	あ、これは、テイラーさんのいぬですか?	
\\	あ、これは、テイラーさんのいぬですか? 
\\	はい。
\\	です。	
\\	はい。
\\	です。 
\\	これ
\\	それ
\\	いただきます。
\\	おいしい
\\	あれ
\\	すごい
\\	なんですか。
\\	はい、どうぞ!	
\\	はい、どうぞ! 
\\	うわ~ すごーい!	
\\	うわ~ すごーい! 
\\	いただきます。	
\\	いただきます。 
\\	うーん。おいしい!	
\\	うーん。おいしい! 
\\	テイラーさん、これは、なんですか?	
\\	テイラーさん、これは、なんですか? 
\\	大丈夫
\\	ミートローフ
\\	肉
\\	大丈夫ですか
\\	これはなんですか?	
\\	これはなんですか? 
\\	それはミートローフです。	
\\	それはミートローフです。 
\\	ミートローフ・・・?	
\\	ミートローフ・・・? 
\\	はい。にくです。	
\\	はい。にくです。 
\\	・・・にくはだいじょうぶですか?	
\\	・・・にくはだいじょうぶですか? 
\\	はい!
\\	うーん。おいしい!	
\\	はい!
\\	うーん。おいしい! 
\\	あそこ
\\	すみません
\\	どうも
\\	どこ
\\	トイレ
\\	ごちそうさまでした!	
\\	ごちそうさまでした! 
\\	あの・・・すみません。トイレはどこですか?	
\\	あの・・・すみません。トイレはどこですか? 
\\	トイレはあそこです。	
\\	トイレはあそこです。 
\\	あそこ?どうも。
\\	あそこ?どうも。
\\	本当に
\\	でも
\\	大丈夫ですか
\\	大丈夫
\\	グラス
\\	かおりさん、ほんとうにありがとうございます。	
\\	かおりさん、ほんとうにありがとうございます。 
\\	いえいえ。あ!
\\	すみません!	
\\	いえいえ。あ!
\\	すみません! 
\\	だいじょうぶですか?	
\\	だいじょうぶですか? 
\\	わたしはだいじょうぶです。でも…グラス…。	
\\	わたしはだいじょうぶです。でも…グラス…。 
\\	だいじょうぶです。	
\\	だいじょうぶです。 
\\	ほんとうにすみません…。	
\\	ほんとうにすみません…。 
\\	夏
\\	暑い
\\	むし暑い
\\	日本
\\	あー、あつい。	
\\	あー、あつい。 
\\	カリフォルニアのなつはあついですね。	
\\	カリフォルニアのなつはあついですね。 
\\	日本のなつもあついですか。	
\\	日本のなつもあついですか。 
\\	はい。	
\\	はい。 
\\	日本のなつは、むしあついです。	
\\	"日本のなつは、むしあついです。 
\\	む・し・あ・つ・い?	
\\	"む・し・あ・つ・い? 
\\	家族
\\	アルバム
\\	誰
\\	妹
\\	かわいい
\\	若い
\\	僕の
\\	えっと
\\	これは、ぼくのかぞくのアルバムです。	
\\	これは、ぼくのかぞくのアルバムです。 
\\	へー
\\	へー
\\	これは、だれですか?	
\\	これは、だれですか? 
\\	ぼくのいもうとです。	
\\	ぼくのいもうとです。 
\\	へー。かわいいー。	
\\	へー。かわいいー。 
\\	これは?	
\\	これは? 
\\	あ、えっと...ぼくです。	
\\	あ、えっと...ぼくです。 
\\	え!わかい!!	
\\	え!わかい!! 
\\	どこ
\\	こっち
\\	才
\\	ごめん
\\	何才
\\	うるさい
\\	ねぇ、テイラーさん、サラちゃんはどこですか?	
\\	ねぇ、テイラーさん、サラちゃんはどこですか? 
\\	こっちです。	
\\	こっちです。 
\\	あ、かわいい~。	
\\	あ、かわいい~。 
\\	ちいさいですね。なんさいですか?	
\\	ちいさいですね。なんさいですか? 
\\	1さいです。	
\\	1さいです。 
\\	ごめんごめん!うるさい?ごめんね~。	
\\	ごめんごめん!うるさい?ごめんね~。 
\\	大好き
\\	嫌い
\\	男の人
\\	いないいないばあ
\\	いないいない・・・ばぁ!いないいない・・・ばぁ!	
\\	いないいない・・・ばぁ!いないいない・・・ばぁ! 
\\	サラは、「いないいないばあ」がだいすきです。	
\\	"サラは、「いないいないばあ」がだいすきです。 
\\	へー。じゃ、いないいない。。。ばあ!	
\\	へー。じゃ、いないいない。。。ばあ! 
\\	チキン
\\	好き
\\	ドライブスルー
\\	お腹がすきました
\\	アレルギー
\\	嫌い
\\	あそこ
\\	おなかがすきました。
\\	おなかがすきました。
\\	あ、あそこに、ドライブスルーがあります。	
\\	あ、あそこに、ドライブスルーがあります。 
\\	フライドチキン、すきですか?	
\\	フライドチキン、すきですか? 
\\	フライドチキン・・・。	
\\	フライドチキン・・・。 
\\	きらいですか?	
\\	きらいですか? 
\\	アレルギーがあります。	
\\	アレルギーがあります。 
\\	チキンアレルギー?	
\\	チキンアレルギー? 
\\	ここ
\\	ゴリラ
\\	います
\\	ぼく
\\	ライオン
\\	動物
\\	あそこ
\\	かわいい
\\	どこ
\\	赤ちゃん
\\	どうぶつ…いますか?	
\\	どうぶつ…いますか? 
\\	…あ、あそこ!ライオンがいます。	
\\	…あ、あそこ!ライオンがいます。 
\\	あー!ライオンのあかちゃんもいますね。	
\\	あー!ライオンのあかちゃんもいますね。 
\\	かわいい!	
\\	かわいい! 
\\	あ、ゴリラがいます。	
\\	あ、ゴリラがいます。 
\\	え?どこ?	
\\	え?どこ? 
\\	ここです。	
\\	ここです。 
\\	え?ぼく? 
\\	え?ぼく? 
\\	お腹
\\	どうぞ
\\	バナナ
\\	食べます
\\	大好き
\\	じゃあ 
\\	じゃ
\\	お腹がすきました
\\	そうですか。
\\	あ、サラちゃん、おなかがすきましたか?	
\\	あ、サラちゃん、おなかがすきましたか? 
\\	テイラーさん、サラちゃんはバナナをたべますか。	
\\	テイラーさん、サラちゃんはバナナをたべますか。 
\\	はい。たべます。サラはバナナがだいすきです。	
\\	はい。たべます。サラはバナナがだいすきです。 
\\	そうですか。はい、サラちゃん。どうぞ。	
\\	そうですか。はい、サラちゃん。どうぞ。 
\\	テイラーさん、マトリーはバナナをたべますか。	
\\	テイラーさん、マトリーはバナナをたべますか。 
\\	何
\\	飲みます
\\	ワイン
\\	ビール
\\	ちょっと
\\	買います
\\	まさとさんは、なにをのみますか。	
\\	まさとさんは、なにをのみますか。 
\\	ワインですか?ビールですか。	
\\	ワインですか?ビールですか。 
\\	ぼくは、ビールをのみます。	
\\	ぼくは、ビールをのみます。 
\\	じゃ、ビールをかいます。	
\\	じゃ、ビールをかいます。 
\\	かおりさんは?かおりさんもビールをのみますか?	
\\	かおりさんは?かおりさんもビールをのみますか? 
\\	ええ。ちょっと。	
\\	ええ。ちょっと。 
\\	ギャンブル
\\	今夜
\\	来ます
\\	行きます
\\	カジノ
\\	いってらっしゃい。
\\	テイラーさん、あれはなんですか?	
\\	テイラーさん、あれはなんですか? 
\\	ああ、あれは、カジノです。	
\\	ああ、あれは、カジノです。 
\\	ギャンブルがすきですか。	
\\	ギャンブルがすきですか。 
\\	ええ。すきですよ。	
\\	ええ。すきですよ。 
\\	じゃ、こんや、カジノにいきますか。	
\\	じゃ、こんや、カジノにいきますか。 
\\	いいですね。	
\\	いいですね。 
\\	かおりさん、こんや、ぼくとまさとさんはカジノにいきます。	
\\	かおりさん、こんや、ぼくとまさとさんはカジノにいきます。 
\\	かおりさんもきますか?	
\\	かおりさんもきますか? 
\\	いってらっしゃい!	
\\	いってらっしゃい! 
\\	おいしい
\\	日本語
\\	待ってください。
\\	からい
\\	とうがらし
\\	英語
\\	うーん。おいしい。 
\\	からい!	
\\	うーん。おいしい。 
\\	からい! 
\\	それはハラペーニョです。	
\\	それはハラペーニョです。 
\\	ハラペーニョ?ハラペーニョはにほんごでなんですか?	
\\	"ハラペーニョ?ハラペーニョはにほんごでなんですか? 
\\	ちょっとまってください。	
\\	ちょっとまってください。 
\\	ああ、ハラペーニョはにほんごでとうがらしです。	
\\	ああ、ハラペーニョはにほんごでとうがらしです。 
\\	ああー。とうがらし・・・。	
\\	ああー。とうがらし・・・。 
\\	本当
\\	来年
\\	妹
\\	メキシコ
\\	6月
\\	やった
\\	いい
\\	結婚式
\\	おめでとうございます
\\	行きます
\\	らいねん、メキシコにいきます。	
\\	らいねん、メキシコにいきます。 
\\	どうして?	
\\	どうして? 
\\	ぼくのいもうとのけっこんしきがあります。	
\\	ぼくのいもうとのけっこんしきがあります。 
\\	おめでとうございます。	
\\	おめでとうございます。 
\\	いいですね。いついきますか。	
\\	いいですね。いついきますか。 
\\	6がつです。	
\\	6がつです。 
\\	へー。ちえみさんもいきますか?	
\\	へー。ちえみさんもいきますか? 
\\	はい!サラもいきます。マトリーもいきます。	
\\	はい!サラもいきます。マトリーもいきます。 
\\	マトリーも?すごい。	
\\	マトリーも?すごい。 
\\	そのあと、にほんにいきます。	
\\	そのあと、にほんにいきます。 
\\	ほんとう?やった!	
\\	ほんとう?やった! 
\\	おめでとう。
\\	あけましておめでとう。
\\	あけましておめでとうございます。
\\	おめでとうございます。
\\	よろしく おねがいします
\\	ちゃん
\\	くん
\\	ことしも よろしく おねがいします。
\\	ことしも よろしく。
\\	も
\\	ことし
\\	おとうさん
\\	ぼく
\\	わたし
\\	かいしゃいん
\\	メキシコ
\\	はじめまして。
\\	がくせい
\\	しゅふ
\\	こんにちは。
\\	メキシコじん
\\	じん
\\	ほんとうに
\\	そうですね。
\\	きょう
\\	さむい
\\	せんせい
\\	ね
\\	さようなら。
\\	ごぜん
\\	6じ
\\	じ
\\	いま
\\	こんばんは。
\\	なんじ
\\	おはようございます。
\\	はい
\\	こんばんは、テイラーさん。	
\\	こんばんは、テイラーさん。 
\\	かおりさん!おはよう ございます。	
\\	かおりさん!おはよう ございます。 
\\	いま、なんじ ですか。	
\\	いま、なんじ ですか。 
\\	6じ です。	
\\	6じ です。 
\\	ごぜん?	
\\	ごぜん? 
\\	はい。ごぜん 6じ です。	
\\	はい。ごぜん 6じ です。 
\\	と
\\	げんきですか。
\\	げんき
\\	も
\\	おげんきですか。
\\	ドル
\\	たかい
\\	えん
\\	78えん
\\	いま
\\	そうですか。
\\	ほんとうに
\\	いくら
\\	1ドル
\\	いま、えん は ほんとうに たかい です...。	
\\	いま、えん は ほんとうに たかい です...。 
\\	そう です か?1ドル は いくら です か。	
\\	そう です か?1ドル は いくら です か。 
\\	1ドル は 78えん です。	
\\	1ドル は 78えん です。 
\\	78えん?たかい!	
\\	78えん?たかい! 
\\	いもうと
\\	けっこんしき
\\	そのあと
\\	6がつ
\\	いつ
\\	にほん
\\	きます
\\	メキシコ
\\	いつ にほん に きますか。	
\\	いつ にほん に きますか。 
\\	6がつです。メキシコ と にほん に いきます。	
\\	6がつです。メキシコ と にほん に いきます。 
\\	メキシコ?	
\\	メキシコ? 
\\	はい。いもうと の けっこんしき に いきます。	
\\	はい。いもうと の けっこんしき に いきます。 
\\	そのあと、にほん に いきます。	
\\	そのあと、にほん に いきます。 
\\	あー!	
\\	あー! 
\\	メキシコの
\\	おねがいします
\\	これ
\\	なん
\\	それ
\\	パスポート
\\	なんですか。
\\	テキーラ
\\	おみやげ
\\	アルコール
\\	つぎの
\\	すみません。
\\	でんしゃ
\\	なんじ
\\	はん
\\	タクシー
\\	じ
\\	どこ
\\	あーあ。	
\\	あーあ。 
\\	すみません。	
\\	すみません。 
\\	つぎ の でんしゃ は なんじ です か。	
\\	つぎ の でんしゃ は なんじ です か。 
\\	2じ はん です。	
\\	2じ はん です。 
\\	2じ はん?	
\\	2じ はん? 
\\	すみません、いま なんじ です か。	
\\	すみません、いま なんじ です か。 
\\	1じ はん です。	
\\	1じ はん です。 
\\	すみません。	
\\	すみません。 
\\	タクシーは どこですか。	
\\	タクシーは どこですか。 
\\	ケチャップ
\\	と
\\	あります
\\	おねがいします
\\	すみません。
\\	これ
\\	えん
\\	ありがとうございます。
\\	これ と これ と これ、おねがいします。	
\\	これ と これ と これ、おねがいします。 
\\	852えん です。	
\\	852えん です。 
\\	はい。	
\\	はい。 
\\	すみません。2えん、ありますか。	
\\	すみません。2えん、ありますか。 
\\	はい。	
\\	はい。 
\\	ありがとう ございます。	
\\	ありがとう ございます。 
\\	すみません。ケチャップ、ありますか。	
\\	すみません。ケチャップ、ありますか。 
\\	えん
\\	あります
\\	おつり
\\	いいえ。
\\	1660えんです。	
\\	1660えんです。 
\\	はい。2000えん。	
\\	はい。2000えん。 
\\	60えん、ありますか。	
\\	60えん、ありますか。 
\\	いいえ、ありません。	
\\	いいえ、ありません。 
\\	おつり、340えんです。	
\\	おつり、340えんです。 
\\	テイラーさん?まさとです!	
\\	テイラーさん?まさとです! 
\\	だいじょうぶです
\\	どこ
\\	いま
\\	もしもし
\\	だいじょうぶ
\\	に
\\	います
\\	マクドナルド
\\	どこ
\\	わかりました
\\	かい
\\	あそこ
\\	2かい
\\	マクドナルド
\\	エスカレーター
\\	はい、マクドナルド です。	
\\	はい、マクドナルド です。 
\\	マクドナルド は どこ に ありますか。	
\\	マクドナルド は どこ に ありますか。 
\\	マクドナルド は 2かい に あります。	
\\	マクドナルド は 2かい に あります。 
\\	2かい…。わかりました。	
\\	2かい…。わかりました。 
\\	すみません。エスカレーター は どこ に ありますか?	
\\	すみません。エスカレーター は どこ に ありますか? 
\\	エスカレーター?あそこ に あります。	
\\	エスカレーター?あそこ に あります。 
\\	カリフォルニア
\\	ざんねん
\\	おひさしぶりです。
\\	ここ
\\	ひさしぶり
\\	ようこそ
\\	あれ?
\\	くるま
\\	なに
\\	あるき
\\	うち
\\	でんしゃ
\\	とおい
\\	バス
\\	いきます
\\	いらっしゃい
\\	どうぞ
\\	くつ
\\	スリッパ
\\	おじゃまします
\\	ぬぎます
\\	みんな
\\	にく
\\	たべません
\\	いただきます。
\\	あれ?
\\	ちょっと
\\	ベジタリアン
\\	たべます
\\	きらい
\\	ビーガン
\\	きのう
\\	いきます
\\	メール
\\	よろしく
\\	げんき
\\	あした
\\	よみます
\\	みんな
\\	おかあさん
\\	きます
\\	かいました
\\	いきました
\\	きょう
\\	でも
\\	かいます
\\	いろいろな
\\	とうきょうゲームショウ
\\	ゲーム
\\	します
\\	モバイルゲーム
\\	もういちど
\\	すき(な)
\\	あの・・・
\\	おねがいします
\\	いちど
\\	おたく
\\	もう
\\	ちょっと
\\	だいすき
\\	します
\\	やすみます
\\	おさけ
\\	のみます
\\	おみやげ
\\	つかれます
\\	オッケイ
\\	メキシコの
\\	いただきます。
\\	テキーラ
\\	あれ?
\\	ふじさん
\\	だいじょうぶ
\\	だめ
\\	あし
\\	さる
\\	あぶない
\\	かお
\\	います
\\	いたい
\\	よ
\\	たのしい
\\	ハイキング
\\	でも
\\	ただいま。
\\	へんな
\\	さむい
\\	あたたかい
\\	かお
\\	おかえりなさい。
\\	つかれます
\\	ありがとうございました。
\\	ひこうき
\\	また
\\	おせわになりました。
\\	またね。
\\	おげんきで。
\\	たのしい
\\	あいます
\\	ほんとうに
\\	メール
\\	ひこうき は なんじ です か。	
\\	ひこうき は なんじ です か。 
\\	5じ はん です。	
\\	5じ はん です。 
\\	まさとさん、かおりさん、おせわ に なりました。	
\\	まさとさん、かおりさん、おせわ に なりました。 
\\	ほんとう に、ありがとう ございました。	
\\	ほんとう に、ありがとう ございました。 
\\	いいえ。たのしかったです。	
\\	いいえ。たのしかったです。 
\\	サラちゃん、またね。	
\\	サラちゃん、またね。 
\\	また、メール します。	
\\	また、メール します。 
\\	はい。また、あいましょう。	
\\	はい。また、あいましょう。 
\\	おげんきで。	
\\	おげんきで。 
\\	建国記念の日
\\	敬老の日
\\	成人の日
\\	勤労感謝の日
\\	こどもの日
\\	餅
\\	豆腐
\\	納豆
\\	おにぎり
\\	たい焼き
\\	緑茶
\\	アイスコーヒー
\\	日本酒
\\	ラムネ
\\	チューハイ
\\	みそ汁
\\	ハンバーグ
\\	カレーライス
\\	天ぷら
\\	なべ
\\	だんご
\\	ようかん
\\	桜餅
\\	ラムネ
\\	プリン
\\	りんご
\\	すいか
\\	さくらんぼ
\\	柿
\\	みかん
\\	きゅうり
\\	玉ねぎ
\\	ピーマン
\\	なすび
\\	にんじん
\\	書道
\\	初詣
\\	祭り
\\	俳句
\\	お盆
\\	犬
\\	うさぎ
\\	猫
\\	亀
\\	猿
\\	桜
\\	松
\\	菊
\\	あじさい
\\	イチョウ
\\	山
\\	海
\\	島
\\	川
\\	火山
\\	東京
\\	長崎
\\	大阪
\\	札幌
\\	広島
\\	野球
\\	バレーボール
\\	サッカー
\\	相撲
\\	柔道
\\	温泉
\\	買い物
\\	遊園地
\\	読書
\\	映画
\\	金閣寺
\\	富士山
\\	東大寺
\\	広島平和記念公園
\\	秋葉原
\\	仏壇
\\	畳
\\	神棚
\\	こたつ
\\	玄関
\\	正露丸
\\	湿布
\\	ムヒ
\\	風邪薬
\\	頭痛薬
\\	電車
\\	飛行機
\\	バス
\\	新幹線
\\	地下鉄
\\	コンビニ
\\	居酒屋
\\	リフレクソロジー
\\	カラオケ
\\	まんが喫茶
\\	ダイソー
\\	マツモトキヨシ
\\	ドン・キホーテ
\\	セブンイレブン
\\	ツタヤ
\\	梅干し
\\	ようかん
\\	抹茶
\\	餅
\\	せんべい
\\	聖徳太子
\\	織田信長
\\	徳川家光
\\	坂本竜馬
\\	伊藤博文
\\	小澤征爾
\\	坂本九
\\	美空ひばり
\\	氷川きよし
\\	嵐
\\	樋口一葉
\\	紫式部
\\	川端康成
\\	村上春樹
\\	大江健三郎
\\	すみません
\\	そうです
\\	初めまして
\\	です
\\	どうぞよろしく
\\	すみません。山口さんですか。	
\\	すみません。山口さんですか。 
\\	はい、そうです。	
\\	はい、そうです。 
\\	初めまして。渡辺です。	
\\	初めまして。渡辺です。 
\\	初めまして。山口です。	
\\	初めまして。山口です。 
\\	どうぞよろしく。	
\\	どうぞよろしく。 
\\	どうぞよろしく。	
\\	どうぞよろしく。 
\\	先生
\\	朝
\\	こんばんは。
\\	失礼しました
\\	夜
\\	ニューヨーク
\\	おはよう
\\	おはようございます
\\	渡辺
\\	昼
\\	渡辺先生おはようございます。	
\\	渡辺先生おはようございます。 
\\	おはよう、山口さん。	
\\	おはよう、山口さん。 
\\	渡辺先生、ニューヨークは夜です。	
\\	渡辺先生、ニューヨークは夜です。 
\\	あっ、失礼しました。こんばんは。	
\\	あっ、失礼しました。こんばんは。 
\\	お久しぶりです
\\	そうですね。
\\	しばらくですね
\\	お元気ですか。
\\	はい、元気ですよ
\\	おかげさまで、 元気です
\\	お久しぶりです。	
\\	お久しぶりです。 
\\	そうですね。	
\\	そうですね。 
\\	お元気ですか?	
\\	お元気ですか? 
\\	はい、元気ですよ。	
\\	はい、元気ですよ。 
\\	こちら
\\	学校
\\	の
\\	友達
\\	ジェニー
\\	学生
\\	わかりません
\\	もう一度お願いします
\\	ゆっくりお願いします
\\	先生、	
\\	先生、 
\\	こちらは学校の友達のジェニーです。	
\\	こちらは学校の友達のジェニーです。 
\\	初めまして。	
\\	初めまして。 
\\	ジェニーです。	
\\	ジェニーです。 
\\	どうぞよろしくお願いします。	
\\	どうぞよろしくお願いします。 
\\	初めまして。	
\\	初めまして。 
\\	あけましておめでとうございます。
\\	今年もどうぞよろしくお願いします。
\\	あけおめ。
\\	ことよろ。
\\	聞く
\\	引く
\\	そうです
\\	よくできました
\\	頑張ってください
\\	ちがいます
\\	違います
\\	皆さん
\\	皆さん聞いてください。	
\\	皆さん聞いてください。 
\\	はい〜。	
\\	はい〜。 
\\	8引く5は、清さん?	
\\	8引く5は、清さん? 
\\	4です。	
\\	4です。 
\\	違います。智子さん。	
\\	違います。智子さん。 
\\	3です。	
\\	3です。 
\\	はい、そうです。次は6引く4は、清さん?	
\\	はい、そうです。次は6引く4は、清さん? 
\\	1です。	
\\	1です。 
\\	違います。智子さん。	
\\	違います。智子さん。 
\\	2です。	
\\	2です。 
\\	はい、よくできました。じゃあ、清さん9引く7は?	
\\	はい、よくできました。じゃあ、清さん9引く7は? 
\\	3です。	
\\	3です。 
\\	じゃあ、頑張ってくださいね。	
\\	じゃあ、頑張ってくださいね。 
\\	はい、頑張ります。	
\\	はい、頑張ります。 
\\	見る
\\	黒板
\\	違います
\\	たす
\\	見てください
\\	う〜ん
\\	よくできました
\\	頑張ってください
\\	皆さん、黒板を見てください。	
\\	皆さん、黒板を見てください。 
\\	はい~。	
\\	はい~。 
\\	清さん、8たす5は?	
\\	清さん、8たす5は? 
\\	13です。	
\\	13です。 
\\	はい、そうです。	
\\	はい、そうです。 
\\	次は智子さん、7たす4は?	
\\	次は智子さん、7たす4は? 
\\	15です。	
\\	15です。 
\\	う~ん、違います。清さん。	
\\	う~ん、違います。清さん。 
\\	11です。	
\\	11です。 
\\	そうですね。よくできました
\\	そうですね。よくできました
\\	じゃあ、智子さん、9たす1は?	
\\	じゃあ、智子さん、9たす1は? 
\\	う~んと.....20です。	
\\	う~んと.....20です。 
\\	う~ん、もう少しですね。	
\\	う~ん、もう少しですね。 
\\	はい!はい!はい!	
\\	はい!はい!はい! 
\\	はい清さん。	
\\	はい清さん。 
\\	10です。	
\\	10です。 
\\	よくできました。	
\\	よくできました。 
\\	智子さんも頑張ってくださいね。	
\\	智子さんも頑張ってくださいね。 
\\	今日
\\	かけ算
\\	問題
\\	問題を言ってください
\\	言う
\\	かける
\\	先生
\\	失礼しました
\\	頑張ってください
\\	今日はかけ算です。	
\\	今日はかけ算です。 
\\	清さん、問題を言ってください。	
\\	清さん、問題を言ってください。 
\\	は〜い。7かける6は?	
\\	は〜い。7かける6は? 
\\	42です。	
\\	42です。 
\\	はい、そうですね。	
\\	はい、そうですね。 
\\	次は智子さん、問題を言ってください。	
\\	次は智子さん、問題を言ってください。 
\\	12かける6は?	
\\	12かける6は? 
\\	72です。	
\\	72です。 
\\	違います。	
\\	違います。 
\\	60です。	
\\	60です。 
\\	ええー。先生違います!	
\\	ええー。先生違います! 
\\	あら!そうですね。失礼しました。72ですね。	
\\	あら!そうですね。失礼しました。72ですね。 
\\	先生、頑張ってください!	
\\	先生、頑張ってください! 
\\	時
\\	お久しぶり
\\	午前
\\	から
\\	今
\\	午後
\\	もしもし
\\	何
\\	もしもし健二さん、美智子です。お久しぶり!元気?	
\\	もしもし健二さん、美智子です。お久しぶり!元気? 
\\	ああー、美智子.....何?.....今午前3時だよ。	
\\	ああー、美智子.....何?.....今午前3時だよ。 
\\	えっ、今日本は午後3時。	
\\	えっ、今日本は午後3時。 
\\	でも、ワシントン
\\	は今午前3時。	
\\	でも、ワシントン
\\	は今午前3時。 
\\	ワシントン
\\	いつから??	
\\	ワシントン
\\	いつから??  
\\	眠い
\\	昨日
\\	わくわく
\\	食べ物
\\	協会
\\	おやすみなさい。
\\	(びっくり声で)いつからワシントン
\\	にいるの?	
\\	(びっくり声で)いつからワシントン
\\	にいるの? 
\\	(眠い声で)ああー、昨日から。	
\\	(眠い声で)ああー、昨日から。 
\\	(わくわくの声)えっ、ワシントン
\\	はどう?食べ物はどう?	
\\	(わくわくの声)えっ、ワシントン
\\	はどう?食べ物はどう? 
\\	ああ〜。眠い〜。	
\\	ああ〜。眠い〜。 
\\	伸び
\\	眠い
\\	おなかがすく
\\	まだ
\\	寝る
\\	(伸びをする声)ウーン、おはようございます!	
\\	(伸びをする声)ウーン、おはようございます! 
\\	(眠い声で)なんだよ。	
\\	(眠い声で)なんだよ。 
\\	おなかがすいてきました。	
\\	おなかがすいてきました。 
\\	おなかがすいた?	
\\	おなかがすいた? 
\\	え~と、2月ですね。	
\\	え~と、2月ですね。 
\\	まだ2月!	
\\	まだ2月! 
\\	子供
\\	味
\\	お腹が空く
\\	美味しい
\\	まずい
\\	ただいま!	
\\	ただいま! 
\\	おかえりなさい。	
\\	おかえりなさい。 
\\	あー、おなかがすいた。	
\\	あー、おなかがすいた。 
\\	じゃあ、このクッキーをどうぞ。	
\\	じゃあ、このクッキーをどうぞ。 
\\	ワア〜おいしい!	
\\	ワア〜おいしい! 
\\	スウェーデン語
\\	べんきょう
\\	楽しい
\\	悪くない
\\	点
\\	難しい
\\	易しい
\\	スウェーデン語の勉強は楽しいですか。	
\\	スウェーデン語の勉強は楽しいですか。 
\\	は〜い。楽しいです。	
\\	は〜い。楽しいです。 
\\	でも、むずかしいです。	
\\	でも、むずかしいです。 
\\	えっ、むずかしくないよ。やさしいよ。	
\\	えっ、むずかしくないよ。やさしいよ。 
\\	えー?やさしくないよ〜。むずかしいよ。	
\\	えー?やさしくないよ〜。むずかしいよ。 
\\	テストはいつも80点です。	
\\	テストはいつも80点です。 
\\	そ、そ、そんな....80点は悪くないよ。	
\\	そ、そ、そんな....80点は悪くないよ。 
\\	私はいつも60点です。	
\\	私はいつも60点です。 
\\	えー!それはよくないね。もっと頑張ってね。	
\\	えー!それはよくないね。もっと頑張ってね。 
\\	誕生日
\\	そして
\\	見る
\\	歌舞伎
\\	午後
\\	更に
\\	旦那
\\	どこ
\\	誰
\\	朝ご飯
\\	遅い
\\	かなり
\\	おめでとうございます
\\	フランス料理
\\	誕生日おめでとうございます。	
\\	誕生日おめでとうございます。 
\\	おはよう!ありがとう。今何時?	
\\	おはよう!ありがとう。今何時? 
\\	9時45分でございます。かなり遅いですよ。	
\\	9時45分でございます。かなり遅いですよ。 
\\	朝ご飯は10時からですよ。私が作りますよ。	
\\	朝ご飯は10時からですよ。私が作りますよ。 
\\	朝ご飯を作るの?どういう風の吹き回し!	
\\	朝ご飯を作るの?どういう風の吹き回し! 
\\	更に、午後の一時に私たちは歌舞伎を見ます。	
\\	更に、午後の一時に私たちは歌舞伎を見ます。 
\\	電話
\\	自動販売機
\\	辺
\\	隣
\\	公衆
\\	あの~、すみません。	
\\	あの~、すみません。 
\\	はい。	
\\	はい。 
\\	この辺に公衆電話はありますか。	
\\	この辺に公衆電話はありますか。 
\\	公衆電話ですか?	
\\	公衆電話ですか? 
\\	ええ。	
\\	ええ。 
\\	えーっと…ああ、コンビニの前に自動販売機がありますね。	
\\	えーっと…ああ、コンビニの前に自動販売機がありますね。 
\\	公衆電話はその隣にありますよ。	
\\	公衆電話はその隣にありますよ。 
\\	ありがとうございます。	
\\	ありがとうございます。 
\\	何時
\\	電話
\\	失礼
\\	どこ
\\	まだ
\\	今何時だ?	
\\	今何時だ? 
\\	あの子はまだか。	
\\	あの子はまだか。 
\\	まだ?藤原さんに電話するわ。。。	
\\	まだ?藤原さんに電話するわ。。。 
\\	もしもし藤原です。	
\\	もしもし藤原です。 
\\	高橋ですが、すみません。智子はそちらにいますか。	
\\	高橋ですが、すみません。智子はそちらにいますか。 
\\	いいえ。こちらにはいませんよ。	
\\	いいえ。こちらにはいませんよ。 
\\	ありがとうございます。	
\\	ありがとうございます。 
\\	はい、失礼します。	
\\	はい、失礼します。 
\\	あの子はどこ?	
\\	あの子はどこ? 
\\	教える
\\	むすめ
\\	何時まで
\\	図書館
\\	わかった。(電話をかける音)	
\\	わかった。(電話をかける音) 
\\	何?教えて。	
\\	何?教えて。 
\\	もしもし、高橋ですが、すみません。むすめの高橋智子はそちらにいますか。	
\\	もしもし、高橋ですが、すみません。むすめの高橋智子はそちらにいますか。 
\\	高橋様、どうもこんばんは。智子ちゃんはいますよ。	
\\	高橋様、どうもこんばんは。智子ちゃんはいますよ。 
\\	良かったです。図書館は何時までですか。	
\\	良かったです。図書館は何時までですか。 
\\	11時までです。	
\\	11時までです。 
\\	はい、ありがとうございます。	
\\	はい、ありがとうございます。 
\\	大丈夫
\\	感動的
\\	本
\\	いらっしゃいませ
\\	すみません
\\	ティッシュ
\\	優しい
\\	お礼
\\	いらっしゃいませ!	
\\	いらっしゃいませ! 
\\	客:あれっ、すみません。大丈夫ですか。	
\\	(泣きながら)はい、大丈夫です。	
\\	(泣きながら)はい、大丈夫です。 
\\	どうしたんですか。	
\\	どうしたんですか。 
\\	(泣きながら)この本。。。	
\\	(泣きながら)この本。。。 
\\	その本ですか。	
\\	その本ですか。 
\\	はい、そうです。この本は感動的です。(泣きながら)	
\\	はい、そうです。この本は感動的です。(泣きながら) 
\\	あっ、ティッシュどうぞ。	
\\	あっ、ティッシュどうぞ。 
\\	ありがとう。おやさしいんですね。お礼にこの本をどうぞ。	
\\	ありがとう。おやさしいんですね。お礼にこの本をどうぞ。 
\\	いいんですか。	
\\	いいんですか。 
\\	はい。	
\\	はい。 
\\	ええ~?ももたろう?	
\\	ええ~?ももたろう? 
\\	はい、感動なんです。	
\\	はい、感動なんです。 
\\	部屋
\\	机
\\	椅子
\\	パソコン
\\	ゴキブリ
\\	時々
\\	小林さんの部屋には何がありますか。	
\\	小林さんの部屋には何がありますか。 
\\	えーっと、机と、椅子と、ベッドと、パソコンと・・・それにテレビもあります。	
\\	えーっと、机と、椅子と、ベッドと、パソコンと・・・それにテレビもあります。 
\\	それだけですか。じゃあ、小林さんの部屋には何がいますか。	
\\	それだけですか。じゃあ、小林さんの部屋には何がいますか。 
\\	えーっと、何もいません。あっそうそう、時々ゴキブリがいます。	
\\	えーっと、何もいません。あっそうそう、時々ゴキブリがいます。 
\\	今週
\\	休み
\\	金曜日
\\	木曜日
\\	ゴールデンウィーク
\\	テスト
\\	試験
\\	火曜日
\\	予定
\\	待ち遠しい
\\	すみません。竹内先生、今週の予定は何ですか。	
\\	すみません。竹内先生、今週の予定は何ですか。 
\\	そうですね。今日は火曜日ですね。	
\\	そうですね。今日は火曜日ですね。 
\\	はい、そうです。	
\\	はい、そうです。 
\\	えと、今日は試験があります。明日はテストがあります。今週はゴールデンウィークですから、木曜日と金曜日は休みです。	
\\	えと、今日は試験があります。明日はテストがあります。今週はゴールデンウィークですから、木曜日と金曜日は休みです。 
\\	木曜日が、待ち遠しいです。	
\\	木曜日が、待ち遠しいです。 
\\	日本
\\	チーズバーガー
\\	大体
\\	焼肉
\\	くらい
\\	有名
\\	牛
\\	高い
\\	万
\\	鈴木さん、日本ではチーズバーガーはいくらですか。	
\\	鈴木さん、日本ではチーズバーガーはいくらですか。 
\\	そうですね、大体300円くらいですね。	
\\	そうですね、大体300円くらいですね。 
\\	焼肉の食べ放題は?	
\\	焼肉の食べ放題は? 
\\	そうですね、大体5,000円くらい。	
\\	そうですね、大体5,000円くらい。 
\\	あの有名な神戸牛は?	
\\	あの有名な神戸牛は? 
\\	高いですよ!大体1万円くらいです。	
\\	高いですよ!大体1万円くらいです。 
\\	デパート
\\	家具売場
\\	紳士服
\\	婦人服
\\	階
\\	(デパートで)すみません。家具はどこですか。	
\\	(デパートで)すみません。家具はどこですか。 
\\	家具売場は6階です。	
\\	家具売場は6階です。 
\\	すみません。紳士服はどこですか。	
\\	すみません。紳士服はどこですか。 
\\	はい、3階でございます。	
\\	はい、3階でございます。 
\\	すみません。婦人服は何階ですか。	
\\	すみません。婦人服は何階ですか。 
\\	はい、5階です。	
\\	はい、5階です。 
\\	病院
\\	昏睡状態
\\	どこ
\\	誰
\\	看護婦
\\	から
\\	いる
\\	去年
\\	なんで
\\	(病院で昏睡状態になっていた人がおきる。)起きた!	
\\	(病院で昏睡状態になっていた人がおきる。)起きた! 
\\	ここはどこですか。	
\\	ここはどこですか。 
\\	ここは東京病院ですよ。	
\\	ここは東京病院ですよ。 
\\	あなたは誰ですか。	
\\	あなたは誰ですか。 
\\	鈴木です。看護婦です。	
\\	鈴木です。看護婦です。 
\\	いつから私はここにいるんですか。	
\\	いつから私はここにいるんですか。 
\\	去年からですよ。	
\\	去年からですよ。 
\\	何でここにいるんですか。	
\\	何でここにいるんですか。 
\\	それは。。。	
\\	それは。。。 
\\	病院長
\\	お疲れ様
\\	ちょっと
\\	教える
\\	初めまして
\\	失礼
\\	鈴木さん、お疲れ様でした。ここまででいいです。	
\\	鈴木さん、お疲れ様でした。ここまででいいです。 
\\	わかりました。	
\\	わかりました。 
\\	鈴木さん、ちょっと待ってください。どこへ行くのですか。教えてください。	
\\	鈴木さん、ちょっと待ってください。どこへ行くのですか。教えてください。 
\\	失礼します。	
\\	失礼します。 
\\	あなたはだれですか。	
\\	あなたはだれですか。 
\\	初めまして。病院長の田中です。	
\\	初めまして。病院長の田中です。 
\\	初めまして。私は、私は、私は誰ですか。	
\\	初めまして。私は、私は、私は誰ですか。 
\\	昭和
\\	生まれ
\\	出身
\\	家族
\\	奥さん
\\	公安調査庁
\\	役人
\\	首相
\\	お名前は高山覚です。昭和45年の生まれです。群馬県の出身です。ご家族が3人います。	
\\	お名前は高山覚です。昭和45年の生まれです。群馬県の出身です。ご家族が3人います。 
\\	家族?	
\\	家族? 
\\	そうですよ。奥さんと娘さんがいます。	
\\	そうですよ。奥さんと娘さんがいます。 
\\	今二人はどこですか。教えてください。	
\\	今二人はどこですか。教えてください。 
\\	まず首相に電話します。ちょっと待ってください。	
\\	病院長
\\	忙しい
\\	首相
\\	大丈夫
\\	記憶喪失
\\	起きる
\\	申しわけありません
\\	もしもし、東京病院の田中です。お忙しいところ申しわけありません。	
\\	もしもし、東京病院の田中です。お忙しいところ申しわけありません。 
\\	田中さん、久しぶりですね。どうしました?	
\\	田中さん、久しぶりですね。どうしました? 
\\	高山さんのことですが、彼が起きました!	
\\	高山さんのことですが、彼が起きました! 
\\	まさか!彼は大丈夫ですか?!	
\\	まさか!彼は大丈夫ですか?! 
\\	大丈夫ですが、記憶喪失です。	
\\	大丈夫ですが、記憶喪失です。 
\\	記憶喪失ですか。。じゃ、こうしましょう…	
\\	記憶喪失ですか。。じゃ、こうしましょう… 
\\	お待たせしました
\\	どうぞ
\\	わかる
\\	名前
\\	二人
\\	本当
\\	電話する
\\	家族
\\	携帯電話
\\	ごゆっくり
\\	お待たせしました。はい、どうぞ。	
\\	お待たせしました。はい、どうぞ。 
\\	なんですか、それは。	
\\	なんですか、それは。 
\\	携帯電話です。どうぞ。	
\\	携帯電話です。どうぞ。 
\\	どういうことですか。	
\\	どういうことですか。 
\\	高山さんはご家族に電話します。	
\\	高山さんはご家族に電話します。 
\\	本当ですか。	
\\	本当ですか。 
\\	本当ですよ。	
\\	本当ですよ。 
\\	でも、二人の名前が、わかりません。	
\\	でも、二人の名前が、わかりません。 
\\	電話を打っている音	
\\	はい、高山です。。。。。もしもし	
\\	はい、高山です。。。。。もしもし 
\\	夏美?私だけど、	
\\	夏美?私だけど、 
\\	さとし!	
\\	さとし! 
\\	ようこそ
\\	ファッション コーディネーター
\\	イタリア人
\\	学生
\\	えっと
\\	カウンセラー
\\	こんにちは。
\\	ヘアスタイリスト
\\	大学院生
\\	わたし
\\	どうぞよろしく おねがいします
\\	社長
\\	ようこそ!スタイルユーです。よろしくおねがいします。私は、スタイルユーの社長、南夏見です。大学院生です。	
\\	ようこそ!スタイルユーです。よろしくおねがいします。私は、スタイルユーの社長、南夏見です。大学院生です。 
\\	しゅうです。ヘアスタイリストです。どうぞよろしく。	
\\	しゅうです。ヘアスタイリストです。どうぞよろしく。 
\\	こんにちは。わたしは、北川冬果です。わたしは、カウンセラーです。よろしくおねがいします。	
\\	こんにちは。わたしは、北川冬果です。わたしは、カウンセラーです。よろしくおねがいします。 
\\	えっと・・・東春人です。えっとえっと・・・学生です。	
\\	えっと・・・東春人です。えっとえっと・・・学生です。 
\\	ファブリツィオです。ファッションコーディネーターです。イタリア人です。どうぞよろしくおねがいします。	
\\	ファブリツィオです。ファッションコーディネーターです。イタリア人です。どうぞよろしくおねがいします。 
\\	おはようございます
\\	インターン
\\	さん
\\	アメリカ人
\\	はい
\\	インターンです。どうぞよろしくおねがいします。	
\\	あ、おはようございます。キム・・・ん?えっと・・・。	
\\	あ、おはようございます。キム・・・ん?えっと・・・。 
\\	ミ ヨンです。	
\\	ミ ヨンです。 
\\	ミ ヨンさんですか。	
\\	ミ ヨンさんですか。 
\\	はい。ミ ヨンです。	
\\	はい。ミ ヨンです。 
\\	アメリカ人ですか。	
\\	はい、そうです。	
\\	はい、そうです。 
\\	ベジタリアン
\\	と
\\	豆
\\	野菜
\\	とり肉
\\	とり
\\	サーモン
\\	魚
\\	肉
\\	ウエイトレス
\\	乾杯
\\	すみません。これは肉ですか。	
\\	すみません。これは肉ですか。 
\\	いいえ。肉ではありません。魚です。サーモンです。	
\\	いいえ。肉ではありません。魚です。サーモンです。 
\\	そうですか。えっと、これは肉ですか。	
\\	そうですか。えっと、これは肉ですか。 
\\	はい、そうです。とり肉です。	
\\	はい、そうです。とり肉です。 
\\	これは肉ですか。	
\\	これは肉ですか。 
\\	いえいえ、肉じゃないです。野菜と豆です。	
\\	いえいえ、肉じゃないです。野菜と豆です。 
\\	ファブリツィオさんは、ベジタリアンですか。	
\\	ファブリツィオさんは、ベジタリアンですか。 
\\	風邪
\\	多分
\\	それ
\\	これ
\\	駄目
\\	も
\\	煙草
\\	薬
\\	何
\\	ハクション
\\	しゅうさん、おはようございま・・・(ハクション!ゴホゴホ)。	
\\	しゅうさん、おはようございま・・・(ハクション!ゴホゴホ)。 
\\	風邪ですか、冬果さん。	
\\	風邪ですか、冬果さん。 
\\	たぶん。(ゴホゴホ!)	
\\	たぶん。(ゴホゴホ!) 
\\	それは何ですか。	
\\	それは何ですか。 
\\	え?これ?これは薬です。	
\\	え?これ?これは薬です。 
\\	それも薬ですか。	
\\	それも薬ですか。 
\\	あ・・・これはたばこです。	
\\	あ・・・これはたばこです。 
\\	たばこはだめです。冬果さん。	
\\	たばこはだめです。冬果さん。 
\\	コンピューター
\\	パスワード
\\	会社
\\	電話
\\	番号
\\	何
\\	-の
\\	メールアドレス
\\	すみません、東さん。コンピューターの パスワードは 何ですか?	
\\	すみません、東さん。コンピューターの パスワードは 何ですか? 
\\	コンピューターのパスワード?えっと、会社の電話番号です。	
\\	コンピューターのパスワード?えっと、会社の電話番号です。 
\\	え?	
\\	え? 
\\	コンピューターの パスワードは 会社の 電話番号です。	
\\	コンピューターの パスワードは 会社の 電話番号です。 
\\	会社の 電話番号は 何ですか。	
\\	会社の 電話番号は 何ですか。 
\\	03-345-6788です。	
\\	03-345-6788です。 
\\	ぜろ さん さん よん ご ろく なな はち はち・・・	
\\	ぜろ さん さん よん ご ろく なな はち はち・・・ 
\\	いくら
\\	お客さん
\\	ここ
\\	円
\\	ショップ
\\	下さい
\\	店員
\\	いらっしゃいませー。いらっしゃいませー。	
\\	いらっしゃいませー。いらっしゃいませー。 
\\	すみません、これはいくらですか。	
\\	すみません、これはいくらですか。 
\\	100円です。	
\\	100円です。 
\\	じゃこれはいくらですか。	
\\	じゃこれはいくらですか。 
\\	・・・お客さん、ここは100円ショップです。これも100円。それも100円。あれも100円です。	
\\	・・・お客さん、ここは100円ショップです。これも100円。それも100円。あれも100円です。 
\\	あ、そうですか。すみません。じゃ、これとこれを下さい。	
\\	あ、そうですか。すみません。じゃ、これとこれを下さい。 
\\	店員
\\	いらっしゃいませ
\\	ありがとうございます
\\	傘
\\	税込み
\\	下さい
\\	円
\\	いらっしゃいませ。	
\\	いらっしゃいませ。 
\\	すみません。あの傘は いくらですか。	
\\	すみません。あの傘は いくらですか。 
\\	000円でございます。	
\\	000円でございます。 
\\	この傘は いくらですか。	
\\	この傘は いくらですか。 
\\	000円でございます。	
\\	000円でございます。 
\\	じゃ、これを下さい。	
\\	じゃ、これを下さい。 
\\	ありがとうございます。税込み 29400円でございます。	
\\	ありがとうございます。税込み 29400円でございます。 
\\	雨
\\	イタリア
\\	デザイナー
\\	へえ
\\	すごい
\\	どこ
\\	どこの
\\	それはファブリツィオさんの傘ですか。	
\\	それはファブリツィオさんの傘ですか。 
\\	はい。イタリアの傘です。イタリアのデザイナーの傘です。	
\\	はい。イタリアの傘です。イタリアのデザイナーの傘です。 
\\	へぇー。すごい。	
\\	へぇー。すごい。 
\\	それはどこの傘ですか。	
\\	それはどこの傘ですか。 
\\	え?私の傘?100円ショップの傘です。	
\\	え?私の傘?100円ショップの傘です。 
\\	どうぞ
\\	ありがとうございます
\\	また
\\	バーニーズ ニューヨーク
\\	ごめん
\\	いくら
\\	お客さん
\\	傘です。どうぞ。	
\\	傘です。どうぞ。 
\\	ありがとうございます。じゃ、また。	
\\	ありがとうございます。じゃ、また。 
\\	雷	
\\	ない! 私の傘!バーニーズ ニューヨークの傘! ない!	
\\	ない! 私の傘!バーニーズ ニューヨークの傘! ない! 
\\	え・・・あれはファブリツィオの傘ですか? ごめん! い・・・いくらですか?	
\\	え・・・あれはファブリツィオの傘ですか? ごめん! い・・・いくらですか? 
\\	28000円!	
\\	28000円! 
\\	に まん はっせん えん?!	
\\	に まん はっせん えん?! 
\\	今日
\\	暑い
\\	でも
\\	とても
\\	涼しい
\\	寒い
\\	ちょうど
\\	いい
\\	暖かい
\\	おはようございます。	
\\	おはようございます。 
\\	おはようございます。冬果さん。今日は暑いですね。	
\\	おはようございます。冬果さん。今日は暑いですね。 
\\	そうですね。でも、ここはとても涼しいですね。(ハクション!)	
\\	そうですね。でも、ここはとても涼しいですね。(ハクション!) 
\\	寒いですか。	
\\	寒いですか。 
\\	寒くないです。ちょうどいいです。(ハクション!ハクション!)	
\\	寒くないです。ちょうどいいです。(ハクション!ハクション!) 
\\	静か(な)
\\	いつも
\\	特別(な)
\\	髪
\\	まっすぐ(な)
\\	きれい(な)
\\	だけ
\\	もちろん
\\	静かですね。	
\\	静かですね。 
\\	そうですね。	
\\	そうですね。 
\\	ここは、いつも静かですか。	
\\	ここは、いつも静かですか。 
\\	静かじゃないです。今日は特別です。冬果さんの髪はまっすぐですね。とてもきれいです。	
\\	静かじゃないです。今日は特別です。冬果さんの髪はまっすぐですね。とてもきれいです。 
\\	そうですか。髪だけですか。	
\\	そうですか。髪だけですか。 
\\	え、いやいや。もちろん、冬果さんもきれいです。	
\\	え、いやいや。もちろん、冬果さんもきれいです。 
\\	どうも。	
\\	どうも。 
\\	どんな
\\	高い
\\	財布
\\	素敵(な)
\\	格好いい
\\	うれしい
\\	どういたしまして
\\	ない。私の財布!ない!	
\\	ない。私の財布!ない! 
\\	どんな財布ですか。	
\\	どんな財布ですか。 
\\	高い財布です。イタリアのデザイナーの財布です。とても素敵な財布です。	
\\	高い財布です。イタリアのデザイナーの財布です。とても素敵な財布です。 
\\	あれですか。	
\\	あれですか。 
\\	あー!そうです!私の財布です。うれしい!ありがとう。春人!	
\\	あー!そうです!私の財布です。うれしい!ありがとう。春人! 
\\	どういたしまして。とても格好いい財布ですね。	
\\	どういたしまして。とても格好いい財布ですね。 
\\	ない!私の傘!ない!	
\\	ない!私の傘!ない! 
\\	夏見さんの傘はどんな傘ですか。	
\\	夏見さんの傘はどんな傘ですか。 
\\	社長
\\	美味しい
\\	お茶
\\	タイプ
\\	人
\\	女の人
\\	かわいい
\\	若い
\\	にぎやか
\\	いつも
\\	元気
\\	全然
\\	はい、どうぞ。お茶です。	
\\	はい、どうぞ。お茶です。 
\\	どうも。ああ、美味しい。ファブリツィオさん、スタイルユーの社長は どんな人ですか。	
\\	どうも。ああ、美味しい。ファブリツィオさん、スタイルユーの社長は どんな人ですか。 
\\	とても 元気な人です。オフィスは いつも にぎやかです。	
\\	とても 元気な人です。オフィスは いつも にぎやかです。 
\\	若いですか。	
\\	若いですか。 
\\	はい。とても 若いです。かわいい人ですよ。	
\\	はい。とても 若いです。かわいい人ですよ。 
\\	ああ、女の人ですか。ファブリツィオさんの タイプですか。	
\\	ああ、女の人ですか。ファブリツィオさんの タイプですか。 
\\	いいえ。全然。	
\\	いいえ。全然。 
\\	運転手
\\	半
\\	横浜
\\	タクシー
\\	すみません
\\	今
\\	何時
\\	渋滞
\\	お客さん
\\	ホテル
\\	横浜ホテルまで。	
\\	横浜ホテルまで。 
\\	はい。	
\\	はい。 
\\	すみません。今 何時ですか。	
\\	すみません。今 何時ですか。 
\\	6時です。	
\\	6時です。 
\\	うわー。お客さん、渋滞ですよ。	
\\	うわー。お客さん、渋滞ですよ。 
\\	渋滞ですか・・・。すみません。今 何時ですか。	
\\	渋滞ですか・・・。すみません。今 何時ですか。 
\\	えっと、6時半です。	
\\	えっと、6時半です。 
\\	仕事
\\	パーティー
\\	から
\\	友達
\\	八時半
\\	六時半
\\	暇
\\	ビリーズ ブートキャンプ
\\	うち
\\	忙しい
\\	サルサ
\\	今夜
\\	映画
\\	銀座
\\	夜
\\	だめ
\\	します
\\	今日の 夜、暇ですか。	
\\	今日の 夜、暇ですか。 
\\	ごめんなさい。今日の 夜は だめです。東さんと 銀座で 映画を みます。	
\\	ごめんなさい。今日の 夜は だめです。東さんと 銀座で 映画を みます。 
\\	へぇー。東と。銀座で。映画を。へぇー。いいですね。・・・あ、冬果さん、今日の夜、暇ですか。	
\\	へぇー。東と。銀座で。映画を。へぇー。いいですね。・・・あ、冬果さん、今日の夜、暇ですか。 
\\	今夜?暇じゃないです。ファブリツィオと サルサを します。	
\\	今夜?暇じゃないです。ファブリツィオと サルサを します。 
\\	へぇー。ファブリツィオと。サルサを。へぇー。いいですね。あ、南・・・。	
\\	へぇー。ファブリツィオと。サルサを。へぇー。いいですね。あ、南・・・。 
\\	私は 忙しいです。うちで ビリーズ ブートキャンプを します。	
\\	私は 忙しいです。うちで ビリーズ ブートキャンプを します。 
\\	楽しい
\\	とても
\\	うち
\\	来ます
\\	から
\\	今
\\	地下鉄
\\	行きます
\\	バー
\\	友達
\\	タクシー
\\	帰ります
\\	電車
\\	もう
\\	サルサは とても 楽しいですね。あ、もう十二時ですね。冬果さんは、電車で 帰りますか。	
\\	サルサは とても 楽しいですね。あ、もう十二時ですね。冬果さんは、電車で 帰りますか。 
\\	うーん。タクシーで 帰ります。ファブリツィオは?	
\\	うーん。タクシーで 帰ります。ファブリツィオは? 
\\	私は 友達の バーへ 行きます。	
\\	私は 友達の バーへ 行きます。 
\\	今から?電車で 行きますか。	
\\	今から?電車で 行きますか。 
\\	はい。今から。地下鉄で 行きます。冬果さんも、来ますか。	
\\	はい。今から。地下鉄で 行きます。冬果さんも、来ますか。 
\\	いえいえいえいえ。私は うちへ 帰ります。	
\\	いえいえいえいえ。私は うちへ 帰ります。 
\\	駅
\\	回送電車
\\	電車
\\	どこ
\\	車庫
\\	次
\\	わかります
\\	来ます
\\	終電
\\	新宿
\\	すみません
\\	駅員
\\	行きます
\\	すみません。これは 新宿へ 行きますか。	
\\	すみません。これは 新宿へ 行きますか。 
\\	行きません。これは 回送電車です。	
\\	行きません。これは 回送電車です。 
\\	回送電車?!回送電車は どこへ 行きますか。	
\\	回送電車?!回送電車は どこへ 行きますか。 
\\	え?回送電車は...車庫へ いきます。わかりますか。	
\\	え?回送電車は...車庫へ いきます。わかりますか。 
\\	はいはい、わかります。じゃ、次の 電車は 何時に 来ますか。	
\\	はいはい、わかります。じゃ、次の 電車は 何時に 来ますか。 
\\	12時25分に 来ます。終電です。	
\\	12時25分に 来ます。終電です。 
\\	一度
\\	だめ(な)
\\	大丈夫
\\	電話
\\	もしもし
\\	会社
\\	来ます
\\	今
\\	もう
\\	午後
\\	休みます
\\	風邪
\\	お願いします
\\	また
\\	(ピー) えっと、スタイルユーの 東です。大丈夫ですか。また、電話します。	
\\	(ピー) えっと、スタイルユーの 東です。大丈夫ですか。また、電話します。 
\\	はい。	
\\	はい。 
\\	あ、もしもし。ファブリツィオ?スタイルユーの 南 夏見です。今日、何時に会社へ来ますか。	
\\	あ、もしもし。ファブリツィオ?スタイルユーの 南 夏見です。今日、何時に会社へ来ますか。 
\\	は?え?今、何時ですか。	
\\	は?え?今、何時ですか。 
\\	もう 午後1時です。	
\\	もう 午後1時です。 
\\	ええ?本当ですか?ああ、今日、会社を休みます。風邪です。ケホケホゴホゴホ。	
\\	ええ?本当ですか?ああ、今日、会社を休みます。風邪です。ケホケホゴホゴホ。 
\\	だめだめ。だめです。今日、二時に お客さんとミーティングをします。	
\\	だめだめ。だめです。今日、二時に お客さんとミーティングをします。 
\\	え?え?すみません。もう一度、お願いします。いつ誰とどこで何をしますか。	
\\	え?え?すみません。もう一度、お願いします。いつ誰とどこで何をしますか。 
\\	ホラー映画
\\	映画
\\	英語
\\	話します
\\	無理
\\	お願いします。
\\	また
\\	後で
\\	ホラー映画ですか?映画は みます。でも、ホラー映画は…。	
\\	ホラー映画ですか?映画は みます。でも、ホラー映画は…。 
\\	(電話)	
\\	はい、南です。は?え?えっとーぉ。あのぉー。ジャスト ア モメント プリーズ。ねぇ、ファブリツィオ、英語の電話!お願い!	
\\	はい、南です。は?え?えっとーぉ。あのぉー。ジャスト ア モメント プリーズ。ねぇ、ファブリツィオ、英語の電話!お願い! 
\\	英語?!無理、無理!私は イタリア語と フランス語と スペイン語と ポルトガル語と 日本語は 話します。でも 英語は 話しません。	
\\	英語?!無理、無理!私は イタリア語と フランス語と スペイン語と ポルトガル語と 日本語は 話します。でも 英語は 話しません。 
\\	あ、は~い。すみません、秋さん。また、後で。	
\\	あ、は~い。すみません、秋さん。また、後で。 
\\	実家
\\	来週
\\	偶然
\\	食べます
\\	毎日
\\	美味しい
\\	うどん
\\	高松
\\	四国
\\	出身
\\	松山
\\	あ、そうそう、南さん。私、来週、会社を 休みます。実家へ 帰ります。	
\\	あ、そうそう、南さん。私、来週、会社を 休みます。実家へ 帰ります。 
\\	あー、そうですよね。冬果さんの出身は どこですか。	
\\	あー、そうですよね。冬果さんの出身は どこですか。 
\\	四国の高松です。四国のうどんは おいしいですよ。私は東京では うどんを食べませんが、四国では 毎日うどんを 食べます。	
\\	四国の高松です。四国のうどんは おいしいですよ。私は東京では うどんを食べませんが、四国では 毎日うどんを 食べます。 
\\	ワオ。偶然ですね。来週、私も 四国へ 行きます。高松へは 行きませんが、松山へは 行きます。	
\\	ワオ。偶然ですね。来週、私も 四国へ 行きます。高松へは 行きませんが、松山へは 行きます。 
\\	お疲れさま
\\	彼氏
\\	時間
\\	用事
\\	チケット
\\	無料
\\	お金
\\	あります
\\	コンサート
\\	ビヨンセ
\\	います
\\	じゃ、お疲れさまです。	
\\	じゃ、お疲れさまです。 
\\	ミヨンちゃん。あ、あの、来週、ビヨンセのコンサートが あります。来週の金曜日の夜、暇ですか。	
\\	ミヨンちゃん。あ、あの、来週、ビヨンセのコンサートが あります。来週の金曜日の夜、暇ですか。 
\\	はい...でも、お金が ありません。	
\\	はい...でも、お金が ありません。 
\\	お金? あ、大丈夫です。無料のチケットが あります。	
\\	お金? あ、大丈夫です。無料のチケットが あります。 
\\	あ、でも、あのぉ…そうそう、用事が あります。	
\\	あ、でも、あのぉ…そうそう、用事が あります。 
\\	用事...ですか。...時間が...ありませんか。	
\\	用事...ですか。...時間が...ありませんか。 
\\	好き(な)
\\	だって
\\	お店
\\	お客様
\\	まだ
\\	うるさい
\\	スプレー
\\	トイレ
\\	ゴキブリ
\\	これでいい。
\\	彼氏は いませんが、好きな人は います。ごめんなさい。じゃ、お疲れ様です。	
\\	彼氏は いませんが、好きな人は います。ごめんなさい。じゃ、お疲れ様です。 
\\	あ…お疲れ様。	
\\	あ…お疲れ様。 
\\	キャー!ゴキブリ!ゴキブリ!ゴ・キ・ブ・リ!ト、トイレにゴキブリがいます!秋、ゴキブリ!ゴキブリが いますぅ。	
\\	キャー!ゴキブリ!ゴキブリ!ゴ・キ・ブ・リ!ト、トイレにゴキブリがいます!秋、ゴキブリ!ゴキブリが いますぅ。 
\\	(カチャ)	
\\	しぃーっ!ファブリツィオ!うるさい!まだ、お店に お客様が います。	
\\	しぃーっ!ファブリツィオ!うるさい!まだ、お店に お客様が います。 
\\	(バシッ!/プチッ)	
\\	(バタン!)	
\\	何人
\\	白ワイン
\\	ブルゴーニュ
\\	おにぎり
\\	ラーメン
\\	生ビール
\\	注文
\\	どうぞ
\\	こちら
\\	二人
\\	それから
\\	いらっしゃい!何人ですか。	
\\	いらっしゃい!何人ですか。 
\\	二人です。	
\\	二人です。 
\\	こちらへ どうぞ。ご注文は?	
\\	こちらへ どうぞ。ご注文は? 
\\	あ、生ビールを 一つ 下さい。それから、ラーメン 一つと おにぎり 三つ お願いします。ファブリツィオは?	
\\	あ、生ビールを 一つ 下さい。それから、ラーメン 一つと おにぎり 三つ お願いします。ファブリツィオは? 
\\	うーん。野菜ラーメンを 下さい。それから、フランスの ブルゴーニュの 白ワインは ありますか。	
\\	うーん。野菜ラーメンを 下さい。それから、フランスの ブルゴーニュの 白ワインは ありますか。 
\\	すみませんが、ワインは ちょっと…。	
\\	すみませんが、ワインは ちょっと…。 
\\	ラーメン屋
\\	小さい
\\	店員
\\	テーブル
\\	いす
\\	店
\\	大丈夫
\\	たくさん
\\	南さん、このラーメン屋、店員さんが たくさんいますね。	
\\	南さん、このラーメン屋、店員さんが たくさんいますね。 
\\	ファブリツィオ!シーッ!	
\\	ファブリツィオ!シーッ! 
\\	うーん。テーブルが、一つ二つ三つ。 テーブルが 三つ ありますねぇ。	
\\	うーん。テーブルが、一つ二つ三つ。 テーブルが 三つ ありますねぇ。 
\\	ファブリツィオ!シーッ!ファブリツィオ!	
\\	ファブリツィオ!シーッ!ファブリツィオ! 
\\	それから、椅子が、一つ二つ三つ四つ五つ六つ七つ八つ九つ!九つ あります。	
\\	それから、椅子が、一つ二つ三つ四つ五つ六つ七つ八つ九つ!九つ あります。 
\\	荷物
\\	いくつ
\\	出発
\\	大きい
\\	時計
\\	後ろ
\\	本屋
\\	右
\\	次のお客様、こちらへ どうぞ。	
\\	次のお客様、こちらへ どうぞ。 
\\	一つです。	
\\	一つです。 
\\	出発は ゲート
\\	です。あそこに、大きい時計が ありますね。	
\\	出発は ゲート
\\	です。あそこに、大きい時計が ありますね。 
\\	大きい時計、大きい時計…あ、はい。	
\\	大きい時計、大きい時計…あ、はい。 
\\	あの時計の後ろに ゲート
\\	が あります。	
\\	あの時計の後ろに ゲート
\\	が あります。 
\\	元気(な)
\\	南
\\	九州
\\	地図
\\	上
\\	下
\\	広島
\\	寺
\\	有名(な)
\\	うどん屋
\\	所
\\	西
\\	はぁ・・・。	
\\	はぁ・・・。 
\\	どうしましたか?秋。元気が ありませんね。	
\\	どうしましたか?秋。元気が ありませんね。 
\\	ええ。ちょっと...。ファブリツィオは 元気ですね。	
\\	ええ。ちょっと...。ファブリツィオは 元気ですね。 
\\	はい。明日から、四国へ 行きます。	
\\	はい。明日から、四国へ 行きます。 
\\	いい所ですよ。うどん屋が たくさん ありますね。	
\\	いい所ですよ。うどん屋が たくさん ありますね。 
\\	四国は 広島の下に ありますね。	
\\	四国は 広島の下に ありますね。 
\\	地図に 上と下は ないですよ...。	
\\	地図に 上と下は ないですよ...。 
\\	トヨタ
\\	ホンダ
\\	車
\\	好きな
\\	正直
\\	興味
\\	白い
\\	ランボルギーニ
\\	スズキ
\\	これは どこの 車ですか。	
\\	これは どこの 車ですか。 
\\	さあ...トヨタ...あれ、ホンダかな...。	
\\	さあ...トヨタ...あれ、ホンダかな...。 
\\	ええ。好きですよ。冬果さんは どんな車が 好きですか。	
\\	ええ。好きですよ。冬果さんは どんな車が 好きですか。 
\\	うーん。正直、興味が ありませんね。	
\\	うーん。正直、興味が ありませんね。 
\\	あ、でも、白い車が 好きです。	
\\	あ、でも、白い車が 好きです。 
\\	フェラーリ
\\	大好き
\\	まあまあ
\\	免許
\\	嫌い
\\	運転
\\	危ない
\\	つまらない
\\	ドイツ
\\	大嫌い
\\	ランボルギーニは イタリアの車です。とても、とても、とても 高い車です。	
\\	ランボルギーニは イタリアの車です。とても、とても、とても 高い車です。 
\\	私は イタリアの車が 大好きです。フェラーリも 好きです。	
\\	私は イタリアの車が 大好きです。フェラーリも 好きです。 
\\	ふう。ごめん ごめん。大丈夫?	
\\	ふう。ごめん ごめん。大丈夫? 
\\	あ、あのぉ...冬果さんは 運転が 好きですか。	
\\	あ、あのぉ...冬果さんは 運転が 好きですか。 
\\	運転?大嫌いですよ。ファブリツィオは 運転が 好きですか。	
\\	運転?大嫌いですよ。ファブリツィオは 運転が 好きですか。 
\\	道後温泉
\\	ランチ
\\	お腹
\\	温泉
\\	疲れます
\\	目
\\	頭
\\	痛い
\\	首
\\	到着
\\	食べます
\\	(バタン!車から降りる)	
\\	イタタタ...。首が 痛いです。頭も 痛いです。	
\\	イタタタ...。首が 痛いです。頭も 痛いです。 
\\	私は 目が 痛いです。疲れましたね。さあ、温泉に 行きましょう。	
\\	私は 目が 痛いです。疲れましたね。さあ、温泉に 行きましょう。 
\\	お…お腹が 痛いです。	
\\	お…お腹が 痛いです。 
\\	え?お腹が 痛い?大丈夫?	
\\	え?お腹が 痛い?大丈夫? 
\\	(グー)うーん。お腹 ペコペコです。	
\\	(グー)うーん。お腹 ペコペコです。 
\\	ははは、じゃ、ランチを 食べましょう。	
\\	ははは、じゃ、ランチを 食べましょう。 
\\	浴衣
\\	超
\\	スタイル
\\	長い
\\	足
\\	有名(な)
\\	顔
\\	細い
\\	歌手
\\	高い
\\	背
\\	サイズ
\\	デザイン
\\	すごく
\\	この浴衣は デザインが かわいいですが、サイズが 小さいですね。	
\\	この浴衣は デザインが かわいいですが、サイズが 小さいですね。 
\\	ああ、ファブリツィオは 背が高い…	
\\	ああ、ファブリツィオは 背が高い… 
\\	あ! あれ!歌手の コロダクルミですよ。	
\\	あ! あれ!歌手の コロダクルミですよ。 
\\	あの人は 有名ですか。	
\\	あの人は 有名ですか。 
\\	女の人
\\	髪
\\	短い
\\	それとも
\\	気
\\	強い
\\	声
\\	風邪
\\	タイプ
\\	(ランチ)ミヨンちゃんは どんな人が 好きですか。	
\\	(ランチ)ミヨンちゃんは どんな人が 好きですか。 
\\	私は、頭がいい人が好きですね。	
\\	私は、頭がいい人が好きですね。 
\\	え?えっとぉ...。うーん。	
\\	え?えっとぉ...。うーん。 
\\	髪が長い人が 好きですか…それとも、髪が短い人が 好きですか。	
\\	髪が長い人が 好きですか…それとも、髪が短い人が 好きですか。 
\\	いや、えっと...。うーん。気が強い人が 好きですよ。	
\\	いや、えっと...。うーん。気が強い人が 好きですよ。 
\\	気が 強い人...声が 大きい人...。	
\\	気が 強い人...声が 大きい人...。 
\\	ハッ、ハ、ハックション!うー。風邪かなあ。	
\\	ハッ、ハ、ハックション!うー。風邪かなあ。 
\\	面白い
\\	本当に
\\	明るい
\\	頭がいい
\\	泊まる
\\	古い
\\	仕事
\\	家
\\	うち
\\	あー、面白い。ファブリツィオさんは、本当に 明るくて 楽しい人ですね。	
\\	あー、面白い。ファブリツィオさんは、本当に 明るくて 楽しい人ですね。 
\\	クルミさんも、頭がよくて 面白いです。	
\\	クルミさんも、頭がよくて 面白いです。 
\\	今夜、松山に 泊まりますか。	
\\	今夜、松山に 泊まりますか。 
\\	いえ、高松です。冬果さんの うちに 泊まります。	
\\	いえ、高松です。冬果さんの うちに 泊まります。 
\\	ちょっと!	
\\	ちょっと! 
\\	動物
\\	家族
\\	新鮮(な)
\\	刺身
\\	スマート(な)
\\	ハンサム(な)
\\	お入りください
\\	おじゃまします
\\	こんばんは
\\	お上がりください
\\	お帰りなさい。
\\	ただいま
\\	にぎやか(な)
\\	(ガラガラ)	
\\	ただいまぁ。	
\\	ただいまぁ。 
\\	お帰り。	
\\	お帰り。 
\\	うわー。動物がたくさんいますね。	
\\	うわー。動物がたくさんいますね。 
\\	あ、冬果ちゃん、お帰りなさい	
\\	あ、冬果ちゃん、お帰りなさい 
\\	いらっしゃーい。あ、冬果ちゃんの 彼氏?	
\\	いらっしゃーい。あ、冬果ちゃんの 彼氏? 
\\	父
\\	わざわざ
\\	もの
\\	つまらない
\\	申します
\\	おばあさん
\\	お母さん
\\	お父さん
\\	妹
\\	弟
\\	祖母
\\	母
\\	冷たい
\\	ファブリツィオ、うちの父と 母と 祖母と 弟と 妹です。	
\\	ファブリツィオ、うちの父と 母と 祖母と 弟と 妹です。 
\\	お父さん、お母さん、おばあさん、弟さん、妹さんですね。	
\\	お父さん、お母さん、おばあさん、弟さん、妹さんですね。 
\\	ファブリツィオと 申します。どうぞ よろしく おねがい いたします。	
\\	ファブリツィオと 申します。どうぞ よろしく おねがい いたします。 
\\	あら、すみません。わざわざ、ありがとう ございます。	
\\	あら、すみません。わざわざ、ありがとう ございます。 
\\	新しい
\\	夢
\\	住所
\\	名前
\\	ビザ
\\	クレジットカード
\\	現金
\\	革
\\	落とします
\\	警察官
\\	アメリカン エキスプレス
\\	・・・ない。財布が ありません。私の新しくて ゴージャスな財布!	
\\	・・・ない。財布が ありません。私の新しくて ゴージャスな財布! 
\\	あのぉ、財布を 落としました。	
\\	あのぉ、財布を 落としました。 
\\	そうですか。どんな財布 ですか。	
\\	そうですか。どんな財布 ですか。 
\\	白くて 大きい財布です。イタリアの 革の財布です。	
\\	白くて 大きい財布です。イタリアの 革の財布です。 
\\	財布の中に いくらありましたか。	
\\	財布の中に いくらありましたか。 
\\	現金は ありませんでしたが、ビザと アメリカン エキスプレスのクレジット カードが ありました。	
\\	現金は ありませんでしたが、ビザと アメリカン エキスプレスのクレジット カードが ありました。 
\\	(アラーム)	
\\	はっ!夢ですかぁ。	
\\	はっ!夢ですかぁ。 
\\	コーヒー
\\	お代わり
\\	苦い
\\	最高
\\	飲みます
\\	飲む
\\	行く
\\	こんぴらさん
\\	ファブリツィオさん。コーヒーのお代わり、いかがですか?	
\\	ファブリツィオさん。コーヒーのお代わり、いかがですか? 
\\	あ、はい。お願いします。	
\\	あ、はい。お願いします。 
\\	あら、ありがとう、ファブリツィオさん。	
\\	あら、ありがとう、ファブリツィオさん。 
\\	(コーヒーをいれる)	
\\	今日、どこ 行くの?	
\\	今日、どこ 行くの? 
\\	急
\\	階段
\\	する
\\	来る
\\	ちょっと
\\	失礼
\\	失礼します。
\\	まだ ですか。急で 長い階段ですね。ふー。	
\\	まだ ですか。急で 長い階段ですね。ふー。 
\\	(電話)	
\\	もしもし?あ、お母さん。	
\\	もしもし?あ、お母さん。 
\\	今?こんぴらさん。	
\\	今?こんぴらさん。 
\\	え?今から お父さんと 来るの?	
\\	え?今から お父さんと 来るの? 
\\	うん。そっか。	
\\	うん。そっか。 
\\	うん、そうそう。じゃ、どうするの?	
\\	うん、そうそう。じゃ、どうするの? 
\\	食う
\\	います
\\	マジ
\\	あそこ
\\	もちろん
\\	絶対に
\\	待つ
\\	待ちます
\\	うどん
\\	列
\\	いる
\\	冬果さん、この長い列は 何ですか。	
\\	冬果さん、この長い列は 何ですか。 
\\	うどんの列ですよ。	
\\	うどんの列ですよ。 
\\	とても 長いですよ。	
\\	とても 長いですよ。 
\\	いっぱい
\\	一人
\\	大体
\\	位
\\	かけうどん
\\	天ぷら
\\	どうして
\\	ふー、お腹がいっぱいです。	
\\	ふー、お腹がいっぱいです。 
\\	ファブリツィオさんは 何を 食べましたか。	
\\	ファブリツィオさんは 何を 食べましたか。 
\\	かけうどんを 食べました。	
\\	かけうどんを 食べました。 
\\	天ぷらは、食べましたか。	
\\	天ぷらは、食べましたか。 
\\	いえ、食べませんでした。	
\\	いえ、食べませんでした。 
\\	何?天ぷらを食べませんでしたぁぁ??ど、ど、どうして。	
\\	何?天ぷらを食べませんでしたぁぁ??ど、ど、どうして。 
\\	驚く
\\	暗い
\\	道
\\	全然
\\	大変(な)
\\	分かります
\\	分かる
\\	すぐに
\\	人気
\\	先週
\\	怖い
\\	でも、うどんは本当に 美味しかったです。	
\\	でも、うどんは本当に 美味しかったです。 
\\	それは、よかったです。	
\\	それは、よかったです。 
\\	お父さんはこのお店によく来ますか。	
\\	お父さんはこのお店によく来ますか。 
\\	ええ、先週も来ました。	
\\	ええ、先週も来ました。 
\\	たぶん、来週も 来ます。	
\\	たぶん、来週も 来ます。 
\\	土産
\\	一六タルト
\\	甘い
\\	天気
\\	寒い
\\	内緒
\\	下手(な)
\\	ファン
\\	ただいまぁ。みなさん。	
\\	ただいまぁ。みなさん。 
\\	四国の お土産、一六タルトです。	
\\	四国の お土産、一六タルトです。 
\\	私と 冬果さんから です。どうぞ。	
\\	私と 冬果さんから です。どうぞ。 
\\	うわー。ありがとうございます。天気は どうでしたか。	
\\	うわー。ありがとうございます。天気は どうでしたか。 
\\	毎日、良い天気でした。でも、ちょっと寒かったです。	
\\	毎日、良い天気でした。でも、ちょっと寒かったです。 
\\	へぇー。道後温泉には 行きましたか。	
\\	へぇー。道後温泉には 行きましたか。 
\\	はい。とても 良い温泉でした。	
\\	はい。とても 良い温泉でした。 
\\	うどんは 食べましたか。	
\\	うどんは 食べましたか。 
\\	もちろん、食べました。安くて すごく美味しかったです。	
\\	もちろん、食べました。安くて すごく美味しかったです。 
\\	成田
\\	空港
\\	パスポート
\\	お願いします
\\	すみません
\\	もう
\\	一度
\\	スーツケース
\\	ゆっくり
\\	パスポート、お願いします。	
\\	パスポート、お願いします。 
\\	え?すみません。もう一度、お願いします。	
\\	え?すみません。もう一度、お願いします。 
\\	パ ス ポート、お願い します。	
\\	パ ス ポート、お願い します。 
\\	あ、はい。	
\\	あ、はい。 
\\	スーツケース、お願いします。	
\\	スーツケース、お願いします。 
\\	すみません。...ゆっくり、お願いします。	
\\	すみません。...ゆっくり、お願いします。 
\\	これ
\\	何
\\	薬
\\	お土産
\\	係員
\\	アスピリン
\\	これは 何ですか。	
\\	これは 何ですか。 
\\	薬です。...アスピリンです。	
\\	薬です。...アスピリンです。 
\\	これは、何ですか?	
\\	これは、何ですか? 
\\	お土産です。	
\\	お土産です。 
\\	はい、オッケーです。	
\\	はい、オッケーです。 
\\	オッケー...	
\\	オッケー... 
\\	留学
\\	違います
\\	センター
\\	いいえ
\\	さん
\\	あ、あの...、ラリーさん ですか。	
\\	あ、あの...、ラリーさん ですか。 
\\	いいえ、違います。	
\\	いいえ、違います。 
\\	うわぁぁぁあぁ、すみません。	
\\	うわぁぁぁあぁ、すみません。 
\\	あ、あのぉ、ラリーさんですか。	
\\	あ、あのぉ、ラリーさんですか。 
\\	ラリー...あ、はい。ロリー ネイラーです。	
\\	ラリー...あ、はい。ロリー ネイラーです。 
\\	はじめまして。
\\	よろしく お願いします。
\\	よろしく
\\	私の
\\	名前
\\	はじめまして。ロリーです。	
\\	はじめまして。ロリーです。 
\\	よろしく お願いします。ラリーさん。	
\\	よろしく お願いします。ラリーさん。 
\\	あ、あの...ラリーじゃありません。ロリーです。	
\\	あ、あの...ラリーじゃありません。ロリーです。 
\\	あ...すみません。ロリー ハンマーさん ですか。	
\\	あ...すみません。ロリー ハンマーさん ですか。 
\\	重い
\\	大きい
\\	痛い
\\	普通
\\	馬鹿力
\\	馬鹿
\\	力
\\	よいしょ
\\	これは ロリーさんの スーツケースですか。大きいですね。	
\\	これは ロリーさんの スーツケースですか。大きいですね。 
\\	(ドサッ)	
\\	うわっ、馬鹿力...。	
\\	うわっ、馬鹿力...。 
\\	ばか...じから... 馬鹿?私は 馬鹿じゃないですよ。	
\\	ばか...じから... 馬鹿?私は 馬鹿じゃないですよ。 
\\	え?あ、いえいえ。馬鹿ではありません。「ばかじから」です。	
\\	え?あ、いえいえ。馬鹿ではありません。「ばかじから」です。 
\\	ばかじからは 何ですか。	
\\	ばかじからは 何ですか。 
\\	定食
\\	ご飯
\\	卵
\\	大丈夫
\\	魚
\\	からあげ
\\	さしみ
\\	カツ丼
\\	ぶた肉
\\	とり肉
\\	肉
\\	店員
\\	屋
\\	大盛り
\\	すみません。注文、お願いします。	
\\	すみません。注文、お願いします。 
\\	さしみ定食 お願いします。	
\\	さしみ定食 お願いします。 
\\	あのぉ。これはなんですか。	
\\	あのぉ。これはなんですか。 
\\	それは、からあげ定食です。	
\\	それは、からあげ定食です。 
\\	からあ...げ?えっと...肉ですか。魚ですか。	
\\	からあ...げ?えっと...肉ですか。魚ですか。 
\\	肉です。とり肉です。	
\\	肉です。とり肉です。 
\\	ああ、とりですか..。じゃ、これは 何ですか。	
\\	ああ、とりですか..。じゃ、これは 何ですか。 
\\	それは、カツ丼です。ぶた肉と 卵です。	
\\	それは、カツ丼です。ぶた肉と 卵です。 
\\	でも
\\	とても
\\	鍵
\\	古い
\\	お風呂
\\	小さい
\\	トイレ
\\	電気
\\	あそこ
\\	そこ
\\	ここ
\\	号室
\\	部屋
\\	アパート
\\	便利(な)
\\	(ロリーのアパート)	
\\	(ロリーのアパート) 
\\	ロリーさんの 部屋は 205号室です。ここです。	
\\	ロリーさんの 部屋は 205号室です。ここです。 
\\	電気は そこ。トイレは あそこ。	
\\	電気は そこ。トイレは あそこ。 
\\	トイレは ここ、、ですか。	
\\	トイレは ここ、、ですか。 
\\	あ、、、小さい トイレですね。	
\\	あ、、、小さい トイレですね。 
\\	お風呂は あそこです。それから、部屋の 鍵は これです。はい、どうぞ。	
\\	お風呂は あそこです。それから、部屋の 鍵は これです。はい、どうぞ。 
\\	おはようございます。
\\	蒸し暑い
\\	とても
\\	分かります
\\	暑い
\\	エレベーター
\\	虫
\\	おはようございます。	
\\	おはようございます。 
\\	あ…おはようございます。	
\\	あ…おはようございます。 
\\	蒸し暑いですね。	
\\	蒸し暑いですね。 
\\	むし...あ...つい?むし?虫?!	
\\	むし...あ...つい?むし?虫?! 
\\	あ、と て も 暑いですね。	
\\	あ、と て も 暑いですね。 
\\	あ、はいはい。分かります。分かります。とても 暑いですね。	
\\	あ、はいはい。分かります。分かります。とても 暑いですね。 
\\	留学生
\\	学生
\\	学
\\	英語
\\	専攻
\\	本当
\\	何年生
\\	月
\\	大学
\\	学生ですか。	
\\	学生ですか。 
\\	はい。月大学(つきだいがく)の留学生です。	
\\	はい。月大学(つきだいがく)の留学生です。 
\\	うそ?月大(つきだい)?私も 月大(つきだい)の 学生です。	
\\	うそ?月大(つきだい)?私も 月大(つきだい)の 学生です。 
\\	本当?何年生ですか。	
\\	本当?何年生ですか。 
\\	二年生です。専攻は英語学です。	
\\	二年生です。専攻は英語学です。 
\\	どこ
\\	頑張ってください。
\\	後で
\\	じゃあ 
\\	じゃ
\\	あそこ
\\	ありがとう ございました。
\\	また
\\	じゃあ、ロリーさん、また後で。	
\\	じゃあ、ロリーさん、また後で。 
\\	あ、あのぉ、静さん、留学センターはどこですか。	
\\	あ、あのぉ、静さん、留学センターはどこですか。 
\\	え?留学センター?ああ、あそこです。	
\\	え?留学センター?ああ、あそこです。 
\\	あそこですか。ありがとうございました。じゃ、また、後で。	
\\	あそこですか。ありがとうございました。じゃ、また、後で。 
\\	じゃあ、また。 頑張ってくださいね。	
\\	じゃあ、また。 頑張ってくださいね。 
\\	こんにちは
\\	留学生
\\	お客さん
\\	携帯
\\	サイン
\\	電話番号
\\	登録
\\	受付
\\	今
\\	何時
\\	十一時
\\	オリエンテーション
\\	から
\\	今何時ですか。	
\\	今何時ですか。 
\\	十一時です。	
\\	十一時です。 
\\	十一時ですか…。	
\\	十一時ですか…。 
\\	ロリーさん、大丈夫ですか。留学生のオリエンテーションは十一時からですよ。	
\\	ロリーさん、大丈夫ですか。留学生のオリエンテーションは十一時からですよ。 
\\	え?本当ですか。じゃ、また後で。	
\\	え?本当ですか。じゃ、また後で。 
\\	デパート
\\	店員
\\	靴
\\	売り場
\\	四階
\\	何階
\\	いらっしゃいませ。	
\\	いらっしゃいませ。 
\\	すみません。靴は どこですか。	
\\	すみません。靴は どこですか。 
\\	靴...ですか。靴売り場は 四階です。	
\\	靴...ですか。靴売り場は 四階です。 
\\	四階ですか。ありがとうございます。	
\\	四階ですか。ありがとうございます。 
\\	あ、何階ですか。	
\\	あ、何階ですか。 
\\	四階、お願いします。	
\\	四階、お願いします。 
\\	サイズ
\\	大きい
\\	ちょっと
\\	足
\\	お待ちください。
\\	ちょうど
\\	いい
\\	下さい
\\	どう
\\	すみません。大きいサイズを お願いします。	
\\	すみません。大きいサイズを お願いします。 
\\	サイズは 何ですか。	
\\	サイズは 何ですか。 
\\	ちょっと...分かりません。	
\\	ちょっと...分かりません。 
\\	そうですか...。じゃ、足を お願いします。うーん。26ですね。お待ちください。	
\\	そうですか...。じゃ、足を お願いします。うーん。26ですね。お待ちください。 
\\	はい、どうぞ。 サイズはどうですか。	
\\	はい、どうぞ。 サイズはどうですか。 
\\	東京
\\	東京タワー
\\	メートル
\\	じゃあ 
\\	じゃ
\\	へえ
\\	水木さん、あれは、何ですか。	
\\	水木さん、あれは、何ですか。 
\\	え?どれですか。	
\\	え?どれですか。 
\\	あれです。	
\\	あれです。 
\\	ああ、あれは、東京タワーです。三百三十三メートルです。	
\\	ああ、あれは、東京タワーです。三百三十三メートルです。 
\\	へぇー。じゃあ、あれは何ですか。	
\\	へぇー。じゃあ、あれは何ですか。 
\\	あれは、サンシャイン六十です。	
\\	あれは、サンシャイン六十です。 
\\	百
\\	そうです
\\	それから
\\	札
\\	一万
\\	千
\\	一円
\\	十
\\	円
\\	玉
\\	これは、一円。一円玉。これは五円。五円玉。十円。十円玉。五十円。五十円玉。百円。百円玉。五百円。五百円玉。	
\\	これは、一円。一円玉。これは五円。五円玉。十円。十円玉。五十円。五十円玉。百円。百円玉。五百円。五百円玉。 
\\	はい。一、五、十、五十、百、五百。	
\\	はい。一、五、十、五十、百、五百。 
\\	そうです。これは千円。千円札。二千円。二千円札。五千円。五千円札。それから、これは一万円。一万円札。	
\\	そうです。これは千円。千円札。二千円。二千円札。五千円。五千円札。それから、これは一万円。一万円札。 
\\	千、二千、五千、一万。	
\\	千、二千、五千、一万。 
\\	そうそう。	
\\	そうそう。 
\\	夏
\\	日本
\\	何月
\\	たぶん
\\	七月
\\	八月
\\	九月
\\	秋
\\	あのぉ、日本の夏は何月から何月までですか。	
\\	あのぉ、日本の夏は何月から何月までですか。 
\\	えっと、たぶん...七月から九月までです。	
\\	えっと、たぶん...七月から九月までです。 
\\	七月と八月と九月ですか・・・。	
\\	七月と八月と九月ですか・・・。 
\\	ええ?ちがいますよ。九月は秋ですよ。夏は、六月から八月までです。	
\\	ええ?ちがいますよ。九月は秋ですよ。夏は、六月から八月までです。 
\\	ええ?そうですか。	
\\	ええ?そうですか。 
\\	会計
\\	ちょっと
\\	待ってください。
\\	大丈夫
\\	カード
\\	お会計、一万 三千 百 二十五 円です。	
\\	お会計、一万 三千 百 二十五 円です。 
\\	ちょっと 待ってください。はい、一万円。	
\\	ちょっと 待ってください。はい、一万円。 
\\	えっと...千円、二千円、三千円。あ...ちょ、ちょっと待ってください。	
\\	えっと...千円、二千円、三千円。あ...ちょ、ちょっと待ってください。 
\\	はい。大丈夫ですよ。	
\\	はい。大丈夫ですよ。 
\\	あ、はい百円。十円、二十円。えっと...一円、二円、三円、四円...	
\\	あ、はい百円。十円、二十円。えっと...一円、二円、三円、四円... 
\\	どちら様
\\	こんばんは
\\	来週
\\	金曜日
\\	日曜日
\\	暇
\\	土曜日
\\	はい。どちら様ですか。	
\\	はい。どちら様ですか。 
\\	あ、私です。静です。	
\\	あ、私です。静です。 
\\	こんばんは。静さん。	
\\	こんばんは。静さん。 
\\	こんばんは。ロリーさん。あのぉ、来週の金曜日から日曜日まで、暇ですか。	
\\	こんばんは。ロリーさん。あのぉ、来週の金曜日から日曜日まで、暇ですか。 
\\	来週の金曜日と土曜日と日曜日?はい。暇ですよ。	
\\	来週の金曜日と土曜日と日曜日?はい。暇ですよ。 
\\	松本
\\	あずさ二号
\\	長野県
\\	今日
\\	まず
\\	予定
\\	切符
\\	自由席
\\	枚
\\	大人
\\	行きます
\\	松本まで、大人八枚お願いします。	
\\	松本まで、大人八枚お願いします。 
\\	八時のあずさ二号ですか。	
\\	八時のあずさ二号ですか。 
\\	はい。自由席、お願いします。	
\\	はい。自由席、お願いします。 
\\	はい、ロリーさんの切符です。	
\\	はい、ロリーさんの切符です。 
\\	ありがとうございます。今日の予定は?	
\\	ありがとうございます。今日の予定は? 
\\	まず、長野県の松本に行きます。	
\\	まず、長野県の松本に行きます。 
\\	富士山
\\	きれい(な)
\\	私たち
\\	行きます
\\	電車
\\	今日
\\	静さん、あれは富士山ですか。	
\\	静さん、あれは富士山ですか。 
\\	え?あ、そうです。きれいですね。	
\\	え?あ、そうです。きれいですね。 
\\	今日、私たちは富士山へ行きますか。	
\\	今日、私たちは富士山へ行きますか。 
\\	いえ、行きません。	
\\	いえ、行きません。 
\\	そうですか。じゃ、どこへ行きますか。	
\\	そうですか。じゃ、どこへ行きますか。 
\\	ええっとぉ。。。水木さん、今日、どこに行きますか。	
\\	ええっとぉ。。。水木さん、今日、どこに行きますか。 
\\	皆さん
\\	食べます
\\	二時間
\\	今
\\	りんご園
\\	午後
\\	自由時間
\\	半
\\	松本城
\\	(お)そば
\\	皆さん、ここは、松本城です。	
\\	皆さん、ここは、松本城です。 
\\	では、今から、ニ時半まで自由時間です。午後、りんご園に行きます。	
\\	では、今から、ニ時半まで自由時間です。午後、りんご園に行きます。 
\\	はーい。	
\\	はーい。 
\\	今、十二時半...。じゃ、自由時間はニ時間ですね。	
\\	今、十二時半...。じゃ、自由時間はニ時間ですね。 
\\	静さん、何を食べますか。	
\\	静さん、何を食べますか。 
\\	実は
\\	ねこ
\\	卵
\\	バター
\\	うん
\\	かわいそう
\\	あります
\\	アレルギー
\\	そば
\\	花粉症
\\	実は、そばアレルギーがあります。	
\\	実は、そばアレルギーがあります。 
\\	ええ?かわいそう...。	
\\	ええ?かわいそう...。 
\\	静さんは、アレルギーがありますか。	
\\	静さんは、アレルギーがありますか。 
\\	うん。卵アレルギー、バターアレルギー、ねこアレルギー…	
\\	うん。卵アレルギー、バターアレルギー、ねこアレルギー… 
\\	それから、花粉症もあります。	
\\	それから、花粉症もあります。 
\\	地図
\\	時間
\\	お元気ですか。
\\	声
\\	行き
\\	乗る
\\	川中島
\\	降りる
\\	バス
\\	あれ?ここは...どこですか?	
\\	あれ?ここは...どこですか? 
\\	ロリーさん、地図がありますか。	
\\	ロリーさん、地図がありますか。 
\\	え?地図?ありません。	
\\	え?地図?ありません。 
\\	あ、もう、二時半ですね。静さん、時間がありませんよ。	
\\	あ、もう、二時半ですね。静さん、時間がありませんよ。 
\\	(電話)	
\\	(電話) 
\\	もしもし。あ、水木さん。こんにちは。お元気ですか。	
\\	もしもし。あ、水木さん。こんにちは。お元気ですか。 
\\	バス
\\	駅
\\	行きます
\\	分かります
\\	江戸
\\	みます
\\	たくさん
\\	侍
\\	います
\\	すみません。このバスは長野駅へ行きますか。	
\\	すみません。このバスは長野駅へ行きますか。 
\\	あのお...分かりますか。	
\\	あのお...分かりますか。 
\\	これは長野駅に行きますか。	
\\	これは長野駅に行きますか。 
\\	…行きません。	
\\	…行きません。 
\\	これは..江戸に行きます。	
\\	これは..江戸に行きます。 
\\	江戸?…。あのお…。	
\\	江戸?…。あのお…。 
\\	ろ、ロリーさん、見ましたか?	
\\	ろ、ロリーさん、見ましたか? 
\\	カウンセラー
\\	旅行
\\	しゅみ
\\	みなさん
\\	話します
\\	日本語
\\	中国語
\\	中国人
\\	ぼく
\\	自己紹介
\\	留学
\\	出身
\\	こんにちは。水木一男です。	
\\	こんにちは。水木一男です。 
\\	私は 留学カウンセラーです。	
\\	私は 留学カウンセラーです。 
\\	よろしく おねがいします。	
\\	よろしく おねがいします。 
\\	じゃ、みなさん、自己紹介、おねがいします。	
\\	じゃ、みなさん、自己紹介、おねがいします。 
\\	風邪
\\	赤坂
\\	電話番号
\\	住所
\\	案内
\\	病院
\\	近く
\\	寒い
\\	たぶん
\\	森
\\	風邪ですか。	
\\	風邪ですか。 
\\	ええ...。たぶん(ゴホッ)...。	
\\	ええ...。たぶん(ゴホッ)...。 
\\	ううっ、寒い。近くに、病院、ありますか。	
\\	ううっ、寒い。近くに、病院、ありますか。 
\\	はい。104 電話番号案内 です。	
\\	はい。104 電話番号案内 です。 
\\	月大学の 近くに 病院、ありますか。	
\\	月大学の 近くに 病院、ありますか。 
\\	住所はどこですか。	
\\	住所はどこですか。 
\\	東京都 赤坂 3-4-4です。	
\\	東京都 赤坂 3-4-4です。 
\\	3-4-5に 森病院が あります。	
\\	3-4-5に 森病院が あります。 
\\	電話番号は 何番ですか。	
\\	電話番号は 何番ですか。 
\\	お待ちください。	
\\	お待ちください。 
\\	電話番号は 
\\	です。	
\\	電話番号は 
\\	です。  
\\	看護師
\\	診察
\\	時間
\\	土日
\\	休み
\\	朝
\\	昼
\\	平日
\\	はい、森病院です。	
\\	はい、森病院です。 
\\	すみません。診察は 何時から 何時まで ですか。	
\\	すみません。診察は 何時から 何時まで ですか。 
\\	診察は 月曜日から 金曜日。	
\\	診察は 月曜日から 金曜日。 
\\	時間は 午前九時から 午後一時半まで と 午後三時半から 午後六時までです。	
\\	時間は 午前九時から 午後一時半まで と 午後三時半から 午後六時までです。 
\\	土日は お休みです。	
\\	土日は お休みです。 
\\	看護師
\\	予約
\\	申します
\\	初めて
\\	保険証
\\	病院
\\	さっき
\\	こんにちは。ご予約は ありますか。	
\\	こんにちは。ご予約は ありますか。 
\\	はい。ロリーと申します。(ゴホッ)	
\\	はい。ロリーと申します。(ゴホッ)  
\\	さっき、電話しました。	
\\	さっき、電話しました。 
\\	四時から 予約を しました。	
\\	四時から 予約を しました。 
\\	四時から…。	
\\	四時から…。 
\\	あ、ロリーネイラーさんですね。	
\\	あ、ロリーネイラーさんですね。 
\\	診察室
\\	いす
\\	のど
\\	痛い
\\	吐き気
\\	熱
\\	度
\\	頭
\\	背中
\\	ロリーさん。ロリーネイラーさん。診察室へ どうぞ。	
\\	ロリーさん。ロリーネイラーさん。診察室へ どうぞ。 
\\	こんにちは。お願いします。	
\\	こんにちは。お願いします。 
\\	ロリーネイラーさんですね。あ、いすへ どうぞ。どうしましたか。	
\\	ロリーネイラーさんですね。あ、いすへ どうぞ。どうしましたか。 
\\	のどが 痛いです。	
\\	のどが 痛いです。 
\\	熱が39.5度あります。	
\\	熱が39.5度あります。 
\\	あ、あと、頭と 背中と お腹が 痛いです。	
\\	あ、あと、頭と 背中と お腹が 痛いです。 
\\	ふーん。吐き気は?	
\\	ふーん。吐き気は? 
\\	吐き気...。	
\\	吐き気...。 
\\	看護師
\\	本当
\\	会計
\\	領収書
\\	処方箋
\\	おつり
\\	お大事に。
\\	大事
\\	高い
\\	お会計、一万三千六百円です。	
\\	お会計、一万三千六百円です。 
\\	いち、いち、いち、いち、一万三千六百円。	
\\	いち、いち、いち、いち、一万三千六百円。 
\\	本当ですか。高いですね...。	
\\	本当ですか。高いですね...。 
\\	あ、領収書、お願いします。	
\\	あ、領収書、お願いします。 
\\	近く
\\	薬局
\\	コンビニ
\\	となり
\\	薬
\\	屋
\\	通行人
\\	あの、すみません、近くに 薬屋は ありますか?	
\\	あの、すみません、近くに 薬屋は ありますか? 
\\	薬屋...。ああ、薬局ですね。	
\\	薬屋...。ああ、薬局ですね。 
\\	薬局は あそこです。	
\\	薬局は あそこです。 
\\	どこですか?	
\\	どこですか? 
\\	あそこに、コンビニが ありますね。	
\\	あそこに、コンビニが ありますね。 
\\	コンビニ...あ、はいはい。あります。	
\\	コンビニ...あ、はいはい。あります。 
\\	薬局は コンビニの となりです。	
\\	薬局は コンビニの となりです。 
\\	薬局
\\	ここ
\\	位
\\	時間
\\	お金
\\	かかります
\\	どのくらい
\\	処方箋
\\	薬
\\	まんが喫茶
\\	いらっしゃいませ。	
\\	いらっしゃいませ。 
\\	あのぉ...。薬を お願いします。	
\\	あのぉ...。薬を お願いします。 
\\	これが あります。	
\\	これが あります。 
\\	あ...処方箋ですね。お待ちください。	
\\	あ...処方箋ですね。お待ちください。 
\\	あの...どのくらい かかりますか。	
\\	あの...どのくらい かかりますか。 
\\	ん?お金ですか。時間ですか。	
\\	ん?お金ですか。時間ですか。 
\\	時間です。	
\\	時間です。 
\\	時間 どのくらい かかりますか。	
\\	時間 どのくらい かかりますか。 
\\	落とします
\\	タイプ
\\	おしゃれ(な)
\\	ティファニー
\\	きれい(な)
\\	とても
\\	私の
\\	あなたの
\\	あなた
\\	イヤリング
\\	人
\\	あ、落としましたよ。	
\\	あ、落としましたよ。 
\\	え?	
\\	え? 
\\	これ、あなたの イヤリング ですね。	
\\	これ、あなたの イヤリング ですね。 
\\	落としましたよ。	
\\	落としましたよ。 
\\	え?ああ、そうです。私の です。	
\\	え?ああ、そうです。私の です。 
\\	すみません。ありがとう ございます。	
\\	すみません。ありがとう ございます。 
\\	とても きれいですね。	
\\	とても きれいですね。 
\\	は?	
\\	は? 
\\	去年
\\	べんきょう
\\	一年間
\\	四年
\\	前
\\	はじめて
\\	日本
\\	じょうず
\\	日本語
\\	上海
\\	北京
\\	へぇー。じゃ、りゅうさんは 中国人?わかりませんでした。	
\\	へぇー。じゃ、りゅうさんは 中国人?わかりませんでした。 
\\	はい。中国の 上海から 来ました。	
\\	はい。中国の 上海から 来ました。 
\\	日本語、じょうずですねぇ。	
\\	日本語、じょうずですねぇ。 
\\	日本は 初めてですか。	
\\	日本は 初めてですか。 
\\	いえ、四年前に 来ました。	
\\	いえ、四年前に 来ました。 
\\	一年間、日本で べんきょうしました。	
\\	一年間、日本で べんきょうしました。 
\\	へぇー。私も、去年 中国に 行きましたよ。	
\\	へぇー。私も、去年 中国に 行きましたよ。 
\\	中国語
\\	メールアドレス
\\	楽しい
\\	漢字
\\	かんたん
\\	むずかしい
\\	発音
\\	ちょっと
\\	わかります
\\	連絡
\\	中国語が わかりますか。	
\\	中国語が わかりますか。 
\\	ちょっと わかります。	
\\	ちょっと わかります。 
\\	でも、中国語の 発音は むずかしいですね。	
\\	でも、中国語の 発音は むずかしいですね。 
\\	日本語は むずかしかったですか。	
\\	日本語は むずかしかったですか。 
\\	まだ、むずかしいですよ。	
\\	まだ、むずかしいですよ。 
\\	あ、でも漢字は むずかしくなかったです。	
\\	あ、でも漢字は むずかしくなかったです。 
\\	こんばんは
\\	遠い
\\	インフルエンザ
\\	飲みます
\\	おやすみなさい。
\\	友だち
\\	ハンサム(な)
\\	おしゃれ(な)
\\	すてき(な)
\\	はい。もしもし。	
\\	はい。もしもし。 
\\	あ...水木さん。こんばんは。	
\\	あ...水木さん。こんばんは。 
\\	はい、行きました。	
\\	はい、行きました。 
\\	いえ、遠くなかったです。	
\\	いえ、遠くなかったです。 
\\	ええ、飲みました。	
\\	ええ、飲みました。 
\\	はい、そうです。インフルエンザでした。	
\\	はい、そうです。インフルエンザでした。 
\\	大きい
\\	面白い
\\	ちがいます
\\	細い
\\	かっこいい
\\	やさしい
\\	おとなしい
\\	彼氏
\\	うそ
\\	ああ...大きくて 面白い ドリューさん?	
\\	ああ...大きくて 面白い ドリューさん? 
\\	ちがいます。ちがいます。	
\\	ちがいます。ちがいます。 
\\	中国人で...細くて かっこいい リュウさん。	
\\	中国人で...細くて かっこいい リュウさん。 
\\	ああー。リュウさん。	
\\	ああー。リュウさん。 
\\	やさしくて おとなしい人ですね。	
\\	やさしくて おとなしい人ですね。 
\\	もしもし
\\	こんばんは
\\	緊張する
\\	明日
\\	カラオケ
\\	映画
\\	みます
\\	とりあえず
\\	会います
\\	もしもし、静さん。	
\\	もしもし、静さん。 
\\	はい。	
\\	はい。 
\\	こんばんは。リュウです。	
\\	こんばんは。リュウです。 
\\	あ...こんばんは...。	
\\	あ...こんばんは...。 
\\	電話番号、ありがとうございました。...はは。緊張しますね。	
\\	電話番号、ありがとうございました。...はは。緊張しますね。 
\\	あのぉ、明日、カラオケをしませんか。	
\\	あのぉ、明日、カラオケをしませんか。 
\\	カラオケですか...。 うーん。	
\\	カラオケですか...。 うーん。 
\\	あ...。好きじゃないですか?	
\\	あ...。好きじゃないですか? 
\\	じゃ、映画をみませんか。	
\\	じゃ、映画をみませんか。 
\\	スポーツ
\\	天気
\\	スノーボード
\\	今度
\\	一緒に
\\	スノボ
\\	好き(な)
\\	大好き(な)
\\	おはようございます、静さん。いい天気ですね。	
\\	おはようございます、静さん。いい天気ですね。 
\\	静さん…し・ず・かさん。	
\\	静さん…し・ず・かさん。 
\\	え?ああ、ロリーさん。	
\\	え?ああ、ロリーさん。 
\\	ああ、そうですね。いい天気ですね。	
\\	ああ、そうですね。いい天気ですね。 
\\	静さんはスポーツが好きですか。	
\\	静さんはスポーツが好きですか。 
\\	虫
\\	僕
\\	雪
\\	代わりに
\\	ラフティング
\\	嫌い
\\	昔
\\	おぼれます
\\	無理
\\	山
\\	ライフジャケット
\\	キャンプ
\\	スキー
\\	へぇー。スノボですか。いいですね。	
\\	へぇー。スノボですか。いいですね。 
\\	僕もスキーが大好きです。	
\\	僕もスキーが大好きです。 
\\	あ、でも、まだ雪がありますか。	
\\	あ、でも、まだ雪がありますか。 
\\	さあ…。	
\\	さあ…。 
\\	代わりに、今度ラフティングに行きませんか。	
\\	代わりに、今度ラフティングに行きませんか。 
\\	私は水が嫌いです。	
\\	私は水が嫌いです。 
\\	昔、おぼれました。	
\\	昔、おぼれました。 
\\	でも、ライフジャケット、ありますよ。	
\\	でも、ライフジャケット、ありますよ。 
\\	インド
\\	吸います
\\	名
\\	タバコ
\\	料理店
\\	料理
\\	カレー
\\	何名
\\	さっき
\\	本当に
\\	遅れる
\\	入ります
\\	静さん、すみません。遅れました。	
\\	静さん、すみません。遅れました。 
\\	本当にすみません。	
\\	本当にすみません。 
\\	いえ、大丈夫です。私もさっき来ました。	
\\	いえ、大丈夫です。私もさっき来ました。 
\\	じゃ、行きましょうか。	
\\	じゃ、行きましょうか。 
\\	何を食べましょうか。	
\\	何を食べましょうか。 
\\	カレーを食べませんか。	
\\	カレーを食べませんか。 
\\	カレー…あ、じゃあ、あのインド料理店に入りましょうか。	
\\	カレー…あ、じゃあ、あのインド料理店に入りましょうか。 
\\	いらっしゃいませ。何名様ですか。	
\\	いらっしゃいませ。何名様ですか。 
\\	二名です。	
\\	二名です。 
\\	注文
\\	コーヒー
\\	ライス
\\	ナン
\\	セット
\\	チキン
\\	カレー
\\	エビ
\\	お客様
\\	紅茶
\\	飲み物
\\	ラッシー
\\	ご注文は。	
\\	ご注文は。 
\\	チキンカレーセットお願いします。	
\\	チキンカレーセットお願いします。 
\\	ナンとライス、どちらがいいですか。	
\\	ナンとライス、どちらがいいですか。 
\\	ナンをお願いします。	
\\	ナンをお願いします。 
\\	お飲み物は?	
\\	お飲み物は? 
\\	何がありますか。	
\\	何がありますか。 
\\	コーヒー、紅茶、ラッシー があります。	
\\	コーヒー、紅茶、ラッシー があります。 
\\	ラッシーをください。	
\\	ラッシーをください。 
\\	お客様は?	
\\	お客様は? 
\\	私は...エビカレー、ナン、ラッシー お願いします。	
\\	私は...エビカレー、ナン、ラッシー お願いします。 
\\	お店
\\	入ります
\\	おごります
\\	ごちそうさま。
\\	初めて
\\	割り勘
\\	一緒
\\	別々
\\	会計
\\	結構です。
\\	お会計、三千円です。	
\\	お会計、三千円です。 
\\	別々でお願いします。	
\\	別々でお願いします。 
\\	おごりますよ。	
\\	おごりますよ。 
\\	一緒で、お願いします。	
\\	一緒で、お願いします。 
\\	いいですよ。結構です。	
\\	いいですよ。結構です。 
\\	割り勘で、お願いします。	
\\	割り勘で、お願いします。 
\\	いいですよ。今日は初めてのデートです。	
\\	いいですよ。今日は初めてのデートです。 
\\	店員
\\	ドレッシング
\\	オニオン
\\	セサミ
\\	ようこそ
\\	しょうゆ
\\	おすすめ
\\	~抜き
\\	野菜
\\	嫌い(な)
\\	パン
\\	かしこまりました。
\\	わさび
\\	カフェイノベーディブへようこそ。	
\\	カフェイノベーディブへようこそ。 
\\	ええっと・・・えびアボカド。	
\\	ええっと・・・えびアボカド。 
\\	かしこまりました。パンは?	
\\	かしこまりました。パンは? 
\\	セサミ。	
\\	セサミ。 
\\	かしこまりました。	
\\	かしこまりました。 
\\	嫌いな野菜、ありますか。	
\\	嫌いな野菜、ありますか。 
\\	ああ、オニオン抜き。	
\\	ああ、オニオン抜き。 
\\	ドレッシングは?	
\\	ドレッシングは? 
\\	うーんと・・・。おすすめは何ですか?	
\\	うーんと・・・。おすすめは何ですか? 
\\	わさびじょうゆです。	
\\	わさびじょうゆです。 
\\	じゃ、それ、お願いします。	
\\	じゃ、それ、お願いします。 
\\	カプチーノ
\\	ショート
\\	スモール
\\	ソイミルク
\\	アイス
\\	ホット
\\	店内
\\	持ち帰り
\\	豆乳
\\	お願いします
\\	サイズ
\\	かしこまりました
\\	カプチーノください。	
\\	カプチーノください。 
\\	ホットとアイスがありますが・・・。	
\\	ホットとアイスがありますが・・・。 
\\	アイスカプチーノをお願いします。	
\\	アイスカプチーノをお願いします。 
\\	サイズは?	
\\	サイズは? 
\\	スモール。	
\\	スモール。 
\\	あ、ショートですね。	
\\	あ、ショートですね。 
\\	あ、あと、豆乳でお願いします。	
\\	あ、あと、豆乳でお願いします。 
\\	はい。ソイミルクですね。かしこまりました。	
\\	はい。ソイミルクですね。かしこまりました。 
\\	お持ち帰りですか?	
\\	お持ち帰りですか? 
\\	店内で。	
\\	店内で。 
\\	お客様
\\	会います
\\	飲みます
\\	お酒
\\	全然
\\	読みます
\\	本
\\	みます
\\	映画
\\	どうも
\\	嘘つき
\\	コーヒーのお客様。	
\\	コーヒーのお客様。 
\\	あ、僕です。どうも。	
\\	あ、僕です。どうも。 
\\	・・・静さんは、よく映画をみますか。	
\\	・・・静さんは、よく映画をみますか。 
\\	・・・いえ、あまり。	
\\	・・・いえ、あまり。 
\\	・・・よく本を読みますか。	
\\	・・・よく本を読みますか。 
\\	いえ、全然読みません。	
\\	いえ、全然読みません。 
\\	・・・じゃ、お酒はどうですか。	
\\	・・・じゃ、お酒はどうですか。 
\\	よく飲みますか。	
\\	よく飲みますか。 
\\	あのぉ・・・リュウさんは、ロリーさんとよく会いますか。	
\\	あのぉ・・・リュウさんは、ロリーさんとよく会いますか。 
\\	女
\\	お世話になりました。
\\	さようなら
\\	帰ります
\\	全部
\\	どうして
\\	ききます
\\	タイプ
\\	彼女
\\	嘘
\\	嘘じゃないです。	
\\	嘘じゃないです。 
\\	ロリーは僕の彼女じゃないです。女友達です。	
\\	ロリーは僕の彼女じゃないです。女友達です。 
\\	全然タイプじゃないですよ。	
\\	全然タイプじゃないですよ。 
\\	へぇー。そうですか。	
\\	へぇー。そうですか。 
\\	聞きましたか。ロリーさん。	
\\	聞きましたか。ロリーさん。 
\\	ええ。聞きました。	
\\	ええ。聞きました。 
\\	色々
\\	一緒に
\\	いただきます。
\\	すごい
\\	焼きます
\\	クッキー
\\	手作り
\\	中
\\	お礼
\\	しょっぱい
\\	はーい、あ、ロリーさん。	
\\	はーい、あ、ロリーさん。 
\\	こんばんは。静さん。	
\\	こんばんは。静さん。 
\\	色々ありがとうございました。	
\\	色々ありがとうございました。 
\\	これ、どうぞ。お礼です。	
\\	これ、どうぞ。お礼です。 
\\	ええ?いいですよ・・・。	
\\	ええ?いいですよ・・・。 
\\	でも・・・中は何ですか?	
\\	でも・・・中は何ですか? 
\\	手作りクッキーです。	
\\	手作りクッキーです。 
\\	クッキーを焼きました。	
\\	クッキーを焼きました。 
\\	楽しい
\\	会います
\\	また
\\	色々
\\	メール
\\	毎日
\\	こちらこそ
\\	パスポート
\\	お元気で。
\\	空港
\\	皆さん
\\	水木さん、静さん、色々お世話になりました。	
\\	水木さん、静さん、色々お世話になりました。 
\\	こちらこそ。毎日、楽しかったです。	
\\	こちらこそ。毎日、楽しかったです。 
\\	メールしますね。	
\\	メールしますね。 
\\	また、会いましょう。	
\\	また、会いましょう。 
\\	はい。また日本に来ます。	
\\	はい。また日本に来ます。 
\\	皆さん、お元気で。	
\\	皆さん、お元気で。 
\\	ロリーさんもお元気で。	
\\	ロリーさんもお元気で。 
\\	本当にありがとうございました。じゃあ、また。	
\\	本当にありがとうございました。じゃあ、また。 
\\	おはよう
\\	おはようございます
\\	先生
\\	ごめん
\\	ごめんなさい
\\	すみません
\\	おはよう、赤木さん、赤木くん。	
\\	おはよう、赤木さん、赤木くん。 
\\	おはようございます。	
\\	おはようございます。 
\\	おはよう、黒川先生。	
\\	おはよう、黒川先生。 
\\	ケント、「ございます」。	
\\	"ケント、「ございます」。 
\\	ああ、ごめん、先生。おはようございます。	
\\	ああ、ごめん、先生。おはようございます。 
\\	ケント、ケント、「すみません」。	
\\	"ケント、ケント、「すみません」。 
\\	よろしく
\\	彼氏
\\	君
\\	これ
\\	いとこ
\\	私の
\\	私
\\	白山君!	
\\	白山君! 
\\	おお、まどかちゃん。おはよ。	
\\	おお、まどかちゃん。おはよ。 
\\	白山君、私のいとこ、ケント君。	
\\	白山君、私のいとこ、ケント君。 
\\	ケント、これ、私の彼氏、白山君。	
\\	ケント、これ、私の彼氏、白山君。 
\\	よろしく。	
\\	よろしく。 
\\	あ、よろしく。	
\\	あ、よろしく。 
\\	お父さん
\\	日本人
\\	お母さん
\\	イギリス人
\\	俺
\\	僕
\\	みなさん。こちらは、赤木ケント君。	
\\	みなさん。こちらは、赤木ケント君。 
\\	赤木君のお父さんは、日本人。お母さんはイギリス人です。	
\\	赤木君のお父さんは、日本人。お母さんはイギリス人です。 
\\	おれは・・・	
\\	おれは・・・ 
\\	ぼくは・・・	
\\	ぼくは・・・ 
\\	そば
\\	腹
\\	へった
\\	これ
\\	それ
\\	山菜
\\	食堂
\\	いただきます。
\\	ああ、腹へった。	
\\	ああ、腹へった。 
\\	ねぇ。私も、お腹へった。	
\\	ねぇ。私も、お腹へった。 
\\	いただきます。	
\\	いただきます。 
\\	いただきまーす。	
\\	いただきまーす。 
\\	まどか、それ、そば?	
\\	まどか、それ、そば? 
\\	うん、そば。	
\\	うん、そば。 
\\	これ、山菜そば。	
\\	これ、山菜そば。 
\\	さんさい?	
\\	さんさい? 
\\	うん。	
\\	うん。 
\\	山菜
\\	おいしい
\\	野菜
\\	カレー
\\	いまいち
\\	まあまあ
\\	山菜は何?	
\\	"山菜は何? 
\\	うーん。。。山の野菜?	
\\	うーん。。。山の野菜? 
\\	へー。それ、おいしい?	
\\	へー。それ、おいしい? 
\\	うん、まあまあ。	
\\	うん、まあまあ。 
\\	ケントのは何?カレー?	
\\	ケントのは何?カレー? 
\\	うん。カツカレー。	
\\	うん。カツカレー。 
\\	それ、おいしい?	
\\	それ、おいしい? 
\\	うーん。いまいち。	
\\	うーん。いまいち。 
\\	ここ
\\	体育館
\\	大きい
\\	新しい
\\	古い
\\	プール
\\	そこ
\\	ここは、体育館。	
\\	ここは、体育館。 
\\	ふーん。あまり、おおきくないね。	
\\	ふーん。あまり、おおきくないね。 
\\	そうね。	
\\	そうね。 
\\	あたらしい?	
\\	あたらしい? 
\\	あたらしくない。ふるい。	
\\	あたらしくない。ふるい。 
\\	へー。	
\\	へー。 
\\	敬語
\\	彼氏
\\	いとこ
\\	ハンサム(な)
\\	ふつう
\\	先輩
\\	まどか!	
\\	まどか! 
\\	ああ、えみ先輩。こんにちは。	
\\	ああ、えみ先輩。こんにちは。 
\\	彼氏?	
\\	彼氏? 
\\	彼氏じゃないです。いとこです。	
\\	彼氏じゃないです。いとこです。 
\\	ケント、こちら、青田エミ先輩。	
\\	ケント、こちら、青田エミ先輩。 
\\	はじめまして。ケントです。	
\\	はじめまして。ケントです。 
\\	どーも。うわ~。ケント君、ハンサム~。	
\\	どーも。うわ~。ケント君、ハンサム~。 
\\	ハンサムじゃない。ふつう、ふつう。	
\\	ハンサムじゃない。ふつう、ふつう。 
\\	ケント!敬語!	
\\	ケント!敬語! 
\\	ああ、ハンサムじゃないです。ふつうです。	
\\	ああ、ハンサムじゃないです。ふつうです。 
\\	授業
\\	先生
\\	大丈夫
\\	英語
\\	難しい
\\	他
\\	簡単
\\	何
\\	赤木君、赤木ケントくん。	
\\	赤木君、赤木ケントくん。 
\\	なに?	
\\	なに? 
\\	「なに」じゃない、「なんですか」です。	
\\	"「なに」じゃない、「なんですか」です。 
\\	すみません。	
\\	すみません。 
\\	なんですか。先生。	
\\	なんですか。先生。 
\\	大丈夫ですか。	
\\	大丈夫ですか。 
\\	授業は難しいですか。	
\\	授業は難しいですか。 
\\	英語は難しくないです。	
\\	英語は難しくないです。 
\\	好き(な)
\\	嫌い(な)
\\	大好き(な)
\\	大嫌い(な)
\\	ただいま
\\	おかえりなさい
\\	ただいま~。	
\\	ただいま~。 
\\	ただいま~。	
\\	ただいま~。 
\\	おかえりなさい。	
\\	おかえりなさい。 
\\	・・・ママ今日・・・カレー?	
\\	・・・ママ今日・・・カレー? 
\\	そう。ケント君、カレー好き?	
\\	そう。ケント君、カレー好き? 
\\	あ・・・	
\\	あ・・・ 
\\	嫌い?	
\\	嫌い? 
\\	電話
\\	元気
\\	お母さん
\\	お父さん
\\	こんばんは
\\	もしもし
\\	うん
\\	ふーん
\\	はい、赤木です。	
\\	はい、赤木です。 
\\	あら。こんばんは。元気ですか?・・・	
\\	あら。こんばんは。元気ですか?・・・ 
\\	ええ。ケント君も元気ですよ。・・・	
\\	ええ。ケント君も元気ですよ。・・・ 
\\	はーい。	
\\	はーい。 
\\	ケント君、ケント君。お父さんから電話。	
\\	ケント君、ケント君。お父さんから電話。 
\\	あ、もしもし?・・・	
\\	あ、もしもし?・・・ 
\\	うん。元気。お父さんは?・・・	
\\	うん。元気。お父さんは?・・・ 
\\	ふーん。お母さんも元気?	
\\	ふーん。お母さんも元気? 
\\	今
\\	午前
\\	楽しい
\\	学校
\\	けど
\\	両親
\\	やさしい
\\	おやすみ
\\	今?九時。	
\\	今?九時。 
\\	・・・うん、午後九時半。イギリスは今、何時?	
\\	・・・うん、午後九時半。イギリスは今、何時? 
\\	・・・十二時半?午前?へー。	
\\	・・・十二時半?午前?へー。 
\\	・・・学校?うん、難しい。	
\\	・・・学校?うん、難しい。 
\\	英語は簡単だけど、他は難しいよ。	
\\	英語は簡単だけど、他は難しいよ。 
\\	・・・うん、楽しいよ。	
\\	・・・うん、楽しいよ。 
\\	まどかちゃんも、まどかちゃんの両親も、やさしいよ。	
\\	まどかちゃんも、まどかちゃんの両親も、やさしいよ。 
\\	・・・じゃあね。はい。おやすみ。	
\\	・・・じゃあね。はい。おやすみ。 
\\	明日
\\	夜
\\	暇
\\	ママ
\\	誕生日
\\	二十日
\\	一日
\\	ケント、明日の夜、ひま?	
\\	ケント、明日の夜、ひま? 
\\	うん。でも、なんで。	
\\	うん。でも、なんで。 
\\	明日、ママの誕生日。	
\\	明日、ママの誕生日。 
\\	何日?	
\\	何日? 
\\	3月20日。	
\\	3月20日。 
\\	はつか?	
\\	はつか? 
\\	うん。にじゅうにち。	
\\	うん。にじゅうにち。 
\\	ケントのたんじょうびはいつ?	
\\	ケントのたんじょうびはいつ? 
\\	7月1日。	
\\	7月1日。 
\\	ああ、ついたちね。	
\\	ああ、ついたちね。 
\\	花屋
\\	ケーキ
\\	いらっしゃいませ
\\	チョコレート
\\	いらっしゃいませ。	
\\	いらっしゃいませ。 
\\	これ、ください。	
\\	これ、ください。 
\\	いらっしゃいませ。	
\\	いらっしゃいませ。 
\\	すみません。	
\\	すみません。 
\\	このチョコレートケーキを二つ、このチーズケーキを三つください。	
\\	このチョコレートケーキを二つ、このチーズケーキを三つください。 
\\	チョコレートケーキを二つとチーズケーキを三つですね。	
\\	チョコレートケーキを二つとチーズケーキを三つですね。 
\\	はい。	
\\	はい。 
\\	お誕生日おめでとうございます。
\\	おめでとう。
\\	ありがとう。
\\	仏様
\\	花
\\	うれしい
\\	ママ、誕生日おめでとう。	
\\	ママ、誕生日おめでとう。 
\\	あら?ケーキ?ありがとう!	
\\	あら?ケーキ?ありがとう! 
\\	おばさん、お誕生日おめでとうございます。	
\\	おばさん、お誕生日おめでとうございます。 
\\	これ、お花です。	
\\	これ、お花です。 
\\	あら・・・あ	
\\	あら・・・あ 
\\	なにこれー。	
\\	なにこれー。 
\\	ケント、これ、仏様のお花だよ。	
\\	ケント、これ、仏様のお花だよ。 
\\	なにそれ?	
\\	なにそれ? 
\\	まどか、大丈夫よ。	
\\	まどか、大丈夫よ。 
\\	ありがとう、ケント君。うれしいわ。	
\\	ありがとう、ケント君。うれしいわ。 
\\	紅茶
\\	飲む
\\	コーラ
\\	ミルク
\\	砂糖
\\	使う
\\	いる
\\	お願い
\\	どうぞ
\\	ママ、紅茶、飲む?	
\\	ママ、紅茶、飲む? 
\\	うん。お願い。	
\\	うん。お願い。 
\\	ミルクとおさとう、使う?	
\\	ミルクとおさとう、使う? 
\\	うん。使う。	
\\	うん。使う。 
\\	はい、どうぞ。	
\\	はい、どうぞ。 
\\	ケントも紅茶、飲む?	
\\	ケントも紅茶、飲む? 
\\	いや。コーラ飲む。	
\\	いや。コーラ飲む。 
\\	コーラとケーキ?	
\\	コーラとケーキ? 
\\	めちゃくちゃ
\\	うまい
\\	甘い
\\	もの
\\	食べる
\\	食べます
\\	もっと
\\	このケーキ、めちゃくちゃうまい!	
\\	このケーキ、めちゃくちゃうまい! 
\\	あら、ケント君、甘いもの食べるの?	
\\	あら、ケント君、甘いもの食べるの? 
\\	はい。食べます。	
\\	はい。食べます。 
\\	もっと食べる?	
\\	もっと食べる? 
\\	うん。おかわり。	
\\	うん。おかわり。 
\\	・・・あ・・・おかわりおねがいします。	
\\	・・・あ・・・おかわりおねがいします。 
\\	はい、どうぞ。	
\\	はい、どうぞ。 
\\	あいさつ
\\	うまい
\\	いい
\\	サッカー
\\	する
\\	行く
\\	おーっす	
\\	おーっす 
\\	あ・・・おはよ。	
\\	あ・・・おはよ。 
\\	おーす?	
\\	おーす? 
\\	あいさつだよ。あいさつ。おーっす。	
\\	あいさつだよ。あいさつ。おーっす。 
\\	おーっす。	
\\	おーっす。 
\\	そうそう。うまいうまい。	
\\	そうそう。うまいうまい。 
\\	なぁ、ケント。ケント、サッカーする?	
\\	なぁ、ケント。ケント、サッカーする? 
\\	うん。するよ。	
\\	うん。するよ。 
\\	今日の昼、サッカーするよ。来る?	
\\	今日の昼、サッカーするよ。来る? 
\\	いいの?行く、行く!やったー。	
\\	いいの?行く、行く!やったー。 
\\	やった
\\	昼休み
\\	後で
\\	学食
\\	~から
\\	パン
\\	買う
\\	じゃあね
\\	やったーお昼休みだー。	
\\	やったーお昼休みだー。 
\\	ケント、学食行く?	
\\	ケント、学食行く? 
\\	行かない。	
\\	行かない。 
\\	なんで?	
\\	なんで? 
\\	サッカーするから。	
\\	サッカーするから。 
\\	ふーん。おなかすかないの?	
\\	ふーん。おなかすかないの? 
\\	後で、パン買うよ。じゃあね。	
\\	後で、パン買うよ。じゃあね。 
\\	眠い
\\	起きる
\\	何時
\\	出ます
\\	明日
\\	朝
\\	おばさん
\\	おやすみなさい。
\\	寝る
\\	もう
\\	絶対
\\	ふぁー。ねむい・・・。	
\\	ふぁー。ねむい・・・。 
\\	もう、ねる。お休みなさい。	
\\	もう、ねる。お休みなさい。 
\\	お休みー。	
\\	お休みー。 
\\	お休みなさい。	
\\	お休みなさい。 
\\	あ、おばさん、明日の朝、サッカーするから、六時に出ます。	
\\	あ、おばさん、明日の朝、サッカーするから、六時に出ます。 
\\	じゃ、何時に起きるの?	
\\	じゃ、何時に起きるの? 
\\	五時半に起きます。	
\\	五時半に起きます。 
\\	ぜったい、起きないよ。	
\\	ぜったい、起きないよ。 
\\	起きるよ。	
\\	起きるよ。 
\\	あと
\\	十分
\\	遅刻する
\\	まじ
\\	今
\\	あと、10分・・・。	
\\	あと、10分・・・。 
\\	ケント・・・来ないなぁ・・・。	
\\	ケント・・・来ないなぁ・・・。 
\\	ケント!ケント!遅刻するよ。	
\\	ケント!ケント!遅刻するよ。 
\\	大丈夫・・・遅刻しないよ。(ムニャムニャ)	
\\	大丈夫・・・遅刻しないよ。(ムニャムニャ) 
\\	・・・え?今、何時?	
\\	・・・え?今、何時? 
\\	七時半。	
\\	七時半。 
\\	ええ?!まじで?!	
\\	ええ?!まじで?! 
\\	本当に
\\	練習
\\	無理
\\	月曜日
\\	いつ
\\	次
\\	おそようございます。ケントさん。	
\\	おそようございます。ケントさん。 
\\	あーごめん!本当にごめん!	
\\	あーごめん!本当にごめん! 
\\	次は行く。絶対行く。	
\\	次は行く。絶対行く。 
\\	無理、無理。	
\\	無理、無理。 
\\	次の練習はいつ?	
\\	次の練習はいつ? 
\\	月曜の朝。	
\\	月曜の朝。 
\\	おはよー。	
\\	おはよー。 
\\	おはようございます。エミ先輩。	
\\	おはようございます。エミ先輩。 
\\	おーっす。	
\\	おーっす。 
\\	ケント!	
\\	ケント! 
\\	部
\\	もちろん
\\	土日
\\	すごい
\\	夕方
\\	金曜日
\\	練習する
\\	毎朝
\\	入る
\\	うそ!
\\	じゃ、ケント君はサッカー部に入るの?	
\\	じゃ、ケント君はサッカー部に入るの? 
\\	はい。入ります。	
\\	はい。入ります。 
\\	毎朝、練習するの?	
\\	毎朝、練習するの? 
\\	はい。月曜日から金曜日まで朝と夕方、練習します。	
\\	はい。月曜日から金曜日まで朝と夕方、練習します。 
\\	すごーい。月から金まで?	
\\	すごーい。月から金まで? 
\\	土日は?	
\\	土日は? 
\\	もちろん、練習するよ。	
\\	もちろん、練習するよ。 
\\	え?うそ!?ほんと?	
\\	え?うそ!?ほんと? 
\\	おごる
\\	返す
\\	小銭
\\	すぐ
\\	細かい
\\	もう少し
\\	一万
\\	金
\\	悪い
\\	財布
\\	会計
\\	肉まん
\\	ある
\\	(ぐー)腹へったー!	
\\	(ぐー)腹へったー! 
\\	なぁ、ケント、おごるから、肉まん食べない?	
\\	なぁ、ケント、おごるから、肉まん食べない? 
\\	いいね。	
\\	いいね。 
\\	肉まん、2つください。	
\\	肉まん、2つください。 
\\	はい。お会計315円です。	
\\	はい。お会計315円です。 
\\	はい・・・あれ?財布がない。	
\\	はい・・・あれ?財布がない。 
\\	ケント、悪い。金ある?	
\\	ケント、悪い。金ある? 
\\	あ、ある。はい。	
\\	あ、ある。はい。 
\\	1万円・・・ですか。	
\\	1万円・・・ですか。 
\\	もう少し細かいのありませんか。	
\\	もう少し細かいのありませんか。 
\\	ケント、小銭ある?	
\\	ケント、小銭ある? 
\\	ああ、あるある、はい。500円。	
\\	ああ、あるある、はい。500円。 
\\	サンキュー。すぐ返す。	
\\	サンキュー。すぐ返す。 
\\	学校
\\	いる
\\	練習中
\\	どこ
\\	なるほど
\\	電話する
\\	携帯
\\	はい、赤木です。	
\\	はい、赤木です。 
\\	あ、まどかちゃん?ケントの父です。	
\\	あ、まどかちゃん?ケントの父です。 
\\	ケントいる?	
\\	ケントいる? 
\\	今、ちょっと・・・いません。	
\\	今、ちょっと・・・いません。 
\\	いないの?どこにいるの?	
\\	いないの?どこにいるの? 
\\	学校にいます。	
\\	学校にいます。 
\\	サッカーの練習中です。	
\\	サッカーの練習中です。 
\\	あ、なるほど。	
\\	あ、なるほど。 
\\	じゃ、ケントの携帯に電話するね。	
\\	じゃ、ケントの携帯に電話するね。 
\\	はい。お願いします。	
\\	はい。お願いします。 
\\	遠い
\\	もって行く
\\	試合
\\	週末
\\	何
\\	京都
\\	仕事
\\	今週
\\	時間
\\	残念
\\	もしもし?	
\\	もしもし? 
\\	あ、ケント?今時間ある?	
\\	あ、ケント?今時間ある? 
\\	うん。あるけど・・・何?	
\\	うん。あるけど・・・何? 
\\	今週、お父さん、仕事で京都に行くから、京都に来ないか?	
\\	今週、お父さん、仕事で京都に行くから、京都に来ないか? 
\\	京都?遠いなぁ。	
\\	京都?遠いなぁ。 
\\	週末、サッカーの試合があるから・・・。	
\\	週末、サッカーの試合があるから・・・。 
\\	そうか・・・残念だなぁ。	
\\	そうか・・・残念だなぁ。 
\\	母さんのショートブレッドを持って行くけど・・・。	
\\	母さんのショートブレッドを持って行くけど・・・。 
\\	え?あ・・・行く。行く。	
\\	え?あ・・・行く。行く。 
\\	下さい
\\	どうも
\\	どうも有り難う。
\\	はい
\\	いいえ。
\\	すみません。
\\	お願いします。
\\	赤坂
\\	大丈夫
\\	駄目(な)
\\	蛸
\\	鮪
\\	海栗
\\	お薦めをお願いします。
\\	いらっしゃいませ。
\\	いらっしゃい。
\\	たこは大丈夫?	
\\	たこは大丈夫? 
\\	大丈夫。	
\\	大丈夫。 
\\	まぐろは大丈夫?	
\\	まぐろは大丈夫? 
\\	大丈夫。	
\\	大丈夫。 
\\	ウニは大丈夫?	
\\	ウニは大丈夫? 
\\	駄目、駄目、駄目。	
\\	駄目、駄目、駄目。 
\\	乾杯!
\\	頂きます。
\\	美味しい
\\	凄い。
\\	お腹いっぱい。
\\	お会計
\\	・でした 
\\	ご馳走様。
\\	乾杯	
\\	乾杯 
\\	頂きます	
\\	頂きます 
\\	美味しい	
\\	美味しい 
\\	凄い	
\\	凄い 
\\	お腹いっぱい	
\\	お腹いっぱい 
\\	お会計	
\\	お会計 
\\	ご馳走様	
\\	ご馳走様 
\\	乾杯	
\\	乾杯 
\\	頂きます	
\\	頂きます 
\\	美味しい	
\\	美味しい 
\\	凄い	
\\	凄い 
\\	お腹いっぱい	
\\	お腹いっぱい 
\\	お会計	
\\	お会計 
\\	ご馳走様	
\\	ご馳走様 
\\	お手洗い
\\	お名前
\\	彼
\\	誰
\\	トイレ
\\	駅弁
\\	新幹線
\\	タクシー
\\	京王プラザ
\\	まで
\\	どうも有り難うございます。
\\	有り難うございます。
\\	有り難う。
\\	どうも。
\\	乗り場
\\	タクシー乗り場
\\	領収書
\\	英語
\\	大丈夫
\\	~行き
\\	地名
\\	違います
\\	朝飯
\\	前
\\	朝飯前
\\	ゆっくり
\\	言います
\\	お願いします。
\\	日本語で
\\	すみません。これは英語で何と言いますか。	
\\	すみません。これは英語で何と言いますか。 
\\	うーん。	
\\	うーん。 
\\	じゃぁ、これは日本語で何と言いますか。	
\\	じゃぁ、これは日本語で何と言いますか。 
\\	これはたこわさびです。	
\\	これはたこわさびです。 
\\	すみません。ゆっくりおねがいします。	
\\	すみません。ゆっくりおねがいします。 
\\	これはたこわさびです。	
\\	これはたこわさびです。 
\\	ゆっくりおねがいします。	
\\	ゆっくりおねがいします。 
\\	た・こ・わ・さ・び。	
\\	た・こ・わ・さ・び。 
\\	た・こ・わ・さ・び?	
\\	た・こ・わ・さ・び? 
\\	そうです。	
\\	そうです。 
\\	ありがとうございます。	
\\	ありがとうございます。 
\\	どういたしまして。	
\\	どういたしまして。 
\\	読みます
\\	何
\\	これ
\\	それ
\\	英語
\\	肉
\\	喫煙
\\	禁煙
\\	お願いします
\\	~抜き
\\	かしこまりました。
\\	いらっしゃいませ。何名様ですか。	
\\	いらっしゃいませ。何名様ですか。 
\\	三人です。	
\\	三人です。 
\\	はい。かしこまりました。喫煙ですか。禁煙ですか。	
\\	はい。かしこまりました。喫煙ですか。禁煙ですか。 
\\	禁煙です。	
\\	禁煙です。 
\\	はい。かしこまりました。	
\\	はい。かしこまりました。 
\\	ホテル
\\	旅館
\\	浴衣
\\	民宿
\\	田舎
\\	ペンション
\\	ビジネスホテル
\\	カプセルホテル
\\	「です」	でございます
\\	いらっしゃいませ。
\\	ようこそ
\\	名
\\	お客様
\\	綴り
\\	名前
\\	泊
\\	グランドホテルへようこそ、いらっしゃいませ。	
\\	グランドホテルへようこそ、いらっしゃいませ。 
\\	チェックインお願いします。	
\\	チェックインお願いします。 
\\	はい。お名前をお願いします。	
\\	はい。お名前をお願いします。 
\\	ロバーツです。	
\\	ロバーツです。 
\\	お名前のつづりをお願いします。	
\\	お名前のつづりをお願いします。 
\\	ありがとうございます。ロバーツ様、2名様3泊でよろしいでしょうか?	
\\	ありがとうございます。ロバーツ様、2名様3泊でよろしいでしょうか? 
\\	はい。	
\\	はい。 
\\	キーでございます。	
\\	キーでございます。 
\\	ありがとう。	
\\	ありがとう。 
\\	下さい
\\	エイアンドダブリュウ
\\	ファーストキッチン
\\	ウェンディーズ
\\	フレッシュネスバーガー
\\	ケンタッキー
\\	マクドナルド
\\	ファーストフード
\\	持ち帰り
\\	テイクアウト
\\	店内装飾
\\	ブルーシールアイスクリーム
\\	いらっしゃいませ。	
\\	いらっしゃいませ。 
\\	英語のメニューをお願いします。	
\\	英語のメニューをお願いします。 
\\	店内でお召し上がりですか。	
\\	店内でお召し上がりですか。 
\\	店内で。・ここで。・テイクアウト。・持ち帰り。	
\\	店内で。・ここで。・テイクアウト。・持ち帰り。 
\\	ごめんなさい。
\\	注文
\\	どうぞ
\\	単品
\\	決まり
\\	円
\\	座る
\\	いらっしゃいませ。店内でお召し上がりですか。	
\\	いらっしゃいませ。店内でお召し上がりですか。 
\\	ご注文をどうぞ。	
\\	ご注文をどうぞ。 
\\	ご注文はお決まりですか。	
\\	ご注文はお決まりですか。 
\\	モスチーズバーガーを一つお願いします。	
\\	モスチーズバーガーを一つお願いします。 
\\	ポテトを単品でお願いします。	
\\	ポテトを単品でお願いします。 
\\	オニオン抜きでお願いします。	
\\	オニオン抜きでお願いします。 
\\	アイスティを氷なしでお願いします。	
\\	アイスティを氷なしでお願いします。 
\\	以上です。	
\\	以上です。 
\\	五百三十円でございます。	
\\	五百三十円でございます。 
\\	手紙
\\	封筒
\\	葉書
\\	絵葉書
\\	航空便
\\	エアメール
\\	船便
\\	普通郵便
\\	速達
\\	銀行
\\	両替
\\	トラベラー(ズ)チェック
\\	現金
\\	交換レート
\\	手数料
\\	お金
\\	換える
\\	郵便局
\\	円
\\	両替
\\	警察
\\	呼ぶ
\\	泥棒
\\	ひったくり
\\	痴漢
\\	助けて。
\\	救急車
\\	痛い
\\	胸焼け
\\	売り切れ
\\	酔い止め
\\	目薬
\\	薬局
\\	風邪薬
\\	頭痛薬
\\	胃薬
\\	江戸っ子
\\	栄養補給
\\	医薬品
\\	医薬部外品
\\	マムシ
\\	スッポン
\\	スズメバチ
\\	真っすぐ
\\	曲がる
\\	左
\\	右
\\	前
\\	後ろ
\\	となり
\\	横
\\	すみません、東京駅はどこですか。	
\\	すみません、東京駅はどこですか。 
\\	まっすぐ行って右に曲がって左にあります。	
\\	まっすぐ行って右に曲がって左にあります。 
\\	すみません、皇居はどこですか。	
\\	すみません、皇居はどこですか。 
\\	まっすぐ行ってあそこにあります。	
\\	まっすぐ行ってあそこにあります。 
\\	すみません、日枝神社はどこですか。	
\\	すみません、日枝神社はどこですか。 
\\	まっすぐ行って左に曲がってそこにあります。	
\\	まっすぐ行って左に曲がってそこにあります。 
\\	いらっしゃいませ。
\\	バス停
\\	駅
\\	銀行
\\	側
\\	右側
\\	信号
\\	3番目
\\	ここ
\\	今
\\	私達
\\	地図
\\	少々
\\	行きたい
\\	公園
\\	上野
\\	タクシー乗り場
\\	いらっしゃいませ!	
\\	いらっしゃいませ! 
\\	すみません、上野公園に行きたいんですが。	
\\	すみません、上野公園に行きたいんですが。 
\\	少々お待ちください。こちらが地図です。私達は今ここです。	
\\	少々お待ちください。こちらが地図です。私達は今ここです。 
\\	ちょっと
\\	お釣り
\\	押して
\\	入れて
\\	お金
\\	そば
\\	暖かい
\\	違う
\\	全部
\\	美味しい
\\	どれ
\\	客
\\	店員
\\	分からない
\\	忘れないでください。
\\	すみません、ちょっと分からないんですが。	
\\	すみません、ちょっと分からないんですが。 
\\	はい、何がいいですか。	
\\	はい、何がいいですか。 
\\	どれが美味しいですか。	
\\	どれが美味しいですか。 
\\	全部美味しいですよ。	
\\	全部美味しいですよ。 
\\	はい、全部お願いします。	
\\	はい、全部お願いします。 
\\	いや、違う違う。じゃ、暖かいものがいいですか。	
\\	いや、違う違う。じゃ、暖かいものがいいですか。 
\\	はい。	
\\	はい。 
\\	じゃ、そばが美味しいですよ。	
\\	じゃ、そばが美味しいですよ。 
\\	はい、お願いします。	
\\	はい、お願いします。 
\\	お金をここに入れて、ここを押してください。お釣りを忘れないでください。	
\\	お金をここに入れて、ここを押してください。お釣りを忘れないでください。 
\\	どうもありがとう。	
\\	どうもありがとう。 
\\	天気
\\	明後日
\\	明日
\\	近くにある
\\	百円ショップ
\\	降る
\\	かなり
\\	必要
\\	傘
\\	雨
\\	京都
\\	おはようございます。	
\\	おはようございます。 
\\	おはようございます。	
\\	おはようございます。 
\\	すみません、今日の天気はどうですか。	
\\	すみません、今日の天気はどうですか。 
\\	今日は雨です。	
\\	今日は雨です。 
\\	傘が必要ですか。	
\\	傘が必要ですか。 
\\	そうですね。かなり降ってます。	
\\	そうですね。かなり降ってます。 
\\	百円ショップは近くにありますか。	
\\	百円ショップは近くにありますか。 
\\	ありますよ。そこをまっすぐ行って、右にあります。	
\\	ありますよ。そこをまっすぐ行って、右にあります。 
\\	ありがとうございます。	
\\	ありがとうございます。 
\\	あ!お天気おじさんだ。	
\\	あ!お天気おじさんだ。 
\\	お!こんにちは。	
\\	お!こんにちは。 
\\	すみません、今日の天気はどうですか。	
\\	すみません、今日の天気はどうですか。 
\\	今日は雨です。	
\\	今日は雨です。 
\\	明日の天気はどうですか。	
\\	明日の天気はどうですか。 
\\	明日も雨です。	
\\	明日も雨です。 
\\	明後日の京都の天気はどうですか。	
\\	明後日の京都の天気はどうですか。 
\\	明後日の京都の天気も雨です。	
\\	明後日の京都の天気も雨です。 
\\	えー?	
\\	えー? 
\\	ジャケット
\\	必要
\\	傘
\\	午前中
\\	でしょう
\\	晴れ
\\	晴れる
\\	正午
\\	午後
\\	朝
\\	午前
\\	上着
\\	コート
\\	何か
\\	すみません。今日の天気はどうですか。	
\\	すみません。今日の天気はどうですか。 
\\	午前中は雨ですが、午後は晴れます。	
\\	午前中は雨ですが、午後は晴れます。 
\\	傘は必要ですか。	
\\	傘は必要ですか。 
\\	そうですね。大丈夫でしょう。	
\\	そうですね。大丈夫でしょう。 
\\	どうもありがとう。	
\\	どうもありがとう。 
\\	会話2	
\\	すみません。	
\\	すみません。 
\\	はい。	
\\	はい。 
\\	明日、富士山に登るんですが、静岡の天気はどうですか。	
\\	明日、富士山に登るんですが、静岡の天気はどうですか。 
\\	明日、静岡は晴れですよ。	
\\	明日、静岡は晴れですよ。 
\\	何が必要ですか。	
\\	何が必要ですか。 
\\	そうですね。寒いので上着が必要でしょう。	
\\	そうですね。寒いので上着が必要でしょう。 
\\	どうもありがとうございます。	
\\	どうもありがとうございます。 
\\	フィファワールドカップ
\\	シュート
\\	ヘディング
\\	フリーキック
\\	間接フリーキック
\\	コーナーキック
\\	ペナルティーキック
\\	パス
\\	タックル
\\	イエローカード
\\	レッドカード
\\	ゴールキーパー
\\	退場
\\	ハーフタイム
\\	クリア
\\	反則
\\	ファール
\\	日本代表
\\	ドイツ大会
\\	ジーコジャパン
\\	サムライブルー
\\	中田英寿
\\	中村俊輔
\\	高原直泰
\\	川口能活
\\	自殺点
\\	オウンゴール
\\	得点
\\	ゴール
\\	オフサイド
\\	スローイン
\\	電車
\\	快速電車
\\	急行電車
\\	特急電車
\\	短期滞在
\\	鉄道
\\	鉄
\\	地下
\\	地下鉄
\\	新幹線
\\	各駅停車
\\	みどりの窓口
\\	禁煙席
\\	喫煙席
\\	通路側の席
\\	窓側の席
\\	席
\\	自由席
\\	指定席
\\	グリーン車
\\	往復
\\	片道
\\	改札口
\\	予約
\\	すみません。	
\\	すみません。 
\\	はい。	
\\	はい。 
\\	みどりの窓口はどこにありますか。	
\\	みどりの窓口はどこにありますか。 
\\	ここまっすぐ行って左の改札口の隣にあります。	
\\	ここまっすぐ行って左の改札口の隣にあります。 
\\	会話2	
\\	すみません、京都へ行きたいんですが。	
\\	すみません、京都へ行きたいんですが。 
\\	はい、かしこまりました。	
\\	はい、かしこまりました。 
\\	パスがありますが。	
\\	パスがありますが。 
\\	はい、かしこまりました。	
\\	はい、かしこまりました。 
\\	禁煙席を予約したいんですが。	
\\	禁煙席を予約したいんですが。 
\\	はい、少々お待ちください。二時十分発車です。三番線でございます。ありがとうございます。	
\\	はい、少々お待ちください。二時十分発車です。三番線でございます。ありがとうございます。 
\\	会話3	
\\	すみません、京都までお願いします。	
\\	すみません、京都までお願いします。 
\\	はい、かしこまりました。	
\\	はい、かしこまりました。 
\\	パスがありますが。	
\\	パスがありますが。 
\\	はい、少々お待ちください。	
\\	はい、少々お待ちください。 
\\	窓側の席をお願いします。	
\\	窓側の席をお願いします。 
\\	はい。二時十分発車です。三番線でございます。ありがとうございます。	
\\	はい。二時十分発車です。三番線でございます。ありがとうございます。 
\\	いくら
\\	まで
\\	秋葉原
\\	円
\\	駅員
\\	客
\\	独身
\\	困る
\\	すみません、秋葉原までいくらですか。	
\\	すみません、秋葉原までいくらですか。 
\\	秋葉原までは百三十円でございます。	
\\	秋葉原までは百三十円でございます。 
\\	もう一度おねがいします。	
\\	もう一度おねがいします。 
\\	百三十円でございます。	
\\	百三十円でございます。 
\\	すみません、書いてください。	
\\	すみません、書いてください。 
\\	はい、どうぞ。	
\\	はい、どうぞ。 
\\	すみません。	
\\	すみません。 
\\	日本語がちょっと分からないんですが。	
\\	日本語がちょっと分からないんですが。 
\\	秋葉原まで行きたいんですが。	
\\	秋葉原まで行きたいんですが。 
\\	海の日
\\	乗り換え
\\	コインロッカー
\\	成田空港
\\	山手線
\\	日暮里
\\	京成スカイライナー
\\	すみません。成田空港へ行きたいんですが。	
\\	すみません。成田空港へ行きたいんですが。 
\\	山手線で日暮里へ行って、日暮里で乗り換えて、京成スカイライナーで一本です。	
\\	山手線で日暮里へ行って、日暮里で乗り換えて、京成スカイライナーで一本です。 
\\	すみません。どこで乗り換えたらいいですか。	
\\	すみません。どこで乗り換えたらいいですか。 
\\	日暮里です。	
\\	日暮里です。 
\\	山手線は何番ですか
\\	山手線は何番ですか
\\	13番です。そちらです。	
\\	13番です。そちらです。 
\\	ありがとうございます。	
\\	ありがとうございます。 
\\	バス
\\	両替
\\	いくら
\\	取ってください
\\	乗車券
\\	お客さん
\\	四人分
\\	三人分
\\	二人分
\\	どっち
\\	行きます
\\	行き
\\	バス停
\\	次の
\\	海老
\\	大豆
\\	甲殻類
\\	貝類
\\	貝
\\	魚
\\	柑橘類
\\	お酒
\\	小麦
\\	乳製品
\\	牛乳
\\	ナッツ
\\	玉子(卵)
\\	しょうゆ
\\	すみません。シーフードピザ一つお願いします。	
\\	すみません。シーフードピザ一つお願いします。 
\\	はい。かしこまりました。	
\\	はい。かしこまりました。 
\\	すみません。海老アレルギーなんです。	
\\	すみません。海老アレルギーなんです。 
\\	いいえ。入っていません。	
\\	いいえ。入っていません。 
\\	はい、入っています。	
\\	はい、入っています。 
\\	海老抜きでお願いします。	
\\	海老抜きでお願いします。 
\\	はい、かしこまりました。	
\\	はい、かしこまりました。 
\\	はじめまして。
\\	ずんだもち
\\	牛タン
\\	モンテネグロ
\\	セルビア
\\	ベネズエラ
\\	フランス
\\	アルゼンチン
\\	レバノン
\\	ナイジェリア
\\	お願いします
\\	よろしく
\\	~人
\\	国名
\\	名前
\\	私
\\	仙台
\\	写真
\\	とる
\\	一緒に
\\	バスケットボール
\\	イタリア
\\	リトアニア
\\	ニュージーランド
\\	パナマ
\\	プエルトリコ
\\	カタール
\\	セネガル
\\	方法
\\	探す
\\	日本
\\	ドイツ
\\	応援
\\	チーム
\\	アメリカ
\\	トルコ
\\	スペイン
\\	スロべニア
\\	アンゴラ
\\	ブラジル
\\	中国
\\	席
\\	ちぐささん、どこのチームを応援していますか。	
\\	ちぐささん、どこのチームを応援していますか。 
\\	日本です。	
\\	日本です。 
\\	すみません。浜松アリーナへ行きたいんですが。	
\\	すみません。浜松アリーナへ行きたいんですが。 
\\	ここから東海道線に乗って、天竜川で降りてください。そこから、歩いて5分です。	
\\	ここから東海道線に乗って、天竜川で降りてください。そこから、歩いて5分です。 
\\	すみません。もう一度お願いします。ゆっくりお願いします。	
\\	すみません。もう一度お願いします。ゆっくりお願いします。 
\\	ここから東海道線に乗って、天竜川で降りてください。そこから、歩いて5分です。	
\\	ここから東海道線に乗って、天竜川で降りてください。そこから、歩いて5分です。 
\\	ありがとうございます。すみません。他の方法はありますか。	
\\	ありがとうございます。すみません。他の方法はありますか。 
\\	はい、ここから、バスに乗って浜松アリーナ駅で降りてください。そこから、歩いて1分です。	
\\	はい、ここから、バスに乗って浜松アリーナ駅で降りてください。そこから、歩いて1分です。 
\\	ありがとうございました。	
\\	ありがとうございました。 
\\	アパート
\\	反対側
\\	札
\\	乗車券
\\	反対
\\	側
\\	右側
\\	左側
\\	バスカード
\\	会話1	
\\	会話1 
\\	すみません。バス停を探しているんですが。	
\\	すみません。バス停を探しているんですが。 
\\	このアパートの反対側にありますよ。	
\\	このアパートの反対側にありますよ。 
\\	ありがとうございます。	
\\	ありがとうございます。 
\\	会話2	
\\	会話2 
\\	すみません。このバスは池袋に行きますか。	
\\	すみません。このバスは池袋に行きますか。 
\\	いいえ、この次のバスに乗ってください。	
\\	いいえ、この次のバスに乗ってください。 
\\	はい。ありがとうございます。	
\\	はい。ありがとうございます。 
\\	会話3	
\\	会話3 
\\	次は池袋でございます。	
\\	次は池袋でございます。 
\\	すみません。次はどこですか。	
\\	すみません。次はどこですか。 
\\	次は池袋です。	
\\	次は池袋です。 
\\	ありがとうございます。	
\\	ありがとうございます。 
\\	会話4	
\\	会話4 
\\	開く
\\	お入りください
\\	入る
\\	どうぞ
\\	失礼する
\\	失礼
\\	邪魔する
\\	邪魔
\\	開いている
\\	御免ください
\\	開いてるよ。	
\\	開いてるよ。 
\\	お邪魔します。	
\\	お邪魔します。 
\\	お邪魔します。	
\\	お邪魔します。 
\\	失礼します。	
\\	失礼します。 
\\	どうぞお入りください。	
\\	どうぞお入りください。 
\\	御免ください。	
\\	御免ください。 
\\	どうぞ
\\	邪魔
\\	邪魔する
\\	つまらない
\\	わざわざ
\\	構う
\\	こんにちは。	
\\	こんにちは。 
\\	あ、高瀬さん、どうぞお上がり下さい。	
\\	あ、高瀬さん、どうぞお上がり下さい。 
\\	お邪魔します。これ、つまらない物ですがどうぞ。	
\\	お邪魔します。これ、つまらない物ですがどうぞ。 
\\	どうもすみません。わざわざありがとうございます。	
\\	どうもすみません。わざわざありがとうございます。 
\\	お寺
\\	お参り
\\	仕方
\\	本堂
\\	合わせる
\\	建物
\\	最初
\\	すみません。お参りの仕方を教えてください。	
\\	すみません。お参りの仕方を教えてください。 
\\	最初にこの水で手を洗います。その後本堂に行きます。	
\\	最初にこの水で手を洗います。その後本堂に行きます。 
\\	すみません、もう一度お願いします。ゆっくりお願いします。	
\\	すみません、もう一度お願いします。ゆっくりお願いします。 
\\	最初に、この水で手を洗います。	
\\	最初に、この水で手を洗います。 
\\	最初に、この水で手を洗います。はいわかりました。	
\\	最初に、この水で手を洗います。はいわかりました。 
\\	その後本堂に行きます。	
\\	その後本堂に行きます。 
\\	すみません、本堂は何ですか?	
\\	すみません、本堂は何ですか? 
\\	あの建物のことです。	
\\	あの建物のことです。 
\\	はい、わかりました。	
\\	はい、わかりました。 
\\	そこでお金を入れて、手を合わせてお参りします。	
\\	そこでお金を入れて、手を合わせてお参りします。 
\\	本殿
\\	鈴
\\	鳴らす
\\	礼をする
\\	叩く
\\	二回
\\	すみません。お参りの仕方を教えてください。	
\\	すみません。お参りの仕方を教えてください。 
\\	あ、いいですよ。最初にこの水で手を洗います。	
\\	あ、いいですよ。最初にこの水で手を洗います。 
\\	すみません、もう一度お願いします。ゆっくりお願いします。	
\\	すみません、もう一度お願いします。ゆっくりお願いします。 
\\	最初にこの水で手を洗います。	
\\	最初にこの水で手を洗います。 
\\	最初にこの水で手を洗います。はいわかりました。	
\\	最初にこの水で手を洗います。はいわかりました。 
\\	その後本殿に行きます。	
\\	その後本殿に行きます。 
\\	すみません、本殿は何ですか?	
\\	すみません、本殿は何ですか? 
\\	あの建物のことです。	
\\	あの建物のことです。 
\\	はい、わかりました。	
\\	はい、わかりました。 
\\	そこでお金を入れて、鈴を鳴らします。	
\\	そこでお金を入れて、鈴を鳴らします。 
\\	そこでお金を入れて、鈴を鳴らします。	
\\	そこでお金を入れて、鈴を鳴らします。 
\\	待ち合わせ
\\	宜しい
\\	待つ
\\	お客様
\\	いっらしゃいませ!	
\\	いっらしゃいませ! 
\\	待ち合わせなんですが。	
\\	待ち合わせなんですが。 
\\	お客様のお名前をよろしいですか。	
\\	お客様のお名前をよろしいですか。 
\\	武田です。	
\\	武田です。 
\\	少々お待ちください。	
\\	少々お待ちください。 
\\	はい、どうぞ、こちらへ。	
\\	はい、どうぞ、こちらへ。 
\\	唐揚げ
\\	焼きそば
\\	焼うどん
\\	塩焼きそば
\\	ソース焼きそば
\\	申し訳ございません
\\	頼む
\\	お待たせしました、唐揚げと、焼きそばと、焼うどんでございます。	
\\	お待たせしました、唐揚げと、焼きそばと、焼うどんでございます。 
\\	すみません、焼うどんは注文していないんですけど。	
\\	すみません、焼うどんは注文していないんですけど。 
\\	え、申し訳ございません。失礼しました。	
\\	え、申し訳ございません。失礼しました。 
\\	忘れ物
\\	色
\\	赤
\\	忘れる
\\	物
\\	パスポート
\\	財布
\\	携帯
\\	傘
\\	すみません。忘れ物を探しているんですが。	
\\	すみません。忘れ物を探しているんですが。 
\\	こちらへ。	
\\	こちらへ。 
\\	はい、こんにちは。忘れ物ですか?	
\\	はい、こんにちは。忘れ物ですか? 
\\	はい、傘を。	
\\	はい、傘を。 
\\	何色ですか?	
\\	何色ですか? 
\\	赤です。	
\\	赤です。 
\\	これですか?	
\\	これですか? 
\\	はい、ありがとうございます。	
\\	はい、ありがとうございます。 
\\	お土産
\\	一番人気
\\	人気
\\	家族
\\	見せる
\\	いらっしゃいませ。	
\\	いらっしゃいませ。 
\\	家族にお土産を買いたいんですが。	
\\	家族にお土産を買いたいんですが。 
\\	こちらです。	
\\	こちらです。 
\\	いくらですか。	
\\	いくらですか。 
\\	2000円です。	
\\	2000円です。 
\\	見せてください。	
\\	見せてください。 
\\	はい、どうぞ。	
\\	はい、どうぞ。 
\\	名物
\\	1500円
\\	いらっしゃいませ。	
\\	いらっしゃいませ。 
\\	家族にお土産を買いたいんですが。	
\\	家族にお土産を買いたいんですが。 
\\	こちらです。	
\\	こちらです。 
\\	いくらですか。	
\\	いくらですか。 
\\	1500円です。	
\\	1500円です。 
\\	見せてください。	
\\	見せてください。 
\\	はい、どうぞ。	
\\	はい、どうぞ。 
\\	一括払い
\\	払い
\\	一回
\\	何回
\\	分括払い
\\	リボ払い
\\	すみません、カードは大丈夫ですか。	
\\	すみません、カードは大丈夫ですか。 
\\	はい、大丈夫です。	
\\	はい、大丈夫です。 
\\	一回です。	
\\	一回です。 
\\	両替
\\	五千円
\\	二千円
\\	千円
\\	五百円
\\	百円
\\	五十円
\\	十円
\\	五円
\\	一円
\\	崩す
\\	お断りする
\\	お断り
\\	一万円
\\	すみません。両替お願いします。	
\\	すみません。両替お願いします。 
\\	申し訳ございません。	
\\	申し訳ございません。 
\\	すみません。これを崩してください。	
\\	すみません。これを崩してください。 
\\	テレホンカード
\\	千円
\\	カード
\\	円
\\	千
\\	いっらしゃいませ。
\\	公衆
\\	公衆電話
\\	いっらしゃいませ!	
\\	いっらしゃいませ! 
\\	テレホンカードはありますか?	
\\	テレホンカードはありますか? 
\\	はい、あります。	
\\	はい、あります。 
\\	1000円のカードをお願いします。	
\\	1000円のカードをお願いします。 
\\	はい、どうぞ。	
\\	はい、どうぞ。 
\\	公衆電話はどこですか。	
\\	公衆電話はどこですか。 
\\	借りる
\\	五千円カード
\\	三千円カード
\\	プリペイドフォン
\\	プリペイド携帯電話
\\	クレジットカード
\\	パスポート
\\	携帯
\\	携帯電話
\\	少々
\\	いらっしゃいませ。	
\\	いらっしゃいませ。 
\\	携帯を借りたいんですが。	
\\	携帯を借りたいんですが。 
\\	はい。	
\\	はい。 
\\	英語でも大丈夫ですか。	
\\	英語でも大丈夫ですか。 
\\	はい、少々お待ちください。	
\\	はい、少々お待ちください。 
\\	いらっしゃいませ。	
\\	いらっしゃいませ。 
\\	プリペイドフォンはありますか。	
\\	プリペイドフォンはありますか。 
\\	はい、あります。	
\\	はい、あります。 
\\	いくらですか。	
\\	いくらですか。 
\\	3000円と7000円です。	
\\	3000円と7000円です。 
\\	3000円をお願いします。	
\\	3000円をお願いします。 
\\	はい、かしこまりました。	
\\	はい、かしこまりました。 
\\	3000円カードをお願いします。	
\\	3000円カードをお願いします。 
\\	はい、かしこまりました。	
\\	はい、かしこまりました。 
\\	温める
\\	お願いします
\\	円
\\	下さい
\\	すみません。これをお願いします。	
\\	すみません。これをお願いします。 
\\	はい、720円でございます。	
\\	はい、720円でございます。 
\\	すみません。これを温めて下さい。	
\\	すみません。これを温めて下さい。 
\\	はい、どうぞ。	
\\	はい、どうぞ。 
\\	はじめまして
\\	びっくり
\\	お上手
\\	日本語
\\	とても
\\	ジョニー・デップ
\\	くださる
\\	お越しくださいました
\\	ようこそ
\\	難しい
\\	はじめまして。ようこそお越しくださいました。	
\\	はじめまして。ようこそお越しくださいました。 
\\	はじめまして。ジョニー・デップです。よろしくお願いします。	
\\	はじめまして。ジョニー・デップです。よろしくお願いします。 
\\	とても日本語がお上手ですね。びっくりしました。	
\\	とても日本語がお上手ですね。びっくりしました。 
\\	いいえ。日本語はとても難しいです。	
\\	いいえ。日本語はとても難しいです。 
\\	今回
\\	映画
\\	プロモーション
\\	来日
\\	皆さん
\\	一言
\\	是非
\\	今回は、新しい映画のプロモーションのための来日ですね。	
\\	今回は、新しい映画のプロモーションのための来日ですね。 
\\	はい。そうです。	
\\	はい。そうです。 
\\	では、日本のファンの皆さんに一言お願いします。	
\\	では、日本のファンの皆さんに一言お願いします。 
\\	日本の皆さん、こんにちは!皆さんにまたお会いできて、とてもうれしいです。是非、新しい「パイレーツオブカリビアン」を見てください!	
\\	"日本の皆さん、こんにちは!皆さんにまたお会いできて、とてもうれしいです。是非、新しい「パイレーツオブカリビアン」を見てください! 
\\	大好き
\\	きれい
\\	国
\\	そして
\\	食べ物
\\	とても
\\	おいしい
\\	いかが
\\	今回は何回目の来日ですか?	
\\	今回は何回目の来日ですか? 
\\	3回目です。	
\\	3回目です。 
\\	日本はいかがですか?	
\\	日本はいかがですか? 
\\	日本が大好きです。日本はきれいな国です。そして、食べ物がとてもおいしいです。	
\\	日本が大好きです。日本はきれいな国です。そして、食べ物がとてもおいしいです。 
\\	大勢
\\	うれしい
\\	多い
\\	子ども
\\	感激
\\	来る
\\	皆さん
\\	空港
\\	ファン
\\	本当に
\\	日本に大勢のファンがいますね。	
\\	日本に大勢のファンがいますね。 
\\	空港に大勢の皆さんが来てくれました。とても感激しました!	
\\	空港に大勢の皆さんが来てくれました。とても感激しました! 
\\	子どもたちのファンも多いですよね。	
\\	子どもたちのファンも多いですよね。 
\\	とてもうれしいです。日本のファンの皆さん、本当にありがとうございます!	
\\	とてもうれしいです。日本のファンの皆さん、本当にありがとうございます! 
\\	滞在中
\\	プライベート
\\	京都
\\	歴史
\\	興味
\\	是非
\\	神社
\\	滞在中、プライベートで、行ってみたいところはありますか。	
\\	滞在中、プライベートで、行ってみたいところはありますか。 
\\	はい。京都に行ってみたいです。	
\\	はい。京都に行ってみたいです。 
\\	なぜですか?	
\\	なぜですか? 
\\	日本の歴史に興味があります。	
\\	日本の歴史に興味があります。 
\\	東京でも、明治神宮という大きな神社がありますよ。	
\\	東京でも、明治神宮という大きな神社がありますよ。 
\\	そうですか!是非、行ってみます。ありがとうございます。	
\\	そうですか!是非、行ってみます。ありがとうございます。 
\\	少し
\\	個人的な
\\	聞く
\\	生まれ変わる
\\	俳優
\\	道
\\	選ぶ
\\	簡単
\\	戻る
\\	少し個人的なことを聞いてもよろしいですか?	
\\	少し個人的なことを聞いてもよろしいですか? 
\\	はい。どうぞ。	
\\	はい。どうぞ。 
\\	もし、生まれ変わったら、また俳優の道を選びますか?	
\\	もし、生まれ変わったら、また俳優の道を選びますか? 
\\	すみません。もう少し簡単な日本語でお願いします。	
\\	すみません。もう少し簡単な日本語でお願いします。 
\\	もう一度子どもに戻ったら、また俳優になりたいですか?	
\\	もう一度子どもに戻ったら、また俳優になりたいですか? 
\\	なるほど、はい、通訳者お願いします。	
\\	なるほど、はい、通訳者お願いします。 
\\	以上
\\	お疲れ様
\\	実は
\\	大感激
\\	滞在
\\	インタビューは以上です。長い時間、お疲れ様でした。	
\\	インタビューは以上です。長い時間、お疲れ様でした。 
\\	お疲れ様でした。	
\\	お疲れ様でした。 
\\	実は、私はジョニーさんの大ファンなので、お話ができて大感激です!	
\\	実は、私はジョニーさんの大ファンなので、お話ができて大感激です! 
\\	私もとても楽しかったです。ありがとうございました。	
\\	私もとても楽しかったです。ありがとうございました。 
\\	それでは、日本での滞在を楽しんでください!	
\\	それでは、日本での滞在を楽しんでください! 
\\	またお会いできる日を楽しみにしています。	
\\	またお会いできる日を楽しみにしています。 
\\	鳥居
\\	入り口
\\	そっくり(な)
\\	サイン
\\	似ている
\\	明治神宮はどこですか?	
\\	明治神宮はどこですか? 
\\	あの鳥居が入り口です。	
\\	あの鳥居が入り口です。 
\\	ありがとうございます。	
\\	ありがとうございます。 
\\	パターン
\\	あー、いつも間違われます。彼は私の兄です。そっくりでしょ。	
\\	あー、いつも間違われます。彼は私の兄です。そっくりでしょ。 
\\	キャー!!ジョニーデップの弟さん!サインください!	
\\	キャー!!ジョニーデップの弟さん!サインください! 
\\	パターン
\\	似てるだけです。	
\\	似てるだけです。
\\	なーんだ。	
\\	なーんだ。 
\\	握手
\\	お願い
\\	喜んで
\\	そろそろ
\\	応援
\\	すみません、サインをいただけますか?	
\\	すみません、サインをいただけますか? 
\\	はい、いいですよ。	
\\	はい、いいですよ。 
\\	あのー、握手もお願いできますか?	
\\	あのー、握手もお願いできますか? 
\\	(握手する)	
\\	はーい、皆さん、そろそろお時間ですので、これで最後になります。	
\\	はーい、皆さん、そろそろお時間ですので、これで最後になります。 
\\	エー!!そんなぁー!	
\\	エー!!そんなぁー! 
\\	本当にごめんなさい。これからも応援よろしくお願いします!	
\\	本当にごめんなさい。これからも応援よろしくお願いします! 
\\	ロビー
\\	お礼
\\	あふれる
\\	お陰
\\	時間
\\	出発
\\	一言
\\	過ごす
\\	戻る
\\	うわぁ、ロビーがファンであふれていますよ。	
\\	うわぁ、ロビーがファンであふれていますよ。 
\\	うれしいなぁ。お礼を言う時間はあるかな?	
\\	うれしいなぁ。お礼を言う時間はあるかな? 
\\	(出発ロビーの入り口にて)	
\\	皆さん、見送りにきてくれて本当にありがとうございます!皆さんのお陰で、とても楽しく過ごすことができました。また必ず皆さんに会いに日本に戻ってきます。さようなら!	
\\	皆さん、見送りにきてくれて本当にありがとうございます!皆さんのお陰で、とても楽しく過ごすことができました。また必ず皆さんに会いに日本に戻ってきます。さようなら! 
\\	どうも
\\	ありがとう。
\\	どうもありがとうございます。
\\	どういたしまして
\\	いいえ、いいえ。
\\	とんでもないです
\\	こちらこそ
\\	これ下さい。
\\	切符
\\	それ
\\	これ
\\	下さい
\\	違います
\\	お願い
\\	お願いします。
\\	それ下さい。
\\	地図
\\	おはようございます
\\	今日は
\\	今晩は
\\	今日
\\	おかげ様で。
\\	はい
\\	お元気ですか。
\\	元気
\\	さようなら
\\	バイバイ。
\\	じゃあね
\\	失礼します。
\\	失礼致します。
\\	お疲れ様でした。
\\	トイレはどこですか?
\\	トイレ
\\	どこ
\\	お手洗い
\\	女
\\	男
\\	すみません。
\\	分かりません。
\\	日本語
\\	日本語は分かりません。
\\	スペイン語
\\	英語
\\	話せます
\\	スペイン語
\\	フランス語
\\	もう
\\	一度
\\	言う
\\	下さい
\\	ゆっくり
\\	話す
\\	すみません
\\	ごめんなさい
\\	いらっしゃいませ。
\\	ありがとう ございました。
\\	いらっしゃい。
\\	何
\\	様
\\	二
\\	名
\\	禁煙
\\	喫煙
\\	どちら
\\	一
\\	杯
\\	枚
\\	名
\\	十
\\	九
\\	八
\\	七
\\	六
\\	五
\\	四
\\	三
\\	二
\\	個
\\	まける
\\	下さい
\\	もっと
\\	お店
\\	あっちの
\\	円
\\	円
\\	百
\\	千
\\	万
\\	ご縁
\\	五円
\\	ネットカフェ
\\	パスワード
\\	ユーザーネーム
\\	します
\\	接続する
\\	ネット
\\	あります
\\	ワイファイ
\\	この
\\	どこ
\\	あるいは
\\	漫画喫茶
\\	お願い
\\	2枚
\\	成田空港
\\	まで
\\	切符
\\	バス停
\\	名前
\\	は
\\	何
\\	です
\\	か
\\	私
\\	の
\\	マイケル
\\	ピーター
\\	キャサリン
\\	マルコ
\\	名前は何ですか。	
\\	名前は何ですか。 
\\	私の名前は、マイケルです。	
\\	私の名前は、マイケルです。 
\\	私の名前は、ピーターです。	
\\	私の名前は、ピーターです。 
\\	私の名前は、キャサリンです。	
\\	私の名前は、キャサリンです。 
\\	私の名前は、マルコです。	
\\	私の名前は、マルコです。 
\\	です
\\	どこ
\\	は
\\	出身
\\	か
\\	アメリカ
\\	メキシコ
\\	フランス
\\	ベトナム
\\	出身はどこですか。	
\\	出身はどこですか。 
\\	アメリカです。	
\\	アメリカです。 
\\	メキシコです。	
\\	メキシコです。 
\\	フランスです。	
\\	フランスです。 
\\	ベトナムです。	
\\	ベトナムです。 
\\	マドリード
\\	リオデジャネイロ
\\	ニューヨーク
\\	どこ
\\	住んでいます
\\	か
\\	に
\\	ソウル
\\	どこに住んでいますか。	
\\	どこに住んでいますか。 
\\	ニューヨークに住んでいます。	
\\	ニューヨークに住んでいます。 
\\	ソウルに住んでいます。	
\\	ソウルに住んでいます。 
\\	マドリードに住んでいます。	
\\	マドリードに住んでいます。 
\\	リオデジャネイロに住んでいます。	
\\	リオデジャネイロに住んでいます。 
\\	銀行員
\\	か
\\	は
\\	仕事
\\	何
\\	教師
\\	会社員
\\	弁護士
\\	です
\\	仕事は何ですか。	
\\	仕事は何ですか。 
\\	教師です。	
\\	教師です。 
\\	会社員です。	
\\	会社員です。 
\\	弁護士です。	
\\	弁護士です。 
\\	銀行員です。	
\\	銀行員です。 
\\	話せません
\\	を
\\	日本語
\\	ちょっと
\\	はい
\\	話せます
\\	いいえ
\\	か
\\	まあまあ
\\	流暢に
\\	全然
\\	日本語を話せますか。	
\\	日本語を話せますか。 
\\	はい、ちょっと話せます。	
\\	はい、ちょっと話せます。 
\\	いいえ、話せません。	
\\	いいえ、話せません。 
\\	はい、まあまあ話せます。	
\\	はい、まあまあ話せます。 
\\	はい、流暢に話せます。	
\\	はい、流暢に話せます。 
\\	いいえ、全然話せません。	
\\	いいえ、全然話せません。 
\\	十年
\\	半年
\\	二年
\\	間
\\	一か月
\\	日本語
\\	どのくらい
\\	か
\\	を
\\	勉強しています
\\	どのくらい日本語を勉強していますか。	
\\	どのくらい日本語を勉強していますか。 
\\	一か月間です。	
\\	一か月間です。 
\\	二年間です。	
\\	二年間です。 
\\	半年間です。	
\\	半年間です。 
\\	十年間です。	
\\	十年間です。 
\\	日本
\\	日本語
\\	を
\\	学びました
\\	か
\\	どこ
\\	学校
\\	語学学校
\\	大学
\\	で
\\	どこで日本語を学びましたか。	
\\	どこで日本語を学びましたか。 
\\	学校で学びました。	
\\	学校で学びました。 
\\	日本で学びました。	
\\	日本で学びました。 
\\	語学学校で学びました。	
\\	語学学校で学びました。 
\\	大学で学びました。	
\\	大学で学びました。 
\\	食べられません
\\	体
\\	おいしい
\\	いい
\\	大好き
\\	食べられます
\\	日本食
\\	まあまあ
\\	が
\\	に
\\	か
\\	日本食が食べられますか。	
\\	日本食が食べられますか。 
\\	はい、おいしいです。	
\\	はい、おいしいです。 
\\	はい、大好きです。	
\\	はい、大好きです。 
\\	はい、体にいいです。	
\\	はい、体にいいです。 
\\	まあまあです。	
\\	まあまあです。 
\\	寿司
\\	か
\\	特に
\\	が
\\	好き
\\	ラーメン
\\	日本食
\\	たこ焼き
\\	です
\\	日本食が好きですか。	
\\	日本食が好きですか。 
\\	はい、特に寿司が好きです。	
\\	はい、特に寿司が好きです。 
\\	はい、特にたこ焼きが好きです。	
\\	はい、特にたこ焼きが好きです。 
\\	はい、特にしゃぶしゃぶが好きです。	
\\	はい、特にしゃぶしゃぶが好きです。 
\\	はい、特にラーメンが好きです。	
\\	はい、特にラーメンが好きです。 
\\	すっぱい
\\	おいしい
\\	辛い
\\	です
\\	か
\\	どう
\\	苦い
\\	とても
\\	どうですか。	
\\	どうですか。 
\\	とてもおいしいです。	
\\	とてもおいしいです。 
\\	辛いです。	
\\	辛いです。 
\\	すっぱいです。	
\\	すっぱいです。 
\\	苦いです。	
\\	苦いです。 
\\	七
\\	五
\\	です
\\	四
\\	十
\\	八
\\	か
\\	何歳
\\	六
\\	二
\\	歳
\\	何歳ですか。	
\\	何歳ですか。 
\\	18歳です。	
\\	18歳です。 
\\	25歳です。	
\\	25歳です。 
\\	47歳です。	
\\	47歳です。 
\\	60歳です。	
\\	60歳です。 
\\	日
\\	か
\\	は
\\	いつ
\\	誕生日
\\	月
\\	です
\\	誕生日は、いつですか。	
\\	誕生日は、いつですか。 
\\	8月22日です。	
\\	8月22日です。 
\\	5月30日です。	
\\	5月30日です。 
\\	9月3日です。	
\\	9月3日です。 
\\	11月24日です。	
\\	11月24日です。 
\\	姉
\\	妹
\\	兄
\\	は
\\	いる
\\	兄弟
\\	弟
\\	か
\\	兄弟はいますか。	
\\	兄弟はいますか。 
\\	妹がいます。	
\\	妹がいます。 
\\	兄がいます。	
\\	兄がいます。 
\\	姉がいます。	
\\	姉がいます。 
\\	弟がいます。	
\\	弟がいます。 
\\	初めまして
\\	国籍
\\	名前
\\	片仮名
\\	じゃあね
\\	またね
\\	さようなら
\\	人 (じん)
\\	インド
\\	オーストラリア
\\	イギリス
\\	カナダ
\\	アメリカ
\\	宜しくお願いします
\\	私
\\	日本
\\	初めまして。私は夏子です。宜しくお願いします。	
\\	初めまして。私は夏子です。宜しくお願いします。 
\\	初めまして。私はピーターです。宜しくお願いします。	
\\	初めまして。私はピーターです。宜しくお願いします。 
\\	酒井
\\	一徳
\\	川本
\\	夏子
\\	ゆっくり
\\	はじめまして。私は酒井一徳です。宜しくお願いします。	
\\	はじめまして。私は酒井一徳です。宜しくお願いします。 
\\	こちらこそ、はじめまして。私は川本夏子です。どうぞ宜しくお願いします。	
\\	こちらこそ、はじめまして。私は川本夏子です。どうぞ宜しくお願いします。 
\\	私
\\	貴方
\\	彼
\\	彼女
\\	あなたたち
\\	彼ら
\\	彼女ら
\\	です
\\	私達
\\	アメリカ人
\\	ノルウェー人
\\	も
\\	面白い
\\	楽しい
\\	忙しい
\\	寒い
\\	日本人
\\	私は夏子です。あなたはアメリカ人ですか。	
\\	私は夏子です。あなたはアメリカ人ですか。 
\\	はい。私はアメリカ人です。	
\\	はい。私はアメリカ人です。 
\\	私はピーターです。あなたは日本人ですか。	
\\	私はピーターです。あなたは日本人ですか。 
\\	はい。私は日本人です。	
\\	はい。私は日本人です。 
\\	貴方
\\	オードリー・ヘップバーン
\\	マドンナ
\\	トム・ハンクス
\\	ニコラス・ケイジ
\\	ブラッド・ピット
\\	さん
\\	何処
\\	学生
\\	忙しい
\\	誰
\\	サンドラ・ブロック
\\	川本夏子さんはどこですか。	
\\	川本夏子さんはどこですか。 
\\	川本さんは忙しいです。	
\\	川本さんは忙しいです。 
\\	酒井一徳さんはどこですか。	
\\	酒井一徳さんはどこですか。 
\\	酒井さんも忙しいです。	
\\	酒井さんも忙しいです。 
\\	あなたは誰ですか。	
\\	あなたは誰ですか。 
\\	私は鈴木さくらです。	
\\	私は鈴木さくらです。 
\\	よろしくお願いします。	
\\	よろしくお願いします。 
\\	よろしくお願いします。	
\\	よろしくお願いします。 
\\	サンタさん
\\	メリークリスマス
\\	天皇
\\	誕生日
\\	天皇誕生日
\\	表参道
\\	サンタクロース
\\	紅葉
\\	です
\\	でございます
\\	お早う
\\	今晩は
\\	今日は
\\	すみません
\\	素敵(な)
\\	有り難う
\\	とても
\\	美しい
\\	うるさい
\\	すみません。あなたは素敵ですね。	
\\	すみません。あなたは素敵ですね。 
\\	ありがとう。	
\\	ありがとう。 
\\	あなたはとても素敵ですね。	
\\	あなたはとても素敵ですね。 
\\	ありがとう。	
\\	ありがとう。 
\\	あなたはきれいですね。	
\\	あなたはきれいですね。 
\\	ありがとう。	
\\	ありがとう。 
\\	あなたはとてもきれいですね。	
\\	あなたはとてもきれいですね。 
\\	ありがとう。	
\\	ありがとう。 
\\	あなたは美しいです。	
\\	あなたは美しいです。 
\\	ありがとう。	
\\	ありがとう。 
\\	あなたはとても美しいです。	
\\	あなたはとても美しいです。 
\\	あなた達はうるさい!	
\\	あなた達はうるさい! 
\\	明ける
\\	おめでとう。
\\	今年
\\	宜しく
\\	お願い
\\	大晦日
\\	紅白
\\	歌
\\	年越し
\\	蕎麦
\\	お年玉
\\	合戦
\\	イタリア
\\	タイ
\\	フランス
\\	料理
\\	好き
\\	何
\\	日本
\\	日本料理
\\	モスバーガー
\\	とても
\\	インド
\\	日本が好きですか。	
\\	日本が好きですか。 
\\	はい、とても好きです。	
\\	はい、とても好きです。 
\\	日本料理が好きですか。	
\\	日本料理が好きですか。 
\\	はい、とても好きです。	
\\	はい、とても好きです。 
\\	何が好きですか	
\\	何が好きですか 
\\	モスバーガーが好きです。	
\\	モスバーガーが好きです。 
\\	嫌い
\\	豆腐
\\	好き
\\	大嫌い
\\	刺身
\\	納豆
\\	寿司
\\	豆腐は好きですか?	
\\	豆腐は好きですか? 
\\	はい、とても好きです。	
\\	はい、とても好きです。 
\\	寿司は好きですか?	
\\	寿司は好きですか? 
\\	いいえ、好きではありません。	
\\	いいえ、好きではありません。 
\\	刺身は好きですか?	
\\	刺身は好きですか? 
\\	いいえ、嫌いです。	
\\	いいえ、嫌いです。 
\\	納豆は好きですか	
\\	納豆は好きですか 
\\	いいえ、大嫌いです。	
\\	いいえ、大嫌いです。 
\\	美味しい
\\	大好き
\\	すごい
\\	ものすごい
\\	よ
\\	これは何ですか。	
\\	これは何ですか。 
\\	これは寿司です。	
\\	これは寿司です。 
\\	寿司はおいしいですか。	
\\	寿司はおいしいですか。 
\\	はい、おいしいですよ。	
\\	はい、おいしいですよ。 
\\	寿司は好きですか。	
\\	寿司は好きですか。 
\\	はい、好きです。	
\\	はい、好きです。 
\\	それは何ですか。	
\\	それは何ですか。 
\\	それは刺身です。	
\\	それは刺身です。 
\\	刺身はおいしいですか。	
\\	刺身はおいしいですか。 
\\	はい、すごくおいしいですよ。	
\\	はい、すごくおいしいですよ。 
\\	刺身は好きですか。	
\\	刺身は好きですか。 
\\	はい、とても好きです。	
\\	はい、とても好きです。 
\\	あれは何ですか。	
\\	あれは何ですか。 
\\	あれは天ぷらです。	
\\	あれは天ぷらです。 
\\	天ぷらはおいしいですか。	
\\	天ぷらはおいしいですか。 
\\	はい、ものすごくおいしいです。	
\\	はい、ものすごくおいしいです。 
\\	天ぷらは好きですか。	
\\	天ぷらは好きですか。 
\\	はい、大好きです。	
\\	はい、大好きです。 
\\	出身
\\	イギリスのマンチェスター
\\	カナダのトロント
\\	ニュージーランドのオークランド
\\	ブラジルのサンパウロ
\\	ノルウェーのオスロ
\\	ご出身はどこですか。	
\\	ご出身はどこですか。 
\\	アメリカのニューヨークです。ご出身はどこですか。	
\\	アメリカのニューヨークです。ご出身はどこですか。 
\\	日本の東京です。	
\\	日本の東京です。 
\\	フィッシュ アンド チップス
\\	相撲
\\	クリケット
\\	カレー
\\	野球
\\	ハンバーガー
\\	サッカー
\\	サーモン
\\	アイススケート
\\	私はエイミーです。	
\\	私はエイミーです。 
\\	どこから来ましたか。	
\\	どこから来ましたか。 
\\	私はアメリカから来ました。	
\\	私はアメリカから来ました。 
\\	私はラウルです。	
\\	私はラウルです。 
\\	どこから来ましたか。	
\\	どこから来ましたか。 
\\	私はインドから来ました。	
\\	私はインドから来ました。 
\\	私は智子です。	
\\	私は智子です。 
\\	どこから来ましたか。	
\\	どこから来ましたか。 
\\	私は日本から来ました。	
\\	私は日本から来ました。 
\\	私はエドワードです。	
\\	私はエドワードです。 
\\	どこから来ましたか。	
\\	どこから来ましたか。 
\\	私はイギリスから来ました。	
\\	私はイギリスから来ました。 
\\	私はケリーです。	
\\	私はケリーです。 
\\	どこから来ましたか。	
\\	どこから来ましたか。 
\\	私はカナダから来ました。	
\\	私はカナダから来ました。 
\\	度
\\	寒い
\\	今日
\\	でしょう
\\	おはよう。	
\\	おはよう。 
\\	おはよう。	
\\	おはよう。 
\\	今日は寒いですね。	
\\	今日は寒いですね。 
\\	そうですね。	
\\	そうですね。 
\\	今日は何度ですか。	
\\	今日は何度ですか。 
\\	今日は三度でしょう。	
\\	今日は三度でしょう。 
\\	なんて寒い!	
\\	なんて寒い! 
\\	絶好調
\\	元気一杯
\\	すごく
\\	元気
\\	まあまあ
\\	あまり
\\	駄目
\\	お元気ですか。	
\\	お元気ですか。 
\\	元気ですよ。お元気ですか。	
\\	元気ですよ。お元気ですか。 
\\	元気ですよ。	
\\	元気ですよ。 
\\	元気?	
\\	元気? 
\\	元気じゃない。	
\\	元気じゃない。 
\\	彼女
\\	私の
\\	の
\\	鞄
\\	グッチ
\\	ルイ・ヴィトン
\\	コーチ
\\	さくら、さくら!	
\\	さくら、さくら! 
\\	あー、ピーター!	
\\	あー、ピーター! 
\\	元気ですか。	
\\	元気ですか。 
\\	元気!元気ですか。	
\\	元気!元気ですか。 
\\	元気ですよ。ええ、それは何ですか。	
\\	元気ですよ。ええ、それは何ですか。 
\\	これは私の鞄です。	
\\	これは私の鞄です。 
\\	あなたの鞄ですか。	
\\	あなたの鞄ですか。 
\\	はい、そうです。	
\\	はい、そうです。 
\\	すてきですね。新しいですか。	
\\	すてきですね。新しいですか。 
\\	はい、そうです。	
\\	はい、そうです。 
\\	今
\\	携帯電話
\\	携帯
\\	電話
\\	だけ
\\	それだけ
\\	はい。	
\\	はい。 
\\	もしもし、ピーターです。さくらですか。	
\\	もしもし、ピーターです。さくらですか。 
\\	はい、さくらです。	
\\	はい、さくらです。 
\\	明けましておめでとうございます。	
\\	明けましておめでとうございます。 
\\	明けましておめでとうございます。	
\\	明けましておめでとうございます。 
\\	今年もよろしくお願いします。	
\\	今年もよろしくお願いします。 
\\	よろしくお願いします。	
\\	よろしくお願いします。 
\\	元気ですか。	
\\	元気ですか。 
\\	はい、元気ですよ。元気ですか。	
\\	はい、元気ですよ。元気ですか。 
\\	絶好調です。今どこですか。	
\\	絶好調です。今どこですか。 
\\	東京です。今どこですか。	
\\	東京です。今どこですか。 
\\	ニューヨークです。	
\\	ニューヨークです。 
\\	東京は寒いですよ。ニューヨークは寒いですか。	
\\	東京は寒いですよ。ニューヨークは寒いですか。 
\\	はい。ニューヨークもとても寒いです。じゃ、それだけです。	
\\	はい。ニューヨークもとても寒いです。じゃ、それだけです。 
\\	じゃあね。	
\\	じゃあね。 
\\	じゃあね。	
\\	じゃあね。 
\\	めちゃめちゃ
\\	良い
\\	悪い
\\	まあまあ
\\	調子
\\	お元気ですか。	
\\	お元気ですか。 
\\	元気です。	
\\	元気です。 
\\	お元気ですか。	
\\	お元気ですか。 
\\	お陰様で、元気ですよ。	
\\	お陰様で、元気ですよ。 
\\	調子はどうですか。	
\\	調子はどうですか。 
\\	朝飯前
\\	姉
\\	兄
\\	弟
\\	妹
\\	母
\\	父
\\	人
\\	前
\\	飯
\\	朝
\\	多い
\\	さくら、ご家族は何人ですか。	
\\	さくら、ご家族は何人ですか。 
\\	4人です。父と母と妹と私です。	
\\	4人です。父と母と妹と私です。 
\\	6人です。父と母と兄と姉と弟と私です。	
\\	6人です。父と母と兄と姉と弟と私です。 
\\	多いですね。	
\\	多いですね。 
\\	優しい
\\	楽しい
\\	可愛い
\\	忙しい
\\	美しい
\\	格好いい
\\	面白い
\\	さくら、ご家族は4人ですね。	
\\	さくら、ご家族は4人ですね。 
\\	そうです。	
\\	そうです。 
\\	そうです。	
\\	そうです。 
\\	これ
\\	お酒
\\	私
\\	妹
\\	兄
\\	母
\\	父
\\	写真
\\	家族
\\	サッカー
\\	これは何ですか。	
\\	これは何ですか。 
\\	これ?これは私の家族の写真です。	
\\	これ?これは私の家族の写真です。 
\\	ご家族は何人ですか。	
\\	ご家族は何人ですか。 
\\	5人です。父と母と兄と妹と私です。	
\\	5人です。父と母と兄と妹と私です。 
\\	多いですね。彼は誰ですか。	
\\	多いですね。彼は誰ですか。 
\\	彼は私の兄です。彼はサッカーが好きです。	
\\	彼は私の兄です。彼はサッカーが好きです。 
\\	彼女は誰ですか。	
\\	彼女は誰ですか。 
\\	彼女は私の妹です。彼女は忙しいです。	
\\	彼女は私の妹です。彼女は忙しいです。 
\\	彼は誰ですか。	
\\	彼は誰ですか。 
\\	彼は私の父です。彼はお酒が好きです。	
\\	彼は私の父です。彼はお酒が好きです。 
\\	彼女は誰ですか。	
\\	彼女は誰ですか。 
\\	彼女は私の母です。彼女は楽しいです。	
\\	彼女は私の母です。彼女は楽しいです。 
\\	何
\\	さくら
\\	夏子
\\	も
\\	本当
\\	私
\\	さくら、何歳ですか。	
\\	さくら、何歳ですか。 
\\	私は29歳です。夏子、何歳ですか。	
\\	私は29歳です。夏子、何歳ですか。 
\\	私も29歳です。ピーター、何歳ですか。	
\\	私も29歳です。ピーター、何歳ですか。 
\\	私も29歳です。	
\\	私も29歳です。 
\\	本当ですか。	
\\	本当ですか。 
\\	おはようございます
\\	かなり
\\	隣
\\	スーパー
\\	近い
\\	とても
\\	家
\\	あそこ
\\	そこ
\\	どこ
\\	ここ
\\	鈴木さくら
\\	友だち
\\	遠い
\\	おはようございます。	
\\	おはようございます。 
\\	さくら、私の友達、一徳です。	
\\	さくら、私の友達、一徳です。 
\\	初めまして、鈴木さくらです。よろしくお願いします。	
\\	初めまして、鈴木さくらです。よろしくお願いします。 
\\	こちらこそ、酒井一徳です。初めまして。よろしくお願いします。ここはどこですか。	
\\	こちらこそ、酒井一徳です。初めまして。よろしくお願いします。ここはどこですか。 
\\	ここは私の家ですよ。	
\\	ここは私の家ですよ。 
\\	ピーター、あなたの家はどこですか。	
\\	ピーター、あなたの家はどこですか。 
\\	私の家はそこですよ。とても近いでしょう?	
\\	私の家はそこですよ。とても近いでしょう? 
\\	一徳、あなたの家はどこですか。	
\\	一徳、あなたの家はどこですか。 
\\	私の家はあそこです。スーパーの隣です。かなり遠いです。	
\\	私の家はあそこです。スーパーの隣です。かなり遠いです。 
\\	いらっしゃいませ。
\\	ベルト
\\	ブランド
\\	グッチ
\\	鞄
\\	プラダ
\\	帽子
\\	ドルチェアンドガッバーナ
\\	いらっしゃいませ。	
\\	いらっしゃいませ。 
\\	すみません、このベルトはどのブランドですか。	
\\	すみません、このベルトはどのブランドですか。 
\\	このベルトですか。このベルトはグッチですね。	
\\	このベルトですか。このベルトはグッチですね。 
\\	その鞄はどのブランドですか。	
\\	その鞄はどのブランドですか。 
\\	その鞄はプラダですね。	
\\	その鞄はプラダですね。 
\\	あの帽子はどのブランドですか。	
\\	あの帽子はどのブランドですか。 
\\	あの帽子はドルチェアンドガッバーナです。	
\\	あの帽子はドルチェアンドガッバーナです。 
\\	ああ、どうも。	
\\	ああ、どうも。 
\\	お昼
\\	オッケー
\\	刺身
\\	かなり
\\	あまり
\\	とても
\\	豚カツ
\\	結構
\\	寿司
\\	店
\\	問題
\\	お昼の時間ですね。	
\\	お昼の時間ですね。 
\\	この店はおいしいです。寿司は好きですか。	
\\	この店はおいしいです。寿司は好きですか。 
\\	大好きですよ。	
\\	大好きですよ。 
\\	私はあまり好きじゃないですね。	
\\	私はあまり好きじゃないですね。 
\\	じゃ、その店もけっこうおいしいです。とんかつは好きですか。	
\\	じゃ、その店もけっこうおいしいです。とんかつは好きですか。 
\\	私は大好きです。とても、とても、とても好きです。	
\\	私は大好きです。とても、とても、とても好きです。 
\\	私はあまり・・・	
\\	私はあまり・・・ 
\\	じゃ、あの店もかなりおいしいですよ。刺身は好きですか。	
\\	じゃ、あの店もかなりおいしいですよ。刺身は好きですか。 
\\	問題ないです。	
\\	問題ないです。 
\\	私もオッケーです。	
\\	私もオッケーです。 
\\	刺身でいいですね。	
\\	刺身でいいですね。 
\\	デパ地下
\\	ずつ
\\	美味しい
\\	一番
\\	本
\\	それ
\\	これ
\\	幾ら
\\	すみません
\\	如何ですか
\\	焼き鳥
\\	ございます
\\	ここはどこですか。	
\\	ここはどこですか。 
\\	デパ地下です。	
\\	デパ地下です。 
\\	やきとり、やきとり、いかがですか。	
\\	やきとり、やきとり、いかがですか。 
\\	すみません、これはいくらですか。	
\\	すみません、これはいくらですか。 
\\	それは1本100円です。	
\\	それは1本100円です。 
\\	これはいくらですか。	
\\	これはいくらですか。 
\\	それは1本200円です。	
\\	それは1本200円です。 
\\	それはいくらですか。	
\\	それはいくらですか。 
\\	これは1本300円です。一番おいしいですよ!	
\\	これは1本300円です。一番おいしいですよ! 
\\	じゃ、1本ずつお願いします。	
\\	じゃ、1本ずつお願いします。 
\\	600円でございます。	
\\	600円でございます。 
\\	ありがとう。	
\\	ありがとう。 
\\	ありがとうございます。	
\\	ありがとうございます。 
\\	結構
\\	色々
\\	ございます
\\	胃
\\	薬
\\	安い
\\	強い
\\	風邪薬
\\	なかなか
\\	有る
\\	いらっしゃいませ!	
\\	いらっしゃいませ! 
\\	すみません、風邪薬はありますか?	
\\	すみません、風邪薬はありますか? 
\\	はい、ありますよ。いろいろありますよ。この風邪薬はなかなか強いです。その風邪薬は安いです。あれも結構いいです。	
\\	はい、ありますよ。いろいろありますよ。この風邪薬はなかなか強いです。その風邪薬は安いです。あれも結構いいです。 
\\	この風邪薬は一番強いですか?	
\\	この風邪薬は一番強いですか? 
\\	はい、そうです。	
\\	はい、そうです。 
\\	じゃ、これお願いします。	
\\	じゃ、これお願いします。 
\\	550円でございます。	
\\	550円でございます。 
\\	今
\\	いる(います)
\\	携帯
\\	の方
\\	ちょっと
\\	またあとで。
\\	新宿
\\	に
\\	もしもし、川本夏子です。鈴木さくらさんお願いします。	
\\	もしもし、川本夏子です。鈴木さくらさんお願いします。 
\\	今、いません。	
\\	今、いません。 
\\	あっ、そうですか。どうもありがとうございます。	
\\	あっ、そうですか。どうもありがとうございます。 
\\	じゃあ、携帯の方ですね。	
\\	じゃあ、携帯の方ですね。 
\\	もしもし、さくらですか?	
\\	もしもし、さくらですか? 
\\	はい、さくらです。	
\\	はい、さくらです。 
\\	今どこにいますか?	
\\	今どこにいますか? 
\\	今新宿にいます!	
\\	今新宿にいます! 
\\	今忙しいですか?	
\\	今忙しいですか? 
\\	はい、ちょっと忙しいです。	
\\	はい、ちょっと忙しいです。 
\\	じゃ、またあとで!	
\\	じゃ、またあとで! 
\\	お願いします。	
\\	お願いします。 
\\	バス
\\	遅い
\\	何時
\\	分
\\	バスは遅いですね。	
\\	バスは遅いですね。 
\\	そうですね。	
\\	そうですね。 
\\	すみません。今何時ですか?	
\\	すみません。今何時ですか? 
\\	今は3時です。	
\\	今は3時です。 
\\	ありがとうございました。	
\\	ありがとうございました。 
\\	どういたしまして。	
\\	どういたしまして。 
\\	昼ご飯
\\	食べる
\\	ちょっと
\\	気持ち
\\	悪い
\\	大丈夫
\\	何で
\\	今12時ですよ。お昼の時間です!	
\\	今12時ですよ。お昼の時間です! 
\\	ピーターさんはお昼ご飯を食べますか?	
\\	ピーターさんはお昼ご飯を食べますか? 
\\	いいえ、今日は食べません。	
\\	いいえ、今日は食べません。 
\\	なんでですか?	
\\	なんでですか? 
\\	今日は、ちょっと気持ち悪いです。	
\\	今日は、ちょっと気持ち悪いです。 
\\	大丈夫ですか?	
\\	大丈夫ですか? 
\\	大丈夫です。ありがとうございます。	
\\	大丈夫です。ありがとうございます。 
\\	半
\\	パン
\\	拉麺
\\	今何時ですか?	
\\	今何時ですか? 
\\	12時半です。	
\\	12時半です。 
\\	お昼を何時に食べますか?	
\\	お昼を何時に食べますか? 
\\	1時はどうですか?	
\\	1時はどうですか? 
\\	いいですよ。	
\\	いいですよ。 
\\	何を食べますか?	
\\	何を食べますか? 
\\	私は日本料理が好きです。日本料理を食べませんか?	
\\	私は日本料理が好きです。日本料理を食べませんか? 
\\	そうですね。私はイタリア料理が好きです。イタリア料理を食べませんか?	
\\	そうですね。私はイタリア料理が好きです。イタリア料理を食べませんか? 
\\	日本料理はものすごく美味しいですよ。	
\\	日本料理はものすごく美味しいですよ。 
\\	イタリア料理もものすごく美味しいですよ。	
\\	イタリア料理もものすごく美味しいですよ。 
\\	じゃあ、インド料理は?	
\\	じゃあ、インド料理は? 
\\	いいですね。	
\\	いいですね。 
\\	天気
\\	ずっと
\\	土曜日
\\	言い訳
\\	もういい
\\	出る
\\	家を出ます
\\	今日何時に家を出ますか?	
\\	今日何時に家を出ますか? 
\\	今日は家を出ません。ずっと家にいます。	
\\	今日は家を出ません。ずっと家にいます。 
\\	出ません!?なんでですか?	
\\	出ません!?なんでですか? 
\\	今日は寒いですよ。	
\\	今日は寒いですよ。 
\\	さむい!今日は寒くないですよ!今日は土曜日ですよ!天気もいいし!	
\\	さむい!今日は寒くないですよ!今日は土曜日ですよ!天気もいいし! 
\\	いや、今日は寒いです!	
\\	いや、今日は寒いです! 
\\	それは言い訳です!今日は寒くないです!	
\\	それは言い訳です!今日は寒くないです! 
\\	今日は本当に寒いです!	
\\	今日は本当に寒いです! 
\\	寒くないです!	
\\	寒くないです! 
\\	寒いです!	
\\	寒いです! 
\\	もういい!私は出ます。	
\\	もういい!私は出ます。 
\\	一緒に
\\	映画
\\	作品
\\	開演
\\	池袋
\\	もちろん
\\	今日、暇ですか?	
\\	今日、暇ですか? 
\\	えええ、どうしてですか?	
\\	えええ、どうしてですか? 
\\	今日一緒に映画を見ませんか?	
\\	今日一緒に映画を見ませんか? 
\\	今日ですか?	
\\	今日ですか? 
\\	はい、今夜ですけど。	
\\	はい、今夜ですけど。 
\\	そうですね….	
\\	そうですね…. 
\\	この映画はおもしろくて、たのしくて、とてもよい作品ですよ!	
\\	この映画はおもしろくて、たのしくて、とてもよい作品ですよ! 
\\	本当ですか?	
\\	本当ですか? 
\\	もちろんです!	
\\	もちろんです! 
\\	映画は何時ですか?	
\\	映画は何時ですか? 
\\	8時40分開演です。	
\\	8時40分開演です。 
\\	どこで見ますか?	
\\	どこで見ますか? 
\\	池袋で見ます。	
\\	池袋で見ます。 
\\	じゃあ、いいですよ。	
\\	じゃあ、いいですよ。 
\\	ハッピーエンド
\\	無理
\\	マジですか
\\	新作
\\	ジュリアロバーツ
\\	駄目
\\	コメディ
\\	アクション
\\	つまらない
\\	ロマンチック
\\	くだらない
\\	別々
\\	今夜一緒に映画を見ませんか?	
\\	今夜一緒に映画を見ませんか? 
\\	いいですよ。どの映画がいいですか?	
\\	いいですよ。どの映画がいいですか? 
\\	じゃ、アクション映画がいいです。	
\\	じゃ、アクション映画がいいです。 
\\	アクション映画はつまらないです。	
\\	アクション映画はつまらないです。 
\\	つまらない?つまらなくないです。あなたはどの映画が好きですか?	
\\	つまらない?つまらなくないです。あなたはどの映画が好きですか? 
\\	私はロマンチックな映画がいいです。ジュリアロバーツの新作はどうですか?	
\\	私はロマンチックな映画がいいです。ジュリアロバーツの新作はどうですか? 
\\	マジですか?ロマンチックな映画はくだらないです!	
\\	マジですか?ロマンチックな映画はくだらないです! 
\\	くだらなくないです!ハッピーエンドが好きです。	
\\	くだらなくないです!ハッピーエンドが好きです。 
\\	じゃ、コメディは?	
\\	じゃ、コメディは? 
\\	だめです。ラブストーリーがいいです。	
\\	だめです。ラブストーリーがいいです。 
\\	私には無理です。	
\\	私には無理です。 
\\	じゃ、別々に見ますか?	
\\	じゃ、別々に見ますか? 
\\	いいですよ。	
\\	いいですよ。 
\\	おかえりなさい
\\	犬
\\	やっぱり
\\	最高
\\	秋
\\	春
\\	涼しい
\\	暖かい
\\	うるさい
\\	静か
\\	ただいま
\\	猫
\\	静かな夜ですね。	
\\	静かな夜ですね。 
\\	そうですね。今夜は静かで暖かいです。春の夜は大好きです!	
\\	そうですね。今夜は静かで暖かいです。春の夜は大好きです! 
\\	素敵な夜ですね!	
\\	素敵な夜ですね! 
\\	そうですね!今夜は最高です!	
\\	そうですね!今夜は最高です! 
\\	やっぱり!また!あのうるさい犬!犬は好きですけど、あの犬は大嫌いです!	
\\	やっぱり!また!あのうるさい犬!犬は好きですけど、あの犬は大嫌いです! 
\\	赤い
\\	靴
\\	青い
\\	ドレス
\\	綺麗
\\	白い
\\	ジャケット
\\	センス
\\	この赤い靴は素敵ですね!	
\\	この赤い靴は素敵ですね! 
\\	そうですね。	
\\	そうですね。 
\\	ピーター、この赤い靴は素敵ですか?	
\\	ピーター、この赤い靴は素敵ですか? 
\\	はい、素敵です!	
\\	はい、素敵です! 
\\	やっぱり、素敵ではありません。	
\\	やっぱり、素敵ではありません。 
\\	このドレスはきれいですね!	
\\	このドレスはきれいですね! 
\\	そうですね!	
\\	そうですね! 
\\	ピーター、この青いドレスはきれいですか?	
\\	ピーター、この青いドレスはきれいですか? 
\\	はい、きれいです。	
\\	はい、きれいです。 
\\	やっぱり、きれいではありません。	
\\	やっぱり、きれいではありません。 
\\	この白いジャケットはどうですか?	
\\	この白いジャケットはどうですか? 
\\	素敵ではありません!	
\\	素敵ではありません! 
\\	えぇぇぇ!	
\\	えぇぇぇ! 
\\	これはとてもきれいです!	
\\	これはとてもきれいです! 
\\	やっぱりピーターはセンスがありません!	
\\	やっぱりピーターはセンスがありません! 
\\	有名
\\	一番好き
\\	素晴らしい
\\	絵
\\	いっぱい
\\	初めて
\\	作品
\\	美術館
\\	クロード・モネ
\\	展覧会
\\	お入りください.
\\	日本の美術館は初めてです。	
\\	日本の美術館は初めてです。 
\\	私も!日本人ですけど、今日が初めてです。	
\\	私も!日本人ですけど、今日が初めてです。 
\\	今日は、クロード・モネの展覧会です!有名な作品がいっぱいあります。	
\\	今日は、クロード・モネの展覧会です!有名な作品がいっぱいあります。 
\\	どうぞ、お入りください。	
\\	どうぞ、お入りください。 
\\	ここです。これ、これ。この絵は素晴らしいです!黄色と白と緑のコントラストが素晴らしいです。	
\\	ここです。これ、これ。この絵は素晴らしいです!黄色と白と緑のコントラストが素晴らしいです。 
\\	そうですね。きれいで素敵で、とても好きです。	
\\	そうですね。きれいで素敵で、とても好きです。 
\\	これは私の一番好きな作品です!	
\\	これは私の一番好きな作品です! 
\\	まあ、悪くないです。	
\\	まあ、悪くないです。 
\\	やっぱり、ピーターはセンスがありません。	
\\	やっぱり、ピーターはセンスがありません。 
\\	終電
\\	電車
\\	次(の)
\\	問題
\\	隣
\\	カプセルホテル
\\	すみません。今何時ですか?	
\\	すみません。今何時ですか? 
\\	今、12 時です。	
\\	今、12 時です。 
\\	次の電車は何時に来ますか?	
\\	次の電車は何時に来ますか? 
\\	えぇぇ、次の電車は来ません。	
\\	えぇぇ、次の電車は来ません。 
\\	えぇぇぇ!	
\\	えぇぇぇ! 
\\	終電は11 時59 分ですよ。	
\\	終電は11 時59 分ですよ。 
\\	本当ですか!?	
\\	本当ですか!? 
\\	大丈夫です。問題ありません。隣にカプセルホテルがあります。	
\\	大丈夫です。問題ありません。隣にカプセルホテルがあります。 
\\	えぇぇぇ!	
\\	えぇぇぇ! 
\\	音
\\	テニス
\\	テレビ
\\	サーブ
\\	はい、夏子です。	
\\	はい、夏子です。 
\\	もしもし、カゼですけど。	
\\	もしもし、カゼですけど。 
\\	あっ、どうも。	
\\	あっ、どうも。 
\\	今日何をしますか?	
\\	今日何をしますか? 
\\	今日ですか?今日はテレビを見ます。	
\\	今日ですか?今日はテレビを見ます。 
\\	そうですか。今日一緒にテニスをしませんか?	
\\	そうですか。今日一緒にテニスをしませんか? 
\\	テニスですか?私はテニスが大嫌いです!	
\\	テニスですか?私はテニスが大嫌いです! 
\\	そうですか??	
\\	そうですか?? 
\\	(テニスボールの音)	
\\	あれ、今の音は何ですか?	
\\	あれ、今の音は何ですか? 
\\	テレビです!	
\\	テレビです! 
\\	夏子!あなたのサーブですよ!	
\\	夏子!あなたのサーブですよ! 
\\	それもテレビですか?	
\\	それもテレビですか? 
\\	そうですね。	
\\	そうですね。 
\\	原宿
\\	にぎやか
\\	麻雀
\\	お金
\\	必要
\\	今度
\\	夏子ですけど。	
\\	夏子ですけど。 
\\	今どこにいますか?	
\\	今どこにいますか? 
\\	今、原宿にいます!カゼは今どこにいますか?	
\\	今、原宿にいます!カゼは今どこにいますか? 
\\	今、家にいます。今夜は何をしますか?	
\\	今、家にいます。今夜は何をしますか? 
\\	今夜は原宿にいます!楽しくて、にぎやかで、原宿は大好きです!カゼは何をしますか?	
\\	今夜は原宿にいます!楽しくて、にぎやかで、原宿は大好きです!カゼは何をしますか? 
\\	今夜、家で麻雀をします!あなたのお金が必要です。すみません。あなたが必要です。	
\\	今夜、家で麻雀をします!あなたのお金が必要です。すみません。あなたが必要です。 
\\	麻雀!!麻雀はつまらなくて、大嫌いです!今夜ずっと原宿にいます。	
\\	麻雀!!麻雀はつまらなくて、大嫌いです!今夜ずっと原宿にいます。 
\\	ここに来ませんか?とても楽しいです!	
\\	ここに来ませんか?とても楽しいです! 
\\	原宿!!人が多くて、うるさくて大嫌いです。また今度!	
\\	原宿!!人が多くて、うるさくて大嫌いです。また今度! 
\\	そうですね。また今度。	
\\	そうですね。また今度。 
\\	勉強する
\\	英語
\\	日本語
\\	得意
\\	苦手
\\	試験
\\	心配
\\	今夜
\\	一緒に
\\	明後日
\\	ピーターは英語が得意ですね。明日英語の試験があります。	
\\	ピーターは英語が得意ですね。明日英語の試験があります。 
\\	とても心配です!今夜一緒に英語を勉強しませんか?	
\\	とても心配です!今夜一緒に英語を勉強しませんか? 
\\	そうですね。明後日、日本語の試験があります。	
\\	そうですね。明後日、日本語の試験があります。 
\\	私もとても心配です。今夜と明日はちょっと忙しいです。	
\\	私もとても心配です。今夜と明日はちょっと忙しいです。 
\\	じゃ、今夜一緒に英語を勉強します。	
\\	じゃ、今夜一緒に英語を勉強します。 
\\	明日は一緒に日本語を勉強します。どうですか?	
\\	明日は一緒に日本語を勉強します。どうですか? 
\\	いいですね。そうします!今夜何時がいいですか?	
\\	いいですね。そうします!今夜何時がいいですか? 
\\	そうですねぇ、今夜七時はどうですか?	
\\	そうですねぇ、今夜七時はどうですか? 
\\	いいです。	
\\	いいです。 
\\	月曜日
\\	火曜日
\\	水曜日
\\	木曜日
\\	金曜日
\\	土曜日
\\	日曜日
\\	明々後日
\\	時ですよ!数学の授業は9 時5 分からですよ。	
\\	時ですよ!数学の授業は9 時5 分からですよ。 
\\	今日は何曜日ですか?	
\\	今日は何曜日ですか? 
\\	今日は火曜日ですよ。	
\\	今日は火曜日ですよ。 
\\	本当ですか?火曜日は嫌いです。学校は大嫌いです!	
\\	本当ですか?火曜日は嫌いです。学校は大嫌いです! 
\\	私は火曜日が大好きです!数学と経済と歴史と化学!	
\\	私は火曜日が大好きです!数学と経済と歴史と化学! 
\\	えぇぇ、やっぱり金曜日がいいです。	
\\	えぇぇ、やっぱり金曜日がいいです。 
\\	でも、授業は嫌いでしょう?金曜日にはまだ授業がありますよ。土曜日がいいでしょう?	
\\	でも、授業は嫌いでしょう?金曜日にはまだ授業がありますよ。土曜日がいいでしょう? 
\\	いいえ、金曜日にはあなたがいませんから、金曜日が大好きです!	
\\	いいえ、金曜日にはあなたがいませんから、金曜日が大好きです! 
\\	来月
\\	来週
\\	歌舞伎
\\	用事
\\	来年
\\	台湾料理
\\	今日は何をしますか?	
\\	今日は何をしますか? 
\\	そうですね。ちょっと用事があります。	
\\	そうですね。ちょっと用事があります。 
\\	明日は暇ですか?	
\\	明日は暇ですか? 
\\	そうですね。明日もちょっと用事があります。	
\\	そうですね。明日もちょっと用事があります。 
\\	水曜日は一緒に歌舞伎を見ませんか?	
\\	水曜日は一緒に歌舞伎を見ませんか? 
\\	そうですね。水曜日も用事があります。	
\\	そうですね。水曜日も用事があります。 
\\	木曜日は一緒に台湾料理を食べませんか?	
\\	木曜日は一緒に台湾料理を食べませんか? 
\\	木曜日はちょっと忙しいですね。ちょっと用事があります。	
\\	木曜日はちょっと忙しいですね。ちょっと用事があります。 
\\	金曜日は時間がありますか?	
\\	金曜日は時間がありますか? 
\\	また用事です。	
\\	また用事です。 
\\	土曜日は?	
\\	土曜日は? 
\\	ちょっとですね。	
\\	ちょっとですね。 
\\	日曜日は?	
\\	日曜日は? 
\\	用事です。	
\\	用事です。 
\\	来週、来月、来年は?	
\\	来週、来月、来年は? 
\\	(携帯の着信音)	
\\	はい。夏子ですけど。今日ですか?今日は大丈夫です。	
\\	はい。夏子ですけど。今日ですか?今日は大丈夫です。 
\\	えー!用事は!?	
\\	えー!用事は!? 
\\	これが私の用事です。	
\\	これが私の用事です。 
\\	聞く
\\	教える
\\	実は
\\	試合
\\	違う
\\	分かる
\\	用例
\\	どうですか?	
\\	どうですか? 
\\	とても楽しいです!野球は大好きです!	
\\	とても楽しいです!野球は大好きです! 
\\	えっ、ちょっと違います。これはサッカーですけど。	
\\	えっ、ちょっと違います。これはサッカーですけど。 
\\	あっ、そうですか。すみません。スポーツの試合は初めてです。	
\\	あっ、そうですか。すみません。スポーツの試合は初めてです。 
\\	初めてですか?	
\\	初めてですか? 
\\	そうです。実は、私はスポーツを見ませんし、しません。	
\\	そうです。実は、私はスポーツを見ませんし、しません。 
\\	うれしいです。スポーツは大丈夫です。私があなたに教えます!	
\\	うれしいです。スポーツは大丈夫です。私があなたに教えます! 
\\	反対
\\	正解
\\	だけ
\\	ヒント
\\	離婚
\\	結婚
\\	記念日
\\	おめでとう。
\\	降参
\\	今日は何の日ですか。	
\\	今日は何の日ですか。 
\\	今日は水曜日ですよ!正解でしょう?	
\\	今日は水曜日ですよ!正解でしょう? 
\\	それだけですか。	
\\	それだけですか。 
\\	日、水曜日です。正解ですか。	
\\	日、水曜日です。正解ですか。 
\\	それだけですか。	
\\	それだけですか。 
\\	月15 日、水曜日でしょう?	
\\	月15 日、水曜日でしょう? 
\\	もう一度聞きます!今日は何の日ですか。	
\\	もう一度聞きます!今日は何の日ですか。 
\\	わかりません。降参です。	
\\	わかりません。降参です。 
\\	じゃ、ヒントをあげます。離婚の反対は何ですか。	
\\	じゃ、ヒントをあげます。離婚の反対は何ですか。 
\\	あー!結婚記念日!おめでとう!	
\\	あー!結婚記念日!おめでとう! 
\\	到着
\\	太る
\\	遊ぶ
\\	かき氷
\\	乗る
\\	紫
\\	乗り物
\\	ピンク
\\	遊園地
\\	綿菓子
\\	出発
\\	観覧車
\\	ジェットコースター
\\	到着!私は遊園地が大好きです!	
\\	到着!私は遊園地が大好きです! 
\\	私も遊園地が大好きです!	
\\	私も遊園地が大好きです! 
\\	最初何に乗りますか。	
\\	最初何に乗りますか。 
\\	最初は食べます。	
\\	最初は食べます。 
\\	えぇぇぇ、食べますか。乗り物に乗りません?	
\\	えぇぇぇ、食べますか。乗り物に乗りません? 
\\	ん。	
\\	り物に乗ります。	
\\	かき氷。今日は最高です!	
\\	乗ります。6時にここで会いますか?	
\\	です。	
\\	です。 
\\	やっぱり遊園地は最高です!	
\\	やっぱり遊園地は最高です! 
\\	応援
\\	叫ぶ
\\	休み
\\	顔
\\	話題
\\	失礼
\\	全く
\\	頑張れ
\\	今夜見ますか。	
\\	今夜見ますか。 
\\	ずっと応援します!高瀬さんは?	
\\	と叫びます!明日は仕事に行きません。	
\\	きですか。	
\\	もちろん、イチローです!彼の顔が好きです。	
\\	もちろん、イチローです!彼の顔が好きです。 
\\	すか。	
\\	えぇぇ、何の話題ですか。	
\\	えぇぇ、何の話題ですか。 
\\	野球!野球!野球!野球!	
\\	野球!野球!野球!野球! 
\\	えー、野球?三月に?	
\\	えー、野球?三月に? 
\\	えぇぇ、失礼ですね!全く!頑張れ、日本!	
\\	えぇぇ、失礼ですね!全く!頑張れ、日本! 
\\	出身
\\	ところで
\\	どちら
\\	紹介
\\	こちらこそ
\\	こちら
\\	紹介する
\\	よろしく
\\	そちら
\\	あちら
\\	海外
\\	川本さん、お久しぶりです。	
\\	川本さん、お久しぶりです。 
\\	お久しぶりですね。	
\\	お久しぶりですね。 
\\	この方はお知り合いですか。	
\\	この方はお知り合いですか。 
\\	です。	
\\	ーと申します。どうぞよろしくお願いします。	
\\	ちらでしょうか。	
\\	とても寒いです。	
\\	日本の冬も寒いでしょう?	
\\	日本の冬も寒いでしょう? 
\\	はい。とても寒いです。	
\\	はい。とても寒いです。 
\\	ところで、今日は何をしますか。	
\\	ところで、今日は何をしますか。 
\\	士ですけどよろしいですか?	
\\	富士…	
\\	富士… 
\\	いいですね!	
\\	いいですね! 
\\	了解
\\	麻婆豆腐
\\	あかん
\\	弱肉強食
\\	ニート
\\	職業
\\	お好み焼き
\\	広島風
\\	山登り
\\	宝くじ
\\	趣味
\\	女優
\\	俳優
\\	既婚者
\\	独身者
\\	一期一会
\\	デーティングゲームへようこそ!	
\\	デーティングゲームへようこそ! 
\\	ありがとうございます。よろしくお願いします!	
\\	ありがとうございます。よろしくお願いします! 
\\	夏子さんはどんなタイプが好きですか。	
\\	夏子さんはどんなタイプが好きですか。 
\\	そうですねぇ、頭がよくて、かっこよくて、楽しい人がいいです。	
\\	そうですねぇ、頭がよくて、かっこよくて、楽しい人がいいです。 
\\	了解!では独身者を三人紹介します。	
\\	了解!では独身者を三人紹介します。 
\\	寂しい
\\	中止
\\	聞こえる
\\	稲妻
\\	雷
\\	雨
\\	~人で
\\	今日は雨です。雷が聞こえます。今日の試合は中止ですから、どうしましょうか。	
\\	今日は雨です。雷が聞こえます。今日の試合は中止ですから、どうしましょうか。 
\\	そうですね。高瀬、どうしますか。	
\\	そうですね。高瀬、どうしますか。 
\\	雨の日は映画を見ます。今日は映画を見ましょう!	
\\	雨の日は映画を見ます。今日は映画を見ましょう! 
\\	いいですね。	
\\	いいですね。 
\\	高瀬さん、かわいい友達はいませんか。	
\\	ますよ。一人呼びましょうか。	
\\	じゃ、四人で映画を見ましょう!	
\\	じゃ、四人で映画を見ましょう! 
\\	いいですよ。今、友達に電話します。	
\\	いいですよ。今、友達に電話します。 
\\	値札
\\	スーツ
\\	万
\\	以上わかりません
\\	くらい(ぐらい)
\\	以上
\\	アルマーニ
\\	千
\\	でしょうか。	
\\	そうですね。999円ぐらい。	
\\	そうですね。999円ぐらい。 
\\	はぁ!?999円?大丈夫ですか。	
\\	はぁ!?999円?大丈夫ですか。 
\\	分かりません。	
\\	そうですか。なるほど。	
\\	そうですか。なるほど。 
\\	万6 千円でしょう。	
\\	万6 千円でしょう。 
\\	すごいです。正解です。	
\\	すごいです。正解です。 
\\	もちろん!値札がここにあります。	
\\	もちろん!値札がここにあります。 
\\	「です」	でございます
\\	ジョーク
\\	冗談
\\	キャッチ
\\	国立劇場
\\	人気
\\	過ごす
\\	良い一日
\\	シェークスピア
\\	から)	ので
\\	はい。夏子です。	
\\	はい。夏子です。 
\\	もしもし。シェークスピア高橋でございます。	
\\	もしもし。シェークスピア高橋でございます。 
\\	はぁ!?	
\\	はぁ!? 
\\	今日はいい天気で風も気持ち良いので、一緒に良い一日を過ごしましょう!	
\\	今日はいい天気で風も気持ち良いので、一緒に良い一日を過ごしましょう! 
\\	もしもし、もしもし!もしもし?あれ~!?	
\\	もしもし、もしもし!もしもし?あれ~!? 
\\	はい。夏子です。	
\\	はい。夏子です。 
\\	もしもし!またシェークスピア高橋でございます。	
\\	もしもし!またシェークスピア高橋でございます。 
\\	誰?	
\\	誰? 
\\	デーティングゲームで一番人気、独身者の高橋です!よろしくお願いします。	
\\	デーティングゲームで一番人気、独身者の高橋です!よろしくお願いします。 
\\	はいはい。	
\\	はいはい。 
\\	では、今夜八時に国立劇場で会いましょう。今夜私達は一緒に歌舞伎を見ます!	
\\	では、今夜八時に国立劇場で会いましょう。今夜私達は一緒に歌舞伎を見ます! 
\\	はぁ!?歌舞伎は好きじゃないです。	
\\	はぁ!?歌舞伎は好きじゃないです。 
\\	冗談でしょう!?歌舞伎はみんな、大好きです。では、八時に。	
\\	冗談でしょう!?歌舞伎はみんな、大好きです。では、八時に。 
\\	あっ、ちょっとキャッチが。	
\\	あっ、ちょっとキャッチが。 
\\	どうぞ!	
\\	どうぞ! 
\\	はい。	
\\	はい。 
\\	もしもし。デーティングゲームで二番人気の独身者の山崎雄太です。今夜暇ですか?	
\\	もしもし。デーティングゲームで二番人気の独身者の山崎雄太です。今夜暇ですか? 
\\	飛行機
\\	便
\\	成田空港
\\	着陸
\\	上手
\\	泊まる
\\	急ですが
\\	ハチ公
\\	渋谷
\\	後10 分で日本ですね。	
\\	後10 分で日本ですね。 
\\	そうですね。	
\\	そうですね。 
\\	こんにちは。私はピーターです。初めまして。よろしくお願いします。	
\\	こんにちは。私はピーターです。初めまして。よろしくお願いします。 
\\	日本語がお上手ですね。	
\\	日本語がお上手ですね。 
\\	いいえ、そんなことはありません。	
\\	いいえ、そんなことはありません。 
\\	私は高瀬です。よろしくお願いします。日本は初めてですか。	
\\	私は高瀬です。よろしくお願いします。日本は初めてですか。 
\\	そうですね。ひとりでちょっと恐いです。渋谷のホテルに泊まります。	
\\	そうですね。ひとりでちょっと恐いです。渋谷のホテルに泊まります。 
\\	渋谷ですか。	
\\	渋谷ですか。 
\\	そうです。	
\\	そうです。 
\\	私の家は渋谷にあります!	
\\	私の家は渋谷にあります! 
\\	そうですか。急ですが、明日暇ですか。	
\\	そうですか。急ですが、明日暇ですか。 
\\	暇です。	
\\	暇です。 
\\	明日、会いませんか。	
\\	明日、会いませんか。 
\\	いいですよ。	
\\	いいですよ。 
\\	えぇぇぇ、渋谷ですか。私の家も渋谷にあります!初めまして。よしです!	
\\	えぇぇぇ、渋谷ですか。私の家も渋谷にあります!初めまして。よしです! 
\\	よろしく!明日3人で遊びましょう!何時がいいですか。2時はどうですか。	
\\	よろしく!明日3人で遊びましょう!何時がいいですか。2時はどうですか。 
\\	えぇぇ・・・	
\\	えぇぇ・・・ 
\\	んんん…	
\\	んんん… 
\\	よっしゃ!2時にハチ公前で会いましょう!	
\\	よっしゃ!2時にハチ公前で会いましょう! 
\\	は、はい。	
\\	は、はい。 
\\	いいですけど・・・	
\\	いいですけど・・・ 
\\	とりあえず
\\	半
\\	ぴったり
\\	いっぱい
\\	出発
\\	出発進行
\\	ここがハチ公前ですか。	
\\	ここがハチ公前ですか。 
\\	そうです。	
\\	そうです。 
\\	人がいっぱいいますね。	
\\	人がいっぱいいますね。 
\\	でしょう。*ものすごい人がいますよ。ところで、あの人は遅いですね!	
\\	でしょう。*ものすごい人がいますよ。ところで、あの人は遅いですね! 
\\	到着。ぴったりです!どうも、こんにちは。	
\\	到着。ぴったりです!どうも、こんにちは。 
\\	はぁー?今2 時半です!	
\\	はぁー?今2 時半です! 
\\	そうです。ぴったり2 時半です。	
\\	そうです。ぴったり2 時半です。 
\\	約束は2 時ですよ。	
\\	約束は2 時ですよ。 
\\	あ~、では、今日何しますか?	
\\	あ~、では、今日何しますか? 
\\	そうですね。	
\\	そうですね。 
\\	とりあえず109へ行きましょう!	
\\	とりあえず109へ行きましょう! 
\\	109は何ですか?	
\\	109は何ですか? 
\\	109は有名なデパートです。かわいい服がいっぱいあります。そして、かわいい女性がいっぱいいます。行きましょう!	
\\	109は有名なデパートです。かわいい服がいっぱいあります。そして、かわいい女性がいっぱいいます。行きましょう! 
\\	どこにありますか?	
\\	どこにありますか? 
\\	あそこにあります。見えますか?	
\\	あそこにあります。見えますか? 
\\	はい、見えます。	
\\	はい、見えます。 
\\	じゃ、行きましょう。出発進行!	
\\	じゃ、行きましょう。出発進行! 
\\	嘘
\\	約束
\\	余裕
\\	きっと
\\	そろそろ
\\	もうすぐ
\\	紹介
\\	閉める
\\	玉三郎
\\	すみません。今何時ですか?	
\\	すみません。今何時ですか? 
\\	7時58分です。	
\\	7時58分です。 
\\	うそでしょう!	
\\	うそでしょう! 
\\	はぁ?大丈夫ですか。	
\\	はぁ?大丈夫ですか。 
\\	大丈夫です。すみません。8時の約束なんですけど。でも、まだ2分あります。余裕です。夏子はきっと来ます!問題ありません!きっと来ます!	
\\	大丈夫です。すみません。8時の約束なんですけど。でも、まだ2分あります。余裕です。夏子はきっと来ます!問題ありません!きっと来ます! 
\\	お客様、そろそろ今夜の歌舞伎が始まります。	
\\	お客様、そろそろ今夜の歌舞伎が始まります。 
\\	だめです!夏子がもうすぐ来ます!	
\\	だめです!夏子がもうすぐ来ます! 
\\	誰ですか?	
\\	誰ですか? 
\\	夏子です。後で紹介します!	
\\	夏子です。後で紹介します! 
\\	じゃ、ドアを閉めます。	
\\	じゃ、ドアを閉めます。 
\\	オッケー、後で行きます!	
\\	オッケー、後で行きます! 
\\	うそでしょう!?	
\\	うそでしょう!? 
\\	やっぱり、玉三郎はいいね。	
\\	やっぱり、玉三郎はいいね。 
\\	いいですね。	
\\	いいですね。 
\\	かっこいいね。	
\\	かっこいいね。 
\\	交差点
\\	世界
\\	専門家
\\	大学
\\	博士
\\	おたく
\\	どう違いますか
\\	この交差点はすごいですね。	
\\	この交差点はすごいですね。 
\\	そうですね。人が多いです。	
\\	そうですね。人が多いです。 
\\	そうですよ!世界一にぎやかな交差点ですよ!	
\\	そうですよ!世界一にぎやかな交差点ですよ! 
\\	本当ですか。	
\\	本当ですか。 
\\	本当ですよ!私は渋谷の専門家です。	
\\	本当ですよ!私は渋谷の専門家です。 
\\	学生ですか。	
\\	学生ですか。 
\\	いいえ。渋谷博士です。	
\\	いいえ。渋谷博士です。 
\\	出身大学はどこですか。	
\\	出身大学はどこですか。 
\\	違います。大学ではないです。	
\\	違います。大学ではないです。 
\\	やっぱり!渋谷おたくでしょう。	
\\	やっぱり!渋谷おたくでしょう。 
\\	博士とおたくはどう違いますか。	
\\	博士とおたくはどう違いますか。 
\\	やっぱり、ピーター、二人で行きましょう。	
\\	やっぱり、ピーター、二人で行きましょう。 
\\	霧
\\	曇り
\\	晴れ
\\	雪
\\	雨
\\	ロンドン
\\	ロス(ロサンゼルス)
\\	ベルリン
\\	トロント
\\	本気
\\	予報
\\	天気
\\	お天気お姉さん
\\	このテレビはすごいですね!	
\\	このテレビはすごいですね! 
\\	でしょう?600チャンネルもありますよ。	
\\	でしょう?600チャンネルもありますよ。 
\\	すごいですね。私は天気予報が好きです。世界の天気を見ましょう。	
\\	すごいですね。私は天気予報が好きです。世界の天気を見ましょう。 
\\	本気ですか。	
\\	本気ですか。 
\\	本気ですよ。トロントは雨です。ベルリンは雪です。ロスは晴れです。	
\\	本気ですよ。トロントは雨です。ベルリンは雪です。ロスは晴れです。 
\\	東京はくもりです。ロンドンは霧です。楽しいです!	
\\	東京はくもりです。ロンドンは霧です。楽しいです! 
\\	何がそんなに楽しいんですか。	
\\	何がそんなに楽しいんですか。 
\\	もちろんお天気おねえさんです。	
\\	もちろんお天気おねえさんです。 
\\	あ、やっぱり。	
\\	あ、やっぱり。 
\\	到着
\\	ビール
\\	メートル
\\	息が苦しい
\\	苦しい
\\	くもり
\\	息
\\	痛い
\\	景色
\\	遅い
\\	秒
\\	水
\\	頂上
\\	ワイン
\\	でしょう?	
\\	水、水お願いします。	
\\	水、水お願いします。 
\\	いです。夏子遅いですね。	
\\	す。	
\\	っ、私の家が見えます!	
\\	富士山は何メートルですか。	
\\	富士山は何メートルですか。 
\\	ましょう。	
\\	ここで飲みますか。ここは寒くて・・・	
\\	ここで飲みますか。ここは寒くて・・・ 
\\	ますか。	
\\	はい。見えます。	
\\	はい。見えます。 
\\	次、あの山に登ります。行きましょう!	
\\	次、あの山に登ります。行きましょう! 
\\	えー、冗談でしょう?	
\\	えー、冗談でしょう? 
\\	いいえ、冗談は嫌いです。	
\\	いいえ、冗談は嫌いです。 
\\	竹の子
\\	ベジタリアン
\\	野菜
\\	焼肉
\\	~について
\\	食生活
\\	黒酢
\\	レポーターのじゅんです。	
\\	レポーターのじゅんです。 
\\	そうですね、私は毎日焼き肉を食べます。	
\\	そうですね、私は毎日焼き肉を食べます。 
\\	なるほど。私も今日から竹の子を食べます。	
\\	なるほど。私も今日から竹の子を食べます。 
\\	真珠
\\	たまに
\\	煙草
\\	肺
\\	質問
\\	ストレス
\\	解消
\\	北岳
\\	ハイキング
\\	ロッククライミング
\\	素潜り
\\	時々
\\	レポーターのじゅんです。	
\\	レポーターのじゅんです。 
\\	そうですね。	
\\	そうですね。 
\\	ハイキングは体にいいですね。	
\\	ハイキングは体にいいですね。 
\\	違います。ロッククライミングをします。ハイキングではないです。	
\\	違います。ロッククライミングをします。ハイキングではないです。 
\\	そうですか!すごいですね!	
\\	そうですか!すごいですね! 
\\	すもぐりもします。肺にとてもいいです。いつも真珠を採ります。	
\\	すもぐりもします。肺にとてもいいです。いつも真珠を採ります。 
\\	すごいですね。	
\\	すごいですね。 
\\	でも、ときどきタバコを吸います。そして、お酒もたまに飲みます。	
\\	でも、ときどきタバコを吸います。そして、お酒もたまに飲みます。 
\\	たまに、ですか?	
\\	たまに、ですか? 
\\	はい、はい。よく飲みます。	
\\	はい、はい。よく飲みます。 
\\	小林さんは元気いっぱいですね。	
\\	小林さんは元気いっぱいですね。 
\\	ありがとうございます。	
\\	ありがとうございます。 
\\	こ、困ります・・・	
\\	こ、困ります・・・ 
\\	三番目
\\	週末
\\	伊豆
\\	ドライブ
\\	運転
\\	知る
\\	全然
\\	暇な時
\\	謙譲語
\\	(と)申します
\\	競馬
\\	最初
\\	平日
\\	はい。	
\\	はい。 
\\	もしもし、こんにちは、初めまして。	
\\	もしもし、こんにちは、初めまして。 
\\	こちらこそ、初めまして。川本夏子です。よろしくお願いします。	
\\	こちらこそ、初めまして。川本夏子です。よろしくお願いします。 
\\	川本さんは暇な時には何をしますか。	
\\	川本さんは暇な時には何をしますか。 
\\	ええ?私ですか。暇な時ですか。	
\\	ええ?私ですか。暇な時ですか。 
\\	もちろん。川本さんのことを全然知りませんから。	
\\	もちろん。川本さんのことを全然知りませんから。 
\\	ああ、夏子でいいですよ。私の趣味は運転することです。	
\\	ああ、夏子でいいですよ。私の趣味は運転することです。 
\\	へぇぇ、私は伊豆が大好きで、車にも興味があります。	
\\	へぇぇ、私は伊豆が大好きで、車にも興味があります。 
\\	いいですよ! 行きましょう!	
\\	いいですよ! 行きましょう! 
\\	よっしゃ!	
\\	よっしゃ! 
\\	着く
\\	ばれる
\\	門限
\\	かしら
\\	ばらす
\\	遅れる
\\	ちゃんと
\\	破る
\\	守る
\\	あのコンサートは最高でした。	
\\	あのコンサートは最高でした。 
\\	そうですね。あ!後もう少しで家に着きます。	
\\	そうですね。あ!後もう少しで家に着きます。 
\\	何の音かしら。ああ、また遅れましたよ。	
\\	何の音かしら。ああ、また遅れましたよ。 
\\	やや、ばれましたよ、おじいさん。	
\\	やや、ばれましたよ、おじいさん。 
\\	おじいさん、顔が真っ白。	
\\	おじいさん、顔が真っ白。 
\\	やややや、それもばれましたよ、おばあさん。	
\\	やややや、それもばれましたよ、おばあさん。 
\\	私達はちゃんと電話しましたよ。	
\\	私達はちゃんと電話しましたよ。 
\\	本当ですか。	
\\	本当ですか。 
\\	6時に/7時に電話しましたよ。	
\\	6時に/7時に電話しましたよ。 
\\	やっぱり!おじいさん、おばあさん、門限は何時ですか。	
\\	やっぱり!おじいさん、おばあさん、門限は何時ですか。 
\\	はいはい、十時です。	
\\	はいはい、十時です。 
\\	これからは門限を守りましょう。	
\\	これからは門限を守りましょう。 
\\	分かりました。	
\\	分かりました。 
\\	ただいま。
\\	出会い
\\	フェラガモ
\\	バチカン
\\	パンテオン
\\	コロッセオ
\\	宇宙
\\	旅行
\\	お帰り
\\	結局
\\	ただいま!	
\\	ただいま! 
\\	お帰り。	
\\	お帰り。 
\\	最高でした。	
\\	最高でした。 
\\	どこへ行きましたか。	
\\	どこへ行きましたか。 
\\	ローマへ行きました。	
\\	ローマへ行きました。 
\\	コロッセオはどうでしたか。	
\\	コロッセオはどうでしたか。 
\\	コロッセオには行きませんでした。	
\\	コロッセオには行きませんでした。 
\\	バチカンには行きましたか。	
\\	バチカンには行きましたか。 
\\	いいえ、行きませんでした。	
\\	いいえ、行きませんでした。 
\\	パンテオンは見ましたか。	
\\	パンテオンは見ましたか。 
\\	いいえ、見ませんでした。	
\\	いいえ、見ませんでした。 
\\	ええ?じゃあ、ローマには何がありましたか。	
\\	ええ?じゃあ、ローマには何がありましたか。 
\\	グッチでしょう。	
\\	グッチでしょう。 
\\	プラダでしょう。	
\\	プラダでしょう。 
\\	フェラガモでしょう	
\\	フェラガモでしょう 
\\	買い物ですか。 他には?	
\\	買い物ですか。 他には? 
\\	素敵な出会いです。	
\\	素敵な出会いです。 
\\	結局男探しですか。	
\\	結局男探しですか。 
\\	ラジオ
\\	独りぼっち
\\	構う
\\	ごめんなさい
\\	かかる
\\	完了
\\	準備中
\\	準備
\\	交換留学生
\\	交換
\\	充電
\\	電池
\\	ピクニック
\\	はい。	
\\	はい。 
\\	もしもし。夏子ですけど。	
\\	もしもし。夏子ですけど。 
\\	はいはい。今から行きます。食べ物と飲み物を買いました。	
\\	はいはい。今から行きます。食べ物と飲み物を買いました。 
\\	本当にごめんなさい。うっかり忘れました。今日、別の予定がありました。	
\\	本当にごめんなさい。うっかり忘れました。今日、別の予定がありました。 
\\	ああ、そうですか。全然かまいませんよ。	
\\	ああ、そうですか。全然かまいませんよ。 
\\	本当にごめんなさい。今度埋め合わせします。	
\\	本当にごめんなさい。今度埋め合わせします。 
\\	その必要はないですよ。来週末はどうですか。	
\\	その必要はないですよ。来週末はどうですか。 
\\	来週は絶対大丈夫です。よろしくお願いします。	
\\	来週は絶対大丈夫です。よろしくお願いします。 
\\	楽勝
\\	鋭い
\\	鈍い
\\	何とか
\\	殆ど
\\	むずかしい
\\	簡単
\\	みきちゃん、試験はどうでしたか。	
\\	みきちゃん、試験はどうでしたか。 
\\	そうですね。試験は・・・	
\\	そうですね。試験は・・・ 
\\	楽勝でした。本当に簡単でした。	
\\	楽勝でした。本当に簡単でした。 
\\	へえ?あの試験は難しかったです。ものすごく大変でした。	
\\	へえ?あの試験は難しかったです。ものすごく大変でした。 
\\	私もあの試験は難しかったと思います。	
\\	私もあの試験は難しかったと思います。 
\\	何?鈴木君はとんでもない人ですね。	
\\	何?鈴木君はとんでもない人ですね。 
\\	物足りない
\\	〜初の
\\	満々
\\	自信
\\	満月
\\	満点
\\	点数
\\	将来
\\	楽しみ
\\	結果
\\	優しい
\\	易しい
\\	満足
\\	史上初
\\	ああ鈴木君、昨日のテストのことですが。	
\\	ああ鈴木君、昨日のテストのことですが。 
\\	ああ、あの試験は易しかったですね。とにかく物足りなかったです。	
\\	ああ、あの試験は易しかったですね。とにかく物足りなかったです。 
\\	じゃあ、結果が楽しみですね。	
\\	じゃあ、結果が楽しみですね。 
\\	もう結果は分かりますから、楽しみではありません。	
\\	もう結果は分かりますから、楽しみではありません。 
\\	私はもう点数を見ましたよ。	
\\	私はもう点数を見ましたよ。 
\\	満点でしょう?
\\	組のトップでしょう?	
\\	満点でしょう?
\\	組のトップでしょう? 
\\	いや、鈴木君は自信満々だね。しかし・・・	
\\	いや、鈴木君は自信満々だね。しかし・・・ 
\\	違いますか。	
\\	違いますか。 
\\	んんん、実は、学校初のれい点でした。	
\\	んんん、実は、学校初のれい点でした。 
\\	ええ?れい点?	
\\	ええ?れい点? 
\\	お待たせ
\\	良かった
\\	下見
\\	結婚式場
\\	パパ
\\	通う
\\	来年
\\	自動車学校
\\	自動車
\\	運転免許
\\	式場
\\	夏子さん、お待たせしました。よろしくお願いします。	
\\	夏子さん、お待たせしました。よろしくお願いします。 
\\	こちらこそよろしくお願いします	
\\	こちらこそよろしくお願いします 
\\	こちらです。どうぞどうぞ	
\\	こちらです。どうぞどうぞ 
\\	凄い車ですね。格好いいです。で、でも。	
\\	凄い車ですね。格好いいです。で、でも。 
\\	どうかしましたか。	
\\	どうかしましたか。 
\\	ちょっと。この方は?	
\\	ちょっと。この方は? 
\\	ああ、もちろん、私の父です。	
\\	ああ、もちろん、私の父です。 
\\	どうも初めまして、夏子さん。話は良く聞きますよ。	
\\	どうも初めまして、夏子さん。話は良く聞きますよ。 
\\	私はまだ運転免許はありません。来年から自動車学校へ通います。	
\\	私はまだ運転免許はありません。来年から自動車学校へ通います。 
\\	は?	
\\	は? 
\\	うん、健治がいい子で本当に良かったです。	
\\	うん、健治がいい子で本当に良かったです。 
\\	ね、パパ、今日は結婚式場の下見に行こう。	
\\	ね、パパ、今日は結婚式場の下見に行こう。 
\\	それじゃ行きましょう。	
\\	それじゃ行きましょう。 
\\	ええ?そんな・・・?	
\\	ええ?そんな・・・? 
\\	蒸し暑い
\\	安全(な)
\\	人気者
\\	楽(な)
\\	気に入る
\\	気が変わる
\\	忘れる
\\	絶対に
\\	だるい
\\	安全第一
\\	今日何時がいいですか?	
\\	今日何時がいいですか? 
\\	あれは今日ですか?	
\\	あれは今日ですか? 
\\	うん、あれは今日ですよ。	
\\	うん、あれは今日ですよ。 
\\	でも、今日は蒸し暑くて、夏ですし。もうだるいです。	
\\	でも、今日は蒸し暑くて、夏ですし。もうだるいです。 
\\	でも、貴方は約束しましたよ。絶対に今週と言いました。忘れましたか。	
\\	"でも、貴方は約束しましたよ。絶対に今週と言いました。忘れましたか。 
\\	でも、気が変わりました。ここが気に入りました。ここが大好きです。	
\\	でも、気が変わりました。ここが気に入りました。ここが大好きです。 
\\	ここのどこが好きですか。	
\\	ここのどこが好きですか。 
\\	食べ物が美味しくて、仕事は楽で、人気者だし。何よりも安全ですから。	
\\	食べ物が美味しくて、仕事は楽で、人気者だし。何よりも安全ですから。 
\\	しかし。。。	
\\	しかし。。。 
\\	狭い
\\	新鮮
\\	空
\\	風
\\	冷たい
\\	海
\\	広い
\\	青い
\\	懐かしい
\\	故郷
\\	毎週
\\	暑い
\\	魚
\\	ここは狭くて、暑くて、食べ物も美味しくなくて、本当の家ではないです。	
\\	ここは狭くて、暑くて、食べ物も美味しくなくて、本当の家ではないです。 
\\	はい、はい。わかりました。とにかく、来週にしましょう。	
\\	はい、はい。わかりました。とにかく、来週にしましょう。 
\\	毎週同じことを言います。やっぱり帰る気はないんですね。	
\\	毎週同じことを言います。やっぱり帰る気はないんですね。 
\\	そんなことはないです。ただ最近忙しいです。一番人気ですから。	
\\	そんなことはないです。ただ最近忙しいです。一番人気ですから。 
\\	忙しくてとか、だるくてとか、いつも言い訳を言いますね。故郷が懐かしくないんですか。	
\\	"忙しくてとか、だるくてとか、いつも言い訳を言いますね。故郷が懐かしくないんですか。 
\\	本気ですか。え、え。無理ですよ。	
\\	本気ですか。え、え。無理ですよ。 
\\	あああああ	
\\	あああああ  
\\	ペンギン
\\	犯人
\\	捕まえる
\\	動物園
\\	考え
\\	理解出来ない
\\	理解出来る
\\	出来る
\\	理解(する)
\\	何となく
\\	泣き出す
\\	走り出す
\\	飛び越える
\\	乗り越える
\\	今の、見ましたか。	
\\	今の、見ましたか。 
\\	どこですか。	
\\	どこですか。 
\\	あのペンギン、壁を飛び越えました。	
\\	あのペンギン、壁を飛び越えました。 
\\	嘘でしょう。	
\\	嘘でしょう。 
\\	ほら!あそこです。	
\\	ほら!あそこです。 
\\	本当だ!大丈夫でしょうか。	
\\	本当だ!大丈夫でしょうか。 
\\	十分ぐらい前からずっとおかしくて、いきなり走り出しました。	
\\	十分ぐらい前からずっとおかしくて、いきなり走り出しました。 
\\	おかしいですけど、私は何となく理解できます。	
\\	おかしいですけど、私は何となく理解できます。 
\\	皆
\\	答え
\\	本州
\\	北海道
\\	九州
\\	沖縄
\\	四国
\\	大きい
\\	~の中
\\	島
\\	次
\\	多い
\\	人口
\\	生徒
\\	質問
\\	正解
\\	はい皆さん、質問の時間ですよ。	
\\	はい皆さん、質問の時間ですよ。 
\\	やった!いぇーい!	
\\	やった!いぇーい! 
\\	東京と、大阪はどちらの人口が多いですか。	
\\	東京と、大阪はどちらの人口が多いですか。 
\\	はい、川村先生、大阪は東京より人口が多いです。	
\\	はい、川村先生、大阪は東京より人口が多いです。 
\\	いいえ、違います。東京の人口は大阪より多いです。じゃ、次の質問、	
\\	いいえ、違います。東京の人口は大阪より多いです。じゃ、次の質問、 
\\	四国は沖縄より大きいでしょう。	
\\	四国は沖縄より大きいでしょう。 
\\	そうです。そして、九州は四国より大きいです。	
\\	そうです。そして、九州は四国より大きいです。 
\\	そう、そう。北海道は九州より大きいです。	
\\	そう、そう。北海道は九州より大きいです。 
\\	本州は北海道より大きいですから、	
\\	本州は北海道より大きいですから、 
\\	答えは本州です!	
\\	答えは本州です! 
\\	正解!	
\\	正解! 
\\	やった!	
\\	やった! 
\\	いぇーい!	
\\	いぇーい! 
\\	徳永
\\	不~
\\	背が高い
\\	背
\\	センチ
\\	身長
\\	貯金
\\	欲しい
\\	一千万
\\	高い
\\	前
\\	悪い
\\	年収
\\	藤沢
\\	不合格
\\	初めまして。徳永です。	
\\	初めまして。徳永です。 
\\	はい。藤沢です。年収はおいくらですか。	
\\	はい。藤沢です。年収はおいくらですか。 
\\	500万円です。	
\\	500万円です。 
\\	まあ、悪くないですね。あなたの年収は前の人より高いですけど、	
\\	まあ、悪くないですね。あなたの年収は前の人より高いですけど、 
\\	で、でも、貯金があります。	
\\	で、でも、貯金があります。 
\\	あら!そうですか。でも、前の人はあなたよりハンサムです。	
\\	あら!そうですか。でも、前の人はあなたよりハンサムです。 
\\	ハー。	
\\	ハー。 
\\	身長は何センチですか。	
\\	身長は何センチですか。 
\\	178です。	
\\	178です。 
\\	うん、悪くないですね。あなたは前の人より背が高いですけど、	
\\	うん、悪くないですね。あなたは前の人より背が高いですけど、 
\\	地理
\\	終わる
\\	試す
\\	鋭い
\\	建物
\\	軽い
\\	エッフェル塔
\\	タワー
\\	東京都庁
\\	新宿
\\	問題
\\	はい、皆さん、今日は地理の問題です。	
\\	はい、皆さん、今日は地理の問題です。 
\\	新宿にある東京都庁です。	
\\	新宿にある東京都庁です。 
\\	いいえ、違います。東京タワーの方が高いです。	
\\	いいえ、違います。東京タワーの方が高いです。 
\\	えっ、しかし・・・	
\\	えっ、しかし・・・ 
\\	フランスのエッフェル塔より9メートルも高いです。	
\\	フランスのエッフェル塔より9メートルも高いです。 
\\	しかし・・・	
\\	しかし・・・ 
\\	更にエッフェル塔より軽いです。	
\\	更にエッフェル塔より軽いです。 
\\	しかし、東京タワーはビルではありません。	
\\	しかし、東京タワーはビルではありません。 
\\	ええ・・・	
\\	ええ・・・ 
\\	東京タワーは建物ですよ。	
\\	東京タワーは建物ですよ。 
\\	鋭いですね。先生は君たちを試したんだ。ハハハ。これで授業終わります。	
\\	鋭いですね。先生は君たちを試したんだ。ハハハ。これで授業終わります。 
\\	でも、まだ三十分ありますよ。	
\\	でも、まだ三十分ありますよ。 
\\	無理(な)
\\	両方
\\	経験
\\	技術
\\	大事(な)
\\	オムレツ
\\	難しい
\\	絶対
\\	コツ
\\	できますよ!絶対できます!	
\\	できますよ!絶対できます! 
\\	無理ですよ。あなたには無理です。絶対できません!	
\\	無理ですよ。あなたには無理です。絶対できません! 
\\	簡単ですよ。楽勝です。誰でもできます。	
\\	簡単ですよ。楽勝です。誰でもできます。 
\\	あなたは時々おかしいですよ。これは難しいです。	
\\	あなたは時々おかしいですよ。これは難しいです。 
\\	絶対できます!よっしゃ!行きますよ!	
\\	絶対できます!よっしゃ!行きますよ! 
\\	ほら、オムレツは難しいです。	
\\	ほら、オムレツは難しいです。 
\\	可愛い
\\	素敵
\\	ハイハイする
\\	出来る
\\	歩く
\\	歳
\\	赤ちゃん
\\	笑顔
\\	可愛い赤ちゃんですね。	
\\	可愛い赤ちゃんですね。 
\\	ありがとうございます。	
\\	ありがとうございます。 
\\	こんにちは!・・・こんにちは。あれ、	
\\	こんにちは!・・・こんにちは。あれ、 
\\	まだ、1歳ですよ。まだ、話すことはできません!	
\\	まだ、1歳ですよ。まだ、話すことはできません! 
\\	歩くことは出来ますか。	
\\	歩くことは出来ますか。 
\\	いいえ、できません。でも、ハイハイすることができます。	
\\	いいえ、できません。でも、ハイハイすることができます。 
\\	それだけですか。つまらないですね。	
\\	それだけですか。つまらないですね。 
\\	そんなことはありません。ほら。	
\\	そんなことはありません。ほら。 
\\	うわー、かわいい笑顔ですね。	
\\	うわー、かわいい笑顔ですね。 
\\	注文
\\	チョコレート
\\	パイナップル
\\	ピーマン
\\	キノコ
\\	茄子
\\	追加
\\	アンチョビ
\\	トッピング
\\	以上
\\	ピザ
\\	マルガリータ
\\	畏まりました
\\	いかが
\\	すみません。注文お願いします。	
\\	すみません。注文お願いします。 
\\	はい。かしこまりました。	
\\	はい。かしこまりました。 
\\	マルガリータピザ一つお願いします。以上です。	
\\	マルガリータピザ一つお願いします。以上です。 
\\	お飲み物はいかがですか。	
\\	お飲み物はいかがですか。 
\\	お水でいいです。二つお願いします。	
\\	お水でいいです。二つお願いします。 
\\	はい。かしこまりました。	
\\	はい。かしこまりました。 
\\	すみません。トッピングにアンチョビの追加はできますか。	
\\	すみません。トッピングにアンチョビの追加はできますか。 
\\	はい、できますよ。	
\\	はい、できますよ。 
\\	なすも追加できますか。	
\\	なすも追加できますか。 
\\	はい、大丈夫ですよ。	
\\	はい、大丈夫ですよ。 
\\	あと、キノコとピーマンとパイナップルとチョコレートもお願いします。	
\\	あと、キノコとピーマンとパイナップルとチョコレートもお願いします。 
\\	はい、かしこまりました。	
\\	はい、かしこまりました。 
\\	本気ですか。それは本当においしいですか。	
\\	本気ですか。それは本当においしいですか。 
\\	はい、おいしいです。	
\\	はい、おいしいです。 
\\	全部食べることができますか。	
\\	全部食べることができますか。 
\\	もちろん。よしさんは?	
\\	もちろん。よしさんは? 
\\	絶対できません。私には無理です。すみません。	
\\	絶対できません。私には無理です。すみません。 
\\	はい。	
\\	はい。 
\\	マルガリータピザもう一つお願いします。	
\\	マルガリータピザもう一つお願いします。 
\\	あ、よかったです。	
\\	あ、よかったです。 
\\	梅雨
\\	海外
\\	落ちる
\\	最善
\\	梅雨明け
\\	梅雨入り
\\	半ば
\\	旅行会社
\\	決める
\\	値段
\\	涼しい
\\	国内
\\	大嫌い
\\	降る
\\	最悪
\\	旅行
\\	梅雨が始まりましたよ。	
\\	梅雨が始まりましたよ。 
\\	そうですね。最悪です。毎日雨が降ります。超ブルーです。	
\\	そうですね。最悪です。毎日雨が降ります。超ブルーです。 
\\	じゃ、旅行へ行きましょう!	
\\	じゃ、旅行へ行きましょう! 
\\	いいですね。	
\\	いいですね。 
\\	私も行きます!	
\\	私も行きます! 
\\	じゃ、どこがいいですか。	
\\	じゃ、どこがいいですか。 
\\	国内がいいです。私は涼しい北海道へ行きたいです!	
\\	国内がいいです。私は涼しい北海道へ行きたいです! 
\\	北海道、いいですね。	
\\	北海道、いいですね。 
\\	暑くて、気持ちがいい沖縄がいいです。私は沖縄へ行きたいです。	
\\	暑くて、気持ちがいい沖縄がいいです。私は沖縄へ行きたいです。 
\\	沖縄もいいですね。じゃ、値段で決めましょう。	
\\	沖縄もいいですね。じゃ、値段で決めましょう。 
\\	オッケー。	
\\	オッケー。 
\\	旅行会社へ行きましょう。	
\\	旅行会社へ行きましょう。 
\\	最近
\\	決まり
\\	代理
\\	予約
\\	泳ぐ
\\	現在
\\	安い
\\	人気
\\	ハワイ
\\	決まる
\\	いらっしゃいませ。どうぞ。	
\\	いらっしゃいませ。どうぞ。 
\\	私たちは旅行へ行きたいんですが。	
\\	私たちは旅行へ行きたいんですが。 
\\	最近ハワイ旅行は人気がありますが。	
\\	最近ハワイ旅行は人気がありますが。 
\\	私たちは国内旅行がいいです。沖縄と北海道、どちらが安いですか。	
\\	私たちは国内旅行がいいです。沖縄と北海道、どちらが安いですか。 
\\	そうですね。現在、北海道は沖縄より安いですよ。	
\\	そうですね。現在、北海道は沖縄より安いですよ。 
\\	じゃ、決まりですね。	
\\	じゃ、決まりですね。 
\\	北海道。。。やっぱり北海道へ行きたくないです。	
\\	北海道。。。やっぱり北海道へ行きたくないです。 
\\	ハー!	
\\	ハー! 
\\	北海道は冬がいいです。私は泳ぎたいです!	
\\	北海道は冬がいいです。私は泳ぎたいです! 
\\	はい、はい、分かりました。じゃ、沖縄にしましょう。じゃ、沖縄を三人予約お願いします。	
\\	はい、はい、分かりました。じゃ、沖縄にしましょう。じゃ、沖縄を三人予約お願いします。 
\\	はい、ありがとうございます!	
\\	はい、ありがとうございます! 
\\	同窓会
\\	卒業
\\	以来
\\	恥ずかしがり屋
\\	昔
\\	やっと
\\	同期会
\\	クラス会
\\	やっと同窓会ですね。川村さん、お久しぶりです。	
\\	やっと同窓会ですね。川村さん、お久しぶりです。 
\\	ああー、渡辺君、お久しぶりです。	
\\	ああー、渡辺君、お久しぶりです。 
\\	10年ぶりですね。	
\\	10年ぶりですね。 
\\	ああ、そうですね。卒業以来ですね。	
\\	ああ、そうですね。卒業以来ですね。 
\\	実は、あなたに言いたいことがあります。	
\\	実は、あなたに言いたいことがあります。 
\\	え、何ですか。	
\\	え、何ですか。 
\\	ずっと昔から言いたかったけど、私は恥ずかしがり屋だったから	
\\	ずっと昔から言いたかったけど、私は恥ずかしがり屋だったから 
\\	実は、私もあなたに言いたいことがあります。	
\\	実は、私もあなたに言いたいことがあります。 
\\	え、私に?何ですか。	
\\	え、私に?何ですか。 
\\	ロボット
\\	僕
\\	完了
\\	入れる
\\	箱
\\	拾う
\\	まっすぐ
\\	命令
\\	言葉
\\	宿題
\\	あの新しいロボットは凄いですよ。	
\\	あの新しいロボットは凄いですよ。 
\\	どのロボットですか。	
\\	どのロボットですか。 
\\	これです。これです。見てください。	
\\	これです。これです。見てください。 
\\	これですか。どこが凄いんですか。	
\\	これですか。どこが凄いんですか。 
\\	このロボットは言葉で命令することができます。	
\\	このロボットは言葉で命令することができます。 
\\	へー。	
\\	へー。 
\\	行きますよ。まっすぐ行って、あのゴミを拾って、あのゴミ箱に入れてください。	
\\	行きますよ。まっすぐ行って、あのゴミを拾って、あのゴミ箱に入れてください。 
\\	かしこまりました。完了しました。	
\\	かしこまりました。完了しました。 
\\	すごいですね!	
\\	すごいですね! 
\\	でしょう。	
\\	でしょう。 
\\	僕の宿題もできるでしょうか。	
\\	僕の宿題もできるでしょうか。 
\\	社長
\\	痛い
\\	罰金
\\	儲かる
\\	買う
\\	売る
\\	下がる
\\	上がる
\\	枚
\\	みかん
\\	証券
\\	仕方(が)ない
\\	お手洗い
\\	申し訳ありません。
\\	会議中
\\	トイレが近い
\\	まもなく10時になります。	
\\	まもなく10時になります。 
\\	社長、会議中申し訳ありませんが、ちょっとお手洗いに行ってきます。	
\\	社長、会議中申し訳ありませんが、ちょっとお手洗いに行ってきます。 
\\	田中君、また君か。トイレ近いな。仕方ない。	
\\	田中君、また君か。トイレ近いな。仕方ない。 
\\	失礼します。	
\\	失礼します。 
\\	はい、よし証券です。	
\\	はい、よし証券です。 
\\	100枚フロリダ蜜株を買ってください。買って、買って。	
\\	100枚フロリダ蜜株を買ってください。買って、買って。 
\\	あ、田中さん。はい、買いました。あ、10ポイント上がりました。	
\\	あ、田中さん。はい、買いました。あ、10ポイント上がりました。 
\\	じゃ、売って、売って、売ってください。	
\\	じゃ、売って、売って、売ってください。 
\\	売りました。あ、20ポイント下がりました。	
\\	売りました。あ、20ポイント下がりました。 
\\	1000枚買って、買って。早く買ってください。	
\\	1000枚買って、買って。早く買ってください。 
\\	買いました。あ!50ポイント上がりました。	
\\	買いました。あ!50ポイント上がりました。 
\\	じゃ、売って、売って、売ってください!	
\\	じゃ、売って、売って、売ってください! 
\\	さすが田中さん。今日も100万円儲かりました。	
\\	さすが田中さん。今日も100万円儲かりました。 
\\	また電話します。よろしくお願いします。	
\\	また電話します。よろしくお願いします。 
\\	はい、失礼します。	
\\	はい、失礼します。 
\\	失礼します。	
\\	失礼します。 
\\	田中君、遅いよ。罰金1000円ね。次は2000円だからね。	
\\	田中君、遅いよ。罰金1000円ね。次は2000円だからね。 
\\	すみませんでした、社長。1000円ですか。痛いですね。	
\\	すみませんでした、社長。1000円ですか。痛いですね。 
\\	先に
\\	レディ
\\	ファースト
\\	教科書
\\	返す
\\	お先にどうぞ。	
\\	お先にどうぞ。 
\\	いいえ、どうぞ、どうぞ、言ってください。	
\\	いいえ、どうぞ、どうぞ、言ってください。 
\\	渡辺君のを先に聞きたいです。	
\\	渡辺君のを先に聞きたいです。 
\\	レディファーストで。	
\\	レディファーストで。 
\\	じゃ、一緒に言いましょう。	
\\	じゃ、一緒に言いましょう。 
\\	じゃ、「せいの」で	
\\	じゃ、「せいの」で 
\\	せいの	
\\	せいの 
\\	好きです。	
\\	好きです。 
\\	私の数学の教科書を返してください。	
\\	私の数学の教科書を返してください。 
\\	ああ、おお、私も数学が好きです!	
\\	ああ、おお、私も数学が好きです! 
\\	あ、ちょっと待って!まだ話があります・・・	
\\	あ、ちょっと待って!まだ話があります・・・ 
\\	いよいよ
\\	厳しい
\\	水着
\\	実は
\\	泳ぐ
\\	海
\\	本番
\\	確か
\\	今日何しますか。	
\\	今日何しますか。 
\\	今日?天気を見てください!いよいよ夏が来ました。	
\\	今日?天気を見てください!いよいよ夏が来ました。 
\\	海ですか。あまり行きたくないです。	
\\	海ですか。あまり行きたくないです。 
\\	まあ、聞いてください。海に行って、泳いで、かき氷を食べて、まさに日本の夏ですよ。	
\\	まあ、聞いてください。海に行って、泳いで、かき氷を食べて、まさに日本の夏ですよ。 
\\	そうですけど。実は、今、水着がちょっと厳しいんです。	
\\	そうですけど。実は、今、水着がちょっと厳しいんです。 
\\	ああ、確かに厳しいですね。	
\\	ああ、確かに厳しいですね。 
\\	先週末
\\	娘
\\	死ぬ
\\	帰り道
\\	遭う
\\	踏む
\\	叫ぶ
\\	殴る
\\	あし
\\	先週末はどうでしたか。	
\\	先週末はどうでしたか。 
\\	最悪でした!本当に最悪でした。	
\\	最悪でした!本当に最悪でした。 
\\	どうしましたか。	
\\	どうしましたか。 
\\	娘の金魚が死んで、彼女はずっと元気がありませんでした。	
\\	娘の金魚が死んで、彼女はずっと元気がありませんでした。 
\\	かわいそうな痴漢です。そいつは大丈夫でしたか。	
\\	かわいそうな痴漢です。そいつは大丈夫でしたか。 
\\	まあね。	
\\	まあね。 
\\	どうしましたか。	
\\	どうしましたか。 
\\	そいつの足を踏んで、顔を殴って、助けてって叫びました。	
\\	"そいつの足を踏んで、顔を殴って、助けてって叫びました。 
\\	え、あなたが叫びましたか。	
\\	え、あなたが叫びましたか。 
\\	いいえ、そいつが叫びました。それで、警察が来ました。	
\\	いいえ、そいつが叫びました。それで、警察が来ました。 
\\	救急車も来たでしょう。	
\\	救急車も来たでしょう。 
\\	まあね。そいつ、全治二週間らしいですよ。	
\\	まあね。そいつ、全治二週間らしいですよ。 
\\	さっき
\\	ついさっき
\\	冗談
\\	ジョーク
\\	親父ギャグ
\\	だまされやすい
\\	勇気
\\	理由
\\	いました!	
\\	いました! 
\\	さっきはすみませんでした。本当にすみませんでした。	
\\	さっきはすみませんでした。本当にすみませんでした。 
\\	えっ、本気にしましたか。ただの冗談ですよ。	
\\	えっ、本気にしましたか。ただの冗談ですよ。 
\\	ええ!冗談ですか。	
\\	ええ!冗談ですか。 
\\	もちろん。数学の教科書?だまされやすい人ですね。	
\\	もちろん。数学の教科書?だまされやすい人ですね。 
\\	ひどいなあ。やりすぎですよ。	
\\	ひどいなあ。やりすぎですよ。 
\\	その理由は私も同じ気持ちだからです。	
\\	その理由は私も同じ気持ちだからです。 
\\	ベビーシッター
\\	三男
\\	次男
\\	長男
\\	三女
\\	次女
\\	お嬢さん
\\	長女
\\	宿題
\\	息子
\\	宝物
\\	安心
\\	お、そろそろ映画が始まるぞ。	
\\	お、そろそろ映画が始まるぞ。 
\\	待って...やっぱり、心配だわ。電話する。	
\\	待って...やっぱり、心配だわ。電話する。 
\\	いいよ。(電話をかける音、人が出る音)	
\\	いいよ。(電話をかける音、人が出る音) 
\\	はい。亀田です。	
\\	はい。亀田です。 
\\	もしもし、亀田です。	
\\	もしもし、亀田です。 
\\	こんばんは。	
\\	こんばんは。 
\\	私の宝物たちは大丈夫ですか。	
\\	私の宝物たちは大丈夫ですか。 
\\	大丈夫ですよ。	
\\	大丈夫ですよ。 
\\	何をしてますか。	
\\	何をしてますか。 
\\	息子さんは宿題をやっています。上のお嬢さんは携帯で友達と話しています。	
\\	息子さんは宿題をやっています。上のお嬢さんは携帯で友達と話しています。 
\\	ありがとう。安心しました。よろしくね。じゃ。	
\\	ありがとう。安心しました。よろしくね。じゃ。 
\\	失礼します。	
\\	失礼します。 
\\	相変わらず
\\	ネット
\\	インターネット
\\	かわいい子
\\	大人しい子
\\	素直な子
\\	うるさい子
\\	いたずらっ子
\\	いじめっ子
\\	賢い子
\\	いい子
\\	悪い子
\\	遺伝子
\\	みんなは何していますか。	
\\	みんなは何していますか。 
\\	私の息子は宿題をやっています。	
\\	私の息子は宿題をやっています。 
\\	テレビは見ていませんか。	
\\	テレビは見ていませんか。 
\\	もちろん。見ていません。私の息子はテレビは見ません。	
\\	もちろん。見ていません。私の息子はテレビは見ません。 
\\	はあ?	
\\	はあ? 
\\	でも、あなたの娘たちは相変わらず悪い子です。	
\\	でも、あなたの娘たちは相変わらず悪い子です。 
\\	私たちの子供でしょう?	
\\	私たちの子供でしょう? 
\\	いいえ。悪い子はあなたの。いい子は私の。	
\\	いいえ。悪い子はあなたの。いい子は私の。 
\\	はあ?	
\\	はあ? 
\\	やっぱり、あなたの遺伝子が悪いですね。	
\\	やっぱり、あなたの遺伝子が悪いですね。 
\\	迷惑
\\	全くもう
\\	鴬嬢
\\	会社
\\	環境
\\	〜について
\\	自動車
\\	生命
\\	保険
\\	結構
\\	間に合う
\\	近所
\\	自分
\\	はい。	
\\	はい。 
\\	もしもし。迷惑保険会社の高原でございます。あ、今日は・・・	
\\	もしもし。迷惑保険会社の高原でございます。あ、今日は・・・ 
\\	間に合っています。	
\\	間に合っています。 
\\	あ、はい、かしこまりました。失礼します。(切られる。)	
\\	あ、はい、かしこまりました。失礼します。(切られる。) 
\\	はい。	
\\	はい。 
\\	もしもし。迷惑保険会社の高原でございます。	
\\	もしもし。迷惑保険会社の高原でございます。 
\\	結構です。	
\\	結構です。 
\\	はい、失礼します。(切られる。)	
\\	はい、失礼します。(切られる。) 
\\	はい。	
\\	はい。 
\\	もしもし。迷惑保険会社の高原でございます。(切られる。)	
\\	もしもし。迷惑保険会社の高原でございます。(切られる。) 
\\	この仕事は大嫌い。	
\\	この仕事は大嫌い。 
\\	はい。	
\\	はい。 
\\	もしもし。迷惑保険会社の高原でございます。	
\\	もしもし。迷惑保険会社の高原でございます。 
\\	どうも。	
\\	どうも。 
\\	今日は保険について・・・	
\\	今日は保険について・・・ 
\\	鶯嬢ですね。	
\\	鶯嬢ですね。 
\\	もしもし。誰ですか。	
\\	もしもし。誰ですか。 
\\	迷惑保険会社の高原でございます。今日は保険について・・・	
\\	迷惑保険会社の高原でございます。今日は保険について・・・ 
\\	おばあさん、高田さんは鶯嬢ですね。ね。	
\\	おばあさん、高田さんは鶯嬢ですね。ね。 
\\	ありがとうございます。	
\\	ありがとうございます。 
\\	おじいさん!あなたはナンパしていますか。	
\\	おじいさん!あなたはナンパしていますか。 
\\	いいえ。ナンパしていません。ただ、会話しています。	
\\	いいえ。ナンパしていません。ただ、会話しています。 
\\	まったくもう!そして、鴬嬢さん、自分の男を見つけてください!保険はいりません!失礼します。	
\\	まったくもう!そして、鴬嬢さん、自分の男を見つけてください!保険はいりません!失礼します。 
\\	もういい、こんな仕事もうやめる。	
\\	もういい、こんな仕事もうやめる。 
\\	知り合う
\\	相変わらず
\\	恋
\\	落ちる
\\	面接
\\	横浜
\\	受ける
\\	お互い
\\	頑張る
\\	明日友達に会いに東京へ行きます。	
\\	明日友達に会いに東京へ行きます。 
\\	どういう友達ですか。	
\\	どういう友達ですか。 
\\	ネット友達です。明日初めて会います。	
\\	ネット友達です。明日初めて会います。 
\\	どこで知り合ったんですか。	
\\	どこで知り合ったんですか。 
\\	ミクシィです。彼女が大好きです。	
\\	ミクシィです。彼女が大好きです。 
\\	相変わらず、すぐに恋に落ちますね。	
\\	相変わらず、すぐに恋に落ちますね。 
\\	明日一緒に行きませんか。	
\\	明日一緒に行きませんか。 
\\	ごめん。明日面接があります。横浜へ受けに行きます。	
\\	ごめん。明日面接があります。横浜へ受けに行きます。 
\\	頑張ってください。	
\\	頑張ってください。 
\\	お互い頑張りましょう!	
\\	お互い頑張りましょう! 
\\	上野
\\	楽しみにする
\\	かぶる
\\	連れる
\\	シャツ
\\	格好
\\	向かう
\\	そちら
\\	公園
\\	超かわいい
\\	はい。	
\\	はい。 
\\	もしもし。今上野公園にきています。今どこですか。	
\\	もしもし。今上野公園にきています。今どこですか。 
\\	ああ、今そちらへ向かっています。	
\\	ああ、今そちらへ向かっています。 
\\	どんな格好していますか。	
\\	どんな格好していますか。 
\\	白い帽子をかぶっていて、赤いシャツを着ていて、犬をつれています。	
\\	白い帽子をかぶっていて、赤いシャツを着ていて、犬をつれています。 
\\	犬ですか!	
\\	犬ですか! 
\\	うん、もちろん!犬は好きでしょう?	
\\	うん、もちろん!犬は好きでしょう? 
\\	実は、あまり。。。	
\\	実は、あまり。。。 
\\	そうですね。 楽しみにしています!	
\\	そうですね。 楽しみにしています! 
\\	じゃ。後5分で。じゃね。	
\\	じゃ。後5分で。じゃね。 
\\	じゃね。	
\\	じゃね。 
\\	ほら
\\	それより
\\	散歩する
\\	たこ焼き
\\	鰹節
\\	鰹
\\	築地
\\	いました!	
\\	いました! 
\\	やっと来ました。	
\\	やっと来ました。 
\\	ほら、かわいいでしょう。	
\\	ほら、かわいいでしょう。 
\\	それより、今日何をしますか。	
\\	それより、今日何をしますか。 
\\	上野公園でこの子のお散歩をしてから、お昼を食べに行きます。	
\\	上野公園でこの子のお散歩をしてから、お昼を食べに行きます。 
\\	いいですね。何がいいですか。	
\\	いいですね。何がいいですか。 
\\	タイ料理はどうですか。	
\\	タイ料理はどうですか。 
\\	私には辛すぎます。	
\\	私には辛すぎます。 
\\	じゃ、たこ焼きは?	
\\	じゃ、たこ焼きは? 
\\	普通すぎます。	
\\	普通すぎます。 
\\	じゃ、築地でお寿司はどうですか。	
\\	じゃ、築地でお寿司はどうですか。 
\\	いいですね。行きましょう。でも、犬は?	
\\	いいですね。行きましょう。でも、犬は? 
\\	大丈夫。この子、マグロも大好きだから。	
\\	大丈夫。この子、マグロも大好きだから。 
\\	紙飛行機
\\	紙
\\	骨
\\	折る
\\	半分
\\	三角形
\\	四角形
\\	四角
\\	三角関係
\\	三角
\\	紙皿
\\	紙コップ
\\	紙風船
\\	完成
\\	皆さん、今日は一緒に紙飛行機作りをやりましょう。	
\\	皆さん、今日は一緒に紙飛行機作りをやりましょう。 
\\	やー!	
\\	やー! 
\\	最初に紙を取ってください。まずは紙を半分に折ってください。そして、その紙をさらに半分に折ってください。	
\\	最初に紙を取ってください。まずは紙を半分に折ってください。そして、その紙をさらに半分に折ってください。 
\\	出来た!	
\\	出来た! 
\\	特急
\\	チェック
\\	東京証券取引所
\\	株価
\\	日本経済新聞
\\	読売新聞
\\	朝日新聞
\\	新聞
\\	暖かい
\\	涼しい
\\	窓
\\	満席
\\	空席
\\	席
\\	急行
\\	空く
\\	すみません。この席は空いていますか。	
\\	すみません。この席は空いていますか。 
\\	はい、空いていますよ。	
\\	はい、空いていますよ。 
\\	あ、すみません。暑いですね。	
\\	あ、すみません。暑いですね。 
\\	そうですね。	
\\	そうですね。 
\\	すみませんが、窓を開けてもいいですか。	
\\	すみませんが、窓を開けてもいいですか。 
\\	いいですけど。	
\\	いいですけど。 
\\	ありがとうございます。ああ、涼しいですね。気持ちいい。	
\\	ありがとうございます。ああ、涼しいですね。気持ちいい。 
\\	ああ。	
\\	ああ。 
\\	ああ、今日の新聞ですか。	
\\	ああ、今日の新聞ですか。 
\\	はい、そうです。	
\\	はい、そうです。 
\\	すみません。ちょっと見てもいいですか。昨日の株価をチェックしたいんですが。	
\\	すみません。ちょっと見てもいいですか。昨日の株価をチェックしたいんですが。 
\\	いいですけど。	
\\	いいですけど。 
\\	ありがとうございます!	
\\	ありがとうございます! 
\\	撮る
\\	非常口
\\	入口
\\	出口
\\	消す
\\	卒業証明書
\\	身分証明書
\\	立入禁止
\\	撮影禁止
\\	禁止
\\	写真撮影
\\	映画撮影
\\	撮影
\\	顔パス
\\	すみません。ここで写真を撮ってもいいですか。	
\\	すみません。ここで写真を撮ってもいいですか。 
\\	(写真の音)	
\\	ちょっと!だめですよ!何してるんですか。だめと言ったでしょう。	
\\	ちょっと!だめですよ!何してるんですか。だめと言ったでしょう。 
\\	すっ、すみません。日本語がわかりません!	
\\	すっ、すみません。日本語がわかりません! 
\\	ちょっと!身分証明書を出してください。	
\\	ちょっと!身分証明書を出してください。 
\\	はい、どうぞ。	
\\	はい、どうぞ。 
\\	あなた、日本人でしょう!	
\\	あなた、日本人でしょう! 
\\	はい、そうです。東京出身です。お父さんは北海道出身。お母さんは九州出身。	
\\	はい、そうです。東京出身です。お父さんは北海道出身。お母さんは九州出身。 
\\	もういいから。今の写真を消して、帰ってください。	
\\	もういいから。今の写真を消して、帰ってください。 
\\	はい、わかりました。	
\\	はい、わかりました。 
\\	出口はこちらです。	
\\	出口はこちらです。 
\\	留守電
\\	接続
\\	合図
\\	以内
\\	伝言
\\	接続する
\\	彼女たち遅いな〜。	
\\	彼女たち遅いな〜。 
\\	本当に来るのかな〜。もう11時だぜ。	
\\	本当に来るのかな〜。もう11時だぜ。 
\\	じゃあ、もう一度彼女たちに電話する。	
\\	じゃあ、もう一度彼女たちに電話する。 
\\	あ〜。もしもし。よしですけど。今11時です。	
\\	あ〜。もしもし。よしですけど。今11時です。 
\\	預かる
\\	流れる
\\	消去する
\\	保存する
\\	場合
\\	再生
\\	操作方法
\\	確認
\\	圏外
\\	店はどこ?	
\\	店はどこ? 
\\	わからない。	
\\	わからない。 
\\	一時間以上捜しているのに。	
\\	一時間以上捜しているのに。 
\\	もう一回電話する。	
\\	もう一回電話する。 
\\	また圏外だ。	
\\	また圏外だ。 
\\	あの二人はどこにいるのかな?	
\\	あの二人はどこにいるのかな? 
\\	あっ。留守電が入っている。	
\\	あっ。留守電が入っている。 
\\	あ〜。もしもし。よしですけど。今11時です。確かに約束時間は10時ですけど。たけさんといつまでも待っていますので、よろしくお願いします。	
\\	あ〜。もしもし。よしですけど。今11時です。確かに約束時間は10時ですけど。たけさんといつまでも待っていますので、よろしくお願いします。 
\\	二人がまだ待ってる。どうする?	
\\	二人がまだ待ってる。どうする? 
\\	説明
\\	一応
\\	お帰り!・・・あっ、その顔、大丈夫ですか。	
\\	お帰り!・・・あっ、その顔、大丈夫ですか。 
\\	ううん、あまり。	
\\	ううん、あまり。 
\\	どうでしたか。ネット友達と会いましたか。	
\\	どうでしたか。ネット友達と会いましたか。 
\\	うん、一応。	
\\	うん、一応。 
\\	で、どうでしたか。	
\\	で、どうでしたか。 
\\	聞かないでください。	
\\	聞かないでください。 
\\	ええ、どうしましたか。彼女のことが好きなんでしょう?	
\\	ええ、どうしましたか。彼女のことが好きなんでしょう? 
\\	言わないでください。	
\\	言わないでください。 
\\	どうしましたか。	
\\	どうしましたか。 
\\	今、話したくないです。明日説明します。	
\\	今、話したくないです。明日説明します。 
\\	では、明日話しましょう!	
\\	では、明日話しましょう! 
\\	プロジェクト
\\	順調
\\	進む
\\	大会
\\	優勝
\\	優先
\\	命
\\	全員
\\	参加
\\	本田さん、どうも。今月のプロジェクトはどうですか。	
\\	本田さん、どうも。今月のプロジェクトはどうですか。 
\\	順調に進んでいます。そちらのプロジェクトはどうですか。	
\\	順調に進んでいます。そちらのプロジェクトはどうですか。 
\\	負けないですよ!私たちも進んでいます。	
\\	負けないですよ!私たちも進んでいます。 
\\	今回は行きません。私たちはプロジェクト優先。会社が命ですから。	
\\	今回は行きません。私たちはプロジェクト優先。会社が命ですから。 
\\	でも、部長たちは全員参加するんですが。	
\\	でも、部長たちは全員参加するんですが。 
\\	ゴルフの結果より、プロジェクトの結果の方が大事でしょう。	
\\	ゴルフの結果より、プロジェクトの結果の方が大事でしょう。 
\\	そうですよね。さすが本田さん。	
\\	そうですよね。さすが本田さん。 
\\	急に
\\	締め切り
\\	褒める
\\	次回
\\	今回
\\	本田さん、おはようございます。今日はすばらしい天気ですね。	
\\	本田さん、おはようございます。今日はすばらしい天気ですね。 
\\	そうですね。実は、急に用事ができて、プロジェクトは全然進んでいないです。	
\\	そうですね。実は、急に用事ができて、プロジェクトは全然進んでいないです。 
\\	あ、大変ですね。	
\\	あ、大変ですね。 
\\	いやー、金曜日の締め切りまでにはぎりぎり終わるでしょう。今回は鈴木さんの勝ちですね。	
\\	いやー、金曜日の締め切りまでにはぎりぎり終わるでしょう。今回は鈴木さんの勝ちですね。 
\\	ふ、まあね。ま、次回もあるから、頑張ってよ。	
\\	ふ、まあね。ま、次回もあるから、頑張ってよ。 
\\	本田さん、先週末はすばらしいゴルフでした!優勝おめでとう!社長も褒めてますよ。	
\\	本田さん、先週末はすばらしいゴルフでした!優勝おめでとう!社長も褒めてますよ。 
\\	なるほど
\\	持ってくる
\\	連れて行く
\\	連れてくる
\\	かなり
\\	飼う
\\	キレる
\\	悲惨(な)
\\	逃げ出す
\\	叫び出す
\\	持って行く
\\	彼女とどうでしたか。	
\\	彼女とどうでしたか。 
\\	寿司を食べに行きましたけど、	
\\	寿司を食べに行きましたけど、 
\\	犬ですか。	
\\	犬ですか。 
\\	はい、彼女は犬を飼っていて、その犬を連れてきました。	
\\	はい、彼女は犬を飼っていて、その犬を連れてきました。 
\\	なるほど、わかりました。それから?	
\\	なるほど、わかりました。それから? 
\\	店員が犬はだめだと言いました。それで彼女が切れました。	
\\	店員が犬はだめだと言いました。それで彼女が切れました。 
\\	マジですか。	
\\	マジですか。 
\\	大変でした。	
\\	大変でした。 
\\	そうですか。私は自分の面接はひどいと思いました。	
\\	そうですか。私は自分の面接はひどいと思いました。 
\\	そう、そう、面接はどうでしたか。	
\\	そう、そう、面接はどうでしたか。 
\\	まあ、今度教えます。	
\\	まあ、今度教えます。 
\\	部長
\\	随分
\\	すっかり
\\	還暦
\\	魚座
\\	星座
\\	誕生日
\\	本部長
\\	次長
\\	課長
\\	係長
\\	社員
\\	アイディア
\\	今日、部長の誕生日ですね。	
\\	今日、部長の誕生日ですね。 
\\	ああ、すっかり忘れてました!	
\\	ああ、すっかり忘れてました! 
\\	確か今年は還暦ですよ。部長に何をあげましょうか。	
\\	確か今年は還暦ですよ。部長に何をあげましょうか。 
\\	還暦ですか。そうですね。ワインはどうですか。	
\\	還暦ですか。そうですね。ワインはどうですか。 
\\	いいですね。2006年ものはどうですか。	
\\	いいですね。2006年ものはどうですか。 
\\	いいアイディアですね。1946年ものより、随分安いし。決まりです!	
\\	いいアイディアですね。1946年ものより、随分安いし。決まりです! 
\\	今夜の誕生日パーティで渡しましょう!	
\\	今夜の誕生日パーティで渡しましょう! 
\\	そうしましょう。	
\\	そうしましょう。 
\\	もう
\\	思い出す
\\	そんな
\\	手を貸す
\\	手伝う
\\	残す
\\	残る
\\	着く
\\	品川
\\	どうも
\\	片付ける
\\	はい、山森です。	
\\	はい、山森です。 
\\	もしもし、前原ですけど。	
\\	もしもし、前原ですけど。 
\\	ああ、どうも。私はもう品川に着きました!後どのぐらいで着きますか。	
\\	ああ、どうも。私はもう品川に着きました!後どのぐらいで着きますか。 
\\	実は、まだ会社にいます。	
\\	実は、まだ会社にいます。 
\\	ええ、大丈夫ですか。	
\\	ええ、大丈夫ですか。 
\\	え、大丈夫ですけど、まだ仕事が残っています。	
\\	え、大丈夫ですけど、まだ仕事が残っています。 
\\	大丈夫です。手伝ってあげますよ。	
\\	大丈夫です。手伝ってあげますよ。 
\\	ええ、そんな。いいですよ。	
\\	ええ、そんな。いいですよ。 
\\	いいから。今から行きます。30分後に着きます。	
\\	いいから。今から行きます。30分後に着きます。 
\\	ああ、ありがとうございます。	
\\	ああ、ありがとうございます。 
\\	早く片付けましょう。仕事はどれですか。	
\\	早く片付けましょう。仕事はどれですか。 
\\	はい、これです。	
\\	はい、これです。 
\\	え?百ページもあるんですか。	
\\	え?百ページもあるんですか。 
\\	ええ?ちょっと待って。	
\\	ええ?ちょっと待って。 
\\	部下
\\	生まれ変わる
\\	ケチ
\\	者
\\	ロマンチック
\\	同じ
\\	ただいま。	
\\	ただいま。 
\\	お帰りなさい。パーティーはどうでした?	
\\	お帰りなさい。パーティーはどうでした? 
\\	パーティーは最高でしたけど、プレゼントは最悪でした。	
\\	パーティーは最高でしたけど、プレゼントは最悪でした。 
\\	まあ、どういうことですか。	
\\	まあ、どういうことですか。 
\\	二人の部下に2006年もののワインをもらいました!	
\\	二人の部下に2006年もののワインをもらいました! 
\\	あら、違いますよ。きっと還暦だからでしょう?今年また生まれ変わりますから。	
\\	あら、違いますよ。きっと還暦だからでしょう?今年また生まれ変わりますから。 
\\	何を言っているんですか。ただのケチでしょう!	
\\	何を言っているんですか。ただのケチでしょう! 
\\	そんなこと言わないでください。だって、私からのプレゼントも同じものなんですから。	
\\	そんなこと言わないでください。だって、私からのプレゼントも同じものなんですから。 
\\	ひどい。私の奥さんまでケチだ。	
\\	ひどい。私の奥さんまでケチだ。 
\\	鉛筆
\\	貸す
\\	他
\\	本
\\	男子
\\	特別
\\	料金
\\	領収書
\\	皆さん、試験は五分後に始まります。	
\\	皆さん、試験は五分後に始まります。 
\\	やばい。鉛筆がない。ああ、本田さんがいっぱい持ってる。本田さん、鉛筆を貸してください。	
\\	やばい。鉛筆がない。ああ、本田さんがいっぱい持ってる。本田さん、鉛筆を貸してください。 
\\	いいですよ。どうぞ。	
\\	いいですよ。どうぞ。 
\\	ありがとうございます!	
\\	ありがとうございます! 
\\	ね、山田さん、他にも鉛筆を持ってますか。	
\\	ね、山田さん、他にも鉛筆を持ってますか。 
\\	いいえ。本田さんにこの鉛筆を貸してもらいました。	
\\	いいえ。本田さんにこの鉛筆を貸してもらいました。 
\\	本田さん!一本お願いします!	
\\	本田さん!一本お願いします! 
\\	はい、一本500円でございます。男子には特別料金をもらいます。	
\\	はい、一本500円でございます。男子には特別料金をもらいます。 
\\	はあ!特別料金?どういうことですか。	
\\	はあ!特別料金?どういうことですか。 
\\	試験は後2分です。	
\\	試験は後2分です。 
\\	ああ、今から一本、1000円です。	
\\	ああ、今から一本、1000円です。 
\\	はい、はい、一本お願いします。	
\\	はい、はい、一本お願いします。 
\\	どうもありがとうございます。領収書はいいですか。	
\\	どうもありがとうございます。領収書はいいですか。 
\\	結構です。	
\\	結構です。 
\\	人気者
\\	暑中見舞い
\\	返事
\\	面倒くさい
\\	さっさと
\\	~枚
\\	宛
\\	暑中見舞いの返事を書きましょう。	
\\	暑中見舞いの返事を書きましょう。 
\\	面倒くさいな〜。	
\\	面倒くさいな〜。 
\\	さっさと,書きましょう!	
\\	さっさと,書きましょう! 
\\	はいはい。今年は何枚来てる。	
\\	はいはい。今年は何枚来てる。 
\\	100枚も来ました。	
\\	100枚も来ました。 
\\	100枚も!私達は人気者だね。	
\\	100枚も!私達は人気者だね。 
\\	大丈夫よ。99枚は私宛、あなたにはたった1枚です。	
\\	大丈夫よ。99枚は私宛、あなたにはたった1枚です。 
\\	やった!さすがお母さん!お母さんは毎年暑中見舞いをくれます。	
\\	やった!さすがお母さん!お母さんは毎年暑中見舞いをくれます。 
\\	メール
\\	初~
\\	デート
\\	~の御陰で
\\	時間を過ごす
\\	輝く
\\	ね、山野さん、ちょっと手伝ってくれませんか。	
\\	ね、山野さん、ちょっと手伝ってくれませんか。 
\\	いいですよ。なんでしょうか。	
\\	いいですよ。なんでしょうか。 
\\	昨日はひろこさんと初デートでした。今からメールを送りたいです。これはどうですか。	
\\	昨日はひろこさんと初デートでした。今からメールを送りたいです。これはどうですか。 
\\	どう思いますか。	
\\	どう思いますか。 
\\	う〜ん。何も言うことはありません。でも、これはどうですか。	
\\	う〜ん。何も言うことはありません。でも、これはどうですか。 
\\	さすが山野さん。	
\\	さすが山野さん。 
\\	盛岡
\\	人形焼き
\\	名曲
\\	名画
\\	名物
\\	冷麺
\\	岩手山
\\	わんこそば
\\	岩手県
\\	舟っこ流し
\\	お帰りなさい。お二人とも盛岡はどうでしたか。	
\\	お帰りなさい。お二人とも盛岡はどうでしたか。 
\\	とてもよかったです!	
\\	とてもよかったです! 
\\	楽しくて、食べ物が美味しくて、また行きたいです。	
\\	楽しくて、食べ物が美味しくて、また行きたいです。 
\\	何が美味しかったですか。	
\\	何が美味しかったですか。 
\\	冷麺です。冷麺を食べたことがありますか。	
\\	冷麺です。冷麺を食べたことがありますか。 
\\	いいえ、食べたことはありませんが、テレビで見たことがあります。	
\\	いいえ、食べたことはありませんが、テレビで見たことがあります。 
\\	そうです。	
\\	そうです。 
\\	いいですね。私も食べたいです。	
\\	いいですね。私も食べたいです。 
\\	はい、どうぞ。お土産です。	
\\	はい、どうぞ。お土産です。 
\\	盛岡冷麺ですか。いいんですか。	
\\	盛岡冷麺ですか。いいんですか。 
\\	もちろん!	
\\	もちろん! 
\\	最後
\\	確認
\\	戸締まり
\\	わざと
\\	嫌われ者
\\	夏休みの旅行に出発する前に最後の確認です!	
\\	夏休みの旅行に出発する前に最後の確認です! 
\\	はい!よっしゃ!	
\\	はい!よっしゃ! 
\\	戸締まりはしましたか。	
\\	戸締まりはしましたか。 
\\	オッケーです!	
\\	オッケーです! 
\\	お金とパスポートを持ちましたか。	
\\	お金とパスポートを持ちましたか。 
\\	オッケーです!	
\\	オッケーです! 
\\	子供達?	
\\	子供達? 
\\	いるよ!	
\\	いるよ! 
\\	おばあさん?	
\\	おばあさん? 
\\	ああ、ちょっと待ってください。	
\\	ああ、ちょっと待ってください。 
\\	やっぱり私は嫌われ者ですね。	
\\	やっぱり私は嫌われ者ですね。 
\\	バーベキュー
\\	道具
\\	揃える
\\	揃う
\\	書類
\\	準備
\\	欠席
\\	今週末、バーベキューをしましょう。	
\\	今週末、バーベキューをしましょう。 
\\	今週末ですか。	
\\	今週末ですか。 
\\	うん、今週末。土曜日は晴れだし、他のメンバーたちも暇で、私はもう道具を揃えました。	
\\	うん、今週末。土曜日は晴れだし、他のメンバーたちも暇で、私はもう道具を揃えました。 
\\	来週の月曜日から新しい仕事を始めるので、来週から忙しくなります。	
\\	来週の月曜日から新しい仕事を始めるので、来週から忙しくなります。 
\\	でも、来週は来週です。まだ今週末は暇でしょう。	
\\	でも、来週は来週です。まだ今週末は暇でしょう。 
\\	いいえ。書類を準備したいです。ごめんなさい。今回は、欠席します。	
\\	いいえ。書類を準備したいです。ごめんなさい。今回は、欠席します。 
\\	インタビュー
\\	頃
\\	馬鹿馬鹿しい
\\	独占
\\	夢
\\	叶う
\\	まもなく、石川さんにインタビューを行います。ああ、彼女が来ました!	
\\	まもなく、石川さんにインタビューを行います。ああ、彼女が来ました! 
\\	う〜ん。音楽が好きだからです。	
\\	う〜ん。音楽が好きだからです。 
\\	子供の頃について教えてください。	
\\	子供の頃について教えてください。 
\\	そうですね。子供の頃、夢が三つありました。	
\\	そうですね。子供の頃、夢が三つありました。 
\\	その三つとは何でしょうか。	
\\	その三つとは何でしょうか。 
\\	第一に、有名になって、第二に、綺麗になって、	
\\	第一に、有名になって、第二に、綺麗になって、 
\\	いいえ、その三つの夢は叶いました!	
\\	いいえ、その三つの夢は叶いました! 
\\	自慢
\\	手作り
\\	餃子
\\	材料
\\	もしもし。ともちゃん?	
\\	もしもし。ともちゃん? 
\\	うん。	
\\	うん。 
\\	あのね、今夜お料理を作るの。一緒に食べない?	
\\	あのね、今夜お料理を作るの。一緒に食べない? 
\\	いいね。何食べるの?	
\\	いいね。何食べるの? 
\\	あたしの自慢の手作り餃子。	
\\	あたしの自慢の手作り餃子。 
\\	食べたい!一緒に材料を買いに行くよ。	
\\	食べたい!一緒に材料を買いに行くよ。 
\\	本当に?ありがとう!	
\\	本当に?ありがとう! 
\\	もしもし。ともや?	
\\	もしもし。ともや? 
\\	うん。	
\\	うん。 
\\	あのさあ、俺今夜料理を作るんだ。一緒に食べない?	
\\	あのさあ、俺今夜料理を作るんだ。一緒に食べない? 
\\	いいねぇ。何食べるの?	
\\	いいねぇ。何食べるの? 
\\	俺の自慢の手作り餃子だよ。	
\\	俺の自慢の手作り餃子だよ。 
\\	食べたい!一緒に材料を買いに行くよ。	
\\	食べたい!一緒に材料を買いに行くよ。 
\\	よろしく!	
\\	よろしく! 
\\	彫刻
\\	彫刻家
\\	刻む
\\	雪の彫刻
\\	彫る
\\	真面目
\\	マジ
\\	祭り
\\	雪
\\	戌年
\\	麻布十番
\\	残念
\\	(女性形)昨日何で麻布十番祭りに来なかったの?	
\\	(女性形)昨日何で麻布十番祭りに来なかったの? 
\\	昨日は会社にずっといたの。	
\\	昨日は会社にずっといたの。 
\\	日曜日なのに。	
\\	日曜日なのに。 
\\	うん。最近忙しいの。お祭りはどうだった?	
\\	うん。最近忙しいの。お祭りはどうだった? 
\\	すごかったよ。人が多くて食べ物が美味しくて	
\\	すごかったよ。人が多くて食べ物が美味しくて 
\\	雪の彫刻?	
\\	雪の彫刻? 
\\	うん。戌年だから、犬の形だったよ。	
\\	うん。戌年だから、犬の形だったよ。 
\\	すごいね。行きたかったな。残念。	
\\	すごいね。行きたかったな。残念。 
\\	(男性形)昨日何で麻布十番祭りに来なかったの?	
\\	(男性形)昨日何で麻布十番祭りに来なかったの? 
\\	昨日はずっと会社にいたから。	
\\	昨日はずっと会社にいたから。 
\\	日曜日なのに。	
\\	日曜日なのに。 
\\	うん。最近忙しくて。お祭りはどうだった?	
\\	うん。最近忙しくて。お祭りはどうだった? 
\\	すごかったよ。人が多くて食べ物が美味くて	
\\	すごかったよ。人が多くて食べ物が美味くて 
\\	雪の彫刻?マジで?	
\\	雪の彫刻?マジで? 
\\	うん。戌年だから、犬の形だったよ。	
\\	うん。戌年だから、犬の形だったよ。 
\\	すごいね。行きたかったな。残念。	
\\	すごいね。行きたかったな。残念。 
\\	まもなく
\\	特に
\\	混む
\\	女性専用車両
\\	ラッシュ
\\	安心
\\	お盆休み
\\	今日は電車が特に混んでいるね。	
\\	今日は電車が特に混んでいるね。 
\\	お盆休みは終わったからね。	
\\	お盆休みは終わったからね。 
\\	女性専用車両があって良かったね。	
\\	女性専用車両があって良かったね。 
\\	そう、そう。	
\\	そう、そう。 
\\	混んでいるけど、安心が一番よ。	
\\	混んでいるけど、安心が一番よ。 
\\	ジム
\\	いや
\\	砂浜
\\	ジョギング
\\	運動
\\	気分
\\	田舎
\\	筋トレ
\\	最近、ジムへ行ってる?	
\\	最近、ジムへ行ってる? 
\\	いや、あまり行ってない。最近、公園と砂浜でジョギングしてる。	
\\	いや、あまり行ってない。最近、公園と砂浜でジョギングしてる。 
\\	そう?僕はジムがいい。	
\\	そう?僕はジムがいい。 
\\	部屋
\\	彼氏
\\	教える
\\	携帯
\\	馬鹿
\\	怒る
\\	質問
\\	暇
\\	予定
\\	彼女
\\	おねえちゃん、ちょっといい?	
\\	おねえちゃん、ちょっといい? 
\\	あっ、部屋に入らないで!そこから話して。	
\\	あっ、部屋に入らないで!そこから話して。 
\\	今日暇?	
\\	今日暇? 
\\	馬鹿な質問しないでよ!今日は忙しいのよ。	
\\	馬鹿な質問しないでよ!今日は忙しいのよ。 
\\	今日何するの?	
\\	今日何するの? 
\\	今日?デートよ。他には何?	
\\	今日?デートよ。他には何? 
\\	いやっ、それだけ。	
\\	いやっ、それだけ。 
\\	あら、私の携帯どこかしら?(足音がする)	
\\	あら、私の携帯どこかしら?(足音がする) 
\\	あ、じゃあ失礼します(携帯を切る)。	
\\	あ、じゃあ失礼します(携帯を切る)。 
\\	ちょっと私の携帯で何してるの?今誰と話してたの?	
\\	ちょっと私の携帯で何してるの?今誰と話してたの? 
\\	おねえちゃんの彼氏。	
\\	おねえちゃんの彼氏。 
\\	何言ったのよ!?	
\\	何言ったのよ!? 
\\	おねえちゃんの今日の予定を教えたんだよ。今日はおねえちゃんは忙しいって。	
\\	おねえちゃんの今日の予定を教えたんだよ。今日はおねえちゃんは忙しいって。 
\\	他の
\\	浮気者
\\	にいちゃん
\\	青森
\\	女の人
\\	泣く
\\	気にする
\\	機嫌
\\	にいちゃん、ちょっといい?	
\\	にいちゃん、ちょっといい? 
\\	部屋に入るな!	
\\	部屋に入るな! 
\\	何で?どうしたの?	
\\	何で?どうしたの? 
\\	ああ、ごめん。今機嫌が悪いんだ。	
\\	ああ、ごめん。今機嫌が悪いんだ。 
\\	どうしたの?	
\\	どうしたの? 
\\	彼女が他の男とデートするんだ。	
\\	彼女が他の男とデートするんだ。 
\\	大丈夫。大丈夫。泣くな、にいちゃん。	
\\	大丈夫。大丈夫。泣くな、にいちゃん。 
\\	え?って、ここは青森じゃないか。	
\\	え?って、ここは青森じゃないか。 
\\	やった
\\	ぎりぎり
\\	セーフ
\\	出番
\\	ストライク
\\	三振
\\	惜しい
\\	(ボールを打った音)	
\\	走れ!走れ!行け!やった!ぎりぎりセーフだった。	
\\	走れ!走れ!行け!やった!ぎりぎりセーフだった。 
\\	わかりました。打ってきます。	
\\	わかりました。打ってきます。 
\\	ストライク、ストライク、ストライク、三振バッターアウト、ゲームセット。	
\\	ストライク、ストライク、ストライク、三振バッターアウト、ゲームセット。 
\\	後もう少しでホームランだったのに。惜しかったなー。	
\\	後もう少しでホームランだったのに。惜しかったなー。 
\\	いや、それは無理だと思うけど、ま、いいか、明日があるさ。	
\\	いや、それは無理だと思うけど、ま、いいか、明日があるさ。 
\\	助ける
\\	助かる
\\	籠る
\\	しっかり
\\	汚い
\\	懲りる
\\	やつ
\\	整理
\\	冷たい
\\	来てくれてありがとう。お兄ちゃんを助けてあげて。	
\\	来てくれてありがとう。お兄ちゃんを助けてあげて。 
\\	誠司!誠司!いるのか?入るよ!やっぱりいた。	
\\	誠司!誠司!いるのか?入るよ!やっぱりいた。 
\\	や、どうも健太。	
\\	や、どうも健太。 
\\	まったく。まだこの部屋にこもってるのか。1ヶ月も経っているのに。	
\\	まったく。まだこの部屋にこもってるのか。1ヶ月も経っているのに。 
\\	いやー、まだ気持ちの整理が着かなくて。まだ時間かかるよ。	
\\	いやー、まだ気持ちの整理が着かなくて。まだ時間かかるよ。 
\\	ほら、鏡で自分を見ろよ。	
\\	ほら、鏡で自分を見ろよ。 
\\	見たくない。	
\\	見たくない。 
\\	じゃあ、これでもかぶれ。	
\\	じゃあ、これでもかぶれ。 
\\	おー、つめて。ありがとう。おかげで目が覚めたよ。よし彼女を取り戻しに行くぞ。	
\\	おー、つめて。ありがとう。おかげで目が覚めたよ。よし彼女を取り戻しに行くぞ。 
\\	懲りないやつだなー、お前は。	
\\	懲りないやつだなー、お前は。 
\\	あ、すいません。	
\\	あ、すいません。 
\\	エレベーター
\\	返事
\\	賢い
\\	階
\\	足音
\\	ありがとうございます!!	
\\	何階ですか?	
\\	何階ですか? 
\\	47階お願いします。ああ、どうもすみません。今日は涼しいですね。	
\\	47階お願いします。ああ、どうもすみません。今日は涼しいですね。 
\\	あ、はい、そうですね。。。。	
\\	あ、はい、そうですね。。。。 
\\	彼女を向いて
\\	あの。。。。。。	
\\	彼女を向いて
\\	あの。。。。。。 
\\	すみません。間違えました。25階お願いします。	
\\	すみません。間違えました。25階お願いします。 
\\	逃げる
\\	どこか
\\	可笑しい
\\	こいつ
\\	怖い
\\	間違う
\\	早く
\\	降りる
\\	足音
\\	ありがとうございます!!	
\\	何階ですか?	
\\	何階ですか? 
\\	47階お願いします。ああ、どうもすみません。今日は涼しいですね。	
\\	47階お願いします。ああ、どうもすみません。今日は涼しいですね。 
\\	あ、はい、そうですね。。。。	
\\	あ、はい、そうですね。。。。 
\\	自分に対して
\\	この人どこかおかしいわ。こいつ何考えてるの?なんかちょっと怖い。いや、怖い。やあ、こっち見てる。	
\\	自分に対して
\\	この人どこかおかしいわ。こいつ何考えてるの?なんかちょっと怖い。いや、怖い。やあ、こっち見てる。 
\\	あの。。。。。。	
\\	あの。。。。。。 
\\	自分に対して
\\	逃げよう!	
\\	自分に対して
\\	逃げよう! 
\\	曲がる
\\	味覚音痴
\\	運動音痴
\\	方向音痴
\\	音痴
\\	単純
\\	確かめる
\\	方向
\\	迷う
\\	機械音痴
\\	ええと。ちょっと待ってください。	
\\	ええと。ちょっと待ってください。 
\\	また迷ってるの??	
\\	また迷ってるの?? 
\\	いや、迷ってない。ただ方向を確かめてるの。	
\\	いや、迷ってない。ただ方向を確かめてるの。 
\\	迷ってる! でも大丈夫、私が分かってるから。まっすぐ行こう。	
\\	迷ってる! でも大丈夫、私が分かってるから。まっすぐ行こう。 
\\	まっすぐ? 違うよ。右に曲がろう。	
\\	まっすぐ? 違うよ。右に曲がろう。 
\\	右じゃない。よしじゃんけんで決めよう。	
\\	右じゃない。よしじゃんけんで決めよう。 
\\	じゃんけんぽん	
\\	じゃんけんぽん 
\\	よし右。お前いつも出すの一緒だよな。	
\\	よし右。お前いつも出すの一緒だよな。 
\\	もう、男は単純なんだから。	
\\	もう、男は単純なんだから。 
\\	ワンちゃん
\\	方法
\\	真ん中
\\	スタート
\\	おやつ
\\	演技
\\	裏切り者
\\	家のワンちゃんはパパとママ、どっちが好きかな?	
\\	家のワンちゃんはパパとママ、どっちが好きかな? 
\\	当然ママよ!	
\\	当然ママよ! 
\\	いや、そんなことはない。	
\\	いや、そんなことはない。 
\\	じゃ、決める方法があるわ。ママはこっち。パパはあっち。犬は真ん中。よしスタート。	
\\	じゃ、決める方法があるわ。ママはこっち。パパはあっち。犬は真ん中。よしスタート。 
\\	こっち来い、小次郎。パパが愛してるよ。いい子だから。	
\\	こっち来い、小次郎。パパが愛してるよ。いい子だから。 
\\	こっち来て、小次郎ちゃん、 おやつ上げるわよ。こっち、こっちよ、こっちでしょ。	
\\	こっち来て、小次郎ちゃん、 おやつ上げるわよ。こっち、こっちよ、こっちでしょ。 
\\	もう毎週同じだわん。つまらないわん。でも美味しいご飯の為に、	
\\	もう毎週同じだわん。つまらないわん。でも美味しいご飯の為に、 
\\	小次郎ちゃん、やっぱりいい子ね。	
\\	小次郎ちゃん、やっぱりいい子ね。 
\\	小次郎め、この裏切り者。	
\\	小次郎め、この裏切り者。 
\\	食べ方
\\	皿
\\	箸
\\	ねた
\\	若いもん
\\	勝手に
\\	ひっくり返す
\\	すみません。お寿司の食べ方を教えてくれませんか。	
\\	すみません。お寿司の食べ方を教えてくれませんか。 
\\	はい、お醤油をこのお皿に入れて、お箸で取って、	
\\	はい、お醤油をこのお皿に入れて、お箸で取って、 
\\	違う!本当のお寿司の食べ方は手で食べるんだよ。	
\\	違う!本当のお寿司の食べ方は手で食べるんだよ。 
\\	ヘー、そうなんですか。	
\\	ヘー、そうなんですか。 
\\	まったく、最近の若いもんは分かってないな。ま、いいか、勝手に食え。	
\\	まったく、最近の若いもんは分かってないな。ま、いいか、勝手に食え。 
\\	毎晩
\\	物凄く
\\	効く
\\	電話する
\\	任せる
\\	起こす
\\	あの音は何?	
\\	あの音は何? 
\\	隣の人なんだ。毎晩毎晩、もの凄くうるさいんだ。	
\\	隣の人なんだ。毎晩毎晩、もの凄くうるさいんだ。 
\\	どうしよう?	
\\	どうしよう? 
\\	わからない。手紙も書いた。話しにも行ったけど、まったく効かないんだ。	
\\	わからない。手紙も書いた。話しにも行ったけど、まったく効かないんだ。 
\\	じゃ、警察に電話しよう。	
\\	じゃ、警察に電話しよう。 
\\	いや、問題を起こしたくない。	
\\	いや、問題を起こしたくない。 
\\	じゃ、まかせて、俺が話しに行く。	
\\	じゃ、まかせて、俺が話しに行く。 
\\	疲れる
\\	目
\\	暖かい
\\	-本
\\	面白い映画だったね。次の映画は?	
\\	面白い映画だったね。次の映画は? 
\\	次?今1時だよ。もう寝ようよ。	
\\	次?今1時だよ。もう寝ようよ。 
\\	寝たくない。もう一本見ようよ。一緒に見ようよ。	
\\	寝たくない。もう一本見ようよ。一緒に見ようよ。 
\\	目が疲れた。もう見られない。暖かいベッドに入りたい〜。おやすみ。	
\\	目が疲れた。もう見られない。暖かいベッドに入りたい〜。おやすみ。 
\\	あ、待って。寝る前にビールもう一本買ってきて。	
\\	あ、待って。寝る前にビールもう一本買ってきて。 
\\	え、ほんとに?	
\\	え、ほんとに? 
\\	虫
\\	動物園
\\	お弁当
\\	ニワトリ
\\	ヒヨコ
\\	猿
\\	豚
\\	せみ
\\	蜂
\\	昆虫
\\	動物園は楽しい!虫もいっぱい。	
\\	動物園は楽しい!虫もいっぱい。 
\\	どれ?	
\\	どれ? 
\\	ブーンブン。	
\\	ブーンブン。 
\\	あれは蜂よ。	
\\	あれは蜂よ。 
\\	ママ、あれは何?	
\\	ママ、あれは何? 
\\	どれ?	
\\	どれ? 
\\	ミーンミンミンミン。	
\\	ミーンミンミンミン。 
\\	あれはせみよ。	
\\	あれはせみよ。 
\\	ママ、あれは何?	
\\	ママ、あれは何? 
\\	どれ?	
\\	どれ? 
\\	ブーブー。	
\\	ブーブー。 
\\	あれは豚よ。	
\\	あれは豚よ。 
\\	ママ、あれは何?	
\\	ママ、あれは何? 
\\	どれ?	
\\	どれ? 
\\	キーキー。	
\\	キーキー。 
\\	あれは猿よ。	
\\	あれは猿よ。 
\\	ママ、あれは何?	
\\	ママ、あれは何? 
\\	どれ?	
\\	どれ? 
\\	ピヨピヨ。	
\\	ピヨピヨ。 
\\	あれはヒヨコよ。	
\\	あれはヒヨコよ。 
\\	ママ、あれは何?	
\\	ママ、あれは何? 
\\	どれ?	
\\	どれ? 
\\	コケコッコー。	
\\	コケコッコー。 
\\	あれはニワトリよ。	
\\	あれはニワトリよ。 
\\	ママ、お腹すいた。	
\\	ママ、お腹すいた。 
\\	じゃあ、お弁当にしましょう。	
\\	じゃあ、お弁当にしましょう。 
\\	隣の者
\\	有名
\\	芸能人
\\	開けろ!	
\\	開けろ! 
\\	どなたですか。	
\\	どなたですか。 
\\	隣の者ですが。	
\\	隣の者ですが。 
\\	はい、ちょっと待って下さい。	
\\	はい、ちょっと待って下さい。 
\\	ええ、あなたは隣の人ではありません。あなたは誰ですか。。。どうしましたか。話せますか。聞こえますか。口を使えますか。もしもし?	
\\	ええ、あなたは隣の人ではありません。あなたは誰ですか。。。どうしましたか。話せますか。聞こえますか。口を使えますか。もしもし? 
\\	あ、あなたは有名な芸能人ですね。	
\\	あ、あなたは有名な芸能人ですね。 
\\	そうですけど、何か用ですか?	
\\	そうですけど、何か用ですか? 
\\	と、とてもすばらしい音楽ですね。もっとボリュームを上げてください。	
\\	と、とてもすばらしい音楽ですね。もっとボリュームを上げてください。 
\\	それだけですか?では、忙しいので。	
\\	それだけですか?では、忙しいので。 
\\	し、失礼します。	
\\	し、失礼します。 
\\	二日酔い
\\	会議
\\	会議中
\\	家族会議
\\	中止
\\	連絡
\\	日程
\\	変更
\\	住所変更
\\	会議は2時からだよね。	
\\	会議は2時からだよね。 
\\	いいえ、中止になりました。	
\\	いいえ、中止になりました。 
\\	中止?私は何も聞いてないぞ。	
\end{CJK}
\end{document}