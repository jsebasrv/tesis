\documentclass[8pt]{extreport} 
\usepackage{hyperref}
\usepackage{CJKutf8}
\begin{document}
\begin{CJK}{UTF8}{min}
\\	問題	
\\	もんだい			
\\	問題が一つあります。	
\\	問題 
\\	もんだい が ひとつ あります			
\\	開発	
\\	かいはつ			
\\	ダムの開発に住民は反対しています。	
\\	開発 
\\	だむ の かいはつ に じゅうみん は はんたい して います			
\\	事件	
\\	じけん			
\\	その事件の犯人はまだ捕まっていない。	
\\	事件 
\\	その じけん の はんにん は まだ つかまって いない			
\\	必要	
\\	ひつよう			
\\	私にはたくさんのお金が必要だ。	
\\	必要 
\\	わたし に は たくさん の おかね が ひつよう だ			
\\	企業	
\\	きぎょう			
\\	彼女はアメリカの企業で働いています。	
\\	企業 
\\	かのじょ は あめりか の きぎょう で はたらいて います			
\\	行う	
\\	おこなう			
\\	その会社は来月、キャンペーンを行う。	
\\	行う 
\\	その かいしゃ は らいげつ きゃんぺーん を おこなう			
\\	現在	
\\	げんざい			
\\	現在の気温は30度です。	
\\	現在 
\\	げんざい の きおん は さんじゅうど です			
\\	利用	
\\	りよう			
\\	私はよく図書館を利用します。	
\\	利用 
\\	わたし は よく としょかん を りよう します			
\\	首相	
\\	しゅしょう			
\\	持つ	
\\	もつ			
\\	この車はよく持っているね。	
\\	持つ 
\\	この くるま は よく もって いる ね			
\\	政府	
\\	せいふ			
\\	そのデモについて政府は何もしなかった。	
\\	政府 
\\	その でも に ついて せいふ は なにも しなかった			
\\	発表	
\\	はっぴょう			
\\	合格者が発表された。	
\\	発表 
\\	ごうかくしゃ が はっぴょう された			
\\	開く	
\\	あく			
\\	電車のドアが開きました。	
\\	開く 
\\	でんしゃ の どあ が あきました			
\\	受ける	
\\	うける			
\\	彼は就職試験を受けた。	
\\	受ける 
\\	かれ は しゅうしょく しけん を うけた			
\\	より	
\\	これから、より一層努力します。	
\\	より 
\\	これから より いっそう どりょく します			
\\	昨年	
\\	さくねん			
\\	昨年は地震が多い年でした。	
\\	昨年 
\\	さくねん は じしん が おおい とし でした			
\\	情報	
\\	じょうほう			
\\	学生たちはインターネットでいろいろな情報を集めた。	
\\	情報 
\\	がくせいたち は いんたーねっと で いろいろな じょうほう を あつめた			
\\	示す	
\\	しめす			
\\	彼は新製品に興味を示している。	
\\	示す 
\\	かれ は しんせいひん に きょうみ を しめして いる			
\\	建設	
\\	けんせつ			
\\	新しいビルの建設が始まった。	
\\	建設 
\\	あたらしい びる の けんせつ が はじまった			
\\	販売	
\\	はんばい			
\\	前売券は窓口で販売しています。	
\\	販売 
\\	まえうりけん は まどぐち で はんばい して います			
\\	治る	
\\	なおる			
\\	けがはもう治りましたか。	
\\	治る 
\\	けが は もう なおりました か			
\\	機能	
\\	きのう			
\\	このソフトにはいろいろな機能があります。	
\\	機能 
\\	この そふと に は いろいろな きのう が あります			
\\	研究	
\\	けんきゅう			
\\	彼は何年も地震の研究をしている。	
\\	研究 
\\	かれ は なんねん も じしん の けんきゅう を して いる			
\\	一方	
\\	いっぽう			
\\	ここは一方通行です。	
\\	一方 
\\	ここ は いっぽう つうこう です			
\\	ドル	
\\	この服は300ドルしました。	
\\	ドル 
\\	この ふく は さんびゃくどる しました			
\\	システム	
\\	全てのシステムが停止した。	
\\	システム 
\\	すべて の しすてむ が ていし した			
\\	頃	
\\	ころ			
\\	子供の頃、よくその公園で遊んだ。	
\\	頃 
\\	こども の ころ よく その こうえん で あそんだ			
\\	対する	
\\	たいする			
\\	その質問に対する答えが見つからなかった。	
\\	対する 
\\	その しつもん に たいする こたえ が みつからなかった			
\\	方法	
\\	ほうほう			
\\	いい方法を思いつきました。	
\\	方法 
\\	いい ほうほう を おもいつきました			
\\	以上	
\\	いじょう			
\\	飛行機が1時間以上遅れた。	
\\	以上 
\\	ひこうき が いちじかん いじょう おくれた			
\\	最近	
\\	さいきん			
\\	それは最近話題の本ですね。	
\\	最近 
\\	それ は さいきん わだい の ほん です ね			
\\	今後	
\\	こんご			
\\	今後ともよろしくお願いします。	
\\	今後 
\\	こんご とも よろしく おねがい します			
\\	世界	
\\	せかい			
\\	私は世界旅行をしたい。	
\\	世界 
\\	わたし は せかい りょこう を したい			
\\	用いる	
\\	もちいる			
\\	彼はその詩を用いて自分の気持ちを伝えた。	
\\	用いる 
\\	かれ は その し を もちいて じぶん の きもち を つたえた			
\\	語る	
\\	かたる			
\\	今夜は大いに語りましょう。	
\\	語る 
\\	こんや は おおいに かたりましょう			
\\	コンピューター	
\\	新しいコンピューターを買った。	
\\	コンピューター 
\\	あたらしい こんぴゅーたー を かった			
\\	中心	
\\	ちゅうしん			
\\	ここが建物の中心です。	
\\	中心 
\\	ここ が たてもの の ちゅうしん です			
\\	やる	
\\	やる			
\\	犬にえさをやった。	
\\	やる 
\\	いぬ に えさ を やった			
\\	意味	
\\	いみ			
\\	それはどういう意味ですか。	
\\	意味 
\\	それ は どういう いみ です か			
\\	計画	
\\	けいかく			
\\	彼は一人旅の計画を立てた。	
\\	計画 
\\	かれ は ひとりたび の けいかく を たてた			
\\	影響	
\\	えいきょう			
\\	私は彼から大きな影響を受けました。	
\\	受ける 
\\	影響 
\\	わたし は かれ から おおき な えいきょう を うけました			
\\	増える	
\\	ふえる			
\\	この町は人口が増えた。	
\\	増える 
\\	この まち は じんこう が ふえた			
\\	選ぶ	
\\	えらぶ			
\\	良い家を選ぶのは難しい。	
\\	選ぶ 
\\	よい いえ を えらぶ の は むずかしい			
\\	経済	
\\	けいざい			
\\	大学で経済を勉強しました。	
\\	経済 
\\	だいがく で けいざい を べんきょう しました			
\\	導入	
\\	どうにゅう			
\\	会社で新しいシステムを導入した。	
\\	導入 
\\	かいしゃ で あたらしい しすてむ を どうにゅう した			
\\	生活	
\\	せいかつ			
\\	日本での生活は楽しいです。	
\\	生活 
\\	にほん で の せいかつ は たのしい です			
\\	進める	
\\	すすめる			
\\	早く授業を進めましょう。	
\\	進める 
\\	はやく じゅぎょう を すすめましょう			
\\	関係	
\\	かんけい			
\\	彼はその事件に関係がない。	
\\	関係 
\\	かれ は その じけん に かんけい が ない			
\\	続ける	
\\	つづける			
\\	仕事を続けてください。	
\\	続ける 
\\	しごと を つづけて ください			
\\	ロボット	
\\	彼はロボットの研究をしています。	
\\	ロボット 
\\	かれ は ロボット の けんきゅう を しています 。			
\\	ほとんど	
\\	お金がほとんどありません。	
\\	ほとんど 
\\	おかね が ほとんど ありません			
\\	国内	
\\	こくない			
\\	この携帯電話が使えるのは国内だけです。	
\\	国内 
\\	この けいたい でんわ が つかえる の は こくない だけ です			
\\	分野	
\\	ぶんや			
\\	音楽は彼の得意な分野です。	
\\	分野 
\\	おんがく は かれ の とくい な ぶんや です			
\\	会社	
\\	かいしゃ			
\\	彼は小さな会社に勤めています。	
\\	会社 
\\	かれ は ちいさ な かいしゃ に つとめて います			
\\	関する	
\\	かんする			
\\	その問題に関する記事を読みました。	
\\	関する 
\\	その もんだい に かんする きじ を よみました			
\\	状態	
\\	じょうたい			
\\	ここは道の状態がとても悪いです。	
\\	状態 
\\	ここ は みち の じょうたい が とても わるい です			
\\	大統領	
\\	だいとうりょう			
\\	フランスの大統領は誰ですか。	
\\	大統領 
\\	ふらんす の だいとうりょう は だれ です か			
\\	対象	
\\	たいしょう			
\\	このアンケートは大学生が対象です。	
\\	対象 
\\	この あんけーと は だいがくせい が たいしょう です			
\\	処理	
\\	しょり			
\\	事務的な処理に1週間かかります。	
\\	処理 
\\	じむてき な しょり に いっしゅうかん かかります			
\\	女性	
\\	じょせい			
\\	そのパーティーに女性は何人来ますか。	
\\	女性 
\\	その ぱーてぃー に じょせい は なんにん きます か			
\\	家	
\\	いえ			
\\	ここが私の家です。	
\\	家 
\\	ここ が わたし の いえ です			
\\	パソコン	
\\	彼はパソコンを2台持っています。	
\\	パソコン 
\\	かれ は ぱそこん を にだい もって います			
\\	解決	
\\	かいけつ			
\\	トラブルがやっと解決した。	
\\	解決 
\\	とらぶる が やっと かいけつ した			
\\	実現	
\\	じつげん			
\\	夢を実現するには努力が必要です。	
\\	必要 
\\	実現 
\\	ゆめ を じつげん する に は どりょく が ひつよう です			
\\	多く	
\\	おおく			
\\	毎年多くの人が海外へ旅行する。	
\\	多く 
\\	まいとし おおく の ひと が かいがい へ りょこう する			
\\	動き	
\\	うごき			
\\	話	
\\	はなし			
\\	あなたの話は面白いね。	
\\	話 
\\	あなた の はなし は おもしろい ね			
\\	価格	
\\	かかく			
\\	ガソリンの価格がどんどん上がっている。	
\\	価格 
\\	がそりん の かかく が どんどん あがって いる			
\\	上がる	
\\	あがる			
\\	私たちは2階に上がった。	
\\	上がる 
\\	わたしたち は にかい に あがった			
\\	もう	
\\	コーヒーをもう一杯ください。	
\\	もう 
\\	こーひー を もう いっぱい ください			
\\	集める	
\\	あつめる			
\\	弟は切手を集めています。	
\\	集める 
\\	おとうと は きって を あつめて います			
\\	声	
\\	こえ			
\\	彼は大きな声で話した。	
\\	声 
\\	かれ は おおき な こえ で はなした			
\\	初めて	
\\	はじめて			
\\	東京に来るのは初めてです。	
\\	初めて 
\\	とうきょう に くる の は はじめて です 。			
\\	明らか	
\\	あきらか			
\\	明らかに彼が悪い。	
\\	明らか 
\\	あきらか に かれ が わるい			
\\	一部	
\\	いちぶ			
\\	計画を一部変更しましょう。	
\\	計画 
\\	一部 
\\	けいかく を いちぶ へんこう しましょう			
\\	調査	
\\	ちょうさ			
\\	私たちがその問題を調査しています。	
\\	調査 
\\	わたしたち が その もんだい を ちょうさ して います			
\\	党	
\\	とう			
\\	変わる	
\\	かわる			
\\	信号が青に変わりました。	
\\	変わる 
\\	しんごう が あお に かわりました。			
\\	社長	
\\	しゃちょう			
\\	おばは小さな会社の社長です。	
\\	社長 
\\	おば は ちいさ な かいしゃ の しゃちょう です			
\\	全国	
\\	ぜんこく			
\\	次は全国のお天気です。	
\\	全国 
\\	つぎ は ぜんこく の おてんき です			
\\	地域	
\\	ちいき			
\\	この地域は雨が多い。	
\\	地域 
\\	この ちいき は あめ が おおい			
\\	重要	
\\	じゅうよう			
\\	これは重要な書類です。	
\\	重要 
\\	これ は じゅうよう な しょるい です			
\\	姿	
\\	すがた			
\\	遠くに彼女の姿が見えた。	
\\	姿 
\\	とおく に かのじょ の すがた が みえた			
\\	先ず	
\\	まず			
\\	帰ったら先ず手を洗いましょう。	
\\	先ず 
\\	かえったら まず て を あらいましょう 。			
\\	昭和	
\\	しょうわ			
\\	私の両親は昭和生まれです。	
\\	昭和 
\\	わたし の りょうしん は しょうわうまれ です			
\\	社会	
\\	しゃかい			
\\	これは大きな社会問題になっている。	
\\	社会 
\\	これ は おおき な しゃかい もんだい に なって いる			
\\	プログラム	
\\	受付でプログラムを受け取った。	
\\	プログラム 
\\	うけつけ で ぷろぐらむ を うけとった			
\\	期待	
\\	きたい			
\\	みんな私たちに期待しています。	
\\	期待 
\\	みんな わたしたち に きたい して います			
\\	力	
\\	ちから			
\\	今の首相はあまり力がない。	
\\	首相 
\\	力 
\\	いま の しゅしょう は あまり ちから が ない			
\\	お相撲さんは、みんな力持ちだ。	
\\	力 
\\	おすもうさん は みんな ちからもち だ			
\\	除く	
\\	のぞく			
\\	部長を除く全員が土曜日も働いた。	
\\	除く 
\\	ぶちょう を のぞく ぜんいん が どようび も はたらいた			
\\	一般	
\\	いっぱん			
\\	一般の方はこちらの席へどうぞ。	
\\	一般 
\\	いっぱん の かた は こちら の せき へ どうぞ			
\\	今回	
\\	こんかい			
\\	まあ今回は許してあげよう。	
\\	今回 
\\	まあ こんかい は ゆるして あげよう			
\\	普及	
\\	ふきゅう			
\\	ゴミのリサイクルが普及している。	
\\	普及 
\\	ごみ の りさいくる が ふきゅう して いる			
\\	表現	
\\	ひょうげん			
\\	彼は歌で自分の気持ちを表現した。	
\\	表現 
\\	かれ は うた で じぶん の きもち を ひょうげん した			
\\	作業	
\\	さぎょう			
\\	作業するにはもっと広いスペースが必要だ。	
\\	作業 
\\	さぎょう する に は もっと ひろい すぺーす が ひつよう だ			
\\	予定	
\\	よてい			
\\	今日の予定を教えてください。	
\\	予定 
\\	きょう の よてい を おしえて ください			
\\	工場	
\\	こうじょう			
\\	彼は食品工場で働いています。	
\\	工場 
\\	かれ は しょくひん こうじょう で はたらいて います			
\\	よく	
\\	よく答えが分かりましたね。	
\\	よく 
\\	よく こたえ が わかりました ね			
\\	生産	
\\	せいさん			
\\	ボルドーはワインの生産で有名だ。	
\\	生産 
\\	ぼるどー は わいん の せいさん で ゆうめい だ			
\\	政権	
\\	せいけん			
\\	政権が交代した。	
\\	政権 
\\	せいけん が こうたい した			
\\	技術	
\\	ぎじゅつ			
\\	彼は非常に高い技術を持っている。	
\\	技術 
\\	かれ は ひじょう に たかい ぎじゅつ を もって いる			
\\	まま	
\\	電気がついたままですよ。	
\\	まま 
\\	でんき が ついた まま です よ			
\\	完成	
\\	かんせい			
\\	新しいホームページが完成した。	
\\	完成 
\\	あたらしい ほーむぺーじ が かんせい した			
\\	変化	
\\	へんか			
\\	今年は変化の多い年でした。	
\\	変化 
\\	ことし は へんか の おおい とし でした			
\\	存在	
\\	そんざい			
\\	宇宙人は存在すると思いますか。	
\\	存在 
\\	うちゅうじん は そんざい する と おもいます か			
\\	参加	
\\	さんか			
\\	明日は市民マラソンに参加します。	
\\	参加 
\\	あす は しみん まらそん に さんか します			
\\	テレビ	
\\	土地	
\\	とち			
\\	ここは父の土地です。	
\\	土地 
\\	ここ は ちち の とち です			
\\	減る	
\\	へる			
\\	体重がかなり減りました。	
\\	減る 
\\	たいじゅう が かなり へりました			
\\	検討	
\\	けんとう			
\\	今日中にこの問題を検討してください。	
\\	検討 
\\	きょう じゅう に この もんだい を けんとう して ください			
\\	政治	
\\	せいじ			
\\	私は政治に関心がある。	
\\	政治 
\\	わたし は せいじ に かんしん が ある			
\\	実施	
\\	じっし			
\\	現在、スペシャルキャンペーンを実施中です。	
\\	実施 
\\	げんざい すぺしゃるきゃんぺーん を じっしちゅう です			
\\	開始	
\\	かいし			
\\	運動会は9時開始です。	
\\	開始 
\\	うんどうかい は くじ かいし です			
\\	備える	
\\	そなえる			
\\	災害に備えて大量の水を買い込んだ。	
\\	備える 
\\	さいがい に そなえて たいりょう の みず を かいこんだ			
\\	輸入	
\\	ゆにゅう			
\\	これはイタリアから輸入した服です。	
\\	輸入 
\\	これ は いたりあ から ゆにゅう した ふく です			
\\	最大	
\\	さいだい			
\\	これは世界最大の船です。	
\\	世界 
\\	最大 
\\	これ は せかい さいだい の ふね です			
\\	競争	
\\	きょうそう			
\\	2社は互いに競争している。	
\\	競争 
\\	にしゃ は たがい に きょうそう して いる			
\\	部分	
\\	ぶぶん			
\\	この部分は問題ないです。	
\\	問題 
\\	部分 
\\	この ぶぶん は もんだい ない です			
\\	消える	
\\	きえる			
\\	突然、電気が消えた。	
\\	消える 
\\	とつぜん でんき が きえた			
\\	家族	
\\	かぞく			
\\	うちは五人家族です。	
\\	家族 
\\	うち は ごにん かぞく です			
\\	内容	
\\	ないよう			
\\	この本の内容を説明してください。	
\\	内容 
\\	この ほん の ないよう を せつめい して ください			
\\	比べる	
\\	くらべる			
\\	今月と先月の売上を比べた。	
\\	比べる 
\\	こんげつ と せんげつ の うりあげ を くらべた			
\\	生まれる	
\\	うまれる			
\\	姉夫婦に男の子が生まれました。	
\\	生まれる 
\\	あね ふうふ に おとこのこ が うまれました			
\\	与える	
\\	あたえる			
\\	そのニュースは彼に大きなショックを与えた。	
\\	与える 
\\	その にゅーす は かれ に おおき な しょっく を あたえた			
\\	選挙	
\\	せんきょ			
\\	彼は選挙に出るつもりだ。	
\\	選挙 
\\	かれ は せんきょ に でる つもり だ			
\\	対立	
\\	たいりつ			
\\	その2社は対立しています。	
\\	対立 
\\	その にしゃ は たいりつ して います			
\\	理由	
\\	りゆう			
\\	遅れた理由を教えてください。	
\\	理由 
\\	おくれた りゆう を おしえて ください			
\\	構成	
\\	こうせい			
\\	システムの構成を変えてみました。	
\\	構成 
\\	しすてむ の こうせい を かえて みました			
\\	管理	
\\	かんり			
\\	彼女が私のスケジュールを管理している。	
\\	管理 
\\	かのじょ が わたし の すけじゅーる を かんり して いる			
\\	増加	
\\	ぞうか			
\\	島の人口は年々増加しています。	
\\	増加 
\\	しま の じんこう は ねんねん ぞうか して います			
\\	改革	
\\	かいかく			
\\	彼は行政を改革したいと思っている。	
\\	改革 
\\	かれ は ぎょうせい を かいかく したい と おもって いる			
\\	会談	
\\	かいだん			
\\	会談の内容が発表されました。	
\\	発表 
\\	内容 
\\	会談 
\\	かいだん の ないよう が はっぴょう されました			
\\	会長	
\\	かいちょう			
\\	初めに会長が挨拶した。	
\\	会長 
\\	はじめ に かいちょう が あいさつ した			
\\	ソフト	
\\	このソフトで日本語を勉強することができます。	
\\	ソフト 
\\	この そふと で にほんご を べんきょう する こと が できます			
\\	提供	
\\	ていきょう			
\\	彼がパーティー会場を提供してくれました。	
\\	提供 
\\	かれ が ぱーてぃー かいじょう を ていきょう して くれました			
\\	ただ	
\\	この温泉はただです。	
\\	ただ 
\\	この おんせん は ただ です			
\\	これら	
\\	今日はこれらの問題について話し合います。	
\\	これら 
\\	きょう は これら の もんだい に ついて はなしあいます			
\\	広がる	
\\	ひろがる			
\\	留学してから私の世界が広がった。	
\\	広がる 
\\	りゅうがく して から わたし の せかい が ひろがった			
\\	特徴	
\\	とくちょう			
\\	この曲線がこの車の特徴です。	
\\	特徴 
\\	この きょくせん が この くるま の とくちょう です			
\\	調べる	
\\	しらべる			
\\	この単語の意味を辞書で調べましょう。	
\\	調べる 
\\	この たんご の いみ を じしょ で しらべましょう			
\\	強調	
\\	きょうちょう			
\\	彼は良いところだけを強調した。	
\\	強調 
\\	かれ は よい ところ だけ を きょうちょう した			
\\	制度	
\\	せいど			
\\	来年から新しい制度が始まります。	
\\	制度 
\\	らいねん から あたらしい せいど が はじまります			
\\	方針	
\\	ほうしん			
\\	今後の方針が決まった。	
\\	方針 
\\	こんご の ほうしん が きまった			
\\	従来	
\\	じゅうらい			
\\	このプリンターは従来のものより速い。	
\\	従来 
\\	この ぷりんたー は じゅうらい の もの より はやい			
\\	いずれ	
\\	いずれまたお会いしましょう。	
\\	いずれ 
\\	いずれ また おあい しましょう			
\\	事故	
\\	じこ			
\\	彼は事故で怪我をしました。	
\\	事故 
\\	かれ は じこ で けが を しました			
\\	電話	
\\	でんわ			
\\	あとで電話します。	
\\	電話 
\\	あとで でんわ します			
\\	外国	
\\	がいこく			
\\	母はまだ外国に行ったことがありません。	
\\	外国 
\\	はは は まだ がいこく に いった こと が ありません			
\\	目的	
\\	もくてき			
\\	彼が来た目的が分かりません。	
\\	目的 
\\	かれ が きた もくてき が わかりません			
\\	自由	
\\	じゆう			
\\	今日は自由な時間が多い。	
\\	自由 
\\	きょう は じゆう な じかん が おおい			
\\	資金	
\\	しきん			
\\	私たちは今、結婚資金を貯めています。	
\\	資金 
\\	わたしたち は いま けっこん しきん を ためて います			
\\	銀行	
\\	ぎんこう			
\\	銀行は3時まで開いています。	
\\	銀行 
\\	ぎんこう は さんじ まで あいて います			
\\	事実	
\\	じじつ			
\\	それは全て事実ですか。	
\\	事実 
\\	それ は すべて じじつ です か			
\\	決定	
\\	けってい			
\\	会議で重要な決定がありました。	
\\	決定 
\\	かいぎ で じゅうよう な けってい が ありました			
\\	方向	
\\	ほうこう			
\\	あの人たちは皆、同じ方向を見ている。	
\\	方向 
\\	あの ひとたち は みな おなじ ほうこう を みて いる			
\\	住宅	
\\	じゅうたく			
\\	ここは静かな住宅地だ。	
\\	住宅 
\\	ここ は しずかな じゅうたくち だ			
\\	代表	
\\	だいひょう			
\\	党の代表が質問に答えました。	
\\	党 
\\	代表 
\\	とう の だいひょう が しつもん に こたえました			
\\	彼がクラスの代表だ。	
\\	代表 
\\	かれ が くらす の だいひょう だ			
\\	激しい	
\\	はげしい			
\\	激しい雨が降っています。	
\\	激しい 
\\	はげしい あめ が ふって います			
\\	株	
\\	かぶ			
\\	最近株を始めました。	
\\	最近 
\\	株 
\\	さいきん かぶ を はじめました			
\\	交渉	
\\	こうしょう			
\\	今、値段を交渉しています。	
\\	交渉 
\\	いま ねだん を こうしょう して います			
\\	結果	
\\	けっか			
\\	試合の結果を早く知りたい。	
\\	結果 
\\	しあい の けっか を はやく しりたい			
\\	全体	
\\	ぜんたい			
\\	全体の80パーセントが完成しました。	
\\	完成 
\\	全体 
\\	ぜんたい の はちじゅっぱーせんと が かんせい しました			
\\	後	
\\	のち			
\\	後に彼は総理大臣になりました。	
\\	後 
\\	のち に かれ は そうり だいじん に なりました			
\\	命令	
\\	めいれい			
\\	彼女は命令に従わなかった。	
\\	命令 
\\	かのじょ は めいれい に したがわなかった			
\\	十分	
\\	じゅうぶん			
\\	お金はまだ十分あります。	
\\	十分 
\\	おかね は まだ じゅうぶん あります			
\\	あまり	
\\	あまり			
\\	私はテレビをあまり見ません。	
\\	テレビ 
\\	あまり 
\\	わたし は てれび を あまり みません			
\\	このビールはあまり美味しくありません。	
\\	あまり 
\\	この びーる は あまり おいしく ありません			
\\	経営	
\\	けいえい			
\\	我が社の経営はうまくいっています。	
\\	経営 
\\	わがしゃ の けいえい は うまく いって います			
\\	原因	
\\	げんいん			
\\	この事故の原因は何ですか。	
\\	事故 
\\	原因 
\\	このじこ の げんいん は なん です か			
\\	写真	
\\	しゃしん			
\\	写真は良い思い出になります。	
\\	写真 
\\	しゃしん は いい おもいで に なります			
\\	立場	
\\	たちば			
\\	彼女は自分の立場を分かっていない。	
\\	立場 
\\	かのじょ は じぶん の たちば を わかって いない			
\\	対策	
\\	たいさく			
\\	一緒に対策を考えましょう。	
\\	対策 
\\	いっしょ に たいさく を かんがえましょう			
\\	繰り返す	
\\	くりかえす			
\\	彼女は同じ間違いを繰り返した。	
\\	繰り返す 
\\	かのじょ は おなじ まちがい を くりかえした			
\\	評価	
\\	ひょうか			
\\	最近、彼の評価が上がった。	
\\	評価 
\\	さいきん かれ の ひょうか が あがった			
\\	なお	
\\	なお、雨の場合は中止です。	
\\	なお 
\\	なお あめ の ばあい は ちゅうし です			
\\	計算	
\\	けいさん			
\\	その計算は間違っている。	
\\	計算 
\\	その けいさん は まちがって いる			
\\	投資	
\\	とうし			
\\	私は4つの会社に投資しています。	
\\	投資 
\\	わたし は よっつ の かいしゃ に とうし して います			
\\	可能	
\\	かのう			
\\	20キロのダイエットは可能だと思いますか。	
\\	可能 
\\	にじゅっきろ の だいえっと は かのう だ と おもいます か			
\\	加える	
\\	くわえる			
\\	塩、コショウを加えてください。	
\\	加える 
\\	しお こしょう を くわえて ください			
\\	種類	
\\	しゅるい			
\\	バラには色々な種類があります。	
\\	種類 
\\	ばら に は いろいろ な しゅるい が あります			
\\	強化	
\\	きょうか			
\\	国は国語教育を強化しています。	
\\	強化 
\\	くに は こくご きょういく を きょうか して います			
\\	意見	
\\	いけん			
\\	あなたの意見が聞きたいです。	
\\	意見 
\\	あなた の いけん が ききたい です			
\\	伴う	
\\	ともなう			
\\	その仕事は危険を伴う。	
\\	伴う 
\\	その しごと は きけん を ともなう			
\\	新聞	
\\	しんぶん			
\\	今日の新聞、どこに置いた?	
\\	新聞 
\\	きょう の しんぶん どこ に おいた			
\\	文章	
\\	ぶんしょう			
\\	彼は文章がとてもうまい。	
\\	文章 
\\	かれ は ぶんしょう が とても うまい			
\\	目立つ	
\\	めだつ			
\\	彼女の大きな帽子はとても目立つ。	
\\	目立つ 
\\	かのじょ の おおき な ぼうし は とても めだつ			
\\	設計	
\\	せっけい			
\\	兄は船の設計をしています。	
\\	設計 
\\	あに は ふね の せっけい を して います			
\\	構造	
\\	こうぞう			
\\	この建物の構造は複雑です。	
\\	構造 
\\	この たてもの の こうぞう は ふくざつ です			
\\	国会	
\\	こっかい			
\\	国会が再開した。	
\\	国会 
\\	こっかい が さいかい した			
\\	ほぼ	
\\	仕事がほぼ終わりました。	
\\	ほぼ 
\\	しごと が ほぼ おわりました			
\\	相手	
\\	あいて			
\\	まず相手の動きを見ましょう。	
\\	動き 
\\	相手 
\\	まず あいて の うごき を みましょう			
\\	試合の相手は誰ですか。	
\\	相手 
\\	しあい の あいて は だれ です か			
\\	発生	
\\	はっせい			
\\	交差点で事故が発生した。	
\\	発生 
\\	こうさてん で じこ が はっせい した			
\\	採用	
\\	さいよう			
\\	その会社は女性を多く採用している。	
\\	女性 
\\	採用 
\\	その かいしゃ は じょせい を おおく さいよう して いる			
\\	残る	
\\	のこる			
\\	料理がたくさん残りました。	
\\	残る 
\\	りょうり が たくさん のこりました			
\\	段階	
\\	だんかい			
\\	この段階では、決断するのはまだ早い。	
\\	段階 
\\	この だんかい で は けつだん する の は まだ はやい			
\\	報告	
\\	ほうこく			
\\	昨日の会議について報告があります。	
\\	報告 
\\	きのう の かいぎ に ついて ほうこく が あります			
\\	商品	
\\	しょうひん			
\\	この商品はよく売れている。	
\\	よく 
\\	商品 
\\	この しょうひん は よく うれて いる			
\\	都市	
\\	とし			
\\	東京は日本一大きな都市です。	
\\	都市 
\\	とうきょう は にほんいち おおき な とし です			
\\	病院	
\\	びょういん			
\\	病院はどこですか。	
\\	病院 
\\	びょういん は どこ です か			
\\	規模	
\\	きぼ			
\\	この動物園は日本一の規模です。	
\\	規模 
\\	この どうぶつえん は にっぽんいち の きぼ です			
\\	業者	
\\	ぎょうしゃ			
\\	引っ越しを業者に頼んだ。	
\\	業者 
\\	ひっこし を ぎょうしゃ に たのんだ			
\end{CJK}
\end{document}