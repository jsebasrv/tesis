\documentclass[8pt]{extreport} 
\usepackage{hyperref}
\usepackage{CJKutf8}
\begin{document}
\begin{CJK}{UTF8}{min}
\\	それ	
\\	それ
\\	はとってもいい話だ。	
\\	それ 
\\	それ
\\	は とっても いい はなし だ	
\\	見る	
\\	みる	
\\	私は絵を
\\	見る
\\	のが好きです。	
\\	見る 
\\	わたし は え を 
\\	みる
\\	の が すき です	
\\	円	
\\	えん	
\\	そこに大きな
\\	円
\\	を描いて。	
\\	円 
\\	そこ に おおき な 
\\	えん
\\	を かいて	
\\	多い	
\\	おおい	
\\	京都にはお寺が
\\	多い
\\	多い 
\\	きょうと に は おてら が 
\\	おおい
\\	家	
\\	うち	
\\	家
\\	に遊びに来てください。	
\\	家 
\\	うち
\\	に あそび に きて ください	
\\	これ	
\\	これ
\\	をください。	
\\	これ 
\\	これ
\\	を ください	
\\	新しい	
\\	あたらしい	
\\	私	
\\	わたし	
\\	私
\\	が行きましょう。	
\\	私 
\\	わたし
\\	が いきましょう	
\\	仕事	
\\	しごと	
\\	3月は
\\	仕事
\\	が忙しい。	
\\	仕事 
\\	さんがつ は 
\\	しごと
\\	が いそがしい	
\\	始める	
\\	はじめる	
\\	テストを
\\	始めて
\\	ください。	
\\	始める 
\\	てすと を 
\\	はじめて
\\	ください	
\\	なる	
\\	彼は医者に
\\	なりました
\\	なる 
\\	かれ は いしゃ に 
\\	なりました
\\	する	
\\	出る	
\\	でる	
\\	今朝は早く家を
\\	出ました
\\	出る 
\\	けさ は はやく いえ を 
\\	でました
\\	使う	
\\	つかう	
\\	このパソコンを
\\	使って
\\	ください。	
\\	使う 
\\	この ぱそこん を 
\\	つかって
\\	ください	
\\	ところ	
\\	私は友達の
\\	ところ
\\	に泊まった。	
\\	ところ 
\\	わたし は ともだち の 
\\	ところ
\\	に とまった	
\\	思う	
\\	おもう	
\\	私もそう
\\	思います
\\	思う 
\\	わたし も そう 
\\	おもいます
\\	持つ	
\\	もつ	
\\	私は車を
\\	持っています
\\	持つ 
\\	わたし は くるま を 
\\	もって います
\\	時間	
\\	じかん	
\\	今は
\\	時間
\\	がありません。	
\\	時間 
\\	いま は 
\\	じかん
\\	が ありません	
\\	今年	
\\	ことし	
\\	今年
\\	はイタリアに旅行したい。	
\\	今年 
\\	ことし
\\	は いたりあ に りょこう したい	
\\	知る	
\\	しる	
\\	誰もその話を
\\	知らない
\\	知る 
\\	だれ も その はなし を 
\\	しらない
\\	作る	
\\	つくる	
\\	今、朝ご飯を
\\	作っています
\\	作る 
\\	いま あさごはん を 
\\	つくって います
\\	同じ	
\\	おなじ	
\\	彼の日本語のレベルは私と
\\	同じ
\\	位だ。	
\\	同じ 
\\	かれ の にほんご の れべる は わたし と 
\\	おなじ
\\	くらい だ	
\\	今	
\\	いま	
\\	彼は
\\	今
\\	、勉強しています。	
\\	今 
\\	かれ は 
\\	いま
\\	べんきょう して います	
\\	続く	
\\	つづく	
\\	工事は3月まで
\\	続きます
\\	続く 
\\	こうじ は さんがつ まで 
\\	つづきます
\\	国	
\\	くに	
\\	未だ	
\\	まだ	
\\	やる	
\\	一緒に宿題を
\\	やろう
\\	やる 
\\	いっしょ に しゅくだい を 
\\	やろう
\\	言う	
\\	いう	
\\	上司が「一杯、飲もう。」と
\\	言った
\\	言う 
\\	じょうし が いっぱい のもう と 
\\	いった
\\	少ない	
\\	すくない	
\\	今年は雨が
\\	少ない
\\	です。	
\\	少ない 
\\	ことし は あめ が 
\\	すくない 
\\	です 。	
\\	行く	
\\	いく	
\\	決める	
\\	きめる	
\\	帰国することに
\\	決めました
\\	決める 
\\	きこく する こと に 
\\	きめました
\\	高い	
\\	たかい	
\\	これがこの町で一番
\\	高い
\\	ビルです。	
\\	高い 
\\	これ が この まち で いちばん 
\\	たかい
\\	びる です	
\\	考える	
\\	かんがえる	
\\	よく
\\	考えて
\\	ください。	
\\	考える 
\\	よく 
\\	かんがえて
\\	ください	
\\	感じる	
\\	かんじる	
\\	膝に痛みを
\\	感じます
\\	感じる 
\\	ひざ に いたみ を 
\\	かんじます
\\	子供	
\\	こども	
\\	電車で
\\	子供
\\	が騒いでいた。	
\\	子供 
\\	でんしゃ で 
\\	こども
\\	が さわいで いた	
\\	そう	
\\	私も
\\	そう
\\	思います。	
\\	そう 
\\	わたし も 
\\	そう
\\	おもいます	
\\	大学	
\\	だいがく	
\\	もう	
\\	彼は
\\	もう
\\	帰りました。	
\\	もう 
\\	かれ は 
\\	もう 
\\	かえりました 。	
\\	近く	
\\	ちかく	
\\	駅の
\\	近く
\\	で食事をした。	
\\	近く 
\\	えき の 
\\	ちかく
\\	で しょくじ を した	
\\	入れる	
\\	いれる	
\\	彼はかばんに手帳を
\\	入れた
\\	入れる 
\\	かれ は かばん に てちょう を 
\\	いれた
\\	学生	
\\	がくせい	
\\	彼は真面目な
\\	学生
\\	です。	
\\	学生 
\\	かれ は まじめ な 
\\	がくせい
\\	です	
\\	午後	
\\	ごご	
\\	明日の
\\	午後
\\	、お客様が来る。	
\\	午後 
\\	あす の 
\\	ごご
\\	おきゃくさま が くる	
\\	長い	
\\	ながい	
\\	彼女の髪はとても
\\	長い
\\	長い 
\\	かのじょ の かみ は とても 
\\	ながい
\\	本	
\\	ほん	
\\	本
\\	を1冊買いました。	
\\	本 
\\	ほん
\\	を いっさつ かいました	
\\	よく	
\\	彼女は
\\	よく
\\	旅行に行きます。	
\\	よく 
\\	かのじょ は 
\\	よく
\\	りょこう に いきます	
\\	彼女	
\\	かのじょ	
\\	彼女
\\	は
\\	です。	
\\	彼女 
\\	かのじょ
\\	は おーえる です	
\\	聞く	
\\	きく	
\\	彼女はラジオを
\\	聞いています
\\	聞く 
\\	かのじょ は らじお を 
\\	きいて います
\\	どう	
\\	あなたは
\\	どう
\\	思いますか。	
\\	どう 
\\	あなた は 
\\	どう
\\	おもいます か	
\\	生きる	
\\	いきる	
\\	皆一生懸命
\\	生きている
\\	生きる 
\\	みんな いっしょうけんめい 
\\	いきて いる
\\	言葉	
\\	ことば	
\\	顔	
\\	かお	
\\	彼はタオルで
\\	顔
\\	を拭きました。	
\\	顔 
\\	かれ は たおる で 
\\	かお
\\	を ふきました	
\\	どこ	
\\	あの本を
\\	どこ
\\	に置きましたか。	
\\	どこ 
\\	あの ほん を 
\\	どこ
\\	に おきました か	
\\	一つ	
\\	ひとつ	
\\	それを
\\	一つ
\\	ください。	
\\	それ 
\\	一つ 
\\	それ を 
\\	ひとつ
\\	ください	
\\	あげる	
\\	この本、あなたに
\\	あげます
\\	あげる 
\\	この ほん あなた に 
\\	あげます
\\	こう	
\\	こう
\\	小さい字は読めない。	
\\	こう 
\\	こう
\\	ちいさい じ は よめない	
\\	学校	
\\	がっこう	
\\	学校
\\	は8時半に始まります。	
\\	学校 
\\	がっこう
\\	は はちじはん に はじまります	
\\	くれる	
\\	友達が誕生日プレゼントを
\\	くれた
\\	くれる 
\\	ともだち が たんじょうび ぷれぜんと を 
\\	くれた
\\	起きる	
\\	おきる	
\\	私は毎朝6時に
\\	起きます
\\	起きる 
\\	わたし は まいあさ ろくじ に 
\\	おきます
\\	春	
\\	はる	
\\	今年の
\\	春
\\	は暖かいね。	
\\	春 
\\	ことし の 
\\	はる
\\	は あたたかい ね	
\\	午前	
\\	ごぜん	
\\	午前
\\	9時のニュースです。	
\\	午前 
\\	ごぜん
\\	くじ の にゅーす です	
\\	買う	
\\	かう	
\\	郵便局で切手を
\\	買いました
\\	買う 
\\	ゆうびんきょく で きって を 
\\	かいました
\\	別	
\\	べつ	
\\	終わる	
\\	おわる	
\\	宿題は
\\	未だ
\\	終わっていません。	
\\	未だ 
\\	終わる 
\\	しゅくだい は 
\\	まだ
\\	おわって いません	
\\	会議は4時に
\\	終わります
\\	終わる 
\\	かいぎ は よじ に 
\\	おわります
\\	若い	
\\	わかい	
\\	彼はまだ
\\	若い
\\	です。	
\\	若い 
\\	かれ は まだ 
\\	わかい
\\	です	
\\	かなり	
\\	彼は
\\	かなり
\\	英語が上手です。	
\\	かなり 
\\	かれ は 
\\	かなり
\\	えいご が じょうず です	
\\	置く	
\\	おく	
\\	彼はかばんをいすの上に
\\	置きました
\\	置く 
\\	かれ は かばん を いす の うえ に 
\\	おきました
\\	住む	
\\	すむ	
\\	彼は会社の近くに
\\	住んでいる
\\	住む 
\\	かれ は かいしゃ の ちかく に 
\\	すんで いる
\\	戻る	
\\	もどる	
\\	今、会社に
\\	戻ります
\\	戻る 
\\	いま かいしゃ に 
\\	もどります
\\	働く	
\\	はたらく	
\\	後	
\\	あと	
\\	仕事の
\\	後
\\	、映画を見た。	
\\	見る 
\\	仕事 
\\	後 
\\	しごと の 
\\	あと
\\	えいが を みた	
\\	先生	
\\	せんせい	
\\	私は日本語の
\\	先生
\\	になりたいです。	
\\	先生 
\\	わたし は にほんご の 
\\	せんせい
\\	に なりたい です	
\\	立つ	
\\	たつ	
\\	彼はステージに
\\	立った
\\	立つ 
\\	かれ は すてーじ に 
\\	たった
\\	気	
\\	き	
\\	彼は意外に
\\	気
\\	が小さい。	
\\	気 
\\	かれ は いがい に 
\\	き
\\	が ちいさい	
\\	呼ぶ	
\\	よぶ	
\\	ウェイターを
\\	呼びましょう
\\	呼ぶ 
\\	うぇいたー を 
\\	よびましょう
\\	もっと	
\\	もっと
\\	近くに来てください。	
\\	近く 
\\	もっと 
\\	もっと
\\	ちかく に きて ください	
\\	大学
\\	に行ってもっと勉強したいです。	
\\	大学 
\\	もっと 
\\	だいがく
\\	に いって もっと べんきょう したい です	
\\	帰る	
\\	かえる	
\\	家に
\\	帰ろう
\\	家 
\\	帰る 
\\	うち に 
\\	かえろう
\\	気持ち	
\\	きもち	
\\	彼の
\\	気持ち
\\	が分からない。	
\\	気持ち 
\\	かれ の 
\\	きもち
\\	が わからない	
\\	飛ぶ	
\\	とぶ	
\\	鳥が
\\	飛んでいます
\\	飛ぶ 
\\	とり が 
\\	とんで います
\\	車	
\\	くるま	
\\	彼の車は
\\	新しい
\\	新しい 
\\	車 
\\	かれ の くるま は 
\\	あたらしい
\\	弟が
\\	車
\\	を買った。	
\\	車 
\\	おとうと が 
\\	くるま
\\	を かった	
\\	変える	
\\	かえる	
\\	旅行の日程を
\\	変えました
\\	変える 
\\	りょこう の にってい を 
\\	かえました
\\	広い	
\\	ひろい	
\\	彼の家はとても
\\	広い
\\	広い 
\\	かれ の いえ は とても 
\\	ひろい
\\	分かる	
\\	わかる	
\\	この
\\	言葉
\\	の意味が分かりません。	
\\	言葉 
\\	分かる 
\\	この 
\\	ことば
\\	の いみ が わかりません	
\\	数	
\\	かず	
\\	グラスの
\\	数
\\	が足りません。	
\\	数 
\\	ぐらす の 
\\	かず
\\	が たりません	
\\	近い	
\\	ちかい	
\\	そこ	
\\	走る	
\\	はしる	
\\	彼は毎晩3キロ
\\	走っています
\\	走る 
\\	かれ は まいばん さんきろ 
\\	はしって います
\\	売る	
\\	うる	
\\	彼は家を
\\	売った
\\	売る 
\\	かれ は いえ を 
\\	うった
\\	朝	
\\	あさ	
\\	気持ちのいい
\\	朝
\\	です。	
\\	気持ち 
\\	朝 
\\	きもち の いい 
\\	あさ
\\	です	
\\	教える	
\\	おしえる	
\\	彼は数学を
\\	教えています
\\	教える 
\\	かれ は すうがく を 
\\	おしえて います
\\	歩く	
\\	あるく	
\\	駅まで
\\	歩きましょう
\\	歩く 
\\	えき まで 
\\	あるきましょう
\\	直ぐ	
\\	すぐ	
\\	直ぐ
\\	行きます。	
\\	行く 
\\	直ぐ 
\\	すぐ
\\	いきます	
\\	会う	
\\	あう	
\\	また
\\	会いましょう
\\	会う 
\\	また 
\\	あいましょう
\\	部屋	
\\	へや	
\\	私の
\\	部屋
\\	は2階にあります。	
\\	部屋 
\\	わたし の 
\\	へや
\\	は にかい に あります	
\\	起こる	
\\	おこる	
\\	町で大事件が
\\	起こりました
\\	起こる 
\\	まち で だいじけん が 
\\	おこりました
\\	秋	
\\	あき	
\\	難しい	
\\	むずかしい	
\\	この本は
\\	難しい
\\	ですね。	
\\	本 
\\	難しい 
\\	この ほん は 
\\	むずかしい
\\	です ね	
\\	送る	
\\	おくる	
\\	彼の家に荷物を
\\	送りました
\\	送る 
\\	かれ の いえ に にもつ を 
\\	おくりました
\\	死ぬ	
\\	しぬ	
\\	乗る	
\\	のる	
\\	駅からはタクシーに
\\	乗って
\\	ください。	
\\	乗る 
\\	えき からは たくしー に 
\\	のって
\\	ください	
\\	安い	
\\	やすい	
\\	この服はとても
\\	安かった
\\	安い 
\\	この ふく は とても 
\\	やすかった
\\	書く	
\\	かく	
\\	彼に手紙を
\\	書きました
\\	書く 
\\	かれ に てがみ を 
\\	かきました
\\	話す	
\\	はなす	
\\	母と電話で
\\	話しました
\\	話す 
\\	はは と でんわ で 
\\	はなしました
\\	頭	
\\	あたま	
\\	今朝から
\\	頭
\\	が痛い。	
\\	頭 
\\	けさ から 
\\	あたま
\\	が いたい	
\\	体	
\\	からだ	
\\	私は
\\	体
\\	が丈夫だ。	
\\	体 
\\	わたし は 
\\	からだ
\\	が じょうぶ だ	
\\	ここ	
\\	駅はここから
\\	近い
\\	です。	
\\	近い 
\\	ここ 
\\	えき は ここ から 
\\	ちかい
\\	です	
\\	ここ
\\	に本があります。	
\\	ここ 
\\	ここ
\\	に ほん が あります	
\\	待つ	
\\	まつ	
\\	あなたが来るのを
\\	待っています
\\	待つ 
\\	あなた が くる の を 
\\	まって います
\\	低い	
\\	ひくい	
\\	彼は背が
\\	低い
\\	低い 
\\	かれ は せ が 
\\	ひくい
\\	もらう	
\\	彼女からプレゼントを
\\	もらいました
\\	もらう 
\\	かのじょ から ぷれぜんと を 
\\	もらいました
\\	食べる	
\\	たべる	
\\	昨日タイカレーを
\\	食べました
\\	食べる 
\\	きのう たいかれー を 
\\	たべました
\\	早い	
\\	はやい	
\\	まだ学校へ行くには
\\	早い
\\	時間です。	
\\	早い 
\\	まだ がっこう へ いく に は 
\\	はやい
\\	じかん です	
\\	兄	
\\	あに	
\\	兄
\\	は水泳が得意です。	
\\	兄 
\\	あに
\\	は すいえい が とくい です	
\\	名前	
\\	なまえ	
\\	あなたの
\\	名前
\\	を教えてください。	
\\	教える 
\\	名前 
\\	あなた の 
\\	なまえ
\\	を おしえて ください	
\\	夫	
\\	おっと	
\\	私の
\\	夫
\\	はサラリーマンです。	
\\	夫 
\\	わたし の 
\\	おっと
\\	は さらりーまん です	
\\	一	
\\	いち	
\\	一
\\	から始めましょう。	
\\	始める 
\\	一 
\\	いち
\\	から はじめましょう	
\\	結婚	
\\	けっこん	
\\	彼女は
\\	秋
\\	に結婚します。	
\\	する 
\\	秋 
\\	結婚 
\\	かのじょ は 
\\	あき
\\	に けっこん します	
\\	彼女は来月
\\	結婚
\\	します。	
\\	結婚 
\\	かのじょ は らいげつ 
\\	けっこん
\\	します	
\\	親	
\\	おや	
\\	親
\\	の愛は有り難い。	
\\	親 
\\	おや
\\	の あい は ありがたい	
\\	古い	
\\	ふるい	
\\	私は
\\	古い
\\	車が好きです。	
\\	古い 
\\	わたし は 
\\	ふるい
\\	くるま が すき です	
\\	美しい	
\\	うつくしい	
\\	この絵は
\\	美しい
\\	です。	
\\	美しい 
\\	この え は 
\\	うつくしい
\\	です	
\\	足	
\\	あし	
\\	彼は
\\	足
\\	が長い。	
\\	足 
\\	かれ は 
\\	あし
\\	が ながい	
\\	次	
\\	つぎ	
\\	起こす	
\\	おこす	
\\	明日、6時に
\\	起こして
\\	ください。	
\\	起こす 
\\	あす ろくじ に 
\\	おこして
\\	ください	
\\	見せる	
\\	みせる	
\\	別
\\	の本も見せてください。	
\\	別 
\\	見せる 
\\	べつ
\\	の ほん も みせて ください	
\\	その写真を
\\	見せて
\\	ください。	
\\	見せる 
\\	その しゃしん を 
\\	みせて
\\	ください	
\\	娘	
\\	むすめ	
\\	私の
\\	娘
\\	はアメリカにいます。	
\\	娘 
\\	わたし の 
\\	むすめ
\\	は あめりか に います	
\\	とても	
\\	この本は
\\	とても
\\	おもしろい。	
\\	とても 
\\	この ほん は 
\\	とても
\\	おもしろい	
\\	少し	
\\	すこし	
\\	私の
\\	国
\\	について少しお話しましょう。	
\\	国 
\\	少し 
\\	わたし の 
\\	くに
\\	に ついて すこし おはなし しましょう	
\\	少し
\\	疲れました。	
\\	少し 
\\	すこし
\\	つかれました	
\\	質問	
\\	しつもん	
\\	質問の意味は
\\	分かりました
\\	か。	
\\	分かる 
\\	質問 
\\	しつもん の いみ は 
\\	わかりました
\\	か	
\\	質問
\\	のある方はどうぞ。	
\\	質問 
\\	しつもん
\\	の ある かた は どうぞ	
\\	楽しむ	
\\	たのしむ	
\\	色	
\\	いろ	
\\	すてきな
\\	色
\\	のセーターですね。	
\\	色 
\\	すてき な 
\\	いろ
\\	の せーたー です ね	
\\	皆	
\\	みんな	
\\	みんな
\\	にお菓子をあげましょう。	
\\	みんな 
\\	みんな
\\	に おかし を あげましょう	
\\	取る	
\\	とる	
\\	テストでいい点を
\\	取った
\\	取る 
\\	てすと で いい てん を 
\\	とった
\\	勉強	
\\	べんきょう	
\\	私は日本語を
\\	勉強
\\	しています。	
\\	勉強 
\\	わたし は にほんご を 
\\	べんきょう
\\	して います	
\\	出来る	
\\	できる	
\\	彼女はイタリア語が
\\	出来ます
\\	出来る 
\\	かのじょ は いたりあご が 
\\	できます
\\	短い	
\\	みじかい	
\\	彼は足が
\\	短い
\\	短い 
\\	かれ は あし が 
\\	みじかい
\\	落ちる	
\\	おちる	
\\	猿も木から
\\	落ちる
\\	落ちる 
\\	さる も き から 
\\	おちる
\\	息子	
\\	むすこ	
\\	うちの
\\	息子
\\	は大学1年生です。	
\\	息子 
\\	うち の 
\\	むすこ
\\	は だいがく いちねんせい です	
\\	白い	
\\	しろい	
\\	彼は歯が
\\	白い
\\	白い 
\\	かれ は は が 
\\	しろい
\\	誰	
\\	だれ	
\\	誰
\\	と会ってみたいですか。	
\\	誰 
\\	だれ
\\	と あって みたい です か	
\\	いつも	
\\	飛行機	
\\	ひこうき	
\\	息子は
\\	飛行機
\\	のおもちゃが好きです。	
\\	息子 
\\	飛行機 
\\	むすこ は 
\\	ひこうき
\\	の おもちゃ が すき です	
\\	勝つ	
\\	かつ	
\\	今日はヤンキースが
\\	勝った
\\	勝つ 
\\	きょう は やんきーす が 
\\	かった
\\	病気	
\\	びょうき	
\\	祖父が
\\	病気
\\	になった。	
\\	なる 
\\	病気 
\\	そふ が 
\\	びょうき
\\	に なった	
\\	犬が病気で
\\	死にました
\\	死ぬ 
\\	病気 
\\	いぬ が びょうき で 
\\	しにました
\\	冬	
\\	ふゆ	
\\	カナダの
\\	冬
\\	はとても寒いです。	
\\	冬 
\\	かなだ の 
\\	ふゆ
\\	は とても さむい です	
\\	年	
\\	とし	
\\	新しい
\\	年
\\	が始まりました。	
\\	新しい 
\\	年 
\\	あたらしい 
\\	とし
\\	が はじまりました	
\\	重い	
\\	おもい	
\\	このかばんは
\\	重い
\\	です。	
\\	重い 
\\	この かばん は 
\\	おもい
\\	です	
\\	胸	
\\	むね	
\\	胸
\\	に少し痛みがあります。	
\\	胸 
\\	むね
\\	に すこし いたみ が あります	
\\	読む	
\\	よむ	
\\	彼女は雑誌を
\\	読んでいます
\\	読む 
\\	かのじょ は ざっし を 
\\	よんで います
\\	払う	
\\	はらう	
\\	私が
\\	払いましょう
\\	払う 
\\	わたし が 
\\	はらいましょう
\\	軽い	
\\	かるい	
\\	この靴はとても
\\	軽い
\\	軽い 
\\	この くつ は とても 
\\	かるい
\\	見つける	
\\	みつける	
\\	新しい仕事を
\\	見つけました
\\	見つける 
\\	あたらしい しごと を 
\\	みつけました
\\	いつ	
\\	次
\\	はいつ会いましょうか。	
\\	会う 
\\	次 
\\	いつ 
\\	つぎ
\\	は いつ あいましょう か	
\\	彼女は
\\	いつ
\\	来ますか。	
\\	いつ 
\\	かのじょ は 
\\	いつ
\\	きます か	
\\	友達	
\\	ともだち	
\\	私には
\\	友達
\\	がたくさんいます。	
\\	友達 
\\	わたし に は 
\\	ともだち
\\	が たくさん います	
\\	忘れる	
\\	わすれる	
\\	約束を
\\	忘れないで
\\	ください。	
\\	忘れる 
\\	やくそく を 
\\	わすれない で
\\	ください	
\\	酒	
\\	さけ	
\\	彼女は
\\	酒
\\	に強い。	
\\	酒 
\\	かのじょ は 
\\	さけ
\\	に つよい	
\\	どちら	
\\	肉と魚と
\\	どちら
\\	が好きですか。	
\\	どちら 
\\	にく と さかな と 
\\	どちら
\\	が すき です か	
\\	姉	
\\	あね	
\\	姉は銀行で
\\	働いて
\\	います。	
\\	働く 
\\	姉 
\\	あね は ぎんこう で 
\\	はたらいて
\\	います 。	
\\	姉
\\	は大学生です。	
\\	姉 
\\	あね
\\	は だいがくせい です	
\\	覚える	
\\	おぼえる	
\\	妹は平仮名を全部
\\	覚えました
\\	覚える 
\\	いもうと は ひらがな を ぜんぶ 
\\	おぼえました
\\	狭い	
\\	せまい	
\\	私の部屋は
\\	狭い
\\	です。	
\\	部屋 
\\	狭い 
\\	わたし の へや は 
\\	せまい
\\	です	
\\	好き	
\\	すき	
\\	私はワインが
\\	好き
\\	です。	
\\	好き 
\\	わたし は わいん が 
\\	すき
\\	です	
\\	建てる	
\\	たてる	
\\	私たちは来年、家を
\\	建てます
\\	建てる 
\\	わたしたち は らいねん いえ を 
\\	たてます
\\	どれ	
\\	この中で
\\	どれ
\\	が好きですか。	
\\	どれ 
\\	この なか で 
\\	どれ
\\	が すき です か	
\\	赤い	
\\	あかい	
\\	赤い
\\	バラを買いました。	
\\	買う 
\\	赤い 
\\	あかい
\\	ばら を かいました	
\\	今日	
\\	きょう	
\\	今日は一人の時間を
\\	楽しみたい
\\	楽しむ 
\\	今日 
\\	きょう は ひとり の じかん を 
\\	たのしみたい
\\	今日
\\	は仕事がありません。	
\\	今日 
\\	きょう
\\	は しごと が ありません	
\\	着る	
\\	きる	
\\	今日はスーツを
\\	着ています
\\	着る 
\\	きょう は すーつ を 
\\	きて います
\\	笑う	
\\	わらう	
\\	赤ん坊が
\\	笑っています
\\	笑う 
\\	あかんぼう が 
\\	わらって います
\\	一番	
\\	いちばん	
\\	彼はクラスで
\\	一番
\\	背が高い。	
\\	高い 
\\	一番 
\\	かれ は くらす で 
\\	いちばん
\\	せ が たかい	
\\	授業	
\\	じゅぎょう	
\\	今日は日本語の
\\	授業
\\	があります。	
\\	授業 
\\	きょう は にほんご の 
\\	じゅぎょう
\\	が あります	
\\	週	
\\	しゅう	
\\	その次の
\\	週
\\	は空いてますか。	
\\	次 
\\	週 
\\	その つぎ の 
\\	しゅう
\\	は あいてます か	
\\	漢字	
\\	かんじ	
\\	漢字
\\	は中国から来ました。	
\\	漢字 
\\	かんじ
\\	は ちゅうごく から きました	
\\	自転車	
\\	じてんしゃ	
\\	毎日、駅まで
\\	自転車
\\	で行きます。	
\\	自転車 
\\	まいにち えき まで 
\\	じてんしゃ
\\	で いきます	
\\	電車	
\\	でんしゃ	
\\	私は
\\	電車
\\	で通学しています。	
\\	電車 
\\	わたし は 
\\	でんしゃ
\\	で つうがく して います	
\\	探す	
\\	さがす	
\\	彼は郵便局を
\\	探していました
\\	探す 
\\	かれ は ゆうびんきょく を 
\\	さがして いました
\\	紙	
\\	かみ	
\\	紙
\\	と鉛筆はありますか。	
\\	紙 
\\	かみ
\\	と えんぴつ は あります か	
\\	歌う	
\\	うたう	
\\	私たちは大きな声で
\\	歌いました
\\	歌う 
\\	わたしたち は おおき な こえ で 
\\	うたいました
\\	遅い	
\\	おそい	
\\	前の車はとても
\\	遅い
\\	遅い 
\\	まえ の くるま は とても 
\\	おそい
\\	首	
\\	くび	
\\	きりんの
\\	首
\\	は長い。	
\\	首 
\\	きりん の 
\\	くび
\\	は ながい	
\\	速い	
\\	はやい	
\\	彼は走るのが
\\	速い
\\	走る 
\\	速い 
\\	かれ は はしる の が 
\\	はやい
\\	一緒に	
\\	いっしょに	
\\	一緒に
\\	帰りましょう。	
\\	帰る 
\\	一緒に 
\\	いっしょに
\\	かえりましょう	
\\	友達と一緒に宿題を
\\	した
\\	する 
\\	一緒に 
\\	ともだち と いっしょ に しゅくだい を 
\\	した
\\	日記	
\\	にっき	
\\	私は毎日、
\\	日記
\\	を付けています。	
\\	日記 
\\	わたし は まいにち 
\\	にっき
\\	を つけて います	
\\	今月	
\\	こんげつ	
\\	今月
\\	はとても忙しい。	
\\	今月 
\\	こんげつ
\\	は とても いそがしい	
\\	遊ぶ	
\\	あそぶ	
\\	子供たちが公園で
\\	遊んでいる
\\	遊ぶ 
\\	こどもたち が こうえん で 
\\	あそんで いる
\\	遠い	
\\	とおい	
\\	家から学校までは
\\	遠い
\\	です。	
\\	遠い 
\\	いえ から がっこう まで は 
\\	とおい
\\	です	
\\	弱い	
\\	よわい	
\\	その子は体が少し
\\	弱い
\\	体 
\\	弱い 
\\	その こ は からだ が すこし 
\\	よわい
\\	耳	
\\	みみ	
\\	彼女は
\\	耳
\\	がよく聞こえません。	
\\	耳 
\\	かのじょ は 
\\	みみ
\\	が よく きこえません	
\\	座る	
\\	すわる	
\\	そこ
\\	に座ってください。	
\\	そこ 
\\	座る 
\\	そこ
\\	に すわって ください	
\\	私は窓側の席に
\\	座った
\\	座る 
\\	わたし は まどがわ の せき に 
\\	すわった
\\	右	
\\	みぎ	
\\	右
\\	のポケットにハンカチが入っています。	
\\	右 
\\	みぎ
\\	の ぽけっと に はんかち が はいって います	
\\	浴びる	
\\	あびる	
\\	私は朝、シャワーを
\\	浴びます
\\	私 
\\	朝 
\\	浴びる 
\\	わたし は あさ しゃわー を 
\\	あびます
\\	肩	
\\	かた	
\\	肩
\\	が凝りました。	
\\	肩 
\\	かた
\\	が こりました	
\\	飲む	
\\	のむ	
\\	友達とお酒を
\\	飲んでいます
\\	友達 
\\	酒 
\\	飲む 
\\	ともだち と おさけ を 
\\	のんで います
\\	寝る	
\\	ねる	
\\	もう
\\	寝よう
\\	寝る 
\\	もう 
\\	ねよう
\\	消す	
\\	けす	
\\	昼間は電気を
\\	消して
\\	ください。	
\\	消す 
\\	ひるま は でんき を 
\\	けして
\\	ください	
\\	元気	
\\	げんき	
\\	彼女は
\\	いつも
\\	元気だ。	
\\	いつも 
\\	元気 
\\	かのじょ は 
\\	いつも
\\	げんき だ	
\\	おかげさまで
\\	元気
\\	です。	
\\	元気 
\\	おかげさま で 
\\	げんき
\\	です	
\\	負ける	
\\	まける	
\\	私たちのチームはその試合で
\\	負けた
\\	負ける 
\\	わたしたち の ちーむ は その しあい で 
\\	まけた
\\	全部	
\\	ぜんぶ	
\\	それ、
\\	全部
\\	ください。	
\\	全部 
\\	それ 
\\	ぜんぶ
\\	ください	
\\	去年	
\\	きょねん	
\\	私は
\\	去年
\\	フランスへ行った。	
\\	去年 
\\	わたし は 
\\	きょねん
\\	ふらんす へ いった	
\\	引く	
\\	ひく	
\\	このドアは
\\	引いて
\\	ください。	
\\	引く 
\\	この どあ は 
\\	ひいて
\\	ください	
\\	図書館	
\\	としょかん	
\\	日曜日は図書館に
\\	行きます
\\	行く 
\\	図書館 
\\	にちようび は としょかん に 
\\	いきます
\\	図書館
\\	で料理の本を借りた。	
\\	図書館 
\\	としょかん
\\	で りょうり の ほん を かりた	
\\	上げる	
\\	あげる	
\\	彼は荷物をあみだなに
\\	上げた
\\	上げる 
\\	かれ は にもつ を あみだな に 
\\	あげた
\\	緑	
\\	みどり	
\\	この町には
\\	緑
\\	がたくさんあります。	
\\	緑 
\\	この まち に は 
\\	みどり
\\	が たくさん あります	
\\	腕	
\\	うで	
\\	彼の
\\	腕
\\	は太い。	
\\	腕 
\\	かれ の 
\\	うで
\\	は ふとい	
\\	ドア	
\\	ドア
\\	を開けてください。	
\\	ドア 
\\	どあ
\\	を あけて ください	
\end{CJK}
\end{document}