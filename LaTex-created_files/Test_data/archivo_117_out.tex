\documentclass[8pt]{extreport} 
\usepackage{hyperref}
\usepackage{CJKutf8}
\begin{document}
\begin{CJK}{UTF8}{min}
\\	十二時半です。	十二時半
\\	。	です	
\\	十二 時半[じはん]です。	
\\	学生です。	学生
\\	。	です	
\\	学生[がくせい]です。	
\\	日本語です。	日本語
\\	。	です	
\\	日本語[にほんご]です。	
\\	専攻は日本語です。	専攻
\\	日本語です。	は	
\\	専攻[せんこう]は 日本語[にほんご]です。	
\\	私はスー・キムです。	
\\	スー・キムです。	私は	
\\	私[わたし]はスー・キムです。	
\\	山下さんは先生です。	山下さんは
\\	。	先生です	
\\	山下[やました]さんは 先生[せんせい]です。	
\\	メアリーさんはアメリカ人です。	メアリさん
\\	です。	はアメリカ人	
\\	メアリーさんはアメリカ 人[じん]です。	
\\	留学生ですか。	留学生です
\\	。	か	
\\	留学生[りゅうがくせい]ですか。	
\\	専攻は何ですか。	専攻は
\\	ですか。	何	
\\	なん, 
\\	です. 
\\	なに 
\\	なんじん 
\\	専攻[せんこう]は 何[なん]ですか。	
\\	(専攻は)英語です。	(専攻は)
\\	。	英語です	
\\	専攻[せんこう]は) 英語[えいご]です。	
\\	今何時ですか。	今
\\	ですか。	何時	
\\	なん, 
\\	です. 
\\	なに 
\\	なんじん 
\\	今[いま] 何[なん] 時[じ]ですか。	
\\	(今)九時です。	(今)
\\	です。	くじ	(かな)9 
\\	今[いま])九 時[じ]です。	
\\	メアリーさんは何歳ですか。	メアリーさんは
\\	ですか。	何歳	
\\	なん, 
\\	です. 
\\	なに 
\\	なんじん 
\\	メアリーさんは 何[なん] 歳[さい]ですか。	
\\	十九歳です。	
\\	です。	十九歳	
\\	十九 歳[さい]です。	
\\	何年生ですか。	
\\	ですか。	何年生	
\\	なん, 
\\	です. 
\\	なに 
\\	なんじん 
\\	何[なん] 年生[ねんせい]ですか。	
\\	二年生です。	
\\	です。	二年生	
\\	二 年生[ねんせい]です。	
\\	電話番号は何ですか。	電話番号は
\\	。	何ですか	
\\	なん, 
\\	です. 
\\	なに 
\\	なんじん 
\\	電話[でんわ] 番号[ばんごう]は 何[なに]ですか。	
\\	186の7343です。	
\\	7343です。	の	
\\	186の7343です。	
\\	武さんの電話番号	
\\	電話番号
\\	武さんの	
\\	の, 
\\	武[たけし]さんの 電話[でんわ] 番号[ばんごう]	
\\	大学の先生	
\\	先生
\\	大学の	
\\	の, 
\\	大学[だいがく]の 先生[せんせい]	
\\	日本語の学生	
\\	日本語
\\	の学生	
\\	の, 
\\	日本語[にほんご]の 学生[がくせい]	
\\	日本の大学	
\\	大学
\\	日本の	
\\	の, 
\\	日本[にっぽん]の 大学[だいがく]	
\\	武さんのお母さんは高校の先生です。	武さん
\\	先生です。	のお母さんは高校の	
\\	の, 
\\	武[たけ]さんのお 母[かあ]さんは 高校[こうこう]の 先生[せんせい]です。	
\\	先生:スミスさん?学生:はい。	先生:スミスさん?学生:
\\	。	はい	
\\	先生[せんせい]:スミスさん? 学生[がくせい]:はい。	
\\	私の電話番号は37-8667です。	私の電話番号
\\	37-8667です。	は	
\\	私[わたし]の 電話[でんわ] 番号[ばんごう]は378667です。	
\\	一分	
\\	いっぷん	(かな)
\\	一 分[ふん]	
\\	一歳	
\\	いっさい	(かな)
\\	一 歳[さい]	
\\	二い	
\\	にい	(かな)
\\	二い	
\\	三分	
\\	さんぷん	(かな)
\\	三 分[ふん]	
\\	四年生	
\\	年生	よ	(かな)
\\	四 年生[ねんせい]	
\\	四時	
\\	時	よ	(かな)
\\	四 時[じ]	
\\	四月	
\\	しがつ	(かな)
\\	四月[しがつ]	
\\	四分	
\\	よんぷん	(かな)
\\	四 分[ふん]	
\\	五う	
\\	ごう	(かな)
\\	五う	
\\	六分	
\\	ろっぷん	(かな)
\\	六 分[ふん]	
\\	七時	
\\	しちじ	(かな)7 
\\	七 時[じ]	
\\	八分	
\\	はっぷん	(かな)8 
\\	八 分[ふん]	
\\	八歳	
\\	はっさい	(かな)
\\	八 歳[さい]	
\\	十分	
\\	じゅっぷん/じっぷん	(かな)
\\	十 分[ふん]	
\\	いいえ	
\\	の 
\\	の 
\\	これはいくらですか。	
\\	はいくらですか。	これ	
\\	これはいくらですか。	
\\	それは三千円です。	
\\	は三千円です。	それ	
\\	それは三 千[せん] 円[えん]です。	
\\	あれは私のぺんです。	
\\	は私のぺんです。	あれ	
\\	あれは 私[わたし]のぺんです。	
\\	これは私のぺんです。	これ
\\	です。	は私のぺん	
\\	これは 私[わたし]のぺんです。	
\\	それは私のぺんです。	
\\	私のぺんです。	それは	
\\	それは 私[わたし]のぺんです。	
\\	どれですか。	
\\	ですか。	どれ	
\\	どれですか。	
\\	どれがあなたのぺんですか。	
\\	あなたのぺんですか。	どれが	
\\	どれ 
\\	何 
\\	は	どれがあなたのぺんですか。	
\\	この時計はいくらですか。	
\\	はいくらですか。	この時計	
\\	この 時計[とけい]はいくらですか。	
\\	その時計は三千円です。	
\\	三千円です。	その時計は	
\\	その 時計[とけい]は三 千[せん] 円[えん]です。	
\\	あの時計は三千五百円です。	
\\	時計は三千五百円です。	あの	
\\	あの 時計[とけい]は三 千[せん]五 百[ひゃく] 円[えん]です。	
\\	どの時計が三千五百円ですか。	
\\	三千五百円ですか。	どの時計が	
\\	どの 
\\	は	どの 時計[とけい]が三 千[せん]五 百[ひゃく] 円[えん]ですか。	
\\	すみません。郵便局はどこですか。	すみません。郵便局
\\	ですか。	はどこ	
\\	すみません。 郵便[ゆうびん] 局[きょく]はどこですか。	
\\	(郵便局は)あそこです。	(郵便局は)
\\	です。	あそこ	
\\	郵便[ゆうびん] 局[きょく]は)あそこです。	
\\	これはだれのかばんですか。	これは
\\	のかばんですか。	だれ	
\\	これはだれのかばんですか。	
\\	それはスーさんのかばんです。	それは
\\	です。	スーさんのかばん	
\\	それはスーさんのかばんです。	
\\	武さんは日本人です。	武さんは
\\	です。	日本人	
\\	武[たけ]さんは 日本人[にっぽんじん]です。	
\\	美智子さんも日本人です。	美智子さん
\\	日本人です。	も	
\\	美智子[みちこ]さんも 日本人[にっぽんじん]です。	
\\	山田さんは学生じゃないです。	山田さんは学生
\\	。	じゃないです	
\\	山田[やまだ]さんは 学生[がくせい]じゃないです。	
\\	山田さんは学生じゃありません。	山田さんは
\\	。	学生じゃありません	
\\	山田[やまだ]さんは 学生[がくせい]じゃありません。	
\\	山田さんは学生ではありません。	山田さんは学生
\\	。	ではありません	
\\	山田[やまだ]さんは 学生[がくせい]ではありません。	
\\	リーさんの専攻は文学ですね。	リーさんの専攻は
\\	。	文学ですね	
\\	リーさんの 専攻[せんこう]は 文学[ぶんがく]ですね。	
\\	これは肉じゃないですね。	これは
\\	。	肉じゃないですね	
\\	これは 肉[にく]じゃないですね。	
\\	とんかつは魚じゃないですよ。	とんかつは
\\	。	魚じゃないですよ	
\\	とんかつは 魚[さかな]じゃないですよ。	
\\	スミスさんはイギリスじんですよ。	スミスさんは
\\	。	イギリスじんですよ	
\\	スミスさんはイギリスじんですよ。	
\\	じゃあ、その時計を下さい。	じゃあ、その時計
\\	。	を下さい	
\\	じゃあ、その 時計[とけい]を 下[くだ]さい。	
\\	じゃあ、これをお願いします。	じゃあ、これ
\\	。	をお願いします	
\\	じゃあ、これをお 願[ねが]いします。	
\\	お名前をどうぞ。	お名前
\\	。	をどうぞ	
\\	お 
\\	お名前. どうぞ 
\\	お 名前[なまえ]をどうぞ。	
\\	三百	
\\	さんびゃく	(かな)300	
\\	三 百[びゃく]	
\\	六百	
\\	ろっぴゃく	(かな)600	
\\	六 百[ぴゃく]	
\\	八百	
\\	はっぴゃく	(かな)800	
\\	八 百[ぴゃく]	
\\	三千	
\\	さんぜん	(かな)3000	
\\	三 千[ぜん]	
\\	八千	
\\	はっせん	(かな)8000	
\\	八 千[せん]	
\\	晩御飯、食べた?	晩御飯、
\\	食べた	
\\	晩[ばん] 御飯[ごはん]、 食[た]べた?	
\\	スーさんは、高校の時めがねをかけていたと言っていました。	スーさんは、高校の時めがねを
\\	と言っていました。	かけていた	
\\	いる 
\\	スーさんは、 高校[こうこう]の 時[とき]めがねをかけていたと 言[い]っていました。	
\\	(私は)トムさんがやったと思います。	(私は)トムさんが
\\	と思います。	やった	
\\	私[わたし]は)トムさんがやったと 思[おも]います。	
\\	あそこで本を読んでいる学生はみちこさんです。	
\\	学生はみちこさんです。	あそこで本を読んでいる	
\\	あそこで 本[ほん]を 読[よ]んでいる 学生[がくせい]はみちこさんです。	
\\	髪が長い人	
\\	人	髪が長い	
\\	髪[かみ]が 長[なが]い 人[ひと]	
\\	めがねをかけている人	
\\	人	めがねをかけている	
\\	めがねをかけている 人[ひと]	
\\	猫が好きな人	
\\	人	猫が好きな	
\\	猫[ねこ]が 好[す]きな 人[ひと]	
\\	あそこで写真を撮っている人(は誰ですか。)	
\\	人(誰ですか。)	あそこで写真を撮っている	
\\	あそこで 写真[しゃしん]を 撮[と]っている 人[ひと](は 誰[だれ]ですか。)	
\\	たばこを吸わない人(が好きです。)	
\\	人(が好きです。)	たばこを吸わない	
\\	たばこを 吸[す]わない 人[ひと](が 好[す]きです。)	
\\	私はまだ宿題をしていません。	私は
\\	宿題をして
\\	。	まだ…いません	
\\	私[わたし]はまだ 宿題[しゅくだい]をしていません。	
\\	スーさんはまだ起きていません。	スーさんは
\\	おきて
\\	。	まだ…いません	
\\	スーさんはまだ 起[お]きていません。	
\\	私はまだ昼ご飯を食べていません。	私は
\\	昼ご飯を食べて
\\	。	まだ…いません	
\\	私[わたし]はまだ 昼[ひる]ご 飯[はん]を 食[た]べていません。	
\\	明日試験があるから、私は今晩勉強します。	明日試験がある
\\	、私は今晩勉強します。	から	
\\	から 
\\	明日[あした] 試験[しけん]があるから、 私[わたし]は 今晩[こんばん] 勉強[べんきょう]します。	
\\	寒かったから、出掛けませんでした。	寒かった
\\	、出掛けませんでした。	から	
\\	から 
\\	寒[さむ]かったから、 出掛[でか]けませんでした。	
\\	中国のほうが、日本より、大きいです。	中国
\\	、日本
\\	、大きいです。	のほうが、より	
\\	のほうが 
\\	より 
\\	中国[ちゅうごく]のほうが、 日本[にっぽん]より、 大[おお]きいです。	
\\	バスと電車とどっちのほうがやすいですか。	バスと電車と
\\	やすいですか。	どっちのほうが・どちらのほうが	
\\	どちらのほうが 
\\	どっちのほうが, 
\\	どちら 
\\	どっち. 
\\	バスと 電車[でんしゃ]とどっちのほうがやすいですか。	
\\	ロシアとフランスと日本の中で、どこが一番寒いですか。	ロシアとフランスと日本
\\	、どこが
\\	寒いですか。	の中で、一番	
\\	のほう 
\\	どっち 
\\	だれ、どれ、何、いつ 
\\	どこ 
\\	ロシアとフランスと 日本[にっぽん]の 中[なか]で、どこが 一番[いちばん] 寒[さむ]いですか。	
\\	季節の中でいつが一番好きですか。	季節
\\	、いつが
\\	好きですか。	の中で、一番	
\\	のほう 
\\	どっち 
\\	だれ、どれ、何、いつ 
\\	どこ 
\\	季節[きせつ]の 中[なか]でいつが 一番[いちばん] 好[す]きですか。	
\\	私は黒いセーターを持っています。赤いのも持っています。	私は黒いセーターを持っています。赤い
\\	も持っています。	の	
\\	私[わたし]は 黒[くろ]いセーターを 持[も]っています。 赤[あか]いのも 持[も]っています。	
\\	安い辞書を買いに行きました。でもいいのがありませんでした。	安い辞書を買いに行きました。でもいい
\\	がありませんでした。	の	
\\	安[やす]い 辞書[じしょ]を 買[か]いに 行[い]きました。でもいいのがありませんでした。	
\\	「これはスーさんのかばんですか。」「いいえ、それはメアリーさんのです。」	「これはスーさんのかばんですか。」「いいえ、それはメアリーさん
\\	です。」	の	
\\	「これはスーさんのかばんですか。」「いいえ、それはメアリーさんのです。」	
\\	りんごとみかんとさくらんぼの中で、どれが一番好きですか。	りんごとみかんとさくらんぼ
\\	好きですか。	の中で、どれが一番	
\\	どれ 
\\	りんごとみかんとさくらんぼの 中[なか]で、どれが 一番[いちばん] 好[す]きですか。	
\\	果物の中で、何が一番好きですか。	果物
\\	好きですか。	の中で、何が一番	
\\	何 
\\	果物[くだもの]の 中[なか]で、 何[なに]が 一番[いちばん] 好[す]きですか。	
\\	アメリカのアイスクリームのほうが日本のよりおいしいです。	アメリカのアイスクリームのほうが日本
\\	よりおいしいです。	の	
\\	アメリカのアイスクリームのほうが 日本[にっぽん]のよりおいしいです。	
\\	(私は)週末にたけしさんとテニスをするつもりです。	(私は)週末にたけしさんと
\\	です。	テニスをするつもり	
\\	私[わたし]は) 週末[しゅうまつ]にたけしさんとテニスをするつもりです。	
\\	山下先生は明日大学に来ないつもりです。	山下先生は明日大学に
\\	です。	来ないつもり	
\\	山下[やました] 先生[せんせい]は 明日[あした] 大学[だいがく]に 来[こ]ないつもりです。	
\\	お寺を見に行くつもりでしたが、天気が良くなかったから、行きませんでした。	お寺を
\\	でしたが、天気が良くなかったから、行きませんでした。	見に行くつもり	
\\	お 寺[てら]を 見[み]に 行[い]くつもりでしたが、 天気[てんき]が 良[よ]くなかったから、 行[い]きませんでした。	
\\	日本語の勉強が楽しくなりました。	日本語の勉強が
\\	。	楽しくなりました	
\\	楽しい 
\\	楽しくなる, い 
\\	く 
\\	日本語[にほんご]の 勉強[べんきょう]が 楽[たの]しくなりました。	
\\	日本語の勉強が好きになりました。	日本語の勉強が
\\	。	好きになりました	
\\	好きな 
\\	好きになる, な 
\\	に 
\\	日本語[にほんご]の 勉強[べんきょう]が 好[す]きになりました。	
\\	どこへ行きましたか。いいえ、どこへも行きませんでした。	
\\	行きましたか。いいえ、
\\	行きませんでした。	どこかへ、どこへも	
\\	へ、に、で 
\\	も 
\\	どこへ 行[い]きましたか。いいえ、どこへも 行[い]きませんでした。	
\\	だれかに会いましたか。いいえ、だれにも会いませんでした。	
\\	に会いましたか。いいえ、
\\	会いませんでした。	だれかに、だれにも	
\\	へ、に、で 
\\	も 
\\	だれかに 会[あ]いましたか。いいえ、だれにも 会[あ]いませんでした。	
\\	はしでご飯を食べます。	はし
\\	ご飯を食べます。	で	
\\	はしでご 飯[はん]を 食[た]べます。	
\\	日本語で話しましょう。	日本語
\\	話しましょう。	で	
\\	日本語[にほんご]で 話[はな]しましょう。	
\\	バスで駅まで行きました。	バス
\\	駅まで行きました。	で	
\\	バスで 駅[えき]まで 行[い]きました。	
\\	テレビで映画を見ました。	テレビ
\\	映画を見ました。	で	
\\	テレビで 映画[えいが]を 見[み]ました。	
\\	午前と午後の便がありますが…	午前と午後の便があります
\\	が/けど	
\\	が、けど 
\\	どちらいいですか, 
\\	午前[ごぜん]と 午後[ごご]の 便[びん]がありますが…	
\\	今度の週末は、映画を見たいです。	今度の週末は、映画を
\\	です。	見たい	
\\	たい 
\\	を 
\\	が 
\\	を 
\\	今度[こんど]の 週末[しゅうまつ]は、 映画[えいが]を 見[み]たいです。	
\\	いつか中国に行きたいです。	いつか中国に
\\	です。	行きたい	
\\	たい 
\\	を 
\\	が 
\\	を 
\\	いつか 中国[ちゅうごく]に 行[い]きたいです。	
\\	あの人には会いたくないです。	あの人には
\\	です。	会いたくない	
\\	たいと思っています 
\\	たいです。
\\	あの 人[ひと]には 会[あ]いたくないです。	
\\	セーターが買いたかったから、デパートに行きました。	セーターが
\\	から、デパートに行きました。	買いたかった	
\\	たいと思っています 
\\	たいです。
\\	セーターが 買[か]いたかったから、デパートに 行[い]きました。	
\\	メアリーさんはチベットに行きたいと言っていました。	メアリーさんはチベットに
\\	。	行きたいと言っていました	
\\	メアリーさんはチベットに 行[い]きたいと 言[い]っていました。	
\\	メアリーさんは着物を着たがっています。	メアリーさんは着物を
\\	。	着たがっています	
\\	たがっている 
\\	たい. 
\\	を, たがっている 
\\	を 
\\	たい 
\\	を 
\\	が.	メアリーさんは 着物[きもの]を 着[き]たがっています。	
\\	たけしさんはハンバーガーを食べたがっています。	たけしさんはハンバーガーを
\\	。	食べたがっています	
\\	たがっている 
\\	たい. 
\\	を, たがっている 
\\	を 
\\	たい 
\\	を 
\\	が.	たけしさんはハンバーガーを 食[た]べたがっています。	
\\	大阪で買い物をしたり、晩ご飯を食べたりします。	大阪で買い物を
\\	、晩ご飯を
\\	。	したり、食べたりします。	
\\	大阪[おおさか]で 買い物[かいもの]をしたり、 晩[ばん]ご 飯[はん]を 食[た]べたりします。	
\\	週末は、勉強したり、友達と話したりしました。	週末は、
\\	、友達と
\\	。	勉強したり、話したりしました。	
\\	週末[しゅうまつ]は、 勉強[べんきょう]したり、 友達[ともだち]と 話[はな]したりしました。	
\\	踊ったり、音楽を聞いたりするのが好きです。	
\\	、音楽を
\\	のが好きです。	踊ったり、聞いたりする	
\\	踊[おど]ったり、 音楽[おんがく]を 聞[き]いたりするのが 好[す]きです。	
\\	富士山に登ったことがあります。	富士山に登った
\\	。	ことがあります	
\\	富士山[ふじさん]に 登[のぼ]ったことがあります。	
\\	たけしさんは授業を休んだことがありません。	たけしさんは授業を休んだ
\\	。	ことがありません	
\\	たけしさんは 授業[じゅぎょう]を 休[やす]んだことがありません。	
\\	「エーロッパに行ったことがありますか。」「はい、あります。」	「エーロッパに行った
\\	か。」
\\	。」	ことがあります、はい、あります。・はい、行ったことがあります。	
\\	はい、行ったことがあります 
\\	はい、あります. はい、ことがあります 
\\	「エーロッパに 行[い]ったことがありますか。」「はい、あります。」	
\\	京都や奈良に行きました。	京都
\\	奈良に行きました。	や	
\\	京都[きょうと]や 奈良[なら]に 行[い]きました。	
\\	ラメンやうどんや照り焼きが好きです。	ラメン
\\	うどん
\\	照り焼きが好きです。	や、や	
\\	ラメンやうどんや 照[て]り 焼[や]きが 好[す]きです。	
\\	「山下先生はテレビを見ますか。」「いいえ、テレビは見ません。」	「山下先生はテレビを見ますか。」「いいえ、テレビ
\\	見ません。」	は	
\\	を, が 
\\	は 
\\	私は 
\\	は 
\\	は 
\\	テレビ 
\\	コーヒー 
\\	山下[やました] 先生[せんせい]はテレビを 見[み]ますか。」「いいえ、テレビは 見[み]ません。」	
\\	「コーヒーが好きですか。」「いいえ、コーヒーは好きじゃないです。」	「コーヒーが好きですか。」「いいえ、コーヒー
\\	好きじゃないです。」	は	
\\	を, が 
\\	は 
\\	私は 
\\	は 
\\	は 
\\	テレビ 
\\	コーヒー 
\\	「コーヒーが 好[す]きですか。」「いいえ、コーヒーは 好[す]きじゃないです。」	
\\	英語では話したくないです。	英語
\\	話したくないです。	では	
\\	は 
\\	で、に 
\\	英語[えいご]では 話[はな]したくないです。	
\\	広島には行ったことがありません。	広島
\\	行ったことがありません。	には	
\\	は 
\\	で、に 
\\	広島[ひろしま]には 行[おこな]ったことがありません。	
\\	私はその人に一回だけ会ったことがあります。	私はその人に一回
\\	会ったことがあります。	だけ	
\\	だけ 
\\	私[わたし]はその 人[ひと]に一 回[かい]だけ 会[あ]ったことがあります。	
\\	一つだけ質問があります。	
\\	質問があります。	一つだけ	
\\	だけ 
\\	一[ひと]つだけ 質問[しつもん]があります。	
\\	三十分だけ寝ました。	
\\	寝ました。	三十分だけ	
\\	だけ 
\\	三十 分[ふん]だけ 寝[ね]ました。	
\\	晩御飯にサラダを食べました。	晩御飯
\\	サラダを食べました。	に	
\\	に 
\\	晩[ばん] 御飯[ごはん]にサラダを 食[た]べました。	
\\	お土産に絵葉書を買いました。	お土産
\\	絵葉書を買いました。	に	
\\	に 
\\	お 土産[みやげ]に 絵葉書[えはがき]を 買[か]いました。	
\\	湖までドライブに行きました/ドライブしました。	湖まで
\\	。	ドライブに行きました/ドライブしました	
\\	ドライブ 
\\	湖[みずうみ]までドライブに 行[い]きました/ドライブしました。	
\\	日本で車を運転したことがありますか。	日本で
\\	ことがありますか。	車を運転した	
\\	運転する 
\\	日本[にっぽん]で 車[くるま]を 運転[うんてん]したことがありますか。	
\\	夕べ怖い夢を見ました。	夕べ怖い
\\	。	夢を見ました	
\\	見る 
\\	持っている 
\\	ある 
\\	夕[ゆう]べ 怖[こわ]い 夢[ゆめ]を 見[み]ました。	
\\	夢を持っています 
\\	夢があります。	
\\	。	夢を持っています 
\\	夢があります。	
\\	見る 
\\	持っている 
\\	ある 
\\	夢[ゆめ]を 持[も]っています/ 夢[ゆめ]があります。	
\\	東京にはデパートがたくさんがあります。	東京
\\	デパートがたくさんがあります。	には	
\\	東京にデパートがたくさんあります。
\\	東京[とうきょう]にはデパートがたくさんがあります。	
\\	バスが来なかったんです。	バスが来なかった
\\	。	んです	
\\	んです 
\\	バスが 来[こ]なかったんです。	
\\	明日テストがあるんです。	明日テストがある
\\	。	んです	
\\	んです 
\\	明日[あした]テストがあるんです。	
\\	トイレに行きたいんだ。	トイレに行きたい
\\	。	んだ	
\\	んです 
\\	トイレに 行[い]きたいんだ。	
\\	成績がよくないんだ。	成績がよくない
\\	。	んだ	
\\	成績[せいせき]がよくないんだ。	
\\	試験が終わったんだ。	試験が終わった
\\	。	んだ	
\\	試験[しけん]が 終[お]わったんだ。	
\\	どうして彼と別れたんですか。	どうして彼と別れた
\\	。	んですか	
\\	どうして 
\\	、どうした 
\\	どうして 彼[かれ]と 別[わか]れたんですか。	
\\	彼、ぜんぜんお風呂に入らないんです。	彼、ぜんぜんお風呂に入らない
\\	。	んです	
\\	彼[かれ]、ぜんぜんお 風呂[ふろ]に 入[はい]らないんです。	
\\	どうしたんですか。	どうした
\\	。	んですか	
\\	どうしたんですか 
\\	んです 
\\	が 
\\	は	どうしたんですか。	
\\	猫が死んだんです。	猫が死んだ
\\	。	んです	
\\	猫[ねこ]が 死[し]んだんです。	
\\	「新しい靴だね」「ええ。イタリアも靴なんだ」	「新しい靴だね」「ええ。イタリアの
\\	なんだ	
\\	な 
\\	な 
\\	んです.	
\\	新[あたら]しい 靴[くつ]だね」「ええ。イタリアも 靴[くつ]なんだ」	
\\	「とてもいい教科書ですね」「ええ。私の大学の先生が書いたんです。」	「とてもいい教科書ね。」「ええ。私の大学の先生が書いた
\\	。」	んだ	
\\	んです 
\\	「とてもいい 教科書[きょうかしょ]ですね」「ええ。 私[わたし]の 大学[だいがく]の 先生[せんせい]が 書[か]いたんです。」	
\\	早く起き過ぎました。	早く起き
\\	。	過ぎました	
\\	早[はや]く 起[お]き 過[す]ぎました。	
\\	食べ過ぎてはいけません。	食べ
\\	はいけません。	過ぎて	
\\	食[た]べ 過[す]ぎてはいけません。	
\\	この本は高過ぎます。	この本は
\\	。	高すぎます	
\\	い 
\\	すぎます	この 本[ほん]は 高[たか] 過[す]ぎます。	
\\	この町は静かすぎます。	この町は
\\	。	静かすぎます	
\\	な 
\\	すぎます	この 町[まち]は 静[しず]かすぎます。	
\\	もっと野菜を食べたほうがいいですよ。	もっと野菜を食べた
\\	よ。	ほうがいいです	
\\	ほうがいいです 
\\	もっと 野菜[やさい]を 食[た]べたほうがいいですよ。	
\\	従業を休まないほうがいいですよ。	従業を休まない
\\	よ。	ほうがいいです	
\\	ほうがいいです 
\\	従業[じゅうぎょう]を 休[やす]まないほうがいいですよ。	
\\	いつも日本語で話すので、日本語が上手になりました。	いつも日本語で話す
\\	、日本語が上手になりました。	ので	
\\	から	
\\	いつも 日本語[にほんご]で 話[はな]すので、 日本語[にほんご]が 上手[じょうず]になりました。	
\\	宿題がたくさんあったので、昨日の夜、寝ませんでした。	宿題がたくさんあった
\\	、昨日の夜、寝ませんでした。	ので	
\\	から	
\\	宿題[しゅくだい]がたくさんあったので、 昨日[きのう]の 夜[よる]、 寝[ね]ませんでした。	
\\	その人はいじわるなので、嫌いです。	その人はいじわる
\\	、嫌いです。	なので	
\\	から	
\\	ので 
\\	な-
\\	な 
\\	んです	その 人[ひと]はいじわるなので、 嫌[きら]いです。	
\\	今日は日曜日なので、銀行は休みです。	今日は日曜日
\\	、銀行は休みです。	なので	
\\	から	
\\	ので 
\\	な-
\\	な 
\\	んです	今日[きょう]は 日曜日[にちようび]なので、 銀行[ぎんこう]は 休[やす]みです。	
\\	試験があるんから、勉強しなければいけません/なきゃいけません。	試験があるんから、勉強
\\	。	しなければいけません/なきゃいけません	
\\	なきゃ 
\\	なければ 
\\	なくちゃいけません、なくてはいけません、ないといけません. 
\\	なりません 
\\	いけません 
\\	なければ 
\\	なきゃ 
\\	いけません 
\\	~なきゃ、~なくちゃ、~ないと.	試験[しけん]があるんから、 勉強[べんきょう]しなければいけません/なきゃいけません。	
\\	今朝は、六時に起きなきゃいけませんでした。	今朝は、六時に
\\	。	起きなきゃいけませんでした。	
\\	今朝[けさ]は、六 時[じ]に 起[お]きなきゃいけませんでした。	
\\	毎日、練習しなきゃいけないんです。	毎日、練習
\\	んです。	しなきゃいけない	
\\	毎日[まいにち]、 練習[れんしゅう]しなきゃいけないんです。	
\\	明日は雨が降るでしょう。	明日は雨が
\\	。	降るでしょう	
\\	でしょう 
\\	い-
\\	明日[あした]は 雨[あめ]が 降[ふ]るでしょう。	
\\	明日は雨が降らないでしょう。	明日は雨が
\\	。	降らないでしょう	
\\	でしょう 
\\	い-
\\	明日[あした]は 雨[あめ]が 降[ふ]らないでしょう。	
\\	北海道は寒いでしょう。	北海道は
\\	。	寒いでしょう	
\\	でしょう 
\\	い-
\\	北海道[ほっかいどう]は 寒[さむ]いでしょう。	
\\	北海道は寒くないでしょう。	北海道は
\\	。	寒くないでしょう	
\\	でしょう 
\\	い-
\\	北海道[ほっかいどう]は 寒[さむ]くないでしょう。	
\\	山下先生は魚が好きでしょう。	山下先生は魚が
\\	。	好きでしょう	
\\	でしょう 
\\	な-
\\	~なでしょう、~のでしょう、~だでしょう。	山下[やました] 先生[せんせい]は 魚[さかな]が 好[す]きでしょう。	
\\	山下先生は魚が好きじゃないでしょう。	山下先生は魚が
\\	。	好きじゃないでしょう	
\\	でしょう 
\\	な-
\\	~なでしょう、~のでしょう、~だでしょう。	山下[やました] 先生[せんせい]は 魚[さかな]が 好[す]きじゃないでしょう。	
\\	あの人はオーストラリア人でしょう。	あの人は
\\	。	オーストラリア人でしょう	
\\	でしょう 
\\	な-
\\	~なでしょう、~のでしょう、~だでしょう。	あの 人[ひと]はオーストラリア 人[じん]でしょう。	
\\	あの人はオーストラリア人じゃないでしょう。	あの人は
\\	。	オーストラリア人じゃないでしょう	
\\	でしょう 
\\	な-
\\	~なでしょう、~のでしょう、~だでしょう。	あの 人[ひと]はオーストラリア 人[じん]じゃないでしょう。	
\\	日本語と韓国語と、どっちのほうが難しいでしょうか。	日本語と韓国語と、どっちのほうが難しい
\\	。	でしょうか	
\\	日本語[にほんご]と 韓国[かんこく] 語[ご]と、どっちのほうが 難[むずか]しいでしょうか。	
\\	たけしさんは興味があるだろうと思います。	たけしさんは興味がある
\\	と思います。	だろう	でしょう 
\\	だろう 
\\	たけしさんは 興味[きょうみ]があるだろうと 思[おも]います。	
\\	ジョン、中国語が分かるでしょ?これ、読んで。	ジョン、中国語が分かる
\\	これ、読んで。	でしょ	でしょう 
\\	でしょう 
\\	でしょ) 
\\	ジョン、 中国[ちゅうごく] 語[ご]が 分[わ]かるでしょ?これ、 読[よ]んで。	
\\	私は日本語が話せます。	私は日本語が話
\\	。	せます	
\\	私[わたし]は 日本語[にほんご]が 話[はな]せます。	
\\	私は泳げないんです	私は泳
\\	んです。	げない	
\\	んです)	
\\	私[わたし]は 泳[およ]げないんです	
\\	雨が降ったので、海に行けませんでした。	雨が降ったので、海に行
\\	。	けませんでした	
\\	雨[あめ]が 降[ふ]ったので、 海[うみ]に 行[い]けませんでした。	
\\	明日試験があるから、今日は勉強しなきゃいけないんです。	明日試験がある
\\	、今日は勉強しなきゃいけないんです。	ので、から	
\\	ので 
\\	から 
\\	明日[あした] 試験[しけん]があるから、 今日[きょう]は 勉強[べんきょう]しなきゃいけないんです。	
\\	日本語はおもしろいし、先生はいいし、私は日本語の授業が大好きです。	日本語はおもしろい
\\	、先生はいい
\\	、私は日本語の授業が大好きです。	し、し	
\\	日本語[にほんご]はおもしろいし、 先生[せんせい]はいいし、 私[わたし]は 日本語[にほんご]の 授業[じゅぎょう]が 大好[だいす]きです。	
\\	日本語の授業は楽しいそうです。	日本語の授業は
\\	です。	楽しいそう	
\\	楽しそう 
\\	楽しいそう 
\\	日本語[にほんご]の 授業[じゅぎょう]は 楽[たの]しいそうです。	
\\	先生はとても親切だそうです。	先生はとても親切
\\	です。	だそう	
\\	そうです 
\\	先生[せんせい]はとても 親切[しんせつ]だそうです。	
\\	その日は授業がなかったそうです。	その日は授業がなかった
\\	です。	そう	
\\	そうです 
\\	その 日[ひ]は 授業[じゅぎょう]がなかったそうです。	
\\	天気予報によると、台風が来るそうです。	天気予報によると、台風が
\\	です。	来るそう	
\\	~によると 
\\	トムさんによると 
\\	新聞によると 
\\	天気予報によると 
\\	天気[てんき] 予報[よほう]によると、 台風[たいふう]が 来[く]るそうです。	
\\	メアリーさん、今日は忙しいって。明日、試験があるんだって。	メアリーさん、今日は忙しい
\\	。明日、試験があるんだ
\\	。	って、って	
\\	って 
\\	と言っている 
\\	そうです	メアリーさん、 今日[きょう]は 忙[いそが]しいって。 明日[あした]、 試験[しけん]があるんだって。	
\\	あきらさんは何て言ってた?	あきらさんは何
\\	言ってた?	て	
\\	って 
\\	と 
\\	言う	あきらさんは 何[なに]て 言[い]ってた?	
\\	チョコレートを食べ過ぎたって言ってた。	チョコレートを食べ
\\	言ってた。	過ぎたって	
\\	って 
\\	と 
\\	言う	チョコレートを 食[た]べ 過[す]ぎたって 言[い]ってた。	
\\	日本に行ったら、着物を買います。	日本に
\\	、着物を買います。	行ったら	
\\	たら
\\	来週試験があったら、今度の週末は勉強したほうがいいですよ。
\\	日本[にっぽん]に 行[い]ったら、 着物[きもの]を 買[か]います。	
\\	天気がよかったら、散歩に行きます。	天気が
\\	、散歩に行きます。	よかったら	
\\	たら
\\	来週試験があったら、今度の週末は勉強したほうがいいですよ。
\\	天気[てんき]がよかったら、 散歩[さんぽ]に 行[い]きます。	
\\	山下先生に会ったら、このことを聞こうと思います。	山下先生に
\\	、このことを聞こうと思います。	会ったら	
\\	たら
\\	来週試験があったら、今度の週末は勉強したほうがいいですよ。
\\	山下[やました] 先生[せんせい]に 会[あ]ったら、このことを 聞[き]こうと 思[おも]います。	
\\	日本人だったら、この言葉を知っているでしょう。	日本人
\\	、この言葉を知っているでしょう。	だったら	
\\	たら
\\	来週試験があったら、今度の週末は勉強したほうがいいですよ。
\\	日本人[にっぽんじん]だったら、この 言葉[ことば]を 知[し]っているでしょう。	
\\	宝くじに当たったら、アムネスティにお金を送ります。	宝くじに
\\	、アムネスティにお金を送ります。	当たったら	
\\	たら
\\	来週試験があったら、今度の週末は勉強したほうがいいですよ。
\\	宝[たから]くじに 当[あ]たったら、アムネスティにお 金[かね]を 送[おく]ります。	
\\	今晩、家に帰ったら、電話します。	今晩、家に
\\	、電話します。	帰ったら	
\\	今晩[こんばん]、 家[いえ]に 帰[かえ]ったら、 電話[でんわ]します。	
\\	宿題が終わったら、遊びに行きましょう。	宿題が
\\	、遊びに行きましょう。	終わったら	
\\	宿題[しゅくだい]が 終[お]わったら、 遊[あそ]びに 行[い]きましょう。	
\\	私が猫だったら、一日中寝ているでしょう。	私が猫
\\	、一日中寝ているでしょう。	だったら	
\\	私[わたし]が 猫[ねこ]だったら、一 日[にち] 中[ちゅう] 寝[ね]ているでしょう。	
\\	お金があったら、車をかうんですけど。	お金が
\\	、車をかうんですけど。	あったら	
\\	お 金[かね]があったら、 車[くるま]をかうんですけど。	
\\	靴を脱がなくてもいいです。	靴を
\\	。	脱がなくてもいいです	
\\	も 
\\	なくてもいい 
\\	なくていい 
\\	靴[くつ]を 脱[ぬ]がなくてもいいです。	
\\	プレゼントは高くなくてもいいです。	プレゼントは
\\	。	高くなくてもいいです	
\\	も 
\\	なくてもいい 
\\	なくていい 
\\	プレゼントは 高[たか]くなくてもいいです。	
\\	私の父はカーネルおじさんみたいです。	私の父はカーネルおじさん
\\	。	みたいです	
\\	そうです 
\\	私[わたし]の 父[ちち]はカーネルおじさんみたいです。	
\\	あの人はゴリラみたいです。	あの人はゴリラ
\\	。	みたいです	
\\	そうです 
\\	あの 人[ひと]はゴリラみたいです。	
\\	雨が降ったみたいです。	雨が
\\	。	降ったみたいです	
\\	そうです 
\\	雨[あめ]が 降[ふ]ったみたいです。	
\\	あの人はお腹が空いているみたいです。	あの人は
\\	。	お腹が空いているみたいです	
\\	そうです 
\\	あの 人[ひと]はお 腹[なか]が 空[あ]いているみたいです。	
\\	あの人は昨日の夜寝なかったみたいです。	あの人は昨日の夜
\\	。	寝なかったみたいです	
\\	そうです 
\\	あの 人[ひと]は 昨日[きのう]の 夜[よる] 寝[ね]なかったみたいです。	
\\	先生は明日学校に来ないみたいです。	先生は明日学校に
\\	。	来ないみたいです	
\\	そうです 
\\	先生[せんせい]は 明日[あした] 学校[がっこう]に 来[こ]ないみたいです。	
\\	財布を忘れたみたいです。	財布を
\\	。	忘れたみたいです	
\\	みたいです 
\\	財布[さいふ]を 忘[わす]れたみたいです。	
\\	国に帰る前に、もう一度東京に行きます。	国に
\\	、もう一度東京に行きます。	帰る前に	
\\	前に 
\\	国[くに]に 帰[かえ]る 前[まえ]に、もう 一度[いちど] 東京[とうきょう]に 行[い]きます。	
\\	日本に来る前に、一学期日本語を勉強しました。	日本に
\\	、一学期日本語を勉強しました。	来る前に	
\\	前に 
\\	日本[にっぽん]に 来[く]る 前[まえ]に、一 学期[がっき] 日本語[にほんご]を 勉強[べんきょう]しました。	
\\	勉強してから、友達に手紙を書きました。	
\\	、友達に手紙を書きました。	勉強してから	
\\	てから 
\\	勉強[べんきょう]してから、 友達[ともだち]に 手紙[てがみ]を 書[か]きました。	
\\	けんさんが来てから、食べましょう。	けんさんが
\\	、食べましょう。	来てから	
\\	てから 
\\	けんさんが 来[き]てから、 食[た]べましょう。	
\\	猫が死んでから、とても寂しいです。	猫が
\\	、とても寂しいです。	死んでから	
\\	てから 
\\	猫[ねこ]が 死[し]んでから、とても 寂[さび]しいです。	
\\	留学生です
\\	。	留学生です
\\	。	か	
\\	留学生[りゅうがくせい]です
\\	じゅあ、その時計を下さい。	じゅあ、その時計
\\	。	を下さい	
\\	じゅあ、その 時計[とけい]を 下[くだ]さい。	
\\	二十三万四千五百六十七	
\\	二十三万四千五百六十七	
\\	二十三 万[まん]	
\\	和はよくテレビを見ます。	和はよくテレビを
\\	。	見ます	
\\	和[わ]はよくテレビを 見[み]ます。	
\\	メアリーさんは時々朝ご飯を食べません。	メアリーさんは時々朝ご飯を
\\	。	食べません	
\\	メアリーさんは 時々[ときどき] 朝[あさ]ご 飯[はん]を 食[た]べません。	
\\	私は明日京都に行きます。	私は明日京都に
\\	。	行きます	
\\	私[わたし]は 明日[あした] 京都[きょうと]に 行[い]きます。	
\\	スーさんは今日家に帰りません。	スーさんは今日家に
\\	。	帰りません	
\\	スーさんは 今日[きょう] 家[か]に 帰[かえ]りません。	
\\	コーヒーを飲みます。	コーヒー
\\	。	を飲みます	
\\	コーヒーを 飲[の]みます。	
\\	音楽を聞きます。	音楽
\\	聞きます。	を	
\\	音楽[おんがく]を 聞[き]きます。	
\\	テレビを見ます。	テレビ
\\	見ます。	を	
\\	テレビを 見[み]ます。	
\\	図書館で本を読みます。	
\\	本を読みます。	図書館で	
\\	図書館[としょかん]で 本[ほん]を 読[よ]みます。	
\\	家でテレビを見ます。	
\\	テレビを見ます。	家で	
\\	家[いえ]でテレビを 見[み]ます。	
\\	私は今日学校に行きません。	私は今日
\\	行きません。	学校へ	
\\	に 
\\	私[わたし]は 今日[きょう] 学校[がっこう]に 行[い]きません。	
\\	私は家に帰ります。	私は
\\	帰ります。	家に	
\\	へ 
\\	私[わたし]は 家[いえ]に 帰[かえ]ります。	
\\	日曜日に京都に行きます。	
\\	京都に行きます。	日曜日に	
\\	日曜日[にちようび]に 京都[きょうと]に 行[い]きます。	
\\	十一時に寝ます。	
\\	寝ます。	十一時に	
\\	十一 時[じ]に 寝[ね]ます。	
\\	四時ごろ(に)寝ます。	
\\	寝ます。	四時ごろに	
\\	四 時[じ]ごろ(に) 寝[ね]ます。	
\\	火曜日に行きます。	
\\	行きます。	火曜日に	
\\	火曜日[かようび]に 行[い]きます。	
\\	十時四十五分に起きます。	
\\	起きます。	十時四十五分に	
\\	十 時[じ]四十五 分[ふん]に 起[お]きます。	
\\	九月に帰ります。	
\\	帰ります。	九月に	
\\	九月[くがつ]に 帰[かえ]ります。	
\\	明日来ます。	
\\	来ます。	明日	
\\	明日[あした] 来[き]ます。	
\\	毎晩テレビを見ます。	
\\	テレビを見ます。	毎晩	
\\	毎晩[まいばん]テレビを 見[み]ます。	
\\	朝(に)新聞を読みます。	
\\	新聞を読みます。	朝(に)	
\\	朝[あさ](に) 新聞[しんぶん]を 読[よ]みます。	
\\	週末(に)何をしますか。	
\\	何をしますか。	週末(に)	
\\	週末[しゅうまつ](に) 何[なに]をしますか。	
\\	昼ご飯を食べませんか。	昼ご飯を
\\	。	食べませんか	
\\	昼[ひる]ご 飯[はん]を 食[た]べませんか。	
\\	いいですね。	
\\	。	いいですね。	
\\	ませんか	
\\	いいですね。	
\\	テニスをしませんか。	テニスを
\\	。	しませんか	
\\	テニスをしませんか。	
\\	うーん、ちょうと。	うーん、
\\	。	ちょうと	
\\	うーん、ちょうと。	
\\	私は時々喫茶店に行きます。	私は時々喫茶店に
\\	。	行きます	
\\	私[わたし]は 時々[ときどき] 喫茶店[きっさてん]に 行[い]きます。	
\\	私は残念テレビを見ません。	私は残念テレビを
\\	。	見ません	
\\	私[わたし]は 残念[ざんねん]テレビを 見[み]ません。	
\\	武さんはあまり勉強しません。	武さんは
\\	勉強しません。	あまり	
\\	武[たけ]さんはあまり 勉強[べんきょう]しません。	
\\	メアリーさん、週末はたいてい、何をしますか。	メアリーさん、
\\	たいてい、何をしますか。	週末は	
\\	メアリーさん、 週末[しゅうまつ]はたいてい、 何[なに]をしますか。	
\\	今日は京都に行きます。	
\\	京都に行きます。	今日は	
\\	今日[きょう]は 京都[きょうと]に 行[い]きます。	
\\	晩ご飯は?	晩ご飯
\\	は	
\\	晩[ばん]ご 飯[はん]は?	
\\	ちょっと下さい。	
\\	下さい。	ちょっと	
\\	ちょっと 下[くだ]さい。	
\\	ちょっと待ってください。	
\\	待ってください。	ちょっと	
\\	ちょっと 待[ま]ってください。	
\\	土曜日に映画をみませんか。 
\\	土曜日は、ちょっと。	
\\	土曜日に映画をみませんか。 
\\	。	土曜日は、ちょっと	
\\	土曜日[どようび]に 映画[えいが]をみませんか。 
\\	土曜日[どようび]は、ちょっと。	
\\	あそこにマクルドナルドがあります。	あそこにマクルドナルド
\\	。	があります	
\\	あそこにマクルドナルドがあります。	
\\	テレビがありません。	テレビ
\\	ありません。	が	
\\	テレビがありません。	
\\	テレビじゃありません。	テレビ
\\	ありません。	じゃ	
\\	テレビじゃありません。	
\\	時間がありますか。	時間が
\\	。	ありますか	
\\	時間[じかん]がありますか。	
\\	火曜日にテストがあります。	火曜日にテスト
\\	。	があります	
\\	火曜日[かようび]にテストがあります。	
\\	明日は日本語のクラスがありません。	明日は
\\	ありません。	日本語のクラスが	
\\	明日[あした]は 日本語[にほんご]のクラスがありません。	
\\	明日京都でお祭りがあります。	明日京都
\\	お祭りがあります。	で	
\\	明日[あした] 京都[きょうと]でお 祭[まつ]りがあります。	
\\	あそこに留学生がいます。	あそこに留学生
\\	。	がいます	
\\	あそこに 留学生[りゅうがくせい]がいます。	
\\	日本人の友達がいます。	日本人の友達が
\\	。	います	
\\	日本人[にっぽんじん]の 友達[ともだち]がいます。	
\\	体育館はどこですか。	体育館は
\\	ですか。	どこ	
\\	体育館[たいいくかん]はどこですか。	
\\	(銀行は)あのデパートの前です。	(銀行は)
\\	です。	あのデパートの前	
\\	銀行[ぎんこう]は)あのデパートの 前[まえ]です。	
\\	銀行は図書館の隣です。	銀行は
\\	です。	図書館の隣	
\\	銀行[ぎんこう]は 図書館[としょかん]の 隣[となり]です。	
\\	傘はテーブルのそばです。	傘は
\\	そばです。	テーブルの	
\\	傘[かさ]はテーブルのそばです。	
\\	レストランはデパートと病院の間です。	レストランは
\\	です。	デパートと病院の間	
\\	レストランはデパートと 病院[びょういん]の 間[ま]です。	
\\	私はモスバーガーの向いでメアリーさんを待ちました。	私はモスバーガー
\\	メアリーさんを待ちました。	の向いで	
\\	前 
\\	向い 
\\	私[わたし]はモスバーガーの 向[むか]いでメアリーさんを 待[ま]ちました。	
\\	山下先生は桜大学の学生でした。	山下先生は桜大学の学生
\\	。	でした	
\\	山下[やました] 先生[せんせい]は 桜[さくら] 大学[だいがく]の 学生[がくせい]でした。	
\\	あれは日本の映画じゃなかったです。	あれは日本の映画
\\	。	じゃなかったです	
\\	あれは 日本[にっぽん]の 映画[えいが]じゃなかったです。	
\\	メアリーさんは九時ごろ家に帰りました。	メアリーさんは九時ごろ家に
\\	。	帰りました	
\\	メアリーさんは九 時[じ]ごろ 家[か]に 帰[かえ]りました。	
\\	私は昨日日本語を勉強しませんでした。	私は昨日日本語を
\\	。	勉強しませんでした	
\\	私[わたし]は 昨日[きのう] 日本語[にほんご]を 勉強[べんきょう]しませんでした。	
\\	私は昨日京都に行きました。	私は昨日京都に
\\	。	行きました	
\\	私[わたし]は 昨日[きのう] 京都[きょうと]に 行[い]きました。	
\\	山下先生も昨日京都に行きました。	山下先生
\\	昨日京都に行きました。	も	
\\	山下[やました] 先生[せんせい]も 昨日[きのう] 京都[きょうと]に 行[い]きました。	
\\	メアリーさんは靴を買います。	メアリーさん
\\	買います。	は靴を	
\\	メアリーさんは 靴[くつ]を 買[か]います。	
\\	メアリーさんはかばんも買いました。	メアリーさんはかばん
\\	買いました。	も	
\\	メアリーさんはかばんも 買[か]いました。	
\\	私は先週京都に行きました。	私は先週京都
\\	。	に行きました	
\\	私[わたし]は 先週[せんしゅう] 京都[きょうと]に 行[い]きました。	
\\	大阪にも行きました。	大阪
\\	行きました。	にも	
\\	大阪[おおさか]にも 行[い]きました。	
\\	ロバートさんは土曜日にパーテイーに来ました。	ロバートさんは土曜日にパーテイーに
\\	。	来ました	
\\	ロバートさんは 土曜日[どようび]にパーテイーに 来[き]ました。	
\\	日曜日にもパーテイーに行きました。	
\\	パーテイーに行きました。	日曜日にも	
\\	日曜日[にちようび]にもパーテイーに 行[い]きました。	
\\	メアリーさんはそこで武さんを一時間待ちました。	メアリーさんはそこで武さんを
\\	待ちました。	一時間	
\\	メアリーさんはそこで 武[たけ]さんを一 時間[じかん] 待[ま]ちました。	
\\	私は昨日日本語を三時間ぐらい勉強しました。	私は昨日日本語を
\\	勉強しました。	三時間ぐらい	
\\	私[わたし]は 昨日[きのう] 日本語[にほんご]を三 時間[じかん]ぐらい 勉強[べんきょう]しました。	
\\	昨日七時間半寝ました。	昨日
\\	寝ました。	七時間半	
\\	昨日[きのう]七 時間[じかん] 半[はん] 寝[ね]ました。	
\\	私は京都で写真をたくさん撮りました。	私は京都で
\\	写真を
\\	撮りました。	たくさん	
\\	私[わたし]は 京都[きょうと]で 写真[しゃしん]をたくさん 撮[と]りました。	
\\	果物をたくさん食べました。	
\\	食べました。	果物をたくさん/たくさん果物を	
\\	果物[くだもの]をたくさん 食[た]べました。	
\\	日本語と英語を話します。	日本語
\\	英語を話します。	と	
\\	日本語[にほんご]と 英語[えいご]を 話[はな]します。	
\\	京都と大阪に行きました。	京都
\\	に行きました。	と大阪	
\\	京都[きょうと]と 大阪[おおさか]に 行[い]きました。	
\\	メアリーさんはスーさんと韓国に行きます。	メアリーさんは
\\	韓国に行きます。	スーさんと	
\\	メアリーさんはスーさんと 韓国[かんこく]に 行[い]きます。	
\\	昨日面白い映画を見ました。	昨日
\\	映画を見ました。	面白い	
\\	昨日[きのう] 面白[おもしろ]い 映画[えいが]を 見[み]ました。	
\\	山下先生は怖い先生です。	山下先生は
\\	先生です。	怖い	
\\	山下[やました] 先生[せんせい]は 怖[こわ]い 先生[せんせい]です。	
\\	京都できれいな写真を撮りました。	京都で
\\	写真を撮りました。	きれいな	
\\	京都[きょうと]できれいな 写真[しゃしん]を 撮[と]りました。	
\\	山下先生は元気な先生です。	山下先生は
\\	先生です。	元気な	
\\	山下[やました] 先生[せんせい]は 元気[げんき]な 先生[せんせい]です。	
\\	沖縄の海はとてもきれいでした。	沖縄の海は
\\	きれいでした。	とても	
\\	沖縄[おきなわ]の 海[うみ]はとてもきれいでした。	
\\	この部屋はちょっと暑いです。	この部屋は
\\	暑いです。	ちょっと	
\\	この 部屋[へや]はちょっと 暑[あつ]いです。	
\\	ロバートさんは日本語のクラスが好きです。	ロバートさんは日本語のクラス
\\	です。	が好き	
\\	ロバートさんは 日本語[にほんご]のクラスが 好[す]きです。	
\\	山下先生は魚がきれいです。	山下先生は魚
\\	です。	がきれい	
\\	山下[やました] 先生[せんせい]は 魚[さかな]がきれいです。	
\\	私は野菜は好きですが、肉はきれいです。	私は野菜
\\	ですが、肉
\\	です。	は好き、はきれい	
\\	私[わたし]は 野菜[やさい]は 好[す]きですが、 肉[にく]はきれいです。	
\\	たけしさんはメアリーさんのことが好きです。	たけしさんはメアリーさん
\\	です。	のことが好き	
\\	のことが 
\\	が。	たけしさんはメアリーさんのことが 好[す]きです。	
\\	たけしさんはコーヒーが大好きです。	たけしさんはコーヒー
\\	です。	が大好き	
\\	たけしさんはコーヒーが 大好[だいす]きです。	
\\	キムさんは納豆が大きれいです。	キムさんは納豆
\\	です。	が大きれい	
\\	キムさんは 納豆[なっとう]が 大[だい]きれいです。	
\\	好きでも嫌いでもないです。	
\\	。	好きでも嫌いでもないです	
\\	好[す]きでも 嫌[きら]いでもないです。	
\\	これは私の好きな本です。	これは私の
\\	です。	好きな本	
\\	これは 私[わたし]の 好[す]きな 本[ほん]です。	
\\	一緒に図書館で勉強しましょう。	一緒に図書館で勉強
\\	。	しましょう	
\\	一緒[いっしょ]に 図書館[としょかん]で 勉強[べんきょう]しましょう。	
\\	喫茶店でコーヒーを飲みましょうか。	喫茶店でコーヒーを
\\	。	飲みましょうか	
\\	喫茶店[きっさてん]でコーヒーを 飲[の]みましょうか。	
\\	リーさんは切手を三枚買いました。	リーさんは
\\	買いました。	切手を三枚	
\\	リーさんは 切手[きって]を三 枚[まい] 買[か]いました。	
\\	教科書を読んでください。	教科書を
\\	ください。	読んで	
\\	教科書[きょうかしょ]を 読[よ]んでください。	
\\	すみません。ちょっと教えてください。	すみません。ちょっと
\\	ください。	教えて	
\\	すみません。ちょっと 教[おし]えてください。	
\\	窓を開けて。	窓を
\\	。	開けて	
\\	窓[まど]を 開[あ]けて。	
\\	教科書を見てもいいですか。	教科書を
\\	ですか。	見てもいい	
\\	も 
\\	食べていいです。	教科書[きょうかしょ]を 見[み]てもいいですか。	
\\	ここで写真を撮ってはいけません。	ここで写真を
\\	。	撮ってはいけません	
\\	は 
\\	も 
\\	~てもいいです。	ここで 写真[しゃしん]を 撮[と]ってはいけません。	
\\	ノートを借りて、コピーします。	ノートを
\\	、コピーします。	借りて	
\\	ノートを 借[か]りて、コピーします。	
\\	今日は、六次に起きて、勉強しました。	今日は、六次に
\\	、勉強しました。	起きて	
\\	今日[きょう]は、六 次[じ]に 起[お]きて、 勉強[べんきょう]しました。	
\\	食堂に行って、昼ご飯を食べましょう。	食堂に
\\	、昼ご飯を食べましょう。	行って	
\\	食堂[しょくどう]に 行[い]って、 昼[ひる]ご 飯[はん]を 食[た]べましょう。	
\\	バスに乗って、会社に行きます。	バスに
\\	、会社に行きます。	乗って	
\\	バスに 乗[の]って、 会社[かいしゃ]に 行[い]きます。	
\\	教科書を忘れて、すみません。	教科書を
\\	、すみません。	忘れて	
\\	教科書[きょうかしょ]を 忘[わす]れて、すみません。	
\\	私は今晩勉強します。明日テストがありますから。	私は今晩勉強します。明日テストがあります
\\	。	から	
\\	私[わたし]は 今晩[こんばん] 勉強[べんきょう]します。 明日[あした]テストがありますから。	
\\	バスに乗りましょう。タクシーは高いですから。	バスに乗りましょう。タクシーは高いです
\\	。	から	
\\	バスに 乗[の]りましょう。タクシーは 高[たか]いですから。	
\\	(私が)やりましょうか。	(私が)
\\	。	やりましょうか	
\\	私[わたし]が)やりましょうか。	
\\	荷物を持ちましょうか。	荷物を
\\	。	持ちましょうか	
\\	荷物[にもつ]を 持[も]ちましょうか。	
\\	スーさんは今勉強しています。	スーさんは今
\\	。	勉強しています	
\\	スーさんは 今[こん] 勉強[べんきょう]しています。	
\\	たけしさんは英語の本を読んでいます。	たけしさんは英語の本を
\\	。	読んでいます	
\\	たけしさんは 英語[えいご]の 本[ほん]を 読[よ]んでいます。	
\\	私は英語を教えています。	私は英語を
\\	。	教えています	
\\	私[わたし]は 英語[えいご]を 教[おし]えています。	
\\	メアリーさんは毎日日本語を勉強しています。	メアリーさんは毎日日本語を
\\	。	勉強しています	
\\	メアリーさんは 毎日[まいにち] 日本語[にほんご]を 勉強[べんきょう]しています。	
\\	山下先生は結婚しています。	山下先生は
\\	。	結婚しています	
\\	山下[やました] 先生[せんせい]は 結婚[けっこん]しています。	
\\	みちこさんは窓の近くに座っています。	みちこさんは窓の近くに
\\	。	座っています	
\\	みちこさんは 窓[まど]の 近[ちか]くに 座[すわ]っています。	
\\	スーさんはお金をたくさん持っています。	スーさんはお金をたくさん
\\	。	持っています	
\\	スーさんはお 金[かね]をたくさん 持[も]っています。	
\\	山下先生は母を知っています。	山下先生は母を
\\	。	知っています	
\\	山下[やました] 先生[せんせい]は 母[はは]を 知[し]っています。	
\\	トムさんはちょっと太っています。	トムさんはちょっと
\\	。	太っています	
\\	トムさんはちょっと 太[ふと]っています。	
\\	私の弟はとても痩せています。	私の弟はとても
\\	。	痩せています	
\\	私[わたし]の 弟[おとうと]はとても 痩[や]せています。	
\\	メアリーさんは
\\	シャツを着ています。	メアリーさんは
\\	シャツを
\\	。	着ています	
\\	メアリーさんは 
\\	シャツ[てぃーしゃつ]を 着[き]ています。	
\\	お父さんは起きています。	お父さんは
\\	。	起きています	
\\	お 父[とう]さんは 起[お]きています。	
\\	家族は東京に住んでいます。	家族は東京に
\\	。	住んでいます	
\\	家族[かぞく]は 東京[とうきょう]に 住[す]んでいます。	
\\	私の姉は日本の会社に勤めています。	私の姉は日本の会社に
\\	。	勤めています	
\\	私[わたし]の 姉[あね]は 日本[にっぽん]の 会社[かいしゃ]に 勤[つと]めています。	
\\	中国に行っています。	中国に
\\	。	行っています	
\\	中国[ちゅうごく]に 行[い]っています。	
\\	家に来ています。	家に
\\	。	来ています	
\\	家[いえ]に 来[き]ています。	
\\	トムさんは髪が長いです。	トムさんは
\\	です。	髪が長い	
\\	トムさんは 髪[かみ]が 長[なが]いです。	
\\	あの店の食べ物は安くて、おいしいです。	あの店の食べ物は
\\	です。	安くて、おいしい	
\\	あの 店[みせ]の 食べ物[たべもの]は 安[やす]くて、おいしいです。	
\\	ホテルはきれいで、よかったです。	ホテルは
\\	です。	きれいで、よかった	
\\	ホテルはきれいで、よかったです。	
\\	山下先生は日本人で、五十歳ぐらいです。	山下先生は
\\	です。	日本人で、五十歳ぐらい	
\\	山下[やました] 先生[せんせい]は 日本人[にっぽんじん]で、五十 歳[さい]ぐらいです。	
\\	デパートにかばんを買いに行きました。	デパートに
\\	行きました。	かばんを買いに	
\\	デパートにかばんを 買[か]いに 行[い]きました。	
\\	メアリーさんは日本に日本語を勉強しに来ました。	メアリーさんは日本に
\\	来ました。	日本語を勉強しに	
\\	メアリーさんは 日本[にっぽん]に 日本語[にほんご]を 勉強[べんきょう]しに 来[き]ました。	
\\	私のクラスに(は)スウェーデン人の学生が一人います。	私のクラスに(は)スウェーデン人の学生が
\\	います。	一人	
\\	私[わたし]のクラスに(は)スウェーデン 人[じん]の 学生[がくせい]が一 人[にん]います。	
\\	どんな音楽を聞く?	どんな音楽を
\\	聞く	
\\	どんな 音楽[おんがく]を 聞[き]く?	
\\	メアリーさんは二年生。	メアリーさんは
\\	。	二年生	
\\	メアリーさんは二年生だよ。
\\	だ)	メアリーさんは二 年生[ねんせい]。	
\\	スーさんは、明日試験があると言っていました。	スーさんは、明日試験がある
\\	。	と言っていました	
\\	言っていました 
\\	と言いました 
\\	スーさんは、 明日[あした] 試験[しけん]があると 言[い]っていました。	
\\	(私は)たけしさんはメアリーさんが好きだと思います。	(私は)たけしさんはメアリーさんが好きだ
\\	。	と思います	
\\	私[わたし]は)たけしさんはメアリーさんが 好[す]きだと 思[おも]います。	
\\	(私は)メアリーさんはたけしさんが好きじゃないと思います。	(私は)メアリーさんはたけしさんが好き
\\	。	じゃないと思います	
\\	~ないと思います。	
\\	私[わたし]は)メアリーさんはたけしさんが 好[す]きじゃないと 思[おも]います。	
\\	ここで写真を撮らないでください。	ここで写真を
\\	ください。	撮らないで	
\\	ここで 写真[しゃしん]を 撮[と]らないでください。	
\\	(私は)日本語を勉強するのが好きです。	(私は)日本語を
\\	です。	勉強するのが好き	
\\	私[わたし]は) 日本語[にほんご]を 勉強[べんきょう]するのが 好[す]きです。	
\\	(私は)部屋を掃除するのが嫌いです。	(私は)部屋を
\\	です。	掃除するのが嫌い	
\\	私[わたし]は) 部屋[へや]を 掃除[そうじ]するのが 嫌[きら]いです。	
\\	ロバートさんは料理を作るのが上手です。	ロバートさんは料理を
\\	です。	作るのが上手	
\\	ロバートさんは 料理[りょうり]を 作[つく]るのが 上手[じょうず]です。	
\\	たけしさんは英語を話すのが下手です。	たけしさんは英語を
\\	です。	話すのが下手	
\\	たけしさんは 英語[えいご]を 話[はな]すのが 下手[へた]です。	
\\	私は日本語を話すのがとくいです。	私は日本語を話すのが
\\	です。	とくい	
\\	~がとくいです、~がにがてです。	私[わたし]は 日本語[にほんご]を 話[はな]すのがとくいです。	
\\	私は日本語を話すのがにがてです。	私は日本語を話すのが
\\	です。	にがて	
\\	~がとくいです、~がにがてです。	私[わたし]は 日本語[にほんご]を 話[はな]すのがにがてです。	
\\	ロバートさんが沖縄に行きました。	ロバートさん
\\	沖縄に行きました。	が	
\\	ロバートさんが 沖縄[おきなわ]に 行[い]きました。	
\\	誰が沖縄に行きましたか。	
\\	沖縄に行きましたか。	誰が	
\\	誰は沖縄に行きましたか。	誰[だれ]が 沖縄[おきなわ]に 行[い]きましたか。	
\\	どのクラスが面白いですか。	どのクラス
\\	面白いですか。	が	
\\	どのクラスが 面白[おもしろ]いですか。	
\\	日本語のクラスが面白いです。	日本語のクラス
\\	面白いです。	が	
\\	日本語[にほんご]のクラスが 面白[おもしろ]いです。	
\\	(このクラス)誰がめがねをかけていますか。	(このクラス)誰
\\	めがねをかけていますか。	が	
\\	(このクラス) 誰[だれ]がめがねをかけていますか。	
\\	山下先生がめがねをかけています。	山下先生
\\	めがねをかけています。	が	
\\	山下[やました] 先生[せんせい]がめがねをかけています。	
\\	猫が何か持って来ました。	猫が
\\	持って来ました。	何か	
\\	猫[ねこ]が 何[なに]か 持[も]って 来[き]ました。	
\\	猫は何か食べましたか。	猫は
\\	食べましたか。	何か	
\\	猫[ねこ]は 何[なに]か 食[た]べましたか。	
\\	いいえ、猫は何も食べませんでした。	いいえ、猫は
\\	。	何も食べませんでした	
\\	いいえ、 猫[ねこ]は 何[なに]も 食[た]べませんでした。	
\end{CJK}
\end{document}