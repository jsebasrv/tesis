\documentclass[8pt]{extreport} 
\usepackage{hyperref}
\usepackage{CJKutf8}
\begin{document}
\begin{CJK}{UTF8}{min}
\\	社会事情と科学的精神
\\	石原純
\\	【テキスト中に現れる記号について】
\\	ルビ
\\	(例)正鵠《せいこく》
\\	入力者注 主に外字の説明や、傍点の位置の指定
\\	(数字は、
\\	0213の面区点番号または
\\	、底本のページと行数)
\\	(例)※[#ローマ数字1、1-13-21]
\\	近時において世界はあらゆる混乱に陥り、すべての国家は険悪な難路を歩みつつあること周知のごとくである。我々の周囲においても思想の錯雑紛糾せること今日のごときは未だかつて見ないといってよいほどであり、したがってすべての人々がこれに多大な関心をもたないわけにゆかない有様になっている。しかも最も恐るべきことには、我々を取り囲むところの全雰囲気がいつかしら一定の偏向を示そうとするかのごとくに見えるのである。我々は正常な雰囲気の状態においてこそ、安らかに生命を保続することができるのであるのに、これがいちじるしく常態を外れるにあたって、そこに多くの憂慮すべき事情の現われることを虞れねばならないであろう。あまりに多く酸素の欠乏せる大気のなかに、我々は窒息を覚えねばならなかったであろうし、もしまた反対に、酸素の過剰に出遇うならば、各人はいたずらに昂奮して無意味な乱舞に陥るかも知れない。だが、すでに非常時の声が我々の国内に漲って以来、この国土における雰囲気はどんな変化を示しつつあるのか。とくにいわゆる五・一五事件や二・二六事件のごとき悲痛なる体験を経来って、そこには明らかにファシズム的色彩が漸次濃厚に達しつつあるではないか。かくのごときものは果して我々の避け得られない運命であったのかどうか、我々は今日においてまさに我々みずからを正視する必要があるであろう。
\\	ここにすべての人々の奮起すべき絶対的な理由がある。この危機に際してはすべての社会事情に対して出来得る限り誤らない検討批判が行なわれなくてはならない。威力に脅かされ、いたずらに黙して退くがごときは、まさに良心的な国民の責を果さないものとさえいわねばならないであろう。だが、我々は同時に絶大な困難を予想する。もしこの検討批判にして独断的に陥り、もしくは極めて浅薄な見解に終始するならば、それはかえって国運の前途に災いするにさえ至ること確かであるからである。しかも独断的でありまたは浅薄なるものはほとんどすべての場合に、当然そうであることの自覚を欠いて現われるのである。ここにあらゆる危険性が包蔵される。我々の信ずるところではこれを避ける唯一の道は科学的精神に徹底することである。あらゆる事情を科学的に検討し批判することによってのみ、我々は我々に可能な限りにおいて、最も望ましいものを見出すことができるであろう。
\\	だが、このことは抽象的に主張するのは容易であるが、一々の具体的事実についてこれを行なおうとする場合に、異常な困難を伴うのはいうまでもない。これはもとより社会事情の極めて複雑なるのによるのであるが、しかしそれにしても主要な社会的政策においてもし科学的精神に反するようなものがあったとしたならば、これを痛論駁撃してその実行を避けしめることは、とくにこの科学的精神を重んずるものの当然なすべきところでなければならない。この意味において昨年(昭和十一年)十月刊行の『改造』誌上に田辺元博士が「科学政策の矛盾」と題して論ぜられたところのものは、大いに一般の注目を惹いた。その主旨はここに再び挙げるまでもないことと思うが、要するに近時文政当局者などによって知識偏重教育の弊害打破がしきりに策せられるにかかわらず、他方では国防充実の目的をもって自然科学を奨励しようとすることの根本的な矛盾を指摘し、今日においてわが国家を危殆に導くものは断じて知識偏重によるのではなく、かえって知識欠乏に由来するのであるとなし、科学的精神の重要性を極力説いたものであった。
\\	今日事実を正視するものにしてこの論旨に反対するものは恐らくあり得ないのであって、したがって田辺博士は単に当然の言をなしたに過ぎないとさえ評するものもある。しかし我々は田辺博士のこの所説が何故に多数の人々を強く感激せしめたかを明らかにする必要がある。それはすなわち現時の世情においてすでに一般者はある威力に脅かされて沈黙を強いられ、その当然と思惟するところのものをさえあえて口にすることのできない陰鬱さをはなはだしく感じているからである。しかも田辺博士が指摘せるごとく、「生来科学を愛好し知識を尊重し」科学的精神に徹すべきはずの科学者自身すらが、一般事物に対してはこの精神を欠如し、「ただ自己の研究に必要なる研究費さえ豊富に支給する政府であるならば、他にいかなる不合理を行なうもあえて関知するところでない」とすることすら公然の事実である。この間にあって、現に重要な社会的公職にある田辺博士みずからがその最も良心的な態度をもって、しかもむしろはなはだ痛烈な言辞を連ねて、あえて政府当局の重大な誤謬を明らかにしたのは、すでにそれだけで痛快を感ぜしめずには置かないのであった。我々は何よりもこのことを大いに称揚すべきである。爾後田辺博士のその他の所説について種々の異見を述べることをもっぱらとした人々もあるけれども、この所説の重大な使命が上述の点に存するものであることを正しく解する限り、いたずらにその論難に走るがごときは、とくにこの時局に際して正鵠《せいこく》を失したものであるといわねばならない。
\\	田辺博士の所説に次いで、同年十二月の『中央公論』において、小倉金之助博士執筆の「自然科学者の任務」に接することのできたのは、また我々の予期せざる幸福であった。ここでは再び現時の「反科学主義との強力なる抗争」の必要が説かれ、これを遂行するためには精神的な団結を待たねばならないとし、非科学的な内部闘争や感情的な対立に終始することの不可なる所以が明らかに示されたのは、これまた当然の言であるとはいうものの、しかし大いに時宜に適する切実な勧告でなければならなかった。とくに小倉博士がこれにおいて自然科学者と社会科学者との密接な提携の重要であるのを論ぜられたのは、極めて有用な警告である。この点に関しては、具体的な実践において種々の困難が現われはしないかと思われるけれども、何らかのよき指導によってこれが達せられるならば、それは最も望ましいところでなければならない。
\\	要するに、田辺、小倉両博士の所論が多大の反響を喚《よ》んだ所以のものは、先に述べたように、我々の周囲に現に拡がっているところのいかにも息苦しい圧迫的な雰囲気に対して、いくらかでもこれを緩和しようとする我々の切実な願望をそれによって卒直に表現することができた点にあるのであった。実際にこの数年間人々はいかにしても堪え難い重圧のもとに沈黙させられねばならなかった。かの喧ましい国体明徴観念の叫びやこれに伴う日本精神発揚の声の盛んであるなかには、何かしら不気味な底流の渦巻いているのは、蔽うべからざる事実である。実際政治の上においては、それが庶政一新の標語の奥に侵入したり、防共協定の影を妖しげに彩ったり、あるいは財政の準戦時体制化として頭を擡げようとしたり、あらゆる事件に伴ってまさに表に浮動しようとする形勢を蔵している。この時にあたって我々はこの底流の実体を明確につきとめ、これを正しく検討批判し、その間において我々に真に必要であるものを採り、しからざるものを打ち摧《くだ》いて、この暗冥な雰囲気を明朗化することが絶対に必要なのである。これがためには我々は飽くまでも正しい科学的精神をもって対抗し、あらゆる機会を利用して民衆を導くことが大切である。これこそ真に国家を憂えるものの重大な責務ではないであろうか。私はこの意味において田辺、小倉両博士のごときと相通ずる幾百幾千の論説が書かれて、その声を社会に普《あま》ねからしめんことを切望して止まないのである。そこには同一の主旨が幾度繰返されてもよいとさえ考えられる。それがますます繰返されることによって社会にはおのずから確固たる信念が始めて保持されるに至るであろうからである。それゆえに私はこれらの所論に対して、それが多少とも自己の捧持する見解と異なるものを含んでいるからといって、ことさらにかような点を挙げて反駁することにのみ急である人々を厳に戒めたいと思うのである。これこそ大局を通観することのできない近視眼者であって、またいたずらに頭上の蠅を気にしてその足を滑らす愚者にも似ているであろう。
\\	ここでしかし私は問題を正しく捉えるために、何故に現時の雰囲気が我々に重圧を感ぜしめているかについて、なお具体的に言及する必要があると考える。最も卒直にいうならば、それはファシズム的統制に対する一種の危惧であり、かつかようなものに対する我々人間の本能的反抗に存することは、恐らく確実である。いかなる人間も彼が一個の独立な人格意志を所有する限りにおいて、ある程度の自由を欲しないものはないであろう。しかしこれと同時に、彼が一定に組織づけられた社会もしくは国家の一員である限り、ある程度の統制を受けないわけにゆかないのもいうまでもない。そこでこの自由と統制との限界が極めて重大な問題なのである。それは単なる自由主義と統制主義との争いではない。我々にとってはさらに極めて深刻なる課題なのである。もとよりかような限界は社会もしくは国家形態によって決定せられるわけであるが、この限界をどこに持来すべきかということが現在の政治の動向を支配する重点であると同時に、これが何らかの転換の気勢を示しながらなお不明に残されていることにおいて民衆の危惧がかかっているのである。わが国においてこれをいかに処理すべきかという点に現時のあらゆる悩みが依存しているのであると考えられる。
\\	この問題は、その解決の理論的にいかに困難であるにもせよ、現実的な歴史的過程はそれをただ一つの方途において導いてゆくよりほかはないわけで、それだけに我々はその過程を軽々しく経過せしめてはならないのであり、そしてこれがために出来得る限り科学的精神に背馳することのない状態を望まなくてはならないのである。我々の最も慎重に慮《おもんぱか》るべきは、実際上にどれだけ多く正当な科学的精神を反映せしめ得るかという点に存するのであって、しかも国家の安危さえもこれに関わることを思うならば、この極めて重大な時機に際して最もよく科学的精神を解すべき科学者たちのすべてがまさに何らかの力を尽すべきであると思われる。社会に浸潤する卑俗的風潮のなかには、もっぱら愛国の美名を借りて単に自己の好むところを遮二無二現ぜしめようとする蒙昧主義がある。彼らの間にはいたずらに過去の英雄的言動に憧憬して、甘んじて恐るべき強力独裁政治を誘致しようとする危険性をも含んでいる。すでに頻発せる数度のテロリズム行為のごときはその一先鋒を示すものにほかならない。これらの禍を未然に避けることは、真に国を憂えるものの念頭に置かれねばならないところであろう。
\\	社会を正しく導かんがために、社会科学の研究は絶対に必要である。それは自然を利用せんがために自然科学の研究を欠くことのできないのと少しも異なる処はない。しかるに現時においては、すでに田辺博士の論ぜられたごとく、それが誤れる知識偏重打破の名のもとにいちじるしく抑圧せられ、ほとんど全くその自由をさえ失うに至ったのはそもそも何故であるかを、我々はまず深く考えねばならない。私の考えるところによれば、その責の大半は本来マルクシズムに含まれる一つの根本的欠陥と、そしてこれを盲信して急激に実行に移そうとした従来のマルクシストらの罪とに帰せられねばならないと思う。
\\	従来の社会科学的理論の一つとして、マルクシズムは確かにその最も優れたものであり得ることを私は認める。我々はそれから我々のかつて知らなかった多くの知識を教えられることを否定するわけにゆかない。だが、私がその根本的欠陥と考えるところのものは、それにおいて個々の社会の特殊形態の成立の可能性がほとんど無視せられている点にあると思う。このことに関して、私はまず自然現象における一つの本質的な関係について注意を喚び起すことを適切と考える。
\\	自然においては、そのあらゆる現象に対してつねに普遍的な法則が成立している。我々が従来数理的に取扱うことのできる多くの場合にそれらが微分方程式として示されることは周知のとおりであるが、現象が実在的に起るにあたってはつねにある始原条件と環境条件とに適合することが必要であり、そしてこれらの条件の如何に応じて方程式はそれぞれの特殊解を与えるのである。かような特殊解は、時に外面的には互いに全く異なる過程をさえ示すことは、とくに注目に値いする事実である。さらに多数の個々の対象的要素が種々の対象物を構成するにあたって、たとえその構成機構は同一の理論に従うにしても、なおそこには極めて多種類の特殊的なものが具体化されることはつねである。かくて我々は自然においてほとんど無数の物質や生物の発生を見るのである。自然は最も普遍的な一般的法則に支配されながら、何故に唯一種の物質や生物を創らなかったかを我々は十分に理解しなくてはならない。
\\	この点において深く考えるものにとっては、人間の形作る国家や社会が決してただ一通りのものではあり得ないことをむしろ当然とするであろう。しかもそれらは単に歴史的に時代を隔てて現われるのみではなく、互いに同時的にも存在し得ることにおいてかえって重要な意味が存するのである。人間にはすでに生物学的に互いに異なる人種的差別がある。また地理的ならびに歴史的環境の相違によるところの民族的差別がある。おのおのの社会や国家は必ずしも同一の人種や民族をもって組織されるとは限らないが、しかしおのおのの社会とおのおのの国家とはそれぞれ固有の始原および環境条件を具えている。それゆえに、たとえこれらの組織に関してある普遍的法則が存在するとしても、なおその個々の形態を異にするのはかえって当然でなければならない。私はいかなる社会科学理論においても、この普遍的法則と、そしてそれが種々の特殊条件のもとに異なった社会および国家を形作る所以とが明確に示されねばならないと思う。
\\	ところが、私の見る限りにおいて、マルクシズムの社会理論にあっては、あたかも歴史の必然的過程がいずれの社会をも唯一の予期された形態に導かねばならないことのみが強調される。そしてこれと異なるものの存在をあえて否定しようとさえするのである。我々は生物学において生物進化の理論を聴くが、しかしその個々の進化の方向は決して一定に制限せられることはないのであった。人間の社会において果してどうであるかはすでに重大なる疑問でなければならないのであり、これを考究するためには、特殊条件に関するさらに深甚かつ忠実なる考察を必要とするのである。現実の問題に関しては、個々の国家や社会の対立が最も重大なる関係をもつ限りにおいて、とくにこの事実を除外することができないはずであるのに、強いてこれを無視するところにマルクシズムの理論の、根本的な欠陥が存するのであると考えられる。しかもこの理論を実践しつつあるソヴィエト聯邦と直接に境を接する近隣において、とくに民族的特殊性を高調する国家社会主義的政治が現われるに至ったことは、これに対する一種の皮肉であると見られないこともない。
\\	さらにソヴィエト聯邦のごときは、すでに理論的にただ一通りの社会形態のみをもって必然的なものとする上から、それがたとえある仮定にのみ依存するものではあっても、これを実践せんがために当然他の仮定の考察を許容し得ない事情に陥り、よって絶対的な独裁的抑圧を実施しないわけにはゆかないのであった。これと同時にわが国においてマルクシズムの理論を輸入せる人々は、これを公式的に盲信し、ソヴィエト聯邦における一切の有様をそのまま即時に踏襲することに専念した。そして理論と実践との不可分を主張して、これに赴くところの手段の可否をさえ問わない危険行為をも勃発せしめた。このことがわが国における社会科学研究の上にいかに強く禍いしたかは、今において正しく反省されねばならないであろう。実際に社会科学の研究は、かくてわずかにその萌芽を示そうとして、すでに踏みにじられてしまったともいわれるであろう。だが、私は真の社会科学理論がその普遍的法則と特殊条件とを厳に区別する限りにおいて、いかなる政治形態のもとにおいても、その研究の許容せらるべき可能性をもたなくてはならないはずであると考えるのである。特殊条件を無視したり、理論と実践とを混淆したりするのは、社会科学者が十分に客観的であるべき科学的精神に欠けることによるところの、彼ら自身の罪であると断ぜられる。社会科学の研究の困難である種々の事由を私はなお想像することはできるが、しかしこれを自由に救う道は恐らく絶無ではないであろう。
\\	他方において民族性の問題に関して、なお少しく言及する必要がある。それはすなわち歴史的に民族性として現われているところのものといえども、なおそれぞれの時代における環境に依存するものであることを見逃してはならないということである。近ごろ日本精神なるものについてしきりに語られるのを聞くが、その内容を検討するに当っては、どれだけが真に日本民族に固有なものであり、これがいかにして環境に応じて現われたかを、正しく見究めることははなはだ大切である。西洋科学が輸入せられるまでは、わが国においてほとんど科学の存しなかったのは事実であるが、それは確かに環境に左右せられたものであり、それゆえに、我々の民族は本来科学的精神を欠いているとはいうことができないであろう。現に明治以後わが国における科学の顕著な発達は、およそ日本精神を科学的精神から引離してのみ見ることの不当を証するものであるといわねばならない。だが、しかし少なくとも現在においては、遺憾ではあるが、我々はとくに科学的精神において優れているとは、いかにしても言うことができないのである。自然科学の実用的利用はすでに相当な程度にまで進んではいるが、その大部分は先進諸国への模倣に帰せられるのであって、純粋に創意的な科学的研究に至っては、これを誇るに足るもののなお極めて乏しいのは否定せられない。しかしながらこのことによってさえもなお我々の民族が科学的精神において劣っていると結論するのは早計である。なぜなら、我々の従来の長い歴史が科学の必要をさほどに感ぜしめないような環境において育てられて来たのであり、したがって科学を展開せしめるための抽象的論理的思考に多く慣らされていなかったのであるから、これを十分に獲得するがためになおある年月を経過せねばならないのは、恐らく人間の心理の自然的発展過程から見て当然であり、今日においてかような発展の不可能を予め理由づける何ものをももたないからである。しかもいずれにしても現時の我々の環境は、ますます科学的精神の必要を増大せしめているのは疑うことのできない事実であり、それゆえに我々は絶大な努力をもってこの環境に適応するように我々自身を向わしめねばならないのである。環境に適応するもののみが将来の発展を約束し得ることは、これこそ進化学の普遍的原理である。
\\	社会科学に比べて自然科学の研究がなお多く自由に残されているのは、この学の性質上当然のことである。しかしこれとても強力政治のもとにはなはだしく制限せられることのあるのは、すでにロシヤやドイツなどの例において我々の見るとおりである。だが、かような状態はその最も極端な場合にのみ見られるものであって、しかもそれがやがて科学研究の衰微を結果しない訳にゆかないであろうことは、恐らく当然に予想せられるところである。なぜなら、我々の科学研究のためには人間精神の最も自由にしてかつ広汎なる活動を必要とすることはいうまでもないのであり、これを一定の方向にのみ束縛し、もしくは自己の周りに墻壁を築くことによって国際的協力を否定して、しかもよい結果を望むというのは、これこそ明白な矛盾を犯すものでなければならないからである。この事理はいかなる人々によっても解せられないはずはないと考えられるにかかわらず、あえてこれを没却して科学の国際性に眼を蔽うものの現われるに至ったのは、やはり極めて特殊な環境の然らしめたものであると解しないわけにゆかない。彼らにあってはすなわち、科学研究の如何を考慮するよりは、なお一層焦眉の急に迫られるところの国情に臨まねばならなかったのであり、これによって止むを得ずあらゆる犠牲を忍び、その科学研究をまでも抑圧するほどの政治形態に陥ったのである。これは実に彼らの至大な悲しむべき不幸ではあっても、断じてその誇りではないはずである。理論に長じたドイツ人はかほど深刻な窮境に立って、なお科学の民族性なるものを理論づけ、よってみずからの立場を擁護しようとさえしているが、将来において、そのいわゆるドイツ科学がいかなる特質と優越とを示し得るかは、はなはだしい疑問である。これに反してソヴィエト聯邦において主張せられるプロレタリア科学にあっては、それがなお国際的性質を具える点において普遍的であり得るけれども、しかし純粋に客観的であるべき自然科学の本質に関してまである階級的差別を強要しようとするのは、同じく政治的理由に基づく一種の欺瞞にほかならないと思われる。
\\	勿論、自然科学といえども、その実用上の目的に関しては、それぞれこれに応ずる内容の要求せられねばならないことは明らかであり、そして種々の場合に異なった有様において奨励せられるのも当然である。現在わが国においてとくに国防の目的をもってこれが奨励せられているのも、その一つである。だが、しかし自然科学における偉大な発見はつねに全く予期せられない処にのみ起るものであり、そしてまた最初は何らの実用をも伴わないこれらの純粋に知識的な発見が後に多大な実用を持来すに至ることは、すでに十分に我々の経験したところであるのを思うならば、直接の実用のみを目的として自然科学を奨励することのいかに誤ったものであるかを悟ることができるであろう。それはあたかも個人に対してその将来のための教育や修養を奨める代りにただちに職業的実務につかしめることを利とするのと同様である。実用のみをめざすこともしかし現在の事情がそれを切実に必要とする限りにおいては止むを得ないのであるが、現在と共に併せて将来を考慮することのさらに重要であるのを我々はつねに忘れてはならない。わが国の実情を見るならば、上に述べたように、科学的精神においてはなお一般的に進んでいるとは決して言われないのであって、ましてたとえ理論的に歪曲されているとはいえ、ドイツ人がドイツ科学を高唱するがごとき意味において語るべき何らの民族的科学をさえわが国において全く有しないのは確かな事実である。したがってそのままの貧困な科学的精神にわが国の将来を委ねることのいかに心細いかを十分に知らねばならない。それゆえに今日においていかに現在の国防や産業のために自然科学が奨励せられねばならないとしても、我々はそれと同等以上に一般的奨励のつねに行なわれねばならないことを痛感するのである。まして近ごろすでに夙《はや》く科学研究に関する統制の声の聞かれるがごときは、この見地において我々の最も遺憾とするところである。科学的精神の涵養のごときは、これを一時の急に向って行なおうとすることは到底不可能であり、それが長い歴史的過程を必要とするのは既述のごとくであることから見て、我々は何よりもまずこの点に重きを置かねばならなかったのであろう。そしてこれこそ真に国家の将来を憂えるものの慮るべきところであると信ずる。
\\	[#地から2字上げ](『科学と社会文化』岩波書店・所収)
\\	底本:「現代日本思想大系 
\\	科学の思想※[#ローマ数字1、1-13-21]」筑摩書房
\\	1964(昭和39)年9月25日発行
\\	底本の親本:「科学と社会文化」岩波書店
\\	1937(昭和12)年12月20日第1刷発行
\\	初出:「科学ペン」
\\	1937(昭和12)年3月
\\	※冒頭の編者による解説は省略しました。
\\	入力:川山隆
\\	校正:高瀬竜一
\\	2016年6月10日作成
\\	青空文庫作成ファイル:
\\	このファイルは、インターネットの図書館、青空文庫
\\	で作られました。入力、校正、制作にあたったのは、ボランティアの皆さんです。
\end{CJK}
\end{document}