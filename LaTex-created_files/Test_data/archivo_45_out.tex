\documentclass[8pt]{extreport} 
\usepackage{hyperref}
\usepackage{CJKutf8}
\begin{document}
\begin{CJK}{UTF8}{min}
\\	"荷物を降[お]ろすのを
\\	手伝っ
\\	てもらえませんか
\\	荷物を降[お]ろすのを
\\	手伝っ
\\	"山田さん、この書類の書き方を
\\	教え
\\	てもらえないでしょうか
\\	山田さん、この書類の書き方を
\\	教え
\\	"先生、私たちの結婚式でスピーチを
\\	し
\\	ていたでけないでしょうか
\\	先生、私たちの結婚式でスピーチを
\\	し
\\	あの病院
\\	は
\\	きれいで、まるで
\\	ホテル
\\	のようです
\\	あの病院
\\	きれいで、まるで
\\	ホテル
\\	私の母
\\	は
\\	いつも元気で、明るくて、
\\	太陽
\\	のようです
\\	私の母
\\	いつも元気で、明るくて、
\\	太陽
\\	奈良
\\	のような
\\	町
\\	が好きです。歴史があって、おもしろいですから。
\\	奈良[なら]
\\	町
\\	が好きです。歴史があって、おもしろいですから。
\\	"彼は田舎の父親から来た手紙を読んで、
\\	子供
\\	のように
\\	泣いた
\\	彼は田舎の父親から来た手紙を読んで、
\\	子供
\\	泣いた
\\	この例
\\	のように
\\	書いてください
\\	この例
\\	書いてください
\\	"私は
\\	ジュース
\\	のような
\\	お酒
\\	しか飲まない。
\\	私は
\\	ジュース
\\	お酒
\\	しか飲まない。
\\	"若いときに
\\	留学する
\\	ことは
\\	いい経験になると思います。
\\	若いときに
\\	留学する
\\	いい経験になると思います。
\\	"いいサービスをするためには、
\\	お客様の気持ちを考える
\\	ことが
\\	大切です。
\\	いいサービスをするためには、
\\	お客様の気持ちを考える
\\	大切です。
\\	次の大会で勝つ
\\	ことを
\\	目指して頑張っています。
\\	次の大会で勝つ
\\	目指して頑張っています。
\\	正月に
\\	食べる料理
\\	を
\\	何
\\	と
\\	言いますか
\\	おせち料理
\\	と言います
\\	正月に
\\	食べる料理
\\	何
\\	おせち料理
\\	"正月に
\\	神社やお寺に行くこと
\\	を
\\	初詣で
\\	と言います
\\	正月に
\\	神社やお寺に行くこと
\\	初詣で
\\	村上春樹[むらかみはるき]
\\	という
\\	作家
\\	を知っていますか。
\\	…ええ。「ノルウエイの森」を書いた人ですね。
\\	村上春樹[むらかみはるき]
\\	作家
\\	を知っていますか。
\\	…ええ。「ノルウエイの森」を書いた人ですね。
\\	南太平洋[たいへいよう]に世界で一番早く朝が来る国があります。
\\	何
\\	という
\\	国
\\	ですか。
\\	トンガ
\\	王国[おうこく]
\\	という
\\	国
\\	です。
\\	南太平洋[たいへいよう]に世界で一番早く朝が来る国があります。
\\	何
\\	国
\\	ですか。
\\	トンガ王国[おうこく]
\\	国
\\	です。
\\	"ワットさんの研究室は
\\	いつ
\\	行っ
\\	ても
\\	きちんと整理[せいり]してあります。
\\	ワットさんの研究室は
\\	行っ
\\	きちんと整理してあります。
\\	"連休中は
\\	どこへ
\\	行っ
\\	ても
\\	人がいっぱいだ。
\\	連休中は
\\	行っ
\\	人がいっぱいだ。
\\	"熱があるので、
\\	何
\\	を食べ
\\	ても
\\	おいしくない。
\\	熱があるので、
\\	を食べ
\\	おいしくない。
\\	だれ
\\	に聞い
\\	ても
\\	「日本は食べ物が高い」と言う。
\\	に聞い
\\	「日本は食べ物が高い」と言う。
\\	どんなに
\\	立派な人
\\	でも
\\	欠点はあります。
\\	立派[りっぱ]な人
\\	欠点はあります。
\\	"これは
\\	ペン
\\	じゃなくて
\\	チョコレート
\\	です。食べられますよ。
\\	これは
\\	ペン
\\	チョコレート
\\	です。食べられますよ。
\\	京都ではお寺お見ましょうか。
\\	お寺
\\	じゃなくて
\\	若い人が行くようなにぎやかな所
\\	に行きたいです。
\\	京都ではお寺お見ましょうか。
\\	お寺
\\	若い人が行くようなにぎやかな所
\\	に行きたいです。
\\	"3時の飛行機に乗らなければなりません。それで、私は
\\	急いでいろ
\\	のです
\\	3時の飛行機に乗らなければなりません。それで、私は
\\	急いでいろ
\\	このレポートは一人で
\\	書いた
\\	のではありません
\\	このレポートは一人で
\\	書いた
\\	"窓ガラスをふ
\\	いたら
\\	、部屋が明るくな
\\	った
\\	窓ガラスをふ
\\	くと
\\	、部屋が明るくな
\\	った
\\	"窓ガラスをふ
\\	、部屋が明るくな
\\	窓ガラスをふ
\\	、部屋が明るくな
\\	"冷たい牛乳を飲
\\	んだら
\\	、おなかが痛くなってしま
\\	った
\\	冷たい牛乳を飲
\\	むと
\\	、おなかが痛くなってしま
\\	った
\\	冷たい牛乳を飲
\\	、おなかが痛くなってしま
\\	冷たい牛乳を飲
\\	、おなかが痛くなってしま
\\	"家へ帰
\\	ったら
\\	、母から荷物が届
\\	いた
\\	家へ帰
\\	ると
\\	、母から荷物が届
\\	いた
\\	家へ帰
\\	、母から荷物が届
\\	家へ帰
\\	、母から荷物が届
\\	"冷蔵庫の中を見
\\	たら
\\	、何も入っていません
\\	でした
\\	冷蔵庫の中を見
\\	ると
\\	、何も入っていません
\\	でした
\\	冷蔵庫[れいぞうこ]の中を見
\\	、何も入っていません
\\	冷蔵庫[れいぞうこ]の中を見
\\	、何も入っていません
\\	外来語
\\	というのは
\\	日本語になった
\\	外国語
\\	のことだ
\\	"外来語
\\	日本語になった外国語
\\	というのは
\\	汚い、きつい、危険な
\\	仕事
\\	のことだ
\\	汚い、きつい、危険な
\\	仕事
\\	「地球にやさしい」
\\	というのは
\\	どいう意味ですか。
\\	…地球を守るために、
\\	地球を汚さない
\\	ということです
\\	「地球にやさしい」
\\	どいう意味ですか。
\\	…地球を守るために、
\\	地球を汚さない
\\	禁煙
\\	というのは
\\	タバコを
\\	吸ってはいけない
\\	ということです
\\	禁煙
\\	タバコを
\\	吸ってはいけない
\\	日本語を学習する外国人が増えている
\\	という
\\	記事
\\	を読みました。
\\	日本語を学習する外国人が増えている
\\	記事
\\	を読みました。
\\	ゴミを分けて出す
\\	という
\\	規則
\\	はなかなか守られていません。
\\	ゴミを分けて出す
\\	規則
\\	はなかなか守られていません。
\\	彼が会社を辞めた
\\	という
\\	うわさ
\\	を聞きました。
\\	彼が会社を辞めた
\\	うわさ
\\	を聞きました。
\\	"学生に
\\	図書館で物を食べない
\\	ように
\\	注意
\\	しました。
\\	"学生に
\\	図書館で物を食べてはいけまんせん
\\	と
\\	注意しました
\\	この仕事を今日中にやる
\\	ように
\\	頼まれました
\\	この仕事を今日中にやってください
\\	と
\\	頼まれました
\\	"子供たちに
\\	早く寝る
\\	ように
\\	言いました
\\	"子供たちに
\\	早く寝なさい
\\	と
\\	言いました
\\	"私は父に
\\	もっと勉強する
\\	ように
\\	言われました
\\	私は父に
\\	もっと勉強する
\\	言われました
\\	"管理人[かんりにん]さんに何か言われたんですか。
\\	ええ、
\\	ゴミは月曜日以外には出さない
\\	ように
\\	注意されました
\\	管理人[かんりにん]さんに何か言われたんですか。
\\	ええ、
\\	ゴミは月曜日以外には出さない
\\	注意されました
\\	このロボット
\\	本物の犬
\\	みたいだ
\\	このロボット
\\	本物の犬
\\	"てんぷらが好きだという外国の方が多いですね。
\\	…ええ。
\\	てんぷら
\\	みたいな
\\	料理
\\	は世界中にありますから。
\\	てんぷらが好きだという外国の方が多いですね。
\\	…ええ。
\\	てんぷら
\\	料理
\\	は世界中にありますから。
\\	"子供のころから
\\	鳥
\\	みたいに
\\	空を
\\	飛びたい
\\	と思っていた。
\\	子供のころから
\\	鳥
\\	空を
\\	飛びたい
\\	と思っていた。
\\	お急がしい
\\	ところ
\\	、すみません。ちょっとお願いがあるんですが。
\\	お急がしい
\\	、すみません。ちょっとお願いがあるんですが。
\\	お休みの
\\	ところ
\\	、手伝ってくださって、ありがとうごさいました。
\\	お休みの
\\	、手伝ってくださって、ありがとうごさいました。
\\	"すみません。写真を1枚
\\	取らせてもらえませんか
\\	"すみません。写真を1枚
\\	取る
\\	"先生、留学生の皆さんにちょっとインタビュー
\\	させてまらえないでしょうか
\\	"先生、留学生の皆さんにちょっとインタビュー
\\	します
\\	"月曜日の店長会議で報告
\\	させていただけませんか
\\	"月曜日の店長会議で報告(します)
\\	"課長、今度の仕事
\\	はぜひ私[わたくし]に担当
\\	させていただけないでしょうか
\\	"課長、今度の仕事はぜひ私に担当
\\	します
\\	"今日は時間がないので、この仕事は明日
\\	する
\\	ことにします
\\	"今日は時間がないので、この仕事は明日
\\	する
\\	ます。
\\	"この問題については、もっとみんなの意見を聞いたほうがいいと思います。
\\	…そうですね。じゃ、来週、もう一度
\\	話し合う
\\	ことにしましょう
\\	この問題については、もっとみんなの意見を聞いたほうがいいと思います。
\\	…そうですね。じゃ、来週、もう一度
\\	話し合う
\\	~ょう。
\\	"今まで車で通勤していましたが、来月から電車で
\\	通う
\\	ことにしました
\\	"今まで車で通勤していましたが、来月から電車で
\\	通う
\\	~ました。
\\	"私は毎晩寝る前に日記を
\\	書く
\\	ことにしています
\\	"私は毎晩寝る前に日記を
\\	書く
\\	~ます。
\\	"子供には、一日[いちにち]に2時間以上はテレビを
\\	見せない
\\	ことにしています
\\	"子供には、一日に2時間以上はテレビを見せない
\\	~ます。
\\	"日曜日は朝食[ちょうしょく]を
\\	食べ
\\	たあと
\\	、どこえも行かず家でテレビを見て
\\	いました
\\	"日曜日は朝食を
\\	食べ
\\	、どこえも行かず家でテレビを見て
\\	いました
\\	"じゃ、来週の金曜日会議が
\\	終わっ
\\	た
\\	あと
\\	あとで
\\	お会いしましょう
\\	か。
\\	"じゃ、来週の金曜日会議が
\\	終わっ
\\	お会いしましょう
\\	か。
\\	"夫の転勤
\\	で広島[ひろしま]へ
\\	引っ越す
\\	ことになりました
\\	"夫の転勤で広島へ
\\	引っ越[こ]す
\\	~ました。
\\	"来年の4月にアジアの10か国の代表が集まって、経済問題について
\\	話し合う
\\	ことになった
\\	"来年の4月にアジアの10か国の代表が集まって、経済問題について
\\	話し合う
\\	た。
\\	"田中さんは家庭
\\	の事情[じじょう]で中国旅行には
\\	参加しない
\\	ことになりました
\\	"田中さんは家庭の事情で中国旅行には
\\	参加しない
\\	~ました。
\\	"この幼稚園[ようちえん]では、昼ご飯のあと、20分昼寝
\\	する
\\	ことになっています
\\	"この幼稚園では、昼ご飯のあと、20分昼寝
\\	する
\\	~ます。
\\	"いつまで日本にいるんですか。
\\	…来年3月に帰国
\\	する
\\	ことになっています
\\	いつまで日本にいるんですか。
\\	…来年3月に帰国
\\	する
\\	~ます。
\\	"祖母は今90歳です。祖母
\\	に
\\	はいつまでも元気で長生き
\\	して欲しいです
\\	"祖母は今90歳です。祖母
\\	はいつまでも元気で長生き(します)
\\	"早く景気
\\	が
\\	よく
\\	なってほしいです
\\	ね。
\\	"早く景気
\\	よく(なります)
\\	ね。
\\	"自然を守るために、この以上森の木を
\\	切らないでほしい
\\	"自然を守るために、この以上森の木を(切りません)
\\	"雨が
\\	降りそうな
\\	日
\\	は、洗濯[せんたく]をしません。
\\	"雨が(降ります)
\\	日
\\	は、洗濯をしません。
\\	"隣[となり]の家に
\\	怖そうな
\\	犬
\\	がいますね。
\\	…ああ、あの犬、本当はおとなしいんです。
\\	となりの家に(こわい)
\\	犬
\\	がいますね。
\\	…ああ、あの犬、本当はおとなしいんです。
\\	"みんな
\\	楽しそうに
\\	歌ったり
\\	、踊ったりしています。
\\	"みんな(楽しい)
\\	歌ったり
\\	、踊ったりしています。
\\	"君の誕生日のパーティーなのに、
\\	楽しくなさそう
\\	だね。
\\	…彼女に振られたんだ。
\\	君の誕生日のパーティーなのに、(楽しくない)
\\	だね。
\\	…彼女にふられたんだ。
\\	"この
\\	シャツ、500円だったんです。安いでしょう?
\\	…うん。でも、あまり
\\	丈夫[じょうぶ]じゃなさそう
\\	だね。
\\	この
\\	シャツ、500円だったんです。安いでしょう?
\\	…うん。でも、あまり(丈夫じゃない)
\\	だね。
\\	"意見がいろいろ出ているね、ミーティングはまだ
\\	終わりそうもありません
\\	"意見がいろいろ出ているね、ミーティングはまだ(終わりません)
\\	"電車が来ましたよ。
\\	…うわ、込んでいるね。
\\	座れそうもない
\\	な。
\\	電車が来ましたよ。
\\	…うわ、込んでいるね。(座れない)
\\	な。
\\	"市役所の説明によると、電器製品を捨てる場合はお金を
\\	払わなければなった
\\	ということです
\\	"市役所の説明によると、電器製品を捨てる場合はお金を
\\	払わなければなった
\\	"部長に30分ほど遅れると伝えてください。
\\	…はい、分かりました。30分ほど
\\	遅れる
\\	ということですね
\\	部長に30分ほど遅れると伝えてください。
\\	…はい、分かりました。30分ほど遅れる
\\	"近くにいた人[ひと]の話によると、トラックから急に荷物が
\\	落[お]ちた
\\	ということです
\\	近くにいた人の話によると、トラックから急に荷物が
\\	落[お]ちた
\\	(手紙文)
\\	先日、ワンさんに会いました。ワンさんから先生に
\\	宜しく
\\	とのことです
\\	(手紙文)
\\	先日、ワンさんに会いました。ワンさんから先生に
\\	宜しく
\\	"今年、「日本語能力試験」を受ける
\\	の
\\	…うん、受けるつもり
\\	だけど
\\	、君はどする
\\	の
\\	今年、「日本語能力試験」を受ける
\\	んですか
\\	…うん、受けるつもり
\\	ですが
\\	、君はどする
\\	んですか
\\	"どうして泣いている
\\	の
\\	誰かとけんかした
\\	の
\\	…ううん。お母さんがいない
\\	の
\\	どうして泣いている
\\	んですか
\\	誰かとけんかした
\\	んですか
\\	…ううん。お母さんがいない
\\	んです
\\	"どしたの?
\\	…うるさくて、目が
\\	覚め
\\	ちゃった
\\	の。
\\	どしたの?
\\	…うるさくて、目が覚め
\\	てしまった
\\	の。
\\	行っ
\\	ちゃいます
\\	行っ
\\	てしまいます
\\	読ん
\\	じゃった
\\	読ん
\\	でしまった
\\	見
\\	ちゃった
\\	見
\\	てしまった
\\	"エアコン、消そうか?
\\	。。。ううん、そのまま
\\	つけ
\\	といて
\\	エアコン、消そうか?
\\	。。。ううん、そのまま
\\	つけ
\\	ておいてください
\\	見
\\	ときます
\\	見
\\	ておきます
\\	作っ
\\	とこう
\\	作っ
\\	ておこう
\\	読ん
\\	どいて
\\	ください
\\	読ん
\\	でおいてください
\\	"ジョギング、
\\	続け
\\	てる
\\	続け
\\	てる
\\	よ。
\\	ジョギング、
\\	続け
\\	ている
\\	続け
\\	ている
\\	よ。
\\	走っ
\\	てる
\\	走っ
\\	ている
\\	飲ん
\\	でる
\\	飲ん
\\	でいる
\\	"こっち、
\\	見
\\	てない
\\	"こっち、
\\	見
\\	ていない
\\	"朝ご飯は食べたくなかったのに、
\\	食べさせられた
\\	"朝ご飯は食べたくなかったのに、
\\	(食べた
\\	"日本では、お酒を
\\	飲ませられる
\\	ことが多い。
\\	"日本では、お酒を
\\	飲む
\\	ことが多い。
\\	"二時間も
\\	待たせられた
\\	"二時間も
\\	待った
\\	"親せきに毎日宿題を
\\	させられる
\\	"親せき
\\	に
\\	毎日宿題を
\\	する
\\	"行く 
\\	行か 
\\	行かされる
\\	行かせられる
\\	"行く→ 
\\	"立つ 
\\	立た 
\\	立たされる
\\	立たせられる
\\	"立つ 
\\	"食べる 
\\	食べさせる 
\\	食べさせられる
\\	"食べる 
\\	"する 
\\	させる 
\\	させられる
\\	"する 
\\	"来[き]る 
\\	来させる 
\\	来[こ]させられる
\\	"来る 
\\	"会社の朝礼で、月に1回「思うこと」を
\\	話させられます
\\	"会社の朝礼で、月に1回「思うこと」を
\\	話します
\\	"中学校に入ったとき、校歌を
\\	覚えさせられました
\\	"中学校に入ったとき、校歌を
\\	覚えました
\\	"会社に入ったとき、3か月間
\\	敬語[けいご]の練習を
\\	させられました
\\	"会社に入ったとき、3か月間敬語の練習を
\\	しました
\\	"小学校のとき、本を読んだあと、いつも先生に感想文を
\\	書かされました
\\	"小学校のとき、本を読んだあと、いつも先生に感想文を
\\	書きました
\\	"太郎[たろう]君
\\	は
\\	先生
\\	に
\\	掃除[そうじ]を
\\	させられました
\\	"太郎君
\\	は
\\	先生
\\	に
\\	そう除を
\\	しました
\\	"地球は宇宙
\\	で最も[もっとも]美しい星の一つ
\\	である
\\	"地球は宇宙で最も美しい星の一つ
\\	だ
\\	"20世紀は戦争の世紀
\\	であった
\\	。21世紀は平和
\\	であって
\\	欲しい。
\\	"20世紀は戦争の世紀
\\	だった
\\	。21世紀は平和
\\	で
\\	欲しい。。
\\	"世界中の人々が平和を願っている
\\	のである
\\	"世界中の人々が平和を願っている
\\	のだ
\\	"失敗
\\	は成功の母
\\	である
\\	"失敗は成功の母
\\	だ
\\	"世界の若者が
\\	集まり
\\	、環境[かんきょう]問題について話し会った。
\\	世界の若者が
\\	集まって
\\	環
\\	境問題について話し会った。
\\	"北海道
\\	の7月の摩周湖[ましゅうこ]は、
\\	深く
\\	青く
\\	静かで
\\	、絵のように美しかった。
\\	北海道の7月のま周湖は、
\\	深くて
\\	青くて
\\	静かで
\\	、絵のように美しかった。
\\	"東京には様々な国の人が住んで
\\	おり
\\	、外国語を聞く機会が多い。
\\	東京には様々な国の人が住んで
\\	います
\\	、外国語を聞く機会が多い。
\\	"若いころはお金も仕事も
\\	なく
\\	、生活は苦しかった。しかし、夢があった。
\\	若いころはお金も仕事も
\\	なくて
\\	、生活は苦しかった。しかし、夢があった。
\\	"子供は友達が持っているのと同じものを
\\	欲しがります
\\	子供は友達が持っているのと同じものを
\\	欲し
\\	"母は地震のニュースを聞くと、とても
\\	不安
\\	がります
\\	母は地震のニュースを聞くと、とても
\\	不安
\\	"父は新しい製品が出ると、すぐに
\\	買いたがる
\\	父は新しい製品が出ると、すぐに
\\	買いたい
\\	"最近、結婚
\\	したがらない
\\	若者が増えています。
\\	最近、結婚
\\	したくない
\\	若者が増えています。
\\	"息子が新しい電子辞書を
\\	欲しがっています
\\	息子が新しい電子辞書を
\\	欲し
\\	"太郎[たろう]君は試合に負けて、とても
\\	残念がっていました
\\	太郎[たろう]君は試合に負けて、とても
\\	残念でした
\\	"娘は歌手に
\\	なりたがっています
\\	娘は歌手に
\\	なりたいです
\\	携帯[けいたい]電話が便利な
\\	こと
\\	は
\\	認めるが、なくてはならない物ではない。
\\	携帯[けいたい]電話が便利な
\\	は
\\	認めるが、なくてはならない物ではない。
\\	この島に空港が造られる
\\	こと
\\	について
\\	どう思いますか。
\\	この島に空港が造られる
\\	について
\\	どう思いますか。
\\	大学を出て一人で生活してみて、現実は  厳[きび]しい
\\	ということ
\\	を
\\	した。
\\	大学を出て一人で生活してみて、現実は厳しい
\\	を
\\	した。
\\	あなたを一番愛しているのはご両親だ
\\	ということ
\\	を
\\	忘れないでください。
\\	あなたを一番愛しているのはご両親だ
\\	を
\\	忘れないでください。
\\	"明日は早く出かけるから、もう
\\	寝ないと
\\	。/ もう
\\	寝なくちゃ
\\	明日は早く出かけるから、もう
\\	(寝る)
\\	。/ もう
\\	(寝る)
\\	中1
\\	"田中さんにメールの返事を
\\	しなくちゃ
\\	しないと
\\	田中さんにメールの返事を
\\	(する)
\\	(する)
\\	中1
\\	試験が終わった!今日は
\\	飲んじゃおう
\\	試験が終わった!今日は
\\	(飲んでしまおう)
\\	中1
\\	"これ洗濯
\\	しといて
\\	これ洗濯
\\	(しておいてください)____
\\	中1
\\	"今日は、
\\	春
\\	らしい
\\	暖かい日でした。
\\	今日は、
\\	春
\\	暖かい日でした。
\\	中1
\\	"私は
\\	女性
\\	らしい
\\	洋服はあまり着ない。
\\	"私は
\\	女性
\\	洋服はあまり着ない。
\\	中1
\\	"あの小学生は、
\\	大人
\\	っぽい
\\	みたい 
\\	あの小学生は、
\\	大人
\\	中1
\\	"この料理は
\\	油
\\	っぽ
\\	くて
\\	いやだ。
\\	みたい 
\\	この料理は
\\	油
\\	くて
\\	いやだ。
\\	中1
\\	"皆様、
\\	ご存じ
\\	の
\\	ように、
\\	試験の内容が変わります。
\\	とうりに 
\\	皆様、
\\	ご存じ
\\	の
\\	試験の内容が変わります。
\\	中1
\\	"前に
\\	話しした
\\	ように、
\\	来年から授業料が値上がります。
\\	とうりに 
\\	前に
\\	話しした
\\	来年から授業料が値上がります。
\\	中1
\\	"明日はもっと早く
\\	来るように
\\	""なさい
\\	明日はもっと早く
\\	(来なさい)
\\	中1
\\	"ここでタバコを
\\	吸わないように。
\\	""なさい
\\	"ここでタバコを
\\	(吸わないでください)
\\	中1
\\	"合格
\\	します
\\	ように
\\	合格
\\	(する)
\\	ように
\\	中1
\\	"母は病気が
\\	治ります
\\	ように
\\	母は病気が
\\	(治る)
\\	ように
\\	中1
\\	"この店のお客さんは、
\\	女性
\\	ばかり
\\	ですね。
\\	この店のお客さんは、
\\	女性
\\	ですね。
\\	中1
\\	"弟は、テレビを
\\	見て
\\	ばかり
\\	いる
\\	弟は、テレビを
\\	見て
\\	いる
\\	中1
\\	"このコンサートは、会員
\\	だけしか
\\	入れ
\\	ません
\\	このコンサートは、会員
\\	はいれ
\\	ません
\\	中1
\\	"今日はお客様が一人
\\	だけしか
\\	来
\\	ませんでした
\\	今日はお客様が一人
\\	来
\\	ませんでした
\\	中1
\\	"ひらがな
\\	さえ
\\	書
\\	け
\\	ない
\\	んですから、漢字なんて
\\	書
\\	け
\\	ません
\\	ひらがな
\\	書
\\	け
\\	ない
\\	んですから、漢字なんて
\\	書
\\	け
\\	ません
\\	中1
\\	"そんなこと、子供
\\	で
\\	さえ
\\	知っている。
\\	"そんなこと、子供
\\	で
\\	知っている。
\\	中1
\\	"明日
\\	こそ
\\	勉強するぞ!
\\	は 
\\	明日
\\	勉強するぞ!
\\	中1
\\	"今日は涼しい
\\	というより
\\	寒いくらいだった。
\\	というか
\\	今日は涼しい
\\	寒いくらいだった。
\\	中1
\\	"あの学生はできない
\\	というか
\\	、やる気がないのでしょう。
\\	というより
\\	あの学生はできない
\\	、やる気がないのでしょう。
\\	中1
\\	"前の車は遅すぎて、走る
\\	というより
\\	はって
\\	いるようだ。
\\	というか
\\	前の車は遅すぎて、走る
\\	はっているようだ。
\\	中1
\\	"京都
\\	というと
\\	お寺をイメージします。
\\	といえば
\\	といったら
\\	京都
\\	、お寺をイメージします。
\\	中1
\\	"日本の食べ物
\\	といえば
\\	おすしがいちばん有名だと思います。
\\	というと
\\	といったら
\\	日本の食べ物
\\	、おすしがいちばん有名だと思います。
\\	中1
\\	"夏の果物
\\	といったら
\\	やっぱりスイカだね。
\\	というと
\\	といえば
\\	夏の果物
\\	、やっぱりスイカだね。
\\	中1
\\	"週末は旅行しました。旅行
\\	といっても
\\	、近くの温泉に行っただけですが。
\\	週末は旅行しました。旅行
\\	、近くの温泉に行っただけですが。
\\	中1
\\	"キミさんの顔色がよくないのをみて、どこか体が悪い
\\	のじゃないかと思った
\\	キミさんの顔色がよくないのをみて、どこか体が悪い
\\	中1
\\	"自転車の事故が増えたことを聞いて、規則を守らない人が多い
\\	のじゃないかと思った
\\	自転車の事故が増えたことを聞いて、規則を守らない人が多い
\\	中1
\\	"散歩
\\	の
\\	ついでに
\\	、この手紙を出してきてくれませんか。
\\	散歩
\\	の
\\	、この手紙を出してきてくれませんか。
\\	中1
\\	"郵便局へ行った
\\	ついでに
\\	、はがきを買ってきた。
\\	郵便局へ行った
\\	、はがきを買ってきた。
\\	中1
\\	"買い物の
\\	たび[度]に
\\	、袋をたくさんもらう。
\\	買い物の
\\	、袋をたくさんもらう。
\\	中1
\\	"この曲を聞く
\\	たびに
\\	、ふるさとを思い出す。
\\	この曲を聞く
\\	、古里を思い出す。
\\	中1
\\	"窓を開け
\\	た
\\	とたん
\\	、強い風が入ってきた。 
\\	途端[とたん]
\\	窓を開け
\\	た
\\	、強い風が入ってきた。
\\	中1
\\	"お酒を飲ん
\\	だ
\\	とたん
\\	、顔が赤くなった。 
\\	途端[とたん]
\\	お酒を飲ん
\\	だ
\\	、顔が赤くなった。
\\	中1
\\	"食事の
\\	最中[さいちゅう]に
\\	、お客さんがきた。
\\	食事の
\\	、お客さんがきた。
\\	中1
\\	"会議をしている
\\	最中[さいちゅう]に
\\	、携帯電話が鳴った。
\\	会議をしている
\\	、携帯電話が鳴った。
\\	中1
\\	2足す3は5
\\	2プラス3イコール5
\\	足し算 
\\	中1
\\	5引く3は2
\\	5マイナス3イコール2
\\	引き算 
\\	中1
\\	4かける2は8
\\	4かける2イコール8
\\	かけ算 
\\	中1
\\	8わる4は2
\\	8わる4イコール2
\\	割り算 
\\	中1
\\	"あなた
\\	にとって
\\	、いちばん大切なものは何ですか。
\\	あなた
\\	、いちばん大切なものは何ですか。
\\	中2
\\	"このあたりの人々
\\	にとっては
\\	、車は
\\	必需品[ひつじゅひん]です。
\\	このあたりの人々
\\	、車は必需品
\\	です。
\\	中2
\\	"年の
\\	わりには
\\	若く見える。
\\	年の
\\	若く見える。
\\	中2
\\	"お金がない
\\	(と言っている)
\\	わりに
\\	、よく買い物をするね。
\\	お金がない
\\	、よく買い物をするね。
\\	中2
\\	"知っている
\\	くせに
\\	、教えてくれない。
\\	のに
\\	知っている
\\	、教えてくれない。
\\	中2
\\	"元気
\\	な
\\	くせに
\\	、病気のふりをしている。
\\	のに
\\	元気
\\	な
\\	、病気のふりをしている。
\\	中2
\\	"お化粧
\\	なんか
\\	してはいけません。
\\	お化粧する
\\	なんて
\\	はいけません。
\\	お化粧
\\	してはいけません。
\\	中2
\\	"納豆[なっとう]
\\	なんて
\\	嫌いだ。
\\	納豆
\\	嫌いだ。
\\	中2
\\	"「泣いているの?」「泣いて
\\	なんか
\\	いないよ!」
\\	「泣いているの?」「泣いて
\\	いないよ!」
\\	中2
\\	"日本語でスピーチ
\\	など
\\	できません。 
\\	等[など]
\\	日本語でスピーチ
\\	できません。
\\	中2
\\	"先生
\\	の
\\	おかげで
\\	合格
\\	できました
\\	先生の
\\	合格できました。
\\	中2
\\	"日本へ来た
\\	おかげで
\\	、日本語に上手になった。
\\	日本へ来た
\\	、日本語に上手になった。
\\	中2
\\	"バスが送れた
\\	せいで
\\	、約束の時間に間に合わなかった。
\\	バスが送れた
\\	、約束の時間に間に合わなかった。
\\	中2
\\	"疲れた
\\	せいか
\\	、頭が痛い。 
\\	せい[所為]で
\\	疲れた
\\	、頭が痛い。
\\	中2
\\	"車で来たので、ビール
\\	の
\\	かわりに
\\	ジュースをください。
\\	車で来たので、ビールの
\\	ジュースをください。
\\	中2
\\	"日曜日に働いた
\\	かわりに
\\	、今日休みを取りました。
\\	日曜日に働いた
\\	、今日休みを取りました。
\\	中2
\\	"出張中の部長
\\	にかわって
\\	、私がごあいさつさせていただきます。 
\\	にかわり
\\	出張中の部長
\\	、私がごあいさつさせていただきます。
\\	中2
\\	"野球
\\	にかわり
\\	、サッカーがさかんになってきた。 
\\	にかわって
\\	野球
\\	、サッカーが盛んになってきた。
\\	中2
\\	"それは
\\	米粒[こめぶつ]
\\	くらいの
\\	大きさです。
\\	それは米粒
\\	大きさです。
\\	中2
\\	"この洗剤は、おもしろい
\\	ほど
\\	汚れが落ちる。
\\	この洗剤は、おもしろい
\\	汚れが落ちる。
\\	中2
\\	"あきる
\\	ほど
\\	食べた。
\\	あきる
\\	食べた。
\\	中2
\\	"宿題が多すぎて、泣きたい
\\	ぐらいだ
\\	宿題が多すぎて、泣きたい
\\	中2
\\	"若い人
\\	ほど
\\	朝寝坊をする。
\\	若い人
\\	朝寝坊をする。
\\	中2
\\	"伝統的[でんとてき]なものは、古い
\\	ほど
\\	価値[かち]がある。
\\	伝統的なものは、古い
\\	価値[かち]がある。
\\	中2
\\	"彼のことを知れ
\\	ば
\\	知る
\\	ほど
\\	好きになる。
\\	彼のことを知れ
\\	知る
\\	好きになる。
\\	中2
\\	"荷物は少
\\	なければ
\\	少ない
\\	ほど
\\	いい。
\\	荷物は少
\\	少ない
\\	いい。
\\	中2
\\	"彼女
\\	くらい
\\	ほど
\\	親切な人
\\	はいない
\\	彼女
\\	親切な人
\\	いる
\\	中2
\\	"今年
\\	ほど
\\	くらい
\\	雪の降った年
\\	はなかった
\\	今年
\\	雪の降った年
\\	あった
\\	中2
\\	"君が謝る
\\	ことはない
\\	よ。
\\	君が謝る
\\	よ。
\\	"来る
\\	ことはありません
\\	来る
\\	は_____
\\	"田中さんから電話があって、少し遅れる
\\	ということです
\\	田中さんから電話があって、少し遅れる
\\	"試験の結果は70
\\	、つもり合格
\\	ということです
\\	試験の結果は70
\\	、つもり合格
\\	"暖かくして、ゆっくり休む
\\	ことだ
\\	暖かくして、ゆっくり休む
\\	"無理をしない
\\	ことです
\\	無理をしない
\\	"合格できたら、
\\	どんなに
\\	うれしい
\\	ことか
\\	合格できたら、
\\	どんなに
\\	うれしい
\\	何度
\\	注意した
\\	ことか
\\	何度
\\	注意した
\\	"国へ帰るのは、来週だ(った)
\\	っけ
\\	国へ帰るのは、来週だ
\\	中2
\\	"明日のパーティーのこと、話し(まし)た
\\	っけ
\\	明日のパーティーのこと、話した
\\	中2
\\	"できるまで、やる
\\	しかない
\\	できるまで、やる
\\	中2
\\	"直せないから、新しいのを買う
\\	しかなかった
\\	直せないから、新しいのを買う
\\	かった
\\	中2
\\	"「田中さん、結婚してる
\\	んだって
\\	。」「へー、知らなかった。」
\\	「田中さん、結婚してる
\\	。」「へー、知らなかった。」
\\	中2
\\	"「試験
\\	範囲[はんい]は、教科書の最初から50パージまで
\\	だって
\\	「試験
\\	範囲[はんい]は、教科書の最初から50パージまで
\\	中2
\\	"「どうして食べないの?」「だって、まずい
\\	んだもん
\\	「どうして食べないの?」「だって、まずい
\\	中2
\\	"今日の試験、できなかった……。勉強しなかった
\\	んだもん
\\	、しかたがない。
\\	今日の試験、できなかった……。勉強しなかった
\\	、しかたがない。
\\	中2
\\	"もっと日本語がうまく話せ
\\	たらいいなあ
\\	もっと日本語がうまく話せ
\\	なあ
\\	中2
\\	"明日、雨が降らない
\\	といいなあ
\\	明日、雨が降らない
\\	なあ
\\	中2
\\	"遅刻してしまった。もっと早く家を出れ
\\	ばよかった
\\	遅刻してしまった。もっと早く家を出れ
\\	中2
\\	"田中さんにあんなことを言わな
\\	ければよかった
\\	田中さんにあんなことを言わな
\\	けれ_____
\\	中2
\\	"パーティー、楽しかったよ。君も行け
\\	ば
\\	よかった
\\	のに
\\	パーティー、楽しかったよ。君も行け
\\	よかった
\\	中2
\\	"安かっ
\\	たら
\\	買う
\\	のに
\\	安かっ
\\	買う
\\	中2
\\	"バス、早く来ない
\\	かなあ
\\	バス、早く来ない
\\	中2
\\	"この試験、うまくいく
\\	かなあ
\\	この試験、うまくいく
\\	中2
\\	"今日、富士山が見える
\\	かなあ
\\	今日、富士山が見える
\\	中2
\\	"あの人は寂し
\\	げ
\\	な目をしている。
\\	""そうな/に
\\	あの人は寂し
\\	な目をしている。
\\	中3
\\	"彼は何か言いた
\\	げ
\\	だった。(言いたい+そう)
\\	""そうな/に
\\	"彼は何か言いた
\\	だった。
\\	中3
\\	"私は子供
\\	の頃[ころ]、病気
\\	がち
\\	だった。
\\	私は子供のころ、病気
\\	だった。
\\	中3
\\	"最近、彼は仕事を休み
\\	がち
\\	だ。
\\	最近、彼は仕事を休み
\\	だ。
\\	中3
\\	"年を取ると、忘れ
\\	っぽく
\\	なる。
\\	年を取ると、忘れ
\\	なる。
\\	中3
\\	"このコート、デザインはいいけれど
\\	、生地[きじ]が安
\\	っぽい
\\	ね。
\\	このコート、デザインはいいけれど
\\	、生地[きじ]が安
\\	ね。
\\	中3
\\	"残業続きで疲れ
\\	気味[ぎみ]
\\	だ。
\\	残業続きで疲れ
\\	だ。
\\	中3
\\	"新入社員
\\	は緊張
\\	気味[ぎみ]
\\	の顔をしていた。
\\	新入社員
\\	は緊張
\\	の顔をしていた。
\\	中3
\\	"上手に泳ぐ人を見た
\\	とき
\\	、「魚のようだ」
\\	と言います
\\	上手に泳ぐ人を見た
\\	、「魚のようだ」
\\	中2
\\	たとえ
\\	猫が手伝ってくれ
\\	ても
\\	、それほど役に立つ
\\	とは
\\	思えません。
\\	猫が手伝ってくれ
\\	、それほど役に立つ
\\	思えません。
\\	中2
\\	たとえ
\\	高い物を使っ
\\	ても
\\	、それほど
\\	直れる
\\	とは
\\	思えません。
\\	高い物を使っ
\\	、それほど
\\	直れる
\\	思えません。
\\	中2
\\	"面白い言い方はいろいろありますが、
\\	「猫
\\	の小判[こばん]」という言い方
\\	ほど
\\	面白いもの
\\	はない
\\	でしょう。
\\	面白い言い方はいろいろありますが、「猫
\\	の小判[こばん]」という言い方
\\	面白いもの
\\	でしょう。
\\	中2
\\	"私の国に山はたくさんあるのですが、
\\	「モンブラン」
\\	ほど
\\	高い山
\\	はない
\\	でしょう。
\\	私の国に山はたくさんあるのですが、「モンブラン」
\\	高い山
\\	でしょう。
\\	中2
\\	"昔からの友達から招待状が来る
\\	と
\\	家族のだれかが結婚する
\\	ようで
\\	うれしい
\\	です。
\\	昔からの友達から招待状が来る
\\	、家族のだれかが結婚する
\\	うれしいです。
\\	中2
\\	"スポーツをする
\\	と
\\	ボルトになる
\\	ようで
\\	元気になります
\\	スポーツをする
\\	、ボルトになる
\\	元気になります。
\\	中2
\\	"母は父がいないと楽だと言っていました。もちろんこれは冗談で、本当は
\\	寂し
\\	がっているんです
\\	。父が電話をかけてくると喜ぶ
\\	ので
\\	分かります。
\\	母は父がいないと楽だと言っていました。もちろんこれは冗談で、本当は
\\	(寂しい)
\\	んです。父が電話をかけてくると喜ぶ
\\	分かります。
\\	中2
\\	できるだけ
\\	手紙を書く
\\	ようにしています
\\	できるだけ
\\	手紙を書く
\\	しています。
\\	中2
\\	"傘を持って行かないと、雨が降られるかもしれませんよ。
\\	…傘
\\	なんて
\\	いらない
\\	"傘を持って行かないと、雨が降られるかもしれませんよ。
\\	…傘
\\	いらない
\\	中2
\\	"辞典がなければこれは読めないでしょう。
\\	…辞典
\\	なんて
\\	なくてもいいです
\\	辞典がなければこれは読めないでしょう。
\\	…辞典
\\	なくてもいいです
\\	中2
\\	"御飯のとき
\\	いつもうちに帰っているかな
\\	とか
\\	この番組を見てるかな
\\	とか
\\	言っています。
\\	御飯のとき
\\	いつもうちに帰っているかな
\\	この番組を見てるかな
\\	言っています。
\\	中2
\\	"子供がマッチで遊んでいたらどうしますか。
\\	…すぐ
\\	やめさせます
\\	使役形[しえきけい]	
\\	子供がマッチで遊んでいたらどうしますか。
\\	…すぐ
\\	(やめます)
\\	中3
\\	"今日は八時まで仕事をしてくれますか。
\\	…すみません。今日は疲れているので、
\\	帰らせてください
\\	使役形[しえきけい]	
\\	今日は八時まで仕事をしてくれますか。
\\	…すみません。今日は疲れているので、
\\	(帰る)
\\	ください
\\	中3
\\	"日本に
\\	来た
\\	ばかり
\\	なので、何も分かりません。
\\	日本に来た
\\	なので、何も分かりません。
\\	中3
\\	"食事
\\	した
\\	ばかり
\\	なのに、もうお腹が空いてしまいました。
\\	食事した
\\	なのに、もうお腹が空いてしまいました。
\\	中3
\\	"どうして我慢できるのだろうかと不思議
\\	でたまりませんでした
\\	どうして我慢できるのだろうかと不思議
\\	(たまらない→
\\	中3
\\	"子供が生まれたのが、うれし
\\	くてたまりません
\\	子供が生まれたのが、うれし
\\	(たまらない→ 
\\	中3
\\	"日本人は礼儀正しい
\\	と聞いていた
\\	のに、そう
\\	でも
\\	ありませんでした。
\\	日本人は礼儀正しい
\\	のに、そう
\\	ありませんでした。
\\	中3
\\	"タイの料理は辛い
\\	と聞いていた
\\	のですが、
\\	やはり
\\	そうでした。
\\	タイの料理は辛い
\\	のですが、
\\	そうでした。
\\	中3
\\	帰れる
\\	ものなら
\\	、今すぐ、国へ
\\	帰りたい
\\	(帰る)___
\\	、今すぐ、国へ
\\	帰りたい
\\	中3
\\	やれる
\\	もんなら
\\	やってみろ
\\	(やる)
\\	やってみろ
\\	中3
\\	"上着を脱いでもいいですか。暑い
\\	ものですから
\\	から, 
\\	上着を脱いでもいいですか。暑い
\\	中3
\\	"遅くなってごめん。道路が
\\	混[こ]んでいた
\\	もんだから
\\	から, 
\\	遅くなってごめん。道路が
\\	混[こ]んでいた
\\	中3
\\	"パーティーには行かなかったよ。
\\	女 
\\	子 
\\	だって
\\	、知らなかった(んだ)
\\	もの
\\	男:
\\	知らなかった(んだ)
\\	し
\\	女 
\\	子
\\	パーティーには行かなかったよ。
\\	だって
\\	、知らなかった
\\	中3
\\	女 
\\	子 
\\	しょうがないよ。子ども(なん)だ
\\	もん
\\	男 
\\	子どもだ
\\	し
\\	女 
\\	子
\\	しょうがないよ。子どもだ
\\	中3
\\	"車の免許
\\	は
\\	持っている
\\	ものの
\\	、ほとんど運転したがない。
\\	車の免許は持っている
\\	、ほとんど運転したがない。
\\	中3
\\	"申し込み
\\	は
\\	した
\\	ものの
\\	、試験を受けるかどか
\\	未定[みてい]だ。
\\	申し込み
\\	は
\\	した
\\	、試験を受けるかどか
\\	未定[みてい]だ。
\\	中3
\\	"春と
\\	は
\\	いう
\\	ものの
\\	、まだ寒い。
\\	春とはいう
\\	、まだ寒い。
\\	中3
\\	"うちには、
\\	車
\\	はもとより
\\	自転車
\\	も
\\	ないな。
\\	うちには、
\\	車
\\	自転車
\\	ないな。
\\	中3
\\	"レタスは
\\	サラダで
\\	食
\\	べるの
\\	はもとより
\\	いためて
\\	も
\\	おいしい。
\\	レタスは
\\	サラダで食べるの
\\	は____
\\	いためて
\\	おいしい。
\\	中3
\\	"母の料理は
\\	見た目
\\	はともかく
\\	味
\\	は
\\	おいしい。
\\	母の料理は
\\	見た目
\\	味
\\	おいしい。
\\	中3
\\	"あの
\\	女優[じょゆう]は、
\\	顔
\\	は
\\	ともかくして
\\	演技[えんぎ 
\\	が
\\	すばらしい。
\\	あの
\\	女優[じょゆう]は、
\\	顔は
\\	して
\\	演技[えんぎ]
\\	すばらしい。
\\	中3
\\	"私の英語力では、
\\	旅行
\\	はまだしも
\\	留学
\\	なんて無理です。
\\	私の英語力では、
\\	旅行
\\	は____
\\	留学
\\	なんて無理です。
\\	中3
\\	10分や20分
\\	ならまだしも
\\	1時間
\\	も待てません。
\\	10分や20分
\\	なら____
\\	1時間
\\	も待てません
\\	中3
\\	"皆さん、
\\	仕事の話
\\	は抜[ぬ]きにして
\\	楽しく飲みましょう。
\\	皆さん、
\\	仕事の話
\\	して
\\	楽しく飲みましょう。
\\	中3
\\	お世辞[せじ
\\	抜[ぬ]きに
\\	、君の日本語は本当にうまいよ。
\\	お世辞[せじ]
\\	、君の日本語は本当にうまいよ。
\\	中3
\\	田中さん
\\	抜[ぬ]きで(は)
\\	パーティーは
\\	始まれません
\\	田中さん
\\	パーティーは
\\	始まれません
\\	中3
\\	"子供のことが心配
\\	でたまらない
\\	子供のことが心配
\\	で_____
\\	中3
\\	"家族に会いたく
\\	てたまらない
\\	家族に会いたく
\\	て_____
\\	中3
\\	今日は何もすることがなくて、ひま
\\	でしょうがない
\\	で仕方[しかた]がない
\\	今日は何もすることがなくて、ひま
\\	で______
\\	中3
\\	そんな方法では、時間がかかっ
\\	てしょうがない
\\	で仕方[しかた]がない
\\	そんな方法では、時間がかかっ
\\	て______
\\	中3
\\	"隣の工事がうるさく
\\	てかなわない
\\	隣の工事が
\\	うるさく
\\	て_____
\\	中3
\\	"私の家は、駅から遠くて不便
\\	でかわらない
\\	私の家は、駅から遠くて不便
\\	で_____
\\	中3
\\	"友人の結婚式に出席できないが、残念
\\	でならない
\\	友人の結婚式に出席できないが、残念
\\	で____
\\	中3
\\	"チョウさんは幸せ
\\	でならない
\\	チョウさんは幸せ
\\	で____
\\	中3
\\	"食事の時は、話は
\\	しない
\\	ものです
\\	食事の時は、話は
\\	(する)
\\	中3
\\	"プレゼントはくれた人の前で
\\	あける
\\	ものです
\\	プレゼントはくれた人の前で
\\	あける
\\	中3
\\	"トマトの皮のような小さいことでもいろいろ
\\	違う
\\	ものです
\\	ね。
\\	トマトの皮のような小さいことでもいろいろ
\\	違う
\\	ね。
\\	中3
\\	"「へえ、三か月でずいぶん
\\	細くなる
\\	ものです
\\	ね。」
\\	「へえ、三か月でずいぶん
\\	細くなる
\\	ね。」
\\	中3
\\	"世界は広いのだから、料理も食べ方もいろいろ
\\	ある
\\	のは当たり前だ
\\	世界は広いのだから、料理も食べ方もいろいろ
\\	ある
\\	のは_____
\\	中3
\\	"あの人は世界中旅行しているのだから、
\\	お金持ち
\\	な
\\	のは当たり前だ
\\	あの人は世界中旅行しているのだから、
\\	お金持ち
\\	のは_____
\\	中3
\\	"食べ
\\	てみると
\\	思っていたより
\\	おいしかったということもよくある。
\\	食べて
\\	と
\\	(思う)
\\	おいしかったということもよくある。
\\	中3
\\	"住みにくい国だと思っていたが、
\\	行って
\\	みると
\\	思っていたより
\\	住みやすかった。
\\	住みにくい国だと思っていたが、
\\	行って
\\	と
\\	(思う)
\\	住みやすかった。
\\	中3
\\	どうしたら
\\	その土地の人と友達になれるの
\\	か
\\	と聞いてみると、「そこの人たちの食べるものお一緒にたべる
\\	ことだ
\\	という答えが返ってきた。
\\	その土地の人と友達になれるの
\\	と聞いてみると、「そこの人たちの食べるものお一緒にたべる
\\	という答えが返ってきた。
\\	中3
\\	"先生に、
\\	どうしたら
\\	いい成績をもらうの
\\	か
\\	と聞いてみると、「なまけない
\\	ことだ
\\	という答えが返ってきた。
\\	先生に、
\\	いい成績をもらうの
\\	と聞いてみると、「なまけない
\\	という答えが返ってきた。
\\	中3
\\	"食べ
\\	ないことはない
\\	が、あまり好きじゃない。
\\	食べ
\\	ない_____
\\	が、あまり好きじゃない。
\\	中3
\\	"お酒やタバコをやめるのは難しいが、
\\	やめられ
\\	ないことはない
\\	お酒やタバコをやめるのは難しいが、
\\	やめられ
\\	ない_____
\\	中3
\\	"毎日、漢字を4つか5つなら、
\\	覚えられ
\\	ないこともない
\\	毎日、漢字を4つか5つなら
\\	、覚えられ
\\	ない_____
\\	中3
\\	"この会社を辞める人の気持ちがわから
\\	ないこともない
\\	この会社を辞める人の気持ちがわから
\\	ない_____
\\	中3
\\	"態度[たいど]の悪い店員に
\\	、一言[ひとこと]文句を
\\	言
\\	わないではいられなかった
\\	"態度[たいど]の悪い店員に、一言文句を
\\	言
\\	わない______かった
\\	中3
\\	"工事の音がうるさくて、耳を
\\	ふさ
\\	がないではいられない
\\	工事の音がうるさくて、耳を
\\	ふさ
\\	がない_______
\\	中3
\\	"職場でいやなことがあると、酒を
\\	飲まずにはいられない
\\	職場でいやなことがあると、酒を
\\	(飲む)
\\	には_____
\\	中3
\\	"あの映画を見たら、誰でも
\\	感動[かんどう]
\\	せずにはいられない
\\	あの映画を見たら、誰でも
\\	感動[かんどう]
\\	(します)
\\	には_____
\\	中3
\\	"ビザが切れたので、国に
\\	帰らねばならない
\\	ビザが切れたので、国に
\\	(帰る)
\\	ならない
\\	中3
\\	"それが真実であることを
\\	確かめねばならない
\\	それが真実であることを
\\	(確かめる)
\\	ならない
\\	中3
\\	"この悲惨[ひさん]な体験[たいけん]を決[けっ]して
\\	忘れ
\\	てはならない
\\	この悲惨な体験を決して
\\	(忘れます)
\\	ならない
\\	中3
\\	"失敗を人のせいに
\\	してはなりません
\\	失敗を人のせいに
\\	(します)
\\	なりません
\\	中3
\\	"田中君、遅いなあ。もう
\\	待っ
\\	てられない
\\	から先に行こう。
\\	田中君、遅いなあ。もう
\\	待っ
\\	て____
\\	から先に行こう。
\\	中3
\\	"忙しくてテレビなんか
\\	見
\\	てらんない
\\	見
\\	てられない
\\	"忙しくてテレビなんか
\\	見
\\	て____
\\	中3
\\	"連休だけれど、もうすぐ試験があるから、
\\	遊ん
\\	でばかりはいられない
\\	連休だけれど、もうすぐ試験があるから、
\\	遊ん
\\	で___は_____
\\	中3
\\	"もう親に
\\	頼[たよ]っ
\\	てばかりはいられない
\\	もう親に
\\	頼[たよ]っ
\\	て___は_____
\\	中3
\\	"努力
\\	のかいが
\\	あって
\\	、希望の大学に合格した。
\\	努力
\\	の____って
\\	、希望の大学に合格した。
\\	中4
\\	"すてきな人と結婚できた。この年まで
\\	待った
\\	かいがあった
\\	すてきな人と結婚できた。この年まで待った
\\	った
\\	中4
\\	"手術
\\	のかいもなく
\\	、愛犬[あいけん]が死んでしまった。
\\	かいがなく
\\	手術
\\	の____く
\\	、愛犬が死んでしまった。
\\	中4
\\	"パクさんは彼女がいないからいつも
\\	生きる
\\	かいがない
\\	と言っている。
\\	かいもない
\\	パクさんは彼女がいないからいつも生きる
\\	い
\\	と言っている。
\\	中4
\\	やりがい
\\	のある仕事
\\	(やります)
\\	がい
\\	のある仕事
\\	中4
\\	教えがい
\\	のある生徒
\\	(教えます)
\\	がい
\\	のある生徒
\\	中4
\\	"借金
\\	してまで
\\	海外旅行に行くなんて、私には信じられない。
\\	借金
\\	までして
\\	借金
\\	(します)
\\	海外旅行に行くなんて、私には信じられない。
\\	中4
\\	"いやな思い
\\	までして
\\	、彼女と付き合うことないよ。
\\	いやな思い
\\	、彼女と付き合うことないよ。
\\	中4
\\	"親にうそを
\\	ついてまで
\\	遊びに行きたくない。
\\	親にうそを
\\	(つきます)
\\	遊びに行きたくない。
\\	中4
\\	"「柔道」は、長く伝えられてきた形や決まりを大切に
\\	してきました
\\	「柔道」は、長く伝えられてきた形や決まりを大切に
\\	(します)
\\	中4
\\	"母は父がなくなってから、一人で店を
\\	開いてきました
\\	母は父がなくなってから、一人で店を
\\	(開きます)
\\	中4
\\	"その技術を使う人の心の持ち方
\\	まで
\\	も問題にしているのです。
\\	その技術を使う人の心の持ち方
\\	も問題にしているのです。
\\	中4
\\	"とてもお腹が空いていてうちの猫
\\	まで
\\	食べてしまいました。
\\	とてもお腹が空いていてうちの猫
\\	食べてしまいました。
\\	中4
\\	"いろいろな所へ旅行ができていいですね。
\\	…ええ、でも、旅行
\\	といっても
\\	、仕事で行くの大変です。
\\	いろいろな所へ旅行ができていいですね。
\\	…ええ、でも、旅行
\\	、仕事で行くの大変です。
\\	中4
\\	"彼はなかなか来ませんね。もう三十分も過ぎたのに。
\\	もしかすると
\\	時間を間違えた
\\	の
\\	かもしれません。
\\	彼はなかなか来ませんね。もう三十分も過ぎたのに。
\\	ると
\\	時間を間違えた
\\	かもしれません。
\\	中4
\\	"あの人、最近元気がなくて、寂しそうですね。
\\	もしかすると
\\	ふられた
\\	の
\\	かもしれません。
\\	あの人、最近元気がなくて、寂しそうですね。
\\	ると
\\	ふられた
\\	かもしれません。
\\	中4
\\	"読み
\\	かけの
\\	本が何冊もある。
\\	読み
\\	本が何冊もある。
\\	中4
\\	"彼女は何か言い
\\	かけて
\\	やめた。
\\	彼女は何か言い
\\	やめた。
\\	中4
\\	"私は、子供のとき、病気で死に
\\	かけた
\\	ことがある。
\\	私は、子供のとき、病気で死に
\\	ことがある。
\\	中4
\\	"この小説はおもしろくて、1日で読み
\\	切った
\\	この小説はおもしろくて、1日で
\\	(読みます)
\\	中4
\\	"彼は
\\	疲れ切った
\\	顔をしている。
\\	彼は
\\	(疲れます)
\\	た
\\	顔をしている。
\\	中4
\\	"博士でも間違うことは
\\	あり
\\	える
\\	ありうる
\\	博士でも間違うことはあり
\\	中4
\\	"これが私の
\\	知り
\\	うる
\\	すべての情報です。
\\	える
\\	これが私の知り
\\	すべての情報です。
\\	中4
\\	"中国語がわかるのは
\\	あり
\\	えない
\\	中国語がわかるのは
\\	あり
\\	中4
\\	"大変な仕事でも、
\\	やり
\\	抜き
\\	自信があります。
\\	大変な仕事でも、やり
\\	自信があります。
\\	中4
\\	"彼はあきらめずにゴールまで
\\	走り
\\	抜いた
\\	彼はあきらめずにゴールまで
\\	走り
\\	た
\\	中4
\\	"特別料金は五百円だから、コーヒーを一杯飲んだ
\\	つもりで
\\	、特急電車に乗った。
\\	特別料金は五百円だから、コーヒーを一杯飲んだ
\\	、特急電車に乗った。
\\	中4
\\	"新しい車を買った
\\	つもりで
\\	、銀行にお金を入れた。
\\	新しい車を買った
\\	、銀行にお金を入れた。
\\	中4
\\	"死んだ
\\	つもりで
\\	、頑張ります。
\\	死んだ
\\	、頑張ります。
\\	中4
\\	"店員は優しく言った
\\	つもりだったが
\\	、「~しなさい」と聞こえてしまった。
\\	店員は優しく言った
\\	だった_
\\	、「~しなさい」と聞こえてしまった。
\\	中4
\\	"確かめた
\\	つもりだったが
\\	、間違えてしまった。
\\	確かめた
\\	だった_
\\	、間違えてしまった。
\\	中4
\\	"あまり飲めないのに、
\\	飲まされました
\\	飲ませられました
\\	あまり飲めないのに、
\\	(飲む)
\\	中4
\\	"歌で上手ではないのに、
\\	歌わされた
\\	歌わせられた
\\	歌で上手ではないのに、
\\	(歌う)
\\	中4
\\	"アナウンスの言葉を聞いて、言葉の使い方を
\\	考えさせられた
\\	アナウンスの言葉を聞いて、言葉の使い方を
\\	(考える)
\\	中4
\\	"頑張っている人を見て、自分も頑張ろうという気持ちに
\\	させられた
\\	頑張っている人を見て、自分も頑張ろうという気持ちに
\\	(する)
\\	中4
\\	"忘れない
\\	うちに
\\	、メモしておこう。
\\	忘れない
\\	、メモしておこう。
\\	中4
\\	"桜がきれいな
\\	うちに
\\	、お花見に行きましょう。
\\	桜がきれいな
\\	、お花見に行きましょう。
\\	中4
\\	"授業が
\\	終わる
\\	か
\\	終わらない
\\	かのうちに
\\	、彼は教室を出た。
\\	授業が
\\	終わる
\\	終わらない
\\	、彼は教室を出た。
\\	中4
\\	"布団に
\\	入る
\\	か
\\	入らない
\\	かのうちに
\\	眠ってしまう。
\\	布団に
\\	入る
\\	入らない
\\	眠ってしまう。
\\	中4
\\	"日本にいる
\\	限り
\\	、日本語は必要だ。
\\	日本にいる
\\	、日本語は必要だ。
\\	中4
\\	"彼女がいる
\\	限り
\\	、ゲームでも遊びができない。
\\	彼女がいる
\\	、ゲームでも遊びができない。
\\	中4
\\	"70歳以上の方
\\	に限り
\\	、入場無料。
\\	70歳以上の方
\\	、入場無料。
\\	中4
\\	"本日
\\	に限り
\\	半額[はんがく]です。
\\	本日
\\	、半額です。
\\	中4
\\	"最近は、女性
\\	に限らず
\\	男性
\\	も
\\	化粧をする。
\\	最近は、女性
\\	男性
\\	化粧をする。
\\	中4
\\	"雨だ。今日
\\	に限って
\\	傘を持ってこなかった。
\\	雨だ。今日
\\	傘を持ってこなかった。
\\	中4
\\	"うちの子
\\	に限って
\\	悪いことをするはずがない。
\\	うちの子
\\	悪いことをするはずがない。
\\	中4
\\	食べずに
\\	働く。
\\	(食べる)
\\	に
\\	働く。
\\	中4
\\	"準備運動を
\\	せずに
\\	泳ぐと危険です。
\\	準備運動を
\\	(する)
\\	に
\\	泳ぐと危険です。
\\	中4
\\	"忙しくて寝る時間
\\	さえ
\\	ないのに、遊びに行けるわけがない。
\\	も 
\\	忙しくて寝る時間
\\	ないのに、遊びに行けるわけがない。
\\	中4
\\	泣い
\\	てなんか
\\	いないよ。
\\	(泣く)
\\	いないよ。
\\	中4
\\	"「ちょっとお茶でも飲んで休もうか。」
\\	「お茶なんか要らない、
\\	休ん
\\	でなんか
\\	いられないよ。」
\\	「ちょっとお茶でも飲んで休もうか。」
\\	「お茶なんか要らない、
\\	(休む)
\\	いられないよ。」
\\	中4
\\	"もうバスはない。歩いて
\\	しかない
\\	もうバスはない。歩いて
\\	中4
\\	"誰もいない。私がやる
\\	しかない
\\	誰もいない。私がやる
\\	中4
\\	"隅田[すみだ]川
\\	において
\\	花火大会が行われるため、交通機関は混雑するでしょう。
\\	隅田川
\\	花火大会が行われるため、交通機関は混雑するでしょう。
\\	中4
\\	"市民文化センター
\\	において
\\	、交流会が開かれます。
\\	市民文化センター
\\	、交流会が開かれます。
\\	中4
\\	"お盆休みのラッシュ
\\	に加えて
\\	事故まで起こり、高速道路はひどい渋滞になった。
\\	お盆[ぼん]休みのラッシュ
\\	事故まで起こり、高速道路はひどい渋滞になった。
\\	中4
\\	"兄は頭のいいの
\\	に加えて
\\	、スポーツもよくできる。
\\	兄は頭のいいの
\\	、スポーツもよくできる。
\\	中4
\\	"コーヒーの値段は店
\\	によって
\\	違う。
\\	コーヒーの値段は店
\\	違う。
\\	中4
\\	"「おかしいなあ、10個買った
\\	はず
\\	なのに。」
\\	「あ、ごめん、1個食べちゃった。」
\\	「おかしいなあ、10個買った
\\	なのに。」
\\	「あ、ごめん、1個食べちゃった。」
\\	中4
\\	"あの子は今年
\\	二十歳[はたち]
\\	の
\\	はず
\\	だ。
\\	あの子は今年
\\	二十歳の
\\	だ。
\\	中4
\\	"こちらで
\\	おかけになって
\\	お待ち下さい。
\\	こちらで
\\	(かける)
\\	お待ち下さい。
\\	中4
\\	"「お客様
\\	、何[なに]に
\\	なさいます
\\	か。」
\\	「ランチ、お願いします。」
\\	「お客様、何に
\\	(する)
\\	か。」
\\	「ランチ、お願いします。」
\\	中4
\\	"社長はゴルフを
\\	なさいます
\\	が、私は
\\	いたしません
\\	社長はゴルフを
\\	(します)
\\	が、私は
\\	(しません)
\\	中4
\\	"戦争が
\\	なくなりますように
\\	戦争が
\\	(なくなる)
\\	ように
\\	中4
\\	"「ここは写真を
\\	とっちゃ
\\	いけないんだよ。」
\\	「あ、そうなんだ。」
\\	"「ここは写真を
\\	(とっては)
\\	いけない
\\	んだよ。」
\\	「あ、そうなんだ。」
\\	中4
\\	"それ、
\\	触っちゃいけない
\\	よ。
\\	それ、
\\	(さわいでは)
\\	いけない
\\	よ。
\\	中4
\\	"あんなやり方ではうまくいかない
\\	だろうと思う
\\	あんなやり方ではうまくいかない
\\	と思う
\\	中4
\\	"あなた
\\	さえ
\\	いれ
\\	ば
\\	、幸せ
\\	です。
\\	あなた
\\	いれ
\\	、幸せです。
\\	中4
\\	"携帯電話を買った。
\\	これ
\\	さえ
\\	あれ
\\	ば
\\	、時計も
\\	カメラもいらない。
\\	携帯電話を買った。これ
\\	あれ
\\	、時計もカメラもいらない。
\\	中4
\\	"この薬を
\\	飲み
\\	さえ
\\	すれ
\\	ば
\\	、すぐに治
\\	ります。
\\	この薬を
\\	飲み
\\	すれ
\\	、すぐに治ります。
\\	中4
\\	"子供がかわいい
\\	からこそ
\\	、しかるんです。
\\	子供がかわいい
\\	から__
\\	、しかるんです。
\\	中4
\\	"努力した
\\	からこそ
\\	、成功したんです。
\\	努力した
\\	から__
\\	、成功したんです。
\\	中4
\\	"厳しく注意したのは、あなたのことを
\\	思えばこそ
\\	です。
\\	厳しく注意したのは、あなたのことを
\\	思えば__
\\	です。
\\	中4
\\	"お互いに信頼し
\\	てこそ
\\	、悩みも言える。
\\	お互いに信頼し
\\	て__
\\	、悩みも言える。
\\	中4
\\	"親になっ
\\	てこそ
\\	、親の
\\	苦労[くろう]がわかる。
\\	親になっ
\\	て__
\\	、親の
\\	苦労[くろう]がわかる。
\\	中4
\\	"物価は上がる
\\	ばかり
\\	だ。
\\	物価は上がる
\\	だ。
\\	中4
\\	"状況は悪くなる
\\	ばかり
\\	だ。
\\	状況は悪くなる
\\	だ。
\\	中4
\\	"事故がないようにと
\\	ただ
\\	祈り
\\	ばかり
\\	だ。
\\	事故がないようにと
\\	祈り
\\	ばかり
\\	だ。
\\	中4
\\	"このパンは焼き
\\	たて
\\	なので、切れません。
\\	たばかり	
\\	このパンは焼き
\\	なので、切れません。
\\	中4
\\	"炊[た]き
\\	たて
\\	の御飯はおいしい。
\\	たばかり	
\\	炊き
\\	の御飯はおいしい。
\\	中4
\\	"「あれから、彼女に会いましたか。」
\\	「いえ、あれ
\\	(っ)
\\	きり
\\	です。」
\\	だけ	
\\	「あれから、彼女に会いましたか。」
\\	「いえ、あれ
\\	き_
\\	です。」
\\	中4
\\	"二人
\\	きり
\\	で話そう。
\\	だけ	
\\	二人
\\	で話そう。
\\	中4
\\	"この魚は骨
\\	まで
\\	食べられる。
\\	も 
\\	この魚は骨
\\	食べられる。
\\	中4
\\	"雨に降られて、下着
\\	まで
\\	ぬれた。
\\	も 
\\	雨に降られて、下着
\\	ぬれた。
\\	中4
\\	"ドアを開け
\\	っぱなし
\\	にするな。
\\	まま)	
\\	ドアを開け
\\	にするな。
\\	中4
\\	"祖母は電気もテレビもつけ
\\	っぱなし
\\	で寝ている。
\\	まま)	
\\	祖母は電気もテレビもつけ
\\	で寝ている。
\\	中4
\\	"知っているのに知らない
\\	ふりをする
\\	知っているのに知らない
\\	をする
\\	中4
\\	"彼女は親切な
\\	ふりをして
\\	人をだました。
\\	彼女は親切な
\\	をして
\\	人をだました。
\\	中4
\\	"「試験、どうだったのかなあ。」
\\	「発表は明日だけれど、だめ
\\	らしい
\\	よ。」
\\	「試験、どうだったのかなあ。」
\\	「発表は明日だけれど、だめ
\\	よ。」
\\	中4
\\	"田中さん、会社を辞める
\\	らしい
\\	ですよ。
\\	田中さん、会社を辞める
\\	ですよ。
\\	中4
\\	"家を出
\\	ようとしたときに
\\	、急に雨が降ってきた。
\\	家を出
\\	ようと__とき_
\\	、急に雨が降ってきた。
\\	中4
\\	"乗
\\	ろうとしたとき
\\	、電車のドアが閉まった。
\\	乗
\\	ろう___とき
\\	、電車のドアが閉まった。
\\	中4
\\	"こわ
\\	がらない
\\	で。
\\	怖
\\	ない
\\	で。
\\	中4
\\	"友達がいれば、
\\	それだけで
\\	秘密を言える。
\\	友達がいれば、
\\	そ____
\\	秘密を言える。
\\	中4
\\	"声を聞けば、
\\	それだけで
\\	幸せだ。
\\	声を聞けば、
\\	そ____
\\	幸せだ。
\\	中4
\\	"「駅の近くに、新しい病院ができましたね。」
\\	「大きな病院ですね。
\\	ところで
\\	、病院
\\	と言えば
\\	、山田さんが入院したそうですよ。」
\\	「駅の近くに、新しい病院ができましたね。」
\\	「大きな病院ですね。
\\	ところで
\\	、病院
\\	と___
\\	、山田さんが入院したそうですよ。」
\\	中4
\\	"いくら外見を飾っても、中身
\\	がなければ
\\	何にもならない
\\	いくら外見を飾っても、中身が
\\	な___
\\	何にもならない
\\	中4
\\	"いくら高い品物でも、丈夫
\\	じゃなければ
\\	役に立たない
\\	いくら高い品物でも、丈夫
\\	じゃなけ__
\\	役に立たない
\\	中4
\\	"窓を開ける
\\	と同時に
\\	虫が入った。
\\	窓を開ける
\\	と__に
\\	虫が入った。
\\	中4
\\	"自分の意見も言う
\\	と同時に
\\	ほかの人の考えもよく聞くことが大切だ。
\\	自分の意見も言う
\\	と__に
\\	ほかの人の考えもよく聞くことが大切だ。
\\	中4
\\	"大きすぎる親の期待は、子供
\\	にしたら
\\	大変だ。
\\	にすれば
\\	にしてみたら
\\	大きすぎる親の期待は、子供
\\	にし__
\\	大変だ。
\\	中4
\\	"犬や猫はかわいいが、飼っていない人
\\	にすれば
\\	迷わくなこともある。
\\	にしたら
\\	犬や猫はかわいいが、飼っていない人
\\	に___
\\	迷惑なこともある。
\\	中4
\\	"もし、それが本当だ
\\	としたら
\\	、夢のようです。
\\	とすれば
\\	もし、それが本当だ
\\	と___
\\	、夢のようです。
\\	中4
\\	"あの人が今も生きている
\\	とすれば
\\	、もう90歳になっているでしょう。
\\	としたら
\\	あの人が今も生きている
\\	と___
\\	、もう90歳になっているでしょう。
\\	中4
\\	"行く
\\	としても
\\	、旅行社としてか行けない。
\\	行く
\\	と___
\\	、旅行社としてか行けない。
\\	中4
\\	"この祭りは市民の社会参加
\\	を
\\	目的
\\	として
\\	始められた
\\	この祭りは市民の社会参加
\\	目的
\\	とし_
\\	始められた
\\	中4
\\	"山田さん
\\	を
\\	リーダー
\\	とする
\\	サークル
\\	を作る。
\\	山田さん
\\	リーダー
\\	とす_
\\	サークル
\\	を作る。
\\	中4
\\	"環境問題
\\	を
\\	テーマ
\\	とした
\\	テレビ番組
\\	をみる。
\\	環境問題
\\	テーマ
\\	とし_
\\	テレビ番組
\\	をみる。
\\	中4
\\	"家族
\\	と共に
\\	日本へ来た。
\\	一緒に 
\\	家族
\\	と__
\\	日本へ来た。
\\	中4
\\	"自動車が普及する
\\	と共に
\\	、事故も増加した。
\\	一緒に 
\\	自動車が普及する
\\	と__
\\	、事故も増加した。
\\	中4
\\	"人口の増加
\\	にともない
\\	、資源[しげん]の
\\	消費量[しょうひりょう]も増やす。
\\	にともなって
\\	人口の増加
\\	にと___
\\	、資源の
\\	消費量[しょうひりょう]も増やす。
\\	中4
\\	"マラソン大会が行われる
\\	にともなって
\\	、この道路は通行止めになります。
\\	マラソン大会が行われる
\\	にと____
\\	、この道路は通行止めになります。
\\	中4
\\	"年を取る
\\	につれて
\\	、体のいろいろな機能が
\\	低下[ていか]する。
\\	年を取る
\\	につ__
\\	、体のいろいろな機能が
\\	低下する。
\\	中4
\\	"町の発展
\\	につれて
\\	、自然が少なくなった。
\\	町の発展
\\	につ__
\\	、自然が少なくなった。
\\	中4
\\	"北へ行く
\\	に従[したが]って
\\	、紅葉[こうよう]は早くなる。
\\	北へ行く
\\	にし____
\\	、紅葉は早くなる。
\\	中4
\\	"予定表
\\	に従[したが]い
\\	、学習を進める。
\\	に従[したが]って
\\	予定表
\\	にし___
\\	、学習を進める。
\\	中4
\\	せっかく
\\	宿題をしたのに、忘れてきてしまいました。
\\	宿題をしたのに、忘れてきてしまいました。
\\	中4
\\	せっかく
\\	のプレゼントだから、今日から使いましょう。
\\	のプレゼントだから、今日から使いましょう。
\\	中4
\\	"母親が買ってきた赤いシャツを
\\	どしても
\\	着よう
\\	としなかった
\\	母親が買ってきた赤いシャツを
\\	どして_
\\	(着ます)___
\\	とし___た
\\	中4
\\	"娘は寝る時間になっても、
\\	どうしても
\\	寝よう
\\	としなかった
\\	"娘は寝る時間になっても、
\\	どうして_
\\	(寝ます)___
\\	とし___た
\\	中4
\\	"私が子供のころには、黄色いセーターを着たりすると「男の癖に」と
\\	言われた
\\	ものだ
\\	私が子供のころには、黄色いセーターを着たりすると「男の癖に」と
\\	言われた
\\	中4
\\	"「この音楽、ご存知でしょう。」
\\	「ええ、どこかで聞いた
\\	ように思った
\\	んです。」
\\	と思う 
\\	「この音楽、ご存知でしょう。」
\\	「ええ、どこかで聞いた
\\	思った
\\	んです。」
\\	中4
\\	"「あれ、宿題を忘れたんですか。」
\\	「ええ、バッグに入れた
\\	ように思った
\\	んですが。」
\\	と思う 
\\	「あれ、宿題を忘れたんですか。」
\\	「ええ、バッグに入れた
\\	思った
\\	んですが。」
\\	中4
\\	"友達が言った
\\	とおり
\\	、日本の物価が高い。
\\	友達が言った
\\	と__
\\	、日本の物価が高い。
\\	中4
\\	"自分の思う
\\	とおりに
\\	生きることは難しい。
\\	自分の思う
\\	と__に
\\	生きることは難しい。
\\	中4
\\	"希望
\\	どおり
\\	希望
\\	ど__
\\	中4
\\	"店員に言われる
\\	まま
\\	、契約書にサインをしてしまった。
\\	店員に言われる
\\	、契約書にサインをしてしまった。
\\	中4
\\	"足に向く
\\	ままに
\\	歩く。
\\	足に向く
\\	に
\\	歩く。
\\	中4
\\	"驚いた
\\	ことに
\\	、彼女は同じ相手と二回も再婚しているんです。
\\	驚いた
\\	に
\\	、彼女は同じ相手と二回も再婚しているんです。
\\	中4
\\	"おもしろい
\\	ことに
\\	、隣の家も、その隣の家も、うちと同じ
\\	名字[みょうじ]なのです。
\\	おもしろい
\\	に
\\	、隣の家も、その隣の家も、うちと同じ
\\	名字[みょうじ]なのです。
\\	中4
\\	"緊張の
\\	あまり
\\	、体が震えた。
\\	きんちょうの
\\	あ__
\\	、体が震えた。
\\	中4
\\	"心配する
\\	あまり
\\	、母は寝込んでしまった。
\\	心配する
\\	あ__
\\	、母は寝込んでしまった。
\\	中4
\\	"日曜日は、家で
\\	を
\\	見たりして
\\	過ごしています。
\\	日曜日は、家で
\\	を
\\	見__して
\\	過ごしています。
\\	中4
\\	"うそを
\\	ついたりして
\\	はいけません。
\\	うそを
\\	つい__して
\\	はいけません。
\\	中4
\\	"友達の赤ちゃんを預かったのですが、
\\	泣かれて困りました
\\	友達の赤ちゃんを預かったのですが、
\\	(泣く)
\\	困りました
\\	中4
\\	"今日中にレポートを
\\	書き上げます
\\	今日中にレポートを
\\	(書く)
\\	上げます
\\	中4
\\	"寒い
\\	わけだ
\\	。雪が降っている。
\\	寒い
\\	だ
\\	。雪が降っている。
\\	中4
\\	"定価[ていか]が1万円で、2
\\	割引[わりび]きだから8千円になる
\\	わけだ
\\	定価[ていか]が1万円で、2割引きだから8千円になる
\\	だ
\\	中4
\\	"欲しくない
\\	わけではない
\\	けれど、お金がないから買えないんです。
\\	欲しくない
\\	では__
\\	けれど、お金がないから買えないんです。
\\	中4
\\	"テレビはつけてあるが、見ている
\\	わけでもない
\\	テレビはつけてあるが、見ている
\\	でも__
\\	中4
\\	"うちの子がそんなことをする
\\	わけがない
\\	うちの子がそんなことをする
\\	ない
\\	中4
\\	"ほめられて、うれしくない
\\	わけがない
\\	ほめられて、うれしくない
\\	ない
\\	中4
\\	"重要[じゅうよう]な会議があるので、病気でも会社を休む
\\	わけにはいかない
\\	重要[じゅうよう]な会議があるので、病気でも会社を休む
\\	には__ない
\\	中4
\\	"税金[ぜいきん]は、高くても払わない
\\	わけにはいかない
\\	税金[ぜいきん]は、高くても払わない
\\	には__ない
\\	中4
\\	"彼は酔っ払って
\\	騒いだ
\\	あげく
\\	、寝てしまった。
\\	彼は酔っ払って
\\	騒いだ
\\	、寝てしまった。
\\	中5
\\	"何度も手術
\\	した
\\	あげく
\\	、治らないと言われた。
\\	何度も手術した
\\	、治らないと言われた。
\\	中5
\\	悩んだ
\\	末[すえ]
\\	、進学ではなく
\\	就職することにした。
\\	悩んだ
\\	、進学ではなく
\\	就職[しゅうしょく]することにした。
\\	中5
\\	"市民との
\\	話の
\\	末[すえ]
\\	、その建物の建設は中止となった。
\\	市民との
\\	話の
\\	、その建物の建設は中止となった。
\\	中5
\\	"あの人は忙しい人で、
\\	来た
\\	かと思ったら
\\	、もう帰ってしまった。
\\	あの人は忙しい人で、
\\	来た
\\	と思___
\\	、もう帰ってしまった。
\\	中5
\\	"花が
\\	咲いた
\\	かと思ったら
\\	、もう枯れてしまった。
\\	花が
\\	咲[さ]いた
\\	と思___
\\	、もう枯れてしまった。
\\	中5
\\	"会いたいと思っていた
\\	ところです
\\	とき 
\\	"会いたいと思っていた
\\	です
\\	中5
\\	"病院で
\\	検査した
\\	ところ
\\	、異常[いじょう]はなかった。
\\	病院で
\\	検査した
\\	、異常[いじょう]はなかった。
\\	中5
\\	"駅に
\\	問い合わせた
\\	ところ
\\	、忘れ物は届いていた。
\\	駅に
\\	問い合わせた
\\	、忘れ物は届いていた。
\\	中5
\\	"工場で火事があり、仕事
\\	どころではなかった
\\	工場で火事があり、仕事
\\	では____
\\	中5
\\	"風邪がひどくて、遊びに行く
\\	どころじゃない
\\	風邪がひどくて、遊びに行く
\\	じゃ__
\\	中5
\\	"「夏休みは取れそう?」
\\	「忙しくて夏休み
\\	どころか
\\	日曜日
\\	も
\\	休めないよ。」
\\	「そう。うちの会社は忙しい
\\	どころか
\\	仕事がなくて困っているんだ。」
\\	「夏休みは取れそう?」
\\	「忙しくて夏休み
\\	ころ_
\\	日曜日
\\	も
\\	休めないよ。」
\\	「そう。うちの会社は忙しい
\\	ころ_
\\	仕事がなくて困っているんだ。」
\\	中5
\\	"猫がえさを食べ
\\	ようとしない
\\	ので、心配だ。
\\	猫がえさを食べ
\\	しない
\\	ので、心配だ。
\\	中5
\\	"田中さんは留学し
\\	ようとして
\\	いる。
\\	"田中さんは留学し
\\	として
\\	いる。
\\	中5
\\	"病気になっ
\\	て始めて
\\	健康の大切さを知りました。
\\	病気に
\\	(なる)
\\	始めて
\\	健康の大切さを知りました。
\\	中5
\\	"私は電気をつけた
\\	まま
\\	、いつの間にか眠ってしまいました。
\\	私は電気をつけた
\\	、いつの間にか眠ってしまいました。
\\	中5
\\	"「彼女は私の妹の子供です。」
\\	つまり
\\	、めい(だ)
\\	と
\\	いうことですね。」
\\	「彼女は私の妹の子供です。」
\\	、めい
\\	いうことですね。」
\\	中5
\\	"六月
\\	から
\\	七月
\\	にかけて
\\	雨が多いです。
\\	からーまで 
\\	六月
\\	から
\\	七月
\\	雨が多いです。
\\	中5
\\	"十代から二十代
\\	にかけて
\\	おもしろい時代です。
\\	からーまで 
\\	十代から二十代
\\	おもしろい時代です。
\\	中5
\\	"急にお客が来たので、急いで部屋を片付ける
\\	やら
\\	お茶を出す
\\	やらで
\\	大変だ。
\\	とかーとか 
\\	急にお客が来たので、急いで部屋を片付ける
\\	お茶を出す
\\	で
\\	大変だ。
\\	中5
\\	"せっかく海へ行ったのにごみ
\\	やら
\\	クラゲ
\\	やらで
\\	大変でした。
\\	とかーとか 
\\	せっかく海へ行ったのにごみ
\\	クラゲ
\\	で
\\	大変でした。
\\	中5
\\	"この手紙の日本語は間違い
\\	だらけ
\\	です。
\\	この手紙の日本語は間違い
\\	です。
\\	中5
\\	"ちゃんと畳んでおかなかったから、服がしわ
\\	だらけ
\\	になってしまった。
\\	ちゃんと畳んでおかなかったから、服がしわ
\\	になってしまった。
\\	中5
\\	"うちの子は遊びに
\\	行った
\\	きり
\\	帰って来ない。
\\	うちの子は遊びに
\\	行った
\\	帰って来ない。
\\	中5
\\	"息子は自分の部屋に
\\	入った
\\	きり
\\	出てこない。
\\	息子は自分の部屋に
\\	入った
\\	出てこない。
\\	中5
\\	"母は入院している父を
\\	つき
\\	(っ)
\\	きり
\\	で看病[かんびょう]している。
\\	母は入院している父を
\\	(つきます)__
\\	きり
\\	で看病している。
\\	中5
\\	"電車が混んでいて、東京まで立ち
\\	っぱなし
\\	だった。
\\	電車が混んでいて、東京まで立ち
\\	っ___
\\	だった。
\\	中5
\\	"危なかった。
\\	もう少しで
\\	事故になる
\\	ところだった
\\	危なかった。
\\	もう_しで
\\	事故になる
\\	だった
\\	中5
\\	もう少しで
\\	川があふれる
\\	ところだった
\\	もう_しで
\\	川があふれる
\\	だった
\\	中5
\\	"電話くれ
\\	たら
\\	すぐに手伝いに
\\	行った
\\	のに
\\	、どうして言わなかったの?
\\	電話くれ
\\	たら
\\	すぐに手伝いに
\\	行った
\\	、どうして言わなかったの?
\\	中5
\\	"田中さんの妹は
\\	美人
\\	というより
\\	かわいい
\\	女性です。
\\	田中さんの妹は美人
\\	いう__
\\	、かわいい女性です。
\\	中5
\\	"私は留学生
\\	として
\\	日本に来ましたが、日本でしゅう職をして、結婚もしました。
\\	私は留学生
\\	と__
\\	日本に来ましたが、日本でしゅう職をして、結婚もしました。
\\	中5
\\	"学校を辞めました。
\\	と(いうの)は
\\	、父が亡くなって働かなければならないからです。
\\	学校を辞めました。
\\	いうの_
\\	、父が亡くなって働かなければならない
\\	からです
\\	中5
\\	"来月、引っ越します。
\\	と(いうの)は
\\	、転勤になった
\\	からです
\\	来月、引っ越します。
\\	いうの_
\\	、転勤になった
\\	からです
\\	中5
\\	"「泳げますか。」
\\	泳げる
\\	ことは
\\	泳げます
\\	が
\\	、ぜんぜん速くないです。」
\\	「泳げますか。」
\\	泳げる
\\	泳げます
\\	、ぜんぜん速くないです。」
\\	中5
\\	"その本は
\\	高い
\\	ことは
\\	高い
\\	けれど
\\	、必要なので買わなければならない。
\\	その本は
\\	高い
\\	高い
\\	、必要なので買わなければならない。
\\	中5
\\	"日本には来た
\\	ものの
\\	、言葉が分からず困っている。
\\	日本には来た
\\	、言葉が分からず困っている。
\\	中5
\\	"時間が経つ
\\	につれて
\\	、何だか気味が悪くなってきた。
\\	時間が経つ
\\	につ__
\\	、何だか気味が悪くなってきた。
\\	中5
\\	"学校では先生
\\	として
\\	働いています。
\\	学校では先生
\\	と__
\\	働いています。
\\	中5
\\	"家でお母さん
\\	として
\\	子供たちに愛されています。
\\	家でお母さん
\\	と__
\\	子供たちに愛されています。
\\	中5
\\	"情報通信技術の開発が進め
\\	ば
\\	進む
\\	ほど
\\	、情報の管理が難しくなってくる。
\\	情報通信技術の開発が進め
\\	進む
\\	、情報の管理が難しくなってくる。
\\	中5
\\	"専門家[せんもんか]の予想
\\	に反して
\\	、景気は回復し始めた。
\\	専門家[せんもんか]の予想
\\	に_して
\\	、景気は回復し始めた。
\\	中5
\\	"我々の期待
\\	に反して
\\	、新商品はあまり売れなかった。
\\	我々の期待
\\	に_して
\\	、新商品はあまり売れなかった。
\\	中5
\\	"パソコンは便利
\\	な
\\	反[半]面
\\	、トラブルも多い。
\\	一方
\\	パソコンは便利
\\	な
\\	、トラブルも多い。
\\	中5
\\	"彼はすぐれた学者
\\	である
\\	反[半]面
\\	、精神的に弱い人間だ。
\\	一方
\\	彼はすぐれた学者
\\	である
\\	、精神的に弱い人間だ。
\\	中5
\\	"この服は水に強い
\\	一方
\\	、熱に弱い。
\\	反面
\\	この服は水に強い
\\	、熱に弱い。
\\	中5
\\	"この辺りは、静か
\\	な
\\	一方
\\	、不便である。
\\	反面
\\	この辺りは、静か
\\	な
\\	、不便である。
\\	中5
\\	"大気中の
\\	2[シーオーツー]濃度[のうど]の上昇と共に、地球の
\\	温暖化[おんだんか]は進む
\\	一方だ
\\	大気中の
\\	2[シーオーツー]濃度[のうど]の上昇と共に、地球の
\\	温暖化[おんだんか]は進む
\\	だ
\\	中5
\\	"悪い天気の
\\	せい[所為]で、野菜の値段は上がる
\\	一方だ
\\	悪い天気のせいで、野菜の値段は上がる
\\	だ
\\	中5
\\	"彼は仕事ができる
\\	上に
\\	優しい。
\\	彼は仕事ができる
\\	優しい。
\\	中5
\\	"昨日は寒かった
\\	上に
\\	、風が強かった。
\\	昨日は寒かった
\\	、風が強かった。
\\	中5
\\	"よく考えた
\\	上で
\\	決めます。
\\	よく考えた
\\	決めます。
\\	中5
\\	"家族と相談の
\\	上で
\\	お返事します。
\\	上、
\\	家族と相談の
\\	お返事します。
\\	中5
\\	"キャプテンに選ばれた
\\	上は
\\	、頑張るしかない。
\\	キャプテンに選ばれた
\\	、頑張るしかない。
\\	中5
\\	"入学する
\\	上は
\\	、卒業したい。
\\	入学する
\\	、卒業したい。
\\	中5
\\	"天気図
\\	の上では
\\	春なのに、実際はまだ寒い。
\\	上[じょう]は
\\	の
\\	天気図
\\	上__
\\	春なのに、実際はまだ寒い。
\\	中5
\\	"彼らは書類
\\	の上では
\\	夫婦だ。
\\	上[じょう]は
\\	の
\\	彼らは書類
\\	上__
\\	夫婦だ。
\\	中5
\\	"その学生は、漢字
\\	ばかりか
\\	、ひらがな
\\	も
\\	カタカナ
\\	も
\\	書けません。
\\	ばかりではなく
\\	その学生は、漢字
\\	か
\\	、ひらがな
\\	カタカナ
\\	書けません。
\\	中5
\\	"田中さんは英語
\\	ばかりではなく
\\	、中国語やフランス語
\\	も
\\	話せる。
\\	ばかりか
\\	田中さんは英語
\\	ではなく
\\	、中国語やフランス語
\\	話せる。
\\	中5
\\	"今日のコンサート何時から
\\	だっけ
\\	今日のコンサート何時から
\\	だっ_
\\	中5
\\	"これ、前に話し
\\	たっけ
\\	これ、前に話し
\\	たっ_
\\	中5
\\	たとえ
\\	親に反対され
\\	ても
\\	、私は彼と結婚します。
\\	た__
\\	親に反対され
\\	て_
\\	、私は彼と結婚します。
\\	中5
\\	"その学生は、講義を最後まで聞かない
\\	うちに
\\	教室を出て行った。
\\	その学生は、講義を最後まで聞かない
\\	教室を出て行った。
\\	中5
\\	"暗くならない
\\	うちに
\\	家に帰りなさい。
\\	暗くならない
\\	家に帰りなさい。
\\	中5
\\	"そんなに夜遅く、子供を外出させる
\\	べきではない
\\	そんなに夜遅く、子供を外出させる
\\	ない
\\	中5
\\	"悪いのは君だから、謝る
\\	べきだ
\\	悪いのは君だから、謝る
\\	中5
\\	"僕が東大を受けても落ちる
\\	に決まっている
\\	僕が東大を受けても落ちる
\\	に____いる
\\	中5
\\	"それはうそ
\\	に決まっている
\\	それはうそ
\\	に____いる
\\	中5
\\	"急いでいる
\\	ものですから
\\	、お先に失礼します。
\\	急いでいる
\\	ですから
\\	、お先に失礼します。
\\	中6
\\	"「いつタバコをやめになったのですか。」
\\	「彼女にやめられると言われた
\\	の
\\	がきっかけ
\\	なんです。」
\\	「いつタバコをやめになったのですか。」
\\	「彼女にやめられると言われた
\\	の
\\	が____
\\	なんです。」
\\	中6
\\	"映画館でドラエモンを見た人が私と彼だけだった
\\	の
\\	がきっかけ
\\	になって、彼と付き合い始めたんです。
\\	映画館でドラエモンを見た人が私と彼だけだった
\\	の
\\	が____
\\	になって、彼と付き合い始めたんです。
\\	中6
\\	"田川さん
\\	にとって
\\	、山口さんを訪ねる月曜の午後はとても楽しい
\\	一時[ひととき]なのである。
\\	田川さん
\\	に___
\\	、山口さんを訪ねる月曜の午後はとても楽しい一時なのである。
\\	中6
\\	"それは、
\\	初心者[しょしんしゃ]
\\	向けの
\\	教科書です。
\\	それは、
\\	初心者[しょしんしゃ]
\\	の
\\	教科書です。
\\	中6
\\	"この製品の生産は、国内
\\	向けも
\\	、海外輸出
\\	向けも
\\	、共に増加した。
\\	"この製品の生産は、国内
\\	も
\\	、海外輸出
\\	も
\\	、共に増加した。
\\	中6
\\	"この店の料理は、量が少なく見た目がきれいなので、女性
\\	向きだ
\\	この店の料理は、量が少なく見た目がきれいなので、女性
\\	だ
\\	中6
\\	"花火大会は
\\	天気
\\	次第[しだい]で
\\	中止になる場合もあります。
\\	花火大会は天気
\\	で
\\	中止になる場合もあります。
\\	中6
\\	"うまくいくかどかは
\\	本人
\\	次第[しだい]だ
\\	うまくいくかどかは本人
\\	だ
\\	中6
\\	"では、田中が
\\	戻り
\\	次第
\\	、お電話
\\	させます。
\\	では、田中が
\\	戻り
\\	、お電話させます。
\\	中6
\\	決まり
\\	次第
\\	、ご連絡
\\	いたします。
\\	決まり
\\	、ご連絡いたします。
\\	中6
\\	"担当がかわりましたので、あいさつに
\\	伺った
\\	次第です
\\	担当がかわりましたので、あいさつに
\\	伺った
\\	です
\\	中6
\\	"日時の変更について、
\\	お知らせする
\\	次第です
\\	日時の変更について、お知らせする
\\	です
\\	中6
\\	"客の意見
\\	にこたえて
\\	、営業時間を延長する。
\\	客の意見
\\	に____
\\	、営業時間を延長する。
\\	中6
\\	"親の期待
\\	にこたえ
\\	、頑張った。
\\	にこたえて
\\	親の期待
\\	に____
\\	頑張った。
\\	中6
\\	"目上の人
\\	に対して
\\	、そういう言い方は失礼ですよ。
\\	目上の人
\\	に_して
\\	、そういう言い方は失礼ですよ。
\\	中6
\\	"生徒の態度
\\	に対して
\\	、不満がある。
\\	生徒の態度
\\	に_して
\\	、不満がある。
\\	中6
\\	"都市で人口が増えている
\\	の
\\	に対して
\\	、農村では減っている。
\\	都市で人口が増えている
\\	の
\\	に_して
\\	、農村では減っている。
\\	中6
\\	"子供の
\\	飲酒[いんしゅ]は法律
\\	により
\\	禁
\\	じられている。
\\	によって
\\	子供の
\\	飲酒[いんしゅ]は法律
\\	に__
\\	禁じられている。
\\	中6
\\	"この地震
\\	による
\\	被害は最大だった。
\\	で
\\	この地震
\\	に__
\\	被害
\\	は最大だった。
\\	中6
\\	"彼は犯罪事件
\\	にかかわって
\\	捕まれた。
\\	彼は犯罪事件
\\	に_____
\\	捕まれた。
\\	中6
\\	"息子はマスコミ
\\	にかかわる
\\	仕事をしています。
\\	息子はマスコミ
\\	に____
\\	仕事をしています。
\\	中6
\\	"命
\\	にかかわる
\\	病気です。
\\	命
\\	に____
\\	病気です。
\\	中6
\\	"悪いことと知り
\\	ながら
\\	、犯罪を繰り替えした。
\\	つつ
\\	悪いことと知り
\\	な__
\\	、犯罪を繰り替えした。
\\	中6
\\	"「狭い
\\	ながら
\\	楽しいわが
\\	家[や]」という言葉がある。
\\	「狭い
\\	な__
\\	楽しいわが
\\	家[や]」という言葉がある。
\\	中6
\\	"彼女は忙しい
\\	つつ
\\	(も)
\\	長[なが]電話をしている。
\\	ながら
\\	彼女は忙しい
\\	、長電話をしている。
\\	中6
\\	"今日こそ頑張ろうと思い
\\	つつ
\\	(も)
\\	、まだ勉強しなかった。
\\	ながら
\\	今日こそ頑張ろうと思い
\\	、まだ勉強しなかった。
\\	中6
\\	"先生と相談し
\\	つつ
\\	、将来を決めたいと思う。
\\	ながら)	
\\	先生と相談し
\\	、将来を決めたいと思う。
\\	中6
\\	"医療はますます進歩し
\\	つつある
\\	医療はますます進歩し
\\	ある
\\	中6
\\	"新しいウイルスによる被害は、全国に広がり
\\	つつある
\\	新しいウイルスによる被害は、全国に広がり
\\	ある
\\	中6
\\	"知らない
\\	くせして
\\	、知っているようなことを言うな。
\\	知らない
\\	て
\\	、知っているようなことを言うな。
\\	中6
\\	"大学生の
\\	くせして
\\	、そんなことも知らないの?
\\	大学生の
\\	て
\\	、そんなことも知らないの?
\\	中6
\\	"「小林さんはまだですか。もう三時になります。」
\\	「二時前に家を出たそうだから、もうすぐ来る
\\	はずです
\\	よ。」
\\	「小林さんはまだですか。もう三時になります。」
\\	「二時前に家を出たそうだから、もうすぐ来る
\\	です
\\	よ。」
\\	中6
\\	"「ジョンさんに出す手紙なんですが、日本語でいいでしょうか。」
\\	「もう3年間勉強したから、わかる
\\	はずです
\\	よ。」
\\	「ジョンさんに出す手紙なんですが、日本語でいいでしょうか。」
\\	「もう3年間勉強したから、わかる
\\	です
\\	よ。」
\\	中6
\\	"毎日学校へ来ている
\\	からといって
\\	、日本語が上手になる
\\	わけではない
\\	毎日学校へ来ている
\\	からと___
\\	、日本語が上手になる
\\	ではない
\\	中6
\\	"肩書きが立派だ
\\	からといって
\\	、お金持ちの
\\	わけではない
\\	肩書きが立派だ
\\	からと___
\\	、お金持ちの
\\	ではない
\\	中6
\\	"「漢字はいくつぐらい読めますか。」
\\	「漢字
\\	どころか
\\	、ひらがな
\\	さえ
\\	まだ読めないんです。」
\\	「漢字はいくつぐらい読めますか。」
\\	「漢字
\\	ど___
\\	、ひらがな
\\	まだ読めないんです。」
\\	中6
\\	"「この方のご住所をご存しですか。」
\\	「住所
\\	どこらか
\\	、名前
\\	さえ
\\	分かりません。」
\\	「この方のご住所をご存しですか。」
\\	「住所
\\	ど___
\\	、名前
\\	分かりません。」
\\	中6
\\	"そんなことを
\\	す
\\	べきではない
\\	するべき
\\	そんなことをす
\\	ではない
\\	中6
\\	"もっと勉強
\\	する
\\	べきだった
\\	もっと勉強
\\	する
\\	だった
\\	中6
\\	"あんなことと
\\	言う
\\	べきではなかった
\\	あんなことと言う
\\	ではなかった
\\	中6
\\	"いやな仕事でも、生活のためには
\\	続け
\\	ざるをえない
\\	いやな仕事でも、生活のためには
\\	続け
\\	を_ない
\\	中6
\\	"この戦争は間違いだったと
\\	言わ
\\	ざるをえない
\\	この戦争は間違いだったと
\\	言わ
\\	を_ない
\\	中6
\\	"明日、ここで卒業式が行われる
\\	ことになっている
\\	明日、ここで卒業式が行われる
\\	ことに_____
\\	中6
\\	"60点以上が合格という
\\	ことになっている
\\	60点以上が合格という
\\	ことに_____
\\	中6
\\	"単なる言う間違い
\\	に過ぎない
\\	単なる言う間違い
\\	に__ない
\\	中6
\\	"すみません、私はただのアルバイト
\\	に過ぎない
\\	すみません、私はただのアルバイト
\\	に__ない
\\	中6
\\	"図書館の利用
\\	にあたって
\\	図書カードが必要です。
\\	にあたり、
\\	図書館の利用
\\	に____
\\	図書カードが必要です。
\\	中6
\\	"アルバイトをする
\\	にあたっては
\\	、学業や体に無理のないようにすること。
\\	アルバイトをする
\\	に____は
\\	、学業や体に無理のないようにすること。
\\	中6
\\	"資料[しりょう]
\\	に沿[そ]って
\\	、ご説明いたします。
\\	資料[しりょう]
\\	に___
\\	、ご説明いたします。
\\	中6
\\	"お客様ひとりひとりのご希望
\\	に沿[そ]った
\\	旅行プレゼンをお作りします。
\\	お客様ひとりひとりのご希望
\\	に___
\\	旅行プレゼンをお作りします。
\\	中6
\\	"開店
\\	に
\\	先[さき]立ち
\\	、関係者だけのパーティーが行われた。
\\	開店
\\	に_立ち
\\	、関係者だけのパーティーが行われた。
\\	中6
\\	"新製品の開発
\\	に先[さき]立って
\\	、アンケート調査を
\\	行[おこな]った。
\\	新製品の開発
\\	に_立って
\\	、アンケート調査を行った。
\\	中6
\\	"この台風の影響で、関東地方は広い
\\	範囲[はんい]
\\	にわたって
\\	大雨となるでしょう。
\\	この台風の影響で、関東地方は広い
\\	範囲[はんい]
\\	に____
\\	大雨となるでしょう。
\\	中6
\\	"わが社は20年間
\\	にわたって
\\	この薬の研究開発に努めてきた。
\\	にわたり、
\\	わが社は20年間
\\	に____
\\	この薬の研究開発に努めてきた。
\\	中6
\\	"彼は色々な所へ行っ
\\	ては
\\	写真を撮っています。
\\	彼は色々な所へ
\\	(行く)__
\\	は
\\	写真を撮っています。
\\	中6
\\	"姉は体重計の数字を見
\\	ては
\\	、「ダイエットしよう」と言っています。
\\	姉は体重計の数字を
\\	(見る)_
\\	は
\\	、「ダイエットしよう」と言っています。
\\	中6
\\	"お巡りさんを見かけ
\\	ては
\\	、逃げている。
\\	お巡りさんを
\\	(見かける)___
\\	は
\\	、逃げている。
\\	中6
\\	"「小林さんがあのマンションを買っていたそうですよ。」
\\	「あんな高いマンションが
\\	まさか
\\	買える
\\	はずがありません
\\	「小林さんがあのマンションを買っていたそうですよ。」
\\	「あんな高いマンションが
\\	ま__
\\	買える
\\	がありません
\\	中6
\\	"あんなに親切な人が
\\	まさか
\\	存在する
\\	はずがない
\\	あんなに親切な人が
\\	ま__
\\	存在する
\\	がない
\\	中6
\\	"先生に伺ってみ
\\	たところ
\\	、分からないという返事でした。
\\	先生に伺って
\\	(みる)_
\\	と__
\\	、分からないという返事でした。
\\	中6
\\	"空港に電話をして聞いてみ
\\	たところ
\\	、誰もでなかった。
\\	空港に電話をして聞いて
\\	(みる)_
\\	と__
\\	、誰もでなかった。
\\	中6
\\	"何回も練習している
\\	うちに
\\	だんだん分かってきました。
\\	何回も練習している
\\	に
\\	だんだん分かってきました。
\\	中6
\\	"そんなにたくさんの料理、いくら僕でも食べ
\\	きれない
\\	よ。
\\	そんなにたくさんの料理、いくら僕でも食べ
\\	ない
\\	よ。
\\	中6
\\	"そのパソコンは修理しても直れないのだから、捨てる
\\	よりほかない
\\	しかない	
\\	そのパソコンは修理しても直れないのだから、捨てる
\\	よ___ない
\\	中6
\\	"明日は手術の日だ。医者に
\\	任[まか]せる
\\	よりほかない
\\	しかない	
\\	明日は手術の日だ。医者に
\\	任[まか]せる
\\	よ___ない
\\	中6
\\	"真面目な彼が休むなんて、何かあった
\\	に違いない
\\	きっと。。。だ	
\\	真面目な彼が休むなんて、何かあった
\\	に__ない
\\	中6
\\	"あの二人は、よく似ているから、親子
\\	に違いない
\\	きっと。。。だ	
\\	あの二人は、よく似ているから、親子
\\	に__ない
\\	中6
\\	"美人が
\\	必ずしも
\\	幸せになれる
\\	とは限らない
\\	"美人が
\\	必ず__
\\	幸せになれる
\\	とは__ない
\\	中6
\\	"独身の人が
\\	必ずしも
\\	寂しい
\\	とは限らない
\\	独身の人が
\\	必ず__
\\	寂しい
\\	とは__ない
\\	中6
\\	"新聞やテレビの言うことなど、
\\	少しも
\\	信じられ
\\	ない
\\	あいまい...
\\	新聞やテレビの言うことなど、
\\	少__
\\	信じられ
\\	中6
\\	"人が何と言おうと、
\\	少しも
\\	気になら
\\	ない
\\	あいまい...
\\	人が何と言おうと、
\\	少__
\\	気になら
\\	中6
\\	"時間に追われたくない
\\	から
\\	、時計を持たなくなった
\\	わけです
\\	時間に追われたくない
\\	から
\\	、時計を持たなくなった
\\	です
\\	中7
\\	"「ごめんなさい。きのうは出かけてしまって、家にいなかったのです。」
\\	「ああ、
\\	それで
\\	電話にでなかった
\\	というわけですか
\\	"「ごめんなさい。きのうは出かけてしまって、家にいなかったのです。」
\\	「ああ、
\\	それで
\\	電話にでなかった
\\	という__です_
\\	中7
\\	"「トマさんは今度新しい車を買ったそうですよ。」
\\	「ああ、
\\	それで
\\	奥さんはいらいらしていた
\\	というわけですか
\\	「トマさんは今度新しい車を買ったそうですよ。」
\\	「ああ、
\\	それで
\\	奥さんはいらいらしていた
\\	という__です_
\\	中7
\\	"「あのう、さっきのお話について、いろいろ質問があるのですが。」
\\	つまり
\\	、全然わからなかった
\\	というわけですね
\\	「あのう、さっきのお話について、いろいろ質問があるのですが。」
\\	つまり
\\	、全然わからなかった
\\	という__です_
\\	中7
\\	"「昨日は財布をなくしたり、駅で転んだりして大変でした。」
\\	つまり
\\	昨日はあまりいい日ではなかった
\\	というわけですね
\\	「昨日は財布をなくしたり、駅で転んだりして大変でした。」
\\	つ__
\\	昨日はあまりいい日ではなかった
\\	という__です_
\\	中7
\\	"来ると約束して
\\	おきながら
\\	、けいこさんは来ませんでした。
\\	来ると約束して
\\	な__
\\	、けいこさんは来ませんでした。
\\	中7
\\	"いつも人の世話になって
\\	おきながら
\\	、責任をとらない。
\\	いつも人の世話になって
\\	な__
\\	、責任をとらない。
\\	中7
\\	"何かあ
\\	ろうと
\\	決してあきらめないでください。
\\	~ても	
\\	何か
\\	(ある)
\\	と
\\	決してあきらめないでください。
\\	中7
\\	"どこのだれと結婚し
\\	ようと
\\	、チョウさんとはできない。
\\	~ても	
\\	どこのだれと結婚
\\	(する)
\\	と
\\	、チョウさんとはできない。
\\	中7
\\	"彼女を
\\	傷つけるまい
\\	と思っていたのに、また悪いことを言ってしまった。
\\	彼女を
\\	きずつける__
\\	と思っていたのに、また悪いことを言ってしまった。
\\	中7
\\	"タバコを
\\	続けるまい
\\	と決めていたのに、続けてしまった。
\\	タバコを
\\	続ける__
\\	と決めていたのに、続けてしまった。
\\	中7
\\	"一日でこの本の文法全部なんて、
\\	覚えられ
\\	っこない
\\	一日でこの本の文法全部なんて、
\\	覚えられ
\\	っ___
\\	中7
\\	"今の実力では、試験に
\\	受かり
\\	っこない
\\	今の実力では、試験に
\\	受かり
\\	っ___
\\	中7
\\	"そんなひどいことなら、あの人なら
\\	言い
\\	かねない
\\	そんなひどいことなら、あの人なら
\\	言い
\\	か___
\\	中7
\\	"この問題を解決せずに放っておいたら、国際問題に
\\	なり
\\	かねない
\\	この問題を解決せずに放っておいたら、国際問題に
\\	なり
\\	か___
\\	中7
\\	"申し訳ありませんが、私に
\\	わかり
\\	かねます
\\	申し訳ありませんが、私に
\\	わかり
\\	か___
\\	中7
\\	"あなたの意見には賛成
\\	し
\\	かねる
\\	あなたの意見には賛成
\\	し
\\	か__
\\	中7
\\	"これは
\\	信じ
\\	がたい
\\	話だが、事実である。
\\	これは
\\	信じ
\\	話だが、事実である。
\\	中7
\\	"人が人の命を奪うなんて、
\\	許し
\\	がたい
\\	人が人の命を奪うなんて、
\\	許し
\\	中7
\\	"富士山が見える
\\	ことから
\\	、この町は
\\	富士見町[ふじみちょう]という名前がついた。
\\	富士山が見える
\\	ら
\\	、この町は
\\	富士見町[ふじみちょう]という名前がついた。
\\	中7
\\	"ここは桜の
\\	名所[めいしょ]である
\\	ことから
\\	、春に花見客が大勢やってくる。
\\	ここは桜の
\\	名所[めいしょ]である
\\	ら
\\	、春に花見客が大勢やってくる。
\\	中7
\\	"いつも遅刻する彼
\\	のことだから
\\	、もうすぐ現れる
\\	だろう
\\	いつも遅刻する彼
\\	の____ら
\\	、もうすぐ現れる
\\	だろう
\\	中7
\\	"親切な林さん
\\	のことだから
\\	、頼めば手伝ってくれよ。
\\	親切な林さん
\\	の____ら
\\	、頼めば手伝ってくれよ。
\\	中7
\\	"雨は休む
\\	ことなく
\\	降り続いた。
\\	雨は休む
\\	こと__
\\	降り続いた。
\\	中7
\\	"時は止まる
\\	ことなく
\\	流れる。
\\	時は止まる
\\	こと__
\\	流れる。
\\	中7
\\	"やってみ
\\	ないことには
\\	、できるかどうかわから
\\	ない
\\	やってみ
\\	こと__
\\	、できるかどうかわから
\\	中7
\\	"実物を見
\\	ないことには
\\	、買う気にはなれ
\\	ない
\\	実物を見
\\	こと__
\\	、買う気にはなれ
\\	中7
\\	"こんなことでいらいらする
\\	ようでは
\\	、胃がいくつあっても足りない。
\\	こんなことでいらいらする
\\	は
\\	、胃がいくつあっても足りない。
\\	中7
\\	"こんな簡単な漢字も読めない
\\	ようでは
\\	3を合格しないだろう。
\\	こんな簡単な漢字も読めない
\\	は
\\	3を合格しないだろう。
\\	中7
\\	"早く花が咲くように
\\	と
\\	、毎日水をやっている。
\\	と願って	
\\	早く花が咲くように
\\	、毎日水をやっている。
\\	中7
\\	"人気者になるように
\\	と
\\	、劇を習っている。
\\	と願って	
\\	人気者になるように
\\	、劇を習っている。
\\	中7
\\	"どんなに頑張っ
\\	たところで
\\	、社長になれるわけではない。
\\	どんなに頑張っ
\\	た___で
\\	、社長になれるわけではない。
\\	中7
\\	"どんなにやってみ
\\	たところで
\\	、すぐにはできないだろう。
\\	どんなにやってみ
\\	た___で
\\	、すぐにはできないだろう。
\\	中7
\\	"もう春だから、少し寒くなっても、雪は降る
\\	まい
\\	と思う。
\\	ないだろうと思う	
\\	もう春だから、少し寒くなっても、雪は
\\	降る
\\	と思う
\\	中7
\\	"丈夫な人だから、病気になっても、死ぬ
\\	まい
\\	と思っていた。
\\	ないだろうと思う	
\\	丈夫な人だから、病気になっても、死ぬ
\\	と思っていた。
\\	中7
\\	"相手のチームは弱い。勝って
\\	当然だ
\\	当たり前だ
\\	て-
\\	相手のチームは弱い。勝って
\\	だ
\\	中7
\\	"君が裏切ったのだから、彼女が怒る
\\	のももっともだ
\\	君が裏切ったのだから、彼女が怒る
\\	のも____だ
\\	中7
\\	"あなたがそう言う
\\	のはもっともだ
\\	あなたがそう言う
\\	のは____だ
\\	中7
\\	"彼の財産はない
\\	も当然だ
\\	彼の財産はない
\\	も__だ
\\	中7
\\	"彼とは2年も一緒に暮らしていて、夫婦
\\	も当然なんです
\\	彼とは2年も一緒に暮らしていて、夫婦
\\	も__なんです
\\	中7
\\	"君は仕事がある
\\	だけましだ
\\	よ。俺は首になってしまった。
\\	君は仕事がある
\\	だ___だ
\\	よ。俺は首になってしまった。
\\	中7
\\	"僕のアパートは、狭くて高いけれど、便利な
\\	だけましだ
\\	僕のアパートは、狭くて高いけれど、便利な
\\	だ___だ
\\	中7
\\	"火事で家が燃えてしまったけど、死なかった
\\	だけましだ
\\	火事で家が燃えてしまったけど、死なかった
\\	だ___だ
\\	中7
\\	"ここは
\\	一流[いちりゅう]ホテル
\\	だけあって
\\	快適だった。
\\	ここは
\\	一流[いちりゅう]ホテル
\\	あって
\\	快適だった。
\\	中7
\\	"このバッグは安い
\\	だけに
\\	すぐに壊れてしまった。
\\	このバッグは安い
\\	に
\\	すぐに壊れてしまった。
\\	中7
\\	"このいすは丈夫だ。高かった
\\	だけのことはある
\\	このいすは丈夫だ。高かった
\\	のこと_ある
\\	中7
\\	"あの人は、学校の成績がいい
\\	ばかりでなく
\\	スポーツ
\\	も
\\	できる。
\\	あの人は、学校の成績がいい
\\	スポーツ
\\	も
\\	できる。
\\	中7
\\	"この辺りは、空気
\\	ばかりでなく
\\	水
\\	も
\\	汚染[おせん]されている。
\\	この辺りは、空気
\\	水
\\	も
\\	汚染されている。
\\	中7
\\	"その映画は日本
\\	のみならず
\\	、外国
\\	も
\\	よく知られている。
\\	その映画は日本
\\	の____
\\	、外国
\\	も
\\	よく知られている。
\\	中7
\\	"この料理は味が良い
\\	のみならず
\\	、見た目
\\	も
\\	美しい。
\\	この料理は味が良い
\\	の____
\\	、見た目
\\	も
\\	美しい。
\\	中7
\\	"あの飛行機に乗った
\\	ばかりに
\\	、彼は死んだ。
\\	あの飛行機に乗った
\\	、彼は死んだ。
\\	中7
\\	"英語の先生が嫌いな
\\	ばかりに
\\	、英語も嫌いになってしまった。
\\	英語の先生が嫌いな
\\	、英語も嫌いになってしまった。
\\	中7
\\	"「それ、五千円ぐらいですか。」
\\	「ええ、
\\	その通り
\\	、五千円です。よくわかりましたね。」
\\	「それ、五千円ぐらいですか。」
\\	「ええ、
\\	、五千円です。よくわかりましたね。」
\\	中7
\\	"よく付き合っ
\\	た上で
\\	、その人を判断するようにしている。
\\	よく付き合っ
\\	た__
\\	、その人を判断するようにしている。
\\	中7
\\	"みんなの意見を聞い
\\	た上で
\\	、まとめたほうがいい。
\\	みんなの意見を聞い
\\	た__
\\	、まとめたほうがいい。
\\	中7
\\	"休みの過ごし方は人
\\	によって
\\	のんびりしたり、旅行したりです。
\\	休みの過ごし方は人
\\	に___
\\	のんびりしたり、旅行したりです。
\\	中7
\\	"車が増えること
\\	によって
\\	、空気は今よりもっときたなくなるだろう。
\\	車が増えること
\\	に___
\\	、空気は今よりもっときたなくなるだろう。
\\	中7
\\	"たった一枚の写真
\\	によって
\\	、彼女に怒られた。
\\	たった一枚の写真
\\	に___
\\	、彼女に怒られた。
\\	中7
\\	"今日はお祝いだ。みんなで飲
\\	もうではないか
\\	もうじゃないか
\\	今日はお祝いだ。みんなで飲
\\	もう__ない_
\\	中7
\\	"誰もやらないなら、僕がやってみ
\\	ようじゃないか
\\	ようではないか
\\	"誰もやらないなら、僕がやってみ
\\	よう__ない_
\\	中7
\\	"バスの事故でたくさんの若者が亡くなったそうだ。何と言ったらいいのか、
\\	言い
\\	ようがない
\\	バスの事故でたくさんの若者が亡くなったそうだ。
\\	何と言ったらいいのか、
\\	言い
\\	が__
\\	中7
\\	"この作文は、日本語がめちゃくちゃで
\\	直し
\\	ようがない
\\	この作文は、日本語がめちゃくちゃで
\\	直し
\\	が__
\\	中7
\\	"葬式[そうしき]が始まると雨が降り出した。まるで空が泣いている
\\	かのようだ
\\	葬式が始まると雨が降り出した。まるで空が泣いている
\\	の__だ
\\	中7
\\	"怖いものでも見た
\\	かのように
\\	、彼女は震えていた。
\\	怖いものでも見た
\\	の__に
\\	、彼女は震えていた。
\\	中7
\\	"最近仕事がすごく忙しくて、サークルに
\\	行け
\\	そうにない
\\	よ。
\\	最近仕事がすごく忙しくて、サークルに
\\	行け
\\	にない
\\	よ。
\\	中7
\\	"まだまだ
\\	帰れ
\\	そうにない
\\	まだまだ
\\	帰れ
\\	にない
\\	中7
\\	"お申し込み
\\	に際しては
\\	、写真が必要
\\	と[に]なります。
\\	とき 
\\	お申し込み
\\	に__ては
\\	、写真が必要となります。
\\	中8
\\	"ビビ氏は日本を
\\	訪問[ほうもん]する
\\	に際し
\\	、喜びを語った。
\\	とき 
\\	ビビ氏は日本を
\\	訪問[ほうもん]する
\\	に_し
\\	、喜びを語った。
\\	中8
\\	"この工事は市の計画
\\	に基づいて
\\	進められます。
\\	この工事は市の計画
\\	に___て
\\	進められます。
\\	中8
\\	"先週の就職説明会では、先輩たちの経験
\\	に基づく
\\	話が聞けてよかった。
\\	先週の就職説明会では、先輩たちの経験
\\	に__く
\\	話が聞けてよかった。
\\	中8
\\	"収入
\\	に応じて
\\	、税金の額は変わる。
\\	収入
\\	に__て
\\	、税金の額は変わる。
\\	中8
\\	"それぞれの国には、それぞれの気候や文化
\\	に応じた
\\	料理がある。
\\	それぞれの国には、それぞれの気候や文化
\\	に__た
\\	料理がある。
\\	中8
\\	"ペットを飼わないという条件
\\	の下で
\\	、アパートを借りた。
\\	ペットを飼わないという条件
\\	の_で
\\	、アパートを借りた。
\\	中8
\\	"山中先生の
\\	ご指導[しどう]
\\	の下
\\	、研究論文を書いています。
\\	山中先生の
\\	ご指導[しどう]
\\	の_
\\	、研究論文を書いています。
\\	中8
\\	"日本に
\\	来て
\\	以来
\\	、母の料理を食べていない。
\\	日本に
\\	来て
\\	、母の料理を食べていない。
\\	中8
\\	"入学
\\	以来
\\	、一度も授業を休んでいない。笑
\\	入学
\\	、一度も授業を休んでいない。
\\	中8
\\	"試験を受ける
\\	以上
\\	、いい点を取りたい。 
\\	からには
\\	試験を受ける
\\	、いい点を取りたい。
\\	中8
\\	"日本へ来た
\\	以上は
\\	、日本語ができるようになりたい。 
\\	からには
\\	日本へ来た
\\	は
\\	、日本語ができるようになりたい。
\\	中8
\\	"約束した
\\	からには
\\	、守るべきだ。
\\	以上
\\	約束した
\\	は
\\	、守るべきだ。
\\	中8
\\	"試合に出る
\\	からには
\\	、勝ちたい。 
\\	以上
\\	試合に出る
\\	は
\\	、勝ちたい。
\\	中8
\\	"来日の
\\	折には
\\	、ぜひこちらに
\\	立[た]ち寄[よ]りください。
\\	来日の
\\	には
\\	、ぜひこちらに
\\	立[た]ち寄[よ]りください。
\\	中8
\\	"次にお目にかかった
\\	折に
\\	、お借りしていた本を返しします。
\\	次にお目にかかった
\\	に
\\	、お借りしていた本を返しします。
\\	中8
\\	"「これからの若者は
\\	どうある
\\	べきだと思いますか。」
\\	「そうですね、勉強はもちろん、趣味をたくさん持つべきですね。」
\\	「これからの若者は
\\	べきだと思いますか。」
\\	「そうですね、勉強はもちろん、趣味をたくさん持つべきですね。」
\\	中8
\\	"父が料理の本を買ってきた。時々は自分でも料理を作りたい
\\	というのである
\\	父が料理の本を買ってきた。時々は自分でも料理を作りたい
\\	と__ので__
\\	中8
\\	"会社では、一年に一回社員旅行がある。みんなで旅行をして、仲良くなろう
\\	というのである
\\	会社では、一年に一回社員旅行がある。みんなで旅行をして、仲良くなろう
\\	と__ので__
\\	中8
\\	"外の寒さ
\\	にもかかわらず
\\	、こたつの暖かさは心を和やかにする。
\\	外の寒さ
\\	にも_____
\\	、こたつの暖かさは心を和やかにする。
\\	中8
\\	"真面目である
\\	にもかかわらず
\\	、昇進しない。
\\	真面目である
\\	にも_____
\\	、昇進しない。
\\	中8
\\	"歯が痛くて、ご飯は
\\	とても
\\	食べられ
\\	なかった
\\	ので、コーヒーで済ませた。
\\	歯が痛くて、ご飯は
\\	食べられ
\\	なかった
\\	ので、コーヒーで済ませた。
\\	中8
\\	"怖くて
\\	とても
\\	見られ
\\	ない
\\	ので、内容がわからない。
\\	怖くて
\\	見られ
\\	ない
\\	ので、内容がわからない。
\\	中8
\\	"自分は我慢しても、
\\	せめて
\\	子供たちには個室を与えようとしてきた。
\\	自分は我慢しても、
\\	子供たちには個室を与えようとしてきた。
\\	中8
\\	"客
\\	の立場
\\	から言うと
\\	、この店は入り口が狭くて入りにくい。しかし、店
\\	の側
\\	から言えば
\\	管理しやすい。
\\	客
\\	の立場
\\	か____
\\	、この店は入り口が狭くて入りにくい。しかし、店
\\	の側
\\	か____
\\	管理しやすい。
\\	中8
\\	"親
\\	の立場
\\	から言うと
\\	、子供のことを心配して当然だが子供
\\	の側
\\	から言えば
\\	、それがうるさいと感じるものだ。
\\	親
\\	の立場
\\	か____
\\	、子供のことを心配して当然だが子供
\\	の側
\\	か____
\\	、それがうるさいと感じるものだ。
\\	中8
\\	"症状[しょうじょう]
\\	からすると
\\	、心臓の病気かもしれません。
\\	症状
\\	か____
\\	、心臓の病気かもしれません。
\\	中8
\\	"周りの態度
\\	からすると
\\	、あの方が社長ではないでしょうか。
\\	周りの態度
\\	か____
\\	、あの方が社長ではないでしょうか。
\\	中8
\\	"彼は服装
\\	からして
\\	だらしない。きっと、他の面同じだろう。
\\	彼は服装
\\	か___
\\	だらしない。きっと、他の面同じだろう。
\\	中8
\\	"あの映画は、題名
\\	からして
\\	悲しそうだ。
\\	あの映画は、題名
\\	か___
\\	悲しそうだ。
\\	中8
\\	"日本の習慣には、外国人
\\	から見ると
\\	妙[みょう]なものもあるだろう。
\\	日本の習慣には、外国人
\\	か___と
\\	妙なものもあるだろう。
\\	中8
\\	"昔の人
\\	から見ると
\\	、現代人の生活のリズムは早すぎるかもしれない。
\\	昔の人
\\	か___と
\\	、現代人の生活のリズムは早すぎるかもしれない。
\\	中8
\\	"恋人に会えない時は、
\\	せめて
\\	声だけでも聞きたい。
\\	恋人に会えない時は、
\\	声だけでも聞きたい。
\\	中8
\\	"好きだ
\\	からといって
\\	、同じ食品ばかり食べるのはよくない。
\\	好きだ
\\	か____て
\\	、同じ食品ばかり食べるのはよくない。
\\	中8
\\	"お金持ちだ
\\	からといって
\\	、偉い
\\	わけではない
\\	からって
\\	お金持ちだ
\\	か____て
\\	、偉い
\\	わけではない
\\	中8
\\	"契約書の内容をよく読ん
\\	でからでないと
\\	、サインはできません。 
\\	でからでなければ
\\	契約書の内容をよく読ん
\\	で_____と
\\	、サインはできません。 
\\	中8
\\	"親の許可をもらっ
\\	てからでなければ
\\	、申し込めない。 
\\	てからでないと
\\	親の許可をもらっ
\\	て______ば
\\	、申し込めない。
\\	中8
\\	"札幌駅
\\	から
\\	すすきの駅
\\	にかけて
\\	、たくさんの居酒屋やバーがあります。
\\	からーまで 
\\	札幌駅
\\	すすきの駅
\\	、たくさんの居酒屋やバーがあります。
\\	中8
\\	"足の早さ
\\	に
\\	かけては
\\	、彼は町で一番だ。
\\	足の早
\\	さ
\\	に___は
\\	、彼は町で一番だ。
\\	中8
\\	"歌のうま
\\	さ
\\	にかけては
\\	、彼女に勝てる人はない。
\\	歌のうま
\\	さ
\\	に___は
\\	、彼女に勝てる人はない。
\\	中8
\\	"「毎日五時半に帰るのですか。」
\\	「はい、よほどのことが
\\	ない限り
\\	、そうします。」
\\	「毎日五時半に帰るのですか。」
\\	「はい、よほどのことが
\\	ない__
\\	、そうします。」
\\	中8
\\	"「この川の水を飲んでも大丈夫ですか。」
\\	「いや、検討が終わら
\\	ない限り
\\	、飲まない方がいいと思います。」
\\	「この川の水を飲んでも大丈夫ですか。」
\\	「いや、検討が終わら
\\	ない__
\\	、飲まない方がいいと思います。」
\\	中8
\\	"「ここは、こんな
\\	ふうに
\\	してください。」
\\	わかった
\\	というふうに
\\	、彼女にニッコリうなずいた。
\\	「ここは、こんな
\\	に
\\	してください。」
\\	わかった
\\	という__に
\\	、彼女にニッコリうなずいた。
\\	中8
\\	"よく見ていてください。こんな
\\	ふうに
\\	すれば、お客さんはうれしいと思う。
\\	よく見ていてください。こんな
\\	に
\\	すれば、お客さんはうれしいと思う。
\\	中8
\\	"「いつ行きますか。土曜日ですか、それとも日曜日ですか。」
\\	どちらにしても
\\	私は行けないと思います。」
\\	「いつ行きますか。土曜日ですか、それとも日曜日ですか。」
\\	ど_____も
\\	私は行けないと思います。」
\\	中8
\\	"花火をしました。
\\	もっとも
\\	花火
\\	といっても
\\	線香[せんこう]花火ですが。
\\	"花火をしました。
\\	もっとも
\\	花火
\\	と____
\\	線香[せんこう]花火ですが。
\\	中8
\\	"今夜の花火大会は、雨で中止だ
\\	とか
\\	今夜の花火大会は、雨で中止だ
\\	中8
\\	"今、インフルエンザが流行っている
\\	とか
\\	今、インフルエンザが流行っている
\\	中8
\\	"あんなひどいところ、二度と行く
\\	まい
\\	(と思います)。
\\	あんなひどいところ、二度と行く
\\	中8
\\	"海外旅行では、絶対に買った水以外は飲む
\\	まい
\\	と思った。
\\	海外旅行では、絶対に買った水以外は飲む
\\	と思った。
\\	中8
\\	"彼には私の気持ちは分かる
\\	まい
\\	ないだろう	
\\	彼には私の気持ちは分かる
\\	中8
\\	"上級者[じょうきゅうしゃ]でも、この問題はでき
\\	まい
\\	ないだろう	
\\	上級者[じょうきゅうしゃ]でも、この問題はでき
\\	中8
\\	"本当のことを話
\\	そうか
\\	話
\\	すまいか
\\	迷ったが、話した。
\\	"本当のことを話
\\	そう_
\\	話
\\	す___
\\	迷ったが、話した。
\\	中8
\\	"食べたいけど、太りたくないし、食べ
\\	ようか
\\	食べ
\\	るまい
\\	か考えているところです。 
\\	食べ
\\	ようか
\\	食べ
\\	まい
\\	食べたいけど、太りたくないし、食べ
\\	か
\\	食べ
\\	まい
\\	か考えているところです。
\\	中8
\\	"あのチームは勝つ
\\	に決まって
\\	いる。
\\	あのチームは勝つ
\\	に____
\\	いる。
\\	中8
\\	"彼が犯人
\\	に
\\	相違[そうい]ない
\\	違いない 
\\	彼が犯人
\\	に__ない
\\	中8
\\	"相手が子供から
\\	と言って
\\	、簡単に勝ってる
\\	とは限らない
\\	相手が子供から
\\	と言って
\\	、簡単に勝ってる
\\	は__ない
\\	中8
\\	"あまり使わない表現だが、必ずしもテストに出ない
\\	とは限らない
\\	あまり使わない表現だが、必ずしもテストに出ない
\\	は__ない
\\	中8
\\	"全力をつくした。あとは祈る
\\	よりほかない
\\	しかない 
\\	全力をつくした。あとは祈る
\\	ほか__
\\	中8
\\	"合格できたのは、彼の努力の
\\	結果
\\	にほかならない
\\	合格できたのは、彼の努力の
\\	結果
\\	ほか__ない
\\	中8
\\	"彼があなたをからかうのは、まさに
\\	愛情[あいじょう]表現
\\	にほかならない
\\	彼があなたをからかうのは、まさに
\\	愛情[あいじょう]表現
\\	ほか__ない
\\	中8
\\	"わが国では、野球
\\	をはじめ
\\	、サッカーやテニスなど、様々なスポーツが盛んです。 
\\	をはじめとして
\\	わが国では、野球
\\	を___
\\	、サッカーやテニスなど、様々なスポーツが盛んです。
\\	中9
\\	"会議には中国
\\	をはじめ
\\	、アジアの国々が参加した。 
\\	をはじめとして
\\	会議には中国
\\	を___
\\	、アジアの国々が参加した。
\\	中9
\\	"新しいマンションの建設
\\	をめぐって
\\	、住民と建設会社との
\\	対立
\\	が続いている。
\\	マンションの建設
\\	をめぐる
\\	対立
\\	は、何ヶ月も続いている。
\\	について 
\\	"新しいマンションの建設
\\	を____
\\	、住民と建設会社との
\\	対立
\\	が続いている。
\\	中9
\\	"憲法改正[けんぽうかいせい]
\\	をめぐって
\\	、長い間、
\\	論争[ろんそう]
\\	が続いた。
\\	について 
\\	憲法
\\	改正[かいせい]
\\	を____
\\	、長い間、
\\	論争[ろんそう]
\\	が続いた。
\\	中9
\\	"森さんの主張には、その点
\\	において
\\	疑問[ぎもん]があります。
\\	森さんの主張には、その点
\\	に___
\\	疑問[ぎもん]があります。
\\	中9
\\	"国際社会
\\	における
\\	わが国の役割を考える。
\\	国際社会
\\	に___
\\	わが国の役割を考える。
\\	中9
\\	"電話かメール
\\	にて
\\	ご連絡ください。
\\	電話かメール
\\	ご連絡ください。
\\	中9
\\	"空港到着後、バス
\\	にて
\\	登別温泉へ向かいます。
\\	空港到着後、バス
\\	登別温泉へ向かいます。
\\	中9
\\	"日本人が優勝したのは、今回が初めて
\\	ではないだろうか
\\	日本人が優勝したのは、今回が初めて
\\	では__だろう_
\\	中9
\\	"この地域の人口は、少しずつ減っ
\\	てきている
\\	この地域の人口は、少しずつ減っ
\\	て__いる
\\	中9
\\	"暖かくなっ
\\	てきています
\\	暖かくなっ
\\	て__います
\\	中9
\\	"遅れるから電話をし
\\	ないといけない
\\	ないとだめです
\\	なければならない	
\\	遅れるから電話をし
\\	ないと____
\\	中9
\\	"有名なブランドだからこれがいい
\\	というわけではない
\\	有名なブランドだからこれがいい
\\	という__では__
\\	中9
\\	"この店の野菜は安い。
\\	なぜなら
\\	すぐ裏の畑で作っている
\\	からだ
\\	この店の野菜は安い。
\\	な___
\\	すぐ裏の畑で作っている
\\	からだ
\\	中9
\\	"私は海に行きたくないんです。
\\	なぜなら
\\	泳げない
\\	からです
\\	私は海に行きたくないんです。
\\	な___
\\	泳げない
\\	からです
\\	中9
\\	"この件の説明は以上です。
\\	なお
\\	詳細[しょうさい]についてはプリントをご覧ください。
\\	この件の説明は以上です。
\\	詳細[しょうさい]についてはプリントをご覧ください。
\\	中9
\\	"今日はこれで終わります。
\\	なお
\\	次回[じかい]の
\\	日時[にちじ]は……
\\	今日はこれで終わります。
\\	次回[じかい]の日時は……
\\	中9
\\	"ファックス、
\\	あるいは
\\	メールでお知らせください。
\\	ファックス、
\\	あ___
\\	メールでお知らせください。
\\	中9
\\	"来週の火曜日の午後はどうですか。
\\	あるいは
\\	水曜日の午前でもかまいませんが……。
\\	来週の火曜日の午後はどうですか。
\\	あ___
\\	水曜日の午前でもかまいませんが……。
\\	中9
\\	"ビザ
\\	および
\\	パスポートについては、以下をご覧ください。
\\	ビザ
\\	パスポートについては、以下をご覧ください。
\\	中9
\\	"図書館の中では携帯電話
\\	および
\\	ゲーム機械を使ってはいけません。
\\	図書館の中では携帯電話
\\	ゲーム機械を使ってはいけません。
\\	中9
\\	"だれもが、そのまま試合が終わると思っていた。
\\	ところが
\\	、最後の五分にすごいドラマがあった。
\\	だれもが、そのまま試合が終わると思っていた。
\\	が
\\	、最後の五分にすごいドラマがあった。
\\	中9
\\	"確かに郵便で送った。
\\	ところが
\\	、届いていないと言われた。
\\	確かに郵便で送った。
\\	が
\\	、届いていないと言われた。
\\	中9
\\	"「会社を経営するのに、一番大切なことは……。」
\\	「優秀な人材
\\	さえ
\\	確保でき
\\	れば
\\	、後は、あまり問題ないですよ。」
\\	「会社を経営するのに、一番大切なことは……。」
\\	「優秀な人材
\\	確保でき
\\	、後は、あまり問題ないですよ。」
\\	中9
\\	"日本経済はその
\\	たびに
\\	これを乗り切ってきた。
\\	日本経済はその
\\	に
\\	これを乗り切ってきた。
\\	中9
\\	"世界経済
\\	における
\\	日本の役割は重要になっている。
\\	世界経済
\\	に___
\\	日本の役割は重要になっている。
\\	中9
\\	"法
\\	の下で
\\	はすべての人は平等だ。 
\\	法
\\	の_で
\\	はすべての人は平等だ。 
\\	中9
\\	"経済発展にばかり目を向けることによって、価値あるものを
\\	見失い
\\	つつある
\\	経済発展にばかり目を向けることによって、価値あるものを
\\	見失い
\\	つ___
\\	中9
\\	"彼女は人目
\\	も
\\	かま[構]わず
\\	子供のように泣いた。 
\\	もかまわないで
\\	彼女は人目
\\	も____
\\	子供のように泣いた。
\\	中9
\\	"彼は靴が
\\	脱げる
\\	の
\\	も
\\	かま[構]わず
\\	走り続いた。 
\\	もかまわないで
\\	彼は靴が
\\	脱げる
\\	の
\\	も____
\\	走り続いた。
\\	中9
\\	"新しい携帯は高い
\\	にもかかわらず
\\	、飛ぶように売れている。
\\	新しい携帯は高い
\\	も____ず
\\	、飛ぶように売れている。
\\	中9
\\	"雨
\\	にもかかわらず
\\	、大勢の人が集まった。
\\	雨
\\	も____ず
\\	、大勢の人が集まった。
\\	中9
\\	"来る来ない
\\	にかかわらず
\\	、連絡をください。
\\	来る来ない
\\	に____ず
\\	、連絡をください。
\\	中9
\\	"国籍
\\	に
\\	かか[関]わらず
\\	、誰でもこの学校に申し込むことができます。
\\	国籍
\\	に____ず
\\	、誰でもこの学校に申し込むことができます。
\\	中9
\\	"年齢
\\	を問わず
\\	多くの人々が集まった。
\\	年齢
\\	を__ず
\\	多くの人々が集まった。
\\	中9
\\	"経験の
\\	有無[うむ]
\\	を問わず
\\	、募集。
\\	経験の
\\	有無[うむ]
\\	を__ず
\\	、募集。
\\	中9
\\	"当社の商品は、一つ一つが手作りです。
\\	したがって
\\	、一度にお送りできる量も限られるのです。
\\	当社の商品は、一つ一つが手作りです。
\\	し____
\\	、一度にお送りできる量も限られるのです。
\\	中9
\\	"教授は急用で来られません。
\\	したがって
\\	講義は中止です。
\\	教授は急用で来られません。
\\	し____
\\	講義は中止です。
\\	中9
\\	"この季節は、目がかゆい
\\	やら
\\	鼻水が出る
\\	やら
\\	、大変です。
\\	とかーとか 
\\	この季節は、目がかゆい
\\	鼻水が出る
\\	、大変です。
\\	中9
\\	"写真を見る
\\	につけ
\\	、(歌を聞く
\\	につけ
\\	、国を思い出す。
\\	写真を見る
\\	に__
\\	、歌を聞く
\\	に__
\\	、国を思い出す。
\\	中9
\\	"いい
\\	につけ
\\	悪い
\\	につけ
\\	、子供は親に似る。
\\	いい
\\	に__
\\	悪い
\\	に__
\\	、子供は親に似る。
\\	中9
\\	"行く
\\	にしろ
\\	行かない
\\	にしろ
\\	、連絡してください。
\\	行く
\\	に__
\\	行かない
\\	に__
\\	、連絡してください。
\\	中9
\\	"受験はしない
\\	にしろ
\\	、勉強はしなさい。
\\	受験はしない
\\	に__
\\	、勉強はしなさい。
\\	中9
\\	"彼は勉強
\\	も
\\	でき
\\	れば
\\	スポーツ
\\	も
\\	でき
\\	る
\\	彼は勉強
\\	も
\\	(できる)
\\	スポーツ
\\	も
\\	(できる)
\\	中9
\\	"私は歌
\\	も
\\	下手
\\	なら
\\	ダンス
\\	も
\\	下手
\\	だ
\\	私は歌
\\	も
\\	下手
\\	ダンス
\\	も
\\	下手
\\	中9
\\	"いろいろな意見が出たが、
\\	要[よう]するに
\\	皆、私の意見に反対なのだ。
\\	いろいろな意見が出たが、
\\	要[よう]__に
\\	皆、私の意見に反対なのだ。
\\	中9
\\	"彼は一度も
\\	入賞[にゅうしょう]しなかった。
\\	要[よう]するに
\\	、才能がなかったということだ。
\\	彼は一度も
\\	入賞[にゅうしょう]しなかった。
\\	する_
\\	、才能がなかったということだ。
\\	中9
\\	"小学生、
\\	すなわち
\\	6歳以上の子供は有料になる。
\\	小学生、
\\	す___
\\	6歳以上の子供は有料になる。
\\	中9
\\	"このペットボトルには1,000ミリリットル、
\\	すなわち
\\	1リットルの水が入っています。
\\	このペットボトルには1,000ミリリットル、
\\	す___
\\	1リットルの水が入っています。
\\	中9
\\	"薬は苦い
\\	ものだ
\\	薬は苦い
\\	だ
\\	中9
\\	"地震のときは誰でも慌てる
\\	ものだ
\\	地震のときは誰でも慌てる
\\	だ
\\	中9
\\	"大地震や
\\	津波[つなみ]には遭いたくない
\\	ものだ
\\	大地震や
\\	津波[つなみ]には遭いたくない
\\	だ
\\	中9
\\	"同じ仕事をして欲しい
\\	ものだ
\\	同じ仕事をして欲しい
\\	だ
\\	中9
\\	"目上の人に対して、失礼な話し方をする
\\	もんじゃない
\\	ものではない
\\	目上の人に対して、失礼な話し方をする
\\	も___ない
\\	中9
\\	"口の中にものを入れたまましゃべる
\\	ものではない
\\	もんじゃない
\\	口の中にものを入れたまましゃべる
\\	も___ない
\\	中9
\\	"今日中にこれを全部終わらせるのは
\\	無理
\\	というものだ
\\	今日中にこれを全部終わらせるのは
\\	無理
\\	いう__だ
\\	中9
\\	"夜中に電話をしてくるのは
\\	非常識[ひじょうしき]
\\	というものだ
\\	夜中に電話をしてくるのは
\\	非常識[ひじょうしき]
\\	いう__だ
\\	中9
\\	"「駅前の新しい
\\	カフェ
\\	、どうだった?」
\\	「あんな
\\	カフェ
\\	、二度と行く
\\	もんか
\\	「駅前の新しい
\\	カフェ
\\	、どうだった?」
\\	「あんな
\\	カフェ
\\	、二度と行く
\\	も__
\\	中9
\\	"「なつみさんって、
\\	かわいい
\\	よね。」
\\	かわいい
\\	もんか
\\	。怒ったら
\\	鬼[おに]のように怖いだ。」
\\	「なつみさんって、
\\	かわいい
\\	よね。」
\\	かわいい
\\	も__
\\	。怒ったら
\\	鬼[おに]のように怖いだ。」
\\	中9
\\	"東京
\\	を中心に
\\	関東地方は台風の影響で風が強くなっています。
\\	東京
\\	を__に
\\	関東地方は台風の影響で風が強くなっています。
\\	中9
\\	"この店はスキー用品
\\	を中心とした
\\	スポーツ専門店です。
\\	この店はスキー用品
\\	を__とした
\\	スポーツ専門店です。
\\	中9
\\	"感謝の
\\	気持ち
\\	をこめて
\\	編んだマフラーです。
\\	感謝の
\\	気持ち
\\	を__て
\\	編んだマフラーです。
\\	中9
\\	"愛
\\	をこめて
\\	カードを贈る。
\\	愛
\\	を__て
\\	カードを贈る。
\\	中9
\\	"友人
\\	を通じて
\\	彼と知り合った。
\\	友人
\\	を__て
\\	彼と知り合った。
\\	中9
\\	"テレビのニュース
\\	を通じて
\\	その事件を知った。
\\	テレビのニュース
\\	を__て
\\	その事件を知った。
\\	中9
\\	"ここでは一年
\\	を通じて
\\	美しい花が見られます。
\\	ここでは一年
\\	を__て
\\	美しい花が見られます。
\\	中9
\\	"地図
\\	を
\\	頼[たよ]りに
\\	、友達に教えてもらったレストランに行った。
\\	地図
\\	を__に
\\	、友達に教えてもらったレストランに行った。
\\	中9
\\	"祖父はつえ
\\	を
\\	頼[たよ]りにして
\\	歩いている。
\\	祖父はつえ
\\	を__にして
\\	歩いている。
\\	中9
\\	"地震の際には、窓ガラスが割れたり壁が倒れたりする
\\	恐れがある
\\	地震の際には、窓ガラスが割れたり壁が倒れたりする
\\	が__
\\	中9
\\	"台風19号は今夜、
\\	四国[しこく]に
\\	上陸[じょうりく]
\\	の
\\	恐れがある
\\	台風19号は今夜、四国に
\\	上陸[じょうりく]
\\	の
\\	が__
\\	中9
\\	"満員電車で毎日通勤するのはつらい
\\	ものがある
\\	満員電車で毎日通勤するのはつらい
\\	が_る
\\	中9
\\	"ここまで完成していろいろのにあきらめなければならないなんて、残念な
\\	ものがある
\\	ここまで完成していろいろのにあきらめなければならないなんて、残念な
\\	が_る
\\	中9
\\	"何でも多
\\	ければ
\\	いい
\\	というものでもない
\\	何でも多
\\	け__
\\	いい
\\	いう__でも__
\\	中9
\\	"親
\\	なら
\\	何を言ってもいい
\\	というものではない
\\	親
\\	何を言ってもいい
\\	いう__では__
\\	中9
\\	"最近、変なメールがたくさんくる。
\\	どうにか
\\	なら
\\	ないものか
\\	最近、変なメールがたくさんくる。
\\	どうにか
\\	なら
\\	もの_
\\	中9
\\	"デジカメが壊れた。なんとか治ら
\\	ないもの(だろう)か
\\	デジカメが壊れた。なんとか治ら
\\	もの(だろう)_
\\	中9
\\	"この小説は事実
\\	をもとに
\\	書かれた。
\\	この小説は事実
\\	もと_
\\	書かれた。
\\	中9
\\	"失敗
\\	をもとにして
\\	発明する。
\\	失敗
\\	もと_して
\\	発明する。
\\	中9
\\	"この機械はただ今調整中
\\	につき
\\	、ご使用になりません。
\\	この機械はただ今調整中
\\	に__
\\	、ご使用になりません。
\\	中9
\\	"本日は祭日
\\	につき
\\	、休業させていただきます。
\\	本日は祭日
\\	に__
\\	、休業させていただきます。
\\	中9
\\	"大学入学
\\	をきっかけに
\\	引っ越す。
\\	大学入学
\\	き___に
\\	引っ越す。
\\	中9
\\	"病気
\\	をきっかけに
\\	酒をやめた。
\\	病気
\\	き___に
\\	酒をやめた。
\\	中9
\\	"卒業
\\	を
\\	契機[けいき]に
\\	独立する。
\\	おきっかけに	
\\	卒業
\\	を__に
\\	独立する。
\\	中9
\\	"昨年の事故
\\	を
\\	契機[けいき]として
\\	、安全対策が強化された。
\\	おきっかけに	
\\	昨年の事故
\\	を__として
\\	、安全対策が強化された。
\\	中9
\\	"受験
\\	の
\\	際に
\\	、写真が必要です。
\\	受験
\\	の
\\	に
\\	、写真が必要です。
\\	中9
\\	"申し込んだ
\\	際
\\	、住所を間違って書いてしまった。
\\	申し込んだ
\\	、住所を間違って書いてしまった。
\\	中9
\\	"この時計は高かった。
\\	それなのに
\\	すぐに壊れた。
\\	この時計は高かった。
\\	それ__に
\\	すぐに壊れた。
\\	中9
\\	"一生懸命勉強している。
\\	それなのに
\\	成績はよくならない。
\\	一生懸命勉強している。
\\	れなの_
\\	成績はよくならない。
\\	中9
\\	"外は大雨だ。
\\	それでも
\\	出かけないとはいけない。
\\	外は大雨だ。
\\	それ__
\\	出かけないとはいけない。
\\	中9
\\	"みんなに反対されている。
\\	それでも
\\	私は彼と結婚したい。
\\	みんなに反対されている。
\\	でも
\\	私は彼と結婚したい。
\\	中9
\\	"「道がすごく渋滞しているようだよ。」
\\	それなら
\\	、電車で行こう。」
\\	「道がすごく渋滞しているようだよ。」
\\	それ__
\\	、電車で行こう。」
\\	中9
\\	"「今、それ、やりたくない。」
\\	それなら
\\	やらなくていいよ。」
\\	「今、それ、やりたくない。」
\\	それ__
\\	やらなくていいよ。」
\\	中9
\\	"父は働きすぎた。
\\	それで
\\	病気になった。
\\	父は働きすぎた。
\\	で
\\	病気になった。
\\	中9
\\	"「田中君、インフルエンザにかっかたんだって。」
\\	それで
\\	学校を休んだんだね。」
\\	「田中君、インフルエンザにかっかたんだって。」
\\	で
\\	学校を休んだんだね。」
\\	中9
\\	"「竹林ちゃんにお金を貸してくれて言われたんだ。」
\\	「へえー。
\\	(それ)で
\\	、貸してあげたの?」
\\	「竹林ちゃんにお金を貸してくれて言われたんだ。」
\\	「へえー。
\\	それ_
\\	、貸してあげたの?」
\\	中9
\\	"「今日、面接を受けたんだ。」
\\	(それ)で
\\	、どうだったの?」
\\	「今日、面接を受けたんだ。」
\\	、どうだったの?」
\\	中9
\\	"20人
\\	足らず
\\	の田舎の小さな町工場ですが。
\\	20人
\\	の田舎の小さな町工場ですが。
\\	中9
\\	"運動なしで、1ヶ月
\\	足らず
\\	で4キロ
\\	減[げん]!
\\	運動なしで、1ヶ月
\\	で4キロ
\\	減[げん]!
\\	中9
\\	"今だけ!
\\	約
\\	割引!
\\	今だけ!
\\	割引!
\\	中9
\\	"「お嬢さん、もう大学をご卒業されましたでしょう?」
\\	それが
\\	、まだなんですよ。」
\\	「お嬢さん、もう大学をご卒業されましたでしょう?」
\\	が
\\	、まだなんですよ。」
\\	中9
\\	"「来週から旅行でしたよね。」
\\	それが
\\	家の事情で中止になったんです。」
\\	「来週から旅行でしたよね。」
\\	そ__
\\	家の事情で中止になったんです。」
\\	中9
\\	"明日は車が
\\	混[こ]むらしい。
\\	そこで
\\	我々は朝早く出発するつもりだ。
\\	明日は車が
\\	混[こ]むらしい。
\\	で
\\	我々は朝早く出発するつもりだ。
\\	中9
\\	"「今度南米に旅行します。
\\	そこで
\\	お願いがあるのですが……。」
\\	「今度南米に旅行します。
\\	そ__
\\	お願いがあるのですが……。」
\\	中9
\\	そういえば
\\	、池田君、元気かな?」
\\	そう___
\\	、池田君、元気かな?」
\\	中9
\\	"「いい家ですね。
\\	そういえば
\\	、お父様は
\\	設計士[せっけいし]さんでしたよね。」
\\	「いい家ですね。
\\	ば
\\	、お父様は
\\	設計士[せっけいし]さんでしたよね。」
\\	中9
\\	"タバコをやめると
\\	誓[ちか]った。
\\	それが
\\	、たった三日でまた吸ってしまった。
\\	それなのに	
\\	タバコをやめると
\\	誓った。
\\	それ_
\\	、たった三日でまた吸ってしまった。
\\	中9
\\	"日本に2年間ぐらい住もうと思った。
\\	それが
\\	今は4年目になった。
\\	それなのに	
\\	日本に2年間ぐらい住もうと思った。
\\	が
\\	今は4年目になった。
\\	中9
\\	"「今日の授業、面白かったね。」
\\	「うん、すごくためになった。
\\	それはそうと
\\	、今度の試験いつだっけ?」
\\	「今日の授業、面白かったね。」
\\	「うん、すごくためになった。
\\	それは___
\\	、今度の試験いつだっけ?」
\\	中9
\\	"「春になりましたね。
\\	それはそうと
\\	、小林さんの息子さん、大学受かったでしょうか。」
\\	「春になりましたね。
\\	そうと
\\	、小林さんの息子さん、大学受かったでしょうか。」
\\	中9
\\	"今朝車庫のドアが開けられない
\\	ほど
\\	夜の間に雪が降りました。
\\	今朝車庫のドアが開けられない
\\	夜の間に雪が降りました。
\\	中9
\\	"「疲れていますか。」
\\	「いや、それ
\\	ほど
\\	ではありません。散歩を続けましょうよ。」
\\	「疲れていますか。」
\\	「いや、それ
\\	ではありません。散歩を続けましょうよ。」
\\	中9
\\	"きれい
\\	だ
\\	とは言えませんが
\\	、面白いでしょう。
\\	な
\\	のではありませんが
\\	きれい
\\	だ
\\	言えません_
\\	、面白いでしょう。
\\	中9
\\	"あなたが嫌い
\\	な
\\	のではありませんが
\\	、一緒にあのパーティーに行きたくないです。
\\	だ
\\	とは言えませんが
\\	あなたが嫌いな
\\	ありません_
\\	、一緒にあのパーティーに行きたくないです。
\\	中9
\\	"「母の兄、
\\	すなわち
\\	僕の叔父ですが……。」
\\	「母の兄、
\\	す___
\\	僕のおじですが……。」
\\	中9
\\	"これは難しいチャレンジだ。
\\	だが
\\	失敗を恐れてはいけない。
\\	しかし 
\\	けれども	
\\	これは難しいチャレンジだ。
\\	失敗を恐れてはいけない。
\\	中9
\\	"生活は貧しい。
\\	だが
\\	幸せだ。
\\	しかし 
\\	けれども	
\\	生活は貧しい。
\\	幸せだ。
\\	中9
\\	"「何を怒ってるの?」
\\	だって
\\	、約束を破ったじゃないか。」
\\	「何を怒ってるの?」
\\	、約束を破ったじゃないか。」
\\	中9
\\	"「テストの点、よかったんだって?」
\\	「うん。
\\	だって
\\	やさしかったんだもん。」
\\	「テストの点、よかったんだって?」
\\	「うん。
\\	やさしかったんだもん。」
\\	中9
\\	"「彼は、まだ来ませんね。
\\	ということは
\\	、欠席ということですね。」
\\	つまり	
\\	「彼は、まだ来ませんね。
\\	こと_
\\	、欠席ということですね。」
\\	中9
\\	"「私はもうお酒を飲めます。」
\\	ということは
\\	二十歳を過ぎてるんだね。」
\\	つまり	
\\	「私はもうお酒を飲めます。」
\\	こと_
\\	二十歳を過ぎてるんだね。」
\\	中9
\\	"「今日は家を出られないんです。
\\	というのは
\\	父の具合が悪くなりまして……。」 
\\	とは
\\	「今日は家を出られないんです。
\\	と___は
\\	父の具合が悪くなりまして……。」
\\	中9
\\	"「僕は卵を食べないんです。
\\	というのは
\\	、アレルギーがあるんです。」 
\\	とは
\\	「僕は卵を食べないんです。
\\	と___は
\\	、アレルギーがあるんです。」
\\	中9
\\	"彼は
\\	真面目[まじめ]で誠実[せいじつ]な人だ。
\\	したがって
\\	みんなから
\\	信頼[しんらい]されている。
\\	彼は
\\	真面目
\\	で誠実[せいじつ]な人だ。
\\	し____
\\	みんなから信頼
\\	されている。
\\	中9
\\	"全商品[ぜんしょうひん]3割引です。
\\	ただし
\\	、この棚の商品は除きます。
\\	全商品3割引です。
\\	、この棚の商品は除きます。
\\	中9
\\	"明日9時に
\\	集合[しゅうごう]です。
\\	但し[ただし]
\\	、雨の場合は中止です。
\\	明日9時に集合です。
\\	、雨の場合は中止です。
\\	中9
\\	"「品[しな]はいい。
\\	ただ[只]
\\	値段が高すぎる。」
\\	「品はいい。
\\	値段が高すぎる。」
\\	中9
\\	"「僕はかまわない。
\\	ただ
\\	妻は何と言うか。」
\\	「僕はかまわない。
\\	妻は何と言うか。」
\\	中9
\\	"窓が割れていない
\\	ところから
\\	、泥棒が
\\	鍵[かぎ]を持っていたことが分かります。
\\	ことから
\\	うえに
\\	以上
\\	窓が割れていない
\\	と___ら
\\	、泥棒がかぎを持っていたことが分かります。
\\	中9
\\	"「雨が降っていますか。」
\\	「人々が傘をさしていない
\\	ところから
\\	、降っていないと思いますよ。
\\	ことから
\\	うえに
\\	以上
\\	「雨が降っていますか。」
\\	「人々が傘をさしていない
\\	と___ら
\\	、降っていないと思いますよ。
\\	中9
\\	"テレビを
\\	見
\\	ることがあります
\\	テレビを
\\	(見る)
\\	ことが____
\\	中9
\\	"佐藤さんは映画館に
\\	行
\\	くことはありません
\\	が、お家で
\\	を
\\	見
\\	ることが
\\	よく
\\	あります
\\	佐藤さんは映画館に
\\	(行く)
\\	こと_ありません
\\	が、お家で
\\	を
\\	(見る)
\\	こと_
\\	よく
\\	あります
\\	中9
\\	"検査の
\\	前夜[ぜんや]から
\\	飲食[いんしょく]禁止です。
\\	もっとも
\\	水を飲んでもかまいません。
\\	検査の
\\	前夜[ぜんや]から
\\	飲食[いんしょく]禁止です。
\\	水を飲んでもかまいません。
\\	中9
\\	"全員が参加しなければなりません。
\\	もっとも
\\	病気の場合は別です。
\\	全員が参加しなければなりません。
\\	病気の場合は別です。
\\	中9
\\	"「これで授業を終わります。
\\	さて
\\	来週の予定ですが……。」
\\	「これで授業を終わります。
\\	来週の予定ですが……。」
\\	中9
\\	"「以上、今日のニュースをお伝えしました。
\\	さて
\\	次に天気予報です。」
\\	「以上、今日のニュースをお伝えしました。
\\	次に天気予報です。」
\\	中9
\\	"薬を塗った。
\\	すると
\\	痛みが治まった。
\\	薬を塗った。
\\	と
\\	痛みが治まった。
\\	中9
\\	"窓を開けた。
\\	すると
\\	蛾[が]が入ってきた。
\\	窓を開けた。
\\	す__
\\	蛾[が]が入ってきた。
\\	中9
\\	"彼女は美人で頭がいい。
\\	しかも
\\	性格もいい。
\\	彼女は美人で頭がいい。
\\	し__
\\	性格もいい。
\\	しかも
\\	中9
\\	"先日泊まったホテルは、古くて高かった。
\\	しかも
\\	サービスが悪かったので、もう二度と行かない。
\\	先日泊まったホテルは、古くて高かった。
\\	も
\\	サービスが悪かったので、もう二度と行かない。
\\	しかも
\\	中9
\\	"日本の夏は暑いし、
\\	おまけに
\\	湿気も多いです。
\\	日本の夏は暑いし、
\\	お___
\\	湿気も多いです。
\\	中9
\\	"あの
\\	蕎麦[そば]屋は高いしまずい。
\\	おまけに
\\	サービスも悪い。
\\	あのそば屋は高いしまずい。
\\	お____
\\	サービスも悪い。
\\	中9
\\	"最近は、ペットを飼う家が増えているようです。
\\	ちなみに
\\	うちにも犬が1匹と猫が2匹います。
\\	最近は、ペットを飼う家が増えているようです。
\\	ち___
\\	うちにも犬が1匹と猫が2匹います。
\\	中9
\\	"燃えるゴミは月曜日と木曜日、燃えないゴミは金曜日に出してください。
\\	ちなみに
\\	資源ごみの
\\	収集日[しゅうしゅうひ]は第3水曜日です。
\\	燃えるゴミは月曜日と木曜日、燃えないゴミは金曜日に出してください。
\\	ち___
\\	資源ごみの
\\	収集日[しゅうしゅうひ]は第3水曜日です。
\\	中9
\\	"大学生
\\	の
\\	三[さん]分の
\\	一[いち]
\\	は全然勉強しません。
\\	大学生
\\	の
\\	は全然勉強しません。
\\	中10
\\	"今晩のお客さん
\\	の五分の四
\\	は日本人です。
\\	"今晩のお客さん
\\	の
\\	は日本人です。
\\	中10
\\	"絞り
\\	たて
\\	の牛乳はおいしい。
\\	たばかり	
\\	しぼり
\\	の牛乳はおいしい。
\\	中10
\\	"あの女性は一度やると決めたら、何があってもやめ
\\	たりしない
\\	あの女性は一度やると決めたら、何があってもやめ
\\	し__
\\	中10
\\	"「お母さん、怒らない?」
\\	「試験の点数なんかで怒っ
\\	たりしません
\\	よ。」
\\	"「お母さん、怒らない?」
\\	「試験の点数なんかで怒っ
\\	し___
\\	よ。」
\\	中10
\\	"会場内は禁煙
\\	につき
\\	、おタバコはご
\\	遠慮[えんりょ]ください。
\\	会場内は禁煙
\\	に__
\\	、おタバコはご
\\	遠慮ください。
\\	中10
\\	"工場の見学は予約された方
\\	に限ります
\\	工場の見学は予約された方
\\	に____
\\	中10
\\	"彼
\\	に限って
\\	浮気[うわき]なんてするはずがない。
\\	彼
\\	に___
\\	浮気[うわき]なんてするはずがない。
\\	中10
\\	"発言するしない
\\	にかかわらず
\\	、自分の意見はまとめておくべきです。
\\	発言するしない
\\	に_____
\\	、自分の意見はまとめておくべきです。
\\	中10
\\	"試験中は携帯電話の電源を切る
\\	こと
\\	筆記用具[ひっきようぐ]以外は机の上に置かない
\\	こと
\\	試験中は携帯電話の電源を切る
\\	筆記用具[ひっきようぐ]以外は机の上に置かない
\\	中10
\\	"「加藤さんは結婚する
\\	んだって
\\	「そうらしですね。」
\\	「加藤さんは結婚する
\\	「そうらしですね。」
\\	中10
\\	"彼ほ日本語がほとんど話せない
\\	ながらも
\\	、身振りで言いたいことを伝えようとしていた。
\\	も 
\\	彼ほ日本語がほとんど話せない
\\	も
\\	身振り[みぶり]で言いたいことを伝えようとしていた。
\\	中10
\\	"休暇は1年に12日あります。
\\	つまり
\\	1ヶ月に1日は休める
\\	ってことだ
\\	ということです
\\	休暇は1年に12日あります。
\\	1ヶ月に1日は休める
\\	ことだ
\\	中10
\\	"「冬の寒い朝ってなかなかベッドから出られない
\\	よね
\\	「うん。」
\\	「冬の寒い朝ってなかなかベッドから出られない
\\	「うん。」
\\	中10
\\	"「ポテトチップスって食べ始めると、なかなかやめられないんだ
\\	よね
\\	「本当に。」
\\	「ポテトチップスって食べ始めると、なかなかやめられないんだ
\\	「本当に。」
\\	中10
\\	"何も食べ
\\	ないで
\\	、学校へ行きました。
\\	何も食べ
\\	なくて
\\	、学校に寝てしまいました。
\\	何も食べ
\\	な__
\\	、学校へ行きました。
\\	何も食べ
\\	な__
\\	、学校に寝てしまいました。
\\	中10
\\	"少しも休
\\	まないで
\\	休まずに
\\	、勉強しています。
\\	少しも
\\	休ま___
\\	休ま
\\	、勉強しています。
\\	中10
\\	"そのどれもが、「ドラゴンボール」
\\	といった
\\	ヒット作品を目指している。
\\	などのような
\\	そのどれもが、「ドラゴンボール」
\\	いっ_
\\	ヒット作品を目指している。
\\	中10
\\	"五月5日には「ちまき」「かしわもち」
\\	といった
\\	昔からの菓子を食べる習慣がある。
\\	などのような
\\	五月5日には「ちまき」「かしわもち」
\\	いっ_
\\	昔からの菓子を食べる習慣がある。
\\	中10
\\	"1秒にも満たない動作の間に主人公の頭に浮かんだ光景が10分間
\\	に(も)わたって
\\	描かれる。
\\	1秒にも満たない動作の間に主人公の頭に浮かんだ光景が10分間
\\	に____
\\	描かれる。
\\	中10
\\	"砂漠[さばく]は東西4500キロ
\\	にわたって
\\	広がっている。
\\	砂漠[さばく]は東西4500キロ
\\	に____
\\	広がっている。
\\	中10
\\	"3年間ずっとアルバイトとして
\\	働く
\\	うちに
\\	、仕事を認められて社員になることができた。
\\	(ている) 
\\	(辞書) 
\\	3年間ずっとアルバイトとして
\\	働く
\\	に
\\	、仕事を認められて社員になることができた。
\\	中10
\\	"この時計は
\\	使っている
\\	うちに
\\	、自然に動かなくなってしまった。
\\	(ている) 
\\	(辞書) 
\\	この時計は
\\	使っている
\\	に
\\	、自然に動かなくなってしまった。
\\	中10
\\	"子供
\\	にとって
\\	生まれたときから存在しているアニメは、今やなくてはならない
\\	娯楽[ごらく]となっている。
\\	子供
\\	に___
\\	生まれたときから存在しているアニメは、今やなくてはならない娯楽となっている。
\\	中10
\\	"海外で日本のテレビアニメが受ける
\\	わけ
\\	とは
\\	何だろうか。
\\	というのは
\\	って
\\	海外で日本のテレビアニメが受けるわけ
\\	何だろうか。
\\	中10
\\	"江戸時代
\\	において
\\	もっと力を持っていたのは誰だろうか。
\\	で 
\\	"江戸時代
\\	に___
\\	もっと力を持っていたのは誰だろうか。
\\	中10
\\	"このチーズは、価格は前と同じだが、20グラム少なくなっている。値上げをした
\\	わけである
\\	わけだ
\\	このチーズは、価格は前と同じだが、20グラム少なくなっている。値上げをした
\\	で__
\\	中10
\\	"内村さんは毎日のように、ヨガ、マッサージ、スポーツジムに通っている。元気
\\	な
\\	わけである
\\	わけだ
\\	内村さんは毎日のように、ヨガ、マッサージ、スポーツジムに通っている。元気
\\	な
\\	で__
\\	中10
\\	"道路を広げる計画には反対意見が多い。実現は難しい
\\	のではないだろうか
\\	と思っています	
\\	道路を広げる計画には反対意見が多い。実現は難しい
\\	の__ない___か
\\	中10
\\	"情報が少なすぎて不安だ。もう少し情報がもらえたら、住民も安心できる
\\	のではないだろうか
\\	と思っています	
\\	情報が少なすぎて不安だ。もう少し情報がもらえたら、住民も安心できる
\\	では__だろう_
\\	中10
\\	"「銀河鉄道999[ぎんがてつどうスリーナイン]」って、どんな話だった
\\	っけ
\\	「銀河鉄道999」って、どんな話だった
\\	中10
\\	"クレアは
\\	鉄郎の暖かい手に触れて、「血の通った体になりたい」って悲し
\\	げ
\\	に言うんだ。
\\	クレアは
\\	鉄郎[てつろう]の暖かい手に触れて、「血の通った体になりたい」って悲し
\\	に言うんだ。
\\	中10
\\	"子供のとき川に落ちて
\\	以来
\\	、水が怖くて今も泳げない。
\\	子供のとき川に落ちて
\\	、水が怖くて今も泳げない。
\\	中10
\\	"日本には富士山
\\	をはじめ
\\	、たくさんの美しい山がある。
\\	日本には富士山
\\	を___
\\	、たくさんの美しい山がある。
\\	中10
\\	"子育ては夫婦の協力
\\	の下で
\\	行われるべきだ。
\\	子育ては夫婦の協力
\\	で
\\	行われるべきだ。
\\	中10
\\	"この温泉は、日本人
\\	はもとより
\\	、外国人に
\\	も
\\	大変人気があります。
\\	もちろん	
\\	この温泉は、日本人
\\	は____
\\	、外国人に
\\	大変人気があります
\\	中10
\\	"犯罪防止[はんざいぼうし]
\\	はもとより
\\	、地域の交通安全
\\	も
\\	警察の大切な仕事だ。
\\	もちろん	
\\	犯罪防止
\\	は____
\\	、地域の交通安全
\\	警察の大切な仕事だ。
\\	中10
\\	"さっきのお前の態度は何だ。人が話しているときには、ちゃんと聞く
\\	もんだ
\\	ぞ。
\\	ものだ
\\	よ
\\	さっきのお前の態度は何だ。人が話しているときには、ちゃんと聞く
\\	ぞ。
\\	中10
\\	"日本の農業を理解する
\\	上で
\\	気候や地形に対する理解も必要だ。
\\	日本の農業を理解する
\\	気候や地形に対する理解も必要だ。
\\	中10
\\	"この洗濯機は、旧型
\\	ながら
\\	とても使いやすい。
\\	この洗濯機は、旧型
\\	とても使いやすい。
\\	中10
\\	"「緑の会」は環境
\\	保護[ほご]活動
\\	を
\\	目的
\\	とする
\\	市民の組織だ。
\\	「緑の会」は環境
\\	保護活動
\\	目的
\\	と__
\\	市民の組織だ。
\\	中10
\\	"結婚
\\	を
\\	前提[ぜんてい]
\\	に
\\	彼女に交際を申し込んだ。
\\	として
\\	結婚
\\	前提[ぜんてい]
\\	彼女に交際を申し込んだ。
\\	中10
\\	"韓国のドラマを
\\	きっかけ
\\	として、韓国文化に関心を持つようになった。
\\	韓国のドラマを
\\	として、韓国文化に関心を持つようになった。
\\	中10
\\	"日本での就職を希望する
\\	からには
\\	、しっかり企業研究をしたほうがいい。
\\	以上は
\\	日本での就職を希望する
\\	か___
\\	、しっかり企業研究をしたほうがいい。
\\	中10
\\	"あのレストランはおいしい
\\	わけでもない
\\	のに、いつも混んでいる。
\\	あのレストランはおいしい
\\	でも__
\\	のに、いつも混んでいる。
\\	中11
\\	"彼は一言も文句言う
\\	ことなく
\\	、重い荷物を運んでいた。
\\	彼は一言も文句言う
\\	こ___
\\	、重い荷物を運んでいた。
\\	中11
\\	"今回は優勝することができましたが、これで満足する
\\	ことなく
\\	、さらに努力を続けます。これからも、応援よろしくお願いします。
\\	今回は優勝することができましたが、これで満足する
\\	こ___
\\	、さらに努力を続けます。これからも、応援よろしくお願いします。
\\	中11
\\	"多くの家庭で収入が減少している
\\	にもかかわらず
\\	、貯金額は増加している。
\\	多くの家庭で収入が減少している
\\	に_____ず
\\	、貯金額は増加している。
\\	中11
\\	"入社後は企業人
\\	として
\\	の
\\	自覚[じかく]を持って行動してください。
\\	入社後は企業人
\\	の自覚を持って行動してください。
\\	中11
\\	"仕事をしている
\\	限り
\\	、嫌なことももちろんあるが、そこから学ぶことも多い。
\\	仕事をしている
\\	、嫌なことももちろんあるが、そこから学ぶことも多い。
\\	中11
\\	"僕が知っている
\\	限りでは
\\	、電器製品はこの店が一番安いです。
\\	僕が知っている
\\	では
\\	、電器製品はこの店が一番安いです。
\\	中11
\\	"同僚から聞いた
\\	限りでは
\\	、今度の部長は仕事に厳しいらしいよう。
\\	同僚から聞いた
\\	では
\\	、今度の部長は仕事に厳しいらしいよう。
\\	中11
\\	"日本は食料を輸入に頼
\\	らざるを得ない
\\	状態だ。
\\	日本は食料を輸入に頼
\\	ら__を___
\\	状態だ。
\\	中11
\\	"試合の結果
\\	はともかく
\\	(として)
\\	、最後まで全力で
\\	戦[たたか]うことができたの満足だ。
\\	試合の結果
\\	は____として
\\	、最後まで全力で
\\	戦[たたか]うことができたの満足だ。
\\	中11
\\	"今の仕事は、給料
\\	はともかく
\\	、やりがいがあるいい仕事だと思っています。
\\	今の仕事は、給料
\\	は____
\\	、やりがいがあるいい仕事だと思っています。
\\	中11
\\	"寝不足で運転したら事故を起し
\\	かねない
\\	よ。
\\	寝不足で運転したら事故を起し
\\	よ。
\\	中11
\\	"お年寄りはちょっと転んだだけでも骨折し
\\	かねない
\\	から、注意が必要だ。
\\	お年寄りはちょっと転んだだけでも骨折し
\\	から、注意が必要だ。
\\	中11
\\	"日本で最も古い大学が京都にある
\\	という
\\	そうです	
\\	日本で最も古い大学が京都にある
\\	中11
\\	電球は省エネ
\\	性能[せいのう]や寿命の長さで優れている。
\\	普通の電球の8分の1から5分の1の電気代で済み、寿命は40倍ある
\\	という
\\	そうです	
\\	電球は省エネ
\\	性能[せいのう]や寿命の長さで優れている。
\\	普通の電球の8分の1から5分の1の電気代で済み、寿命は40倍ある
\\	中11
\\	"隣のうちのお嬢さんは会う
\\	たび[度]に
\\	きれいになっている。
\\	隣のうちのお嬢さんは会う
\\	に
\\	きれいになっている。
\\	中11
\\	"今日の講演会
\\	に関して
\\	ご意見のある方はこの紙に書いて出口の箱にお入れください。
\\	について	
\\	今日の講演会
\\	に_し_
\\	ご意見のある方はこの紙に書いて出口の箱にお入れください。
\\	中11
\\	"鈴木氏は今度の選挙に出る
\\	のではないか
\\	鈴木氏は今度の選挙に出る
\\	ではない_
\\	中11
\\	"様々な意見が出て会議が混乱しているので、調整が必要
\\	なのではないか
\\	と思う
\\	様々な意見が出て会議が混乱しているので、調整が必
\\	要
\\	の__
\\	か
\\	と思う
\\	中11
\\	"彼はまだお酒が飲まない年齢だ。
\\	未成年[みせいねん]
\\	のだ
\\	んです
\\	のである
\\	彼はまだお酒が飲まない年齢だ。
\\	未成年[みせいねん]
\\	中11
\\	"この商品は国内では販売されていない。言い換えれば、海外でしか買えない
\\	のである
\\	のだ
\\	この商品は国内では販売されていない。言い換えれば、海外でしか買えない
\\	中11
\\	"確かに優勝しましたが、
\\	国民栄誉賞[こくみんえいようしょう]をいただく
\\	ほどのものじゃありません
\\	確かに優勝しましたが、
\\	国民栄誉賞[こくみんえいようしょう]をいただく
\\	の__じゃありません
\\	中11
\\	"「朝食を食べましたか。」
\\	「朝食
\\	という
\\	ほどのものじゃない
\\	けど、バナナを食べました。」
\\	「朝食を食べましたか。」
\\	「朝食
\\	という
\\	の__じゃない
\\	けど、バナナを食べました。」
\\	中11
\\	"狭い庭なんですよ。庭といえる
\\	ほどのものじゃありません
\\	狭い庭なんですよ。庭といえる
\\	の__じゃありません
\\	中11
\\	"この店はパンを売れる
\\	だけじゃなく
\\	、パンの作り方教室も開いている。
\\	この店はパンを売れる
\\	じゃ__
\\	、パンの作り方教室も開いている。
\\	中11
\\	"ボランティアの活動は相手のため
\\	だけ
\\	で[じゃ]なく
\\	、自分のためにもなることがわかった。
\\	ボランティアの活動は相手のため
\\	で__
\\	、自分のためにもなることがわかった。
\\	中11
\\	"この絵は絵
\\	というより
\\	、まるで写真のようだ。
\\	この絵は絵
\\	と____
\\	、まるで写真のようだ。
\\	中11
\\	"「星の王子様」は子供向け
\\	というより
\\	、大人のための本だ。
\\	「星の王子様」は子供向け
\\	と____
\\	、大人のための本だ。
\\	中11
\\	"新入社員が入って、君たちも先輩になるのですから、いつまでも甘え
\\	てはいられません
\\	よ。
\\	新入社員が入って、君たちも先輩になるのですから、いつまでも甘え
\\	は______
\\	よ。
\\	中11
\\	"やっ
\\	てらんねぇ
\\	よ。
\\	やっ
\\	ちゃいられない
\\	よ。
\\	"やっていられませんよ。
\\	中11
\\	"その会社は消費者の要望
\\	に応じて
\\	、商品の品質検査を強化した。
\\	その会社は消費者の要望
\\	に__て
\\	、商品の品質検査を強化した。
\\	中11
\\	"時代の変化
\\	に応じて
\\	若者の文化や考え方も変わる。
\\	時代の変化
\\	に__て
\\	若者の文化や考え方
\\	も
\\	変わる。
\\	中11
\\	のトラブル
\\	による
\\	被害は、この銀行者にとどまらない。
\\	で	
\\	のトラブル
\\	に__
\\	被害は、この銀行者にとどまらない。
\\	中11
\\	"外部からの情報引き出しによってか、データ
\\	流失[りゅうしつ]が起きたもの
\\	とみられる
\\	と考える 
\\	外部からの情報引き出しによってか、データ
\\	流失[りゅうしつ]が起きたもの
\\	と____
\\	中11
\\	"自動車業界は
\\	都南[とうなん]アジアでの
\\	需要[じゅよう]はまだまだ伸びる
\\	とみている
\\	と考えている 
\\	自動車業界は
\\	都南[とうなん]アジアでの
\\	需要はまだまだ伸びる
\\	と____
\\	中11
\\	"本日は年末の忙しい時期
\\	にもかかわらず
\\	、こんなに多くの方にお集まりいただきありがとうございます。
\\	のに	
\\	本日は年末の忙しい時期
\\	に______
\\	、こんなに多くの方にお集まりいただきありがとうございます。
\\	中11
\\	"政府
\\	は
\\	景気が回復するまでは消費税を上げない
\\	としている
\\	"政府
\\	景気が回復するまでは消費税を上げない
\\	と____
\\	中12
\\	"学校側
\\	は
\\	少子化[しょうしか]に備えてカリキュラムを見直す
\\	としている
\\	学校側
\\	少子化[しょうしか]に備えてカリキュラムを見直す
\\	と____
\\	中12
\\	"社名の変更
\\	と共に
\\	制服
\\	も
\\	新しいデザインになった。
\\	社名の変更
\\	と_に
\\	制服
\\	も
\\	新しいデザインになった。
\\	中12
\\	"身分証明証[みぶんしょうめいしょ]が必要かどうか確かめ
\\	たところ
\\	、不要だということだった。
\\	身分証明証[みぶんしょうめいしょ]が必要かどうか確かめ
\\	た___
\\	、不要だということだった。
\\	中12
\\	"新しい出会いを期待
\\	しつつ
\\	、パーティーに出かけた。
\\	ながら	
\\	新しい出会いを期待
\\	(する)
\\	、パーティーに出かけた。
\\	中11
\\	"台風で
\\	電線[でんせん]が切れ、この町は
\\	全域[ぜんいき]
\\	にわたって
\\	停電した。
\\	台風で
\\	電線[でんせん]が切れ、この町は
\\	全域[ぜんいき]
\\	に____
\\	停電した。
\\	中11
\\	"海で日焼けして首
\\	から
\\	背中
\\	にかけて
\\	、赤くなってしまった。
\\	海で日焼けして首
\\	背中
\\	に___
\\	、赤くなってしまった。
\\	中11
\\	"噴火[ふんか]の
\\	おそれがあります
\\	ので、避難してください。
\\	噴火の
\\	があります
\\	ので、避難してください。
\\	中11
\\	"科学技術の進歩
\\	と共に
\\	、宇宙の
\\	謎[なぞ]が明らかになっていくだろう。
\\	科学技術の進歩
\\	と_に
\\	、宇宙の
\\	謎[なぞ]が明らかになっていくだろう。
\\	中12
\\	"驚いた
\\	ことに
\\	その子は難しい文章をすらすら読んだ。
\\	驚いた
\\	に
\\	その子は難しい文章をすらすら読んだ。
\\	中12
\\	"残念な
\\	ことに
\\	、試験に落ちてしまった。
\\	残念な
\\	に
\\	、試験に落ちてしまった。
\\	中12
\\	"家族の協力
\\	を抜きに
\\	(して)
\\	、私の成功はなかった。
\\	家族の協力
\\	を__に
\\	、私の成功はなかった。
\\	中12
\\	"この計画は彼
\\	抜きに
\\	は進められない。
\\	この計画は彼
\\	に
\\	は進められない。
\\	中12
\\	何かにつけ
\\	、近所の人にお世話になっている。
\\	何____
\\	、近所の人にお世話になっている。
\\	中12
\\	"合格できたのは、日々の努力の結果
\\	にほかならない
\\	合格できたのは、日々の努力の結果
\\	ならない
\\	中12
\\	"電気料金が
\\	あんまり
\\	高いもんだ
\\	から
\\	、調べてもらったら、やっぱり電力会社の間違いだった。
\\	電気料金が
\\	高いもんだ
\\	、調べてもらったら、やっぱり電力会社の間違いだった。
\\	中12
\\	"電話をかけていた相手の言葉遣いが
\\	あんまり
\\	失礼だった
\\	から
\\	、思わず切ってしまった。
\\	電話をかけていた相手の言葉遣いが
\\	失礼だった
\\	、思わず切ってしまった。
\\	中12
\\	"たばこの火がカーテンに燃え移っていた。気づくのが遅れたら、火事に
\\	なるところだった
\\	たばこの火がカーテンに燃え移っていた。気づくのが遅れたら、火事に
\\	(なる?)
\\	ところ
\\	(だ?)
\\	中12
\\	"「明日は漢字のテストだよ。」
\\	「あっ、そうだね。忘れ
\\	るところだった
\\	。ありがとう。」
\\	「明日は漢字のテストだよ。」
\\	「あっ、そうだね。
\\	(忘れる?)
\\	ところ
\\	(だ?)
\\	。ありがとう。」
\\	中12
\\	"デートの約束をしている日
\\	に限って
\\	、残業を頼まれる。
\\	デートの約束をしている日
\\	に___
\\	、残業を頼まれる。
\\	中12
\\	"彼は「ごめん」と言っ
\\	た
\\	きり
\\	、黙ってしまった。
\\	彼は「ごめん」と言っ
\\	た
\\	、黙ってしまった。
\\	中12
\\	"妻は赤ん坊の世話に
\\	かかり
\\	っきり
\\	なので、掃除や洗濯は俺がしている。
\\	妻は赤ん坊の世話に
\\	(かかる)
\\	っきり
\\	なので、掃除や洗濯は俺がしている。
\\	中12
\\	"彼女が一人
\\	きり
\\	で夜道を歩くのは危険だ。
\\	彼女が一人
\\	で夜道を歩くのは危険だ。
\\	中12
\\	"「海水浴どうだった?楽しかった?」
\\	「人が多くて、ゆっくり泳ぐ
\\	どころじゃなかった
\\	よ。」
\\	「海水浴どうだった?楽しかった?」
\\	「人が多くて、ゆっくり泳ぐ
\\	じゃなかった
\\	よ。」
\\	中12
\\	"「授業が終わったらカラオケ行かない?」
\\	「カラオケ
\\	どころじゃない
\\	よ!レポート、書かなきゃ。明日締切りなんだ。」
\\	「授業が終わったらカラオケ行かない?」
\\	「カラオケ
\\	じゃない
\\	よ!レポート、書かなきゃ。明日締切りなんだ。」
\\	中12
\\	"水泳教室に通って
\\	(は)
\\	いる
\\	ものの
\\	、いまだに25メートルしか泳げない。
\\	のに 
\\	水泳教室に通って
\\	(は)
\\	いる
\\	、いまだに25メートルしか泳げない。
\\	中12
\\	"うれしい
\\	ことに
\\	、うちの学校が合唱コンクールで優勝したんですよ。
\\	うれしい
\\	、うちの学校が合唱コンクールで優勝したんですよ。
\\	中12
\\	"今人気のエリナはモデル
\\	にしては
\\	背が高いほうではない。
\\	今人気のエリナはモデル
\\	して_
\\	背が高いほうではない。
\\	中12
\\	"このお弁当は300円
\\	にしては
\\	量も多いし味もいい。
\\	このお弁当は300円
\\	に__は
\\	量も多いし味もいい。
\\	中12
\\	"週末、台風が来そうで心配だ。運動会が中止になったら、楽しみにしている娘が
\\	どんなに
\\	がっかりする
\\	ことか
\\	週末、台風が来そうで心配だ。運動会が中止になったら、楽しみにしている娘が
\\	どんなに
\\	がっかりする
\\	か
\\	中12
\\	"人は私のことを頭がいいと言うけど、合格するために、
\\	どれだけ
\\	勉強した
\\	ことか
\\	。私の努力は誰も知らないでしょうね。
\\	人は私のことを頭がいいと言うけど、合格するために、
\\	どれだけ
\\	勉強した
\\	こと_
\\	。私の努力は誰も知らないでしょうね。
\\	中12
\\	"日本は47都道府県
\\	から
\\	成[な]って
\\	いる。
\\	日本は47都道府県
\\	か____
\\	いる。
\\	中12
\\	"10人の科学者
\\	から
\\	な[成]る
\\	研究グループによって、調査が行われた。
\\	10人の科学者
\\	か___
\\	研究グループによって、調査が行われた。
\\	中12
\\	"北海道のお土産
\\	としては
\\	、クッキーやチョコレートなどが有名である。
\\	北海道のお土産
\\	と___
\\	、クッキーやチョコレートなどが有名である。
\\	中12
\\	"マンガのテーマ
\\	としては
\\	、「恋愛」や「冒険」など
\\	好[この]まれる。
\\	マンガのテーマ
\\	と___
\\	、「恋愛」や「冒険」など
\\	好[この]まれる。
\\	中12
\\	"家の中でテレビを長時間つけているのは教育
\\	上[じょう]
\\	よくない。
\\	"家の中でテレビを長時間つけているのは教育
\\	よくない。
\\	中12
\\	"雨の日に傘をさして自転車に乗るのは交通安全
\\	上[じょう]
\\	、非常に危険である。
\\	雨の日に傘をさして自転車に乗るのは交通安全
\\	、非常に危険である。
\\	中12
\\	"この会社は、工場を海外に移したこと
\\	により
\\	、コストを下げるに成功した。
\\	(で 
\\	この会社は、工場を海外に移したこと
\\	、コストを下げるに成功した。
\\	中12
\\	"夫にスーパーの袋を捨てないように注意した
\\	ことから
\\	、けんかになった。
\\	夫にスーパーの袋を捨てないように注意した
\\	こ___
\\	、けんかになった。
\\	中12
\\	"発掘調査[はっくつちょうさ]で指輪やネックレスが発見された
\\	ことから
\\	、この墓は身分の高い人のものだと考えられる。
\\	発掘調査[はっくつちょうさ]で指輪やネックレスが発見された
\\	こ___
\\	、この墓は身分の高い人のものだと考えられる。
\\	中12
\\	"頂上まであとすこしのところで吹雪に遭い、引き返
\\	さざるを得なかった
\\	頂上まであとすこしのところで吹雪に遭い、引き返
\\	さ__を_なかった
\\	中12
\\	"参加者が予想よりはるかに少なかった。残念だが、今日のエベントは失敗だと言
\\	わざるを得ない
\\	参加者が予想よりはるかに少なかった。残念だが、今日のエベントは失敗だと言
\\	わ__を___
\\	中12
\\	"子供を持っ
\\	てはじめて
\\	親のありがたさが分かった。
\\	子供を持っ
\\	て____
\\	親のありがたさが分かった。
\\	中13
\\	"就職し
\\	てはじめて
\\	お金を
\\	稼[かせ]ぐことの大変さを知りました。
\\	就職し
\\	て____
\\	お金を
\\	稼ぐことの大変さを知りました。
\\	中13
\\	"「お父さんは優しいよ。お母さんのほうが怖い。」
\\	「優太[ゆうた]
\\	ってば
\\	ったら
\\	は 
\\	「お父さんは優しいよ。お母さんのほうが怖い。」
\\	「優太[ゆうた]
\\	中13
\\	"お母さん
\\	ったら
\\	子
\\	、どうして子供の名前を間違えて呼ぶのよ。たった3人なのに。
\\	ってば
\\	は 
\\	お母さん
\\	、どうして子供の名前を間違えて呼ぶのよ。たった3人なのに。
\\	中13
\\	"彼女のピアノの腕は
\\	素人[しろうと]
\\	にしては
\\	相当のものだ。
\\	彼女のピアノの腕は素人
\\	に___
\\	相当のものだ。
\\	中13
\\	"このレポートは
\\	一晩[ひとばん]で書いた
\\	にしては
\\	よくできている。
\\	このレポートは一晩で書いた
\\	に___
\\	よくできている。
\\	中13
\\	"俺は
\\	負けず嫌い[まけずぎらい]だ。ゲームでも何でも、やる
\\	からには
\\	勝たなければならないと思う。
\\	俺は負けず嫌いだ。ゲームでも何でも、やる
\\	か___
\\	勝たなければならないと思う。
\\	中13
\\	"10時だ。子供はもう寝る時間
\\	だろっ
\\	。歯を磨いて、ベッドに入りなさい。 
\\	だろう
\\	だろ
\\	でしょ
\\	でしょう
\\	10時だ。子供はもう寝る時間
\\	。歯を磨いて、ベッドに入りなさい。
\\	中13
\\	"飲みに行こうって誘ったのは君
\\	だろ
\\	。今日になってキャンセルなんて、ひどいよ。
\\	でしょ
\\	飲みに行こうって誘ったのは君
\\	。今日になってキャンセルなんて、ひどいよ。
\\	中13
\\	"そちらのご都合
\\	さえ
\\	よ
\\	ければ
\\	、明日伺わせていただきます。
\\	そちらのご都合
\\	よ
\\	け__
\\	、明日伺わせていただきます。
\\	中13
\\	"残念だったね、桜ちゃん。転び
\\	さえ
\\	しな
\\	ければ
\\	、一位だったなのに…
\\	残念だったね、桜ちゃん。転び
\\	しな
\\	け__
\\	、一位だったなのに…
\\	中13
\\	"出張の予定だったが、大雪で飛行機が
\\	欠航[けっこう]してしまったので
\\	行き
\\	ようがない
\\	出張の予定だったが、大雪で飛行機が
\\	欠航[けっこう]してしまったので
\\	行き
\\	ない
\\	中13
\\	"ごめん、ごめん。携帯電話をなくちゃって、連絡
\\	し
\\	ようがなかった
\\	んだ。
\\	ごめん、ごめん。携帯電話をなくちゃって、連絡
\\	し
\\	なかった
\\	んだ。
\\	中13
\\	"さっきのお客さん、あれこれ
\\	試着[しちゃく]した
\\	あげく
\\	、何も買わずに帰っちゃって…
\\	さっきのお客さん、あれこれ
\\	試着[しちゃく]した
\\	、何も買わずに帰っちゃって…
\\	中13
\\	"税金を無駄遣いした
\\	あげく
\\	、消費税を引き上げるなんて許せない。
\\	税金を無駄遣いした
\\	、消費税を引き上げるなんて許せない。
\\	中13
\\	"「どうも、すみませ~ん」
\\	「何、笑ってるんだ!謝るときにはへらへら笑う
\\	もんじゃない
\\	「どうも、すみませ~ん」
\\	「何、笑ってるんだ!謝るときにはへらへら笑う
\\	じゃ__
\\	中13
\\	"「忙しいから、休みなんてもらえないだろうな。」
\\	「今日、課長機嫌がいいから、
\\	頼む
\\	だけ
\\	頼んで
\\	み
\\	たら?」
\\	「忙しいから、休みなんてもらえないだろうな。」
\\	「今日、課長機嫌がいいから、
\\	頼む
\\	頼んで
\\	たら?」
\\	中13
\\	"このドレス、素敵だよね。似合わないかもしれないけど、
\\	着る
\\	だけ
\\	着て
\\	み
\\	よう
\\	かな。
\\	このドレス、素敵だよね。似合わないかもしれないけど、
\\	着る
\\	着て
\\	よう
\\	かな。
\\	中13
\\	"背が2センチ足りない
\\	ばかりに
\\	、警察官になれなかった。
\\	背が2センチ足りない
\\	に
\\	、警察官になれなかった。
\\	中13
\\	"虫に刺されたくらいで病院に行く
\\	ことはない
\\	よ。2、3日で治るから。
\\	虫に刺されたくらいで病院に行く
\\	こと_ない
\\	よ。2、3日で治るから。
\\	中13
\\	"いちごはレモン
\\	に比べて
\\	、ビタミン
\\	が多いんだって。ほんとかな?
\\	より	
\\	いちごはレモン
\\	に___
\\	、ビタミン
\\	が多いんだって。ほんとかな?
\\	中13
\\	"「今度のアルバイト、それを袋に入れるだけ?楽そうね。」
\\	「楽
\\	な
\\	もんか
\\	。1日に何千個も入れるんだよ。」
\\	「今度のアルバイト、それを袋に入れるだけ?楽そうね。」
\\	「楽
\\	な
\\	も__
\\	。1日に何千個も入れるんだよ。」
\\	中13
\\	"「日本のチーム、優勝できたよね。」
\\	「優勝!?それは期待しすぎ
\\	というもの
\\	だろう。」
\\	「日本のチーム、優勝できたよね。」
\\	「優勝!?それは期待しすぎ
\\	と____
\\	だろう。」
\\	中13
\\	"私が社長を批判したなんて、とんでもない。それは誤解
\\	というもの
\\	ですよ。
\\	私が社長を批判したなんて、とんでもない。それは誤解
\\	と____
\\	ですよ。
\\	中13
\\	"害虫[がいちゅう]による松の被害は県内全域
\\	に及んで
\\	おり、元の状態に回復するにはかなりに時間がかかるだろう。
\\	害虫[がいちゅう]による松の被害は県内全域
\\	に___
\\	おり、元の状態に回復するにはかなりに時間がかかるだろう。
\\	中13
\\	"議論は内国問題に
\\	留[とど]まらず国際問題
\\	に
\\	まで
\\	及び
\\	、今回の会議は非常に実りのあるものとなった。
\\	議論は内国問題に
\\	留まらず国際問題
\\	に
\\	まで
\\	、今回の会議は非常に実りのあるものとなった。
\\	中13
\\	"あの学生は基礎的な学力があるし、努力家だから、これから伸びる
\\	可能性がある
\\	あの学生は基礎的な学力があるし、努力家だから、これから伸びる
\\	がある
\\	中13
\\	"一度失敗すると、あとのつけは
\\	数百年[すうひゃくねん]に及ぶ
\\	可能性がある
\\	一度失敗すると、あとのつけは数百年に及ぶ
\\	がある
\\	中13
\\	"「生きるべきか死ぬべきかそれが問題だ」。
\\	この
\\	言葉
\\	はシェークスピアの「ハムレット」に出てくるものだ。
\\	「生きるべきか死ぬべきかそれが問題だ」。
\\	の
\\	言葉
\\	はシェークスピアの「ハムレット」に出てくるものだ。
\\	中13
\\	"「ワールドカップ2014でドイツが優勝した」。
\\	この
\\	ニュース
\\	を聞いたとき私はブラジルの北にいた。
\\	「ワールドカップ2014でドイツが優勝した」。
\\	の
\\	ニュース
\\	を聞いたとき私はブラジルの北にいた。
\\	中13
\\	"今回の災害は今後の
\\	防災[ぼうさい]を考える
\\	上で
\\	、非常に重要なものとなるに違いない。
\\	今回の災害は今後の
\\	防災[ぼうさい]を考える
\\	、非常に重要なものとなるに違いない。
\\	中13
\\	"人間が成長する
\\	上で
\\	、愛情は欠かせないものだ。
\\	人間が成長する
\\	、愛情は欠かせないものだ。
\\	中13
\\	"あの時謝ったけれど、時間が経つ
\\	につれて
\\	、腹が立ってきた。
\\	あの時謝ったけれど、時間が経つ
\\	に___
\\	、腹が立ってきた。
\\	中13
\\	"調べが進む
\\	につれて
\\	、事実が明らかになると思われる。
\\	調べが進む
\\	に___
\\	、事実が明らかになると思われる。
\\	中13
\\	"悲しい
\\	ことに
\\	、インドネシアには
\\	絶滅[ぜつめつ]の恐れ
\\	の
\\	ある
\\	鳥類[ちょうるい]
\\	が141種もいます。
\\	"悲しい
\\	、インドネシアには
\\	絶滅[ぜつめつ]の恐れのある
\\	鳥類[ちょうるい]が141種もいます。
\\	中13
\\	"やけどの
\\	恐れがあります
\\	から、この機械に絶対に触らないでください。
\\	やけどの
\\	があります
\\	から、この機械に絶対に触らないでください。
\\	中13
\\	"アマゾン
\\	熱帯雨林[ねったいうりん]の
\\	素晴[すば]らしさは言う
\\	までもありません
\\	アマゾン
\\	熱帯雨林[ねったいうりん]の
\\	素晴[すば]らしさは言う
\\	ありません
\\	中13
\\	"彼女の返事は聞く
\\	までもない
\\	。イエスに決まっている。
\\	"彼女の返事は聞く
\\	ない
\\	。イエスに決まっている。
\\	中13
\\	"彼の息子なら大丈夫だろう。会う
\\	までもない
\\	さ。
\\	"彼の息子なら大丈夫だろう。会う
\\	ない
\\	さ。
\\	中13
\\	"小学生の頃プラネタリウムを見たこと
\\	がきっかけで
\\	、宇宙に興味を持つようになった。
\\	小学生の頃プラネタリウムを見たこと
\\	が____で
\\	、宇宙に興味を持つようになった。
\\	中13
\\	"市長
\\	をはじめ
\\	、皆様のご協力で今日のこの日を迎えることができました。
\\	市長
\\	を___
\\	、皆様のご協力で今日のこの日を迎えることができました。
\\	中13
\\	"奥様
\\	にしても
\\	ご主人がノーベル賞するとは当日まで知らなかったということです。
\\	奥様
\\	して_
\\	ご主人がノーベル賞するとは当日まで知らなかったということです。
\\	中13
\\	"フランスの南西では水不足が続いているので、風呂の水ひとつ
\\	にしても
\\	使う量が
\\	制限[せいげん]されているらしい。
\\	フランスの南西では水不足が続いているので、風呂の水ひとつ
\\	して_
\\	使う量が
\\	制限されているらしい。
\\	中13
\\	"この病気のことは家族に
\\	さえ
\\	相談できない。
\\	この病気のことは家族に
\\	相談できない。
\\	中13
\\	"疲れたとき
\\	は
\\	寝る
\\	に限る
\\	疲れたとき
\\	寝る
\\	限る
\\	中13
\\	"和菓子[わがし]
\\	は
\\	京都
\\	に限る
\\	和菓子[わがし]
\\	京都
\\	限る
\\	中13
\\	"ワインを買ってきて。赤
\\	でも
\\	白
\\	でも
\\	いいけどイタリアのワインね。
\\	"ワインを買ってきて。赤
\\	白
\\	いいけどイタリアのワインね。
\\	中13
\\	"彼は中国語
\\	でも
\\	韓国語
\\	でも
\\	理解できる。
\\	"彼は中国語
\\	韓国語
\\	理解できる。
\\	中13
\\	"さすがに大学院生
\\	とあって
\\	、どの論文を読めばいいか、よく知っている。
\\	"さすがに大学院生
\\	あ__
\\	、どの論文を読めばいいか、よく知っている。
\\	中13
\\	"水曜日は女性が割引料金で見られる
\\	とあって
\\	、映画館は仕事帰りの女性ばかりだ。
\\	"水曜日は女性が割引料金で見られる
\\	あ__
\\	、映画館は仕事帰りの女性ばかりだ。
\\	中13
\\	"押す
\\	やら
\\	引く
\\	やら
\\	いろいろやってみたが、このドアはいっこうに開けない。
\\	押す
\\	引く
\\	いろいろやってみたが、このドアはいっこうに開けない。
\\	中13
\\	"姉
\\	といい
\\	、兄
\\	といい
\\	、みんな会社員になってしまった。父の店を守るのは私以外にいない。
\\	姉
\\	と__
\\	、兄
\\	と__
\\	、みんな会社員になってしまった。父の店を守るのは私以外にいない。
\\	中14
\\	"ここは、味
\\	といい
\\	サービス
\\	といい
\\	、最高のレストランだ。
\\	"ここは、味
\\	と__
\\	サービス
\\	と__
\\	、最高のレストランだ。
\\	中14
\\	"工事中
\\	につき
\\	バス停の場所を移動しました。
\\	工事中
\\	に__
\\	バス停の場所を移動しました。
\\	中14
\\	"俺は金閣寺
\\	というような
\\	派手なお寺より、
\\	三千院[さんぜんいん]
\\	といったような
\\	地味なお寺のほうが好きだ。 
\\	といった
\\	俺は金閣寺
\\	ような
\\	派手なお寺より、
\\	三千院[さんぜんいん]
\\	ような
\\	地味なお寺のほうが好きだ。
\\	中14
\\	"医師からの説明は、入院前、手術前、手術後
\\	といった
\\	段階で丁寧にいたします。 
\\	というような
\\	といったような
\\	医師からの説明は、入院前、手術前、手術後
\\	と___
\\	段階で丁寧に致します。
\\	中14
\\	"子供の病気を直したい
\\	ばかりに
\\	、父親は無理をして働き、とうとう病気になってしまった。
\\	子供の病気を直したい
\\	に
\\	、父親は無理をして働き、とうとう病気になってしまった。
\\	中14
\\	"ローマ
\\	にしても
\\	アテネ
\\	にしても
\\	、古代遺跡が多く残る都市では地下鉄を作るのに時間がかかる。 
\\	にしろ
\\	にしろ
\\	ローマ
\\	して_
\\	アテネ
\\	して_
\\	、古代遺跡が多く残る都市では地下鉄を作るのに時間がかかる。
\\	中14
\\	"この先生のゼミに入るためには、中国語
\\	にしろ
\\	、韓国語
\\	にしろ
\\	、アジアの言葉を最低一つ勉強しなければならない。
\\	にしても
\\	にしても
\\	この先生のゼミに入るためには、中国語
\\	し_
\\	、韓国語
\\	し_
\\	、アジアの言葉を最低一つ勉強しなければならない。
\\	中14
\\	"研究者にとって「しつこさ」
\\	も
\\	長所
\\	なら
\\	、「あきらめの早さ」
\\	も
\\	長所
\\	だ。場合によって、この二つを使い分ける必要がある。
\\	研究者にとって「しつこさ」
\\	長所
\\	、「あきらめの早さ」
\\	長所
\\	だ。場合によって、この二つを使い分ける必要がある。
\\	中14
\\	"医者が1人しかいないクリニック
\\	も
\\	病院
\\	なら
\\	、何十もの
\\	診療科[しんりょうか]
\\	も
\\	病院
\\	である。自分の
\\	病状[びょうじょう]に合わせて病院を選ぶことが必要だ。
\\	医者が1人しかいないクリニック
\\	病院
\\	、何十もの
\\	診療科[しんりょうか]
\\	病院
\\	である。自分の
\\	病状[びょうじょう]に合わせて病院を選ぶことが必要だ。
\\	中14
\\	何
\\	を食べる
\\	にせよ
\\	、栄養のバランスを考えることが必要だ。
\\	何
\\	を食べる
\\	に__
\\	、栄養のバランスを考えることが必要だ。
\\	中14
\\	何
\\	を選択する
\\	にせよ
\\	、返事をメールで知らせてください。
\\	何
\\	を選択する
\\	に__
\\	、返事をメールで知らせてください。
\\	中14
\\	"最後の問題はそれまでの問題
\\	と違って
\\	かなり難しい。
\\	最後の問題はそれまでの問題
\\	と___
\\	かなり難しい。
\\	中14
\\	"彼女はおしゃべりな姉
\\	と違って
\\	無口[むくち]な女性だ。
\\	彼女はおしゃべりな姉
\\	違っ_
\\	無口[むくち]な女性だ。
\\	中14
\\	"東日本が濃い味が好まれる
\\	のに対して
\\	、西日本は薄味が好まれる。
\\	東日本が濃い味が好まれる
\\	に_して
\\	、西日本は薄味が好まれる。
\\	中14
\\	"都市の人口は増えている
\\	のに対して
\\	農村[のそん]の人口は減ってきている。
\\	都市の人口は増えている
\\	に_して
\\	農村[のそん]の人口は減ってきている。
\\	中14
\\	"自分で決めた
\\	のだから
\\	、最後まであきらめずに頑張りなさい。
\\	自分で決めた
\\	だ__
\\	、最後まであきらめずに頑張りなさい。
\\	中14
\\	"急いでください。時間がない
\\	んですから
\\	急いでください。時間がない
\\	です__
\\	中14
\\	"忙しい
\\	からか
\\	、お金がない
\\	からか
\\	、最近松島さんがゴルフに来なくなった。
\\	ためか
\\	忙しい
\\	から_
\\	、お金がない
\\	から_
\\	、最近松島さんがゴルフに来なくなった。
\\	中14
\\	"寝不足
\\	からか
\\	、一日中、頭が痛かった。
\\	ためか
\\	寝不足
\\	か
\\	、一日中、頭が痛かった。
\\	中14
\\	"彼女は自信家でプライドが高い
\\	反面
\\	、傷つきやすく、他人の評価を気にする性格だった。
\\	彼女は自信家でプライドが高い
\\	、傷つきやすく、他人の評価を気にする性格だった。
\\	中14
\\	"工業の発展は人類の生活を豊かにした
\\	反面
\\	、美しい自然を
\\	破壊[はかい]することになった。
\\	工業の発展は人類の生活を豊かにした
\\	、美しい自然を
\\	破壊[はかい]することになった。
\\	中14
\\	"デュランさんは20年以上日本に住んでいる
\\	だけあって
\\	、日本語はペラペラだ。
\\	デュランさんは20年以上日本に住んでいる
\\	け___
\\	、日本語はペラペラだ。
\\	中14
\\	"建築家の自宅
\\	だけあって
\\	、おしゃれで機能的につくられている。
\\	"建築家の自宅
\\	け___
\\	、おしゃれで機能的につくられている。
\\	中14
\\	"インフルエンザが流行っている
\\	ためか
\\	、病院の
\\	待合室[まちあいしつ]は混雑している。
\\	からか
\\	"インフルエンザが流行っている
\\	ため_
\\	、病院の
\\	待合室[まちあいしつ]は混雑している。
\\	中14
\\	"若い
\\	だけに
\\	、なんでもすぐに覚えられる。
\\	"若い
\\	に
\\	、なんでもすぐに覚えられる。
\\	中14
\\	"きっと合格すると期待していた
\\	だけに
\\	、不合格の知らせにがっかりした。
\\	"きっと合格すると期待していた
\\	に
\\	、不合格の知らせにがっかりした。
\\	中14
\\	"どこか具合でも悪い
\\	のか
\\	、朝から渡辺さんは元気がない。
\\	"どこか具合でも悪い
\\	か
\\	、朝から渡辺さんは元気がない。
\\	中14
\\	"忙しい
\\	のか
\\	、最近、黒川ちゃんから連絡が来ない。
\\	忙しい
\\	の_
\\	、最近、黒川ちゃんから連絡が来ない。
\\	中14
\\	"だれかとけんかでもした
\\	のか
\\	、娘が学校へ行きたくないと言った。
\\	"だれかとけんかでもした
\\	か
\\	、娘が学校へ行きたくないと言った。
\\	中14
\\	"一生懸命頼んだ
\\	ものの
\\	、結局引き受けてはもらえなかった。
\\	"一生懸命頼んだ
\\	の
\\	、結局引き受けてはもらえなかった。
\\	中14
\\	"世の中には
\\	型
\\	にあら
\\	ざる
\\	ものはない
\\	、といってもいいすぎではない。
\\	世の中には型
\\	にあら
\\	ものはない
\\	、といってもいいすぎではない。
\\	中14
\\	"歴史にはまだまだ
\\	知られ
\\	ざる
\\	事実があるはずだ。
\\	歴史にはまだまだ
\\	知られ
\\	事実があるはずだ。
\\	中14
\\	"上は宗教から、芸術
\\	から
\\	、生活
\\	に至るまで
\\	型
\\	にはまってないものは一つとしてありません。
\\	上は宗教から、芸術
\\	から
\\	、生活
\\	に__まで
\\	、型にはまってないものは一つとしてありません。
\\	中14
\\	"自転車のねじ
\\	から
\\	人工衛星の部分
\\	に至るまで
\\	、どれもこの工場で作っています。
\\	自転車のねじ
\\	から
\\	人工衛星の部分
\\	に__まで
\\	、どれもこの工場で作っています。
\\	中14
\\	"クラシック
\\	から
\\	に至るまで
\\	、当店ではどんなジャンルの音楽でもご用意しております。
\\	クラシック
\\	至るまで
\\	、当店ではどんなジャンルの音楽でもご用意しております。
\\	中14
\\	"その竹の一片に彼の肉体と精神をまかせ
\\	きった
\\	ことと思います。
\\	その竹の一片に彼の肉体と精神をまかせ
\\	ことと思います。
\\	中14
\\	"彼はマラソンで42.195
\\	を走り
\\	きった
\\	彼はマラソンで42.195
\\	を走り
\\	中14
\\	"赤ちゃんは安心し
\\	きった
\\	表情で母親の胸で眠っている。
\\	赤ちゃんは安心し
\\	表情で母親の胸で眠っている。
\\	中14
\\	"お見舞いの品を選ぶ
\\	上で
\\	、気をつけなければならないことはどんなことですか。
\\	辞書形)	
\\	お見舞いの品を選ぶ
\\	上_
\\	、気をつけなければならないことはどんなことですか。
\\	中14
\\	"次回の授業には、この論文を読んだ
\\	上で
\\	参加してください。
\\	た形)	
\\	次回の授業には、この論文を読んだ
\\	上_
\\	参加してください。
\\	中14
\\	"株式会社
\\	とはいうものの
\\	、社員は5人しかいない。
\\	"株式会社
\\	と_いう__の
\\	、社員は5人しかいない。
\\	中14
\\	"コンサートには観客が100人くらいはくるだろうと思っていたが、100人
\\	どころか
\\	200人も来た。
\\	コンサートには観客が100人くらいはくるだろうと思っていたが、100人
\\	ど___
\\	200人も来た。
\\	中14
\\	"夕方になっても雨はやむ
\\	どころか
\\	、ますます激しくなった。
\\	夕方になっても雨はやむ
\\	ど___
\\	、ますます激しくなった。
\\	中14
\\	"慌て
\\	ては
\\	、普段できることも失敗しますよ。落ち着いてください。
\\	慌て
\\	は
\\	、普段できることも失敗しますよ。落ち着いてください。
\\	中14
\\	"全員が参加し
\\	ては
\\	、会場に入れなくなる。
\\	全員が参加し
\\	は
\\	、会場に入れなくなる。
\\	中14
\\	"それが、永遠の別れになるとは、かみ
\\	ならぬ
\\	私には、予想もできなかった。
\\	ではない
\\	それが、永遠の別れになるとは、神
\\	なら_
\\	私には、予想もできなかった。
\\	中14
\\	"それは、しかし天才
\\	ならぬ
\\	我々にとって、唯一の、宮崎へ近づく道でおります。
\\	ではない
\\	それは、しかし天才
\\	ぬ
\\	我々にとって、唯一の、宮崎へ近づく道でおります。
\\	中14
\\	"いつか宇宙に行きたいと思っていたが、それがついに夢
\\	ならぬ
\\	現実となった。
\\	ではない
\\	いつか宇宙に行きたいと思っていたが、それがついに夢
\\	なら_
\\	現実となった。
\\	中14
\\	"非常用として3日分の水と
\\	食糧[しょくりょう]を蓄えておき
\\	さえ
\\	すれ
\\	ば
\\	、あとは何とかなる。
\\	非常用として3日分の水と食糧を蓄えておき
\\	すれ
\\	、あとは何とかなる。
\\	中14
\\	"似ている声はありますが、調べてみると同じ声は
\\	一
\\	つ
\\	として
\\	あり
\\	ません
\\	"似ている声はありますが、調べてみると同じ声は
\\	つ
\\	あり
\\	ません
\\	中14
\\	"皆が
\\	励[はげ]まし合った結果、
\\	一
\\	人
\\	として
\\	やめた人はい
\\	なかった
\\	"皆が
\\	励まし合った結果、
\\	人
\\	やめた人はい
\\	なかった
\\	中14
\\	"相手が「うん」と言わぬ
\\	以上
\\	、あきらめるしかありません。
\\	"相手が「うん」と言わぬ
\\	、あきらめるしかありません。
\\	中14
\\	"家賃が払えない
\\	以上
\\	、出て行くしかない。
\\	"家賃が払えない
\\	、出て行くしかない。
\\	中14
\\	"俺が病気にでもな
\\	らぬ
\\	限り
\\	、この店は売りません。
\\	俺が病気にでもな
\\	らぬ
\\	、この店は売りません。
\\	中14
\\	"あきらめ
\\	ない
\\	限り
\\	、チャンスは必ず来ると思う。
\\	あきらめ
\\	ない
\\	、チャンスは必ず来ると思う。
\\	中14
\\	"どんなに生活に困っても、子供の学費のため貯金してきたこのお金を使う
\\	わけにはいかない
\\	わけにはゆかない
\\	"どんなに生活に困っても、子供の学費のため貯金してきたこのお金を使う
\\	には__ない
\\	中14
\\	"失業中[しつぎょうちゅう]だからといって、親に頼る
\\	わけにはいきません
\\	わけにはゆきません
\\	失業中だからといって、親に頼る
\\	には__ません
\\	中14
\\	"驚き
\\	のあまりに
\\	、手に持っていたカップを落としてしまった。
\\	驚き
\\	の___に
\\	、手に持っていたカップを落としてしまった。
\\	中14
\\	"山中先生は、明日試験があることを知っていた
\\	くせに
\\	、教えてくられなかった。
\\	山中先生は、明日試験があることを知っていた
\\	に
\\	、教えてくられなかった。
\\	中14
\\	"お前、自分では料理が作れない
\\	くせに
\\	、いつもあたしが作った料理に文句を言っちゃう。
\\	お前、自分では料理が作れない
\\	に
\\	、いつもあたしが作った料理に文句を言っちゃう。
\\	中14
\\	"約束を破っ
\\	てみろ
\\	、絶対に許さないからな。
\\	約束を破っ
\\	て__
\\	、絶対に許さないからな。
\\	中14
\\	"古里の公園の桜はもう散っている
\\	に違いない
\\	古里の公園の桜はもう散っている
\\	違い__
\\	中14
\\	"郊外
\\	は
\\	都心
\\	に比べて
\\	緑が多い。
\\	より, 
\\	郊外
\\	都心
\\	に__て
\\	緑が多い。
\\	中14
\\	"お金って、なかなか
\\	貯[た]まら
\\	ないもんです
\\	ね。
\\	お金って、なかなか
\\	貯[た]まら
\\	ない__です
\\	ね。
\\	中14
\\	"出された食事は残す
\\	ものではない
\\	出された食事は残す
\\	ではない
\\	中14
\\	"チロ!チロ!どこにいるんだ?おー、い
\\	た
\\	、い
\\	た
\\	、こんなとこにいたのか。
\\	チロ!チロ!どこにいるんだ?おー、い
\\	、い
\\	、こんなとこにいたのか。
\\	中14
\\	"「どうしてうそをついたの?」
\\	だって
\\	、だれも僕の言うことを聞いてくれない
\\	んだし
\\	男
\\	んだもん
\\	女、子
\\	「どうしてうそをついたの?」
\\	だ__
\\	、だれも僕の言うことを聞いてくれない
\\	んだ_
\\	中14
\\	"いくら状況を説明し
\\	たところで
\\	、警察は信じない
\\	だろう
\\	いくら状況を説明し
\\	ところ_
\\	、警察は信じない
\\	だろう
\\	中14
\\	"きれいに片付け
\\	たところで
\\	、子供がすぐ散らかすんだから意味がないよ。
\\	きれいに片付け
\\	ところ_
\\	、子供がすぐ散らかすんだから意味がないよ。
\\	中14
\\	"この山の自然を愛す
\\	ればこそ
\\	、観光客の数を厳しく
\\	制限[せいげん]しているのです。
\\	この山の自然を愛す
\\	こそ
\\	、観光客の数を厳しく
\\	制限しているのです。
\\	中14
\\	"親友であ
\\	ればこそ
\\	、お互いの欠点を
\\	指摘[してき]し合えるのだ。
\\	親友であ
\\	こそ
\\	、お互いの欠点を指摘し合えるのだ。
\\	中14
\\	"「このメイク、ちょっと派手すぎる?」
\\	「うーん。派手
\\	じゃないこと
\\	も[は]ないけど
\\	、パーティーなんだから、いいんでしょう?」
\\	「このメイク、ちょっと派手すぎる?」
\\	「うーん。派手
\\	じゃない__も__けど
\\	、パーティーなんだから、いいんでしょう?」
\\	中14
\\	「そろそろ12時だけど、
\\	昼飯[ひるめし]、どうする?」
\\	「腹、減って
\\	ないこと
\\	も[は]ないけど
\\	もうすこし後でもいいよ。」
\\	「そろそろ12時だけど、昼飯、どうする?」
\\	「腹、減って
\\	こと_ない__
\\	もうすこし後でもいいよ。」
\\	中14
\\	"悲しいときは
\\	泣きたい
\\	だけ
\\	泣けば
\\	いいよ。
\\	悲しいときは
\\	泣きたい
\\	泣けば
\\	いいよ。
\\	中14
\\	"今日とれたトマトだよ。(持って)
\\	ほしい
\\	だけ
\\	持って
\\	いっていいよ。
\\	"今日とれたトマトだよ。
\\	ほしい
\\	持って
\\	いっていいよ。
\\	中14
\\	"「今日は俺がおごるぜ。」
\\	「いえ、とんでもない。今日はおごっていただく
\\	わけにはいきません
\\	。先輩の祝いですから、僕達が出します。」
\\	「今日は俺がおごるぜ。」
\\	「いえ、とんでもない。今日はおごっていただく
\\	には__ません
\\	。先輩の祝いですから、僕達が出します。」
\\	中14
\\	"申し込みは郵送
\\	のみ
\\	の受け付けとなります。
\\	だけ 
\\	申し込みは郵送
\\	の受け付けとなります。
\\	中14
\\	"太枠内[ふとわくない]
\\	のみ
\\	ご記入ください。
\\	だけ 
\\	太枠内[ふとわくない]
\\	ご記入ください。
\\	中14
\\	"いつまでも若い
\\	つもりで
\\	徹夜してると体を壊すよ。
\\	いつまでも若い
\\	つもり_
\\	徹夜してると体を壊すよ。
\\	中14
\\	"旅行に行っ
\\	た
\\	つもりで
\\	、この「列車の旅」の
\\	を見て、楽しみましょう。
\\	旅行に行っ
\\	た
\\	つもり_
\\	、この「列車の旅」の
\\	を見て、楽しみましょう。
\\	中14
\\	"「ここにあるスーパーの袋の山、何だよ。」
\\	「あら、袋
\\	だって
\\	必要なのよ。」
\\	「ここにあるスーパーの袋の山、何だよ。」
\\	「あら、袋
\\	必要なのよ。」
\\	中14
\\	"鈴木さんはスポーツが得意だから、サッカー
\\	だって
\\	野球
\\	だって
\\	何でもできます。
\\	鈴木さんはスポーツが得意だから、サッカー
\\	野球
\\	何でもできます。
\\	中14
\\	"「ずいぶん長いことお祈りしたね。」
\\	「今年
\\	こそ
\\	、いい人に出会えますようにってお願いしたの。」
\\	「ずいぶん長いことお祈りしたね。」
\\	「今年
\\	、いい人に出会えますようにってお願いしたの。」
\\	中14
\\	"この本は子供向けだが、逆に、大人に
\\	こそ
\\	読んでもらいたい。
\\	この本は子供向けだが、逆に、大人に
\\	読んでもらいたい。
\\	中14
\\	"スペイン語が話せる
\\	といっても
\\	、日常会話に困らない程度です。
\\	スペイン語が話せる
\\	いって_
\\	、日常会話に困らない程度です。
\\	中14
\\	"病院へ行くほどではない
\\	にしろ
\\	、風邪をひいて体がだるい。
\\	にせよ
\\	病院へ行くほどではない
\\	に__
\\	、風邪をひいて体がだるい。
\\	中14
\\	"ほんの短い期間だった
\\	にせよ
\\	、海外で一人暮らしを経験できたことはよかった。
\\	にしろ
\\	"ほんの短い期間だった
\\	に__
\\	、海外で一人暮らしを経験できたことはよかった。
\\	中14
\\	"病気になっ
\\	てからでないと
\\	、健康のありがたみは分からない。
\\	病気になっ
\\	て__でない_
\\	、健康のありがたみは分からない。
\\	中14
\\	"高校を卒業し
\\	てからでないと
\\	、アルバイトをやらせてもらえなかった。
\\	高校を卒業し
\\	から___と
\\	、アルバイトをやらせてもらえなかった。
\\	中14
\\	"パソコンは修理が終わり
\\	次第
\\	、お送りします。
\\	パソコンは修理が終わり
\\	、お送りします。
\\	中14
\\	"落し物が見つかり
\\	次第
\\	、こちらからお電話します。
\\	落し物が見つかり
\\	、こちらからお電話します。
\\	中14
\\	"大地震が起こっ
\\	てからというもの
\\	、いつも地面が揺れているような気がする。
\\	大地震が起こっ
\\	て__という__
\\	、いつも地面が揺れているような気がする。
\\	中14
\\	"退職し
\\	てからというもの
\\	、暇で仕方がない。
\\	退職し
\\	から___もの
\\	、暇で仕方がない。
\\	中14
\\	"買い物
\\	ついでに
\\	銀行でお金をおろしてきた。
\\	買い物
\\	に
\\	銀行でお金をおろしてきた。
\\	中14
\\	"友達の結婚式で函館へ行く
\\	ついでに
\\	、とや湖に寄って火山を見てきたい。
\\	友達の結婚式で函館へ行く
\\	に
\\	、とや湖に寄って火山を見てきたい。
\\	中14
\\	"便利さという点
\\	から見ると
\\	、やはり田舎より都会のほうが暮らしやすい。
\\	便利さという点
\\	見る_
\\	、やはり田舎より都会のほうが暮らしやすい。
\\	中14
\\	"仕事を求めて都会に出る若者がいる
\\	一方
\\	(で)
\\	、故郷に戻って就職する若者もいる。
\\	仕事を求めて都会に出る若者がいる
\\	、故郷に戻って就職する若者もいる。
\\	中14
\\	"富良野はブドウの栽培に適している
\\	ことから
\\	、ワイン作りが盛んだ。
\\	富良野はブドウの栽培に適している
\\	から
\\	、ワイン作りが盛んだ。
\\	中14
\\	"ロボコンは初めのころは
\\	の番組で、大学や高専の学生
\\	を対象に
\\	行われていた。
\\	ロボコンは初めのころは
\\	の番組で、大学や高専の学生
\\	対象_
\\	行われていた。
\\	中14
\\	"テレビの午後の番組は主に主婦
\\	を対象に
\\	組まれている。
\\	テレビの午後の番組は主に主婦
\\	対象_
\\	組まれている。
\\	中14
\\	"ここは温泉
\\	ばかりでなく
\\	、釣りや山登りも楽しめます。
\\	だけでなく	
\\	ここは温泉
\\	でなく
\\	、釣りや山登りも楽しめます。
\\	中14
\\	"子供の反抗は、大人になるための第一歩
\\	にほかならない
\\	子供の反抗は、大人になるための第一歩
\\	に__なら__
\\	中14
\\	"このような事故が起きたのは、会社の管理体制が甘かったから
\\	にほかなりません
\\	このような事故が起きたのは、会社の管理体制が甘かったから
\\	ほか__ません
\\	中14
\\	"参加するかどうか
\\	はともかく
\\	、申し込みだけはしておこう。
\\	参加するかどうか
\\	は___く
\\	、申し込みだけはしておこう。
\\	中14
\\	"ブス
\\	なのはともかく
\\	、口が臭くては我慢できないよ。
\\	ブス
\\	なのは___く
\\	、口が臭くては我慢できないよ。
\\	中14
\\	"医者になる
\\	ためには
\\	、国家試験に合格しなければならない。
\\	医者になる
\\	ため__
\\	、国家試験に合格しなければならない。
\\	中14
\\	"勝利[しょうり]
\\	のためには
\\	、全員の力を合わせることが必要だ。
\\	勝利[しょうり]
\\	ため__
\\	、全員の力を合わせることが必要だ。
\\	中14
\\	"チンパンジーは道具が使える
\\	ことから
\\	、人間に最も近いと考えている。
\\	チンパンジーは道具が使える
\\	こ___
\\	、人間に最も近いと考えている。
\\	中15
\\	"難民問題は
\\	人道的[じんどうてき]な問題であろ
\\	のみならず
\\	近隣[きんりん]諸国にも影響を及ばす政治的な
\\	側面[そくめん]もある。
\\	だけでなく 
\\	難民問題は人道的な問題であろ
\\	の____
\\	近隣[きんりん]諸国にも影響を及ばす政治的な
\\	側面[そくめん]もある。
\\	中15
\\	"仏教は、中国、日本、韓国、タイ
\\	といった
\\	アジアの国で広く信仰されている。
\\	など 
\\	仏教は、中国、日本、韓国、タイ
\\	とい__
\\	アジアの国で広く信仰されている。
\\	中15
\\	"暑くなる
\\	に
\\	従[したが]って
\\	体調[たいちょう]を崩す人が増えた。
\\	暑くなる
\\	に____て
\\	、体調を崩す人が増えた。
\\	中15
\\	"銃がどこにあるか、
\\	考え
\\	得る
\\	場所はすべて捜したが、まったく手がかりがつかめなかった。
\\	うる
\\	える
\\	銃がどこにあるか、
\\	考え
\\	場所はすべて捜したが、まったく手がかりがつかめなかった。
\\	中15
\\	"普通の人が宇宙へ行ける日が来るなんて、100年前には想像
\\	し
\\	得[え]なかった
\\	ことだ。
\\	普通の人が宇宙へ行ける日が来るなんて、100年前には想像
\\	し
\\	なかった
\\	ことだ。
\\	中15
\\	"実験結果は予想
\\	に反する
\\	ものだったので、関係者はがっかりした様子だった。
\\	実験結果は予想
\\	に_する
\\	ものだったので、関係者はがっかりした様子だった。
\\	中15
\\	"修理
\\	に関する
\\	問い合わせはサービスセンターまでお電話かメールでご連絡ください。
\\	について 
\\	修理
\\	に_する
\\	問い合わせはサービスセンターまでお電話かメールでご連絡ください。
\\	中15
\\	"来日前は留学に期待する
\\	反面
\\	、不安も大きかった。
\\	来日前は留学に期待する
\\	、不安も大きかった。
\\	中15
\\	"お札にはその国の歴史
\\	上[じょう]
\\	の人物の顔が描かれていることが多い。
\\	お札にはその国の歴史
\\	の人物の顔が描かれていることが多い。
\\	中15
\\	"あの二人は表面
\\	上[じょう]
\\	は親しそうに見えるけど、本当はあまり仲がよくないんだ。
\\	あの二人は表面
\\	は親しそうに見えるけど、本当はあまり仲がよくないんだ。
\\	中15
\\	"世界規模での人口移動が進み
\\	つつある
\\	現在、共生の意識がますます必要になっている。
\\	世界規模での人口移動が進み
\\	つつ__
\\	現在、共生の意識がますます必要になっている。
\\	中15
\\	"環境対策のためにも、夏
\\	に限らず
\\	、年間を通して
\\	節電[せつでん]を心がけるべきだ。
\\	だけでなく、ほかにも… 
\\	環境対策のためにも、夏
\\	限__
\\	、年間を通して節電を心がけるべきだ。
\\	中15
\\	"演劇は
\\	決して
\\	華やかなだけの世界では
\\	ない
\\	ということを覚えておいて欲しい。
\\	全く、全然、絶対に 
\\	演劇は
\\	して
\\	華やかなだけの世界では
\\	ということを覚えておいて欲しい。
\\	中15
\\	"アフロヘアーの青年が山口
\\	五郎[ごろう]
\\	のもとで
\\	尺八[しゃくはち]修行[しゅぎょう]を始めた。
\\	アフロヘアーの青年が山口五郎
\\	の___
\\	尺八修行を始めた。
\\	中15
\\	"この地域では、昔から
\\	そう
\\	であったように、
\\	共同[きょうどう]で
\\	田植[たう]えをする。
\\	この地域では、昔から
\\	う
\\	であったように、
\\	共同で
\\	田植えをする。
\\	中15
\\	"誰でも
\\	そう
\\	だが、子供を持って初めて親の有りがたみを知る。
\\	誰でも
\\	う
\\	だが、子供を持って初めて親の有りがたみを知る。
\\	中15
\\	"気をつける。このあたりは毒ヘビがいる
\\	ぞ
\\	男 
\\	気をつける。このあたりは毒ヘビがいる
\\	中15
\\	"おーい。ここにあった
\\	ぞ
\\	男 
\\	おーい。ここにあった
\\	中15
\\	"遅く帰ってきた娘の顔を見て、ホッとする
\\	と同時に
\\	腹が立った。
\\	遅く帰ってきた娘の顔を見て、ホッとする
\\	と__に
\\	腹が立った。
\\	中15
\\	"邦楽[ほうがく]は日本の民族音楽である
\\	と同時に
\\	人類全体の財産である。
\\	邦楽は日本の民族音楽である
\\	同時_
\\	人類全体の財産である。
\\	中15
\\	"私にはとても無理な仕事だったので、断る
\\	しかない
\\	私にはとても無理な仕事だったので、断る
\\	ない
\\	中15
\\	"苦労
\\	の
\\	末
\\	、画家はやっと作品を完成させることができた。
\\	苦労
\\	の
\\	、画家はやっと作品を完成させることができた。
\\	中15
\\	"去年の夏
\\	以来
\\	、めぐみちゃんは一度も会っていない。
\\	…してからずっと… 
\\	去年の夏
\\	、めぐみちゃんは一度も会っていない。
\\	中15
\\	"このカバンはとてもよくできていて、偽物とは思えない
\\	くらい
\\	だ。
\\	ほど, 
\\	このカバンはとてもよくできていて、偽物とは思えない
\\	だ。
\\	中15
\\	"今日はお客さんのために心
\\	をこめて
\\	歌います。
\\	…の気持ちを持って
\\	今日はお客さんのために心
\\	を___
\\	歌います。
\\	中15
\\	"頭脳は使え
\\	ば
\\	使った
\\	だけ
\\	柔らかくなる。
\\	頭脳は使え
\\	使った
\\	柔らかくなる。
\\	中15
\\	"苦労は大き
\\	ければ
\\	大きい
\\	だけ
\\	財産になる。
\\	苦労は大き
\\	大きい
\\	財産になる。
\\	中15
\\	"お金の話を持ち出し
\\	た
\\	とたんに
\\	、相手が怒り出した。 
\\	途端[とたん]に
\\	お金の話を持ち出し
\\	た
\\	に
\\	、相手が怒り出した。
\\	中15
\\	"民主主義[みんしゅしゅぎ]だ
\\	からといって
\\	、何でも数で決めていいわけでは
\\	ない
\\	民主主義[みんしゅしゅぎ]だ
\\	といっ_
\\	、何でも数で決めていいわけでは
\\	ない
\\	中15
\\	お見えになる お越しになる	尊敬語  
\\	来る			中15
\\	おいでになる	尊敬語  
\\	行く 来る いる			中15
\\	ておいでになる	尊敬語  
\\	ている			中15
\\	ご覧に入れる	謙譲語 
\\	見せる			中15
\\	承[うけたまわ]る	謙譲語 
\\	聞く 引き受ける			中15
\\	拝借[はいしゃく]する	謙譲語 
\\	借りる			中15
\\	存じる	謙譲語 
\\	思う 知っている			中15
\\	申し伝える	謙譲語 
\\	伝言する			中15
\\	てまいる	謙譲語 
\\	ていく 
\\	てくる			中15
\\	"横浜から
\\	お越し
\\	の大山様、佐藤様が
\\	お待ちです
\\	ので、1階の受け付けまで
\\	お越しください
\\	お見えください
\\	横浜から
\\	お越し
\\	の大山様、佐藤様が
\\	お
\\	(待つ)
\\	です
\\	ので、1階の受け付けまで
\\	(来る)
\\	ください
\\	中15
\\	"本日の
\\	ご予約
\\	は山本が
\\	承りました
\\	。ありがとうございました。
\\	本日の
\\	ご予約
\\	は山本が
\\	(引け受けました)
\\	。ありがとうございました。
\\	中15
\\	"この資料、長い間
\\	拝借した
\\	まま
\\	お返し
\\	もせず、大変申し訳ありませんでした。 
\\	はいしゃく
\\	この資料、長い間
\\	(借りた)
\\	まま
\\	お返し
\\	もせず、大変申し訳ありませんでした。
\\	中15
\\	"「こちらまで、お車で
\\	おいでになりました
\\	か。」
\\	「いいえ、電車で
\\	参りました
\\	「こちらまで、お車で
\\	(来ました)
\\	か。」
\\	「いいえ、電車で
\\	(来ました)
\\	中15
\\	"「休みの日はいつも何をし
\\	ておいでになりますか
\\	「趣味のゴルフをし
\\	ております
\\	「休みの日はいつも何をし
\\	て
\\	(います)
\\	か。」
\\	「趣味のゴルフをし
\\	て
\\	(います)
\\	中15
\\	"田中はただ今、席を外し
\\	ております
\\	ので、後ほどこちらからご連絡するように
\\	申し伝えます
\\	田中はただ今、席を外し
\\	ております
\\	ので、後ほどこちらからご連絡するように
\\	(伝言します)
\\	中15
\\	"ただ今から皆様に
\\	ご覧に入れます
\\	のは、イルカのショーでございます。
\\	ただ今から皆様に
\\	(見せます)
\\	のは、イルカのショーでございます。
\\	中15
\\	"プレゼント用のラッピングはあちらのカウンターで
\\	承っております
\\	"プレゼント用のラッピングはあちらのカウンターで
\\	(引き受けて)
\\	おります
\\	中15
\\	"この資料をコピーし
\\	てまいります
\\	ので、少々お待ちください。
\\	この資料をコピーし
\\	て
\\	(行きます)
\\	ので、少々お待ちください。
\\	中15
\\	"こちらの商品はきっとご満足いただけると
\\	存じます
\\	こちらの商品はきっとご満足いただけると
\\	(思います)
\\	中15
\\	"事故の情報をもう一度詳しく
\\	お
\\	話し
\\	願えます
\\	か。
\\	ください 
\\	事故の情報をもう一度詳しく
\\	お
\\	話し
\\	か。
\\	中15
\\	"見学ご希望の方は、この書類に
\\	ご
\\	記入
\\	願います
\\	ください 
\\	見学ご希望の方は、この書類に
\\	ご
\\	記入
\\	中15
\\	"ご迷惑をお掛け致しましたことを深くおわび
\\	申し上げます
\\	する 
\\	ご迷惑をお掛け致しましたことを深くおわび
\\	中15
\\	"またのご来店を心よりお待ち
\\	申し上げて
\\	おります。
\\	する 
\\	またのご来店を心よりお待ち
\\	おります。
\\	中15
\\	"こちらの事情をご理解頂け
\\	ればと思います
\\	こちらの事情をご理解
\\	(いただく)
\\	と思います
\\	中15
\\	"皆さんのご意見お聞かせくださ
\\	ればと思います
\\	皆さんのご意見お聞かせくださ
\\	れば_____
\\	中15
\\	につきまして
\\	にとりまして
\\	におきまして
\\	~といたしまして
\\	~に関しまして
\\	について
\\	にとって
\\	において
\\	~として
\\	~に関して
\\	中15
\\	"彼氏は具合が悪いのに、医者に行き
\\	もせずに
\\	仕事を続けている。
\\	彼氏は具合が悪いのに、医者に行き
\\	せず_
\\	仕事を続けている。
\\	中16
\\	"彼は上司の許可を
\\	得
\\	もせずに
\\	、新しいプロジェクトを進めた。
\\	彼は上司の許可を
\\	(得る)
\\	も__に
\\	、新しいプロジェクトを進めた。
\\	中16
\\	"どんな大金持ち
\\	といえども
\\	、お金で解決できない悩みがあるはずだ。
\\	といっても 
\\	どんな大金持ち
\\	と___も
\\	、お金で解決できない悩みがあるはずだ。
\\	中16
\\	"名医[めいい]
\\	といえども
\\	、全ての患者を救うことはできない。
\\	といっても 
\\	名医[めいい]
\\	と___も
\\	、全ての患者を救うことはできない。
\\	中16
\\	よほど
\\	英語が堪能な人
\\	でも
\\	、そう簡単には訳せないだろう。
\\	どんなに~であっても 
\\	よ__
\\	英語が堪能な人
\\	で_
\\	、そう簡単には訳せないだろう。
\\	中16
\\	よほど
\\	不器用[ぶきよう]な人
\\	でも
\\	、この機械を使えば、ちゃんとした物が作ればはずだ。
\\	どんなに~であっても 
\\	よ__
\\	不器用な人
\\	で_
\\	、この機械を使えば、ちゃんとした物が作ればはずだ。
\\	中16
\\	"朝のラッシュを見ると、日本人が
\\	いかに
\\	我慢強い
\\	か
\\	が分かる。
\\	非常に~である 
\\	朝のラッシュを見ると、日本人が
\\	い__
\\	我慢強い
\\	が分かる。
\\	中16
\\	"自然の力の前では人間の存在など
\\	いかに
\\	小さなもの
\\	か
\\	知った。
\\	非常に~である 
\\	自然の力の前では人間の存在など
\\	い__
\\	小さなもの
\\	知った。
\\	中16
\\	"お嬢さんが
\\	近々[ちかぢか]結婚なさる
\\	とか
\\	。おめでとうございます。
\\	そうだ	
\\	お嬢さんが近々結婚なさる
\\	。おめでとうございます。
\\	中16
\\	"先日のゴルフ大会では社長が優勝なさった
\\	とか
\\	そうだ	
\\	先日のゴルフ大会では社長が優勝なさった
\\	中16
\\	"経済の専門家
\\	に言わせれば
\\	、円はこれからもっと高くなるらしい。
\\	その人の意見では 
\\	経済の専門家
\\	に言____
\\	、円はこれからもっと高くなるらしい。
\\	中16
\\	"口の悪い弟
\\	に言わせると
\\	長[ちょう]」がつく人間は信用してはいけないそうだ。
\\	その人の意見では 
\\	口の悪い弟
\\	に言____
\\	長[ちょう]」がつく人間は信用してはいけないそうだ。
\\	中16
\\	"予想ではなく、経験
\\	に基づいて
\\	判断しました。
\\	予想ではなく、経験
\\	に___て
\\	判断しました。
\\	中16
\\	"人口増加によって、地球温暖化はますます進む
\\	と言える
\\	のではないでしょうか。
\\	と判断できる, 
\\	人口増加によって、地球温暖化はますます進む
\\	と___
\\	のではないでしょうか。
\\	中16
\\	"お金があれば幸せだ
\\	と言える
\\	のでしょうか。?
\\	と判断できる, 
\\	お金があれば幸せだ
\\	と___
\\	のでしょうか。
\\	中16
\\	"英語は小さい時から学ばせたほうがいいという意見もある
\\	一方
\\	で、きちんと
\\	母語[ぼご]を学んでからにしたほうがいいという意見もある。
\\	英語は小さい時から学ばせたほうがいいという意見もある
\\	で
\\	、きちんと母語を学んでからにしたほうがいいという意見もある。
\\	中16
\\	"バリアフリーとは
\\	障害[しょうがい]を持った人やお年寄り
\\	に限らず
\\	、誰でもが快適に利用できるということです。
\\	だけでなく 
\\	バリアフリーとは
\\	障害を持った人やお年寄り
\\	に限__
\\	、誰でもが快適に利用できるということです。
\\	中16
\\	"関係者が情報を共有すべきだと考え、皆様にお知らせ
\\	次第です
\\	関係者が情報を共有すべきだと考え、皆様にお知らせ
\\	です
\\	中16
\\	"私[わたくし]どもだけではどうしようもなく、こうしてお願いに参った
\\	次第でございます
\\	私どもだけではどうしようもなく、こうしてお願いに参った
\\	でございます
\\	中16
\\	"それ
\\	をもって
\\	「客観的評価」
\\	とされている
\\	ことに私たちはあまり疑問を
\\	抱[いだ]きません。
\\	それ
\\	を___
\\	「客観的評価」
\\	とされている
\\	ことに私たちはあまり疑問を
\\	抱[いだ]きません。
\\	中16
\\	"出席率、授業中の発表、レポート
\\	をもって
\\	評価[ひょうか]とします。
\\	出席率、授業中の発表、レポート
\\	を___
\\	評価とします。
\\	中16
\\	"拍手
\\	をもって
\\	、賛成をいただいたものといたします。
\\	拍手
\\	を___
\\	、賛成をいただいたものといたします。
\\	中16
\\	"小社
\\	におきましては
\\	目下[もっか]『私の死亡記事』というネクロロジー集を
\\	編纂中[へんさんちゅう]です。
\\	では 
\\	小社
\\	に_____は
\\	目下[もっか]『私の死亡記事』というネクロロジー集を
\\	編纂中[へんさんちゅう]です。
\\	中16
\\	"経済成長期の日本
\\	におきましては
\\	、収入が2~3年で倍になることもあった。
\\	では 
\\	経済成長期の日本
\\	に_____は
\\	、収入が2~3年で倍になることもあった。
\\	中16
\\	"今のうちにエネルギー政策を変更しないと、将来重大な問題が起こり
\\	得[う]る
\\	ことができる 
\\	今のうちにエネルギー政策を変更しないと、将来重大な問題が起こり
\\	中16
\\	"磯飛さんがにこにこしている。待ち望んでいたお子さんが生まれた
\\	のであろう
\\	のだろう
\\	磯飛さんがにこにこしている。待ち望んでいたお子さんが生まれた
\\	の____
\\	中16
\\	"片島さんの部屋の電気が消えている。彼は出かけている
\\	のだろう
\\	のであろう
\\	片島さんの部屋の電気が消えている。彼は出かけている
\\	の___
\\	中16
\\	"エコロジーは世界中で必要な思想だ
\\	と思われる
\\	と考えられる
\\	と思う 
\\	エコロジーは世界中で必要な思想だ
\\	と____
\\	中16
\\	"彼の指摘は本社の経営上の問題の本質をついている
\\	と思われる
\\	と考えられる
\\	と思う, 
\\	彼の指摘は本社の経営上の問題の本質をついている
\\	と____
\\	中16
\\	"村には電気は
\\	なかった
\\	。ろうそくはたとえあった
\\	としても
\\	高価[こうか]でとても買えなかった。(だから、夜は勉強できなかった。)
\\	村には電気は
\\	った
\\	。ろうそくはたとえあった
\\	とし__
\\	高価[こうか]でとても買えなかった
\\	中16
\\	"パワーポイントで作成したファイルを受け取ったのですが、開くことでき
\\	なかった
\\	り、開いた
\\	としても
\\	内容が読み取りません。(だから、困っています。)
\\	パワーポイントで作成したファイルを受け取ったのですが、開くことでき
\\	った
\\	り、開いた
\\	とし__
\\	内容が読み取りません
\\	中16
\\	"上司や同僚がまだ仕事をしているので帰
\\	ろうにも
\\	帰
\\	れない
\\	上司や同僚がまだ仕事をしているので
\\	(帰る)
\\	にも
\\	(帰る)
\\	中16
\\	"パスワードが分からないので、データを見
\\	ようにも
\\	見
\\	られない
\\	パスワードが分からないので、データを見
\\	よう__
\\	見
\\	られない
\\	中16
\\	"これでは子供を産
\\	もうにも
\\	産
\\	めない
\\	と思うのですが。
\\	これでは子供を
\\	(産む)
\\	にも
\\	(産む)
\\	と思うのですが。
\\	中16
\\	"映画「王様のスピーチ」はタイトルの
\\	わりに
\\	はおもしろかった。
\\	映画「王様のスピーチ」はタイトルの
\\	に
\\	はおもしろかった。
\\	中16
\\	"この王様は幼い頃、いじめられた
\\	わりに
\\	はまっすぐな性格をしている。
\\	この王様は幼い頃、いじめられた
\\	に
\\	はまっすぐな性格をしている。
\\	中16
\\	"豊かな国は貧しい国を
\\	援助[えんじょ]する
\\	べきだ
\\	豊かな国は貧しい国を援助する
\\	だ
\\	中16
\\	"子供の前で夫婦げんかを
\\	す
\\	べきではない
\\	する
\\	べきではない
\\	子供の前で夫婦げんかを
\\	す
\\	ではない
\\	中16
\\	"友人の秘密を他人に話す
\\	べきではない
\\	友人の秘密を他人に話す
\\	ではない
\\	中16
\\	"治す医療、
\\	というより
\\	、人間がもともと持っている回復する力に働きかける医療が求められている。
\\	治す医療、
\\	と____
\\	、人間がもともと持っている回復する力に働きかける医療が求められている。
\\	中16
\\	"歴史を学ぶことは、過去を知る
\\	というより
\\	、よりよい未来を
\\	築[きず]くためなのです。
\\	歴史を学ぶことは、過去を知る
\\	と____
\\	、よりよい未来を
\\	築[きず]くためなのです。
\\	中16
\\	"「どうしたの?顔、はれてるよ。」
\\	「虫歯を抜いたら、痛い
\\	のなんのって
\\	。何も食べられないんだ。」
\\	「どうしたの?顔、はれてるよ。」
\\	「虫歯を抜いたら、痛い
\\	なん___
\\	。何も食べられないんだ。」
\\	中16
\\	"「突然部長に呼ばれてさ、部長のお嬢さんと見合いしないかって。びっくりした
\\	のなんのって
\\	「で、見合いするの?」
\\	「突然部長に呼ばれてさ、部長のお嬢さんと見合いしないかって。びっくりした
\\	の__のって
\\	「で、見合いするの?」
\\	中16
\\	"「昨日のハイキング、思いのほか大変だったね。」
\\	「ほんと、ぶっ続けで5時間歩きっぱなし。疲れた
\\	のなんのって
\\	、最後はもう一歩も歩けないっていう感じだったようね。」
\\	「昨日のハイキング、思いのほか大変だったね。」
\\	「ほんと、ぶっ続けで5時間歩きっぱなし。疲れた
\\	なん___
\\	、最後はもう一歩も歩けないっていう感じだったようね。」
\\	中16
\\	"このベンチ、ペンキ
\\	塗りたて
\\	だって。
\\	このベンチ、ペンキ
\\	塗り__
\\	だって。
\\	中16
\\	"お母さんっ
\\	てば
\\	、いつも勝手に僕の部屋に入るのよ。
\\	お母さん
\\	っ__
\\	、いつも勝手に僕の部屋に入るのよ。
\\	中16
\\	"このパソコン
\\	ってば
\\	、しょっちゅうフリーズするんだ。
\\	ったら
\\	このパソコン
\\	っ__
\\	、しょっちゅうフリーズするんだ。
\\	中16
\\	"ほこりが鼻に入るとくしゃみが出て、自然にそれを外へ出す
\\	ようになっています
\\	ほこりが鼻に入るとくしゃみが出て、自然にそれを外へ出す
\\	ように___います
\\	中16
\\	"このライターは
\\	着火[ちゃっか]部分を固くして、子供がいたずらしても火がつかない
\\	ようになっています
\\	このライターは
\\	着火[ちゃっか]部分を固くして、子供がいたずらしても火がつかない
\\	なっています
\\	中16
\\	"「このチョコ、1粒1,000円もするんだよ。」
\\	「え!本当?じゃあ、おいしい
\\	わけ
\\	ようね。」
\\	「このチョコ、1粒1,000円もするんだよ。」
\\	「え!本当?じゃあ、おいしい
\\	ようね。」
\\	中16
\\	"「お宅は郊外だから、静かでしょう。」
\\	「静か
\\	どころか
\\	騒音がひどいんですよ。近くに高速道路があるから。。。」
\\	「お宅は郊外だから、静かでしょう。」
\\	「静か
\\	騒音がひどいんですよ。近くに高速道路があるから。。。」
\\	中16
\\	"おしゃれに全然気を使わない
\\	ようじゃ
\\	、社会人としてまずいんじゃない?」
\\	おしゃれに全然気を使わない
\\	よう__
\\	、社会人としてまずいんじゃない?」
\\	中16
\\	"締め切りを守れない
\\	ようじゃ
\\	、漫画家としてやっていけないよ。
\\	締め切りを守れない
\\	じゃ
\\	、漫画家としてやっていけないよ。
\\	中16
\\	"料理は1つ作るのにこんなに時間がかかる
\\	ようじゃ
\\	、一人暮らしは無理かな?
\\	料理は1つ作るのにこんなに時間がかかる
\\	よう__
\\	、一人暮らしは無理かな?
\\	中16
\\	"昔僕は好きな女の子の前で悪
\\	ぶって
\\	いた。本当は好きだって言う勇気がなかっただけなんだ。
\\	昔僕は好きな女の子の前で
\\	(悪い)
\\	いた。本当は好きだって言う勇気がなかっただけなんだ。
\\	中16
\\	"「あいつ、先生の前だといい子
\\	ぶる
\\	けど、掃除サボるし、宿題も誰かの写してるんだぜ。」
\\	「まったく、頭に来るよな。」
\\	「あいつ、先生の前だといい子
\\	けど、掃除サボるし、宿題も誰かの写してるんだぜ。」
\\	「まったく、頭に来るよな。」
\\	中16
\\	"よく知らないやつ
\\	に限って
\\	、偉そうなことを言う。
\\	よく知らないやつ
\\	に___
\\	、偉そうなことを言う。
\\	中16
\\	"高い車に乗っている人
\\	に限って
\\	、安いアパートに住んでるんだって、堀さんは言っていたけど、本当かな。
\\	高い車に乗っている人
\\	に___
\\	、安いアパートに住んでるんだって、堀さんは言っていたけど、本当かな。
\\	中16
\\	"仕事でも何でも自分一人で悩まないで、誰かに相談する
\\	ことです
\\	よ。
\\	仕事でも何でも自分一人で悩まないで、誰かに相談する
\\	です
\\	よ。
\\	中16
\\	"今日の森林火災で、この辺の
\\	木
\\	という
\\	木
\\	は、一本残らず燃えてしまった。
\\	今日の森林火災で、この辺の
\\	木
\\	木
\\	は、一本残らず燃えてしまった。
\\	中16
\\	"赤坂監督はこの映画で、今年の映画関連の
\\	賞
\\	という
\\	賞
\\	を独占した。
\\	赤坂監督はこの映画で、今年の映画関連の
\\	賞
\\	賞
\\	を独占した。
\\	中16
\\	"桜の季節には
\\	道
\\	という
\\	道
\\	に観光客があふれ、地元の人間にとっては迷惑な話だ。
\\	桜の季節には
\\	道
\\	道
\\	に観光客があふれ、地元の人間にとっては迷惑な話だ。
\\	中16
\\	"漫画に夢中になっていて、友達が教えてくれなかったら、乗り過ごす
\\	ところだった
\\	漫画に夢中になっていて、友達が教えてくれなかったら、
\\	(乗り過ごす)
\\	ところ
\\	(です)
\\	中16
\\	"祖父の病気はもう少し発見が遅れていたら、
\\	手遅[ておく]れになる
\\	ところだった
\\	祖父の病気はもう少し発見が遅れていたら、手遅れに
\\	(なる)__
\\	ところ
\\	(です)
\\	中16
\\	"「カメラが壊れちゃって、修理代が15,000円もするんだ。」
\\	000円も払う
\\	くらいなら
\\	、新しいのを買ったほうがいいね。」
\\	「カメラが壊れちゃって、修理代が15,000円もするんだ。」
\\	000円も払う
\\	く____
\\	、新しいのを買ったほうがいいね。」
\\	中16
\\	"一時間待つ
\\	くらいなら
\\	ほかの店に行こう。
\\	一時間待つ
\\	く____
\\	ほかの店に行こう。
\\	中16
\\	"「レポート、締め切りに合わないよ。どうしよう。」
\\	「そんなに焦る
\\	くらいなら
\\	、もっと早く書き始めればよかったのに。」
\\	「レポート、締め切りに合わないよ。どうしよう。」
\\	「そんなに焦る
\\	く____
\\	、もっと早く書き始めればよかったのに。」
\\	中16
\\	"「もうすぐ冬だね、私、寒いの苦手なんだ。」
\\	「暑いより、寒い
\\	ほうが
\\	まし
\\	だよ。寒いときには服を着ればいいんだから。」
\\	「もうすぐ冬だね、私、寒いの苦手なんだ。」
\\	「暑いより、寒い
\\	ほうが
\\	だよ。寒いときには服を着ればいいんだから。」
\\	中16
\\	"「彼、何を作っても何も言わずに食べるだけなの。」
\\	「ちゃんと食べてくれるならいいじゃない。『まずい』って言われる
\\	より
\\	まし
\\	だ。
\\	「彼、何を作っても何も言わずに食べるだけなの。」
\\	「ちゃんと食べてくれるならいいじゃない。『まずい』って言われる
\\	より
\\	だ。
\\	中16
\\	"「ダンさんって才能があるよね。」
\\	「僕もそう思う。彼女の発表には素晴らしい
\\	ものがある
\\	よね。」
\\	「ダンさんって才能があるよね。」
\\	「僕もそう思う。彼女の発表には素晴らしい
\\	が__
\\	よね。」
\\	中16
\\	"「この町、ずいぶん変わりましたね。」
\\	「ええ、便利になったんですが、違う町になってしまって、寂しい
\\	ものがあります
\\	よ。」
\\	「この町、ずいぶん変わりましたね。」
\\	「ええ、便利になったんですが、違う町になってしまって、寂しい
\\	あります
\\	よ。」
\\	中16
\\	"環境保護への関心は高まっているが、
\\	代替[だいたい]エネルギーの普及は簡単には進む
\\	まい
\\	環境保護への関心は高まっているが、
\\	代替[だいたい]エネルギーの普及は簡単には進む
\\	中16
\\	"ストレスを感じない人はいる
\\	まい
\\	ストレスを感じない人はいる
\\	中16
\\	"彼女に振られた
\\	直後[ちょくご]は、もう恋なんかする
\\	まいと思って
\\	いたが…。
\\	彼女に振られた直後は、もう恋なんかする
\\	と思って
\\	いたが…。
\\	中16
\\	"母は
\\	何か
\\	につけ
\\	、心配して電話している。
\\	母は
\\	何か
\\	に__
\\	、心配して電話している。
\\	中16
\\	"環境汚染のニュースを
\\	聞く
\\	につけ
\\	、健康への影響に不安を感じる。
\\	環境汚染のニュースを
\\	聞く
\\	に__
\\	、健康への影響に不安を感じる。
\\	中16
\\	"この部屋は、広い
\\	わりに
\\	家賃が安い。
\\	この部屋は、広い
\\	に
\\	家賃が安い。
\\	中16
\\	"ソンさんは映画が好きだと言う
\\	わりに
\\	は、映画のことを知らない。
\\	ソンさんは映画が好きだと言う
\\	に
\\	は、映画のことを知らない。
\\	中16
\\	"大原さん、テニスが嫌いだと言っていた
\\	わりに
\\	は、熱心に練習していますね。
\\	大原さん、テニスが嫌いだと言っていた
\\	に
\\	は、熱心に練習していますね。
\\	中16
\\	"彼女が
\\	立つ
\\	か
\\	立たない
\\	かのうちに
\\	、次は自分が座るという態度を周りに示す。
\\	彼女が
\\	立つ
\\	立たない
\\	うち_
\\	、次は自分が座るという態度を周りに示す。
\\	中16
\\	"早朝[そうちょう]から開店セールに並んだ人たちはドアが
\\	開く
\\	か
\\	開かない
\\	かのうちに
\\	、どっと店内になでれ込んだ。
\\	早朝[そうちょう]から開店セールに並んだ人たちはドアが
\\	開く
\\	開かない
\\	うち_
\\	、どっと店内になでれ込んだ。
\\	中16
\\	"映画のラストシーンで、主人公が寂し
\\	げ
\\	に微笑んだのが印象的だった。
\\	そう, 
\\	映画のラストシーンで、主人公が寂し
\\	に微笑んだのが印象的だった。
\\	中16
\\	"彼女はしかれている間、何か言いた
\\	げ
\\	だったが、結局何も言わなかった。
\\	そう, 
\\	彼女はしかれている間、何か言いた
\\	だったが、結局何も言わなかった。
\\	中16
\\	"海で子供を助けたことが新聞に載って、嬉しい
\\	やら
\\	恥ずかしい
\\	やら
\\	や…など、たり…たりする 
\\	海で子供を助けたことが新聞に載って、嬉しい
\\	恥ずかしい
\\	中16
\\	"青空を飛んでいた鳥は突然海に
\\	潜[もぐ]っ
\\	た
\\	かと思うと
\\	、魚を口にくわえて出てきた。
\\	青空を飛んでいた鳥は突然海に
\\	潜っ
\\	た
\\	思う_
\\	、魚を口にくわえて出てきた。
\\	中16
\\	"工場でドンと大きいな音がし
\\	た
\\	かと思うと
\\	、真っ赤な火が燃えて広がった。
\\	工場でドンと大きいな音がし
\\	た
\\	思う_
\\	、真っ赤な火が燃えて広がった。
\\	中16
\\	"政治の目的は国民の幸福
\\	にほかならない
\\	政治の目的は国民の幸福
\\	ならない
\\	中16
\\	"事故を起こしたのは
\\	労働条件[ろうどうじょうけん]が厳しかったから
\\	にほかならない
\\	と、彼は裁判で主張した。
\\	事故を起こしたのは
\\	労働条件[ろうどうじょうけん]が厳しかったから
\\	に__なら__
\\	と、彼は裁判で主張した。
\\	中16
\\	"人類は地球上の生物のわずか2
\\	を発見した
\\	にすぎず
\\	、全ての生物を確認、分類するのは不可能と言われているそうだ。
\\	人類は地球上の生物のわずか2
\\	を発見した
\\	に__ず
\\	、全ての生物を確認、分類するのは不可能と言われているそうだ。
\\	中16
\\	"世界人口の2割を占める
\\	にすぎない
\\	先進国の人々が、
\\	の6割を
\\	排出[はいしゅつ]していると言われている。
\\	世界人口の2割を占める
\\	に__ない
\\	先進国の人々が、
\\	の6割を
\\	排出[はいしゅつ]していると言われている。
\\	中16
\\	"先日は熱が40度も出た
\\	上に
\\	下痢[げり]が止まらず、本当に大変でした。
\\	先日は熱が40度も出た
\\	上_
\\	下痢が止まらず、本当に大変でした。
\\	中16
\\	"工事現場の仕事は危険がともなう
\\	重労働[じゅうろうどう]である
\\	上に
\\	賃金[ちんぎん]も低いので、どの現場でも人
\\	手不足[でぶそく]になっているらしい。
\\	工事現場の仕事は危険がともなう
\\	重労働[じゅうろうどう]である
\\	上_
\\	賃金[ちんぎん]も低いので、どの現場でも人
\\	手不足になっているらしい。
\\	中16
\\	"「来月の富士山日帰りバスツアーは席がまだありますか。」
\\	「はい、まだございますが、
\\	ある
\\	といいましても
\\	、残りわずかですので、お早めにご予約ください。」
\\	「来月の富士山日帰りバスツアーは席がまだありますか。」
\\	「はい、まだございますが、
\\	ある
\\	といっても[敬語?])
\\	、残りわずかですので、お早めにご予約ください。」
\\	中16
\\	"嫌なことは忘れる
\\	に限ります
\\	よ。
\\	嫌なことは忘れる
\\	に____
\\	よ。
\\	中16
\\	"運動の後は、はちみつとレモンのジュース
\\	に限る
\\	運動の後は、はちみつとレモンのジュース
\\	に__
\\	中16
\\	"このタオル、高い
\\	だけあって
\\	肌ざわりがすごくいいんだ。
\\	このタオル、高い
\\	だ____
\\	肌ざわりがすごくいいんだ。
\\	中16
\\	"この町は文化遺産されている
\\	だけあって
\\	、住民の環境保護に対する意識も高い。
\\	この町は文化遺産されている
\\	だ____
\\	、住民の環境保護に対する意識も高い。
\\	中16
\\	"好き
\\	にしろ
\\	、嫌い
\\	にしろ
\\	、健康のために野菜は毎日とらなきゃだめだよ。
\\	にせよ
\\	しても
\\	好き
\\	に__
\\	、嫌い
\\	に__
\\	、健康のために野菜は毎日とらなきゃだめだよ。
\\	中16
\\	"大学院で研究しようと思ったら、
\\	理系[りけい]
\\	にしろ
\\	文系[ぶんけい]
\\	にしろ
\\	、英語力は絶対必要だよ。
\\	にせよ
\\	にしても
\\	大学院で研究しようと思ったら、
\\	理系[りけい]
\\	に__
\\	文系[ぶんけい]
\\	に__
\\	、英語力は絶対必要だよ。
\\	中16
\\	"えっ!今日ハイキングに行くの!?テレビで台風が来る
\\	とか
\\	(言ってた)
\\	よ。
\\	えっ!今日ハイキングに行くの!?テレビで台風が来る
\\	言ってたよ。
\\	中17
\\	"今度の展覧会では「平和」というテーマ
\\	にそった
\\	作品を
\\	展示[てんじ]しています。
\\	"今度の展覧会では「平和」というテーマ
\\	に___
\\	作品を展示しています。
\\	中17
\\	"国民の皆様のご期待
\\	にそう
\\	よう、努力致します。
\\	国民の皆様のご期待
\\	に__
\\	よう、努力致します。
\\	中17
\\	"日本酒造り
\\	にかけては
\\	タンさんの右に出る者はいない。
\\	日本酒造り
\\	に___は
\\	タンさんの右に出る者はいない。
\\	中17
\\	"彼は映画俳優だが、ピアノの演奏
\\	にかけても
\\	すばらしい才能を持っている。
\\	彼は映画俳優だが、ピアノの演奏
\\	に___も
\\	すばらしい才能を持っている。
\\	中17
\\	"田舎では、車が
\\	ないことには
\\	生活できない。
\\	田舎では、車が
\\	こと__
\\	生活できない。
\\	中17
\\	"使っていただか
\\	ないことには
\\	、この商品のよさはご理解いただけないので、サンプルをご用意させていただきました。
\\	使っていただか
\\	こと__
\\	、この商品のよさはご理解いただけないので、サンプルをご用意させていただきました。
\\	中17
\\	"日本人は自分の
\\	意思[いし]をはっきり言わないので誤解
\\	され
\\	がち
\\	だ。
\\	日本人は自分の意思をはっきり言わないので誤解
\\	され
\\	だ。
\\	中17
\\	"初めて会ったとき、日下部ちゃんは遠慮
\\	がち
\\	に俺のラインを聞いた。
\\	初めて会ったとき、日下部ちゃんは遠慮
\\	に俺のラインを聞いた。
\\	中17
\\	"タバコは体に悪いからやめようと
\\	思い
\\	つつ
\\	も
\\	、つい吸ってしまう。
\\	タバコは体に悪いからやめようと
\\	思い
\\	、つい吸ってしまう。
\\	中17
\\	"「偽者のでは…?」と
\\	疑い
\\	つつ
\\	も
\\	、安さにひかれて買ってしまった。
\\	「偽者のでは…?」と
\\	疑い
\\	、安さにひかれて買ってしまった。
\\	中17
\\	"もし、生まれ変われる
\\	としたら
\\	、俺は女性になりたい。
\\	とすれば
\\	とすると
\\	もし、生まれ変われる
\\	と___
\\	、俺は女性になりたい。
\\	中17
\\	"家を買う
\\	としたら
\\	、郊外の庭付きの一戸建てがいい。
\\	とすると
\\	とすれば
\\	家を買う
\\	と___
\\	、郊外の庭付きの一戸建てがいい。
\\	中17
\\	"登山ルートは天候
\\	次第
\\	で変更する場合もありますので、ご
\\	了承[りょうしょう]ください。
\\	登山ルートは天候
\\	で変更する場合もありますので、ご了承ください。
\\	中17
\\	"今度の
\\	審査[しんさ]の結果
\\	次第
\\	で、国から研究資金がもらえるかどうか決まる。
\\	今度の審査の結果
\\	で、国から研究資金がもらえるかどうか決まる。
\\	中17
\\	"グローバル化が進んで、ご学力の必要性は高まる
\\	一方だ
\\	ばかりだ
\\	グローバル化が進んで、ご学力の必要性は高まる
\\	だ
\\	中17
\\	"風雨[ふうう]は
\\	強[つよ]まる
\\	一方で
\\	漁[りょう]に出た
\\	漁船[ぎょせん]がまだ帰らず、関係者は心配している。
\\	ばかりだ
\\	風雨[ふうう]は強まる
\\	で
\\	漁[りょう]に出た
\\	漁船[ぎょせん]がまだ帰らず、関係者は心配している。
\\	中17
\\	"レストランの開店
\\	に先立[さきだ]って
\\	、試食会が開かれた。
\\	レストランの開店
\\	に__って
\\	、試食会が開かれた。
\\	中16
\\	"新しい機械の導入
\\	に先立ち
\\	、工場内で説明会が行われた。
\\	新しい機械の導入
\\	に__ち
\\	、工場内で説明会が行われた。
\\	中16
\\	"契約書の内容をご確認
\\	の上
\\	(で)
\\	、こちらに
\\	署名[しょめい]と
\\	印鑑[いんかん]をお願い致します。
\\	~してから	
\\	契約書の内容をご確認
\\	の__
\\	、こちらに
\\	署名[しょめい]と
\\	印鑑[いんかん]をお願い致します。
\\	中16
\\	"駅前の
\\	再開発[さいかいはつ]については、住民の皆さんの意見をまとめた
\\	上で
\\	、市に
\\	要望書[ようぼうしょ]を提出したいと思います。
\\	~してから	
\\	駅前の
\\	再開発については、住民の皆さんの意見をまとめた
\\	、市に
\\	要望書を提出したいと思います。
\\	中16
\\	"そのアイドルはコンサートの最後にアンコール
\\	にこたえて
\\	もう1
\\	曲[きょく]歌った。
\\	そのアイドルはコンサートの最後にアンコール
\\	に____
\\	もう1
\\	曲歌った。
\\	中16
\\	"ワールドカップで高橋選手はスポーターの
\\	声援[せいえん]
\\	にこたえて
\\	大活躍[だいかつやく]した。
\\	ワールドカップで高橋選手はスポーターの
\\	声援[せいえん]
\\	に____
\\	大活躍した。
\\	中16
\\	"ごめん!財布落としちゃって…。食事をおごる話は
\\	なかったことにして
\\	くれる。
\\	ごめん!財布落としちゃって…。食事をおごる話は
\\	(ない)
\\	こと_して
\\	くれる。
\\	中17
\\	"友達にもらった写真を自分で撮っ
\\	たことにして
\\	、コンテストに応募して、入賞しちゃった人がいたらしいよ。
\\	友達にもらった写真を自分で撮っ
\\	た__に__
\\	、コンテストに応募して、入賞しちゃった人がいたらしいよ。
\\	中17
\\	"子供のとき、嫌いなおかずをいつも犬にやって自分で食べ
\\	たことにしていた
\\	のを母は知らなかった。
\\	子供のとき、嫌いなおかずをいつも犬にやって自分で
\\	(食べる)
\\	こと_していた
\\	のを母は知らなかった。
\\	中17
\\	"あの人は小説家
\\	の
\\	つもり
\\	らしいけど、書いた小説が出版されたことという話を聞いたことがない。
\\	あの人は小説家
\\	の
\\	らしいけど、書いた小説が出版されたことという話を聞いたことがない。
\\	中17
\\	"説明書通りにやっている
\\	つもり
\\	なのに、どうしてもうまくできない。
\\	説明書通りにやっている
\\	なのに、どうしてもうまくできない。
\\	中17
\\	"柔道を始めたのは、子供のころいつもけんかで兄に負けるのが、くやしく
\\	てならなかった
\\	からです。
\\	柔道を始めたのは、子供のころいつもけんかで兄に負けるのが、悔しく
\\	て___かった
\\	からです。
\\	中17
\\	"私のように年をとってくると、古里の山々が懐かしく思い出され
\\	てならない
\\	んですよ。
\\	私のように年をとってくると、古里の山々が懐かしく思い出され
\\	て____
\\	んですよ。
\\	中17
\\	"不老不死[ふろうふし]の願いがかなう
\\	ものなら
\\	、俺はいくらでも金を出す。
\\	不老不死[ふろうふし]の願いがかなう
\\	もの__
\\	、俺はいくらでも金を出す。
\\	中17
\\	"子供の頃から星が好きだったので、行ける
\\	ものなら
\\	宇宙旅行に行ってみたいと思っています。
\\	子供の頃から星が好きだったので、行ける
\\	なら
\\	宇宙旅行に行ってみたいと思っています。
\\	中17
\\	"子犬が
\\	渡
\\	ろうか
\\	渡る
\\	まいか
\\	と小川の前でうろうろしている。
\\	"子犬が(渡る)_
\\	か
\\	(渡る)__
\\	まいか
\\	と小川の前でうろうろしている。
\\	中17
\\	"高橋さんは夏休みに地元へ
\\	帰
\\	ろうか
\\	帰る
\\	まいか
\\	と悩んでいるらしい。
\\	高橋さんは夏休みに地元へ
\\	帰
\\	ろう_
\\	帰る
\\	と悩んでいるらしい。
\\	中17
\\	"この商店街で30年頑張ってきましたが、悩んだ
\\	末
\\	に店を閉めることにしました。
\\	この商店街で30年頑張ってきましたが、悩んだ
\\	に店を閉めることにしました。
\\	中17
\\	"子供がいるので、
\\	たとえ
\\	給料が高く
\\	ても
\\	、土日に休めない仕事はしたくなく。
\\	子供がいるので、
\\	給料が高く
\\	ても
\\	、土日に休めない仕事はしたくなく。
\\	中17
\\	たとえ
\\	どんなに小さい子供
\\	でも
\\	、悪いことをしたら、謝らなければならないことを教える必要がある。
\\	どんなに小さい子供
\\	でも
\\	、悪いことをしたら、謝らなければならないことを教える必要がある。
\\	中17
\\	"この条件では、鈴木
\\	商事[しょうじ]からの合併の申し出は受け入れ
\\	がたい
\\	この条件では、鈴木
\\	商事[しょうじ]からの合併の申し出は受け入れ
\\	中17
\\	"いい記録を出すために
\\	薬物[やくぶつ]を使用するのは、フェアプレイの精神に反する
\\	許し
\\	がたい
\\	行為だ。
\\	いい記録を出すために
\\	薬物[やくぶつ]を使用するのは、フェアプレイの精神に反する
\\	許し
\\	行為だ。
\\	中17
\\	"住民の迷惑
\\	もかまわず
\\	、夜中にバイクで走りまわるのはやめて欲しい。
\\	住民の迷惑
\\	も____
\\	、夜中にバイクで走りまわるのはやめて欲しい。
\\	中17
\\	"僕たちは虫に刺されるの
\\	もかまわず
\\	、森の中で毎日遊んでいた。
\\	僕たちは虫に刺されるの
\\	も____
\\	、森の中で毎日遊んでいた。
\\	中17
\\	"電車の中で足を踏まれて、痛みさ
\\	の
\\	あまり
\\	声も出なかった。
\\	電車の中で足を踏まれて、痛みさ
\\	の
\\	声も出なかった。
\\	中17
\\	"当店では一つ一つのお菓子を、心
\\	を込めて
\\	手作りしております。
\\	当店では一つ一つのお菓子を、心
\\	を___
\\	手作りしております。
\\	中17
\\	"クラスメートはお互いに親しみ
\\	を込めて
\\	、ニックネームで呼び合っている。
\\	クラスメートはお互いに親しみ
\\	を___
\\	、ニックネームで呼び合っている。
\\	中17
\\	"今度できたスーパーは
\\	品数[しなかず]
\\	も
\\	多けれ
\\	ば
\\	値段
\\	も
\\	安いので、大人気だ。
\\	今度できたスーパーは品数
\\	多けれ
\\	値段
\\	安いので、大人気だ。
\\	中17
\\	"一度と決めたらには、どんなに困難があっても最後までやり
\\	ぬく
\\	覚悟です。
\\	一度と決めたらには、どんなに困難があっても最後までやり
\\	覚悟です。
\\	中17
\\	"父が祖父から受け継ぎ、守り
\\	ぬいた
\\	この店を、これからは僕がもっと大きく育てていくつもりだ。
\\	父が祖父から受け継ぎ、守り
\\	た
\\	この店を、これからは僕がもっと大きく育てていくつもりだ。
\\	中17
\\	"娘も昔は「パパ、大好き!」と言ってくれて、可愛かった
\\	ものだ
\\	が…。
\\	娘も昔は「パパ、大好き!」と言ってくれて、可愛かった
\\	だ
\\	が…。
\\	中17
\\	"うちの猫はちっとも家に帰ってこない。困った
\\	ものだ
\\	うちの猫はちっとも家に帰ってこない。困った
\\	だ
\\	中17
\\	"遺産
\\	をめぐって
\\	、兄弟の争いが起こることを父は心配している。
\\	遺産
\\	を____
\\	、兄弟の争いが起こることを父は心配している。
\\	中17
\\	"どこのマンションでも、
\\	改修[かいしゅう]工事
\\	をめぐる
\\	話し合いは、なかなかまとまらないものだ。
\\	どこのマンションでも、
\\	改修[かいしゅう]工事
\\	を___
\\	話し合いは、なかなかまとまらないものだ。
\\	中17
\\	"うちのお店は若い人向けの服を中心に扱っているので、20~30代のお客様が多い
\\	わけだ
\\	うちのお店は若い人向けの服を中心に扱っているので、20~30代のお客様が多い
\\	だ
\\	中17
\\	"寮では毎日一緒に生活する
\\	わけだ
\\	から、お互いにルールを守ることが大切だ。
\\	寮では毎日一緒に生活する
\\	だ
\\	から、お互いにルールを守ることが大切だ。
\\	中17
\\	"開会
\\	にあたって
\\	、一言ご
\\	挨拶[あいさつ]申し上げます。
\\	開会
\\	に____
\\	、一言ご挨拶申し上げます。
\\	中17
\\	"研修を始める
\\	にあたり
\\	、社員としての
\\	心構[こころが]えについてお話しします。
\\	研修を始める
\\	に___
\\	、社員としての
\\	心構[こころが]えについてお話しします。
\\	中17
\\	"これからのサービス
\\	に加え
\\	、新たなサービスを企画して、お客様のニーズに
\\	応[こた]えたい。
\\	これからのサービス
\\	に__
\\	、新たなサービスを企画して、お客様のニーズに応えたい。
\\	中17
\\	"家を買う場合は、不動産屋の
\\	手数料[てすうりょう]
\\	に加え
\\	、税金や引っ越し費用など、
\\	購入[こうにゅう]代金のほかにも様々な経費がかかる。
\\	家を買う場合は、不動産屋の手数料
\\	に__
\\	、税金や引っ越し費用など、購入代金のほかにも様々な経費がかかる。
\\	中17
\\	"仮[かり]に私を悲しませないためのうそだ
\\	としても
\\	、私は絶対許せない。
\\	としたって
\\	仮に私を悲しませないためのうそだ
\\	と___
\\	、私は絶対許せない。
\\	中17
\\	"無理して働いて病気になった
\\	としても
\\	会社は
\\	補償[ほしょう]してくれないよ。
\\	としたって
\\	無理して働いて病気になった
\\	と___
\\	会社は
\\	補償[ほしょう]してくれないよ。
\\	中17
\\	"集めた資料
\\	を基づいて
\\	、論文を書いた。
\\	集めた資料
\\	を___て
\\	、論文を書いた。
\\	中17
\\	"この映画はある地方に伝わる伝説
\\	をもとにして
\\	作られたと言われている。
\\	を基づいて	
\\	この映画はある地方に伝わる伝説
\\	を___して
\\	作られたと言われている。
\\	中17
\\	"どんな健康法も、続け
\\	てこそ
\\	効果がある。
\\	どんな健康法も、続け
\\	て__
\\	効果がある。
\\	中17
\\	"相手に分かりやすく説明でき
\\	てこそ
\\	、本当の知識と言えるのです。
\\	相手に分かりやすく説明でき
\\	て__
\\	、本当の知識と言えるのです。
\\	中17
\\	"イギリスの人気小説が日本語に翻訳されました。
\\	それ
\\	が今年日本語でベストセラーになりました。
\\	"イギリスの人気小説が日本語に翻訳されました。
\\	れ
\\	が今年日本語でベストセラーになりました。
\\	中17
\\	"リクさん、少し太った
\\	んじゃない
\\	のではありませんか, 
\\	"リクさん、少し太った
\\	中17
\\	"元気がないですね。何か困っていることがある
\\	んじゃない
\\	ですか。
\\	のではありませんか, 
\\	"元気がないですね。何か困っていることがある
\\	ですか。
\\	中17
\\	"あの信号を左へ曲がっ
\\	たところに
\\	、郵便局があります。
\\	"あの信号を左へ曲がっ
\\	た___に
\\	、郵便局があります。
\\	中17
\\	"改札を出て、階段を上っ
\\	たところで
\\	、待っていてください。
\\	"改札を出て、階段を上っ
\\	た___で
\\	、待っていてください。
\\	中17
\\	"家を出
\\	ようとした
\\	とき、電話がかかってきた。
\\	"家を出
\\	とした
\\	とき、電話がかかってきた。
\\	中17
\\	"雨がやんだので、桜を撮
\\	ろうとした
\\	ら、カメラの電池が切れてしまった。
\\	"雨がやんだので、桜を撮
\\	とした
\\	ら、カメラの電池が切れてしまった。
\\	中17
\\	"妻は紅茶が好きで、お茶やコーヒーを飲
\\	もうとしない
\\	"妻は紅茶が好きで、お茶やコーヒーを飲
\\	としない
\\	中17
\\	"人の話を聞
\\	こうとしない
\\	人は、いつまでたっても自分の考えを変えることができません。
\\	"人の話を聞
\\	としない
\\	人は、いつまでたっても自分の考えを変えることができません。
\\	中17
\\	"この店ではクレジットカードが使える
\\	のだろうか
\\	"この店ではクレジットカードが使える
\\	の___か
\\	中17
\\	"このクラスでは日本語で話すチャンスがとても少ない。こんな勉強で会話が上手になる
\\	のだろう
\\	か。
\\	"このクラスでは日本語で話すチャンスがとても少ない。こんな勉強で会話が上手になる
\\	の___か
\\	中17
\\	"友達
\\	と
\\	の
\\	旭川旅行は、とても楽しかったです。
\\	"友達
\\	と
\\	旭川旅行は、とても楽しかったです。
\\	中17
\\	"国の両親
\\	から
\\	の
\\	手紙を読んで、泣いてしまった。
\\	"国の両親
\\	から
\\	手紙を読んで、泣いてしまった。
\\	中17
\\	"先生
\\	へ
\\	の
\\	お土産は何がいいでしょうか。 
\\	に
\\	への
\\	"先生
\\	に
\\	お土産は何がいいでしょうか。
\\	中17
\\	"アジアの経済はこれからますます発展する
\\	だろう
\\	だろうと思う
\\	"アジアの経済はこれからますます発展する
\\	中17
\\	"この実験にはあと2,3週間はかかる
\\	だろう
\\	だろうと思う
\\	"この実験にはあと2,3週間はかかる
\\	中17
\\	"お金を入れるとボタンに電気がつき
\\	ます
\\	から
\\	、それを押してください。
\\	"お金を入れるとボタンに電気がつき
\\	ます
\\	、それを押してください。
\\	中17
\\	"10分ぐらいで戻ってき
\\	ます
\\	から
\\	、ここで待っていてくれますか。
\\	"10分ぐらいで戻ってき
\\	ます
\\	、ここで待っていてくれますか。
\\	中17
\\	"留学生
\\	の
\\	が
\\	描いた絵を見た。
\\	"留学生
\\	描いた絵を見た。
\\	中17
\\	"三井さん
\\	が
\\	の
\\	作ったケーキはとてもおいしかった。
\\	"三井さん
\\	作ったケーキはとてもおいしかった。
\\	中17
\\	"店の前に「本日休業」
\\	って
\\	書いてありました。 
\\	て
\\	と	
\\	店の前に「本日休業」
\\	書いてありました。
\\	中17
\\	"昨日、竹谷さん
\\	って
\\	人が来ましたよ。
\\	という	
\\	昨日、竹谷さん
\\	人が来ましたよ。
\\	中17
\\	"ねえ、函館
\\	って
\\	、どんな街?
\\	ねえ、函館
\\	、どんな街?
\\	中17
\\	"仕事が忙しいので、今夜のパーティーに出る
\\	つもりはない
\\	出ないつもりだ
\\	仕事が忙しいので、今夜のパーティーに出る
\\	は__
\\	中17
\\	"「1週間くらい休みを取ったらどうですか?」
\\	「いえ、その
\\	つもりはありません
\\	「1週間くらい休みを取ったらどうですか?」
\\	「いえ、その
\\	は_____
\\	中17
\\	"電話する
\\	つもりだったが
\\	、忘れてしまいました。すみません。
\\	電話する
\\	だった_
\\	、忘れてしまいました。すみません。
\\	中17
\\	"「鍵がかけていませんでしたよ。」
\\	「すみません、かけた
\\	つもり
\\	でした。」
\\	「鍵がかけていませんでしたよ。」
\\	「すみません、かけた
\\	でした。」
\\	中17
\\	"本当の研究発表
\\	の
\\	つもり
\\	で、みんなの前で話してください。
\\	本当の研究発表
\\	の
\\	で、みんなの前で話してください。
\\	中17
\\	"私は一生懸命やっ
\\	ている
\\	つもり
\\	です。
\\	やっているつもり 
\\	やるつもり 
\\	私は一生懸命やっ
\\	ている
\\	です。
\\	中17
\\	"この猫は一日中、寝て
\\	ばかり
\\	いる。
\\	この猫は一日中、寝て
\\	いる。
\\	中17
\\	"子供のとき、母に「勉強しろ」
\\	とか
\\	「たくさん食べなさい」
\\	とか
\\	よく言われました。
\\	と
\\	言う
\\	子供のとき、母に「勉強しろ」
\\	「たくさん食べなさい」
\\	よく言われました
\\	中17
\\	"暗くなって、星が見え
\\	てきた
\\	暗くなって、星が見え
\\	て__
\\	中17
\\	"隣の家からいいにおいがし
\\	てきた
\\	隣の家からいいにおいがし
\\	て__
\\	中17
\\	"兄が旅行から帰っ
\\	てきた
\\	兄が旅行から帰っ
\\	て
\\	(きた/いった
\\	中17
\\	"授業のあと、学生達はうちへ帰っ
\\	ていった
\\	授業のあと、学生達はうちへ帰っ
\\	て
\\	(きた/いった
\\	中17
\\	"新聞に
\\	こ
\\	んなことが書いてあった。最近の日本人は家族みんなで休日にコンピューターゲームを楽しみそうだ。
\\	新聞に
\\	んなことが書いてあった。最近の日本人は家族みんなで休日にコンピューターゲームを楽しみそうだ。
\\	中17
\\	"この薬は一日2回飲ま
\\	なくてはならない
\\	飲ま
\\	なくてはいけない
\\	なければならない	
\\	この薬は一日2回飲ま
\\	なくては____
\\	中17
\\	"レポートは日本語で
\\	なくてはなりません
\\	なくてはいけません
\\	なければならない	
\\	レポートは日本語で
\\	なくては_____
\\	中17
\\	"熱が下がったら、薬を飲ま
\\	なくてもかまわない
\\	なくてもいい	
\\	熱が下がったら、薬を飲ま
\\	なくても_____
\\	中17
\\	"作文は長く
\\	なくてもかまいません
\\	なくてもいい	
\\	作文は長く
\\	なくても_____
\\	中17
\\	立たなくちゃいけない、立たなくちゃ
\\	立たなきゃ
\\	立たなくてはいけない。 			中17
\\	"「申し込みはどうするんですか?」
\\	「この紙に名前を書く
\\	だけでいい
\\	んです。」
\\	「申し込みはどうするんですか?」
\\	「この紙に名前を書く
\\	で__
\\	んです。」
\\	中17
\\	"「お父さんにの誕生日のプレゼントは何がいい
\\	かな
\\	「セーターはどう
\\	かな
\\	「お父さんにの誕生日のプレゼントは何がいい
\\	「セーターはどう
\\	中17
\\	"「3時までにこの資料を全部コピーしなければならないんだけど、手伝ってくれ
\\	ない
\\	かな
\\	「いいよ。」
\\	「3時までにこの資料を全部コピーしなければならないんだけど、手伝ってくれ
\\	ない
\\	「いいよ。」
\\	中17
\\	"私の絵
\\	なんか
\\	みんなに見せないでください。絵が下手なんです。
\\	なんて 
\\	なんか
\\	など	
\\	私の絵
\\	みんなに見せないでください。絵が下手なんです。
\\	中17
\\	"先生が3年も前に事故に遭って亡くなった
\\	なんて
\\	、知りませんでした。
\\	先生が3年も前に事故に遭って亡くなった
\\	、知りませんでした。
\\	中17
\\	"殴[なぐ]って、弟
\\	を
\\	泣
\\	かせた
\\	ことがある。
\\	殴って、弟
\\	を
\\	(泣った)
\\	ことがある。
\\	中17
\\	"テストで100点を取って、母
\\	を
\\	びっくり
\\	させた
\\	テストで100点を取って、母
\\	を
\\	びっくり
\\	(した)
\\	中17
\\	"子供が書いた作文はすばらしく、関心
\\	させられた
\\	子供が書いた作文はすばらしく、関心
\\	(した)
\\	中17
\\	"何度買っても宝くじが当たらず、がっかり
\\	させられた
\\	何度買っても宝くじが当たらず、がっかり
\\	(した)
\\	中17
\\	"日曜大工でいすを作る
\\	なら
\\	、まず材料に良い木を選ばなくてはいきません。
\\	日曜大工でいすを作る
\\	、まず材料に良い木を選ばなくてはいきません。
\\	中17
\\	"頭が痛い
\\	なら
\\	、この薬を飲むといいですよ。
\\	頭が痛い
\\	、この薬を飲むといいですよ。
\\	中17
\\	"「リユさんはお母さんに何と言いましたか。」
\\	「7時に起こし
\\	てくれ
\\	といいました。」
\\	てください	
\\	「リユさんはお母さんに何と言いましたか。」
\\	「7時に起こし
\\	て
\\	(ください)
\\	といいました。」
\\	中17
\\	"店長:しぶや君、この資料をコピーし
\\	てくれ
\\	店長:しぶや君、この資料をコピーし
\\	て
\\	(ください)
\\	中17
\\	"電車に乗っている
\\	あいだ
\\	、本を読んでいた。
\\	電車に乗っている
\\	、本を読んでいた。
\\	中17
\\	"夏休みの
\\	あいだ
\\	、ずっと国に帰っていた。
\\	夏休みの
\\	、ずっと国に帰っていた。
\\	中17
\\	"旅行の
\\	あいだに
\\	、アパートに泥棒が入った。
\\	旅行の
\\	に
\\	、アパートに泥棒が入った。
\\	中17
\\	"先生が来
\\	るまでに
\\	、掃除を終わらせた。
\\	先生が
\\	(来る)
\\	までに
\\	、掃除を終わらせた。
\\	中17
\\	"眼鏡をかけ
\\	た
\\	山中先生
\\	本を読んでいるやつ、本を読んだやつ
\\	ている 
\\	眼鏡を(かけている)__
\\	山中先生
\\	中17
\\	"曲がっ
\\	た
\\	線
\\	本を読んでいるやつ、本を読んだやつ
\\	ている 
\\	(曲がっている)
\\	線
\\	中17
\\	"季節
\\	によって
\\	景色が変わる。
\\	季節
\\	に___
\\	景色が変わる。
\\	中17
\\	"昨夜の地震にびっくりして、下着
\\	の
\\	まま
\\	、外に出た。
\\	昨夜の地震にびっくりして、下着
\\	の
\\	、外に出た。
\\	中17
\\	"急にドアが開いたのは、だれかがボタンを押した
\\	からだ
\\	急にドアが開いたのは、だれかがボタンを押した
\\	だ
\\	中17
\\	"この人形は大きい目
\\	をしています
\\	この人形は大きい目
\\	を_____
\\	中17
\\	"このパンは帽子みたいな形
\\	をしている
\\	このパンは帽子みたいな形
\\	を____
\\	中17
\\	何をお読みですか。	
\\	"何を読んでいますか。
\\	中17
\\	いい時計をお持ちですね。	
\\	"いい時計を持っていますね。
\\	中17
\\	時間がおありですか。	
\\	"時間がありますか。
\\	中17
\\	部長は何時にお着きですか。	
\\	"部長は何時に着きますか。
\\	中17
\\	今、お帰りでしたか。	
\\	"今、帰りましたか。
\\	中17
\\	おいでです	
\\	"行く、いる、来る 
\\	中17
\\	お越しです、お見えです	
\\	"来る 
\\	中17
\\	お召しです	
\\	"食べる 
\\	中17
\\	お休みです	
\\	"寝る 
\\	中17
\\	お住まいです	
\\	"住んでいる 
\\	中17
\\	ご存じです	
\\	"知っている 
\\	中17
\\	ここに座ってもかまいませんか。	
\\	"ここに座ってもいいですか。
\\	中17
\\	"山中先生は厳しいですか。
\\	…ええ、でも、鈴木先生
\\	ほど
\\	厳し
\\	くない
\\	山中先生は厳しいですか。
\\	…ええ、でも、鈴木先生
\\	厳し
\\	くない
\\	中18
\\	"このレストランは人気があるそうだが、料理は思った
\\	ほど
\\	おいし
\\	くなかった
\\	このレストランは人気があるそうだが、料理は思った
\\	おいしくなかった。
\\	中18
\\	"この島で見る星
\\	ほど
\\	美しいもの
\\	はない
\\	この島で見る星
\\	美しいもの
\\	はない
\\	中18
\\	"大雪が降った
\\	ために
\\	、空港が使えなくなりました。
\\	大雪が降った
\\	に
\\	、空港が使えなくなりました。
\\	中18
\\	"機械が使えなくなったのは、不備がある
\\	ためだ
\\	機械が使えなくなったのは、不備がある
\\	だ
\\	中18
\\	"もう少し安
\\	ければ
\\	、買っ
\\	た
\\	んですが。
\\	もう少し安
\\	(買う)
\\	んですが。
\\	中18
\\	"お金があ
\\	れば
\\	、このスマホが買え
\\	た
\\	のに。
\\	あ
\\	ったら
\\	お金があ
\\	、このスマホが
\\	(買える)
\\	のに。
\\	中18
\\	"子供が8人もいたから、生活は楽ではなかった
\\	はずだ
\\	"子供が8人もいたから、生活は楽ではなかった
\\	だ
\\	中18
\\	"「加藤先生は来ますか。」
\\	「はい、その
\\	はずだ
\\	「加藤先生は来ますか。」
\\	「はい、その
\\	だ
\\	中18
\\	"あんなに練習したんだから、今日の試合は負ける
\\	はずがない
\\	あんなに練習したんだから、今日の試合は負ける
\\	が__
\\	中18
\\	"階段の前に1週間前から赤い自転車が置いてある。レモンさんも赤い自転車を持っているが、今修理に出てある。だからこの自転車はレモンさんの自転車の
\\	はずがない
\\	階段の前に1週間前から赤い自転車が置いてある。レモンさんも赤い自転車を持っているが、今修理に出てある。だからこの自転車はレモンさんの自転車の
\\	が__
\\	中18
\\	"「鍵がかかっていなかったよ。」
\\	「そんな
\\	はずはありません
\\	「鍵がかかっていなかったよ。」
\\	「そんな
\\	は_____
\\	中18
\\	"パーティーに出ない
\\	はずだった
\\	が、部長が都合が悪くなったので、私が出席することになった。
\\	パーティーに出ない
\\	だった
\\	が、部長が都合が悪くなったので、私が出席することになった。
\\	中18
\\	"旅行に行く
\\	はずだった
\\	。しかし、病気で行けなくなった。
\\	旅行に行く
\\	だった
\\	。しかし、病気で行けなくなった。
\\	中18
\\	"このエレベーターは古いから、たまに
\\	止まる
\\	こと
\\	が[も]ある
\\	このエレベーターは古いから、たまに
\\	止まる
\\	ある
\\	中18
\\	"母の料理はいつもおいしいが、時々
\\	おいしくない
\\	こと
\\	も[が]ある
\\	母の料理はいつもおいしいが、時々
\\	おいしくない
\\	ある
\\	中18
\\	"調査
\\	した[の]結果
\\	、この町の人口が減ってきていることがわかりました。
\\	調査
\\	結果
\\	、この町の人口が減ってきていることがわかりました。
\\	中18
\\	"両親と
\\	話し合った[話し合いの]結果
\\	、アメリカへ留学することに決めました。
\\	両親と
\\	(話し合う)
\\	結果
\\	、アメリカへ留学することに決めました。
\\	中18
\\	歩き出す	
\\	"歩く
\\	中18
\\	読み出す	
\\	読む
\\	中18
\\	"先生がいらっしゃったら、食べ
\\	始めましょう
\\	出す 
\\	先生がいらっしゃったら、
\\	(食べよう)
\\	中18
\\	"本を読
\\	み始めて
\\	ください。
\\	出す 
\\	本を
\\	(読んで)
\\	ください。
\\	中18
\\	"急に雨が降
\\	り出した
\\	急に雨が
\\	(降った)
\\	中18
\\	"宿題の作文を書
\\	き終わる
\\	前に、友達が遊びに来た。
\\	宿題の作文を
\\	(書く)
\\	前に、友達が遊びに来た。
\\	中18
\\	"5分間走
\\	り続けて
\\	ください。
\\	5分間
\\	(走って)
\\	ください。
\\	中18
\\	"今日の料理は塩を入れ
\\	忘れた
\\	ので、おいしくない。
\\	今日の料理は塩を
\\	(入れた)
\\	ので、おいしくない。
\\	中18
\\	"困ったときこそ助け
\\	合う
\\	ことが大切だ。
\\	困ったときこそ
\\	(助ける)
\\	ことが大切だ。
\\	中18
\\	"部屋の空気を入れ
\\	換えた
\\	部屋の空気を
\\	(入れた)
\\	中18
\\	"申し込む学生が10人以上にならなければ、この旅行は中止
\\	ということになる
\\	申し込む学生が10人以上にならなければ、この旅行は中止
\\	と__ことに__
\\	中18
\\	"今夜カレーを食べれば、3日続けてカレーを食べた
\\	ということになる
\\	今夜カレーを食べれば、3日続けてカレーを食べた
\\	いう___なる
\\	中18
\\	"だんだん春らしくなって
\\	きました
\\	だんだん春らしくなって
\\	た
\\	中18
\\	"これからは、日本で働く外国人が増えて
\\	いく
\\	でしょう。
\\	これからは、日本で働く外国人が増えて
\\	でしょう。
\\	中18
\\	"「今日は恋人の誕生日なんだ。」
\\	「電話でもかけてあげ
\\	たら
\\	いかがですか[どう?]。」
\\	「今日は恋人の誕生日なんだ。」
\\	「電話でもかけてあげ
\\	いかがですか[どう?]。」
\\	中18
\\	"「少し熱があるみたい…。」
\\	「薬を飲んで、今日は早く寝
\\	たら
\\	どう?」
\\	寝
\\	たら
\\	「少し熱があるみたい…。」
\\	「薬を飲んで、今日は早く寝
\\	どう?」
\\	中18
\\	"子供に食べさせる野菜は、値段が安い
\\	より
\\	、安全な
\\	ほうが
\\	いい。
\\	子供に食べさせる野菜は、値段が安い
\\	、安全な
\\	いい。
\\	中18
\\	"今日は、北海道
\\	より
\\	東京の
\\	ほうが
\\	気温が低かったです。
\\	今日は、北海道
\\	東京の
\\	気温が低かったです。
\\	中18
\\	"高橋さんの家はいかにも日本の家
\\	らしい
\\	家です。
\\	高橋さんの家はいかにも日本の家
\\	家です。
\\	中18
\\	"文句を言うのはあなた
\\	らしくない
\\	文句を言うのはあなた
\\	くない
\\	中18
\\	"新聞によると、昨日の朝韓国で大きい地震があった
\\	らしい
\\	新聞によると、昨日の朝韓国で大きい地震があった
\\	中18
\\	"パーティーが始めた
\\	らしい
\\	。会場の中からにぎやかな声が聞こえてくる。
\\	パーティーが始めた
\\	。会場の中からにぎやかな声が聞こえてくる。
\\	中18
\\	"会社の代表
\\	として
\\	、お客さんに新しい商品の説明をした。
\\	会社の代表
\\	、お客さんに新しい商品の説明をした。
\\	中18
\\	"その男は先週の土曜日にこの店に来て、一言も話
\\	せずに
\\	酒を飲んでいた。
\\	ないで	
\\	その男は先週の土曜日にこの店に来て、一言も話
\\	に
\\	酒を飲んでいた。
\\	中18
\\	"子供の熱が下が
\\	らず
\\	、心配しました。
\\	なくて 
\\	子供の熱が下が
\\	ら_
\\	、心配しました。
\\	中18
\\	"この寺は今まで2回火事で焼け
\\	ている
\\	この寺は今まで2回火事で焼け
\\	中18
\\	"小原さんは小学校のときアメリカで住
\\	んでいる
\\	。だから、英語の発音がきれいだ。
\\	小原さんは小学校のときアメリカで住
\\	。だから、英語の発音がきれいだ。
\\	中18
\\	"これ
\\	なんか
\\	いかがですか[どう?]
\\	これ
\\	いかがですか[どう?]
\\	中18
\\	"「次の議長はだれがいいかな。」
\\	「片島君
\\	なんか
\\	いいと思うよ。」
\\	「次の議長はだれがいいかな。」
\\	「片島君
\\	いいと思うよ。」
\\	中18
\\	"急いでいた
\\	ものですから
\\	、鍵をかけるのを忘れてしまいました。
\\	急いでいた
\\	です__
\\	、鍵をかけるのを忘れてしまいました。
\\	中18
\\	"私
\\	は[が]
\\	先生
\\	に
\\	注意された。
\\	"先生は私を注意した。
\\	中18
\\	"私
\\	は[が]
\\	部長
\\	に
\\	仕事を頼まれた。
\\	"部長は私に仕事を頼んだ。
\\	中18
\\	私
\\	は[が]泥棒に財布を盗まれた。	
\\	"泥棒が私の財布を盗んだ。
\\	中18
\\	"あなた
\\	に
\\	そこに立てられると、前が見えません。
\\	"あなたがそこに立つと
\\	、前が見えません。
\\	中18
\\	"私は猫
\\	に
\\	急に死なれて、大学に行けなかった。
\\	"私の猫が急に死んで
\\	、大学に行けなかった。
\\	中18
\\	"こんな所に信号
\\	を
\\	作られて、車が渋滞するようになってしまった。
\\	こんな所に信号を
\\	(作る)
\\	、車が渋滞するようになってしまった。
\\	中18
\\	"私の家の前にゴミ
\\	を
\\	捨てられて困っています。
\\	私の家の前にゴミを
\\	(捨てる)
\\	困っています。
\\	中18
\\	"映画を見ているとき、笑っ
\\	たり
\\	泣い
\\	たり
\\	しました。
\\	映画を見ているとき、笑っ
\\	泣い
\\	しました。
\\	中18
\\	"この廊下は人が通ると、電気がつい
\\	たり
\\	消え
\\	たり
\\	します。
\\	この廊下は人が通ると、電気がつい
\\	消え
\\	します。
\\	中18
\\	"この店の食べ物は種類が多くて、甘かっ
\\	たり
\\	、辛かっ
\\	たり
\\	します。
\\	この店の食べ物は種類が多くて、甘かっ
\\	、辛かっ
\\	します。
\\	中18
\\	"服が脱ぎ
\\	っぱなし
\\	だ。片付けないから、部屋が汚い。
\\	服が脱ぎ
\\	っ___
\\	だ。片付けないから、部屋が汚い。
\\	中18
\\	"こらっ。ドアが開け
\\	っぱなし
\\	だよ。早く閉めなさい。
\\	こらっ。ドアが開け
\\	っ___
\\	だよ。早く閉めなさい。
\\	中18
\\	"地図の説明が丁寧
\\	な
\\	おかげで
\\	、待ち合わせの場所がすぐにわかりました。
\\	地図の説明が丁寧
\\	な
\\	お___
\\	、待ち合わせの場所がすぐにわかりました。
\\	中18
\\	"風邪薬
\\	の
\\	せいで
\\	、眠くなった。
\\	風邪薬
\\	の
\\	で
\\	、眠くなった。
\\	中18
\\	"カムムさんはお酒を飲んだ
\\	みたい
\\	です。顔が赤いです。
\\	カムムさんはお酒を飲んだ
\\	です。顔が赤いです。
\\	中18
\\	"資料が届いた
\\	よう
\\	ですので、事務室に取りに行って参ります。
\\	資料が届いた
\\	ですので、事務室に取りに行って参ります。
\\	中18
\\	"私は
\\	どちらかと言えば
\\	、好き嫌いはあまりない
\\	ほう
\\	だ。
\\	私は
\\	と言えば
\\	、好き嫌いはあまりない
\\	だ。
\\	中18
\\	"「食事しながらお酒を飲みますか。」
\\	「そうですね。いつもではありませんが、
\\	どちらかと言えば
\\	、飲む
\\	ほう
\\	です。」
\\	「食事しながらお酒を飲みますか。」
\\	「そうですね。いつもではありませんが、
\\	と言えば
\\	、飲む
\\	です。」
\\	中18
\\	"この辺りには高い店が多いのですが、この店は
\\	どちらかと言えば
\\	、安い
\\	ほう
\\	です。
\\	この辺りには高い店が多いのですが、この店は
\\	と言えば
\\	、安い
\\	です。
\\	中18
\\	どうか[どうぞ]
\\	大きい地震が起き
\\	ません
\\	ように
\\	どうか[どうぞ]
\\	大きい地震が起き
\\	ません
\\	に
\\	中18
\\	"先生も
\\	どうぞ[どうか]
\\	風邪をひかれ
\\	ません
\\	ように
\\	"先生も
\\	どうぞ[どうか]
\\	風邪をひかれ
\\	ません
\\	に
\\	中18
\\	"全員か参加し
\\	てみろ
\\	、会場があふれてしまうよ。
\\	全員か参加し
\\	て__
\\	、会場があふれてしまうよ。
\\	中19
\\	"努力
\\	次第で
\\	、夢は実現する。
\\	努力
\\	次第_
\\	、夢は実現する。
\\	中19
\\	"教師のアイディア
\\	次第で
\\	、生徒の学力は伸びる。
\\	教師のアイディア
\\	で
\\	、生徒の学力は伸びる。
\\	中19
\\	"クラス全員が来る
\\	としたら
\\	、椅子が3つ足りない、隣の教室から持ってこよう。
\\	とすれば
\\	とすると
\\	クラス全員が来る
\\	と___
\\	、椅子が3つ足りない、隣の教室から持ってこよう。
\\	中19
\\	"天気予報のとおりに明日大雨だ
\\	とすると
\\	、花見の予定は変更しなければならない。
\\	としたら 
\\	とすれば
\\	天気予報のとおりに明日大雨だ
\\	と___
\\	、花見の予定は変更しなければならない。
\\	中19
\\	"国の母が入院した。できる
\\	ものなら
\\	、今すぐにも帰
\\	りたい
\\	国の母が入院した。できる
\\	もの__
\\	、今すぐにも帰
\\	り__。
\\	中19
\\	"プライドの高い伊藤さんを少しいでも批難し
\\	ようものなら
\\	、彼は起こるだろう。
\\	プライドの高い伊藤さんを少しいでも批難し
\\	もの__
\\	、彼は起こるだろう。
\\	中19
\\	"息子は「ただいま」と言った
\\	(か)
\\	と思ったら
\\	、もうベッドで横になっていた。
\\	と思うと
\\	息子は「ただいま」と言った
\\	かと____
\\	、もうベッドで横になっていた。
\\	中19
\\	"母はテレビをみながら泣いている
\\	(か)
\\	と思ったら
\\	、突然笑い始めた。 
\\	思うと
\\	母はテレビをみながら泣いている
\\	かと____
\\	、突然笑い始めた。
\\	中19
\\	"彼は宝石を手に取って見
\\	るか
\\	見
\\	ないかのうちに
\\	その価値を言い当ててしまう。
\\	彼は宝石を手に取って見
\\	か
\\	見
\\	かの__に
\\	その価値を言い当ててしまう。
\\	中19
\\	"この試験を受ける
\\	に際して
\\	、以下の書類を提出してください。
\\	の先に, 
\\	この試験を受ける
\\	際__
\\	、以下の書類を提出してください。
\\	中19
\\	"政府の能力は、非常時代
\\	に際して
\\	どのような素早く行動できるかで判断できる。
\\	の先に, 
\\	政府の能力は、非常時代
\\	に_して
\\	どのような素早く行動できるかで判断できる。
\\	中19
\\	"留学する
\\	にあたって
\\	、パスポートとビザを
\\	申請[しんせい]した。
\\	にあたり
\\	留学する
\\	に____
\\	、パスポートとビザを申請した。
\\	中19
\\	"コンピューター
\\	なしで
\\	、仕事をするのは難しい。
\\	が存在しない状態で	
\\	コンピューター
\\	で
\\	、仕事をするのは難しい。
\\	中19
\\	"許可
\\	なしで
\\	この部屋を使わないでください。
\\	が存在しない状態で
\\	許可
\\	なし_
\\	この部屋を使わないでください。
\\	中19
\\	"自分が正しいと思うことは、迷う
\\	ことなく
\\	やるべきだ。
\\	しないで、 
\\	自分が正しいと思うことは、迷う
\\	こと__
\\	やるべきだ。
\\	中19
\\	"日本に来てから大学に入るまで、一日も休む
\\	ことなく
\\	日本語の勉強を続けた。
\\	しないで、 
\\	日本に来てから大学に入るまで、一日も休む
\\	なく
\\	日本語の勉強を続けた。
\\	中19
\\	"会社の先輩は、文句を言う
\\	つつも
\\	、いつも私の仕事を手伝ってくれた。
\\	会社の先輩は、文句を言う
\\	、いつも私の仕事を手伝ってくれた。
\\	中19
\\	"彼女は化粧が落ちるの
\\	もかまわず
\\	、泣き続けた。
\\	彼女は化粧が落ちるの
\\	も___ず
\\	、泣き続けた。
\\	中19
\\	"講義も終わった
\\	ことだから
\\	、みんなで食事に行こう。
\\	講義も終わった
\\	だから
\\	、みんなで食事に行こう。
\\	中19
\\	"ここでは石油
\\	に代わる
\\	新しい
\\	燃料[ねんりょう]を使っている。
\\	ここでは石油
\\	に___
\\	新しい
\\	燃料[ねんりょう]を使っている。
\\	中19
\\	"社長
\\	に代わって
\\	、部長が来年度の計画をご説明します。
\\	社長
\\	(のかわりに・にかわって?)
\\	、部長が来年度の計画をご説明します。			中19
\\	"大統領は支援者の
\\	声援[せいえん]
\\	にこたえて
\\	手を振った。
\\	大統領は支援者の
\\	声援[せいえん]
\\	に____
\\	手を振った。
\\	中19
\\	"起業
\\	に先立つ
\\	資金は親から援助してもらった。
\\	起業
\\	に___
\\	資金は親から援助してもらった。
\\	中19
\\	"日本での生活が長くなる
\\	に従って
\\	日本の文化にも詳しくなった。
\\	日本での生活が長くなる
\\	に___
\\	日本の文化にも詳しくなった。
\\	中19
\\	"この国では医学の進歩
\\	に伴う
\\	高齢化が進んでいる。
\\	この国では医学の進歩
\\	に__
\\	高齢化が進んでいる。
\\	中19
\\	"社員たちは社長
\\	に対して
\\	給料を上がって欲しいと訴えた。
\\	社員たちは社長
\\	に_して
\\	給料を上がって欲しいと訴えた。
\\	中19
\\	"オリンピックの開催
\\	を
\\	契機[けいき]として
\\	都市整備が急ピッチで進まれた。
\\	オリンピックの開催
\\	を__として
\\	都市整備が急ピッチで進まれた。
\\	中19
\\	"県大会での優勝
\\	を
\\	契機[けいき]に
\\	今度は全国大会での優勝を目指す。
\\	県大会での優勝
\\	を__に
\\	今度は全国大会での優勝を目指す。
\\	中19
\\	"実話
\\	をもとにして
\\	映画を作った。
\\	実話
\\	を__にして
\\	映画を作った。
\\	中19
\\	"兄は何度も入学試験に失敗し
\\	たあげく
\\	、とうとう大学への進学をあきらめてしまった。
\\	兄は何度も入学試験に失敗し
\\	た___
\\	、とうとう大学への進学をあきらめてしまった。
\\	中19
\\	"東京の
\\	賃貸[ちんたい]マンションは狭い
\\	上に
\\	値段も高い。
\\	東京の
\\	賃貸[ちんたい]マンションは狭い
\\	値段も高い。
\\	中19
\\	"授業料を
\\	免除[めんじょ]される
\\	かわりに
\\	、学校の事務の仕事を手伝うことになった。
\\	授業料を
\\	免除[めんじょ]される
\\	、学校の事務の仕事を手伝うことになった。
\\	中19
\\	"福井県は眼鏡の生産
\\	にかけては
\\	全国一を誇っている。
\\	福井県は眼鏡の生産
\\	に___は
\\	全国一を誇っている。
\\	中19
\\	"子供
\\	にしたら
\\	ビールは単なる苦い飲み物でしかない。
\\	にすれば
\\	子供
\\	に___
\\	ビールは単なる苦い飲み物でしかない。
\\	中19
\\	"周囲の期待
\\	に反して
\\	、結局彼らは結婚しなかった。
\\	に反し
\\	周囲の期待
\\	に___
\\	、結局彼らは結婚しなかった。
\\	中19
\\	"ワサビ
\\	抜き
\\	のお寿司なんて食べたくない。
\\	ワサビ
\\	のお寿司なんて食べたくない。
\\	中19
\\	"この店ではメーカー・
\\	車種
\\	しゃしゅ]
\\	を問わず
\\	高額[こうがく]でバイクの買い取りを行っている。
\\	この店ではメーカー・
\\	車種
\\	しゃしゅ]
\\	を___
\\	高額[こうがく]でバイクの買い取りを行っている。
\\	中19
\\	"この大学は医学
\\	を中心として
\\	理解の学部が人気だ。
\\	この大学は医学
\\	を__として
\\	理解の学部が人気だ。
\\	中19
\\	"ディズニーランドは、子供
\\	はもちろん
\\	大人も楽しめる。
\\	もとより
\\	ディズニーランドは、子供
\\	は____
\\	大人も楽しめる。
\\	中19
\\	"墓地[ぼち]の建設
\\	をめぐって
\\	周辺の住民が反対運動を起こしている。
\\	墓地[ぼち]の建設
\\	を____
\\	周辺の住民が反対運動を起こしている。
\\	中19
\\	"この写真を見る
\\	につけ
\\	昔のことを思い出す。
\\	この写真を見る
\\	つ_
\\	昔のことを思い出す。
\\	中19
\\	"先月の売上げは約300万円、今月は合計およそ400万円であった。
\\	したがって
\\	、わずか1ヶ月で30
\\	以上伸びたことになる。
\\	先月の売上げは約300万円、今月は合計およそ400万円であった。
\\	し____
\\	、わずか1ヶ月で30
\\	以上伸びたことになる。
\\	中19
\\	"「天気予報によると明日は大雨になりそうだって。」
\\	「えっ、そう。
\\	だとすると
\\	、明日のお花見は無理かもしれないね。」
\\	だとすれば
\\	だとしたら
\\	「天気予報によると明日は大雨になりそうだって。」
\\	「えっ、そう。
\\	する_
\\	、明日のお花見は無理かもしれないね。」
\\	中19
\\	"仕事は9時からだが、私は8時までに会社に着くように出かける。
\\	なぜかというと
\\	、早い時間のほうが電車が空いていて快適だから。
\\	なぜなら
\\	仕事は9時からだが、私は8時までに会社に着くように出かける。
\\	かと___
\\	、早い時間のほうが電車が空いていて快適だから。
\\	中19
\\	"伊藤さんと渋谷さんは誰からもうらやましがれるカップルだった。
\\	それなのに
\\	、結婚してからはうまくいかなくなって、2年後に離婚してしまった。
\\	伊藤さんと渋谷さんは誰からもうらやましがれるカップルだった。
\\	そ____
\\	、結婚してからはうまくいかなくなって、2年後に離婚してしまった。
\\	中19
\\	"佐藤さんは優秀な会社員で、英語と中国語がぺらぺらで、スポーツも料理もできる。
\\	要するに
\\	万能[ばんのう]な女性だ。
\\	佐藤さんは優秀な会社員で、英語と中国語がぺらぺらで、スポーツも料理もできる。
\\	する_
\\	万能[ばんのう]な女性だ。
\\	中19
\\	"この学部では「スポーツ科学」は必要科目です。
\\	すなわち
\\	、この科目を取らなければ卒業できないのです。
\\	この学部では「スポーツ科学」は必要科目です。
\\	す___
\\	、この科目を取らなければ卒業できないのです。
\\	中19
\\	"昭和は大きく戦前と戦後に分けられる。
\\	いわば
\\	異なる2つの時代が1つの名前で呼ばれているようなものだ。
\\	昭和は大きく戦前と戦後に分けられる。
\\	い__
\\	異なる2つの時代が1つの名前で呼ばれているようなものだ。
\\	中19
\\	"卵は安くて調理が簡単な食材だ。
\\	しかも
\\	、栄養が豊富である。
\\	卵は安くて調理が簡単な食材だ。
\\	し__
\\	、栄養が豊富である。
\\	中19
\\	"日下部ちゃんは小学校1年生なのに家で留守番ができる。
\\	そればかりか
\\	、掃除や夕食の買い物までするそうだ。
\\	そればかりでなく
\\	日下部ちゃんは小学校1年生なのに家で留守番ができる。
\\	それ____
\\	、掃除や夕食の買い物までするそうだ。
\\	中19
\\	"次回は郊外学習の予定です。
\\	もっとも
\\	、雨が降ったら中止ですが。
\\	次回は郊外学習の予定です。
\\	も___
\\	、雨が降ったら中止ですが。
\\	中19
\\	"夕食まで自由時間です。
\\	ただし
\\	、外に出るときは必ず連絡してください。
\\	夕食まで自由時間です。
\\	た__
\\	、外に出るときは必ず連絡してください。
\\	中19
\\	"パーティーは7時から食堂で行いますので、
\\	お集まりください
\\	なお
\\	、参加費は無料です。
\\	パーティーは7時から食堂で行いますので、お集まりください。
\\	、参加費は無料です。
\\	中19
\\	"「コーヒー、飲む?
\\	それとも
\\	、お茶?」
\\	「コーヒー、飲む?
\\	そ___
\\	、お茶?」
\\	中19
\\	"「俺はハンサムなのだろうか。
\\	それとも
\\	、そうではないのだろうか。」
\\	「俺はハンサムなのだろうか。
\\	そ___
\\	、そうではないのだろうか。」
\\	中19
\\	"今日予約している店は魚料理がおいしいんです。……
\\	さて
\\	、みなさん
\\	揃[そろ]いましたね。そろそろでかけましょうか。
\\	今日予約している店は魚料理がおいしいんです。……
\\	、みなさん
\\	揃[そろ]いましたね。そろそろでかけましょうか。
\\	中19
\\	"昨日はひどい天気だったね。せっかくの休みなのにどこへも行けなかったよ。
\\	それはそうと
\\	、今日、漢字のテストがあるんだっけ? 
\\	それはさておき
\\	昨日はひどい天気だったね。せっかくの休みなのにどこへも行けなかったよ。
\\	そ_____
\\	、今日、漢字のテストがあるんだっけ?
\\	中19
\\	さん「今日は道が込んでるね。」
\\	さん「そうそう、宿題やった?難しかったよね。半分以上分からなかった。」
\\	さん
\\	それにしても
\\	、込んでるね。今日は何かあるのかなあ。」
\\	さん「今日は道が込んでるね。」
\\	さん「そうそう、宿題やった?難しかったよね。半分以上分からなかった。」
\\	さん
\\	それ____
\\	、込んでるね。今日は何かあるのかなあ。」
\\	中19
\\	"先生の意見は理解
\\	し
\\	がたい
\\	ものばかりだ。
\\	先生の意見は理解
\\	し
\\	ものばかりだ。
\\	中19
\\	"人のまねをして書いたレポートはおもしろくないものにな
\\	り
\\	がち
\\	だ。
\\	人のまねをして書いたレポートはおもしろくないものにな
\\	り
\\	だ。
\\	中19
\\	"最近ちょっと太り
\\	気味
\\	なの。ダイエットしなくちゃ。
\\	最近ちょっと太り
\\	なの。ダイエットしなくちゃ。
\\	中19
\\	"大量の数字は人間には扱
\\	い
\\	づらい
\\	ので、計算を任せるためにコンピューターが開発されたのである。
\\	大量の数字は人間には扱
\\	い
\\	ので、計算を任せるためにコンピューターが開発されたのである。
\\	中19
\\	"政府が出した
\\	改革案[かいかくあん]は問題
\\	だらけ
\\	だ。
\\	政府が出した
\\	改革案[かいかくあん]は問題
\\	だ。
\\	中19
\\	"今朝から熱
\\	っぽい
\\	今朝から熱
\\	っ__
\\	中19
\\	"この家は高齢者
\\	向
\\	き
\\	に
\\	作られている。
\\	この家は高齢者
\\	に
\\	作られている。
\\	中19
\\	"このパンフレットは外国人
\\	向
\\	け
\\	に
\\	、分かりやすい日本語で書かれています。
\\	このパンフレットは外国人
\\	に
\\	、分かりやすい日本語で書かれています。
\\	中19
\\	"まず、彼を言うことを聞
\\	こうではないか
\\	まず、彼を言うことを聞
\\	ではない_
\\	中19
\\	"どの学習者にも日本語の発音がやさしい
\\	とは限らない
\\	どの学習者にも日本語の発音がやさしい
\\	は_らない
\\	中19
\\	"世界的な食糧危機が起こら
\\	ないとも限らない
\\	世界的な食糧危機が起こ_
\\	と__らない
\\	中19
\\	"いい就職先が見つから
\\	ないとも限らない
\\	から、真面目に努力を続けるべきだ。
\\	"いい就職先が見つか_
\\	とも_らない
\\	から、真面目に努力を続けるべきだ。
\\	中19
\\	"この計画に問題があると考えられ
\\	なくはない
\\	この計画に問題があると考えられ
\\	はない
\\	中19
\\	"この
\\	漫才[まんざい]コンビはおもしろ
\\	くなくはない
\\	。しかし、他にもっとおもしろいコンビがある。
\\	この
\\	漫才[まんざい]コンビはおもしろ
\\	はない
\\	。しかし、他にもっとおもしろいコンビがある。
\\	中19
\\	"「彼は財布を盗んだのですか。」
\\	「いいえ、彼は財布を盗んだ
\\	のではありません
\\	。他の人が盗んだのです。」
\\	「彼は財布を盗んだのですか。」
\\	「いいえ、彼は財布を盗んだ
\\	ありません
\\	。他の人が盗んだのです。」
\\	中19
\\	"その本を買い
\\	はしなかったが
\\	、おもしろそうだったので、図書館で借りて読んだ。
\\	その本を買い
\\	しなかった_
\\	、おもしろそうだったので、図書館で借りて読んだ。
\\	中19
\\	"彼女はあまり多くのことを話し
\\	はしないが
\\	、話し方は上手だ。
\\	彼女はあまり多くのことを話し
\\	しない_
\\	、話し方は上手だ。
\\	中19
\\	"ケーキが大好きな
\\	洋子[ようこ]さんが、その店のこのケーキのことを知らない
\\	わけがない
\\	っこない
\\	ケーキが大好きな
\\	洋子[ようこ]さんが、その店のこのケーキのことを知らない
\\	が__
\\	中19
\\	"この問題はかなり難しい。彼女に
\\	解[と]け
\\	っこない
\\	よ。
\\	る
\\	わけがない
\\	この問題はかなり難しい。彼女に
\\	解[と]け
\\	っ___
\\	よ。
\\	中19
\\	"断水になると、料理のし
\\	ようがない
\\	断水になると、料理のし
\\	が__
\\	中19
\\	"毎日10
\\	歩いて学校に通っている彼はすごいと
\\	しか
\\	言い
\\	ようがない
\\	毎日10
\\	歩いて学校に通っている彼はすごいとしか言い
\\	が__
\\	中19
\\	"今日はパーティーの準備で忙しくて、美容院に行く
\\	どころではなかった
\\	今日はパーティーの準備で忙しくて、美容院に行く
\\	ではなかった
\\	中19
\\	"今回明らかになったのは問題全体の一部
\\	にすぎない
\\	今回明らかになったのは問題全体の一部
\\	に____
\\	中19
\\	"今回の大統領の発言は外国に誤解を与え
\\	兼ねない
\\	今回の大統領の発言は外国に誤解を与え
\\	(兼ねる/兼ねない)
\\	中19
\\	"ご依頼の件はお引き受けし
\\	兼ねます
\\	ご依頼の件はお引き受けし
\\	(兼ねます/兼ねません)
\\	中19
\\	"お酒を飲んで楽しくなって、歌を歌
\\	わずにはいられなかった
\\	お酒を飲んで楽しくなって、歌を歌
\\	わず__いられ____
\\	中19
\\	"のどが渇いて、水が飲みたく
\\	てしょうがない
\\	てしかたがない
\\	のどが渇いて、水が飲みたく
\\	て
\\	(しょう/しよう)
\\	がない
\\	中19
\\	"彼の言っていることに
\\	嘘[うそ]があるような気がし
\\	てならない
\\	彼の言っていることに
\\	嘘[うそ]があるような気がし
\\	て____
\\	中19
\\	"締切りまで時間がないので、とにかく今、分かっていることを論文に書
\\	くほかない
\\	締切りまで時間がないので、とにかく今、分かっていることを論文に書
\\	く__ない
\\	中19
\\	"今日は道路が込んでるわね。バス、時間通りに来る
\\	かしら
\\	今日は道路が込んでるわね。バス、時間通りに来る
\\	か__
\\	中19
\\	"あれ、財布がない。どこに置いたの
\\	かしら
\\	あれ、財布がない。どこに置いたの
\\	か__
\\	中19
\\	"彼は服装
\\	からして
\\	、学校の先生には見えない。 
\\	からいうと
\\	からすると
\\	からみると
\\	からみても
\\	彼は服装
\\	から__
\\	、学校の先生には見えない。
\\	中19
\\	"一面にひまわりの花が咲いていて、その部分が燃えている
\\	かのようだ
\\	一面にひまわりの花が咲いていて、その部分が燃えている
\\	の__だ
\\	中19
\\	"彼女の絵には見る人の心を強く動かす
\\	ものがある
\\	彼女の絵には見る人の心を強く動かす
\\	がある
\\	中19
\\	"今のライブスタイルを変えない限り、ゴミは増える
\\	一方だ
\\	今のライブスタイルを変えない限り、ゴミは増える
\\	だ
\\	中19
\\	"「山口さん、
\\	得意先
\\	からまだ帰ってこないの?遅いね。」
\\	「まだどこかでコーヒーでも飲んでる
\\	にきまって
\\	るよ。」
\\	「山口さん、
\\	得意先
\\	からまだ帰ってこないの?遅いね。」
\\	「まだどこかでコーヒーでも飲んでる
\\	に__って
\\	るよ。」
\\	中19
\\	"環境
\\	破壊は人間の
\\	身勝手[みがって]な行動の結果
\\	に相違ない
\\	環境
\\	破壊[はかい]は人間の
\\	身勝手[みがって]な行動の結果
\\	に__
\\	ない
\\	中19
\\	"料理上手になるためには、とにかく美味しいものを食べて味を覚え
\\	る
\\	ことだ
\\	料理上手になるためには、とにかく美味しいものを食べて味を覚え
\\	る
\\	だ
\\	中19
\\	"今日の試合に負けたからって、がっかりする
\\	ことはない
\\	よ。次で頑張ればいいんだから。
\\	今日の試合に負けたからって、がっかりする
\\	こと_ない
\\	よ。次で頑張ればいいんだから。
\\	中19
\\	"「手術の
\\	必要
\\	が
\\	あります
\\	か。」
\\	「いいえ。その
\\	必要
\\	は
\\	ありません
\\	。薬で
\\	治療[ちりょう]できます。」
\\	「手術の
\\	必要
\\	あります
\\	か。」
\\	「いいえ。その
\\	必要
\\	ありません
\\	。薬で
\\	治療できます。」
\\	中19
\\	"この本は高いので買う
\\	には及びません
\\	。必要なところをコピーしてください。
\\	この本は高いので買う
\\	には__ません
\\	。必要なところをコピーしてください。
\\	中19
\\	"お忙しいでしょうから、わざわざ来て頂く
\\	には及びません
\\	お忙しいでしょうから、わざわざ来て頂く
\\	には__ません
\\	中19
\\	"重田さんは夏休みに夫婦でアフリカでかけるらしい。うらやましい
\\	限りだ
\\	重田さんは夏休みに夫婦でアフリカでかけるらしい。うらやましい
\\	りだ
\\	中19
\\	"楽しみにしていた
\\	同窓会[どうそうかい]が地震の影響で中止になってしまった。残念な
\\	限りだ
\\	楽しみにしていた
\\	同窓会[どうそうかい]が地震の影響で中止になってしまった。残念な
\\	りだ
\\	中19
\\	"恋人と結婚式を挙げたときの感激
\\	といったらなかった
\\	恋人と結婚式を挙げたときの感激
\\	いった_なかった
\\	中19
\\	"大勢の人がいるところで転んでしまった。恥ずかしい
\\	といったらなかった
\\	大勢の人がいるところで転んでしまった。恥ずかしい
\\	と___らなかった
\\	中19
\\	"あなたと再会できる日を
\\	どんなに
\\	待った
\\	ことか
\\	あなたと再会できる日を
\\	どんなに
\\	待った
\\	か
\\	中19
\\	"漢字が書けるようになるまでに、
\\	何度
\\	練習した
\\	ことか
\\	漢字が書けるようになるまでに、
\\	何度
\\	練習した
\\	こと_
\\	中19
\\	"太郎、水道の水が出しっぱなし
\\	じゃないか
\\	。早く止めなさい。
\\	太郎、水道の水が出しっぱなし
\\	ない_
\\	。早く止めなさい。
\\	中19
\\	"電話がかかってきたとき、俺は眠り
\\	かけて
\\	いた。
\\	電話がかかってきたとき、俺は眠り
\\	いた。
\\	中19
\\	"机の上に食べ
\\	かけ
\\	のリンゴが置いてあった。
\\	机の上に食べ
\\	のリンゴが置いてあった。
\\	中19
\\	"池の氷が溶け
\\	つつある
\\	池の氷が溶け
\\	ある
\\	中19
\\	"仕事を引き受けたら、最後までやり
\\	ぬく
\\	ことが必要だ。
\\	仕事を引き受けたら、最後までやり
\\	ことが必要だ。
\\	中19
\\	"彼女は会社の
\\	不満[ふまん]を言い
\\	尽くして
\\	退職した。
\\	彼女は会社の
\\	不満[ふまん]を言い
\\	して
\\	退職した。
\\	中19
\\	彼は親が残してくれた800万円を半年で使い
\\	尽くして
\\	しまった。
\\	彼は親が残してくれた800万円を半年で使い
\\	して
\\	しまった。
\\	中19
\\	"今、旅行の準備をし
\\	ている最中
\\	だ。
\\	今、旅行の準備をし
\\	ている__
\\	だ。
\\	中19
\\	"晩ご飯を作っ
\\	ている最中
\\	に、彼女から電話がかかってきた。
\\	晩ご飯を作っ
\\	ている__
\\	に、彼女から電話がかかってきた。
\\	中19
\\	"今回のコンサートツアーは、名古屋で開催される
\\	の
\\	を
\\	皮切[かわき]りにして
\\	、全国20都市を回る予定です。
\\	今回のコンサートツアーは、名古屋で開催される
\\	の
\\	を___にして
\\	、全国20都市を回る予定です。
\\	上1
\\	"彼は、この小説がベストセラーになった
\\	の
\\	を
\\	皮切[かわき]りとして
\\	、次々と人気シリーズを生み出していた。
\\	彼は、この小説がベストセラーになった
\\	の
\\	を___として
\\	、次々と人気シリーズを生み出していた。
\\	上1
\\	"パンダの前足の形は大変珍しい
\\	とあって
\\	、遺伝学の研究対象として注目されている。(指は6本)
\\	パンダの前足の形は大変珍しい
\\	と___
\\	、遺伝学の研究対象として注目されている。
\\	上1
\\	"あのダ・ヴィンチのモナリザが見られる
\\	とあって
\\	、開館前から長い列ができたという。
\\	あのダ・ヴィンチのモナリザが見られる
\\	と___
\\	、開館前から長い列ができたという。
\\	上1
\\	"凍った湖の上でスケートは、北国
\\	ならではの
\\	楽しい遊びだ。
\\	凍った湖の上でスケートは、北国
\\	なら__の
\\	楽しい遊びだ。
\\	上1
\\	"大企業にはない、中小企業
\\	ならではの
\\	良さについて考える。
\\	大企業にはない、中小企業
\\	では_
\\	良さについて考える。
\\	上1
\\	"この山の桜の季節もいいが、それ
\\	にもまして
\\	紅葉の頃が美しい。
\\	この山の桜の季節もいいが、それ
\\	に__して
\\	紅葉の頃が美しい。
\\	上1
\\	"両親が私の言葉を信じてくれたことが、何
\\	にもまして
\\	ありがたかった。
\\	両親が私の言葉を信じてくれたことが、何
\\	に__して
\\	ありがたかった。
\\	上1
\\	"家を買うなら、床下
\\	から
\\	屋根
\\	に至るまで
\\	、専門家に細かくチェックしてもらったほうがいいですよ。
\\	家を買うなら、床下
\\	から
\\	屋根
\\	に__まで
\\	、専門家に細かくチェックしてもらったほうがいいですよ。
\\	上1
\\	"竹は、
\\	工芸品[こうげいひん]から衣類・食品
\\	に至るまで
\\	様々な製品に使われている。
\\	竹は、工芸品から衣類・食品
\\	に__まで
\\	様々な製品に使われている。
\\	上1
\\	"さすが元バレリーナ。立っている姿
\\	からして
\\	美しい。
\\	さすが元バレリーナ。立っている姿
\\	して
\\	美しい。
\\	上1
\\	"グルメの友人は、朝食のパン
\\	からして
\\	有名店のものを選ぶ。
\\	グルメの友人は、朝食のパン
\\	して
\\	有名店のものを選ぶ。
\\	上1
\\	"今年の国民生活時間調査によると、新聞を読んでいる40代の男性は41
\\	、30代は23
\\	、20代
\\	に
\\	至[いた]っては
\\	だった。
\\	今年の国民生活時間調査によると、新聞を読んでいる40代の男性は41
\\	、30代は23
\\	、20代
\\	に_っては
\\	だった。
\\	上1
\\	"彼は侍の家に生まれたが、剣もちょっと習った程度で、弓や
\\	槍[やり]
\\	に
\\	至[いた]っては
\\	触ったことさえないという男だった。
\\	彼は侍の家に生まれたが、剣もちょっと習った程度で、弓や
\\	槍[やり]
\\	に_っては
\\	触ったことさえないという男だった。
\\	上1
\\	"世界広し
\\	といえども
\\	源氏[げんじ]物語』のような
\\	壮大[そうだい]な恋愛小説他にないだろう。
\\	といっても 
\\	でも	
\\	世界広し
\\	とい__も
\\	源氏[げんじ]物語』のような
\\	壮大[そうだい]な恋愛小説他にないだろう。
\\	上1
\\	"山中村の住民は先月の洪水により
\\	未[いま]だに避難所で生活
\\	を
\\	余儀[よぎ]なくされている
\\	"山中村の住民は先月の洪水により
\\	未[いま]だに避難所で生活
\\	を__なく_れている
\\	上1
\\	"景気悪化の伴い、工場は
\\	閉鎖[へいさ]
\\	を
\\	余儀[よぎ]なくされた
\\	景気悪化の伴い、工場は
\\	閉鎖[へいさ]
\\	を__なく_れた
\\	上1
\\	"国を離れて10年経つが、1日
\\	たりとも
\\	故郷を忘れたことは
\\	ない
\\	国を離れて10年経つが、1日
\\	た__も
\\	故郷を忘れたことは
\\	上1
\\	"私たちの税金は1円
\\	たりとも
\\	無駄に使って欲しく
\\	ない
\\	私たちの税金は1円
\\	りと_
\\	無駄に使って欲しく
\\	上1
\\	"叔母[おば]は、周りの人にの気持ちを考えない
\\	きらいがあり
\\	、思ったことを何でも口にする。
\\	叔母[おば]は、周りの人にの気持ちを考えない
\\	があり
\\	、思ったことを何でも口にする。
\\	上1
\\	"大型電気店の値下げ競争は、消費者にとってはありがたいが、最近は少々行きすぎの
\\	きらいがある
\\	大型電気店の値下げ競争は、消費者にとってはありがたいが、最近は少々行きすぎの
\\	きらい___
\\	上1
\\	"国際交流エベントの成功ために、広く皆様にご協力をお願い申し上げる
\\	次第です
\\	んです 
\\	国際交流エベントの成功ために、広く皆様にご協力をお願い申し上げる
\\	です
\\	上1
\\	"重要議案は出席者の3分の2以上の賛成
\\	をもって
\\	承認されます。
\\	重要議案は出席者の3分の2以上の賛成
\\	を___
\\	承認されます。
\\	上1
\\	"「毒を
\\	もって
\\	毒を
\\	制[せい]す」というのは、悪を倒すために別の悪を利用することだ。
\\	「毒
\\	を___
\\	毒を
\\	制[せい]す」というのは、悪を倒すために別の悪を利用することだ。
\\	上1
\\	"「皆さん、泊まってくださるお客様
\\	あっての
\\	旅館だということを忘れずに、今日も1日笑顔で頑張りましょう。」
\\	「皆さん、泊まってくださるお客様
\\	あって_
\\	旅館だということを忘れずに、今日も1日笑顔で頑張りましょう。」
\\	上1
\\	"一流シェフは「良い材料
\\	あっての
\\	おいしい料理」とよく口にする。
\\	一流シェフは「良い材料
\\	の
\\	おいしい料理」とよく口にする。
\\	上1
\\	"便利な生活を求め
\\	てやまない
\\	人間の
\\	欲望[よくぼう]が、さまざまな
\\	矛盾[むじゅん]を生み出されている。
\\	便利な生活を求め
\\	て
\\	(やむ)
\\	人間の
\\	欲望[よくぼう]が、さまざまな
\\	矛盾を生み出されている。
\\	上1
\\	"今年1年が平和な年になることを願っ
\\	てやみません
\\	今年1年が平和な年になることを願っ
\\	て
\\	(やむ)
\\	上1
\\	"試合終了のホイッスルが響き、泥
\\	まみれ
\\	の選手たちは雨の中でお互いの
\\	健闘[けんとう]をたたえ合った。
\\	試合終了のホイッスルが響き、泥
\\	の選手たちは雨の中でお互いの
\\	健闘[けんとう]をたたえ合った。
\\	上1
\\	"小麦粉の入ったボウルをひっくり返した子猫は粉
\\	まみれ
\\	になってしまった。
\\	小麦粉の入ったボウルをひっくり返した子猫は粉
\\	になってしまった。
\\	上1
\\	"教師の
\\	再三[さいさん]の注意
\\	をよそに
\\	、学生は授業中も携帯電話をいじっている。
\\	教師の再三の注意
\\	を__に
\\	、学生は授業中も携帯電話をいじっている。
\\	上1
\\	"恋人が泣いて引きとめる
\\	の
\\	をよそに
\\	、彼はカメラを
\\	携[たずさ]えて戦場に向かった。
\\	"恋人が泣いて引きとめる
\\	の
\\	よそ_
\\	、彼はカメラを
\\	携[たずさ]えて戦場に向かった。
\\	上1
\\	"あの子は幼い
\\	なりに
\\	親を助けようと、いろいろ努力している。
\\	あの子は幼い
\\	に
\\	親を助けようと、いろいろ努力している。
\\	上1
\\	"お金がなければない
\\	なりに
\\	、楽しみ方はあるものだ。
\\	お金がなければない
\\	に
\\	、楽しみ方はあるものだ。
\\	上1
\\	"彼が犯人だという証拠は
\\	ないでもない
\\	が、まだ
\\	断定[だんてい]はできない。
\\	彼が犯人だという証拠は
\\	ない__ない
\\	が、まだ
\\	断定はできない。
\\	上1
\\	"「是非に言われれば飲ま
\\	ないでもない
\\	んですが、最近酒は控えているんです。」
\\	「じゃあ、まあ、少しだけ……。」
\\	「是非に言われれば飲ま
\\	でも__
\\	んですが、最近酒は控えているんです。」
\\	「じゃあ、まあ、少しだけ……。」
\\	上1
\\	"「面倒くさい」が口癖の母は、歩いて5分のスーパーに行くのにも車を使い運動不足で医者の
\\	叱[しか]られる
\\	しまつ
\\	だ。
\\	「面倒くさい」が口癖の母は、歩いて5分のスーパーに行くのにも車を使い運動不足で医者の
\\	叱[しか]られる
\\	だ。
\\	上1
\\	"あの女の子は子供の頃から
\\	嘘[うそ]ばかりついていて、最後には詐欺で捕まる
\\	しまつ
\\	だ。
\\	あの女の子は子供の頃から
\\	嘘[うそ]ばかりついていて、最後には詐欺で捕まる
\\	だ。
\\	上1
\\	"目が悪いなら、眼鏡
\\	なり
\\	コンタクト
\\	なり
\\	すればいいのに。
\\	目が悪いなら、眼鏡
\\	コンタクト
\\	すればいいのに。
\\	上1
\\	"今日は俺のおごりだから、牛丼
\\	なり
\\	カレー
\\	なり
\\	、好きに注文していいよ。
\\	今日は俺のおごりだから、牛丼
\\	カレー
\\	、好きに注文していいよ。
\\	上1
\\	"ご主人の昇進や娘さんの結婚など、山田さんの家は
\\	いい
\\	ことずくめ
\\	だ。
\\	ご主人の昇進や娘さんの結婚など、山田さんの家は
\\	いい
\\	ことず__
\\	だ。
\\	上1
\\	"僕の学校は
\\	規則
\\	ずくめ
\\	で、息がつまりそうだ。
\\	僕の学校は
\\	規則
\\	ず__
\\	で、息がつまりそうだ。
\\	上1
\\	"彼女のデザインする服は、どれも個性的
\\	にして
\\	実用的だと
\\	定評[ていひょう]がある。
\\	彼女のデザインする服は、どれも個性的
\\	に__
\\	実用的だと
\\	定評[ていひょう]がある。
\\	上1
\\	"怪盗[かいとう]と呼ばれた男の犯行の
\\	手口[てぐち]は
\\	大胆[だいたん]
\\	にして
\\	、かつ
\\	繊細[せんさい]だった。
\\	怪盗[かいとう]と呼ばれた男の犯行の
\\	手口[てぐち]は
\\	大胆
\\	に__
\\	、かつ
\\	繊細[せんさい]だった。
\\	上1
\\	"彼は
\\	真夏[まなつ]でも、エアコン
\\	はおろか
\\	、扇風機
\\	さえ
\\	使
\\	わない
\\	そうだ。
\\	彼は
\\	真夏[まなつ]でも、エアコン
\\	は__か
\\	、扇風機
\\	さえ
\\	使
\\	そうだ。
\\	上1
\\	"重い物を持とうとして腰を痛め、起き上がること
\\	はおろか
\\	寝返り[ねがえり]を打つこと
\\	も
\\	でき
\\	ない
\\	状態だった。
\\	重い物を持とうとして腰を痛め、起き上がること
\\	は__か
\\	寝返り[ねがえり]を打つこと
\\	も
\\	でき
\\	状態だった。
\\	上1
\\	"本に夢中になって、夜が明けたことに
\\	すら
\\	気づかなかった。
\\	さえ, も	
\\	本に夢中になって、夜が明けたことに
\\	気づかなかった。
\\	上1
\\	"スピーチに
\\	慣[な]れた人で
\\	すら
\\	、話し始めはどきどきするそうだ。
\\	さえ, も	
\\	スピーチに
\\	慣れた人で
\\	、話し始めはどきどきするそうだ。
\\	上1
\\	"お世話になった先輩のお宅へ、お礼
\\	かたがた
\\	新年のご挨拶に伺った。
\\	お世話になった先輩のお宅へ、お礼
\\	か___
\\	新年のご挨拶に伺った。
\\	上1
\\	"上司が入院したので、お見舞い
\\	かたがた
\\	仕事の進め方について相談に行った。
\\	上司が入院したので、お見舞い
\\	か___
\\	仕事の進め方について相談に行った。
\\	上1
\\	"あの優しそうな老人が、強盗事件の犯人だった
\\	とは
\\	あの優しそうな老人が、強盗事件の犯人だった
\\	と_
\\	上1
\\	"有名なコーヒーショップだと聞いてはいたが、コーヒー1杯2,000円
\\	とは
\\	驚いたよ。
\\	有名なコーヒーショップだと聞いてはいたが、コーヒー1杯2,000円
\\	と_
\\	驚いたよ。
\\	上1
\\	"彼女は立ち上がる
\\	なり
\\	、コップのチャンペンを彼の顔にかけた。
\\	が早いか
\\	彼女は立ち上がる
\\	、コップのチャンペンを彼の顔にかけた。
\\	上1
\\	"王子はシンデレラを一目見る
\\	なり
\\	、恋に落ちてしまった。
\\	が早いか
\\	王子はシンデレラを一目見る
\\	、恋に落ちてしまった。
\\	上1
\\	"料理はうまい
\\	わ
\\	、眺めはいい
\\	わ
\\	、あの旅館は最高だったね。
\\	料理はうまい
\\	、眺めはいい
\\	、あの旅館は最高だったね。
\\	上1
\\	"最近、酒井くん、どうしたんだろう。遅刻はする
\\	わ
\\	、宿題は忘れる
\\	わ
\\	最近、酒井くん、どうしたんだろう。遅刻はする
\\	、宿題は忘れる
\\	上1
\\	"店員がドアを開ける
\\	が早いか
\\	、待っていた客がなだれ込んできた。
\\	なり
\\	店員がドアを開ける
\\	が___
\\	、待っていた客がなだれ込んできた。
\\	上1
\\	"魔法使いが
\\	呪文[じゅもん]を
\\	唱[とな]える
\\	が早いか
\\	、王子はたちまちカエルに変わった。
\\	なり
\\	魔法使いが
\\	呪文[じゅもん]を
\\	唱[とな]える
\\	が___
\\	、王子はたちまちカエルに変わった。
\\	上1
\\	"父は肉が焼ける
\\	そばから
\\	私たちの皿にどんどん載せてくれた。
\\	父は肉が焼ける
\\	から
\\	私たちの皿にどんどん載せてくれた。
\\	上1
\\	"ゴミを片付けた
\\	そばから
\\	カラスが散らかすので、道が汚れて困る。
\\	ゴミを片付けた
\\	から
\\	カラスが散らかすので、道が汚れて困る。
\\	上1
\\	"「今渡こそ犯人を捕まえ
\\	ずにはおかない
\\	ぞ」と
\\	警部[けいぶ]は心に誓った。
\\	「今渡こそ犯人を捕まえ
\\	ず__おか__
\\	ぞ」と
\\	警部[けいぶ]は心に誓った。
\\	上1
\\	"当時の大統領は自分の意見に反対する者を排除せ
\\	ずにはおかない
\\	人間だった。
\\	当時の大統領は自分の意見に反対する者を排除せ
\\	ず__おか__
\\	人間だった。
\\	上1
\\	"この本は読み始め
\\	たが最後
\\	、徹夜してでも一気に終わりまで読みたくなる面白さだ。
\\	この本は読み始め
\\	た_最後
\\	、徹夜してでも一気に終わりまで読みたくなる面白さだ。
\\	上1
\\	"彼はアトリエにこまっ
\\	たが最後
\\	寝食[しんしょく]を忘れてしまうので家族は心配している。
\\	彼はアトリエにこまっ
\\	が最後
\\	寝食[しんしょく]を忘れてしまうので家族は心配している。
\\	上1
\\	"「あんな奴、殴っ
\\	てやる
\\	「あんな奴、殴っ
\\	てや_
\\	上1
\\	"「もう我慢できない。こんな会社辞め
\\	てやる
\\	「もう我慢できない。こんな会社辞め
\\	て_る
\\	上1
\\	"そこは一見目立たない店だが、昼休み
\\	ともなると
\\	大勢の客が列を作る有名ラーメン店だ。
\\	そこは一見目立たない店だが、昼休み
\\	なると
\\	大勢の客が列を作る有名ラーメン店だ。
\\	上1
\\	"普段はスポーツに関心のない人も、オリンピック
\\	ともなれば
\\	夢中にかじりつく。
\\	普段はスポーツに関心のない人も、オリンピック
\\	なれば
\\	夢中にかじりつく。
\\	上1
\\	"父親の死後、2人の兄が家や土地をもらったの
\\	にひかえ
\\	、末の弟に残されたのはなぜか一冊の古いノートだけだった。
\\	父親の死後、2人の兄が家や土地をもらったの
\\	ひかえ
\\	、末の弟に残されたのはなぜか一冊の古いノートだけだった。
\\	上1
\\	"あの映画は
\\	前作[ぜんさく]の観客が250万人を越えたの
\\	にひかえ
\\	続編[ぞくへん]は100万人にも届かなかったそうだ。
\\	あの映画は
\\	前作[ぜんさく]の観客が250万人を越えたの
\\	ひかえ
\\	続編[ぞくへん]は100万人にも届かなかったそうだ。
\\	上1
\\	"少子化[しょうしか]の原因は、
\\	晩婚化[ばんこんか]
\\	もさることながら
\\	、教育費の
\\	負担[ふたん]の大きさにもある。
\\	少子化[しょうしか]の原因は、
\\	晩婚化[ばんこんか]
\\	もさること___
\\	、教育費の
\\	負担[ふたん]の大きさにもある。
\\	上1
\\	"学生
\\	街[がい]のレストランは味
\\	もさることながら
\\	、量が多いことが第一条件だと言われる。
\\	学生
\\	街[がい]のレストランは味
\\	も__ことながら
\\	、量が多いことが第一条件だと言われる。
\\	上1
\\	"2倍
\\	であれ
\\	3倍
\\	であれ
\\	、優勝できなければ意味がない。
\\	2倍
\\	で__
\\	3倍
\\	で__
\\	、優勝できなければ意味がない。
\\	上1
\\	"犬
\\	であれ
\\	猫
\\	であれ
\\	、このマンションでペットを飼うことは禁止されています。
\\	犬
\\	で__
\\	猫
\\	で__
\\	、このマンションでペットを飼うことは禁止されています。
\\	上1
\\	"地形や営業構造など
\\	地域[ちいき]の実状
\\	に
\\	即[そく]して
\\	、災害対策を急がなくてはならない。
\\	地形や営業構造など地域の実状
\\	に__して
\\	、災害対策を急がなくてはならない。
\\	上1
\\	"弊社は、時代
\\	に
\\	即[そく]した
\\	営業によって、常に業界をリードして参りました。
\\	弊社は、時代
\\	に_した
\\	営業によって、常に業界をリードして参りました。
\\	上1
\\	"有利な条件で転職するなら、景気が
\\	好天[こうてん]している今
\\	をおいて他にない
\\	有利な条件で転職するなら、景気が
\\	好天[こうてん]している今
\\	を___他にない
\\	上1
\\	"地球の
\\	生態系[せいたいけい]を
\\	保全[ほぜん]し、環境を守ることができるのは、人類
\\	をおいて他にない
\\	地球の
\\	生態系[せいたいけい]を
\\	保全[ほぜん]し、環境を守ることができるのは、人類
\\	を___他にない
\\	上1
\\	"環境保護の必要性は、改めて世論に問う
\\	までもない
\\	ことだ。
\\	環境保護の必要性は、改めて世論に問う
\\	まで_ない
\\	ことだ。
\\	上1
\\	"彼は何を言いたいかは聞く
\\	までもない
\\	よ。
\\	彼は何を言いたいかは聞く
\\	もない
\\	よ。
\\	上1
\\	"明日のロケット打ち上げは、天候
\\	いかん
\\	で
\\	延期になる可能性が出てきました。
\\	明日のロケット打ち上げは、天候
\\	で
\\	延期になる可能性が出てきました。
\\	上1
\\	"今回のツアーは政治情勢
\\	いかん
\\	によっては
\\	、中止になる場合もあります。
\\	今回のツアーは政治情勢
\\	によっては
\\	、中止になる場合もあります。
\\	上1
\\	"時給1,000円はもらえ
\\	ないまでも
\\	、850円はもらえたい。
\\	時給1,000円はもらえ
\\	ない__も
\\	、850円はもらえたい。
\\	上1
\\	"始めてテントに寝たが、
\\	快適[かいてき]とは言え
\\	ないまでも
\\	、思ったほど悪くなかった。
\\	始めてテントに寝たが、快適とは言え
\\	ない__も
\\	、思ったほど悪くなかった。
\\	上1
\\	"温泉
\\	といい
\\	、スキー場
\\	といい
\\	、冬の北海道は観光客にとって、魅力がいっぱいです。
\\	温泉
\\	と__
\\	、スキー場
\\	と__
\\	、冬の北海道は観光客にとって、魅力がいっぱいです。
\\	上1
\\	"このホテルは全く期待はずれだった。サービス
\\	といい
\\	、料理
\\	といい
\\	、ひどいものだ。
\\	このホテルは全く期待はずれだった。サービス
\\	と__
\\	、料理
\\	と__
\\	、ひどいものだ。
\\	上1
\\	"子供に人気の料理と言えば、カレーやハンバーグ
\\	といったところ
\\	でしょう。
\\	子供に人気の料理と言えば、カレーやハンバーグ
\\	といった___
\\	でしょう。
\\	上1
\\	"私が生まれたのは歴史のある古い街で、日本なら奈良や京都
\\	といったところ
\\	です。
\\	私が生まれたのは歴史のある古い街で、日本なら奈良や京都
\\	といった___
\\	です。
\\	上1
\\	"可能
\\	であった
\\	可能でした			上1
\\	"条例
\\	が
\\	制定
\\	される
\\	条例を制定します			上1
\\	"パソコン教育
\\	が行われる
\\	パソコン教育を行います			上1
\\	"忘れては
\\	なるまい
\\	ならない
\\	忘れてはいけません・なりません			上1
\\	欲しい
\\	もらいたい
\\	されたい
\\	欲しいです・ください			上1
\\	あり
\\	おり
\\	し
\\	、~減
\\	り
\\	、~超え
\\	ており
\\	、~少
\\	なく
\\	ず
\\	おらず
\\	せず
\\	~あって、~いって、~して、~減って、~超えていて、~少なくて、
\\	~ないで、~いないで、~しないで
\\	上1
\\	"会議は実施されて
\\	おらず
\\	会議は実施されていないで、~			上1
\\	した
\\	した
\\	試験
\\	をする
\\	、議論
\\	をするなどした
\\	試験
\\	議論
\\	などを行った
\\	したのです
\\	してしまいました
\\	試験したり、議論したりした
\\	上1
\\	"このお弁当、量が少ない
\\	といおうか
\\	、味が薄い
\\	といおうか
\\	、とにかく物足りないんだよ。
\\	というべきか
\\	というか
\\	このお弁当、量が少ない
\\	と___か
\\	、味が薄い
\\	と___か
\\	、とにかく物足りないんだよ。
\\	上1
\\	"あの店員、5万円の
\\	化粧水[けしょうすい]を勧めるなんて、
\\	庶民[しょみん]感覚を知らない
\\	といおうか
\\	というべきか
\\	あの店員、5万円の
\\	化粧水[けしょうすい]を勧めるなんて、庶民感覚を知らない
\\	と___か
\\	上1
\\	"酒井くんは電車が送れた
\\	のをいいことに
\\	、テストが終わる頃、
\\	堂々[どうどう]とやってきた。
\\	酒井くんは電車が送れた
\\	をいい__に
\\	、テストが終わる頃、
\\	堂々[どうどう]とやってきた。
\\	上1
\\	"どうやらうちの猫、飼い主の留守な
\\	のをいいことに
\\	いたずらをしたらしく、部屋がめちゃくちゃだ。
\\	どうやらうちの猫、飼い主の留守な
\\	をいい__に
\\	いたずらをしたらしく、部屋がめちゃくちゃだ。
\\	上1
\\	"こんなくだらない番組、3時間も見ていられる
\\	か
\\	こんなくだらない番組、3時間も見ていられる
\\	上1
\\	"私は医者なんです。薬がないからといって、患者を放っておけます
\\	か
\\	私は医者なんです。薬がないからといって、患者を放っておけます
\\	上1
\\	"「あの人、友達の結婚式に白いドレス着て行ったんだって。」
\\	「え?!物を知らない
\\	にもほどがある
\\	よね。」
\\	「あの人、友達の結婚式に白いドレス着て行ったんだって。」
\\	「え?!物を知らない
\\	に_ほど_ある
\\	よね。」
\\	上1
\\	"中身を減らして、2割引きと言って売るなんて、客をバカにする
\\	にもほどがある
\\	中身を減らして、2割引きと言って売るなんて、客をバカにする
\\	も__がある
\\	上1
\\	"スニーカー
\\	ならまだしも
\\	、サンダルやハイヒールで登山なんて
\\	無茶[むちゃ]だ。
\\	スニーカー
\\	なら__しも
\\	、サンダルやハイヒールで登山なんて
\\	無茶[むちゃ]だ。
\\	上1
\\	"電気代が上がった。それだけ
\\	ならまだしも
\\	、4月から家賃を2,000円上げると言われて、困っている。
\\	電気代が上がった。それだけ
\\	ならまだ__
\\	、4月から家賃を2,000円上げると言われて、困っている。
\\	上1
\\	"きちんと挨拶するのはビジネスマナー
\\	以前
\\	の常識だと思いますよ。
\\	きちんと挨拶するのはビジネスマナー
\\	以_
\\	の常識だと思いますよ。
\\	上1
\\	"「このキノコ、きれいね。おいしいかな。」
\\	「ちょっと待ってよ。おいしいおいしくない
\\	以前
\\	に、食べても大丈夫なの?」
\\	「このキノコ、きれいね。おいしいかな。」
\\	「ちょっと待ってよ。おいしいおいしくない
\\	前
\\	に、食べても大丈夫なの?」
\\	上1
\\	"この車は
\\	燃費[ねんぴ]も良く、
\\	洗練[せんれん]されたデザインで、
\\	売れる
\\	べくして
\\	売れた
\\	と言える。
\\	この車は
\\	燃費[ねんぴ]も良く、
\\	洗練[せんれん]されたデザインで、
\\	売れる
\\	して
\\	売れた
\\	と言える。
\\	上1
\\	"指摘された欠陥を
\\	放置[ほうち]していたのだから、これは
\\	起こる
\\	べくして
\\	起こった
\\	事故だ。
\\	指摘された欠陥を
\\	放置[ほうち]していたのだから、これは
\\	起こる
\\	して
\\	起こった
\\	事故だ。
\\	上1
\\	"これは
\\	国宝級[こくほうきゅう]の仏像なのです。
\\	物
\\	が
\\	物な
\\	だけに
\\	、普通の運送業者には頼りません。
\\	これは
\\	国宝級[こくほうきゅう]の仏像なのです。
\\	物
\\	が
\\	物な
\\	に
\\	、普通の運送業者には頼りません。
\\	上1
\\	"着ている
\\	服
\\	が
\\	服な
\\	だけに
\\	、こんな高級レストランに入りづらい。
\\	着ている
\\	服
\\	が
\\	服な
\\	に
\\	、こんな高級レストランに入りづらい。
\\	上1
\\	"キティちゃん好きの彼女は服
\\	といわず
\\	、文具
\\	といわず
\\	、全部キティちゃんグッズで統一している。
\\	キティちゃん好きの彼女は服
\\	といわ_
\\	、文具
\\	といわ_
\\	、全部キティちゃんグッズで統一している。
\\	上1
\\	"彼の部屋は、床
\\	といわず
\\	ベッド
\\	といわず
\\	、いろいろなものが散乱しています。
\\	彼の部屋は、床
\\	といわ_
\\	ベッド
\\	といわ_
\\	、いろいろなものが散乱しています。
\\	上1
\\	"家賃が高いのも困るけど、
\\	安かっ
\\	たら
\\	安かっ
\\	たで
\\	何か問題がありそうで不安だよね。
\\	家賃が高いのも困るけど、
\\	安かっ
\\	た_
\\	安かっ
\\	た_
\\	何か問題がありそうで不安だよね。
\\	上1
\\	"部屋にほこりがたまれば文句を言うし、掃除を
\\	し
\\	たら
\\	し
\\	たで
\\	、「勝手に入った」と怒るし、全く高校生の息子は扱いにくい。
\\	部屋にほこりがたまれば文句を言うし、掃除を
\\	し
\\	た_
\\	し
\\	た_
\\	、「勝手に入った」と怒るし、全く高校生の息子は扱いにくい。
\\	上1
\\	筆者	
\\	私・僕			上1
\\	本稿[ほんこう]
\\	小稿・
\\	拙稿[せっこう]
\\	私のレポート
\\	私の論文			上1
\\	述べる 
\\	論述[ろんじゅつ]する	
\\	書く・言う
\\	上1
\\	前述[ぜんじゅつ]した・
\\	上述[じょうじゅつ]した
\\	後述[こうじゅつ]する
\\	前に述べた・言った
\\	後で述べる・言う			上1
\\	"佐藤は~と述べた。 敬語
\\	佐藤先生のアドバイスに深く感謝いたします。 敬語
\\	佐藤先生は~とおっしゃいました
\\	佐藤先生のアドバイスに深く感謝いたします。
\\	上1
\\	管見[かんけん](では)	
\\	私の知っている範囲			上1
\\	当該[とうがい]
\\	機関で
\\	その・この・あの機関で			上1
\\	"日本
\\	あるいは
\\	もしくは
\\	または
\\	米国
\\	日本
\\	か
\\	米国			上1
\\	"韓国
\\	および
\\	インド
\\	韓国
\\	と
\\	インド			上1
\\	2011年2月
\\	北海道富良野
\\	市/町
\\	気温は-6°
\\	である
\\	昨年の2月
\\	この町
\\	寒い
\\	上1
\\	提示しなくてはならない
\\	惑星ではない
\\	その状態を
\\	記[しる]しておく
\\	提示しなくちゃならない
\\	惑星じゃない
\\	その状態を
\\	記[しる]しとく
\\	上1
\\	それゆえ・したがって	
\\	それで・だから			上1
\\	しかし・しかしながら	
\\	でも・だけど			上1
\\	なお・ただし・また	
\\	あと			上1
\\	非常に・大変	
\\	とても・すごく			上1
\\	多数・数多く・多く	
\\	いっぱい・たくさん			上1
\\	少し・
\\	若干[じゃっかん]・多少	
\\	ちょっと			上1
\\	大学のような機関
\\	このようなデータ
\\	大学みたいな機関
\\	こんなデータ
\\	上1
\\	討論などで	
\\	討論[とうろん]なんかで			上1
\\	どちら・いずれの調査	
\\	どっちの調査			上1
\\	"言い訳ばかりしている政治家の話は聞く
\\	にたえない
\\	言い訳ばかりしている政治家の話は聞く
\\	に__ない
\\	上1
\\	"一目も構わず電車の中で化粧している女性の姿は見る
\\	にたえない
\\	と祖母は嘆く。
\\	一目も構わず電車の中で化粧している女性の姿は見る
\\	に_えない
\\	と祖母は嘆く。
\\	上1
\\	"帰宅途中で彼女に捕ま
\\	ろうものなら
\\	、1時間はおしゃべりに付き合わせる。
\\	帰宅途中で彼女に捕ま
\\	ものなら
\\	、1時間はおしゃべりに付き合わせる。
\\	上1
\\	"このまま不景気が続
\\	こうものなら
\\	、我々のような
\\	零細[れいさい]企業は
\\	軒並[のきな]みつぶれてしまう。
\\	このまま不景気が続
\\	こう__なら
\\	、我々のような
\\	零細[れいさい]企業は
\\	軒並[のきな]みつぶれてしまう。
\\	上1
\\	"渡辺先生のような英語が上手
\\	ならいざしらず
\\	、僕に会議の通訳なんて無理ですよ。
\\	渡辺先生のような英語が上手
\\	ならい_しらず
\\	、僕に会議の通訳なんて無理ですよ。
\\	上1
\\	"学生時代
\\	ならいざしらず
\\	、君ももう社会人になったのだから、少しは大人としての自覚を持つべきじゃないか。
\\	学生時代
\\	ならいざ_らず
\\	、君ももう社会人になったのだから、少しは大人としての自覚を持つべきじゃないか。
\\	上1
\\	"学校に行けない子どもたちために、何か支援はでき
\\	ないもの
\\	(だろう)
\\	か
\\	学校に行けない子どもたちために、何か支援はでき
\\	ないもの_
\\	上1
\\	"お忙しいことは存じておりますが、一度を話だけでも聞いて頂け
\\	ない
\\	もの
\\	でしょう
\\	か
\\	お忙しいことは存じておりますが、一度を話だけでも聞いて頂け
\\	ない
\\	もの
\\	でしょう_
\\	上1
\\	"癌[がん]に限らず、病気の発見は早い
\\	に越したことはない
\\	癌[がん]に限らず、病気の発見は早い
\\	に_したことはない
\\	上1
\\	"家を借りるなら車庫
\\	に越したことはない
\\	けど、近くに駐車場があれば、我慢するよ。
\\	家を借りるなら車庫
\\	に_したことはない
\\	けど、近くに駐車場があれば、我慢するよ。
\\	上1
\\	"ペット可のマンション
\\	とはいえ
\\	、どんな動物でも飼えるわけではない。
\\	ペット可のマンション
\\	とは__
\\	、どんな動物でも飼えるわけではない。
\\	上1
\\	"いくら虫が苦手だ
\\	とはいえ
\\	、そんなに
\\	殺虫剤[さっちゅうざい]をまくと体に悪いよ。
\\	いくら虫が苦手だ
\\	はいえ
\\	、そんなに
\\	殺虫剤[さっちゅうざい]をまくと体に悪いよ。
\\	上1
\\	"時給が上がるといっても期待しないほうがいいよ。せいぜい50円
\\	といったところだ
\\	よ。
\\	時給が上がるといっても期待しないほうがいいよ。せいぜい50円
\\	といった___だ
\\	よ。
\\	上1
\\	"「ご趣味はピアノだと伺いましたが……。」
\\	「いやいや、やっと右手でドレミが弾ける
\\	といったところです
\\	よ。」
\\	「ご趣味はピアノだと伺いましたが……。」
\\	「いやいや、やっと右手でドレミが弾ける
\\	といっ_ところです
\\	よ。」
\\	上1
\\	"人間だれでも
\\	言う
\\	に
\\	言え
\\	ない
\\	悩みがあるものだ。
\\	人間だれでも
\\	言う
\\	言え
\\	悩みがあるものだ。
\\	上1
\\	"娘が私に寄りかかって寝てしまったので、
\\	動く
\\	に
\\	動け
\\	ない
\\	肩が凝ってしまった。
\\	娘が私に寄りかかって寝てしまったので、
\\	動く
\\	動け
\\	肩がこってしまった。
\\	上1
\\	"彼は、地方出張
\\	にかこつけて
\\	、どうやら恋人に会いに行っているらしい。
\\	彼は、地方出張
\\	に__つけて
\\	、どうやら恋人に会いに行っているらしい。
\\	上1
\\	"要するに、雪と桜とか季節の何か
\\	にかこつけて
\\	、集まって騒ぎたいんだろう。
\\	要するに、雪と桜とか季節の何か
\\	にかこ__て
\\	、集まって騒ぎたいんだろう。
\\	上1
\\	"お暑い
\\	ところ
\\	申し訳ございませんが、
\\	節電[せつでん]のためエアコンの温度は28度に設定させて頂いております。
\\	お暑い
\\	申し訳ございませんが、
\\	節電[せつでん]のためエアコンの温度は28度に設定させて頂いております。
\\	上1
\\	"このような権威ある賞をいただきまして、誠に
\\	光栄[こうえい]
\\	の至り
\\	でございます。
\\	このような権威ある
\\	賞をいただきまして、誠に
\\	光栄[こうえい]
\\	の_り
\\	でございます。
\\	上1
\\	"就任パーティーの
\\	席上[せきじょう]で新社長の名前を間違えるとは、思い返しても
\\	赤面[せきめん]
\\	の至り
\\	だ。
\\	就任パーティーの
\\	席上[せきじょう]で新社長の名前を間違えるとは、思い返しても
\\	赤面[せきめん]
\\	至り
\\	だ。
\\	上1
\\	"鈴木二郎殿。4月1日
\\	をもって
\\	、第2営業部勤務を命じる。
\\	をもちまして
\\	鈴木二郎殿。4月1日
\\	を__て
\\	、第2営業部勤務を命じる
\\	上1
\\	"第22回卒業式は、これ
\\	をもちまして
\\	、終了いたします。
\\	をもって
\\	第22回卒業式は、これ
\\	を____て
\\	、終了いたします。
\\	上1
\\	"新人
\\	の
\\	こととて
\\	失礼があったらを許しください。
\\	~なので	
\\	新人
\\	の
\\	こ__て
\\	失礼があったらを許しください。
\\	上1
\\	"何分にも田舎
\\	の
\\	こととて
\\	山菜[さんさい]料理しかありませんが、どうぞゆっくりしていってください。
\\	~なので	
\\	何分にも田舎
\\	の
\\	こ__て
\\	山菜[さんさい]料理しかありませんが、どうぞゆっくりしていってください。
\\	上1
\\	"温かくご指導いただきまして
\\	感謝の
\\	念[ねん]
\\	にたえません
\\	温かくご指導いただきまして
\\	感謝の
\\	念[ねん]
\\	に_えません
\\	上1
\\	"念願[ねんがん]の世界大会優勝がかない、本当に
\\	喜び
\\	にたえません
\\	念願[ねんがん]の世界大会優勝がかない、本当に
\\	喜び
\\	に_えません
\\	上1
\\	"プロであれアマチュアであれ、スポーツ選手
\\	たる者
\\	、正々堂々と戦うことを常に忘れてはならない。
\\	プロであれアマチュアであれ、スポーツ選手
\\	者
\\	、正々堂々と戦うことを常に忘れてはならない。
\\	上1
\\	"企業の営業者
\\	たる者
\\	は、働く人たちが能力を
\\	発揮[はっき]できるよう環境を整えるべきである。
\\	企業の営業者
\\	者
\\	は、働く人たちが能力を
\\	発揮できるよう環境を整えるべきである。
\\	上1
\\	"たった2点足りなかったばかりに不合格だなんて、悔しい
\\	限りだ
\\	たった2点足りなかったばかりに不合格だなんて、悔しい
\\	りだ
\\	上1
\\	"貴重な本を特別に貸してやったのに、汚されて、
\\	腹立[はらた]たし
\\	限りだ
\\	貴重な本を特別に貸してやったのに、汚されて、
\\	腹立[はらた]たし
\\	りだ
\\	上1
\\	"厳しいと言っても、社長も鬼
\\	ではあるまいし
\\	、ちゃんと話せばわかってもらえるはずだ。
\\	"厳しいと言っても、社長も鬼
\\	ではある__し
\\	、ちゃんと話せばわかってもらえるはずだ。
\\	上1
\\	"「上村さん、コピー、終わりました。次は何をしたらいいですか。」
\\	「新入社員
\\	じゃあるまいし
\\	、少しは自分で考えるよ。」
\\	"「上村さん、コピー、終わりました。次は何をしたらいいですか。」
\\	「新入社員
\\	じゃ__まい_
\\	、少しは自分で考えるよ。」
\\	上1
\\	"店員は、「早く帰れ」と言
\\	わ
\\	んばかり
\\	に空いた皿を片付けはじめた。
\\	"店員は、「早く帰れ」と言
\\	わ
\\	ばかり
\\	に空いた皿を片付けはじめた。
\\	上1
\\	"彼女は、今にも泣き出
\\	さ
\\	んばかり
\\	の顔をして、部屋を飛び出していった。
\\	"彼女は、今にも
\\	(泣き出す)
\\	んばかり
\\	の顔をして、部屋を飛び出していった。
\\	上1
\\	"新型の医療機器をそろえても、使いこなせる技術者がいな
\\	ければそれまでだ
\\	いなかっ
\\	たらそれまでだ
\\	"新型の医療機器をそろえても、使いこなせる技術者がいな
\\	それ__だ
\\	上2
\\	"あきらめ
\\	たらそれまでだ
\\	。努力し続ければ夢は必ずかなう。
\\	あきらめ
\\	ればそれまでだ
\\	"あきらめ
\\	たら__までだ
\\	。努力し続ければ夢は必ずかなう。
\\	上2
\\	"もっと早く健康診断を受けていれば手術をしないですんだ
\\	ものを
\\	"もっと早く健康診断を受けていれば手術をしないですんだ
\\	もの_
\\	上2
\\	"散歩は朝の涼しいときにすればいい
\\	ものを
\\	、暑い盛りに出て行くから、
\\	熱中症[ねっちゅうしょう]なんかになるんだよ。
\\	"散歩は朝の涼しいときにすればいい
\\	もの_
\\	、暑い盛りに出て行くから、
\\	熱中症[ねっちゅうしょう]なんかになるんだよ。
\\	上2
\\	"うちの子
\\	ときたら
\\	、いつもゲームばかりやっていて、声をかけても返事もしない。
\\	"うちの子
\\	と__ら
\\	、いつもゲームばかりやっていて、声をかけても返事もしない。
\\	上2
\\	"最近の若い人
\\	ときたら
\\	、電車の中で床に座り込んだりして、恥ずかしくないのかしら。
\\	"最近の若い人
\\	きた_
\\	、電車の中で床に座り込んだりして、恥ずかしくないのかしら。
\\	上2
\\	"今から急い
\\	だところで
\\	、間に合うわけがないよ。
\\	"今から
\\	(急ぐ)
\\	ところで
\\	、間に合うわけがないよ。
\\	上2
\\	"私が言っ
\\	たところで
\\	、彼の気持ちは変わらないだろう。
\\	"私が言っ
\\	た___で
\\	、彼の気持ちは変わらないだろう。
\\	上2
\\	どんなに
\\	ひどいけがをし
\\	よう
\\	が[と]
\\	、アイスホッケーはやめられない。
\\	でも 
\\	どんなに
\\	ひどいけがを
\\	(する)
\\	が
\\	、アイスホッケーはやめられない。
\\	上2
\\	"お前が
\\	どこへ行
\\	こう
\\	と[が]
\\	、俺の知ったことか。勝手にしろ!
\\	でも 
\\	"お前が
\\	どこへ行
\\	こう_
\\	、俺の知ったことか。勝手にしろ!
\\	上2
\\	"お客が
\\	来
\\	よう
\\	と[が]
\\	来
\\	るまい
\\	と[が]
\\	、部屋はいつも片付けておけ。
\\	"お客が
\\	来
\\	よう_
\\	来
\\	るまい_
\\	、部屋はいつも片付けておけ。
\\	上2
\\	"警官が
\\	みてい
\\	よう
\\	が[と]
\\	い
\\	るまい
\\	が[と]
\\	、交通ルールを守るべきだ。
\\	"警官が
\\	みてい
\\	よう_
\\	い
\\	るまい_
\\	、交通ルールを守るべきだ。
\\	上2
\\	病気
\\	なら
\\	病気
\\	で
\\	、おとなしく寝てなきゃだめじゃない。
\\	病気
\\	病気
\\	、おとなしく寝てなきゃだめじゃない。
\\	上2
\\	"「すみません、今日の飲み会はやっぱり行けなくなってしまって……。」
\\	来られない
\\	なら
\\	来られない
\\	で
\\	、早く連絡してくれればよかったのに……。もう予約取り消しできないんだよ。」
\\	"「すみません、今日の飲み会はやっぱり行けなくなってしまって……。」
\\	来られない
\\	来られない
\\	、早く連絡してくれればよかったのに……。もう予約取り消しできないんだよ。」
\\	上2
\\	"どんな
\\	判決[はんけつ]が出ても、被害者
\\	にしたら
\\	、納得できるものではないだろう。
\\	にすれば
\\	にしても
\\	"どんな
\\	判決[はんけつ]が出ても、被害者
\\	に___
\\	、納得できるものではないだろう。
\\	上2
\\	"人員削減[じんいんさくげん]は会社側
\\	にしても
\\	メリットばかりとは言えない。
\\	にすれば
\\	にしたら
\\	"人員削減は会社側
\\	に___
\\	メリットばかりとは言えない。
\\	上2
\\	"地下鉄が止まっていたら、バスで行く
\\	までのことだ
\\	。心配はいらないよ。
\\	"地下鉄が止まっていたら、バスで行く
\\	のことだ
\\	。心配はいらないよ。
\\	上2
\\	"息子も娘も
\\	継[つ]ぐ気がないなら、工場は閉める
\\	までのことだ
\\	"息子も娘も継ぐ気がないなら、工場は閉める
\\	まで___だ
\\	上2
\\	"(学生
\\	自治[じち]会により)講演会
\\	が行われた
\\	学生
\\	自治[じち]会が講演会を行った。			上2
\\	変更される
\\	計画される
\\	変更する
\\	計画する
\\	上2
\\	"炭酸[たんさん]ガスの増加が大気の温度
\\	を
\\	上昇
\\	させる
\\	(炭酸[たんさん]ガスの増加により)大気の温度が上昇する。			上2
\\	"を行
\\	かせる
\\	を進歩
\\	させる
\\	が行く
\\	が進歩する
\\	上2
\\	"問題
\\	が
\\	解決する 
\\	問題
\\	を
\\	解決する 
\\	プラン
\\	が
\\	完成する 
\\	プラン
\\	を
\\	完成する 
\\	問題が解決する
\\	プランが完成する
\\	上2
\\	"私は、友人
\\	が
\\	予定時間に来ないとき、怒る。
\\	私は、友人
\\	(は・が?)
\\	予定時間に来ないとき、怒る。			上2
\\	"子供
\\	が
\\	初めて教育を受ける教育機関は、幼稚園である。
\\	子供
\\	(は・が?)
\\	初めて教育を受ける教育機関は、幼稚園である。			上2
\\	"家の前に毎朝たばこの
\\	吸い殻[すいがら]を拾ってられて、腹が立つ
\\	といったらない
\\	立つっ
\\	たらない
\\	"家の前に毎朝たばこの
\\	吸い殻[すいがら]を拾ってられて、腹が立つ
\\	といった___
\\	上2
\\	"弟にカードゲームで負けるなんて、悔しい
\\	ったらありじゃない
\\	といったらない
\\	"弟にカードゲームで負けるなんて、悔しい
\\	といった___
\\	上2
\\	"成績が悪いので、奨学金に
\\	応募しようにも
\\	応募
\\	できない
\\	"成績が悪いので、奨学金に
\\	応募
\\	しよう__
\\	応募
\\	できない
\\	上2
\\	"彼女に告白
\\	しようにも
\\	、チャンスがなくてなかなか
\\	できない
\\	"彼女に告白
\\	しよう__
\\	、チャンスがなくてなかなかでき
\\	ない
\\	上2
\\	"失恋したぐらいで、やけになって食べ
\\	まくる
\\	奴の気がしれないよ。
\\	"失恋したぐらいで、やけになって食べ
\\	ま_る
\\	奴の気がしれないよ。
\\	上2
\\	"3時間もカラオケで歌い
\\	まくって
\\	、声がなくなった。
\\	"3時間もカラオケで歌い
\\	ま_って
\\	、声がなくなった。
\\	上2
\\	"どんなに素晴らしい小説
\\	にしたって
\\	、出版業界で売れなきゃ認められないよ。
\\	にしたところで
\\	"どんなに素晴らしい小説
\\	に____
\\	、出版業界で売れなきゃ認められないよ。
\\	上2
\\	"進学か就職か、いずれ
\\	にしたって
\\	、自分で決めなければならない。
\\	にしたところで
\\	"進学か就職か、いずれ
\\	に____
\\	、自分で決めなければならない。
\\	上2
\\	"あの頑固な教授に、俺のこの研究論文を絶対に認めさせ
\\	てみせる
\\	"あの頑固な教授に、俺のこの研究論文を絶対に認めさせ
\\	て___
\\	上2
\\	"部長、僕、次こそいい条件で、契約を取っ
\\	てみせます
\\	"部長、僕、次こそいい条件で、契約を取っ
\\	て____
\\	上2
\\	"川面[かわも]に落ちた紅葉が浮き
\\	つ
\\	沈み
\\	つ
\\	流れて行くのを二人で見ていた。
\\	"川面[かわも]に落ちた紅葉が浮き
\\	沈み
\\	流れて行くのを二人で見ていた。
\\	上2
\\	"花火大会は押し
\\	つ
\\	押しされ
\\	つ
\\	で、すごい人ごみだったが、楽しかった。
\\	"花火大会は押し
\\	押しされ
\\	で、すごい人ごみだったが、楽しかった。
\\	上2
\\	"窓の外を
\\	見る
\\	ともなく
\\	見て
\\	いたら、猫がかわいい子猫を連れて歩いてきた。
\\	"窓の外を
\\	見る
\\	と___
\\	見て
\\	いたら、猫がかわいい子猫を連れて歩いてきた。
\\	上1
\\	聞く
\\	ともなく
\\	ラジオを
\\	聞いて
\\	いたら、故郷の町の名前が出てきて驚いた。
\\	聞く
\\	と___
\\	ラジオを
\\	聞いて
\\	いたら、故郷の町の名前が出てきて驚いた。
\\	上1
\\	"島の環境を改善する
\\	べく
\\	島民[とうみん]は様々な取り組みをしている。
\\	"島の環境を改善する
\\	島民[とうみん]は様々な取り組みをしている。
\\	上1
\\	"今年こそロケットを完成させる
\\	べく
\\	、研究者たちは努力を続けている。
\\	"今年こそロケットを完成させる
\\	、研究者たちは努力を続けている。
\\	上1
\\	"結婚して
\\	からというもの
\\	、彼は仕事終わるとまっすぐ家に帰るようになった。
\\	"結婚して
\\	という__
\\	、彼は仕事終わるとまっすぐ家に帰るようになった。
\\	上2
\\	"その歌を聞いて
\\	からというもの
\\	、メロディーが頭から離れない。
\\	"その歌を聞いて
\\	という__
\\	、メロディーが頭から離れない。
\\	上2
\\	"ゲーム機しか知らない子供にも、昔
\\	ながら
\\	の
\\	遊びを伝えたい。
\\	"ゲーム機しか知らない子供にも、昔
\\	の
\\	遊びを伝えたい。
\\	上2
\\	"仏陀[ぶっだ]は生まれ
\\	ながら
\\	に
\\	非凡[ひぼん]な才能を
\\	発揮[はっき]したそうです。
\\	"仏陀[ぶっだ]は生まれ
\\	に
\\	非凡[ひぼん]な才能を
\\	発揮したそうです。
\\	上2
\\	"「助けていただき、ありがとうございました。」
\\	「いやいや、医者として当然のことをし
\\	たまでです
\\	"「助けていただき、ありがとうございました。」
\\	「いやいや、医者として当然のことをし
\\	た__です
\\	上2
\\	"聞かれたから答え
\\	たまでで
\\	、別に深い意味はないよ。
\\	"聞かれたから答え
\\	た__で
\\	、別に深い意味はないよ。
\\	上2
\\	"子供たちは、
\\	連日[れんじつ]の暑さ
\\	をものともせず
\\	、元気に遊びまわっている。
\\	"子供たちは、
\\	連日[れんじつ]の暑さ
\\	をもの__せず
\\	、元気に遊びまわっている。
\\	上2
\\	"激流[げきりゅう]
\\	をものともせず
\\	、彼はカヌーで川をくだっていた。 
\\	"激流[げきりゅう]
\\	ものとも__
\\	、彼はカヌーで川をくだっていた。
\\	上2
\\	"友達は高校で数学を教える
\\	かたわら
\\	、小説を書いている。
\\	"友達は高校で数学を教える
\\	か___
\\	、小説を書いている。
\\	上2
\\	"最近は教育の
\\	かたわら
\\	、インターネットでビジネスをする女性が増えている。
\\	"最近は教育の
\\	か___
\\	、インターネットでビジネスをする女性が増えている。
\\	上2
\\	"風も涼しくなり、徐々に秋
\\	ぬいて
\\	参りましたが、いかがお過ごしでしょうか。
\\	"風も涼しくなり、徐々に秋
\\	いて
\\	参りましたが、いかがお過ごしでしょうか。
\\	上2
\\	"彼は上司のせいで苦労したが、非難
\\	ぬいた
\\	ことは一言も言わず、
\\	黙々[もくもく]と働いていた。
\\	"彼は上司のせいで苦労したが、非難
\\	いた
\\	ことは一言も言わず、
\\	黙々[もくもく]と働いていた。
\\	上2
\\	"電話を切る
\\	や
\\	否[いな]や
\\	、刑事は部屋を飛び出して事件現場に向かった。 
\\	や
\\	とすぐ 
\\	~たとたん	"電話を切る
\\	や_や
\\	、刑事は部屋を飛び出して事件現場に向かった。
\\	上2
\\	"優勝の瞬間、大川選手は「やった」と叫ぶ
\\	や
\\	否[いな]や
\\	、コーチの元へ駆け寄った。
\\	や
\\	とすぐ 
\\	~たとたん	"優勝の瞬間、大川選手は「やった」と叫ぶ
\\	や_や
\\	、コーチの元へ駆け寄った。
\\	上2
\\	"お菓子
\\	ごとき
\\	で兄弟げんかするな。
\\	"お菓子
\\	とき
\\	で兄弟げんかするな。
\\	上2
\\	"こんな問題
\\	ごとき
\\	、小学生でも解ける。
\\	"こんな問題
\\	とき
\\	、小学生でも解ける。
\\	上2
\\	"なかなか連絡が来ないので落ちた
\\	かと思いきゃ
\\	、今日になって合格通知が届いた。
\\	"なかなか連絡が来ないので落ちた
\\	かと思い__
\\	、今日になって合格通知が届いた。
\\	上1
\\	"住宅
\\	街[がい]のマンションだから静か
\\	だ[か]と思いきゃ
\\	、遅くまで人通りが多くてうるさかった。
\\	"住宅
\\	街[がい]のマンションだから静か
\\	だ[か]と思い__
\\	、遅くまで人通りが多くてうるさかった。
\\	上1
\\	"国民の生命を守
\\	らんがため
\\	、彼は敢えて危険を伴う
\\	任務[にんむ]を引き受けた。
\\	"国民の生命を守
\\	らんが__
\\	、彼は敢えて危険を伴う
\\	任務[にんむ]を引き受けた。
\\	上1
\\	"全宇宙を
\\	征服[せいふく]
\\	せんがため
\\	、彼女は大宇宙
\\	船団[せんだん]を
\\	率[ひき]いて飛び立った。
\\	"全宇宙を征服
\\	(する)
\\	んがため
\\	、彼女は大宇宙
\\	船団[せんだん]を
\\	率[ひき]いて飛び立った。
\\	上1
\\	"首相の入院を今朝のニュースで知った。
\\	首相の病名はまだ公表されていない。
\\	首相の今年中の職務
\\	への
\\	復帰[ふっき]
\\	については
\\	、今に時点ではわからない。
\\	首相が入院したことを、今朝のニュースで知った。
\\	首相どんな病気なのか、まだ公表されていない。
\\	首相は今年中に職務に
\\	復帰[ふっき]できるだろうか、今に時点ではわからない。
\\	首相が入院した。それを今朝のニュースで知った。
\\	首相どんな病気なのか。まだ公表されていない。
\\	首相は今年中に職務に
\\	復帰[ふっき]できるだろうか。今に時点ではわからない。
\\	上2
\\	"友達
\\	と
\\	の
\\	旅行
\\	外国
\\	から
\\	の
\\	輸入
\\	来年大学院
\\	へ
\\	の
\\	進学
\\	明日
\\	まで
\\	の
\\	出張
\\	日本
\\	で
\\	の
\\	就職
\\	家族
\\	への
\\	愛
\\	他人
\\	への
\\	依存
\\	友達と旅行をすること
\\	外国から輸入すること
\\	来年大学院へ進むこと
\\	明日まで出張すること
\\	日本で就職すること
\\	家族を愛すること
\\	他人に依存すること
\\	上2
\\	"憲法で男女が平等の権利を持つ
\\	こと
\\	が認められない。 
\\	憲法で男女が平等の権利を持つ
\\	(こと・を?)
\\	が認められない。			上2
\\	"子供たちが大勢で歌っている
\\	の
\\	が
\\	聞こえる
\\	聞こえる、見える、感じる、
\\	子供たちが大勢で歌っている
\\	(こと・の?)
\\	が聞こえる。			上2
\\	"私が彼に会った
\\	の
\\	は、昨日です。
\\	私が彼に会った
\\	(こと・の?)
\\	は、昨日です。 
\\	上2
\\	"兄弟に残された
\\	遺書[いしょ]には次
\\	のごとく
\\	記[しる]されていた。
\\	のように	
\\	兄弟に残された
\\	遺書[いしょ]には次
\\	のご__
\\	記[しる]されていた。
\\	上2
\\	"宝くじで
\\	大金[たいきん]を手に入れたが、
\\	湯水[ゆみず]
\\	のごとく
\\	使い続け、1年後には元の貧乏生活に戻ってしまった。
\\	のように	"宝くじで大金を手に入れたが、
\\	湯水[ゆみず]
\\	のご__
\\	使い続け、1年後には元の貧乏生活に戻ってしまった。
\\	上2
\\	"砂に描いた絵は、強風により、一瞬
\\	にして
\\	消え去った。
\\	"砂に描いた絵は、強風により、一瞬
\\	にし_
\\	消え去った。
\\	上2
\\	"妹は交通事故に遭ったが、幸い
\\	にして
\\	軽いけがだったので入院せずにすんだ。
\\	"妹は交通事故に遭ったが、幸い
\\	にし_
\\	軽いけがだったので入院せずにすんだ。
\\	上2
\\	"あんな高いところから飛び
\\	降[お]りるなんて、想像する
\\	だに
\\	恐ろしい。
\\	"あんな高いところから飛び
\\	降[お]りるなんて、想像する
\\	に
\\	恐ろしい。
\\	上2
\\	"その
\\	山寺[やまてら]は、見る
\\	だに
\\	不気味[ぶきみ]な雰囲気に包まれていた。
\\	"その
\\	山寺[やまてら]は、見る
\\	に
\\	不気味[ぶきみ]な雰囲気に包まれていた。
\\	上2
\\	"予想
\\	だにしなかった
\\	コンピューターのシステム障害が発生し、担当者は対応に追われた。
\\	"予想
\\	だに_なかった
\\	コンピューターのシステム障害が発生し、担当者は対応に追われた。
\\	上2
\\	"私はいい企画だと思ったが、社内ではいっこ
\\	だにされなかった
\\	"私はいい企画だと思ったが、社内ではいっこ
\\	だ_されなかった
\\	上2
\\	"どうして君が仕事をやめたいと言い出したのか、理解
\\	し
\\	かねる
\\	んだけどね。
\\	できない 
\\	"どうして君が仕事をやめたいと言い出したのか、理解
\\	し
\\	んだけどね。
\\	上2
\\	"明日
\\	を限りに
\\	離れ離れ[はなればなれ]になる二人は夜
\\	徹[てっ]して語り明けした。
\\	"明日
\\	限りに
\\	離れ離れ[はなればなれ]になる二人は夜
\\	徹[てっ]して語り明けした。
\\	上2
\\	"「今日
\\	を限りに
\\	キャンベルはやめる」と、彼は今年だけでも3回は言った。
\\	"「今日
\\	限りに
\\	キャンベルはやめる」と、彼は今年だけでも3回は言った。
\\	上2
\\	"100人
\\	からいる
\\	けが人の中には、命にかかわる
\\	重傷者[じゅうしょうしゃ]もいます。
\\	"100人
\\	から__
\\	けが人の中には、命にかかわる
\\	重傷者[じゅうしょうしゃ]もいます。
\\	上2
\\	"小さな子供が、10
\\	からある
\\	旅行かばんを一生懸命運ぼうとしている。
\\	"小さな子供が、10
\\	から__
\\	旅行かばんを一生懸命運ぼうとしている。
\\	上2
\\	"政府は、国民の安全
\\	にかかわる
\\	情報はすぐに公表すべきだ。
\\	"政府は、国民の安全
\\	に____
\\	情報はすぐに公表すべきだ。
\\	上2
\\	"火災などの非常時
\\	にあって
\\	は、落ち着いて行動することがまず大事だ。
\\	"火災などの非常時
\\	に___
\\	は、落ち着いて行動することがまず大事だ。
\\	上2
\\	"このような緊急時
\\	にあって
\\	、医師は
\\	速[すみ]やかな決断を
\\	迫[せま]られます。
\\	"このような緊急時
\\	に___
\\	、医師は
\\	速[すみ]やかな決断を
\\	迫[せま]られます。
\\	上2
\\	"確かに、見
\\	ようによっては
\\	非情な行為かもしれません。
\\	"確かに、見
\\	ように___は
\\	非情な行為かもしれません。
\\	上2
\\	"説明不足だと、受け取り
\\	ようによっては
\\	、誤解を招く恐れがあるよ。
\\	"説明不足だと、受け取り
\\	ように___は
\\	、誤解を招く恐れがあるよ。
\\	上2
\\	"ペンキ塗りたて。座る
\\	べからず
\\	~な 
\\	"ペンキ塗りたて。座る
\\	らず
\\	上2
\\	"飲酒運転は、許す
\\	べからざる
\\	行為だ。
\\	~な 
\\	"飲酒運転は、許す
\\	らざる
\\	行為だ。
\\	上2
\\	"やるべきことはすべてやったんでしょう。あとはただ結果を待つ
\\	のみ
\\	だね。
\\	"やるべきことはすべてやったんでしょう。あとはただ結果を待つ
\\	だね。
\\	上2
\\	"聞こえてくるのはただ
\\	草原[そうげん]を渡る風の音
\\	のみ
\\	だった。
\\	"聞こえてくるのはただ草原を渡る風の音
\\	だった。
\\	上2
\\	ゆえに
\\	である。
\\	~ため	
\\	に
\\	である。
\\	上2
\\	"日本
\\	カワウソ
\\	は毛皮が美しい
\\	がゆえに
\\	乱獲[らんかく]され、絶滅してしまった。
\\	~ため	"日本カワウソは毛皮が美しい
\\	が__に
\\	乱獲され、絶滅してしまった。
\\	上2
\\	"この絵は、リアルな
\\	描写[びょうしゃ]
\\	に
\\	幻想的[げんそうてき]な世界
\\	が
\\	相[あい]まって
\\	、忘れがたい印象を与える。
\\	"この絵は、リアルな
\\	描写[びょうしゃ]に
\\	幻想的[げんそうてき]な世界
\\	が_まって
\\	、忘れがたい印象を与える。
\\	上2
\\	"文学として読めないがゆえに、その
\\	筆[ふで]の線が
\\	余白[よはく]
\\	と
\\	相[あい]まって
\\	作り出す空間の
\\	美[び]を純粋に鑑賞できるのである。
\\	"文学として読めないがゆえに、その
\\	筆[ふで]の線が
\\	余白[よはく]
\\	と_まって
\\	作り出す空間の
\\	美[び]を純粋に鑑賞できるのである。
\\	上2
\\	"新しいタイプの芸術に対して、評価が分かれるのは推測
\\	にかたくない
\\	"新しいタイプの芸術に対して、評価が分かれるのは推測
\\	にかた___
\\	上2
\\	"リーダーたるもの、苦しい決断を
\\	迫[せま]られるであろうことは予想
\\	にかたくない
\\	"リーダーたるもの、苦しい決断を
\\	迫[せま]られるであろうことは予想
\\	に__くない
\\	上2
\\	"真っ先に
\\	救命[きゅうめい]ボートに乗るとは
\\	船長[せんちょう]
\\	としてあるまじき
\\	行為だ。
\\	"真っ先に救命ボートに乗るとは
\\	船長[せんちょう]
\\	としてあるま_き
\\	行為だ。
\\	上2
\\	"こじきからお金をだまし取るなんて、人
\\	として
\\	許す
\\	まじき
\\	ことだ。
\\	"こじきからお金をだまし取るなんて、人
\\	として
\\	許す
\\	ま_き
\\	ことだ。
\\	上2
\\	"このまま森林伐採を続けていたら、自然災害を引き起こさ
\\	ずにはすまない
\\	だろう。
\\	"このまま森林伐採を続けていたら、自然災害を引き起こさ
\\	ずにはす___
\\	だろう。
\\	上2
\\	"欠陥品を売ったのだから消費者に非難され
\\	ないではすまない
\\	でしょう。
\\	"欠陥品を売ったのだから消費者に非難され
\\	ではすまない
\\	でしょう。
\\	上2
\\	"親猫が、子猫のために大きな犬と戦った。これが愛情
\\	でなくてなんだろう
\\	"親猫が、子猫のために大きな犬と戦った。これが愛情
\\	でなくて__だろう
\\	上2
\\	"10階から
\\	転落[てんらく]した幼児が無事だったとは、これが奇跡
\\	でなくてなんだろう
\\	"10階から
\\	転落[てんらく]した幼児が無事だったとは、これが奇跡
\\	でなくて__だろう
\\	上2
\\	"アルバイトとはいえ、面接中にメールを確認すると非常識
\\	極[きわ]まる
\\	極[きわ]まりない
\\	"アルバイトとはいえ、面接中にメールを確認すると非常識
\\	まる
\\	上2
\\	"「人間とは何か」などという
\\	難解[なんかい]
\\	極[きわ]まる
\\	問いかけに、一体何と答えればいいのだろう。
\\	極[きわ]まりない
\\	"「人間とは何か」などという
\\	難解[なんかい]
\\	まる
\\	問いかけに、一体何と答えればいいのだろう。
\\	上2
\\	"災害で故郷を失った人々の姿に涙
\\	を禁じえなかった
\\	"災害で故郷を失った人々の姿に涙
\\	を_じえなかった
\\	上2
\\	"長年ご指導いただきました教授のご
\\	退官[たいかん]にあたり、一同、
\\	惜別[せきべつ]の
\\	念[ねん]
\\	を禁じえません
\\	"長年ご指導いただきました教授のご
\\	退官[たいかん]にあたり、一同、
\\	惜別[せきべつ]の念
\\	を_じえません
\\	上2
\\	"ホームドクターに信頼
\\	にたる
\\	眼科の専門医を紹介してもらった。
\\	"ホームドクターに信頼
\\	に_る
\\	眼科の専門医を紹介してもらった。
\\	上2
\\	"ここは温泉も料理もすばらしく、部屋も落ち着いた雰囲気で、推薦する
\\	にたる
\\	旅館だ。
\\	"ここは温泉も料理もすばらしく、部屋も落ち着いた雰囲気で、推薦する
\\	に_る
\\	旅館だ。
\\	上2
\\	"当時母がどんな気持ちだったのか、子供の私に知る
\\	べくもない
\\	ことだった。
\\	"当時母がどんな気持ちだったのか、子供の私に知る
\\	べく_ない
\\	ことだった。
\\	上2
\\	"アジアが世界経済の
\\	鍵[かぎ]であることは、疑う
\\	べくもない
\\	"アジアが世界経済の鍵であることは、疑う
\\	べく_ない
\\	上2
\\	"「努力
\\	なくして
\\	、成功なし」というが、運というものもあるのではないだろうか。
\\	"「努力
\\	して
\\	、成功なし」というが、運というものもあるのではないだろうか。
\\	上2
\\	"家族の同意
\\	なくして
\\	、手術は行えないことになっている。
\\	"家族の同意
\\	して
\\	、手術は行えないことになっている。
\\	上2
\\	"➀自転車通学のよい
\\	ところ
\\	は経済的な
\\	こと
\\	である。
\\	経済的であることである
\\	➁この食堂の良い
\\	点
\\	は、ご飯のおかわりができる
\\	こと
\\	である。
\\	③日本に来て
\\	一番
\\	困った
\\	こと
\\	は言葉だった。
\\	首尾一貫[しゅびいっかん]	"➀自転車通学のよい
\\	ところ
\\	は経済的な
\\	である。
\\	➁この食堂の良い
\\	点
\\	は、ご飯のおかわりができる
\\	である。
\\	③日本に来て
\\	一番
\\	困った
\\	こと
\\	は言葉だった。
\\	上3
\\	"私の
\\	考え
\\	では、このような規則は良くない。
\\	私は、このような規則は良くない
\\	と思う
\\	首尾一貫[しゅびいっかん]	
\\	私の
\\	考えは
\\	、このような規則は良くない
\\	と思う
\\	。			上3
\\	"大学食堂の一階は禁煙になっているので空気がきれいだ
\\	が
\\	、2階は禁煙になっていないから、空気が汚い
\\	。それで、私
\\	は2階に行ったことがない。
\\	一つの文、一つの内容	大学食堂の一階は禁煙なので空気がきれいで、2階は禁煙になっていないから、私はそ行ったことがない。			上3
\\	引用に使われる表現			上3
\\	"よく目にする
\\	2万5千
\\	分の
\\	地形図が一般地図である。
\\	"よく目にする
\\	2万5千__1
\\	地形図が一般地図である。
\\	上3
\\	分類の表現	
\\	上4
\\	5つのパターン			上4
\\	上4
\end{CJK}
\end{document}