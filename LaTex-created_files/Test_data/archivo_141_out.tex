\documentclass[8pt]{extreport} 
\usepackage{hyperref}
\usepackage{CJKutf8}
\begin{document}
\begin{CJK}{UTF8}{min}
\\	昨日の喧嘩で何も言
\\	ばかりに、平気な顔をしている。	昨日の喧嘩で何も言わなかったばかりに、平気な顔をしている。	きのうのケンカでなにもいわなかったばかりに、へいきなかおをしている。	
\\	「ばかり」 
\\	昨日[きのう]の 喧嘩[けんか]で 何[なに]も 言[い]わなかったばかりに、 平気[へいき]な 顔[かお]をしている。		
\\	早く帰
\\	べき。	早く帰るべき。	はやくかえるべき。	
\\	「べく」 
\\	早[はや]く 帰[かえ]るべき。		
\\	最近のウィルスは強力で、プログラムを実行
\\	する 
\\	が
\\	する]まいが、ページを見るだけで感染するらしい。	最近のウィルスは強力で、プログラムを実行しようがしまいが、ページを見るだけで感染するらしい。	さいきんのウィルスはきょうりょくで、プログラムをじっこうしようがしまいが、ページをみるだけでかんせんするらしい。	
\\	最近[さいきん]のウィルスは 強力[きょうりょく]で、プログラムを 実行[じっこう]しようがしまいが、ページを 見[み]るだけで 感染[かんせん]するらしい。		
\\	相手は剣の達人だ。そう簡単には勝
\\	まい。	相手は剣の達人だ。そう簡単には勝てまい。	あいてはつるぎのたつじんだ。そうかんたんにはかてまい。	
\\	相手[あいて]は 剣[けん]の 達人[たつじん]だ。そう 簡単[かんたん]には 勝[か]てまい。		
\\	企業内の顧客データを利用
\\	、彼の行方を調べることが出来た。。	企業内の顧客データを利用し、彼の行方を調べることが出来た。	きぎょうないのこきゃくデータをりようし、かれのゆくえをしらべることができた。	
\\	企業[きぎょう] 内[ない]の 顧客[こきゃく]データを 利用[りよう]し、 彼[かれ]の 行方[ゆくえ]を 調[しら]べることが 出来[でき]た。		
\\	今後50年、人間が直面するであろう問題に正面から向き合って、自ら解決をはかりつつ、そのノウハウが次の産業となるシナリオを考えたい。	今後50年、人間が直面するであろう問題に正面から向き合って、自ら解決をはかりつつ、そのノウハウが次の産業となるシナリオを考えたい。	こんご50ねん、にんげんがちょくめんするであろうもんだいにしょうめんからむきあって、みずからかいけつをはかりつつ、そのノウハウがつぎのさんぎょうとなるシナリオをかんがえたい。	
\\	「であろう」 
\\	今後[こんご] 
\\	年[ねん]、 人間[にんげん]が 直面[ちょくめん]するであろう 問題[もんだい]に 正面[しょうめん]から 向き合[むきあ]って、 自[みずか]ら 解決[かいけつ]をはかりつつ、そのノウハウが 次[つぎ]の 産業[さんぎょう]となるシナリオを 考[かんが]えたい。		
\\	コーディングが好きな開発者は、ちゃんとしたドキュメント作成と十分なテストを怠
\\	きらいがある。	コーディングが好きな開発者は、ちゃんとしたドキュメント作成と十分なテストを怠るきらいがある。	コーディングがすきなかいはつしゃは、ちゃんとしたドキュメントさくせいとじゅうぶんなテストをおこたるきらいがある。	
\\	「きらいがある」	コーディングが 好[す]きな 開発[かいはつ] 者[しゃ]は、ちゃんとしたドキュメント 作成[さくせい]と 十分[じゅうぶん]なテストを 怠[おこた]るきらいがある。		
\\	そんな謎
\\	めいた顔をされても、うまく説明できないよ。。	そんな謎めいた顔をされても、うまく説明できないよ。	そんななぞめいたかおをされても、うまくせつめいできないよ。	
\\	「めく」 
\\	そんな 謎[なぞ]めいた 顔[かお]をされても、うまく 説明[せつめい]できないよ。		
\\	妻はルイヴィトンのバッグを欲
\\	がっているんだけど、そんなもん、買えるわけないでしょう!	妻はルイヴィトンのバッグを欲しがっているんだけど、そんなもん、買えるわけないでしょう!	つまはルイヴィトンのバッグをほしがっているんだけど、そんなもん、かえるわけないでしょう!	
\\	「~がる」	妻[つま]はルイヴィトンのバッグを 欲[ほ]しがっているんだけど、そんなもん、 買[か]えるわけないでしょう!		
\\	そんな無茶な手段は認
\\	まい!	そんな無茶な手段は認めますまい!	そんなむちゃなしゅだんはみとめますまい!	
\\	そんな 無茶[むちゃ]な 手段[しゅだん]は 認[みと]めますまい!		
\\	預金者が大手銀行を相手取って訴訟を起こすケースも出ており、金融庁は被害者の救済を優先させて、金融機関に犯罪防止対策の強化を促すべきだと判断。	預金者が大手銀行を相手取って訴訟を起こすケースも出ており、金融庁は被害者の救済を優先させて、金融機関に犯罪防止対策の強化を促すべきだと判断。	よきんものがおおてぎんこうをあいてとってそしょうをおこすケースもでており、きんゆうちょうはひがいしゃのきゅうさいをゆうせんさせて、きんゆうきかんにはんざいぼうしたいさくのきょうかをうながすべきだとはんだん。	
\\	「べき」 
\\	預金[よきん] 者[しゃ]が 大手[おおて] 銀行[ぎんこう]を 相手取[あいてど]って 訴訟[そしょう]を 起[お]こすケースも 出[で]ており、 金融[きんゆう] 庁[ちょう]は 被害[ひがい] 者[しゃ]の 救済[きゅうさい]を 優先[ゆうせん]させて、 金融[きんゆう] 機関[きかん]に 犯罪[はんざい] 防止[ぼうし] 対策[たいさく]の 強化[きょうか]を 促[うなが]すべきだと 判断[はんだん]。		
\\	お弁当を買うお金
\\	さえなかった。	お弁当を買うお金さえなかった。	おべんとうをかうおきんさえなかった。	
\\	「(で)さえ」 
\\	お 弁当[べんとう]を 買[か]うお 金[かね]さえなかった。		
\\	この天才の私
\\	すら分からなかった。	この天才の私ですら分からなかった。	このてんさいのわたしですらわからなかった。	
\\	「(で)すら」 
\\	「(で)さえ」 
\\	この 天才[てんさい]の 私[わたし]ですら 分[わ]からなかった。		
\\	国土交通省は2年後に利用率を70%まで引き上げる考えで、買い替え時に利用する気になるかどうかがカギになりそうだ。	国土交通省は2年後に利用率を70%まで引き上げる考えで、買い替え時に利用する気になるかどうかがカギになりそうだ。	こくどこうつうしょうは2ねんごにりようりつを70%までひきあげるかんがえで、かいかえときにりようするきになるかどうかがカギになりそうだ。	
\\	「である」 
\\	国土[こくど] 交通省[こうつうしょう]は 2[に] 年[ねん] 後[ご]に 利用[りよう] 率[りつ]を 7[なな] 0[ぜろ] 
\\	[ぱーせんと]まで 引き上[ひきあ]げる 考[かんが]えで、 買い替[かいか]え 時[じ]に 利用[りよう]する 気[き]になるかどうかがカギになりそうだ。		
\\	教科書を読
\\	そばから忘れてしまうので勉強ができない。	教科書を読むそばから忘れてしまうので勉強ができない。	きょうかしょをよむそばからわすれてしまうのでべんきょうができない。	
\\	「そばから」 
\\	教科書[きょうかしょ]を 読[よ]むそばから 忘[わす]れてしまうので 勉強[べんきょう]ができない。		
\\	あの人は、二度と嘘を付
\\	まいと誓ったのです。	あの人は、二度と嘘を付くまいと誓ったのです。	あのにんは、にどとうそをつくまいとちかったのです。	
\\	あの 人[ひと]は、 二度[にど]と 嘘[うそ]を 付[つ]くまいと 誓[ちか]ったのです。		
\\	彼女は朝早く起こされるのを嫌
\\	がるタイプです。	彼女は朝早く起こされるのを嫌がるタイプです。	かのじょはあさはやくおこされるのをいやがるタイプです。	
\\	「~がる」	彼女[かのじょ]は 朝[あさ] 早[はや]く 起[お]こされるのを 嫌[いや]がるタイプです。		
\\	先生と相談
\\	あげく、退学することにした。	先生と相談のあげく、退学することにした。	せんせいとそうだんのあげく、たいがくすることにした。	
\\	「~あげく(挙句)」 
\\	先生[せんせい]と 相談[そうだん]のあげく、 退学[たいがく]することにした。		
\\	私は緊張しすぎて、ちらっと見ること
\\	すら出来ませんでした。	私は緊張しすぎて、ちらっと見ることすら出来ませんでした。	わたしはきんちょうしすぎて、ちらっとみることすらできませんでした。	
\\	「(で)すら」 
\\	「(で)さえ」 
\\	私[わたし]は 緊張[きんちょう]しすぎて、ちらっと 見[み]ることすら 出来[でき]ませんでした。		
\\	多くの大学生は、締切日ぎりぎりまで、宿題を
\\	やる, 
\\	きらいがある。	多くの大学生は、締切日ぎりぎりまで、宿題をやらないきらいがある。	おおくのだいがくせいは、しめきりびぎりぎりまで、しゅくだいをやらないきらいがある。	
\\	「きらいがある」	多[おお]くの 大学生[だいがくせい]は、 締切[しめきり] 日[び]ぎりぎりまで、 宿題[しゅくだい]をやらないきらいがある。		
\\	いつも皮肉
\\	めいた言い方をしたら、みんなを嫌がらせるよ。。	いつも皮肉めいた言い方をしたら、みんなを嫌がらせるよ。	いつもひにくめいたいいかたをしたら、みんなをいやがらせるよ。 いやがる【嫌がる】 〔拒否する〕 
\\	「めく」 
\\	いつも 皮肉[ひにく]めいた 言い方[いいかた]をしたら、みんなを 嫌[いや]がらせるよ。		
\\	安全措置を忘
\\	べからず。	安全措置を忘れるべからず。	あんぜんそちをわすれるべからず。	
\\	「べからず」 
\\	安全[あんぜん] 措置[そち]を 忘[わす]れるべからず。		
\\	ゴミ捨
\\	べからず。	ゴミ捨てるべからず。	ゴミすてるべからず。	
\\	「べからず」 
\\	ゴミ 捨[す]てるべからず。		
\\	花火は、火薬と金属の粉末を混ぜたものに火を付け、燃焼時の火花を楽しむためのもの。	花火は、火薬と金属の粉末を混ぜたものに火を付け、燃焼時の火花を楽しむためのもの。	はなびは、かやくときんぞくのふんまつをまぜたものにひをづけ、ねんしょうときのひばなをたのしむためのもの。	
\\	花火[はなび]は、 火薬[かやく]と 金属[きんぞく]の 粉末[ふんまつ]を 混[ま]ぜたものに 火[ひ]を 付[つ]け、 燃焼[ねんしょう] 時[じ]の 火花[ひばな]を 楽[たの]しむためのもの。		
\\	このままでは、個人情報が漏洩
\\	する]かねないので、速やかに対応をお願い致します。	このままでは、個人情報が漏洩しかねないので、速やかに対応をお願い致します。	このままでは、こじんじょうほうがろうえいしかねないので、すみやかにたいおうをおねがいいたします。	
\\	「~かねる」	このままでは、 個人[こじん] 情報[じょうほう]が 漏洩[ろうせつ]しかねないので、 速[すみ]やかに 対応[たいおう]をお 願[ねが]い 致[いた]します。		
\\	何かを買う前に本当に必要かどうかをよく考
\\	べきだ。	何かを買う前に本当に必要かどうかをよく考えるべきだ。	なにかをかうまえにほんとうにひつようかどうかをよくかんがえるべきだ。	
\\	「べき」 
\\	何[なに]かを 買[か]う 前[まえ]に 本当[ほんとう]に 必要[ひつよう]かどうかをよく 考[かんが]えるべきだ。		
\\	父親は病気
\\	がちで、みんなが心配している。	父親は病気がちで、みんなが心配している。	ちちおやはびょうきがちで、みんながしんぱいしている。	
\\	「~がち」	父親[ちちおや]は 病気[びょうき]がちで、みんなが 心配[しんぱい]している。		
\\	「あんたと関係ない」と言
\\	ばかりに彼女は彼を無視していた。	「あんたと関係ない」と言わんばかりに彼女は彼を無視していた。	「あんたとかんけいない」といわんばかりにかのじょはかれをむししていた。	
\\	「ばかり」 
\\	「あんたと 関係[かんけい]ない」と 言[い]わんばかりに 彼女[かのじょ]は 彼[かれ]を 無視[むし]していた。		
\\	大学
\\	おろか、高校すら卒業しなかった。	大学はおろか、高校すら卒業しなかった。	だいがくはおろか、こうこうすらそつぎょうしなかった。	
\\	「おろか」 
\\	大学[だいがく]はおろか、 高校[こうこう]すら 卒業[そつぎょう]しなかった。		
\\	やむ
\\	を得ない事由により手続が遅れた場合、必ずご連絡下さい。	やむを得ない事由により手続が遅れた場合、必ずご連絡下さい。	やむをえないじゆうによりてつづきがおくれたばあい、かならずごれんらくください。	
\\	「やむを得ない」	やむを 得[え]ない 事由[じゆう]により 手続[てつづき]が 遅[おく]れた 場合[ばあい]、 必[かなら]ずご 連絡[れんらく] 下[くだ]さい。		
\\	もちろん、生徒数減少の現在、学科の新設は困難であろうが、職業科の統廃合や科内コースの改編などで時代に合わせた変革が求められているはずである。	もちろん、生徒数減少の現在、学科の新設は困難であろうが、職業科の統廃合や科内コースの改編などで時代に合わせた変革が求められているはずである。	もちろん、せいとすうげんしょうのげんざい、がっかのしんせつはこんなんであろうが、しょくぎょうかのとうはいごうやかないコースのかいへんなどでじだいにあわせたへんかくがもとめられているはずである。	
\\	「であろう」 
\\	もちろん、 生徒[せいと] 数[すう] 減少[げんしょう]の 現在[げんざい]、 学科[がっか]の 新設[しんせつ]は 困難[こんなん]であろうが、 職業[しょくぎょう] 科[か]の 統廃合[とうはいごう]や 科[か] 内[ない]コースの 改編[かいへん]などで 時代[じだい]に 合[あ]わせた 変革[へんかく]が 求[もと]められているはずである。		
\\	打合せは毎週2時から始
\\	はずじゃないですか?	打合せは毎週2時から始まるはずじゃないですか?	うちあわせはまいしゅう2ときからはじまるはずじゃないですか?	
\\	「はず」 
\\	打合[うちあわ]せは 毎週[まいしゅう]2 時[じ]から 始[はじ]まるはずじゃないですか?		
\\	彼は油
\\	まみれになりながら、車の修理に頑張りました。	彼は油まみれになりながら、車の修理に頑張りました。	かれはあぶらまみれになりながら、くるまのしゅうりにがんばりました。	
\\	「まみれ」 
\\	彼[かれ]は 油[あぶら]まみれになりながら、 車[くるま]の 修理[しゅうり]に 頑張[がんば]りました。		
\\	留守
\\	がちなご家庭には、犬よりも、猫の方がお勧めです。	留守がちなご家庭には、犬よりも、猫の方がお勧めです。	るすがちなごかていには、いぬよりも、ねこのほうがおすすめです。	
\\	「~がち」	留守[るす]がちなご 家庭[かてい]には、 犬[いぬ]よりも、 猫[ねこ]の 方[ほう]がお 勧[すす]めです。		
\\	漢字
\\	おろか、平仮名さえ読めないよ!	漢字はおろか、平仮名さえ読めないよ!	かんじはおろか、ひらがなさえよめないよ!	
\\	「おろか」 
\\	漢字[かんじ]はおろか、 平仮名[ひらがな]さえ 読[よ]めないよ!		
\\	あいつが大学に入
\\	うが入
\\	まいが、俺とは関係ないよ。	あいつが大学に入ろうが入るまいが、俺とは関係ないよ。	あいつがだいがくにいろうがいるまいが、おれとはかんけいないよ。	
\\	あいつが 大学[だいがく]に 入[にゅう]ろうが 入[はい]るまいが、 俺[おれ]とは 関係[かんけい]ないよ。		
\\	「人」の漢字
\\	すら知らない生徒は、いないでしょ!	「人」の漢字すら知らない生徒は、いないでしょ!	「にん」のかんじすらしらないせいとは、いないでしょ!	
\\	「人」 
\\	「(で)すら」 
\\	「(で)さえ」 
\\	人[ひと]」の 漢字[かんじ]すら 知[し]らない 生徒[せいと]は、いないでしょ!		
\\	運動を始めるのが早かろうが遅かろうが、健康にいいというのは変わりません。	運動を始めるのが早かろうが遅かろうが、健康にいいというのは変わりません。	うんどうをはじめるのがはやかろうがおそかろうが、けんこうにいいというのはかわりません。	
\\	「かろう」 
\\	「い」 
\\	運動[うんどう]を 始[はじ]めるのが 早[はや]かろうが 遅[おそ]かろうが、 健康[けんこう]にいいというのは 変[か]わりません。		
\\	この料理は美味
\\	はずだったが、焦げちゃって、まずくなった。	この料理は美味いはずだったが、焦げちゃって、まずくなった。	このりょうりはおいしいはずだったが、こげちゃって、まずくなった。	
\\	「はず」 
\\	この 料理[りょうり]は 美味[うま]いはずだったが、 焦[こ]げちゃって、まずくなった。		
\\	色々予定してあるから、今年は楽しいクリスマス
\\	はず。	色々予定してあるから、今年は楽しいクリスマスのはず。	いろいろよていしてあるから、こんねんはたのしいクリスマスのはず。	
\\	「はず」 
\\	色々[いろいろ] 予定[よてい]してあるから、 今年[ことし]は 楽[たの]しいクリスマスのはず。		
\\	みんなイタリアに行
\\	がってるんだけど、私の予算で行けるかどうかはとても怪しい。	みんなイタリアに行きたがってるんだけど、私の予算で行けるかどうかはとても怪しい。	みんなイタリアにいきたがってるんだけど、わたしのよさんでいけるかどうかはとてもあやしい。	
\\	「~がる」	みんなイタリアに 行[い]きたがってるんだけど、 私[わたし]の 予算[よさん]で 行[い]けるかどうかはとても 怪[あや]しい。		
\\	紅葉が始まり、すっかり秋
\\	めいた空気になってきた。	紅葉が始まり、すっかり秋めいた空気になってきた。	こうようがはじまり、すっかりあきめいたくうきになってきた。	
\\	「めく」 
\\	紅葉[こうよう]が 始[はじ]まり、すっかり 秋[あき]めいた 空気[くうき]になってきた。		
\\	早く帰
\\	べく、準備をし始めた。	早く帰るべく、準備をし始めた。	はやくかえるべく、じゅんびをしはじめた。	
\\	「べく」 
\\	早[はや]く 帰[かえ]るべく、 準備[じゅんび]をし 始[はじ]めた。		
\\	私
\\	さえ出来れば、あんたには楽ちんでしょう。	私でさえ出来れば、あんたには楽ちんでしょう。	わたしでさえできれば、あんたにはらくちんでしょう。	
\\	「(で)さえ」 
\\	私[わたし]でさえ 出来[でき]れば、あんたには 楽[らく]ちんでしょう。		
\\	言語は簡単にマスターできること
\\	言語は簡単にマスターできることではない。	げんごはかんたんにマスターできることではない。	
\\	「である」	言語[げんご]は 簡単[かんたん]にマスターできることではない。		
\\	休日であろうが、なかろうが、この仕事では関係ないみたい。	休日であろうが、なかろうが、この仕事では関係ないみたい。	きゅうじつであろうが、なかろうが、このしごとではかんけいないみたい。	
\\	「かろう」 
\\	「い」 
\\	休日[きゅうじつ]であろうが、なかろうが、この 仕事[しごと]では 関係[かんけい]ないみたい。		
\\	これは不公平
\\	でしょうか。	これは不公平ではないでしょうか。	これはふこうへいではないでしょうか。	
\\	「である」	これは 不公平[ふこうへい]ではないでしょうか。		
\\	上司の話を聞くと、どうしても海外に出張を
\\	する]を得ないようです。	上司の話を聞くと、どうしても海外に出張をせざるを得ないようです。	じょうしのはなしをきくと、どうしてもかいがいにしゅっちょうをせざるをえないようです。	
\\	「~ざるを得ない」	上司[じょうし]の 話[はなし]を 聞[き]くと、どうしても 海外[かいがい]に 出張[しゅっちょう]をせざるを 得[え]ないようです。		
\\	このレストランは安いと思
\\	きや、会計は5千円以上だった!	このレストランは安いと思いきや、会計は5千円以上だった!	このレストランはやすいとおもいきや、かいけいは5せんえんいじょうだった!	
\\	「思いきや」 
\\	このレストランは 安[やす]いと 思[おも]いきや、 会計[かいけい]は 
\\	千[せん] 円[えん] 以上[いじょう]だった!		
\\	「食べてみよう」と言
\\	が早いか、口の中に放り込んだ。	「食べてみよう」と言ったが早いか、口の中に放り込んだ。	「たべてみよう」といったがはやいか、くちのなかにほうりこんだ。	
\\	「が早いか」 
\\	食[た]べてみよう」と 言[い]ったが 早[はや]いか、 口[くち]の 中[なか]に 放り込[ほうりこ]んだ。		
\\	封筒には写真が数枚入
\\	、手紙が添えられていた。	封筒には写真が数枚入っており、手紙が添えられていた。	ふうとうにはしゃしんがすうまいはいっており、てがみがそえられていた。	
\\	封筒[ふうとう]には 写真[しゃしん]が 数[すう] 枚[まい] 入[はい]っており、 手紙[てがみ]が 添[そ]えられていた。		
\\	このテレビがこれ以上壊れたら、新しいのを買
\\	を得ないな。	このテレビがこれ以上壊れたら、新しいのを買わざるを得ないな。	このテレビがこれいじょうこわれたら、あたらしいのをかわざるをえないな。	
\\	「~ざるを得ない」	このテレビがこれ 以上[いじょう] 壊[こわ]れたら、 新[あたら]しいのを 買[か]わざるを 得[え]ないな。		
\\	ずっと我慢してきたが、この状態だと歯医者さんに行
\\	を得ない。	ずっと我慢してきたが、この状態だと歯医者さんに行かざるを得ない。	ずっとがまんしてきたが、このじょうたいだとはいしゃさんにいかざるをえない。	
\\	「~ざるを得ない」	ずっと 我慢[がまん]してきたが、この 状態[じょうたい]だと 歯医者[はいしゃ]さんに 行[い]かざるを 得[え]ない。		
\\	この埃
\\	だらけのテレビをちゃんと拭いてくれない?	この埃だらけのテレビをちゃんと拭いてくれない?	このほこりだらけのテレビをちゃんとぬぐいてくれない?	
\\	「だらけ」 
\\	この 埃[ほこり]だらけのテレビをちゃんと 拭[ふ]いてくれない?		
\\	体によくないと思
\\	つつ、最近は全然運動してない。	体によくないと思いつつ、最近は全然運動してない。	からだによくないとおもいつつ、さいきんはぜんぜんうんどうしてない。	
\\	「~つつ」	体[からだ]によくないと 思[おも]いつつ、 最近[さいきん]は 全然[ぜんぜん] 運動[うんどう]してない。		
\\	今後もお客様との対話の窓口として、より充実して行
\\	べく 努力してまいります。	今後もお客様との対話の窓口として、より充実していくべく 努力してまいります。	こんごもおきゃくさまとのたいわのまどぐちとして、よりじゅうじつしていくべくどりょくしてまいります。	
\\	「べく」 
\\	今後[こんご]もお 客様[きゃくさま]との 対話[たいわ]の 窓口[まどぐち]として、より 充実[じゅうじつ]していくべく 努力[どりょく]してまいります。		
\\	二日酔いで痛む頭を押
\\	つつ、トイレに入った。	二日酔いで痛む頭を押さえつつ、トイレに入った。	ふつかよいでいたむあたまをおさえつつ、トイレにいっった。	
\\	「~つつ」	二日酔[ふつかよ]いで 痛[いた]む 頭[あたま]を 押[お]さえつつ、トイレに 入[はい]った。		
\\	この場ではちょっと決
\\	かねますので、また別途会議を設けましょう。	この場ではちょっと決めかねますので、また別途会議を設けましょう。	このばではちょっときめかねますので、またべっとかいぎをもうけましょう。	
\\	「~かねる」	この 場[ば]ではちょっと 決[き]めかねますので、また 別途[べっと] 会議[かいぎ]を 設[もう]けましょう。		
\\	そう簡単に直せ
\\	はずがないよ。	そう簡単に直せるはずがないよ。	そうかんたんになおせるはずがないよ。	
\\	「はず」 
\\	そう 簡単[かんたん]に 直[なお]せるはずがないよ。		
\\	宿題が多すぎて、トイレに行く時間
\\	さえなかった。	宿題が多すぎて、トイレに行く時間さえなかった。	しゅくだいがおおすぎて、トイレにいくじかんさえなかった。	
\\	「(で)さえ」 
\\	宿題[しゅくだい]が 多[おお]すぎて、トイレに 行[い]く 時間[じかん]さえなかった。		
\\	子供が掃除
\\	そばから散らかすから、もう諦めたくなった。	子供が掃除するそばから散らかすから、もう諦めたくなった。	こどもがそうじするそばからちらかすから、もうあきらめたくなった。	
\\	「そばから」 
\\	子供[こども]が 掃除[そうじ]するそばから 散[ち]らかすから、もう 諦[あきら]めたくなった。		
\\	テレビを見
\\	ながら、寝ちゃダメよ!	テレビを見ながら、寝ちゃダメよ!	テレビをみながら、ねちゃダメよ!	
\\	「~つつ」	テレビを 見[み]ながら、 寝[ね]ちゃダメよ!		
\\	肉を食
\\	ようにしている。	肉を食べないようにしている。	にくをたべないようにしている。	
\\	肉[にく]を 食[た]べないようにしている。		
\\	女の子と共通の話題ができて、自分の体も健康になる。いいことずくめですよ。	女の子と共通の話題ができて、自分の体も健康になる。いいことずくめですよ。	おんなのこときょうつうのわだいができて、じぶんのからだもけんこうになる。いいことずくめですよ。	
\\	「ずくめ」 
\\	女の子[おんなのこ]と 共通[きょうつう]の 話題[わだい]ができて、 自分[じぶん]の 体[からだ]も 健康[けんこう]になる。いいことずくめですよ。		
\\	今の日本では、雇用慣行が崩
\\	つつある。	今の日本では、雇用慣行が崩れつつある。	いまのにっぽんでは、こようかんこうがくずれつつある。	
\\	「~つつ」	今[いま]の 日本[にっぽん]では、 雇用[こよう] 慣行[かんこう]が 崩[くず]れつつある。		
\\	時間が
\\	ある 
\\	うが
\\	ある]まいが、間に合わせるしかない。	時間があろうがあるまいが、間に合わせるしかない。	じかんがあろうがあるまいが、まにあわせるしかない。	
\\	時間[じかん]があろうがあるまいが、 間に合[まにあ]わせるしかない。		
\\	その時までは決して彼に会
\\	まいと心に決めていた。	その時までは決して彼に会うまいと心に決めていた。	そのときまではけっしてかれにあうまいとこころにきめていた。	
\\	その 時[とき]までは 決[けっ]して 彼[かれ]に 会[あ]うまいと 心[こころ]に 決[き]めていた。		
\\	純物質が混じりあっている物質
\\	。	純物質が混じりあっている物質である。	こんごうぶつとは、2しゅるいいじょうのじゅんぶっしつがまじりあっているぶっしつである。	
\\	「である」 
\\	純[じゅん] 物質[ぶっしつ]が 混[ま]じりあっている 物質[ぶっしつ]である。		
\\	ボールは爆発
\\	する]ばかりに、膨らんでいた。	ボールは爆発せんばかりに、膨らんでいた。	ボールはばくはつせんばかりに、ふくらんでいた。	
\\	「ばかり」 
\\	ボールは 爆発[ばくはつ]せんばかりに、 膨[ふく]らんでいた。		
\\	搭乗のアナウンスが聞
\\	や否や、みんながゲートの方へ走り出した。	搭乗のアナウンスが聞こえるや否や、みんながゲートの方へ走り出した。	とうじょうのアナウンスがきこえるやいなや、みんながゲートのほうへはしりだした。	
\\	「や/や否や」 
\\	搭乗[とうじょう]のアナウンスが 聞[き]こえるや 否[いな]や、みんながゲートの 方[ほう]へ 走り出[はしりだ]した。		
\\	白
\\	ずくめ団体は去年ニューズになっていた。	白ずくめ団体は去年ニューズになっていた。	しろずくめだんたいはきょねんニューズになっていた。	
\\	「ずくめ」 
\\	白[しろ]ずくめ 団体[だんたい]は 去年[きょねん]ニューズになっていた。		
\\	携帯を2年間使ってたら、傷
\\	だらけになった。	携帯を2年間使ってたら、傷だらけになった。	けいたいを2ねんかんつかってたら、きずだらけになった。	
\\	「だらけ」 
\\	携帯[けいたい]を2 年間[ねんかん] 使[つか]ってたら、 傷[きず]だらけになった。		
\\	「食べてみよう」と言
\\	が早いか、口の中に放り込んだ。	「食べてみよう」と言うが早いか、口の中に放り込んだ。	「たべてみよう」というがはやいか、くちのなかにほうりこんだ。	
\\	「が早いか」 
\\	食[た]べてみよう」と 言[い]うが 早[はや]いか、 口[くち]の 中[なか]に 放り込[ほうりこ]んだ。		
\\	彼女は、教授の姿を見
\\	が早いか、教室から逃げ出した。	彼女は、教授の姿を見るが早いか、教室から逃げ出した。	かのじょは、きょうじゅのすがたをみるがはやいか、きょうしつからにげだした。	
\\	「が早いか」 
\\	彼女[かのじょ]は、 教授[きょうじゅ]の 姿[すがた]を 見[み]るが 早[はや]いか、 教室[きょうしつ]から 逃げ出[にげだ]した。		
\\	私の顔を見
\\	や否や、何か言おうとした。	私の顔を見るや否や、何か言おうとした。	わたしのかおをみるやいなや、なにかいおうとした。	
\\	「や/や否や」 
\\	私[わたし]の 顔[かお]を 見[み]るや 否[いな]や、 何[なに]か 言[い]おうとした。		
\\	吾輩は猫
\\	。	吾輩は猫である。	わがはいはねこである。	
\\	「である」 
\\	吾輩[わがはい]は 猫[ねこ]である。		
\\	どんな商品でもネットで販売するだけで売上が伸びるというものではなかろう。	どんな商品でもネットで販売するだけで売上が伸びるというものではなかろう。	どんなしょうひんでもネットではんばいするだけでうりあがのびるというものではなかろう。	
\\	「かろう」 
\\	「い」 
\\	どんな 商品[しょうひん]でもネットで 販売[はんばい]するだけで 売上[うりあげ]が 伸[の]びるというものではなかろう。		
\\	たった1キロを走っただけで、汗
\\	まみれになるのは情けない。	たった1キロを走っただけで、汗まみれになるのは情けない。	たった1キロをはしっただけで、あせまみれになるのはなさけない。	
\\	「まみれ」 
\\	たった1キロを 走[はし]っただけで、 汗[あせ]まみれになるのは 情[なさ]けない。		
\\	うちの子供はプールに入るのを理由もなく怖
\\	がる。	うちの子供はプールに入るのを理由もなく怖がる。	うちのこどもはプールにいるのをりゆうもなくこわがる。	
\\	「~がる」	うちの 子供[こども]はプールに 入[はい]るのを 理由[りゆう]もなく 怖[こわ]がる。		
\\	早くきてよ!何を恥
\\	がっているの?	早くきてよ!何を恥ずかしがっているの?	はやくきてよ!なにをはずかしがっているの?	
\\	「~がる」	早[はや]くきてよ! 何[なに]を 恥[は]ずかしがっているの?		
\\	試験に合格
\\	べく、皆一生懸命に勉強している。	試験に合格すべく、皆一生懸命に勉強している。	しけんにごうかくすべく、みないっしょうけんめいにべんきょうしている。	
\\	「べく」 
\\	試験[しけん]に 合格[ごうかく]すべく、 皆[みな] 一生懸命[いっしょうけんめい]に 勉強[べんきょう]している。		
\\	お金
\\	さえあれば、何でも出来るよ。	お金さえあれば、何でも出来るよ。	おきんさえあれば、なんでもできるよ。	
\\	「(で)さえ」 
\\	お 金[かね]さえあれば、 何[なに]でも 出来[でき]るよ。		
\\	自分の過ちを認
\\	さえしなければ、問題は解決しないよ。	自分の過ちを認めさえしなければ、問題は解決しないよ。	じぶんのあやまちをみとめさえしなければ、もんだいはかいけつしないよ。	
\\	「(で)さえ」 
\\	自分[じぶん]の 過[あやま]ちを 認[みと]めさえしなければ、 問題[もんだい]は 解決[かいけつ]しないよ。		
\\	結婚
\\	おろか、2ヶ月付き合って、結局別れてしまった。	結婚はおろか、2ヶ月付き合って、結局別れてしまった。	けっこんはおろか、2がつつきあって、けっきょくわかれてしまった。	
\\	「おろか」 
\\	結婚[けっこん]はおろか、 
\\	ヶ月[かげつ] 付き合[つきあ]って、 結局[けっきょく] 別[わか]れてしまった。		
\\	家に帰ったら、すぐパソコンを使
\\	がる。	家に帰ったら、すぐパソコンを使いたがる。	いえにかえったら、すぐパソコンをつかいたがる。	
\\	「~がる」	家[いえ]に 帰[かえ]ったら、すぐパソコンを 使[つか]いたがる。		
\\	明日に行くのを止
\\	う。	明日に行くのを止めよう。	あしたにいくのをやめよう。	
\\	明日[あした]に 行[い]くのを 止[と]めよう。		
\\	散歩
\\	がてら、タバコを買いに行きました。	散歩がてら、タバコを買いに行きました。	さんぽがてら、タバコをかいにいきました。	
\\	「~がてら」 
\\	散歩[さんぽ]がてら、タバコを 買[か]いに 行[い]きました。		
\\	彼は漫画マニアだから、これらをもう全部読
\\	はずだよ。	彼は漫画マニアだから、これらをもう全部読んだはずだよ。	かれはまんがマニアだから、これらをもうぜんぶよんだはずだよ。	
\\	「はず」 
\\	彼[かれ]は 漫画[まんが]マニアだから、これらをもう 全部[ぜんぶ] 読[よ]んだはずだよ。		
\\	例え国のためであっても、国民を騙
\\	べきではないと思う。	例え国のためであっても、国民を騙すべきではないと思う。	たとえくにのためであっても、こくみんをだますべきではないとおもう。	
\\	「べき」 
\\	例[たと]え 国[こく]のためであっても、 国民[こくみん]を 騙[だま]すべきではないと 思[おも]う。		
\\	この旅館は、様々な新しい設備が備
\\	、とても快適だった。	この旅館は、様々な新しい設備が備えており、とても快適だった。	このりょかんは、さまざまなあたらしいせつびがそなえており、とてもかいてきだった。	
\\	この 旅館[りょかん]は、 様々[さまざま]な 新[あたら]しい 設備[せつび]が 備[そな]えており、とても 快適[かいてき]だった。		
\\	ビタミンを食
\\	さえすれば、健康が保証されますよ。	ビタミンを食べさえすれば、健康が保証されますよ。	ビタミンをたべさえすれば、けんこうがほしょうされますよ。	
\\	「(で)さえ」 
\\	ビタミンを 食[た]べさえすれば、 健康[けんこう]が 保証[ほしょう]されますよ。		
\\	事情を2時間かけて説明を
\\	する]あげく、納得してもらえなかった	事情を2時間かけて説明をしたあげく、納得してもらえなかった。	じじょうを2じかんかけてせつめいをしたあげく、なっとくしてもらえなかった。	
\\	「~あげく(挙句)」 
\\	事情[じじょう]を 
\\	時間[じかん]かけて 説明[せつめい]をしたあげく、 納得[なっとく]してもらえなかった。		
\\	昼間だから絶対込んでいると思
\\	きや、一人もいなかった。	昼間だから絶対込んでいると思いきや、一人もいなかった。	ひるまだからぜったいこんでいるとおもいきや、ひとりもいなかった。	
\\	「思いきや」 
\\	昼間[ひるま]だから 絶対[ぜったい] 込[こ]んでいると 思[おも]いきや、一 人[にん]もいなかった。		
\\	このドキュメントは間違え
\\	だらけで、全然役に立たない。	このドキュメントは間違えだらけで、全然役に立たない。	このドキュメントはまちがえだらけで、ぜんぜんやくにたたない。	
\\	「だらけ」 
\\	このドキュメントは 間違[まちが]えだらけで、 全然[ぜんぜん] 役に立[やくにた]たない。		
\\	博物館を見
\\	がてらに、お土産を買うつもりです。	博物館を見がてらに、お土産を買うつもりです。	はくぶつかんをけんがてらに、おみやげをかうつもりです。	
\\	「~がてら」 
\\	博物館[はくぶつかん]を 見[み]がてらに、お 土産[みやげ]を 買[か]うつもりです。		
\\	新しい仕事が見つからないのでやむ
\\	を得ない状態です。	新しい仕事が見つからないのでやむを得ない状態です。	あたらしいしごとがみつからないのでやむをえないじょうたいです。	
\\	「やむを得ない」	新[あたら]しい 仕事[しごと]が 見[み]つからないのでやむを 得[え]ない 状態[じょうたい]です。		
\\	電気製品の発展につれて、ハードディスクの容量はますます大きく
\\	なる]つつある。	電気製品の発展につれて、ハードディスクの容量はますます大きくなりつつある。	でんきせいひんのはってんにつれて、ハードディスクのようりょうはますますおおきくなりつつある。	
\\	「~つつ」	電気[でんき] 製品[せいひん]の 発展[はってん]につれて、ハードディスクの 容量[ようりょう]はますます 大[おお]きくなりつつある。		
\\	私は寒
\\	がり屋だから、ミネソタで暮らすのは辛かった。	私は寒がり屋だから、ミネソタで暮らすのは辛かった。	わたしはさむがりやだから、ミネソタでくらすのはつらかった。	
\\	「~がる」	私[わたし]は 寒[さむ]がり 屋[や]だから、ミネソタで 暮[く]らすのは 辛[つら]かった。		
\\	確定申告は忘
\\	がちな手続のひとつだ。	確定申告は忘れがちな手続のひとつだ。	かくていしんこくはわすれがちなてつづきのひとつだ。	
\\	「~がち」	確定[かくてい] 申告[しんこく]は 忘[わす]れがちな 手続[てつづき]のひとつだ。		
\end{CJK}
\end{document}