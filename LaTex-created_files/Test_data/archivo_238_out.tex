\documentclass[8pt]{extreport} 
\usepackage{hyperref}
\usepackage{CJKutf8}
\begin{document}
\begin{CJK}{UTF8}{min}
\\	ごますり			
\\	言わぬが花			
\\	花に嵐		(あらし)	
\\	花より団子		だんご	
\\	芋(の子)を洗うよう		(いも)(あら)	
\\	転がる石に苔むさず		(ころ)(いし)(こけ)	
\\	実る程頭の下がる稲穂かな		(みの)(ほどあたま)(さ)(いなほ)	
\\	根回し		(ねまわ)	
\\	根も花もない		(ね)(はな)	
\\	桜			
\\	山椒は小粒でも辛い		(さんしょう)(こつぶ)(から)	
\\	高嶺の花		(たかね)(はな)	
\\	竹を割ったよう		(たけ)(わ)	
\\	瓜二つ		(うりふた)	
\\	寄らば大樹の陰		(よ)(たいじゅ)(かげ)	
\\	雨が降ろうと槍が降ろうと		(あめ)(ふ)(やり)(ふ)	
\\	雨降って地固まる		(あめふり)(ちかた)	
\\	風の便り		(かぜ)(たよ)	
\\	雲をつかむよう		(くも)	
\\	昔のことは水に流して		(みず)(なが)	
\\	水の泡		(みず)(あわ)	
\\	水を打ったよう		(みず)(う)	
\\	水商売		(みずしょうばい)	
\\	焼け石に水		(や)(いし)(みず)	
\\	足もとから鳥が立つ		(あし)(とり)(た)	
\\	羽根を伸ばす		(はね)(の)	
\\	借りてきた猫のよう		(か)(ねこ)	
\\	猫も杓子も		(ねこ)(しゃくし)	
\\	猫に小判		(ねこ)(こばん)	
\\	猫のひたい		(ねこ)	
\\	猫の手も借りたい		(ねこ)(て)(か)	
\\	同じ穴のむじな		(おな)(あな)	
\\	雀の涙		(すずめ)(なみだ)	
\\	立つ鳥後を濁さず		(た)(とりご)(にご)	
\\	鶴の一声		(つる)(ひとこえ)	
\\	馬の骨		(うま)(ほね)	
\\	ごまめの歯ぎしり		(は)	
\\	蜂の巣をつついたような騒がしい。		(はち)(す)	
\\	ひっぱりダコ			
\\	井の中の蛙(大海の知らず)		(い)(なか)(かわず)(たいかい)(し)	
\\	まな板の上の鯉		(いた)(うえ)(こい)	
\\	さばを読んでる		(よ)	
\\	蚊のなくような声		(か)(こえ)	
\\	虫の息		(むし)(いき)	
\\	泣き面に蜂		(な)(つら)(はち)	
\\	たで食う虫も好き好き		(く)(むし)すきずき	
\\	あばたもえくぼ			
\\	あぐらをかく			
\\	足もとを見る		(あし)	
\\	足を洗う			
\\	頭が下がる			
\\	合わせる顔がない			
\\	腹芸		(はらげい)	
\\	以心伝心		(いしんでんしん)	
\\	顔が広い		(かお)(ひろ)	
\\	肩身がせまい		(かたみ)	
\\	腰が低い		(こし)(ひく)	
\\	耳が痛い		(みみ)(いた)	
\\	良薬口に苦し		(りょうやくぐち)(にが)	
\\	心臓が強い		(しんぞう)(つよ)	
\\	白い目で見る		(め)(み)	
\\	面の皮が厚い		(つら)(かわ)(あつ)	
\\	腕をみがく		(うで)	
\\	後ろ髪を引かれる思い		(うし)(がみ)(ひ)(おも)	
\\	後ろ指をさされるよう		(うし)(ゆび)	
\\	石の上にも三年		(いし)(うえ)(さんねん)	
\\	十人十色		(じゅうにんといろ)	
\\	二枚舌を使う		(にまいじた)(つか)	
\\	二の足を踏む		(に)(あし)(ふ)	
\\	女三人寄れば姦しい		(おんなさんにんよ)(かしま)	
\\	三人寄れば文殊の知恵		(さんにんよ)(もんじゅ)(ちえ)	
\\	八方美人		(はっぽうびじん)	
\\	ちりも積もれば山となる		(つ)(やま)	
\\	ひと旗あげる		(はた)	
\\	ひとり相撲を取る		(ずもう)(と)	
\\	すし詰め (の電車)		(づ)	
\\	穴があったら入りたい		(あな)(はい)	
\\	石橋をたたいて渡る		(いしばし)(わた)	
\\	草分け		(くさわ)	
\\	おのぼりさん			
\\	住めば都		(す)(みやこ)	
\\	渡りに舟		(わた)(ふね)	
\\	朝飯前		(あさめしまえ)	
\\	ばかは死ななきゃ治らない		(し)(なお)	
\\	ちゃんぽん			
\\	出る杭は打たれる		(で)(くい)(う)	
\\	あんな箱入り娘をアメリカを出すなんて。。。		(はこい)(むすめ)	
\\	重箱の隅を(ようじで)つつく		(じゅうばこ)(すみ)	
\\	片棒をかつぐ		(かたぼう)	
\\	光陰矢のごとし		(こういんや)	
\\	どうせ腰かけで勤めたんだから。		(こし)	
\\	臭いものにはふたをする		(くさ)	
\\	のれんに腕押し		(うでお)	
\\	同じ釜の飯を食った		(おな)(かま)(めし)(く)	
\\	サジを投げる		(さじ)(な)	
\\	刺身のつま		(さしみ)	
\\	袖の下		(そで)(した)	
\\	賄賂を贈る		
\\	砂をかむよう		(すな)	
\\	太鼓判を押す		(たいこばん)(お)	
\\	有終の美を飾る		(ゆうしゅう)(び)(かざ)	
\\	窓際族		まどぎわぞく	
\\	手際-
\end{CJK}
\end{document}