\documentclass[8pt]{extreport} 
\usepackage{hyperref}
\usepackage{CJKutf8}
\begin{document}
\begin{CJK}{UTF8}{min}
\\	床の間 
\\	和室の床が一段高くなった部分
\\	障子(しょうじ) 
\\	ふすま 
\\	畳(たたみ)
\\	あぐらをかく
\\	*ひざを抱えて座る
\\	聞き入る
\\	聞き流す
\\	*聞き逃す
\\	聞こえる
\\	聞く
\\	腰掛ける [私はベンチに腰掛けたの。]
\\	耳が遠い
\\	耳につく
\\	耳を傾ける
\\	*耳を澄ます
\\	正座する
\\	背伸びする
\\	*座り込む
\\	爪先立ちをする
\\	突っ立つ
\\	横座り (よこずわり)
\\	ちょこんと
\\	*でんと
\\	*どっかり(と)
\\	がばっと
\\	へたへた(と)
\\	*むっくと
\\	ぺたんと
\\	しゃんと
\\	呆れる
\\	呆気に取られる
\\	びっくりする
\\	ぎょっとする
\\	仰天する
\\	息を飲む
\\	目を疑う
\\	脅かす (おびやかす)
\\	舌を巻く
\\	カンヒザクラ [寒緋桜] 
\\	しだれざくら [枝垂れ桜] 
\\	ソメイヨシノ [染井吉野] 
\\	オオシマザクラ [大島桜] 
\\	エドヒガン [江戸彼岸] 
\\	みみずく
\\	たか
\\	格子戸 [こうしど]
\\	はと (鳩) [公園の鳩にえさをやったの。]
\\	きのこ/キノコ [この山ではきのこが採れますよ]
\\	株 [かぶ]
\\	御節料理 [おせちりょうり]
\\	かかと [靴のかかとで彼の足を踏んでしまったの。]
\\	すずめ (雀) [朝は雀の声で目が覚めます。]
\\	バッタ
\\	目覚まし時計 (めざましどけい) [7時に目覚まし時計が鳴りました。]
\\	かま (釜)
\\	箸 (はし) [箸を上手に使えるよ。]
\\	扇子 (せんす)
\\	垣根 (かきね) [その家の周りは垣根がめぐらされている。]
\\	生垣 (いけがき)
\\	こけ
\\	しゃもじ
\\	神宮 (じんぐう)
\\	*キッチン 
\\	*台所 
\\	(だいどころ) [お母さんは台所にいます。] [彼女はキッチンで夕食を作っています。]
\\	電子レンジ (でんしレンジ)
\\	(お)茶わん [茶わんにご飯をよそいました。]
\\	ワイングラス
\\	コーヒーカップ
\\	湯飲み(茶わん) (ゆのみ) [湯飲みにはお茶を入れてお茶碗にはご飯をよそいます。]
\\	*ガスレンジ *ガスコンロ
\\	レバー [レバーを右に回すと締まります。]
\\	流し (ながし) [流しが臭うのできれいにしたよ。]
\\	歯医者 (はいしゃ) [私は歯医者が嫌いです。]
\\	聴診器 (ちょうしんき)
\\	冷蔵庫 (れいぞうこ) [飲み物は冷蔵庫にあります。]
\\	リビング 居間 (いま) [父は居間でテレビを見ている。]
\\	雨戸 
\\	窓ガラス (あまど)
\\	窓ガラス 
\\	網戸 
\\	雨戸
\\	網戸 (あみど)
\\	網棚 (あみだな)
\\	円盤 (えんばん)
\\	蛙 (かえる) [田んぼの中でかえるが合唱している。]
\\	くし [彼はくしで髪をとかした。]
\\	蛇口 (じゃぐち) [彼は水道の蛇口をひねったの。]
\\	蛇 (へび) [へびがカエルを捕まえたの。]
\\	椅子 (いす) [その椅子は変わった形をしている。]
\\	コード [コードが短くてコンセントに届きません。]
\\	コンセント [プラグをコンセントに挿し込んだよ。]
\\	電池 (でんち) [新しい電池を入れましょう。]
\\	エアコン [このエアコンはあまり電力を使いません。]
\\	ヒーター
\\	*じゅうたん *カーペット [じゅうたんを掃除しなければならない。] [私たちは床に新しいカーペットを敷いた。]
\\	スイッチ *スイッチを入れる *スイッチを切る [彼はカーラジオのスイッチを入れた。]
\\	*クーラー *冷房 くーらー 
\\	冷房 が 効いている [今年の夏はクーラーがよく売れた。] [冷房を入れてください。]
\\	ジャガイモ [ポテトサラダを作るから、ジャガイモを買ってきて。]
\\	玉ネギ (たまねぎ) [私は玉ねぎが嫌いです。]
\\	ねぎ (葱) [みそ汁にねぎを入れました。]
\\	おかず [晩ご飯のおかずは何?]
\\	調味料 (ちょうみりょう) [塩、コショウはよく使われる調味料です。]
\\	しょうゆ (醤油) [もう少し醤油を足してください。]
\\	サラダ油
\\	島 (しま) [この島は観光で成り立っています。]
\\	おおさじ こさじ スプーン
\\	さじ (匙)
\\	包丁 (ほうちょう) [この包丁は切れが良いな。]
\\	はかり (秤) [はかりで小麦粉の分量を量ったよ。]
\\	天ぷら油 (てんぷらあぶら)
\\	酢 (す) [お酢を入れすぎて酸っぱい。]
\\	ラップ
\\	なべ (鍋) [鍋にスープが入っているよ。]
\\	ふた (蓋)
\\	おたま (御玉)
\\	フライパン [フライパンで目玉焼きを作ったよ。]
\\	灰皿 (はいざら) [灰皿をください。]
\\	エプロン *エプロンをする/かける/つける [エプロンをしないと服が汚れます。]
\\	ほうき (箒) [ほうきで庭を掃除しました。]
\\	ぞうきん (雑巾) *ぞうきんをぬらす *ぞうきんをしぼる
\\	バケツ [小さなバケツで水を汲んだの。]
\\	靴下 (くつした) [この靴下は3足で1000円です。]
\\	掃除機 (そうじき) *掃除機をかける [掃除機が壊れた。]
\\	鶏 (にわとり) [その鶏は毎朝鳴きます。]
\\	回数券 (かいすうけん) [バスの回数券を買いました。]
\\	定期券 (ていきけん) [定期券は1万2千円でした。]
\\	特急券 (とっきゅうけん)
\\	乗車券 (じょうしゃけん)
\\	改札口 (かいさつぐち) [改札口で会いましょう。]
\\	かし (樫) 
\\	[~の木]
\\	運転免許証 (うんてんめんきょしょう)
\\	トランク *トランクに荷物を積む
\\	シートベルト *シートベルトをしめる
\\	エンジン *エンジンをかける [車のエンジンが調子悪い。]
\\	バックミラー *バックミラーで確認をする
\\	ハンドル *ハンドルをにぎる *ハンドルを右に切る [この車は左ハンドルです。]
\\	ワイパー *ワイパーを動かす
\\	フロントガラス
\\	サイドミラー
\\	ライト *ライトをつける
\\	タイヤ [タイヤを交換してください。]
\\	高速道路 *高速道路に乗る [高速道路は混んでたよ。]
\\	線路 (せんろ) *線路を越える [子猫が線路に迷い込んだぞ。]
\\	渋滞 (じゅうたい) [今日、交通渋滞にはまってしまった。]
\\	突き当たり (つきあたり) *突き当たりを右に曲がる [この先は突き当たりです。]
\\	あなぐま (穴熊)
\\	たぬき (狸)
\\	がけ (崖) [彼らは垂直のがけを登り始めたの。]
\\	コンビニ [コンビニは至る所にあります。]
\\	山里 (やまざと)
\\	とらばさみ [~をしかける]
\\	口紅 (くちべに) *口紅を塗る/つける [赤い口紅を買いました。]
\\	長靴 (ながぐつ)
\\	むち (鞭)
\\	包帯 (ほうたい) [彼女は手に包帯を巻いていたんだ。]
\\	おり (檻)
\\	ゴムまり (ゴム毬)
\\	たんす (箪笥) [昨日、たんすを買いました。]
\\	みみず (蚯蚓)
\\	松明 (たいまつ)
\\	ひいらぎ (柊)
\\	七面鳥 (しちめんちょう)
\\	*判子 *印鑑 [ここに判子を押してください。] [ここに受け取りの印鑑をお願いします。]
\\	鶴 (つる) [湖にツルがいますよ。]
\\	浄水器 (じょうすいき)
\\	パンフレット
\\	蛍光灯 (けいこうとう) [古い蛍光灯を取り替えてください。]
\\	電球 (でんきゅう) [電球が切れたので交換しましょう。]
\\	蛍 (ほたる)
\\	硬貨 (こうか)
\\	*紙幣(しへい) *(お)札
\\	シール [机にシールを貼った。]
\\	溝 (みぞ) [車が溝にはまったの。]
\\	つり革 (つりかわ)
\\	入れ物 *プラスチックの入れ物 [荷物が多いので大きな入れ物が必要です。]
\\	作業服 (さぎょうふく)
\\	総菜 (そうざい)
\\	調理パン (ちょうりパン)
\\	食肉 (しょくにく)
\\	通帳 (つうちょう) [銀行の通帳を持ってきてください。]
\\	ネックレス [彼女は金のネックレスをしています。]
\\	*イヤリング *ピアス [彼女はすてきなイヤリングをしている。]
\\	歯磨き粉 (はみがきこ)
\\	しま (縞)
\\	足袋 (たび)
\\	浴衣 (ゆかた)
\\	望遠鏡 (ぼうえんきょう)
\\	鉄砲 (てっぽう)
\\	電卓 (でんたく) [この電卓はポケットに入ります。]
\\	風呂敷 (ふろしき)
\\	瀬戸物 (せともの)
\\	座敷 (ざしき) [明日はお座敷での宴会になります。]
\\	太鼓 (たいこ)
\\	請求書 (せいきゅうしょ)
\\	*物差し *定規 (じょうぎ) [30センチの物差しをください。]
\\	井戸 (いど)
\\	ボルト
\\	鉢巻き (はちまき)
\\	はっぴ
\\	だんじり
\\	万年筆 (まんねんひつ) [父に万年筆をプレゼントしました。]
\\	あめ (飴) [缶に飴が入っています。]
\\	いか [私はイカの刺し身が大好きです。]
\\	稲 (いね) [日本は稲の品種が多いよ。]
\\	うどん (饂飩) 
\\	小麦粉に少量の塩を加え、水でこね、薄く延ばして細く切ったものをゆでた食品。
\\	[私はお昼にうどんを食べました。]
\\	温室 (おんしつ)
\\	温度計 (おんどけい)
\\	蚊 (か) [蚊に足を刺された。]
\\	聞き違える
\\	柳 (やなぎ)
\\	羊 (ひつじ)
\\	金槌 (かなづち)
\\	かみそり (剃刀) [かみそりの刃で指を切りました。]
\\	かゆ (粥)
\\	かるた
\\	かわら (瓦)
\\	乾電池 (かんでんち)
\\	機関車 (きかんしゃ)
\\	金閣寺 (きんかくじ)
\\	競馬 (けいば)
\\	顕微鏡 (けんびきょう)
\\	碁 (ご)
\\	琴 (こと)
\\	酒場 (さかば)
\\	蜂 (はち) [友達が蜂に刺されたんだ。]
\\	図鑑 (ずかん)
\\	鈴 (すず)
\\	栓抜き (せんぬき) [栓抜きはどこですか。]
\\	洗面所 (せんめんじょ) [洗面所で顔を洗った。]
\\	草履 (ぞうり)
\\	そば (蕎麦) [お昼にそばを食べた。]
\\	そろばん
\\	ちりがみ (塵紙)
\\	デッサン
\\	てぬぐい (手拭い)
\\	伝言板 (でんごんばん)
\\	テント [みんなでテントを張りました。]
\\	灯台 (とうだい)
\\	*寝巻 *寝間着 (ねまき)
\\	のこぎり (鋸)
\\	はす (蓮)
\\	*ばね *バネ
\\	パンツ [彼女はスカートよりパンツが似合うね。]
\\	ファスナー
\\	ブローチ
\\	へそ [カエルにはへそがないんだ。]
\\	穂 (ほ)
\\	ほり (堀・濠)
\\	ミシン
\\	みつ (蜜)
\\	松 (まつ)
\\	もも (腿)
\\	ヨット [ヨットでクルージングを楽しみましたわ。]
\\	ラケット [テニスのラケットを買いました。]
\\	れんが (煉瓦)
\\	ろうそく [バースデーケーキにろうそくを立てました。]
\\	綿 (わた)
\\	椀 (わん) [その椀は地元の職人によって作られた。]
\\	碗 (わん)
\\	弓 (ゆみ)
\\	堤 (つつみ)
\\	バイオリン [彼女はバイオリンを習っています。]
\\	ガーゼ
\\	バッター
\\	ぬいぐるみ (縫い包み)
\\	ストロー
\\	ガードレール
\\	オーブン
\\	まんじゅう (饅頭) [彼はおまんじゅうを一口で食べたの。]
\\	ポリバケツ
\\	つばめ [つばめが飛んでいますね。]
\\	豚カツ (とんかつ) [昼食にとんかつを食べました。]
\\	はえ [はえがうるさく飛んでいるね。]
\\	ぎょうざ [ここのぎょうざは美味しいよ。]
\\	コタツ [寒いのでこたつを出しました。]
\\	カボチャ [かぼちゃのスープは美味しいです。]
\\	きゅうり (胡瓜) [きゅうりに味噌をつけて食べたの。]
\\	梅干 (うめぼし) [うちでは、朝食には必ず梅干しが出ます。]
\\	キャベツ [キャベツの千切りを添えたよ。]
\\	車椅子 [彼女は車椅子に乗っています。]
\\	てんぷら [私はえびのてんぷらが好きです。]
\\	渦 (うず)
\\	ドラム缶 (どらむかん)
\\	モルモット
\\	振りそで
\\	テトラポッド
\\	瀬戸内海 (せとないかい)
\\	乳母車 (うばぐるま)
\\	蚊帳 (かや)
\\	壷 (つぼ)
\\	うちわ [父はうちわで扇いでいるの。]
\\	ちり取り
\\	花びら
\\	げた
\\	ねじ回し
\\	おむつ 
\\	ストロボ 
\\	そり 
\\	担架 (たんか) 
\\	ちょう (蝶)
\\	手錠 (てじょう) 
\\	風車 (ふうしゃ) 
\\	ブーツ
\\	ワラ (藁) 
\\	しか (鹿)
\\	のみ (鑿) 
\\	恐竜 (きょうりゅう) 
\\	青魚 (あおざかな) 
\\	トビウオ 
\\	[トビウオのジャンプは沖天の勢いがあった。]
\\	炊飯器 (すいはんき) [新製品の炊飯器を買った。]
\\	あぶみ (鐙) [鐙は、馬具の一つ。乗馬で用いる。]
\\	飛行艇 (ひこうてい) [飛行艇を発進させる。]
\\	いなりずし (稲荷寿司) [いなりずしは俵の形をしている。]
\\	唐草模様 (からくさもよう) [仏像の台に唐草模様が彫られている。]
\\	マグカップ 
\\	[このマグカップは陶製だ。]
\\	肘掛け椅子 (ひじかけいす)
\\	マット 
\\	[玄関にマットを敷く。]
\\	風鈴 (ふうりん) 
\\	裸体画 (らたいが)
\\	きりん (麒麟) [きりんの首は長い。]
\\	水差し (みずさし)
\\	体重計 (たいじゅうけい)
\\	道具箱 (どうぐばこ)
\\	紙やすり
\\	ポケットナイフ
\\	水準器 (すいじゅんき)
\\	かんな (鉋)
\\	電気ドリル
\\	万力 (まんりき)
\\	やすり (鑢) 
\\	手おの (ておの)
\\	レンチ
\\	糸のこぎり (糸鋸) 
\\	ドライバー
\\	ナット
\\	座金 (ざがね) ワッシャー
\\	スパナ 
\\	ニンニク 
\\	ピーマン
\\	カリフラワー 
\\	アスパラガス 
\\	ハツカダイコン (二十日大根) 
\\	レタス 
\\	ビートルート 
\\	ズッキーニ 
\\	クレソン 
\\	芽キャベツ (めキャベツ) 
\\	セロリ 
\\	ブロッコリー 
\\	カブラ 
\\	いちご (苺)
\\	ピーナッツ
\\	ココナツ ココナッツ
\\	パイナップル
\\	ブドウ ぶどう
\\	桃 (もも)
\\	アボカド
\\	パパイヤ 
\\	ライチ 
\\	洋ナシ 洋梨 (ようなし) 
\\	ライム 
\\	マンゴー
\\	グレープフルーツ 
\\	ホチキス
\\	パンチ
\\	クリップ
\\	三角定規 (さんかくじょうぎ) 
\\	分度器 (ぶんどき) 
\\	鋏 (はさみ)
\\	鉛筆削り (えんぴつけずり) 
\\	標識 (ひょうしき) [その標識によると、ここでは
\\	ターンできない。]
\\	郵便ポスト (ゆうびんポスト)
\\	マンホールのふた
\\	バス待合所
\\	街灯 (がいとう)
\\	街路名標識
\\	交通信号 (こうつうしんごう)
\\	ロータリー 
\\	タンカー
\\	ワゴン車 (ワゴンしゃ)
\\	セメント車 
\\	小型トラック (こがたトラック) 
\\	フォークリフト 
\\	セダン 
\\	オープンカー 
\\	ステーションワゴン 
\\	ハッチバック 
\\	ガソリンスタンド
\\	さいころ 
\\	コウモリ (蝙蝠) 
\\	ハリネズミ (針鼠) 
\\	リス 
\\	きつね (狐)
\\	山羊 (ヤギ) 
\\	ロバ 
\\	おおかみ (狼)
\\	ラクダ 
\\	ゴリラ 
\\	シマウマ 
\\	トラ (虎)
\\	ヒョウ 
\\	サイ 
\\	カバ 
\\	アザラシ 
\\	イルカ (海豚) 
\\	サメ 
\\	クラゲ 
\\	イセエビ 
\\	カニ 
\\	ヒトデ 
\\	トカゲ 
\\	ワニ 
\\	トンボ 
\\	クモ 
\\	アリ
\\	ゴキブリ 
\\	カマキリ 
\\	カタツムリ 
\\	サソリ 
\\	ヒヨコ
\\	めんどり (雌鶏,雌鳥)
\\	ワシ 
\\	カラス 
\\	ハチドリ 
\\	おうむ オオム (鸚鵡) 
\\	ダチョウ 
\\	ペンギン 
\\	クジャク 
\\	フラミンゴ 
\\	ガチョウ 
\\	アヒル 
\\	カモメ 
\\	ハクチョウ 
\\	言葉の壁 [言葉の壁を乗り越えるのは並大抵のものではありません。]
\\	ヤミ市場 [最近のヤミ市場でドルは、1ドル当たり21,400ドン程度で取引されている。]
\\	ファーストインプレッション 
\\	寝惚け眼 (ねぼけまなこ) *寝ぼけ眼をこする
\\	直射日光 (ちょくしゃにっこう)
\\	けいか (螢火)
\\	怪光 (かいこう)
\\	こもれび (木洩れ陽) (木洩れ日)
\\	薄光 (はっこう)
\\	かっちゅう (甲冑)
\\	月光 (げっこう)
\\	清い月影 (きよいつきかげ)
\\	ブルゾン 
\\	ブラジャー 
\\	車両部品
\\	ウエストコート 
\\	タンクトップ 
\\	アノラック 
\\	プラカード 
\\	熱帯低気圧 (ねったいていきあつ) 
\\	はめ倒す (はめたおす)
\\	会飲 (かいいん)
\\	居候 (いそうろう)
\\	絡み付く (からみつく)
\\	麻薬 (まやく)
\\	悪弊 (あくへい)
\\	タイ
\\	アジ
\\	サヨリ
\\	カツオ 
\\	カステラ 
\\	乾パン (かんパン)
\\	ムーンケーキ
\\	ウォームギア 
\\	キャスター 
\\	キャタピラー 
\\	立方体 (りっぽうたい) 
\\	円柱 (えんちゅう) 
\\	円錐 (えんすい) 
\\	角錐 (かくすい) 
\\	水鉄砲 (みずでっぽう) 
\\	水中眼鏡 (すいちゅうめがね) 
\\	回りどうろう
\\	チェック
\\	切り花 (きりばな) [切り花でしたら こちらの花なんか 明るくていいと思いますけど]
\\	パラボラアンテナ 
\\	懐中電灯 (かいちゅうでんとう)
\\	まな板 (まないた) [魚を切った後、まな板を洗ったの。]
\\	泰山木 (たいさんぼく)
\end{CJK}
\end{document}