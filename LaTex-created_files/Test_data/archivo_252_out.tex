\documentclass[8pt]{extreport} 
\usepackage{hyperref}
\usepackage{CJKutf8}
\begin{document}
\begin{CJK}{UTF8}{min}
\\	「ナオミは自分の完璧なスタイルをひけらかしたがる。私も彼女みたいに痩せていればなあ。彼女が羨ましいわ。」 「ダイエットすれば。」	
\\	体形(スタイル) 
\\	ダイエットを始める(中である)
\\	ナオミに一目ぼれした。3か月後、思い切って彼女にプロポーズしてみたが振られてしまった。	
\\	~に恋をする
\\	思い切ってやってみる 
\\	~のお願い事をする
\\	偶然にその人に出会ったの。優しくて頭もいいの。おまけに独身!	
\\	彼の甥は、謙虚で思いやりのある人になるよう育てられた。	
\\	甥 
\\	~を育てる 
\\	謙虚な
\\	人は外見ではなく、むしろ中身で判断されるべきだ。	
\\	というよりむしろ
\\	の外見、見た目 
\\	本当の自分、現在の自分
\\	「大切なのは肩書なんかじゃない。絶対そうだ。」とボブは語気を強めて言った。	
\\	こそが~だ 
\\	重要である
\\	自分の力で成功をつかんだ人が言ったように、「先見の明のある人は最後には成功する。」	
\\	~が言うように 
\\	自らの努力で成功した 
\\	を持つ 
\\	成功する
\\	若さが永遠のものではないことを忘れてはいけない。	
\\	永遠の
\\	いったいなぜみんながそれ程までにボブを軽蔑するのか、僕にはわからない。	
\\	軽蔑
\\	彼のことを笑わずにはいられない。ばかな間違いを繰り返すからね。きっと彼は懲りないんだろうな。	
\\	~をバカにして笑う 
\\	最も可能性の低い 
\\	教訓を得る、懲りる
\\	さらに厄介なことに、彼は私たちに迷惑をかけていることさえわかっていない。	
\\	さらに悪いことに 
\\	~を自覚している
\\	慌てていたから、手袋を間違えて裏返しにはめてしまった。	
\\	~なので 
\\	慌てている 
\\	裏返しに 
\\	間違えて、誤って
\\	デイブはデブでドジ。浅い小川を飛び越えたときに、つまずいて足首を捻挫したんだ。	
\\	不器用な、ドジな 
\\	~を捻挫する
\\	「僕、高所恐怖症なんだ。」 「弱虫っ!」	
\\	臆病者、弱虫
\\	消極的なその男性が、人前で自己表現することはまずない。	
\\	消極的な 
\\	自分を表現する
\\	迷信はすべて根拠のないものだということが、ニックに言われてよく分かった。	
\\	~を納得させる
\\	注射のことを考えると、怖くて震えてしまう。	
\\	~のことを考えると 
\\	注射
\\	ボブは根が怠け者だから、与えられた仕事をしばしば怠る。	
\\	本当は
\\	ボブが働いている間、ジェニファーは家でくだらないメロドラマに夢中になっていた。	
\\	連続メロドラマ
\\	ボブはマンガ好きだけど、僕にはそれほどいいものとは思えない。実際、つまらないよ。	
\\	を~だと評価する 
\\	実際には
\end{CJK}
\end{document}