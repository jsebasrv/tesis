\documentclass[8pt]{extreport} 
\usepackage{hyperref}
\usepackage{CJKutf8}
\begin{document}
\begin{CJK}{UTF8}{min}
\\	電子レンジ	
\\	でんし
\\	湯飲み(茶わん)	
\\	ゆのみ ちゃわん
\\	窓ガラス	
\\	まど
\\	雨戸	
\\	あまど
\\	天井	
\\	てんじょう
\\	水道の蛇口をひねる	
\\	すいどうのじゃぐちをひねる
\\	水が氷る	
\\	みずがこおる
\\	冷凍して保存する	
\\	れいとうしてほぞんする
\\	残り物を温める	
\\	のこりものをあたためる
\\	ビールを冷やす	
\\	ひやす
\\	冷えている	
\\	ひえている
\\	電源を入れる	
\\	でんげん
\\	じゅうたんを敷く	
\\	しく
\\	冷房が効いている	
\\	れいぼうがきいている
\\	暖房をつける	
\\	だんぼう
\\	日当たり	
\\	ひあたり
\\	おかず	
\\	栄養	
\\	えいよう
\\	量る	
\\	はかる
\\	調味料	
\\	ちょうみりょう
\\	酢	
\\	す
\\	少々	
\\	しょうしょう
\\	皮を剥く	
\\	かわをむく
\\	材料を刻む	
\\	ざいりょうをきざむ
\\	大きめに切る	
\\	味噌汁がぬるくなる	
\\	みそしる
\\	かぶせる	
\\	ラップでくるむ	
\\	大さじ	
\\	おおさじ
\\	包丁	
\\	ほうちょう
\\	まな板	
\\	まないた
\\	幅	
\\	はば
\\	ふた	
\\	炊飯器	
\\	すいはんき
\\	しゃもじ	
\\	お湯を沸かす	
\\	おゆをわかす
\\	お湯を冷ます	
\\	おゆをさます
\\	ご飯を炊く	
\\	ごはんをたく
\\	油を熱する	
\\	あぶらをねっする
\\	茹でる	
\\	ゆでる
\\	炒める	
\\	いためる
\\	蒸す	
\\	むす
\\	混ぜる	
\\	まぜる
\\	焦げる	
\\	こげる
\\	ひっくり返す/裏返す	
\\	ひっくりかえす・うらがえす
\\	加える	
\\	くわ
\\	胡椒を振る	
\\	こしょうをふる
\\	味見をする・味を見る	
\\	あぎみをする・あじをみる
\\	できあがる	
\\	フライ返し	
\\	ふらいがえし
\\	掃除機	
\\	そうじき
\\	ほうき	
\\	ちりとり	
\\	ぞうきん	
\\	どける	
\\	部屋を散らかす	
\\	へやをちらかす
\\	片付ける	
\\	かたづけ
\\	こぼす	
\\	こぼれる	
\\	ほこりがたまる	
\\	雪が積もる	
\\	ゆきがつもる
\\	掃く	
\\	はく
\\	床を拭く	
\\	ゆかをふく
\\	家具を磨く	
\\	かぐをみがく
\\	ブラシで擦る	
\\	ぶらしでこする
\\	絞る	
\\	しぼる
\\	濡らす	
\\	ぬらす
\\	燃えるゴミ	
\\	もえるごみ
\\	面倒くさい	
\\	めんどうくさい
\\	物置にしまう	
\\	ものおきにしまう
\\	揃える	
\\	そろえる
\\	洗濯物がたまる	
\\	せんたくもの
\\	濡れている	
\\	ぬれている
\\	乾く	
\\	かわく
\\	湿っている	
\\	しめっている
\\	干す	
\\	ほす
\\	乾燥機で乾かす	
\\	かんそうきでかわかす
\\	真っ白に仕上がる	
\\	まっしろにしあがる
\\	たたむ	
\\	服を汚す	
\\	ふくをよごす
\\	そで	
\\	洗剤を入れる	
\\	せんざいをいれる
\\	服をつるす	
\\	ふく
\\	しわ伸ばす	
\\	のばす
\\	色が落ちる	
\\	いろがおちる
\\	湿っている	
\\	可愛らしい	
\\	かわいらしい
\\	行儀がいい	
\\	ぎょうぎ
\\	利口な・かしこい	
\\	りこう
\\	人懐っこい	
\\	ひとなつっこい
\\	生意気	
\\	なまいき
\\	甘やかす	
\\	あまやかす
\\	かわいがる	
\\	おんぶする	
\\	抱っこする	
\\	だっこ
\\	吠える	
\\	ほえる
\\	しっぽを振る	
\\	ふる
\\	人になれる	
\\	ひと
\\	唸る	
\\	うなる
\\	穴を掘る	
\\	あなをほる
\\	穴を埋める	
\\	あなをうめる
\\	幼児	
\\	ようじ
\\	わがままに育つ	
\\	そだつ
\\	頭をなでる	
\\	あたま
\\	お尻をたたく・ぶつ	
\\	おしり
\\	いたずらをする	
\\	放す	
\\	はなす
\\	迷子になる	
\\	まいご
\\	母親の後をついていく	
\\	ははおやのあと
\\	犬を飼う	
\\	いぬをかう
\\	散歩に連れていく	
\\	さんぽにつれていく
\\	犬にえさをやる・与える	
\\	いぬ。あたえる
\\	平日	
\\	へいじつ
\\	祝日・祭日	
\\	しゅくじつーさいじつ
\\	一昨年	
\\	さくねん・おととし
\\	再来年	
\\	さらいねん
\\	先々月	
\\	せんせんげつ
\\	先々週	
\\	せんせんしゅう
\\	再来週	
\\	さらいしゅう
\\	先日	
\\	せんじつ
\\	暮れ	
\\	くれ
\\	休暇を取る	
\\	きゅうかをとる
\\	予定	
\\	よてい
\\	日にち	
\\	ひにち
\\	日程	
\\	にってい
\\	調整する	
\\	ちょうせい
\\	変更する	
\\	へんこう
\\	ずらす	
\\	旅行会社に申し込む	
\\	りょこうがいしゃにもうしこむ
\\	予約を取り消す	
\\	よやくをとりけす
\\	荷物を詰める	
\\	にもつをつめる
\\	海外旅行	
\\	かいがいりょこう
\\	国内旅行	
\\	こくないりょこう
\\	二泊三日	
\\	にはくみっか
\\	温泉	
\\	おんせん
\\	旅館	
\\	りょかん
\\	旅館に宿泊する	
\\	りょかんにしゅくはく
\\	列車の時刻を調べる	
\\	れっしゃのじこくをしらべる
\\	時刻表	
\\	じこくひょう
\\	東京発横浜行き	
\\	とうきょうはつよこはまいき
\\	私鉄	
\\	してつ
\\	新幹線	
\\	しんかんせん
\\	座席を指定する	
\\	ざせきをしていする
\\	指定席	
\\	していせき
\\	窓側	
\\	まどがわ
\\	自由席	
\\	じゆうせき
\\	通路側	
\\	つうろがわ
\\	乗車券	
\\	じょうしゃけん
\\	片道切符	
\\	かたみちきっぷ
\\	特急券	
\\	とっきゅうけん
\\	回数券	
\\	かいすうけん
\\	定期券	
\\	ていきけん
\\	往復切符	
\\	おうふくきっぷ
\\	特急料金	
\\	とっきゅうりょうきん
\\	改札口を通る	
\\	かいさつぐちをとおる
\\	切符を支払い戻す	
\\	きっぷをしはらいもどす
\\	支払い戻す	
\\	しはらいもどす
\\	混雑している	
\\	こんざつ
\\	白線の内側に下がる	
\\	はくせんのうちがわにさがる
\\	外側	
\\	そとがわ
\\	発車する	
\\	はっしゃ
\\	乗車する	
\\	じょうしゃ
\\	下車する	
\\	げしゃ
\\	電車の中はがらがらだ	
\\	でんしゃのなか
\\	空っぽ	
\\	からっぽ
\\	満員電車	
\\	まんいんでんしゃ
\\	通勤する	
\\	つうきん
\\	電車が揺れる	
\\	でんしゃがゆれる
\\	始発電車	
\\	しはつでんしゃ
\\	終電	
\\	しゅうでん
\\	始発駅から乗る	
\\	しはつえきからのる
\\	終点で降りる	
\\	しゅうてんでおりる
\\	急行	
\\	きゅうこう
\\	各駅停車	
\\	かくえきていしゃ
\\	乗り換え	
\\	のりかえ
\\	乗り遅れる	
\\	のりおくれる
\\	乗り過ごす	
\\	のりすごす
\\	乗り越す	
\\	のりこす
\\	乗り越し	
\\	のりこし
\\	運賃を精算する	
\\	うんちんをせいさん
\\	車掌	
\\	しゃしょう
\\	車内	
\\	しゃない
\\	携帯電話	
\\	けいたいでんわ
\\	使用を控える	
\\	しようをひかえる
\\	年寄り・老人	
\\	としより・ろうじん
\\	席を譲る	
\\	せきをゆずる
\\	体の不自由な人	
\\	からだのふじゆうなひと
\\	ちかんにあう	
\\	駅の売店	
\\	えきのばいてん
\\	優先席	
\\	ゆうせんせき
\\	運転免許証	
\\	うんてんめんきょしょう
\\	レンタカーを借りる	
\\	かりる
\\	トランクに荷物を積む	
\\	にもつをつむ
\\	人を乗せる	
\\	ひとをのせる
\\	シートベルトをしめる	
\\	エンジンをかける	
\\	確認する	
\\	かくにん
\\	ハンドルを握る	
\\	にぎる
\\	アクセルを踏む	
\\	ふむ
\\	ブレーキをかける	
\\	急ブレーキ	
\\	きゅう
\\	ライトをつける	
\\	ワイパーを動かす	
\\	うごかす
\\	ガソリンスタンドによる	
\\	高速道路	
\\	こうそくどうろ
\\	タイヤがパンクする	
\\	道路が混んでいる	
\\	どうろがこんでいる
\\	道路が渋滞している	
\\	どうろがじゅうたい
\\	すいている	
\\	回り道	
\\	まわりみち
\\	遠回り	
\\	とおまわり
\\	近道	
\\	ちかみち
\\	突き当りを左に曲がる	
\\	つきあたりをひだりにまがる
\\	バスが停車する	
\\	ていしゃ
\\	線路を越える	
\\	せんろをこえる
\\	トラックを追い越す	
\\	おいこす
\\	時速100キロで走る	
\\	じそく。はしる
\\	スピード違反で捕まる	
\\	いはんでつかまる
\\	駐車違反をする	
\\	ちゅうしゃいはん
\\	信号を無視する	
\\	しんごうをむし
\\	車をぶつける	
\\	くるま
\\	猫をひく	
\\	ねこ
\\	交通事故にあう	
\\	こうつうじこ
\\	車に気を付ける	
\\	くるまにきをつける
\\	ななめに横断する	
\\	おうだん
\\	道路を横切る	
\\	どうろをよこぎる
\\	お金を貯める	
\\	おかねをためる
\\	貯金する	
\\	ちょきん
\\	預金する	
\\	よきん
\\	給料	
\\	きゅうりょう
\\	引き出す・下ろす	
\\	ひきだす・おろす
\\	預ける	
\\	あずける
\\	通帳記入をする	
\\	つうちょうきにゅう
\\	暗証番号	
\\	あんしょうばんごう
\\	銀行の本店	
\\	ぎんこうのほんてん
\\	支店	
\\	してん
\\	口座を開く	
\\	こうざをひらく
\\	家賃を振り込む	
\\	やちんをふりこむ
\\	支払い	
\\	しはらい
\\	公共料金	
\\	こうきょうりょうきん
\\	商品	
\\	しょうひん
\\	代金	
\\	だいきん
\\	宛名	
\\	あてな
\\	窓口で両替をする	
\\	まどぐちでりょうがえ
\\	差出人名	
\\	さしだしにんめい
\\	送金する	
\\	そうきん
\\	手紙を速達出だす	
\\	てがみをそくたつでだす
\\	仕送りする	
\\	しおくり
\\	手数料がかかる	
\\	てすうりょう
\\	送料がかかる	
\\	そうりょう
\\	交際する	
\\	こうさい
\\	付き合う	
\\	つきあう
\\	仲直りする	
\\	なかなおり
\\	仲がいい	
\\	なか
\\	仲良し	
\\	なかよし
\\	コンサートに誘う	
\\	さそう
\\	嫌がる	
\\	いやがる
\\	ふる	
\\	気になる	
\\	き
\\	嫌いになる	
\\	きらい
\\	羨ましい	
\\	うらやましい
\\	好かれる	
\\	す
\\	嫌われる	
\\	きら
\\	嫌がられる	
\\	いや
\\	気がある	
\\	き
\\	あこがれる	
\\	夢中になる	
\\	むちゅう
\\	恋をする	
\\	こい
\\	女性	
\\	じょせい
\\	お互いに気に入る	
\\	おたがいにきにいる
\\	都合をつける	
\\	つごう
\\	都合をつく	
\\	つごう
\\	付き合い	
\\	つきあい
\\	買い物に付き合う	
\\	かいものにつきあう
\\	恋人	
\\	こいびと
\\	彼氏	
\\	かれし
\\	彼女	
\\	かのじょ
\\	結婚を申し込む	
\\	けっこんをもうしこむ
\\	断る	
\\	ことわる
\\	曖昧な返事をする	
\\	あいまいなへんじ
\\	曖昧	
\\	あいまい
\\	朝の支度をする	
\\	あさのしたく
\\	浴びる	
\\	あびる
\\	歯を磨く	
\\	はをみがく
\\	下着	
\\	したぎ
\\	上着	
\\	うわぎ
\\	歯磨き粉をつける	
\\	はみがきこ
\\	髭をそる	
\\	ひげ
\\	ドライヤーで髪を乾かす	
\\	かみをかわかす
\\	髪をとく	
\\	かみ
\\	化粧をする	
\\	けしょう
\\	口紅を塗る	
\\	くちべにをぬる
\\	服を着替える	
\\	ふくをきがえる
\\	おしゃれをする	
\\	紐を結ぶ	
\\	ひもをむすぶ
\\	マフラーをまく・する	
\\	手袋をはめる・する	
\\	てぶくろ
\\	指輪をはめる・する	
\\	ゆびわ
\\	解く	
\\	ほどく
\\	袖無し	
\\	そでなし
\\	無地	
\\	むじ
\\	柄・模様	
\\	がら・もよう
\\	しま	
\\	水玉	
\\	みずたま
\\	長袖	
\\	ながそで
\\	似合う	
\\	にあう
\\	気に入る	
\\	きにいる
\\	お気に入り	
\\	おきにいり
\\	好む	
\\	このむ
\\	目立つ色	
\\	めだついろ
\\	派手	
\\	はで
\\	流行っている	
\\	はやっている
\\	サイズを確かめる・確認する	
\\	たしかめる・かくにん
\\	緩い	
\\	ゆるい
\\	品物をかごに入れる	
\\	しなもの
\\	きつい	
\\	サイズが合う	
\\	あう
\\	流行り・流行	
\\	はやり・りゅうこう
\\	セール・バーゲン	
\\	特売品	
\\	とくばいひん
\\	売り切れる	
\\	うりきれる
\\	本物	
\\	ほんもの
\\	偽物	
\\	にせもの
\\	得をする	
\\	とく
\\	損をする	
\\	そん
\\	家族そろって	
\\	かぞく
\\	家族全員	
\\	かぞくぜんいん
\\	子供を連れる	
\\	こどもをつれる
\\	腹が減る	
\\	はらがへる
\\	外食をする	
\\	がいしょく
\\	昼食をする	
\\	ちゅうしょく
\\	雰囲気	
\\	ふんいき
\\	満席	
\\	まんせき
\\	空いている	
\\	すいている
\\	列に並ぶ	
\\	れつにならぶ
\\	禁煙席	
\\	きんえんせき
\\	喫煙席	
\\	きつえんせき
\\	行列ができる	
\\	ぎょうれつ
\\	量が多い	
\\	りょうがおおい
\\	量が少ない	
\\	りょうがすくない
\\	注文が決まる	
\\	ちゅうもんがきまる
\\	注文する	
\\	ちゅうもん
\\	注文を取る・受ける	
\\	ちゅうもんをとる・うける
\\	デザートを追加する	
\\	ついか
\\	コーヒーのお変わり	
\\	おかわり
\\	無料	
\\	むりょう
\\	食器を下げる	
\\	しょっきをさげる
\\	料理を残す	
\\	りょうりをのこそ
\\	もったいない	
\\	何名様ですか	
\\	なんめいさま
\\	申し訳ありません	
\\	もうしわけ
\\	宴会を開く	
\\	へんかいをひらく
\\	飲み会	
\\	のみかい
\\	送別会	
\\	そうべつかい
\\	歓迎会	
\\	かんげいかい
\\	飲み放題	
\\	のみほうだい
\\	お酒を注ぐ	
\\	おさけをつぐ
\\	お湯を注ぐ	
\\	おゆをそそぐ
\\	乾杯する	
\\	かんぱい
\\	盛り上がる	
\\	もりあがる
\\	土が盛り上がる	
\\	つちがもりあがる
\\	ビールの泡	
\\	あわ
\\	溢れる	
\\	あふれる
\\	こぼす	
\\	こぼれる	
\\	一気に飲む	
\\	いっきにのむ
\\	ウイスキーの水割り	
\\	みずわり
\\	おつまみ	
\\	お酒に酔う	
\\	おさけによう
\\	酔っぱらう	
\\	よっぱらう
\\	酔っぱらい	
\\	よっぱらい
\\	暴れる	
\\	あばれる
\\	賑やか	
\\	にぎやか
\\	やかましい	
\\	騒がしい	
\\	さわがしい
\\	騒ぐ	
\\	さわぐ
\\	騒ぞうしい	
\\	そうぞうしい
\\	たばこの煙	
\\	けむり
\\	煙い・煙たい	
\\	けむい・けむたい
\\	グラスを割る	
\\	わる
\\	ビールびんを倒す	
\\	たおす
\\	ビールびんが倒れる	
\\	たおれる
\\	ほかの客に迷惑をかける	
\\	きゃくにめいわく
\\	迷惑がかかる	
\\	めいわく
\\	レジ	
\\	感情を済ませる	
\\	かんじょうをすませる
\\	別々に払う	
\\	べつべつにはらう
\\	割り勘にする	
\\	わりかん
\\	おごろ	
\\	おごり	
\\	現金で払う	
\\	げんきんではらう
\\	かーどの有効期限	
\\	ゆうこうきげん
\\	お札を崩す	
\\	おさつをくずす
\\	お釣りをもらう	
\\	おつり
\\	領収書	
\\	りょうしゅうしょ
\\	贅沢	
\\	ぜいたく
\\	無駄遣いをする	
\\	むだづかい
\\	請求書が届く	
\\	せいきゅうしょをととく
\\	費用がかかる	
\\	ひよう
\\	生活費	
\\	ふそく
\\	交通費	
\\	こうつうひ
\\	不足	
\\	ふそく
\\	生活費が不足する	
\\	せいかつひがふそくする
\\	お金が足りる	
\\	おかねがたりる
\\	足りない	
\\	たりない
\\	食費	
\\	しょくひ
\\	節約	
\\	せつやく
\\	無駄を省く	
\\	むだをはぶく
\\	無駄	
\\	むだ
\\	省く	
\\	はぶく
\\	貧乏になる	
\\	びんぼう
\\	お金持ち	
\\	おかねもち
\\	貧しい生活	
\\	まずしいせいかつ
\\	借金をする	
\\	しゃっきん
\\	科目	
\\	かもく
\\	教科	
\\	きょうか
\\	得意な	
\\	とくいな
\\	苦手な	
\\	にがてな
\\	国語	
\\	こくご
\\	算数	
\\	さんすう
\\	理科	
\\	りか
\\	社会	
\\	しゃかい
\\	体育	
\\	たいいく
\\	入学する	
\\	にゅうがく
\\	私立高校	
\\	しりつこうこう
\\	公立高校	
\\	こうりつだいがく
\\	国立大学	
\\	こくりつだいがく
\\	昼休み	
\\	おひるやすみ
\\	お弁当	
\\	おべんとう
\\	給食	
\\	きゅうしょく
\\	時間割	
\\	じかんわり
\\	入学式	
\\	にゅうがくしき
\\	出席する	
\\	しゅっせき
\\	通学する	
\\	つうがくする
\\	学期	
\\	がっき
\\	前期	
\\	ぜんき
\\	卒業式	
\\	そつぎょうしき
\\	欠席する	
\\	けっせきする
\\	通勤する	
\\	つうきん
\\	今学期	
\\	こんがっき
\\	後期	
\\	こうき
\\	学年	
\\	がくねん
\\	自習する	
\\	じしゅう
\\	鉛筆を削る	
\\	えんぴつをけずる
\\	物差し	
\\	ものさし
\\	定規	
\\	じょうぎ
\\	合計を出す	
\\	ごうけいをだす
\\	測る	
\\	はかる
\\	足し算	
\\	たしざん
\\	引き算	
\\	ひきざん
\\	掛算	
\\	かけざん
\\	割り算	
\\	わりざん
\\	学部	
\\	がくぶ
\\	受験	
\\	じゅけん
\\	文学部	
\\	ぶんがくぶ
\\	経済学部	
\\	けいざいがくぶ
\\	法学部	
\\	ほyがくぶ
\\	理工学部	
\\	りこうがくぶ
\\	医学部	
\\	いがくぶ
\\	学科	
\\	がっか
\\	専攻	
\\	せんこう
\\	経済学	
\\	けいざいがく
\\	政治学	
\\	せいじがく
\\	心理学	
\\	しんるがく
\\	言語学	
\\	げんごがく
\\	物理学	
\\	ぶつりがく
\\	ゼミナール	
\\	単位をとる	
\\	たんい
\\	講義	
\\	こうぎ
\\	補習	
\\	ほしゅう
\\	補講	
\\	ほこう
\\	進学	
\\	しんがく
\\	希望	
\\	きぼう
\\	合格発表	
\\	ごうかくはっぴょう
\\	合格する	
\\	ごうかく
\\	不合格になる	
\\	ふごうかく
\\	授業料	
\\	じゅぎょうりょう
\\	学費	
\\	がくひ
\\	奨学金	
\\	しょうがくきん
\\	教授	
\\	きょうじゅ
\\	学者	
\\	がくしゃ
\\	講義	
\\	こうぎ
\\	間に合う	
\\	まにあう
\\	講義に遅れる・遅刻する	
\\	こうぎにおくれる・ちこく
\\	成績がいい	
\\	せいせき
\\	授業中	
\\	じゅぎょうちゅう
\\	居眠りをする	
\\	いねむり
\\	授業をサボる	
\\	じゅぎょう
\\	単位を落とす	
\\	たんいをおとす
\\	留学する	
\\	りゅうがく
\\	記入する	
\\	きにゅう
\\	見直す	
\\	みなおす
\\	間違う	
\\	まちがう
\\	聞き直す	
\\	ききなおす
\\	レポートをまとめる	
\\	仕上がる	
\\	しあがる
\\	提出する	
\\	ていしゅつ
\\	担当の先生	
\\	たんとうのせんせい
\\	先生に教わる	
\\	せんせいにおそわる
\\	繰り返す	
\\	くりかえす
\\	暗記	
\\	あんき
\\	試験で問われる	。。。
\\	しけんでとわれる
\\	落ち着いて考える	
\\	おちついてかんがえる
\\	聞き直す	
\\	ききなおす
\\	言い直す	
\\	いいなおす
\\	聞き取る	
\\	ききとる
\\	書き取る	
\\	かきとる
\\	付いている	
\\	ついている
\\	付いていない	
\\	ついていない
\\	試験が済む	
\\	しけんがすむ
\\	カンニングする	
\\	下書き	
\\	したがき
\\	清書する	
\\	せいしょ
\\	提出期限	
\\	ていしゅつきげん
\\	提出期限を守る	
\\	ていしゅつきげんをまもる
\\	工場	
\\	こうじょう
\\	店員	
\\	てんいん
\\	英会話講師	
\\	えいかいわこうし
\\	翻訳	
\\	ほんやく
\\	通訳	
\\	つうやく
\\	事務の手伝い	
\\	じむのてつだい
\\	翻訳する	
\\	ほんやく
\\	探す	
\\	さがす
\\	給料	
\\	きゅうりょう
\\	履歴書	
\\	りれきしょ
\\	面接	
\\	めんせつ
\\	従業員	
\\	じゅうぎょういん
\\	仕事仲間	
\\	しごとなかま
\\	時給1200円	
\\	じきゅう
\\	交通費	
\\	こうつうひ
\\	勤務時間	
\\	きんむじかん
\\	お金を稼ぐ	
\\	おかねをかせぐ
\\	お金を儲ける	
\\	おかねをもうける
\\	仕事を済ます	
\\	しごとをすます
\\	仕事を怠ける	
\\	しごとをなまける
\\	怠け者	
\\	なまけもの
\\	首になる	
\\	くび
\\	働き者	
\\	はたらきもの
\\	辞めさせられる	
\\	やめさせられる
\\	初めてパソコンを触る	
\\	はじめてぱそこんをさわる
\\	入門講座を受ける	
\\	にゅうもんこうざをうける
\\	パソコンに慣れる	
\\	ぱそこんになれる
\\	新製品	
\\	しんせいひん
\\	新品	
\\	しんぴん
\\	画面を見る	
\\	がめんをみる
\\	ファイルを開く	
\\	ふぁいるをひらく
\\	年賀状を作る	
\\	ねんがじょうをつくる
\\	初心者	
\\	しょしんしゃ
\\	中古	
\\	ちゅうこ
\\	閉じる	
\\	とじる
\\	世界中と繋がる	
\\	せかいじゅうとつながる
\\	出品する	
\\	しゅっぴんする
\\	情報を得る	
\\	じょうほうをえる
\\	参加する	
\\	さんかする
\\	オークションに品物を出す	
\\	しなものをだす
\\	ホームページを作る	
\\	つくる
\\	ブログを始める	
\\	はじめる
\\	受信する	
\\	じゅしんする
\\	送信する	
\\	そうしんする
\\	返信する	
\\	へんしんする
\\	転送する	
\\	てんそうする
\\	新規作成する	
\\	しんきさくせいする
\\	まとめて送る	
\\	おくる
\\	整理する	
\\	せいりする
\\	入力する	
\\	にゅうりょくする
\\	ローマ字	
\\	ろーまじ
\\	漢字に変換する	
\\	かんじにへんかんする
\\	挿入する	
\\	そうにゅうする
\\	削除する	
\\	さくじょする
\\	宛先	
\\	あてさき
\\	差出人	
\\	さしだしにん
\\	件名	
\\	けんめい
\\	やり取りをする	
\\	やりとり
\\	添付する	
\\	てんぷする
\\	改行する	
\\	かいぎょうする
\\	登録する	
\\	とうろくする
\\	完了する	
\\	かんりょうする
\\	アドレス帳	
\\	あどれすちょう
\\	長男	
\\	ちょうなん
\\	長女	
\\	ちょうじょ
\\	次男	
\\	じなん
\\	次女	
\\	じじょ
\\	三男	
\\	さんなん
\\	三女	
\\	さんじょ
\\	末っ子	
\\	すえっこ
\\	叔父	
\\	おじ
\\	叔母	
\\	おば
\\	同僚	
\\	どうりょう
\\	上司	
\\	じょうし
\\	先輩	
\\	せんぱい
\\	後輩	
\\	こうはい
\\	部長	
\\	ぶちょう
\\	課長	
\\	かちょう
\\	大学生	
\\	だいがくせい
\\	高校生	
\\	こうこうせい
\\	中学生	
\\	ちゅうがくせい
\\	小学生	
\\	しょうがくせい
\\	会社員	
\\	かいしゃいん
\\	主婦	
\\	しゅふ
\\	教師	
\\	きょうし
\\	末っ子	
\\	すえっこ
\\	3人姉妹の一番上	
\\	さんにんしまいのいちばんうえ
\\	親戚	
\\	しんせき
\\	親類	
\\	しんるい
\\	孫	
\\	まご
\\	おい	
\\	めい	
\\	従兄	
\\	いとこ
\\	一人っ子	
\\	ひとりっこ
\\	一人息子	
\\	ひとりむすこ
\\	一人娘	
\\	ひとりむすめ
\\	お生まれはどちらですか	
\\	おうまれ
\\	出身	
\\	しゅっしん
\\	お茶をお持ちます	
\\	おちゃをもちます
\\	遠慮	
\\	えんりょ
\\	無沙汰	
\\	ぶさた
\\	ご主人	
\\	ごしゅじん
\\	お気の毒に	
\\	おきのどく
\\	お目にかかるのを楽しみにしています	
\\	おめに
\\	ご苦労様です	
\\	ごくろうさま
\\	失礼します	
\\	しつれい
\\	賢い	
\\	かしこい
\\	礼儀正しい	
\\	れいぎただしい
\\	正直	
\\	しょうじき
\\	正直に言うと	
\\	しょうじきにいうと
\\	素直	
\\	すなお
\\	純粋	
\\	じゅんすい
\\	純粋な気持ち	
\\	じゅんすいなきもち
\\	かっこいい	
\\	かっこ悪い	
\\	かっこわるい
\\	勇気がある	
\\	ゆうき
\\	積極的	
\\	せっきょくてき
\\	消極的	
\\	しょうきょくてき
\\	落ち着く	
\\	おちつく
\\	おとなしい	
\\	真面目	
\\	まじめ
\\	吞気	
\\	のんき
\\	美人	
\\	びじん
\\	ハンサム	
\\	スマート	
\\	瘦せている	
\\	やせている
\\	意地悪	
\\	いじわる
\\	意地悪をする	
\\	いじわる
\\	噓をつく	
\\	うそ
\\	噓つき	
\\	うそつき
\\	ずるい	
\\	図々しい	
\\	ずうずうしい
\\	しつこい	
\\	しつこく注意する	
\\	ちゅうい
\\	だらしない	
\\	けち	
\\	いい加減	
\\	いいかげん
\\	無責任	
\\	むせきにん
\\	いい加減なことを言う	
\\	いいかげんなことをいう
\\	いい加減にしろ!	
\\	いいかげん
\\	体の調子が悪い	
\\	からだのちょうしがわるい
\\	具合	
\\	ぐあい
\\	めまいがする	
\\	頭痛がする	
\\	ずつう
\\	胃が痛い	
\\	いがいたい
\\	腰が痛い	
\\	こしがいたい
\\	息が苦しい	
\\	いきがくるしい
\\	苦しむ	
\\	くるしむ
\\	鼻がつまる	
\\	はな
\\	鼻がかむ	
\\	はな
\\	吐き気がする	
\\	はきけ
\\	気分が悪い	
\\	きぶんがわるい
\\	気持ちが悪い	
\\	きもちがわるい
\\	吐く	
\\	はく
\\	目がかゆい	
\\	め
\\	目をこする	
\\	め
\\	目をかく	
\\	め
\\	風邪を引いて辛い	
\\	かぜをひいてつらい
\\	アレルギーがある	
\\	患者	
\\	かんじゃ
\\	看護師	
\\	かんごし
\\	お大事に	
\\	おだいじに
\\	お世話になりました	
\\	おせわ
\\	専門の医師	
\\	せんもんのいし
\\	診察	
\\	しんさつ
\\	診察を受ける	
\\	しんさつをうける
\\	歯医者にかかる	
\\	はいしゃ
\\	花粉症になる	
\\	かふんしょう
\\	うがいをする	
\\	けがを治す	
\\	なおす
\\	手の傷	
\\	てのきず
\\	病気の症状	
\\	びょうきのしょうじょう
\\	注射	
\\	ちゅうしゃ
\\	主従をうける	
\\	しゅじゅつ
\\	痛みを我慢する	
\\	いたみをがまんする
\\	我慢	
\\	がまん
\\	やけどをする	
\\	大したことがない	
\\	たい
\\	ましになる	
\\	お見舞いに行く	
\\	おみまいにいく
\\	虫歯	
\\	むしば
\\	にきび	
\\	白髪	
\\	しらが
\\	皴	
\\	しわ
\\	鼻水	
\\	はなみず
\\	鼻水が出る	
\\	はなみずがでる
\\	ため息	
\\	ためいき
\\	ため息をつく	
\\	たまいき
\\	涙を流す	
\\	なみだをながす
\\	汗	
\\	あせ
\\	涎	
\\	よだれ
\\	汗をかく	
\\	あせ
\\	汗臭い	
\\	あせくさい
\\	咳が出る	
\\	せきがでる
\\	吃逆	
\\	しゃっくり
\\	くしゃみ	
\\	欠伸	
\\	あくび
\\	げっぷ	
\\	おなら	
\\	体重が増える	
\\	たいじゅうがふえる
\\	体重	
\\	たいじゅう
\\	体重を増やす	
\\	たいじゅうをふやす
\\	体重が減る	
\\	たいじゅうがでる
\\	体重を減らす	
\\	たいじゅうをへらす
\\	いびきをかく	
\\	足がしびれる	
\\	あし
\\	体が震える	
\\	からだがふるえる
\\	震える	
\\	ふるえる
\\	肩がこる	
\\	かた
\\	癖	
\\	くせ
\\	癖がつく	
\\	くせ
\\	癖になる	
\\	くせ
\\	ストレスがたまる	
\\	年を取る	
\\	としをとる
\\	爪が伸びる	
\\	つめがのびる
\\	髭が生える	
\\	ひげがはえる
\\	髪の毛が抜ける	
\\	かみのけがぬける
\\	歯が生える	
\\	はがはえる
\\	髪を染める	
\\	かみをそめる
\\	はげる	
\\	花が咲く	
\\	はながさく
\\	花が散る	
\\	はながちる
\\	木が枯れる	
\\	きがかれる
\\	山が崩れる	
\\	やまがくずれる
\\	険しい山	
\\	けわしいやま
\\	なだらかな坂	
\\	さか
\\	坂	
\\	さか
\\	箱がつぶれる	
\\	はこ
\\	つぶる	
\\	浮く	
\\	うく
\\	沈む	
\\	しずむ
\\	蒸し暑い	
\\	むしあつい
\\	腐る	
\\	くさる
\\	かびが生える	
\\	かびがはえる
\\	かび	
\\	変な匂いがする	
\\	へんなにおいがする
\\	匂う	
\\	におう
\\	生き物	
\\	いきもの
\\	気持ち悪い生き物	
\\	きもちわるいいきもの
\\	親しい友人	
\\	したしい
\\	怪しい男	
\\	あやしいおとこ
\\	恐ろしい事件	
\\	おそろしいじけん
\\	錆びた	
\\	さびた
\\	錆びた包丁	
\\	さびたほうちょう
\\	鋭い	
\\	するどい
\\	鈍い	
\\	にぶい
\\	のろい車	
\\	のろいくるま
\\	安全な場所	
\\	あんぜんなばしょ
\\	薄暗い部屋	
\\	うすぐらいへや
\\	眩しい明かり	
\\	まぶしいあかり
\\	眩しい	
\\	まぶしい
\\	詳しい	
\\	くわしい
\\	詳しい説明書	
\\	くわしいせつめいしょ
\\	詳しい説明する	
\\	くわしいせつめい
\\	下らない	
\\	くだらない
\\	でたらめに答える	
\\	こたえる
\\	突然	
\\	とつぜん
\\	いきなり	
\\	とたんに	
\\	やっと	
\\	とうとう	
\\	結局	
\\	けっきょく
\\	遂に	
\\	ついに
\\	よく来ます	
\\	きます
\\	大抵	
\\	たいてい
\\	普段は	
\\	ふだんは
\\	普通は	
\\	ふつうは
\\	たまに	
\\	かなり	
\\	結構	
\\	けっこう
\\	物凄く	
\\	ものすごく
\\	めちゃくちゃ	
\\	順番がめちゃくちゃ	
\\	じゅんばん
\\	めちゃくちゃに壊れた	
\\	こわれた
\\	是非・是非とも	
\\	ぜひ・ぜひとも
\\	どうか	
\\	まず	
\\	とにかく	
\\	一度に	
\\	いちどに
\\	いっぺんに片づける	
\\	かたづける
\\	いっぺんに疲れが出た	
\\	つかれがでた
\\	順番	
\\	じゅんばん
\\	うっかりする	
\\	すっかり	
\\	すっかり忘れた	
\\	わすれた
\\	がっかり	
\\	試験に落ちてがっかりする	
\\	しけんにおちて
\\	やっぱり・やはり	
\\	やっぱり思ったとおりだ	
\\	おもった
\\	しっかりしろ!	
\\	すっきり	
\\	部屋がすっきりする	
\\	へや
\\	気分がすっきりする	
\\	きぶん
\\	さっぱり	
\\	お風呂に入ってさっぱりする	
\\	おふろにはいって
\\	この問題はさっぱりわからない	
\\	もんだい
\\	くっすり眠る	
\\	ねむる
\\	にっこり	
\\	にっこり笑う	
\\	わらう
\\	にこにこ	
\\	そっくり	
\\	あの親子はそっくりだ	
\\	おやこ
\\	はっきり	
\\	山がはっきり見える	
\\	やま
\\	のんびり	
\\	のんびり過ごす	
\\	すごす
\\	きちんと	
\\	洋服をきちんと・ちゃんと畳む	
\\	ようふく・たたむ
\\	ふと思い出す	
\\	おもいだす
\\	わざと	
\\	わざと負ける	
\\	まける
\\	ほっと	
\\	母の病気が治ってほっとした	
\\	ははのびょうきがなおって
\\	じっと	
\\	じっとがまんする	
\\	じっと見る	
\\	みる
\\	ざっと	
\\	ざっと掃除する	
\\	そうじ
\\	そっと	
\\	そっと近づく	
\\	ちかづく
\\	胸がドキドキする	
\\	むね
\\	胸	
\\	むね
\\	面接でドキドキ	
\\	めんせつ
\\	渋滞でいらいらする	
\\	じゅうたい
\\	渋滞	
\\	じゅうたい
\\	はきはき答える	
\\	こたえる
\\	彼女ははきはきして感じがいい	
\\	かのじょ。かんじ
\\	ぶつぶつ文句と言う	
\\	もんくという
\\	文句	
\\	もんく
\\	彼は何かぶつぶつ言っている	
\\	かれはなに
\\	変な男がうろうろしている	
\\	へんなおとこ
\\	うろうろ	
\\	物がぶらぶらぐれる	
\\	もの
\\	街をぶらぶらする	
\\	まち
\\	ぎりぎり	
\\	ぎりぎりで最終電車に間に合った	
\\	さいしゅうでんしゃにまにあった
\\	締切	
\\	しめきり
\\	締切ぎりぎりに願書を出した	
\\	しめきり。がんしょをだした
\\	願書	
\\	がんしょ
\\	ぴかぴか	
\\	靴をピカピカに磨く	
\\	くつ・みがく
\\	のろのろ	
\\	電車がのろのろ走っている	
\\	でんしゃ・はしっている
\\	ぺらぺら	
\\	秘密	
\\	ひみつ
\\	秘密をぺらぺらしゃべる	
\\	ひみつ
\\	田中さんは英語がぺらぺらだ	
\\	たなか・えいご
\\	ペラペラの紙		
\\	ぺらぺらの紙	
\\	かみ
\\	お腹ぺこぺこ	
\\	おなか
\\	ますます	
\\	物価はますます上がるだろう	
\\	ぶっか・あがる
\\	だんだん	
\\	だんだん寒くなりますね	
\\	さむく
\\	どんどん	
\\	彼の日本語はどんどん上達している	
\\	かれのにほんご・じょうたつ
\\	そろそろ始めましょう	
\\	はじめましょう
\\	もともと	
\\	彼はもともと気の弱い人だった	
\\	かれはもともときのよわいひと
\\	気の弱い	
\\	きのよわい
\\	この時計はもともと安かったから、壊れても仕方がない	
\\	とけい・やす・こわれてもしかた
\\	たまたま	
\\	たまたま高校時代の友人に会った	
\\	こうこうじだいのゆうじんにあった
\\	事件の結果はまあまあだった	
\\	じけんのけっか
\\	さまざま	
\\	さまざまな商品	
\\	しょうひん
\\	理想	
\\	りそう
\\	理想の人に出会う	
\\	りそうのひとであう
\\	出会い	
\\	であい
\\	出迎え	
\\	でむかえ
\\	客を出迎える	
\\	きゃくをでむかえる
\\	見送る	
\\	みおくる
\\	空を見上げる	
\\	そらをみあげる
\\	通りを見下ろす	
\\	とおりをみおろす
\\	急に立ち止まる	
\\	きゅうにたちどまる
\\	ソファーから立ち上がる	
\\	たちあがる
\\	寺の前を通りかかる	
\\	てらのまえをとおり
\\	郵便局を通り過ぎる	
\\	ゆうびんきょくをとおりすぎる
\\	隣の人に話しかける	
\\	となりのひとにはなし
\\	田中さんと話し合う	
\\	たなか・はなしあう
\\	話し合い	
\\	はなしあい
\\	知り合い	
\\	しりあい
\\	中村さんと知り合う	
\\	なかむ・しりあう
\\	駅で待ち合わせる	
\\	えきでまちあわせる
\\	待ち合わせ	
\\	まちあわせ
\\	問い合わせ	
\\	といあわせ
\\	電話で問い合わせる	
\\	でんわでといあわせる
\\	申し込む	
\\	もうしこむ
\\	旅行会社に申し込む	
\\	りょこうがいしゃにもうしこむ
\\	申し込みを締切	
\\	もうしこみをしめきる
\\	申し込みを受け付ける	
\\	もうしこみをうけつける
\\	受け取り	
\\	うけとり
\\	荷物を受け取る	
\\	にもつをうけとる
\\	取り消し	
\\	とりけし
\\	電池を取り替える	
\\	でんちをとりかえる
\\	予約を取り消す	
\\	よやくをとりけす
\\	を取り出す	
\\	とりだす
\\	子供が飛び出す	
\\	こどもがとびだす
\\	プールに飛び込む	
\\	とびこす
\\	大学を出る・卒業する	
\\	だいがくをでる・そつぎょう
\\	会議に出る・出席する	
\\	かいぎにでる・しゅっせき
\\	食事が出る	
\\	しょくじがでる
\\	結論が出る	
\\	けつろんがでる
\\	元気が出る	
\\	げんきがでる
\\	新製品が出る・発売される	
\\	しんせいひんがでる・はつばい
\\	ボーナスが出る	
\\	でる
\\	ゴキブリが出る・現れる	
\\	でる・あらわれる
\\	新聞に名前が出る	
\\	しんぶんになまえがでる
\\	角を曲がると駅に出る	
\\	かどをまがるとえきにでる
\\	財布	
\\	さいふ
\\	財布を取る・盗む	
\\	さいふをとる・ぬすむ
\\	財布を取りに戻る	
\\	さいふをとりにもどる
\\	運転免許を取る	
\\	うんてんめんきょ・とる
\\	こぴーを取る・する	
\\	とry
\\	睡眠を取る・眠る	
\\	すいみんをとる・ねむる
\\	痛みを取る	
\\	いたみをとる
\\	新入社員を取る	
\\	しんにゅうしゃいんをとる
\\	採用する	
\\	さいよう
\\	病気にかかる・になる	
\\	びょうき
\\	医者にかかる	
\\	いしゃ
\\	をかける	
\\	毛布をかける	
\\	もうふ
\\	心配をかける・心配させる	
\\	しんぱい
\\	席を立つ	
\\	せきをたつ
\\	座る	
\\	すわる
\\	時間がたつ	
\\	じかん
\\	予定を立てる	
\\	よていをたてる
\\	計画する	
\\	けいかく
\\	音を立てる・出す	
\\	おとをたてる・だす
\\	雨が上がる・やむ	
\\	あめがあがる
\\	どうぞお上がり下さい	
\\	おあがりください
\\	どうぞ入ってください	
\\	はいって
\\	風呂から上がる・出る	
\\	ふろからあがる・でる
\\	物価が上がる	
\\	ぶっかがあがる
\\	効果が上がる	
\\	こうかがあがる
\\	初めてスピーチであがる	
\\	はじめて
\\	緊張する	
\\	きんちょう
\\	例をあげる・並べる	
\\	れい・ならべる
\\	結婚式をあげる	
\\	けっこんしき
\\	熱が下がる	
\\	ねつがさがる
\\	黄色い線の内側まで下がる	
\\	きいろいせんのうちがわまでさがる
\\	評判	
\\	ひょうばん
\\	評判が下がる・落ちる	
\\	ひょうばんがさがる・おちる
\\	カーテンが下がる	
\\	さがる
\\	買い物かごを下げる	
\\	かいもの・さげる
\\	一時下げて書く	
\\	いちじさげてかく
\\	テレビの音を下げる・小さくする	
\\	おとをさげる・ちいさく
\\	泥がつく	
\\	どろ
\\	傷がつく	
\\	きず
\\	連絡がつく	
\\	れんらく
\\	都合がつく	
\\	つごう
\\	おまけがつく	
\\	印をつける	
\\	しるし
\\	たばこに火をつける	
\\	ひ
\\	日記をつける・書く	
\\	にっき・かく
\end{CJK}
\end{document}