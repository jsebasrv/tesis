\documentclass[8pt]{extreport} 
\usepackage{hyperref}
\usepackage{CJKutf8}
\begin{document}
\begin{CJK}{UTF8}{min}
\\	音樂に就いて
\\	會津八一
\\	【テキスト中に現れる記号について】
\\	入力者注 主に外字の説明や、傍点の位置の指定
\\	(例)[#ここから1字下げ]
\\	[#ここから1字下げ]
\\	[#ここから41字詰め]
\\	既に美育部を持つてゐる早稻田中學校が新に音樂會を興してその發會式をやらうとする其の日から、又病氣で暫く引き籠る事になつた。私は元來音樂には殆ど無智で趣味も深いとは言へない。けれども相應な希望は持つてゐる。病中ながら、その希望を會員の諸君にも會員外の諸君にも一寸申し上げてみたいと思ふ。
\\	[#ここで字詰め終わり]
\\	[#ここで字下げ終わり]
\\	吾々は何の爲に畫を描くか? かつて美育部の展覽會で私がかう云ふ問題を出し、そして自分でこの問に答へた事がある。吾々が美育部で努めなければならぬ事は、專門家になる爲に繪を描くのではない、人間として繪を描くのであると云ふ事を自分自身にも他人にも明瞭にしておくべき事――則ち是である。
\\	むづかしい修養の爲でなく單なる娯樂の爲に繪を描く人があつても必ずしも咎めない。又後々に專門の畫家になつてもそれはその人の自由である。然しながら今から專門家を氣取る人があるならば、それは警めなければならない。その意味は人間には自然に色彩と形との美しさを追及する欲望がある。その欲望を正しく上品に訓練して行く事が吾々として學生時代は勿論一生涯つとめなければならぬ事である。繪を描く事は心の中からの止むべからざる要求を本にして我々の行ふ修養の一つであつて、外部から餘儀なくせらるる種類の物ではない。又見樣見眞似の流行沙汰ですべきものでもないのである。美育部の會ではざつとこんな事を言つたかと思ふ。
\\	音樂も丁度こんな物であるまいかと思ふ。最近日本の社會へ著るしく音樂の趣味が普及して來たやうである。之は洵に喜ぶべき事で、殊に音樂の如く心の最も深い奧底を動かす力を持つて居る藝術に對する趣味が廣まつて行く事は何よりもうれしい事である。しかし音樂をたのしむ人をよくよく見ると色々な人がある。或る人は餘りに奴隷ではないか? 或る人は餘りにペダンティックであるまいか? 又或る人は單なるエキゾティック趣味の追及者たるに止まるのではあるまいか? 甚だしきに至ると、外見の爲、虚榮の爲、是が餘程ありはせぬかと思ふ。音樂と天才、音樂と夜會、音樂と花環、音樂と横文字、――と連想は何れも輕快ではあるが、必ずしも妙ではない。私が今早中グリークラブの設立を喜ぶのは、今少し率直な切實なそして深刻な、音樂を考へての事である。あらゆる人が人間として是非訓練しなければならぬ音とタイムとの美しさに對する感情の修養、それから生ずる色々の賜物、先づこれを當面の目標としたい。是を捨てて差し當つて我々の眼中に置くべき物はない。世界の大家の名を暗記する事も、その名曲を暗ずる事も、高價な樂器を持つ事も、流行の服裝をする事も、髮の毛を長く延ばす事も、それ等は第二第三の事である。一體如何なる種類の藝術でも人間の自然性の必要から生れ出ぬ物はない。それが原始的状態から次第に專門家の手に移つて發達をとげる。けれどもその專門家の手で却つて人間性の自然に遠ざかる樣になる。其の時に革新の聲が起る。その革新と云ふ事は何時も『原始的状態に戻れ!』と云ふ叫びに外ならぬ。音樂は私はよく知らないけれども、獨逸等は民族として昔から愛好して今でも世界で最も音樂的趣味の豐かな國であるらしい。然るにこの國の或る作曲家が論じてゐるものを見ると、――今では昔程獨逸の政治家や學者、即ち所謂識者が音樂に親しまなくなつた。そして其の主なる理由は音樂が狹い範圍の專門家の獨占に歸してしまひ、そして其の專門家に人間として左程立派な人物も無くなつた爲である。――とかう云つてゐる。若しさうであるならば、その言葉は吾々が最も味つて置かなければならんと思ふ。吾々は行末に音樂の專門家となる事は銘々の自由である。然しながら、差し當つて音樂家めいた人の集まりを作るのが目的であつてはならない。銘々が上品な美しい感情を持つた圓滿な人格に近づく爲の修養である事をくれぐれも忘れてはならない。今の日本の音樂專門家を私は立派な人々でないと云ふのではない。何れも人格者として我々が手本にすべき程の人であるとは尚更言ひ難い。若し世界の音樂クラブが專門家氣取りの青年だけで組織されてゐるとしても、せめて吾々のグリークラブだけでも、之を排斥したい。吾々は何處迄も唯音樂を愛し、音樂に依つて更に教化されたる人間になるやうに決心した團體でなければならぬ。この態度を失はない限り、吾々の樂堂が古びた階段教室であらうとも、諸君のヂャケツの袖が白く手ずれて居ようとも、諸君の技術そのものが玄人めかしくなくとも、我が早中グリークラブは、特色のある、否、音樂會としてはかくあるべき音樂會となり得る事と思ふ。
\\	吾々が鍋島直大侯の薨去を聞いたのはまだ耳新しい事である。その臨終には割合に長い間同じやうな危篤の状態が續いた。或る時一端止つた脈が注射で蘇つて息を吹きかへされた時に、靜に眼を開いて、音樂を聞いてみたいと望まれるらしかつた。それで宮内省の雅樂寮から樂人を頼んで來て、枕許で一曲奏して貰つた。夢見るやうにうつとりとそれを聞いて微かな笑を漏らされた。それから力の無い手で自分で拍子を取り始められた。一曲終ればまた一曲、三曲の雅樂を聞いて侯爵の心は其の日は全く蘇つて了つて、それからずつと翌日まで氣分がよかつた。それから間も無く亡くなられた。侯爵の魂は音樂の海の上を小舟のやうに漂ひつくして果は見えなくなられたのである。何と言ふ清く美しい死であらう! そしてこの清く美しい死は即ち其清く美しい人格を語るものである。鍋島侯はかねて音樂の愛好者であつた。しかしながら勿論專門家ではなかつた。たゞ武士の嗜みとして稽古せられた迄である。けれどもそれが力強く根深く侯爵の心魂を涵して居た。それが臨終の際無意識の間に現れたのである。所が世の中の專門的音樂家にかばかり深き音樂の趣味が發見されるであらうか? 技能の專門的研究、職業的演奏、それは人格そのものとは何等の交渉の無い事が多いのみならず、寧ろ負擔や苦痛である事も珍らしくない。況んや世間の輕浮な音樂熱は一時の流行的現象であるかもしれぬ。吾々は我がグリークラブの創立に際して別に吾々の境地を明かにしておく事が先づ以て何よりも必要であると思ふ。[#地から2字上げ](大正十年十月十日)[#「(大正十年十月十日)」は1段階小さな文字]
\\	底本:「會津八一全集 第七卷」中央公論社
\\	1982(昭和57)年4月25日初版発行
\\	底本の親本:「會津八一全集 第七卷」中央公論社
\\	1969(昭和44)年刊行
\\	初出:「興風」
\\	1922(大正11)年2月
\\	入力:フクポー
\\	校正:鴨川佳一郎
\\	2017年10月25日作成
\\	青空文庫作成ファイル:
\\	このファイルは、インターネットの図書館、青空文庫
\\	で作られました。入力、校正、制作にあたったのは、ボランティアの皆さんです。
\end{CJK}
\end{document}