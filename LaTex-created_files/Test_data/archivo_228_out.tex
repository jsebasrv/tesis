\documentclass[8pt]{extreport} 
\usepackage{hyperref}
\usepackage{CJKutf8}
\begin{document}
\begin{CJK}{UTF8}{min}
\\	拓本の話
\\	會津八一
\\	【テキスト中に現れる記号について】
\\	ルビ
\\	(例)機會《はずみ》
\\	入力者注 主に外字の説明や、傍点の位置の指定
\\	(数字は、
\\	0213の面区点番号または
\\	、底本のページと行数)
\\	(例)※[#二の字点、1-2-22]
\\	二倍の踊り字(「く」を縦に長くしたような形の繰り返し記号)
\\	(例)いろ/\
\\	私は拓本の御話をしやう。
\\	支那では昔からすべて文字で書いたものを大切にするが、誰が書いたところで相當に年月が經てばみんな消えて仕舞ふ。紙でも、絹でも、木でも、――名人が書けば木の中へ何寸も深く字が喰ひ込むなどと昔からいふことであるけれども、其木からが千年も經てば磨滅もする風化もする。無くなつてみれば勿論紀念にもならないし、習字の手本にもならない。そこで金屬や石といふやうな堅いものに刻りつけて、いつまでも保存するやうにすることが、もう隨分古くから行はれて居る。殷や周の銅器の刻文、秦の玉版や石刻の文字、漢魏の碑碣などがそれだ。みな千年萬年の後へ遺すつもりで作られたものだ。その文字のある所へ紙を載せて、その上から油墨で刷つたものが俗にいふ石刷即ち拓本で、もとより古を尚び、文字を大切にする支那のことであるから、この石刷をも、原本の實物のやうに大切にする。かうして拓本を作つて珍重することも支那では隨分古くからやつて居ることで、今日に遺つて居るのではまづ古いところでは唐時代のものであらう。それ以後五代拓、宋拓、元拓、明拓といふやうなわけで、勿論古い程尊ばれる。といふのは、いかに石でも金でも、年月が經てば矢張りいたむ。或は風雨に曝されたり、或は野火や山火事に焦がされたり、或は落雷で碎かれたり、或はまたそんなことが無くとも、餘り屡※[#二の字点、1-2-22]拓本を取つた爲に石が磨滅して仕舞ふといふことは珍らしく無いからである。つまり古いほど完全に近い。隨つて古いほど貴いといふことになる。同じ碑の拓本でも、一枚は人が愛馬を賣つても寶劍を質に入れても手に入れなければならぬと騒ぐのに、他の一枚はたゞで貰つてもほしく無いといふやうな話も出て來る。漢の時代に建てられた西嶽崋山廟の碑は、實物は今は無くなつて了つてゐるのであるが、明時代に取つた拓本が一二枚今日迄遺つて居る。これなどは唯拓本による存在である。この西嶽崋山廟の拓本を二三年前に或る支那人が日本へ賣りに來たことがあるが、なんでも一枚三萬五千圓といふ値段であつた。其時に魏の三體石經の拓本も持つて來た。此石經は遠からぬ昔に土中から掘り出したものであるが、後に間もなく碎けて仕舞つた。そこで碎けないさきの拓本であるといふので一枚二千圓と號して居た。
\\	これまで御話して來ただけでは、何だか支那趣味の骨董談のやうに聞えるかもしれぬが、それこそ心外千萬である。なるほど支那人が文字を大切にする態度には宗教がかつた處もあつて、我々としては一々支那人の通りといふわけにも行くまいが、とにかく古人が文字で書いて遺したものは美術であり、文學であり、同時にまた史料である。美術といふ熟語からが、ファインアートといふ英語の明治初年の直譯であるやうに、今日美術を論じて居る人々は、いつも西洋流の美學や、美術論や、美術史に頭が引張られて居るから、今のところではよほど偉い人で無い限りは、東洋の美術といふものに理解が薄い。ことに文字が東洋の美術の中で占めて居る殆ど最高の地位については、まるで無理解な人が多い。けれどもこれも東洋人が今少し落附いて物を考へる時が來ると共に次第に理解せられる時が來ると私は信じて居るのであるが、支那人の大切にする古代の文字の拓本は、即ち歴代の東洋美術の遺品であると考へ直して見て貰ひたい。これだけのことは、文字の拓本の美術的價値について、取敢へず申述べて見たのであるが、なるほど東洋で珍重された拓本は、これまでは、むしろ文字のあるものに片寄り過ぎて居たかも知れなかつた。しかし近頃は大同とか、天龍山とか、龍門とか、或は朝鮮や日本内地の石佛、又は其他の造型美術の拓本を作ることが行はれて來て、それが我が國の現代の學者、美術家、ことに新興の畫家、彫刻家に強い刺戟を與へて居ることは、目覺しい事實である。それから又、漢魏六朝から唐宋に及ぶ幾千の墓碑や墓誌の文章は其時代々々の精神や樣式を漲らした文學であり、同時にまた正史以上に正確な史料的價値を含んで居ることをよく考へて見なければならない。こんな事を私が今事新しく述べ立てるまでもなく、いやしくも今日眞面目に學問をやつて居る人の間に、拓本の功果を疑つて居る者は無い位の趨勢にはなつて居るのであつて、私の友人の或る學者は拓本する事と、寫眞を撮ることゝ、スケッチをすることの出來ぬ者は考古學や歴史を研究する資格が缺けて居ると、京都大學の學生に教へて居るさうであるが、これは私も全然同感である。寫眞が立體的に奧行きをも寫すのに對して、拓本の平面的なことは一つの短所であらうが、寫眞が實物より小さくなる場合が多いのに、拓本はいつも實物大で、しかも實物とわづかに濡れ紙一重を隔てたばかりの親しみの深い印象を留めて居る。拓本が持つ此強い聯想は到底寫眞の企て及ぶところでない。
\\	話が前へ戻つて繰り返へすやうになるけれども、日本の金石文の拓本のことについて云つてみても、正史であるところの日本書紀の記載に間違ひのあることが、法隆寺金堂の釋迦像の銘文や藥師寺の東塔の※[#「木+察」、第4水準2-15-66]の銘文から知られて來たといふやうなことは、今となつては誰も知る事であるが、此所に一つ面白い例がある。それは私は今、昔奈良の東大寺にあつた二つの唐櫃の銘文の拓本を持つて居るが、其櫃の一つは今は御物となつて正倉院にあるが、他の一方はもう實物は此の世の中から失はれたものと見えて、正倉院にも何處にもありはしない。ところがその失はれた唐櫃の銘文の拓本が私の所にあるといふわけだ。即ちその唐櫃は天にも地にも唯一枚の此拓本によつてのみわづかに存在を續けて居る。そして其銘文によつて、私は、これまで此等の唐櫃に歸せられた製作の時代について、一般學者の推定が實に五六百年も間違つて居たことも斷定し得るのである。實は此唐櫃は本來は二つだけのものでなく、四つあるべきもので、其一ともいふべきものが嘗て大倉氏の集古館に納められてあつたが、あの大震災のために燒けて仕舞つた。他の今一つの唐櫃こそは、長へに失はれて全く行く所を知らないのであるが、何かの機會《はずみ》に、何かの僥倖で、せめて其銘文の拓本でも手に入れるやうなことがあり得たならば、我々の史的研究、ことに東大寺の研究に對して一大光明となるであらう。かう考へて來ると拓本には萬金の値ありといふべきで、しかも其値たるや、斷じて骨董値段ではない。
\\	そこで私は、我が早稻田學園でも、先づ學生が拓本といふものゝ必要を覺り、よく此方法に親しみ、これをよく手に入れておいて貰ひたい希望から私は、少からぬ犧牲を忍んで、昨年の十月は私が年來祕藏して居た奈良時代の美術に關する拓本の大部分を第一學院史學部の學生の手に委ねて展覽會を開いて貰ひ、又十二月には第二學院の學術部の學生をわづらはして日本の古い寺院の瓦に模樣の拓本五六百種で、展覽會を開いて貰つた。すると官私立大學や民間の專門學者研究者が相繼いでやつて來て、參觀者名簿に署名せられた。
\\	その内にも東京博物館の高橋博士や、萬葉學の井上博士の如きは、或は學生の分類を批評したり或は學生の成功を賞讚したりして歸られた。それから十二月に私が關西方面に旅行した時には、第二學院の學生の手に成つた瓦の拓本の繪葉書に對して、京都大學の天沼博士が加へられた眞劍な批評を聞き、それから旅先で見た二つの雜誌『史學雜誌』と『民族』とに、此等の催に對する賞讚的紹介を見た。つまり若い學生達の催でも、かうした學界の專門家達の眞面目な眞劍な興味を喚起したことは明かで、これを以て見ても、拓本の學界に於ける價値を知るに充分である。そこで私は、今年の四月には、更に進んで、支那の漢から六朝時代までの書畫の拓本をあつめて我が大學内で展覽會を開き、今秋は朝鮮の拓本の展覽會を開いて、學生及び世間の學者と共に研究の歩を進めて見たいと思つて居るのであるが、前囘の經驗からこゝに一つ私の不滿に思ふ事がある。それは參觀者の内に、學園内の人の方が學園外の人より比較的少かつた事である。どうか私の常に唱へて止まぬところの、學問の基礎を實際的に、即ち實物の上に置く學風、即ち私の謂ふ實學の態度が、我が學園を支配することの一日も速に實現せんことを祈る。
\\	最後につけ加へていつておきたいのは、拓本の方法である。前囘拓本展觀會の宣傳ビラを方々へ貼らせたときに、この拓本といふ字が讀めない人、したがつて意味のわからぬ人が學園の内外に隨分多かつたやうで、中には會場に來て拓本そのものを見ても、まだその作り方などに就いてよく解らない人が多かつたやうだから、今簡單に方法を話して見れば、拓すべき石碑なら石碑の上に拓すべき紙を載せて、その上を少し濕氣のあるタオルで強く押へつける。――或は豫めタオルを卷いて置いて、それを紙の上へ押しつけながら轉がす方が手際よく行くかも知れぬ、――すると壓力と濕氣の爲めに紙は石面の文字のあらゆる凸凹にまんべんなく喰ひ込む。それから少し時間を措いて、紙の濕氣が少し乾くのを見計つて、饅頭のやうにふつくら[#「ふつくら」に傍点]と作つたタンポに、油墨か――これは其目的で作つたものを賣つて居る――、又はたゞ墨汁をつけて紙の上を輕く叩けば、それで拓本が出來る。こまかい事は實際の經驗上自分で發明するのが何よりだ。これが紙を濕らして取る方法であるが、濕されない種類のものは、實物の上へ紙をよく押しつけて、支那製の雪花墨又は日本製の釣鐘墨といふもので靜かにそして細かに其上を撫で※[#「廴+囘」、第4水準2-12-11]はせば乾いたまゝで拓本が出來るが、西洋人は其代りに石墨などを用ゐるやうである。
\\	拓本の趣味を語れといふ學報記者の註文に對して、私はむしろ拓本の實用と私自身の希望を語つて仕舞つたが、拓本の紙の質が支那、朝鮮、日本、同じ支那でも地方々々で違ふことや、タンポの打ち方や墨の濃淡に從つて表はるゝいろ/\の趣味や、平面だけしか取れぬ筈の拓本に全形を想はせる工夫のあることや、模本贋本の多いこと、その見分け方、拓した時代の見分け方、或は又自分で拓本を取つて居る時に低く續くタンポの音に伴つて起つて來るところの何ともいひ難い微妙な快感や、凡そそれこそほんとに拓本の趣味のことは、いづれ又暇な時に御話をする機會があるでせう。
\\	底本:「日本の名随筆27 墨」作品社
\\	1985(昭和60)年1月25日第1刷発行
\\	1997(平成9)年5月20日第17刷発行
\\	底本の親本:「會津八一全集 第一一巻」中央公論社
\\	1982(昭和57)年10月発行
\\	入力:門田裕志
\\	校正:仙酔ゑびす
\\	2006年11月18日作成
\\	2012年4月11日修正
\\	青空文庫作成ファイル:
\\	このファイルは、インターネットの図書館、青空文庫
\\	で作られました。入力、校正、制作にあたったのは、ボランティアの皆さんです。
\end{CJK}
\end{document}