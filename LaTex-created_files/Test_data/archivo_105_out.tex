\documentclass[8pt]{extreport} 
\usepackage{hyperref}
\usepackage{CJKutf8}
\begin{document}
\begin{CJK}{UTF8}{min}
\\	毎日運動
\\	のに、全然痩せなかった。	毎日運動したのに、全然痩せなかった。	まいにちうんどうしたのに、ぜんぜんやせなかった。	
\\	「のに」 
\\	毎日[まいにち] 運動[うんどう]したのに、 全然[ぜんぜん] 痩[や]せなかった。		
\\	それは、
\\	話だよ。	それは、ありえない話だよ。	それは、ありえないはなしだよ。	
\\	「ある得る」, 
\\	それは、ありえない 話[はなし]だよ。		
\\	去年から背が高
\\	なったね。	去年から背が高くなったね。	きょねんからせがたかくなったね。	
\\	する 
\\	なる 
\\	に 
\\	「なる」 
\\	去年[きょねん]から 背[せ]が 高[たか]くなったね。		
\\	残念だが、今週末は行
\\	。	残念だが、今週末は行けない。	ざんねんだが、こんしゅうまつはいけない。	
\\	が	残念[ざんねん]だが、 今週[こんしゅう] 末[まつ]は 行[い]けない。		
\\	友達に会
\\	ら、買い物に行きます。	友達に会えたら、買い物に行きます。	ともだちにあえたら、かいものにいきます。	
\\	「たら(ば)」	友達[ともだち]に 会[あ]えたら、 買い物[かいもの]に 行[い]きます。		
\\	これは、日本語で何
\\	言
\\	か。	これは、日本語で何と言いますか。	これは、にほんごでなんといいますか。	
\\	これは、 日本語[にほんご]で 何[なに]と 言[い]いますか。		
\\	晩ご飯を作
\\	おく。	晩ご飯を作っておく。	ばんごはんをつくっておく。	
\\	「~ておく」 
\\	晩[ばん]ご 飯[はん]を 作[つく]っておく。		
\\	その部屋はあまり静
\\	よ。	その部屋はあまり静かじゃないですよ。	そのへやはあまりしずかじゃないですよ。	
\\	「です」 
\\	その 部屋[へや]はあまり 静[しず]かじゃないですよ。		
\\	勉強
\\	する, 
\\	ちゃ 。	勉強しなくちゃ 。	べんきょうしなくちゃ。	
\\	「なくて」
\\	勉強[べんきょう]しなくちゃ。		
\\	そんなのは、
\\	よ!	そんなのは、あるかよ!	そんなのは、あるかよ!	
\\	そんなのは、あるかよ!		
\\	先生が学校に行
\\	か教えない?	先生が学校に行ったか教えない?	せんせいががっこうにいったかおしえない?	
\\	「か」 
\\	先生[せんせい]が 学校[がっこう]に 行[い]ったか 教[おし]えない?		
\\	マトリックスを見
\\	けど、面白かった。	マトリックスを見たけど、面白かった。	マトリックスをみたけど、おもしろかった。	
\\	「が」 
\\	「けど」	マトリックスを 見[み]たけど、 面白[おもしろ]かった。		
\\	日本に来て、寿司が食
\\	れるようになった。	日本に来て、寿司が食べられるようになった。	にほんにきて、すしがたべられるようになった。	
\\	する 
\\	なる 
\\	に 
\\	「なる」 
\\	「する」 
\\	日本[にっぽん]に 来[き]て、 寿司[すし]が 食[た]べられるようになった。		
\\	ゆっくり話
\\	ください。	ゆっくり話してください。	ゆっくりはなしてください。	
\\	「~ください」- 
\\	「くださる」	ゆっくり 話[はな]してください。		
\\	パーティーに行かなかったの?/うん、時間が
\\	から。	パーティーに行かなかったの?/うん、時間がなかったから。	パーティーにいかなかったの?/うん、じかんがなかったから。	
\\	「から」 
\\	「ので」	パーティーに 行[い]かなかったの?/うん、 時間[じかん]がなかったから。		
\\	代わりに行
\\	。	代わりに行ってあげる。	かわりにいってあげる。	
\\	「あげる」	代[か]わりに 行[い]ってあげる。		
\\	これ、ちょっと見
\\	いい?	これ、ちょっと見ていい?	これ、ちょっとみていい?	
\\	これ、ちょっと 見[み]ていい?		
\\	話を聞
\\	いますか。/ううん、聞
\\	ない。	話を聞いていますか。/ううん、聞いてない。	はなしをきいていますか。/ううん、きいてない。	
\\	「~てる」 
\\	話[はなし]を 聞[き]いていますか。/ううん、 聞[き]いてない。		
\\	何を読
\\	いる?教科書を読
\\	います。	何を読んでいる?教科書を読んでいます。	なにをよんでいる? /きょうかしょをよんでいます。	
\\	「~ている」 
\\	何[なに]を 読[よ]んでいる? 教科書[きょうかしょ]を 読[よ]んでいます。		
\\	みきちゃんが、明日
\\	来る, 
\\	って。	みきちゃんが、明日こないって。	みきちゃんが、あしたこないって。	
\\	「って」 
\\	「て」 
\\	「という」 
\\	みきちゃんが、 明日[あした]こないって。		
\\	無理矢理に部屋に入
\\	うとしている。	無理矢理に部屋に入ろうとしている。	むりやりにへやにはいろうとしている。	
\\	無理矢理[むりやり]に 部屋[へや]に 入[にゅう]ろうとしている。		
\\	みんなが行
\\	なら私も行く。	みんなが行くなら私も行く。	みんながいくならわたしもいく。	
\\	「なら(ば)」	みんなが 行[い]くなら 私[わたし]も 行[い]く。		
\\	学生
\\	のに、彼女は勉強しない。	学生なのに、彼女は勉強しない。	がくせいなのに、かのじょはべんきょうしない。	
\\	「のに」 
\\	学生[がくせい]なのに、 彼女[かのじょ]は 勉強[べんきょう]しない。		
\\	代わりに行
\\	代わりに行ってくれる?	かわりにいってくれる?	
\\	「くれる」	代[か]わりに 行[い]ってくれる?		
\\	友達が父にいいことを教
\\	。	友達が父にいいことを教えてあげた。	ともだちがちちにいいことをおしえてあげた。	
\\	「あげる」	友達[ともだち]が 父[ちち]にいいことを 教[おし]えてあげた。		
\\	先週、ボブに会
\\	よ。	先週、ボブに会いましたよ。	せんしゅう、ボブにあいましたよ。	
\\	「~ます」 
\\	先週[せんしゅう]、ボブに 会[あ]いましたよ。		
\\	てゆうか、もう帰らないとだめですけど。	てゆうか、もう帰らないとだめですけど。	てゆうか、もうかえらないとだめですけど。	
\\	「ゆう」 
\\	「いう」	てゆうか、もう 帰[かえ]らないとだめですけど。		
\\	来年留学するって智子のこと?	来年留学するって智子のこと?	らいねんりゅうがくするってともこのこと?	
\\	「って」 
\\	「て」 
\\	「という」	来年[らいねん] 留学[りゅうがく]するって 智子[さとこ]のこと?		
\\	食
\\	ば病気になるよ。	食べなければ病気になるよ。	たべなければびょうきになるよ。	
\\	「ば」	食[た]べなければ 病気[びょうき]になるよ。		
\\	遠い所に行
\\	。	遠い所に行かないで。	とおいところにいかないで。	
\\	「~ください」- 
\\	「くださる」 
\\	遠[とお]い 所[ところ]に 行[い]かないで。		
\\	友達のおかげで、映画はただで見
\\	。	友達のおかげで、映画はただで見られた。	ともだちのあかげで、えいがはただでみられた。	
\\	「見える」 
\\	「聞こえる」 
\\	友達[ともだち]のおかげで、 映画[えいが]はただで 見[み]られた。		
\\	私が全部食
\\	。	私が全部食べてあげました。	わたしがぜんぶたべてあげました。	
\\	「くれる」 
\\	「あげる」 
\\	私[わたし]が 全部[ぜんぶ] 食[た]べてあげました。		
\\	ちょっと忙
\\	ので、そろそろ失礼します。	ちょっと忙しいので、そろそろ失礼します。	ちょっといそがしいので、そろそろしつれいします。	
\\	「から」 
\\	「ので」	ちょっと 忙[いそが]しいので、そろそろ 失礼[しつれい]します。		
\\	先生
\\	と、きっと年上なんじゃないですか?	先生だと、きっと年上なんじゃないですか?	せんせいだと、きっととしうえなんじゃないですか?	
\\	「と」	先生[せんせい]だと、きっと 年上[としうえ]なんじゃないですか?		
\\	私が友達にプレゼントを
\\	。	私が友達にプレゼントをあげた。	わたしがともだちにプレゼントをあげた。	
\\	「あげる」	私[わたし]が 友達[ともだち]にプレゼントをあげた。		
\\	時間がなかった。/
\\	からパーティーに行かなかったの?	時間がなかった。/だからパーティーに行かなかったの?	じかんがなかった。/だからパーティーにいかなかったの?	
\\	「から」 
\\	「ので」	時間[じかん]がなかった。/だからパーティーに 行[い]かなかったの?		
\\	今、時間が
\\	って、本当?	今、時間がないって、本当?	いま、じかんがないって、ほんとう?	
\\	「って」 
\\	「と」	今[いま]、 時間[じかん]がないって、 本当[ほんとう]?		
\\	友達に会
\\	ば、買い物に行きます。	友達に会えれば、買い物に行きます。	ともだちにあえれば、かいものにいきます。	
\\	「ば」	友達[ともだち]に 会[あ]えれば、 買い物[かいもの]に 行[い]きます。		
\\	友達は何を
\\	する]いるの?昼ご飯を食
\\	いる。	友達は何をしているの?昼ご飯を食べている。	ともだちはなにをしているの?/ひるごはんをたべている。	
\\	「~ている」 
\\	友達[ともだち]は 何[なに]をしているの? 昼[ひる]ご 飯[はん]を 食[た]べている。		
\\	んで、友達に会う時間がない。	なんで、友達に会う時間がない。	なんで、ともだちにあうじかんがない。	
\\	「から」 
\\	「ので」 
\\	なんで、 友達[ともだち]に 会[あ]う 時間[じかん]がない。		
\\	重い荷物が持
\\	。	重い荷物が持てます。	おもいにもつがもてます。	
\\	重[おも]い 荷物[にもつ]が 持[も]てます。		
\\	学校に行
\\	と。	学校に行かないと。	がっこうにいかないと。	
\\	「と」
\\	学校[がっこう]に 行[い]かないと。		
\\	ボールを落
\\	と落ちる。	ボールを落すと落ちる。	ボールをおとすとおちる。	
\\	「と」	ボールを 落[おと]すと 落[お]ちる。		
\\	独身だ
\\	いうのは、嘘だったの?	独身だというのは、嘘だったの?	どくしんだというのは、うそだったの?	
\\	「いう」 
\\	独身[どくしん]だというのは、 嘘[うそ]だったの?		
\\	お父さんが
\\	が壊れた。	お父さんがくれた時計が壊れた。	おとうさんがくれたとけいがこわれた。	
\\	「~ください」- 
\\	「くださる」	お 父[とう]さんがくれた 時計[とけい]が 壊[こわ]れた。		
\\	全部食
\\	くれますか。	全部食べないでくれますか。	ぜんぶたべないでくれますか。	
\\	「くれる」 
\\	「もらえる」 
\\	全部[ぜんぶ] 食[た]べないでくれますか。		
\\	全部食
\\	ほしいんだけど・・・。	全部食べてほしいんだけど・・・。	ぜんぶたべてほしいんだけど・・・。	
\\	「欲しい」	全部[ぜんぶ] 食[た]べてほしいんだけど・・・。		
\\	もうお金が
\\	って。	もうお金がないって。	もうおかねがないって。	
\\	「って」 
\\	「と」	もうお 金[かね]がないって。		
\\	友達
\\	プレゼントが来た。	友達からプレゼントが来た。	ともだちからプレゼントがきた。	
\\	「から」 
\\	「ので」	友達[ともだち]からプレゼントが 来[き]た。		
\\	千円を貸
\\	くれる?	千円を貸してくれる?	せんえんをかしてくれる。	
\\	「くれる」 
\\	「もらえる」	千[せん] 円[えん]を 貸[か]してくれる?		
\\	運動しているから、強
\\	なる。	運動しているから、強くなる。	うんどうしているから、つよくなる。	
\\	する 
\\	なる 
\\	に 
\\	「なる」 
\\	運動[うんどう]しているから、 強[つよ]くなる。		
\\	今日は何を
\\	する, 
\\	か?/テーマパークに行
\\	今日は何をしようか?/テーマパークに行こう!	あしたはなにをしようか?/テーマパークにいこう!	
\\	今日[きょう]は 何[なに]をしようか?/テーマパークに 行[い]こう!		
\\	デパートに行
\\	が、何も欲しくなかったです。	デパートに行きましたが、何も欲しくなかったです。	デパートにいきましたが、なにもほしくなかったです。	
\\	「が」 
\\	「けど」	デパートに 行[い]きましたが、 何[なに]も 欲[ほ]しくなかったです。		
\\	そんなことは
\\	~うる]	そんなことはありえる	そんなことはありえる	
\\	「ある得る」, 
\\	そんなことはありえる		
\\	そうゆうことじゃないって!	そうゆうことじゃないって!	そうゆうことじゃないって!	
\\	「ゆう」 
\\	「いう」	そうゆうことじゃないって!		
\\	学生がこれを先生に
\\	。	学生がこれを先生にあげる。	がくせいがこれをせんせいにあげる。	
\\	「あげる」	学生[がくせい]がこれを 先生[せんせい]にあげる。		
\\	アリス
\\	って、すごくいい人でしょ?	アリスって、すごくいい人でしょ?	アリスって、すごくいいひとでしょ?	
\\	「って」 
\\	「と」 
\\	「は」	アリ ス[す]って、すごくいい 人[ひと]でしょ?		
\\	暇
\\	ら、遊びに行くよ。	暇だったら、遊びに行くよ。	ひまだったら、あそびにいくよ。	
\\	「たら(ば)」	暇[ひま]だったら、 遊[あそ]びに 行[い]くよ。		
\\	あの人は、本当に
\\	食べる。	あの人は、本当に何でも食べる。	あのひとは、ほんとうになんでもたべる。	
\\	あの 人[ひと]は、 本当[ほんとう]に 何[なに]でも 食[た]べる。		
\\	それでいいと思うなら、そう
\\	する]なよ。	それでいいと思うなら、そうしなよ。	それでいいとおもうなら、そうしなよ。	
\\	「~なさい」 
\\	それでいいと 思[おも]うなら、そうしなよ。		
\\	早く寝
\\	はなりませんでした。	早く寝てはなりませんでした。	はやくねてはなりませんでした。	
\\	「だめ」, 「いけない」, 
\\	「ならない」 
\\	早[はや]く 寝[ね]てはなりませんでした。		
\\	私の部屋は、きれい
\\	、静
\\	、とても好き。	私の部屋は、きれいで、静かで、とても好き。	わたしのへやは、きれいで、しずかで、とてもすき。	
\\	私[わたし]の 部屋[へや]は、きれいで、 静[しず]かで、とても 好[す]き。		
\\	時間がないなら、明日でもいいよ。	もし時間がないなら、明日でもいいよ。	もしじかんがないなら、あしたでもいいよ。	
\\	「もし」 
\\	なら	もし 時間[じかん]がないなら、 明日[あした]でもいいよ。		
\\	智子は来年、海外に行
\\	んだって。	智子は来年、海外に行くんだって。	ともこはらいねん、かいがいにいくんだって。	
\\	「って」 
\\	「と」	智子[さとこ]は 来年[らいねん]、 海外[かいがい]に 行[い]くんだって。		
\\	デパートに行
\\	が、いい物がたくさんありました。	デパートに行きましたが、いい物がたくさんありました。	デパートにいきましたが、いいものがたくさんありました。	
\\	「が」 
\\	「けど」
\\	デパートに 行[い]きましたが、いい 物[もの]がたくさんありました。		
\\	明日、映画を見
\\	行く。	明日、映画を見に行く。	あした、えいがをみにいく。	
\\	明日[あした]、 映画[えいが]を 見[み]に 行[い]く。		
\\	誰が盗んだのか、
\\	知りませんか。	誰が盗んだのか、誰か知りませんか。	だれがぬすんだのか、だれかしりませんか。	
\\	誰[だれ]が 盗[ぬす]んだのか、 誰[だれ]か 知[し]りませんか。		
\\	それを食
\\	はだめ!	それを食べてはだめ!	それをたべてはだめ!	
\\	「だめ」, 「いけない」, 
\\	「ならない」 
\\	ては
\\	それを 食[た]べてはだめ!		
\\	けど、彼がまだ好きなの。	だけど、彼がまだ好きなの。	だけど、かれがまだすきなの。	
\\	「が」 
\\	「けど」	だけど、 彼[かれ]がまだ 好[す]きなの。		
\\	食堂に行
\\	、昼ご飯を食
\\	、昼寝をした。	食堂に行って、昼ご飯を食べて、昼寝をした。	しょくどうにいって、ひるごはんをたべて、ひるねをした。	
\\	食堂[しょくどう]に 行[い]って、 昼[ひる]ご 飯[はん]を 食[た]べて、 昼寝[ひるね]をした。		
\\	学校に行
\\	と友達と会えないよ。	学校に行かないと友達と会えないよ。	がっこうにいかないとともだちとあえないよ。	
\\	「と」	学校[がっこう]に 行[い]かないと 友達[ともだち]と 会[あ]えないよ。		
\\	宿題を
\\	する, 
\\	はいけなかった。	宿題をしなくてはいけなかった。	しゅくだいをしなくてはいけなかった。	
\\	ては
\\	いけない
\\	宿題[しゅくだい]をしなくてはいけなかった。		
\\	ずっと一緒に
\\	たい。	ずっと一緒にいたい。	ずっといっしょにいたい。	
\\	「たい」	ずっと 一緒[いっしょ]にいたい。		
\\	名前はアリス。	あたしの名前はアリス。	あたしのなまえはアリス。	
\\	あたしの 名前[なまえ]はアリス。		
\\	どうして彼が好きなの?優
\\	し、かっこ
\\	し、面白いから。	どうして彼が好きなの?優しいし、かっこいいし、面白いから。	どうしてかれがすきなの?やさしいし、かっこいいし、おもしろいから。	
\\	「し」	どうして 彼[かれ]が 好[す]きなの? 優[やさ]しいし、かっこいいし、 面白[おもしろ]いから。		
\\	遠い所に行
\\	くれ。	遠い所に行かないでくれ。	とおいところにいかないでくれ。	
\\	「~ください」- 
\\	「くださる」 
\\	「くれ」	遠[とお]い 所[ところ]に 行[い]かないでくれ。		
\\	友達に聞
\\	けど、知らなかった。	友達に聞いたけど、知らなかった。	ともだちにきいたけど、しらなかった。	
\\	「が」 
\\	「けど」	友達[ともだち]に 聞[き]いたけど、 知[し]らなかった。		
\\	消しゴムを貸
\\	くれ。	消しゴムを貸してくれ。	けしゴムをかしてくれ。	
\\	「~ください」- 
\\	「くださる」-
\\	「くれ」	消しゴム[けしごむ]を 貸[か]してくれ。		
\\	先生から今日は授業がない
\\	聞
\\	んだけど。	先生から今日は授業がないと聞いたんだけど。	せんせいからきょうはじゅぎょうがないときいたんだけど。	
\\	先生[せんせい]から 今日[きょう]は 授業[じゅぎょう]がないと 聞[き]いたんだけど。		
\\	時間が
\\	からパーティーに行きませんでした。	時間がなかったからパーティーに行きませんでした。	じかんがなかったからパーティーにいきませんでした。	
\\	「から」 
\\	「ので」	時間[じかん]がなかったからパーティーに 行[い]きませんでした。		
\\	全部飲
\\	も構わないよ。	全部飲んでも構わないよ。	ぜんぶのんでもかまわないよ。	
\\	全部[ぜんぶ] 飲[の]んでも 構[かま]わないよ。		
\\	そういう人と一緒に仕事をするのは、嫌だよね。	そういう人と一緒に仕事をするのは、嫌だよね。	そういうひとといっしょにしごとをするのは、いやだよね。	
\\	「いう」 
\\	「こう」、「そう」、「ああ」、
\\	「どう」	そういう 人[ひと]と 一緒[いっしょ]に 仕事[しごと]をするのは、 嫌[いや]だよね。		
\\	ああいう人と結婚できたら、幸せになれると思います。	ああいう人と結婚できたら、幸せになれると思います。	ああいうひととけっこんできたら、しあわせになれるとおもいます。	
\\	「いう」 
\\	「こう」、「そう」、「ああ」、
\\	「どう」	ああいう 人[ひと]と 結婚[けっこん]できたら、 幸[しあわ]せになれると 思[おも]います。		
\\	ご飯を食
\\	きゃ。	ご飯を食べなきゃ。	ごはんをたべなきゃ。	
\\	「なければ」
\\	ご 飯[はん]を 食[た]べなきゃ。		
\\	ルミネ
\\	いうデパートはどこにあるか、知っていますか?	ルミネというデパートはどこにあるか、知っていますか?	ルミネというデパートはどこにあるか、しっていますか?	
\\	「いう」 
\\	ルミネというデパートはどこにあるか、 知[し]っていますか?		
\\	彼は何を言
\\	かわからない。	彼は何を言ったかわからない。	かれはなにをいったかわからない。	
\\	「か」 
\\	彼[かれ]は 何[なに]を 言[い]ったかわからない。		
\\	犬に餌を
\\	あげる]?	犬に餌をやった?	いぬにえさをやった?	
\\	「やる」 
\\	「あげる」	犬[いぬ]に 餌[えさ]をやった?		
\\	時間が
\\	んでパーティーに行かなかった。	時間がなかったんでパーティーに行かなかった。	じかんがなかったんでパーティーにいかなかった。	
\\	「から」 
\\	「ので」 
\\	時間[じかん]がなかったんでパーティーに 行[い]かなかった。		
\\	図書館はどこですか。/図書館
\\	なら、あそこです。	図書館はどこですか。/図書館なら、あそこです。	としょかんはどこですか。/としょかんなら、あそこです。	
\\	「なら(ば)」	図書館[としょかん]はどこですか。/ 図書館[としょかん]なら、あそこです。		
\\	先生が教
\\	んですか。	先生が教えてくれるんですか。	せんせいがおしえてくれるんですか。	
\\	「くれる」 
\\	「あげる」 
\\	先生[せんせい]が 教[おし]えてくれるんですか。		
\\	彼は、これは何
\\	言いましたか。	彼は、これは何だと言いましたか。	かれは、これはなんだといいましたか。	
\\	彼[かれ]は、これは 何[なに]だと 言[い]いましたか。		
\\	この中から
\\	を選ぶの。	この中からどれかを選ぶの。	このなかからどれかをえらぶの。	
\\	この 中[なか]からどれかを 選[えら]ぶの。		
\\	毎日、肉を食
\\	ようになった。	毎日、肉を食べるようになった。	まいにち、にくをたべるようになった。	
\\	する 
\\	なる 
\\	に 
\\	「なる」 
\\	「する」 
\\	毎日[まいにち]、 肉[にく]を 食[た]べるようになった。		
\\	お金が
\\	ある]らいいね。	お金があったらいいね。	おかねがったらいいね。	
\\	「たら(ば)」	お 金[かね]があったらいいね。		
\\	友達からプレゼントを
\\	。	友達からプレゼントをもらった。	ともだちからプレゼントをもらった。	
\\	「もらう」	友達[ともだち]からプレゼントをもらった。		
\\	カレーを食べよう
\\	思ったけど、食べる時間がなかった。	カレーを食べようと思ったけど、食べる時間がなかった。	カレーをたべようとおもったけど、たべるじかんがなかった。	
\\	カレーを 食[た]べようと 思[おも]ったけど、 食[た]べる 時間[じかん]がなかった。		
\\	私は、アリス
\\	言
\\	。	私は、アリスと言います。	わたしは、アリスといいます。	
\\	私[わたし]は、アリスと 言[い]います。		
\\	久しぶりに彼の声が聞
\\	。	久しぶりに彼の声が聞けた。	ひさしぶりにかれのこえがきけた。	
\\	「見える」 
\\	「聞こえる」 
\\	久[ひさ]しぶりに 彼[かれ]の 声[こえ]が 聞[き]けた。		
\\	彼は高校生
\\	聞いたけど、信じられない。	彼は高校生だと聞いたけど、信じられない。	かれはこうこうせいだときいたけど、しんじられない。	
\\	彼[かれ]は 高校生[こうこうせい]だと 聞[き]いたけど、 信[しん]じられない。		
\\	ここに入
\\	はいけません。	ここに入ってはいけません。	ここにはいってはいけません。	
\\	「だめ」, 「いけない」, 
\\	「ならない」 
\\	ここに 入[はい]ってはいけません。		
\\	鈴木さんはどこですか。/もう、家に帰
\\	いる。	鈴木さんはどこですか。/もう、家に帰っている。	すずきさんはどこですか。/もう{うち
\\	いえ}にかえっている。	
\\	鈴木[すずき]さんはどこですか。/もう、 家[いえ]に 帰[かえ]っている。		
\\	ここに座
\\	なさい。	ここに座りなさい。	ここにすわりなさい。	
\\	「~なさい」 
\\	ここに 座[すわ]りなさい。		
\\	お酒を飲
\\	みましたが、すごく眠くなりました。	お酒を飲んでみましたが、すごく眠くなりました。	おさけをのんでみましたが、すごくねむくなりました。	
\\	お 酒[さけ]を 飲[の]んでみましたが、すごく 眠[ねむ]くなりました。		
\\	昼ご飯は、
\\	いいです。	昼ご飯は、どこでもいいです。	ひるごはんは、どこでもいいです。	
\\	昼[ひる]ご 飯[はん]は、どこでもいいです。		
\\	それを
\\	それをくれる?	それをくれる?	
\\	「~ください」- 
\\	「くださる」	それをくれる?		
\\	何を読
\\	る?/教科書を読
\\	います。	何を読んでる?/教科書を読んでいます。	なにをよんでる?/きょうかしょをよんでいます。	
\\	「~てる」 
\\	何[なに]を 読[よ]んでる?/ 教科書[きょうかしょ]を 読[よ]んでいます。		
\\	面白くない映画は見
\\	。	面白くない映画は見ません。	おもしろくないえいがほみません。	
\\	「~ます」 
\\	面白[おもしろ]くない 映画[えいが]は 見[み]ません。		
\\	鉛筆を家に持
\\	くる?	鉛筆を家に持ってくる?	えんぴつを{うち
\\	いえ}にもってくる。	
\\	(行く、来る) 
\\	鉛筆[えんぴつ]を 家[いえ]に 持[も]ってくる?		
\\	私が友達にプレゼントを
\\	。	私が友達にプレゼントをもらった。	わたしがともだちにプレゼントをもらった。	
\\	「もらう」	私[わたし]が 友達[ともだち]にプレゼントをもらった。		
\\	日本語をずっと前から勉強
\\	きて、結局はやめた。	日本語をずっと前から勉強してきて、結局はやめた。	にほんをずっとまえからべんきょうしてきて、けっきょくはやめた。	
\\	(行く、来る) 
\\	日本語[にほんご]をずっと 前[まえ]から 勉強[べんきょう]してきて、 結局[けっきょく]はやめた。		
\\	時間が
\\	、映画を見
\\	。	時間がありまして、映画を見ました。	じかんがありまして、えいがをみました。	
\\	時間[じかん]がありまして、 映画[えいが]を 見[み]ました。		
\\	何を
\\	する]たいですか。	何をしたいですか。	なにをしたいですか。	
\\	「たい」	何[なに]をしたいですか。		
\\	明日は何を食
\\	か?/カレーを食
\\	明日は何を食べようか?/カレーを食べよう!	あしたはなにをたべようか?/カレーをたべよう!	
\\	明日[あした]は 何[なに]を 食[た]べようか?/カレーを 食[た]べよう!		
\\	大きい縫い包みが欲
\\	大きい縫い包みが欲しい!	おおきいぬいぐるみがほしい!	
\\	「欲しい」	大[おお]きい 縫[ぬ]い 包[つつ]みが 欲[ほ]しい!		
\\	これは、先生が
\\	。	これは、先生がくれた。	これは、せんせいがくれた。	
\\	「くれる」	これは、 先生[せんせい]がくれた。		
\\	名前はボブ
\\	。	俺の名前はボブだ。	おれのなまえはボブだ。	
\\	俺[おれ]の 名前[なまえ]はボブだ。		
\\	私は、ハンバーガーとサラダ
\\	します。	私は、ハンバーガーとサラダにします。	わたしは、ハンバーガーとサラダにします。	
\\	する 
\\	なる 
\\	に 
\\	「なる」 
\\	「する」 
\\	私[わたし]は、ハンバーガーとサラダにします。		
\\	ケーキ、食
\\	たくないの?	ケーキ、食べたくないの?	ケーキ、たべたくないの?	
\\	「たい」	ケーキ、 食[た]べたくないの?		
\\	明日は何を食
\\	/カレーを食
\\	明日は何を食べましょうか?/カレーを食べましょう, 
\\	あしたはなにをたべましょうか?/カレーをたべましょう!	
\\	明日[あした]は 何[なに]を 食[た]べましょうか?/カレーを 食[た]べましょう,!		
\\	高い物を買
\\	くれる?	高い物を買わないでくれる?	たかいものをかわないでくれる?	
\\	「くれる」 
\\	「もらえる」 
\\	高[たか]い 物[もの]を 買[か]わないでくれる?		
\\	車を買
\\	の?	車を買ってくれるの?	くるまをかってくれるの?	
\\	「くれる」	車[くるま]を 買[か]ってくれるの?		
\\	ちょっと静かに
\\	する]くれない?	ちょっと静かにしてくれない?	ちょっとしずかにしてくれない?	
\\	「くれる」 
\\	「もらえる」	ちょっと 静[しず]かにしてくれない?		
\\	今週末は、
\\	行かなかった。	今週末は、どこにも行かなかった。	こんしゅうまつは、どこにもいかなかった。	
\\	今週[こんしゅう] 末[まつ]は、どこにも 行[い]かなかった。		
\\	今、時間が
\\	って聞いたんだけど、本当?	今、時間がないって聞いたんだけど、本当?	いま、じかんがないってきいたんだけど、ほんとう?	
\\	「って」 
\\	「と」	今[いま]、 時間[じかん]がないって 聞[き]いたんだけど、 本当[ほんとう]?		
\\	というか、もう帰らないとだめですけど。	というか、もう帰らないとだめですけど。	というか、もうかえらないとだめですけど。	
\\	「という」 
\\	「か」 
\\	というか、もう 帰[かえ]らないとだめですけど。		
\\	お父さんが言った。
\\	「それをください」とお父さんが言った。	「それをください」とおとうさんがいった。	
\\	「~ください」- 
\\	「くださる」	「それをください」とお 父[とう]さんが 言[い]った。		
\\	好きに
\\	する, 
\\	。	好きにしろ。	すきにしろ。	
\\	好[す]きにしろ。		
\\	「今日は授業がない」
\\	先生から聞
\\	んだけど。	「今日は授業がない」と先生から聞いたんだけど。	「きょうはじゅぎょうがない」とせんせいからきいたんだけど。	
\\	今日[きょう]は 授業[じゅぎょう]がない」と 先生[せんせい]から 聞[き]いたんだけど。		
\\	彼の日本語が上手
\\	なった。	彼の日本語が上手になった。	かれのにほんががじょうずになった。	
\\	する 
\\	なる 
\\	に 
\\	「なる」 
\\	「する」 
\\	彼[かれ]の 日本語[にほんご]が 上手[じょうず]になった。		
\\	スプーンを
\\	。	スプーンをちょうだい。	スプーンをちょうだい。	
\\	「~ちょうだい」 
\\	スプーンをちょうだい。		
\\	ここにあるレストランは
\\	美味しくない。	ここにあるレストランはどれも美味しくない。	ここにあるレストランはどれもおいしくない。	
\\	ここにあるレストランはどれも 美味[おい]しくない。		
\\	全部食
\\	もいいよ。	全部食べてもいいよ。	ぜんぶたべてもいいよ。	
\\	全部[ぜんぶ] 食[た]べてもいいよ。		
\\	ということは]、みきちゃんは、今彼氏がいない
\\	ということ]?	てことは、みきちゃんは、今彼氏がいないてこと?	てことは、みきちゃんは、いまかれしがいないてこと?	
\\	「って」 
\\	「て」 
\\	「という」	てことは、みきちゃんは、 今[こん] 彼氏[かれし]がいないてこと?		
\\	楽
\\	ば、私も行かない。	楽しくなければ、私も行かない。	たのしくなければ、わたしもいかない。	
\\	「ば」	楽[たの]しくなければ、 私[わたし]も 行[い]かない。		
\\	「友達」は、英語で
\\	いう意味です。	「友達」は、英語で
\\	という意味です。	「ともだち」は、えいがで
\\	といういみです。	
\\	「いう」 
\\	友達[ともだち]」は、 英語[えいご]で
\\	という 意味[いみ]です。		
\\	全部食
\\	もいいよ。	全部食べなくてもいいよ。	ぜんぶたべなくてもいいよ。	
\\	全部[ぜんぶ] 食[た]べなくてもいいよ。		
\\	お酒を飲
\\	うとしたが、奥さんが止めた。	お酒を飲もうとしたが、奥さんが止めた。	おさけをのもうとしたが、おくさんがとめた。	
\\	お 酒[さけ]を 飲[の]もうとしたが、 奥[おく]さんが 止[と]めた。		
\\	友達のおかげで、映画をただで見
\\	ことができた。	友達のおかげで、映画をただで見ることができた。	ともだちのおかげで、えいがをただでみることができた。	
\\	「見える」 
\\	「聞こえる」 
\\	友達[ともだち]のおかげで、 映画[えいが]をただで 見[み]ることができた。		
\\	食
\\	たくなかったけど食
\\	たくなった。	食べたくなかったけど食べたくなった。	たべたくなかったけどたべtくなった。	
\\	「たい」	食[た]べたくなかったけど 食[た]べたくなった。		
\\	たくさん食
\\	と太るよ。	たくさん食べると太るよ。	たくさんたべるとふとるよ。	
\\	「と」	たくさん 食[た]べると 太[ふと]るよ。		
\\	お好み焼きを始めて食
\\	みたけど、とてもおいしかった!	お好み焼きを始めて食べてみたけど、とてもおいしかった!	おこのみやきをはじめてたべてみたけど、とてもおいしかった!	
\\	お 好み焼[このみや]きを 始[はじ]めて 食[た]べてみたけど、とてもおいしかった!		
\\	映画を見たり、本を読
\\	。	映画を見たり、本を読んだりしない。	えいがをみたり、ほんをよんだりしない。	
\\	「~たりする」	映画[えいが]を 見[み]たり、 本[ほん]を 読[よ]んだりしない。		
\\	先に行
\\	いるよ。/美恵ちゃんは、もう来
\\	いるよ。	先に行っているよ。/美恵ちゃんは、もう来ているよ。	さきにいっているよ。/みえちゃんは、もうきているよ。	
\\	先[さき]に 行[い]っているよ。/ 美恵[みえ]ちゃんは、もう 来[き]ているよ。		
\\	毎日学校に行
\\	とだめです。	毎日学校に行かないとだめです。	まいにちがっこうにいかないとだめです。	
\\	と
\\	だめ
\\	毎日[まいにち] 学校[がっこう]に 行[い]かないとだめです。		
\\	そんなことは
\\	~える]。	そんなことはありうる。	そんなことはありうる。	
\\	「ある得る」, 
\\	そんなことはありうる。		
\\	毎日、肉を食
\\	ようにする。	毎日、肉を食べるようにする。	まいにち、にくをたべるようにする。	
\\	する 
\\	なる 
\\	に 
\\	「なる」 
\\	「する」 
\\	毎日[まいにち]、 肉[にく]を 食[た]べるようにする。		
\\	死
\\	じゃだめだよ!	死んじゃだめだよ!	しんじゃだめだよ。	
\\	「では」
\\	「だめ」
\\	死[し]んじゃだめだよ!		
\\	ここは静
\\	んで、とても穏やかです。	ここは静かなんで、とても穏やかです。	ここはしずかなんで、とてもおだやかです。	
\\	「から」 
\\	「ので」 
\\	ここは 静[しず]かなんで、とても 穏[おだ]やかです。		
\\	映画を見たり、本を読
\\	。	映画を見たり、本を読んだりしなかった。	えいがをみたり、ほんをよんだりしなかった。	
\\	「~たりする」	映画[えいが]を 見[み]たり、 本[ほん]を 読[よ]んだりしなかった。		
\\	鉛筆を持
\\	いる?	鉛筆を持っている?	えんぴつをもっている。	
\\	(行く、来る) 
\\	鉛筆[えんぴつ]を 持[も]っている?		
\\	晩ご飯を食
\\	ね。	晩ご飯を食べませんでしたね。	ばんごはんをたべませんでしたね。	
\\	「~ます」 
\\	晩[ばん]ご 飯[はん]を 食[た]べませんでしたね。		
\\	日本語で話
\\	くれ。	日本語で話してくれ。	にほんごではなしてくれ。	
\\	「~ください」- 
\\	「くださる」 
\\	「くれ」	日本語[にほんご]で 話[はな]してくれ。		
\\	車を買
\\	よ。	車を買ってあげるよ。	くるまをかってあげるよ。	
\\	「あげる」	車[くるま]を 買[か]ってあげるよ。		
\\	私は有名な人
\\	なる。	私は有名な人になる。	わたしはゆうめいなひとになる。	
\\	する 
\\	なる 
\\	に 
\\	「なる」 
\\	「する」 
\\	私[わたし]は 有名[ゆうめい]な 人[ひと]になる。		
\\	お父さんは、早く帰
\\	きました。	お父さんは、早く帰ってきました。	おとうさんは、はやくかえってきました。	
\\	(行く、来る) 
\\	お 父[とう]さんは、 早[はや]く 帰[かえ]ってきました。		
\\	新しいデパートに行
\\	みる。	新しいデパートに行ってみる。	あたらしいデパートにいってみる。	
\\	新[あたら]しいデパートに 行[い]ってみる。		
\\	地下に入って、富士山が見
\\	なった。	地下に入って、富士山が見えなくなった。	ちかにはいって、ふじさんがみえなくなった。	
\\	する 
\\	なる 
\\	に 
\\	「なる」 
\\	「する」 
\\	地下[ちか]に 入[はい]って、 富士山[ふじさん]が 見[み]えなくなった。		
\\	友達は何を
\\	する, 
\\	るの?/昼ご飯を食
\\	る。	友達は何をしてるの?/昼ご飯を食べてる。	ともだちはなにをしてるの?/ひるごはんをたべている。	
\\	「~てる」 
\\	友達[ともだち]は 何[なに]をしてるの?/ 昼[ひる]ご 飯[はん]を 食[た]べてる。		
\\	映画を見
\\	り、本を読
\\	り、昼寝
\\	りする。	映画を見たり、本を読んだり、昼寝したりする。	えいがをみたり、ほんをやんだり、ひるねしたりする。	
\\	「~たりする」	映画[えいが]を 見[み]たり、 本[ほん]を 読[よ]んだり、 昼寝[ひるね]したりする。		
\\	この大学の授業は簡単
\\	り、難
\\	りする。	この大学の授業は簡単だったり、難しかったりする。	このだいがくのじゅぎょうはかんたんだったり、むずかしかったりする。	
\\	「~たりする」	この 大学[だいがく]の 授業[じゅぎょう]は 簡単[かんたん]だったり、 難[むずか]しかったりする。		
\\	銀行に行
\\	らどうですか。	銀行に行ったらどうですか。	ぎんこうにいったらどうですか。	
\\	「ば」 
\\	「たら」 
\\	銀行[ぎんこう]に 行[い]ったらどうですか。		
\\	この質問の答えは、
\\	分かる。	この質問の答えは、誰でも分かる。	このしつもんのこたえは、だれでもわかる。	
\\	この 質問[しつもん]の 答[こた]えは、 誰[だれ]でも 分[わ]かる。		
\\	毎日学校に行
\\	はなりません。	毎日学校に行かなくてはなりません。	まいにちがっこうにいかなくてはなりません。	
\\	ては
\\	ならない
\\	毎日[まいにち] 学校[がっこう]に 行[い]かなくてはなりません。		
\\	夜、遅くまで電話
\\	はならない。	夜、遅くまで電話してはならない。	よる、おそくまででんわしてはならない。	
\\	「だめ」, 「いけない」, 
\\	「ならない」 
\\	夜[よる]、 遅[おそ]くまで 電話[でんわ]してはならない。		
\\	楽
\\	ば、私も行く。	楽しければ、私も行く。	たのしければ、わたしもいく。	
\\	「ば」	楽[たの]しければ、 私[わたし]も 行[い]く。		
\\	宿題をチェック
\\	する]もらいたかったけど、時間がなくて無理だった。	宿題をチェックしてもらいたかったけど、時間がなくて無理だった。	しゅくだいをチェックしてもらいたかったけど、じかんがなくてむりだった。	
\\	「もらう」	宿題[しゅくだい]をチェックしてもらいたかったけど、 時間[じかん]がなくて 無理[むり]だった。		
\\	こんなのを本当に食
\\	こんなのを本当に食べるか?	こんなのをほんとうにたべるか?	
\\	こんなのを 本当[ほんとう]に 食[た]べるか?		
\\	晩ご飯を作
\\	とく。	晩ご飯を作っとく。	ばんごはんをつくっとく。	
\\	「~とく」 
\\	晩[ばん]ご 飯[はん]を 作[つく]っとく。		
\\	ここに来
\\	ください。	ここに来ないでください。	ここにこないでください。	
\\	「~ください」- 
\\	「くださる」	ここに 来[こ]ないでください。		
\\	家に帰
\\	ら、誰もいなかった。	家に帰ったら、誰もいなかった。	いえにかえったら、だれもいなかった。	
\\	「たら(ば)」	家[いえ]に 帰[かえ]ったら、 誰[だれ]もいなかった。		
\\	え?何
\\	って?	え?何だって?	え?なんだって?	
\\	「って」 
\\	「と」	え? 何[なに]だって?		
\\	こんなのを本当に食
\\	こんなのを本当に食べる?	こんなのをほんとうにたべる?	
\\	こんなのを 本当[ほんとう]に 食[た]べる?		
\\	昨日何を食
\\	か忘れた。	昨日何を食べたか忘れた。	きのうなにをたべたかわすれた。	
\\	「か」 
\\	昨日[きのう] 何[なに]を 食[た]べたか 忘[わす]れた。		
\\	これは何だ
\\	言いましたか。	これは何だと言いましたか。	これはなんだといいましたか。	
\\	これは 何[なに]だと 言[い]いましたか。		
\\	アリスさんが言
\\	なら問題ないよ。	アリスさんが言うなら問題ないよ。	アリスさんがいうならもんだいないよ。	
\\	「なら(ば)」	アリスさんが 言[い]うなら 問題[もんだい]ないよ。		
\\	部屋を綺麗に
\\	する]ほしいのよ。	部屋を綺麗にしてほしいのよ。	へやをきれいにしてほしいのよ。	
\\	「欲しい」	部屋[へや]を 綺麗[きれい]にしてほしいのよ。		
\\	イタリア料理を食べに行
\\	。/すみません。ちょっと、お腹がいっぱいです。	イタリア料理を食べに行きませんか。/すみません。ちょっと、お腹がいっぱいです。	イタリアりょうりをたべにいきませんか。/すみません。ちょっと、おなかがいっぱいです。	
\\	イタリア 料理[りょうり]を 食[た]べに 行[い]きませんか。/すみません。ちょっと、お 腹[なか]がいっぱいです。		
\\	主人公が犯人だった
\\	いうのが一番面白かった。	主人公が犯人だったというのが一番面白かった。	しゅじんこうがはんにんだったというのがいちばんおもしろかった。	
\\	「いう」 
\\	主人公[しゅじんこう]が 犯人[はんにん]だったというのが 一番[いちばん] 面白[おもしろ]かった。		
\\	先生が学校に行
\\	か行
\\	か知らない。	先生が学校に行ったか行かなかったか知らない。	せんせいががっこうにいったかいかなかったかしらない。	
\\	「か」 
\\	先生[せんせい]が 学校[がっこう]に 行[い]ったか 行[い]かなかったか 知[し]らない。		
\\	そんなことは
\\	。	そんなことはありえない。	そんなことはありえない。	
\\	「ある得る」, 
\\	そんなことはありえない。		
\\	名前はボブ
\\	。	僕の名前はボブだ。	ぼくのなまえはボブだ。	
\\	僕[ぼく]の 名前[なまえ]はボブだ。		
\\	ここは静
\\	ので、とても穏やかです。	ここは静かなので、とても穏やかです。	ここはすずかなので、とてもおだやかです。	
\\	「から」 
\\	「ので」	ここは 静[しず]かなので、とても 穏[おだ]やかです。		
\\	名前はキム
\\	。	僕の名前はキムです。	ぼくのなまえはキムです。	
\\	僕[ぼく]の 名前[なまえ]はキムです。		
\\	どうしてパーティーに行きませんでしたか。/時間が
\\	からです。	どうしてパーティーに行きませんでしたか。/時間がなかったからです。	どうしてパーティーにいきませんでしたか。/じかんがなかったからです。	
\\	「から」 
\\	「ので」	どうしてパーティーに 行[い]きませんでしたか。/ 時間[じかん]がなかったからです。		
\\	まだ一杯あるから、たくさん食
\\	な。	まだ一杯あるから、たくさん食べな。	まだいっぱいあるから、たくさんたべな。	
\\	「~なさい」 
\\	まだ 一杯[いっぱい]あるから、たくさん 食[た]べな。		
\\	先生が学校に行
\\	かどうか知らない。	先生が学校に行ったかどうか知らない。	せんせいががっこうにいったかどうかしらない。	
\\	「どうか」 
\\	先生[せんせい]が 学校[がっこう]に 行[い]ったかどうか 知[し]らない。		
\\	消しゴムを貸
\\	。	消しゴムを貸して。	けしゴムをかして。	
\\	「~ください」- 
\\	「くださる」 
\\	消しゴム[けしごむ]を 貸[か]して。		
\\	私は医者
\\	なった。	私は医者になった。	わたしはいしゃになった。	
\\	する 
\\	なる 
\\	に 
\\	「なる」 
\\	「する」 
\\	私[わたし]は 医者[いしゃ]になった。		
\\	一年間練習したから、ピアノが弾
\\	ようになった。	一年間練習したから、ピアノが弾けるようになった。	いちねんかんれんしゅうしたから、ピアノがひけるようになった。	
\\	する 
\\	なる 
\\	に 
\\	「なる」 
\\	「する」 
\\	一 年間[ねんかん] 練習[れんしゅう]したから、ピアノが 弾[はじ]けるようになった。		
\\	友達がプレゼントを私に
\\	。	友達がプレゼントを私にくれた。	ともだちがプレゼントをわたしにくれた。	
\\	「くれる」 
\\	「あげる」 
\\	友達[ともだち]がプレゼントを 私[わたし]にくれた。		
\\	友達
\\	プレゼントが来た。	友達だからプレゼントが来た。	ともだちだからプレゼントがきた。	
\\	「から」 
\\	「ので」	友達[ともだち]だからプレゼントが 来[き]た。		
\\	はい、はい、
\\	。	はい、はい、分かった、分かった。	はい、はい、わかった、わかった。	
\\	はい、はい、 分[わ]かった、 分[わ]かった。		
\\	お酒は好きというか、ないと生きていけない。	お酒は好きというか、ないと生きていけない。	おさけはすきというか、ないといきていけない。	
\\	「という」 
\\	「か」 
\\	お 酒[さけ]は 好[す]きというか、ないと 生[い]きていけない。		
\\	毎日、勉強を避
\\	うとする。	毎日、勉強を避けようとする。	まいにち、べんきょうをさけようとする。	
\\	毎日[まいにち]、 勉強[べんきょう]を 避[さ]けようとする。		
\\	子犬はとても好
\\	。	子犬はとても好きです。	こいぬはとてもすきです。	
\\	「です」 
\\	子犬[こいぬ]はとても 好[す]きです。		
\\	早く寝
\\	うとしたけど、結局は徹夜した。	早く寝ようとしたけど、結局は徹夜した。	はやくねようとしたけど、けっきょくはてつやした。	
\\	早[はや]く 寝[ね]ようとしたけど、 結局[けっきょく]は 徹夜[てつや]した。		
\\	アメリカに行
\\	ら、たくさん太りました。	アメリカに行ったら、たくさん太りました。	アメリカにいったら、たくさんふとりました。	
\\	「たら(ば)」	アメリカに 行[い]ったら、たくさん 太[ふと]りました。		
\\	友達に会
\\	ば、買い物に行きます。	友達に会えれば、買い物に行きます。	ともだちにあえれば、かいものにいきます。	
\\	「ば」 
\\	友達[ともだち]に 会[あ]えれば、 買い物[かいもの]に 行[い]きます。		
\\	落書きを書
\\	ください。	落書きを書かないでください。	らくがきをかかないでください。	
\\	「~ください」- 
\\	「くださる」	落書[らくが]きを 書[か]かないでください。		
\\	千円を貸
\\	もらえる?	千円を貸してもらえる?	せんえんをかしてもらえる?	
\\	「くれる」 
\\	「もらえる」	千[せん] 円[えん]を 貸[か]してもらえる?		
\\	彼女は、学生
\\	、先生だ。	彼女は、学生じゃなくて、先生だ。	彼女は、がくせいじゃなくて、せんせいだ。	
\\	彼女[かのじょ]は、 学生[がくせい]じゃなくて、 先生[せんせい]だ。		
\\	リブート
\\	いうのは、パソコンを再起動する
\\	いうことです。	リブートというのは、パソコンを再起動するということです。	リブートというのは、パソコンをさいきどうするということです。	
\\	「いう」 
\\	リブートというのは、パソコンを 再[さい] 起動[きどう]するということです。		
\\	明日、大学に行
\\	。	明日、大学に行きます。	あした、だいがくにいきます。	
\\	「~ます」 
\\	明日[あした]、 大学[だいがく]に 行[い]きます。		
\\	もう信
\\	。	もう信じられない。	もうしんじられない。	
\\	もう 信[しん]じられない。		
\\	多分行かないと思う。というか、お金がないから、行けない。	多分行かないと思う。というか、お金がないから、行けない。	たぶんいかないとおもう。というか、おかねがないから、いけない。	
\\	「という」 
\\	「か」 
\\	多分[たぶん] 行[い]かないと 思[おも]う。というか、お 金[かね]がないから、 行[い]けない。		
\\	旅行の計画は終った?/うん、切符を買ったし、ホテルの予約も
\\	する]ある。	旅行の計画は終った?/うん、切符を買ったし、ホテルの予約もしてある。	りょこうのけいかくはおわった?/うん、きっぷをかったし、ホテルのよやくもしてある。	
\\	「~てある」 
\\	旅行[りょこう]の 計画[けいかく]は 終[おわ]った?/うん、 切符[きっぷ]を 買[か]ったし、ホテルの 予約[よやく]もしてある。		
\\	たまにご両親と話
\\	ばどう?	たまにご両親と話せばどう?	たまにごりょうしんとはなせばどう?	
\\	「ば」 
\\	「たら」 
\\	たまにご 両親[りょうしん]と 話[はな]せばどう?		
\\	私は学生
\\	ので、お金がないんです。	私は学生なので、お金がないんです。	わたしはがくせいなので、おかながないんです。	
\\	「から」 
\\	「ので」	私[わたし]は 学生[がくせい]なので、お 金[かね]がないんです。		
\\	みきちゃんは、あんたの彼女でしょう?/う~ん、彼女というか、友達というか、なんというか・・・	みきちゃんは、あんたの彼女でしょう?/う~ん、彼女というか、友達というか、なんというか・・・	みきちゃんは、あんたのかのじょでしょう?/う~ん、かのじょというか、ともだちというか、なんというか・・・	
\\	「という」 
\\	「か」 
\\	みきちゃんは、あんたの 彼女[かのじょ]でしょう?/う~ん、 彼女[かのじょ]というか、 友達[ともだち]というか、なんというか・・・		
\\	この魚は、鯛
\\	いいます。	この魚は、鯛といいます。	このさかなは、たいといいます。	
\\	「いう」 
\\	この 魚[さかな]は、 鯛[たい]といいます。		
\\	友達は
\\	遅れる。	友達はいつも遅れる。	ともだちはいつもおくれる。	
\\	友達[ともだち]はいつも 遅[おく]れる。		
\\	しないとだめだよ。/
\\	とはいっても]、時間がないからできないよ。	しないとだめだよ。/だって、時間がないからできないよ。	しないとだめだよ。/だって、じかんがないからできないよ。	
\\	「だって」 
\\	「とはいっても」	しないとだめだよ。/だって、 時間[じかん]がないからできないよ。		
\\	あんたは、いつもこういう時に来るんだから、困るんだよ。	あんたは、いつもこういう時に来るんだから、困るんだよ。	あんたは、いつもこういうときにくるんだから、こまるんだよ。	
\\	「いう」 
\\	「こう」、「そう」、「ああ」、
\\	「どう」	あんたは、いつもこういう 時[とき]に 来[く]るんだから、 困[こま]るんだよ。		
\\	昨日、友達が遊
\\	きた。	昨日、友達が遊びにきた。	きのう、ともだちがあそびにきた。	
\\	昨日[きのう]、 友達[ともだち]が 遊[あそ]びにきた。		
\\	学生
\\	ら、学生割引で買えます。	学生だったら、学生割引で買えます。	がくせいだったら、がくせいわりびきでかえます。	
\\	「たら(ば)」	学生[がくせい]だったら、 学生[がくせい] 割引[わりびき]で 買[か]えます。		
\\	そんなのは、
\\	か]	そんなのは、あるの?	そんなのは、あるの?	
\\	(の)
\\	そんなのは、あるの?		
\\	それを
\\	。	それをください。	それをください。	
\\	「~ください」- 
\\	「くださる」	それをください。		
\\	どうして友達じゃないんですか?先生
\\	し、年上
\\	し・・・。	どうして友達じゃないんですか?先生だし、年上だし・・・。	どうしてともだちじゃないんですか?せんせいだし、ねんうえだし・・・。	
\\	「し」	どうして 友達[ともだち]じゃないんですか? 先生[せんせい]だし、 年上[としうえ]だし・・・。		
\\	その部屋はあまり静
\\	。	その部屋はあまり静かじゃないです。	そのへやはあまりしずかじゃないです。	
\\	「です」 
\\	その 部屋[へや]はあまり 静[しず]かじゃないです。		
\\	一生懸命、頑張
\\	いく!	一生懸命、頑張っていく!	いっしょうけんめい、がんばっていく!	
\\	(行く、来る) 
\\	一生懸命[いっしょうけんめい]、 頑張[がんば]っていく!		
\\	ここに名前を
\\	。	ここに名前を書いてちょうだい。	ここになまえをかいてちょうだい。	
\\	「~ちょうだい」 
\\	ここに 名前[なまえ]を 書[か]いてちょうだい。		
\\	もう帰
\\	いい?	もう帰っていい?	もうかえっていい?	
\\	もう 帰[かえ]っていい?		
\\	アリスが、「寒い」と言
\\	。	アリスが、「寒い」と言った。	アリスが、「さむい」といった。	
\\	アリスが、
\\	寒[さむ]い」と 言[い]った。		
\\	駅の方へ走
\\	いった。	駅の方へ走っていった。	えきのほうへはしっていった。	
\\	(行く、来る) 
\\	駅[えき]の 方[ほう]へ 走[はし]っていった。		
\\	今、どこに行こう
\\	考えている。	今、どこに行こうかと考えている。	いま、どこにいこうかとかんがえている。	
\\	今[いま]、どこに 行[い]こうかと 考[かんが]えている。		
\\	周りがうるさくて、彼が言っていることがあんまり聞
\\	。	周りがうるさくて、彼が言っていることがあんまり聞こえなかった。	まわりがうるさくて、かれがいっていることがあんまりきこえなかった。	
\\	「見える」 
\\	「聞こえる」 
\\	周[まわ]りがうるさくて、 彼[かれ]が 言[い]っていることがあんまり 聞[き]こえなかった。		
\\	電池を買
\\	おきます。	電池を買っておきます。	でんちをかっておきます。	
\\	「~ておく」 
\\	電池[でんち]を 買[か]っておきます。		
\\	何
\\	言いましたか。	何と言いましたか。	なんといいましたか。	
\\	何[なに]と 言[い]いましたか。		
\\	話を聞
\\	いますか。/ううん、聞
\\	いない。	話を聞いていますか。/ううん、聞いていない。	はなしをきいていますか。/ううん、きいていない。	
\\	「~ている」 
\\	話[はなし]を 聞[き]いていますか。/ううん、 聞[き]いていない。		
\\	みきちゃんが洋介と別れたんだって。/ということは、みきちゃんは、今彼氏がいないということ?/ そう。そういうこと。	みきちゃんが洋介と別れたんだって。/ということは、みきちゃんは、今彼氏がいないということ?/ そう。そういうこと。	みきちゃんがようすけとわかれたんだって。/ということは、みきちゃんは、いまかれしがいないということ?/ そう。そういうこと。	
\\	「という」 
\\	「こと」 
\\	みきちゃんが 洋介[ようすけ]と 別[わか]れたんだって。/ということは、みきちゃんは、 今[こん] 彼氏[かれし]がいないということ?/そう。そういうこと。		
\\	電気を消
\\	と暗くなる。	電気を消すと暗くなる。	でんきをけすとくらくなる。	
\\	「と」	電気[でんき]を 消[け]すと 暗[くら]くなる。		
\\	宿題を
\\	する, 
\\	といけない。	宿題をしないといけない。	しゅくだいをしないといけない。	
\\	と
\\	いけない
\\	宿題[しゅくだい]をしないといけない。		
\\	「寒い」
\\	アリスが田中に言った。	「寒い」とアリスが田中に言った。	「さむい」とアリスがたなかにいった。	
\\	寒[さむ]い」とアリスが 田中[たなか]に 言[い]った。		
\\	お母さんは
\\	。母は買い物に行きました。	お母さんはどこですか。母は買い物に行きました。	おかあさんはどこですか。はははかいものにいきますた。	
\\	お 母[かあ]さんはどこですか。 母[はは]は 買い物[かいもの]に 行[い]きました。		
\\	先週に見た映画は、とても面白
\\	。	先週に見た映画は、とても面白かったです。	せんしゅうにみたえいがは、とてもおもしろかったです。	
\\	「です」 
\\	先週[せんしゅう]に 見[み]た 映画[えいが]は、とても 面白[おもしろ]かったです。		
\\	日本人はお酒に弱い
\\	いうのは本当?	日本人はお酒に弱いというのは本当?	にほんじんはおさけによわいというのはほんとう?	
\\	「いう」 
\\	日本人[にっぽんじん]はお 酒[さけ]に 弱[よわ]いというのは 本当[ほんとう]?		
\\	温泉に行
\\	たい。	温泉に行きたい。	おんせんにいきたい。	
\\	「たい」	温泉[おんせん]に 行[い]きたい。		
\\	漢字で書
\\	ください。	漢字で書いてください。	かんじでかいてください。	
\\	「~ください」- 
\\	「くださる」	漢字[かんじ]で 書[か]いてください。		
\\	明日
\\	って、雨が降
\\	んだって。	明日って、雨が降るんだって。	あしたって、あめがふるんだって。	
\\	「って」 
\\	「と」 
\\	「は」	明日[あした]って、 雨[あめ]が 降[ふ]るんだって。		
\\	友達が父にいいことを教
\\	。	友達が父にいいことを教えてくれた。	ともだちがちちにいいことをおしえてくれた。	
\\	「くれる」	友達[ともだち]が 父[ちち]にいいことを 教[おし]えてくれた。		
\\	食堂に行
\\	、昼ご飯を食
\\	、昼寝をする。	食堂に行って、昼ご飯を食べて、昼寝をする。	しょくどうにいって、ひるごはんをたべて、ひるねをする。	
\\	食堂[しょくどう]に 行[い]って、 昼[ひる]ご 飯[はん]を 食[た]べて、 昼寝[ひるね]をする。		
\\	冬に入って、コートを着ている人が増
\\	いきます。	冬に入って、コートを着ている人が増えていきます。	ふゆにはいって、コートをきているひとがふえていきます。	
\\	(行く、来る) 
\\	冬[ふゆ]に 入[はい]って、コートを 着[き]ている 人[ひと]が 増[ふ]えていきます。		
\\	色々な人と付き合
\\	きたけど、いい人はまだ見つからない。	色々な人と付き合ってきたけど、いい人はまだ見つからない。	いろいろなひととつきあってきたけど、いいひとはまだみつからない。	
\\	(行く、来る) 
\\	色々[いろいろ]な 人[ひと]と 付き合[つきあ]ってきたけど、いい 人[ひと]はまだ 見[み]つからない。		
\\	富士山が登
\\	。	富士山が登れた。	ふじさんがのぼれた。	
\\	富士山[ふじさん]が 登[のぼ]れた。		
\\	映画を見たり、本を読
\\	。	映画を見たり、本を読んだりした。	えいがをみたり、ほんをよんだりした。	
\\	「~たりする」	映画[えいが]を 見[み]たり、 本[ほん]を 読[よ]んだりした。		
\\	よく聞
\\	なさい!	よく聞きなさい!	よくききなさい!	
\\	「~なさい」 
\\	よく 聞[き]きなさい!		
\\	道は分
\\	か。	道は分かりますか。	みちはわかりますか。	
\\	道[みち]は 分[わ]かりますか。		
\\	今日は暇
\\	けど、明日は忙しい。	今日は暇だけど、明日は忙しい。	きょうはひまだけど、あしたはいそがしい。	
\\	「が」 
\\	「けど」	今日[きょう]は 暇[ひま]だけど、 明日[あした]は 忙[いそが]しい。		
\\	彼が寝坊したことも
\\	ね。	彼が寝坊したこともありうるね。	かれがねぼうしたこともありうるね。	
\\	「ある得る」, 
\\	彼[かれ]が 寝坊[ねぼう]したこともありうるね。		
\\	今日、知
\\	。	今日、知りました。	きょう、しりました。	
\\	今日[きょう]、 知[し]りました。		
\\	日本語で話
\\	。	日本語で話して。	にほんごではなして。	
\\	「~ください」- 
\\	「くださる」 
\\	日本語[にほんご]で 話[はな]して。		
\\	ので、友達に会う時間がない。	なので、友達に会う時間がない。	なので、ともだちにあうじかんがない。	
\\	「から」 
\\	「ので」	なので、 友達[ともだち]に 会[あ]う 時間[じかん]がない。		
\\	昨日、時間が
\\	。	昨日、時間がなかったんです。	きのう、じかんがなかったんです。	
\\	「です」 
\\	昨日[きのう]、 時間[じかん]がなかったんです。		
\\	犯人を
\\	で見ましたか。	犯人をどこかで見ましたか。	はんにんをどこかでみましたか。	
\\	犯人[はんにん]をどこかで 見[み]ましたか。		
\\	友達が私にプレゼントを
\\	。	友達が私にプレゼントをくれた。	ともだちがわたしにプレゼントをくれた。	
\\	「くれる」	友達[ともだち]が 私[わたし]にプレゼントをくれた。		
\\	漢字を書
\\	もらえませんか。	漢字を書いてもらえませんか。	かんじをかいてもらえませんか。	
\\	「くれる」 
\\	「もらえる」	漢字[かんじ]を 書[か]いてもらえませんか。		
\\	大学に行かないって、どういう意味なの?	大学に行かないって、どういう意味なの?	だいがくにいかないって、どういういみなの?	
\\	「いう」 
\\	「こう」、「そう」、「ああ」、
\\	「どう」	大学[だいがく]に 行[い]かないって、どういう 意味[いみ]なの?		
\\	名前はキム
\\	。	私の名前はキムです。	わたしのなまえはキムです。	
\\	私[わたし]の 名前[なまえ]はキムです。		
\\	毎日学校に行
\\	ばいけません。	毎日学校に行かなければいけません。	まいにちがっこうにいかなければいけません。	
\\	ば
\\	いけない
\\	毎日[まいにち] 学校[がっこう]に 行[い]かなければいけません。		
\\	他にいいものがたくさんあるけど、やっぱりこれ
\\	する。	他にいいものがたくさんあるけど、やっぱりこれにする。	ほかにいいものがたくさんあるけど、やっぱりこれにする。	
\\	する 
\\	なる 
\\	に 
\\	「なる」 
\\	「する」 
\\	他[た]にいいものがたくさんあるけど、やっぱりこれにする。		
\\	漢字は書
\\	か?	漢字は書けますか?	かんじはかけますか。	
\\	漢字[かんじ]は 書[か]けますか?		
\\	その部屋はあまり静
\\	よ。	その部屋はあまり静かじゃありませんよ。	そのへやはあまりしずかじゃありませんよ。	
\\	ある)	
\\	「です」 
\\	その 部屋[へや]はあまり 静[しず]かじゃありませんよ。		
\\	この質問の答えは、
\\	知らない。	この質問の答えは、誰も知らない。	このしつもんのこたえは、だれもしらない。	
\\	この 質問[しつもん]の 答[こた]えは、 誰[だれ]も 知[し]らない。		
\\	先生がこれを学生に
\\	。	先生がこれを学生にくれる。	せんせいがこれをがくせいにくれる。	
\\	「くれる」	先生[せんせい]がこれを 学生[がくせい]にくれる。		
\\	その時計は私から
\\	のよ。	その時計は私からもらったのよ。	そのとけいはわたしからもらったのよ。	
\\	「もらう」	その 時計[とけい]は 私[わたし]からもらったのよ。		
\\	今日は晴れて、富士山が見
\\	。	今日は晴れて、富士山が見える。	きょうははれて、ふじさんがみえる。	
\\	「見える」 
\\	「聞こえる」 
\\	今日[きょう]は 晴[は]れて、 富士山[ふじさん]が 見[み]える。		
\\	この歌を知
\\	いますか?	この歌を知っていますか?	このうたをしっていますか。	
\\	この 歌[うた]を 知[し]っていますか?		
\\	先生が教
\\	んですか。	先生が教えてあげるんですか。	せんせいがおしえてあげるんですか。	
\\	「くれる」 
\\	「あげる」 
\\	先生[せんせい]が 教[おし]えてあげるんですか。		
\\	電池を買
\\	ときます。	電池を買っときます。	だんちをかっときます。	
\\	「~とく」 
\\	電池[でんち]を 買[か]っときます。		
\\	来年留学するというのは、智子のこと?	来年留学するというのは、智子のこと?	らいねんりゅうがくするというのは、ともこのこと?	
\\	「って」 
\\	「て」 
\\	「という」, 
\\	来年[らいねん] 留学[りゅうがく]するというのは、 智子[さとこ]のこと?		
\\	ここに入
\\	ちゃだめだよ。	ここに入っちゃだめだよ。	ここにはいっちゃだめだよ。	
\\	「ては」
\\	だめ
\\	ここに 入[はい]っちゃだめだよ。		
\\	勉強をなるべく避
\\	うと思った。	勉強をなるべく避けようと思った。	べんきょうをなるべくさけようとおもった。	
\\	勉強[べんきょう]をなるべく 避[さ]けようと 思[おも]った。		
\\	勉強をたくさんしたから、頭がよ
\\	なった。	勉強をたくさんしたから、頭がよくなった。	べんきょうをたくさんしたから、あたまがよくなった。	
\\	する 
\\	なる 
\\	に 
\\	「なる」 
\\	勉強[べんきょう]をたくさんしたから、 頭[あたま]がよくなった。		
\\	鉛筆を学校へ持
\\	いく?	鉛筆を学校へ持っていく?	えんぴつをがっこうへもっていく?	
\\	(行く、来る) 
\\	鉛筆[えんぴつ]を 学校[がっこう]へ 持[も]っていく?		
\\	がおいしいクッキーを全部食べた。	誰かがおいしいクッキーを全部食べた。	だれかがおいしいクッキーをぜんぶたべた。	
\\	誰[だれ]かがおいしいクッキーを 全部[ぜんぶ] 食[た]べた。		
\\	行かなくてもいいよ。/
\\	とはいっても]、みんな行くって。私も行かないと。	行かなくてもいいよ。/だって、みんな行くって。私も行かないと。	いかなくてもいいよ。/だって、みんないくって。わたしもいかないと。	
\\	「だって」 
\\	「とはいっても」	行[い]かなくてもいいよ。/だって、みんな 行[い]くって。 私[わたし]も 行[い]かないと。		
\\	犬と遊
\\	たいですか。	犬と遊びたいですか。	いぬとあそびたいですか。	
\\	「たい」	犬[いぬ]と 遊[あそ]びたいですか。		
\\	全部飲
\\	も大丈夫だよ。	全部飲んでも大丈夫だよ。	ぜんぶのんでもだいじょうぶだよ。	
\\	全部[ぜんぶ] 飲[の]んでも 大丈夫[だいじょうぶ]だよ。		
\\	海外に行
\\	ことになった。	海外に行くことになった。	かいがいにいくことになった。	
\\	する 
\\	なる 
\\	に 
\\	「なる」 
\\	「する」 
\\	海外[かいがい]に 行[い]くことになった。		
\\	広島のお好み焼きを食
\\	みたい!	広島のお好み焼きを食べてみたい!	ひろしまのおこのみやきをたべてみたい!	
\\	広島[ひろしま]のお 好み焼[このみや]きを 食[た]べてみたい!		
\\	海外に行
\\	ことにした。	海外に行くことにした。	かいがいにいくことにした。	
\\	する 
\\	なる 
\\	に 
\\	「なる」 
\\	「する」 
\\	海外[かいがい]に 行[い]くことにした。		
\\	宿題を
\\	する, 
\\	ばだめだった。	宿題をしなければだめだった。	しゅくだいをしなければだめだった。	
\\	ば
\\	だめ
\\	宿題[しゅくだい]をしなければだめだった。		
\\	これは先生に
\\	。	これは先生にあげる。	これはせんせいにあげる。	
\\	「あげる」	これは 先生[せんせい]にあげる。		
\\	毎日ジムに行
\\	うと決めた。	毎日ジムに行こうと決めた。	まいにちジムにいこうときめた。	
\\	毎日[まいにち]ジムに 行[い]こうと 決[き]めた。		
\\	お金が
\\	ある]ばいいね。	お金があればいいね。	おかねがあればいいね。	
\\	「(ば)」	お 金[かね]があればいいね。		
\\	準備はどうですか。準備は、もう
\\	する]あるよ。	準備はどうですか。準備は、もうしてあるよ。	じゅんびはどうですか。/じゅんびは、もうしてあるよ。	
\\	「~てある」 
\\	準備[じゅんび]はどうですか。 準備[じゅんび]は、もうしてあるよ。		
\\	今日は何を
\\	する, 
\\	/テーマパークに行
\\	今日は何をしましょうか?/テーマパークに行きましょう!	あしたはなにをしましょうか?/テーマパークにいきましょう!	
\\	今日[きょう]は 何[なに]をしましょうか?/テーマパークに 行[い]きましょう!		
\\	というか]、もう帰らないとだめですけど。	ていうか、もう帰らないとだめですけど。	ていうか、もうかえらないとだめですけど。	
\\	「って」 
\\	「て」 
\\	「という」	ていうか、もう 帰[かえ]らないとだめですけど。		
\\	お金が
\\	ある]ばいいね。	お金があればいいね。	おかねがあればいいね。	
\\	「ば」	お 金[かね]があればいいね。		
\\	これは、なん
\\	いう魚ですか。	これは、なんという魚ですか。	これは、なんというさかなですか。	
\\	「いう」 
\\	これは、なんという 魚[さかな]ですか。		
\\	田中さんは、お金持
\\	、かっこよ
\\	、魅力的ですね。	田中さんは、お金持ちで、かっこよくて、魅力的ですね。	たなかさんは、おかねもちで、かっこよくて、みりょくてきですね。	
\\	田中[たなか]さんは、お 金持[かねも]ちで、かっこよくて、 魅力[みりょく] 的[てき]ですね。		
\\	よかったら、映画を観に行きますか?	もしよかったら、映画を観に行きますか?	もしよかったら、えいがをみにいきますか?	
\\	「もし」 
\\	たら	もしよかったら、 映画[えいが]を 観[かん]に 行[い]きますか?		
\\	これは友達に買
\\	もらった。	これは友達に買ってもらった。	これはともだちにかってもらった。	
\\	「もらう」	これは 友達[ともだち]に 買[か]ってもらった。		
\\	もう時間がないわ。	もう時間がないわ。	もうじかんがないわ。	
\\	「わ」、「ぞ」、「ぜ」、「かしら」	もう 時間[じかん]がないわ。		
\\	あっち行
\\	あっち行け!	あっちいけ!	
\\	あっち 行[い]け!		
\\	いい大学に行
\\	かな?	いい大学に行けるかな?	いいだいがくにいけるかな?	
\\	「な」 
\\	「さ」	いい 大学[だいがく]に 行[い]けるかな?		
\\	今日は雨が降
\\	かな?	今日は雨が降るかな?	きょうはあめがふるかな?	
\\	「な」 
\\	「さ」	今日[きょう]は 雨[あめ]が 降[ふ]るかな?		
\\	それを食
\\	な!	それを食べるな!	それをたべるな!	
\\	それを 食[た]べるな!		
\\	早く酒を持ってきて
\\	くれる, 
\\	。	早く酒を持ってきてくれ。	はやくさけをもってきてくれ。	
\\	早[はや]く 酒[さけ]を 持[も]ってきてくれ。		
\\	今、図書館に行くんだよ
\\	な。/うん、なんで?	今、図書館に行くんだよな。/うん、なんで?	いま、としょかんにいくんだよな。/うん、 なんで?	
\\	「な」 
\\	「さ」	今[いま]、 図書館[としょかん]に 行[い]くんだよな。/うん、なんで?		
\\	いい大学に入
\\	かしら?	いい大学に入れるかしら?	いいだいがくにはいれるかしら?	
\\	「わ」、「ぞ」、「ぜ」、「かしら」	いい 大学[だいがく]に 入[い]れるかしら?		
\\	変なことを言
\\	な!	変なことを言うな!	へんなことをいうな!	
\\	変[へん]なことを 言[い]うな!		
\\	日本語は、たくさん勉強したけどな。まだ全然分からない。/大丈夫よ。きっと分かるようになるからさ。/ならいいけどな。	日本語は、たくさん勉強したけどな。まだ全然分からない。/大丈夫よ。きっと分かるようになるからさ。/ならいいけどな。	にほんごは、たくさんべんきょうしたけどな。まだぜんぜんわからない。/だいじょうぶよ。きっとわかるようになるからさ。/ならいいけどな。	
\\	「な」 
\\	「さ」	日本語[にほんご]は、たくさん 勉強[べんきょう]したけどな。まだ 全然[ぜんぜん] 分[わ]からない。/ 大丈夫[だいじょうぶ]よ。きっと 分[わ]かるようになるからさ。/ならいいけどな。		
\\	これで、もう終わりだぜ。	これで、もう終わりだぜ。	これで、もうおわりだぜ。	
\\	「わ」、「ぞ」、「ぜ」、「かしら」	これで、もう 終[お]わりだぜ。		
\\	おい、行くぞ!	おい、行くぞ!	おい、いくぞ!	
\\	「わ」、「ぞ」、「ぜ」、「かしら」	おい、 行[い]くぞ!		
\end{CJK}
\end{document}