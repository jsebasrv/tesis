\documentclass[8pt]{extreport} 
\usepackage{hyperref}
\usepackage{CJKutf8}
\begin{document}
\begin{CJK}{UTF8}{min}
\\	外国語	
\\	がいこくご			
\\	外国語を習うのは難しい。	
\\	外国語 
\\	がいこくご を ならう の は むずかしい			
\\	着物	
\\	きもの			
\\	彼女は着物がよく似合います。	
\\	着物 
\\	かのじょ は きもの が よく にあいます			
\\	牛乳	
\\	ぎゅうにゅう			
\\	私は毎朝、牛乳を飲む。	
\\	牛乳 
\\	わたし は まいあさ ぎゅうにゅう を のむ			
\\	住所	
\\	じゅうしょ			
\\	この住所に行ってください。	
\\	住所 
\\	この じゅうしょ に いって ください			
\\	先週	
\\	せんしゅう			
\\	先週は海に行った。	
\\	先週 
\\	せんしゅう は うみ に いった			
\\	飛行場	
\\	ひこうじょう			
\\	バスが飛行場に着きました。	
\\	飛行場 
\\	ばす が ひこうじょう に つきました			
\\	救急車	
\\	きゅうきゅうしゃ			
\\	誰か救急車を呼んでください。	
\\	救急車 
\\	だれか きゅうきゅうしゃ を よんで ください			
\\	下りる	
\\	おりる			
\\	そこの階段を下りてください。	
\\	下りる 
\\	そこ の かいだん を おりて ください			
\\	果物	
\\	くだもの			
\\	デザートに果物を食べましょう。	
\\	果物 
\\	でざーと に くだもの を たべましょう			
\\	間違い	
\\	まちがい			
\\	この文には間違いがあります。	
\\	間違い 
\\	この ぶん に は まちがい が あります			
\\	あなた	
\\	これはあなたのカバンですか。	
\\	あなた 
\\	これ は あなた の かばん です か			
\\	映画館	
\\	えいがかん			
\\	彼と近くの映画館に行きました。	
\\	映画館 
\\	かれ と ちかく の えいがかん に いきました			
\\	踊る	
\\	おどる			
\\	彼女はクラブで踊るのが好きです。	
\\	踊る 
\\	かのじょ は くらぶ で おどる の が すき です			
\\	怖い	
\\	こわい			
\\	私は犬が怖いです。	
\\	怖い 
\\	わたし は いぬ が こわい です			
\\	鏡	
\\	かがみ			
\\	彼女は鏡を見て髪を直した。	
\\	鏡 
\\	かのじょ は かがみ を みて かみ を なおした			
\\	草	
\\	くさ			
\\	庭に草が生えている。	
\\	草 
\\	にわ に くさ が はえて いる			
\\	敬語	
\\	けいご			
\\	お客様には敬語を使いなさい。	
\\	敬語 
\\	おきゃくさま に は けいご を つかいなさい			
\\	欠席	
\\	けっせき			
\\	風邪のため今日は欠席します。	
\\	欠席 
\\	かぜ の ため きょう は けっせき します			
\\	スイッチ	
\\	彼はカーラジオのスイッチを入れた。	
\\	スイッチ 
\\	かれ は かーらじお の すいっち を いれた			
\\	掃除	
\\	そうじ			
\\	週末は部屋の掃除をしました。	
\\	掃除 
\\	しゅうまつ は へや の そうじ を しました			
\\	畳	
\\	たたみ			
\\	そのホテルには畳の部屋がある。	
\\	畳 
\\	その ほてる に は たたみ の へや が ある			
\\	速さ	
\\	はやさ			
\\	新幹線の速さはどれ位ですか。	
\\	速さ 
\\	しんかんせん の はやさ は どれ くらい です か			
\\	林	
\\	はやし			
\\	私たちは林の中に入っていった。	
\\	林 
\\	わたしたち は はやし の なか に はいって いった			
\\	プレゼント	
\\	誕生日にプレゼントをもらいました。	
\\	プレゼント 
\\	たんじょうび に ぷれぜんと を もらいました			
\\	弁当	
\\	べんとう			
\\	今日は弁当を持ってきました。	
\\	弁当 
\\	きょう は べんとう を もって きました			
\\	いつでも	
\\	いつでもうちに来てください。	
\\	いつでも 
\\	いつでも うち に きて ください			
\\	寂しい	
\\	さびしい			
\\	これは寂しい曲ですね。	
\\	寂しい 
\\	これ は さびしい きょく です ね			
\\	少しも	
\\	すこしも			
\\	あなたは少しも悪くない。	
\\	少しも 
\\	あなた は すこしも わるく ない			
\\	尊敬	
\\	そんけい			
\\	祖父は家族みんなに尊敬されています。	
\\	尊敬 
\\	そふ は かぞく みんな に そんけい されて います			
\\	テニス	
\\	彼らはよくテニスをしています。	
\\	テニス 
\\	かれら は よく てにす を して います			
\\	広さ	
\\	ひろさ			
\\	その家の広さはどれ位ですか。	
\\	広さ 
\\	その いえ の ひろさ は どれ くらい です か			
\\	読み	
\\	よみ			
\\	母に読み書きを習いました。	
\\	読み 
\\	はは に よみかき を ならいました			
\\	植える	
\\	うえる			
\\	庭にバラを植えました。	
\\	植える 
\\	にわ に ばら を うえました			
\\	我慢	
\\	がまん			
\\	彼のわがままには我慢できません。	
\\	我慢 
\\	かれ の わがまま に は がまん できません			
\\	国籍	
\\	こくせき			
\\	私は日本国籍です。	
\\	国籍 
\\	わたし は にほん こくせき です			
\\	こちら	
\\	受付はこちらです。	
\\	こちら 
\\	うけつけ は こちら です			
\\	恥ずかしい	
\\	はずかしい			
\\	とても恥ずかしかった。	
\\	恥ずかしい 
\\	とても はずかしかった			
\\	ボタン	
\\	ボタンを押してください。	
\\	ボタン 
\\	ぼたん を おして ください			
\\	洋服	
\\	ようふく			
\\	今日は洋服を買いに行きます。	
\\	洋服 
\\	きょう は ようふく を かい に いきます			
\\	油	
\\	あぶら			
\\	水と油は混ざらない。	
\\	油 
\\	みず と あぶら は まざらない			
\\	おかしい	
\\	彼の話はおかしかった。	
\\	おかしい 
\\	かれ の はなし は おかしかった			
\\	余る	
\\	あまる			
\\	夕食の料理がたくさん余った。	
\\	余る 
\\	ゆうしょく の りょうり が たくさん あまった			
\\	コート	
\\	新しいコートでテニスをしました。	
\\	コート 
\\	あたらしい こーと で てにす を しました			
\\	正直	
\\	しょうじき			
\\	彼女はとても正直だ。	
\\	正直 
\\	かのじょ は とても しょうじき だ			
\\	ドラマ	
\\	私はドラマを見るのが好きです。	
\\	ドラマ 
\\	わたし は どらま を みる の が すき です			
\\	ビザ	
\\	学生ビザを持っています。	
\\	ビザ 
\\	がくせい びざ を もって います			
\\	水道	
\\	すいどう			
\\	東京は水道の水が不味い。	
\\	水道 
\\	とうきょう は すいどう の みず が まずい			
\\	店員	
\\	てんいん			
\\	あの店員はとても親切です。	
\\	店員 
\\	あの てんいん は とても しんせつ です			
\\	電池	
\\	でんち			
\\	新しい電池を入れましょう。	
\\	電池 
\\	あたらしい でんち を いれましょう			
\\	ポケット	
\\	財布をポケットにしまった。	
\\	ポケット 
\\	さいふ を ぽけっと に しまった			
\\	握手	
\\	あくしゅ			
\\	彼らは握手をした。	
\\	握手 
\\	かれら は あくしゅ を した			
\\	固い	
\\	かたい			
\\	私の上司は頭が固い。	
\\	固い 
\\	わたし の じょうし は あたま が かたい			
\\	そろそろ	
\\	そろそろ始めましょうか。	
\\	そろそろ 
\\	そろそろ はじめましょう か			
\\	ぶどう	
\\	私はぶどうが好きです。	
\\	ぶどう 
\\	わたし は ぶどう が すき です			
\\	迷う	
\\	まよう			
\\	道に迷いました。	
\\	迷う 
\\	みち に まよいました			
\\	押さえる	
\\	おさえる			
\\	ドアを押さえてください。	
\\	押さえる 
\\	どあ を おさえて ください			
\\	見物	
\\	けんぶつ			
\\	皆で東京見物をした。	
\\	見物 
\\	みんな で とうきょう けんぶつ を した			
\\	知らせる	
\\	しらせる			
\\	皆に会議の日にちを知らせた。	
\\	知らせる 
\\	みんな に かいぎ の ひにち を しらせた			
\\	神社	
\\	じんじゃ			
\\	京都には神社がたくさんある。	
\\	神社 
\\	きょうと に は じんじゃ が たくさん ある			
\\	注射	
\\	ちゅうしゃ			
\\	彼は注射があまり好きではありません。	
\\	注射 
\\	かれ は ちゅうしゃ が あまり すき で は ありません			
\\	めったに	
\\	彼女はめったに怒りません。	
\\	めったに 
\\	かのじょ は めったに おこりません 。			
\\	恋人	
\\	こいびと			
\\	彼は恋人を失った。	
\\	恋人 
\\	かれ は こいびと を うしなった			
\\	財布	
\\	さいふ			
\\	会社に財布を忘れた。	
\\	財布 
\\	かいしゃ に さいふ を わすれた			
\\	作文	
\\	さくぶん			
\\	日本語で作文を書きました。	
\\	作文 
\\	にほんご で さくぶん を かきました			
\\	鳴る	
\\	なる			
\\	今朝、5時に電話が鳴った。	
\\	鳴る 
\\	けさ ごじ に でんわ が なった			
\\	真面目	
\\	まじめ			
\\	彼は真面目な人です。	
\\	真面目 
\\	かれ は まじめ な ひと です			
\\	申し込む	
\\	もうしこむ			
\\	彼女はそのセミナーに申し込んだ。	
\\	申し込む 
\\	かのじょ は その せみなー に もうしこんだ			
\\	冷蔵庫	
\\	れいぞうこ			
\\	飲み物は冷蔵庫にあります。	
\\	冷蔵庫 
\\	のみもの は れいぞうこ に あります			
\\	気に入る	
\\	きにいる			
\\	新しい靴がとても気に入りました。	
\\	気に入る 
\\	あたらしい くつ が とても きにいりました			
\\	クリスマス	
\\	クリスマスにはケーキを食べます。	
\\	クリスマス 
\\	くりすます に は けーき を たべます			
\\	ネクタイ	
\\	父の日にネクタイをプレゼントした。	
\\	ネクタイ 
\\	ちちのひ に ねくたい を ぷれぜんと した			
\\	厚さ	
\\	あつさ			
\\	私は板の厚さを測った。	
\\	厚さ 
\\	わたし は いた の あつさ を はかった			
\\	受け付ける	
\\	うけつける			
\\	郵便物は5時まで受け付けています。	
\\	受け付ける 
\\	ゆうびんぶつ は ごじ まで うけつけて います			
\\	誕生日	
\\	たんじょうび			
\\	彼女の誕生日は7月16日です。	
\\	誕生日 
\\	かのじょ の たんじょうび は しちがつ じゅうろくにち です			
\\	独身	
\\	どくしん			
\\	私の兄はまだ独身です。	
\\	独身 
\\	わたし の あに は まだ どくしん です			
\\	苦い	
\\	にがい			
\\	私は苦いコーヒーが好きです。	
\\	苦い 
\\	わたし は にがい こーひー が すき です			
\\	バイオリン	
\\	彼女はバイオリンを習っています。	
\\	バイオリン 
\\	かのじょ は ばいおりん を ならって います			
\\	パスポート	
\\	パスポートが見つからない。	
\\	パスポート 
\\	ぱすぽーと が みつからない			
\\	バッグ	
\\	私は黒いバッグを持っています。	
\\	バッグ 
\\	わたし は くろい ばっぐ を もって います			
\\	今まで	
\\	いままで			
\\	今までどこにいたのですか。	
\\	今まで 
\\	いままで どこ に いた の です か			
\\	切符	
\\	きっぷ			
\\	東京までの切符を買った。	
\\	切符 
\\	とうきょう まで の きっぷ を かった			
\\	ゴム	
\\	ゴムが伸びてしまった。	
\\	ゴム 
\\	ごむ が のびて しまった			
\\	ジェット機	
\\	ジェットき			
\\	ジェット機が飛んでいる。	
\\	ジェット機 
\\	じぇっとき が とんで いる			
\\	少々	
\\	しょうしょう			
\\	塩を少々入れてください。	
\\	少々 
\\	しお を しょうしょう いれて ください			
\\	随分	
\\	ずいぶん			
\\	随分早く終わったね。	
\\	随分 
\\	ずいぶん はやく おわった ね			
\\	引っ張る	
\\	ひっぱる			
\\	娘が私の手を引っ張った。	
\\	引っ張る 
\\	むすめ が わたし の て を ひっぱった			
\\	帽子	
\\	ぼうし			
\\	暑いので帽子を被りましょう。	
\\	帽子 
\\	あつい の で ぼうし を かぶりましょう			
\\	焼ける	
\\	やける			
\\	肉が焼けました。	
\\	焼ける 
\\	にく が やけました			
\\	鉛筆	
\\	えんぴつ			
\\	鉛筆を貸して下さい。	
\\	鉛筆 
\\	えんぴつ を かして ください			
\\	会話	
\\	かいわ			
\\	親子の会話は大切です。	
\\	会話 
\\	おやこ の かいわ は たいせつ です			
\\	ギター	
\\	彼女はギターが得意です。	
\\	ギター 
\\	かのじょ は ぎたー が とくい です			
\\	茶	
\\	ちゃ			
\\	暑さ	
\\	あつさ			
\\	今年の夏は暑さが厳しい。	
\\	暑さ 
\\	ことし の なつ は あつさ が きびしい			
\\	胃	
\\	い			
\\	昨日から胃が痛い。	
\\	胃 
\\	きのう から い が いたい			
\\	椅子	
\\	いす			
\\	そのお年寄りは椅子に座った。	
\\	椅子 
\\	その お としより は いす に すわった 。			
\\	お茶	
\\	おちゃ			
\\	私たちは毎日お茶を飲みます。	
\\	茶 
\\	お茶 
\\	わたしたち は まいにち おちゃ を のみます			
\\	お茶を入れましょうか。	
\\	お茶 
\\	おちゃ を いれましょう か			
\\	踊り	
\\	おどり			
\\	この踊りは易しいですよ。	
\\	踊り 
\\	この おどり は やさしい です よ			
\\	禁煙	
\\	きんえん			
\\	彼は今、禁煙しています。	
\\	禁煙 
\\	かれ は いま きんえん して います			
\\	次男	
\\	じなん			
\\	次男は今、海外にいます。	
\\	次男 
\\	じなん は いま かいがい に います			
\\	セーター	
\\	このセーターはウールだ。	
\\	セーター 
\\	この せーたー は うーる だ			
\\	引っ越す	
\\	ひっこす			
\\	来月、大阪に引っ越します。	
\\	引っ越す 
\\	らいげつ おおさか に ひっこします			
\\	丸い	
\\	まるい			
\\	地球は丸い。	
\\	丸い 
\\	ちきゅう は まるい。			
\\	写る	
\\	うつる			
\\	このカメラはよく写りますよ。	
\\	写る 
\\	この かめら は よく うつります よ			
\\	喫茶店	
\\	きっさてん			
\\	喫茶店でコーヒーを飲んだ。	
\\	喫茶店 
\\	きっさてん で こーひー を のんだ			
\\	全然	
\\	ぜんぜん			
\\	この本は全然面白くなかった。	
\\	全然 
\\	この ほん は ぜんぜん おもしろく なかった			
\\	やっぱり	
\\	やっぱり旅が大好きだ。	
\\	やっぱり 
\\	やっぱり たび が だいすき だ			
\\	週刊誌	
\\	しゅうかんし			
\\	その週刊誌はゴシップばかりだ。	
\\	週刊誌 
\\	その しゅうかんし は ごしっぷ ばかり だ			
\\	棚	
\\	たな			
\\	大きな棚はとても便利です。	
\\	棚 
\\	おおき な たな は とても べんり です			
\\	震える	
\\	ふるえる			
\\	彼は寒くて震えていました。	
\\	震える 
\\	かれ は さむくて ふるえて いました			
\\	磨く	
\\	みがく			
\\	靴を磨いてください。	
\\	磨く 
\\	くつ を みがいて ください			
\\	カーテン	
\\	朝、カーテンを開けた。	
\\	カーテン 
\\	あさ かーてん を あけた			
\\	火事	
\\	かじ			
\\	火事です。119番に電話してください。	
\\	火事 
\\	かじ です ひゃくじゅうきゅうばん に でんわ して ください			
\\	シャツ	
\\	このシャツはアイロンが必要だ。	
\\	シャツ 
\\	この しゃつ は あいろん が ひつよう だ			
\\	ナイフ	
\\	ナイフで手を切った。	
\\	ナイフ 
\\	ないふ で て を きった			
\\	皆さん	
\\	みなさん			
\\	皆さんにお話があります。	
\\	皆さん 
\\	みなさん に おはなし が あります			
\\	次女	
\\	じじょ			
\\	うちの次女は春から中学生です。	
\\	次女 
\\	うち の じじょ は はる から ちゅうがくせい です			
\\	貯金	
\\	ちょきん			
\\	貯金は十分にあります。	
\\	貯金 
\\	ちょきん は じゅうぶん に あります			
\\	バイク	
\\	兄はバイクが大好きです。	
\\	バイク 
\\	あに は ばいく が だいすき です			
\\	預かる	
\\	あずかる			
\\	荷物を預かってください。	
\\	預かる 
\\	にもつ を あずかって ください			
\\	いつか	
\\	僕はいつかアフリカに行きたい。	
\\	いつか 
\\	ぼく は いつか あふりか に いきたい			
\\	オレンジ	
\\	私はオレンジが好きです。	
\\	オレンジ 
\\	わたし は おれんじ が すき です			
\\	噛む	
\\	かむ			
\\	もっとよく噛みなさい。	
\\	噛む 
\\	もっと よく かみなさい			
\\	きつい	
\\	このシャツは少しきついです。	
\\	きつい 
\\	この しゃつ は すこし きつい です			
\\	削る	
\\	けずる			
\\	彼は家族のために仕事の時間を削った。	
\\	削る 
\\	かれ は かぞく の ため に しごと の じかん を けずった			
\\	越える	
\\	こえる			
\\	私たちは高い山を越えました。	
\\	越える 
\\	わたしたち は たかい やま を こえました			
\\	小包	
\\	こづつみ			
\\	フランスの友達から小包が届いた。	
\\	小包 
\\	ふらんす の ともだち から こづつみ が とどいた			
\\	新年	
\\	しんねん			
\\	新年明けましておめでとうございます。	
\\	新年 
\\	しんねん あけまして おめでとう ございます			
\\	配達	
\\	はいたつ			
\\	彼は新聞配達をしている。	
\\	配達 
\\	かれ は しんぶん はいたつ を して いる			
\\	動物園	
\\	どうぶつえん			
\\	昨日子供たちと動物園に行きました。	
\\	動物園 
\\	きのう こどもたち と どうぶつえん に いきました			
\\	封筒	
\\	ふうとう			
\\	その手紙を封筒に入れた。	
\\	封筒 
\\	その てがみ を ふうとう に いれた			
\\	危ない	
\\	あぶない			
\\	その道は車が多くて危ない。	
\\	危ない 
\\	その みち は くるま が おおくて あぶない			
\\	可愛い	
\\	かわいい			
\\	彼女の赤ちゃんは可愛いです。	
\\	可愛い 
\\	かのじょ の あかちゃん は かわいい です			
\\	さっき	
\\	さっきの話を続けましょう。	
\\	さっき 
\\	さっき の はなし を つづけましょう			
\\	宿題	
\\	しゅくだい			
\\	宿題、手伝ってくれない?	
\\	宿題 
\\	しゅくだい てつだって くれない			
\\	テキスト	
\\	テキストを読んでください。	
\\	テキスト 
\\	てきすと を よんで ください			
\\	ビニール	
\\	ゴミはそのビニール袋に入れてください。	
\\	ビニール 
\\	ごみ は その びにーるぶくろ に いれて ください			
\\	りんご	
\\	りんごを一つください。	
\\	りんご 
\\	りんご を ひとつ ください			
\\	アルバム	
\\	私は彼のアルバムを見た。	
\\	アルバム 
\\	わたし は かれ の あるばむ を みた			
\\	お祖父さん	
\\	おじいさん			
\\	私のお祖父さんは毎日散歩します。	
\\	お祖父さん 
\\	わたし の おじいさん は まいにち さんぽ します 。			
\\	水泳	
\\	すいえい			
\\	母は健康のために水泳をしている。	
\\	水泳 
\\	はは は けんこう の ため に すいえい を して いる			
\\	混む・込む	
\\	こむ			
\\	電車が込んでいる。	
\\	混む・込む 
\\	でんしゃ が こんで いる			
\\	柔道	
\\	じゅうどう			
\\	私は柔道を習っています。	
\\	柔道 
\\	わたし は じゅうどう を ならって います			
\\	スカート	
\\	彼女はあまりスカートははかない。	
\\	スカート 
\\	かのじょ は あまり すかーと は はかない			
\\	取れる	
\\	とれる			
\\	シャツのボタンが取れた。	
\\	ボタン 
\\	取れる 
\\	しゃつ の ぼたん が とれた			
\\	ペン	
\\	ペンを貸してください。	
\\	ペン 
\\	ぺん を かして ください			
\\	物	
\\	もの			
\\	彼女の家には物がたくさんあります。	
\\	物 
\\	かのじょ の いえ に は もの が たくさん あります			
\\	柔らかい	
\\	やわらかい			
\\	布団がとても柔らかい。	
\\	柔らかい 
\\	ふとん が とても やわらかい			
\\	郵便	
\\	ゆうびん			
\\	さっき郵便が届きました。	
\\	さっき 
\\	郵便 
\\	さっき ゆうびん が とどきました			
\\	割れる	
\\	われる			
\\	コップが落ちて割れた。	
\\	割れる 
\\	こっぷ が おちて われた			
\\	駅員	
\\	えきいん			
\\	駅員に聞きましょう。	
\\	駅員 
\\	えきいん に ききましょう			
\\	おしゃべり	
\\	妹はとてもおしゃべりです。	
\\	おしゃべり 
\\	いもうと は とても おしゃべり です			
\\	切手	
\\	きって			
\\	ここに切手を貼ってください。	
\\	切手 
\\	ここ に きって を はって ください			
\\	緑色	
\\	みどりいろ			
\\	彼は緑色のジャケットを着ています。	
\\	緑色 
\\	かれ は みどりいろ の じゃけっと を きて います			
\\	毎朝	
\\	まいあさ			
\\	私は毎朝ジョギングをします。	
\\	毎朝 
\\	わたし は まいあさ じょぎんぐ を します			
\\	曲げる	
\\	まげる			
\\	ひざを曲げてください。	
\\	曲げる 
\\	ひざ を まげて ください			
\\	幼稚園	
\\	ようちえん			
\\	娘は幼稚園に通っています。	
\\	幼稚園 
\\	むすめ は ようちえん に かよって います			
\\	きっと	
\\	明日はきっと雨が降ります。	
\\	きっと 
\\	あした は きっと あめ が ふります			
\\	理科	
\\	りか			
\\	私は理科が得意です。	
\\	理科 
\\	わたし は りか が とくい です			
\\	下着	
\\	したぎ			
\\	私は下着を手で洗う。	
\\	下着 
\\	わたし は したぎ を て で あらう			
\\	涼しい	
\\	すずしい			
\\	夕方は涼しくなりますよ。	
\\	涼しい 
\\	ゆうがた は すずしく なります よ			
\\	太る	
\\	ふとる			
\\	私の姉はすぐ太ります。	
\\	太る 
\\	わたし の あね は すぐ ふとります			
\\	冷房	
\\	れいぼう			
\\	冷房を入れてください。	
\\	冷房 
\\	れいぼう を いれて ください			
\\	別れる	
\\	わかれる			
\\	駅で友だちと別れました。	
\\	別れる 
\\	えき で ともだち と わかれました			
\\	空く	
\\	あく			
\\	後ろの席が空いています。	
\\	空く 
\\	うしろ の せき が あいて います			
\\	鍵	
\\	かぎ			
\\	出かける時は鍵を掛けてください。	
\\	鍵 
\\	でかける とき は かぎ を かけて ください			
\\	傘	
\\	かさ			
\\	電車に傘を忘れた。	
\\	傘 
\\	でんしゃ に かさ を わすれた			
\\	出掛ける	
\\	でかける			
\\	主人はもう出掛けました。	
\\	出掛ける 
\\	しゅじん は もう でかけました			
\\	毎晩	
\\	まいばん			
\\	姉は毎晩日記を書いています。	
\\	毎晩 
\\	あね は まいばん にっき を かいて います			
\\	真ん中	
\\	まんなか			
\\	道の真ん中に人が立っている。	
\\	真ん中 
\\	みち の まんなか に ひと が たって いる			
\\	指輪	
\\	ゆびわ			
\\	彼女に指輪をプレゼントしました。	
\\	指輪 
\\	かのじょ に ゆびわ を ぷれぜんと しました			
\\	幾つ	
\\	いくつ			
\\	娘さんは幾つになりましたか。	
\\	幾つ 
\\	むすめさん は いくつ に なりました か			
\\	腐る	
\\	くさる			
\\	リンゴが腐っている。	
\\	腐る 
\\	りんご が くさって いる			
\\	貧乏	
\\	びんぼう			
\\	彼は昔は貧乏だった。	
\\	貧乏 
\\	かれ は むかし は びんぼう だった			
\\	間違える	
\\	まちがえる			
\\	電話番号を間違えました。	
\\	間違える 
\\	でんわ ばんごう を まちがえました			
\\	メニュー	
\\	メニューをください。	
\\	メニュー 
\\	めにゅー を ください			
\\	アイスクリーム	
\\	弟はアイスクリームが大好きです。	
\\	アイスクリーム 
\\	おとうと は あいすくりーむ が だいすき です			
\\	おしゃれ	
\\	あの子はおしゃれだね。	
\\	おしゃれ 
\\	あの こ は おしゃれ だ ね			
\\	休憩	
\\	きゅうけい			
\\	少し休憩しましょう。	
\\	休憩 
\\	すこし きゅうけい しましょう			
\\	舌	
\\	した			
\\	舌を出してください。	
\\	舌 
\\	した を だして ください			
\\	ジュース	
\\	このジュースは甘すぎる。	
\\	ジュース 
\\	この じゅーす は あま すぎる			
\\	素晴らしい	
\\	すばらしい			
\\	素晴らしい景色ですね。	
\\	素晴らしい 
\\	すばらしい けしき です ね			
\\	ゼロ	
\\	今日の交通事故はゼロです。	
\\	ゼロ 
\\	きょう の こうつう じこ は ぜろ です			
\\	朝刊	
\\	ちょうかん			
\\	今日の朝刊に面白い記事があった。	
\\	朝刊 
\\	きょう の ちょうかん に おもしろい きじ が あった			
\\	丁寧	
\\	ていねい			
\\	彼女はいつも丁寧に仕事をします。	
\\	丁寧 
\\	かのじょ は いつも ていねい に しごと を します			
\\	苦手	
\\	にがて			
\\	私は料理が苦手です。	
\\	苦手 
\\	わたし は りょうり が にがて です			
\\	日	
\\	ひ			
\\	夏は日が長い。	
\\	日 
\\	なつ は ひ が ながい			
\\	引っ越し	
\\	ひっこし			
\\	去年、引っ越ししました。	
\\	引っ越し 
\\	きょねん ひっこし しました			
\\	歩道	
\\	ほどう			
\\	歩道を歩きましょう。	
\\	歩道 
\\	ほどう を あるきましょう			
\\	折れる	
\\	おれる			
\\	強風で木の枝が折れた。	
\\	折れる 
\\	きょうふう で き の えだ が おれた			
\\	見送り	
\\	みおくり			
\\	駅に友人の見送りに行きました。	
\\	見送り 
\\	えき に ゆうじん の みおくり に いきました			
\\	夕刊	
\\	ゆうかん			
\\	そのニュースは夕刊で見ました。	
\\	夕刊 
\\	その にゅーす は ゆうかん で みました			
\\	この間	
\\	このあいだ			
\\	この間彼女に会った。	
\\	この間 
\\	このあいだ かのじょ に あった			
\\	冷える	
\\	ひえる			
\\	クーラーで体が冷えた。	
\\	冷える 
\\	くーらー で からだ が ひえた			
\\	ピンク	
\\	娘がピンクのドレスを着ている。	
\\	ピンク 
\\	むすめ が ぴんく の どれす を きて いる			
\\	グラス	
\\	これはきれいなグラスですね。	
\\	グラス 
\\	これ は きれい な ぐらす です ね			
\\	毛	
\\	け			
\\	猫の毛がセーターに付いた。	
\\	毛 
\\	ねこ の け が せーたー に ついた			
\\	黒板	
\\	こくばん			
\\	答えを黒板に書いてください。	
\\	黒板 
\\	こたえ を こくばん に かいて ください			
\\	ダンス	
\\	彼女はダンスが大好きです。	
\\	ダンス 
\\	かのじょ は だんす が だいすき です			
\\	積もる	
\\	つもる			
\\	雪が積もっています。	
\\	積もる 
\\	ゆき が つもって います			
\\	半年	
\\	はんとし			
\\	日本に来て半年になります。	
\\	半年 
\\	にっぽん に きて はんとし に なります			
\end{CJK}
\end{document}