\documentclass[8pt]{extreport} 
\usepackage{hyperref}
\usepackage{CJKutf8}
\begin{document}
\begin{CJK}{UTF8}{min}
\\	ガリレオ・ガリレイ
\\	石原純
\\	【テキスト中に現れる記号について】
\\	ルビ
\\	(例)云《い》う
\\	ルビの付く文字列の始まりを特定する記号
\\	(例)数年|経《へ》て
\\	入力者注 主に外字の説明や、傍点の位置の指定
\\	(例)[#7字下げ]
\\	[#7字下げ]緒言[#「緒言」は中見出し]
\\	自然をふかく研究して、そのなかから新しい法則を見つけ出すということは、人間にとっての最も大きなよろこびであり、之《これ》によって自然の限りなく巧妙なはたらきを味わい知るということは、わたしたちの心を何よりもけだかく、美しくすることのできる真実の道でもあります。昔から偉大な科学者たちは世のなかの一切の栄誉などにかかわることなく、ひたすらに自然のなかにつき入ってその秘密をさぐることに熱中しました。そこにはいろいろな苦心が重ねられたのでありましたが、それでも世界のなかで誰も知らない事がらを、自分だけがつきとめたというすばらしい喜びは、それまでの並々ならぬ困難をつぐなって余りあるものに違いなかったのでした。そして、このようにして科学は時代とともに絶えず進んで来たのでしたが、それが今日どれほど多く世のなかの人の役に立っているかは、誰も知っている通りであります。この事をよく考えて見るならば、わたしたちがふだんの生活において科学を利用して非常な便利を得ているにつけても、今までの科学者たちの多大の苦心に対して心からの感謝をささげないではすまないのでありましょう。
\\	ところで、そのなかでも特に深く想い起されるのは、このような科学の進むべき正しい道をはっきりとわたしたちに示してくれた最初の科学者のことであります。科学は自然におけるいろいろなはたらきを研究してゆく学問であることは、上にも述べた通りであり、またそういう意味での自然の研究はごく古い時代からあったには違いないのですが、実際にその研究をどのような方法で進めてゆくべきかと云《い》うことを明らかにしたのは、十六世紀から十七世紀の前半にわたってイタリヤで名だかかったガリレオ・ガリレイであったということは、今日一般に認められている処《ところ》であって、その意味でこのガリレイは自然科学の先祖とあがめられているのです。それで私はここで幾《いく》らかのすぐれた科学者の事蹟《じせき》について皆さんにお話しして見《み》ようとするのに当《あた》って、まずガリレイのことから始めるのが、当然の順序であると考えるのです。
\\	[#7字下げ]ピザにおけるガリレイ[#「ピザにおけるガリレイ」は中見出し]
\\	ピザというのは、イタリヤの中部からやや北方にある都会で、そこにはヅオモ大寺と呼ばれる大きな寺院があり、そのなかに名だかい斜塔が立っています。十六七世紀頃にはかなりに盛んな町であったのですが、ガリレイはこの町で一五六四年の二月十五日に生まれました。父はヴィンセンツォ・ガリレイという人で、その家は以前にはイタリヤの貴族であってフィレンツェという都市に住んでいたのでしたが、この頃には零落《れいらく》してピザに移住していたのだと云《い》われています。それで生活にも余裕がなかったので、父はその息子のガリレオが育つにつれて、将来は商人にでもして家を興《おこ》してゆこうと考えたのでしたが、どうも息子が学問を好むので、ピザの大学で医学を学ばせることにしたのでした。ところがガリレオは医者になるのも好まなかったらしく、幼年の頃から好きな数学の講義を廊下で熱心に立ち聞きしているという有様なので、或《あ》る公爵家の家庭教師がそれを知って数学と物理学とを学ばせるように父親をも説得したということです。これで見てもガリレオが生来純粋の学問をどれほど望んでいたかがわかるわけです。それでともかくもガリレオは喜んで学業に励《はげ》みましたが、一五八九年になって、或《あ》る侯爵の推薦でこのピザの大学の数学教授に任命されました。それが僅《わず》かに二十五歳のことでありますから、彼の学才のいかにすぐれていたかが想察《そうさつ》されるのです。
\\	さてガリレイはその後一五九一年まで二年間この大学の教職に就《つ》いていましたが、その間に既《すで》にいろいろの研究にとりかかり、特に有名な自由落下の法則をまず最初に見つけ出しました。之《これ》はいろいろの物体が地球の上で自由に落ちる場合に、その速さがどう変ってゆくかを示す法則なのです。この問題について、その頃まではなお一般に昔のギリシャ時代の哲学者であったアリストテレスの説が信ぜられていたので、それによると比重の大きいものほど速く落ちるというので、例えば鉄片と木片とを同時に落すと、鉄片の方が遥《はる》かに速く落ちるということになりますが、ガリレイはそれを疑って、ともかく事実をたしかに突きとめなくてはならないと考えて、いろいろ実験を行って見たのでした。この実験をピザの斜塔で行ったということが話には伝わっていますが、それにはどうも確かな証拠はないようです。しかし、何《いず》れにしても、そのような実験からガリレイが自由落下の法則を見つけ出したのには違いないのでしょう。つまりガリレイは最初から科学では自然の事実に基づかなくてはいけないという信念を強く持っていたのでした。
\\	もう一つ有名な伝説として、ガリレイがピザの大寺院のなかでその天井からつり下げられている吊灯の揺れるのを見て、その往復する時間が揺れ方の大小に係わらないことを見つけ出したということが話されて居《お》り、之《これ》は彼の学生時代のことだと云《い》われていますが、之《これ》もよほど疑わしいので、現在この寺院にある青銅の吊灯にある銘を見ると、それより数年後の日附がしるされているのです。ですからこの伝説そのままはやはり信ぜられないのですが、同じく実験の上からガリレイが振子の揺れ方に関する法則を見つけ出したということだけは確かだと考えられています。ここでも彼は事実をいろいろ調べてその法則に到達したのに違いないのです。
\\	この頃には時計といってもごく粗雑なものしかなかったので、その後は医者が病人の脈搏《みゃくはく》の速さを測るのに、かような振子をつかった脈搏計《みゃくはくけい》というものをつくって、それを使ったそうで、これはなかなかおもしろい事がらだと思われます。
\\	[#7字下げ]壮年時代[#「壮年時代」は中見出し]
\\	ピザの大学でガリレイは教授ではありましたが、その俸給はごく少くて、ようやく自分一人が生活するにも足りない程度でした。ところが一五九一年に父が歿《な》くなったので、その家族を扶養しなくてはならなくなり、その儘《まま》では過ごすことができなくなったので、そこで以前にピザにゆく時に世話になった侯爵がまた彼のために奔走し、そのおかげで翌年バドーヴァの大学に転任することになりました。
\\	パドーヴァの大学にはその後十八年間在職しましたが、この時期こそガリレイの生涯において最も幸福な、また最も精根を尽して研究に専心《せんしん》することのできた時代であったのでした。その頃彼の学識の高いことはヨーロッパの諸国に広く伝えられたので、その名声を慕って諸国からたくさんの学徒が集まって来て、その講義は千人を容《い》れるだけの大講堂で行っても、なお狭くて収容しきれない程であったということでした。ところがそうなると、授業に費す時間がどうしても多くなって、それだけ自分の研究が妨げられるので、彼はようやくもっと自由の時間をもつことのできるような地位を望むようになり、一六一〇年になって再びピザに戻り、今度はそこで最も名誉のある「大公国の第一哲学者」として迎えられました。
\\	パドーヴァ時代にガリレイは、コペルニクスの書物を読んで、その学説の正しいことを感じ、自分でも之《これ》を研究してみたいと望んだのでした。コペルニクスという人はポーランドの国の僧侶であったのですが、イタリヤへ来て学問を修め、その後帰国してから、有名な地動説を称《とな》え、その書物は一五四三年に彼の没《ぼっ》する直前に出版されて、それから世に広まったのでしたが、その頃の宗教家のはげしい非難に遇《あ》って、殆《ほと》んど禁止の運命に置かれていたのでした。宗教家の反対というのはキリスト教の聖書に、我々人間は神にかたどってつくられたものであり、そしてこの人間の住んでいる地球は宇宙の中心にあって、あらゆる天体はそれをめぐっているということが記されているのに、コペルニクスの地動説では、太陽のまわりを地球が廻《まわ》っていると説くので、これは神聖な聖書にそむく虚偽異端の説であるというのでした。ガリレイは併《しか》し、この宇宙の正しい事実を言いあらわす科学こそ神の栄光と偉大さとをいとも驚くべくもの語るものであって、之《これ》を禁圧《きんあつ》するのは、それこそかえって神の意志に背くものであるという強固な信条のもとに、寧《むし》ろコペルニクスの説を肯定しようとしたのでした。併《しか》しその頃の宗教家たちには、そのようなすぐれた思想のわかる筈《はず》はありません。かえって自分たちの狭い考えに捉われて、依然として之《これ》に反対していました。
\\	ところが、その当時ドイツにヨハンネス・ケプラーというすぐれた若い学者があって、オーストリーのグラーツ大学で数学の講師をしていましたが、この人が惑星の軌道について研究した結果をガリレイの許《もと》に送って来ました。このケプラーは有名な惑星運動の法則を立てた人ですが、その仕事はずっと後に完成したので、この時の研究というのはそれ以前のものに過ぎなかったのですが、それでもガリレイは之《これ》に非常な興味を感じ、彼に親愛に充《み》ちた返書を送りました。そのなかには、「私はコペルニクスの運命を恐れています。彼は少数の人たちからは不朽の栄誉を得たとしても、愚者に充《み》ちた大多数の民衆にとっては軽蔑と汚辱との対象にしか過ぎないでしょう」と云《い》う言葉が記されています。
\\	その後ガリレイは天体観測を自分で行おうと考え、オランダで発明された望遠鏡の話を聞いて、それと同様のものを製作し、望遠鏡でいろいろな星を観測しました。之《これ》は一六〇九年のことで、その結果として月に高い山のあることや、銀河がたくさんの星の集まりであること、木星には四つの月が附随していること、金星、水星が月と同じように盈《み》ち虧《か》けを示すこと、太陽に黒点のあることなどを見つけ出し、それらの事がらからコペルニクスの説の真であることをますます確信するようになりました。
\\	[#7字下げ]宗教裁判とその晩年[#「宗教裁判とその晩年」は中見出し]
\\	ところが一六一〇年に、ガリレイがピザに帰ってからは、その地がローマ法王の直接の管下に属するだけに、ますます宗教家たちの反対が強くなり、異端説を主張するのをひどく責めるようになりました。その間にガリレイは、その誤解を説き、また科学と宗教との異なることを示そうとしてあらゆる努力を費しましたが、それは到底当時の人々の耳には入らなかったので、また中にはガリレイの名声の高いのを嫉《ねた》む人々の策謀などもそれに混って来て、遂には大僧正の命令で地動説を称《とな》えてはならないということを警告されました。之《これ》は一六一六年のことでしたが、その後も併《しか》しガリレイは自分の信念だけは変えませんでした。併《しか》しただ当分のうちはできるだけ事を荒立てないように黙って過ごしましたが、数年|経《へ》てからは事情もいくらか違って来たので、一六二九年になって問答の形式で普通に「天文対話」と呼ばれている書物を著し一六三二年に之《これ》を出版しました。
\\	ところがこの書物についてある僧侶がローマ法王に讒言《ざんげん》したので、法王は宗教裁判所に審査させることになり、その結果この讒言《ざんげん》は通らなかったのでしたが、ガリレイは之《これ》によって大僧正の以前の警告を無視しているという判決が下されて、ローマに出頭を命ぜられました。ガリレイはこの時|既《すで》に七十歳に近い老年で、おまけに病身で衰弱していましたが、その冬の寒い季節に止《や》むなく旅に出かけ、翌年の二月にようやくローマに到着しました。併《しか》し疲労が甚《はなは》だしいので暫《しばら》くの間静養が許され、四月になって裁判所で審問が始まりました。
\\	この審判の結果は、ガリレイの書物の領布を[#「領布を」はママ]禁じ、地動説を放棄することを条件として閑居《かんきょ》を命ぜられたので、その宣告の日には自分でその判決文を読んで宣誓のために署名をさせられたのでした。それからガリレイはフィレンツェの自分の家に帰って、そこに閉じこもって晩年を送りましたが、この間の彼の生活は実に寂しい有様ですごされました。その一人娘のバージニアが彼の病苦をやさしく慰めはしたものの、その後まもなく彼に先き立って没《な》くなりました。でも、ガリレイの唯一つの慰めはその科学上の研究にあったので、これ迄《まで》に行ったいろいろな研究をまとめて、それを一六三八年に出版しました。之《これ》は普通に「力学対話」と呼ばれていますが、以前の「天文対話」と同じように問答の形式に書かれているので、そこに始めて科学研究の正しい道が示されている点で非常に重要な書物なのであります。
\\	ところが、ガリレイ自身はそれ以前から眼をわずらっていて、この書物が出版された頃にはもはや両眼とも全く盲目になっていて、せっかくの自分の書物を見ることができなかったと云《い》うのですから、実に惨《いた》ましい極みでもありました。それでも彼の精神は最後までしっかりしていたとのことで、以前からガリレイのためになみなみならぬ心尽しをされたトスカナの大公爵はいつも彼の病床を見舞われて慰問をなし、有名な詩人ミルトンなども彼を訪ずれました。そしてその病床には最も忠実な弟子であったヴィヴィアニ及《およ》びトリチェリが絶えず傍に侍していたということです。かくてガリレイは一六四二年の一月八日に、七十八歳の高齢でこの世を去りましたが、一生を科学のために尽した満足をもって安らかにその生を終ったのでありましょう。ガリレイの死後にも寺院はなお迫害を加えていたのですが、後の時代になってはかえってそれとは反対に誰しもがガリレイの不朽の功績をたたえるようになったのですから、この事はあらゆる人々にとっての絶大な教訓でなければなりません。つまりそれはガリレイが何等《なんら》の私心もなく、ひたすらに真理のために尽した偉大な仕事のおかげによるのです。しかも科学の上での真理は永遠に消え失せることもなく、人間の社会が進めば進むほど、ますますそのなかにすばらしい輝きを増してくるようになるのです。今ではフィレンツェの聖十字院のなかにガリレイの立派な墓碑が立てられ、博物館にはその立像が置かれ、彼の製作した望遠鏡やその他の器械が陳列されて、そぞろに彼の面影を偲《しの》ばしめています。
\\	底本:「偉い科學者」實業之日本社
\\	1942(昭和17)年10月10日発行
\\	※「旧字、旧仮名で書かれた作品を、現代表記にあらためる際の作業指針」に基づいて、底本の表記をあらためました。
\\	「於て」は「おいて」に、「於ける」は「おける」に、「先づ」は「まず」に、「漸く」は「ようやく」に、「却って」は「かえって」に、「益々」は「ますます」に、置き換えました。
\\	※読みにくい言葉、読み誤りやすい言葉に振り仮名を付しました。底本には振り仮名が付されていません。
\\	※国立国会図書館デジタルコレクション
\\	で公開されている当該書籍画像に基づいて、作業しました。
\\	※「歿」と「没」の混在は、底本通りです。
\\	※イタリヤの地名「ピザ」は現在日本では「ピサ」と表現されていますが、底本通りとしました。
\\	入力:高瀬竜一
\\	校正:
\\	2019年1月29日作成
\\	青空文庫作成ファイル:
\\	このファイルは、インターネットの図書館、青空文庫
\\	で作られました。入力、校正、制作にあたったのは、ボランティアの皆さんです。
\end{CJK}
\end{document}