\documentclass[8pt]{extreport} 
\usepackage{hyperref}
\usepackage{CJKutf8}
\begin{document}
\begin{CJK}{UTF8}{min}
\\	父親	父親は今年50歳になる	父親[ちちおや]			
\\	母親		母親[ははおや]			
\\	長女	家内は長女で、兄弟がいません	長女[ちょうじょ]			
\\	長男	姉は元気な長男が生まれた	長男[ちょうなん]			
\\	次女	次女は母に似ている	次女[じじょ]			
\\	次男	次男はとても背が高い	次男[じなん]			
\\	末っ子	彼は5人兄弟に末っ子です	末っ子[すえっこ]			
\\	一人っ子	私の友達は、一人っ子が多い。	一人っ子[ひとりっこ]			
\\	一人娘	父親は一人娘結婚式で泣いた	一人娘[ひとりむすめ]			
\\	女 
\\	良 
\\	夫婦	
\\	さん夫婦は、毎日散歩している	夫婦[ふうふ]			
\\	大 
\\	女 
\\	帚 
\\	ご夫妻	部長ご夫妻の家に招待された	ご 夫妻[ふさい]			
\\	親戚	
\\	さんと私は親類です	親戚[しんせき]			親類:
\\	厂 
\\	上 
\\	小 
\\	戈 
\\	先祖	家の先祖に、有名な大統領がいたそうだ	先祖[せんぞ]			
\\	尊敬します	私は両親をとても尊敬している	尊敬[そんけい]します			
\\	いとこ	今日は、いとこが遊びに来る	いとこ			
\\	甥	甥が生まれて、私も叔父さんになった	甥[おい]			
\\	生 
\\	男 
\\	姪	甥も姪も、とても可愛い	姪[めい]			
\\	女 
\\	至 
\\	そっくりな	妹の顔は父親にそっくりだ	そっくりな			
\\	女性	理想の女性と結婚する	女性[じょせい]	
\\	理想: りそう
\\	忄 
\\	生 
\\	高齢	祖母は高齢だが、まだとても元気だ	高齢[こうれい]			
\\	歯 
\\	令 
\\	年上	年上の友達、彼女は私より三つ年上だ	年上[と年上え]			
\\	先輩	田中さんと私は同じ年だが、職場では彼のほうが先輩だ	先輩[せんぱい]			
\\	後輩:こうはい
\\	上司	上司に相談してから決定する	上司[じょうし]			
\\	決定する:けっていする
\\	相手	相手の目を見て話す	相手[あいて]			
\\	仲	私は山本さんと仲がいい	仲[なか]			
\\	生年月日	書類に生年月日を記入する	生年月日[せいねんがっぴ]			
\\	記入する:きにゅうする
\\	年が明ける	年の初めに1年の計画を立てる。	年[とし]が 明[あ]ける			
\\	年が終わる:
\\	出身	ご出身はどちらですか 
\\	ベトナムです 私はHCM大学の出身です	出身[しゅっしん]			
\\	成長	子供の成長を喜ぶ	成長[せいちょう]			
\\	成人	日本では二十歳以上の人を成人という	成人[せいじん]			
\\	退学	病気で大学を退学した	退学[たいがく]			
\\	艮 
\\	退職	母の介護のため、退職を決めた。	退職[たいしょく]			
\\	介護 
\\	かいご
\\	の 
\\	艮 
\\	耳 
\\	戠 
\\	失業	会社が倒産して失業した	失業[しつぎょう]			
\\	倒産:とうさん
\\	丿 
\\	夫 
\\	通勤	私は毎日1時間かけて通勤している	通勤[つうきん]			
\\	学歴	学歴が高くても、実力があるかどうかは分からない	学歴[がくれき]			
\\	止 
\\	観光	来日の目的は観光です	観光[かんこう]			
\\	来日:らいにち
\\	見 
\\	兀 
\\	帰省	お盆にはふるさとに帰省する日本人が多い	帰省[きせい]			
\\	故郷:ふるさと
\\	帰宅	毎日忙しくて帰宅が遅い	帰宅[きたく]			
\\	刂 
\\	帚 
\\	宀 
\\	乇	
\\	化粧	あなたは毎日、お化粧に何分ぐらいかけていますか	化粧[けしょう]			
\\	亻 
\\	匕 
\\	米 
\\	庄 
\\	成功	実験に成功する	成功[せいこう]			実験:じっけん
\\	戈 
\\	工 
\\	力 
\\	失敗	このパソコンを買ったのは失敗だった	失敗[しっぱい]			
\\	丿 
\\	夫 
\\	貝 
\\	攴 
\\	徹夜	徹夜で勉強する	徹夜[てつや]			
\\	身長	兄は身長が高い	身長[しんちょう]			
\\	体重	体重を測る 父の体重は60キロだ	体重[たいじゅう]			
\\	千 
\\	里 
\\	読書	趣味は読書です	読書[どくしょ]			
\\	冗談	冗談を言ったら、みんなが本気にした	冗談[じょうだん]			
\\	本気:ほんき
\\	な, 
\\	おしゃべり	授業中に隣の人とおしゃべりしていて、先生に怒られた	おしゃべり			
\\	我慢	痛くても我慢する	我慢[がまん]			
\\	丿 
\\	戈 
\\	亅 
\\	忄 
\\	曼 
\\	希望	最後まで希望を捨ててはいけない	希望[きぼう]			
\\	乂 
\\	布 
\\	王 
\\	賛成	賛成の人は手を挙げてください	賛成[さんせい]			
\\	挙げる:あげる
\\	貝 
\\	想像	想像と現実は違う	想像[そうぞう]			
\\	努力	一生懸命、努力をする	努力[どりょく]			
\\	奴 
\\	力 
\\	太陽	太陽が昇って、暖かくなった。	太陽[たいよう]			
\\	昇る:のぼる
\\	大 
\\	丶 
\\	昜 
\\	日 
\\	升 
\\	地球	地球の環境が悪化している	地球[ちきゅう]			
\\	土 
\\	也 
\\	王 
\\	求 
\\	王 
\\	睘 
\\	土 
\\	竟 
\\	湿度	日本の夏は特に湿度が高いので、蒸し暑いよね。	湿度[しつど]			
\\	蒸し暑い:むしあつい
\\	氵 
\\	广 
\\	廿 
\\	又 
\\	艹 
\\	烝 
\\	梅雨	6月から7月は梅雨の時期だ	梅雨[つゆ]			
\\	時期:じき
\\	木 
\\	毎 
\\	カビ	梅雨の時期はカビが生えやすい。	カビ			
\\	暖房	寒いので暖房をつける	暖房[だんぼう]			
\\	日 
\\	爰 
\\	戸 
\\	方 
\\	皮	りんごの皮を剥いて食べる	皮[かわ]			
\\	剥く:むく
\\	刂 
\\	警察	自転車を盗まれたので、警察に届けた。	警察[けいさつ]			
\\	盗む:ぬすむ
\\	敬 
\\	言 
\\	宀 
\\	祭 
\\	次 
\\	皿 
\\	作者	小説の作者を知っていますか。	作者[さくしゃ]			
\\	作品	学生の作品をロビーに展示する	作品[さくひん]			
\\	展示する:てんじする
\\	尸 
\\	廾 
\\	制服	日本の中学校には制服がある学校が多い	制服[せいふく]			
\\	刂 
\\	月 
\\	洗剤	洗剤で食器を洗う	洗剤[せんざい]			
\\	食器:しょっき
\\	氵 
\\	先 
\\	斉 
\\	刂 
\\	底	あの人は心の底では何を考えているか分からない	底[そこ]			
\\	广 
\\	氐 
\\	地下	大都市は地下の開発が進んでいる。	地下[ちか]			
\\	道路	日本では、車は道路の左側を走る	道路[どうろ]			
\\	首 
\\	足 
\\	各 
\\	坂	坂を上る	坂[さか]			
\\	土 
\\	反 
\\	煙	火事の時は、煙に注意して逃げてください	煙[けむり]			
\\	火 
\\	灰	紙が燃えて灰になる	灰[はい]			
\\	厂 
\\	火 
\\	判	書類に判を押す	判[はん]			
\\	半 
\\	刂 
\\	前半ー後半	映画の前半は退屈だったが、後半は面白かった	前半[ぜんはん]ー 後半[こうはん]			
\\	退屈:たいくつ
\\	尸 
\\	出 
\\	最高	この店は今日、開店以来最高の売り上げを記録した	最高[さいこう]			
\\	最低	クラスで最低の点を取ってしまった。	最低[さいてい]			
\\	性質	この布は燃えにくい性質を持っている	性質[せいしつ]			
\\	布:ぬの
\\	順番	大きい商品から順番に並べる	順番[じゅんばん]			
\\	川 
\\	頁 
\\	釆 
\\	田 
\\	番	次は私の番だ。	番[ばん]			
\\	生	回転ずしを試したら?生の魚は避けて、卵や納豆、キュウリや調理したエビ、焼いたウナギなんかのすしを食べられるのよ	生[なま]	
\\	試す:ためす
\\	避ける:よける
\\	キュウリ:
\\	調理:ちょうり
\\	ウナギ:
\\	渇きます	喉が渇いた	渇[かわ]きます			
\\	氵 
\\	曷 
\\	嗅ぎます	食べ物の匂いを嗅ぎました	嗅[か]ぎます			
\\	口 
\\	臭 
\\	叩きます	子供のお尻を叩きました	叩[はた]きます			
\\	口 
\\	卩 
\\	尸 
\\	九 
\\	殴ります	教師が生徒を殴って、問題になっている。	殴[なぐ]			
\\	区 
\\	殳 
\\	蹴ります	ボールを蹴ります。 彼は怒ると殴ったり蹴ったりする	蹴[け]ります			
\\	殴る:なぐる
\\	足 
\\	就 
\\	抱きます	子供を両手でしっかりと抱く	抱[いだ]きます			
\\	しっかり 
\\	倒れます	父は働きすぎて倒れてしまった	倒[たお]れます			
\\	亻 
\\	到 
\\	倒します	花瓶を倒して割れてしまった	倒[たお]します			
\\	亻 
\\	到 
\\	起きます。起こります	今日は8時に起きた。 父は毎晩遅くまで起きているようだ 最近なかなかやる気が起こらない	起きます。起こります			やる気:やるき
\\	起こします	家の子は今朝起こしてもなかなか起きない。	起[お]こします			
\\	尋ねる	交番で警官に市役所までの道を尋ねた	尋[たず]ねる	
\\	寸 
\\	叫びます	大声で叫んだが、相手は気づかずにいってしまった。	叫[さけ]びます	
\\	口 
\\	丩	
\\	黙る	先生が質問したが、だれも答えないで黙っている。	黙[だま]る	
\\	黒 
\\	数える	この寺は、日本で最も古い寺の一つに数えられている	数[かぞ]える			
\\	最も:もっとも
\\	攴 
\\	乾く	風が強かったので、外に干した洗濯物はすぐに乾いた	乾[かわ]く			かわく
\\	干す:ほす
\\	乾かす	ドライヤーで濡れた髪を乾かした。	乾[かわ]かす			
\\	ドライヤー
\\	畳む	洗濯物を畳む	畳[たた]む			たたむ
\\	田 
\\	奢る	昨日は後輩に焼肉を奢って	奢[おご]る	
\\	おごる
\\	大 
\\	者 
\\	預かる	旅行に行く友達から犬を預かることになった。	預[あず]かる	
\\	預かる:あずかる
\\	予 
\\	頁 
\\	預ける	銀行にお金を預けると、利子がつく	預[あづ]ける	
\\	預ける:あずける
\\	教わる	この料理の作り方は母から教わりました。	教[おそ]わる			
\\	断る	急用ができたので、先生に断って早退させてもらった	断[ことわ]る	
\\	急用:きゅうよう
\\	早退する:そうたい
\\	斤 
\\	見つかる	なくなったと思っていた指輪が、ソファーの下で見つかった	見[み]つかる			
\\	ソファー:
\\	見つける	なくなったと思っていた指輪をソファーのしたで見つけた	見[み]つける	
\\	見つける:みつける
\\	捕まる	犯人が警察に捕まった。 タクシーが捕まる。	捕[つか]まる			捕まる:つかまる
\\	扌 
\\	甫 
\\	捕まえる	川で魚を捕まえる タクシーを捕まえる	捕[つか]まえる			捕まえる:つかまえる
\\	扌 
\\	甫 
\\	乗る	毎朝、電車に乗って学校に通っている	乗[の]る			乗る:のる
\\	千 
\\	木 
\\	乗せる	子供を車に乗せて、学校まで送って行った。	乗[の]せる	
\\	乗せる:のせる
\\	千 
\\	木 
\\	降りる。下りる	電車を降ります。2階から1階に下りる。	降[お]りる					
\\	夅 
\\	一 
\\	卜 
\\	降ろす。下ろす	あの銀行の前で降ろしてください。 銀行口座 から3万円を下ろした。	降[お]ろす。 下[お]ろす	
\\	降ろす:おろす
\\	下ろす:おろす
\\	銀行口座:ぎんこうこうざ
\\	夅 
\\	一 
\\	卜 
\\	直る	発音の癖がなかなか直らない	直[なお]る	
\\	癖:くせ
\\	疒 
\\	辟 
\\	直す	先生が作文を直してくれました。 日本語を英語に直す。	直[なお]す			直す:なおす
\\	産む。生む	妻が先日元気な女の子を産んだ。 
\\	彼の努力がこの新記録を生んだ。 
\\	産[う]む。 生[う]む			産む:うむ
\\	生む:うむ
\\	生 
\\	出会う	駅で偶然大学時代の友人と出会った。	出会[であ]う	
\\	出会う:であう
\\	偶然:ぐうぜん
\\	な, 
\\	亻 
\\	禺 
\\	灬 
\\	訪ねる	さっきあなたを訪ねて、お客さんがいらっしゃいましたよ。	訪[たず]ねる	
\\	訪ねる:たずねる
\\	言 
\\	方 
\\	効く	この薬は頭痛によく効く	効[き]く			
\\	効く:きく
\\	頭痛:ずつう
\\	交 
\\	力 
\\	豆 
\\	頁 
\\	疒 
\\	甬 
\\	はやる	この冬は赤い色がはやっている。	はやる			
\\	経つ	日本へ来てから10年が経った	経[た]つ			経つ:たつ
\\	糸 
\\	圣 
\\	すれ違う	この道は狭いので、自動車がすれ違うのは大変だ。	すれ 違[ちが]う			すれ違う:すれちがう
\\	韋 
\\	離れる	危ないから、ストーブから離れて遊びなさい	離[はな]れる	
\\	離れる:はなれる
\\	离 
\\	隹 
\\	离 
\\	隹 
\\	離す	テストの時は、机を離して並べる	離[はな]す	
\\	離す:はなす
\\	离 
\\	隹 
\\	ぶつかる	道で自転車にぶつかって怪我をした	ぶつかる	
\\	打つかる:ぶつかる
\\	ぶつける	運転していて、車を電柱にぶつけてしまった	ぶつける			ぶつける:
\\	電柱:でんちゅう
\\	木 
\\	主 
\\	零れる	手が震えてお茶が零れてしまった。	零[こぼ]れる	
\\	零れる:こぼれる
\\	震える:ふるえる
\\	雨 
\\	令 
\\	雨 
\\	辰 
\\	零す	コープを倒して水を零してしまった	零[こぼ]す	
\\	零す:こぼす
\\	雨 
\\	令 
\\	拭く	布巾でテーブルの上を拭いておいてください	拭[ふ]く	
\\	拭く:ふく
\\	布巾:ふきん
\\	扌 
\\	式 
\\	片付く。片付ける	大掃除をして、やっと部屋が片付いた 洗った食器を食器棚に片付ける	片付[かたづ]ける			片付ける:かたづける
\\	片付く:かたづく
\\	包む	プレゼントをきれいな紙で包んでおいてください。	包[つつ]む	
\\	包む:くるむ
\\	勹 
\\	己 
\\	貼る	壁にポスターを貼る	張[は]る	
\\	貼る:はる
\\	無くなる	部屋のかぎが無くなってしまった。	無[な]くなる	
\\	無くなる:なくなる
\\	無くす	パスポートを無くして困っている	無[な]くす			無くす:なくす
\\	足りる	この収入では生活するのに全然足りない	足[た]りる	
\\	足りる:たりる
\\	残る	料理を作りすぎて、たくさん残ってしまった	残[のこ]る	
\\	残る:のこる
\\	歹 
\\	残す	父は私たちに多くの財産を残してくれた	残[のこ]す	
\\	残す:のこす
\\	財産:ざいさん
\\	腐る	腐ったものを食べて、お腹を壊してしまった	腐[くさ]る			腐る:くさる
\\	府 
\\	肉 
\\	剥ける	海で日焼けをして、背中の皮が剥けた。	剥[む]ける			剥ける:むける
\\	日焼け:ひやけ
\\	背中:せなか
\\	刂 
\\	剥く	柿は皮を剥いてからでないと、食べられません	剥[む]く	
\\	剥く:むく
\\	柿:かき
\\	木 
\\	市 
\\	積もる	昨日降った雪が積もっている	積[つ]もる	
\\	積もる:つもる
\\	禾 
\\	責 
\\	積む	机の上に本がたくさん積んである	積[つ]む	
\\	積む:つむ
\\	禾 
\\	責 
\\	空く	靴下に穴が空いてしまった. 
\\	平日は忙しいですが、土曜日なら空いています. 
\\	空[す]く			空く:すく
\\	空ける	壁に穴を空ける	空[あ]ける	
\\	空ける:あける
\\	穴:あな
\\	下がる	熱|温度|値段|成績 が下がる。 お皿をお下げしてもよろしいですか。	下[さ]がる	
\\	冷える	寒いところに長くいたので、手足が冷えてしまった 
\\	ジュースは冷えていないとおいしくない 
\\	冷[ひ]える	
\\	手足:てあし
\\	冷やす:ひやす
\\	温める:あたためる
\\	冷える:ひえる
\\	暖まる:あたたまる
\\	冷ます:さます
\\	暖める:あたためる
\\	冷める:さめる
\\	冷やす	果物は冷やして食べたほうがおいしい 
\\	こっちへ来て火で体を温めたらどう? 
\\	冷[ひ]やす			冷やす:ひやす
\\	温める:あたためる
\\	冷える:ひえる
\\	冷ます:さます
\\	暖める:あたためる
\\	冷める:さめる
\\	温める:あたためる
\\	冷める	この料理は冷めるとおいしくない。	冷[さ]める	
\\	冷やす:ひやす
\\	温める:あたためる
\\	冷える:ひえる
\\	暖まる:あたたまる
\\	冷ます:さます
\\	暖める:あたためる
\\	冷める:さめる
\\	冷ます	お風呂のおゆが熱くなりすぎた。少し冷ましてから入ろう。	冷[さ]ます	
\\	冷やす:ひやす
\\	温める:あたためる
\\	冷える:ひえる
\\	暖まる:あたたまる
\\	冷ます:さます
\\	暖める:あたためる
\\	冷める:さめる
\\	燃える	希望に燃えて大学に入った。	燃[も]える	
\\	燃える:もえる
\\	燃やす	メンバー全員が試合にファイトを燃やしている	燃[も]やす			燃やす:もやす
\\	ファイト
\\	燃 
\\	訓: も.える も.やす も.す 音: ネン 
\\	火 
\\	然 
\\	沸く	お湯が沸く 歌手が登場して会場が沸いた	沸[わ]く	
\\	沸く:わく
\\	登場:とうじょう
\\	沸かす	湯を沸かす おもしろいことを言って会場をわかした	沸[わ]かす	
\\	沸かす:わかす
\\	鳴る	雷が鳴っている	鳴[な]る			鳴る:なる
\\	雷:かみなり
\\	雨 
\\	田 
\\	鳴らす	ブザーを鳴らす	鳴[な]らす	
\\	鳴らす:ならす
\\	ブザー
\\	役立つ	今度の新入社員はあまり役に立たない	役立[やくだ]つ			役立つ:やくだつ
\\	役立てる	インターネットの情報を、勉強や仕事の役に立てる	役立[やくだ]てる			役立てる:やくだてる
\\	飾り	クリスマスの飾りを買った	飾[かざ]り	
\\	飾り:かざり
\\	遊ぶ	かくれんぼは子供の遊びだ	遊[あそ]ぶ	
\\	かくれんぼ
\\	集まり	あした、マンションの住人の集まりがある	集[あつ]まり	
\\	集まり:あつまり
\\	教え	母の教えを今でも思い出す	教[おし]え	
\\	教え:おしえ
\\	考え	いい考えがある	考[かんが]え	
\\	考え:かんがえ
\\	片付け	私は片づけが苦手だ	片付[かたづ]け	
\\	片付け:かたづけ
\\	苦手:にがて
\\	な, 
\\	決まり	学校で新しい決まりが作られた	決[き]まり	
\\	決まり:きまり
\\	騒ぎ	昨日、駅前で騒ぎがあった	騒[さわ]ぎ	
\\	騒ぎ:さわぎ
\\	頼み	あなたに頼みがあるんです	頼[たの]み	
\\	頼み:たのみ
\\	暮れ	年の暮れは忙しい	暮[く]れ	
\\	暮れ:くれ
\\	の, 
\\	莫 
\\	日 
\\	夢中	子供のころ、SF漫画に夢中になった。	夢中[むちゅう]な	
\\	夢中:むちゅう
\\	な, 
\\	退屈	他の観客は笑いながら見ていたが、私は退屈だった	退屈[たいくつ]	
\\	退屈:たいくつ
\\	な, 
\\	平気	彼女は、どんなに悪口を言われても平気な顔をしている。	平気[へいき]	
\\	平気:へいき
\\	な, 
\\	悪口:わるくち
\\	苦しい	苦しい試合だったが、何とか勝つことができた 
\\	経営がうまくいかず、社長は苦しい立場に置かれた 
\\	苦[くる]しい			
\\	立場に置く:
\\	苦しさ 
\\	苦しみ 
\\	寝苦しい 
\\	聞き苦しい 
\\	見苦しい 
\\	苦しむ 
\\	苦しめる 
\\	平気な	彼女は、どんなに悪口を言われても平気な顔をしている 
\\	体力があるので、10キロぐらい歩くのは平気だ 
\\	平気[へいき]な			平気:へいき
\\	な, 
\\	うらやましい	才能の豊かな人がうらやましい 
\\	弟はゲームをたくさん持っている友達をうらやましがっている 
\\	うらやましい			才能:さいのう
\\	豊かな:ゆたかな
\\	うらやましさ 
\\	うらやむ 
\\	痒い	蚊に刺されて首が痒い	痒[かゆ]い	
\\	痒い:かゆい
\\	蚊:か
\\	刺す:さす
\\	首:くび
\\	おとなしい	うちの犬はおとなしくて、決して人に吼えない 
\\	私が買い物をしてる間、子供はおとなしく待っていた 
\\	おとなしい			
\\	決して:けっして
\\	吼える:ほえる
\\	間:あいだ
\\	我慢強い	我慢強い性格 
\\	チャンスがやって来るのを我慢強く待つ 
\\	我慢強[がまんづよ]い			
\\	しんぼう強い 
\\	辛抱強い	
\\	正直な	彼は正直な人だから、うそをつくことができない 
\\	どうして悪戯したの。正直に話しなさい 
\\	正直[しょうじき]な			
\\	悪戯する:いたずらする
\\	不正直な 
\\	亜 
\\	心 
\\	虚 
\\	戈 
\\	けちな	あの人はとてもけちだ 
\\	けちな人は嫌われる 
\\	けちな			けちな:
\\	我が儘な	我が儘な行動 
\\	彼女は我が儘だ 
\\	我が儘に振る舞う 
\\	我[わ]が儘な			我が儘:わがまま
\\	な, 
\\	振る舞う:ふるまう
\\	丿 
\\	戈 
\\	亅 
\\	亻 
\\	盡 
\\	扌 
\\	辰 
\\	舛 
\\	積極的な	日本語を身につけるため、積極的に日本人の友達を作った 
\\	今の首相は、教育問題にあまり積極的ではない 
\\	積極[せっきょく] 的[てき]な			積極的:せっきょくてき
\\	な
\\	身につける:みにつける
\\	身につく:みにつく
\\	禾 
\\	責 
\\	木 
\\	亟 
\\	消極的	今の首相は、教育問題に消極的だ	消極[しょうきょく] 的[てき]	
\\	消極的:しょうきょくてき
\\	な
\\	氵 
\\	肖 
\\	木 
\\	亟 
\\	満足	アルバイトだけでは満足な収入は得られない 
\\	頑張ったので2位でも満足だ 
\\	満足[まんぞく]			
\\	不満な, 不満足な 
\\	満足 
\\	不満	この小説は面白かったが、終わり方がちょっと不満だ 
\\	あのレストランの料理は、味に不満はないが量が少ない 
\\	不満[ふまん]			
\\	不安	最近、体の調子がよくない。悪い病気ではないかと不安だ 
\\	新しい仕事がなかなか見つかれず、不安な毎日を過ごしている 
\\	不安[ふあん]						
\\	楽な	もう少し楽な暮らしがしたい 
\\	肩こりがひどいのでマッサージをしてもらったら、楽になった 
\\	この仕事は、電話を受けるだけでいいのでとても楽だ 
\\	楽[らく]な						
\\	面倒	ご入会には面倒な手続きがいりません 
\\	雨が降っているので、出かけるのが面倒だ 
\\	ご面倒をおかけして、申し訳ありません 
\\	面倒[めんどう]			
\\	意外	事故を調査するうちに意外な事実が分かった 
\\	彼が結婚していて子供もいるとは意外だった 
\\	意外[いがい]			
\\	結構	お味はいかがですか。大変結構です 
\\	この前は、結構な品をいただき、ありがとうございました 
\\	結構[けっこう]		。	
\\	木 
\\	冓 
\\	派手	派手な化粧 
\\	この服は派手すぎる 
\\	派手に騒ぐ 
\\	派手[はで]			派手:はで
\\	な, 
\\	地味:じみ
\\	な, 
\\	地味	地味なデザイン 
\\	あの人は若いのに、地味な格好ばかりしている 
\\	地味[じみ]			地味:じみ
\\	な, 
\\	格好:かっこう
\\	御洒落	彼はいつもスーツをおしゃれに着こなしている 
\\	駅前におしゃれなレストランができたわよ 
\\	御[お] 洒落[しゃれ]			御洒落:おしゃれ
\\	な, 
\\	彳 
\\	卸 
\\	氵 
\\	西 
\\	艹 
\\	洛 
\\	変	変なメールが届いたのですぐに消した 
\\	停電の後、テレビの画面が変になった 
\\	変[へん]			
\\	停電:ていでん
\\	おかしい 
\\	不思議	この村には不思議な話が伝わっている 
\\	誰もいない部屋から声が聞こえてくるとは不思議だ 
\\	不思議[ふしぎ]			不思議:ふしぎ
\\	な, 
\\	増し	咳が酷かったが、含嗽をしたら前より増しになった 
\\	失敗するかもしれないが、何もやらないよりは増しだ 
\\	文章が間違いだらけだ、もう少し増しなものが書けないのだろうか 
\\	増[ま]し			増し:まし
\\	な
\\	咳:せき
\\	酷い:ひどい
\\	含嗽:うがい
\\	今 
\\	口 
\\	口 
\\	無駄	落ちるとわかっているのに試験を受けるのは、無駄なことだ 
\\	必要ないものを買ってお金を無駄に使ってしまった 
\\	無駄[むだ]			無駄:むだ
\\	な, 
\\	自由	赤ちゃんがいるので、自由な時間がほとんどない 
\\	ここにあるパソコンはご自由にお使いください 
\\	自由[じゆう]						
\\	不自由	目/耳/足/体が不自由な人 
\\	地震でガスと電気が止まり、しばらく不自由な生活をした 
\\	パソコンがないと仕事をするのに不自由だ 
\\	不自由[ふじゆう]			不自由:ふじゆう
\\	な, 
\\	温まる	お風呂に入ると体が温まる	温[あたた]まる	
\\	温まる:あたたまる
\\	温める	エアコンをつけて部屋を暖める 
\\	スープを温める 
\\	温[あたた]める			温める:あたためる
\\	高まる	台風が近づき、波が高まっている 
\\	若者の政治への関心が高まっている 
\\	高[たか]まる			高まる 
\\	たかまる 
\\	高める	最近若者はコミュニケーション能力を高める必要がある 
\\	評価/人気/やる気/教育水準を高める 
\\	高[たか]める			高める:たかめる
\\	水準:すいじゅん
\\	強まる	台風の勢力はますます強まっている 
\\	雨/風/火/力/揺れが強まる 
\\	強[つよ]まる			強まる:つよまる
\\	勢力:せいりょく
\\	益々:ますます
\\	揺れ:ゆれ
\\	扌 
\\	䍃	
\\	強める	火/力/自信を強める 
\\	強[つよ]める			強める:つよめる
\\	弱まる	疲れていると抵抗力が弱まって、風邪にかかりやすくなる 
\\	雨/風/力/勢力/影響が弱まる 
\\	弱[よわ]まる			弱まる:よわまる
\\	抵抗:ていこう
\\	扌 
\\	氐 
\\	扌 
\\	亢 
\\	弱める	材料が柔らかくなったら、火を弱めてゆっくり煮込んでください 
\\	勢いを弱める 
\\	弱[よわ]める			弱める:よわめる
\\	勢い:いきおい
\\	柔らかい:やわらかい
\\	煮込み:にこみ
\\	広まる	その噂は1日で会社中に広まった	広[ひろ]まる	
\\	広まる:ひろまる
\\	噂:うわさ
\\	広める	明治時代に、ヨーロッパから帰国した留学生たちが、日本に西洋文化を広 めた	広[ひろ]める	
\\	広める:ひろめる
\\	深まる	子供ができた、二人の愛情はますます深った 
\\	秋が深まり、紅葉の美しい季節になった 
\\	深[ふか]まる			愛情:あいじょう
\\	季節:きせつ
\\	深める	交流を深め、相手の国のことを理解することが大切だ 
\\	オリンピックを通じ、世界各国との友好関係を深めたいと思います 
\\	深[ふか]める			深める:ふかめる
\\	交流:こうりゅう
\\	理解:りかい
\\	友好:ゆうこう
\\	家庭	田中課長は仕事では厳しいが、家庭では優しいお父さんだそうだ 
\\	早く結婚して、温かい家庭を持ちたい 
\\	家庭[かてい]			家庭:かてい
\\	協力	家族で協力して祖母の介護をした 
\\	アンケート調査にご協力ください 
\\	協力[きょうりょく]			協力:きょうりょく
\\	介護:かいご
\\	感謝	アドバイスしてくれた先生に、とても感謝している	感謝[かんしゃ]	
\\	感謝:かんしゃ
\\	アドバイス:
\\	咸 
\\	心 
\\	言 
\\	射 
\\	お礼	お世話になった方にお礼をした 
\\	入学祝のお礼にお菓子を送った 
\\	お 礼[れい]			礼:れい
\\	お辞儀	お客様には丁寧にお辞儀をしましょう	お 辞儀[じぎ]	
\\	お辞儀:おじぎ
\\	握手	コンサートで歌手と握手した	握手[あくしゅ]	
\\	握手:あくしゅ
\\	扌 
\\	屋 
\\	意地悪	好きな子にはつい意地悪してしまう	意地悪[いじわる]	
\\	意地悪:いじわる
\\	な, 
\\	悪戯	悪戯をして先生に怒られた	悪戯[いたずら]	
\\	悪戯:いたずら
\\	な
\\	節約	生活費の節約 
\\	電気や水を節約する 
\\	節約[せつやく]			節約:せつやく
\\	反省	1日の反省を日記に書く 
\\	あなたのせいで、みんな迷惑したんですよ、少しは反省しなさい 
\\	反省[はんせい]			反省:はんせい
\\	実行	この計画は実行が難しい 
\\	作戦を実行する 
\\	実行[じっこう]			作戦:さくせん
\\	実行:じっこう
\\	発達	体や心が発達する 
\\	文明が発達する 
\\	発達した台風が近づいている 
\\	発達[はったつ]			発達:はったつ
\\	出場	リンピックへの出場が決まった 
\\	全国大会に出場する 
\\	出場[しゅつじょう]			出場:しゅつじょう
\\	活躍	友人たちの活躍を聞くたびに、嬉しく思う 田中選手の活躍を期待する 
\\	活躍[かつやく]			活躍:かつやく
\\	嬉しい:うれしい
\\	競争	どちらがいい成績をとるか、友達と競争した 
\\	競争が厳しい 
\\	競争が激しい 
\\	競争[きょうそう]			競争:きょうそう
\\	拍手	ステージの歌手に拍手する 
\\	ゲストを拍手で迎える 
\\	拍手[はくしゅ]			拍手:はくしゅ
\\	テージ:
\\	舞台:ぶたい
\\	ゲスト:
\\	交換	名刺を交換する 
\\	意見を交換する 
\\	時計の電池が切れたので交換した 
\\	交換[こうかん]			交換:こうかん
\\	流行	新しい流行 
\\	若者の間でプログが流行している 
\\	インフルエンザは流行する 
\\	流行[りゅうこう]			流行:りゅうこう
\\	宣伝	バーゲンセールの宣伝が始まった 
\\	新商品をテレビで宣伝する 
\\	宣伝[せんでん]			宣伝:せんでん
\\	広告	新製品の広告がのっていた 
\\	広告が出る 
\\	広告が載せる 
\\	広告[こうこく]			広告:こうこく
\\	注目	有名歌手の結婚が注目を集めている 
\\	新しいファッションに注目する 
\\	注目を浴びる 
\\	注目[ちゅうもく]			注目:ちゅうもく
\\	浴びる:あびる
\\	ファッション:
\\	通訳	英語を日本語に通訳する 
\\	国際会議の通訳になる 
\\	首相の通訳を務める 
\\	通訳[つうやく]			通訳:つうやく
\\	務める:つとめる
\\	翻訳	日本語の小説を英語に翻訳する 
\\	翻訳家になる 
\\	翻訳[ほんやく]			翻訳:ほんやく
\\	番 
\\	羽 
\\	伝言	留守番電話に伝言を残す 
\\	欠席した人に伝言する 
\\	伝言[でんごん]			伝言:でんごん
\\	伝言を残す:
\\	報告	出張の報告 
\\	報告を纏める 
\\	学校に試合の結果を報告する 
\\	報告[ほうこく]						
\\	録画	テレビ番組をビデオに録画する	録画[ろくが]	
\\	混雑	デパートは、大勢の人で混雑している	混雑[こんざつ]	
\\	混雑:こんざつ
\\	デパート:
\\	氵 
\\	昆 
\\	隹 
\\	渋滞	道路が渋滞していて、会義に遅刻した	渋滞[じゅうたい]	
\\	渋滞:じゅうたい
\\	氵 
\\	氵 
\\	帯 
\\	衝突	電車の衝突の場面を見た 
\\	バスがトラックと衝突した 
\\	クラスで意見の衝突がある 
\\	部長と課長が衝突して、周囲が困っている 
\\	衝突[しょうとつ]			衝突:しょうとつ
\\	行 
\\	重 
\\	穴 
\\	大 
\\	停電	雷が落ちて停電した	停電[ていでん]	
\\	緊張	面接では緊張して、うまく答えられなかった 
\\	試合の前なので、みんな緊張している 
\\	緊張[きんちょう]			緊張:きんちょう
\\	自信	体力に自信がある 
\\	この成績なら大丈夫ですよ。もっと自信をもってください 
\\	自信[じしん]			自信:じしん
\\	自慢	自慢の料理を作る 
\\	母親は皆に有名大学に入った息子を自慢している 
\\	私は体が丈夫なのが自慢だ 
\\	自慢[じまん]			自慢:じまん
\\	感心	チンさんの進歩の速さに感心した 太郎君はよく親の手伝いをする、感心な子供だ 
\\	感心[かんしん]			感心:かんしん
\\	な, 
\\	感動	パラリンピックを見て、とても感動させられた 
\\	私はピカソの絵に感動し、自分も画家になりたいと思った 
\\	感動[かんどう]			感動:かんどう
\\	パラリンピック
\\	画家:がか
\\	興奮	試合を見ていた観客たちは、興奮して大声を出した	興奮[こうふん]	
\\	興奮:こうふん
\\	感想	ご感想はいかがですか 
\\	感想を述べる 
\\	感想[かんそう]			感想:かんそう
\\	述べる:のべる
\\	朮 
\\	予想	選挙の結果を予想する 
\\	予想が当たる 
\\	予想を裏切る: 
\\	予想[よそう]			予想:よそう
\\	選挙:せんきょ
\\	裏切る 裏切る:うらぎる
\\	訓: うら 音: リ 
\\	衣 
\\	原因	今、警察が事故の原因を調べている	原因[げんいん]	
\\	団体	博物館で団体のチケットを買う	団体[だんたい]	
\\	博物館:はくぶつかん
\\	団体:だんたい
\\	税金	年度末に税金を払う 
\\	税金を納める 
\\	税金[ぜいきん]			税金:ぜいきん
\\	納める:おさめる
\\	責任	この失敗はあなたの責任ではない 
\\	社長は倒産の責任をとって辞めた 
\\	事故の責任を追及する 
\\	責任[せきにん]			責任:せきにん
\\	追及する:ついきゅう
\\	題名	コンクールに出す作品に題名をつける	題名[だいめい]	
\\	題名:だいめい
\\	コンクール:
\\	条件	運転免許を持っていることが採用の条件だ 
\\	このアルバイトは条件がいい 
\\	条件[じょうけん]			条件:じょうけん
\\	採用:さいよう
\\	期間	申し込み期間は12月1日から3日まです	期間[きかん]	
\\	期間:きかん
\\	倍	値段が倍になる 
\\	薬の量を倍にする 
\\	太陽の大きさは地球の約100倍だ 
\\	倍[ばい]						
\\	近道	駅への近道を通る 
\\	遅刻しそうなので、近道する 
\\	近道[ちかみち]			
\\	遠回り 
\\	回り	池の周りを歩く: 
\\	駅の周りにはビルが多い 
\\	周りの人と仲良くする 
\\	回[まわ]り			回り:まわり
\\	穴	地面に穴を掘る 
\\	壁に穴をあける 
\\	穴の開いた靴下を捨てる 
\\	穴[あな]						
\\	列	入口の前には長い列ができていた 
\\	タクシーの列 
\\	1列に並ぶ 
\\	列[れつ]			列:れつ
\\	幅	道の幅が広い 
\\	値上がりの幅が大きい 
\\	交際の幅が広い 
\\	幅[はば]			幅:はば
\\	範囲:はんい
\\	範囲	試験の範囲 
\\	彼女は趣味の範囲が広い 
\\	わかる範囲で答えてください 
\\	範囲[はんい]			範囲:はんい
\\	中身	箱を開けて中身を見る 
\\	話の中身が理解できない 
\\	中身[なかみ]			中身:なかみ
\\	特徴	商品の特徴を確かめる 
\\	彼女は特徴のある顔をしている 
\\	特徴[とくちょう]			特徴:とくちょう
\\	普通	パスポートの申請には1週間ぐらいかかるのが普通た 
\\	私は普通、朝食にはパンを食べる 
\\	田中さんって、ちょっと変じゃない。そう、普通の人だと思うけど 
\\	普通[ふつう]			申請:しんせい
\\	当たり前	そんな酷いことをされたら、怒るのが当たり前だ	当たり前[あたりまえ]	
\\	当たり前:あたりまえ
\\	偽	偽の銀行員に騙されてお金を取られた	偽[にせ]	
\\	偽:にせ
\\	の, 
\\	騙す:だます
\\	亻 
\\	為 
\\	別	恋愛と結婚は別だ 
\\	別の意見を述べる 
\\	欠席者には別に連絡する 
\\	男性と女性を別に調査する 
\\	別[べつ]						
\\	国籍	私はずっとアメリカで暮らしているが、国籍は日本だ	国籍[こくせき]	
\\	国籍:こくせき
\\	東洋	東洋の文化と西洋の文化を比べる	東洋[とうよう]	
\\	西洋	西洋音楽がすきです	西洋[せいよう]	
\\	国際	日本企業の海外支社でのトラブルが国際問題に発展した	国際[こくさい]	
\\	企業:きぎょう
\\	支社:ししゃ
\\	トラブル
\\	発展:はってん
\\	自然	山や海へ行って、自然の中で過ごす 
\\	好きな人と一緒にいたいと思うのは、自然なことだ 
\\	意地が悪い人は、自然に友達が少なくなる 
\\	自然[しぜん]						
\\	宗教	世界にはいろいろな宗教がある	宗教[しゅうきょう]	
\\	宗教:しゅうきょう
\\	愛	私は家族を愛している 
\\	神への愛 
\\	愛[あい]			神:かみ
\\	届く	父から手紙が届く 
\\	本棚の上のほうに手が届かない 
\\	届[とど]く			届く:とどく
\\	届ける	隣の家に旅行のお土産を届けた 
\\	住所が変わった場合は、すぐに学校に届けてください 
\\	届[とど]ける			届ける:とどける
\\	かく	頭をかく: 
\\	痒いところをかいたら血が出てきた 
\\	汗をかく 
\\	恥をかく 
\\	かく			汗をかく:あせをかく
\\	痒い:かゆい
\\	血:ち
\\	恥をかく:はじをかく
\\	掴む	電車で突然腕を掴まれてびっくりした 
\\	コンテストで優勝して、歌手になるチャンスを掴んだ 
\\	宝くじが当たって、大金を掴んだ 
\\	この文章は難しくて、意味が掴みにくい 
\\	掴[つか]む				突然:とつぜん
\\	掴む:つかむ
\\	宝くじに当たって大金を掴んだ。 
\\	あの人の話が長くて要点を掴むのに苦労する。 
\\	腕:うで
\\	握る	子供は怖がって母親の手を強く握った 
\\	母はカラオケが大好きで、マイクを握ると離さない 
\\	赤ちゃんがベットの上で、手を握ったり開いたりしている 
\\	握[にぎ]る			握る:にぎる
\\	押さえる	あの人はおなかを押さえて座っている。腹痛だろうか 
\\	後ろの人のためにドアを押さえて待った 
\\	押[お]さえる			押さえる:おさえる
\\	近づける	車を道の 脇に近づける 
\\	絵に顔を近づけてよく見る 
\\	近[ちか]づける			近づける:ちかづける
\\	脇:わき
\\	合わせる	みんなで力を合わせて頑張りましょう 
\\	手を合わせて祈る 
\\	中年の私には、若い学生たちと話を合わせるのは難しい 
\\	新しいスーツに合わせて靴とバッグも買った 
\\	テレビを見て時計の時間を合わせた 
\\	合[あ]わせる			合わせる:あわせる
\\	雨期	ベトナムの雨期は5月から11月にかけてです	雨期			雨期:うき
\\	乾期	乾期になると森林が燃えやすい	乾期	
\\	乾期:かんき
\\	の
\\	身につける	実際な知識が身につくようにたくさんボランティア活動に参加しています。 日本語を身につけるために、積極的に日本人の友達を作っている	身[み]につける:			身につける:みにつける
\\	身につく:みにつく
\\	当たる	ボールが当たって窓ガラスが割れた 
\\	天気予報が当たる 
\\	うちの南側に大きなビルが建ったので、日が当たらなくなった 
\\	1回戦で去年の優勝チームと当たることになった 
\\	当[あ]たる						
\\	当てる	相手選手のボールが速いのでラケットに当てるのも大変だ 
\\	クイズの答えを当てる 
\\	洗濯物を日に当てて乾かす 
\\	額に手を当てて、熱がないかどうか確かめる 
\\	当[あ]てる			当てる:あてる
\\	ラケット:
\\	クイズ:
\\	日に当てる:ひにあてる
\\	乾かす:かわかす
\\	額:ひたい
\\	比べる	東京と大阪の面積を比べる 
\\	去年に比べて10センチも背が伸びた 
\\	比[くら]べる			比べる:くらべる
\\	似合う	彼女は着物がよく似あう 
\\	彼に似合いそうなネクタイを探した 
\\	似合[にあ]う			似合う:にあう
\\	似る	彼女は母親によく似ている 
\\	英語と中国語の文法は少し似ている 
\\	親子は声が似ることが多い 
\\	似[に]る			似る:にる
\\	親子:おやこ
\\	の
\\	似せる	アイドルに髪型を似せる 
\\	歌手に声を似せて歌う 
\\	似[に]せる			似せる:にせる
\\	アイドル:
\\	髪型:かみがた
\\	分かれる	トイレは、男性用と女性用に分かれている 
\\	クラスで二つのチームに分かれてサッカーをした 
\\	喫煙については、人々の意見が分かれている 
\\	分[わ]かれる			分かれる:わかれる
\\	喫煙:きつえん
\\	分ける	財産を3人の子供に分ける 
\\	クラスでチームを二つに分けてサッカーをした 
\\	分[わ]ける			分ける:わける
\\	足す	味が薄かったので塩を足した 
\\	風呂の湯が少なくなったので、足しておいた 
\\	4に6をたすと10になる 
\\	4足す6は10だ 
\\	足[た]す			足す:たす
\\	薄い:うすい
\\	引く	このドアは押すのではなく、引いて開けるんです 
\\	カーテンを引いて寝る 
\\	レストランでウェイターが椅子を引いてくれた 
\\	わからない言葉に線を引く 
\\	10から8を引きと2になる 
\\	10引く2は8だ 
\\	風邪をひく 
\\	辞書 を引く
\\	引[ひ]く		
\\	引く:ひく
\\	増やす	体重を増やす 
\\	貯金を増やす 
\\	増[ふ]やす			増やす:ふやす
\\	減らす	体重を減らす	減[へ]らす			減らす:へらす
\\	変わる	季節が変わる 
\\	信号が青から赤に変わった 
\\	変[か]わる			変わる:かわる
\\	変える	髪型を変える 
\\	今の人生を変えたい 
\\	彼女は最近髪の色を茶色から金色に変えた 
\\	変[か]える			変える:かえる
\\	茶色:ちゃいろ
\\	代わる、替わる、換わる		代[か]わる、 替[か]わる、 換[か]わる	4月に店長はかわった 
\\	ちょっと出かけてくるので、しばらく受付の仕事をかわってください 
\\	替わる:かわる
\\	代わる:かわる
\\	換わる:かわる
\\	代える、替える、換える	千円札を百円だまにかえてもらった 
\\	古くなった電球を新しいのとかえた 
\\	代[か]える、 替[か]える、 換[か]える						
\\	返る	友達に貸したお金が返ってきた 
\\	返[かえ]る			
\\	返す	この本は2週間以内に図書館に返さなければならない 
\\	使い終わったものは元の場所に返してください 
\\	返[かえ]す			返す:かえす
\\	譲る	電車の中で、お年寄りに席を譲った 
\\	帰国するので、家具を友達に安く譲った 
\\	財産を子供に譲る 
\\	彼は社長の地位を息子に譲って、引退した 
\\	譲[ゆず]る			譲る:ゆずる
\\	権利・財産等 
\\	を 譲 る 
\\	助かる	飛行機が落ちたが、3人が助かった 
\\	この辺は物価が安くてとても助かる 
\\	この書類、コビーしましょうか。ありがとう、助かります 
\\	助[たす]かる			助かる:たすかる
\\	山人:さんじん
\\	助ける	川に落ちた子供を助けた 
\\	誰か助けて 
\\	困っているとき、友達が助けてくれた 
\\	お年寄りがバスを降りるのを助けてあげた 
\\	助[たす]ける			助ける:たすける
\\	いじめる	猫をいじめる 
\\	子供をいじめてはいけません 
\\	いじめる			苛める:いじめる
\\	騙す	彼は独身と騙して5人の女性と付き合った 
\\	ブランド品だと思ったら偽物だった。騙された 
\\	騙[だま]す			騙す:だます
\\	ブランド:
\\	偽物:にせもの
\\	の
\\	盗む	留守中に泥棒に入られて、お金を盗まれてしまった 
\\	盗[ぬす]む			盗む:ぬすむ
\\	泥棒:どろぼう
\\	盗み:ぬすみ
\\	刺さる	指に刺が刺さって痛い 
\\	死体にナイフが刺さっていた 
\\	刺[さ]さる			刺さる:ささる
\\	刺:とげ
\\	死体:したい
\\	の 
\\	刺す	指に刺を刺してしまった 
\\	歩いていたら、いきなり男知らないに刺された 
\\	刺[さ]す			刺す:さす
\\	行き成り:いきなり
\\	刺:とげ
\\	殺す	ゴキブリをスリッパで叩いて殺した 
\\	戦争で、多くの人が殺された 
\\	殺[ころ]す			殺す:ころす
\\	ゴキブリ:
\\	叩く:たたく
\\	隠れる	逃げた犯人は空き家に隠れていた 
\\	月が雲に隠れて見えなくなった 
\\	隠[かく]れる			隠れる:かくれる
\\	隠 
\\	訓: かく.す かく.し かく.れる かか.す よ.る 音: イン オン 
\\	[隱].	
\\	隠す	お金を引き出しの奥に隠した 
\\	子供は恥ずかしがって、帽子で顔を隠してしまった 
\\	隠[かく]す			隠す:かくす
\\	恥ずかしい:はずかしい
\\	埋まる	山が崩れて家が埋まってしまった 
\\	埋[う]まる			埋まる:うずまる
\\	崩れる:くずれる
\\	埋める	穴を掘ってごみを埋めた 
\\	土を入れて穴を埋めた 
\\	解答欄はいちおう全部埋めたが、自信がない 
\\	埋[う]める			埋める:うめる
\\	掘る:ほる
\\	解答欄:かいとうらん
\\	囲む	テーブルを囲んで座る 
\\	ただし答えを〇で囲みなさい 
\\	日本は周りを海に囲まれている 
\\	囲[かこ]む			囲む:かこむ
\\	詰まる	かばんにたくさん荷物が詰まっていて重い 
\\	トイレが詰まった 
\\	前の選手と後ろの選手の差が詰まってきた 
\\	風邪をひいて鼻が詰まっている 
\\	詰[つ]まる			詰まる:つまる
\\	詰める	かばんに荷物をつめる 
\\	弁当箱にご飯を詰める 
\\	前の車との距離を詰める 
\\	詰[つ]める			詰める:つめる
\\	距離:きょり
\\	開く	ドアが開く 
\\	1位と2位の差が、どんどん開いていった 
\\	父は駅前にクリーニング店を開いた 
\\	開[ひら]く			開く:ひらく
\\	閉じる	エレベーターのドアが閉じる 
\\	寒いのでドアを閉じた 
\\	公園の門は夜になると閉じられる 
\\	売り上げが減ったので、店を閉じることにした 
\\	閉[と]じる			閉じる:とじる
\\	飛ぶ	鳥が空を飛んでいる 
\\	みかんの汁が飛ぶ 
\\	ボールが飛んできた 
\\	子供が事故にあったと聞いて、病院へ飛んで行った 
\\	資料のページが飛んでいないかどうか、確認してください 
\\	パソコンのデータが飛んでしまった 
\\	飛[と]ぶ			飛ぶ:とぶ
\\	みかん汁:しる
\\	飛ばす	紙飛行機を飛ばす 
\\	風船を飛ばす 
\\	唾を飛ばして話す 
\\	文章を1行飛ばして読んでしまった 
\\	飛[と]ばす			飛ばす:とばす
\\	風船:ふうせん
\\	唾:つば
\\	振る	手を振る 
\\	肉に塩をふる 
\\	片思いの彼女に告白したが、振られてしまった 
\\	振[ふ]る			振る:ふる
\\	片思い:かたおもい
\\	告白:こくはく
\\	捲る	カードを捲る 
\\	カレンダーを捲る 
\\	捲[めく]る			捲る:めくる
\\	捲 
\\	訓: ま.く ま.くる まく.る めく.る まく.れる 音: ケン 
\\	扌 
\\	巻 
\\	見かける	町で偶然、昔の友達を見かけた 
\\	最近、公衆電話を見かけなくなった 
\\	このりんごは見かけは甘そうだったが、食べてみると酸っぱいかった 
\\	見[み]かける			見かける:みかける
\\	公衆電話:こうしゅうでんわ
\\	確かめる	ファックスが届いたかどうか、相手に電話して確かめた 
\\	飲み会の参加人数を確かめた 
\\	確[たし]かめる			確かめる:たしかめる
\\	ファックス:
\\	確認する 
\\	試す	洋服が似合うかどうか、着て試してみた 
\\	自分の実力を試すために、テストを受けた 
\\	いままで 食べたことのない新しい食べ物を試す 
\\	試[ため]す			試す:ためす
\\	繰り返す	同じ失敗を繰り返してはいけません 
\\	この本は大好きなので、繰り返し読みました 
\\	繰り返[くりかえ]す			繰り返す:くりかえす
\\	訳す	英語を日本語に訳す 
\\	訳[やく]す			訳す:やくす
\\	翻訳する 
\\	通訳する 
\\	許す	酷いことを言われてけんかになったが、相手が誤ったので許してあげた 
\\	罪を許される 
\\	子供には、1日1時間だけゲームをすることを許している 
\\	許[ゆる]す			誤る:あやまる
\\	罪:つみ
\\	な, 
\\	許しを与える 
\\	許しをもらう: 
\\	許可する 
\\	慣れる	日本へ来て半年たって、生活にも慣れた 
\\	生の肉を食べるには慣れていない 
\\	サーカスの動物は、よく人に慣れている 
\\	慣[な]れる			慣れる:なれる
\\	サーカス:
\\	慣らす	準備運転をして体を慣らしてからプールに入ったほうがいい 
\\	像を慣らして芸をさせる 
\\	慣[な]らす			慣らす:ならす
\\	芸:げい
\\	育つ	雨が多い年は、コメがよく育つ 
\\	親は子供が元気に育つのを望んでいる 
\\	あの大学では優秀な研究者がたくさん育っている 
\\	育[そだ]つ			育つ:そだつ
\\	成長する 
\\	育てる	母は5人の子供を育てた 
\\	朝晩水をやって草花を育てている 
\\	あの会社は人材を育ってるのが上手だ 
\\	育[そだ]てる			育てる:そだてる
\\	草花:くさばな
\\	生える	髪が生える 
\\	葉が生える 
\\	生[は]える						
\\	生やす	髭をもっと長く生やすぞ。 
\\	顎鬚を 生やす 
\\	生[は]やす			生やす:はやす
\\	顎髭:あごひげ
\\	汚れる	空気の汚れたところには住みたくない 
\\	汚れがつく 
\\	汚れを落とす 
\\	汚[よご]れる		
\\	汚れる:けがれる
\\	汚す	泥遊びをして服を汚した 
\\	川に水を汚さないようにしよう 
\\	汚[けが]す			汚す:けがす
\\	壊れる	家が壊れる 
\\	壊れた時計を直してもらった 
\\	壊[こわ]れる			壊れる:こわれる
\\	壊す	電気製品を壊す 
\\	息子は新しいおもちゃをすぐに壊してしまう 
\\	飲み過ぎて体を壊してしまった 
\\	壊[こわ]す			壊す:こわす 
\\	電気製品:でんきせいひん
\\	割れる	ガラスは割れる 
\\	羊の蹄は先が二つに割れている 
\\	みんなの意見が割れて、なかなか決まらなかった 
\\	割[わ]れる			割れる:われる
\\	羊:ひつじ
\\	蹄:ひづめ
\\	割る	コップを割る 
\\	卵をわってボウルに入れる 
\\	りんごを二つに割って二人で食べた 
\\	10を5で割ると2になる 
\\	10わる5は2だ 
\\	割[わ]る			割る:わる
\\	折れる	強い風で枝が折れた 
\\	折[お]れる			折れる:おれる
\\	折る	スキーをして、足の骨を折った 
\\	便箋を三つにおって封筒に入れた 
\\	折[お]る			折る:おる
\\	骨:ほね
\\	便箋:びんせん
\\	破れる	本が破れる 
\\	服が破れる 
\\	破[わ]れる			破れる:やぶれる
\\	破る	紙を破る 
\\	約束を破る 
\\	水泳の世界記録を破られた 
\\	破[やぶ]る			破る:やぶる
\\	守る 
\\	曲がる	曲がった道 
\\	この角を右に曲がって50メトルほど行くと、銀行があります 
\\	ネクタイが曲がっています 
\\	腰の曲がったおばあさん 
\\	曲[ま]がる			曲がる:まがる
\\	曲げる	針金を曲げる 
\\	腰を曲げる 
\\	曲[ま]げる			曲げる:まげる
\\	針金:はりがね
\\	外れる	ボタンが外れる 
\\	天気予報が外れる 
\\	矢が窓を外れる 
\\	けがをした選手は、チームのメンバーから外れた 
\\	外[はず]れる			外れる:はずれる
\\	外す	眼鏡を外す 
\\	予想を外す 
\\	監督は、けがをした選手をチームのメンバーから外した 
\\	外[はず]す			外す:はずす
\\	監督:かんとく
\\	の 
\\	監 
\\	音: カン 
\\	皿 
\\	督 
\\	音: トク 
\\	叔 
\\	目 
\\	揺れる	風で木の葉が揺れている 
\\	地震で、地面が大きく揺れた 
\\	かれと結婚するかしないか、気持ちが揺れている 
\\	揺[ゆ]れる			揺れる:ゆれる
\\	地面:じめん
\\	揺らす	子供の乗ったブランコを揺らして遊ばせた 
\\	揺[ゆ]らす			揺らす:ゆらす
\\	流れる	町の中心を大きな川が流れている 
\\	汗が流れる 
\\	川にたくさんのごみが流れている 
\\	彼の部屋にはいつも音楽が流れている 
\\	うわさが流れる 
\\	流[なが]れる			流れる:ながれる
\\	流す	汚れた水を川に流してはいけない 
\\	涙を流す 
\\	洪水で家が流された 
\\	この喫茶店はいつもクラジック音楽を流している 
\\	情報を流す 
\\	と
\\	の間に電気を流す 
\\	流[なが]す			流す:ながす
\\	濡らす	水を零して服を濡らしてしまった 
\\	濡[ぬ]らす			濡らす:ぬらす
\\	迷う	道に迷う 
\\	冬山で迷ってしまい、もう少しで死ぬところだった 
\\	日本で就職するか、国に帰るか、迷ている 
\\	何にしようと迷っている 
\\	迷[まよ]う			迷う:まよう
\\	悩む	就職するか、進学するか、悩んでいる 
\\	彼女が苦しい恋に悩んでいる 
\\	父は腰痛で悩んでいる 
\\	悩[なや]む			悩む:なやむ
\\	腰痛:ようつう
\\	悩みがある 
\\	悩みを打ち上げる 
\\	慌てる	学校に遅れそうになって、慌てて家を出た 
\\	店で財布が見つからなくて慌てた 
\\	慌[あわ]てる			慌てる:あわてる
\\	覚ます	目を覚ますと、もう10時だった 
\\	少し酔いを覚ましてから帰ろう 
\\	覚[さ]ます			覚ます:さます
\\	酔い:よい
\\	グループ	3人のグループで旅行をする 
\\	形容詞は、イ形容詞とナ形容詞の二つのグループに分かれる 
\\	グループ			
\\	プロフェッショナル	プロのサッカー選手になりたい 
\\	彼女の料理はプロ並みだ 
\\	プロ並み:なみ
\\	アマチュア	この絵はアマチュアの作品とは思えないほどすばらしい 
\\	アマチュア			
\\	トレーニング	勝つためには毎日のトレーニングは必要だ 
\\	トレーニング			
\\	マッサージ	頭をマッサージしてもらうと気持ちがいい 
\\	マッサージ			
\\	アドバイス	先輩に仕事のやり方についてアドバイスをする 
\\	何かアドバイスをいただけませんか 
\\	アドバイス			
\\	アイデア	いいアイデアが浮かんだ 
\\	アイデア			浮かぶ:うかぶ
\\	トップ	彼はクラスでトップの成績だ 
\\	日本の科学技術は、世界のトップベルにある 
\\	高橋選手はトップでゴールした 
\\	これからスピーチコンテストを始めます、トップは中国のチンさんです 
\\	新聞のトップを見る 
\\	トップ			
\\	スピード	運転手さん、もう少しスピードを上げてください 
\\	先生の話すスピードが速すぎて理解できない 
\\	スピード			
\\	ラッシュ	年末年始は、帰省ラッシュでチケットがとりにくい 
\\	ラッシュの時間帯に電車に乗るのは大変だ 
\\	ラッシュ			
\\	年末年始:ねんまつねんし
\\	時間帯:じかんたい
\\	ヘルメット	バイクに乗るときは、ヘルメットをかぶる 
\\	ヘルメット			
\\	コンタクトレンズ	コンタクトレンズを使ってるの 
\\	コンタクト			
\\	プラスチック	プラスチック製品は生ごみの中に入れないでください 
\\	プラスチック			
\\	ベランダ	ベランダに洗濯物を干す 
\\	ベランダ			
\\	ベンチ	公園のベンチで休む 
\\	駅のホームのベンチにすわる 
\\	ベンチ			
\\	デザイン	あのドレスは色もデザインもいい 
\\	私たちの学校制服は、有名なデザイナーがデザインしたものだ 
\\	デザイン			
\\	ドレス:
\\	デザイナー: 
\\	バーゲン	デパートで今バーゲンセールをしている 
\\	このバッグ、バーゲンで買ったの 
\\	レジ	スーパーでレジのアルバイトをしている 
\\	レジに並ぶ 
\\	レジ			
\\	レシート	レジでお金を払って、レシートを受け取る 
\\	レシート			
\\	領収書:りょうしゅうしょ
\\	インスタント	インスタント食品が便利だ 
\\	インスタントラーメン 
\\	インスタントコーヒー 
\\	インスタント			
\\	ファストフード	そばは江戸時代のファストフードだった 
\\	ファストフード			
\\	フルーツ	暖かいところには、いろいろなフルーツがある 
\\	フルーツ			
\\	デザート	デザートにアイスクリームを食べる 
\\	デザート			
\\	インターネット	インターネットで世界の環境問題について調べた 
\\	きのうは1日中インターネットをしていた 
\\	インターネット						
\\	チャイム	授業が始まるとき、チャイムが鳴る 
\\	チャイム			
\\	アナウンス	電車の中でアナウンスを聞く 
\\	緊急ニュースを校内でアナウンスする 
\\	アナウンス			
\\	緊急:きんきゅう
\\	な, 
\\	の 
\\	メッセージ	留守番電話にメッセージを残す 
\\	大統領は国民にメッセージを発表した 
\\	この曲には、平和へのメッセージがこ込められている 
\\	メッセージ			
\\	パンフレット	旅行会社でパンフレットをたくさんもらってきた 
\\	パンフレット			
\\	カード	テレホンカードで電話をかける 
\\	ママがカードを1枚とるから、ちゃんと広げてよ 
\\	カード						
\\	インタビュー	勝ったチームの選手にインタビューする 
\\	記者のインタビューに答える 
\\	インタビュー						
\\	データ	研究のために、データを集めている 
\\	データ						
\\	張る	肩が張っているときはこの薬が効く。 
\\	ビール を 飲 んだら 腹 が 張 った。 
\\	もっと ロープ をぴんと 張 ってください。 
\\	張[は]る			張る:はる
\\	濃い	色が濃い 
\\	味が濃い 
\\	二人は濃い仲だ 
\\	髪 の 毛 が 濃 い 
\\	濃[こ]い			濃い:こい
\\	訓: こ.い 音: ノウ 
\\	氵 
\\	農 
\\	毛 
\\	薄い	薄い本 
\\	化粧が薄い 
\\	薄[うす]い			薄い:うすい
\\	訓: うす.い うす- -うす うす.める うす.まる うす.らぐ うす.ら- うす.れる すすき 音: ハク 
\\	艹 
\\	溥 
\\	酸っぱい	レモンは酸っぱい 
\\	古い牛乳のバックをあけたら、酸っぱいにおいがした 
\\	酸[す]っぱい			酸っぱい:すっぱい
\\	レモン:
\\	訓: す.い 音: サン 
\\	酉 
\\	夋	
\\	臭い	納豆は臭いから嫌いだという日本人も多い 
\\	魚を焼いたので台所が臭くなった 
\\	臭[くさ]い			臭い:くさい
\\	臭 
\\	訓: くさ.い -くさ.い にお.う にお.い 音: シュウ 
\\	自 
\\	大 
\\	おかしい	山本君は、授業にいつもおかしいことを言ってみんなを笑わせる 
\\	パソコンの調子がおかしい 
\\	彼女が朝からずっと様子がおかしい 
\\	うちの会社、子供ができたらやめなきゃいけないのよ、そんなおかしいよ 
\\	おかしい			可笑しい:おかしい
\\	格好いい	あの先輩は、かっこいいので人気がある 
\\	年を取っても自分の夢を追いかけるのは、かっこいい生き方だと思う 
\\	格好[かっこう]いい			格好いい:かっこういい
\\	生き方:いきかた
\\	うまい	母は料理がうまい 
\\	この前のスピーチコンテストでは、とてもうまく話せた 
\\	仕事のあとのビールはうまい 
\\	彼は恋人とあまりうまくいっていないようだ 
\\	うまい			美味い:うまい
\\	上手い:うまい
\\	親しい	私は田中さんと親しい 
\\	彼とは留学した大学が同じで親しくなった 
\\	母親同士が親友なので、家族同士も親しく付き合っている 
\\	親[した]しい			親しい:したしい
\\	人と親しい:
\\	同士:どうし
\\	母親同士 親友:しんゆう
\\	詳しい	この地図はとても詳しい 
\\	先生、もう少し詳しく説明していただけませんか 
\\	姉は日本文化にとても詳しい 
\\	詳[くわ]しい			詳しい:くわしい
\\	細かい	新聞の字は細かくて、お年寄りには読みづらい 
\\	この書類を書くときには、細かい注意が必要だ 
\\	細かいお金がないので1万円でお釣りをください 
\\	細[こま]かい			細かい:こまかい
\\	小さい 
\\	こぜに 
\\	浅い	この川は浅いので、子供が泳いでも危なくない 
\\	地震は地下の浅いところで起こった 
\\	経験が浅い 
\\	浅[あさ]い			浅い;あさい
\\	浅 
\\	訓: あさ.い 音: セン 
\\	氵 
\\	[淺].	
\\	固い, 硬い	このパンはとても固い 
\\	私は頭が固いから、いいアイデアがなかなか浮かばない 
\\	瓶の蓋が固くてなかなか開かない 
\\	文章が硬い 
\\	ダイヤモンドは非常に硬いので、工業用に使われている	固[かた]い, 硬[かた]い			硬い:かたい
\\	蓋:ふた
\\	浮く:うく
\\	硬 
\\	訓: かた.い 音: コウ 
\\	石 
\\	更 
\\	浮 
\\	訓: う.く う.かれる う.かぶ む う.かべる 音: フ 
\\	氵 
\\	孚 
\\	温い	温いコーヒーはおいしくない 
\\	冷やしたビールを冷蔵庫から出したままにして置いたら、温くなってしまった 
\\	温[ぬる]い			温い:ぬるい
\\	眩しい	カーテンを開けたら、太陽が眩しかった 
\\	彼女は、最近眩しいほど美しくなった 
\\	眩[まぶ]しい			眩しい:まぶしい
\\	眩 
\\	訓: げん.す くるめ.く まぶ.しい くら.む まど.う めま.い まばゆ.い くれ.る ま.う 音: ゲン カン 
\\	目 
\\	玄 
\\	蒸し暑い	日本の夏は特に湿度が高いので、大変蒸し暑い 
\\	蒸し暑[むしあつ]い			蒸し暑い:むしあつい
\\	訓: む.す む.れる む.らす 音: ジョウ セイ 
\\	艹 
\\	烝 
\\	訓: あつ.い 音: ショ 
\\	日 
\\	者 
\\	夏 
\\	訓: なつ 音: カ ガ ゲ 
\\	夂 
\\	清潔	清潔な服を身につける 
\\	トイレはいつも清潔にしておきたい 
\\	清潔[せいけつ]			清潔:せいけつ
\\	な, 
\\	清 
\\	訓: きよ.い きよ.まる きよ.める 音: セイ ショウ シン 
\\	氵 
\\	青 
\\	潔 
\\	訓: いさぎよ.い 音: ケツ 
\\	氵 
\\	絜 
\\	新鮮	新鮮な魚はおいしい 
\\	山で新鮮な空気を胸いっぱいに吸い込んだ 
\\	新鮮な気持ちで新学期を迎えた 
\\	新鮮[しんせん]			吸い込む:すいこむ
\\	新学期:しんがっき
\\	胸いっぱい:むねいっぱい
\\	胸 
\\	訓: むね むな- 音: キョウ 
\\	月 
\\	匈 
\\	豊か	豊かな自然 
\\	この辺りは、国でもっとも豊かな地方だ 
\\	彼女は想像力が豊かだ 
\\	豊[ゆた]か			豊か:ゆたか
\\	な 
\\	想像力:そうぞうりょく
\\	豊 
\\	訓: ゆた.か とよ 音: ホウ ブ 
\\	曲 
\\	豆 
\\	立派	立派な仕事 
\\	あなたも立派な大人なのだから、自分のことは自分で決めなさい 
\\	いじめは、立派な犯罪だ 
\\	立派[りっぱ]			立派:りっぱ
\\	な 
\\	立派な情報 
\\	犯罪:はんざい
\\	正確	正確な数はわからないが、この観客は5000人ぐらいだろう 
\\	計算は答えを正確に出さなければならない 
\\	山本さんは時間に正確で、待ち合わせに絶対遅れない 
\\	正確[せいかく]				正確:せいかく
\\	な, 
\\	確か	新聞は確かな事実だけを伝えなければならない 
\\	確かにかばんにさいふを入れたはずなのに、いくら探しても見つからない 
\\	あの会社、倒産するらしいですよ、それは確かですか 
\\	確[たし]か			確か:たしか
\\	な, 
\\	確実:かくじつ
\\	な, 
\\	確 
\\	訓: たし.か たし.かめる 音: カク コウ 
\\	石 
\\	寉 
\\	重
\\	今日は午後から重要な会議がある 
\\	情報技術は、将来ますます重要になるだろう 
\\	食料問題は国にとってとても重要だ 
\\	重要[じゅうよう]			重要:じゅうよう
\\	な, 
\\	大事な 
\\	大切な 
\\	必要	子供に必要なのは親の暖かい愛情だ 
\\	海外の仕事でパスポートが必要になった 
\\	必要があれば、このコビー機を使ってください 
\\	必要[ひつよう]			必要:ひつよう
\\	な 
\\	愛情:あいじょう
\\	の 
\\	不必要な 
\\	不要な 
\\	勿体無い	流行遅れでもまだ着られる服を捨てるのは勿体無い 
\\	折角留学したのだから、一所懸命勉強しないともったいない 
\\	勿体[もったい] 無[な]い			勿体無い:もったいない
\\	後れ:おくれ
\\	折角:せっかく
\\	勿 
\\	訓: なか.れ なし 音: モチ ブツ ボツ 
\\	勹 
\\	丿 
\\	体 
\\	訓: からだ かたち 音: タイ テイ 
\\	亻 
\\	本 
\\	無 
\\	訓: な.い 音: ム ブ 
\\	灬 
\\	[無明]	
\\	凄い	昨日の台風はすごかった 
\\	味、どう。うん、すごくおいしいよ 
\\	コンテストで優勝したんだ、それはすごいね 
\\	凄[すご]い			凄い:すごい
\\	凄 
\\	訓: さむ.い すご.い すさ.まじい 音: セイ サイ 
\\	冫 
\\	妻 
\\	酷い	一人を大勢でいじめるとは酷い 
\\	二人の女性と同時に付き合っていたなんて、酷い男だ 
\\	今学期の成績は酷かった 
\\	隠れてたばこを吸って、先生に酷く怒られた 
\\	酷[ひど]い			酷い:ひどい
\\	隠れ:かくれ
\\	激しい	雨が激しい 
\\	入試に失敗して、今まで勉強しなかったことを激しく後悔した 
\\	変化が激しい 
\\	激[はげ]しい			激しい:はげしい
\\	激 
\\	訓: はげ.しい 音: ゲキ 
\\	氵 
\\	敫	
\\	そっくり	兄は父に顔も声もそっくりだ 
\\	有名な画家の絵を真似してそっくりに描いた 
\\	友達だと思って声をかけたら、そっくりな別人だった 
\\	そっくり			そっくり:
\\	な 
\\	画家:がか
\\	の 
\\	真似し:まねし
\\	急	急に歯が痛み出した 
\\	急な用事ができて、国へ帰ることになった 
\\	この川は流れが急だ 
\\	急[きゅう]						
\\	適当	家庭教師を探しているが、適当な人がなかなかいない 
\\	肉と野菜を適当な大きさに切ってカレーを作る 
\\	親が結婚しろと煩くて、そこたびに適当に返事をしている 
\\	適当[てきとう]			適当:てきとう
\\	な, 
\\	家庭教師:かていきょうし
\\	煩い:うるさい
\\	いいかげんな 
\\	適 
\\	訓: かな.う 音: テキ 
\\	特別	彼女には音楽家としての特別な才能がある 
\\	合格祝いに母が特別にごちそうを作ってくれた 
\\	今年の冬は寒いが、今日は特別寒い 
\\	特別[とくべつ]						
\\	完全	土の中から古代の器が完全な形で出てきた 
\\	準備は完全だったのに、結果はよくなかった 
\\	試合は私たちの完全な勝利だった 
\\	実験完全に失敗してしまった 
\\	完全[かんぜん]			完全:かんぜん
\\	な, 
\\	古代:こだい
\\	な, 
\\	器:うつわ
\\	勝利:しょうり
\\	盛ん	私のふるさとは農業が盛んだ 
\\	日本で海外旅行が盛んになったのは80年代だ 最近、大学では公開講座が盛んに開かれるようになった 
\\	盛[さか]ん			盛ん:さかん
\\	な 
\\	公開:こうかい
\\	講座:こうざ
\\	盛 
\\	訓: も.る さか.る さか.ん 音: セイ ジョウ 
\\	成 
\\	皿 
\\	様々	インターネットから様々な情報は得られる 
\\	様々に色が変わる 
\\	何を幸せと思うかは、人によってさまざまだ 
\\	様々[さまざま]						
\\	可能	科学が進歩して、今まで不可能だったことも可能になった 
\\	この成績なら希望の大学に合格することは十分可能だ 
\\	成功のためには可能な限りなんでもするつもりだ 
\\	可能[かのう]			可能:かのう
\\	な, 
\\	基本的	パソコンの基本的な使い方はマニュアルに書いてある 
\\	うちの会社は、基本的に9時から18時まで勤務時間だ 
\\	基本[きほん] 的[てき]			基本的:きほんてき
\\	な 
\\	マニュアル:
\\	勤務:きんむ
\\	国際的	東京で国際的なアニメフェスティバルが開かれた 
\\	あのピアニストは国際的に活躍している 
\\	この大学は留学生が多くてとても国際的ですね 
\\	国際[こくさい] 的[てき]			フェスティバル:
\\	国際的:こくさいてき
\\	な 
\\	ピアニスト:
\\	ぼろぼろ	この家は何十年も住んだので、もうぼろぼろだ 
\\	酷い目にあって身も心もぼろぼろになってしまった 
\\	使い過ぎて、辞書がぼろぼろになった。 
\\	ぼろぼろ			ぼろぼろ:
\\	な, 
\\	非常に	これは非常に高価なものですので、気を付けて運んでください 
\\	非常の際は、この下のレバーを引いてください 
\\	非常[ひじょう]に			非常に:ひじょうに
\\	高価:こうか
\\	な, 
\\	レバー:
\\	大変に	朝夕は電車が大変込むので疲れる 
\\	これまで大変お世話になりました 
\\	計画が中止になったのは大変残念だ 
\\	大変[たいへん]に			朝夕:あさゆう
\\	中止:ちゅうし
\\	とても 
\\	非常に 
\\	ほとんど	村上春樹の小説はほとんど読んだ 
\\	ほとんどの子供たちが携帯電話をもっているそうだ 
\\	来日した時、日本語はほとんどわからなかった 
\\	あの人はほとんど話したことがありません 
\\	ほとんど						
\\	大体	今日の試験は大体できた 
\\	大体の人がそのニュースを知っていた 
\\	1か月の収入は大体15万円ぐらいです 
\\	大体[だいたい]			大体:だいたい
\\	およそ 
\\	可也	頭痛の薬を飲んだら、30分ぐらいで可也よくなった 
\\	きのうの台風で、九州では可也の被害が出たそうだ 
\\	可也			可也:かなり
\\	な, 
\\	ずいぶん	しばらく会わない間に、チンさんはずいぶん日本語が上手になっていた 
\\	怪我の具合はどうですか、ずいぶんよくなりましたが、まだ運動はできま せん 
\\	ずいぶん			随分:ずいぶん
\\	と 
\\	随 
\\	訓: まにま.に したが.う 音: ズイ 
\\	陏	
\\	結構	日曜日なので込んでいるかと思ったら、 結構空いていた 
\\	初めて作った料理だが、 結構おいしくできた 
\\	結構[けっこう]			結構:けっこう
\\	な, 
\\	結 
\\	訓: むす.ぶ ゆ.う ゆ.わえる 音: ケツ ケチ 
\\	糸 
\\	吉 
\\	構 
\\	訓: かま.える かま.う 音: コウ 
\\	木 
\\	冓 
\\	大分	病気は大分よくなった 
\\	仕事がまだ大分残っている 
\\	大分[おおいた]			大分:だいぶん
\\	かなり 
\\	もっと	りんごよりイチゴのほうが好きだ、でも、メロンはもっと好きだ 
\\	もっと大きな声で話してください 
\\	これだけでは足りない、もっとほしい 
\\	もっと			メロン:
\\	イチゴ:
\\	すっかり	桜はもうすっかり散ってしまった 
\\	友達との約束をすっかり忘れた 
\\	病気の具合はどうですか、おかげさまで、すっかりよくなりました 
\\	すっかり			すっかり:
\\	と 
\\	散る:ちる
\\	一杯	会場には子供たちがいっぱいいて、とても賑やかだった 
\\	もう、お腹がいっぱいだ 
\\	姉の部屋は本でいっぱいだ 
\\	窓をいっぱいに開く 
\\	一 杯[はい]						
\\	ぎりぎり	走れば、9時の電車にぎりぎり間に合うだろう 
\\	ぎりぎりまで待ったが、田中君は来なかった 
\\	ぎりぎりで1級に合格することができた 
\\	ぎりぎり			ぎりぎり:
\\	な, 
\\	ぴったり	彼はこの仕事にぴったりだ。 
\\	7時ぴったりに目覚まし時計をセットした 
\\	二人はぴったりとくっ付いて、離れようとしなかった 
\\	そのネクタイ、今日のスーツにぴったりだね 
\\	子供が生まれてから、夫はぴったりとタバコをやめた 
\\	ぴったり			ぴったり
\\	と, 
\\	目覚まし時計:めざましどけい
\\	くっ付く:くっつく
\\	たいてい	日曜日はたいていうちにいます 
\\	たいていの人が田中さんの意見に賛成した 
\\	たいてい			
\\	ほとんど 
\\	同時に	二人は同時にゴールした 
\\	電気が消えて、同時に非常ベルが鳴り始めた 
\\	駅から近いアパートは便利だが、同時に、家賃も高い 
\\	彼女は私の妻であると同時に、仕事のパートナーでもある 
\\	同時[どうじ]に			同時に:どうじに
\\	非常ベル:
\\	パートナー:
\\	前もって	もし欠席する場合は、前もってご連絡します 
\\	デートするときは、いいレストランを前もって調べておくようにしている 
\\	前[まえ]もって			前もって:まえもって
\\	すぐ	チャイムを鳴らすと、すぐにドアが開いた 
\\	うちのすぐ近くで殺人事件があって、とても怖かった 
\\	すぐ			殺人事件:さつじんじけん
\\	もうすぐ	日本へ来て、もうすぐ3年になる 
\\	もうすぐ夏休みだ 
\\	もうすぐ						
\\	突然	子供が突然飛び出してきたので、急ブレーキをかけた 
\\	友人が亡くなった、突然のことで、まだ信じられない 
\\	突然[とつぜん]			突然:とつぜん
\\	な, 
\\	飛び出す:とびだす
\\	ブレーキを掛ける:ブレーキをかける
\\	突 
\\	訓: つ.く 音: トツ カ 
\\	穴 
\\	大 
\\	然 
\\	訓: しか しか.り しか.し さ 音: ゼン ネン 
\\	灬 
\\	あっという間	子供の成長は早い、あっという間に、もう小学生だ 
\\	駅でかばんを盗られてしまった、あっという間の出来事だった 
\\	楽しい時間が過ぎるのは、あっという間だ 
\\	あっという 間[ま]			"あっという間:あっというま 
\\	出来事:できごと
\\	いつの間にか	いつの間にか、外は暗くなっていた 
\\	子供はいつの間にか、私より背が高くなっていた 
\\	いつの間にこんなにお金を使ってしまったのだろう 
\\	いつの 間[ま]にか			いつの間にか:いつのまにか
\\	ずっと	バスより地下鉄で行くほうがずっと速い 
\\	あの人はずっと前に一度会ったことがある 
\\	私は小さいころからずっとピアノを習っている 
\\	ずっと						
\\	相変わらず	あの女優は、年を取った今も相変わらずきれいだ 
\\	久しぶりに会ったが、彼女の気の強さは相変わらずだった 
\\	相変[あいか]わらず			相変わらず:あいかわらず
\\	女優 女優:じょゆう
\\	の 
\\	じょゆう			
\\	次々に	新しいタイプのインスタント食品が次々に発売されている 
\\	あの小説家は次々と新しい作品を発表している 
\\	次々[つぎつぎ]に			次々に:つぎつぎに
\\	タイプ:
\\	どんどん	今日は体調もよく、仕事がどんどん進んだ 
\\	待ってと頼んでも、彼はどんどん先に行ってしまった 
\\	どんどんとドアをノックする大きな音が聞こえた 
\\	どんどん			ノックする:
\\	ますます	朝から降っていた雨は、午後になるとますます強くなった 
\\	彼女、子供のころから可愛かったけど、最近ますますきれいになったね 
\\	ますます						
\\	やっと	30分も待って、やっとバスが来た 
\\	ずいぶん探して、やっと目的地に着くことができた 
\\	安い給料しかもらっていないので、生活が苦しく、食べていくのがやっとだ 
\\	うちの子はまだ小さくて、電車のつり革にやっと手が届くぐらいの身長です 
\\	やっと			つり革:つりかわ
\\	とうとう	長い間使っていた洗濯機がとうとう壊れてしまった 
\\	とうとう人間が月へ行ける日がやって来た 
\\	2時間待ったが、彼はとうとう来なかった 
\\	とうとう						
\\	ついに	若いころの夢をついに実現することができた 
\\	兄は土曜日も休まず働き続け、ついに過労で倒れてしまった 
\\	頑張って練習したが、ついに全国大会に出場することはできなかった 
\\	ついに			遂に:ついに
\\	もちろん	マンションを買った、もちろんローンでした 
\\	あしたのパーティーに行く。もちろん 
\\	彼は中国に10年住んでいたから、もちろん中国語が話せるはずだ 
\\	もちろん			ローン:
\\	当然 
\\	やはり	やはりうちのチームが勝った。予想通りだった 
\\	私のふるさとでは、今でもやはり旧暦で正月を祝うんです 
\\	赤いのがいいなあ。あ、黒いのもいいかなあ。。。やっぱり赤いのにしよう 
\\	やはり			矢張り:やはり
\\	予想通り:よそうどおり
\\	旧暦:きゅうれき
\\	祝う:いわう
\\	旧 
\\	訓: ふる.い もと 音: キュウ 
\\	丨 
\\	日 
\\	暦 
\\	訓: こよみ りゃく 音: レキ 
\\	厂 
\\	林 
\\	日 
\\	きっと	田中さんはいつも遅刻するから、今日もきっと遅れてくるだろう 
\\	あなたなら、きっと合格できる 
\\	今度日本へ来たら、うちに泊まってください。きっとです 
\\	きっと			
\\	必ず 
\\	ぜひ	若い時に、ぜひ留学したいと思っていた 
\\	日本へいらっしゃたら、ぜひ私のうちに泊まってください 
\\	ぜひ						
\\	なるべく	なるべく辞書を見ないで、この本を読んでください 
\\	風邪が流行っているときは、人の多くところへはなるべく行かないほうがいい 
\\	なるべく			成るべく:なるべく
\\	流行る:はやる
\\	案外	今日は曇っているが、案外暖かい 
\\	道が込んでいたが、案外早く着いた 
\\	案外[あんがい]			案外:あんがい
\\	な, 
\\	意外に 
\\	もしかすると	体調が悪いので、もしかすると、明日休むかもしれません 
\\	このごろ成績があがって来たから、もしかしたら大学に合格できるかもしれない 
\\	北村さん、もしかして河内さんのことが好きなんじゃない 
\\	もしかすると			若しかすると:もしかすると
\\	まさか	あの二人、離婚するそうだよ.まさか!あんなに仲良かったのに 
\\	あの成績のいい子林さんが、まさか東京大学に落ちるとは思わなかった 
\\	まさか			真逆:まさか
\\	離婚する:りこん
\\	うっかり	うっかりして、砂糖と塩を間違えて入れてしまった 
\\	買い物に行くのに、うっかり財布を忘れて出かけてしまった 
\\	うっかり			うっかり:
\\	つい	禁煙しようと思うのだが、食事の後などについ吸ってしまう 
\\	疲れていたので、会議中につい居眠りしてしまった 
\\	つい						
\\	思わず	夜道で急に肩を叩かれて、思わずと叫んでしまった 
\\	思[おも]わず			思わず:おもわず
\\	叩く:たたく
\\	ほっと	試験が終わって、ほっとした 
\\	ほっと			ほっとする:
\\	いらいら	急いでいるのにバスがなかなか来なくていらいらした 
\\	いらいら			苛々する:いらいら
\\	のんびり	半年ほどとても忙しかった。温泉へでも行ってのんびりしたい 
\\	定年退職後はふるさとへ帰ってのんびりと暮らしたい 
\\	準備できた。まだ、のんびりしてると間に合わないよ 
\\	のんびり			のんびり:
\\	定年退職:ていねんたいしょく
\\	しっかり	とてもしっかりした論理 
\\	釣り針は魚の口にしっかり刺さっていた 
\\	文章の要点をしっかり押さえなければならない。 
\\	しっかり			聢り:しっかり 
\\	論理:ろんり
\\	釣り針:つりばり
\\	ゆっくり	人工の織物がゆっくりと羊毛産業を圧迫している 
\\	このダンスは、ゆっくりとしたテンポの曲に合う。 
\\	ゆっくり			ゆっくり
\\	織物:おりもの
\\	羊毛:ようもう
\\	圧迫:あっぱく
\\	こってり	生クリームとチーズをたくさん入れたので、このシチューはこってりしている 
\\	こってり叱られる 
\\	こってり			こってり
\\	絞る:しぼる
\\	生クリーム:なまクリーム
\\	チーズ:
\\	シチュー
\\	がっかり	お父さん、入試だめだったよ」「がっかりするな、良雄、運がなかっただめなんだから。」	がっかり			がっかり 
\\	てっきり	田中さんは若そうだからてっきり大学生だと思い込んでいた 
\\	あれ、まだ家にいたんですか。てっきりもう出かけたのだと思っていました。 
\\	てっきり			てっきり 
\\	思い込む:おもいこむ
\\	はっきり	宛先は必ずはっきり書くように 
\\	話し方の先生は、はっきりした発音方法やそのほかの発声技術を教える 
\\	富士山がはっきり見える 
\\	山田さんはイエスかノーかはっきりと言わないから困る 
\\	かぜ薬を飲んだせいか、頭がはっきりしない 
\\	はっきり			はっきり:
\\	と 
\\	宛先:あてさき
\\	発声:はっせい
\\	びっくり	ここであなたに会うなんて驚きです(びっくりしました) 
\\	彼の、写真のような正確で鮮明な記憶にはびっくりした 
\\	びっくり			びっくり
\\	鮮明:せんめい
\\	な, 
\\	記憶:きおく
\\	たっぷり	栄養たっぷりの海藻 
\\	料理はたっぷり作った。 
\\	今朝はたっぷり寝た 
\\	乗り気たっぷりの返事 
\\	たっぷり			たっぷり:
\\	な, 
\\	海藻:かいそう
\\	乗り気:のりき
\\	な, 
\\	藻 
\\	訓: も 音: ソウ 
\\	艹 
\\	澡 
\\	ゆったり	ゆったりとした観光 
\\	ゆったり座って寛ぐ 
\\	あなたの家にいると、まるでふるさとに戻ったようにゆったりと寛ぐ 
\\	ゆったり			ゆったり:
\\	寛ぐ:くつろぐ
\\	すんなり		すんなり			すんなり:
\\	と, 
\\	ぎっしり	箱の中で、マッチがぎっしり入っている 
\\	段ボール箱には、本がぎっしり詰まっている 
\\	ぎっしり			ぎっしり:
\\	生憎 マッチ:
\\	段ボール:だんぼーる
\\	ぐっすり	ぐっすりとよく眠る 
\\	長時間ぐっすり眠る 
\\	ぐっすり			ぐっすり:
\\	と 
\\	深く、十分に眠るようです。
\\	こっそり	書類をこっそり盗み出した 
\\	テストの時は隣の子の答えをこっそり見た。 
\\	社長との会話をこっそりと録音する 
\\	彼は夜中によくこっそり家を抜け出したものだ 
\\	こっそり			こっそり:
\\	と 
\\	盗み出す:ぬすみだす
\\	抜け出す:ぬけだす
\\	きっかり	きっかり_ 時 に 来 る 
\\	時 きっかり 
\\	会議 は3 時 きっかりに 始 まった 
\\	きっかり			きっかり:
\\	さっぱり	彼にはさっぱり洒落が通じなかった. 
\\	この料理は油を使っていないので、さっぱりしていておいしい 
\\	外から帰って、シャワーを浴びたら、さっぱりした 
\\	さっぱり			さっぱり
\\	な, 
\\	洒落:しゃれ
\\	な, 
\\	びっしょり	彼は汗びっしょりだ。 
\\	びっしょり			びっしょり:
\\	な, 
\\	ぐったり	ぐったりした様子だ 
\\	先生をぐったりさせる 
\\	ぐったりいすに座る 
\\	死んだようにぐったりしている 
\\	ぐったり			ぐったり:
\\	にっこり	にっこり笑らう	にっこり			にっこり:
\\	にこにこ	高木さんはガールフレンドから電話をもらって、ニコニコしている 
\\	あの人はいつもニコニコしていて、楽しそうだよ 
\\	にこにこ			
\\	ガールフレンド:
\\	ずっしり	ずっしりと重い	ずっしり			ずっしり:
\\	きっぱり	「会社をやめる」と、佐藤さんはきっぱり言った。 
\\	ヘロイン をきっぱりとやめる 
\\	きっぱりした 返事 
\\	きっぱり			きっぱり:
\\	と, 
\\	ヘロイン:
\\	うんざり	部長の長い話はみんなうんざりしている 在日外国人の多くが『外人』という言葉にうんざりしている 
\\	うんざり			うんざり
\\	ばったり	会議でばったり偶然 高山先生に会う 
\\	ばったり			ばったり: 
\\	ぼんやり	隆君は授業中いつもぼんやりしている 
\\	海の向こうに島がぼんやり見える 
\\	ぼんやり			ぼんやり
\\	と, 
\\	じっくり	結婚についてじっくり思案する 
\\	春の草花を見つけたら、じっくり観察してノートに書き留めるんだ 
\\	じっくり			じっくり:
\\	と 
\\	思案する:しあん
\\	観察:かんさつ
\\	書き留める:かきとめる
\\	演奏	楽器を演奏する 
\\	彼女は生まれて初めて大観衆の前でピアノの演奏した。 
\\	演奏[えんそう]			楽器:がっき
\\	演奏:えんそう
\\	観衆:かんしゅう
\\	演 
\\	音: エン 
\\	氵 
\\	寅 
\\	奏 
\\	訓: かな.でる 音: ソウ 
\\	天 
\\	芸術	美術、音楽、文学、演劇などは、みな、芸術の一種であると言える 
\\	芸術[げいじゅつ]			芸術:げいじゅつ
\\	演劇:えんげき
\\	一種:いっしゅ
\\	劇 
\\	音: ゲキ 
\\	刂 
\\	検査	製品の検査 
\\	病院で脳の検査を受けた 
\\	検査[けんさ]			検査:けんさ
\\	血液	体の中を血液が流れている 
\\	足から心臓に向かって流れる血液 
\\	血液[けつえき]			血液:けつえき
\\	心臓:しんぞう
\\	血 
\\	訓: ち 音: ケツ 
\\	皿 
\\	液 
\\	音: エキ 
\\	氵 
\\	夜 
\\	臓 
\\	訓: はらわた 音: ゾウ 
\\	月 
\\	蔵 
\\	治療	病気を治療した 
\\	虫歯の治療を受けた 
\\	治療[ちりょう]			治療:ちりょう
\\	虫歯:むしば
\\	療 
\\	音: リョウ 
\\	疒 
\\	尞	
\\	症状	風邪の症状は、熱、咳、鼻水など 
\\	症状[しょうじょう]			症状:しょうじょう
\\	鼻水:はなみず
\\	症 
\\	音: ショウ 
\\	疒 
\\	正 
\\	状 
\\	音: ジョウ 
\\	犬 
\\	予防	風邪の予防 
\\	災害を予防する 
\\	予防[よぼう]			災害:さいがい
\\	栄養	栄養のある食べ物 
\\	健康のために、栄養に気をつけましょう 
\\	栄養[えいよう]						
\\	死亡	死亡の原因を調べる 
\\	事故で3名死亡した 
\\	死亡[しぼう]			死亡:しぼう
\\	命	命は大切にしなければならない 
\\	命[いのち]			
\\	生命 
\\	一生	ピカソの一生を調べる 
\\	幸せな一生を送る 
\\	あなたのことは、一生忘れません 
\\	一生[いっしょう]						
\\	誤解	誤解がないようにするには、よく話し合うことが大切だ 
\\	テストで問題の意味を誤解して答えを間違えた 
\\	誤解[ごかい]			誤解:ごかい
\\	誤 
\\	訓: あやま.る -あやま.る 音: ゴ 
\\	言 
\\	呉 
\\	後悔	私は若いころに勉強しなかったことを、とても後悔している 
\\	後悔[こうかい]			後悔:こうかい
\\	訳	遅刻した訳を話す 
\\	二人が別れたわけを知りたい 
\\	彼が言っていることは、わけがわからない 
\\	訳[わけ]			別れる:わかれる
\\	理由 
\\	事情 
\\	意味 
\\	理屈 
\\	態度	あの学生は、授業の態度が悪い 
\\	あいまいな態度をとる 
\\	好きか嫌いか、態度をはっきりさせて 
\\	態度[たいど]			曖昧:あいまい
\\	な, 
\\	癖	私の癖は、困ったとき頭をかくことだ 
\\	正しい形を見て練習しないと、時に変な癖がつく 
\\	癖[くせ]			癖:くせ
\\	癖 
\\	訓: くせ くせ.に 音: ヘキ 
\\	疒 
\\	辟 
\\	礼儀	目上の人の礼儀 
\\	あの人は礼儀を知らない 
\\	礼儀[れいぎ]			礼儀:れいぎ
\\	エチケント 
\\	、マナー 
\\	文句	給料に文句がある 
\\	店に商品について文句を言う 
\\	歌の文句 
\\	小説の文句を引用する 
\\	文句[もんく]			文句:もんく
\\	引用:いんよう
\\	表情	顔の表情 
\\	あの人は表情が豊かだ 
\\	表情を変える 
\\	表情[ひょうじょう]			表情:ひょうじょう
\\	表面	水の表面 
\\	あのビルは表面にタイルが貼ってある 
\\	あの人は表面は優しそうだが、実は意地が悪い 
\\	表面[ひょうめん]			表面:ひょうめん
\\	タイル:
\\	禁煙	この部屋は禁煙です 
\\	子供が生まれるので、禁煙することにした 
\\	禁煙[きんえん]			
\\	喫煙:きつえん
\\	禁 
\\	音: キン 
\\	林 
\\	示 
\\	禁止	美術館の中では、写真を撮ることは禁止されている 
\\	館内への食べもの、飲み物の持ち込むは禁止です 
\\	禁止[きんし]			禁止:きんし
\\	館内:かんない
\\	持ち込む:もちこむ
\\	禁 
\\	音: キン 
\\	林 
\\	示 
\\	完成	建物が完成した 
\\	半年かけて論文を完成させた 
\\	完成[かんせい]						
\\	課題	現在の日本には、高齢化、ごみ問題など、多くの課題がる 
\\	鈴木先生の授業では、毎週課題が出される 
\\	課題[かだい]			課題:かだい
\\	高齢化:こうれいか
\\	現在げんざい
\\	例外	どんな規則にも例外がある 
\\	休むと試験は受けられないが、病気の場合は例外だ 
\\	例外[れいがい]			例外:れいがい
\\	基本	なんの練習でも、基本が大切だ 
\\	基本[きほん]			基本:きほん
\\	訓: もと もとい 音: キ 
\\	其 
\\	土 
\\	記録	先週の会議の記録を読んだ 
\\	この事件を記録に残す必要ある 
\\	北島選手は世界新記録で優勝した 
\\	きのうの雨量は過去最高を記録したそうだ 
\\	記録[きろく]			記録:きろく
\\	記 
\\	訓: しる.す 音: キ 
\\	言 
\\	己 
\\	録 
\\	訓: しる.す と.る 音: ロク 
\\	金 
\\	彔 
\\	出来事	大きな出来事 
\\	毎日の出来事をブログに書く 
\\	出来事[できごと]			出来事:できごと
\\	場面	目の前でトラックとバスが衝突した、その場面が夢に出てきた 
\\	ドラマの場面が変わる 
\\	場面[ばめん]			場面:ばめん
\\	目の前:めのまえ
\\	シーン 
\\	機会	彼女と二人で話したいのだが、なかなか機会がない 
\\	バレンタインデーは愛の告白のいい機会だ 
\\	機会[きかい]			機会:きかい
\\	バレンタインデー:
\\	告白:こくはく
\\	チャンス 
\\	距離	駅からの距離を測る 
\\	ここから学校までは、かなり距離がある 
\\	距離[きょり]			距離:きょり
\\	距 
\\	訓: へだ.たる けづめ 音: キョ 
\\	足 
\\	巨 
\\	離 
\\	訓: はな.れる はな.す 音: リ 
\\	离 
\\	隹 
\\	提案	会議で提案をする 
\\	社長に新しい計画を提案する 
\\	提案[ていあん]			提案:ていあん
\\	やり取り	友達とメールをやり取りする 
\\	情報のやりとり 
\\	やり取[と]り			やり取り:やりとり
\\	交換 
\\	知識	本を読んで知識を身につける 
\\	知識[ちしき]						
\\	実力	試合で実力を出す 
\\	実力[じつりょく]			実力:じつりょく
\\	手段	問題を解決するために必要な手段をとる 
\\	目的のため,手段を選ばない 
\\	手術は最後の手段だ 
\\	手段[しゅだん]			手段:しゅだん
\\	方法 
\\	代表	クラスの代表として会議に参加する 
\\	優勝して、日本の代表になった 
\\	富士山は日本を代表する山 
\\	この映画は
\\	監督の代表的な作品だ 
\\	代表[だいひょう]			代表:だいひょう
\\	監督:かんとく
\\	効果	この薬を飲んだら、すくに効果が出た 
\\	効果[こうか]						
\\	印象	彼女と初めて会った時、優しそうな人だという印象を受けた 
\\	彼は昔、よく先生に怒られていたという印象がある 
\\	印象[いんしょう]			印象:いんしょう
\\	印	地図の目的地に印をつけた 
\\	鳩は平和の印だ 
\\	愛の印に恋人に指輪を贈る 
\\	印[しるし]			印:しるし
\\	目的地:もくてきち
\\	鳩:はと
\\	合図	友達に目で合図を送る 
\\	首肯くことは普通同意の合図である。 
\\	合図[あいず]			合図:あいず
\\	首肯く:うなずく
\\	同意:どうい 
\\	共通	二人の共通の趣味は音楽だ 
\\	横浜と神戸の共通点は、大きな港があることだ 
\\	少子高齢化は先進国に共通する問題だ 
\\	共通[きょうつう]			共通:きょうつう
\\	な, 
\\	共通点 少子高齢化:しょうしこうれいか
\\	先進国:せんしんこく
\\	強調	大事な点を強調して説明する 
\\	過度 の 強調 
\\	国際的強調 
\\	強意強調 
\\	音声強調 
\\	強調[きょうちょう]			強調:きょうちょう
\\	過度:かど
\\	な, 
\\	省略	詳しい説明を省略して簡単に話す 
\\	省略[しょうりゃく]			省略:しょうりゃく
\\	挑戦	難しい課題に挑戦する 
\\	世界記録への挑戦 
\\	テニスの試合で家内に挑戦 する 
\\	挑戦[ちょうせん]			挑戦:ちょうせん
\\	やる気	やる気のある人 
\\	褒められてやる気になる 
\\	やる 気[き]			やる気:やるき
\\	勇気	困難に立ち向かう勇気 
\\	勇気を出してプロポーズする 
\\	勇気[ゆうき]			勇気:ゆうき
\\	困難:こんなん
\\	な, 
\\	立ち向かう:たちむかう
\\	資格	弁護士の資格を取る 
\\	彼女は奨学金をもらう資格が十分ある 
\\	あなたも同じことをしたのだから、あなたに彼を悪く言う資格はない 
\\	資格[しかく]			資格:しかく
\\	弁護士:べんごし
\\	奨学金:しょうがくきん
\\	悪く言う:わるくいう
\\	申請	パスポートの申請 
\\	大使館でビザを申請する 
\\	申請[しんせい]			申請:しんせい
\\	本人	通帳は貯金をする本人が作らなけれなならない 
\\	本人[ほんにん]			本人:ほんにん
\\	の 
\\	通帳:つうちょう
\\	契約	アパートを2年間借りる契約をした 
\\	我が社は今度、
\\	社と契約を結ぶことにした 
\\	契約を結ぶ。
\\	契約[けいやく]			契約:けいやく
\\	結ぶ:むすぶ
\\	証明	銀行口座を開くときには、身分証明が必要だ 
\\	無実を証明 
\\	証明[しょうめい]			証明:しょうめい
\\	銀行口座:ぎんこうこうざ
\\	身分証明:みぶんしょうめい 
\\	変更	予定の変更を行う 
\\	旅行を沖縄から北海道に変更する 
\\	変更[へんこう]			
\\	変える 
\\	保存	食料の保存 
\\	パソコンにデータを保存する 
\\	保存[ほぞん]			保存:ほぞん
\\	保護	保護フィルム: 
\\	自然を保護する 
\\	迷子の保護 
\\	警察に保護される 
\\	病気で働けない生活保護を受けている 
\\	保護[ほご]			保護:ほご
\\	迷子:まいご
\\	資源	資源を有効に利用する 
\\	資源[しげん]			資源:しげん
\\	有効:ゆうこう
\\	な, 
\\	不足	野菜が不足した食事 
\\	戦争で食料が不足する 
\\	不足[ふそく]			不足:ふそく
\\	な, 
\\	食料:しょくりょう
\\	平均	テストの点を平均する 
\\	一日に平均1800キロカロリーの食事をとる 
\\	商品の質が平均している 
\\	平均[へいきん]			平均:ならし
\\	割合	デパートの客は女性の割合が高い 
\\	年に10%の割合で売る上げが伸びている 
\\	割合[わりあい]			割合:わりあい
\\	わりに 
\\	比率 
\\	商売	父は衣料品の商売をしている 
\\	商売[しょうばい]						
\\	生産	日本は農作物の生産が少ない 
\\	工場でカメラを生産する 
\\	生産[せいさん]			生産:せいさん
\\	農作物:のうさくもつ
\\	消費 
\\	消費	牛乳の消費が減っている 
\\	運動してカロリーを消費する 
\\	消費[しょうひ]						
\\	現代	現代は情報の時代だと言われている 
\\	現代[げんだい]						
\\	戦争	
\\	国と 
\\	国の間で戦争が起きた 
\\	戦争が終わって平和になった 
\\	戦争[せんそう]						
\\	上る	坂を上る 
\\	東京タワーに上る 
\\	煙が空に上る 
\\	地震の死者は、5000人に上った 
\\	上[のぼ]る						
\\	下る	山を下る 
\\	下[くだ]る						
\\	進む	出口に向かって進む 
\\	工事は予定通り進んでいる 
\\	日本は科学術が進んでいる 
\\	この時計は5分進んでいる 
\\	4月から高校に進みます 
\\	進[すす]む						
\\	進める	もう少し車を前に進めてください 
\\	会議の準備を進める 
\\	時計を10分進める 
\\	進[すす]める			進める:すすめる
\\	通る	この道は車がたくさん通る 
\\	改札口を通ってホームに上がる 
\\	私の町の真ん中に大きな道が通っている 
\\	この肉はよく火が通っていない 
\\	会議で私の意見が通った 
\\	通[とお]る			通る:とおる
\\	改札口:かいさつぐち
\\	通す	この布は、空気は通すが水は通さない 
\\	針に糸を通す 
\\	隣の県まで鉄道を通す 
\\	すみません、ちょっと通させてください 
\\	客を応接室に通す 
\\	この提案を会議で通したい 
\\	通[とお]す			通す:とおす
\\	糸:いと
\\	応接室:おうせつしつ
\\	超える	この山を越えると隣の県だ 
\\	テストに平均点は80点を超えた 
\\	最高気温が30度を超える日を真夏日という 
\\	超[こ]える			超える:こえる
\\	過ぎる	列車は広島駅を過ぎた 
\\	約束の時間を過ぎても友達は来なかった 
\\	過[よ]ぎる			過ぎる:すぎる
\\	過ごす	大学時代を東京で過ごした 
\\	日曜日は家でテレビを見て過ごすことが多い 
\\	過[す]ごす			過ごす:すごす
\\	移る	黒板の字が見えにくかったので、前の席に移った 
\\	来月から営業部に移ることになった 
\\	日本の首都は19世紀後半に京都から東京に移った 
\\	風邪をひいた.たぶん、田中ンさんから移ったのだと思う 
\\	移[うつ]る			移る:うつる
\\	移す	本社を大阪から東京に移した 
\\	田中さんに風邪を移されてしまった 
\\	移[うつ]す			移す:うつす
\\	連れる	子供を動物園へ連れて行った 
\\	明日娘が恋人を家につれてくることになっている 
\\	連[つ]れる			連れる:つれる
\\	寄る	窓の側に寄って外を見た 
\\	ポスターが左に寄っていますから、直してください 
\\	会社からの帰りに、雑誌を買いに本屋へ寄った 
\\	寄[よ]る		
\\	寄る:よる
\\	側:そば
\\	寄せる	地震の時は、車を道の左側に寄せて止めなければならない 
\\	寄[よ]せる			寄せる:よせる
\\	与える	成績の良い学生に奨学金を与える 
\\	影響を与える 
\\	与[あた]える			与える:あたえる
\\	得る	収入を得る 
\\	会議は許可を得てからでなければ発言してはいけない 
\\	得[え]る			得る:える
\\	勧める	ダイエットしている友人に、一緒にウオーキングをしようと勧めた 
\\	客にお茶を進めた 
\\	勧[すす]める			勧める:すすめる
\\	勧 
\\	訓: すす.める 音: カン ケン 
\\	力 
\\	薦める	先生に薦められた本を買った 
\\	「議長に誰がいいですか」「山田さんを薦めます」 
\\	お薦めの店があったら教えてください 
\\	母親は子供達 を野菜を食べるようにと進 めます 
\\	薦[すす]める			薦める:すすめる
\\	薦 
\\	訓: すす.める 音: セン 
\\	艹 
\\	任せる	責任の重い仕事を新人に任せてみた 
\\	パーティーの司会よろしく頼むよ、任せてください 
\\	任[まか]せる			任せる:まかせる
\\	司会:しかい
\\	守る	法律を守る 
\\	環境を守る 
\\	子供を危険から守る 
\\	守[まも]る						
\\	争う	あの兄弟はいつも小さなことで争っている 
\\	山田選手と高橋選手がトップを争っている 
\\	争[あらそ]う			争う:あらそう
\\	望む	卒業後は、教師になることを望んでいる 
\\	あなたの未来が明るいものであるよう望みます 
\\	望[のぞ]む			望む:のぞむ
\\	信じる	きのう、 幽霊を見たよ、信じられない 
\\	頑張れば成功すると信じている 
\\	友人を信じてお金を預けた 
\\	私は仏教を信じています 
\\	信[しん]じる			幽霊:ゆうれい
\\	仏教:ぶっきょう
\\	幽 
\\	訓: ふか.い かす.か くら.い しろ.い 音: ユウ 
\\	山 
\\	幺 
\\	霊 
\\	訓: たま 音: レイ リョウ 
\\	雨 
\\	通じる	日本に来た時、私の日本語が通じるかどうか心配だった 
\\	何も言わなければ、気持ちは通じない 
\\	山田さんはまじめすぎて、冗談が通じない 
\\	この地下道は駅に通じている 
\\	コンサートチケット申し込み電話がやっと通じた 
\\	通[つう]じる			通じる:つうじる
\\	飽きる	好きなものでも、毎日食べると飽きてしまう 
\\	もうこのゲームは飽きた 
\\	飽[あ]きる						
\\	思いつく	アイディアを思いつく 
\\	スピーチを頼まれたのだが、なかなかいい表現を思いつかない 
\\	思[おも]いつく						
\\	思いやる	父の言葉は厳しかったが、その言葉は子供を思いやる気持ちが表れていた 
\\	思[おも]いやる			
\\	熱中する	母は今カラオケ熱中している 
\\	テレビのボクシング中継に熱中して、つい大声を出してしまった 
\\	熱中[ねっちゅう]する			熱中する:ねっちゅうする
\\	中継:ちゅうけい
\\	継 
\\	訓: つ.ぐ まま- 音: ケイ 
\\	糸 
\\	巻く	マフラーを首に巻く 
\\	指に包帯を巻く 
\\	コードお巻いて片付ける 
\\	巻[ま]く			包帯:ほうたい
\\	コード:
\\	結ぶ	靴の紐を結ぶ 
\\	ネクタイを結ぶ 
\\	髪をリボンで結ぶ 
\\	東京とニューヨークを結ぶ飛行機は、1日何便ですか 
\\	契約を結ぶ 
\\	結[むす]ぶ		
\\	リボン
\\	紐 
\\	訓: ひも 音: チュウ ジュウ 
\\	糸 
\\	丑 
\\	済む	仕事が済んだらすぐ帰る 
\\	きょうは暖かいから、コートを着ないで済んだ 
\\	済[す]む			済む:すむ
\\	済ませる	お金の支払いを済ませる 
\\	朝はパンと牛乳で済ます 
\\	済[す]ませる						
\\	出来る	駅前に新しいスーパーが出来た 
\\	友達が出来た 
\\	2時間もかかって、やっと料理が出来た 
\\	筆記試験は出来たのだが、面接で失敗してしまった 
\\	この椅子は木で出来ている 
\\	出来[でき]る		
\\	筆記試験:ひっきしけん
\\	切れる	このはさみはよく切れる 
\\	定期は切れる 
\\	電池は切れて、ラジオが聞こえなくなった 
\\	料理を作ろうとして、塩がきれていることに気が付いた 
\\	切[き]れる						
\\	切らす	うっかりしていて、さとうを切らしてしまった 
\\	申し訳ありません、名刺を切らしておりまして。。。 
\\	切[き]らす			切らす:きらす
\\	伝わる	彼が結婚するといううわさが伝わって来た 
\\	この地方には昔から伝わる不思議な話がある 
\\	漢字は中国から伝わって来た 
\\	空気がないと音は伝わらない 
\\	伝[つた]わる			伝わる:つたわる
\\	伝える	電話で用事を伝える 
\\	ふるさとの料理を若い人に伝える 
\\	ザビエルがキリスト教を日本に伝えた 
\\	金属は熱をよく伝える 
\\	伝[つた]える		
\\	伝える:つたえる
\\	キリスト教:キリストきょう
\\	金属:きんぞく
\\	属 
\\	訓: さかん つく やから 音: ゾク ショク 
\\	尸 
\\	禹 
\\	[屬]. 
\\	[屬].	
\\	続く	雨の日が続く 
\\	地震が続いて起こる 
\\	続[つづ]く						
\\	続ける	もう3時間も会議
\\	が続けている 
\\	3回続けて遅刻して、先生に怒られた 
\\	続[つづ]ける			続ける:つづける
\\	繋がる	本州と四国は橋で繋ががっている 
\\	この道が駅に繋ががっている 
\\	渋滞で車が1キロも繋ががっている 
\\	やっと電話が繋がった 
\\	繋[つな]がる			繋がる:つながる
\\	繋 
\\	訓: つな.ぐ かか.る か.ける 音: ケイ 
\\	糸 
\\	繋ぐ	この橋は本州と四国をつないでいる 
\\	恋人と手をつないで歩く 
\\	もしもし、102号室をお願いします、はい、おつなぎします 
\\	繋[つな]ぐ			繋ぐ:つなぐ
\\	繋 
\\	訓: つな.ぐ かか.る か.ける 音: ケイ 
\\	糸 
\\	つなげる	このチャンスをぜひ成功につなげたい 
\\	繋[つな]げる			つなげる 
\\	伸びる	髪が伸びる 
\\	体操をしたら、腰が伸びた 
\\	記録が伸びる 
\\	ラーメンが伸びる 
\\	伸[の]びる			伸びる:のびる
\\	体操:たいそう
\\	伸ばす	身長を伸ばす 
\\	アイロンをかけてしわを伸ばす 
\\	練習して、水泳の記録を伸ばす 
\\	伸[の]ばす			伸ばす:のばす
\\	アイロン:
\\	皺:しわ
\\	皺 
\\	訓: しわ しぼ 音: シュウ スウ 
\\	芻 
\\	皮 
\\	延びる	レポートの締め切りが延びてよかった 
\\	2時間の予定だった会議が延びて3時間になった 
\\	鉄道が、隣の市まで延びた 
\\	延[の]びる			延びる:のびる
\\	延ばす	チケットが取れなかったので、帰国の日を延ばした 
\\	みんなが終らないので、先生がテストの時間を10分延ばしてくれた 
\\	道路を20キロ先まで延ばした 
\\	延[の]ばす			延ばす:のばす
\\	重なる	印刷したら、紙が2枚重なって出てきた 
\\	日曜日と祝日が重なると、次の月曜日が休みになる 
\\	重[かさ]なる			重なる:かさなる
\\	印刷いんさつ
\\	印 
\\	訓: しるし -じるし しる.す 音: イン 
\\	卩 
\\	刷 
\\	訓: す.る -ず.り -ずり は.く 音: サツ 
\\	刂 
\\	重ねる	皿を重ねて置いておく 
\\	練習を重ねる 
\\	重[かさ]ねる			重ねる:かさねる
\\	広がる	道幅が広がって、歩きやすくなった 
\\	最近、裾が広がってデザインのスボンがはやっている 
\\	強風のため、火事がどんどん広がっている 
\\	海が広がっている 
\\	広[ひろ]がる			広がる:ひろがる
\\	広げる	お客さんが増えているので、もっと店を広げたい 
\\	勢力を広げる 
\\	机の上に新聞を広げて読む 
\\	広[ひろ]げる			広げる:ひろげる
\\	載る	この荷物は大きすぎて、私の車には載らないだろう 
\\	新聞に、私の書いた記事が載った 
\\	載[の]る			載る:のる
\\	載せる	棚に荷物を載せる 
\\	いい作品は文集に載せますから、頑張って書いてください 
\\	載[の]せる		
\\	載せる:のせる
\\	そろう	全員がそろったら出発しましょう 
\\	カーテンとカーペットの色がそろっている 
\\	みんなの意見がなかなかそろわない 
\\	そろう			揃う:そろう
\\	そろえる	マージャンをするため、メンバーをそろえた 
\\	カーテンとカーペットの色をそろえる 
\\	脱いだ靴はそろえておくのがエチケットだ 
\\	そろえる			マージャン
\\	脱ぐ:ぬぐ
\\	エチケット:
\\	纏まる	3時間話し合って、やっとクラスの意見がまとまった 
\\	あしたスピーチをしなければならないのに、なかなか考えがまとまらない 
\\	纏[まと]まる			纏まる:まとまる
\\	纏 
\\	訓: まつ.わる まと.う まと.める まと.まる まと.い 音: テン デン 
\\	糸 
\\	廛 
\\	纏める	引っ越しの前に、いらなくなったものをまとめて捨てた 
\\	話す前に考えをまとめておく 
\\	来月までに論文をまとめなければならない 
\\	纏[まと]める			纏める:まとめる
\\	纏 
\\	訓: まつ.わる まと.う まと.める まと.まる まと.い 音: テン デン 
\\	糸 
\\	廛 
\\	付く	けがをしてシャツに血がつぃた 
\\	ウオーキングを毎日したら体力がついた 
\\	このおかしには お負けがついている 
\\	このスープには味がついていない 
\\	付[つ]く			お負け:おまけ
\\	付ける	パンにジャムをつける 
\\	日本へ来ていろいろな知識を身につけた 
\\	英語ができない人には通訳をつけます 
\\	塩でスープに味をつける 
\\	付[つ]ける						
\\	貯まる	お金がたまる	貯[た]まる			貯まる:たまる
\\	貯める	ストレスをためる 
\\	宿題をためてしまい、休みの最後の日にまとめてやった 
\\	貯[た]める			貯める:ためる
\\	混ざる/ 交ぜる	材料が十分混ざっていないと、おいしいケーキはできない 
\\	混[ま]ざる						
\\	交ぜる/ 混ぜる	お米に豆を交ぜていた 
\\	赤と黄色を混ぜるとおレジ色になる 
\\	交[ま]ぜる/ 混[ま]ぜる						
\\	溶ける/ 解ける	3時間かかって、やっと問題が解けた 
\\	春になって雪がとけた 
\\	この洗剤は冷たい水に溶けにくい 
\\	溶[と]ける/ 解[と]ける			
\\	溶ける:とける
\\	解ける:とける
\\	解く/ 溶く/ 溶かす	数学の問題を解く 
\\	電子レンジでバターをとかす 
\\	コーヒーにさとうを溶かす 
\\	卵を溶いてフライパンに流す 
\\	解[と]く/ 溶[と]く/ 溶[と]かす						
\\	含む	レモンはビタミン
\\	を多く含んでいる 
\\	この値段には消費税がふくまれている 
\\	含[ふく]む						
\\	含める	うちの家族は、私を含めて5人です 
\\	毎日の昼食代は、飲物も含めると1000円ぐらいだ 
\\	含[ふく]める					含める:ふくめる
\\	抜ける	髪の毛が抜ける 
\\	タイヤの空気が抜けた 
\\	この書類は3ページ目が抜けている 
\\	抜[ぬ]ける			抜ける:ぬける
\\	タイヤ:
\\	抜 
\\	訓: ぬ.く -ぬ.く ぬ.き ぬ.ける ぬ.かす ぬ.かる 音: バツ ハツ ハイ 
\\	扌 
\\	友 
\\	抜く	虫歯を抜く 
\\	体の力をぬく 
\\	朝食をぬく 
\\	マラソンで、前の3人を抜いて、トップになった 
\\	抜[ぬ]く			抜く:ぬく
\\	現れる	犯人は金を取りに現れたところを逮捕された 
\\	新しい薬の効果がすぐに表れた 
\\	現[あらわ]れる						
\\	現す	雲がなくなって、富士山が姿を現した 
\\	新しい薬がすぐに効果を現した 
\\	現[あらわ]す			
\\	表す	気持ちを表す 
\\	地図では郵便局を表す 
\\	表[あらわ]す		
\\	散る	風で桜が散ってしまった 
\\	散[ち]る						
\\	散らす	風が桜を散らしてしまった 
\\	散[ち]らす						
\\	明ける	梅雨があける 
\\	明[あ]ける						
\\	差す	雨がやんで、日が差してきた 
\\	傘をさす 
\\	目薬を差す 
\\	差[さ]す						
\\	パートナー	あの人は仕事のいいパートナーだ 
\\	うちの犬はただのペットではなくて、私のパートナーです 
\\	パートナー			
\\	リーダー	グループのリーダーを決める 
\\	この国に新しいリーダーが誕生した 
\\	リーダー			
\\	ユーモア	ユーモアがある彼はクラスの人気者だ 
\\	旅行の話を、ユーモアたっぷりに話す 
\\	ユーモア			
\\	ショック	彼女にほかの恋人がいて、ショックだった 
\\	シートベルトをすれば、ショックが小さくなる 
\\	ショック						
\\	ストレス	ストレスがたまったときは、カラオケに行く 
\\	ストレスで胃に穴が空いた 
\\	ストレス						
\\	バランス	栄養のバランスがいい食事をする 
\\	このデザインは左右のバランスが悪い 
\\	バランス						
\\	レベル	マリアさんは日本語のレベルが高い 
\\	レベルに別にクラスを分ける 
\\	レベル						
\\	アップ	時給がアップした 
\\	写真をアップでとる 
\\	アップ						
\\	ダウン	ダウンで生活が苦しい 
\\	ダウン						
\\	プラス	漢字を覚えるときは、日本語の勉強にプランになる 
\\	まじめに働いていたら、時給が20プラスされた 
\\	プラス						
\\	マイナス	今月の給料は1万円マイナスだった 
\\	マイナス						
\\	イメージ	私はこの曲から広い海をイメージした 
\\	イメージがいい女優を
\\	に使う 
\\	イメージ						
\\	マスコミ	このニュースはマスコミに注目されている 
\\	マスコミ関係の仕事がしたい 
\\	マスコミ			
\\	プライバシ	現代社会では個人のプライバシーが守られないことが多い 
\\	プライバシ			
\\	ルール	サッカーのルールを覚える 
\\	交通ルールは国によって違う 
\\	ルール			ルール 
\\	マナー	フランス料理のマナーは難しい	マナー			マナー 
\\	スケジュール	今週のスケジュールを確認する 
\\	週末はスケジュールでいっぱいだ 
\\	スケジュール			
\\	タイトル	タイトルをよく見てから文章を読む 
\\	有名な映画のタイトルが思い出せない 
\\	タイトル			
\\	テーマ	この映画のテーマは愛だ 
\\	論文のテーマが決まらない 
\\	テーマ			
\\	ストーリー	あの映画のストーリーを教えてください 
\\	ストーリー			
\\	ヒット	あの歌手の新曲はヒットしている 
\\	ヒット			
\\	ブランド	海外でブランドのバッグを買った 
\\	彼女は一流ブランドのものしか使わない 
\\	ブランド						
\\	レンタル	このすーしケースはレンタルだ 
\\	パーティードレスはレンタルすることにした 
\\	レンタル			
\\	リサイクル	古くなった服をリサイクルに出した 
\\	新聞紙はリサイクルされて、トイレットペーパーなどになる 
\\	リサイクル						
\\	タイプ	このクラスにはいろいろなタイプの学生がいる 
\\	このタイプのパソコンはもう売っていない 
\\	タイプ						
\\	スタイル	彼女はスタイルを気にしてダイエットしている 
\\	スタイル						
\\	セット	テーブルといすをセットで買う 
\\	目ざまし時計を6時にセットした 
\\	セット						
\\	ウイルス	この病気はウイルスによって起こる 
\\	ウイルスが入ってパソコンが故障した 
\\	ウイルス						
\\	ロボット	子供はロボットのアニメは好きだ 
\\	ロボットは工場で危険な作業をする 
\\	ロボット						
\\	デジタル	デジタルの時計は見やすい 
\\	デジタル						
\\	マイク	みんなの前でマイクを持って話す 
\\	マイク						
\\	ブレーキ	ブレーキを踏んでスピードを落とした 
\\	石油の値上がりで、経済の発展にブレーキがかかった 
\\	弟のいたずらがひどいときは、ブレーキをかけるようにしている 
\\	ブレーキ						
\\	ペンキ	かべにペンキを塗る 
\\	ペンキ			
\\	炊ける	ご飯が炊ける 
\\	炊[た]ける						
\\	炊く	ご飯を炊く 
\\	炊[た]く						
\\	煮る	この野菜をスープで煮て、煮えたら塩で味をつけてください 
\\	煮[に]る						
\\	炒める	フライパンに油をひいて、肉と野菜を炒める 
\\	炒[いた]める			
\\	焼ける	よく焼けれいない豚肉は食べないほうがいい 
\\	焼[や]ける						
\\	焼く	フライパンでオムレツを焼く 
\\	トースターでパンを焼く 
\\	炭火で魚を焼く 
\\	焼[や]く			炭:すみ
\\	炭 
\\	訓: すみ 音: タン 
\\	山 
\\	灰 
\\	ゆでる	熱い湯で卵をゆでる 
\\	ゆでる						
\\	揚げる	油でてんぷらを揚げる 
\\	揚[あ]げる						
\\	蒸す	湯を沸かして、その湯気でギョーザを蒸す 
\\	蒸[ふか]す						
\\	ぐっすり	子供はぐっすり眠っていて、起こしてもなかなか起きなかった 
\\	ぐっすり寝たので疲れがとれた 
\\	ぐっすり			
\\	きちんと	背中をまっすぐにして、きちんと座りなさい 
\\	鈴木さんはいつも言われたことをきちんとやる人だ 
\\	明日はきちんとした服で来てください 
\\	きちんと			
\\	じっと	じっと考える 
\\	暑くて、じっと座っていても汗が出てくる 
\\	小さい子供はなかなかじっとしていない 
\\	じっと		
\\	そっと	寝ている人を起こさないように、そっと部屋を出た 
\\	壊さないように、そっと持ってください 
\\	今は誰も話したくないので、そっとしておいてください 
\\	そっと			
\\	別々に	チョコレートとクッキ―は別々に包んでください 
\\	今までは二人でやって来たが、これから別々に道を行くことにした 
\\	別々[べつべつ]に			
\\	それぞれ	みんなはそれぞれ意見を言った 
\\	これらの絵には、それぞれ良いところがある 
\\	うちの家族のそれぞれがパソコンを持っている 
\\	それぞれ			
\\	ただ	子供は何を聞いてもただ泣いているだけだった 
\\	その学校で、私はただ一人の日本人だった 
\\	ただ			
\\	少なくとも	ここから駅まで歩いたら、少なくとも30分はかかるだろう 
\\	毎日予習をしてください、少なくとも、言葉の意味は調べてきてください 
\\	少[すく]なくとも			
\\	決して	このことは、決してほかの人には言わないつもりだ 
\\	最後まで決してあきらめない 
\\	決[けっ]して			
\\	少しも	日運動しているのに、少しも体重が減らない 
\\	あの人がうそをついているときは、少しも思わなかった 
\\	少[すこ]しも			
\\	ちっとも	あの二人、付き合っているらしいよ」「へえ、ちっとも知らないかっ た」 
\\	周りがうるさいので、ちっとも勉強が進まない 
\\	ちっとも			
\\	全く	タンさんが帰国したことを、私は全く知らなかった 
\\	この二つは全く同じような見えるが、実はこちらは偽物だのだ 
\\	上田君は勉強もしなし欠席も多い、まったく困った学生だ 
\\	全[まった]く						
\\	どんなに	「どうして連絡してくれないかったの,どんなに心配したが、わかる」 
\\	どんなに頑張ってもあの人には勝ってない 
\\	どんなに			
\\	どうしても	レポートはどうしても明日までに完成させなければならない 
\\	この問題がどうしてもわからない 
\\	どうしても			
\\	一体	あの人は一体何をしているのだろう 
\\	困った、一体どうしたらいいのだろう 
\\	一体[いったい]			
\\	たった	5000人の会場に、たった1000人しかお客さんが来なかった 
\\	インスタントラーメンは、湯を注いでたった3分でできる 
\\	あの人はたった一度い会っただけなのに、なぜか忘れられない 
\\	たった			
\\	別に	別に用はなかったが、声が聞きたくなって母に電話した 
\\	「何か意見がありますか」「いいえ、別に。。。」 
\\	別[べつ]に			
\\	ほんの	ここから隣町まで、バルでほんの5分しかかからない 
\\	あの子はまだほんの子供で、何か悪いかよくわかっていない 
\\	ずっと日照りが続いていたが、昨日ほんの少し雨が降った 
\\	ほんの			
\\	それで	今朝駅で事故があった。それで、2-3時間電車は遅れた 
\\	「きのう、夜遅く帰ったんです。暗い道を一人で歩いていると、、後ろから、変な男が。。。」「えっ、それで」 
\\	それで			
\\	それで	新しいパソコンが必要になった。それで、銀行から貯金を少しおろすこと にした 
\\	今までの薬では治らなかった。それで、新しい薬を試してみることにした 
\\	それで			
\\	そのうえ	彼女は優秀な研究者だ。そのうえ、性格もいいので、みんなから尊敬され ている 
\\	かぜをひき、そのうえ、お腹も壊して、結局試験を受けられなかった 
\\	そのうえ			
\\	また	彼は銀行員で、また、有名な作家でもある 
\\	お酒は、飲み方によって、健康のためにもなり、また害にもなる 
\\	正月は、多くの人がうちでお祝いをする、また、最近では旅行先や海外で 過ごす人も増えている 
\\	また			
\\	または	この書類は、黒または青のペンで書くこと 
\\	試験に欠席した人は、追試験を受けるか、またはレポートを出してくださ い 
\\	または			
\\	追試験:ついしけん
\\	それとも	コーヒーにしますか、それとも紅茶にしますか 
\\	卒業後は国に帰るか、それとも日本で就職するか、迷ってる 
\\	それとも			
\\	つまり	彼は、父の姉の息子、つまり私のいとこにあたる 
\\	「この仕事が、知識と経験が必要だと思いますが、私にありません」「つ まり、あなたは無理だということですか」 
\\	つまり			
\\	必ず	この書類は明日必ず出だしてください 
\\	「明日のパーティーにぜひきて下さい」「わあ、嬉しい、必ず行きます」 
\\	私は毎朝必ず牛乳を飲むことにしています 
\\	必[かなら]ず			
\\	本当に	優勝できて、本当にうれしい 
\\	私は本当に悪いことはしていないんです、信じてください 
\\	行くと返事をしたが、本当に行きたくない 
\\	本当[ほんとう]に						
\\	絶対	今年は絶対合格する 
\\	明日は大事な試験だから、絶対遅刻できない 
\\	絶対[ぜったい]						
\\	転ぶ	階段で転んで怪我をした	転[ころ]ぶ			転ぶ:ころぶ
\\	見送る	国に帰る友達を空港まで見送った。 
\\	見送[みおく]る			見送る:みおくる
\\	植える	近所の公園にはいろいろな花が植えてあります 
\\	植[う]える			植える:うえる
\\	やり直す	実験がうまく行かなかったので、やり直した 
\\	やり 直[なお]す			やり直す:やりなおす
\\	もう一度やる			
\\	恐ろしい	私は高い所が恐ろしい。 
\\	恐ろしい事件が次々と起こした。 
\\	恐[おそ]ろしい			恐ろしい:おそろしい
\\	恐 
\\	訓: おそ.れる おそ.る おそ.ろしい こわ.い こわ.がる 音: キョウ 
\\	心 
\\	建設	駅の前に建設中のビルは来年に完成するそうだ 
\\	建設[けんせつ]			建設:けんせつ
\\	〜 会社 
\\	むっと	電車の中で携帯電話で話している人に注意したらむっとされた。 
\\	むっと			むっと:
\\	さっと	東京で怪光がさっと現われる 
\\	さっと			颯と:さっと
\\	怪光:かいこう
\\	さっさと	さっさと元に戻しなさい。 
\\	小沢さんは毎日5時になると、さっさと帰る 
\\	ヤンさんは用事をさっさと片付けた 
\\	さっさと			さっさと:
\\	無責任	無責任なことをして、周囲に迷惑をかけてしまった。 
\\	無[む] 責任[せきにん]			無責任:むせきにん
\\	な, 
\\	勝手	私が何を飲むか、勝手に決めないで欲しい。	勝手[かって]			勝手:かって
\\	な, 
\\	考え直す	あの人と結婚は、考え直したほうがいい。 
\\	考え直[かんがえなお]す			考え直す:かんがえなおす
\\	暗記	来週の試験までに、この本の内容全部暗記しなければならない。 
\\	暗記[あんき]			暗記:あんき
\\	アクセス	詳しい情報は、下記のURLにアクセスしてください。 
\\	アクセス			アクセス:
\\	シンプル	あの人はいつも、シンプルなデザインの服を着ている。 
\\	シンプル			
\\	実家	妻は今、実家に帰っている。 
\\	実家[じっか]			実家:じっか
\\	乱暴	そんな荒い運転をしていると、いつか事故を起こすよ。 
\\	この運転手は運転が乱暴だ。 
\\	乱暴[らんぼう]			乱暴:らんぼう
\\	な, 
\\	荒い:あらい
\\	じっくり	大学卒業後どうするかは、じっくりと考えてから決めたい	じっくり			じっくり:
\\	と 
\\	〜(と) 考 える			
\\	集団	狭い道で中学生たちが集団になって歩いているので通れない	集団[しゅうだん]			集団:しゅうだん
\\	集団になる:			
\\	老いる	人間は老いると病気になりやすい	老[お]いる			老いる:おいる
\\	(なる) 
\\	て		
\\	老 
\\	訓: お.いる ふ.ける 音: ロウ 
\\	耂 匕 
\\	カット	時間がなくなってきたので、詳しい説明はカットします	カット			カット:
\\	真剣	二人は結婚するつもりで真剣に付き合っている	真剣[しんけん]			真剣:しんけん
\\	な 
\\	真剣に
\\	損をする	株の値段が下がって損をした	損[そん]をする			損をする:
\\	株:かぶ
\\	損:そん
\\	な, 
\\	損 
\\	訓: そこ.なう そこな.う -そこ.なう そこ.ねる -そこ.ねる 音: ソン 
\\	扌 
\\	員 
\\	岸	大西洋岸から太平洋岸まで 
\\	船を岸から出す 
\\	岸[きし]			岸:きし
\\	岸 
\\	訓: きし 音: ガン 
\\	山 
\\	肌	紙の肌 
\\	日焼けをしすぎて、肌が真っ赤になってしまった。 
\\	肌[はだ]			肌:はだ
\\	燃料	燃料がなくなったら、機会は止まる	燃料[ねんりょう]			燃料:ねんりょう
\\	賢い	この犬はとても賢い ずる賢い者を捕まえるにはその倍の頭が要る 
\\	賢[かしこ]い			賢い:かしこい
\\	ずる賢い:ずるがしこい
\\	剃る	僕は毎朝ひげを剃る。 
\\	剃刀で髭を剃る	剃[そ]る			剃る:そる
\\	剃刀:かみそり
\\	髭:ひげ
\\	剃 
\\	訓: まい そ.る す.る 音: テイ 
\\	弟 
\\	刂 
\\	髭 
\\	訓: ひげ くちひげ 音: シ 
\\	髟 
\\	此 
\\	たっぷり	明日は休日だからたっぷり寝よう	たっぷり			たっぷり:
\\	な, 
\\	規模	大規模な工事 
\\	規模[きぼ]			規模:きぼ
\\	サイン入り	有名な野球選手のサイン入りのボールをもらった。 
\\	サイン 入[い]り			サイン入り:サインいり
\\	の 
\\	引き返す	彼は途中で引き返した 
\\	忘れ物をして途中で家に引き返した 
\\	引き返[ひきかえ]す			引き返す:ひきかえす
\end{CJK}
\end{document}