\documentclass[8pt]{extreport} 
\usepackage{hyperref}
\usepackage{CJKutf8}
\begin{document}
\begin{CJK}{UTF8}{min}
\\	高慢	こうまん	
\\	偏見	へんけん	
\\	嬢	じょう	
\\	▲彼がリー嬢に結婚を申し込んだところ、彼女は承諾した。 
\\	▲彼はブラウン嬢との婚約を発表した。 
\\	▲田中嬢は長年アメリカにいたので英語を話すのが非常にうまい。 
\\	令夫人	れいふじん	
\\	幸福	こうふく	
\\	▲彼は幸福だと思います。 
\\	▲彼は幸福だ。一つには昇進したし、また一つには結婚したばかりだから。 
\\	▲彼はとても幸福そうだ。 
\\	与える	あたえる	
\\	"▲人は天から与えられるのでなければ、なにもうけることはできません。 
\\	▲指導部の交代は、国際政治経済に重要な影響を与える。 
\\	▲現在はTETRAの操作的な意味を 「すべての消費可能なリソースを消費する」と与えている。 
\\	紳士	しんし	
\\	▲彼が話をしている紳士は誰ですか。 
\\	▲彼こそ本物の紳士だ。 
\\	▲彼には紳士らしいところがまるでない。 
\\	自身	じしん	
\\	▲それは君自身の問題だ。 
\\	▲それは女王ご自身から私に与えられた。 
\\	▲それは僕自身でやりたい。 
\\	述べる	のべる	
\\	▲彼女は理由を簡単に述べた。 
\\	▲彼女は友人の結婚披露宴ですばらしい挨拶を述べた。 
\\	▲彼女は彼の有益な助言に対して、感謝の言葉を述べた。 
\\	舞踏会	ぶとうかい	
\\	▲彼は舞踏会を催した。 
\\	▲先週、私達は舞踏会を開きました。 
\\	卿	きょう	
\\	▲どうぞ、ご自分のことをお話下さい、アンソニー卿。 
\\	▲ははあ、それではこの方がエブァンス卿夫人なのだ。 
\\	▲私はアンソニー卿のすばらしい話に夢中になっていました。 
\\	驚く	おどろく	
\\	▲私は彼女がそんなことをしたので驚いている。 
\\	▲私は彼を見て驚いた。 
\\	▲私は物価がとても上がったと聞いて驚いている。 
\\	決心	けっしん	
\\	▲あなたの決心を賞賛しないようなものはほとんどいない。 
\\	▲あなたは決心しなければならない。 
\\	▲お前は教員になる決心をしたのか。 
\\	章	しょう	
\\	▲先生は私に一番長い章を割り当てた。 
\\	▲次の章をよく予習しておきなさい。 
\\	▲今のところ何とか最初の3章を読み終えた。 
\\	馬車	ばしゃ	
\\	▲むこうをすてきな馬車が行くよ。 
\\	▲凶漢が王の馬車に爆弾を投げつけた。 
\\	▲自動車の出現とともに馬車は徐々に姿を消していった。 
\\	話題	わだい	
\\	▲彼は話題を変えた。 
\\	▲彼女の自慢話はもううんざりだわ。他に話題はないのかしら。 
\\	▲彼は話題の豊富な人だ。 
\\	踊る	おどる	
\\	▲音楽に合わせて踊っていた。 
\\	▲歌ったり踊ったりしましょう。 
\\	▲一緒に踊りませんか。 
\\	事実	じじつ	
\\	▲この本は事実を扱っている。 
\\	▲この話は事実に基づいている。 
\\	▲これは事実であって作り話ではありません。 
\\	大佐	たいさ	
\\	▲彼は2年前に、大佐の階級に昇進した。 
\\	▲彼は大佐に昇進した。 
\\	想像	そうぞう	
\\	▲タイムマシーンを持っていると想像して見てください。 
\\	▲テレビがなかったら生活がどんな物になるか想像できますか。 
\\	▲そんな生活は想像できません。 
\\	態度	たいど	
\\	▲彼の態度に問題はない。 
\\	▲彼の態度は180度変わった。 
\\	▲彼の態度には銀行家らしいところが全くない。 
\\	加わる	くわわる	
\\	▲これがごく最近私の蔵書に加わった本です。 
\\	▲ジムを説得して仲間に加わらせようとした。 
\\	▲そういうわけで、彼女は彼らに加わらなかったのです。 
\\	去る	さる	
\\	▲彼女はあたふたとここを去った。 
\\	▲冬は去りました。 
\\	▲彼らは次の日、日本を去る事になっていた。 
\\	スト	スト	
\\	▲郵便がストでおくれている。 
\\	▲明日は鉄道がストですよ。 
\\	▲通勤客はストで足を奪われた。 
\\	抱く	いだく	
\\	▲その子供は母親の腕の中に抱かれて安心した。 
\\	▲労働は経済面の利益、つまり給料を選るのに必要なだけでなく、社会的及び心理的欲求、つまり、社会の利益のために何かをしているという感じを抱くためにも、必要なのである。 
\\	▲母親は赤ん坊を左胸に抱く傾向がある。 
\\	度々	たびたび	
\\	▲彼はたびたび日本の両親に手紙を書きます。 
\\	▲彼はたびたび学校に遅刻する。 
\\	▲彼は度々遅れてきます。 
\\	士官	しかん	
\\	▲彼は海軍士官の任務を受けた。 
\\	▲彼は陸軍士官だ。 
\\	賞賛	しょうさん	
\\	▲彼女の勤勉はまったく賞賛に値する。 
\\	▲彼を知っているものは皆彼を称賛した。 
\\	▲彼女の今度の小説は多いに賞賛されている。 
\\	望む	のぞむ	
\\	▲タケシは英語の勉強のためにロンドンに行くことを望んでいる。 
\\	▲ただ一つ望んだのは、できるだけ早くそこに着くことだった。 
\\	▲たとえどこへ行こうと、努力なしでは成功は望めません。 
\\	期待	きたい	
\\	▲要するに、彼の新しい小説は期待はずれのつまらない作品と言える。 
\\	▲僕らは、彼が土曜の午後姿を見せるのを期待している。 
\\	▲編集者はそれらがタイプされていることを期待するだろう。 
\\	型	がた	
\\	▲暖房と同様に、冷房時の消費電力量を測定し、旧型と省エネ型(2001年製)のエアコンを比較しました。 
\\	▲ひとつのヒントに目を輝かせた女の子は、可愛らしい猫型のシャープペンシルをノートに走らせていく。 
\\	▲コンパクト型にはおしゃれなデザインが多いですね。 
\\	招待	しょうたい	
\\	▲竹内だけは招待に応じなかった。 
\\	▲彼は、たぶん、招待してくれるように仕向けているでしょう。 
\\	▲来たい人は誰でも招待してよろしい。 
\\	財産	ざいさん	
\\	▲父の残した財産で彼は楽に暮らしていける。 
\\	▲彼女は彼の人柄に惹かれているのであって、彼の財産に惹かれているのではない。 
\\	▲彼女は莫大な財産を独占している。 
\\	身	み	
\\	▲冷たい風が彼のコートを通して身にしみた。 
\\	▲まあちょっと私の身になってくださいよ。 
\\	▲彼の美貌と礼儀正しい身のこなしに、私はだまされた。 
\\	兄	あに	
\\	▲彼は兄に劣らず利口だ。 
\\	▲彼は兄に劣らず頭がいい。 
\\	▲彼は兄に追いつこうと走った。 
\\	自尊心	じそんしん	
\\	▲彼女を支えているのは自尊心です。 
\\	▲彼は自尊心を犠牲にしてまでそれを得ようとした。 
\\	▲彼は自尊心が強すぎて、他人にものを尋ねることができない。 
\\	祖母	おば	
\\	▲私の祖母ははたを織るのが好きだ。 
\\	▲私の祖母は健康で、一人暮らしをしている。 
\\	▲私の祖母はオートバイに乗れる。まして自転車はなおさらだ。 
\\	直ぐ様	すぐさま	
\\	▲子供達は新しい環境にすぐさま適応する。 
\\	▲私はすぐさま叔母の見舞いに昨日病院に行った。 
\\	▲ベルが鳴るとジョンはすぐさま飛び起きた。 
\\	に因って	によって	
\\	▲この小説は米国の作家によって書かれた。 
\\	▲この紹介状は叔父によってかかれた。 
\\	▲この新しい政策によって大規模に職業が生まれることが期待される。 
\\	事情	じじょう	
\\	▲事情があって私たちは会うのを取り止めねばならなかった。 
\\	▲事情があって私には、それ以上は言えません。 
\\	▲私は事情があってお手伝いできませんでした。 
\\	晩餐	ばんさん	
\\	▲大抵の晩餐は11時頃に終わります。 
\\	▲彼女は私たちを晩餐に招いた。 
\\	▲晩餐に送れるのは礼儀正しくない。 
\\	許す	ゆるす	
\\	のサンプルを貯える事は許されるべきではない。 
\\	▲彼女が笑ったのはわたしを許したしるしだ。 
\\	▲父は私が泳ぎに行くのを許してくれた。 
\\	微笑	びしょう	
\\	▲彼は微笑で悲しみを隠した。 
\\	▲長身の男はトムを見て微笑した。 
\\	▲彼女の顔に突然微笑が浮かんだ。 
\\	相当	そうとう	
\\	▲それを発明した教授は大学から相当の対価を受ける権利がある。 
\\	▲彼らの結婚費用は相当なものだった。 
\\	▲その新しい書類システムは4000ドル相当のものだ。 
\\	語る	かたる	
\\	▲その話の最も悲しい部分はまだ語られていない。 
\\	▲ユウジは夏休みの間の彼の冒険について話を彼の友人に語った。 
\\	▲何のためらいもなく真実を語る。 
\\	心づかい	こころづかい	
\\	感謝	かんしゃ	
\\	▲引き続きのご支援を感謝いたします。 
\\	▲リッチ氏は彼女の好意に感謝した。 
\\	▲援助していただいて本当に感謝しています。 
\\	機会	きかい	
\\	▲その映画を見る機会がこれまでありませんでした。 
\\	▲その機会は十分に利用された。 
\\	▲それでは、またの機会ということに。 
\\	残念	ざんねん	
\\	▲私が早く去るので彼らは残念がった。 
\\	▲私の友人がここにいないのは残念だ。 
\\	▲残念に思います。 
\\	験	げん	
\\	やって来る	やってくる	
\\	▲私は日本に骨を埋める覚悟でやってきた。 
\\	▲私は不幸は単独にはやってこないと言うことを忘れない。 
\\	▲私は待ちに待ったが、ついにジョンがやってきた。 
\\	好意	こうい	
\\	▲彼は彼女の好意を得ようと願った。 
\\	▲彼は彼女のうわべだけの好意にだまされた。 
\\	▲彼は彼女の好意を得るため出来るだけの事をした。 
\\	細君	さいくん	
\\	▲彼は2年前に細君に死なれた。 
\\	噂	うわさ	
\\	▲噂は結局本当であることが分かった。 
\\	▲噂は結局本当だった。 
\\	▲噂は結局は真実だと分かった。 
\\	がる	がる	
\\	▲私が早く去るので彼らは残念がった。 
\\	其れでは	それでは	
\\	▲それじゃ、彼の顔が立たない。 
\\	▲いや、それじゃだめだ。逆になるようにまわしてごらん。 
\\	▲どう思うって・・・それじゃあ忌憚のない意見を言わせてもらうけどね。 
\\	為す	なす	
\\	軈て	やがて	
\\	▲彼は来る途中ですから、やがて到着するでしょう。 
\\	▲彼もやがて英語が上達するであろう。 
\\	▲彼はやがて平静に戻った。 
\\	同様	どうよう	
\\	▲1939年には、1914年と同様、世界は戦争の危機に瀕していた。 
\\	▲あなたと同様私も芸術家ではないのです。 
\\	▲あなたはこの仕事では私同様専門家ではない。 
\\	真実	しんじつ	
\\	▲彼の裁判で真実がついに明らかになった。 
\\	▲彼の言葉にひとかけらの真実もない。 
\\	▲彼の秘書は真実を知っているようである。 
\\	魅力	みりょく	
\\	▲彼の魅力はルックスではなく性格だ。 
\\	▲彼の魅力に引きつけられた。 
\\	▲彼の魅力って、包容力があるところよね。 
\\	令嬢	れいじょう	
\\	出かける	でかける	
\\	▲彼女はパーティーに出かけるために盛装した。 
\\	▲彼女はパリに旅行に出かけた。 
\\	▲彼女はちょうど買い物に出かけるところです。 
\\	求婚	きゅうこん	
\\	▲アンはヘンリーの求婚を受け入れた。 
\\	▲彼女は彼の求婚を断った。 
\\	▲彼女は彼の求婚に負けた。 
\\	フォード	フォード	
\\	▲私のおじはフォードに乗っている。 
\\	▲彼の車はフォード製です。 
\\	▲僕の錆付いたフォードが故障して、交差点を塞いでしまった。 
\\	ポンド	ポンド	
\\	▲このポンドをドルに替えて下さい。 
\\	▲この牛肉は1ポンドにつき4ドルです。 
\\	▲これらの小包の重さは平均すると2ポンドです。 
\\	名誉	めいよ	
\\	▲招待された事は大変な名誉です。 
\\	▲私達は学校の名誉のために走った。 
\\	▲私は自分の名誉がかかっていると信じた。 
\\	嬢さん	じょうさん	
\\	挨拶	あいさつ	
\\	▲彼は私が居るのに気がついていたが挨拶しなかった。 
\\	▲彼女が歓迎のあいさつを述べた。 
\\	▲クリスは2人の後をついていって、ケイトに陽気に挨拶をしました。 
\\	相続	そうぞく	
\\	▲私の60歳のおばあちゃんは莫大な財産を相続した。 
\\	▲私の60歳の叔母は莫大な財産を相続した。 
\\	▲私の叔母は莫大な財産を相続した。 
\\	結構	けっこう	
\\	▲満足はごちそうと同じ位結構なもの。 
\\	▲いや結構です。たくさんいただきました。 
\\	▲以下の空欄部分にご記入頂くだけで結構です。 
\\	職	しょく	
\\	▲彼はそういう職にうってつけの人であった。 
\\	▲彼はその会社に職を求めた。 
\\	▲彼はすぐに新しい職を辞めた。 
\\	計画	けいかく	
\\	▲彼女はやむを得ずその計画をあきらめた。 
\\	▲彼女はなんとか自分の計画のことを言わずにすませた。 
\\	▲彼女はその計画を成し遂げた。 
\\	非難	ひなん	
\\	▲彼を非難するつもりでないことを急いでつけ加えなければならない。 
\\	▲彼を非難することはできない。 
\\	▲彼を非難しても仕方のないことだ。 
\\	愛す	あいす	
\\	我慢	がまん	
\\	▲私はもはや彼の怠惰に我慢出来ない。 
\\	▲私は協力的でない人たちに我慢できない。 
\\	▲私はもはや彼のふるまいには我慢できない。 
\\	散歩	さんぽ	
\\	▲彼は長い散歩で疲れた。 
\\	▲彼は朝食前に散歩することにしている。 
\\	▲彼は朝食前に散歩した。 
\\	次のように	つぎのように	
\\	▲結論として彼は次のように言った。 
\\	▲医者は次のように指示した。この薬は毎食後服用すること、そして・・・。 
\\	▲次のように記載します。 
\\	確信	かくしん	
\\	▲私は彼の正直さを確信している。 
\\	▲私は彼の無実を確信している。 
\\	▲私は彼の成功を確信している。 
\\	迎える	むかえる	
\\	▲駅に迎えに来るはずの弟のすがたが見えなかった。 
\\	▲群集は勝利者を歓呼して迎えた。 
\\	▲彼らは歓呼して彼を指揮官に迎えた。 
\\	生ける	いける	
\\	▲彼女は花をきれいに生けた。 
\\	▲花瓶には美しいバラの花が生けている。 
\\	▲彼女は花を生けるのが好きです。 
\\	自惚れる	うぬぼれる	
\\	▲人気がでて彼女はうぬぼれた。 
\\	▲褒めると彼女はうぬぼれるよ。 
\\	▲彼女は美人だとうぬぼれている。 
\\	快い	こころよい	
\\	▲私は昨夜快く眠った。 
\\	▲雄弁なその学者は快く討論に参加してくれた。 
\\	▲有名なその外交官は快く委員会に参加してくれた。 
\\	動機	どうき	
\\	▲爆薬製造犯はしばしば世間の注目を集めたいという動機を持っている。 
\\	▲彼がその家に放火した動機は何であったのか。 
\\	▲彼の行為は崇高な理想が動機になっていた。 
\\	教養	きょうよう	
\\	▲彼はいわゆる教養人だ。 
\\	▲全ての教養あるアメリカ人は最初かまたは最後にヨーロッパに行く。 
\\	▲彼は言葉の真の意味において教養のある人だった。 
\\	立派	りっぱ	
\\	▲市長は市政を立派に行った。 
\\	▲私たちはみな彼を立派な学者だと思った。 
\\	▲子供が健康だということは母親の育て方が立派であるということを物語る。 
\\	要求	ようきゅう	
\\	▲それは理論を要求することにしたがっている。 
\\	▲それはほんの小さな家にすぎないが私の要求を完全に満たしている。 
\\	▲それは論理が要求することに従っている。 
\\	評価	ひょうか	
\\	▲誰もが彼女の仕事を高く評価している。 
\\	▲鳥はその鳴き声で分かるように、話し方によって人間は評価される。 
\\	▲当事者の君は彼を結構評価しているけど、傍目八目とはいえ、私の厳しい見方があたっているかもね。 
\\	一体	いったい	
\\	▲いったい何のためにそれをメニューに載せているんだ。 
\\	▲彼は取りあえず借金は返済したらしいが、一体あんな大金を誰が都合したのだろうか。 
\\	▲いったい何がしたいのか。 
\\	告げる	つげる	
\\	▲誠に、誠に、あなたに告げます。 
\\	▲誰が君にそのニュースを告げたのか。 
\\	▲真っ赤な夕焼けは明日の晴天を告げた。 
\\	尊敬	そんけい	
\\	▲彼の勇敢さは尊敬に値する。 
\\	▲彼はアメリカの科学者であるアインシュタインを尊敬している。 
\\	▲彼の親切な行為により彼は人々の尊敬を得た。 
\\	懸念	けねん	
\\	▲私はハリウッド映画の商業主義に懸念を抱かずにはいられない。 
\\	▲世界の航空産業は墜落事故についての懸念を募らせている。 
\\	▲非行、学校嫌い、人格の未発達などの問題が現在以上に顕在化してくることが懸念される。 
\\	牧師館	ぼくしかん	
\\	片刃	かたば	
\\	▲鋭い舌は絶えず用いることによって、ますます鋭利になる片刃の道具である。 
\\	聞かす	きかす	
\\	▲お祖母さんの昔話は耳にたこができるほど聞かされた。 
\\	▲そのことについてできるだけ話を聞かしてください。 
\\	▲彼の死を聞かされて深く心を痛めた。 
\\	優れる	すぐれる	
\\	▲彼は優れた調整役で、ずっとそのように見られていた。 
\\	▲彼は優れた記憶力を発揮してわたしの質問に答えた。 
\\	▲彼は優れた記憶力の持ち主だ。 
\\	舞う	まう	
\\	▲木の葉が庭でくるくる舞っていた。 
\\	▲ヒラヒラと桜の花が舞っている。もうお花見の季節も終わりだな。 
\\	▲風でほこりが舞っていた。 
\\	不幸	ふこう	
\\	▲不幸はあまりにもひどく、読者は涙を抑えることが出来なかった。 
\\	▲不幸にも彼女は5ドルしか持ち合わせがなかった。 
\\	▲不幸の裏には幸いあり。 
\\	半分	はんぶん	
\\	▲形勢は2対1で不利、勝ち目は半分。 
\\	▲今までのところ半分だけ終わりました。 
\\	▲仕事の半分はかたづいた。 
\\	牧師	ぼくし	
\\	▲ボブは牧師になりました。 
\\	▲その牧師は貧乏人のために懸命に働いた。 
\\	▲マーテイン・ルーサー・キング・ジュニアは、黒人牧師を父として生まれた。 
\\	観察	かんさつ	
\\	▲野鳥観察はすばらしい趣味です。 
\\	▲野鳥を観察するのはとても面白い。 
\\	▲野鳥を観察しているところです。 
\\	軽べつ	けいべつ	
\\	ご自身	ごじしん	
\\	些とも	ちっとも	
\\	▲犬なんかちっとも恐くない。 
\\	▲結果にちっとも満足しなかった。 
\\	▲今はちっともうれしくない。 
\\	仄めかす	ほのめかす	
\\	▲首相は早い時期に選挙を行う可能性があるとほのめかした。 
\\	▲彼らのうち誰一人、彼女がほのめかしたことを理解できなかった。 
\\	▲私には彼が何をほのめかしているのかわからなかった。 
\\	一同	いちどう	
\\	▲出演者一同は観客からたちあがっての大かっさいを受けた。 
\\	▲司祭は会衆一同を祝福した。 
\\	▲彼の冗談が一同をどっと笑わせた。 
\\	上品	じょうひん	
\\	▲彼女は美しいが、さらによいことには非常に上品だ。 
\\	▲彼女は非常に上品な女性です。 
\\	▲シックで上品な制服と歴史ある静かな雰囲気が特徴の女子高なのだという。 
\\	不可能	ふかのう	
\\	▲美しいということは、無視することがほとんど不可能な推薦状のようなものである。 
\\	▲彼女は彼の言うことを理解するのが不可能だとわかった。 
\\	▲彼女はもっとのんびりした生活を望んだが、あの状況ではそれは不可能であった。 
\\	交際	こうさい	
\\	▲妹は彼らとの交際を望んでいない。 
\\	▲彼女は私の忠告を無視して彼との交際を続けた。 
\\	▲彼女は一度も女の社会で交際したことがなかった。 
\\	客間	きゃくま	
\\	▲この書斎は客間兼用だ。 
\\	▲夜遅くに、若い男は少女とふたりきりで長い間客間に座っていた。 
\\	少なくとも	すくなくとも	
\\	▲私は少なくとも一週間に一回図書館に行きます。 
\\	▲私は少なくとも月曜日までに宿題を終わらさなければならない。 
\\	▲私はこのペンに少なくとも100ドルは払った。 
\\	忠告	ちゅうこく	
\\	▲彼は我々の忠告に耳を貸さなかった。 
\\	▲彼は我々にそれをしないように忠告した。 
\\	▲彼は我々の忠告を聞きいれなかった。 
\\	権利	けんり	
\\	▲その黒人たちは長いこと権利を奪われてきた。 
\\	▲その法律は我々に年金を受け取る権利を与えている。 
\\	▲だれでも自由を享受する権利を持つ。まして人生を享受する権利はなおさらだ。 
\\	縁組み	えんぐみ	
\\	限	きり	
\\	至る	いたる	
\\	▲今のところ、全体の意見の一致には至っていない。 
\\	▲彼らは、カルカッタからニューヨーク市に至るまで、世界中に支部を持っている。 
\\	▲国勢調査の結果は左派に自己の政策が誤っていたことを信じさせるにいたった。 
\\	然も	しかも	
\\	▲我々は彼を助けなければならない、しかも直ちに。 
\\	▲君は決心しなければならない。しかも今すぐにだ。 
\\	▲家はかなりにみえたし、しかも値段が手頃だった。 
\\	照る	てる	
\\	▲日の照るうちに草を干せ。 
\\	▲降っても照っても明日は出発します。 
\\	▲日の照るうちに干し草を作れ。 
\\	党	とう	
\\	▲インフレ問題が党を分裂させた。 
\\	▲その党は急激にのびた。 
\\	▲その党は主に働く女性から成り立っている。 
\\	バレー	バレー	
\\	▲彼はバレーの選手です。 
\\	▲バレー劇場は私が体の動きを研究することができる場所でなのです。 
\\	▲私はバレーの踊りよりも人間が体を動かす方法に見せられているのです。 
\\	振る舞う	ふるまう	
\\	▲学生であるならそのように振舞いなさい。 
\\	▲彼は陽気に振る舞おうとしたが、私達はみんな彼がどれほど悲しい思いをしているかわかっていた。 
\\	▲彼女がこんな風に振る舞うとは奇妙だ。 
\\	寧ろ	むしろ	
\\	▲問題は費用よりもむしろ時間だ。 
\\	▲問題は費用よりむしろ時間だ。 
\\	▲問題は、その本質よりもむしろ外観である。 
\\	滅多に	めったに	
\\	▲彼はめったにテレビを見ない。 
\\	▲彼はめったにそこへ行かなかった。 
\\	▲彼はめったに家族と食事をしたことがない。 
\\	最早	もはや	
\\	▲彼はもはや子供ではない。 
\\	▲彼はもはや思い出の人となってしまった。 
\\	▲彼はもはや仕事がない。 
\\	了解	りょうかい	
\\	▲二人の間には暗黙の了解があったに違いない。 
\\	▲了解しました。 
\\	▲我々には互いに支持しあおうという暗黙の了解があった。 
\\	以来	いらい	
\\	▲彼は約10年まえにイタリアへ行き、それ以来ずっとそこに住んでいる。 
\\	▲彼は先月以来病気だそうです。 
\\	▲彼は妻が死んで以来酒を飲む癖がついた。 
\\	偶然	ぐうぜん	
\\	▲君はわざと窓をこわしたのか、それとも偶然にか。 
\\	▲古い住所録で偶然、彼の電話番号を見せた。 
\\	▲偶然に頼るな。 
\\	到着	とうちゃく	
\\	▲彼がバス停に到着したとき、バスはもう見えなかった。 
\\	▲彼が最初に到着した者だった。 
\\	▲彼がいつ到着するかは明確には分からない。 
\\	好ましい	このましい	
\\	▲その志望者は試験官に好ましい印象を与えた。 
\\	▲その新作映画の前評判は非常に好ましいものである。 
\\	▲就職の面接には派手なネクタイよりも地味なネクタイの方が好ましい。 
\\	愉快	ゆかい	
\\	▲控え目にいっても、彼は愉快なやつではない。 
\\	▲私たちのクラスではいつも愉快なことが起こっている。 
\\	▲人が多ければ多いほど愉快だ。 
\\	承諾	しょうだく	
\\	▲私は彼らの沈黙を承諾の意にとった。 
\\	▲私は彼の要求を承諾するだろう。 
\\	▲承諾するのが一番だ。 
\\	拒絶	きょぜつ	
\\	▲彼は彼女に結婚してくれと言ったが、彼女は拒絶した。 
\\	▲彼は断固として拒絶した。 
\\	▲彼女が彼の申し込みを拒絶したはずがない。 
\\	無関心	むかんしん	
\\	▲奥さんの問題にどうしてそう無関心でいられるのか。 
\\	▲我が国の青年男女は政治に無関心だ。 
\\	▲君は周囲の人に無関心すぎるよ。 
\\	立ち去る	たちさる	
\\	▲駅員に連絡したが、男はその場を立ち去り、改札口を出て行った。 
\\	▲彼は逃げるように野菜売り場を立ち去って、精肉コーナーへ。 
\\	▲話が退屈だったので彼らは一人また一人と立ち去って行った。 
\\	苦しみ	くるしみ	
\\	▲苦しみなくして利益なし。 
\\	▲私はあなたが味わったのと同じ苦しみに悩んでいる。 
\\	▲私は長い間臥薪嘗胆の苦しみをなめた。 
\\	誤解	ごかい	
\\	▲いずれにせよ、その隔たりは現実のものであり、実際に役に立つものではあるのだが、同時にそれは誤解を招くおそれのあるものでもある。 
\\	▲この社会が若さということにそのような感情的特権を付与するというのは誤解を引き起こしやすいと思う。 
\\	▲この問題はお互いの誤解から生じたものだ。 
\\	説得	せっとく	
\\	▲彼を説得しようとしても無駄なことだ。 
\\	▲彼を説得しようとしても無駄だ。 
\\	▲彼を説得して旅行を中止させるのに苦労しました。 
\\	賢明	けんめい	
\\	▲賢明で助けになってくれる友人ほど貴重な価値をもつ宝はほとんどありません。 
\\	▲健康を犠牲にしてまで働きすぎるのは賢明ではないだろう。 
\\	▲賢明な経営者はコストをぎりぎりまでさげる方法を知っていますよ。 
\\	顔つき	かおつき	
\\	▲その悪い知らせを話した時、ロイはまじめな顔つきをした。 
\\	▲その老人は堂々とした顔つきをしていた。 
\\	▲マーシャにその話をしたとき、いかめしい顔つきをした。 
\\	如何なる	いかなる	
\\	▲我々はいかなる犠牲をはらっても目標を達成せねばならぬ。 
\\	▲言語は、いかなる言葉も必ず個人の心の創作として飛び出してくるという点で、ユニークである。 
\\	▲我々は、自分がしていることが何であろうとそれに目的を与えてくれるような目標あるいは指導的な思想を、今までのいかなる時代にまして、必要としている。 
\\	頂ける	いただける	
\\	▲アレン夫人と話している、あの若いご婦人に紹介していただけないでしょうか。 
\\	▲いつこれを返していただけますか。 
\\	▲いつ降りたらいいのか教えていただけますか。 
\\	一層	いっそう	
\\	▲彼は事態を一層悪くした。 
\\	▲ケンは1人息子なので、父親は一層可愛がった。 
\\	▲彼はその薬を飲んだためにかえっていっそう病気が悪くなった。 
\\	撃つ	うつ	
\\	▲警官がヤンキーに足を撃たれた。 
\\	▲警察は空砲を撃った。 
\\	▲君の予言どおりやつはうってきた。 
\\	同じく	おなじく	
\\	▲ぼくは君と同じく若くない。 
\\	▲金持ちにも貧乏人と同じく悩みがある。 
\\	▲君と同じく彼は愚かでない。 
\\	お嬢さま	おじょうさま	
\\	兼ねる	かねる	
\\	▲先約がありますので残念ながらご招待に応じかねます。 
\\	▲この映画は教育と娯楽をかねている。 
\\	▲どれくらいの時間がかかるのか、正確なところは申し上げかねます。 
\\	示す	しめす	
\\	▲友達が示した方向へ行った。 
\\	▲羅針盤の針は常に北を示す。 
\\	▲良い医者は患者に同情を示す。 
\\	其れ程	それほど	
\\	▲そのニュースはそれほど重要でない。 
\\	▲その映画は笑えるものだったが、見ていてそれほど感動するようなものではなかった。 
\\	▲その仕事を終えるのにはそれほど時間がかからないでしょう。 
\\	大層	たいそう	
\\	▲その野球チームの監督はたいそう厳しくて、選手を思い通りに動かすのが上手だ。 
\\	▲その娘はたいそう美しいので非常に冷静な男でさえも彼女に惹かれる。 
\\	▲たいそうがっかりしたことには、彼女は試験に落ちた。 
\\	確か	たしか	
\\	▲彼は馬の善し悪しを見る目は確かだ。 
\\	▲うん。確か使ってない優待券もどこかにあったはず。 
\\	▲彼が東京へ行ったことはたしかだと思う。 
\\	ディア	ディア	
\\	慰める	なぐさめる	
\\	▲彼の優しい言葉は、私をとても慰めてくれた。 
\\	▲多くの激励の手紙が私の心を慰めてくれました。 
\\	▲彼はもっと悪いことになっていたかもしれないのだと思って自分を慰めた。 
\\	微笑む	ほほえむ	
\\	▲ナンシーはうれしそうに微笑んだ。 
\\	▲なんと言ってよいかわからず、彼女はただほほえむだけだった。 
\\	▲何といって良いか分からず、彼女はただ微笑むだけだった。 
\\	利	り	
\\	▲他の物が漁夫の利を占める。 
\\	▲本命と対抗馬が競り合ったお陰で、ダークホースの彼が漁夫の利を得たってわけだ。 
\\	▲君は漁夫の利を占めようとしているのか。 
\\	別れる	わかれる	
\\	▲私は橋の上で彼と別れた。 
\\	▲私はずっと前に彼女と別れた。 
\\	▲私は主人と別れる事にしました。 
\\	沸く	わく	
\\	▲彼女はお茶を入れる前にお湯が沸くまで待った。 
\\	▲彼のホームランは観衆を沸かせた。 
\\	▲お風呂沸いたわよ。 
\\	主張	しゅちょう	
\\	▲彼は、その莫大な財産は自分が自由に出来るものだと主張した。 
\\	▲彼の無罪を最後まで主張する。 
\\	▲彼の日々の振る舞いは彼の主義主張と合致していない。 
\\	人好き	ひとずき	
\\	人柄	ひとがら	
\\	▲彼はざっくばらんな人柄なので話しやすい。 
\\	▲彼の財産でなく、彼の人柄のために私は彼が好きだ。 
\\	▲人柄のほうが財産よりも重要である。 
\\	人物	じんぶつ	
\\	▲仕立て屋は人物を作る。 
\\	▲私はこの仕事をするのにうってつけの人物を知っています。 
\\	▲社長は新しいセールスマネージャーを入れるため、引き抜き専門の人物を雇った。 
\\	快活	かいかつ	
\\	▲何ごとが起ころうと、いつも快活でいなければならない。 
\\	▲健康と快活さは美を生む。 
\\	▲彼は明朗快活な青年だ。 
\\	怒り	いかり	
\\	▲彼の怒りを恐れて、誰も彼を止めなかった。 
\\	▲彼の怒りは欲求不満から生じた。 
\\	▲彼の頭がさっと怒りの表情に変わるのを見た。 
\\	恋しい	こいしい	
\\	▲あなたはアメリカで日本料理が恋しくなるでしょう。 
\\	▲毎日外食ばかりで済ませていると、家族料理が恋しくなります。 
\\	▲兵士たちは故国が恋しくてならなかった。 
\\	教会	きょうかい	
\\	▲通りの向こう側に教会がある。 
\\	▲中世の教会は肉体をさげすみ霊をたたえた。 
\\	▲昔日曜日には教会に行ったものだ。 
\\	望ましい	のぞましい	
\\	▲原子力は平和目的に利用されることが望ましい。 
\\	▲誠実であって貧しいのは不正な手段で得られた富より望ましい。 
\\	▲誠実にして貧しいのは不正な手段で得た富より望ましい。 
\\	程度	ていど	
\\	▲その程度のことは百も承知だったから、私は少しも驚かなかった。 
\\	▲その程度の金額では我々がどうかなるということはない。 
\\	▲その程度の中傷はほうっておきなさい。 
\\	義務	ぎむ	
\\	▲私達は税金を払う義務がある。 
\\	▲私達は法律に従う義務がある。 
\\	▲私は彼に返事をする義務があった。 
\\	翌日	よくじつ	
\\	▲私は彼に翌日そこへ行くかどうかと聞いた。 
\\	▲特別講義は悪天候のため翌日に延期された。 
\\	▲彼は、翌日ほとんどホテルにいて、友人や支援者と話をした。 
\\	資産	しさん	
\\	▲彼女はちょっとした資産を持っている。 
\\	▲彼女はほんのわずかな資産しか持っていなかった。 
\\	▲彼は資産のおかげでそのクラブの会員になれた。 
\\	領地	りょうち	
\\	現れる	あらわれる	
\\	▲彼女がどちらかの道を進むべきか決められないでいると、偶然彼が現れた。 
\\	▲内面の欠点は必ず表に現れる。 
\\	▲富山湾にはときどき蜃気楼が現われます。 
\\	合せる	あわせる	
\\	方々	かたがた	
\\	▲あの人たちが昨日あなたが会った方々ですか。 
\\	▲わからなかったら方々の人に聞いた方がいいよ。 
\\	▲私どもの新製品は、耳にされた全ての方々から、たいへん注目されています。 
\\	夫れ夫れ	それぞれ	
\\	▲これら最後の英語はそれぞれの話し手の必要に応じて独自の路線で発展しつづけるだろう。 
\\	▲その3人の女の子は、それぞれ賞を獲得した。 
\\	▲これらの語を各々使って文を作りなさい。 
\\	絶えず	たえず	
\\	▲彼は父親を絶えず恐れている。 
\\	▲彼らは顧客との関係を向上させようと絶えず努力している。 
\\	▲彼は絶えず他人の悪口ばかり言っている。 
\\	直ちに	ただちに	
\\	▲その数百年後に、フォークが西洋のテーブルに現れることになるが、ただちに受け入れられたわけではなかった。 
\\	▲その法律は直ちに実施された。 
\\	▲その事故のニュースはラジオでただちに放送された。 
\\	捕らえる	とらえる	
\\	▲我々はその泥棒を捕らえた。 
\\	▲彼はその動物をとらえるためにわなをしかけた。 
\\	▲捕らえられるまで、鳥には鳥捕りの笛がよい音に聞こえる。 
\\	自ら	みずから	
\\	▲しかし、主人公は自らの誤りに気づいている。 
\\	▲神は自らにかたどって人を創造された。 
\\	▲子どもを甘やかして育てたことで、ステラは後に自ら災いを招くことになってしまった。 
\\	一般	いっぱん	
\\	▲太陽に浮かぶほとんどの島は地図に示されているが、海底は一般には知られていない。 
\\	▲彼は一般の読者に人気がある。 
\\	▲世間一般のホテルなみにいえば良い昼食会だった。 
\\	予期	よき	
\\	▲わたしはその事業がうまくやれると予期しております。 
\\	▲映画は、私が予期したように面白かった。 
\\	▲考えられるすべての理由の中で、彼は最も予期しなかった理由を選んだ。 
\\	今一つ	いまひとつ	
\\	受け入れる	うけいれる	
\\	▲彼が私の忠告を受け入れなかったことは間違いだと思う。 
\\	▲日本は自由国家群に受け入れられた。 
\\	▲彼の意見は受け入れられなかった。 
\\	地位	ちい	
\\	▲彼女はその地位を引き受けようとしなかった。 
\\	▲彼女がその地位に就く資格を得るのに、少なくとも2年はかかるであろう。 
\\	▲彼女がその地位にふさわしいということは、だれも疑っていない。 
\\	夫妻	ふさい	
\\	▲デイビス夫妻は、ある年のホテルのクリスマスパーティーに招待された。 
\\	▲ハートの形をした池は国王ご夫妻の自慢の種である。 
\\	▲皇太子夫妻は歓迎の群集の中を通っていかれた。 
\\	好奇心	こうきしん	
\\	▲彼女は好奇心が強かったので、その箱を開けてみた。 
\\	▲彼女は好奇心からというより見栄からテニスを始めた。 
\\	▲彼女は好奇心からそうしたのだ。 
\\	小径	こみち	
\\	弾く	ひく	
\\	▲私はピアノを弾きます。 
\\	▲私はピアノを弾かない。 
\\	▲私はピアノを弾いて2時間すごした。 
\\	打ち明ける	うちあける	
\\	▲彼女は秘密を打ち明けるのに気が進まなかった。 
\\	▲彼女は秘密を打ち明けたくてうずうずしていた。 
\\	▲彼女は彼女の心の中を友人たちに打ち明けた。 
\\	教区	きょうく	
\\	▲教区牧師は法話の時しばしば聖書に言及する。 
\\	敬意	けいい	
\\	▲彼は敬意を表して私の言う事を聞いてくれた。 
\\	▲日本人は必ずしも敬意の表れとしてお辞儀するわけではない。 
\\	▲退職する重役に敬意を表して送別会が開かれた。 
\\	有利	ゆうり	
\\	▲コンピューターが使えるのは有利である。 
\\	▲さあ、くよくよしないで。形勢は君に有利なんだから。 
\\	▲その子たちはよい教育を受けたという有利な点を持っていた。 
\\	滞在	たいざい	
\\	▲健太郎は京都の友達のところに滞在している。 
\\	▲好きなだけ滞在して下さい。 
\\	▲好きなだけ長く私たちの家に滞在しなさい。 
\\	離れる	はなれる	
\\	▲彼女に愛着があって離れられない。 
\\	▲彼女のことが頭から離れない。 
\\	▲彼女の家は道路から少し離れている。 
\\	館	かん	
\\	▲タッソーろう人形館へ行く道を教えて頂けませんか。 
\\	▲この方が、「真砂館」のおかみさんの染葉洋子さん。 
\end{CJK}
\end{document}