\documentclass[8pt]{extreport} 
\usepackage{hyperref}
\usepackage{CJKutf8}
\begin{document}
\begin{CJK}{UTF8}{min}
\\	支那の明器
\\	會津八一
\\	私ほど名実の副はない蒐集家は無い。何か余程いゝものでも沢山持つて居るやうに云ひ囃やされながら、実は是れと云ふほどのものは何も持たない。
\\	小石川に住んで居る頃に――これは十数年も前のことだが――諸国の郷土玩具を集めたことがあつた。六百種もあつたかと思ふ。しかしこれは世間の玩具通などのするやうに、いろいろの変つた物を集めて自慢をするといふのでは無く、其頃しきりに私の考へて居た原始的信仰の研究資料にと思つたのであつた。不幸にして此の玩具の大半は出版部の倉庫の中で洪水を喫つて全滅してしまつた。
\\	次に私が今現に持つて居ていくらか話の種にしてもいゝと思ふのは支那の明器、即ち古墳から発掘される土製の人形や器物の類で、私の持つて居るのは百三四十点にも及んで居る。支那では三代の昔から人の死んだ時に墓の中へ人形を入れて御伴をさせる。所謂「俑」である。人形のほかに鶏や犬や豚や馬や牛などの動物或は器物、時としては建物まで御伴させることが漢時代以後だんだん盛に行はれ、唐に至つて流行を極めた。木で造つて着物を着せたものなどもあつた筈だが、木は長い間に皆な腐つて跡方も無く消え失せるので今日に残つて居るのは極く稀に玉製のものなどもあるが、たいていは土製ばかりである。土製と云つても瓦のやうに焼いて、上から胡粉を塗つて、其上へ墨や絵の具で彩色したものもあるし、唐時代などになると三彩と云つて黄、褐、緑、或は藍色の釉薬をかけた陶製のものもある。此の明器が支那でかれこれ云はれるやうになつたのはあまり古いことではなく、何でも京漢鉄道の敷設の時に古墳を発掘した欧人の技師が初めて見つけ出して、それからだんだん北京の骨董店などに現はれることになつた。最初は殆んど市価のないものであつた。それを有名な考古学者の羅振玉氏が買ひ蒐めて後に『古明器図録』といふ図録を作つた。其頃から世界の学者や鑑賞家の注意を惹いて、今では世界の何処の博物館にも沢山に蔵されて居り、欧米人の手で編輯された図録も沢山に出て居り、従つて研究も広く行はれて居る。日本でも東京帝室博物館や、東西両京の帝国大学、東京美術学校、個人では細川侯爵、校友の反町茂作氏などがいづれも優秀なものを沢山に持つて居られる。横川博士の蒐集は近年宮内省へ献納された。美術的によく出来て居て、色彩が製作当時のまゝで、おまけに形が珍らしいものなどになると数百円から千円以上のものも稀ではない。しかし上海あたりの場末の道具屋の店さきに曝されて居るいかものには一円で二つも三つも呉れてよこすやうなものもある。つまり明器の価格はピンからキリまである。
\\	そこで、なぜ世界の隅々まで、急に此の明器をそれほど珍重するやうになつたかと云ふに、それは少しも無理も無いことで、支那の骨董品として大昔から古銅器即ち鐘鼎の類が非常に尊重されたものであるし、唐宋以後になれば支那特有の絵画も次第に発達して其遺品も今日に於ては豊富に伝へて居る。しかし唐時代以前の美術彫刻はと云へば、これまでは漢時代の画像石か六朝時代の仏像或はその附属物として沙門の像や獅子位のものであつた。ところが一度此の明器の類が続々と出土するに及んで、漢時代ではこれまでの画像石のやうに線彫りでなく、丸彫りの人形や動物、ことに嬉しいのは六朝以後唐時代に至る間の将軍、文官、美人、奴婢、家畜などの風俗的生活が吾々の眼前に見せられることになつた。即ち天地を祀る祭器としての銅器や、装身具としての玉器や、仏教の偶像だけしか無かつた支那美術の畠に、それこそ本統に人間らしい、柔らかい感じの、気のおけない人間生活の彫刻が現はれたわけである。そこで美術上からも考古学上からも、或は唯の物好きからも、欧米人などが、ことに大騒ぎするのは決して無理も無いことである。人によると墓から出たといふ事を、いつまでも気にしてゐる人があるが、千年から二千年も経つた今日に及んでまだそんな事を気にしてゐるやうでは、よくよく学問にも芸術にも因縁の無い連中と云ふよりほかは無い。又無暗に贋物を恐がる人もある。たかゞ土製の人形が、何十円何百円に売れるといふことになれば、墓を掘るまでもなく、偽物を作つて金儲けをすることを知らぬ支那人ではないから、事実贋物は随分沢山ある。支那の或る地方では一村挙つて此の贋物製造を商売にしてゐる所さへあつて、念の入つたことには一旦造り上げて彩色までしたものをわざと土中に埋め、其上から汚い水などを引懸けて、二三年目に掘り出して、いゝ加減に土を落して市へ出すといふやり方もある。また真物から型を抜き取つて、其型で偽物を作つたり、真物は真物でも素焼の所へ後から釉薬をかけるといふやゝこしい法もある。だから支那の市場には夥しい、しかも紛らはしい贋物があるのは事実である。そこであちらを旅行して、そんな現場を見せられて帰つて来た人の土産噺などを聞いて無暗に恐れをなす人のあるのも無理もないことであるが、贋物の多いのは何も明器に限つたわけでは無いし、又支那に限つたわけでも無い。何処の国でも古いものは贋物の方が多い。そこで明器買ひも頗る眉唾であるが、眉に唾ばかりつけても、わからない人には矢張りわからない。北京や上海や何処に行つたことがあつてもそれだけではわからない。支那人でもわからない人は矢張りわからない。しかしわかる人が見れば何でもなく直ぐ見分けがつく。贋物が恐いと尻込みする人は、私は美術がわかりませんと自白して居るのと同じことだから、さういふ人は手を出さぬ方がいゝであらう。
\\	贋物は支那製ばかりでは無く、独逸風の応用化学で巧に三彩の真似をしたものや、また日本製の物もある。或は遥々東京まで来てから、白粉の塗り直し黛の描き直し、着物の染め直しなどをやるのもある。又全く贋物と云ふ意識は無く、一種の尚古趣味から京都あたりの相当な陶工が自分の手腕を見せるつもりで真剣に作つたものもある。それ等も目のある人が見れば何の苦もなく見分けが附くものである。
\\	ところが私は誰も知る貧乏人であるのに今日までに、可なりの数まで集めるには随分骨が折れた。私の手まへとして一個百円前後もする物をいくつも買ふことは出来るわけがない。そこで私は月給のあまりで足りない時は窮余の一策として自分の書いた書画に値段を附けて展覧会を開いて、其収入でやうやく商人の支払を済ませたこともある。さういふ展覧会を私はこれまでに東京の銀座で一度、郷里で三度も開いた。こんな手もとで私があつめたものだから蒐集として人に誇るほどのものは何一つ無い。従つて安物づくめである。それこそゲテモノ展の観がある。しかし私は苟しくも早稲田大学で東洋美術史といふ少し私には荷物の勝つた講義を御引き受けして居る関係から、何も持たぬ、何も知らぬでは済まされないと思つて、とにかく微力の限り、むしろそれ以上を尽したものである。だから何処の役人に対しても、富豪に対しても、蒐集の貧弱を愧ぢる必要は少しも無いつもりである。明器の話は、私としては教場ですべき仕事の一つだから、ここでは先づこれ位のことで止めにする。
\\	私は最近に朝鮮の或る方面から、昔の新羅時代の古瓦を、破片混りではあるが四百個ばかり買入れた。これまで私の手もとにあつた日本や支那の古瓦二百個を加へると六百ほどになる。これも私としては東洋美術史研究の標本であつて、決して道楽三昧でやつて居るわけではないが、とにかく之れも一つの蒐集といへば蒐集であらう。
\\	底本:「日本の名随筆 別巻9 骨董」作品社
\\	1991(平成3)年11月25日第1刷発行
\\	1999(平成11)年8月25日第6刷発行
\\	底本の親本:「會津八一全集 第七巻」中央公論社
\\	1982(昭和57)年4月発行
\\	入力:門田裕志
\\	校正:仙酔ゑびす
\\	2006年11月18日作成
\\	青空文庫作成ファイル:
\\	このファイルは、インターネットの図書館、青空文庫
\\	で作られました。入力、校正、制作にあたったのは、ボランティアの皆さんです。
\end{CJK}
\end{document}