\documentclass[8pt]{extreport} 
\usepackage{hyperref}
\usepackage{CJKutf8}
\begin{document}
\begin{CJK}{UTF8}{min}
\\	読んだ	
\\	読[よ]んだ	
\\	かわいかった	
\\	かわいかった	
\\	静かだった	
\\	静[しず]かだった	
\\	学生だった	
\\	学生[がくせい]だった	
\\	読まなかった	
\\	読[よ]まなかった	
\\	かわいくなかった	
\\	かわいくなかった	
\\	静かじゃなかった	
\\	静[しず]かじゃなかった	
\\	学生じゃなかった	
\\	がくせいジャなかった	
\\	行った	
\\	行[い]った	
\\	よかった	
\\	よかった	
\\	よくなかった	
\\	よくなかった	
\\	私はトムさんがやったと思います。	
\\	私[わたし]はトムさんがやったと 思[おも]います。	
\\	晩御飯、食べた? 
\\	うん、食べた。	
\\	晩[ばん] 御飯[ごはん]、 食[た]べた?  うん、 食[た]べた。	
\\	スーさんは日本語を勉強していると言っていました。	
\\	スーさんは 日本語[にほんご]を 勉強[べんきょう]していると 言[い]っていました。	
\\	おもしろい人	
\\	おもしろい 人[ひと]	
\\	髪が長い人	
\\	髪[かみ]が 長[なが]い 人[ひと]	
\\	めがねをかけている人	
\\	めがねをかけている 人[ひと]	
\\	猫が好きな人	
\\	猫[ねこ]が 好[す]きな 人[ひと]	
\\	私はきのう宿題をしませんでした。 私はまだ宿題をしていません。	
\\	私[わたし]はきのう 宿題[しゅくだい]をしませんでした。 私[わたし]はまだ 宿題[しゅくだい]をしていません。	
\\	まだ~ていません	
\\	まだ~ていません	
\\	寒かったですから、出かけませんでした。 寒かったから、出かけませんでした。 寒かったから、出かけなかった。	
\\	寒[さむ]かったですから、 出[で]かけませんでした。 寒[さむ]かったから、 出[で]かけませんでした。 寒[さむ]かったから、 出[で]かけなかった。	
\\	エルビスプレスリのほうがフランクシナトラよりかっこいいです。	
\\	エルビスプレスリのほうがフランクシナトラよりかっこいいです。	
\\	バスと電車とどっちのほうが安いですか。	
\\	バスと 電車[でんしゃ]とどっちのほうが 安[やす]いですか。	
\\	パバロッティとカレーラスとドミンゴの中で、だれがいちばん歌が上手だと思いますか。 もちろん、パバロッティがいちばん歌が上手です。	
\\	パバロとカレーラスとドミンゴの 中[なか]で、だれがいちばん 歌[うた]が 上手[じょうず]だと 思[おも]いますか。 もちろん、パバロッティがいちばん 歌[うた]が 上手[じょうず]です。	
\\	私は黒いセーターを持っています。赤いのも持っています。	
\\	私[わたし]は 黒[くろ]いセーターを 持[も]っています。 赤[あか]いのも 持[も]っています。	
\\	これはスーさんのかばんですか。 いいえ、それはメアリーさんのです。	
\\	これはスーさんのかばんですか。 いいえ、それはメアリーさんのです。	
\\	(私は)週末にたけしさんとテニスをするつもりです。	
\\	私[わたし]は) 週末[しゅうまつ]にたけしさんとテニスをするつもりです。	
\\	山下先生はあした大学に来ないつもりです。	
\\	山下[やました] 先生[せんせい]はあした 大学[だいがく]に 来[こ]ないつもりです。	
\\	のほうが
\\	より
\\	のほうが 
\\	より
\\	と
\\	とどっちのほうが
\\	と 
\\	とどっちのほうが
\\	の中で]
\\	がいちばん
\\	の 中[なか]で] 
\\	がいちばん
\\	つもりだ	
\\	[たす]つもりだ	
\\	暖かくなる	
\\	暖[あたた]かくなる	
\\	静かになる	
\\	静[しず]かになる	
\\	会社員になる	
\\	会社[かいしゃ] 員[いん]になる	
\\	日本語の勉強が楽しくなりました。	
\\	日本語[にほんご]の 勉強[べんきょう]が 楽[たの]しくなりました。	
\\	日本語の勉強が好きになりました。	
\\	日本語[にほんご]の 勉強[べんきょう]が 好[す]きになりました。	
\\	メアリさんは前より日本語が上手になりました。	
\\	メアリさんは 前[まえ]より 日本語[にほんご]が 上手[じょうず]になりました。	
\\	何か_、だれか_、どこか_ 何_も、だれ_も、どこ_も	
\\	何[なに]か_、だれか_、どこか_ 何[なに]_も、だれ_も、どこ_も	
\\	どこかへ行きましたか。 いいえ、どこへも行きました。	
\\	どこかへ 行[い]きましたか。 いいえ、どこへも 行[い]きました。	
\\	だれかに会いましたか。 いいえ、だれにも会いませんでした。	
\\	だれかに 会[あ]いましたか。 いいえ、だれにも 会[あ]いませんでした。	
\\	何かしましたか。 いいえ、何もしませんでした。	
\\	何[なに]かしましたか。 いいえ、 何[なに]もしませんでした。	
\\	(私たち)はしでごはんを食べます。	
\\	はしでごはんを 食[た]べます。	
\\	日本語で話しましょう。	
\\	日本語[にほんご]で 話[はな]しましょう。	
\\	バスで駅まで行きました。	
\\	バスで 駅[えき]まで 行[い]きました。	
\\	テレビで映画を見ました。	
\\	テレビで 映画[えいが]を 見[み]ました。	
\\	今度の週末は、映画が見たいです。	
\\	今度[こんど]の 週末[しゅうまつ]は、 映画[えいが]が 見[み]たいです。	
\\	いつか中国に行きたいです。	
\\	いつか 中国[ちゅうごく]に 行[い]きたいです。	
\\	あの人にはもう会いたくありません。	
\\	あの 人[ひと]にはもう 会[あ]いたくありません。	
\\	セーターが買いたかったから、デパートに行きました。	
\\	セーターが 買[か]いたかったから、デパートに 行[い]きました。	
\\	メアリさんはトイレに行きたいと言っていました。	
\\	メアリさんはトイレに 行[い]きたいと 言[い]っていました。	
\\	メアリさんはコーヒーを飲みたがっています。	
\\	メアリさんはコーヒーを 飲[の]みたがっています。	
\\	大阪で買い物をしたり、韓国料理を食べたりします。	
\\	大阪[おおさか]で 買い物[かいもの]をしたり、 韓国[かんこく] 料理[りょうり]を 食[た]べたりします。	
\\	たり
\\	たりする	
\\	たり
\\	週末は、勉強したり、友だちと話したりしました。	
\\	週末[しゅうまつ]は、 勉強[べんきょう]したり、 友[とも]だちと 話[はな]したりしました。	
\\	踊ったり、音楽を聞いたりするのが好きです。	
\\	踊[おど]ったり、 音楽[おんがく]を 聞[き]いたりするのが 好[す]きです。	
\\	ヨーロッパに行ったことがありますか。 はい、行ったことがありますか。	
\\	ヨーロッパに 行[い]ったことがありますか。 はい、 行[おこな]ったことがありますか。	
\\	たけしさんは授業を休んだことがありません。	
\\	たけしさんは 授業[じゅぎょう]を 休[やす]んだことがありません。	
\\	京都や奈良に行きました。	
\\	京都[きょうと]や 奈良[なら]に 行[い]きました。	
\\	山下先生はテレビを見ますか。 いいえ、テレビは見ません。 (いいえ、私はテレビを見ません。)	
\\	山下[やました] 先生[せんせい]はテレビを 見[み]ますか。 いいえ、テレビは 見[み]ません。 (いいえ、 私[わたし]はテレビを 見[み]ません。)	
\\	私はその人に一回だけ会ったことがあります。	
\\	私[わたし]はその 人[ひと]に一 回[かい]だけ 会[あ]ったことがあります。	
\\	晩ごはんにサラダを食べました。	
\\	晩[ばん]ごはんにサラダを 食[た]べました。	
\\	おみやげに絵葉書を買いました。	
\\	おみやげに 絵葉書[えはがき]を 買[か]いました。	
\\	東京には大きいデパートがたくさんあります。	
\\	東京[とうきょう]には 大[おお]きいデパートがたくさんあります。	
\\	あしたテストがあります。 あしたテストがあるんです。	
\\	あしたテストがあります。 あしたテストがあるんです。	
\\	トイレに行きたいです。 トイレに行きたいんです。	
\\	トイレに 行[い]きたいです。 トイレに 行[い]きたいんです。	
\\	成績がよくないんです。	
\\	成績[せいせき]がよくないんです。	
\\	試験が終わったんです。	
\\	試験[しけん]が 終[お]わったんです。	
\\	静かです 静かなんです	
\\	静[しず]かです 静[しず]かなんです	
\\	学生です 学生なんです	
\\	学生[がくせい]です 学生[がくせい]なんです	
\\	どうして彼と別れたんですか。 彼、ぜんぜんお風呂に入らないんです。	
\\	どうして 彼[かれ]と 別[わか]れたんですか。 彼[かれ]、ぜんぜんお 風呂[ふろ]に 入[はい]らないんです。	
\\	どうしたんですか。 猫が死んだんです。	
\\	どうしたんですか。 猫[ねこ]が 死[し]んだんです。	
\\	とてもいい教科書ですね。 ええ。 私の大学の先生が書いたんです。	
\\	とてもいい 教科書[きょうかしょ]ですね。 ええ。  私[わたし]の 大学[だいがく]の 先生[せんせい]が 書[か]いたんです。	
\\	食べすぎてはいけません。	
\\	食[た]べすぎてはいけません。	
\\	早く起きすぎました。	
\\	早[はや]く 起[お]きすぎました。	
\\	この本は高すぎます。	
\\	この 本[ほん]は 高[たか]すぎます。	
\\	あの人は親切すぎます。	
\\	あの 人[ひと]は 親切[しんせつ]すぎます。	
\\	もっと野菜を食べたほうがいいですよ。	
\\	もっと 野菜[やさい]を 食[た]べたほうがいいですよ。	
\\	授業を休まないほうがいいですよ。	
\\	授業[じゅぎょう]を 休[やす]まないほうがいいですよ。	
\\	ので 
\\	から)	
\\	ので
\\	いつも日本語で話すので、日本語が上手になりました。	
\\	いつも 日本語[にほんご]で 話[はな]すので、 日本語[にほんご]が 上手[じょうず]になりました。	
\\	宿題がたくさんあったので、きのうの夜、寝ませんでした。	
\\	宿題[しゅくだい]がたくさんあったので、きのうの 夜[よる]、 寝[ね]ませんでした。	
\\	その人はいじわるなので、きらいです。	
\\	その 人[ひと]はいじわるなので、きらいです。	
\\	今日は日曜日なので、銀行は休みです。	
\\	今日[きょう]は 日曜日[にちようび]なので、 銀行[ぎんこう]は 休[やす]みです。	
\\	来週テストがあるから、たくさん勉強しなくちゃいけません。	
\\	来週[らいしゅう]テストがあるから、たくさん 勉強[べんきょう]しなくちゃいけません。	
\\	食べる、食べない、食べなくちゃいけません	
\\	食[た]べる、 食[た]べない、 食[た]べなくちゃいけません	
\\	言う、言わない、言わなくちゃいけません	
\\	言[い]う、 言[い]わない、 言[い]わなくちゃいけません	
\\	する、しない、しなくちゃいけません	
\\	する、しない、しなくちゃいけません	
\\	くる、こない、こなくちゃいけません	
\\	くる、こない、こなくちゃいけません	
\\	今朝は、六時に起きなくちゃいけませんでした。	
\\	今朝[けさ]は、六 時[じ]に 起[お]きなくちゃいけませんでした。	
\\	毎日、練習しなくちゃいけないんです。	
\\	毎日[まいにち]、 練習[れんしゅう]しなくちゃいけないんです。	
\\	明日は雨が降るでしょう。 明日は雨が降らないでしょう。	
\\	明日[あした]は 雨[あめ]が 降[ふ]るでしょう。 明日[あした]は 雨[あめ]が 降[ふ]らないでしょう。	
\\	北海道は寒いでしょう。 北海道は寒かないでしょう。	
\\	北海道[ほっかいどう]は 寒[さむ]いでしょう。 北海道[ほっかいどう]は 寒[かん]かないでしょう。	
\\	山下先生は魚が好きでしょう。 山下先生は魚が好きじゃないでしょう。	
\\	山下[やました] 先生[せんせい]は 魚[さかな]が 好[す]きでしょう。 山下[やました] 先生[せんせい]は 魚[さかな]が 好[す]きじゃないでしょう。	
\\	あの人はオーストラリア人でしょう。 あの人はオーストラリア人じゃないでしょう。	
\\	あの 人[ひと]はオーストラリア 人[じん]でしょう。 あの 人[ひと]はオーストラリア 人[じん]じゃないでしょう。	
\\	日本語と韓国語と、どっちのほうが難しいでしょうか。	
\\	日本語[にほんご]と 韓国[かんこく] 語[ご]と、どっちのほうが 難[むずか]しいでしょうか。	
\\	たけしさんは興味があるだろうと思います。	
\\	たけしさんは 興味[きょうみ]があるだろうと 思[おも]います。	
\\	ジョン、中国語がわかるでしょ?これ、読んで。	
\\	ジョン、 中国[ちゅうごく] 語[ご]がわかるでしょ?これ、 読[よ]んで。	
\\	読む 読みます 読まない 読みさせん	
\\	読[よ]む 読[よ]みます 読[よ]まない 読[よ]みさせん	
\\	かわいい かわいいです かわいくない かわいくありません	
\\	かわいい かわいいです かわいくない かわいくありません	
\\	静かだ 静かです 静かじゃない 静かじゃありません	
\\	静[しず]かだ 静[しず]かです 静[しず]かじゃない 静[しず]かじゃありません	
\\	学生だ 学生です 学生じゃない 学生じゃありません	
\\	学生[がくせい]だ 学生[がくせい]です 学生[がくせい]じゃない 学生[がくせい]じゃありません	
\\	食べない しない 来ない ない	
\\	食[た]べない しない 来[こ]ない ない	
\\	書かない 話さない 待たない	
\\	書[か]かない 話[はな]さない 待[ま]たない	
\\	死なない 読まない 作らない	
\\	死[し]なない 読[よ]まない 作[つく]らない	
\\	泳がない 呼ばない 買わない	
\\	泳[およ]がない 呼[よ]ばない 買[か]わない	
\\	スーさんは、あした試験があると言っていました。	
\\	スーさんは、あした 試験[しけん]があると 言[い]っていました。	
\\	(私は)たけしさんはメアリーさんが好きだと思います。	
\\	私[わたし]は)たけしさんはメアリーさんが 好[す]きだと 思[おも]います。	
\\	ここで写真を撮らないでください。	
\\	ここで 写真[しゃしん]を 撮[と]らないでください。	
\\	(私は)日本語を勉強するのが好きです。	
\\	私[わたし]は) 日本語[にほんご]を 勉強[べんきょう]するのが 好[す]きです。	
\\	(私は)部屋を掃除するのがきらいです。	
\\	私[わたし]は) 部屋[へや]を 掃除[そうじ]するのがきらいです。	
\\	ロバートさんは料理を作るのが上手です。	
\\	ロバートさんは 料理[りょうり]を 作[つく]るのが 上手[じょうず]です。	
\\	たけしさんは英語を話すのが下手です。	
\\	たけしさんは 英語[えいご]を 話[はな]すのが 下手[へた]です。	
\\	ロバートさんが沖縄に行きました。	
\\	ロバートさんが 沖縄[おきなわ]に 行[い]きました。	
\\	だれが沖縄に行きましたか。	
\\	だれが 沖縄[おきなわ]に 行[い]きましたか。	
\\	どのクラスがおもしろいですか。 日本語のクラスがおもしろいです。	
\\	どのクラスがおもしろいですか。 日本語[にほんご]のクラスがおもしろいです。	
\\	(このクラスで)だれがめがねをかけていますか。 山下先生がめがねをかけています。	
\\	(このクラスで)だれがめがねをかけていますか。 山下[やました] 先生[せんせい]がめがねをかけています。	
\\	猫が何か持って来ました。	
\\	猫[ねこ]が 何[なに]か 持[も]って 来[き]ました。	
\\	猫は何か食べましたか。	
\\	猫[ねこ]は 何[なに]か 食[た]べましたか。	
\\	いいえ、猫は何も食べませんでした。	
\\	いいえ、 猫[ねこ]は 何[なに]も 食[た]べませんでした。	
\\	スーさんは今勉強しています。	
\\	スーさんは 今[こん] 勉強[べんきょう]しています。	
\\	たけしさんは英語の本をよんでいます。	
\\	たけしさんは 英語[えいご]の 本[ほん]をよんでいます。	
\\	私は英語を教えています。	
\\	私[わたし]は 英語[えいご]を 教[おし]えています。	
\\	メアリーさんは日本語を勉強しています。	
\\	メアリーさんは 日本語[にほんご]を 勉強[べんきょう]しています。	
\\	山下先生は結婚しています。	
\\	山下[やました] 先生[せんせい]は 結婚[けっこん]しています。	
\\	みちこさんは窓のそばに座っています。	
\\	みちこさんは 窓[まど]のそばに 座[すわ]っています。	
\\	スーさんはお金をたくさん持っています。	
\\	スーさんはお 金[かね]をたくさん 持[も]っています。	
\\	山下先生は英語を知っています。	
\\	山下[やました] 先生[せんせい]は 英語[えいご]を 知[し]っています。	
\\	トムさんはちょっと太っています。	
\\	トムさんはちょっと 太[ふと]っています。	
\\	私の弟はとてもやせています。	
\\	私[わたし]の 弟[おとうと]はとてもやせています。	
\\	メアリーさんは
\\	シャツを着ています。	
\\	メアリーさんは	
\\	お父さんは起きています。	
\\	お 父[とう]さんは 起[お]きています。	
\\	父と母は東京に住んでいます。	
\\	父[ちち]と 母[はは]は 東京[とうきょう]に 住[す]んでいます。	
\\	私の姉は日本の会社に勤めています。	
\\	私[わたし]の 姉[あね]は 日本[にっぽん]の 会社[かいしゃ]に 勤[つと]めています。	
\\	中国に行っています。	
\\	中国[ちゅうごく]に 行[い]っています。	
\\	うちに来ています。	
\\	うちに 来[き]ています。	
\\	食べています 食べていません 食べていました 食べていませんでした	
\\	食[た]べています 食[た]べていません 食[た]べていました 食[た]べていませんでした	
\\	トムさんは髪が長いです。	
\\	トムさんは 髪[かみ]が 長[なが]いです。	
\\	安くて	
\\	安[やす]くて	
\\	よくて	
\\	よくて	
\\	元気で	
\\	元気[げんき]で	
\\	日本人で	
\\	日本人[にっぽんじん]で	
\\	あの店の食べ物は安くて、おいしいです。	
\\	あの 店[みせ]の 食べ物[たべもの]は 安[やす]くて、おいしいです。	
\\	あの人はいつも元気で、おもしろいです。	
\\	あの 人[ひと]はいつも 元気[げんき]で、おもしろいです。	
\\	山下先生は日本人で、四十歳ぐらいです	
\\	山下[やました] 先生[せんせい]は 日本人[にっぽんじん]で、四十 歳[さい]ぐらいです	
\\	デパートにかばんを買いに行きました。	
\\	デパートにかばんを 買[か]いに 行[い]きました。	
\\	メアリーさんは日本に日本語を勉強しに来ました。	
\\	メアリーさんは 日本[にっぽん]に 日本語[にほんご]を 勉強[べんきょう]しに 来[き]ました。	
\\	一人 二人 三人 四人	
\\	一 人[にん] 二 人[にん] 三 人[にん] 四 人[にん]	
\\	私のクラスに(は)スウェーデン人の学生が一人います。	
\\	私[わたし]のクラスに(は)スウェーデン 人[じん]の 学生[がくせい]が一 人[にん]います。	
\\	食べて	
\\	食[た]べて	
\\	会って 待って とって	
\\	会[あ]って 待[ま]って とって	
\\	読んで 遊んで 死んで	
\\	読[よ]んで 遊[あそ]んで 死[し]んで	
\\	書いて 行って 泳いで 話して	
\\	書[か]いて 行[い]って 泳[およ]いで 話[はな]して	
\\	して 来て	
\\	して 来[き]て	
\\	教科書のテープを聞いて下さい	
\\	教科書[きょうかしょ]のテープを 聞[き]いて 下[くだ]さい	
\\	すみません。ちょっと教えて下さい。	
\\	すみません。ちょっと 教[おし]えて 下[くだ]さい。	
\\	教科書を見てもいいですか。はい、見てもいいですよ。	
\\	教科書[きょうかしょ]を 見[み]てもいいですか。 見[み]てもいいですよ。	
\\	いいえ、教科書を見てはいけません。	
\\	いいえ、 教科書[きょうかしょ]を 見[み]てはいけません。	
\\	ノートを借りて、コピーします。	
\\	ノートを 借[か]りて、コピーします。	
\\	今日は、六時に起きて、勉強しました。	
\\	今日[きょう]は、 六[ろく] 時[じ]に 起[お]きて、 勉強[べんきょう]しました。	
\\	食堂に行って、昼ごはんを食べましょう。	
\\	食堂[しょくどう]に 行[い]って、 昼[ひる]ごはんを 食[た]べましょう。	
\\	バスに乗って、会社に行きます。	
\\	バスに 乗[の]って、 会社[かいしゃ]に 行[い]きます。	
\\	教科書を忘れて、すみません。	
\\	教科書[きょうかしょ]を 忘[わす]れて、すみません。	
\\	私は勉強します。あしたテストがありますから。	
\\	私[わたし]は 勉強[べんきょう]します。あしたテストがありますから。	
\\	バスに乗りましょう。タクシーは高いですから。	
\\	バスに 乗[の]りましょう。タクシーは 高[たか]いですから。	
\\	(私が)やりましょうか。	
\\	私[わたし]が)やりましょうか。	
\\	荷物を持ちましょうか。	
\\	荷物[にもつ]を 持[も]ちましょうか。	
\\	きのう、おもしろい映画を見ました。	
\\	きのう、おもしろい 映画[えいが]を 見[み]ました。	
\\	山下先生はこわい先生です。	
\\	山下[やました] 先生[せんせい]はこわい 先生[せんせい]です。	
\\	京都できれいな写真を撮りました。	
\\	京都[きょうと]できれいな 写真[しゃしん]を 撮[と]りました。	
\\	山下先生は元気な先生です。	
\\	山下[やました] 先生[せんせい]は 元気[げんき]な 先生[せんせい]です。	
\\	おもしろいです おもしろくありません おもしろかったです おもしろくありませんでした	
\\	おもしろいです おもしろくありません おもしろかったです おもしろくありませんでした	
\\	いいです よくありません よかったです よくありませんでした	
\\	いいです よくありません よかったです よくありませんでした	
\\	元気です 元気じゃありません 元気でした 元気じゃありませんでした	
\\	元気[げんき]です 元気[げんき]じゃありません 元気[げんき]でした 元気[げんき]じゃありませんでした	
\\	ロバートさんは日本語のクラスが好きです。	
\\	ロバートさんは 日本語[にほんご]のクラスが 好[す]きです。	
\\	山下先生は魚がきらいです。	
\\	山下[やました] 先生[せんせい]は 魚[さかな]がきらいです。	
\\	好きでもきらいでもありません。	
\\	好[す]きでもきらいでもありません。	
\\	これは私の好きなテレビです。	
\\	これは 私[わたし]の 好[す]きなテレビです。	
\\	沖縄の海はとてもきれいでした。	
\\	沖縄[おきなわ]の 海[うみ]はとてもきれいでした。	
\\	この部屋はちょっと暑いです。	
\\	この 部屋[へや]はちょっと 暑[あつ]いです。	
\\	たけしさんはコーヒーが大好きです。	
\\	たけしさんはコーヒーが 大好[だいす]きです。	
\\	キムさんはなっとうが大きらいです。	
\\	キムさんはなっとうが 大[だい]きらいです。	
\\	一緒に図書館で勉強しましょう。	
\\	一緒[いっしょ]に 図書館[としょかん]で 勉強[べんきょう]しましょう。	
\\	喫茶店でコーヒーを飲みましょうか。	
\\	喫茶店[きっさてん]でコーヒーを 飲[の]みましょうか。	
\\	リーさんは切手を三枚買いました。	
\\	リーさんは 切手[きって]を 三[さん]枚[まい] 買[か]いました。	
\\	見られる	
\\	見[み]られる	
\\	行ける 話せる 買える	
\\	行[い]ける 話[はな]せる 買[か]える	
\\	泳げる 遊べる 待てる	
\\	泳[およ]げる 遊[あそ]べる 待[ま]てる	
\\	読める 死ねる 取れる(授業)	
\\	読[よ]める 死[し]ねる 取[と]れる
\\	授業[じゅぎょう])	
\\	来られる できる	
\\	来[こ]られる できる	
\\	出れる 見れる 来れる	
\\	出[で]れる 見[み]れる 来[こ]れる	
\\	私は日本語がはなせます。	
\\	私[わたし]は 日本語[にほんご]がはなせます。	
\\	私は泳げないんです。	
\\	私[わたし]は 泳[およ]げないんです。	
\\	雨が降ったので、海に行けませんでした。	
\\	雨[あめ]が 降[ふ]ったので、 海[うみ]に 行[い]けませんでした。	
\\	見られる 見られない 見られた 見られなかった	
\\	見[み]られる 見[み]られない 見[み]られた 見[み]られなかった	
\\	見られます 見られません 見られました 見られませんでした	
\\	見[み]られます 見[み]られません 見[み]られました 見[み]られませんでした	
\\	漢字 を/が 読める	
\\	漢字[かんじ] を/が  読[よ]める	
\\	仕事 が できる	
\\	仕事[しごと] が できる	
\\	山に登れる	
\\	山[やま]に 登[のぼ]れる	
\\	日本語はおもしろいし、先生はいいし、私は日本語の授業が大好きです。	
\\	日本語[にほんご]はおもしろいし、 先生[せんせい]はいいし、 私[わたし]は 日本語[にほんご]の 授業[じゅぎょう]が 大好[だいす]きです。	
\\	家族から手紙が来たし、彼と電話で話したし、きのうはとてもいい日でした。	
\\	家族[かぞく]から 手紙[てがみ]が 来[き]たし、 彼[かれ]と 電話[でんわ]で 話[はな]したし、きのうはとてもいい 日[ひ]でした。	
\\	国に帰りたいですか。 いいえ、日本の生活は楽しいし、いい友だちがいるし、帰りたくありません。	
\\	国[くに]に 帰[かえ]りたいですか。 いいえ、 日本[にっぽん]の 生活[せいかつ]は 楽[たの]しいし、いい 友[とも]だちがいるし、 帰[かえ]りたくありません。	
\\	物価が安いし、この町の生活は楽です。	
\\	物価[ぶっか]が 安[やす]いし、この 町[まち]の 生活[せいかつ]は 楽[らく]です。	
\\	山下先生はいい先生です。教えるのが上手だし、親切だし。	
\\	山下[やました] 先生[せんせい]はいい 先生[せんせい]です。 教[おし]えるのが 上手[じょうず]だし、 親切[しんせつ]だし。	
\\	おもしろいし 好きだし 学生だし	
\\	し 
\\	おもしろいし 好[す]きだし 学生[がくせい]だし	
\\	このりんごはおいしそうです。	
\\	このりんごはおいしそうです。	
\\	あしたは天気がよさそうです。	
\\	あしたは 天気[てんき]がよさそうです。	
\\	メアリーさんは元気そうでした。	
\\	メアリーさんは 元気[げんき]そうでした。	
\\	この本は難しくなさそうです。	
\\	この 本[ほん]は 難[むずか]しくなさそうです。	
\\	ともこさんはテニスが上手じゃなさそうです。	
\\	ともこさんはテニスが 上手[じょうず]じゃなさそうです。	
\\	暖かそうなセーターを着ています。	
\\	暖[あたた]かそうなセーターを 着[き]ています。	
\\	漢字がわからなかったので、日本人の友だちに聞いてみました。	
\\	漢字[かんじ]がわからなかったので、 日本人[にっぽんじん]の 友[とも]だちに 聞[き]いてみました。	
\\	友だちがあの店のケーキはおいしいと言っていましたから、今度食べてみます。	
\\	友[とも]だちがあの 店[みせ]のケーキはおいしいと 言[い]っていましたから、 今度[こんど] 食[た]べてみます。	
\\	ブラジルに行ったことがありますか。 メキシコなら行ったことがありますが、ブラジルは行ったことがありません。	
\\	ブラジルに 行[い]ったことがありますか。 メキシコなら 行[い]ったことがありますが、ブラジルは 行[い]ったことがありません。	
\\	日本語がわかりますか。 ひらがなならわかります。	
\\	日本語[にほんご]がわかりますか。 ひらがなならわかります。	
\\	に 
\\	に 
\\	私は一週間に三回髪を洗います。	
\\	私[わたし]は 一週間[いっしゅうかん]に 三[さん] 回[かい] 髪[かみ]を 洗[あら]います。	
\\	私は一か月に一回家族に電話をかけます。	
\\	私[わたし]は 一か月[いっかげつ]に 一回[いっかい] 家族[かぞく]に 電話[でんわ]をかけます。	
\\	父は一年に二回旅行をします。	
\\	父[ちち]は 一[いち] 年[ねん]に 二[に] 回[かい] 旅行[りょこう]をします。	
\\	なら に、で、から 
\\	は、が、を	なら 
\\	なら に、で、から 
\\	は、が、を	
\\	いい漢字の辞書がほしいです。	
\\	いい 漢字[かんじ]の 辞書[じしょ]がほしいです。	
\\	子供の時、ゴジラのおもちゃがほしかったです。	
\\	子供[こども]の 時[とき]、ゴジラのおもちゃがほしかったです。	
\\	お金はあまりほしくありません。	
\\	お 金[かね]はあまりほしくありません。	
\\	ロバートさんはコンピューターがほしいと言っています。	
\\	ロバートさんはコンピューターがほしいと 言[い]っています。	
\\	きょうこさんはクラシックの
\\	がほしくないでしょう。	
\\	かもしれません)	きょうこさんはクラシックの 
\\	がほしくないでしょう。	
\\	トムさんは友だちをほしがっています。(ほしがる/ほしがっている)	
\\	トムさんは 友[とも]だちをほしがっています。(ほしがる/ほしがっている)	
\\	あしたは雨が降るかもしれません。	
\\	あしたは 雨[あめ]が 降[ふ]るかもしれません。	
\\	田中さんより、鈴木さんのほうが背が高いかもしれません。	
\\	田中[たなか]さんより、 鈴木[すずき]さんのほうが 背[せ]が 高[たか]いかもしれません。	
\\	あしたは天気がよくないかもしれません。	
\\	あしたは 天気[てんき]がよくないかもしれません。	
\\	トムさんは、子供の時、いじわるだったかもしれません。	
\\	トムさんは、 子供[こども]の 時[とき]、いじわるだったかもしれません。	
\\	トムさんはカナダ人だ。 トムさんはカナダ人_かもしれません。	
\\	トムさんはカナダ 人[じん]だ。 トムさんはカナダ 人[じん]_かもしれません。	
\\	山下先生は犬がきらいだ。 山下先生は犬がきらい_かもしれません。	
\\	山下[やました] 先生[せんせい]は 犬[いぬ]がきらいだ。 山下[やました] 先生[せんせい]は 犬[いぬ]がきらい_かもしれません。	
\\	私はその女の人に花をあげます。	
\\	私[わたし]はその 女[おんな]の 人[ひと]に 花[はな]をあげます。	
\\	その女の人は男の人に時計をあげました。	
\\	その 女[おんな]の 人[ひと]は 男[おとこ]の 人[ひと]に 時計[とけい]をあげました。	
\\	両親が(私に)新しい車をくれるかもしれません。	
\\	両親[りょうしん]が
\\	私[わたし]に) 新[あたら]しい 車[くるま]をくれるかもしれません。	
\\	私は(が)姉に/姉から古い辞書をもらいました。 姉は(が)私に古い辞書をくれました。	
\\	私[わたし]は(が) 姉[あね]に/ 姉[あね]から 古[ふる]い 辞書[じしょ]をもらいました。 姉[あね]は(が)私[わたし]に 古[ふる]い 辞書[じしょ]をくれました。	
\\	もっと勉強したらどうですか。	
\\	もっと 勉強[べんきょう]したらどうですか。	
\\	薬を飲んだらどうですか。	
\\	薬[くすり]を 飲[の]んだらどうですか。	
\\	うちにきませんか。
\\	うちにきたらどうですか。)	
\\	うちにきませんか。
\\	うちにきたらどうですか。)	
\\	私のうちには猫が三匹います。	
\\	私[わたし]のうちには 猫[ねこ]が 三匹[さんびき]います。	
\\	かさを三本買いました。	
\\	かさを 三本[さんぼん] 買[か]いました。	
\\	私の母は猫を三匹も飼っています。	
\\	私[わたし]の 母[はは]は 猫[ねこ]を 三匹[さんびき]も 飼[か]っています。	
\\	きのうのパーティーには学生が二十人も来ました。	
\\	きのうのパーティーには 学生[がくせい]が 二十人[にじゅうにん]も 来[き]ました。	
\\	私は日本語の辞書を一冊しか持っていません。	
\\	私[わたし]は 日本語[にほんご]の 辞書[じしょ]を 一冊[いっさつ]しか 持[も]っていません。	
\\	この会社にはコンピューターが二台しかありません。	
\\	この 会社[かいしゃ]にはコンピューターが 二台[にだい]しかありません。	
\\	食べよう	
\\	食[た]べよう	
\\	行こう 話そう 買おう 泳ごう 遊ぼう	
\\	行[い]こう 話[はな]そう 買[か]おう 泳[およ]ごう 遊[あそ]ぼう	
\\	待とう 読もう 死のう 取ろう	
\\	待[ま]とう 読[よ]もう 死[し]のう 取[と]ろう	
\\	来よう しよう	
\\	来[こ]よう しよう	
\\	あしたは授業がないから、今夜、どこかに食べに行こう。	
\\	あしたは 授業[じゅぎょう]がないから、 今夜[こんや]、どこかに 食[た]べに 行[い]こう。	
\\	結婚しようよ。	
\\	結婚[けっこん]しようよ。	
\\	手伝おうか。	
\\	手伝[てつだ]おうか。	
\\	友だちがおもしろいと行っていたから、この映画を見ようか。	
\\	友[とも]だちがおもしろいと 行[い]っていたから、この 映画[えいが]を 見[み]ようか。	
\\	今度、いつ会おうか。	
\\	今度[こんど]、いつ 会[あ]おうか。	
\\	毎日三時間日本語を勉強しようと思っています。	
\\	毎日[まいにち] 三[さん] 時間[じかん] 日本語[にほんご]を 勉強[べんきょう]しようと 思[おも]っています。	
\\	一万円あげましょう。何に使いますか。 漢字の辞書を買おうと思います。	
\\	一万[いちまん] 円[えん]あげましょう。 何[なに]に 使[つか]いますか。 漢字[かんじ]の 辞書[じしょ]を 買[か]おうと 思[おも]います。	
\\	両親から一万円もらったんですか。何に使うんですか。 漢字の辞書を買おうと思っています。	
\\	両親[りょうしん]から 一万[いちまん] 円[えん]もらったんですか。 何[なに]に 使[つか]うんですか。 漢字[かんじ]の 辞書[じしょ]を 買[か]おうと 思[おも]っています。	
\\	あした試験があるので、今晩勉強しておきます。	
\\	あした 試験[しけん]があるので、 今晩[こんばん] 勉強[べんきょう]しておきます。	
\\	寒くなるから、ストーブをつけておきました。	
\\	寒[さむ]くなるから、ストーブをつけておきました。	
\\	ホテルを予約しておかなくちゃいけません。	
\\	ホテルを 予約[よやく]しておかなくちゃいけません。	
\\	これは去年の誕生日に彼女がくれた本です。	
\\	これは 去年[きょねん]の 誕生[たんじょう] 日[び]に 彼女[かのじょ]がくれた 本[ほん]です。	
\\	父がエイミー・タンが書いた本をくれました。	
\\	父[ちち]がエイミー・タンが 書[か]いた 本[ほん]をくれました。	
\\	私が一番感動した映画は「生きる」です。	
\\	私[わたし]が 一番[いちばん] 感動[かんどう]した 映画[えいが]は
\\	生[い]きる」です。	
\end{CJK}
\end{document}