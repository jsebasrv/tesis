\documentclass[8pt]{extreport} 
\usepackage{hyperref}
\usepackage{CJKutf8}
\begin{document}
\begin{CJK}{UTF8}{min}
\\	この皿はプラスチックで、そして、このコップは紙でできている。	
\\	でできている 
\\	(でできている)
\\	チーズやヨーグルトは牛乳からできます。	
\\	---からできます。 
\\	から 
\\	できる 
\\	富士山の高さは何キロくらいでしょうか。	
\\	さ 
\\	この本のよさがよく分かる。	
\\	よさ 
\\	いい 
\\	-さ。
\\	これはチョコレートのように見えるけれど、消しゴムだから、食べられませんよ。	
\\	のように 
\\	この写真を見れば分かるように、ロンドンには、色々な名所がある。	
\\	のように 
\\	小さい猫の声は、赤ちゃんが泣いているように聞こえます。	
\\	のように 
\\	エジプトはピラミッドやスフィンクスで有名です。	
\\	ーは
\\	で有名です。 
\\	このお寺は古い桜の木があることで知られている。	
\\	---は
\\	ことで知られている。 
\\	この地方は冬は雪が多く、夏はとても暑い。	
\\	く、- 
\\	(多い)
\\	く 
\\	-て 
\\	昨日は友達とレストランで晩御飯を食べ、その後、映画を見にいた。	
\\	ーて 
\\	週末はたいてい両親の家に帰るが、忙しくて帰られないこともある。	
\\	こともある。 
\\	こともある 
\\	も 
\\	ことがある 
\\	文法の説明を読んでも分からないことがある	
\\	ことがある。 
\\	ことがある 
\\	日本人が大好きな桜の花がいつ頃咲くかは、場所によって違います。	
\\	かは
\\	によって違う。 
\\	性格
\\	は人によって違います。	
\\	かは
\\	2によって違う。 
\\	私にとって、読み物が難しいかどうかは、漢字の多さによります。	
\\	ーかどうかは、
\\	2による。 
\\	この地方では、11月になると雪が降り始めます。	
\\	始める。 始める 
\\	友達へのプレゼント。	
\\	の
\\	の
\\	の, 
\\	先生とミーチングに行けない。
\\	先生とのミーチングに行けない。
\\	日本での仕事	
\\	の
\\	の, 
\\	先生とミーチングに行けない。
\\	先生とのミーチングに行けない。
\\	八時からのぱーティ。	
\\	の
\\	の, 
\\	先生とミーチングに行けない。
\\	先生とのミーチングに行けない。
\\	京都までの新幹線(しんかんせん)の切符(きっぷ)	
\\	の
\\	の, 
\\	先生とミーチングに行けない。
\\	先生とのミーチングに行けない。
\\	世界で一番長い川はナイルだと言われているが、アマゾン川だと言う人もいる。	
\\	"ーは
\\	と言われているが
\\	道後温泉は日本の一番古い温泉で、3000年の歴史があると言われています。	
\\	ーは
\\	と言われている。 
\\	日本の目所と言えば、温泉も忘れることはできません。	
\\	"ーと言えば。 
\\	週末はたいてい洗濯とか掃除(とか)をします。	
\\	ーとか(-とか) とか
\\	や 
\\	とか 
\\	漢字を覚える時は、フラッシュカードを作るとか、何回も書くとかするといいです。	
\\	ーとか(-とか) とか
\\	とか 
\\	する. 
\\	や、とか 
\\	パソコンというのは、パーソナルコンピュータのことです。	
\\	ーというのは
\\	のことだ。 
\\	すみません。学生証というのは、何のことですか?	
\\	というのは、何のことですか。 
\\	行事というのは、季節で決まった特に特別に何かを行う意味です。	
\\	というのは、-という意味です 
\\	日本語はひらがなだけでなく、カタカナや漢字も覚えなくてはいけません。	
\\	だけでなく、
\\	も 
\\	"「すみません」は、
\\	という意味だけでなく、
\\	という意味もあります。
\\	だけでなく、
\\	も 
\\	宮崎林のアニメって、いいよね。	
\\	って 
\\	田中さん、5時に来るって言ってたけど、まだ来ないね。	
\\	って言う 
\\	って 
\\	と
\\	明日はテストがないって聞いたけど、本当か。	
\\	って聞く 
\\	って 
\\	と
\\	ここに「静かにして下さい」って書いてあるのがみえませんか。	
\\	って書く 
\\	って 
\\	と
\\	僕の名前は「つとむ」だけづ、友達は「トム」って呼ぶんだ。	
\\	って呼ぶ 
\\	って 
\\	と
\\	「桃太郎」っていう話、聞いたことある?	
\\	1っていう
\\	って 
\\	と
\\	日本語は〜話す相手によって話し方を変えなければいけないことがあります。	
\\	(なければいけない)。  
\\	""いけない”  
\\	まだちょっとしなきゃなんあいことがあるんだ。	
\\	(なきゃなんあい。) 
\\	なければならない (ー
\\	なきゃならない ー
\\	なきゃなんない。) 
\\	ならない 
\\	調べなきゃならないことがあるので、図書館に行って来る。	
\\	(なきゃならない)。 
\\	なければならない (ー
\\	なきゃならない ー
\\	なきゃなんない。)
\\	病院に行って来なくちゃなんない。	
\\	(なくちゃなんない) 
\\	なくてはならないー
\\	なくちゃならないー
\\	なくちゃなんない。
\\	明日はもっと早く起きなくちゃ。	
\\	なくちゃ。 
\\	なくては。 
\\	いけない/ならない 
\\	今学期は日本語やアジアの歴史などのクラスを取っています。	
\\	や 
\\	など (の/、)
\\	など 
\\	や。 
\\	日本の食べ物の中では、寿司やてんぷらなどがす好きです。	
\\	や 
\\	など 
\\	など 
\\	や。 
\\	漢字は日本語の勉強で最も大切なものの一つです。	
\\	の一つです。 
\\	は
\\	{一つ/一人)だ。 
\\	読書は私が一番好きなことの一つだ。	
\\	は
\\	の一つだ。 
\\	モーツアルトは、最も人気がある差客か(さきゃくか=
\\	の一人だ。	
\\	は
\\	の一つだ。 
\\	あなたにとって、一番大切な人は誰ですか。	
\\	にとって 
\\	ペンの代わりに鉛筆(えんぴつ)を使って書いてください。	
\\	の代わりに 
\\	母が病気だったので、(母の)代わりに私が晩ご飯を作ります。	
\\	の代わりに 
\\	の 
\\	父は家族のためにいっしょうけんめい働いている。	
\\	(のために) 
\\	日本の会社で働きたいと思うている。そのためにはもっと日本語が上手にならなくては行けない。	
\\	(のために) 
\\	コンピュータが壊れたために、レポートが書きませんでした。	
\\	のため(に) 
\\	から 
\\	ので。
\\	勉強が忙しいため、友達と合う時間がない。	
\\	のため(に) 
\\	から 
\\	ので。
\\	トムさんは日本語が上手なため、ときどき通訳(つうやく=
\\	を頼まれる(たのまれる)。	
\\	のため(に) 
\\	から 
\\	ので。
\\	九州か四国へ行く。 本か雑誌が(ある)。 友達か家族と旅行に行く。	
\\	か 
\\	か 
\\	その店がおいしかったかまずかったか覚えていますか。	
\\	か
\\	か  
\\	その店、おいしいんでしょ。	
\\	~でしょ 
\\	~でしょう 
\\	でしょう 
\\	だろう)
\\	でしょ。
\\	このケーキ、おいしいだろう。	
\\	~だろう
\\	だろう, 
\\	でしょう。
\\	う 
\\	これは、1年生の学生によく見られる間違いです。	
\\	が見られる。 
\\	見る 
\\	日本人の友達ができてから、日本語が上手に話せるようになった。	
\\	ようになる。 
\\	話すようになる	
\\	ようになる。 
\\	話さないようになる	
\\	ようになる。 
\\	なくなる 
\\	ないようになる 
\\	父は病気になってから、お酒を飲まないようになった。	
\\	ようになる。 
\\	なくなる 
\\	ないようになる 
\\	同じ感じまた間違えた。	
\\	"また また、
\\	この本は面白く、また、勉強にもなる。	
\\	"また また、
\\	海外旅行に行く前にパスポートを取る必要があります。	
\\	必要がある。 
\\	私の国では日本に旅行に行く時、旅行ビザを取る必要はない。	
\\	必要はない。 
\\	は 
\\	雨の場合はテニスの試合はない。その場合、チケットは次の試合に使うことが出来る。	
\\	の場合(は/には) 
\\	(この、その、
\\	場合 場合 
\\	時 
\\	場合 
\\	分からない(時/場合)は、言って下さい。	
\\	場合(に/には) 場合 
\\	時 
\\	場合 
\\	私が取っているのは、中国語ではなくて日本語です。	
\\	1(では/じゃ)なく(て)
\\	ではなくて 
\\	じゃなくて 
\\	ここですか。 
\\	いいえ、ここじゃなくて、あそこです。	
\\	1(では/じゃ)なく(て)
\\	ではなくて 
\\	じゃなくて 
\\	お土産にTシャツを何枚かかった。	
\\	"何+
\\	か 
\\	おいしい日本のみかんをいくつかもらいました。	
\\	"いく+
\\	か 
\\	いくつ 
\\	いく人, いく 
\\	何 
\\	つ。
\\	忘れてしまった 
\\	忘れちゃった。	~てしまった。 
\\	買っておいた 
\\	買っといた。	
\\	ておく 
\\	飲んでしまう。 
\\	飲んじゃう。	〜でしまう/しまった 
\\	見せてあげる。 
\\	見せたげる。	~てあげる。 
\\	このクラスにはキムさんの他に(も)韓国語出来る学生がありますか。	
\\	の他に(も)~~ 
\\	日本では、日本語を勉強する他に、色々な所に旅行にも行きたいと思っています。	
\\	の他に(も)~~ 
\\	夏休みに大学の夏のコースを取った。その他にも、バイトをしたり、日本に行ったりした。	
\\	"(その/あの/この)他に(も)~~ 
\\	週末は宿題がたくさんあったので、勉強の他(に)は何も出来なかった。	
\\	〜他(に)は(〜ない). 
\\	今、私が住んでいるアパートの広さは、この部屋と同じぐらいです。	
\\	と同じぐらい〜 
\\	今度住むアパートは、この部屋と同じぐらいの広さです。	
\\	1と同じくらい/ぐらい
\\	ドラえモンは実はネコ型ロボットなんです。	
\\	の型だ。 
\\	私の血液型は
\\	です。	
\\	の型だ。 
\\	父は1950年型のキャデラックを持っている。	
\\	1型(の)
\\	大阪、京都、それに奈良にも行った。	
\\	1(、/と/に)
\\	2、それに
\\	それに 
\\	日本の夏はとても暑い。それに湿度も高い。	
\\	それに、
\\	それに 
\\	そのアルバイトはあまり大変じゃないし、それに給料もいい。	
\\	1し、それに 
\\	それに 
\\	あの人が来るのなら、私は帰ります。	
\\	1のなら
\\	の 
\\	トムが行くなら、私も行きます。	
\\	1なら
\\	明日なら大丈夫です。	
\\	(の)なら〜。 
\\	の 
\\	日本語なら読めます。	
\\	なら〜。 
\\	の 
\\	友達が迎えに来たら、私は帰ります。	
\\	1たら
\\	たら 
\\	(の)なら, 
\\	姫路城は日本で最も美しいお城の一つとして知られています。	
\\	として 
\\	毎日、新しい漢字を五つ覚えることにしている。	
\\	ことにしている。 
\\	私のアパートでは、ペットを飼ってはいけないことになっています。	
\\	ことになっている。 
\\	ゾウは長い鼻をしています。それに、耳も大きいです。	
\\	は 
\\	をしている。 
\\	1は
\\	2が 
\\	きれいな目をした人。	
\\	をした 
\\	した 
\\	している。
\\	4年間日本語を勉強してきました。	
\\	くる 
\\	日本語の文法が面白くなってきました。	
\\	くる 
\\	これから、もっと暖かくなっていきます。	
\\	いく 
\\	図書館で本を借りて来る。	
\\	くる。 
\\	くる。
\\	日本に引っ越すことになりました。	
\\	-ことになった。 
\\	先生は学生に「漢字を覚えなさい」と言いました。	
\\	と言う。
\\	先生は学生に漢字を覚えるように言いました。	
\\	ように言う。 
\\	私は友達に「ノートを見せて下さい」と頼みました。	
\\	と頼みました。
\\	私は友達にノートを見せてくれるように頼みました。	
\\	くれるように頼む。 
\\	ように頼む 
\\	てくれる 
\\	私はよく両親に[弟をいじめてはいけません」と言われます。	
\\	と言う。
\\	私はよく両親に弟をいじめないように言われます。	
\\	ないように頼む/言う  
\\	母に薬を持って来てくれるように頼もうか。	
\\	くれるように頼む。 
\\	ように頼む 
\\	てくれる 
\\	もし文法が間違ってたら、直してくれない?	
\\	くれる? 
\\	この記事、読むの、手伝ってくれない?	
\\	くれない? 
\\	今勉強中だから、もうちょっと静かにしてもらえる?	
\\	もらえる? 
\\	明日、九校に迎えに来てもらえない?	
\\	もらえない? 
\\	明日からもっと早く家を出るようにします。	
\\	ようにする。 
\\	ようにする 
\\	ことにする 
\\	春休み何をしようかなあ。	
\\	〜かなあ。 
\\	日本語のクラスの外でも、なるべく日本語で話した方がいいよ。	
\\	"なるべく 
\\	なるべく 
\\	なるべく, 
\\	ケーキを作ろうとしたけれど、卵がなかったから、作れなかった。	
\\	とした(が/けれど)〜 
\\	宿題をしようとしたが、友達が来たから、するのをやめた。	
\\	とした(が/けれど)〜 
\\	電車に乗ろうとしたら、目の前でドアが閉まってしまった。	
\\	とした(が/けれど)〜 
\\	映画が好きなので、毎週のように映画を見に行っています。	
\\	"毎
\\	のように;
\\	日、週、月、年,回) 
\\	日本人は一般的に丁寧だと思われている。	
\\	と思われている。 
\\	思われている 
\\	日本の食べ物は体にいいと考えられている。	
\\	と考えられている。 
\\	と考えられている 
\\	今日、小テストがあるなんて、知らなかった。	
\\	なんて(、/は) 
\\	なんて 
\\	など(は)
\\	などということは。
\\	寿司が嫌い日本人などという人は聞いたことがない。	
\\	などという(もの/人/
\\	は 
\\	なんて 
\\	などという人は
\\	ローラーブレードなどというのは簡単だよ。	
\\	などという(もの/人/
\\	は 
\\	なんて 
\\	などというのは
\\	「すごい」や「かわいい」などという言葉は若い人達がよく使う表現だ。	
\\	などという(もの/人/
\\	は 
\\	なんて 
\\	などという人は
\\	こんなまずい料理をおいしいなどと言ったのは誰だ?	
\\	などと 
\\	などということを。 
\\	なんて 
\\	などは
\\	朝、起きたら、私わまずコーヒーを飲む。	
\\	"まず 
\\	この問題について、グループで話して合って下さい。	
\\	合う 
\\	大きいな災害(さいがい)の時は、みんなで助け合うことが大切だ。	
\\	合う 
\\	高校生の時、よく友達と写真を撮り合ったり見せ合ったり話し合ったり信じ合ったりして遊んだ。	
\\	合う 
\\	新しい車を買うためにお金をためています。	
\\	ために
\\	先生は、学生が分かるようにやさしい単語を使って説明した。	
\\	ように
\\	ように 
\\	みんなに聞こえるように、大きい声で話して下さい。	
\\	ように
\\	ように 
\\	朝寝坊をしないように、目覚まし時計をセットしておきます。	
\\	ように
\\	ように 
\\	昔、昔ある所に。。。 ある日、分かるようになった。 ある有名人に合いました。	
\\	ある
\\	ある所 
\\	ある日 
\\	ある有名人 
\\	地球温暖化(ちきゅうおんだんか:
\\	の問題はもっと大きくなっていくのではないでしょうか。	
\\	のではないでしょうか。 
\\	のではないでしょうか 
\\	ではないだろう 
\\	あのクラスは、毎日宿題があるので、大変のではないだろうか。	
\\	のではないだろうか。 
\\	のではないだろうか 
\\	ではないでしょうか。 
\\	このアパートは広いから、二人で住めるんじゃないかな。	
\\	んじゃないかな。 
\\	んじゃないかな 
\\	んではないでしょうか。
\\	今日、雨が降るんではないでしょうか。	
\\	んではないでしょうか。 
\\	んじゃないかな 
\\	んではないでしょうか
\\	困っている人がいたら迷わず助けてあげる。	
\\	ず(に) 
\\	ないで 
\\	ないで 
\\	に 
\\	辞書を見ずに新聞が読めるようになりたいです。	
\\	ず(に) 
\\	ないで 
\\	ないで 
\\	に 
\\	試験はペンで書かずに、鉛筆を使って下さい。	
\\	ず(に) 
\\	ないで 
\\	ないで 
\\	に 
\\	両親に相談せずに、入学することを決めてしまった。	
\\	(入学=
\\	ず(に) 
\\	する 
\\	せずに。 
\\	ないで 
\\	ないで 
\\	に 
\\	日本人は丁寧で親切だ。そういう話をよく聞く。	
\\	(そういう/こういう/ああいう)
\\	動物園でパンダの赤ちゃんが3匹生まれたそうだ。こういうニューズはうれしい。	
\\	(そういう/こういう/ああいう)
\\	こういう, 
\\	トムー。。。の関係って面白いね。 山田ーうん、ああいう関係って、他の国ではあまり見られないだろうね。	
\\	(そういう/こういう/ああいう)
\\	ああいう, 
\\	奈良は日本で一番歴史の古い町の一つと言えるでしょう。	
\\	と言えるでしょう。 
\\	日本語を勉強する外国人は多くなってきていると言えるだろう。	
\\	と言えるだろう。 
\\	バレンタインデーにチョコレートをくれたということは、彼女は君が好きだということなんだよ。	
\\	は
\\	(という)ことなのだ。 
\\	なのだ 
\\	である 
\\	だ。
\\	なの 
\\	なん 
\\	お金持ちになることは、幸せになれるということなのだろう。	
\\	は
\\	(という)ことなのだ。 
\\	なのだ 
\\	である 
\\	だ。
\\	大阪まで新幹線で行きたいんですが、いくらでしょうか。	
\\	〜ん(だけど/ですが) 
\\	んですが 
\\	んだけど。
\\	先生、この文法よく分からないんです。。。	
\\	〜ん(だけど/ですが) 
\\	んですが 
\\	んだけど。
\\	友達の誕生日にプレゼントを送りたいんだけど、何がいいと思う?	
\\	〜ん(だけど/ですが) 
\\	んですが 
\\	んだけど。
\\	子供の時、アニメが大好きだったんだ。それで、日本語を勉強しようと思ったんだ。	
\\	1。それで、
\\	それで 
\\	だから 
\\	トムは明日試験がある。だから今日のパーティーに来ないと思う。	
\\	だから、
\\	だから 
\\	それで 
\\	だから 
\\	世界中、どこに行ってもマクドナルドが食べられる。	
\\	も 
\\	ても 
\\	この漢字何回覚えても、すぐに忘れてしまう。	
\\	も 
\\	ても 
\\	この問題は、いくら考えても分からない。	
\\	も 
\\	ても 
\\	試合に負けてどんなに悔しくても、泣いてはいけない。	
\\	も 
\\	ても 
\\	彼はギターがとても上手だ。どんな曲でも弾ける。	
\\	でも 
\\	ても 
\\	どんな、どこ、いくら、誰、いつ、何
\\	日本にいるうちに、色々な所に旅行に行きたい。	
\\	うちに 
\\	間に 
\\	うちに 
\\	ない 
\\	熱いうちに、どうぞ召し上がって下さい。	
\\	うちに 
\\	間に 
\\	うちに 
\\	ない 
\\	暑くならないうちに、犬の散歩をしてきた方がいいよ。	
\\	うちに 
\\	間に 
\\	うちに 
\\	ない 
\\	子供のうちに外国語を勉強すると発音がよくなるそうだ。	
\\	のうちに 
\\	間に 
\\	うちに 
\\	ない 
\\	できれば医者になりたいが、授業料が高いので難しいかもしれない。	
\\	"できれば/できたら 
\\	できれば 
\\	できたら。
\\	スミスさん、できれば、この英語、直してくれると助かるんだけど。	
\\	"できれば/できたら 
\\	できれば 
\\	できたら。
\\	山田さん、できたら、果物を持って来てくれない。	
\\	"できれば/できたら 
\\	できれば 
\\	できたら。
\\	いま、食べたばかりですから, お腹がいっぱいで、何も食べられません。	
\\	ばかりだ。 
\\	ばかりの。
\\	先週、一年生に成ったばかりの学生のためのオリエンテーションがあった。	
\\	ばかりの 
\\	ばかりの.
\\	一日当たり5千円もらったことになる。	
\\	当たり
\\	一時間当たり10ドル	
\\	当たり
\\	一人当たりのコスト	
\\	当たりの
\\	背が高いのは弟の方だ。でも、足が速いのは僕の方だ。	
\\	のは
\\	の方だ。 
\\	あのレストランは静かだ。でも、食べ物がおいしいのはこのレストランの方だと思う。	
\\	のは
\\	の方だ。 
\\	一週間に一回はプールで泳ぐようにしている。	
\\	は 
\\	は 
\\	3本 
\\	少し), 
\\	仕事が忙しくても、昼ご飯だけは食べて下さいね。	
\\	(だけ/ぐらい/くらい)は 
\\	は 
\\	3本 
\\	少し), 
\\	だけは、ぐらいは、
\\	くらいは 
\\	日本には、本州をはじめ四つの大きな島がある。	
\\	をはじめ 
\\	もうこれ以上は食べられません。	
\\	以上 
\\	18歳以上の人しか見られません。	
\\	以上 
\\	プレゼントは50ドル以下の物にしようと思う。	
\\	以下 
\\	コースを開くためには、学生が五人以上必要です。それ以下の場合は、キャンセルです。	
\\	以下 
\\	以下 
\\	私は、愛さえあれば幸せです。	
\\	さえ 
\\	さえ 
\\	が 
\\	を, 
\\	いい友達さえいれば、他に欲しいものありません。	
\\	さえ 
\\	さえ 
\\	が 
\\	を, 
\\	雨さえ降らなければ、車の運転は怖くない。	
\\	さえ 
\\	さえ 
\\	が 
\\	を, 
\\	20年も乗っていた車がついにうごくなってしまった。	
\\	ついに 
\\	日本の有名な小説がハリウッドで映画化させることになった。	
\\	化する 
\\	化する 
\\	この大学は外国人が多くて、国際化が進んでいる。	
\\	化 
\\	化 
\\	最適化 一般化	さいてきか:
\\	いっぱんか:
\\	化 
\\	化 
\\	先生のオフィスアワーに研究室に行きました。ところが、先生はいらしゃらなかった。	
\\	1。ところが 
\\	ところが 
\\	ところが
\\	でも 
\\	スミスさんは日本に10年も住んでいたから、日本語が上手に話せるに違いない。	
\\	〜に違いない 
\\	日本のマンガをもとにしてたくさんのアニメやゲームが作られている。	
\\	をもとに(して) 
\\	して 
\\	これは、私の子供の時の経験をもとにした小説です。	
\\	をもとにした 
\\	母と同じで、私もえびのアレルギーがあるので、えびはたべられないんです。	
\\	と同じで 
\\	スノーボードはスキーと違って、スティックを使わずにすべります。	
\\	は
\\	と違っう/と同じだ/ 似ている 
\\	多くの国に違って、日本では20歳にならないと投票できない。	
\\	と違って 
\\	田中さんは長い間アメリカに住んでいたので、あまり日本人らしくない。	
\\	〜らしくない/らしくありません 
\\	「男らしい」とか「女らしい」という言葉は、あまり好きではありません。誰でも、その人らしく生きればいいのではないでしょうだ。	
\\	らしい 
\\	は
\\	らしい 
\\	らしい 
\\	らしい 
\\	らしく 
\\	アメリカ的;歴史的な 文学的な;国際的 一般的;	
\\	的 
\\	な-
\\	な 
\\	漢字がもっと簡単に覚えられたらいいのになあ。	
\\	らいいのになあ。 
\\	なあ 
\\	量は多かったけれど、お腹がすいていたから、ほとんど食べてしまった。	
\\	"ほとんど 
\\	おいしい料理だったが、お腹が痛くてほとんど食べられなかった。	
\\	ほとんど〜ない。 
\\	ほとんどの人が彼のことを知っている。	
\\	ほとんどの 
\\	先生の説明を聞いて、分かったような気がしたけど、家に帰ってもう一度勉強し直したら、また分からなくなってしまった。	
\\	~(ような)気がする 
\\	ような 
\\	誰かがドアをノックしたような気がするから、ちょっと見てくれない?	
\\	~(ような)気がする 
\\	ような 
\\	これから公園へサッカーをしに行くところです。	
\\	ところ 
\\	話すところ	
\\	ところ 
\\	ところ 
\\	食べているところ 考えているところ 電話をかけているところ	
\\	いるところ 
\\	ところ 
\\	ちょうど勉強が終わったところに友達から電話がかかってきた。	
\\	ところ 
\\	ところ 
\\	ところに 
\\	試験が終わってから、答えを間違えたことに気がつけました。	
\\	ことに気がつく 
\\	こと 
\\	なかなか面白いことにき気がつきました。	
\\	に気がつく。 
\\	こと 
\\	こと 
\\	子供達が幸せな人生が送れるようにと願っています。	
\\	ように(と)願う 
\\	弟が大学の入学試験に合格できるようにと毎日祈っています。	
\\	ように(と)祈る 
\\	寿司が好きな人もいるし、嫌いな人もいる。	
\\	1も 
\\	し 
\\	2も
\\	とびら)
\\	この教科書は、会話も練習できるし、文化も学べます。	
\\	1も 
\\	し、
\\	2も 
\\	とびら
\\	寿司好きな人もいれば、嫌いな人もいます。	
\\	1も 
\\	も 
\\	この教科書は、会話も練習できれば、文化も学べます。	
\\	1も 
\\	も 
\\	漢字を勉強する時いつも、如何してこんなたくさん漢字があるのだろうかと思う。	
\\	の(だろう/でしょう)か 
\\	絶対に怒らない人がいるのだろうか。私は絶対にそんな人はいないと思う。	
\\	(そんな/あんな/こんな)
\\	そんな/あんな: 
\\	あんな: 
\\	そういう、ああいう、こういう 
\\	学校が終わって、子供達はそれぞれ自分の家に帰って行った。	
\\	"それぞれ 
\\	それぞれ 
\\	それぞれの国には、それぞれの文化や習慣がある。	
\\	"それぞれ の
\\	それぞれ 
\\	来月山田先生が結婚するらしいですよ。	
\\	〜らしい。 
\\	らしい 
\\	先輩によると、あの先生の授業は大変らしい。	
\\	〜らしい。 
\\	によると。 
\\	外国語が上手になりたかったら、あきらめないで毎日勉強し続けることが大切です。	
\\	続く 
\\	たけしさんはコンピュータゲームばかりしていて全然勉強しないそうです。	
\\	ばかりする 
\\	ばかり 
\\	田中さんはマンガばかり読んでいる。 
\\	ばっかり 
\\	ばかり
\\	ルームメイトは、毎日寝てばかりいる。授業に行かなくてもいいのかなあ。	
\\	ばかりいる。 
\\	ばかり 
\\	田中さんはマンガばかり読んでいる。 
\\	ばっかり 
\\	ばかり
\\	田中さんとばかり遊ぶ。	
\\	ばかり 
\\	ばかりいる。 
\\	ばかり 
\\	田中さんはマンガばかり読んでいる。 
\\	ばっかり 
\\	ばかり
\\	「真っ」の発音 真っ赤 真っ青 真北 真っ昼間 真下 真夜中	まっか:
\\	まっさお:
\\	まきた:
\\	まっぴるま:
\\	ました: 
\\	まよなか: 
\\	真っ
\\	真っ
\\	真っ
\\	真っ 
\\	東京は電車矢地下鉄やバスが便利な街なので、車がなくても生活できると言うわけです。	
\\	~わけだ。 
\\	このレストランの料理がおいしくない。 
\\	行きたくないわけですね。	
\\	~わけだ。 
\\	あっ、ヒーターが止まっている。寒いわけだ。	
\\	"〜わけだ。 
\\	今日は忙しいけど、来週の日曜日でよければ、一緒に行けるよ。	
\\	でよければ 
\\	この辞書のサイトは結構便利なので、よく使う。	
\\	結構 
\\	だ) 
\\	結構 
\\	結構上手に書く; 結構誰とでも話す; 結構昔からある	
\\	結構 
\\	結構 
\\	この携帯電話のコマーシャルはなかなか面白い。	
\\	なかなか 
\\	だ) 
\\	なかなか 
\\	結構 
\\	なかなか 
\\	つまらない 
\\	不便. 
\\	なかなか 
\\	これはなかなか上手に描けたと思う。	
\\	なかなか 
\\	なかなか 
\\	結構 
\\	なかなか 
\\	つまらない 
\\	不便. 
\\	なかなか 
\\	ねえ、あの人、どこかで見たことない? 
\\	そう言えば、昔、どこかで会ったことがあるような気がする。	
\\	そう言えば 
\\	この本を読むと、昔のギリシャにも日本のように色々な神様がいたということがよく分かる。	
\\	ということ(が、を、
\\	が、を, 
\\	この映画は午後7時からとなっています。	
\\	は
\\	となっている。 
\\	は
\\	だ 
\\	そうですか。じゃ、前の日の金曜日に、研究室に受けに来て下さい。	
\\	「では/それでは」
\\	では 
\\	それでは 
\\	それ 
\\	では 
\\	それでは 
\\	それじゃ 
\\	じゃ 
\\	はい。それでは、金曜日の先生の研究室にうかがいます。	
\\	「では/それでは」
\\	では 
\\	それでは 
\\	それ 
\\	では 
\\	それでは 
\\	それじゃ 
\\	じゃ 
\\	運動は体にいい。その上、心の健康にもいい。	
\\	その上、
\\	その上 
\\	も 
\\	まで。 
\\	忙しくて、寝る時間さえありません。	
\\	さえ 
\\	さえ 
\\	さえ 
\\	が 
\\	を 
\\	へ 
\\	に 
\\	に, で、と 
\\	から 
\\	南極にさえ行ったことがある。	
\\	さえ 
\\	さえ 
\\	さえ 
\\	が 
\\	を 
\\	へ 
\\	に 
\\	に, で、と 
\\	から 
\\	このサイトは携帯電話向けだから、コンピュータでは見にくい。	
\\	向けだ 
\\	向け 
\\	向け 
\\	子供向けに作られた映画。	
\\	向けに 
\\	向け 
\\	向け 
\\	ビジネス向けの記事が多い。	
\\	向けの 
\\	向け 
\\	向け 
\\	こんなにたくさんの人が読んでいるのだから、面白い本であるに違いない。	
\\	〜である/~ではない 
\\	だ 
\\	-な/-の 
\\	である 
\\	であります/ではありません 
\\	である 
\\	だ 
\\	あの話は本当ではないと言われています。	
\\	〜である/~ではない 
\\	だ 
\\	-な/-の 
\\	である 
\\	であります/ではありません 
\\	である 
\\	だ 
\\	子供の頃嫌いであった食べ物が、大人になって好きになるという話をよく聞く。	
\\	{〜である/ ~であった} 
\\	{~ではない/~ではなかった} である 
\\	だ 
\\	-な/-の 
\\	である 
\\	であります/ではありません 
\\	である 
\\	だ 
\\	私の日本語の先生である田中先生。	
\\	{〜である/ ~であった} 
\\	{~ではない/~ではなかった} である 
\\	だ 
\\	-な/-の 
\\	である 
\\	であります/ではありません 
\\	である 
\\	だ 
\\	新型ではないケータイを使っているので、ケータイで
\\	メールが出来ない。	
\\	{である/であった/ではない/ではなかった} 
\\	である 
\\	だ 
\\	-な/-の 
\\	である 
\\	であります/ではありません 
\\	である 
\\	だ 
\\	生み出す 見つけ出す 掘り出す	うみだす 
\\	みつけだすー 
\\	ほりだす- 
\\	出す 
\\	出す 
\\	始める 
\\	出す 
\\	始める 
\\	出す 
\\	笑い出す 動き出す	わらいだす-
\\	うごきだす- 
\\	出す 
\\	出す 
\\	始める 
\\	出す 
\\	始める 
\\	出す 
\\	アニメが好きな若者で、宮崎駿のアニメを見たことがない人はいない。	
\\	ない 
\\	は{ない/いない} 
\\	ない
\\	は(い)ない 
\\	日本の環境省はアメリカの 
\\	に当たる。	
\\	かんきょうしょう 
\\	は
\\	に当たる。 
\\	英語の
\\	に当たる言葉は、日本語の場合、会社では「社長」、大学では「学長」、米国政府では「大統領」です。	
\\	""しゃちょう” 
\\	""がくちょう” 
\\	""だいとうりょう” 
\\	に当たる
\\	は
\\	だ。 
\\	メールが戻って来たということは、このアドレスはもう使われていないということになる。	
\\	(という)ことになる 
\\	家賃が上がったので、引っ越さなければならないことになってしまった。	
\\	(という)ことになる 
\\	今のうちに勉強しておかないと、後で卒業できないということになるかもしれない。	
\\	(という)ことになる 
\\	日本の学生は高校まではよく勉強するが、大学に入ってからはあまり勉強しない傾向がある。	
\\	(という)傾向がある。 
\\	という 
\\	若い人の方が、年を取った人より、ファーストフードをよく食べるという傾向が見られる。	
\\	(という)傾向が見られる。 
\\	という 
\\	とする。	
\\	もし、タイムマーシンがあるとしたら、どの時代に行って、誰に会ってみたいですか。	
\\	"(もし)
\\	としたら 
\\	としたら 
\\	もし 
\\	この問題が分からないとしたら、それはあまり勉強しなかったということですよ。	
\\	"(もし)
\\	としたら 
\\	としたら 
\\	もし 
\\	机がガタガタして、書きにくい。	
\\	が 
\\	する。 
\\	歯がシクシクする。	
\\	が 
\\	する。 
\\	心がウキウキする。	
\\	が 
\\	する。 
\\	指がズキズキする。	
\\	が 
\\	する。 
\\	兄は自分で料理しないくせに、いつも私の作ったものに文句を言う。	
\\	〜くせに 
\\	くせに 
\\	くせに 
\\	くせに 
\\	なのに、.
\\	書いてる	かいている ~てる 
\\	~ている	
\\	飲んでます	のんでいます 〜てますー
\\	〜ています	
\\	読んでた	よんでいた 〜てたー
\\	〜ていた	
\\	この部屋じゃ小さすぎて、。。。	では 〜じゃー
\\	〜では	
\\	早く起きなくちゃ。。。	起きなくては(いけない/ならない) 〜なくちゃー
\\	〜なくては
\\	話さなきゃ	話さなければ(いけない/ならない)	
\\	忘れちゃう	忘れてしまう	
\\	食べちゃった	食べてしまった	
\\	買っといた	買っておいた 
\\	買っておく)	
\\	寝とく	寝ておく	
\\	宿題も出しといたげる	宿題も出しておいてあげる	
\\	説明したげる	説明してあげる	
\\	出したげた	出してあげた	
\\	あの人が嫌いなわけではないんですが、デートしたいとは思いません。	
\\	{な/だった/じゃない/じゃなかった} わけではない 
\\	この料理はまずいわけではないけど、油が多いから、あまり食べられない。	
\\	(という)わけではない。 
\\	ネコというわけではない	
\\	{という/だった/じゃない/じゃなかった} わけではない 
\\	経済は政治と深い関係がある。	
\\	は 
\\	{と/に}関係がある。 
\\	と 
\\	に。 
\\	の 
\\	関係 
\\	が。
\\	そのトピックに関係のある本はこの列にあります。	
\\	{と/に} 関係 {が/の} ある
\\	と 
\\	に。 
\\	の 
\\	関係 
\\	が。
\\	授業に関係のない質問はしないようにしましょう。	
\\	{と/に} 関係 {が/の} ある
\\	と 
\\	に。 
\\	の 
\\	関係 
\\	が。
\\	それは、僕には関係がないから、知らないなあ。	
\\	は 
\\	{と/に}関係がある。 
\\	と 
\\	に。 
\\	の 
\\	関係 
\\	が。
\\	ハイブリッドの車が増えたことによって、町の空気がきれいなってきました。	
\\	"~ことによって によって 
\\	インターネットで調べることによって、世界中で今起こっていることを知ることが出来る。	
\\	"~ことによって によって 
\\	万有引力の法則はニュートンによって発見されました。	
\\	によって によって 
\\	青と黄色の絵の具を混ぜます。すると、緑色になります。	
\\	すると、
\\	すると 
\\	このアイコンをクリックして下さい。すると、コンピュータの画面の文字が大きくなります。	
\\	すると、
\\	すると 
\\	田中さんは、明日の会議に出られないそうですよ。 
\\	すると、誰か他の人が出なくてはいけないということですね。	
\\	すると、
\\	すると 
\\	空手部では、先輩に言われた通りにしないと、怒られてしまう。	
\\	通り(に) 
\\	通り 
\\	通り 
\\	通り 
\\	とおり 
\\	どおり。
\\	日本のファミリーレストランの店員は、みんなマニュアルの通りに話すから、ロボットみたいだ。	
\\	の 通り(に) 
\\	通り 
\\	通り 
\\	通り 
\\	とおり 
\\	どおり。
\\	指示通りに、ここに答えを書いて下さい。	
\\	しじ どおり	
\\	通り(に) 
\\	通り 
\\	どおり。 
\\	通り 
\\	通り 
\\	通り 
\\	とおり 
\\	どおり。
\\	外国語は話せば話すほど、上手になります。そして、読めば読むほど、単語が増えます。	
\\	ほど 
\\	運動すればするほど、健康になれるし、やせられるから、運動は一石二鳥ですね。	
\\	ほど 
\\	アパートは駅に近ければ近いほど家賃が高くなり、逆に、駅から遠ければ遠いほど安くなる。	
\\	ほど 
\\	親が立派であればあるほど、子供はプレッシャーを感じてしまうようだ。	
\\	であれば 
\\	で)あるほど 
\\	ほど)
\\	いい大学であればあるほど入るのが難しいです。	
\\	であれば 
\\	で)あるほど 
\\	ほど)
\\	便利なら便利なほど...	
\\	なら
\\	なほど 
\\	ある日、うさぎとかめが競走をしました。さて、どちらがレースに勝ったでしょうか。	
\\	さて、
\\	さて 
\\	さて
\\	さて、もう遅いですだら、今日はこれで失礼します。	
\\	さて、
\\	さて 
\\	さて
\\	古代のヨーロッパはローマを中心として発展した。	
\\	を 中心 {と/に} (して) 
\\	を中心(と/に)する 
\\	地球や火星、木星、金星などの惑星は、太陽を中心として回っている	
\\	を 中心 {と/に} (して) 
\\	を中心(と/に)する 
\\	私は、最近、日本の若者言語を中心とした言語の研究をしています。	
\\	1を中心(と/に)した 
\\	1中心の
\\	1を中心(と/に)する 
\\	日本と私の国は、朝と夜が逆だ。いま、こちらは午後9時だが、日本は午前10時だ。	
\\	は 逆 だ。 
\\	逆 
\\	ぎゃく
\\	日本語で数字を100から逆に数えてみて下さい。	
\\	"逆 に 
\\	逆 
\\	ぎゃく
\\	一方通行の道だということを知らないで、逆の方向に進んでしまい、警察につかまってしまった。	
\\	"逆 の 
\\	逆 
\\	ぎゃく
\\	その本なら、大学の図書館に行けば、あるはずだよ。	
\\	はず 
\\	{な/じゃない/だった/じゃなかった) 
\\	{の/じゃない/だった/じゃなかった} 
\\	はず 
\\	はず 
\\	〜ないはずだ 
\\	〜はずがない.
\\	はずがない. 
\\	はずだ 
\\	の 
\\	はず 
\\	この点について、もう一度、説明していただきませんか。	
\\	点 
\\	点 
\\	サービスという点では、日本のデパートは最高だ。	
\\	{という/の} 点で (は) 
\\	点 
\\	この車はガソリンがなくても走れるという点で環境
\\	にいいですが、値段が高いです。	
\\	という点で 
\\	点 
\\	この留学プログラムは、ホームステイが出来るという点が、セールスポイントですね。	
\\	(という)点が 
\\	点 
\\	漢字を覚えば覚えほど、単語の数がますます増えて、ますます日本語が分かるようになります。	
\\	"ますます
\\	ますます 
\\	ますます 
\\	もっともっと 
\\	もっともっとがんばって下さい。
\\	台風が近づいてきたので、風がますます強くなってきた。	
\\	"ますます 
\\	くなる 
\\	ますます 
\\	ますます 
\\	もっともっと 
\\	もっともっとがんばって下さい。
\\	話すのは速すぎので、ますますゆっくり話すようにする。	
\\	"ますます 
\\	ますます 
\\	ますます 
\\	もっともっと 
\\	もっともっとがんばって下さい。
\\	この作文は、コンピュータで書き直して、来週の月曜日に出して下さい。	
\\	直す 
\\	直す, 
\\	作文を書いたことは書いたけれど、まだ間違いがたくさんあるから書き直さなくてはいけない。	
\\	ことは 
\\	(が/けれど) 
\\	私はテニスをすることはしますが、あまり上手じゃありません。	
\\	ことは 
\\	(が/けれど) 
\\	このアパートは駅に近くて便利なことは便利だけれど、家賃はとても高いです。	
\\	{な/だった} ことは 
\\	{だ/だった} が 
\\	あの人は、いい人はいい人だけど、ちょっとがんこ
\\	だね。	
\\	は 
\\	{だ/だった}が 
\\	{だった}ことは 
\\	{だ/だった} が] 
\\	友達が手伝ってくれたおかげで、仕事が早く終わった。	
\\	おかげ 
\\	おかげ 
\\	せい 
\\	ハンサムなせいで、うそを信じて、ばかなことをしました。	
\\	{な/じゃない/だった/じゃなかった} せい/おかげ 
\\	おかげ 
\\	せい 
\\	おかげ 
\\	せい 
\\	シートベルトのおかげで、大丈夫でした。	
\\	{の/じゃない/だった/じゃなかった} おかげ 
\\	おかげ 
\\	せい 
\\	おかげ 
\\	せい 
\\	ゆうべ飲み過ぎたせいで、朝から頭がガンガンする。	
\\	せい 
\\	おかげ 
\\	せい 
\\	犬のせいじゃない!	
\\	{の/じゃない/だった/じゃなかった} せい 
\\	おかげ 
\\	せい 
\\	おかげ 
\\	せい 
\\	日食
\\	を見ているうちに、どんどん空が暗くなってきて、ついに真っ暗になってしまった。	
\\	"どんどん-
\\	どんどん 
\\	ますます
\\	どんどん 
\\	ますます 
\\	私はもう5回も日本に行っています。行くたびに、新しいことを学んで帰ります。	
\\	たびに 
\\	たびに 
\\	たびに 
\\	彼は、デートのたびにおいしいレストランに連れて行ってくれる。	
\\	のたびに 
\\	たびに 
\\	たびに 
\\	15年も飼っていた犬が、年を取ってとうとう死んでしまったので、とても悲しい。	
\\	"とうとう 
\\	とうとう 
\\	ついに 
\\	とうとう 
\\	とうとう. ついに 
\\	この漢字の書き方は、覚えても覚えなくてもいいです。	
\\	も 
\\	なくても 
\\	犬は外でトイレをするから、天気がよくてもよくなくても、毎日散歩に連れて行かなくてはいけない。	
\\	も 
\\	なくても 
\\	この映画は子供でも子供 {じゃ/で} なくても、楽しめます。	
\\	でも 
\\	{じゃ/で} なくても 
\\	このアパートは家賃が安いから、静かでも静か {じゃ/で} なくても、借りようと思っています。	
\\	でも 
\\	{じゃ/で} なくても 
\\	日本の代表的な伝統系能に、能と狂言がある。前者は悲劇で後者は喜劇だ。	
\\	"前者は(〜)
\\	で/
\\	で/}、後者は〜 
\\	この大学には、韓国、中国、台湾、といったアジアの国々からの留学生がたくさんいる。	
\\	{や/、} 
\\	{や/、} といった 
\\	といった 
\\	といった 
\\	などの
\\	のような 
\\	日本語が上手に話せるようになるには、丁寧な話し方やくだけた話し方、敬語といったスピーチレベルが違う話し方の練習もしなければならない。	
\\	といった 
\\	といった 
\\	といった 
\\	などの
\\	のような 
\\	日本に留学したいと思う一方、外国に住むことに不安もある。	
\\	一方(で)、
\\	一方 
\\	一方 
\\	勉強が忙しい一方で、学費のためにアルバートもしなければならないから、毎日本当に大変です。	
\\	一方(で)、
\\	一方 
\\	一方 
\\	日本語を勉強してみたいと思う。一方、フランス語もいいね。	
\\	一方(で) 、
\\	一方 
\\	一方 
\\	将来は、ヨーロッパ、あるいは、アジアで仕事ができたらいいと思っている。	
\\	2、〜)あるいは 
\\	あるいは 
\\	メールを出すか、あるいは、電話をするか、どちらでもいいが、とにかく連絡しなくてはいけない。	
\\	か、あるいは、
\\	か 
\\	あるいは 
\\	バスがなかなか来ない。授業に遅れてしまいそうだ。	
\\	"なかなか 
\\	ない 
\\	なかなか 
\\	来週、母の弟、つまり、叔父が中国から遊びに来ます。	
\\	つまり、
\\	つまり
\\	登校拒否というのは、つまり、いじめなど理由で学校に行かなくなるという意味です。	
\\	(というの)は、つまり〜{だ/(という)ことだ} 
\\	つまり
\\	将来、小学校の先生になりたいと思っています。そこで、ボランティアで子供達を教えてみることにしました。	
\\	そこで, 
\\	そこで 
\\	それで 
\\	そこで そこで 
\\	それで 
\\	寒かった。それで、見に来た人が少なかった。 
\\	それで 
\\	私はウニ以外の寿司は、何でも食べられる。	
\\	以外の 
\\	以外の
\\	の
\\	それ以外の日	
\\	以外の 
\\	以外の
\\	の
\\	週末は、勉強以外に、テニスをしたり映画を見たりします。	
\\	以外に 
\\	以外に 
\\	に 
\\	どんなに大変でも、日本語の勉強は決してやめない。	
\\	"決して
\\	ない/なかった} 決して 
\\	{ません/ませんでした} 
\\	今日の試験は決してやさしくありませんでしたが、みんなよくできていました。	
\\	決して 
\\	{ない/なっかた/ありません/ありませんでした} 
\\	彼は決して悪い人間ではない。ちょっとわがままだけだ。	
\\	決して
\\	{では/じゃ} 
\\	ない/なかった/ありません/ありませんでした}
\\	アルバイトで忙しいという理由で、学校を休む学生がいます。	
\\	という理由で 
\\	これらの問題についてもっと話し合ってみる必要がある。	
\\	ら 
\\	ら 
\\	彼らが言っていることはよく分からない。	
\\	ら 
\\	ら 
\\	男の子が興味を持つオタク文化に対して、カワイイ文化は女の子に人気があるようだ。	
\\	に対して 
\\	に対して 
\\	日本語は文法が難しいのに対して、中国語は発音が難しい。	
\\	のに対して 
\\	に対して 
\\	私の知っている限り、外国人が使いやすい日本語の辞書はないんですよ。	
\\	限り 
\\	私はこの仕事が終わらない限り、家に帰ることが出来ない。	
\\	限り 
\\	本当である限り 本当でない限り	
\\	で {ある/ない} 
\\	限り 
\\	雨が強く降ってきたにもかかわらず、サッカーの試合は続けられた。	
\\	にもかかわらず 
\\	{な/じゃない/だった/じゃなかった} 
\\	{の/じゃない/だった/じゃなかった} 
\\	せっかく日本語を3年も勉強したのだから、ぜひ一度、日本に行ってみたいです。	
\\	"せっかく 
\\	{のだから/のに} 
\\	せっかく 
\\	せっかくケーキを焼いたのに、誰も食べてくれなかったから、がっかりした。	
\\	"せっかく 
\\	{のだから/のに} 
\\	せっかく 
\\	せっかくの野球の試合が雨で中止になってしまった。	
\\	"せっかくの
\\	せっかく 
\\	せっかくですが、今日は仕事が忙しくて。	
\\	"せっかく{ですが/ですけど/だけど/
\\	せっかく 
\\	春と言っても、寒くてまだセーターを着ている。	
\\	{だ/じゃない/だった/じゃなかった} {と/って} 言っても 
\\	この文法は難しいと言っても、全部分からないわけではない。	
\\	{と/って} 言っても 
\\	って 
\\	と 
\\	予定が思うように進まないので、とても困っています。	
\\	"思うように〜ない 
\\	私はチョコレートが死ぬほど好きだ。毎日でも食べる。	
\\	ほど 
\\	ほど 
\\	ほど, 
\\	ほど 
\\	私は日本語が上手に話せることは話せるが、スミスさんほどではない。	
\\	のほど 
\\	ほど 
\\	ほど, 
\\	ほど 
\\	これほどすばらしい絵を見たことがない。	
\\	ほど 
\\	ほど 
\\	ほど, 
\\	ほど 
\\	1年に700万人以上の人が訪れるという。	
\\	という 
\\	子供には、子供の年齢に会った本を与えた方がいいですよ。	
\\	に会った 
\\	病状に会わせて、いろいろな薬が使われる。	
\\	に会わせて 
\\	冬の間、カエルは、まるで死んだように動きません。	
\\	"まるで 
\\	ように 
\\	まるで 
\\	まだ春なのに、今日はまるで夏のように暑い。	
\\	"まるで 
\\	{の/じゃない/だった/じゃなかった} ように 
\\	まるで 
\\	あの小娘は、まるで子猫のようだ。	
\\	〜ようだ。 
\\	まるで 
\\	まるで天子のような女。	
\\	〜ような
\\	まるで 
\\	明日の試験では、言うまでもなく、文法の問題が出ます。	
\\	は)言うまでもなく、
\\	図書館では静かにしなくてはいけないことは、言うまでもない。	
\\	{こと/の} は言うまでもない。 
\\	世界で人気がある日本ポップカルチャーには、アニメは言うまでもなく、ファッションや音楽などもある。	
\\	は言うまでもなく 
\\	も 
\\	日本のトイレには、お湯がお尻を洗うトイレから、ふたが自動で開くトイレまでもある。	
\\	から 
\\	までも 
\\	までも 
\\	この教科書では、漢字や文法だけでなく、日本の現代の文化から歴史までも学べる。	
\\	だけでなく 
\\	まで(も) 
\\	までも 
\\	最近では、歩くだけでなく、サッカーをするロボットまである。	
\\	まで(も) 
\\	までも 
\\	このゲームは、子供でも大人でも楽しむことが出来ます。	
\\	でも 
\\	でも 
\\	も
\\	も 
\\	コーヒーにはたくさんカフェインが入っている。したがって、飲み過ぎると寝られなくなる。	
\\	したがって、
\\	私が彼のことが好きだというのは事実だが、将来彼と結婚しようと思っているわけではない。	
\\	(という)のは事実だ。 
\\	という 
\\	面白さという点から考えると、この小説はあまり面白くないが、でも、歴史的なことが分かるので勉強になる。	
\\	という点から考えると 
\\	このホテルは静かな上に、サービスがとてもいい。	
\\	上に 
\\	上に 
\\	嬉しいことに、奨学金がもらえることになった。	
\\	ことに 
\\	ことに 
\\	ことに 
\\	残念なことに、私達のチームは最後の試合で負けてしまった。	
\\	なことに 
\\	ことに 
\\	ことに 
\\	驚いたことに、この鳥は人間の言葉が話せるらしい。	
\\	ことに 
\\	ことに 
\\	ことに 
\\	腹の立つことには、私、彼が好きです。	
\\	ことに 
\\	ことに 
\\	ことに 
\\	私は日本文学を研究したいと思っている。だが、両親は私に医者になってほしがっちいる。	
\\	だが、
\\	だが 
\\	でも、けれど、けど 
\\	けれども、しかしながら、だが、
\\	が 
\\	しかし 
\\	日本語では、目上の人に対して「あなた」と呼ぶのはとても失礼なことだ。	
\\	に対して 
\\	に対して 
\\	日本語や中国語を勉強して、アジアの国々に対する見方が変わった。	
\\	に対する 
\\	に対して 
\\	に対して 
\\	よく勉強したんだから、100点を取っても、驚くことないよ。	
\\	すること(は)ない 
\\	は 
\\	スミスさんは、日本語が上手だ。なぜかというと、日本に住んでいたからだ。	
\\	"{どうして/なぜ} かというと 〜から {だ/です/
\\	どうして/なぜ 
\\	から だ/です。
\\	私は辛い食べ物が好きだから、どちらというと日本語料理より、韓国料理の方が好きだ。	
\\	"どちらかというと 
\\	次に発表する人は、誰かというと、田中さんです。	
\\	"{いつ/だれ/どのぐらい} かというと 
\\	かというと 
\\	頭が痛いです。 
\\	それなら、すぐに寝た方がいいよ。	
\\	"それなら 
\\	それ 
\\	漢字が難しいと思っている学生は少なくない。	
\\	{は/が/も} 少なくない 
\\	少なくない 
\\	少なくない人 
\\	インタネットを通して、色々な情報を得ることが出来る。	
\\	を通して 
\\	を通して 
\\	昨夜から今朝にかけて、たくさん雨が降った。	
\\	から 
\\	にかけて 
\\	ここから隣の町にかけて、スーパーやレストランが並んでいるので、道がとても混む。	
\\	から 
\\	にかけて 
\\	今の気温は華氏 
\\	で50度だから、摂氏で言うと10度ぐらいだ。	
\\	で言うと 
\\	各国の代表者が集めって、テロの犯罪について会議を開いた。	
\\	"各 
\\	各 
\\	ここには各先生のメールアドレスと、各セクションの教室の番号が書いてあります	
\\	"各 
\\	各 
\\	あのレストランは、安いばかりでなく、とてもおいしい。	
\\	ばかりでなく 
\\	だけでなく
\\	(も). 
\\	京都には日本内からばかりなく、世界中から観光客が来る。	
\\	ばかりでなく 
\\	(も) 
\\	だけでなく
\\	(も). 
\\	このさむらい風の人形は、いくらですか。	
\\	風の 
\\	風 
\\	風 
\\	このドレスは着物風だ。	
\\	風だ。 
\\	風 
\\	風 
\\	言葉は時代と共に変化する。	
\\	と共に 
\\	みんなが同時に話さないで、一人ずつ順番話して下さい。	
\\	ずつ 
\\	ずつ 
\\	ホテルのビュッフェで色々な種類の料理をちょっとずつ食べてみた。	
\\	ずつ 
\\	ずつ 
\\	このゲームは子供から大人に至るまで、誰でも楽しめます。	
\\	から 
\\	に至まで 
\\	友達同士で話し合って下さい。	
\\	同士 {が/で/を} 
\\	自分で出来ることば、自分ですべきだ。	
\\	べき {だ/じゃない/ではない/だった/じゃなかった/ではなかった} 
\\	べき 
\\	する 
\\	べき 
\\	す 
\\	する, 
\\	すべき 
\\	するべき。
\\	空の様子からすると、夕方には雨になるようだ。	
\\	からすると 
\\	思わぬ結果になって、びっくりしている。	
\\	ぬ 
\\	ぬ 
\\	ない 
\\	日本の国土は狭そうに思われているが、数字の上で見ると、そうでもない。	
\\	のうえで 
\\	の上で 
\\	上 
\\	翻訳の上での間違いが、お互いの国の間に誤解を生んでしまった。	
\\	の上での 
\\	の上で 
\\	上 
\\	田中さんは、健康上の問題があって、仕事を辞めた。	
\\	上の 
\\	の上で 
\\	上 
\\	日本の歴史上、明治維新は大きい意味を持っている。	
\\	上 
\\	の上で 
\\	上 
\\	奈良は京都と並んで、古い歴史のある町で、日本の首都だたこともある。	
\\	と並んで 
\\	おそらく来年は、日本に行けるだろう。	
\\	"おそらく〜だろう 
\\	おそらく 
\\	だろう 
\\	でしょう。
\\	彼と私の考えはまったく同じだ。	
\\	"まったく 
\\	/(の)
\\	まったく〜(という)わけではない 
\\	このレポートはまったくだめだ。	
\\	"まったく 
\\	/(の)
\\	まったく〜(という)わけではない 
\\	このカレーは辛すぎて、まったく食べられない。	
\\	"まったく〜ない 
\\	まったく〜(という)わけではない 
\\	納豆は嫌いだが、まったく食べられないというわけではない。	
\\	"まったく〜(という)わけではない 
\\	まったく〜(という)わけではない 
\\	いったいあなたは何を考えてそんなことをしたんですか。	
\\	"いったい 
\\	いったい 
\\	いったいいつ終わるのだろうか。	
\\	"いったい 
\\	いったい 
\\	この絵はモネによって描かれたとされているが、絵の中にモネのサインは入っていない。	
\\	とされている 
\\	家に帰って、いつも第一にすることは、
\\	メールのチェックだ。	
\\	"第一 に 
\\	試験に合格できた第一の理由は、覚えた言葉や漢字が全部試験に出たことだ。	
\\	"第一 の
\\	彼女の第一印象は悪かったのに、なぜか私と彼女はとてもいい友達になった。	
\\	"第一 
\\	私のモットーは「健康が第一」です。	
\\	が第一 だ。 
\\	この大学の図書館には、アジア経済に関する本がたくさんあります。	
\\	に関する 
\\	和紙の利用法に関して調べてみたら、とても面白いことがわかった。	
\\	に関して 
\\	について 
\\	に関して, 
\\	コンピュータの知識に関しては、彼が一番だと思う。	
\\	に関して は 
\\	について 
\\	に関して, 
\\	現代人にとって、コンピュータのない生活は考えられないだろう。	
\\	{は/なんて} 考えられない 
\\	漢字が800も書けるようになるなんて、日本語の勉強を始めた時には考えられなかった。	
\\	{ということは/なんて} 考えられない 
\\	将来は、自分の能力を生かすことが出来る会社で働きたい。	
\\	"生かす 
\\	生かす 
\\	土の中から魚の骨や貝などが出てきたことから、ここは大昔海だったということが分かった。	
\\	"ことから 
\\	ことから 
\\	早く辞書なしで新聞が読めるようになりたい。	
\\	なし で 
\\	今はもうコンピュータなしの生活は考えられない。	
\\	なし の 
\\	ダイエット中だから、今日はデザートなしだ。	
\\	なし だ。 
\\	今年こそ日本に留学したいと思っている。	
\\	こそ 
\\	こそ 
\\	こそ 
\\	は、が 
\\	を 
\\	スタジアムに行って実際に見てこそ本当の面白さが分かるのだ。	
\\	こそ 
\\	こそ 
\\	こそ 
\\	は、が 
\\	を 
\\	大きい音に驚いて、ガムを飲み込んでしまった。	
\\	込む 
\\	込む 
\\	{飲み込む}
\\	昨日は久しぶりに会った友達と一晩中話し込んで、徹夜してしまった。	
\\	込む 
\\	込む 
\\	{話し込む}
\\	その子は水に飛び込む気持ちにはどうしてもなれなかった。	
\\	込む 
\\	込む 
\\	{飛び込む}
\\	自分は絶対に正しいと思い込んでいる人とは、あまりはなしたくない。	
\\	込む 
\\	込む 
\\	{思い込む}
\\	キムさんってさあ、フランス語がペラペラなんだけど、日本語も上手なんだってさ。	
\\	さ(あ) 
\\	さ 
\\	私が住んでいるアパートは、周りが静かなだけでなく安全でもあるので、学生に人気がある。	
\\	でもある 
\\	は 
\\	でもある. 
\\	は 
\\	でもあり、
\\	でもある。 
\\	今日は朝からずっと雨が降った。	
\\	"ずっと 
\\	ずっと 
\\	今から2,3時間、図書館に勉強しに行ってきます。	
\\	今のままだと、地球温暖化は止まらない。	
\\	のままだ 
\\	まま 
\\	あのまま日本に残っていたら、もっと日本語が上手になっただろう。	
\\	まま 
\\	まま 
\\	このお菓子は、冷たいまま食べるより、電子レンジで温めた方がおいしく食べられますよ。	
\\	まま(で)
\\	まま 
\\	これはこのままにする。	
\\	ままにする。 
\\	まま 
\\	彼は20代で会社を作り、30代で、大金持ちになった。	
\\	代 
\\	代 
\\	1980年代の後半から1990年代の初めにかけての日本の経済は「バブル経済」と呼ばれている。	
\\	年代 
\\	代 
\\	年 
\\	日本では秋になると仕事の面接に行くスーツ姿の学生が増える。	
\\	(の)姿 
\\	姿 
\\	母が病気で苦しむ姿を見て、私は将来医者になろうと心に決めた。	
\\	姿 
\\	姿 
\\	今日は朝から、雨が降ってはやみ、降ってはやみしていて嫌な天気だ。	
\\	は 
\\	は
\\	子供の頃は、毎年夏休みになると、家族と海に泳ぎに行ったものだ。	
\\	ものだ 
\\	ものだ 
\\	毎日日本語で話すようにしていると、だんだん上手に話せるようになりますよ。	
\\	"だんだ
\\	だんだん 
\\	~ていく; 〜てくる;〜くなる;〜になる;〜ようになる
\\	暖かくなるにつれて、雪が溶け、木が緑になっていく。	
\\	につれて 
\\	につれて 
\\	技術の進歩につれて、私達の生活は便利になった。	
\\	につれて 
\\	につれて 
\\	今日中にこのレポートを書き上げてしまわなければならない。	
\\	"書き上げる 
\\	上げる 
\\	この荷物は重すぎて誰も持ち上げることは出来ない。	
\\	"持ち上げる 
\\	上げる 
\\	日本語を勉強している学生が、自分達で日本語のブログサイトを立ち上げた。	
\\	"立ち上げる 
\\	上げる) 
\\	上げる 
\\	泥棒はコンビニの店員を縛り上げて、レジのお金を盗んで行った。	
\\	"縛り上げる 
\\	上げる) 
\\	上げる 
\\	この岩を磨き上げてください。	
\\	"磨き上げる 
\\	上げる) 
\\	上げる 
\\	2行目から10行目まで、読み上げた。	
\\	"読み上げる 
\\	上げる) 
\\	上げる 
\\	会社への行きがけに銀行によって、お金をおろして行こう。	
\\	がけに 
\\	がけに 
\\	行く、来る、
\\	帰る 
\\	日本に行ったら 英語は使わないでなるべく日本語だけで生活するつもりだ。	
\\	だけで 
\\	だけで 
\\	この植木は2週間に一度水をやるだけでいいですよ。	
\\	だけで {いい/十分だ/
\\	だけで 
\\	「飲むだけでやせられる薬」などという広告は信じない方がいいですよ。	
\\	だけで 
\\	だけで 
\\	この読み物は、あまりにも漢字が多すぎて、何か書いてあるのかまったく理解できない。	
\\	"あまりに(も) 
\\	あまりに(も) 
\\	隣の家のパーテイがあまりにうるさかったので、警察を連絡した。	
\\	"あまりに(も)
\\	だ} 
\\	あまりに(も) 
\\	あまりに便利な生活は、人間を駄目にすると思う。	
\\	"あまりに(も)
\\	く} 
\\	あまりに(も) 
\\	漢字を覚えるにはフラッシュカードを作るのが一番です。	
\\	には 
\\	には 
\\	には 
\\	(=時には). 
\\	には. 
\\	ために 
\\	には 
\\	ために
\\	ためには
\\	海外旅行には、このスーツケースを持って行くと便利ですよ。	
\\	する) には 
\\	には 
\\	には 
\\	(=時には). 
\\	には. 
\\	ために 
\\	には 
\\	ために
\\	ためには
\\	先生の説明を聞いて、この言葉の意味はだいたい分かったけれど、まだ上手に使えない。	
\\	"だいたい
\\	だいたい 
\\	この大学では、だいたいの学生がアルバイトをしている。	
\\	"だいたいの 
\\	だいたい 
\\	だいたいの数は分かりますが、正確な数は調べてみないと分かりません。	
\\	"だいたい 
\\	だいたい 
\\	だいたい 
\\	7時から大学ホールにおいて学生会議が行います。	
\\	において 
\\	における 
\\	において. 
\\	アジアにおける歴史に関する問題は、解決が難しそうです。	
\\	における 
\\	における 
\\	において. 
\\	暑いのは好きではないが、寒いのもまた苦手だ。	
\\	"もまた 
\\	もまた 
\\	も; 
\\	も.
\\	なんだ、まだ宿題終わってないの。早くしないと、授業がはじまっちゃうよ。	
\\	"なんだ 
\\	なんだ 
\\	ほら、これ、見て。リリーが子犬の時の写真。かわいいねえ。	
\\	"ほら 
\\	ほら 
\\	盲導犬などのサービスドッグを除いて、病院には犬を連れて入ってはいけません。	
\\	を除いて 
\\	除く 
\\	を除いて 
\\	留学中、沖縄を除く日本全国各地を友達と一緒に旅行した。	
\\	を除く 
\\	除く 
\\	を除いて 
\\	昭和25年頃、すなわち 1950年頃の日本は、戦争が終わったすぐ後で、みんな貧乏だった。	
\\	すなわち 
\\	すなわち
\\	受付の仕事は一応私達がやることになっています。	
\\	"一応 
\\	一応 
\\	コンピュータがフリーズする問題は一応解決したけれど、まだ、メールが文字化けするという問題が残っている。	
\\	"一応 
\\	一応 
\\	レポートは一応書き上げたけれど、英語のスペルチェックがまだだ。	
\\	"一応 
\\	一応 
\\	作文はワープロで書いてもいいはずだが、一応、先生に聞いてみよう。	
\\	"一応 
\\	一応 
\\	日本でのインターンシップの経験は、私の人生を大きく左右するだろう。	
\\	を 
\\	左右する 
\\	左右する 
\\	今年の冬は、一月、二月共、あまり寒くなった。	
\\	共(に) 
\\	1も
\\	2も.
\\	インターネットの時代に入って、人々の買物の仕方が変わった。	
\\	に {入り/入って (から)} 
\\	に入る 
\\	に入り 
\\	に入って(から), 
\\	梅雨に入ってから、晴れの日がまったくなくて湿度が高く、気持ちが悪い。	
\\	に {入り/入って (から)} 
\\	に入る 
\\	に入り 
\\	に入って(から), 
\\	江戸時代に入り、和歌より俳句の方が楽しまれるようになった。	
\\	に {入り/入って (から)} 
\\	に入る 
\\	に入り 
\\	に入って(から), 
\\	地球の未来を考えたとき、何よりもまず初めに地球温暖化が問題にされるべきだ。	
\\	が問題に {される/なる} 
\\	を 問題にする 
\\	が問題にされる. 
\\	が問題になる 
\\	私の母は、未だに携帯電話を持たずに生活している。	
\\	"未だに 
\\	まだ 
\\	リサーチを通して、私達がいかにエネルギーを無駄に使っているかを知った。	
\\	"いかに〜か 
\\	いかに 
\\	母の料理がいかにおいしいかは、言葉では説明できない。	
\\	"いかに〜か 
\\	いかに 
\\	この問題をいかに解決すべきか、みんなで話し合わなくては鳴らない。	
\\	"いかに〜か 
\\	いかに 
\\	この論文はすばらしいと思う。しかしながら、賛成できない点もいくつかある。	
\\	しかしながら 
\\	しかしながら 
\\	このカレーはかなり辛いけれど、食べられないことはない。	
\\	"かなり 
\\	ピアノ、お上手ですね。いつ始められたんですか。	
\\	る-
\\	られる 
\\	お
\\	になる 
\\	いらっしゃる 
\\	めし上がる。 
\\	いる 
\\	いられる, 
\\	おられる. 
\\	おられる 
\\	いらっしゃる。
\\	先輩、まだ来られないんですか。	
\\	される;(来=こられる) 
\\	お
\\	になる 
\\	いらっしゃる 
\\	めし上がる。 
\\	いる 
\\	いられる, 
\\	おられる. 
\\	おられる 
\\	いらっしゃる。
\\	先生はテニスをされるんですね。知りませんでした。	
\\	る-
\\	られる 
\\	お
\\	になる 
\\	いらっしゃる 
\\	めし上がる。 
\\	いる 
\\	いられる, 
\\	おられる. 
\\	おられる 
\\	いらっしゃる。
\\	日本では四国に住んでおられたんですか。暖かいし、海もきれいだし、いい所ですよね。	
\\	る-
\\	られる 
\\	お
\\	になる 
\\	いらっしゃる 
\\	めし上がる。 
\\	いる 
\\	いられる, 
\\	おられる. 
\\	おられる 
\\	いらっしゃる。 
\\	よね 
\\	すみません、あの鉛筆をつかわれますか。	
\\	う-
\\	れる 
\\	お
\\	になる 
\\	いらっしゃる 
\\	めし上がる。 
\\	いる 
\\	いられる, 
\\	おられる. 
\\	おられる 
\\	いらっしゃる。
\\	スミス先生は、僕の日本語の先生である森先生を知っておられるそうだ。	
\\	いる: 
\\	おられる 
\\	お
\\	になる 
\\	いらっしゃる 
\\	めし上がる。 
\\	いる 
\\	いられる, 
\\	おられる. 
\\	おられる 
\\	いらっしゃる。
\\	どんなアパートを借りるかは、家賃次第だ。	
\\	次第だ 
\\	次第 
\\	漢字の勉強は覚え方次第で大変じゃなくなりますよ。	
\\	次第で 
\\	次第 
\\	日本人だから漢字をよく知っているとは限らない。	
\\	"〜とは限らない 
\\	とは限らない 
\\	だ 
\\	とは 
\\	自分で、部屋は、毎日ちゃんと掃除して、きちんと片付けておきなさいよ。	
\\	"ちゃんと;きちんと 
\\	ちゃんと 
\\	きちんと 
\\	ちゃんと 
\\	きちんと。
\\	日本語を勉強しているんだから、少なくとも一度は日本に行ってみたいと思っている。	
\\	"少なくとも 
\\	少なくとも 
\\	は 
\\	人に何かしてもらったら、少なくともお礼は言った方がいい。	
\\	"少なくとも 
\\	(は) 少なくとも 
\\	は 
\\	海に行きたい人もいれば、山に行きたい人もいて、旅行の計画が全然決まらない。	
\\	も 
\\	も 
\\	人生は考え方次第で、楽しくもなれば、苦しくも鳴る。	
\\	くも 
\\	ば, 〜も 
\\	この仕事は危険でもあれば大変でもある。しかし、誰かがやらなければならないのだ。	
\\	{に/で} も 
\\	ば、〜も 
\\	子供の頃は、兄弟とよく遊びもすれば、けんかもしたものだ。	
\\	もすれば、
\\	もする 
\\	一週間のうち、三日はアルバイトに行きます。	
\\	のうち(で) 
\\	のうち(で)
\\	一週間の旅行のうち、三日間は東京に、後(の)四日間は京都に行く予定だ。	
\\	"後(の)
\\	後 
\\	後 
\\	後 
\\	の 
\\	後. このプロジェクトには後三人必要だ。
\\	これは私の物です。でも、後の物は全部ルームメートの物です。	
\\	"後の
\\	後 
\\	後 
\\	後 
\\	の 
\\	後. このプロジェクトには後三人必要だ。
\\	私にとって何よりも大切なのは、家族の幸せです。	
\\	"何より(も) 
\\	先生のアドバイスに従って勉強したら、成績がよくなった。	
\\	"〜に従って 
\\	昨日は大雪で、授業に来た学生はたったの四人だった。	
\\	"たった(の)
\\	隣の部屋の音楽が気になってしかたがない。うるさすぎる。	
\\	しかたがない 
\\	てしかたない、てしようがない,
\\	しかたがない 
\\	しようがない 
\\	しかたがない 
\\	お金がないのだから、留学できないのはしかたがない。
\\	昨日は、4時間しか寝られなかったので、眠くてしようがない。	
\\	しようがない 
\\	てしかたない、てしようがない,
\\	しかたがない 
\\	しようがない 
\\	しかたがない 
\\	お金がないのだから、留学できないのはしかたがない。
\\	来年、日本に行けることになってので、嬉しくてしょうがない。	
\\	しょうがない 
\\	てしかたない、てしようがない,
\\	しょうがない 
\\	しかたがない 
\\	お金がないのだから、留学できないのはしかたがない。
\\	昨日から頭が痛くてしかたがない。	
\\	しかたがない/しようがない/しょうがない 
\\	てしかたない、てしようがない,
\\	しょうがない 
\\	しかたがない 
\\	お金がないのだから、留学できないのはしかたがない。
\\	あの先生の講義は退屈でしょうがない。	
\\	でしかたがない/しようがない/しょうがない 
\\	てしかたない、てしようがない,
\\	しょうがない 
\\	しかたがない 
\\	お金がないのだから、留学できないのはしかたがない。
\\	大学に入って以来、高校時代の友達に一度も会っていない。	
\\	以来 
\\	父は10年前に亡くなった。それ以来、母は一人で私達兄弟を育ててくれた。	
\\	いらい 
\\	この製品に使われているものは、何から何までリサイクルできます。	
\\	"何から何まで 
\\	何から何まで 
\\	全部, すべて、
\\	みんな. 何から何まで 
\\	全部、みんな、
\\	すべて。
\\	信じられた人にだまされて、何から何まで信じられなくなった。	
\\	"何から何まで 
\\	すべて、みんな、
\\	全部.
\\	友達と大げんかをしたら、その後、かえって仲良なった。	
\\	"かえって 
\\	楽しみ	
\\	み 
\\	み 
\\	み 
\\	さ-
\\	悲しみ	
\\	み 
\\	み 
\\	み 
\\	さ-
\\	重み	
\\	み 
\\	み 
\\	み 
\\	さ-
\\	暖かみ	
\\	み 
\\	み 
\\	み 
\\	さ-
\\	甘み	
\\	み 
\\	み 
\\	み 
\\	さ-
\\	苦み	
\\	み 
\\	み 
\\	み 
\\	さ-
\\	電車のドアが開くと同時に、多くの人が降りて来た。	
\\	と同時に 
\\	この絵本は読んで面白いと同時に、考えさせられる。	
\\	と同時に 
\\	この車は機能性と同時に安全性についてもよく考えて作られている。	
\\	と同時に 
\\	大切であると同時に、古いです。	
\\	{である/じゃない/ではない} と同時に 
\\	この漢字って、どう読むんだっけ。	
\\	んだっけ っけ 
\\	~ない, んだ 
\\	っけ。
\\	今、どこの大学に留学してたっけ。	
\\	(んだ)っけ っけ 
\\	~ない, んだ 
\\	っけ。
\\	これ、いつ買ったんだっけ。	
\\	(んだ)っけ っけ 
\\	~ない, んだ 
\\	っけ。
\\	この政治家って、衆議院だったっけ。参議院だったっけ。	
\\	だった (んだ) っけ っけ 
\\	~ない, んだ 
\\	っけ。
\\	この電子辞書、いくらだったっけ。	
\\	だった (んだ) っけ っけ 
\\	~ない, んだ 
\\	っけ。
\\	あのレストラン、おいしかったっけ。	
\\	(んだ)っけ っけ 
\\	~ない, んだ 
\\	っけ。
\\	彼、上手じゃなかった(んだ)っけ	
\\	じゃなかった(んだ)っけ っけ 
\\	~ない, んだ 
\\	っけ。
\\	彼、上手(なん)だっけ。	
\\	(なん)だっけ っけ 
\\	~ない, んだ 
\\	っけ。
\\	あの仕事は、立ちっばなしの仕事だから、疲れるに違いない。	
\\	っばなしの 
\\	っばなし 
\\	うちのチームは今日までずっと勝ちっばなしだ。	
\\	っばなしだ/で 
\\	っばなし 
\\	疲れていたので、テレビをつけっばなしにして寝てしまった。	
\\	っばなしにする 
\\	っばなし 
\end{CJK}
\end{document}