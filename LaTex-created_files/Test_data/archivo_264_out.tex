\documentclass[8pt]{extreport} 
\usepackage{hyperref}
\usepackage{CJKutf8}
\begin{document}
\begin{CJK}{UTF8}{min}
\\	學規
\\	会津八一
\\	【テキスト中に現れる記号について】
\\	入力者注 主に外字の説明や、傍点の位置の指定
\\	(例)[#4字下げ]
\\	古い日記や手紙などを、みんな燒いてしまつたので、こまかに時日をいへないが、まだ若い中學教師であつた私が、牛込下戸塚町の素人下宿から、小石川豐川町へ引越して、その時越後から出て來たばかりの三人の書生と初めて所帶を持つたのは、たしか大正のはじめであつた。その時書生たちが机を並べた八疊の間の床の間の壁に、私がその人たちのために作つた四か條の學規といふものを自筆で書いて貼らせた。けれども受驗勉強で夢中になつてゐる書生たちは、誰一人としてそんな文句に目をくれるものもなく、どれほど窮屈な氣持で、これをうとましく思つたものもなかつた。けつきよくこの學規は、私自身のために私が作つて、書いて、そして自分を警しめるだけのものになつてしまつた。それから四十年にも近く、今の老境にはいつても、いつも親しくなつかしい氣持でこの四か條が思ひ出される。
\\	私はもとから理想とか、主義とか、抱負とかいふやうなものがあるのか、ないのか、自分にもはつきりしないが、とにかくそんなことを大ツぴらに口を出していひ立てるのを好かない。そのせいか、私の學規も昔からあるものとはだいぶ樣子がちがふやうだ。これくらゐのところを目安にしてかかるなら、長い一生の末までには、いくらか實行が出來るのではあるまいか。
\\	[#4字下げ]秋艸堂學規
\\	[#ここから1字下げ]
\\	一 ふかくこの生を愛すべし
\\	一 かへりみて己を知るべし
\\	一 學藝を以て性を養ふべし
\\	一 日々新面目あるべし
\\	[#ここで字下げ終わり]
\\	[#地から2字上げ]以上
\\	底本:「會津八一全集 第七卷」中央公論社
\\	1982(昭和57)年4月25日初版発行
\\	※底本のテキストは、著者自筆原稿によります。
\\	入力:フクポー
\\	校正:鴨川佳一郎
\\	2017年6月13日作成
\\	青空文庫作成ファイル:
\\	このファイルは、インターネットの図書館、青空文庫
\\	で作られました。入力、校正、制作にあたったのは、ボランティアの皆さんです。
\end{CJK}
\end{document}