\documentclass[8pt]{extreport} 
\usepackage{hyperref}
\usepackage{CJKutf8}
\begin{document}
\begin{CJK}{UTF8}{min}
\\	人質	ひとじち 
\\	紙一重	
\\	かみひとえ
\\	愛想がいい	
\\	気立てがいい	
\\	潔い	いさぎよい 
\\	潔く罪を認める
\\	情(け)深い	ナサケブカイ 
\\	用心深い	ヨウジンブカイ 
\\	凛々しい	リリシイ 
\\	逞しい	たくましい 
\\	(肉体) ・
\\	を思いやる	
\\	気品がある	
\\	色気がある	
\\	淑やかな	しとやか 
\\	気まぐれな	
\\	近寄りがたい	
\\	無口な	
\\	むくち
\\	陰気な	
\\	インキ
\\	忠実な	まめ 
\\	ちゅうじつ 
\\	生真面目な	き 
\\	勤勉な	キンベン 
\\	大柄な<ー
\\	小柄な	オオガラ 
\\	コガラ 
\\	オオヘイ
\\	有望な	
\\	温和な	オンワ 
\\	気さくな	
\\	おおらかな	
\\	素朴な	ソボク 
\\	(人も疑問も。。。)
\\	物好きな	モノズキ 
\\	孤独な	
\\	臆病な	
\\	馴れ馴れしい	
\\	いやらしい	
\\	卑しい	イヤシイ 
\\	に//
\\	卑しい行い 
\\	しぶとい	
\\	荒っぽい	
\\	金遣いが荒い	カネヅカイガアライ 
\\	お節介な	おせっかい 
\\	おっちょこちょいな	
\\	でしゃばりな	
\\	キザな	
\\	ルーズな	
\\	無礼な	
\\	ブレイ
\\	豪華な	ゴウカ 
\\	ゆったりする	
\\	環境設定	
\\	望ましい	
\\	申し分(が)ない	モウシブン=
\\	快い->快く	こころよく 
\\	申し出	
\\	承諾	
\\	しょうだく 依頼、要求を承諾する
\\	すがすがしい	
\\	爽快	
\\	そうかい
\\	待ち遠しい	マチドオシイ 
\\	心強い<ー
\\	心細い	
\\	空しい・虚しい	むなしい 
\\	切ない	セツナイ 
\\	物足りない	
\\	モノタリナイ
\\	煩わしい	ワズラワシイ 
\\	鬱陶しい	うっとうしい 
\\	うざい  (俗語)	=鬱陶しい、煩わしい
\\	痛ましい	
\\	いたましい事故現場
\\	きまり(が)悪い	
\\	浅ましい	あさましい 
\\	根性	
\\	おっかない	
\\	見苦しい	
\\	(例 態度)
\\	心苦しい	
\\	に呆れる	
\\	気に障る	さわる=
\\	癪に障る	しゃくにさわる 
\\	癇に障る	かんにさわる 
\\	に気兼ねする	きがね 
\\	に, 
\\	船に酔って、胸がムカつく	
\\	嘔く=
\\	恥をかく	
\\	配線	
\\	として・に ふさわしい	
\\	俯く	ウツムク 
\\	瞬きをする	マバタキ 
\\	目を瞑る	ツブル 
\\	一言を呟く	ツブヤク 
\\	囁く	ササヤク 
\\	歌を口ずさむ	
\\	煙にむせる	
\\	食べ物でむせる	
\\	ほっぺたをつねる	
\\	草をむしる	
\\	カレンダーをめくる	
\\	塩をつまむ	
\\	鼻をつまむ	
\\	花を摘む	ツム 
\\	才能の芽を摘む	メをツム 
\\	足をさする	
\\	髪の毛をいじる	イジル 
\\	家具のほこりを叩く	ハタク 
\\	庭をいじる	
\\	(いじる=
\\	水を掬う	スクウ 
\\	杖を突く	ツエをつく 
\\	赤ちゃんをバスタオルで包む	クルム 
\\	紙を丸める	
\\	体を丸める	
\\	マッチを擦る	スル 
\\	指をさす	
\\	横になる	
\\	うつ伏せになる	
\\	うつぶせ
\\	仰向けになる	アオムケ 
\\	体を反らす
\\	曲げる	そらす 
\\	馬にまたがる	
\\	3年にまたがる計画	=渡る 
\\	湯に浸かる	
\\	つかる
\\	門をくぐる	
\\	水たまりを避ける	ヨケル 
\\	水中でもがく・足掻く	スイチュウアガク 
\\	山中を彷徨う	さんちゅう サマヨウ 
\\	投稿手段	とうこう 
\\	出典	
\\	遮断機	しゃだんき 
\\	訃報	ふほう 
\\	人をなだめる	
\\	人を労る	いたわる 
\\	人をちやほやする	
\\	人をおだてる	
\\	人をあざ笑う	
\\	人をねたむ	
\\	人をさらう 人を誘拐する	ユウカイ 
\\	人を裁く	サバク 
\\	人をかばう	
\\	人を励ます	
\\	励む 
\\	ハゲマス 
\\	人をもてなす	
\\	人を冷やかす	ヒ 
\\	人を脅す・威す	オドス<人間
\\	脅かす
\\	人を貶す	ケナス 
\\	人を罵る	ののしる 
\\	人を詰る	ナジル 
\\	人を咎める	トガメル 
\\	人に恋をする	
\\	人に片思い(を)する	
\\	人に尽くす	
\\	人にお小遣いをねだる	コヅカイ 
\\	人に寄付を強いる	シイル 
\\	人に新しいバッグを見せびらかす	
\\	人になつく	
\\	人に親しまれる	シタシマレル 
\\	人にモテる	
\\	人に出くわす	=たまたま
\\	出会う 
\\	人に言いつける	1。
\\	人に告げ口をする	つげぐち 
\\	人に嫌がらせをする	
\\	秘密をバラす	
\\	近所の人と揉める	モメル 
\\	家族を養う	
\\	ヤシナウ
\\	名付ける	ナヅケル 
\\	名付け親	
\\	人見知りする	
\\	へりくだった態度で接する	セッスル=
\\	へりくだる=
\\	話が拗れる	こじれる 
\\	関係が拗れる	コジレル 
\\	風邪をこじらせる	
\\	人を欺く	アザムク 
\\	濯ぐ	ススグ 布巾を濯ぐ 
\\	ユスグ 口をゆすぐ 
\\	コップをゆすぐ 
\\	水を濾す	こす 
\\	ばらまく	
\\	揺する	ユ 
\\	を生やす<ー
\\	が生える	ハヤス 
\\	花を生ける	イケル 
\\	よこす	手紙を寄越す 
\\	金をよこせ! 
\\	逸らす	そらす 目を 
\\	話を 
\\	凝らす	こらす 
\\	工夫を凝らす 
\\	晒す	さらす ー
\\	にー
\\	日光に ー
\\	水に 
\\	辿る	たどる 
\\	防犯カメラ	
\\	据える	すえる ー
\\	防犯カメラ)
\\	据え付ける	すえつける 
\\	添える	そえる 
\\	写真を添える=
\\	添付 
\\	控える	ひかえる ー
\\	例:塩分を=
\\	外出を 
\\	発言を 
\\	例:隣室に=
\\	控え室 
\\	歩む	あゆむ 
\\	彼と別の人生を歩む 
\\	歩み 
\\	進出	
\\	目論む	もくろむ 
\\	侵入 侵略	しんにゅう
\\	住居(じゅうきょ・すまい 
\\	に侵入する シンリャク 
\\	侵略戦争 
\\	阻む	はばむ 
\\	行く手(ゆくて 
\\	を
\\	遮る	さえぎる 
\\	視界(しかい 
\\	を遮る 話を
\\	綴る	つづる ー
\\	綴り 
\\	綴り 
\\	仕掛ける	罠を 
\\	手がける	
\\	仕事)
\\	合わす	=合わせる
\\	交える<ー
\\	交わる	マジエル 
\\	を 
\\	先生を交えて話し合う 
\\	マジワル ー
\\	交わす	カワス 
\\	束ねる	タバネル 
\\	背く	ソムク 
\\	に
\\	逃す	ノガス 
\\	機会を
\\	逃れる	ノガレル 
\\	敵から
\\	促す	ウナガス 
\\	支払いを 
\\	疲労で目が霞む	カスム 
\\	貧血になる	ヒンケツ 
\\	安静にする	
\\	棘が刺さる	トゲ 
\\	めまいがする	
\\	意識が朦朧とする	モウロウ 
\\	意識が遠ざかる ー
\\	を遠ざける	
\\	意識不明になる	
\\	足首を捻挫する	ネンザ 
\\	打ったところが腫れる	
\\	ハレル
\\	足がむくむ	
\\	むくみ 
\\	下痢をする<ー
\\	便秘になる	ゲリ<ー
\\	ベンピ 
\\	歯を矯正する	きょうせい 
\\	蕁麻疹ができる	じんましん 
\\	猫に引っかかれる	
\\	引っ掻く 
\\	強まる<ー
\\	弱まる	
\\	強める 
\\	痛みを和らげる	ヤワラゲル 
\\	が和らぐ
\\	リハビリする リハビリを受ける	
\\	患者に付き添う	ツキソウ 
\\	面会に行く	
\\	癌	がん 
\\	※カタカナで書くことが多い ガン
\\	喘息	ゼンソク 
\\	喘息の発作(ホッサ) 
\\	気管支炎	
\\	気管=
\\	気管支=
\\	肺炎	
\\	皮膚炎	
\\	アトピー性皮膚炎 
\\	うつ病	
\\	認知症	ニンチショう 
\\	アルツハイマー病
\\	烏養繋心役を演じてくださいました田中一成さんのご逝去に際し、心よりお悔やみ申し上げます。謹んでご冥福をお祈りいたします	逝去=セイキョ 
\\	悔やみ=
\\	謹んで=ツツシンデ 
\\	気になれない	
\\	「大丈夫」と強がる	ツヨガル 
\\	くたびれる	
\\	て
\\	バテる	
\\	「知らない」と惚ける	トボケル 
\\	1。年とともに惚ける 2。ピントがぼけた写真	ボケル 1。
\\	2。ピント=
\\	どもる	
\\	しくじる	
\\	入試で
\\	察する	サッスル 1。
\\	捧げる	ササゲル 
\\	怯える	オビエル 
\\	に
\\	こみ上げる	
\\	涙がこみ上げる 
\\	自惚れる	ウヌボレル 
\\	例:彼女は自分が美人だと自惚れている。 
\\	呆然とする	ボウゼン 
\\	赤らめる	顔をあからめる 
\\	凝る ー
\\	凝った	コル 
\\	に ー
\\	懲りる	コリル 
\\	失敗に懲りる
\\	こだわる	
\\	に
\\	「給料が安い」とぼやく	
\\	まごつく	
\\	任す	
\\	を
\\	和む	ナゴム 
\\	心が和む 
\\	こもる	
\\	家にこもる 
\\	体に熱がこもる 
\\	ごまかす	1。話をごまかす 
\\	2。値段をごまかす 
\\	冴える	サエル 1。目がさえる 
\\	2。頭が冴える 
\\	叶う	カナウ 1。願いが叶う 
\\	叶える 2。水泳では彼に叶うものはいない 
\\	彼にはかなわない
\\	意気込む	イキゴム 
\\	「今度こそ勝つぞ!」と意気込む。
\\	老ける	フケル 
\\	年の割には老けて見える 
\\	=老いる(オイル)、年老いた
\\	能率	
\\	掠る	カスる 
\\	弾が耳をかすった ー
\\	かすり傷 
\\	擦り傷(すりきず)
\\	写真がぶれている ぶれる	
\\	ぼやける	
\\	字がぼやけてよく見えない
\\	かさむ	
\\	経費がかさむ 
\\	かさばる	
\\	荷物が嵩張って持ちにくい
\\	重なる	
\\	予定が重なる 
\\	不幸が重なる 
\\	重ねる 
\\	を重ねる	
\\	そびえる	
\\	高層ビルが空にそびえている
\\	たるむ	1。
\\	紐がたるんでいる 2。
\\	気持ちがたるむ 
\\	とろける	
\\	が
\\	はげる	1。禿げる 
\\	頭が禿げる 2。剥げる 
\\	ペンキがはげる
\\	さえずる	
\\	小鳥がさえずる
\\	軋む	キシム 
\\	床が軋む
\\	弾く 弾む 弾	ハジク 
\\	ひく 1。
\\	油は水を弾く 2。
\\	ギターの弦を弾く 弦=ゲン 
\\	ーーー はずむ 
\\	声が ーーー たま 
\\	剥ぐ	ハグ 
\\	魚の皮を剥ぐ
\\	潤う	ウルオウ 
\\	肌が潤う 
\\	駅ができて商店街が潤う
\\	繕う	ツクロウ 1。
\\	綻びを繕う ほころび 
\\	体裁を繕う 体裁=テイサイ 
\\	捗る	ハカドル 
\\	仕事が捗る 
\\	足を引きずる	
\\	反る	そる 
\\	板がそる
\\	剃る	ソル 
\\	髭を剃る 
\\	尖る	トガル 
\\	尖ったナイフ 先の尖った靴 
\\	まぶた	
\\	ずれる	1。メガネがずれる 
\\	2。タイミングがずれる 
\\	いかれる  
\\	俗語
\\	エンジンがいかれる
\\	キイキイ	
\\	こめる	
\\	心を込めて手紙を書いた
\\	染める	
\\	髪を染める ー
\\	が染まる 
\\	尽きる	資源が尽きる 
\\	偏る	偏った考え 
\\	ハマる	
\\	水溜りにハマる 
\\	鍵が鍵穴にはまらない 
\\	型に嵌まる 
\\	ゲームにハマる =熱中する, 
\\	とどまる	
\\	この町にとどまる 物価の上昇がとどまることがない
\\	とどめる	
\\	経費を最小限にとどめる 
\\	記憶にとどめる 
\\	途切れる	トギレル 
\\	途絶える	トダエル 
\\	便り	タヨリ 
\\	蘇る・甦る	ヨミガエル 
\\	(死者) 
\\	有り触れる	
\\	=平凡な(=へいぼん)
\\	廃れる	スタレル 
\\	流行が廃れる
\\	かぶれる	1。
\\	肌がかぶれる 2。
\\	ロックにかぶれる
\\	化ける	バケル 
\\	人間に化ける 
\\	に 女に化ける=女装する
\\	バレる	
\\	嘘がバレる
\\	釣り合う	
\\	引き起こす	
\\	絡む	カラム 
\\	毛糸
\\	が絡む 痰(タン=
\\	が絡む 
\\	酔っ払いに絡まれる 
\\	金が絡む
\\	覆す	クツガエス 
\\	常識を覆す 
\\	見なす	
\\	それは不正
\\	だとみなされている
\\	しのぐ	
\\	夏の暑さをしのぐ
\\	継ぐ	ツグ 
\\	父の仕事を継ぐ 
\\	突(っ)つく	
\\	肘でつつく
\\	縮尺 尺度	シャク 
\\	しゅくしゃく 
\\	解釈 釈明	
\\	翻訳 通訳	『源氏物語』の翻訳 インタビューの通訳
\\	選択 採択	サイタク 
\\	光沢 贅沢	タク 
\\	凍結 解凍	トウ 
\\	病棟 
\\	棟・
\\	棟	トウ 
\\	訓練 試練	シレン 
\\	試練を受ける)
\\	陳列 陳謝	チン 
\\	臨時	リンジ 
\\	維持 支持	イジ 
\\	シジ 
\\	詩人 詩集	シ 
\\	待遇 待望	タイボウ 
\\	特技 特許	
\\	トッキョ
\\	侍	サムライ
\\	区画 区間	クカク 
\\	クカン 
\\	駆使 駆除	クシ 
\\	クジョ 
\\	中枢	チュウスウ 
\\	欧米 欧州	
\\	殴打	オウダ 
\\	損失 失脚	ソンシツ 
\\	シッキャク 
\\	秩序	チツジョ 
\\	法と秩序 ー
\\	秩序を乱す 
\\	製鉄 鉄棒	セイテツ 
\\	テツボウ 
\\	更迭	コウテツ 
\\	大臣(ダイジン 
\\	を更迭する
\\	博物館 博覧会	
\\	薄弱な 希薄な	ハクジャク 
\\	キハク 
\\	束縛	ソクバク 
\\	名簿 家計簿	ボ 
\\	カケイボ 
\\	瞳	ひとみ 
\\	くすぐったい	足の裏がくすぐったい 
\\	渋い 渋柿	0。渋い色 
\\	1。お茶は渋い 
\\	2。渋い顔をする 
\\	ーーーーー しぶがき 
\\	華々しい	ハナバナシイ 
\\	華々しい活躍 
\\	みずみずしい	
\\	野菜
\\	寝苦しい	
\\	夜
\\	息苦しい	グ
\\	彼は息苦しそうだ 息苦しい雰囲気 
\\	暑苦しい	ク 
\\	すばし(っ)こい	
\\	犬
\\	素早い	スバヤイ 
\\	素早く答える
\\	速やかな	スミ 
\\	速やかに退去する 
\\	健全な	
\\	精神
\\	健やかな	スコ 
\\	健やかに育つ 健やかな笑顔 
\\	軽快な	ケイカイ 
\\	曲
\\	好調な	
\\	好調な売れ行き
\\	好評な	コウヒョウ 
\\	滑らかな	ナメ 
\\	肌
\\	細やかな	コマ 
\\	細やかな愛情
\\	鮮やかな	アザ 
\\	色
\\	緩やかな	ユル 
\\	川の流れが緩やかだ
\\	しなやか・な	
\\	体
\\	和やかな	ナゴ 
\\	会話
\\	のどかな	
\\	風景
\\	きらびやかな	
\\	宮殿(きゅうでん=
\\	つぶらな	
\\	瞳
\\	緊急な・の	
\\	用事 
\\	連絡
\\	巧妙な	コウミョウ 
\\	手口
\\	円満な	エンマン 
\\	家庭 ーー
\\	円満に解決する 
\\	巧みな	タク 
\\	演技 ー
\\	言葉巧みに 
\\	言葉巧みに説いた 
\\	透き通った	すきとおった 
\\	ガラス
\\	何気ない	ナニゲナイ 
\\	あっけない	
\\	人生
\\	みすぼらしい	
\\	家
\\	汚らわしい	ケガ 
\\	行為
\\	あくどい	
\\	商売 
\\	危うい	
\\	このままでは合格が危うい
\\	ややこしい	
\\	関係
\\	素っ気ない	ソッケナイ 
\\	返事
\\	脆い	モロい 
\\	紛らわしい	マギ 
\\	言葉
\\	久しい	ヒサ 
\\	彼に久しく会っていない
\\	平たい 平らな	ヒラ 
\\	石、入れ物 ーーー タイ 
\\	場所
\\	儚い	
\\	命
\\	あやふやな	
\\	あやふやな態度をとる 
\\	虚ろな	ウツ 
\\	瞳 
\\	滑稽な	コッケイ 
\\	姿・話
\\	ちっぽけな	
\\	夢
\\	ぞんざいな	
\\	言葉遣い
\\	不気味な	ブキミ 
\\	生き物
\\	不細工な	ブサイク 
\\	形 
\\	無茶な事を言う	
\\	雑な	
\\	仕事
\\	露骨な	ロコツ 
\\	露骨に嫌がる
\\	念入りな	ネンイリ 
\\	計画
\\	密かな	ヒソ 
\\	楽しみ
\\	疎かな	オロソ 
\\	勉強が疎かになる 
\\	ろくな〜ない	
\\	それはろくな仕事ではない 朝からろくに食べていない
\\	ものすごい	外はものすごい風邪だ
\\	おびただしい	
\\	数の虫
\\	甚だしい	ハナハ 
\\	非常識
\\	も甚だしい 勘違いも甚だしい 
\\	限りなく 無限に	
\\	限りなく広がる大草原(だいそうげん=
\\	果てしない	
\\	大地	
\\	極めて	
\\	断然	ダンゼン 
\\	めちゃくちゃな	1。
\\	車がめちゃくちゃに壊れる 2。めちゃくちゃ美味しい (俗語)
\\	完璧な	
\\	圧倒的な	
\\	反対意見が圧倒的に多い
\\	極端な	キョクタン 
\\	意見
\\	大幅な	オオハバ 
\\	増加
\\	半端な<ー
\\	半端ない	ハンパ 
\\	中途半端な	チュウトハンパ 
\\	工事が中途半端に終わる
\\	微かな	カス 
\\	微かに聞こえる
\\	若干	ジャッカン 
\\	若干名の社員を募集する
\\	いかなる	
\\	いかなる時も全力を尽くそう
\\	いたるところ	
\\	いたるところで歓迎される
\\	大げさな	
\\	大まかな	
\\	要点を大まかに話す
\\	大方 だいたい	オオカタ 
\\	講義の大方は理解できる 2。大方、明日には回復するだろう
\\	おおよそ	
\\	おおよその見当(ケントウ=
\\	をつける
\\	概ね	オオムネ 
\\	概ね賛成である
\end{CJK}
\end{document}