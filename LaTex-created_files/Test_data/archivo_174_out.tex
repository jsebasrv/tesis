\documentclass[8pt]{extreport} 
\usepackage{hyperref}
\usepackage{CJKutf8}
\begin{document}
\begin{CJK}{UTF8}{min}
\\	リサイタル	
\\	開演	
\\	かいえん
\\	前売り(券)	
\\	まえうり
\\	入賞	
\\	にゅうしょう
\\	共演者 共演する	
\\	きょうえんしゃ
\\	公演	
\\	こうえん
\\	ギタリスト	
\\	開場	
\\	かいじょう
\\	全席自由	
\\	ぜんせきじゆう
\\	当日(券)	
\\	とうじつ
\\	デビュー	
\\	クラッシック	
\\	ジャズ	
\\	ピアニスト	
\\	演奏する	
\\	えんそう
\\	曲目	
\\	きょくもく
\\	部門 
\\	国際音楽コンクールバイオリン部門	
\\	ぶもん
\\	若さに似合わない高度な技術	
\\	違った一面	
\\	いちめん
\\	批評	
\\	ひひょう
\\	批判	
\\	ひはん
\\	断水	水道が使えなくなること だんすい 
\\	汲む	飲み水とか、洗い物をする水を汲んでおきましょう。 くむ 
\\	水道の工事	
\\	災害派遣	
\\	さいがいはけん
\\	水不足	
\\	みずぶそく
\\	水道管	
\\	すいどうかん
\\	新しいのと取り替える	
\\	赤く濁った水が出る場合は、水道課まで連絡する	
\\	にごる
\\	バケツ	
\\	やかん	
\\	家中水だらけ	
\\	いえじゅうみず
\\	一晩	
\\	ひとばん
\\	災害時	
\\	さいがいじ
\\	節水	
\\	せっすい
\\	お問い合わせ先	
\\	お問い合わせ	
\\	修理工事	
\\	工事に伴い	
\\	汲み置き	
\\	くみおき
\\	施工業者	
\\	せこうぎょうしゃ しこう
\\	雨不足	
\\	あめぶそく
\\	水位	
\\	すいい
\\	一時的	
\\	水洗	
\\	すいせん
\\	耐える	
\\	た・える
\\	常識を働かせて必要な情報だけ読み取ろう	
\\	じょうしき
\\	日帰り旅行	
\\	ひがえり
\\	バーベキュー	
\\	〇〇研修	〇〇
\\	けんしゅう
\\	定員になり次第締め切ります。	
\\	申し込み	
\\	計画を立てる	
\\	小雨決行	
\\	(=大雨中止) しょううけっこう
\\	申込用紙に記入する	
\\	もうしこみようし に きにゅう
\\	参加費	
\\	参加申込書	
\\	申込日	
\\	もうしこみび
\\	参加者名	
\\	国籍	
\\	こくせき
\\	性別	
\\	せいべつ
\\	昼食	
\\	ちゅうしょく
\\	不満がある	
\\	形式 形式的な	
\\	けいしき
\\	時下	
\\	ジカ
\\	ますます	
\\	省略する	
\\	ショウリャク
\\	所属	
\\	私の所属は事務部です。 
\\	ショゾク
\\	連絡先	
\\	レンラクサキ
\\	取引 取引先	
\\	トリヒキ 
\\	詳細	
\\	ショウサイ
\\	株	
\\	カブ
\\	商品の納品	
\\	ショウヒン ノ ノウヒン
\\	発送	
\\	ハッソウ
\\	遅延	
\\	チエン
\\	納期	
\\	ノウキ
\\	入荷	
\\	ニュウカ
\\	配送	
\\	ハイソウ
\\	手配	
\\	テハイ
\\	今度->この度	
\\	都合	
\\	ツゴウ
\\	変更	
\\	ヘンコウ
\\	お届け日	
\\	オトドケビ
\\	他に 
\\	もし五月十八日以降で他にお受け取りのご都合のよろしい日時がございましたら、ご連絡をお願いします。	
\\	ホカニ
\\	お手数ですが お手数をおかけします	
\\	テスウ
\\	このメールに返信で	
\\	ヘンシン
\\	有限会社	
\\	ユウゲンガイシャ
\\	御中 有限会社クラウド御中	
\\	オンチュウ
\\	小計	
\\	しょうけい
\\	送料	
\\	定休 
\\	水曜日定休	
\\	ただ忍耐あるのみだ ただ忍耐あるばかりだ	
\\	ニンタイ
\\	改まった	
\\	あらたまった
\\	拝啓 
\\	敬具	
\\	はいけい 
\\	けいぐ
\\	挨拶	あいさつ
\\	時下、ますますご清栄 (ご健勝) のことと、お喜び申し上げます。	
\\	セイエイ ケンショウ
\\	今後のご活躍を祈り申し上げます	
\\	書類選考	
\\	一次試験 二次試験	
\\	面接 
\\	履歴書=
\\	イチジ
\\	不合格 不採用	ダメだった 
\\	採用 合格	
\\	受かる	
\\	通る	
\\	補講	
\\	ほこう
\\	学生課	
\\	弊社	
\\	へいしゃ
\\	ご応募いただき…	
\\	慎重	
\\	しんちょう
\\	協議	
\\	きょうぎ
\\	実施する	
\\	ジッシ
\\	筆記用具	
\\	ひっき
\\	精算	
\\	交通費が精算いたします) せいさん
\\	日程	
\\	「~させていただくことになりました」	「~することにしました」の謙譲語
\\	遠回しな表現	
\\	とおまわし
\\	入金	
\\	していただければ幸いです	
\\	してください サイワイ 
\\	法的手段 法的措置	
\\	ホウテキシュダン ホウテキソチ
\\	行き違い	・
\\	本状と行き違いにお支払済の場合はお許しください。= 
\\	督促状	
\\	とくそくじょう
\\	代金	
\\	コンビニ後払い	
\\	あとばらい
\\	払い込み用紙	
\\	うっかり	
\\	代表取締役	
\\	ダイヒョウ トリシマリ ヤク
\\	請求する	
\\	せいきゅう
\\	ご入金お確認がとれません。	入金されていない
\\	請求書	
\\	セイキュウショ
\\	金額	
\\	キンガク
\\	送信状 
\\	送信先 
\\	発信者	
\\	ソウシンジョウ ハッシンシャ
\\	予定表に従い、学習を進める	
\\	有効期限:
\\	年
\\	月
\\	日 
\\	年
\\	月
\\	日まで有効	
\\	ユウコウキゲン
\\	無料 半額 
\\	割引 
\\	ムリョウ ハンガク ワリビキ 
\\	店のみ 
\\	全店	
\\	店だけ 
\\	どの店でも
\\	ご注文の際 
\\	お会計前に	
\\	他の割引券/クーポンとは併用できません	ヘイヨウ 
\\	お一人様一回 同伴の方お一人まで [一人分だ]	ドウハン 
\\	時間帯 朝食時間帯	ジカンタイ チョウショクジカンタイ 
\\	調剤 調剤薬局	
\\	チョウザイ
\\	一部	
\\	~等 
\\	雑誌等	~とう 
\\	共通 例)Aと
\\	に共通の内容はどれか。	キョウツウ 
\\	店頭 
\\	店頭表示価格	テントウ 
\\	期間	
\\	対象	タイショウ 
\\	葉書き 
\\	届いたセールのハガキ	
\\	限定 
\\	期間限定	
\\	お得な 
\\	お得なセール	オトク 
\\	特典	トクテン 
\\	ご来店のお客様	
\\	程度 
\\	週三日程度	
\\	早朝 
\\	早朝勤務	ソウチョウ 
\\	ソウチョウキンム 
\\	無用 
\\	心配無用	
\\	値札	ネフダ 
\\	さらに 
\\	店頭表示価格からさらに割引?	
\\	合計額	ゴウケイガク 
\\	会員証	カイインショウ 
\\	もれなく	
\\	記念品	キネンヒン 
\\	本状 
\\	必ずこの本状をお持ちください。	ホンジョウ 
\\	調理補助	チョウリホジョ 
\\	応相談	オウソウダン 
\\	特価	トッカ 
\\	接客	セッキャク 
\\	祝日 ㊗
\\	しゅくじつ 
\\	交通費 
\\	交通費->全額支給      一部支給	
\\	除く	ノゾク 
\\	前後 
\\	10分前後まで	ゼンゴ 
\\	予算	ヨサン 
\\	月3万5千円ぐらい	つき〜 
\\	角部屋	カドヘヤ 
\\	所在地	ショザイチ 
\\	面積	メンセキ 
\\	築年数	
\\	種別 例)アパートとかマンション	シュベツ 
\\	敷金 =保証金	シキキン 
\\	賃料 =家賃	
\\	間取り 例)間取りが悪い家	マドリ 
\\	最寄り駅	モヨリエキ 
\\	余る 例)履歴書の用紙、余ってるから、あげるよ	アマル 
\\	ヨウシ=
\\	資格	シカク 
\\	応募	オウボ 
\\	高校生不可	
\\	費用 例)通勤費用	ヒヨウ 
\\	余分な 
\\	を余分に持ってる	ヨブン 
\\	距離	キョリ 
\\	がある=
\\	要普通免許 =車の普通運転免許証が必要	ヨウフツウメンキョ 
\\	原付 =原動機付自転車	ゲンツキ 
\\	=ゲンドウ
\\	キツキジテンシャ
\\	盗難	トウナン 
\\	防止	
\\	ボウシ
\\	破損	ハソン 
\\	責任を負う	
\\	放置する	ホウチ 
\\	放置自転車 
\\	月極	ツキギメ 
\\	撤去 例)撤去の対象となります。	テッキョ 
\\	及び	オヨビ 
\\	と
\\	及び
\\	駐輪(場)	チュウリン(ジョウ) 
\\	満車	マンシャ 
\\	注意事項	チュウイジコウ 
\\	施設	シセツ 
\\	利用案内	
\\	要領の悪い	ヨウリョウのワルイ 
\\	要領=
\\	無断で	
\\	一切 例)一切責任を負いません	イッサイ 
\\	時間貸(し)	ジカンカシ 
\\	定期契約	テイキケイヤク 
\\	定期利用	
\\	利く 例)学割も利く	キク 
\\	アクを取る	
\\	煮立つ	ニタツ 
\\	ルー	
\\	の素	モト 
\\	ソースミックス	
\\	薄切り	ウスギリ 
\\	角切り	カクギリ 
\\	すくう	
\\	焦がさないように	コガサナイ 
\\	焦げ目がつくように	コゲメ
\\	一旦	イッタン 
\\	仕上げに	シアゲに 
\\	一口大に切る	ヒトクチダイ 
\\	厚手の鍋	アツデのナベ 
\\	ボウル	
\\	耐熱皿	タイネツザラ 
\\	煮込む	ニコム 
\\	茹でる	ユデル 
\\	とろけるチーズ	
\\	溶かす 例)バターを溶かす	トカス 
\\	順に	ジュンに 
\\	熱する	ネッスル 
\\	鶏肉 か 豚肉	
\\	軟らかい=柔らかい	ヤワラカイ 
\\	声〜, 
\\	加工する	
\\	加える	クワエル 
\\	悲惨な	ヒサン 
\\	体験 
\\	経験	タイケン 
\\	ケイケン 
\\	弱火 
\\	中火にかける	ヨワビ 
\\	チュウビ 
\\	シチュー	
\\	炒める	イタメル 
\\	かき混ぜる	
\\	納豆) 
\\	混ぜる 
\\	←→混ざる=混じる 
\\	沸騰する 
\\	沸く	フットウ 
\\	ワク 
\\	無線
\\	完備	
\\	周辺	シュウヘン 
\\	至便	シベン 
\\	直結	チョッケツ 
\\	旅行代理店	
\\	消費税	ショウヒゼイ 
\\	3連泊	
\\	バイキング	
\\	衣替え	コロモガエ 
\\	出張です	シュッチョウ 
\\	素泊まり	スドマリ 
\\	備考 備考欄	ビコウ 
\\	ビコウラン 
\\	最適な	
\\	〜一方だ。 〜つつある。	
\\	=〜の状態だ。
\\	〜に至る。 〜次第だ。 〜始末だ。	〜という結果になった 
\\	1-イタル
\\	2-シダイ 3-シマツ
\\	(薬を)点滴する	テンテキ 
\\	ケアレスミス	
\\	命が奪われる	ウバワレル 
\\	途端	
\\	たとたん 
\\	院長	いんちょう 
\\	命を預かる	
\\	重く受け止める	
\\	他人事	
\\	ひとごと
\\	他人事とはせずに	自分には関係ないことと思わないで
\\	和製英語	ワセイエイゴ 
\\	モーニングコール	
\\	ペーパーテスト	
\\	ベビーカー	
\\	パソコン	
\\	リモコン	
\\	コンビニ	
\\	マナー	
\\	3~どころか	
\\	子供の躾がなっていないどころか、親自身のマナーが酷い(ヒドイ)。
\\	子供の躾がなっていない	シツケ 
\\	親が睨んでくるんですよ。	
\\	雑誌に寄せらせた投書	寄せる=
\\	トウショ 
\\	ベビーカーを畳む	
\\	大きな荷物を抱える	カカエル 
\\	平気で	
\\	乗り込む	
\\	飛行機に 
\\	敵地に 
\\	睨みつける	ニラミツケル 
\\	折りたたむ	
\\	段落	
\\	読み取る	
\\	著者(=ちょしゃ 
\\	の言いたいことを読み取る
\\	近頃	チカゴロ 
\\	お金には不自由しない	
\\	スリルを味わう	
\\	出来事	
\\	盗人にも三分の理	ヌスビト//ヌスット サンブ 
\\	物事	
\\	ものごと
\\	万引きする	マンビキ 
\\	暴行する	
\\	ぼうこう
\\	殺害する	サツガイ 
\\	被害者	ひがいしゃ 
\\	を人間扱いせずに	あつかい=
\\	効率	
\\	のみ	
\\	だらだら(と過ごす)	
\\	特有の	トクユウの 
\\	老後	ロウゴ 
\\	皮肉屋	ヒニクヤ 
\\	反論	はんろん 
\\	中年	ちゅうねん 
\\	資する	
\\	シする
\\	視する	
\\	例ー特別視する 
\\	先頃	さきごろ 
\\	呼称	こしょう 
\\	される 
\\	と
\\	僻む	ひがむ 
\\	覚悟	かくご 
\\	にもかかわらず	
\\	思うに	
\\	コクがある	
\\	学問	
\\	蓄積	ちくせき 
\\	遊び呆ける	
\\	あそびほうける
\\	自制	
\\	じせい
\\	禁欲	きんよく 
\\	銘記する	メイキ 
\\	指示語	シジゴ 
\\	どんどん (減っちゃって)	
\\	居心地悪くなる	
\\	いごこち
\\	罰金を取る	バッキン 
\\	値上がりする	
\\	酔っ払い	ヨッパライ 
\\	勧める	進める ススメル
\\	分煙	ブンエン 
\\	終日禁煙	しゅうじつきんえん 
\\	公共施設	
\\	しせつ
\\	周囲	しゅうい 
\\	付き物	ツキモノ 
\\	肺	ハイ 
\\	くだけた言い方	
\\	砕ける 
\\	〜とは限らない	=ないかもしれない 
\\	〜にすぎない	=だけだ 
\\	〜にほかならない	=確かに〜だ 
\\	〜かねない	=〜しそうだ 
\\	〜わけにはいかない	=できない 
\\	地球儀	
\\	チキュウギ
\\	凸凹する・	デコボコ 
\\	車が凸凹道を走る
\\	出っ張る	デッパル 
\\	腹が 
\\	国際宇宙ステーション	コクサイウチュウ 
\\	ツルツルです	
\\	宇宙飛行士	うちゅうひこうし 
\\	宇宙船	うちゅうせん 
\\	飛び立つ	
\\	地表	チヒョウ 
\\	上空	ジョウクウ 
\\	位置する	
\\	直径	ちょっけい 
\\	およそ	
\\	地上	
\\	の=
\\	に ほかならない	
\\	わずか	
\\	(数が少ない)
\\	ツルツル	
\\	磨く	ミガク 1。
\\	必要に迫られて	セマラレテ 
\\	一種の	
\\	-がらみ	1。…に関係ある, 
\\	2。40がらみの女 
\\	図書館通い	ガヨイ
\\	事故にあう	
\\	くさめくさめ//くしゃみくしゃみ	
\\	に当たる表現
\\	クスクェー	沖縄では子供がくしゃみをするとの表現
\\	言い伝え	
\\	書物	ショモツ 
\\	記述	キジュツ 
\\	方言	ホウゲン 
\\	要注意	
\\	二重否定	
\\	ニジュウヒテイ
\\	〜ないわけにはいかない 〜ざるをえない	どうしても〜しなければならない
\\	高くつく	
\\	ここのところ	
\\	銀行間取引	
\\	通帳	
\\	金利	
\\	ペンキ	
\\	吸い殻	すいがら 
\\	吸い殻入れ
\\	灰皿)
\\	心が温まる	
\\	何だか	
\\	目の前に突きつける	
\\	つきつける
\\	警察手帳	
\\	刑事	
\\	宣教師	センキョウシ 
\\	言ってごらんなさい	=言ってみなさい
\\	乗り合い自動車	バス
\\	中略	
\\	身動き	
\\	身動きが取れない
\\	身動きが取れない
\\	身動きさえできぬ満員	動けないくらい満員
\\	この上(も)ない	
\\	言葉につかえたまま	
\\	パンス・ネエ	鼻眼鏡_ハナメガネ 
\\	鼠色	ねずみ色 
\\	幸福に満ちる	
\\	あらゆる	
\\	あらゆる機会を利用する 
\\	輝かす 
\\	(目を)輝かして	
\\	賢い	カシコイ
\\	賢者	ケンジャ 
\\	は
\\	と入れ違いに
\\	の隣へ腰をかけた。	
\\	情景	ジョウケイ 
\\	もじもじ	
\\	「したいことがはっきり出来ない時にもじもじする」
\\	銀河	
\\	鉄道	
\\	童話	ドウワ 
\\	ぼんやり	
\\	早くも	
\\	勢い	イキオイ 
\\	名指す	ナザス 
\\	星図	セイズ 
\\	ハキハキ	
\\	気の毒がる	
\\	たまらない	
\\	開通	カイツウ 
\\	長蛇の列	
\\	チョウダのレツ
\\	裁判員裁判	サイバンインサイバン 
\\	放火する	ホウカ 
\\	被告	ヒコク 
\\	懲役	チョウエキ 
\\	判決	ハンケツ 
\\	寝袋	ネブクロ 
\\	くるまる	
\\	に)
\\	気の毒に。	
\\	足止めされる	アシドメ 
\\	航空業界	コウクウギョウカイ 
\\	損失	ソンシツ 
\\	額	ガク 
\\	催し物	モヨオシモノ 
\\	多額の利益を得る	
\\	火山噴火	カザンフンカ 
\\	正常化	セイジョウカ 
\\	混乱	コンラン 
\\	欧州	オウシュウ 
\\	航空輸送網	コウクウユソウモウ 
\\	見通し	
\\	[みとおしの悪い道路]//
\\	[来年度の見通し]
\\	火山灰	カザンバイ 
\\	拡散する	カクサン 
\\	欠航	ケッコウ 
\\	構内	
\\	退任	
\\	救済	キュウサイ 
\\	中元	チュウゲン 
\\	世話になった人々に。
\\	推測する	
\\	配慮する	ハイリョ 
\\	配慮に欠ける言動 
\\	産地	サンチ 
\\	製法	セイホウ 
\\	贈り主様	オクリヌシサマ 
\\	主力	シュリョク 
\\	商戦	
\\	百貨店	ヒャッカテン 
\\	目玉商品	メダマ 
\\	こだわり	
\\	に、
\\	のこだわりはない)
\\	有機農法	ユウキノウホウ 
\\	農薬	
\\	栽培する	
\\	サイバイ
\\	冷製	
\\	演出する	
\\	緩める<ー
\\	緩む	ユルメル<ー
\\	ゆるむ=
\\	客に財布の紐を緩める) 
\\	歩調を
\\	大麻	タイマ 
\\	譲渡	ジョウト  
\\	容疑をかける	ヨウギ 
\\	所持する	ショジ 
\\	取り扱い	トリアツカイ 
\\	認めている	
\\	ひき逃げ	
\\	通報する	
\\	罰則	バッソク 
\\	飲酒運転	インシュウンテン 
\\	市道	シドウ 
\\	うつぶせ	
\\	一一〇番	
\\	路面	ロメン 
\\	血痕	ケッコン 血の跡 
\\	捜査する	ソウサ 
\\	身元	ミモト 
\\	身元保証人 
\\	死因	シイン 
\\	特定する	
\\	トクテイ
\\	ついでに	
\\	万歩計	マンポケイ 
\\	付加機能	フカ 
\\	損なう	ソコ 
\\	煙草で健康を損なう 健康を ボールを受け損なう 
\\	分野	ブンヤ 
\\	専門分野
\\	機器	キキ 
\\	型	カタ 
\\	〜割
\\	を占める	シメル 
\\	割合	
\\	大半を占める	
\\	〇〇率	
\\	〜円 等)に達する	
\\	総〇数	
\\	〇
\\	は
\\	を上回る	ウワマワル 
\\	下回る	シタマワル 
\\	わずかに/やや	
\\	はるかに/大きく	
\\	管理する	
\\	いじる	手をつける 
\\	身の回り	ミのマワリ 
\\	設ける	モウケル 
\\	=設置する/ 
\\	機会を設ける 
\\	値上がり	ネアガリ 
\\	発火源	ハッカゲン 
\\	死傷者	シショウシャ 
\\	行為者	コウイシャ 
\\	負傷者	フショウシャ 
\\	政令	セイレイ 
\\	政令指定都市	シレイシテイトシ 
\\	1980年代から1990年代にかけて	
\\	土屋	ツチヤ 
\\	教授	キョウジュ 
\\	定義	テイギ 
\\	我笑う、ゆえに我あり	
\\	称する 名称 自称	ショウ 
\\	弁護士と称する男 
\\	病気だと称して欠席する 
\\	孫と称する人物 メイショウ 
\\	ジショウ 
\\	と->自称の
\\	論理	
\\	論じる	
\\	突拍子もない	トッピョウシ 
\\	ぼんやりと	
\\	スラッシュ	
\\	主題	シュダイ 
\\	映画の主題歌(シュダイカ) 
\\	山のよう(に、な)	
\\	どれだけ	
\\	追加する	ついか 
\\	そこで	
\\	目がチカチカする	
\\	シックハウス症候群	ショウコウグン 
\\	ともに	
\\	木材	モクザイ 
\\	ホルムアルデヒド	
\\	蛋白質	たんぱく質 
\\	特性	トクセイ 
\\	形成する	ケイセイ 
\\	人体	じんたい 
\\	変質	ヘンシツ 
\\	一概に...ない	
\\	(一概に=
\\	僅か	わずか 
\\	満たない	
\\	に (みたない)
\\	物質	
\\	体質	
\\	スズメバチ	
\\	腫れる	ハレル 
\\	アレルギー体質	
\\	ミツバチ	
\\	毒性	
\\	襲う	オソウ 
\\	抗体	コウタイ 
\\	吐き気	ハキケ 
\\	目眩	めまい 
\\	習性	
\\	巣	
\\	ス 鳥の巣 
\\	空き巣 アキス 
\\	空き巣に入られた。
\\	カチカチ	
\\	問題提起	
\\	理屈	りくつ 
\\	一回転	
\\	先頭 
\\	後部	
\\	緩やか	ゆるやか 
\\	一方	
\\	走り抜ける	
\\	本来	
\\	差し掛かる	
\\	1。曲がり角に差し掛かる 
\\	2。子供が思春期に差し掛かる 思春期=ししゅんき 
\\	地点	
\\	ちてん
\\	講習会	
\\	自動体外式除細動器	
\\	心室	
\\	細動	サイドウ 
\\	痙攣	けいれん 
\\	血液	
\\	ケツエキ
\\	設置	
\\	心電図	しんでんず 
\\	測定	そくてい 
\\	仕組み	
\\	世の中の仕組み 
\\	処置	
\\	救命	
\\	事例	
\\	人工呼吸	
\\	応急	
\\	おうきゅう 応急手当 
\\	応急処置をする 
\\	論説(文)	
\\	ラクダ	
\\	分数	
\\	分母	
\\	ぶんぼ
\\	元通り	
\\	損	ソン 
\\	分配	ブノパイ 
\\	利益(りえき 
\\	を分配する
\end{CJK}
\end{document}