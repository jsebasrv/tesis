\documentclass[8pt]{extreport} 
\usepackage{hyperref}
\usepackage{CJKutf8}
\begin{document}
\begin{CJK}{UTF8}{min}
\\	趣味の向上
\\	――青年學生のために――
\\	會津八一
\\	【テキスト中に現れる記号について】
\\	入力者注 主に外字の説明や、傍点の位置の指定
\\	(例)[#3字下げ]
\\	二倍の踊り字(「く」を縦に長くしたような形の繰り返し記号)
\\	(例)そも/\
\\	「それは意見の相違だ」と互に頑張りあつて、相下らない。こんな事は世間の政治家の間などには、珍らしくも無くなつて仕舞つたが、「趣味の相違」といふ捨科白を美術や文學などに心を寄せる人々との間にも折々聞かされるので、其度毎に私はいやな思ひをする。世の中がデモクラチックになつて行くに從つて、意見の相違も重大さを増して來るであらうし、文藝上の事も畢竟趣味の相違に、あらゆる議論が歸着するかもしれぬが、それは究竟地のことであつて、最初から「趣味の相違」を持ち出すのは不謹愼な、そして危險千萬な話である。
\\	趣味には相違といふ事のほかに階級がある。即ち高い低いがある、淺い深いがある、精粗の別がある、あらゆる人のあらゆる趣味を同一の平面上に配列して、それをみんな互角だとするのは寧ろ突飛な、亂暴な仕業だ。或る人は高く、深く、練れた趣味を持ち、又或る人は淺く、低く、なまな趣味よりか持たぬといふ事は、實際目の前にいくらもある事だ。そんな場合にでも『それは趣味の相違です』と澄まし返つて居る譯には行くまい。
\\	低い所から高い方へ登るのは、骨が折れるものだ。しかし骨を折ればこそ高くもなるのだ。高くなればこそ骨も折れるのだ。したがつて骨が折れたゞけの效能もなければならぬ。水は低きに赴く。趣味も、多くの人の信ずる如く、唯だ 
\\	な樂みといふだけを能事とするならば、よし低下するとも、向上などはあり得ない。趣味を享樂そのものと誤解したり、「趣味の相違」を楯に取つて澄ましたりして居れば、低下、墮落は請合である。
\\	よく世間では、趣味の享樂に大騷ぎをして居る一群の人々がある――私自身もそんな連中の一人だと折々誤認されるのであるが――その種類の人達は趣味といふ物を人間の生活から引き離して(つまり人間生活の中から趣味的なエッセンスだけを蒸溜でもして)趣味そのものだけを樂しまうとするのであるが、これは私にとつては、甚だ感服せぬのである。ラティン語の諺に
\\	[#3字下げ]
\\	といふことがある。これは「文學の無い生活は死なり」といふ意味になるが、今では、これから更に一歩を進めて、吾々の生活の中から、文學とか美術とかいふ物だけを引き抽いて、其他は總て捨てゝ仕舞つて、唯だこれだけを樂んで行かうといふ風に、餘程文藝趣味の享樂に重きを置いて考へる傾向が生じて來た。なるほど其れ位の勢でなければ、進歩もなし得ないかもしれぬ。そして又其の人の熱心の程度によつては無理も無い事でもあらうが、ともかくも全的な生活から趣味だけを引き離すことがそも/\吾々を遠い謬見に導き去る第一歩だ。
\\	文學といふ言葉もなく、美術といふ名もなく、只だ此の世の中で何か一つの仕事を見出して、それに從事し、沒頭して居るうちに、何となく己れに特有の樂みが湧く、それも半ば無意識的に。そして樂むといふのでもなく、其癖知らず識らずの間に樂しく日を送ることがあるならば、其時こそ眞に趣味生活の第一歩ではあるまいか。働くことが樂しく、日を送ることが樂しく、生きて居ることが樂しく、そして其より外に何らの樂みも無く、また何等の樂みを求めない。これがほんたうの生活の趣味であり、又趣味の生活である。だから私は、「文藝なき生活は死なり」といふよりも、「生活なき文藝は死なり」といひたい位に思つて居る。
\\	私がこれまでに拜見した坪内逍遙先生の和歌の中に、かういふのがあつて、私は實に感服して居る。
\\	[#3字下げ]人みなのすさびを吾はつとめとす此つとめ無くば吾生けらんや
\\	何人も知る如く、先生は我が國文壇の最高權威であつて、ことに其半生を劇の研究に捧げてゐられる。劇といふものはいはゞ社會の娯樂機關であるから、一般の人は唯だこれを見て樂しむ。そして所謂見物氣分、物見遊山の氣分で、懷手か何かで、いかにも暢氣なものである。ところが其を畢生の研究の對象とせらるゝ先生などは、一生涯の重苦しい負擔のやうに考へて居られる。即ち、先生にとつては、御芝居そのものは最早單純な安易な快樂ではなく、絶えざる苦心と焦慮と、勞役と、憧憬と、向上との對象である。先生の歌は之を歌つて居られる。そして此辛勞と勞作とあればこそ、此一生が先生自身にとつて極めて生き甲斐あるものなりと歌つて居られるのである。私が感服するのは主として此點である。一體、無責任に、面白半分に芝居見物をして樂しむ人々が眞の意味に於て果して樂しいのであらうか。もし樂しいとしても、それが最も高尚な樂みであるのであらうか。それとも、芝居を一生涯手にかけながら、努力で一貫して居られる先生の方が、更により多く、より深く、樂しいのであらうか。
\\	實際、詩人にしろ、小説家にしろ、畫家にしろ、(或は喜劇役者でも、漫畫家でも、落語家でも)他人に樂みを與へるために、自分では何程か苦まぬ人はない、さうして初めて他人にも趣味や快樂を感ぜしめるほどのものが生れ出るのである。……といつても文藝家は總て他人の趣味や感興に媚びるために全力を盡すべきだといふのではない。私のいふのは、自分だけの獨よがりの淺はかな、趣味的滿足では、到底ろくなものが出來やうわけがないと云ふのだ。そんな事で他人が承知せぬことは勿論であるが、よく落ち着いて考へたならば、自分自身をさへ眞に滿足させて居らぬことに氣づかねばならぬ。
\\	若し吾々の生活そのものを樂み、その中に趣味を見出し、時としては他人の爲めに其樂みを作り出すとも、決して、他人の手によつて作り上げられた趣味の供給のみを期待せぬやうな態度を吾々が持ち得るならば、それこそ實に一種崇高な、嚴肅な、眞面目な、積極的な態度と云はねばならぬ。しかるに、若し抽き出されたる趣味そのものゝ享樂のみを要望するならば、それこそ立派な 
\\	で、たゞ酒に狂ひ色に耽るに代ふるに文章や音樂や繪畫を以てするに過ぎない。其態度は極めて消極的。遂に吾等を危險に導かずには置かない。
\\	吾々――ことに青年たる吾々が、志を勵まし、勇を鼓して、望多き門出を試みるときには、老人たちのするやうに、片下りの樂な路ばかりを擇ぶやうではいけない。かならず一歩一歩に高きに登り行くだけの意氣があらねばならぬ。それが樂でないことはわかり切つて居る。しかし、それでこそ、その中にだら/\した下り坂を、底知らずに下つて行くよりも、一層深い、根強い、意味のある樂みも籠つて居るのである。
\\	底本:「會津八一全集 第七卷」中央公論社
\\	1982(昭和57)年4月25日初版発行
\\	底本の親本:「會津八一全集 第七卷」中央公論社
\\	1969(昭和44)年刊行
\\	初出:「興風」
\\	1924(大正13)年12月
\\	入力:フクポー
\\	校正:鴨川佳一郎
\\	2017年10月25日作成
\\	青空文庫作成ファイル:
\\	このファイルは、インターネットの図書館、青空文庫
\\	で作られました。入力、校正、制作にあたったのは、ボランティアの皆さんです。
\end{CJK}
\end{document}