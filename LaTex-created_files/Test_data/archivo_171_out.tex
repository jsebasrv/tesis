\documentclass[8pt]{extreport} 
\usepackage{hyperref}
\usepackage{CJKutf8}
\begin{document}
\begin{CJK}{UTF8}{min}
\\	のたまふ・のたまはす		ハ行四段サ行下二段	おっしゃる(言ふの尊敬語)
\\	仰おほす		サ行下二段	おっしゃる(言ふの尊敬語)お命じになる・命令なさる(命ずの尊敬語)
\\	聞こゆ・聞こえさす		ヤ行上二段サ行下二段	申し上げる(謙譲)(お)~申し上げる(謙譲の補助動詞)
\\	申まうす		サ行四段	申し上げる(言ふの謙譲語)(お)~申し上げる(謙譲の補助動詞)
\\	奏そうす		サ行変格	(帝に)申し上げる(言ふの謙譲語)
\\	啓けいす		サ行変格	(中宮・東宮に)申し上げる(言ふの謙譲語)
\\	承うけたまわる		ラ行四段	お受けする・お聞きする(受く・聞くの謙譲語)
\\	給たまふ		
\\	・ハ行四段
\\	・ハ行下二段	
\\	お与えになる・下さる(与ふの尊敬語)~なさる・お~になる(尊敬の補助動詞)
\\	~(ており)ます(謙譲の補助動詞)
\\	たまはす		サ行下二段	お与えになる・下さる(与ふの尊敬語)
\\	たまはる		ラ行四段	いただく(受く・もらうの尊敬語)お与えになる・下さる(与ふの尊敬語)
\\	召めす		サ行四段	お呼びになる(尊敬)お取り寄せになる(尊敬)お召しになる(着るの尊敬)お乗りになる(尊敬)召し上がる(尊敬)
\\	思おぼす・思し召おぼしめす		サ行四段	お思いになる(尊敬)
\\	大殿籠おほとのごもる		ラ行四段	おやすみになる(尊敬)
\\	候さぶらふ・候さうらふ		ヤ行四段	(貴人の 側そば に)お仕え申し上げる・同候する(仕ふの謙譲語)あります・います・おります(あり・おりの丁寧語)~です・~(でござい)ます・~ています(丁寧の補助動詞)
\\	侍はべり		ラ行変格	(貴人の 側そば に)お仕え申し上げる・同候する(仕ふの謙譲語)あります・います・おります(あり・おりの丁寧語)~です・~(でござい)ます・~ています(丁寧の補助動詞)
\\	奉たてまつる		ラ行四段	差し上げる(尊敬)(お)~申し上げる(謙譲の補助動詞)お召しになる(着るの尊敬語)お召し上がる・お飲みになる(尊敬)お乗りになる(尊敬)
\\	参らす		サ行下二段	差し上げる(謙譲)(お)~申し上げる(謙譲の補助動詞)
\\	参まゐる・詣まうづ		ラ行四段ダ行下二段	参上する(謙譲語)(貴人に何かをして)差し上げる(与ふの謙譲語)召し上がる(尊敬語)
\\	まかる・まかづ		ラ行四段ダ行下二段	(貴所から)退出する・(都から地方へ) 赴おもむく・下る(謙譲)出かけます・参ります(丁寧語)(動詞の上に付き、謙譲や丁寧の意を表す)
\\	遣つかはす		サ行四段	(人や物を)おやりになる(尊敬)行かせる・やる(尊敬語ではない)
\\	仕つか(う)まつる		ラ行四段	お仕え申し上げる(謙譲)(何かを)し申し上げる(「す」の謙譲語)
\\	聞きこし召めす		サ行四段	お聞きになる(尊敬)召し上がる・お飲みになる(尊敬)
\\	しろしめす		サ行四段	知っていらっしゃる(尊敬)お治めになる(尊敬)
\\	おはす・おはします		サ行変格サ行四段	いらっしゃる(尊敬)~(て)いらっしゃる(尊敬の補助動詞)
\\	います・ます・まします・いますかり		サ行四段ラ行変格	いらっしゃる(尊敬)~(て)いらっしゃる(尊敬の補助動詞)
\\	御覧ごらんず		サ行変格	御覧になる(尊敬)
\\	ためらふ		ハ行四段	気持ちを沈める・体を休める・病勢を落ち着かせる迷ってぐずぐずずする・ 躊躇ちゅうちょ する・ためらう (現代語と同じ)
\\	休やすらふ		ハ行四段	ためらう立ち止まる・一息入れる・休息する
\\	語かたらふ		ハ行四段	親しく語る・語り合う親しく交際する・うちとける(男女が)契る・夫婦の関係を結ぶ相談する・仲間に引き入れる・説得する
\\	住む		マ行四段	住む (現代語と同じ)(女のもとに) 通かよう・ 通かよって共に暮らす
\\	止やむ		マ行四段	そのままになる・(そのまま)終わる
\\	移うつろふ		ハ行四段	移りゆく変わってゆく・色あせる心変わりする
\\	見出みいだす		サ行四段	外を見る見つけ出す
\\	もてなす		サ行四段	振る舞う処理する・待遇する・取り扱う・世話をするもてはやす・大切にする
\\	あつかふ		ハ行四段	面倒を見る・世話をする・看病するもてあます
\\	あくがる		ラ行下二段	(魂が体から抜け出て)さまよう・うわの空になる(あてもなく)さまよい歩く・浮かれ出る
\\	合あ(会・逢)ふ		ハ行四段	結婚する・男女が契る
\\	敢あふ		ハ行下二段	(打消を伴い)耐えられない・我慢できない(動詞につき、打消を伴い)~しきれない・~できない(「あへなむ」の形で)かまわないだろう・差し支えないだろう
\\	潮垂しほたる		ラ行下二段	涙を流す・涙で 袖そでが濡ぬれる
\\	掻き暗かきくらす		サ行四段	空を暗くする・辺り一面を暗くする心を暗くする・悲しみにくれる
\\	惑まどふ		ハ行四段	迷う・悩む・途方にくれる・うろたえる・心が乱れる(動詞の連用形について)ひどく(~する)
\\	謀たばかる		ラ行四段	考えをめぐらす・工夫するだます・たくらむ
\\	すさぶ(荒ぶ・進ぶ・遊ぶ)		バ行四段・上二段	(何かに) 興きょうじる・楽しむ・気の向くまま(何かを)する・ 慰なぐさみに(何かを)する
\\	すまふ(争う・辞ふ)		ハ行四段	抵抗する・争う辞退する・断る
\\	学まねぶ		バ行四段	まねる(見聞した物事をそのまま)伝える・書き記す
\\	ねぶ		バ行上二段	年をとる・成長する大人びる・ませる
\\	掟おきつ		タ行下二段	指図さしずする・命令する計画する・決めておく・処理する
\\	愁うれふ		ハ行下二段	(嘆き)訴える・不平を言う
\\	結むすぶ		バ行四段	作る・できる契る・約束する
\\	掬むすぶ		バ行四段	(両手で水などを)すくう
\\	訪とぶらふ		ハ行四段	訪ねる・(様子を)尋ねる・探し求める見舞う
\\	弔とぶらふ		ハ行四段	弔とむらう
\\	弔問ちょうもんする・供養くようする)人の死を悲しみいたんで、遺族におくやみを言う。
\\	やつす		サ行四段	目立たない服装(様子)にする・質素にする・粗末な服装(様子)出家する
\\	障さはる		ラ行四段	差し障さしさわる・ 差し支さしつかえる・妨げられる
\\	畏かしこまる		ラ行四段	恐縮する・恐縮して正座する・お礼を申し上げる・謹慎する
\\	託かこつ		タ行四段	嘆く・不平を言う
\\	分わく		カ行四段	分ける・区別する(現代語と同じ)理解する
\\	慎つつむ		マ行四段	遠慮する・はばかる
\\	あらまほし		シク活用	理想的だ・望ましい
\\	らうらうじ		シク活用	巧みだ・もの慣れている・気がきいている・配慮が行き届いている上品だ・うつくしい
\\	うるせし		ク活用	賢い・優れている巧みだ・優れている
\\	はかばかし		シク活用	しっかりしている・きちんとしているはっきりしている
\\	をさをさし		シク活用	しっかりしている大人びている
\\	さうなし		ク活用	比べるものがない・すばらしい
\\	くまなし		ク活用	暗いところがない・影がない何でも知っている・行きとどいている
\\	ずちなし	(術無し)	ク活用	どうしようもない・手のうちようがない・困り果てる
\\	さがなし		ク活用	意地が悪い・口やかましいいたずらだ
\\	あいなし		ク活用	気に入らない・つまらない・不都合であるなんとなく・わけもなく・むやみに
\\	まさなし		ク活用	よくない・見苦しい
\\	はかなし		ク活用	頼りない・むなしい・はかないちょっとしたことだ・取るに足りない・たわいない
\\	こころづきなし		ク活用	気にくわない・好感がもてない・心ひかれない
\\	あへなし		ク活用	期待はずれだ・落胆している・張り合いがないはかない・どうしようもない
\\	おほけなし		ク活用	身のほど知らずだ・分不相応だ・おそれ多い
\\	よしなし		ク活用	理由がない方法がないつまらない・意味がない関係がない・縁がない
\\	はしたなし		ク活用	中途半端だ・どっちつかずだ(どっちつかずで)いたたまれない・きまり悪い(いたたまれないほど)そっけない・愛想がないはなはだしい・激しい
\\	しどけなし		ク活用	だらしない・乱れている・怠慢だ無造作だ・くつろいでいる・気楽だ・ゆったりしている
\\	いぎたなし		ク活用	寝坊だ・ぐっすり寝込んでいる
\\	ひとわろし		ク活用	みっともない・体裁が悪い
\\	いぶせし		ク活用	うっとうしい・気が晴れない・不快だ気がかりだ・気になる
\\	かたはらいたし		ク活用	苦々しい・腹立たしい・見苦しい・聞き苦しい・みっともない気の毒だ・心苦しいきまり悪い・恥ずかしい
\\	こちたし		ク活用	(噂や評判が)うるさい・煩わしい大げさだ・仰々しい非常に多い・はなはだしい・おびただしい
\\	ねたし		ク活用	くやしい・しゃくだ・にくらしい
\\	けし		シク活用	異様だ・変だ・よくない
\\	わびし		シク活用	つらい・苦しい・やりきれない・困ったことだわびしい・心細い・貧しい
\\	こころぐるし		シク活用	気の毒だ・かわいそうだ・つらい気がかりだ・心配だ
\\	まだし		シク活用	まだ早い・まだその時期ではない・未熟だ
\\	さかし		シク活用	賢い・優れている・巧みだしっかりしている・気丈だ利口ぶっている・こざかしい
\\	まばゆし		ク活用	まぶしい美しい・立派だ・すばらしいきまりが悪い・恥ずかしいみていられない・(目を背けたいほど)いやだ
\\	かたじけなし		ク活用	おそれ多い・申し訳ないおそれ多い・もったいない・ありがたい恥ずかしい・面目ない
\\	かしこし		ク活用	おそれ多い・もったいない・尊い賢い・優れている・立派だ・ありがたい都合がよい・運がよい・うまいはなはだしく
\\	しげし		ク活用	(何かが)多い
\\	すごし		ク活用	気味が悪い・ぞっとするほど恐ろしいさびしい・殺風景だ(ぞっとするほど)すばらしい・うつくしい
\\	いたし		ク活用	とてもよい・すばらしいとても悪い・ひどいとても・はなはだしくそれほど・あまり
\\	見る		マ行上一段	見る・会う・眺める思う・わかる関係を結ぶ・結婚する・妻とする面倒を見る・世話をする
\\	見す		サ行下二段	見せる・見させる結婚させる
\\	見ゆ		ヤ行下二段	見える・思われる見られる・見せる(女性が)結婚する
\\	かいまみる	(垣間見る)	マ行上一段	のぞき見る
\\	よばふ		八行四段	呼び続ける求婚する
\\	好く		カ行四段	風流を好む・芸道に熱中する
\\	わたる		ラ行四段	行く、来る、通る(~で)いらっしゃる (年月を)経過する・時を過ごす
\\	ありく	(歩く)	カ行四段	歩き回る・出歩くあちこちで~してまわる
\\	おこなふ	(行ふ)	ハ行四段	仏道修行をする・勤行する
\\	悩む		マ行四段	病気になる・病気で苦しむ・患う
\\	おこたる	(怠る)	ラ行四段	病気が回復する
\\	おくる	(後る・遅る)	ラ行下二段	人に先立たれる・死に遅れる
\\	ながむ	
\\	眺む・
\\	詠む)	マ行下二段	
\\	もの思いに沈む・物思いにふける・物思いしながらぼんやり見る
\\	(和歌を)口ずさむ
\\	ときめく	(時めく)	カ行四段	寵愛を受ける・かわいがられる
\\	かしづく		カ行四段	大事に養い育てる
\\	めづ	(愛づ)	ダ行下二段	愛する・好む・かわいがる・気に入るほめる・たたえる・感嘆する・賞美する
\\	おどろく		カ行四段	目を覚ますはっと気づくおどろく・びっくりする
\\	こうず	(困ず)	サ行変格	疲れる・苦しい悩む・困る・つらく感じる
\\	おぼゆ	(覚ゆ)	ヤ行下二段	思われる・そのような気がする思い出す・思い出される似る・似ている
\\	聞こゆ		ヤ行下二段	聞こえる・耳に入るうわさされる・評判になる・世間に知られる理解される・わけがわかる
\\	まもる・まぼる	(守る)	ラ行四段	じっと見つめる
\\	たのむ	(頼む)	
\\	マ行四段・
\\	マ行下二段	
\\	あてにする・頼りにする・期待する
\\	あてにさせる・頼りにさせる・期待される
\\	かづく	(被く)	
\\	カ行四段・
\\	カ行下二段	
\\	かぶる 
\\	いただく
\\	かぶせる
\\	与える
\\	ののしる		ラ行四段	大声を出す・大声で騒ぐ評判になる・噂になる勢いが盛んである・羽振りをきかす
\\	やる・やらず	(遣る)	ラ行四段	(人を)行かせる・(物)を送る〈動詞
\\	「やらず」の形で〉 ~しきれない・最後まで~ない
\\	いらふ	(答ふ)	ハ行下二段	答える・返事する
\\	にほふ	(匂ふ)	ハ行四段	美しく映える・美しく輝く香りがする
\\	あきらむ	(明らむ)	マ行下二段	明らかにする・はっきりさせる
\\	ねんず	(念ず)	サ行変格	我慢する祈る
\\	まうく	(設く)	カ行下二段	準備する・用意する
\\	ゐる	
\\	居る・
\\	率いる)	ワ行上一段	
\\	座る
\\	じっとしている・留まる
\\	(~し)ている…しつづける
\\	連れていく・伴う
\\	具す		サ行変格	伴う・連れる・添える
\\	経	(ふ)	ハ行下二段	たつ(経過する)通る(経由する)
\\	さる	
\\	避る・
\\	去る)	ラ行四段	
\\	避ける
\\	(時期が)くる・(そのときに)なる 立ち去る・過ぎ去る・移る
\\	ものす	(物す)	サ行変格	(ものごとを)するいらっしゃる
\\	ならふ		八行四段	慣れる・習慣となる親しくなる・なじむ
\\	しのぶ	
\\	忍ぶ・
\\	偲ぶ)	バ行上二段・四段	
\\	がまんする・気持をこらえる人目につかないようにする
\\	思い出す・恋い慕う
\\	わぶ	(侘ぶ)	バ行上二段	困る・嘆く・つらく思う・思い悩む〈動詞の連用形について〉 ~しかねる・容易に~できない
\\	をかし		シク活用	すばらしい・興味深いおかしい・こっけいだ
\\	よろし		シク活用	悪くはない・すばらしい普通だ・たいしたことはない
\\	ありがたし		ク活用	めったにない・まれである・珍しいすばらしい
\\	つきづきし		シク活用	似つかわしい・ふさわしい
\\	なまめかし		シク活用	上品だ・優美である若々しい・みずみずしい
\\	めでたし		ク活用	すばらしい・立派だ・すぐれている
\\	うるはし		シク活用	きちんとしている・整っている・端正である美しい・見事だ
\\	やむごとなし		ク活用	高貴だ・尊い格別だ・並々でない・優れている・この上ない
\\	おとなし		シク活用	大人らしい・大人びている思慮分別がある・落ち着いている穏やかだ・素直だ
\\	ゆかし		シク活用	見たい・聞きたい・知りたい・心ひかれる恋しい・なつかしい・慕わしい
\\	なつかし		シク活用	親しみ深い・心ひかれる
\\	はづかし	(恥ずかし)	シク活用	恥ずかしい・きまりが悪い・気がひける立派だ・すぐれている
\\	こころにくし		ク活用	奥ゆかしい・上品だ・心ひかれる
\\	うつくし		シク活用	かわいい・いとしい美しい
\\	かなし		シク活用	いとしい・かわいい悲しい・あわれだ・残念だ
\\	らうたし		ク活用	かわいい・いじらしい
\\	めやすし	(目安し)	ク活用	(見た目の)感じがよい・見苦しくない
\\	あやし	
\\	怪し・
\\	賤し)	シク活用	
\\	不思議だ・神秘的だ
\\	身分が低い・卑しい
\\	粗末だ・みすぼらしい
\\	くちをし	(口惜し)	シク活用	残念だ・いやだ・なさけない
\\	あたらし	(惜し)	シク活用	惜しい・もったいない・つまらない
\\	むつかし		シク活用	不快だ・わずらわしい・うっとうしい気味が悪い・恐ろしい
\\	つれなし		ク活用	平然としている・平気だ・素知らぬ顔をしている冷淡だ・薄情だ・無関心である
\\	なめし		ク活用	無礼だ・無作法だ
\\	おどろおどろし		シク活用	大げさだ・仰々しい気味が悪い・恐ろしい
\\	憂し		ク活用	つらい・いやだ・憂鬱だ・憎い
\\	びんなし	(便なし)	ク活用	不都合だ・具合が悪い気の毒だ・かわいそうだ
\\	いとほし		シク活用	気の毒だ・かわいそうだいとしい・かわいい・いじらしい
\\	いはけなし		ク活用	幼い・あどけない・子どもっぽい・幼稚である
\\	さうざうし		シク活用	もの足りない・さびしい・飽き足りない
\\	すさまじ		シク活用	興ざめだ・おもしろくない・魅力がない殺風景だ・寒々としている・荒涼としている
\\	つらし		ク活用	薄情だ・冷淡だ・思いやりがないつらい・耐えがたい
\\	ところせし	(所狭し)	ク活用	窮屈だ・仰々しい気詰まりだ
\\	うしろめたし		ク活用	気がかりだ・不安だ・心もとない
\\	わりなし		ク活用	道理に合わない・ひどい・無理やりだしかたがない・どうしようもない・耐えがたい苦しい・つらい・困っている
\\	本意なし		ク活用	不本意だ・残念だ・がっかりだ・物足りない
\\	あさまし		シク活用	驚くほどだ・意外だあきれるほどだ・がっかりだ情けない・嘆かわしい
\\	めざまし		シク活用	気にくわない・目障りだすばらしい・立派だ・目が覚めるほどすばらしい
\\	いみじ		シク活用	とてもよい・素晴らしい・立派だとても悪い・ひどい・恐ろしいとても・甚だしく
\\	ゆゆし		シク活用	不吉だ・縁起が悪い・君が悪い・恐ろしいとてもよい・すばらしいとても・ 甚はなは だしく
\\	やさし		シク活用	恥ずかしい・きまり悪い優美だ・上品だ・風流だけなげだ・ 殊勝しゅしょうだ・感心だ
\\	しるし		ク活用	はっきりとわかる・顕著だ~のとおりに・予想通りだ
\\	とし	(疾し)	ク活用	早い・速い
\\	ゆくりなし		ク活用	突然だ・思いがけない
\\	おぼつかなし		ク活用	はっきりしない・ぼんやりしている気がかりだ・不安だ・疑わしい待ち遠しい・じれったい
\\	心もとなし		ク活用	はっきりしない・かすかだ気がかりだ・不安だ待ち遠しい・じれったい
\\	あはれなり		ナリ活用	しみじみと心に深く感じられる
\\	つれづれなり	(徒然なり)	ナリ活用	退屈であるものさびしい
\\	すずろなり・そぞろなり	(漫ろなり)	ナリ活用	なんということもない思いがけないむやみやたらだ
\\	まめなり・まめやかなり		ナリ活用	まじめである・誠実である実用的である・実際的なさま
\\	あだなり	(徒なり)	ナリ活用	はかない・頼りにならない・もろい誠実さがない・浮ついている・浮気だ
\\	いたづらなり	(徒らなり)	ナリ活用	役に立たない・むだだ・かいがないむなしい・はかない
\\	優なり		ナリ活用	優れている・すばらしく立派だ上品で美しい・優雅である
\\	あてなり	(貴なり)	ナリ活用	高貴である・身分が高い気品がある・上品である
\\	あからさまなり		ナリ活用	ほんのちょっと・一時的にほんの・少しも・まったく
\\	みそかなり	(密かなり)	ナリ活用	ひそかに・こっそりと
\\	おろかなり	(疎かなり)	ナリ活用	いい加減だ・疎略だ並ひととおりだ~という言葉では言い表せない
\\	をこなり		ナリ活用	愚かだ・まぬけだ・ばかばかしい
\\	むげなり	(無下なり)	ナリ活用	ひどい・最低だ・最悪だむやみに・ひどく身分が低い
\\	なかなかなり		ナリ活用	中途半端だ・不十分だかえってしないほうがよい・なまじっか
\\	消息	(せうそこ)	名詞	手紙訪問の申し入れ・訪問のあいさつ
\\	手		名詞	字・筆跡演奏法・曲手段・方法・やり方傷・負傷
\\	文		名詞	手紙漢詩・漢文・漢字
\\	あそび	(遊び)	名詞	遊び管弦の遊び
\\	うへ	(上)	名詞	帝・天皇奥様~のこと
\\	あほやけ	(公)	名詞	朝廷・政府帝公的なこと
\\	うち	(内裏・内)	名詞	内部宮中・内裏帝・天皇
\\	御前		名詞	御前・お側貴人への敬称
\\	みゆき	(行幸・御幸)	名詞	(帝や天皇の)お出かけ・お出まし
\\	たより	(頼り・便り)	名詞	生活のよりどころ・頼みにできるものつて・縁故・手づるよい機会・都合のよい折・ついで
\\	物語		名詞	世間話・雑談物語
\\	ためし	(例)	名詞	前例・先例
\\	いそぎ	(急ぎ)	名詞	急なこと・急用支度・準備
\\	用意		名詞	気配り・心遣い・配慮容貌・顔立ち
\\	かたち	(形・容貌)	名詞	形・外見・外形容貌・顔立ち
\\	かげ	
\\	影・
\\	陰)	名詞	
\\	光
\\	姿・形
\\	陰・物陰
\\	けしき	(気色)	名詞	様子・ありさま顔色・機嫌・態度・そぶり意向・意中・内意・内実
\\	こころざし	(心ざし・志)	名詞	意志・意向・心が向かうところ愛情・好意・誠意お礼の贈り物・謝礼
\\	本意		名詞	本来の意志・かねてからの願い
\\	こと	(言)	名詞	言葉・言語和歌
\\	わざ	(業)	名詞	こと(事)・行い葬儀・法事
\\	よろづ	(万)	名詞	多くの数・さまざま・万事何ごとにつけても・すべてにわたって
\\	ことわり	(理)	名詞	道理・筋道
\\	ひがこと・ひがごと	(僻事)	名詞	間違い・誤り
\\	そらごと	(空言・虚言)	名詞	嘘・偽り・作りごと
\\	しるし	(験・徴)	名詞	効き目・効果霊験・ご利益前兆・きざし
\\	料	(れう)	名詞	ため・(何かの)ためのもの
\\	禄	(ろく)	名詞	ほうび
\\	としごろ	(年頃・年比・年来)	名詞	数年・数年来・長年
\\	つとめて		名詞	早朝・朝翌朝
\\	世・世の中		名詞	世の中・世間人間関係・男女の仲・夫婦の仲一生・御代
\\	いかで・いかでか		副詞	どうして・どのように~だろうかどうして~だろうか、いや、~ないなんとかして
\\	いかが・いかに		副詞	どのように・どうして~だろうかどうして~だろうか、いや~ないどんなにか・どれほど・なんとまぁ~だろう
\\	など・などか・などて		副詞	なぜ・どうして~か
\\	いつしか		副詞	いつの間にか・早くも(できるだけ)早く
\\	おのづから		副詞	自然と・ひとりで偶然・たまたま万一・もしかして
\\	なほ		副詞	依然としてやはりなんといってもやはり・それでもやはりさらに・いっそう・ますます
\\	いとど		副詞	いっそう・ますますそのうえさらに・ただでさえ・そうでなくとも
\\	げに	(実に)	副詞	本当に・なるほど
\\	かく		副詞	こう・このように
\\	さ		副詞	そう・そのように
\\	しか		副詞	そう・そのように
\\	と		副詞	そのように・あのように
\\	やがて		副詞	そのまま・引き続いてすぐに・すぐさま・ただちに
\\	すなはち		副詞	すぐに・ただちに・即刻つまり・言い換えれば
\\	やうやう		副詞	しだいに・だんだんとやっと・かろうじて
\\	やをら・やはら		副詞	そっと・静かに・おもむろに
\\	なかなか(に)		副詞	なまじ・なまじっかかえって・むしろ
\\	さすがに		副詞	そうはいってもやはりなんといってもやはり
\\	かたみに	(互に)	副詞	お互いに・かわるがわる
\\	うたて		副詞	不快に・いやな感じに気味悪く・異様に怪しく情けない・いやだ
\\	なべて		副詞	一般に・すべて・みな同様に並一通り・普通・平凡
\\	わざと		副詞	わざわざ特に・とりわけ・格別に正式な・本格的な
\\	あまた	(数多)	副詞	たくさん・数多く
\\	ここら・そこら		副詞	たくさん・数多くたいそう・はなはだしく
\\	え		副詞	~することができない
\\	な~そ		副詞	~するな・~しないでくれ・~してはならない
\\	おほかた(~打消)	(大方)	副詞	まったく・少しも・決して(~ない)だいたい・概して (※打消を伴わない)
\\	さらに(~打消)	(更に)	副詞	まったく・少しも・決して(~ない)実に (※打消を伴わない)
\\	世に(~打消)		副詞	まったく・少しも・決して(~ない)実に・非常に (※打消を伴わない)
\\	絶えて(~打消)	(絶えて)	副詞	まったく・少しも・決して(~ない)
\\	つゆ(~打消)		副詞	まったく・少しも・決して(~ない)
\\	ゆめ・ゆめゆめ(~打消・禁止)		副詞	まったく・少しも・決して(~ない)決して(~するな)
\\	つやつや(~打消)		副詞	まったく・少しも・決して(~ない)
\\	をさをさ(~打消)		副詞	ほとんど(~ない)
\\	よも(~打消推量)		副詞	まさか (~ないだろう)
\\	あなかしこ(禁止)		副詞	決して (~するな)
\\	おぼろけなり		ナリ活用	並ひととおりだ・ありきたりだ並ひととおりではない・格別だ
\\	なのめなり	(斜めなり)	ナリ活用	並ひととおりだ・普通だ・平凡だ不十分だ・いい加減だ並ひととおりでない・格別だ
\\	清らなり・けうらなり		ナリ活用	清らかで美しいきちんとしている・整っていて美しい
\\	まほなり		ナリ活用	欠点がなくすぐれている・完全であるまともであるさま
\\	あらはなり	(顕なり)	ナリ活用	はっきり見える・まる見えである明らかである・露骨である
\\	あながちなり	(強ちなり)	ナリ活用	強引だ・無理やりだ・いちずだむやみに・あまりに
\\	せちなり	(切なり)	ナリ活用	切実である・ひたすらだ大切である・さしせまっている
\\	とみなり	(頓なり)	ナリ活用	急だ
\\	うちつけなり		ナリ活用	急だ・突然だ・にわかだ軽率だ・考えが浅い・場当たりだ
\\	さらなり	(更なり)	ナリ活用	言うまでもない・もちろんだ
\\	ねんごろなり・ねむごろなり		ナリ活用	心をこめて丁寧だ・熱心だ親密だ・仲むつまじい
\\	おいらかなり		ナリ活用	穏やかだ・おっとりしている率直である・素直である
\\	あやにくなり		ナリ活用	意地が悪い・にくらしいほどひどい都合が悪い・間が悪い・あいにくだ
\\	おぼえ	(覚え)	名詞	評判・人望・うわさ・評論 寵愛ちょうあい ・信任
\\	ひま	(隙・暇)	名詞	間・すき間絶え間・合間時間的ゆとり・暇・自由な時間
\\	いとま	(暇)	名詞	時間的ゆとり・暇・余裕休むこと・勤めをやめること・休暇別れること
\\	才		名詞	教養・学才・学問技能・才能・美術
\\	よろこび	(喜び)	名詞	喜び・お祝いお礼
\\	こころばへ	(心ばへ)	名詞	気立て・性質心遣い・気配り 趣おもむき ・風情・趣意
\\	こころづくし	(心尽くし)	名詞	もの思いをすること気をもむこと・心労の多いこと
\\	そこ		名詞	あなた(二人称)
\\	ここ		名詞	このわたし(一人称)あなた(二人称)
\\	かれ		名詞	あの人あれ・あのもの
\\	それ		名詞	その人あなた(二人称)
\\	これ		名詞	この人このわたし(一人称)あなた(二人称)
\\	あなた		名詞	あちら・向こう
\\	そなた		名詞	そちら・その方面あなた・お前
\\	こなた		名詞	こちら・こっちこのわたし(一人称)
\\	そのかみ		名詞	その当時・その昔・往時
\\	せうと	(兄人)	名詞	兄・弟
\\	おとうと・おとと	(弟)	名詞	弟・妹
\\	いも	(妹)	名詞	いとしいあなた(妻・恋人)
\\	つま	(夫・妻)	名詞	夫・妻
\\	はらから	(同胞)	名詞	兄弟・姉妹
\\	かたへ	(片方)	名詞	片方・半分・一部分側・傍ら・そば・身近仲間・同僚・兄弟
\\	ほど	(程)	名詞	間・うち・時・ころ距離・広さ・あたり身分・地位・年齢様子・ありさま
\\	かぎり	(限り)	名詞	限度・限界・期限時期・機会・あいだ最後・臨終全部・すべて
\\	きは	(際)	名詞	端・限り・終わり・限界程度・身分
\\	ついで	(序)	名詞	序列・順序機会・折・場合
\\	沙汰		名詞	評議・裁き・処置命令・指図評判・うわさ
\\	とが	(咎・利)	名詞	欠点・短所過失・罪
\\	け	(故)	名詞	ゆえ・ため・せい
\\	よし	(由)	名詞	趣・風情・いきさつ 由緒ゆいしょ ・由来・理由手段・方法・手立てこと・旨
\\	やう	(様)	名詞	様子・ありさま・姿理由・わけ・事情方法・手段・やり方~ことには
\\	ちぎり	(契り)	名詞	約束・契約前世からの約束・因縁・宿縁
\\	ほだし	(絆)	名詞	束縛するもの・障害となるもの
\\	あやめ	(文目)	名詞	(物の)模様・形・区別(ものごとの)道理・筋道・分別
\\	うつつ	(現)	名詞	現実正気
\\	あるじ	
\\	主・
\\	餐)	名詞	
\\	主人・主君
\\	饗応・もてなし・接待
\\	ふるさと	(古里・故郷)	名詞	なじみの土地・生まれ故郷わが家・自宅旧都
\\	さて	(然て)	副詞	そういう状態で・そのまま・そうしてそれ以外・そのほか
\\	さながら	(然ながら)	副詞	そのまま・もとのまますべて・ことごとく・全部
\\	いま	(今)	副詞	まもなく・すぐにさらに・もう
\\	せめて		副詞	無理に・強いて痛切に・切実に・ひどく
\\	むべ・うべ		副詞	なるほど・いかにも
\\	かつ		副詞	一方ではすぐに・たちまち
\\	ひねもす	(終日)	副詞	一日中・朝から晩まで
\\	かまへて	(~打消・禁止)	副詞	決して・絶対に(~ない・~するな)きっと・必ず・なんとかして
\\	あへて	(~打消)	副詞	まったく・少しも・決して(~ない)
\\	かけて	(~打消)	副詞	まったく・少しも・決して(~ない)
\\	さだめて	(~推量)	副詞	きっと・必ず(~だろう)
\end{CJK}
\end{document}