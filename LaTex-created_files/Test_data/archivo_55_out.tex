\documentclass[8pt]{extreport} 
\usepackage{hyperref}
\usepackage{CJKutf8}
\begin{document}
\begin{CJK}{UTF8}{min}
\\	"ライオン
\\	は
\\	猛獣[もうじゅう]だ。
\\	"ライオン
\\	猛獣[もうじゅう]だ。
\\	"は 
\\	は 
\\	が 
\\	は
\\	"この本[ほん]
\\	は
\\	難[むずか]しい。
\\	"この本[ほん]
\\	難[むずか]しい。
\\	"は 
\\	は 
\\	が 
\\	は
\\	"あの人[ひと]
\\	は
\\	、 山田[やまだ]さんのご 主人[しゅじん]です。
\\	"あの人[ひと]
\\	、 山田[やまだ]さんのご 主人[しゅじん]です。
\\	"は 
\\	は 
\\	が 
\\	は
\\	"私[わたし]
\\	は
\\	行[い]きません。
\\	"私[わたし]
\\	行[い]きません。
\\	は 
\\	は
\\	"ボールペン
\\	は
\\	ありません。
\\	"ボールペン
\\	ありません。
\\	は 
\\	は
\\	"お金[かね]
\\	は
\\	僕[ぼく]が 払[はら]うよ。
\\	"お金[かね]
\\	僕[ぼく]が 払[はら]うよ。
\\	は 
\\	は 
\\	"宿題[しゅくだい]
\\	は
\\	もう 済[す]んだ。
\\	"宿題[しゅくだい]
\\	もう 済[す]んだ。
\\	は 
\\	は
\\	"いつも
\\	は
\\	バスで 行[い]きます。
\\	"いつも
\\	バスで 行[い]きます。
\\	は
\\	"あの 人[ひと]
\\	は
\\	もう 付[つ]き 合[あ]いません。
\\	"あの 人[ひと]
\\	もう 付[つ]き 合[あ]いません。
\\	は
\\	て 
\\	で.	"もう 二度[にど] 頼[たの]み
\\	は
\\	しないからさ。
\\	"もう 二度[にど] 頼[たの]み
\\	しないからさ。
\\	は
\\	て 
\\	で.	"薬[くすり]は、まだ 飲[の]んで
\\	は
\\	いないよ。
\\	"薬[くすり]は、まだ 飲[の]んで
\\	いないよ。
\\	は
\\	て 
\\	で.	"その 噂[うわさ]、 嘘[うそ]で
\\	は
\\	ないのよ。
\\	"その 噂[うわさ]、 嘘[うそ]で
\\	ないのよ。
\\	は
\\	"昼[ひる]
\\	は
\\	暑[あつ]く、 夜[よる]
\\	は
\\	寒[さむ]い。それが 砂漠[さばく]の 気候[きこう]だ。
\\	"昼[ひる]
\\	暑[あつ]く、 夜[よる]
\\	寒[さむ]い。それが 砂漠[さばく]の 気候[きこう]だ。
\\	は
\\	"これ
\\	は
\\	私[わたし]の、それ
\\	は
\\	あなたの。 間違[まちが]えないで。
\\	"これ
\\	私[わたし]の、それ
\\	あなたの。 間違[まちが]えないで。
\\	は
\\	"電話[でんわ]をかけるに
\\	は
\\	かけたが、 手紙[てがみ]も 出[だ]しておこう。
\\	"電話[でんわ]をかけるに
\\	かけたが、 手紙[てがみ]も 出[だ]しておこう。
\\	は
\\	"この 本[ほん]、 読[よ]むに
\\	は
\\	読[よ]んだが、 難[むずか]しくてあまりよく 分[わ]からない。
\\	"この 本[ほん]、 読[よ]むに
\\	読[よ]んだが、 難[むずか]しくてあまりよく 分[わ]からない。
\\	は
\\	"人[ひと]
\\	が
\\	いる (います)。
\\	"人[ひと]
\\	いる (います)。
\\	が
\\	"本[ほん]
\\	が
\\	ある (あります)。
\\	"本[ほん]
\\	ある (あります)。
\\	が
\\	"魚[さかな]
\\	が
\\	いる (います)。
\\	"魚[さかな]
\\	いる (います)。
\\	が
\\	"魚[さかな]
\\	が
\\	ある (あります)。
\\	"魚[さかな]
\\	ある (あります)。
\\	が
\\	"私[わたし]はアイスクリーム
\\	が
\\	食[た]べたい。
\\	"私[わたし]はアイスクリーム
\\	食[た]べたい。
\\	が
\\	"私[わたし]は 英語[えいご]
\\	が
\\	話[はな]せます。
\\	"私[わたし]は 英語[えいご]
\\	話[はな]せます。
\\	が
\\	"彼女[かのじょ]は 音楽[おんがく]
\\	が
\\	好[す]きです。
\\	"彼女[かのじょ]は 音楽[おんがく]
\\	好[す]きです。
\\	が
\\	"誰[だれ]
\\	が
\\	来[き]ましたか。
\\	"誰[だれ]
\\	来[き]ましたか。
\\	が
\\	"あなたはどんな 食[た]べ 物[もの]
\\	が
\\	好[す]きですか。
\\	"あなたはどんな 食[た]べ 物[もの]
\\	好[す]きですか。
\\	が
\\	"何時[なじ]
\\	が
\\	いいですか。
\\	"何時[なじ]
\\	いいですか。
\\	が
\\	"波[なみ]
\\	が
\\	高[たか]い。
\\	"波[なみ]
\\	高[たか]い。
\\	が
\\	"私[わたし]は 気分[きぶん]
\\	が
\\	いいです。
\\	"私[わたし]は 気分[きぶん]
\\	いいです。
\\	が
\\	"部屋[へや]
\\	が
\\	暑[あつ]い。
\\	"部屋[へや]
\\	暑[あつ]い。
\\	が
\\	"いい 匂[にお]い
\\	が
\\	する。
\\	"いい 匂[にお]い
\\	する。
\\	が
\\	"甘[あま]い 味[あじ]あ
\\	が
\\	する。
\\	"甘[あま]い 味[あじ]あ
\\	する。
\\	が
\\	"戸[と]
\\	が
\\	閉[し]まる。
\\	"戸[と]
\\	閉[し]まる。
\\	が
\\	"電気[でんき]
\\	が
\\	消[き]えた。
\\	"電気[でんき]
\\	消[き]えた。
\\	が
\\	"月[つき]
\\	が
\\	出[で]た。
\\	"月[つき]
\\	出[で]た。
\\	が
\\	それ 
\\	ところ, 
\\	"それ
\\	が
\\	、 何[なん]と 物凄[ものそご]く 混[こ]んでいたんだ。
\\	"それ
\\	、 何[なん]と 物凄[ものそご]く 混[こ]んでいたんだ。
\\	が
\\	それ 
\\	ところ, 
\\	"僕[ぼく]はもう 諦[あきら]めていた。 ところ
\\	が
\\	ね、 見[み]つかたんだよ、 何[なん]と。
\\	"僕[ぼく]はもう 諦[あきら]めていた。 ところ
\\	ね、 見[み]つかたんだよ、 何[なん]と。
\\	が
\\	"戦争[せんそう] 記念碑[きねんひ]は、 人間[にんげん]の 争[あらそ]うさがを 悲[かな]しむ
\\	が
\\	ごとく 立[た]っていた。
\\	"戦争[せんそう] 記念碑[きねんひ]は、 人間[にんげん]の 争[あらそ]うさがを 悲[かな]しむ
\\	ごとく 立[た]っていた。
\\	がごとく 
\\	ままに 
\\	が
\\	"彼[から]は、 母[はは]のすすめる
\\	が
\\	ままに 技師[ぎし]になった。
\\	"彼[から]は、 母[はは]のすすめる
\\	ままに 技師[ぎし]になった。
\\	がごとく 
\\	ままに 
\\	が
\\	"その 点[てん]です
\\	が
\\	、もっと 詳[くわ]しく 説明[せつめい]して 下さい。
\\	"その 点[てん]です
\\	、もっと 詳[くわ]しく 説明[せつめい]して 下さい。
\\	が
\\	"すでに、 申[もう]し 上[あ]げました
\\	が
\\	、この 計画[けいかく]の 成功[せいこう]は 間違[まちが]いないものと 思[おも]ういます。
\\	"すでに、 申[もう]し 上[あ]げました
\\	、この 計画[けいかく]の 成功[せいこう]は 間違[まちが]いないものと 思[おも]ういます。
\\	が
\\	"僕[ぼく]はベストをつくした
\\	が
\\	、 駄目[だめ]だった。
\\	"僕[ぼく]はベストをつくした
\\	、 駄目[だめ]だった。
\\	が
\\	"昨日[きのう]お 電話[でんわ]しました
\\	が
\\	、お 留守[るす]でした。
\\	"昨日[きのう]お 電話[でんわ]しました
\\	、お 留守[るす]でした。
\\	が
\\	"僕[ぼく]は 一晩中[ひとばんじゅう] 考[かんが]えた
\\	が
\\	、 名案[めいあん]は 浮[う]かばなかった。
\\	"僕[ぼく]は 一晩中[ひとばんじゅう] 考[かんが]えた
\\	、 名案[めいあん]は 浮[う]かばなかった。
\\	が
\\	"僕[ぼく]がどこへ 行[い]こう
\\	が
\\	、ずっと 友達[ともだち]でいよう。
\\	"僕[ぼく]がどこへ 行[い]こう
\\	、ずっと 友達[ともだち]でいよう。
\\	うが
\\	まいが.
\\	が
\\	"走[はし]ろう
\\	が
\\	、 走[はし]るまい
\\	が
\\	、 間[ま]に 合[あ]わないだろう。
\\	"走[はし]ろう
\\	、 走[はし]るまい
\\	、 間[ま]に 合[あ]わないだろう。
\\	うが
\\	まいが.
\\	が
\\	のに.	"雪[ゆき]が 降[ふ]ればいい
\\	が
\\	なあ。
\\	"雪[ゆき]が 降[ふ]ればいい
\\	なあ。
\\	が
\\	のに.	"宝[たから]くじが 当[あ]たったら 嬉[うれ]しい
\\	が
\\	なあ。
\\	"宝[たから]くじが 当[あ]たったら 嬉[うれ]しい
\\	なあ。
\\	が
\\	"社長[しゃちょう]さんにお 目[め]にかかりたいんです
\\	が
\\	"社長[しゃちょう]さんにお 目[め]にかかりたいんです
\\	が
\\	"もう 一日[いちにち] 休暇[きゅうか]をとりたいんです
\\	が
\\	"もう 一日[いちにち] 休暇[きゅうか]をとりたいんです
\\	が
\\	"私[わたし]は、このままでいいと 思[おも]うんです
\\	が
\\	"私[わたし]は、このままでいいと 思[おも]うんです
\\	が
\\	"こううそつきめ
\\	が
\\	"こううそつきめ
\\	が
\\	"このガキめ
\\	が
\\	"このガキめ
\\	が
\\	"確[たし]かにここに 置[お]いておいたんです
\\	が
\\	"確[たし]かにここに 置[お]いておいたんです
\\	が
\\	"電話[でんわ]がかかってくるはずなんだ
\\	が
\\	。そうしたんだろう。
\\	"電話[でんわ]がかかってくるはずなんだ
\\	。そうしたんだろう。
\\	が
\\	"あなたは 学生[がくせい]です。 私[わたし]
\\	も
\\	学生[がくせい]です。
\\	"あなたは 学生[がくせい]です。 私[わたし]
\\	学生[がくせい]です。
\\	も
\\	"これは 西瓜[すいか]で、あれ
\\	も
\\	西瓜[すいか]です。
\\	"これは 西瓜[すいか]で、あれ
\\	西瓜[すいか]です。
\\	も
\\	"私[わたし]は 肉[にく]を 食[た]べました。 私[わたし]は 魚[さかな]
\\	も
\\	食[た]べました。
\\	"私[わたし]は 肉[にく]を 食[た]べました。 私[わたし]は 魚[さかな]
\\	食[た]べました。
\\	も
\\	"私[わたし]はコーヒーを 飲[の]みせん。 私[わたし]はコーラ
\\	も
\\	飲[の]みません。
\\	"私[わたし]はコーヒーを 飲[の]みせん。 私[わたし]はコーラ
\\	飲[の]みません。
\\	も
\\	"僕[ぼく]はピクニックに 行[い]きませんでした。 彼女[かのじょ]
\\	も
\\	ピクニックに 行[い]きませんでした。
\\	"僕[ぼく]はピクニックに 行[い]きませんでした。 彼女[かのじょ]
\\	ピクニックに 行[い]きませんでした。
\\	も
\\	"あなた
\\	も
\\	私[わたし]
\\	も
\\	、 同[おな]じ 学校[がっこう]の 学生[がくせい]です。
\\	"あなた
\\	私[わたし]
\\	、 同[おな]じ 学校[がっこう]の 学生[がくせい]です。
\\	も
\\	"私[わたし]はワイン
\\	も
\\	ビール
\\	も
\\	飲[の]みます。
\\	"私[わたし]はワイン
\\	ビール
\\	飲[の]みます。
\\	も
\\	"その 白[しろ]いの
\\	も
\\	青[あお]いの
\\	も
\\	一つつず 下[くだ]さい。
\\	"その 白[しろ]いの
\\	青[あお]いの
\\	一つつず 下[くだ]さい。
\\	も
\\	"あの 子[こ]はお 父[とう]さんに
\\	も
\\	お 母[かあ]さんに
\\	も
\\	死[し]なれた。
\\	"あの 子[こ]はお 父[とう]さんに
\\	お 母[かあ]さんに
\\	死[し]なれた。
\\	も
\\	"僕[ぼく]は 勉強[べんきょう]するの
\\	も
\\	働[はたら]くの
\\	も
\\	嫌[きら]いだ。
\\	"僕[ぼく]は 勉強[べんきょう]するの
\\	働[はたら]くの
\\	嫌[きら]いだ。
\\	も
\\	"彼女[かのじょ]は 美[うつく]しく
\\	も
\\	あるし 優[やさ]しく
\\	も
\\	ある。
\\	"彼女[かのじょ]は 美[うつく]しく
\\	あるし 優[やさ]しく
\\	ある。
\\	も
\\	"献金[けんきん]は 百[ひゃく] 円[えん]で
\\	も
\\	二百[にひゃく] 円[えん]で
\\	も
\\	結構[けっこう]です。
\\	"献金[けんきん]は 百[ひゃく] 円[えん]で
\\	二百[にひゃく] 円[えん]で
\\	結構[けっこう]です。
\\	も
\\	"日本語[にほんご]で 書[か]いて
\\	も
\\	英語[えいご]で 書[か]いて
\\	も
\\	構[か]いません。
\\	"日本語[にほんご]で 書[か]いて
\\	英語[えいご]で 書[か]いて
\\	構[か]いません。
\\	も
\\	"お父[とう]さん、 僕[ぼく]
\\	も
\\	映画[えいが]に 連[つ]れて 行[い]ってよ。
\\	"お父[とう]さん、 僕[ぼく]
\\	映画[えいが]に 連[つ]れて 行[い]ってよ。
\\	も
\\	"お 母[かあ]さん、このケーキ
\\	も
\\	食[た]べていい?
\\	"お 母[かあ]さん、このケーキ
\\	食[た]べていい?
\\	も
\\	"私[わたし]はもう 十年[じゅうねん]
\\	も
\\	日本[にほん]に 帰[かえ]っていません。
\\	"私[わたし]はもう 十年[じゅうねん]
\\	日本[にほん]に 帰[かえ]っていません。
\\	も
\\	"彼[かれ]は 八人[はちにん]
\\	も
\\	兄弟[きょうだい]がある。
\\	"彼[かれ]は 八人[はちにん]
\\	兄弟[きょうだい]がある。
\\	も
\\	"あの 犬[いぬ]の 背丈[せたけ]ほど
\\	も
\\	ある。
\\	"あの 犬[いぬ]の 背丈[せたけ]ほど
\\	ある。
\\	も
\\	"このつぼは 十万[じゅうまん] 円[えん]
\\	も
\\	するんだって。
\\	"このつぼは 十万[じゅうまん] 円[えん]
\\	するんだって。
\\	も
\\	"この 仕事[しごと]はもう 三日[みっか]
\\	も
\\	あれば 出[で]きるでしょう。
\\	"この 仕事[しごと]はもう 三日[みっか]
\\	あれば 出[で]きるでしょう。
\\	も
\\	"三百万[さんびゃくまん] 円[えん]
\\	も
\\	あればいい 車[くるま]が 買[か]えますよ。
\\	"三百万[さんびゃくまん] 円[えん]
\\	あればいい 車[くるま]が 買[か]えますよ。
\\	も
\\	"誰[だれ]
\\	も
\\	居[い]ませんか。
\\	"誰[だれ]
\\	居[い]ませんか。
\\	も
\\	"今日[きょう]は 店[みせ]はどこ
\\	も
\\	閉[し]まっている。
\\	"今日[きょう]は 店[みせ]はどこ
\\	閉[し]まっている。
\\	も
\\	"私[わたし]は 一度[いちど]
\\	も
\\	スキーをしたことがありせん。
\\	"私[わたし]は 一度[いちど]
\\	スキーをしたことがありせん。
\\	も
\\	"大[たい]したごちそう
\\	も
\\	ございませんが、どうぞ 召[め]し 上[あ]がって 下[くだ]さい。
\\	"大[たい]したごちそう
\\	ございませんが、どうぞ 召[め]し 上[あ]がって 下[くだ]さい。
\\	も
\\	"彼[かれ]はテニスの 選手[せんしゅ]
\\	も
\\	していたらしい。
\\	"彼[かれ]はテニスの 選手[せんしゅ]
\\	していたらしい。
\\	も
\\	"熱[ねつ]が 高[たか]いのに、 妹[いもうと]は 苦[くる]しいと
\\	も
\\	言[い]わないで、じっと 寝[ね]ていました。
\\	"熱[ねつ]が 高[たか]いのに、 妹[いもうと]は 苦[くる]しいと
\\	言[い]わないで、じっと 寝[ね]ていました。
\\	も
\\	"そう
\\	も
\\	泣[な]かれては、 私[わたし]も 困[こま]るよ。
\\	"そう
\\	泣[な]かれては、 私[わたし]も 困[こま]るよ。
\\	こうも, そうも, ああも 
\\	こんなに, そんなに, あんなに).	
\\	も
\\	"外国[がいこく]にいると、ああ
\\	も
\\	日本食[にほんしょく]が 恋[こい]しくなるものだろうか。
\\	"外国[がいこく]にいると、ああ
\\	日本食[にほんしょく]が 恋[こい]しくなるものだろうか。
\\	こうも, そうも, ああも 
\\	こんなに, そんなに, あんなに).	
\\	も
\\	"これはあなた
\\	の
\\	本[ほん]です。
\\	"これはあなた
\\	本[ほん]です。
\\	の
\\	"私[わたし]
\\	の
\\	カバン
\\	の
\\	中[なか]に 地図[ちず]があります。
\\	"私[わたし]
\\	カバン
\\	中[なか]に 地図[ちず]があります。
\\	の
\\	"テーブル
\\	の
\\	下[した]に 犬[いぬ]がいます。
\\	"テーブル
\\	下[した]に 犬[いぬ]がいます。
\\	の
\\	"流[なが]し
\\	の
\\	中[なか]にゴキブリがいるよ!
\\	"流[なが]し
\\	中[なか]にゴキブリがいるよ!
\\	の
\\	"私[わたし]は 昨日[きのう] 革[かわ]
\\	の
\\	手袋[てぶくる]を 買[か]いました。
\\	"私[わたし]は 昨日[きのう] 革[かわ]
\\	手袋[てぶくる]を 買[か]いました。
\\	の
\\	"その 竹[たけ]
\\	の
\\	かごはいくらですか。
\\	"その 竹[たけ]
\\	かごはいくらですか。
\\	の
\\	"あのブロンズ
\\	の
\\	像[ぞう]は 誰[だれ]の 作[さく]ですか。
\\	"あのブロンズ
\\	像[ぞう]は 誰[だれ]の 作[さく]ですか。
\\	の
\\	"これ 川端康成[かわばたやすなり]
\\	の
\\	小説[しょうせつ]です。
\\	"これ 川端康成[かわばたやすなり]
\\	小説[しょうせつ]です。
\\	の
\\	"この 曲[きょく]はマーラー
\\	の
\\	交響曲[こうきょうきょく] 一番[いちばん]です。
\\	"この 曲[きょく]はマーラー
\\	交響曲[こうきょうきょく] 一番[いちばん]です。
\\	の
\\	"あの 人[ひと]は  私[わたし]
\\	の
\\	叔父[おじ]です。
\\	"あの 人[ひと]は  私[わたし]
\\	叔父[おじ]です。
\\	の
\\	"彼女[かのじょ]は 社長[しゃちょう]
\\	の
\\	秘書[ひしょ]です。
\\	"彼女[かのじょ]は 社長[しゃちょう]
\\	秘書[ひしょ]です。
\\	の
\\	"あの 人[ひと]は  私[わたし]の 命[いのち]
\\	の
\\	恩人[おんじん]です。
\\	"あの 人[ひと]は  私[わたし]の 命[いのち]
\\	恩人[おんじん]です。
\\	の
\\	"松岡[まつおか] 先生[せんせい]は 理科[りか]
\\	の
\\	先生[せんせい]です。
\\	"松岡[まつおか] 先生[せんせい]は 理科[りか]
\\	先生[せんせい]です。
\\	の
\\	"これは 日本語[にほんご]
\\	の
\\	教科書[きょうかしょ]です。
\\	"これは 日本語[にほんご]
\\	教科書[きょうかしょ]です。
\\	の
\\	"万有引力[ばんゆういんりょく]
\\	の
\\	法則[ほうそく]は、ニュートンによって 発見[はっけん]された。
\\	"万有引力[ばんゆういんりょく]
\\	法則[ほうそく]は、ニュートンによって 発見[はっけん]された。
\\	の
\\	"私[わたし]は、 校長[こうちょう]
\\	の
\\	坂東[ばんどう]と 申[もう]します。
\\	"私[わたし]は、 校長[こうちょう]
\\	坂東[ばんどう]と 申[もう]します。
\\	の
\\	"ピアニスト
\\	の
\\	佐藤[さとう]さんをご 紹介[しょうかい]いたします。
\\	"ピアニスト
\\	佐藤[さとう]さんをご 紹介[しょうかい]いたします。
\\	の
\\	"ニューヨーク
\\	の
\\	冬[ふゆ]は 寒[さむ]いでしょうねえ。
\\	"ニューヨーク
\\	冬[ふゆ]は 寒[さむ]いでしょうねえ。
\\	の
\\	"熱海[あたみ]
\\	の
\\	海岸[かいがん]は 綺麗[きれい]ですよ。
\\	"熱海[あたみ]
\\	海岸[かいがん]は 綺麗[きれい]ですよ。
\\	の
\\	"京都[きょうと]
\\	の
\\	従兄[いとこ]から 手紙[てがみ]が 来[き]た。
\\	"京都[きょうと]
\\	従兄[いとこ]から 手紙[てがみ]が 来[き]た。
\\	の
\\	"あのおじいさんは 恍惚[こうこつ]
\\	の
\\	人[ひと]になったそうだ。
\\	"あのおじいさんは 恍惚[こうこつ]
\\	人[ひと]になったそうだ。
\\	の
\\	"彼女[かのじょ]は 白衣[はくい]
\\	の
\\	天使[てんし]です。
\\	"彼女[かのじょ]は 白衣[はくい]
\\	天使[てんし]です。
\\	の
\\	"私[わたし]は 日本語[にほんご]
\\	の
\\	手紙[てがみ]が 読[よ]めます。
\\	"私[わたし]は 日本語[にほんご]
\\	手紙[てがみ]が 読[よ]めます。
\\	の
\\	"あれは 緑[みどり]
\\	色[いろ])
\\	の
\\	車[くるま]です。
\\	"あれは 緑[みどり]
\\	色[いろ])
\\	車[くるま]です。
\\	い 
\\	白い 
\\	赤い 
\\	い 
\\	の, 
\\	白の 
\\	赤の	
\\	の
\\	"この 紫[むらさき]
\\	色[いろ])
\\	の
\\	花[はな]は 何[なん]といいますか。
\\	"この 紫[むらさき]
\\	色[いろ])
\\	花[はな]は 何[なん]といいますか。
\\	い 
\\	白い 
\\	赤い 
\\	い 
\\	の, 
\\	白の 
\\	赤の	
\\	の
\\	"今日[きょう]はこのグレー
\\	の
\\	コートを 着[き]て 行[い]こう。
\\	"今日[きょう]はこのグレー
\\	コートを 着[き]て 行[い]こう。
\\	い 
\\	白い 
\\	赤い 
\\	い 
\\	の, 
\\	白の 
\\	赤の	
\\	の
\\	"お 父[とう]さん
\\	の
\\	は、どこにあるのかな。
\\	"お 父[とう]さん
\\	は、どこにあるのかな。
\\	の
\\	"こっち
\\	の
\\	は 甘[あま]いげど、そっち
\\	の
\\	は 辛[から]いよ。
\\	"こっち
\\	は 甘[あま]いげど、そっち
\\	は 辛[から]いよ。
\\	の
\\	"私[わたし]
\\	の
\\	はもう 食[た]べてしまったよ。
\\	"私[わたし]
\\	はもう 食[た]べてしまったよ。
\\	の
\\	"お 刺身[さしみ]は 新[あたら]しい
\\	の
\\	が 美味[おい]しい。
\\	"お 刺身[さしみ]は 新[あたら]しい
\\	が 美味[おい]しい。
\\	の
\\	"そこの 隅[すみ]で 泣[な]いている
\\	の
\\	は 誰[だれ]?
\\	"そこの 隅[すみ]で 泣[な]いている
\\	は 誰[だれ]?
\\	の
\\	"僕[ぼく]、まだ 寝[ね]る
\\	の
\\	はいやだよ。テレビ 見[み]たいから。
\\	"僕[ぼく]、まだ 寝[ね]る
\\	はいやだよ。テレビ 見[み]たいから。
\\	の
\\	"これは 十年[じゅうねん] 前[まえ]
\\	の
\\	新聞[しんぶん]だよ。
\\	"これは 十年[じゅうねん] 前[まえ]
\\	新聞[しんぶん]だよ。
\\	の
\\	"これから 二十分[にじゅっぷん]
\\	の
\\	休憩[きゅうけい]があります。
\\	"これから 二十分[にじゅっぷん]
\\	休憩[きゅうけい]があります。
\\	の
\\	"昔[むかし]
\\	の
\\	人達[ひとたち]はどんな 生活[せいかつ]をしていたのだろう。
\\	"昔[むかし]
\\	人達[ひとたち]はどんな 生活[せいかつ]をしていたのだろう。
\\	の
\\	が, 
\\	"今日[きょう]は 風[かぜ]
\\	の
\\	吹[ふ]く 寒[さむ]い 日[ひ]ですね。
\\	"今日[きょう]は 風[かぜ]
\\	吹[ふ]く 寒[さむ]い 日[ひ]ですね。
\\	の
\\	が, 
\\	"雨[あめ]
\\	の
\\	降[ふ]る 日[ひ]はうちの 中[なか]で 遊[あそ]びましょう。
\\	"雨[あめ]
\\	降[ふ]る 日[ひ]はうちの 中[なか]で 遊[あそ]びましょう。
\\	の
\\	が, 
\\	"これは 南部[なんぶ]さん
\\	の
\\	作曲[さっきょく]した 曲[きょく]です。
\\	"これは 南部[なんぶ]さん
\\	作曲[さっきょく]した 曲[きょく]です。
\\	の
\\	"花[はな]に 例[たと]えれば、 彼女[かのじょ]は 白百合[しらゆり]
\\	の
\\	ような 人[ひと]です。
\\	"花[はな]に 例[たと]えれば、 彼女[かのじょ]は 白百合[しらゆり]
\\	ような 人[ひと]です。
\\	~のような、~のように	
\\	の
\\	"君[きみ]どうかしたの? 病人[びょうにん]
\\	の
\\	ような 青[あお]い 顔[かお]をして。
\\	"君[きみ]どうかしたの? 病人[びょうにん]
\\	ような 青[あお]い 顔[かお]をして。
\\	~のような、~のように	
\\	の
\\	"この 砂[すな]は 砂糖[さとう]
\\	の
\\	ように 白[しろ]いね。
\\	"この 砂[すな]は 砂糖[さとう]
\\	ように 白[しろ]いね。
\\	~のような、~のように	
\\	の
\\	"僕[ぼく]は 鳥[とり]
\\	の
\\	ように 空[さら]を 飛[と]んでみたい。
\\	"僕[ぼく]は 鳥[とり]
\\	ように 空[さら]を 飛[と]んでみたい。
\\	~のような、~のように	
\\	の
\\	"飛行機[ひこうき]が 飛[と]んでいる
\\	の
\\	が 見[み]えます。
\\	"飛行機[ひこうき]が 飛[と]んでいる
\\	が 見[み]えます。
\\	の
\\	"犬[いぬ]の 吠[ほ]える
\\	の
\\	が 聞[き]こえる。
\\	"犬[いぬ]の 吠[ほ]える
\\	が 聞[き]こえる。
\\	の
\\	"君[きみ]とよく 遊[あそ]んだ
\\	の
\\	を 思[おも]い 出[だ]すね。
\\	"君[きみ]とよく 遊[あそ]んだ
\\	を 思[おも]い 出[だ]すね。
\\	の
\\	"私[わたし]はパスポートをなくした
\\	の
\\	に 気[き]が 付[つ]いた。
\\	"私[わたし]はパスポートをなくした
\\	に 気[き]が 付[つ]いた。
\\	の
\\	"さあ、これからお 風呂[ふろ]に 入[はい]る
\\	の
\\	(よ)。
\\	"さあ、これからお 風呂[ふろ]に 入[はい]る
\\	(よ)。
\\	よ 
\\	の
\\	"風邪[かぜ]だから 大人[おとな]しく 寝[ね]ている
\\	の
\\	(よ)。
\\	"風邪[かぜ]だから 大人[おとな]しく 寝[ね]ている
\\	(よ)。
\\	よ 
\\	の
\\	"私[わたし]、みよちゃんと 喧嘩[けんか]しちゃった
\\	の
\\	"私[わたし]、みよちゃんと 喧嘩[けんか]しちゃった
\\	の
\\	"友達[ともだち]が 漫画[まんが]の 本[ほん]を 貸[か]してくれた
\\	の
\\	"友達[ともだち]が 漫画[まんが]の 本[ほん]を 貸[か]してくれた
\\	の
\\	"僕[ぼく]おなかが 空[す]いたから、もう 食[た]べちゃった
\\	の
\\	"僕[ぼく]おなかが 空[す]いたから、もう 食[た]べちゃった
\\	の
\\	"この 手紙[てがみ]、 君[きみ]が 書[か]いた
\\	の
\\	"この 手紙[てがみ]、 君[きみ]が 書[か]いた
\\	~のですか 
\\	の
\\	"明日[あした] 来[こ]ない
\\	の
\\	"明日[あした] 来[こ]ない
\\	~のですか 
\\	の
\\	"この 本[ほん]もう 読[よ]んじゃった
\\	の
\\	"この 本[ほん]もう 読[よ]んじゃった
\\	~のですか 
\\	の
\\	"もう、 誠[まこと]が、よく 来[き]た
\\	の
\\	"もう、 誠[まこと]が、よく 来[き]た
\\	のう 
\\	の
\\	"ありがたいことじゃ。 今年[ことし]は 米[こめ]が 豊作[ほうさく]で
\\	の
\\	"ありがたいことじゃ。 今年[ことし]は 米[こめ]が 豊作[ほうさく]で
\\	のう 
\\	の
\\	"行[い]く
\\	の
\\	行[い]かない
\\	の
\\	と 迷[まと]っていないで、 早[はや]くどっちに 決[き]めたら?
\\	"行[い]く
\\	行[い]かない
\\	と 迷[まと]っていないで、 早[はや]くどっちに 決[き]めたら?
\\	~の~の	
\\	の
\\	"寒[さむ]い
\\	の
\\	暑[あつ]い
\\	の
\\	と 言[い]ってないで、ジョギングは 毎朝[まいあさ]しないさい。
\\	"寒[さむ]い
\\	暑[あつ]い
\\	と 言[い]ってないで、ジョギングは 毎朝[まいあさ]しないさい。
\\	~の~の	
\\	の
\\	~のところ	"お 父[とう]さん、この 算数[さんすう]の 問題[もんだい] 難[むずか]しくて、ここ
\\	ん
\\	ところが 分[か]からないよ。
\\	"お 父[とう]さん、この 算数[さんすう]の 問題[もんだい] 難[むずか]しくて、ここ
\\	ところが 分[か]からないよ。
\\	ん
\\	~のところ	"あ、 僕[ぼく] 砂浜[すなはま]
\\	ん
\\	とこ(ところ)に 帽子[ぼうし] 忘[わす]れてきちゃった。
\\	"あ、 僕[ぼく] 砂浜[すなはま]
\\	とこ(ところ)に 帽子[ぼうし] 忘[わす]れてきちゃった。
\\	ん
\\	"私[わたし]は 海[うみ]
\\	で
\\	泳[およ]ぎました。
\\	"私[わたし]は 海[うみ]
\\	泳[およ]ぎました。
\\	で
\\	"あの 木[き]の 下[した]
\\	で
\\	お 弁当[べんとう]を 食[た]べましょう。
\\	"あの 木[き]の 下[した]
\\	お 弁当[べんとう]を 食[た]べましょう。
\\	で
\\	"私[わたし]は 大学[だいがく]
\\	で
\\	日本語[にほんご]を 勉強[べんきょう]しています。
\\	"私[わたし]は 大学[だいがく]
\\	日本語[にほんご]を 勉強[べんきょう]しています。
\\	で
\\	"デパートで 食事[しょくじ]をしてから、 本屋[ほんや]
\\	で
\\	辞書[じしょ]を 買[か]いました。
\\	"デパートで 食事[しょくじ]をしてから、 本屋[ほんや]
\\	辞書[じしょ]を 買[か]いました。
\\	で
\\	"宮田[みやた]さんは、クラス 中[なか]
\\	で
\\	一番[いちばん] 早[はや]く 走[はし]ります。
\\	"宮田[みやた]さんは、クラス 中[なか]
\\	一番[いちばん] 早[はや]く 走[はし]ります。
\\	で
\\	"この 中[なか]
\\	で
\\	誰[だれ]が 猫[ねこ]の 首[くび]に 鈴[すず]を 付[つ]けに 行[い]くんだい?
\\	"この 中[なか]
\\	誰[だれ]が 猫[ねこ]の 首[くび]に 鈴[すず]を 付[つ]けに 行[い]くんだい?
\\	で
\\	も~_も	"ここ
\\	で
\\	も、あそこ
\\	で
\\	も、 子供達[こともたち]が 遊[あそ]んでいる。
\\	"ここ
\\	も、あそこ
\\	も、 子供達[こともたち]が 遊[あそ]んでいる。
\\	で
\\	も~_も	"今年[ことし]の 夏[なつ]は 海[うみ]
\\	で
\\	も 山[やま]
\\	で
\\	も 事故[じこ]か 続発[ぞくはつ]した。
\\	"今年[ことし]の 夏[なつ]は 海[うみ]
\\	も 山[やま]
\\	も 事故[じこ]か 続発[ぞくはつ]した。
\\	で
\\	"一九四
\\	[せんきゅうひゃくよんじゅう] 年代[ねんだい]
\\	で
\\	は、こんな 歌[うた]が 流行[はや]ったそうだ。
\\	"一九四
\\	[せんきゅうひゃくよんじゅう] 年代[ねんだい]
\\	は、こんな 歌[うた]が 流行[はや]ったそうだ。
\\	では 
\\	でも 
\\	で
\\	"今[いま]
\\	で
\\	も、その 習慣[しゅうかん]はなお 行[おこな]われている。
\\	"今[いま]
\\	も、その 習慣[しゅうかん]はなお 行[おこな]われている。
\\	では 
\\	でも 
\\	で
\\	"彼[かれ]は 六十[ろくじゅう] 歳[さい]
\\	で
\\	退職[たいしょく]した。
\\	"彼[かれ]は 六十[ろくじゅう] 歳[さい]
\\	退職[たいしょく]した。
\\	で
\\	"母[はは]は 二十五[にじゅうご] 歳[さい]
\\	で
\\	結婚[けっこん]した。
\\	"母[はは]は 二十五[にじゅうご] 歳[さい]
\\	結婚[けっこん]した。
\\	で
\\	"この 仕事[しごと]は 二十分[にじゅっぷん]
\\	で
\\	出来[でき]ますか。
\\	"この 仕事[しごと]は 二十分[にじゅっぷん]
\\	出来[でき]ますか。
\\	で
\\	"もう 二週間[にしゅうかん]
\\	で
\\	夏休[なつやす]みになります。
\\	"もう 二週間[にしゅうかん]
\\	夏休[なつやす]みになります。
\\	で
\\	"光[ひかり]は 一秒間[いちびょうかん]
\\	で
\\	地球[ちきゅう]を 七回[ななまわ]り 半[はん]する。
\\	"光[ひかり]は 一秒間[いちびょうかん]
\\	地球[ちきゅう]を 七回[ななまわ]り 半[はん]する。
\\	で
\\	"私[わたし]は 学校[がっこう]へ 自転車[じてんしゃ]
\\	で
\\	通[かよ]っています。
\\	"私[わたし]は 学校[がっこう]へ 自転車[じてんしゃ]
\\	通[かよ]っています。
\\	で
\\	"私[わたし]は 飛行機[ひこうき]
\\	で
\\	ロンドンに 行[い]ってから、 船[ふね]
\\	で
\\	スペインに 行[い]きました。
\\	"私[わたし]は 飛行機[ひこうき]
\\	ロンドンに 行[い]ってから、 船[ふね]
\\	スペインに 行[い]きました。
\\	で
\\	"私[わたし]は 父[ちち]と 電話[でんわ]
\\	で
\\	話[はな]しました。
\\	"私[わたし]は 父[ちち]と 電話[でんわ]
\\	話[はな]しました。
\\	で
\\	"鉛筆[えんぴつ]
\\	で
\\	書[か]いて 下[ください]さい。
\\	"鉛筆[えんぴつ]
\\	書[か]いて 下[ください]さい。
\\	で
\\	"石[せっ]けん
\\	で
\\	よく 手[て]を 洗[あら]いなさいよ。
\\	"石[せっ]けん
\\	よく 手[て]を 洗[あら]いなさいよ。
\\	で
\\	"彼[かれ]はいつもユーモア
\\	で
\\	みんなを 笑[わら]わせます。
\\	"彼[かれ]はいつもユーモア
\\	みんなを 笑[わら]わせます。
\\	で
\\	"日本語[にほんご]
\\	で
\\	話[はな]して 下[ください]さい。
\\	"日本語[にほんご]
\\	話[はな]して 下[ください]さい。
\\	で
\\	"ガラス
\\	で
\\	指[ゆび]を 少[すこ]し 切[き]ってしまいました。
\\	"ガラス
\\	指[ゆび]を 少[すこ]し 切[き]ってしまいました。
\\	で
\\	"その 本[ほん]を 二千[にせん] 円[えん]
\\	で
\\	買[か]いました。
\\	"その 本[ほん]を 二千[にせん] 円[えん]
\\	買[か]いました。
\\	で
\\	"ハワイへはいくら
\\	で
\\	行[い]けますか?
\\	"ハワイへはいくら
\\	行[い]けますか?
\\	で
\\	"僕[ぼく]は 時間[じかん]
\\	で
\\	アルバイトをしている。
\\	"僕[ぼく]は 時間[じかん]
\\	アルバイトをしている。
\\	で
\\	"その 林檎[りんご]は 二つ
\\	で
\\	五百[ごひゃく] 円[えん]です。
\\	"その 林檎[りんご]は 二つ
\\	五百[ごひゃく] 円[えん]です。
\\	で
\\	"私[わたし]達[たち]のチーム 一点[いってん] 差[さ]
\\	で
\\	勝[か]った。
\\	"私[わたし]達[たち]のチーム 一点[いってん] 差[さ]
\\	勝[か]った。
\\	で
\\	"これはプラスチック
\\	で
\\	できた 箱[はこ]です。
\\	"これはプラスチック
\\	できた 箱[はこ]です。
\\	で
\\	"それは 粘土[ねんど]
\\	で
\\	作[つく]った 器[うつわ]です。
\\	"それは 粘土[ねんど]
\\	作[つく]った 器[うつわ]です。
\\	で
\\	"バターはミルク
\\	で
\\	作[つく]ります。
\\	"バターはミルク
\\	作[つく]ります。
\\	で
\\	"私[わたし]は 急[いそ]い
\\	で
\\	出[で]かけました。
\\	"私[わたし]は 急[いそ]い
\\	出[で]かけました。
\\	で
\\	"彼[かれ]は 夢中[むちゅう]
\\	で
\\	走[はし]った。
\\	"彼[かれ]は 夢中[むちゅう]
\\	走[はし]った。
\\	で
\\	"僕[ぼく]は 叱[しか]るつもり
\\	で
\\	言[い]ったのではない。
\\	"僕[ぼく]は 叱[しか]るつもり
\\	言[い]ったのではない。
\\	で
\\	"彼[かれ]は 不注意[ふちゅうい]
\\	で
\\	階段[かいだん]から 落[お]ちた。
\\	"彼[かれ]は 不注意[ふちゅうい]
\\	階段[かいだん]から 落[お]ちた。
\\	で
\\	"僕[ぼく]は 今日[きょう] 風邪[かぜ]
\\	で
\\	学校[がっこう]を 休[やす]んだ。
\\	"僕[ぼく]は 今日[きょう] 風邪[かぜ]
\\	学校[がっこう]を 休[やす]んだ。
\\	で
\\	"寒[さむ]さ
\\	で
\\	手[て]が 凍[こご]えそうだ。
\\	"寒[さむ]さ
\\	手[て]が 凍[こご]えそうだ。
\\	で
\\	"このデザインの 流行[りゅうこう]
\\	で
\\	店[みせ]は 大[だい] 繁盛[はんじょう]だ。
\\	"このデザインの 流行[りゅうこう]
\\	店[みせ]は 大[だい] 繁盛[はんじょう]だ。
\\	で
\\	"新聞[しんぶん]の 天気予報[てんきよほう]
\\	で
\\	見[み]ると、 今日[きょう]は 曇[くもり]だそうだ。
\\	"新聞[しんぶん]の 天気予報[てんきよほう]
\\	見[み]ると、 今日[きょう]は 曇[くもり]だそうだ。
\\	で
\\	"あなたの 時計[とけい]
\\	で
\\	は 今[いま] 何時[なんじ]ですか。
\\	"あなたの 時計[とけい]
\\	は 今[いま] 何時[なんじ]ですか。
\\	で
\\	"彼[かれ]がひどく 悩[なや]んでいるのは、 顔色[かおいろ]
\\	で
\\	分[わ]かる。
\\	"彼[かれ]がひどく 悩[なや]んでいるのは、 顔色[かおいろ]
\\	分[わ]かる。
\\	で
\\	"みかんの 値段[ねだん]は 大[おお]きさ
\\	で
\\	違[ちが]います。
\\	"みかんの 値段[ねだん]は 大[おお]きさ
\\	違[ちが]います。
\\	で
\\	"土地[とち]の 値段[ねだん]は 広[ひろ]さと 便利[べんり]さ
\\	で
\\	変[かわ]ります。
\\	"土地[とち]の 値段[ねだん]は 広[ひろ]さと 便利[べんり]さ
\\	変[かわ]ります。
\\	で
\\	"それは 家族[かぞく]
\\	で
\\	決[き]めたことです。
\\	"それは 家族[かぞく]
\\	決[き]めたことです。
\\	で
\\	"その 新聞[しんぶん]は 学校[がっこう]
\\	で
\\	発行[はっこう]された。
\\	"その 新聞[しんぶん]は 学校[がっこう]
\\	発行[はっこう]された。
\\	で
\\	"コーヒー
\\	を
\\	飲[の]みましょう。
\\	"コーヒー
\\	飲[の]みましょう。
\\	を
\\	"私[わたし]は、 昨日[きのう]、 美味[おい]しい 天[てん]ぷら
\\	を
\\	食[た]べました。
\\	"私[わたし]は、 昨日[きのう]、 美味[おい]しい 天[てん]ぷら
\\	食[た]べました。
\\	を
\\	"僕[ぼく]は 一昨日[おととい]フランス 映画[えいが]
\\	を
\\	見[み]た。
\\	"僕[ぼく]は 一昨日[おととい]フランス 映画[えいが]
\\	見[み]た。
\\	を
\\	"明日[あした] 妹[いもうと]に 電話[でんわ]
\\	を
\\	かけます。
\\	"明日[あした] 妹[いもうと]に 電話[でんわ]
\\	かけます。
\\	を
\\	"先週[せんしゅう]デパートで 素敵[すてき]な 靴[くつ]
\\	を
\\	買[か]いました。
\\	"先週[せんしゅう]デパートで 素敵[すてき]な 靴[くつ]
\\	買[か]いました。
\\	を
\\	~たい、~たくない、~たかった、~たくなかった	"僕[ぼく]はお 酒[さけ]
\\	を
\\	(が) 飲[の]みたいんです。
\\	"僕[ぼく]はお 酒[さけ]
\\	(が) 飲[の]みたいんです。
\\	が 
\\	を 
\\	を
\\	~たい、~たくない、~たかった、~たくなかった	"私[わたし]はアイスクリーム
\\	を
\\	(が) 食[た]べたい。
\\	"私[わたし]はアイスクリーム
\\	(が) 食[た]べたい。
\\	が 
\\	を 
\\	を
\\	"娘[むすめ]にピアノ
\\	を
\\	習[なら]わせています。
\\	"娘[むすめ]にピアノ
\\	習[なら]わせています。
\\	を
\\	"病院[びょういん]の 中[なか]では、 子供達[こどもたち]
\\	を
\\	走[はし]らせないようにして 下[くだ]さい。
\\	"病院[びょういん]の 中[なか]では、 子供達[こどもたち]
\\	走[はし]らせないようにして 下[くだ]さい。
\\	を
\\	~れる(~れます)、~られる(~られます) 
\\	"私[わたし]は 先生[せんせい]に 名前[なまえ]
\\	を
\\	呼[よ]ばれた。
\\	"私[わたし]は 先生[せんせい]に 名前[なまえ]
\\	呼[よ]ばれた。
\\	を
\\	~れる(~れます)、~られる(~られます) 
\\	"私[わたし]は 猫[ねこ]に 魚[さなか]
\\	を
\\	取[と]られた。
\\	"私[わたし]は 猫[ねこ]に 魚[さなか]
\\	取[と]られた。
\\	を
\\	"私[わたし]は 社長[しゃちょう]の 秘書[ひしょ]
\\	を
\\	しています。
\\	"私[わたし]は 社長[しゃちょう]の 秘書[ひしょ]
\\	しています。
\\	~をしている(~をしています)	
\\	を
\\	"父[ちち]は 先生[せんせい]
\\	を
\\	しています。
\\	"父[ちち]は 先生[せんせい]
\\	しています。
\\	~をしている(~をしています)	
\\	を
\\	"あなたは 毎朝[まいあさ] 何時[なんじ]にうち
\\	を
\\	出[で]ますか。
\\	"あなたは 毎朝[まいあさ] 何時[なんじ]にうち
\\	出[で]ますか。
\\	を
\\	"明日[あした]ローマ
\\	を
\\	発[た]ってパリに 行[い]きます。
\\	"明日[あした]ローマ
\\	発[た]ってパリに 行[い]きます。
\\	を
\\	"地下鉄[ちかてつ]
\\	を
\\	降[お]りてから、タクシーでうちに 帰[かえ]りました。
\\	"地下鉄[ちかてつ]
\\	降[お]りてから、タクシーでうちに 帰[かえ]りました。
\\	を
\\	"あなたはいつの 大学[だいがく]
\\	を
\\	卒業[そつぎょう]しますか。
\\	"あなたはいつの 大学[だいがく]
\\	卒業[そつぎょう]しますか。
\\	を
\\	"私[わたし]はスロープ
\\	を
\\	滑[すべ]りました。
\\	"私[わたし]はスロープ
\\	滑[すべ]りました。
\\	を
\\	"あなたはどの 道[みち]
\\	を
\\	通[とお]ってここへ 来[き]ましたか。
\\	"あなたはどの 道[みち]
\\	通[とお]ってここへ 来[き]ましたか。
\\	を
\\	"私[わたし] 達[たち]は、 広[ひろ]い 海[うみ]の 上[うえ] 走[はし]るハイウェイ
\\	を
\\	ドライブしました。
\\	"私[わたし] 達[たち]は、 広[ひろ]い 海[うみ]の 上[うえ] 走[はし]るハイウェイ
\\	ドライブしました。
\\	を
\\	はじめとして
\\	、~_筆頭に
\\	、これを機に
\\	"卒業[そつぎょう] 式[しき]には、 父兄[ふけい]
\\	を
\\	はじめとして、 大勢[おおぜい]の 来賓[らいひん]が 出席[しゅっせき]した。
\\	"卒業[そつぎょう] 式[しき]には、 父兄[ふけい]
\\	はじめとして、 大勢[おおぜい]の 来賓[らいひん]が 出席[しゅっせき]した。
\\	を
\\	はじめとして
\\	、~_筆頭に
\\	、これを機に
\\	"この 会社[かいしゃ]では、 社長[しゃちょう]
\\	を
\\	筆頭[ひっとう]に 毎朝[まいあさ] 体操[たいそう]をしています。
\\	"この 会社[かいしゃ]では、 社長[しゃちょう]
\\	筆頭[ひっとう]に 毎朝[まいあさ] 体操[たいそう]をしています。
\\	を
\\	はじめとして
\\	、~_筆頭に
\\	、これを機に
\\	"これ
\\	を
\\	機[き]に 禁煙[きんえん]をしよう。
\\	"これ
\\	機[き]に 禁煙[きんえん]をしよう。
\\	を
\\	"机[つくえ]の 上[うえ]
\\	に
\\	本[ほん]があります。
\\	"机[つくえ]の 上[うえ]
\\	本[ほん]があります。
\\	に
\\	"私[わたし] 達[たち]は 東京[とうきょう]の 近[ちか]く
\\	に
\\	住[す]んでいます。
\\	"私[わたし] 達[たち]は 東京[とうきょう]の 近[ちか]く
\\	住[す]んでいます。
\\	に
\\	"見[み]てごらん、この 池[いけ]
\\	に
\\	コイが 沢山[たくさん]いるよ。
\\	"見[み]てごらん、この 池[いけ]
\\	コイが 沢山[たくさん]いるよ。
\\	に
\\	"大昔[おおむかし]、 火星[かせい]
\\	に
\\	水[みず]があったそうだ。
\\	"大昔[おおむかし]、 火星[かせい]
\\	水[みず]があったそうだ。
\\	に
\\	"あなたはどこ
\\	に
\\	お 勤[つと]めですか。
\\	"あなたはどこ
\\	お 勤[つと]めですか。
\\	に
\\	"道[みち]
\\	に
\\	雪[ゆき]が 沢山[たくさん] 残[のこ]っています。
\\	"道[みち]
\\	雪[ゆき]が 沢山[たくさん] 残[のこ]っています。
\\	に
\\	"あそこ
\\	に
\\	立[た]っている 男[おとこ]の 人[ひと]は 今川[いまがわ]さんです。
\\	"あそこ
\\	立[た]っている 男[おとこ]の 人[ひと]は 今川[いまがわ]さんです。
\\	に
\\	"あの 壁[かべ]
\\	に
\\	掛[か]かっている 絵[え]は 綺麗[きれい]ですね。
\\	"あの 壁[かべ]
\\	掛[か]かっている 絵[え]は 綺麗[きれい]ですね。
\\	に
\\	"僕[ぼく]はお 父[とう]さん
\\	に
\\	手紙[てがみ]を 書[か]いた。
\\	"僕[ぼく]はお 父[とう]さん
\\	手紙[てがみ]を 書[か]いた。
\\	に
\\	"私[わたし]は 来月[らいげつ]イギリス
\\	に
\\	行[い]きます。
\\	"私[わたし]は 来月[らいげつ]イギリス
\\	行[い]きます。
\\	に
\\	"僕[ぼく]がうち
\\	に
\\	帰[かえ]ったとき、 妹[いもうと]も 学校[がっこう]から 帰[かえ]ってきた。
\\	"僕[ぼく]がうち
\\	帰[かえ]ったとき、 妹[いもうと]も 学校[がっこう]から 帰[かえ]ってきた。
\\	に
\\	"父[ちち]は 会社[かいしゃ]
\\	に
\\	出[で]かけました。
\\	"父[ちち]は 会社[かいしゃ]
\\	出[で]かけました。
\\	に
\\	"景色[けしき]のいい 所[ところ]
\\	に
\\	旅行[りょこう]したいね。
\\	"景色[けしき]のいい 所[ところ]
\\	旅行[りょこう]したいね。
\\	に
\\	"太陽[たいよう]は 西[にし]
\\	に
\\	沈[しず]みます。
\\	"太陽[たいよう]は 西[にし]
\\	沈[しず]みます。
\\	に
\\	"さあ、 飛行機[ひこうき]
\\	に
\\	乗[の]る 時間[じかん]ですよ。
\\	"さあ、 飛行機[ひこうき]
\\	乗[の]る 時間[じかん]ですよ。
\\	に
\\	"来月[らいげつ]スキー
\\	に
\\	行[い]きませんか。
\\	"来月[らいげつ]スキー
\\	行[い]きませんか。
\\	に
\\	"彼[かれ]らは 恋[こい]
\\	に
\\	おちた。
\\	"彼[かれ]らは 恋[こい]
\\	おちた。
\\	に
\\	"森田[もりた] 氏[し]は 政界[せいかい]
\\	に
\\	入[はい]って。
\\	"森田[もりた] 氏[し]は 政界[せいかい]
\\	入[はい]って。
\\	に
\\	"あなたはもう 投票[とうひょう]
\\	に
\\	行[い]きましたか。
\\	"あなたはもう 投票[とうひょう]
\\	行[い]きましたか。
\\	に
\\	"私[わたし]は 午後[ごご] 会議[かいぎ]
\\	に
\\	出[で]ます。
\\	"私[わたし]は 午後[ごご] 会議[かいぎ]
\\	出[で]ます。
\\	に
\\	"信号[しんごう]が 青[あお]
\\	に
\\	なりました。さあ、 渡[わた]りましょう。
\\	"信号[しんごう]が 青[あお]
\\	なりました。さあ、 渡[わた]りましょう。
\\	になる (なります)” 
\\	に変わる (変わります)”	
\\	に
\\	"もう 秋[あき]
\\	に
\\	なりましたね。
\\	"もう 秋[あき]
\\	なりましたね。
\\	になる (なります)” 
\\	に変わる (変わります)”	
\\	に
\\	"私[わたし]の 祖母[そぼ]は 九十歳[きゅうじゅっさい]
\\	に
\\	なりました。
\\	"私[わたし]の 祖母[そぼ]は 九十歳[きゅうじゅっさい]
\\	なりました。
\\	になる (なります)” 
\\	に変わる (変わります)”	
\\	に
\\	"山上[やまがみ] 君[くん]はカメラマン
\\	に
\\	なった。
\\	"山上[やまがみ] 君[くん]はカメラマン
\\	なった。
\\	になる (なります)” 
\\	に変わる (変わります)”	
\\	に
\\	"前[まえ]は 薬屋[くすりや]だったあの 店[みせ]はメガネ 屋[や]
\\	に
\\	変[か]わった。
\\	"前[まえ]は 薬屋[くすりや]だったあの 店[みせ]はメガネ 屋[や]
\\	変[か]わった。
\\	になる (なります)” 
\\	に変わる (変わります)”	
\\	に
\\	"学校[がっこう]は 毎朝[まいあさ] 八時[はちじ] 半[はん]
\\	に
\\	始[はじ]まります。
\\	"学校[がっこう]は 毎朝[まいあさ] 八時[はちじ] 半[はん]
\\	始[はじ]まります。
\\	に
\\	"私[わたし]は 夕方[ゆうがた] 七時[しちじ]
\\	に
\\	帰[かえ]ります。
\\	"私[わたし]は 夕方[ゆうがた] 七時[しちじ]
\\	帰[かえ]ります。
\\	に
\\	"私[わたし]は 休憩[きゅうけい] 時間[じかん]
\\	に
\\	コーヒーを 飲[の]みました。
\\	"私[わたし]は 休憩[きゅうけい] 時間[じかん]
\\	コーヒーを 飲[の]みました。
\\	に
\\	"僕[ぼく]は 来週[らいしゅう]の 金曜日[きんようび]にコンサート
\\	に
\\	行[い]きます。
\\	"僕[ぼく]は 来週[らいしゅう]の 金曜日[きんようび]にコンサート
\\	行[い]きます。
\\	に
\\	"私[わたし]の 母[はは]は 二十歳[はたち]の 時[とき]
\\	に
\\	結婚[けっこん]した。
\\	"私[わたし]の 母[はは]は 二十歳[はたち]の 時[とき]
\\	結婚[けっこん]した。
\\	に
\\	"稲妻[いなづま]が 光[ひか]ると 同時[どうじ]
\\	に
\\	雷鳴[らいめい]が 轟[とどろ]きました。
\\	"稲妻[いなづま]が 光[ひか]ると 同時[どうじ]
\\	雷鳴[らいめい]が 轟[とどろ]きました。
\\	に
\\	"私[わたし]は、 朝[あさ]と 晩[ばん]、 一日[いちにち]
\\	に
\\	二回[にかい] 歯[は]を 磨[みが]きます。
\\	"私[わたし]は、 朝[あさ]と 晩[ばん]、 一日[いちにち]
\\	二回[にかい] 歯[は]を 磨[みが]きます。
\\	に
\\	"山手[やまのて] 線[せん]の 電車[でんしゃ]は 五分[ごふん]おき
\\	に
\\	来[き]ます。
\\	"山手[やまのて] 線[せん]の 電車[でんしゃ]は 五分[ごふん]おき
\\	来[き]ます。
\\	に
\\	"この 写真[しゃしん]のサイズは、 縦[たて] 八[はっ]センチ
\\	に
\\	横[よこ] 一三[じゅうさん]センチです。
\\	"この 写真[しゃしん]のサイズは、 縦[たて] 八[はっ]センチ
\\	横[よこ] 一三[じゅうさん]センチです。
\\	に
\\	"テスト 用紙[ようし]は 一人[ひとり]
\\	に
\\	一枚[いちまい]ずつです。
\\	"テスト 用紙[ようし]は 一人[ひとり]
\\	一枚[いちまい]ずつです。
\\	に
\\	"一度[いちど]
\\	に
\\	全部[ぜんぶ] 食[た]べないで、 少[すこ]しずつお 上[あ]がりなさい。
\\	"一度[いちど]
\\	全部[ぜんぶ] 食[た]べないで、 少[すこ]しずつお 上[あ]がりなさい。
\\	に
\\	"千円[せんえん]
\\	に
\\	ついて 五十円[ごじゅうえん]の 手数料[てすうりょう]がかかります。
\\	"千円[せんえん]
\\	ついて 五十円[ごじゅうえん]の 手数料[てすうりょう]がかかります。
\\	に
\\	"私[わたし]は、 今[いま]エレクトロニクスの 研究[けんきゅう]
\\	に
\\	取[と]り 組[く]んでいます。
\\	"私[わたし]は、 今[いま]エレクトロニクスの 研究[けんきゅう]
\\	取[と]り 組[く]んでいます。
\\	に
\\	"彼[かれ]は 空手[からて]のけいこ
\\	に
\\	夢中[むちゅう]です。
\\	"彼[かれ]は 空手[からて]のけいこ
\\	夢中[むちゅう]です。
\\	に
\\	"私[わたし]は 寒[さむ]さ
\\	に
\\	震[ふる]えた
\\	"私[わたし]は 寒[さむ]さ
\\	震[ふる]えた
\\	に
\\	"僕[ぼく]は 高熱[こうねつ]
\\	に
\\	苦[くる]しんだ。
\\	"僕[ぼく]は 高熱[こうねつ]
\\	苦[くる]しんだ。
\\	に
\\	"僕[ぼく]、おじいちゃんからお 年玉[としだま]
\\	に
\\	二千円[にせんえん]もらったよ。
\\	"僕[ぼく]、おじいちゃんからお 年玉[としだま]
\\	二千円[にせんえん]もらったよ。
\\	に
\\	"私[わたし] 達[たち]は、ハネムーン
\\	に
\\	ハワイへ 行[い]きました。
\\	"私[わたし] 達[たち]は、ハネムーン
\\	ハワイへ 行[い]きました。
\\	に
\\	"いろいろお 世話[せわ]
\\	に
\\	なりました。つまらないものですが、どうぞお 礼[れい]
\\	に
\\	これをお 受[う]け 取[と]り 下[くだ]さい。
\\	"いろいろお 世話[せわ]
\\	なりました。つまらないものですが、どうぞお 礼[れい]
\\	これをお 受[う]け 取[と]り 下[くだ]さい。
\\	に
\\	と, 
\\	"私[わたし]はフライドチキン
\\	に
\\	ハンバーガーがいいわ。
\\	"私[わたし]はフライドチキン
\\	ハンバーガーがいいわ。
\\	に
\\	と, 
\\	"村祭[むらまつ]りは、 子供[こども]
\\	に
\\	若[わか]い
\\	に
\\	お 年寄[としよ]りと、 総出[そうで]でにぎわった。
\\	"村祭[むらまつ]りは、 子供[こども]
\\	若[わか]い
\\	お 年寄[としよ]りと、 総出[そうで]でにぎわった。
\\	に
\\	と, 
\\	"今日[きょう]のパーティー、スカート
\\	に
\\	ブラウスでいいかしら。
\\	"今日[きょう]のパーティー、スカート
\\	ブラウスでいいかしら。
\\	に
\\	"ご主人[しゅじん] 様[さま]
\\	に
\\	もお 元気[げんき]でいらっしゃることと 思[おも]います。
\\	"ご主人[しゅじん] 様[さま]
\\	もお 元気[げんき]でいらっしゃることと 思[おも]います。
\\	に
\\	"首相[しゅしょう]
\\	に
\\	は、 来月[らいげつ]アメリカを 訪問[ほうもん]されるそうです。
\\	"首相[しゅしょう]
\\	は、 来月[らいげつ]アメリカを 訪問[ほうもん]されるそうです。
\\	に
\\	"聞[き]くところ
\\	に
\\	よれば、 新[あたら]しいハイウェイがこの 町[まち]を 通[とお]るそうだ。
\\	"聞[き]くところ
\\	よれば、 新[あたら]しいハイウェイがこの 町[まち]を 通[とお]るそうだ。
\\	によって 
\\	により/...によれば/...によらないで)”	
\\	に
\\	"不注意[ふちゅうい]
\\	に
\\	より、 友達[ともだち]から 借[か]りた 本[ほん]を 汚[よご]してしまった。
\\	"不注意[ふちゅうい]
\\	より、 友達[ともだち]から 借[か]りた 本[ほん]を 汚[よご]してしまった。
\\	によって 
\\	により/...によれば/...によらないで)”	
\\	に
\\	"最近[さいきん]の 研究[けんきゅう]
\\	に
\\	よれば、 銀河[ぎんが]の 中心[ちゅうしん]にはブラックホールがあるといことだ。
\\	"最近[さいきん]の 研究[けんきゅう]
\\	よれば、 銀河[ぎんが]の 中心[ちゅうしん]にはブラックホールがあるといことだ。
\\	によって 
\\	により/...によれば/...によらないで)”	
\\	に
\\	"新郎[しんろう]
\\	に
\\	新婦[しんぷ]
\\	"新郎[しんろう]
\\	新婦[しんぷ]
\\	に
\\	"鬼[おに]
\\	に
\\	金棒[かなぼう]
\\	"鬼[おに]
\\	金棒[かなぼう]
\\	に
\\	"私[わたし]は 寿司屋[すしや]へお 寿司[すし]を 食[た]べ
\\	に
\\	行[い]きます。
\\	"私[わたし]は 寿司屋[すしや]へお 寿司[すし]を 食[た]べ
\\	行[い]きます。
\\	行く、来る、向かう、出発する、帰る、戻る	
\\	に
\\	"私[わたし]は 友達[ともだち]を 迎[むか]え
\\	に
\\	空港[くうこう]へ 行[い]きました。
\\	"私[わたし]は 友達[ともだち]を 迎[むか]え
\\	空港[くうこう]へ 行[い]きました。
\\	行く、来る、向かう、出発する、帰る、戻る	
\\	に
\\	"宿題[しゅくだい]を 忘[わす]れたので、うちへ
\\	宿題[しゅくだい]を) 取[と]り
\\	に
\\	戻[もど]った。
\\	"宿題[しゅくだい]を 忘[わす]れたので、うちへ
\\	宿題[しゅくだい]を) 取[と]り
\\	戻[もど]った。
\\	行く、来る、向かう、出発する、帰る、戻る	
\\	に
\\	"俊男[としお]、お 友達[ともだち]が 遊[あそ]び
\\	に
\\	来[き]ましたよ。
\\	"俊男[としお]、お 友達[ともだち]が 遊[あそ]び
\\	来[き]ましたよ。
\\	行く、来る、向かう、出発する、帰る、戻る	
\\	に
\\	"今日[きょう]は 待[ま]ち
\\	に
\\	待[ま]った 遠足[えんそく]の 日[ひ]です。
\\	"今日[きょう]は 待[ま]ち
\\	待[ま]った 遠足[えんそく]の 日[ひ]です。
\\	に
\\	"その 知[し]らせを 聞[き]いて、 彼女[かのじょ]は 泣[な]き
\\	に
\\	泣[な]いた。
\\	"その 知[し]らせを 聞[き]いて、 彼女[かのじょ]は 泣[な]き
\\	泣[な]いた。
\\	に
\\	れる、られる, 
\\	"私[わたし]はすり
\\	に
\\	財布[さいふ]をすられた。
\\	"私[わたし]はすり
\\	財布[さいふ]をすられた。
\\	に
\\	れる、られる, 
\\	"僕[ぼく]が 父[ちち]
\\	に
\\	叱[しか]られた。
\\	"僕[ぼく]が 父[ちち]
\\	叱[しか]られた。
\\	に
\\	れる、られる, 
\\	"私[わたし]は 車[くるま]
\\	に
\\	泥[どろ]を 跳[は]ねられた。
\\	"私[わたし]は 車[くるま]
\\	泥[どろ]を 跳[は]ねられた。
\\	に
\\	"(私[わたし]
\\	に
\\	あなたの 写真[しゃしん]を 見[み]せて 下[くだ]さい。
\\	"(私[わたし]
\\	あなたの 写真[しゃしん]を 見[み]せて 下[くだ]さい。
\\	に
\\	"妹[いもうと]は 私[わたし]
\\	に
\\	誕生日[たんじょうび]のプレゼンとをくれました。
\\	"妹[いもうと]は 私[わたし]
\\	誕生日[たんじょうび]のプレゼンとをくれました。
\\	に
\\	"私[わたし]は 彼女[かのじょ]
\\	に
\\	フランス 語[ご]を 教[おし]えた。
\\	"私[わたし]は 彼女[かのじょ]
\\	フランス 語[ご]を 教[おし]えた。
\\	に
\\	~てもらう(~てもらいます), 
\\	"私[わたし]は 梅田[うみだ] 先生[せんせい]
\\	に
\\	ピアノを 教[おし]えてもらいました。
\\	"私[わたし]は 梅田[うみだ] 先生[せんせい]
\\	ピアノを 教[おし]えてもらいました。
\\	に
\\	~てもらう(~てもらいます), 
\\	"私[わたし]は 母[はは]
\\	に
\\	サンドイッチを 作[つく]ってもらった。
\\	"私[わたし]は 母[はは]
\\	サンドイッチを 作[つく]ってもらった。
\\	に
\\	~てもらう(~てもらいます), 
\\	"友達[ともだち]
\\	に
\\	引[ひ]っ 越[こ]しの 手伝[てつだ]いに 来[き]てもらった。
\\	"友達[ともだち]
\\	引[ひ]っ 越[こ]しの 手伝[てつだ]いに 来[き]てもらった。
\\	に
\\	"雨[あめ]がひどくて、ビショビショ
\\	に
\\	濡[ぬ]れてしまった。
\\	"雨[あめ]がひどくて、ビショビショ
\\	濡[ぬ]れてしまった。
\\	に
\\	"このパン、からから
\\	に
\\	乾[かわ]いてるね。
\\	"このパン、からから
\\	乾[かわ]いてるね。
\\	に
\\	は~けれど(が), 
\\	"あのレストランの 料理[りょうり]は、 美味[おい]しい
\\	に
\\	は 美味[おい]しいけれど、ちょっと 高[たか]いね。
\\	"あのレストランの 料理[りょうり]は、 美味[おい]しい
\\	は 美味[おい]しいけれど、ちょっと 高[たか]いね。
\\	に
\\	は~けれど(が), 
\\	"返事[へんじ]、する
\\	に
\\	はするけど、もう 少[すこ]し 待[ま]ってね。
\\	"返事[へんじ]、する
\\	はするけど、もう 少[すこ]し 待[ま]ってね。
\\	に
\\	"今[いま] 思[おも]う
\\	に
\\	、 私[わたし]はあの 時[とき]は 少[すこ]し 努力[どりょく]が 足[た]りなかった。
\\	"今[いま] 思[おも]う
\\	、 私[わたし]はあの 時[とき]は 少[すこ]し 努力[どりょく]が 足[た]りなかった。
\\	に
\\	"考[かんが]えてみる
\\	に
\\	、これ 以上[いじょう]の 要求[ようきゅう]は 無理[むり]だろう。
\\	"考[かんが]えてみる
\\	、これ 以上[いじょう]の 要求[ようきゅう]は 無理[むり]だろう。
\\	に
\\	"恋人[こいびと]もいるだろう
\\	に
\\	、 若[わか]い 命[いのち]を 失[うしな]うとは。
\\	"恋人[こいびと]もいるだろう
\\	、 若[わか]い 命[いのち]を 失[うしな]うとは。
\\	~でしょうに、~であろうに、~だろうに、~でしたでしょうに、~だったでしょうに、~だっただろうに	
\\	に
\\	"ひどいお 天気[てんき]だったでしょう
\\	に
\\	。よく 帰[かえ]れたこと。
\\	"ひどいお 天気[てんき]だったでしょう
\\	。よく 帰[かえ]れたこと。
\\	~でしょうに、~であろうに、~だろうに、~でしたでしょうに、~だったでしょうに、~だっただろうに	
\\	に
\\	"そんなに 休[やす]まず 働[はたら]いて、さぞ 疲[つか]れただろう
\\	に
\\	"そんなに 休[やす]まず 働[はたら]いて、さぞ 疲[つか]れただろう
\\	~でしょうに、~であろうに、~だろうに、~でしたでしょうに、~だったでしょうに、~だっただろうに	
\\	に
\\	"こんな 点数[てんすう]しか 取[と]れなかったか。もっと 勉強[べんきょう]しておけばよかっただろう
\\	に
\\	"こんな 点数[てんすう]しか 取[と]れなかったか。もっと 勉強[べんきょう]しておけばよかっただろう
\\	~でしょうに、~であろうに、~だろうに、~でしたでしょうに、~だったでしょうに、~だっただろうに	
\\	に
\\	"私[わたし]は 大阪[おおさか]
\\	へ
\\	行[い]きます。
\\	"私[わたし]は 大阪[おおさか]
\\	行[い]きます。
\\	へ
\\	"彼[かれ]は 駅[えき]の 方[ほう]
\\	へ
\\	歩[ある]いて行[い]きました。
\\	"彼[かれ]は 駅[えき]の 方[ほう]
\\	歩[ある]いて行[い]きました。
\\	へ
\\	"北野[きたの]さんはヨーロッパ
\\	へ
\\	出張[しゅっちょう] 中[ちゅう]です。
\\	"北野[きたの]さんはヨーロッパ
\\	出張[しゅっちょう] 中[ちゅう]です。
\\	へ
\\	"地震[じしん]で 五階[ごかい] 建物[たてもの]が 後[うしろ]
\\	へ
\\	傾[かたむ]いた。
\\	"地震[じしん]で 五階[ごかい] 建物[たてもの]が 後[うしろ]
\\	傾[かたむ]いた。
\\	へ
\\	"私[わたし]は 今[いま] 空港[くうこう]
\\	へ
\\	着[つ]きました。
\\	"私[わたし]は 今[いま] 空港[くうこう]
\\	着[つ]きました。
\\	へ
\\	"いつ 日本[にほん]
\\	へ
\\	お 帰[かえ]りになりましたか。
\\	"いつ 日本[にほん]
\\	お 帰[かえ]りになりましたか。
\\	へ
\\	"僕[ぼく]は 道[みち]に 迷[まよ]ったあげく、 三十分[さんじゅっぷん]も 遅[おく]れて 会議場[かいぎじょう]
\\	へ
\\	辿[たど]り 着[つ]いた。
\\	"僕[ぼく]は 道[みち]に 迷[まよ]ったあげく、 三十分[さんじゅっぷん]も 遅[おく]れて 会議場[かいぎじょう]
\\	辿[たど]り 着[つ]いた。
\\	へ
\\	"これはあなた
\\	へ
\\	のプレゼンとです。
\\	"これはあなた
\\	のプレゼンとです。
\\	へ
\\	"これは 先生[せんせい]
\\	へ
\\	の 手紙[てがみ]だからお 渡[わた]ししてね。
\\	"これは 先生[せんせい]
\\	の 手紙[てがみ]だからお 渡[わた]ししてね。
\\	へ
\\	"宿題[しゅくだい]が 終[お]わったところ
\\	へ
\\	友達[ともだち]が 遊[あそ]びに 来[き]た。
\\	"宿題[しゅくだい]が 終[お]わったところ
\\	友達[ともだち]が 遊[あそ]びに 来[き]た。
\\	~ところへ 
\\	へ
\\	"傘[かざ]を 持[も]たずに 出[で]かけたところ
\\	へ
\\	、 雨[あめ]が 降[ふ]ってきた。
\\	"傘[かざ]を 持[も]たずに 出[で]かけたところ
\\	、 雨[あめ]が 降[ふ]ってきた。
\\	~ところへ 
\\	へ
\\	"私[わたし]はその 話[はなし]を 先生[せんせい]
\\	から
\\	聞[き]きました。
\\	"私[わたし]はその 話[はなし]を 先生[せんせい]
\\	聞[き]きました。
\\	から
\\	"僕[ぼく]は 彼女[かのじょ]
\\	から
\\	ラブレターをもらった。
\\	"僕[ぼく]は 彼女[かのじょ]
\\	ラブレターをもらった。
\\	から
\\	"その本[ほん]を 誰[だれ]
\\	から
\\	借[か]りましたか。
\\	"その本[ほん]を 誰[だれ]
\\	借[か]りましたか。
\\	から
\\	"私[わたし]は 先生[せんせい]
\\	から
\\	褒[ほ]められて 嬉[うれ]しかった。
\\	"私[わたし]は 先生[せんせい]
\\	褒[ほ]められて 嬉[うれ]しかった。
\\	から
\\	"あたな
\\	から
\\	どうぞお 話[はな]し 下[くだ]さい。
\\	"あたな
\\	どうぞお 話[はな]し 下[くだ]さい。
\\	から
\\	"私[わたし]も 早[はや]く 親[おや]
\\	から
\\	独立[どくりつ]したいんです。
\\	"私[わたし]も 早[はや]く 親[おや]
\\	独立[どくりつ]したいんです。
\\	から
\\	"彼[かれ]はポケット
\\	から
\\	車[くるま]の 鍵[かぎ]を 取[と]り 出[だ]した。
\\	"彼[かれ]はポケット
\\	車[くるま]の 鍵[かぎ]を 取[と]り 出[だ]した。
\\	から
\\	"泥棒[どろぼう]は 窓[まど]
\\	から
\\	入[はい]ったらしい。
\\	"泥棒[どろぼう]は 窓[まど]
\\	入[はい]ったらしい。
\\	から
\\	"木[き]
\\	から
\\	林檎[りんご]が 落[お]ちてきた。
\\	"木[き]
\\	林檎[りんご]が 落[お]ちてきた。
\\	から
\\	"太陽[たいよう]は 東[ひがし]
\\	から
\\	昇[のぼ]ります。
\\	"太陽[たいよう]は 東[ひがし]
\\	昇[のぼ]ります。
\\	から
\\	"ホテルの 屋上[おくじょう]
\\	から
\\	富士山[ふじさん]が 見[み]えました。
\\	"ホテルの 屋上[おくじょう]
\\	富士山[ふじさん]が 見[み]えました。
\\	から
\\	"ステンドグラス
\\	から
\\	夕日[ゆうひ]が 差[さ]し 込[こ]んでいる。
\\	"ステンドグラス
\\	夕日[ゆうひ]が 差[さ]し 込[こ]んでいる。
\\	から
\\	"私[わたし]はその 知識[ちしき]を 本[ほん]
\\	から
\\	得[え]ました。
\\	"私[わたし]はその 知識[ちしき]を 本[ほん]
\\	得[え]ました。
\\	から
\\	"日本語[にほんご]の 授業[じゅぎょう]は、 朝[あさ] 九時[くじ]
\\	から
\\	始[はじ]まります。
\\	"日本語[にほんご]の 授業[じゅぎょう]は、 朝[あさ] 九時[くじ]
\\	始[はじ]まります。
\\	から
\\	"来週[らいしゅう]
\\	から
\\	地下鉄[ちかてつ]が 値上[ねあ]げになる。
\\	"来週[らいしゅう]
\\	地下鉄[ちかてつ]が 値上[ねあ]げになる。
\\	から
\\	"私[わたし]は、 忙[いそが]しくて、 昨日[きのう]の 夜[よる]
\\	から
\\	ずっと 寝[ね]ていません。
\\	"私[わたし]は、 忙[いそが]しくて、 昨日[きのう]の 夜[よる]
\\	ずっと 寝[ね]ていません。
\\	から
\\	"七夕[たなばた]は、 昔[むかし]
\\	から
\\	日本[にほん]に 伝[つた]わっている 行事[ぎょうじ]です。
\\	"七夕[たなばた]は、 昔[むかし]
\\	日本[にほん]に 伝[つた]わっている 行事[ぎょうじ]です。
\\	から
\\	"私[わたし]は、 日本[にほん]に 来[き]て
\\	から
\\	もう 二年[にねん]になります。
\\	"私[わたし]は、 日本[にほん]に 来[き]て
\\	もう 二年[にねん]になります。
\\	から
\\	"祖父[そふ]が 亡[な]くなって
\\	から
\\	五年[ごねん]たちました。
\\	"祖父[そふ]が 亡[な]くなって
\\	五年[ごねん]たちました。
\\	から
\\	"この 幼稚園[ようちえん]には、 二歳[にさい]
\\	から
\\	四歳[よんさい]までの 子供[こども] 達[たち]います。
\\	"この 幼稚園[ようちえん]には、 二歳[にさい]
\\	四歳[よんさい]までの 子供[こども] 達[たち]います。
\\	~から~まで 
\\	から
\\	"三時[さんじ]
\\	から
\\	四時[よじ]までの 間[あいだ]に 来[き]て 下[くだ]さい。
\\	"三時[さんじ]
\\	四時[よじ]までの 間[あいだ]に 来[き]て 下[くだ]さい。
\\	~から~まで 
\\	から
\\	"私[わたし]は、その 記事[きじ]を 隅[すみ]
\\	から
\\	隅[すみ]まで 読[よ]みました。
\\	"私[わたし]は、その 記事[きじ]を 隅[すみ]
\\	隅[すみ]まで 読[よ]みました。
\\	~から~まで 
\\	から
\\	"レストランといっても、ピン
\\	から
\\	きりまであります。
\\	"レストランといっても、ピン
\\	きりまであります。
\\	~から~まで 
\\	から
\\	"夕食[ゆうしょく]を 食[た]べて
\\	から
\\	寝[ね]るまでに 四時[よじ] 間[かん]ある。
\\	"夕食[ゆうしょく]を 食[た]べて
\\	寝[ね]るまでに 四時[よじ] 間[かん]ある。
\\	~から~まで 
\\	から
\\	"花火[はなび]が、 次[つぎ]
\\	から
\\	次[つぎ]へと 打[う]ち 上[あ]げられた。
\\	"花火[はなび]が、 次[つぎ]
\\	次[つぎ]へと 打[う]ち 上[あ]げられた。
\\	~から~へ~ 
\\	から
\\	"大波[おおなみ]が、 次[つぎ]
\\	から
\\	次[つぎ]へと 押[お]し 寄[よ]せてきた。
\\	"大波[おおなみ]が、 次[つぎ]
\\	次[つぎ]へと 押[お]し 寄[よ]せてきた。
\\	~から~へ~ 
\\	から
\\	"私[わたし] 達[たち]は 飛行機[ひこうき]で 島[しま]
\\	から
\\	島[しま]へと 飛[と]んだ。
\\	"私[わたし] 達[たち]は 飛行機[ひこうき]で 島[しま]
\\	島[しま]へと 飛[と]んだ。
\\	~から~へ~ 
\\	から
\\	で	"刃物[はもの]は 鋼鉄[こうてつ]
\\	から
\\	作[つく]られます。
\\	"刃物[はもの]は 鋼鉄[こうてつ]
\\	作[つく]られます。
\\	から
\\	で	"パンは 小麦[こむぎ] 粉[こ]
\\	から
\\	作[つく]ります。
\\	"パンは 小麦[こむぎ] 粉[こ]
\\	作[つく]ります。
\\	から
\\	で	"本[ほん]は 紙[かみ]
\\	から
\\	できています。
\\	"本[ほん]は 紙[かみ]
\\	できています。
\\	から
\\	で	"日本[にほん]は 四[よっ]つの 大[おお]きい 島[しま]
\\	から
\\	なっています。
\\	"日本[にほん]は 四[よっ]つの 大[おお]きい 島[しま]
\\	なっています。
\\	から
\\	で	"この 国[くに]の 経済[けいざい]は、 貿易[ぼうえき]による 収入[しゅうにゅう]
\\	から
\\	成[な]り 立[た]っている。
\\	"この 国[くに]の 経済[けいざい]は、 貿易[ぼうえき]による 収入[しゅうにゅう]
\\	成[な]り 立[た]っている。
\\	から
\\	"百台[ひゃくだい]
\\	から
\\	の 大型[おおがた]トラックが 並[なら]んでいる。
\\	"百台[ひゃくだい]
\\	の 大型[おおがた]トラックが 並[なら]んでいる。
\\	から
\\	"千人[せんにん]
\\	から
\\	の 観光[かんこう] 客[きゃく]が、 毎日[まいにち]ここを 訪[おとず]れる。
\\	"千人[せんにん]
\\	の 観光[かんこう] 客[きゃく]が、 毎日[まいにち]ここを 訪[おとず]れる。
\\	から
\\	"お魚[さかな]が 安[やす]かった
\\	から
\\	沢山[たくさん] 買[か]いました。
\\	"お魚[さかな]が 安[やす]かった
\\	沢山[たくさん] 買[か]いました。
\\	から
\\	"今日[きょう]はいいお 天気[てんき]だ
\\	から
\\	、 桜[さくら]は 満開[まんかい]になるだろう。
\\	"今日[きょう]はいいお 天気[てんき]だ
\\	、 桜[さくら]は 満開[まんかい]になるだろう。
\\	から
\\	"私[わたし]はブロッコリは 嫌[きら]いだ
\\	から
\\	、 絶対[ぜったい]に 食[た]べない。
\\	"私[わたし]はブロッコリは 嫌[きら]いだ
\\	、 絶対[ぜったい]に 食[た]べない。
\\	から
\\	"日本語[にほんご]が 上手[じょうじ]になりたい
\\	から
\\	、 一生懸命[いっしょうけんめい] 勉強[べんきょう]します。
\\	"日本語[にほんご]が 上手[じょうじ]になりたい
\\	、 一生懸命[いっしょうけんめい] 勉強[べんきょう]します。
\\	から
\\	"君[きみ]に 友達[ともだち]が 出来[でき]ないのは、うちにばかりいる
\\	から
\\	だよ。
\\	"君[きみ]に 友達[ともだち]が 出来[でき]ないのは、うちにばかりいる
\\	だよ。
\\	から
\\	"僕[ぼく]が 乗[の]り 過[す]ごしたのは、 君[きみ]が 教[おし]えてくれなかった
\\	から
\\	だぞ。
\\	"僕[ぼく]が 乗[の]り 過[す]ごしたのは、 君[きみ]が 教[おし]えてくれなかった
\\	だぞ。
\\	から
\\	"彼[かれ]は、 妻[つま]との 不和[ふわ]
\\	から
\\	、 別居[べっきょ]した。
\\	"彼[かれ]は、 妻[つま]との 不和[ふわ]
\\	、 別居[べっきょ]した。
\\	から
\\	"彼[かれ]は、 高校[こうこう]の 成績[せいせき]
\\	から
\\	、 奨学金[しょうがくきん]をまらうことになった。
\\	"彼[かれ]は、 高校[こうこう]の 成績[せいせき]
\\	、 奨学金[しょうがくきん]をまらうことになった。
\\	から
\\	"あなたの 眼[め]
\\	から
\\	見[み]れば、この 仕事[しごと]は 簡単[かんたん]でしょう。
\\	"あなたの 眼[め]
\\	見[み]れば、この 仕事[しごと]は 簡単[かんたん]でしょう。
\\	から
\\	"この 手紙[てがみ]の 内容[ないよう]
\\	から
\\	判断[はんだん]すると、 彼[かれ]は 課長[かちょう]に 昇進[しょうしん]するらしい。
\\	"この 手紙[てがみ]の 内容[ないよう]
\\	判断[はんだん]すると、 彼[かれ]は 課長[かちょう]に 昇進[しょうしん]するらしい。
\\	から
\\	"医学的[いがくてき] 見地[けんち]
\\	から
\\	すれば、 彼[かれ]の 研究[けんきゅう]はノーベル 賞[しょう]にも 値[あたい]する。
\\	"医学的[いがくてき] 見地[けんち]
\\	すれば、 彼[かれ]の 研究[けんきゅう]はノーベル 賞[しょう]にも 値[あたい]する。
\\	から
\\	"私[わたし]は 嬉[うれ]しさ
\\	から
\\	飛[と]び 上[あ]がった。
\\	"私[わたし]は 嬉[うれ]しさ
\\	飛[と]び 上[あ]がった。
\\	から
\\	"僕[ぼく]は、 工藤[くどう] 君[くん]ちょっとした 言[い]い 合[あ]い
\\	から
\\	喧嘩[けんか]してしまった。
\\	"僕[ぼく]は、 工藤[くどう] 君[くん]ちょっとした 言[い]い 合[あ]い
\\	喧嘩[けんか]してしまった。
\\	から
\\	"彼[かれ]は 責任感[せきにんかん]
\\	から
\\	辞職[じしょく]した。
\\	"彼[かれ]は 責任感[せきにんかん]
\\	辞職[じしょく]した。
\\	から
\\	"お 小遣[こづか]いをもらった
\\	から
\\	といって、 使[つか]い 過[す]ぎると、すぐなくなるよ。
\\	"お 小遣[こづか]いをもらった
\\	といって、 使[つか]い 過[す]ぎると、すぐなくなるよ。
\\	~からといって	
\\	から
\\	"安[やす]い
\\	から
\\	といって、 沢山[たくさん] 買[か]ってはいけません。
\\	"安[やす]い
\\	といって、 沢山[たくさん] 買[か]ってはいけません。
\\	~からといって	
\\	から
\\	"慣[な]れたところだ
\\	から
\\	といって、 一人[ひとり]で 歩[ある]くのは 危険[きけん]です。
\\	"慣[な]れたところだ
\\	といって、 一人[ひとり]で 歩[ある]くのは 危険[きけん]です。
\\	~からといって	
\\	から
\\	"これだけ 勉強[べんきょう]した
\\	から
\\	には、きっと 合格[ごうかく]して 見[み]せるぞ。
\\	"これだけ 勉強[べんきょう]した
\\	には、きっと 合格[ごうかく]して 見[み]せるぞ。
\\	~からには	
\\	から
\\	"外国[がいこく]へ 行[い]く
\\	から
\\	には、そこの 文化[ぶんか]をよく 知[し]らなければならない。
\\	"外国[がいこく]へ 行[い]く
\\	には、そこの 文化[ぶんか]をよく 知[し]らなければならない。
\\	~からには	
\\	から
\\	"この 会社[かいしゃ]に 入社[にゅうしゃ]した
\\	から
\\	には、 一生懸命[いっしょうけんめい] 頑張[がんば]ろう。
\\	"この 会社[かいしゃ]に 入社[にゅうしゃ]した
\\	には、 一生懸命[いっしょうけんめい] 頑張[がんば]ろう。
\\	~からには	
\\	から
\\	"彼女[かのじょ]は 見[み]る
\\	から
\\	に 優[やさ]しそうな 人[ひと]ですね。
\\	"彼女[かのじょ]は 見[み]る
\\	に 優[やさ]しそうな 人[ひと]ですね。
\\	~からに	
\\	から
\\	"それは 聞[き]く
\\	から
\\	に 恐[こわ]い 話[はなし]だ。
\\	"それは 聞[き]く
\\	に 恐[こわ]い 話[はなし]だ。
\\	~からに	
\\	から
\\	"私[わたし]は 九時[くじ]から 五時[ごじ]
\\	まで
\\	仕事[しごと]をします。
\\	"私[わたし]は 九時[くじ]から 五時[ごじ]
\\	仕事[しごと]をします。
\\	~から~まで	
\\	まで
\\	"彼女[かのじょ]は 朝[あさ]から 晩[ばん]
\\	まで
\\	スケートの 練習[れんしゅう]をした。
\\	"彼女[かのじょ]は 朝[あさ]から 晩[ばん]
\\	スケートの 練習[れんしゅう]をした。
\\	~から~まで	
\\	まで
\\	"次[つぎ]のバス
\\	まで
\\	三十分[さんじゅっぷん]もある。
\\	"次[つぎ]のバス
\\	三十分[さんじゅっぷん]もある。
\\	~から~まで	
\\	まで
\\	"父[ちち]は 百歳[ひゃくさい]
\\	まで
\\	生[い]きた。
\\	"父[ちち]は 百歳[ひゃくさい]
\\	生[い]きた。
\\	~から~まで	
\\	まで
\\	"僕[ぼく]は 九歳[きゅうさい]
\\	まで
\\	ローマで 育[そだ]った。
\\	"僕[ぼく]は 九歳[きゅうさい]
\\	ローマで 育[そだ]った。
\\	~から~まで	
\\	まで
\\	"ストライキが 解決[かいけつ]する
\\	まで
\\	電車[でんしゃ]は 動[うご]かない。
\\	"ストライキが 解決[かいけつ]する
\\	電車[でんしゃ]は 動[うご]かない。
\\	~から~まで	
\\	まで
\\	"今[いま]
\\	まで
\\	何[なに]をしていたの?
\\	"今[いま]
\\	何[なに]をしていたの?
\\	~から~まで	
\\	まで
\\	"私[わたし]は、うちから 学校[がっこう]
\\	まで
\\	自転車[じてんしゃ]で 通[かよ]っています。
\\	"私[わたし]は、うちから 学校[がっこう]
\\	自転車[じてんしゃ]で 通[かよ]っています。
\\	まで
\\	"この 汽車[きしゃ]は、 青森[あおもり]
\\	まで
\\	行[い]きます。
\\	"この 汽車[きしゃ]は、 青森[あおもり]
\\	行[い]きます。
\\	まで
\\	"中間[ちゅうかん]テストの 範囲[はんい]は 一課[いっか]から 四課[よんか]
\\	まで
\\	です。
\\	"中間[ちゅうかん]テストの 範囲[はんい]は 一課[いっか]から 四課[よんか]
\\	です。
\\	まで
\\	"どこ
\\	まで
\\	いっらっしゃいますか?
\\	"どこ
\\	いっらっしゃいますか?
\\	まで
\\	"奈良[なら]
\\	まで
\\	の 切符[きっぷ]を 二枚[にまい] 下[くだ]さい。
\\	"奈良[なら]
\\	の 切符[きっぷ]を 二枚[にまい] 下[くだ]さい。
\\	まで
\\	"結婚[けっこん] 式[しき]の 費用[ひよう]は 百万円[ひゃくまんえん]
\\	まで
\\	にしておこう。
\\	"結婚[けっこん] 式[しき]の 費用[ひよう]は 百万円[ひゃくまんえん]
\\	にしておこう。
\\	まで
\\	"ミスサクラ・コンテストの 参加者[さんかしゃ]を 募集[ぼしゅう]します。 年齢[ねんれい]は 二十五歳[にじゅうごさい]
\\	まで
\\	です。
\\	"ミスサクラ・コンテストの 参加者[さんかしゃ]を 募集[ぼしゅう]します。 年齢[ねんれい]は 二十五歳[にじゅうごさい]
\\	です。
\\	まで
\\	"本[ほん]の 貸[か]し 出[だ]しは 一回[いっかい] 三冊[さんさつ]
\\	まで
\\	です。
\\	"本[ほん]の 貸[か]し 出[だ]しは 一回[いっかい] 三冊[さんさつ]
\\	です。
\\	まで
\\	"洪水[こうずい]で 水[みず]が 床[ゆか]の 上[うえ]
\\	まで
\\	来[き]た。
\\	"洪水[こうずい]で 水[みず]が 床[ゆか]の 上[うえ]
\\	来[き]た。
\\	まで
\\	"ご飯[はん]は 最後[さいご]の 一粒[ひとつぶ]
\\	まで
\\	食[た]べなさい。
\\	"ご飯[はん]は 最後[さいご]の 一粒[ひとつぶ]
\\	食[た]べなさい。
\\	まで
\\	"奨学金[しょうがくきん]がもらえるようになる
\\	まで
\\	頑張[がんば]ろう。
\\	"奨学金[しょうがくきん]がもらえるようになる
\\	頑張[がんば]ろう。
\\	まで
\\	"目的[もくてき] 地[ち]が 見[み]つからなくて、 私[わたし]は 足[あし]が 棒[ぼう]のようになる
\\	まで
\\	探[さが]して 歩[ある]いた。
\\	"目的[もくてき] 地[ち]が 見[み]つからなくて、 私[わたし]は 足[あし]が 棒[ぼう]のようになる
\\	探[さが]して 歩[ある]いた。
\\	まで
\\	"友達[ともだち]に
\\	まで
\\	見放[みはな]されるようじゅ、お 終[しま]いだ。
\\	"友達[ともだち]に
\\	見放[みはな]されるようじゅ、お 終[しま]いだ。
\\	まで
\\	"坊主[ぼうず] 憎[にく]けりゃ 袈裟[けさ]
\\	まで
\\	憎[にく]い。
\\	諺[ことわざ])
\\	"坊主[ぼうず] 憎[にく]けりゃ 袈裟[けさ]
\\	憎[にく]い。
\\	諺[ことわざ])
\\	まで
\\	"今朝[けさ] 早起[はやお]きしたのは、タマタマ 小鳥[ことり]の 鳴[な]き 声[ごえ]で 目[め]が 覚[さ]めた
\\	まで
\\	だよ。
\\	"今朝[けさ] 早起[はやお]きしたのは、タマタマ 小鳥[ことり]の 鳴[な]き 声[ごえ]で 目[め]が 覚[さ]めた
\\	だよ。
\\	まで
\\	"私[わたし]が 会社[かいしゃ]で 出世[しゅっせ]できたのは、 運[うん]がよかった
\\	まで
\\	のことです。
\\	"私[わたし]が 会社[かいしゃ]で 出世[しゅっせ]できたのは、 運[うん]がよかった
\\	のことです。
\\	まで
\\	"今度[こんど]はいついらしてくださるかと 思[おも]って…。ちょっとお 電話[でんわ]した
\\	まで
\\	よ。
\\	"今度[こんど]はいついらしてくださるかと 思[おも]って…。ちょっとお 電話[でんわ]した
\\	よ。
\\	まで
\\	"そんなにおっしゃらないでください。 私[わたし]は 只[ただ] 当[あ]たり 前[まえ]のことをした
\\	まで
\\	です。
\\	"そんなにおっしゃらないでください。 私[わたし]は 只[ただ] 当[あ]たり 前[まえ]のことをした
\\	です。
\\	まで
\\	"私[わたし] 達[たち]、 今日[きょう]お 宅[たく]の 隣[となり]に 越[こ]して 参[まい]りましたので、ご 挨拶[あいたつ]
\\	までに
\\	お 伺[うかが]いしました。
\\	"私[わたし] 達[たち]、 今日[きょう]お 宅[たく]の 隣[となり]に 越[こ]して 参[まい]りましたので、ご 挨拶[あいたつ]
\\	お 伺[うかが]いしました。
\\	までに
\\	"御結婚[ごけっこん]おめでとうございます。ほんのお 祝[いわい]のしるし
\\	までに
\\	ささやかな 品[しな]をお 送[おく]りしました。
\\	"御結婚[ごけっこん]おめでとうございます。ほんのお 祝[いわい]のしるし
\\	ささやかな 品[しな]をお 送[おく]りしました。
\\	までに
\\	"今期[こんき]の 利益[りえき]は 百億円[ひゃくおくえん]
\\	までに
\\	達[たっ]した。
\\	"今期[こんき]の 利益[りえき]は 百億円[ひゃくおくえん]
\\	達[たっ]した。
\\	までに
\\	"彼女[かのじょ]の 名前[なまえ]は、 世界[せかい]の 隅々[すみずみ]
\\	までに
\\	知[し]れ 渡[わた]った。
\\	"彼女[かのじょ]の 名前[なまえ]は、 世界[せかい]の 隅々[すみずみ]
\\	知[し]れ 渡[わた]った。
\\	までに
\\	"飛行機[ひこうき]が 出[で]る
\\	までに
\\	は、まだ 一時間[いちじかん]ぐらいある。
\\	"飛行機[ひこうき]が 出[で]る
\\	は、まだ 一時間[いちじかん]ぐらいある。
\\	までに
\\	"この絵[え]を 書[か]き 上[あ]げる
\\	までに
\\	は、あと 一ヶ月[いっかげつ]は 必要[ひつよう]です。
\\	"この絵[え]を 書[か]き 上[あ]げる
\\	は、あと 一ヶ月[いっかげつ]は 必要[ひつよう]です。
\\	までに
\\	"林檎[りんご]
\\	と
\\	ぶどうがテーブルの 上[うえ]にあります。
\\	"林檎[りんご]
\\	ぶどうがテーブルの 上[うえ]にあります。
\\	と
\\	"赤[あか]い 林檎[りんご]
\\	と
\\	緑[みどり]のぶどうもテーブルの 上[うえ]にあります。
\\	"赤[あか]い 林檎[りんご]
\\	緑[みどり]のぶどうもテーブルの 上[うえ]にあります。
\\	と
\\	"私[わたし]が 買[か]った 林檎[りんご]
\\	と
\\	あなたが 買[か]ったぶどうはテーブルの 上[うえ]にあるます。
\\	"私[わたし]が 買[か]った 林檎[りんご]
\\	あなたが 買[か]ったぶどうはテーブルの 上[うえ]にあるます。
\\	と
\\	"スキムミルク
\\	と
\\	普通[ふつう]のミルク
\\	と
\\	どちらを 買[か]いましょうか。
\\	"スキムミルク
\\	普通[ふつう]のミルク
\\	どちらを 買[か]いましょうか。
\\	と 
\\	と 
\\	と
\\	"夏[なつ]
\\	と
\\	冬[ふゆ]
\\	と
\\	では 気温[きぶん]は 大変[たいへん] 違[ちが]います。
\\	"夏[なつ]
\\	冬[ふゆ]
\\	では 気温[きぶん]は 大変[たいへん] 違[ちが]います。
\\	と 
\\	と 
\\	と
\\	"谷[たに] 君[くん]の 書[か]いた 作文[さくぶん]
\\	と
\\	北川[きたがわ] 君[くん]の 書[か]いた 作文[さくぶん]
\\	と
\\	あまたはどちらの
\\	"谷[たに] 君[くん]の 書[か]いた 作文[さくぶん]
\\	北川[きたがわ] 君[くん]の 書[か]いた 作文[さくぶん]
\\	あまたはどちらの
\\	と 
\\	と 
\\	と
\\	"あれからもう 二十年[にじゅうねん]か、この 街[まち]も 昔[むかし]
\\	と
\\	随分[ずいぶん] 変[かわ]ったね。
\\	"あれからもう 二十年[にじゅうねん]か、この 街[まち]も 昔[むかし]
\\	随分[ずいぶん] 変[かわ]ったね。
\\	と 
\\	と 
\\	と
\\	"昨日[きのう]は、 弟[おとうと]
\\	と
\\	川[かわ]へ 泳[およ]ぎに 行[い]きました。
\\	"昨日[きのう]は、 弟[おとうと]
\\	川[かわ]へ 泳[およ]ぎに 行[い]きました。
\\	と
\\	"今日[きょう]お 昼[ひる]に、 僕[ぼく]はお 父[とう]さん
\\	と
\\	お 母[かあ]さん
\\	と
\\	ハンバーガーを 食[た]べました。
\\	"今日[きょう]お 昼[ひる]に、 僕[ぼく]はお 父[とう]さん
\\	お 母[かあ]さん
\\	ハンバーガーを 食[た]べました。
\\	と
\\	"北島[きたじま]さんは 来月[らいげつ] 南田[みなみだ]さん
\\	と
\\	結婚[けっこん]します。
\\	"北島[きたじま]さんは 来月[らいげつ] 南田[みなみだ]さん
\\	結婚[けっこん]します。
\\	と
\\	"彼[かれ]は 恐怖[きょうふ]
\\	と
\\	闘[たたか]って、ついに 勝[か]った。
\\	"彼[かれ]は 恐怖[きょうふ]
\\	闘[たたか]って、ついに 勝[か]った。
\\	と
\\	"水野[みずの] 君[くん]はいろいろな 困難[こんなん]
\\	と
\\	闘[たたか]って、ついに 優勝[ゆうしょう]した。
\\	"水野[みずの] 君[くん]はいろいろな 困難[こんなん]
\\	闘[たたか]って、ついに 優勝[ゆうしょう]した。
\\	と
\\	呼ぶ、なる、名付ける. 
\\	"竹下[たけした] 君[くん]は 工学[こうがく] 部[ぶ]の 学生[がくせい]
\\	と
\\	なった。
\\	"竹下[たけした] 君[くん]は 工学[こうがく] 部[ぶ]の 学生[がくせい]
\\	なった。
\\	と
\\	呼ぶ、なる、名付ける. 
\\	"友達[ともだち]は 僕[ぼく]のことをタッちゃん
\\	と
\\	呼[よ]びます。
\\	"友達[ともだち]は 僕[ぼく]のことをタッちゃん
\\	呼[よ]びます。
\\	と
\\	呼ぶ、なる、名付ける. 
\\	"新[あたら]しい 喫茶店[きっさてん]の 名前[なまえ]を「ドリーム」
\\	と
\\	つけました。
\\	"新[あたら]しい 喫茶店[きっさてん]の 名前[なまえ]を「ドリーム」
\\	つけました。
\\	と
\\	"いよいよ 運動会[うんどうかい]の 日[ひ]
\\	と
\\	なりました。
\\	"いよいよ 運動会[うんどうかい]の 日[ひ]
\\	なりました。
\\	と
\\	"クリスマスの 街頭[がいとう] 募金[ぼきん]は 全部[ぜんぶ]で 百万円[ひゃくまんえん]
\\	と
\\	なった。
\\	"クリスマスの 街頭[がいとう] 募金[ぼきん]は 全部[ぜんぶ]で 百万円[ひゃくまんえん]
\\	なった。
\\	と
\\	"あの 広場[ひろば]は 子供[こども] 達[たち]の 公園[こうえん]
\\	と
\\	なった。
\\	"あの 広場[ひろば]は 子供[こども] 達[たち]の 公園[こうえん]
\\	なった。
\\	と
\\	"そんな 仕事[しごと]は 三日[みっか]
\\	と
\\	かかりません。
\\	"そんな 仕事[しごと]は 三日[みっか]
\\	かかりません。
\\	と
\\	"ひまわりの 種[たね]をまいてから、 一週間[いっしゅうかん]
\\	と
\\	たたないうち 芽[め]が 出[で]てきました。
\\	"ひまわりの 種[たね]をまいてから、 一週間[いっしゅうかん]
\\	たたないうち 芽[め]が 出[で]てきました。
\\	と
\\	"もう 二度[にど]
\\	と
\\	あんな 所[ところ]へは 行[い]かない。
\\	"もう 二度[にど]
\\	あんな 所[ところ]へは 行[い]かない。
\\	と
\\	"君[きみ]が 信[しん]じよう
\\	と
\\	信[しん]じまい
\\	と
\\	、それは 本当[ほんとう]に 起[お]こったんだよ。
\\	"君[きみ]が 信[しん]じよう
\\	信[しん]じまい
\\	、それは 本当[ほんとう]に 起[お]こったんだよ。
\\	(よ)うと~まいと、
\\	であろうと~(で)なかろうと、
\\	かろうと~なかろうと	
\\	と
\\	"それが 事実[じじつ]であろう
\\	と
\\	なかろう
\\	と
\\	、 私[わたし]は 構[かま]わない。
\\	"それが 事実[じじつ]であろう
\\	なかろう
\\	、 私[わたし]は 構[かま]わない。
\\	(よ)うと~まいと、
\\	であろうと~(で)なかろうと、
\\	かろうと~なかろうと	
\\	と
\\	"大[おお]きかろう
\\	と
\\	小[ちい]さかろう
\\	と
\\	味[あじ]さえよければいい。
\\	"大[おお]きかろう
\\	小[ちい]さかろう
\\	味[あじ]さえよければいい。
\\	(よ)うと~まいと、
\\	であろうと~(で)なかろうと、
\\	かろうと~なかろうと	
\\	と
\\	"いつかは 大地震[だいじしん]があるだろう
\\	と
\\	思[おも]う。
\\	"いつかは 大地震[だいじしん]があるだろう
\\	思[おも]う。
\\	と
\\	"期末[きまつ]テストは 二週間[にしゅうかん] 先[さき]
\\	と
\\	決[きま]った。
\\	"期末[きまつ]テストは 二週間[にしゅうかん] 先[さき]
\\	決[きま]った。
\\	と
\\	"お 化[ば]けか
\\	と
\\	思[おも]ったら、 柳[やなぎ]の 木[き]だよ。
\\	"お 化[ば]けか
\\	思[おも]ったら、 柳[やなぎ]の 木[き]だよ。
\\	と
\\	"マラソンは 苦[くる]しかったが、 友達[ともだち]に 遅[おく]れまい
\\	と
\\	一生懸命[いっしょうけんめい] 走[はし]った。
\\	"マラソンは 苦[くる]しかったが、 友達[ともだち]に 遅[おく]れまい
\\	一生懸命[いっしょうけんめい] 走[はし]った。
\\	と
\\	"お 店[みせ]は 今日[きょう]は 開[あ]いてないよ。ほら、
\\	本日[ほんじつ] 休業[きゅうぎょう]」
\\	と
\\	出[で]ているでしょう。
\\	"お 店[みせ]は 今日[きょう]は 開[あ]いてないよ。ほら、
\\	本日[ほんじつ] 休業[きゅうぎょう]」
\\	出[で]ているでしょう。
\\	と
\\	"そこにじっ
\\	と
\\	座[すわ]っていなさい。
\\	"そこにじっ
\\	座[すわ]っていなさい。
\\	と
\\	"廊下[ろうか]をバタバタ
\\	と
\\	走[はし]ってはいけません。
\\	"廊下[ろうか]をバタバタ
\\	走[はし]ってはいけません。
\\	と
\\	"ゴロゴロ
\\	と
\\	雷[かみなり]がなっている。
\\	"ゴロゴロ
\\	雷[かみなり]がなっている。
\\	と
\\	"大[おお]きな 木[き]がどさっ
\\	と
\\	倒[たお]れた。
\\	"大[おお]きな 木[き]がどさっ
\\	倒[たお]れた。
\\	と
\\	"先生[せんせい]が 教室[きょうしつ]に 入[はい]って 来[く]る
\\	と
\\	、 生徒[せいと] 達[たち]は 慌[あわ]てて 席[せき]に 着[つ]いた。
\\	"先生[せんせい]が 教室[きょうしつ]に 入[はい]って 来[く]る
\\	、 生徒[せいと] 達[たち]は 慌[あわ]てて 席[せき]に 着[つ]いた。
\\	と
\\	"僕[ぼく]が 学校[がっこう]に 着[つ]く
\\	と
\\	、 友達[ともだち]が 大勢[おおぜい] 寄[よ]ってきた。
\\	"僕[ぼく]が 学校[がっこう]に 着[つ]く
\\	、 友達[ともだち]が 大勢[おおぜい] 寄[よ]ってきた。
\\	と
\\	"幕[まく]が 上[あ]がる
\\	と
\\	、 観客[かんきゃく]は 水[みず]を 打[う]ったように 静[しず]かになった。
\\	"幕[まく]が 上[あ]がる
\\	、 観客[かんきゃく]は 水[みず]を 打[う]ったように 静[しず]かになった。
\\	と
\\	"雪[ゆき]が 降[ふ]り 始[はじ]める
\\	と
\\	、 動物[どうぶつ] 達[たち]は 冬眠[とうみん]を 始[はじ]める。
\\	"雪[ゆき]が 降[ふ]り 始[はじ]める
\\	、 動物[どうぶつ] 達[たち]は 冬眠[とうみん]を 始[はじ]める。
\\	と
\\	"あんまり 安[やす]い
\\	と
\\	かえって 心配[しんぱい]だ。
\\	"あんまり 安[やす]い
\\	かえって 心配[しんぱい]だ。
\\	と
\\	"君[きみ]が 来[こ]ない
\\	と
\\	つまらないなあ。
\\	"君[きみ]が 来[こ]ない
\\	つまらないなあ。
\\	と
\\	"この 古[ふる]い 戸[と]、 壊[こわ]れる
\\	と
\\	いけないから 早[はや]く 直[なお]しましょう。
\\	"この 古[ふる]い 戸[と]、 壊[こわ]れる
\\	いけないから 早[はや]く 直[なお]しましょう。
\\	と
\\	"雨[あめ]が 降[ふ]る
\\	と
\\	、 水不足[みずぶそく]が 解消[かいしょう]するのですが。
\\	"雨[あめ]が 降[ふ]る
\\	、 水不足[みずぶそく]が 解消[かいしょう]するのですが。
\\	と
\\	"あのうどん 屋[や]に 行[い]く
\\	と
\\	、 美味[おい]しいうどんが 食[た]べられるよ。
\\	"あのうどん 屋[や]に 行[い]く
\\	、 美味[おい]しいうどんが 食[た]べられるよ。
\\	と
\\	"デパートにはあり
\\	と
\\	有らゆる[あ] 品物[しなもの]が 並[なら]んでいる。
\\	"デパートにはあり
\\	有らゆる[あ] 品物[しなもの]が 並[なら]んでいる。
\\	ありとあらゆる~	
\\	と
\\	"彼[かれ]は 昼[ひる]
\\	と
\\	なく 夜[よる]
\\	と
\\	なく 働[はたら]いた。
\\	"彼[かれ]は 昼[ひる]
\\	なく 夜[よる]
\\	なく 働[はたら]いた。
\\	~となく~となく	
\\	と
\\	"物置[ものおき]に 自転車[じてんしゃ]
\\	や
\\	スキーがしまってあります。
\\	"物置[ものおき]に 自転車[じてんしゃ]
\\	スキーがしまってあります。
\\	や
\\	"夏休[なつやす]みで、ホテル
\\	や
\\	旅館[りょかん]はどこも 満員[まんいん]だ。
\\	"夏休[なつやす]みで、ホテル
\\	旅館[りょかん]はどこも 満員[まんいん]だ。
\\	や
\\	"今日[きょう]は 貴[たかし] 君[くん]の 誕生日[たんじょうび]パーティーだったの。 守[まもる] 君[くん]
\\	や
\\	朝子[あさこ]ちゃん
\\	や
\\	智[とも]ちゃん
\\	や
\\	誠[まこと] 君[くん]、 仲良[なかよ]しがみんあ 集[あつ]まったのよ。
\\	"今日[きょう]は 貴[たかし] 君[くん]の 誕生日[たんじょうび]パーティーだったの。 守[まもる] 君[くん]
\\	朝子[あさこ]ちゃん
\\	智[とも]ちゃん
\\	誠[まこと] 君[くん]、 仲良[なかよ]しがみんあ 集[あつ]まったのよ。
\\	や
\\	"彼女[かのじょ]は、 僕[ぼく]の 姿[すがた]を 見[み]る
\\	や
\\	、 怒り[いかり]を 爆発[ばくはつ]させた。
\\	"彼女[かのじょ]は、 僕[ぼく]の 姿[すがた]を 見[み]る
\\	、 怒り[いかり]を 爆発[ばくはつ]させた。
\\	~やいなや 
\\	や
\\	"地震[じしん]が 発生[はっせい]する
\\	や
\\	、 気象庁[きしょうちょう]は 直[ただ]ちに 津波[つなみ] 警報[けいほう]を 出[だ]した。
\\	"地震[じしん]が 発生[はっせい]する
\\	、 気象庁[きしょうちょう]は 直[ただ]ちに 津波[つなみ] 警報[けいほう]を 出[だ]した。
\\	~やいなや 
\\	や
\\	"まあ、もう 少[すこ]し 話[なはし]をしてから 帰[かえ]ろう
\\	や
\\	"まあ、もう 少[すこ]し 話[なはし]をしてから 帰[かえ]ろう
\\	や
\\	"久[ひさ]しぶりに、うなぎでも 食[た]べに 行[い]こう
\\	や
\\	"久[ひさ]しぶりに、うなぎでも 食[た]べに 行[い]こう
\\	や
\\	"もうやめよう
\\	や
\\	ソフトボール。そろそろ 暗[くら]くなってきたよ。
\\	"もうやめよう
\\	、ソフトボール。そろそろ 暗[くら]くなってきたよ。
\\	や
\\	"文子[ふみこ]さん、 僕[ぼく]の 気持[きも]ち 分[わ]かってくれないんだ。まあ 仕方[しかた]がない
\\	や
\\	。そのうち 分[わ]かってくれるさ。
\\	"文子[ふみこ]さん、 僕[ぼく]の 気持[きも]ち 分[わ]かってくれないんだ。まあ 仕方[しかた]がない
\\	。そのうち 分[わ]かってくれるさ。
\\	や
\\	"まあ、いい
\\	や
\\	。もう 一度[いちど]やり 直[なお]そう。
\\	"まあ、いい
\\	。もう 一度[いちど]やり 直[なお]そう。
\\	や
\\	"園田[そのだ] 君[くん]、ガールフレンドが 出来[でき]たんだって。 羨[うらや]ましい
\\	や
\\	"園田[そのだ] 君[くん]、ガールフレンドが 出来[でき]たんだって。 羨[うらや]ましい
\\	や
\\	"あ、ボールが 当[あ]たって 窓[まど]ガラスが 割[わ]れちゃった。どうにもならない
\\	や
\\	。みんなで 謝[あやま]りに 行[い]こうよ。
\\	"あ、ボールが 当[あ]たって 窓[まど]ガラスが 割[わ]れちゃった。どうにもならない
\\	。みんなで 謝[あやま]りに 行[い]こうよ。
\\	や
\\	"芳子[よしこ]
\\	や
\\	、ちょっと 針[はり]に 糸[いと]を 通[とお]しておくれでないかい?
\\	"芳子[よしこ]
\\	、ちょっと 針[はり]に 糸[いと]を 通[とお]しておくれでないかい?
\\	や
\\	"おじいさん
\\	や
\\	、 今日[きょう]はあなたの 誕生日[たんじょうび]ですよ。
\\	"おじいさん
\\	、 今日[きょう]はあなたの 誕生日[たんじょうび]ですよ。
\\	や
\\	またも
\\	まして
\\	"僕[ぼく]は、またも
\\	や
\\	彼女[かのじょ]を 怒[おこ]らせてしまった。
\\	"僕[ぼく]は、またも
\\	彼女[かのじょ]を 怒[おこ]らせてしまった。
\\	またもや、ましてや	
\\	や
\\	またも
\\	まして
\\	"このドレス、 去年[きょねん]でさえ 窮屈[きゅうくつ]だったのよ。まして
\\	や
\\	、 今年[ことし]はまた 太[ふと]ったもの、 入[はい]るはずがないわ。
\\	"このドレス、 去年[きょねん]でさえ 窮屈[きゅうくつ]だったのよ。まして
\\	、 今年[ことし]はまた 太[ふと]ったもの、 入[はい]るはずがないわ。
\\	またもや、ましてや	
\\	や
\\	"赤[あか]い 林檎[りんご]
\\	だけ
\\	三[みっ]つください。
\\	"赤[あか]い 林檎[りんご]
\\	三[みっ]つください。
\\	だけ
\\	"私[わたし]は 一人[ひとり]
\\	だけ
\\	取[と]り 残[のこ]された。
\\	"私[わたし]は 一人[ひとり]
\\	取[と]り 残[のこ]された。
\\	だけ
\\	"人[ひと]は 外見[がいけん]
\\	だけ
\\	では 分[わ]からない。
\\	"人[ひと]は 外見[がいけん]
\\	では 分[わ]からない。
\\	だけ
\\	"ここにあるのは 古[ふる]い 新聞[しんぶん]
\\	だけ
\\	です。
\\	"ここにあるのは 古[ふる]い 新聞[しんぶん]
\\	です。
\\	だけ
\\	"ここ
\\	だけ
\\	の 話[はなし]ですが、 私[わたし]は 南極[なんきょく] 探検[たんけん]を 計画[けいかく]しているんです。
\\	"ここ
\\	の 話[はなし]ですが、 私[わたし]は 南極[なんきょく] 探検[たんけん]を 計画[けいかく]しているんです。
\\	だけ
\\	る、れる、られる 
\\	"私[わたし]は 歩[ある]ける
\\	だけ
\\	歩[ある]きました。
\\	"私[わたし]は 歩[ある]ける
\\	歩[ある]きました。
\\	だけ
\\	る、れる、られる 
\\	"出[できる] 来る
\\	だけ
\\	助[たす]けてあげましょう。
\\	"出[できる] 来る
\\	助[たす]けてあげましょう。
\\	だけ
\\	る、れる、られる 
\\	"やれる
\\	だけ
\\	やってみようよ。
\\	"やれる
\\	やってみようよ。
\\	だけ
\\	"安[やす]ければ 安[やす]い
\\	だけ
\\	いいです。
\\	"安[やす]ければ 安[やす]い
\\	いいです。
\\	ば~だけ	
\\	だけ
\\	"早[はや]くすればする
\\	だけ
\\	有利[ゆうり]だ。
\\	"早[はや]くすればする
\\	有利[ゆうり]だ。
\\	ば~だけ	
\\	だけ
\\	"ケーキ
\\	だけ
\\	でなく、アイスクリームも 食[た]べましょうよ。
\\	"ケーキ
\\	でなく、アイスクリームも 食[た]べましょうよ。
\\	~だけで(じゃ)なく~も	
\\	だけ
\\	"この 部屋[へや]は 広[ひろ]い
\\	だけ
\\	でなく、とても 明[あか]るいね。
\\	"この 部屋[へや]は 広[ひろ]い
\\	でなく、とても 明[あか]るいね。
\\	~だけで(じゃ)なく~も	
\\	だけ
\\	"この 犬[いぬ]は、 上手[じょうず]に 泳[およ]ぐ
\\	だけ
\\	でなく 芸[げい]もうまい。
\\	"この 犬[いぬ]は、 上手[じょうず]に 泳[およ]ぐ
\\	でなく 芸[げい]もうまい。
\\	~だけで(じゃ)なく~も	
\\	だけ
\\	"期末[きまつ]テストまで、 後[あと]たった 一週間[いっしゅうかん]
\\	しか
\\	ありません。
\\	"期末[きまつ]テストまで、 後[あと]たった 一週間[いっしゅうかん]
\\	ありません。
\\	しか
\\	"お 母[かあ]さん、 今日[きょう]のおやつこれ
\\	しか
\\	ないの?
\\	"お 母[かあ]さん、 今日[きょう]のおやつこれ
\\	ないの?
\\	しか
\\	"セミナーのレポートの 締[し]め 切[き]りは 明後日[あさって]。こうなったら 徹夜[てつや]する
\\	しか
\\	ない。
\\	"セミナーのレポートの 締[し]め 切[き]りは 明後日[あさって]。こうなったら 徹夜[てつや]する
\\	ない。
\\	しか
\\	"財布[さいふ]を 盗[ぬす]まれたから、 歩[ある]いて 帰[かえ]る
\\	しか
\\	ない。
\\	"財布[さいふ]を 盗[ぬす]まれたから、 歩[ある]いて 帰[かえ]る
\\	ない。
\\	しか
\\	"もう 一週間[いっしゅうかん]
\\	くらい
\\	で 花[はな]が 咲[さ]くでしょう。
\\	"もう 一週間[いっしゅうかん]
\\	で 花[はな]が 咲[さ]くでしょう。
\\	くらい
\\	"水[みず]をグラスに 半分[はんぶん]
\\	ぐらい
\\	ください。
\\	"水[みず]をグラスに 半分[はんぶん]
\\	ください。
\\	ぐらい
\\	"簡単[かんたん]な 料理[りょうり]
\\	くらい
\\	私[わたし]だって 出来[でき]るわ。
\\	"簡単[かんたん]な 料理[りょうり]
\\	私[わたし]だって 出来[でき]るわ。
\\	くらい
\\	"洗濯[せんたく]
\\	くらい
\\	自分[じぶん]でしなさい。
\\	"洗濯[せんたく]
\\	自分[じぶん]でしなさい。
\\	くらい
\\	"空[そら] 一面[いちめん] 真[ま]っ 赤[か]になる
\\	くらい
\\	の、 見事[みごと]な 夕焼[ゆうや]けでした。
\\	"空[そら] 一面[いちめん] 真[ま]っ 赤[か]になる
\\	の、 見事[みごと]な 夕焼[ゆうや]けでした。
\\	くらい
\\	"主人[しゅじん]はこんなに 遅[おそ]くまで 働[はたら]いている
\\	くらい
\\	だから、よほど 仕事[しごと]が 忙[いそが]しいらしい。
\\	"主人[しゅじん]はこんなに 遅[おそ]くまで 働[はたら]いている
\\	だから、よほど 仕事[しごと]が 忙[いそが]しいらしい。
\\	くらい
\\	"地震[じしん]
\\	くらい
\\	恐[こわ]いものはない。
\\	"地震[じしん]
\\	恐[こわ]いものはない。
\\	くらい
\\	"お 父[とう]さん
\\	くらい
\\	背[せ]が 高[たか]かったらいいのになあ。
\\	"お 父[とう]さん
\\	背[せ]が 高[たか]かったらいいのになあ。
\\	くらい
\\	"そんな 不味[まず]いものを 食[た]べる
\\	くらい
\\	なら、 何[なに]も 食[た]べないほうがましだ。
\\	"そんな 不味[まず]いものを 食[た]べる
\\	なら、 何[なに]も 食[た]べないほうがましだ。
\\	くらい
\\	"期日[きじつ]に 遅[おく]れる
\\	くらい
\\	なら、 徹夜[てつや]をして 完成[かんせい]させたほうがましだ。
\\	"期日[きじつ]に 遅[おく]れる
\\	なら、 徹夜[てつや]をして 完成[かんせい]させたほうがましだ。
\\	くらい
\\	"この 林檎[りんご]は、あの 林檎[りんご]
\\	より
\\	大[おお]きいです。
\\	"この 林檎[りんご]は、あの 林檎[りんご]
\\	大[おお]きいです。
\\	よりか 
\\	よりも	
\\	より
\\	"私[わたし]は、あなた
\\	より
\\	(か・も) 背[せ]が 高[たか]い。
\\	"私[わたし]は、あなた
\\	(か・も) 背[せ]が 高[たか]い。
\\	よりか 
\\	よりも	
\\	より
\\	"今日[きょう]のテストは 思[おも]った
\\	より
\\	(か・も) 難[むずか]しくなかった。
\\	"今日[きょう]のテストは 思[おも]った
\\	(か・も) 難[むずか]しくなかった。
\\	よりか 
\\	よりも	
\\	より
\\	"昨日[きのう]
\\	より
\\	(か・も) 今日[きょう]のほうがずっと 寒[さむ]い。
\\	"昨日[きのう]
\\	(か・も) 今日[きょう]のほうがずっと 寒[さむ]い。
\\	よりか 
\\	よりも	
\\	より
\\	"僕[ぼく] 達[たち]の 学校[がっこう]は、 駅[えき]
\\	より
\\	手前[てまえ]にあります。
\\	"僕[ぼく] 達[たち]の 学校[がっこう]は、 駅[えき]
\\	手前[てまえ]にあります。
\\	より
\\	"まだお 腹[なか]が 空[す]いていないなら、 夕食[ゆうしょく]は 七時[しちじ]
\\	より
\\	後[あと]にしましょう。
\\	"まだお 腹[なか]が 空[す]いていないなら、 夕食[ゆうしょく]は 七時[しちじ]
\\	後[あと]にしましょう。
\\	より
\\	"ホームの 白線[はくせん]
\\	より
\\	内側[うちがわ]でお 待[ま]ち 下[くだ]さい。
\\	"ホームの 白線[はくせん]
\\	内側[うちがわ]でお 待[ま]ち 下[くだ]さい。
\\	より
\\	"どうしても 電話[でんわ]が 通[つう]じない。こうなっては、 中谷[なかたに 
\\	君[くん]の 家[うち]へ 行[い]く
\\	より
\\	ほかない。
\\	"どうしても 電話[でんわ]が 通[つう]じない。こうなっては、 中谷[なかたに 
\\	君[くん]の 家[うち]へ 行[い]く
\\	ほかない。
\\	よりほか 
\\	より
\\	"ケーブルカーが 故障[こしょう]で 動[うご]かない。 直[なお]るまで 待[ま]つ
\\	より
\\	ほかないね。
\\	"ケーブルカーが 故障[こしょう]で 動[うご]かない。 直[なお]るまで 待[ま]つ
\\	ほかないね。
\\	よりほか 
\\	より
\\	"新幹線[しんかんせん]ひかり 一号[いちごう] 博多[はかた] 行[ゆ]きは、 十六[じゅうろく] 番[ばん] 線[せん]
\\	より
\\	発車[はっしゃ]します。
\\	"新幹線[しんかんせん]ひかり 一号[いちごう] 博多[はかた] 行[ゆ]きは、 十六[じゅうろく] 番[ばん] 線[せん]
\\	発車[はっしゃ]します。
\\	から 
\\	より
\\	"今日[きょう]の 会議[かいぎ]は 一時[いちじ]
\\	より
\\	始[はじ]まります。
\\	"今日[きょう]の 会議[かいぎ]は 一時[いちじ]
\\	始[はじ]まります。
\\	から 
\\	より
\\	どこ、誰、何、
\\	"どこ
\\	より
\\	我[わ]が 家[や]が 一番[いちばん]だね。
\\	"どこ
\\	我[わ]が 家[や]が 一番[いちばん]だね。
\\	より
\\	どこ、誰、何、
\\	"まあ、 取[と]り 立[た]てのナシ、 何[なに]
\\	より
\\	のものをありがとうございます。
\\	"まあ、 取[と]り 立[た]てのナシ、 何[なに]
\\	のものをありがとうございます。
\\	より
\\	どこ、誰、何、
\\	"君[きみ]は 誰[だれ]
\\	より
\\	(も) 早[はや]く 走[はし]ったね。
\\	"君[きみ]は 誰[だれ]
\\	(も) 早[はや]く 走[はし]ったね。
\\	より
\\	"ピアノ・コンクールの 優勝者[ゆうしょうしゃ]は 明日[あした]の 最終[さいしゅう] 審査[しんさ]の 結果[けっか]に
\\	より
\\	決[き]めます。
\\	"ピアノ・コンクールの 優勝者[ゆうしょうしゃ]は 明日[あした]の 最終[さいしゅう] 審査[しんさ]の 結果[けっか]に
\\	決[き]めます。
\\	より
\\	"今回[こんかい]の 人口[じんこう] 調査[ちょうさ]に
\\	より
\\	、 日本[にほん]の 人口[じんこう]は 約[やく] 一億[いちおく] 二千万[にせんまん] 人[にん]と 分[わ]かった。
\\	"今回[こんかい]の 人口[じんこう] 調査[ちょうさ]に
\\	、 日本[にほん]の 人口[じんこう]は 約[やく] 一億[いちおく] 二千万[にせんまん] 人[にん]と 分[わ]かった。
\\	より
\\	"斎藤[さいとう] 君[くん]は、ビジネスマンという
\\	より
\\	科学者[かがくしゃ]だ。
\\	"斎藤[さいとう] 君[くん]は、ビジネスマンという
\\	科学者[かがくしゃ]だ。
\\	というより 
\\	より
\\	"大[おお]きいなあ!これは 湖[みずうみ]という
\\	より
\\	海[うみ]だ。
\\	"大[おお]きいなあ!これは 湖[みずうみ]という
\\	海[うみ]だ。
\\	というより 
\\	より
\\	くらい(ぐらい)	"この 仕事[しごと]はあと 一週間[いっしゅうかん]
\\	ほど
\\	あれば 出来[でき] 上[あ]がります。
\\	"この 仕事[しごと]はあと 一週間[いっしゅうかん]
\\	あれば 出来[でき] 上[あ]がります。
\\	ほど
\\	くらい(ぐらい)	"あと 一カ月[いっかげつ]
\\	ほど
\\	で 夏休[なつやす]みになる。
\\	"あと 一カ月[いっかげつ]
\\	で 夏休[なつやす]みになる。
\\	ほど
\\	"彼[かれ]は、 彼女[かのじょ]にふられて、みているのが 気[き]の 毒[どく]な
\\	ほど
\\	しょげている。
\\	"彼[かれ]は、 彼女[かのじょ]にふられて、みているのが 気[き]の 毒[どく]な
\\	しょげている。
\\	ほど
\\	"私[わたし]はコンクールは 優勝[ゆうしょう]して、 涙[なみだ]がでる
\\	ほど
\\	嬉[うれ]しかった。
\\	"私[わたし]はコンクールは 優勝[ゆうしょう]して、 涙[なみだ]がでる
\\	嬉[うれ]しかった。
\\	ほど
\\	"今年[ことし]の 梅雨[つゆ]は、いつも
\\	ほど
\\	雨[あめ]が 多[おお]くない。
\\	"今年[ことし]の 梅雨[つゆ]は、いつも
\\	雨[あめ]が 多[おお]くない。
\\	ほど
\\	"夏休[なつやす]み
\\	ほど
\\	待[ま]ち 遠[どお]しいものはない。
\\	"夏休[なつやす]み
\\	待[ま]ち 遠[どお]しいものはない。
\\	ほど
\\	"なんでも、 練習[れんしゅう]すればする
\\	ほど
\\	上手[じょうず]になる。
\\	"なんでも、 練習[れんしゅう]すればする
\\	上手[じょうず]になる。
\\	~するほど、~ば~ほど	
\\	ほど
\\	"骨董[こっとう] 品[ひん]は、 古[ふる]ければ 古[ふる]い
\\	ほど
\\	価値[かち]がある。
\\	"骨董[こっとう] 品[ひん]は、 古[ふる]ければ 古[ふる]い
\\	価値[かち]がある。
\\	~するほど、~ば~ほど	
\\	ほど
\\	"日本[にほん]に 行[い]くのは、いつになる
\\	か
\\	全[まった]くわからない。
\\	"日本[にほん]に 行[い]くのは、いつになる
\\	全[まった]くわからない。
\\	か
\\	"この 河[かわ]は、 見[み]ただけでは、 海[うみ]なんだ
\\	か
\\	河[かわ]なんだ
\\	か
\\	分[わ]からないくらい 大[おお]きいね。
\\	"この 河[かわ]は、 見[み]ただけでは、 海[うみ]なんだ
\\	河[かわ]なんだ
\\	分[わ]からないくらい 大[おお]きいね。
\\	か
\\	"もうそんな 時間[じかん]か。 外[そと]が 明[あか]るいせい
\\	か
\\	気[き]が 付[つ]かなかった。
\\	"もうそんな 時間[じかん]か。 外[そと]が 明[あか]るいせい
\\	気[き]が 付[つ]かなかった。
\\	か
\\	"昨夜[ゆうべ]は 飲[の]みすぎたの
\\	か
\\	、 今朝[けさ]は 頭[あたま]が 痛[いた]い。
\\	"昨夜[ゆうべ]は 飲[の]みすぎたの
\\	、 今朝[けさ]は 頭[あたま]が 痛[いた]い。
\\	か
\\	"右[みぎ]の
\\	か
\\	左[ひだり]の
\\	か
\\	、 早[はや]く 決[き]めてよ。
\\	"右[みぎ]の
\\	左[ひだり]の
\\	、 早[はや]く 決[き]めてよ。
\\	~か~か 
\\	か
\\	"行[い]くの
\\	か
\\	行[い]かないの
\\	か
\\	、どっちなの?
\\	"行[い]くの
\\	行[い]かないの
\\	、どっちなの?
\\	~か~か 
\\	か
\\	"二度[にど]
\\	か
\\	三度[さんど] 君[きみ]のうちに 電話[でんわ]したけど、 留守[るす]だった。
\\	"二度[にど]
\\	三度[さんど] 君[きみ]のうちに 電話[でんわ]したけど、 留守[るす]だった。
\\	~か~	
\\	か
\\	"私[わたし] 達[たち]は 三日[みっか]
\\	か
\\	四日[よっか] 旅行[りょこう]に 出[で]かけます。
\\	"私[わたし] 達[たち]は 三日[みっか]
\\	四日[よっか] 旅行[りょこう]に 出[で]かけます。
\\	~か~	
\\	か
\\	"そのチャンデー、 一[ひと]つ
\\	か
\\	二[ふた]つ 私[わたし]にも 頂戴[ちょうだい]。
\\	"そのチャンデー、 一[ひと]つ
\\	二[ふた]つ 私[わたし]にも 頂戴[ちょうだい]。
\\	~か~	
\\	か
\\	"昨日[きのう]はどうして 来[こ]なかったのです
\\	か
\\	"昨日[きのう]はどうして 来[こ]なかったのです
\\	か
\\	"戦争[せんそう]を 避[さ]けるのは 不可能[ふかのう]なのだろう
\\	か
\\	"戦争[せんそう]を 避[さ]けるのは 不可能[ふかのう]なのだろう
\\	か
\\	"どちら 様[さま]でいっらっしゃいます
\\	か
\\	"どちら 様[さま]でいっらっしゃいます
\\	か
\\	"分[わ]かった
\\	か
\\	。もう 二度[にど]とするんじゃないよ。
\\	"分[わ]かった
\\	。もう 二度[にど]とするんじゃないよ。
\\	か
\\	"本当[ほんとう]に 明日[あした] 来[き]てくれます
\\	か
\\	"本当[ほんとう]に 明日[あした] 来[き]てくれます
\\	か
\\	"どうしてそんなことが 信[しん]じられるだろう
\\	か
\\	"どうしてそんなことが 信[しん]じられるだろう
\\	か
\\	"こんなことになるとは、 誰[だれ]が 想像[そうぞう]しただろう
\\	か
\\	"こんなことになるとは、 誰[だれ]が 想像[そうぞう]しただろう
\\	か
\\	"こんな 夜中[よなか]に 人[ひと]が 訪[たず]ねてくることがあろう
\\	か
\\	"こんな 夜中[よなか]に 人[ひと]が 訪[たず]ねてくることがあろう
\\	か
\\	"そんなことってあるんでしょう
\\	か
\\	"そんなことってあるんでしょう
\\	か
\\	"本当[ほんとう]に 大丈夫[だいじょうぶ]でしょう
\\	か
\\	"本当[ほんとう]に 大丈夫[だいじょうぶ]でしょう
\\	か
\\	"こんな 時間[じかん]まで 遊[あそ]んでくる 人[ひと]があります
\\	か
\\	"こんな 時間[じかん]まで 遊[あそ]んでくる 人[ひと]があります
\\	か
\\	"あんなに 頼[たの]んだのに、この 手紙[たがみ] 出[だ]すのを 忘[わす]れたの
\\	か
\\	"あんなに 頼[たの]んだのに、この 手紙[たがみ] 出[だ]すのを 忘[わす]れたの
\\	か
\\	"明[あか]るいうちに 帰[かえ]っておいでといってではない
\\	か
\\	"明[あか]るいうちに 帰[かえ]っておいでといってではない
\\	~ではないか/じゃないか 
\\	か
\\	"逆立[さかだ]ち、 昨日[きのう]はできたじゃない
\\	か
\\	"逆立[さかだ]ち、 昨日[きのう]はできたじゃない
\\	~ではないか/じゃないか 
\\	か
\\	"帰[かえ]りに 一杯[いっぱい]やろう
\\	か
\\	"帰[かえ]りに 一杯[いっぱい]やろう
\\	か
\\	"もう 一度[いちど] 確[たし]かめてみようじゃない
\\	か
\\	"もう 一度[いちど] 確[たし]かめてみようじゃない
\\	か
\\	"そんなに 行[い]きたいのなら、 行[い]ってみたらどう
\\	か
\\	"そんなに 行[い]きたいのなら、 行[い]ってみたらどう
\\	~たらどうか 
\\	か
\\	"独[ひと]りで 心配[しんぱい]しているより、 色[いろ]んな 人[ひと]に 聞[き]いてみたらどう
\\	か
\\	"独[ひと]りで 心配[しんぱい]しているより、 色[いろ]んな 人[ひと]に 聞[き]いてみたらどう
\\	~たらどうか 
\\	か
\\	"そう
\\	か
\\	。とうとう 合格[ごうかく]した
\\	か
\\	。よくやったな。
\\	"そう
\\	。とうとう 合格[ごうかく]した
\\	。よくやったな。
\\	か
\\	"なあんだ。 影[かげ]
\\	か
\\	。 幽霊[ゆうれい]かと 思[おも]った。
\\	"なあんだ。 影[かげ]
\\	。 幽霊[ゆうれい]かと 思[おも]った。
\\	か
\\	ね 
\\	な 
\\	"お嬢[じょう]さんのピアノのお 上手[じょうず]なこと! 奥様[おくさま]もピアニストでいらっしゃいます
\\	もの
\\	ね。
\\	"お嬢[じょう]さんのピアノのお 上手[じょうず]なこと! 奥様[おくさま]もピアニストでいらっしゃいます
\\	ね。
\\	もの
\\	ね 
\\	な 
\\	"この 浜辺[はまべ] 懐[なつ]かしいね。 新婚[しんこん] 旅行[りょこう]のときに 来[き]たところだ
\\	もの
\\	ね。
\\	"この 浜辺[はまべ] 懐[なつ]かしいね。 新婚[しんこん] 旅行[りょこう]のときに 来[き]たところだ
\\	ね。
\\	もの
\\	"ねえ、お 母[かあ]さん、あのおもちゃ 買[か]ってよ。 僕[ぼく]どうしても 欲[ほ]しいんだ
\\	もの
\\	"ねえ、お 母[かあ]さん、あのおもちゃ 買[か]ってよ。 僕[ぼく]どうしても 欲[ほ]しいんだ
\\	もの
\\	"今晩[こんばん] 来[き]てくださいませんか。 私[わたし]どうしてもお 会[あ]いしたいんです
\\	もの
\\	"今晩[こんばん] 来[き]てくださいませんか。 私[わたし]どうしてもお 会[あ]いしたいんです
\\	もの
\\	"私[わたし] 成人[せいじん]になったんだ
\\	もの
\\	、 少[すこ]しぐらいお 酒[さけ] 飲[の]んでもいいでしょう?
\\	"私[わたし] 成人[せいじん]になったんだ
\\	、 少[すこ]しぐらいお 酒[さけ] 飲[の]んでもいいでしょう?
\\	もの
\\	"息子[むすこ]は 英語[えいご]がまだよく 分[わ]かりません
\\	もの
\\	、 一人[ひとり]で 買[か]い 物[もの]に 行[い]くのは 無理[むり]ですわ。
\\	"息子[むすこ]は 英語[えいご]がまだよく 分[わ]かりません
\\	、 一人[ひとり]で 買[か]い 物[もの]に 行[い]くのは 無理[むり]ですわ。
\\	もの
\\	"彼[かれ]はポテトチップスを 食[た]べ
\\	ながら
\\	テレビを 見[み]ている。
\\	"彼[かれ]はポテトチップスを 食[た]べ
\\	テレビを 見[み]ている。
\\	ながら
\\	"僕[ぼく]は 音楽[おんがく]を 聞[き]き
\\	ながら
\\	ドライブをした。
\\	"僕[ぼく]は 音楽[おんがく]を 聞[き]き
\\	ドライブをした。
\\	ながら
\\	"彼[かれ]は 大会社[だいがいしゃ]の 社長[しゃちょう]
\\	ながら
\\	、いつも 腰[こし]の 低[ひく]い 人[ひと]だ。
\\	"彼[かれ]は 大会社[だいがいしゃ]の 社長[しゃちょう]
\\	、いつも 腰[こし]の 低[ひく]い 人[ひと]だ。
\\	ながら
\\	"彼女[かのじょ]はあんなに 色々[いろいろ] 苦労[くろう]し
\\	ながら
\\	、それを 苦[く]にしていない。
\\	"彼女[かのじょ]はあんなに 色々[いろいろ] 苦労[くろう]し
\\	、それを 苦[く]にしていない。
\\	ながら
\\	と.	"お 父さ[とう]んがね、 私[わたし]の 作[つく]ったお 料理[りょうり]とっても 美味[おい]しかった 
\\	って
\\	"お 父さ[とう]んがね、 私[わたし]の 作[つく]ったお 料理[りょうり]とっても 美味[おい]しかった 
\\	って
\\	と.	"彼[かれ]ね、 今晩[こんばん] 来[き]てくれないか
\\	って
\\	言[い]うのよ。
\\	"彼[かれ]ね、 今晩[こんばん] 来[き]てくれないか
\\	言[い]うのよ。
\\	って
\\	という, 
\\	"相撲[すもう]
\\	って
\\	スポーツは、なかなか 面白[おもしろ]いね。
\\	"相撲[すもう]
\\	スポーツは、なかなか 面白[おもしろ]いね。
\\	って
\\	という, 
\\	"ニューヨーク
\\	って
\\	街[まち]は、 汚[きたな]いけど 活気[かっき]があるね。
\\	"ニューヨーク
\\	街[まち]は、 汚[きたな]いけど 活気[かっき]があるね。
\\	って
\\	"狼[おおかみ]
\\	って
\\	恐[こわ]いんだね、 犬[いぬ]によく 似[に]てるけど。
\\	"狼[おおかみ]
\\	恐[こわ]いんだね、 犬[いぬ]によく 似[に]てるけど。
\\	って
\\	"明日[あした] 大阪[おおさか]へ 行[い]く
\\	って
\\	本当[ほんとう]?
\\	"明日[あした] 大阪[おおさか]へ 行[い]く
\\	本当[ほんとう]?
\\	って
\\	"いつ 温泉[おんせん]に 行[い]こうか
\\	って
\\	忙[いそが]しいから 当分[とうぶん] 駄目[だめ]だね。
\\	"いつ 温泉[おんせん]に 行[い]こうか
\\	忙[いそが]しいから 当分[とうぶん] 駄目[だめ]だね。
\\	って
\\	"もう 出[で]かけよう
\\	って
\\	君[きみ] 早[はや]いね、 支度[したく]するの。
\\	"もう 出[で]かけよう
\\	君[きみ] 早[はや]いね、 支度[したく]するの。
\\	って
\\	"彼女[かのじょ]を 見[み]かけた
\\	って
\\	いつ、どこで?
\\	"彼女[かのじょ]を 見[み]かけた
\\	いつ、どこで?
\\	って
\\	"医学部[いがくぶ]に 合格[ごうかく]なさったんです
\\	って
\\	おめでとうございます。
\\	"医学部[いがくぶ]に 合格[ごうかく]なさったんです
\\	おめでとうございます。
\\	って
\\	"渡辺[わたなべ]さんちね、 大[おお]きなカラーテレビ 買[か]ったんだ
\\	って
\\	"渡辺[わたなべ]さんちね、 大[おお]きなカラーテレビ 買[か]ったんだ
\\	って
\\	"課長[かちょう]の 話[はなし]では、 今度[こんど]のボーナスいつもより 少[すく]ないんだ
\\	って
\\	"課長[かちょう]の 話[はなし]では、 今度[こんど]のボーナスいつもより 少[すく]ないんだ
\\	って
\\	"お兄[にい]ちゃん 明日[あした]テストがあるから、 僕[ぼく]と 遊[あそ]ぶ 時間[じかん]ないんだ
\\	って
\\	"お兄[にい]ちゃん 明日[あした]テストがあるから、 僕[ぼく]と 遊[あそ]ぶ 時間[じかん]ないんだ
\\	だって	
\\	って
\\	"コンクールね、 早川[はやかわ]さんが 一等[いっとう]だ
\\	って
\\	さ。
\\	"コンクールね、 早川[はやかわ]さんが 一等[いっとう]だ
\\	さ。
\\	だって	
\\	って
\\	とて; 
\\	"一生懸命[いっしょうけんめい] 勉強[べんきょう]した
\\	って
\\	、 飯田[いいだ] 君[くん]には 敵[かな]わない。
\\	"一生懸命[いっしょうけんめい] 勉強[べんきょう]した
\\	、 飯田[いいだ] 君[くん]には 敵[かな]わない。
\\	って
\\	とて; 
\\	"今更[いまさら] 嘆[なげ]いた
\\	って
\\	はじまらない。
\\	"今更[いまさら] 嘆[なげ]いた
\\	はじまらない。
\\	って
\\	"ブラックホール
\\	とは
\\	、 一体[いったい]どんなもののだろう。
\\	"ブラックホール
\\	、 一体[いったい]どんなもののだろう。
\\	とは
\\	"政治[せいじ]
\\	とは
\\	、 政治家[せいじか]にだけ 任[まか]せておいてよいものであろうか。
\\	"政治[せいじ]
\\	、 政治家[せいじか]にだけ 任[まか]せておいてよいものであろうか。
\\	とは
\\	と, 
\\	"私[わたし]、 高見[たかみ] 君[くん]
\\	とは
\\	もう 付[つ]き 合[あ]うのやめたわ。
\\	"私[わたし]、 高見[たかみ] 君[くん]
\\	もう 付[つ]き 合[あ]うのやめたわ。
\\	とは
\\	と, 
\\	"小川[おがわ]さん
\\	とは
\\	、 小学校[しょうがっこう]のときからのお 友達[ともだち]ですの。
\\	"小川[おがわ]さん
\\	、 小学校[しょうがっこう]のときからのお 友達[ともだち]ですの。
\\	とは
\\	"竹内[たけうち] 君[くん]、あんなに 一生懸命[いっしょうけんめい] 勉強[べんきょう]したのに 不合格[ふごうかく]だった
\\	とは
\\	"竹内[たけうち] 君[くん]、あんなに 一生懸命[いっしょうけんめい] 勉強[べんきょう]したのに 不合格[ふごうかく]だった
\\	とは
\\	"もう 今年[ことし]も 十二月[じゅうにがつ]
\\	とは
\\	。 月日[つきひ]のたつのは 何[なん]と 早[はや]いのだろう。
\\	"もう 今年[ことし]も 十二月[じゅうにがつ]
\\	。 月日[つきひ]のたつのは 何[なん]と 早[はや]いのだろう。
\\	とは
\\	"みよちゃんは、ピアノ
\\	とか
\\	絵[え]
\\	とか
\\	お 習字[しゅうじ]
\\	とか
\\	沢山[たくさん]お 稽古[けいこ]にいっているのよ。
\\	"みよちゃんは、ピアノ
\\	絵[え]
\\	お 習字[しゅうじ]
\\	沢山[たくさん]お 稽古[けいこ]にいっているのよ。
\\	とか
\\	"テレビはばかり 見[み]ていないで、 少[すこ]しは 弟[おとうと]の 面倒[めんどう]を 見[み]る
\\	とか
\\	、 犬[いぬ]を 散歩[さんぽ]に 連[つ]れていく
\\	とか
\\	、しなさいよ。
\\	"テレビはばかり 見[み]ていないで、 少[すこ]しは 弟[おとうと]の 面倒[めんどう]を 見[み]る
\\	、 犬[いぬ]を 散歩[さんぽ]に 連[つ]れていく
\\	、しなさいよ。
\\	とか
\\	"あの 家[いえ]は、 一億[いちおく] 円[えん]
\\	とか
\\	で 売[う]れたそうですよ。
\\	"あの 家[いえ]は、 一億[いちおく] 円[えん]
\\	で 売[う]れたそうですよ。
\\	とか
\\	"喬[たかし] 君[くん]のおじいさんは、 百[ひゃく]いくつ
\\	とか
\\	で、まだ 英語[えいご]を 教[おし]えているんですって。
\\	"喬[たかし] 君[くん]のおじいさんは、 百[ひゃく]いくつ
\\	で、まだ 英語[えいご]を 教[おし]えているんですって。
\\	とか
\\	"お 姉[ねえ]ちゃんっ
\\	たら
\\	、まだ 寝[ね]ているよ。
\\	"お 姉[ねえ]ちゃんっ
\\	、まだ 寝[ね]ているよ。
\\	ったら	
\\	たら
\\	"彼[かれ]っ
\\	たら
\\	、もう 新[あたら]しいガールフレンドに 夢中[むちゅう]なんだから。
\\	"彼[かれ]っ
\\	、もう 新[あたら]しいガールフレンドに 夢中[むちゅう]なんだから。
\\	ったら	
\\	たら
\\	"三時[さんじ] 間[かん]しか 寝[ね]てないんだ。 眠[ねむ]いっ
\\	たら
\\	ないよ。
\\	"三時[さんじ] 間[かん]しか 寝[ね]てないんだ。 眠[ねむ]いっ
\\	ないよ。
\\	ったら~ない	
\\	たら
\\	"昨夜[ゆうべ]の 雷[かみなり]っ
\\	たら
\\	今[いま]までになく、ひどかったのよ。
\\	"昨夜[ゆうべ]の 雷[かみなり]っ
\\	今[いま]までになく、ひどかったのよ。
\\	ったら~ない	
\\	たら
\\	"お 兄[にい]ちゃんっ
\\	たら
\\	、 早[はや]く 学校[がっこう]へ 行[い]こうよ。
\\	"お 兄[にい]ちゃんっ
\\	、 早[はや]く 学校[がっこう]へ 行[い]こうよ。
\\	たら
\\	"面白[おもろ]いものがあるよ。 早[はや]く 来[き]てよ。お 父[とう]さんっ
\\	たら
\\	"面白[おもろ]いものがあるよ。 早[はや]く 来[き]てよ。お 父[とう]さんっ
\\	たら
\\	"みんな 待[ま]ってるのよ。 早[はや]く 帰[かえ]って 来[き]なさいっ
\\	たら
\\	"みんな 待[ま]ってるのよ。 早[はや]く 帰[かえ]って 来[き]なさいっ
\\	たら
\\	"まだ 二時[にじ]よ。おやつはもっと 後[あと]でっ
\\	たら
\\	"まだ 二時[にじ]よ。おやつはもっと 後[あと]でっ
\\	たら
\\	"やめてっ
\\	たら
\\	、そんなこと。
\\	"やめてっ
\\	、そんなこと。
\\	たら
\\	"今度[こんど]は 僕[ぼく]にやらせてっ
\\	たら
\\	"今度[こんど]は 僕[ぼく]にやらせてっ
\\	たら
\\	"あら、もう 三杯[さんばい]も 食[た]べた
\\	のに
\\	、またおかわり?
\\	"あら、もう 三杯[さんばい]も 食[た]べた
\\	、またおかわり?
\\	というのに 
\\	いいのに	
\\	のに
\\	"私[わたし]の 帽子[ぼうし]、 確[たし]かにここに 置[お]いた
\\	のに
\\	、ないわ。
\\	"私[わたし]の 帽子[ぼうし]、 確[たし]かにここに 置[お]いた
\\	、ないわ。
\\	というのに 
\\	いいのに	
\\	のに
\\	"夏休[なつやす]みだという
\\	のに
\\	、 宿題[しゅくだい]が 沢山[たくさん]あって 遊[あそ]びに 行[い]けない。
\\	"夏休[なつやす]みだという
\\	、 宿題[しゅくだい]が 沢山[たくさん]あって 遊[あそ]びに 行[い]けない。
\\	というのに 
\\	いいのに	
\\	のに
\\	"君[きみ]が 来[く]ると 分[わ]かっていれば、 僕[ぼく]、うちにいた
\\	のに
\\	"君[きみ]が 来[く]ると 分[わ]かっていれば、 僕[ぼく]、うちにいた
\\	のに
\\	"もう 百[ひゃく] 円[えん]あれば、この 本[ほん]が 買[か]える
\\	のに
\\	なあ。
\\	"もう 百[ひゃく] 円[えん]あれば、この 本[ほん]が 買[か]える
\\	なあ。
\\	のに
\\	"食[た]べないの?せっかく 作[つく]った
\\	のに
\\	"食[た]べないの?せっかく 作[つく]った
\\	のに
\\	"子供[こども]
\\	だって
\\	それくらい 分[わ]かります。
\\	"子供[こども]
\\	それくらい 分[わ]かります。
\\	だって
\\	"親友[しんゆう] 同士[どうし]
\\	だって
\\	喧嘩[けんか]することまる。
\\	"親友[しんゆう] 同士[どうし]
\\	喧嘩[けんか]することまる。
\\	だって
\\	"夏[なつ]
\\	だって
\\	肌寒[はだざむ]い 日[ひ]もある。
\\	"夏[なつ]
\\	肌寒[はだざむ]い 日[ひ]もある。
\\	だって
\\	"明日[あした]
\\	だって
\\	、 明後日[あさって]
\\	だって
\\	、 構[かま]いませんよ。
\\	"明日[あした]
\\	、 明後日[あさって]
\\	、 構[かま]いませんよ。
\\	~だって~だって 
\\	~たって~たって 
\\	だって
\\	"大阪[おおさか]で 買[か]っ
\\	たって
\\	、 東京[とうきょう]で 買[か]っ
\\	たって
\\	、 値段[ねだん] 同[おな]じだよ。
\\	"大阪[おおさか]で 買[か]っ
\\	、 東京[とうきょう]で 買[か]っ
\\	、 値段[ねだん] 同[おな]じだよ。
\\	~だって~だって 
\\	~たって~たって 
\\	だって
\\	"いつ
\\	だって
\\	見[み]られますよ。
\\	"いつ
\\	見[み]られますよ。
\\	だって
\\	"そんな 物[もの]はどこに
\\	だって
\\	あるさ。
\\	"そんな 物[もの]はどこに
\\	あるさ。
\\	だって
\\	"誰[だれ]
\\	だって
\\	それくらい 分[わ]かるはずです。
\\	"誰[だれ]
\\	それくらい 分[わ]かるはずです。
\\	だって
\\	"誰[だれ]
\\	だって
\\	貧乏[びんぼう]はいやだ。
\\	"誰[だれ]
\\	貧乏[びんぼう]はいやだ。
\\	だって
\\	"どんな 所[とろこ]
\\	だって
\\	行[い]きたくない。
\\	"どんな 所[とろこ]
\\	行[い]きたくない。
\\	だって
\\	"そんなことは、いつ
\\	だって
\\	したくない。
\\	"そんなことは、いつ
\\	したくない。
\\	だって
\\	"僕[ぼく]はお 金[かね]なんか 借[か]りていませんよ。
\\	だって
\\	。 忘[わす]れたのかしら。
\\	"僕[ぼく]はお 金[かね]なんか 借[か]りていませんよ。
\\	。 忘[わす]れたのかしら。
\\	だって
\\	"今夜[こんや]も 彼[かれ]とデートなの
\\	だって
\\	。よく 飽[あ]きないわね。あの 二人[ふたり]。
\\	"今夜[こんや]も 彼[かれ]とデートなの
\\	。よく 飽[あ]きないわね。あの 二人[ふたり]。
\\	だって
\\	"彼女[かのじょ]がいい 先生[せんせい]であるこは、 学生[がくせい] 達[たち]に 評判[ひょうばん]がいい
\\	ので
\\	分[わ]かる。
\\	"彼女[かのじょ]がいい 先生[せんせい]であるこは、 学生[がくせい] 達[たち]に 評判[ひょうばん]がいい
\\	分[わ]かる。
\\	の 
\\	で.	
\\	ので
\\	"この 町[まち]の 空気[くうき]がよくないことは、スモッグの 日[ひ]が 多[おお]い
\\	ので
\\	分[わ]かる。
\\	"この 町[まち]の 空気[くうき]がよくないことは、スモッグの 日[ひ]が 多[おお]い
\\	分[わ]かる。
\\	の 
\\	で.	
\\	ので
\\	"そんなにつまでもしょうげているなんて、あなたらしいくない
\\	こと
\\	よ。
\\	"そんなにつまでもしょうげているなんて、あなたらしいくない
\\	よ。
\\	こと
\\	"これだけ 言[い]ってもまだ 分[わ]からないのなら、もう 知[し]らない
\\	こと
\\	よ。
\\	"これだけ 言[い]ってもまだ 分[わ]からないのなら、もう 知[し]らない
\\	よ。
\\	こと
\\	"おや、 珍[めずら]しい 鳥[とり]がいる
\\	こと
\\	"おや、 珍[めずら]しい 鳥[とり]がいる
\\	こと
\\	"まあ、 素敵[すてき]な 絵[え]だ
\\	こと
\\	ねえ。
\\	"まあ、 素敵[すてき]な 絵[え]だ
\\	ねえ。
\\	こと
\\	"これからおじゃまてもいい
\\	こと
\\	"これからおじゃまてもいい
\\	こと
\\	"風邪[かぜ]はもうすっかりよくなった
\\	こと
\\	"風邪[かぜ]はもうすっかりよくなった
\\	こと
\\	"新婚[しんこん] 旅行[りょこう]はヨーロッパですって? 素敵[すてき]だ
\\	こと
\\	"新婚[しんこん] 旅行[りょこう]はヨーロッパですって? 素敵[すてき]だ
\\	こと
\\	"彼女[かのじょ]は 会社[かいしゃ]、 年休[ねんきゅう] 五週間[ごしゅうかん]ですって。 羨[うやら]ましい
\\	こと
\\	"彼女[かのじょ]は 会社[かいしゃ]、 年休[ねんきゅう] 五週間[ごしゅうかん]ですって。 羨[うやら]ましい
\\	こと
\\	"これから 映画[えいが]にでも 行[い]きません
\\	こと
\\	"これから 映画[えいが]にでも 行[い]きません
\\	こと
\\	"遅[おそ]くなったから、タクシーでも 拾[ひろ]わない
\\	こと
\\	"遅[おそ]くなったから、タクシーでも 拾[ひろ]わない
\\	こと
\\	"期日[きじつ]に 送[おく]れずレポートを 提出[ていしゅつ]する
\\	こと
\\	"期日[きじつ]に 送[おく]れずレポートを 提出[ていしゅつ]する
\\	こと
\\	"借[か]りた 金[かね]は 必[かなら]ず 返[かえ]す
\\	こと
\\	"借[か]りた 金[かね]は 必[かなら]ず 返[かえ]す
\\	こと
\\	"今[いま] 出[で]かけれ
\\	ば
\\	間[ま]に 合[あ]いますよ。
\\	"今[いま] 出[で]かけれ
\\	間[ま]に 合[あ]いますよ。
\\	ば
\\	"こんなペンでもよけれ
\\	ば
\\	そうぞ。
\\	"こんなペンでもよけれ
\\	そうぞ。
\\	ば
\\	"雨[あめ]さえ 降[ふ]れ
\\	ば
\\	豊作[ほうさく]じゃ、 間違[まちが]いないのだが。
\\	"雨[あめ]さえ 降[ふ]れ
\\	豊作[ほうさく]じゃ、 間違[まちが]いないのだが。
\\	ば
\\	"コオロギが 鳴[な]き 出[だ]せ
\\	ば
\\	秋[あき]も 間近[まぢか]い。
\\	"コオロギが 鳴[な]き 出[だ]せ
\\	秋[あき]も 間近[まぢか]い。
\\	ば
\\	"夕焼[ゆうや]けになれ
\\	ば
\\	翌日[よくじつ]は 天気[てんき]になる。
\\	"夕焼[ゆうや]けになれ
\\	翌日[よくじつ]は 天気[てんき]になる。
\\	ば
\\	"お 寿司[すし]なら
\\	ば
\\	あの 店[みせ]が 美味[おい]しいですよ。
\\	"お 寿司[すし]なら
\\	あの 店[みせ]が 美味[おい]しいですよ。
\\	~といえは、ならば	
\\	ば
\\	"噂[うわさ]によれ
\\	ば
\\	、 彼女[かのじょ]は 恋人[こいびと]と 別[わか]れたらしい。
\\	"噂[うわさ]によれ
\\	、 彼女[かのじょ]は 恋人[こいびと]と 別[わか]れたらしい。
\\	~といえは、ならば	
\\	ば
\\	"ミステリー 映画[えいが]と 言[い]え
\\	ば
\\	、やっぱりヒッチコックのが 一番[いちばん]だね。
\\	"ミステリー 映画[えいが]と 言[い]え
\\	、やっぱりヒッチコックのが 一番[いちばん]だね。
\\	~といえは、ならば	
\\	ば
\\	~すると、~したところが	"前[まえ]の 仕事[しごと]を 思[おも]え
\\	ば
\\	、 今[いま]の 仕事[しごと]は 楽[らく]なものです。
\\	"前[まえ]の 仕事[しごと]を 思[おも]え
\\	、 今[いま]の 仕事[しごと]は 楽[らく]なものです。
\\	ば
\\	~すると、~したところが	"世論[よろん] 調査[ちょうさ]からみれ
\\	ば
\\	、その 政治[せいじ]の 人気[にんじ]はかなり 落[お]ちたようだ。
\\	"世論[よろん] 調査[ちょうさ]からみれ
\\	、その 政治[せいじ]の 人気[にんじ]はかなり 落[お]ちたようだ。
\\	ば
\\	"女[おんな]いれ
\\	ば
\\	男[おとこ]もいる。
\\	"女[おんな]いれ
\\	男[おとこ]もいる。
\\	~も~ば 
\\	ば
\\	"食[た]べる 物[もの]なけれ
\\	ば
\\	住[す]む 所[ところ]もない。
\\	"食[た]べる 物[もの]なけれ
\\	住[す]む 所[ところ]もない。
\\	~も~ば 
\\	ば
\\	"彼女[かのじょ]はピアノも 弾[ひ]け
\\	ば 絵[え]もかく 多才[たさい]な人[ひと]です。
\\	"彼女[かのじょ]はピアノも 弾[ひ]け
\\	~も~ば 
\\	ば
\\	"東京[とうきょう]には 毎月[まいつき] 出張[しゅっちょう]で 行[い]かなけれ
\\	ば
\\	なりません。
\\	"東京[とうきょう]には 毎月[まいつき] 出張[しゅっちょう]で 行[い]かなけれ
\\	なりません。
\\	ねばらない 
\\	なければならない/なりません.	
\\	ば
\\	"予算[よさん] 内[ない]で 計画[けいかく]を 立[た]てね
\\	ば
\\	ならない。
\\	"予算[よさん] 内[ない]で 計画[けいかく]を 立[た]てね
\\	ならない。
\\	ねばらない 
\\	なければならない/なりません.	
\\	ば
\\	"人[ひと]の 約束[やくそく]ほ 守[まも]らなけれ
\\	ば
\\	ならない。
\\	"人[ひと]の 約束[やくそく]ほ 守[まも]らなけれ
\\	ならない。
\\	ねばらない 
\\	なければならない/なりません.	
\\	ば
\\	"ハロウィーンは、たとえ
\\	ば
\\	日本[にほん]のお 盆[ぼん]のようなものです。
\\	"ハロウィーンは、たとえ
\\	日本[にほん]のお 盆[ぼん]のようなものです。
\\	いわば 
\\	たとえば	
\\	ば
\\	"彼[かれ]は、いわ
\\	ば
\\	生[い]き 字引[じびき]のようなものだ。
\\	"彼[かれ]は、いわ
\\	生[い]き 字引[じびき]のようなものだ。
\\	いわば 
\\	たとえば	
\\	ば
\\	"時間[じかん]があれば、 汽車[きしゃ]で 行[い]くべきだ。なぜなら
\\	ば
\\	その 方[ほう]が 景色[けしき]を 十分[じゅうぶん] 楽[たの]しめるから。
\\	"時間[じかん]があれば、 汽車[きしゃ]で 行[い]くべきだ。なぜなら
\\	その 方[ほう]が 景色[けしき]を 十分[じゅうぶん] 楽[たの]しめるから。
\\	なぜならば	
\\	ば
\\	"人[ひと]に 意地悪[いじわる]をしないほうがいい。なぜなら
\\	ば
\\	いつか 自分[じぶん]に 戻[もど]って 来[く]るからだ。
\\	"人[ひと]に 意地悪[いじわる]をしないほうがいい。なぜなら
\\	いつか 自分[じぶん]に 戻[もど]って 来[く]るからだ。
\\	なぜならば	
\\	ば
\\	"八時[はちじ]の 電車[でんしゃ]に 乗[の]り 遅[おく]れ
\\	ては
\\	、 会議[かいぎ]に 間[ま]に 合[あ]わない。
\\	"八時[はちじ]の 電車[でんしゃ]に 乗[の]り 遅[おく]れ
\\	、 会議[かいぎ]に 間[ま]に 合[あ]わない。
\\	ては
\\	"今[いま]から 始[はじ]め
\\	ては
\\	、 送[おく]れてしまう。
\\	"今[いま]から 始[はじ]め
\\	、 送[おく]れてしまう。
\\	ては
\\	"こんなに 沢山[たくさん]お 土産[みやげ]をいただい
\\	ては
\\	申[もう]し 訳[わけ]ありません。
\\	"こんなに 沢山[たくさん]お 土産[みやげ]をいただい
\\	申[もう]し 訳[わけ]ありません。
\\	ては
\\	"そんなに 頼[たの]まれ
\\	ては
\\	いやとは 言[い]えないね。
\\	"そんなに 頼[たの]まれ
\\	いやとは 言[い]えないね。
\\	ては
\\	"早[はや]まっ
\\	ては
\\	、 損[そん]をするだけだ。
\\	"早[はや]まっ
\\	、 損[そん]をするだけだ。
\\	ては
\\	"赤[あか]ん 坊[ぼう]が 生[う]まれ
\\	ては
\\	、 君[きみ]も 忙[いそが]しくなるね。
\\	"赤[あか]ん 坊[ぼう]が 生[う]まれ
\\	、 君[きみ]も 忙[いそが]しくなるね。
\\	ては
\\	"彼[かれ]は、 小石[こいし]を 拾[ひろ]っ
\\	ては
\\	投[な]げ 拾[ひろ]っ
\\	ては
\\	投[な]げていた。
\\	"彼[かれ]は、 小石[こいし]を 拾[ひろ]っ
\\	投[な]げ 拾[ひろ]っ
\\	投[な]げていた。
\\	ては
\\	"明日[あした]は 遠足[えんそく]なのに、 今夜[こんや]は 雨[あめ]。ともちゃんは、 空[そら]を 見上[みあ]げ
\\	ては
\\	ため 息[いき]をついている。
\\	"明日[あした]は 遠足[えんそく]なのに、 今夜[こんや]は 雨[あめ]。ともちゃんは、 空[そら]を 見上[みあ]げ
\\	ため 息[いき]をついている。
\\	ては
\\	だけ 
\\	のみ.	"その 子犬[こいぬ]はいつも 眠[ねむ]って
\\	ばかり
\\	います。
\\	"その 子犬[こいぬ]はいつも 眠[ねむ]って
\\	います。
\\	ばかり
\\	だけ 
\\	のみ.	"いつまでも 遊[あそ]んで
\\	ばかり
\\	いないで 早[はや]く 職[しょく]でも 見[み]つけなさいよ。
\\	"いつまでも 遊[あそ]んで
\\	いないで 早[はや]く 職[しょく]でも 見[み]つけなさいよ。
\\	ばかり
\\	だけ 
\\	のみ.	"うちに
\\	ばかり
\\	いないで、たまには 外出[がいしゅつ]しよう。
\\	"うちに
\\	いないで、たまには 外出[がいしゅつ]しよう。
\\	ばかり
\\	"そのニュースを 聞[き]いて、 彼[かれ]はただオロオロする
\\	ばかり
\\	でした。
\\	"そのニュースを 聞[き]いて、 彼[かれ]はただオロオロする
\\	でした。
\\	ばかり
\\	"突然[とつぜん]の 停電[ていでん]に 人々[ひとびと]は 右往左往[うおうさおう]する
\\	ばかり
\\	だった。
\\	"突然[とつぜん]の 停電[ていでん]に 人々[ひとびと]は 右往左往[うおうさおう]する
\\	だった。
\\	ばかり
\\	"あの 電車[でんしゃ]に 乗[の]った
\\	ばかり
\\	に、 事故[じこ]に 遭[あ]った。
\\	"あの 電車[でんしゃ]に 乗[の]った
\\	に、 事故[じこ]に 遭[あ]った。
\\	ばかりに	
\\	ばかり
\\	"犯罪[はんざい]の 場[ば]にたまたま 居合[いあ]わせた
\\	ばかり
\\	に、 事件[じけん]に 巻[ま]き 込[こ]まれてしまった。
\\	"犯罪[はんざい]の 場[ば]にたまたま 居合[いあ]わせた
\\	に、 事件[じけん]に 巻[ま]き 込[こ]まれてしまった。
\\	ばかりに	
\\	ばかり
\\	"彼[かれ]は 二[に]メートル
\\	ばかり
\\	もある 大[おお] 男[おとこ]です。
\\	"彼[かれ]は 二[に]メートル
\\	もある 大[おお] 男[おとこ]です。
\\	ほど 
\\	くらい/ぐらい 
\\	ばかり
\\	"旅行[りょこう]の 費用[ひよう]は 全部[ぜんぶ]で 五十万[ごじゅうまん] 円[えん]
\\	ばかり
\\	かかった。
\\	"旅行[りょこう]の 費用[ひよう]は 全部[ぜんぶ]で 五十万[ごじゅうまん] 円[えん]
\\	かかった。
\\	ほど 
\\	くらい/ぐらい 
\\	ばかり
\\	"私[わたし]は 十日[とおか]
\\	ばかり
\\	留守[るす]にします。
\\	"私[わたし]は 十日[とおか]
\\	留守[るす]にします。
\\	ほど 
\\	くらい/ぐらい 
\\	ばかり
\\	"食事[しょくじ]を 作[つく]り 終[お]わって、もう 食[た]べる
\\	ばかり
\\	のところに、 電話[でんわ]がかかってきました。
\\	"食事[しょくじ]を 作[つく]り 終[お]わって、もう 食[た]べる
\\	のところに、 電話[でんわ]がかかってきました。
\\	ばかり
\\	"雨[あめ]が 降[ふ]り 出[だ]さん
\\	ばかり
\\	の 空[そら] 模様[もよう]だ。
\\	"雨[あめ]が 降[ふ]り 出[だ]さん
\\	の 空[そら] 模様[もよう]だ。
\\	ばかり
\\	"私[わたし]は 習[なら]った
\\	ばかり
\\	の 日本語[にほんご]を 使[つか]ってみました。
\\	"私[わたし]は 習[なら]った
\\	の 日本語[にほんご]を 使[つか]ってみました。
\\	~たばかり 
\\	ばかり
\\	"さっき 食[た]べた
\\	ばかり
\\	なのに、またすぐお 腹[なか]が 空[す]いてしまった。
\\	"さっき 食[た]べた
\\	なのに、またすぐお 腹[なか]が 空[す]いてしまった。
\\	~たばかり 
\\	ばかり
\\	"頭[あたま]
\\	ばかり
\\	か 喉[のど]も 痛[いた]い。
\\	"頭[あたま]
\\	か 喉[のど]も 痛[いた]い。
\\	ばかりか	
\\	ばかり
\\	"あの 生徒[せいと]は 教授[きょうじゅ]にゴマをする
\\	ばかり
\\	か、カンニングまでしていい 点[てん]をとろうとするそうだ。
\\	"あの 生徒[せいと]は 教授[きょうじゅ]にゴマをする
\\	か、カンニングまでしていい 点[てん]をとろうとするそうだ。
\\	ばかりか	
\\	ばかり
\\	"男[おとこ]の 子[こ]
\\	ばかり
\\	じゃなく、 女[おんな]の 子[こ]も 大勢[おおぜい]いたよ。
\\	"男[おとこ]の 子[こ]
\\	じゃなく、 女[おんな]の 子[こ]も 大勢[おおぜい]いたよ。
\\	~ばかりで(は)なく~も、~ばかりじゃなく~も	
\\	ばかり
\\	"私[わたし]は、 日本語[にほんご]
\\	ばかり
\\	でなく、 中国語[ちゅうごくご]も 勉強[べんきょう]したい。
\\	"私[わたし]は、 日本語[にほんご]
\\	でなく、 中国語[ちゅうごくご]も 勉強[べんきょう]したい。
\\	~ばかりで(は)なく~も、~ばかりじゃなく~も	
\\	ばかり
\\	"この 手提[てさ]げは、デザインがいい
\\	ばかり
\\	でなく、 値段[ねだん]も 手頃[てごろ]だ。
\\	"この 手提[てさ]げは、デザインがいい
\\	でなく、 値段[ねだん]も 手頃[てごろ]だ。
\\	~ばかりで(は)なく~も、~ばかりじゃなく~も	
\\	ばかり
\\	"喉[のど]が 渇[かわ]いた
\\	ばかり
\\	でなく、お 腹[なか]も 空[す]いた。
\\	"喉[のど]が 渇[かわ]いた
\\	でなく、お 腹[なか]も 空[す]いた。
\\	~ばかりで(は)なく~も、~ばかりじゃなく~も	
\\	ばかり
\\	ばかり, 
\\	"私[わたし]の 財布[さいふ]の 中[なか]には 五百[ごひゃく] 円[えん]
\\	ばかし
\\	しかない。
\\	"私[わたし]の 財布[さいふ]の 中[なか]には 五百[ごひゃく] 円[えん]
\\	しかない。
\\	ばかし 
\\	ばかし
\\	ばかり, 
\\	"これ
\\	ばかし
\\	の 貯金[ちょきん]では、 旅行[りょこう]にも 行[い]けない。
\\	"これ
\\	の 貯金[ちょきん]では、 旅行[りょこう]にも 行[い]けない。
\\	ばかし 
\\	ばかし
\\	ばかり, 
\\	"あの 人[ひと]は、 一日[いちにち] 中[じゅう]お 酒[さけ]
\\	ばかし
\\	飲[の]んでいる。
\\	"あの 人[ひと]は、 一日[いちにち] 中[じゅう]お 酒[さけ]
\\	飲[の]んでいる。
\\	ばかし 
\\	ばかし
\\	ばかり, 
\\	"そんなに 甘[あま]いもの
\\	ばかし
\\	食[た]べていると、 虫歯[むしば]になるよ。
\\	"そんなに 甘[あま]いもの
\\	食[た]べていると、 虫歯[むしば]になるよ。
\\	ばかし 
\\	ばかし
\\	ばかり, 
\\	"日本語[にほんご]のテストがあった
\\	ばかし
\\	なのに、 次[つぎ]はフランス 語[ご]のテストだ。
\\	"日本語[にほんご]のテストがあった
\\	なのに、 次[つぎ]はフランス 語[ご]のテストだ。
\\	ばかし 
\\	ばかし
\\	ばかり, 
\\	"今[いま]いけた
\\	ばかし
\\	の 花[はな]よ。 綺麗[きれい]でしょう。
\\	"今[いま]いけた
\\	の 花[はな]よ。 綺麗[きれい]でしょう。
\\	ばかし 
\\	ばかし
\\	"よく 勉強[べんきょう]した
\\	だけに
\\	、いい 成績[せいせき]で 合格[ごうかく]した。
\\	"よく 勉強[べんきょう]した
\\	、いい 成績[せいせき]で 合格[ごうかく]した。
\\	だけ 
\\	に. 
\\	だけあって	
\\	だけに
\\	"ここは 有名[ゆうめい]な 観光地[かんこうち]
\\	だけに
\\	、 毎日[まいにち] 沢山[たくさん]の 人[ひと]が 訪[おとず]れる。
\\	"ここは 有名[ゆうめい]な 観光地[かんこうち]
\\	、 毎日[まいにち] 沢山[たくさん]の 人[ひと]が 訪[おとず]れる。
\\	だけ 
\\	に. 
\\	だけあって	
\\	だけに
\\	"ここは 赤道[せきどう]に 近[ちか]い
\\	でけに
\\	、すごく 暑[あつ]いね。
\\	"ここは 赤道[せきどう]に 近[ちか]い
\\	、すごく 暑[あつ]いね。
\\	だけ 
\\	に. 
\\	だけあって	
\\	だけに
\\	"彼[かれ]にはもう 合[あ]えないだろうと 諦[あきら]めていた
\\	だけに
\\	、 電話[でんわ]がかかってきた 時[とき]には、とても 嬉[うれ]しかった。
\\	"彼[かれ]にはもう 合[あ]えないだろうと 諦[あきら]めていた
\\	、 電話[でんわ]がかかってきた 時[とき]には、とても 嬉[うれ]しかった。
\\	だけに
\\	"ハリケーンの 被害[ひがい]が 大[おお]きかった
\\	だけに
\\	、その 町[まち]の 復興[ふっこう]ぶりには 目[め]を 見張[ひは]るものがあった。
\\	"ハリケーンの 被害[ひがい]が 大[おお]きかった
\\	、その 町[まち]の 復興[ふっこう]ぶりには 目[め]を 見張[ひは]るものがあった。
\\	だけに
\\	"紙屑[かみくず]
\\	だの
\\	空缶[あきかん]
\\	だの
\\	、 道[みち]に 散[ち]らかっている。
\\	"紙屑[かみくず]
\\	空缶[あきかん]
\\	、 道[みち]に 散[ち]らかっている。
\\	~だの~だの 
\\	だの
\\	"一人[ひとり] 歩[ある]きは 危険[きけん]
\\	だの
\\	、 地下鉄[ちかてつ]は 恐[こわ]い
\\	だの
\\	、ニューヨークに 引[ひ]っ 越[こ]すと 決[き]まったら 友達[ともだち]から 脅[おど]かされた。
\\	"一人[ひとり] 歩[ある]きは 危険[きけん]
\\	、 地下鉄[ちかてつ]は 恐[こわ]い
\\	、ニューヨークに 引[ひ]っ 越[こ]すと 決[き]まったら 友達[ともだち]から 脅[おど]かされた。
\\	~だの~だの 
\\	だの
\\	"先生[せんせい]
\\	でも
\\	分[わ]からないでしょう。
\\	"先生[せんせい]
\\	分[わ]からないでしょう。
\\	でも
\\	"ちょっと
\\	でも
\\	見落[みと]としがあっては 大変[たいへん]だ。
\\	"ちょっと
\\	見落[みと]としがあっては 大変[たいへん]だ。
\\	でも
\\	"あの先生[せんせい]はどんな 難[むずか]しい 問題[もんだい]
\\	でも
\\	解[と]ける。
\\	"あの先生[せんせい]はどんな 難[むずか]しい 問題[もんだい]
\\	解[と]ける。
\\	どんな~でも	
\\	でも
\\	"私[わたし]はどんな 食[た]べ 物[もの]
\\	でも
\\	一度[いちど]は 食[た]べてみたいと 思[おも]います。
\\	"私[わたし]はどんな 食[た]べ 物[もの]
\\	一度[いちど]は 食[た]べてみたいと 思[おも]います。
\\	どんな~でも	
\\	でも
\\	"どこ
\\	でも
\\	いいから 座[すわ]って 下[くだ]さい。
\\	"どこ
\\	いいから 座[すわ]って 下[くだ]さい。
\\	でも
\\	"なん
\\	でも
\\	好[す]きなものを 注文[ちゅうもん]していいよ。
\\	"なん
\\	好[す]きなものを 注文[ちゅうもん]していいよ。
\\	でも
\\	"帰[かえ]りにコーヒー
\\	でも
\\	飲[の]みに 行[い]かない?
\\	"帰[かえ]りにコーヒー
\\	飲[の]みに 行[い]かない?
\\	でも
\\	"ここ
\\	でも
\\	よければお 話[おはなし]しますが。
\\	"ここ
\\	よければお 話[おはなし]しますが。
\\	でも
\\	"せめて 少[すこ]しだけ
\\	でも
\\	貯金[ちょきん]があったらなあ。
\\	"せめて 少[すこ]しだけ
\\	貯金[ちょきん]があったらなあ。
\\	せめて~でも~ 
\\	でも
\\	"腰[こし]を 下[お]ろす 所[ところ]
\\	でも
\\	あったらいいのになあ。
\\	"腰[こし]を 下[お]ろす 所[ところ]
\\	あったらいいのになあ。
\\	せめて~でも~ 
\\	でも
\\	"あの 人[ひと]の 言[い]うことは、まんざら 嘘[うそ]
\\	でも
\\	ない。
\\	"あの 人[ひと]の 言[い]うことは、まんざら 嘘[うそ]
\\	ない。
\\	まんざら~でも 
\\	でも
\\	"この 計画[けいかく]は、 実現[じつげん] 不可能[ふかのう]なわけ
\\	でも
\\	ない。
\\	"この 計画[けいかく]は、 実現[じつげん] 不可能[ふかのう]なわけ
\\	ない。
\\	まんざら~でも 
\\	でも
\\	"こう 人[ひと]が 多[おお]くては、ゆっくり 絵[え]を 観賞[かんしょう]する
\\	どころ
\\	ではない。
\\	"こう 人[ひと]が 多[おお]くては、ゆっくり 絵[え]を 観賞[かんしょう]する
\\	ではない。
\\	どころ
\\	"今[いま]とても 忙[いそが]しくて、 休暇[きゅうか]
\\	どころ
\\	の 騒[さわ]ぎではない。
\\	"今[いま]とても 忙[いそが]しくて、 休暇[きゅうか]
\\	の 騒[さわ]ぎではない。
\\	どころ
\\	"あの 子[こ]は 勉強[べんきょう]する
\\	どころか
\\	、 一日[いちにち] 中[じゅう] 遊[あそ]んでばかりいる。
\\	"あの 子[こ]は 勉強[べんきょう]する
\\	、 一日[いちにち] 中[じゅう] 遊[あそ]んでばかりいる。
\\	どころか
\\	"お湯[ゆ]
\\	どころか
\\	水[みず]さえ 出[で]ない。
\\	"お湯[ゆ]
\\	水[みず]さえ 出[で]ない。
\\	どころか
\\	"親友[しんゆう]といえ
\\	ども
\\	親友[しんゆう]といっても)、 彼女[かのじょ]の 本当[ほんとう]の 気持[きも]ちは 分[わ]からない。
\\	"親友[しんゆう]といえ
\\	親友[しんゆう]といっても)、 彼女[かのじょ]の 本当[ほんとう]の 気持[きも]ちは 分[わ]からない。
\\	ども
\\	"行[い]け
\\	ども
\\	行[い]け
\\	でも
\\	行[い]っても 行[い]っても)、 砂[すな]ばかりの 大[だい] 砂漠[さばく]だった。
\\	"行[い]け
\\	行[い]け
\\	行[い]っても 行[い]っても)、 砂[すな]ばかりの 大[だい] 砂漠[さばく]だった。
\\	ども
\\	"雨[あめ]が 降[ふ]り 出[だ]した。 明日[あした]のピクニックはいいお 天気[てんき]どたいい
\\	がな(がなあ)
\\	"雨[あめ]が 降[ふ]り 出[だ]した。 明日[あした]のピクニックはいいお 天気[てんき]どたいい
\\	が 
\\	な(あ)	
\\	がな(がなあ)
\\	"昨日[きのう]も 彼女[かのじょ]は 電話[でんわ]をかけてこなかった。 今日[きょう]はかけてくると 思[おも]う
\\	がな(がなあ)
\\	"昨日[きのう]も 彼女[かのじょ]は 電話[でんわ]をかけてこなかった。 今日[きょう]はかけてくると 思[おも]う
\\	が 
\\	な(あ)	
\\	がな(がなあ)
\\	"子供[こども]を 学校[がっこう]に 送[おく]り
\\	がてら
\\	、 買[か]い 物[もの]に 行[い]ってこよう。
\\	"子供[こども]を 学校[がっこう]に 送[おく]り
\\	、 買[か]い 物[もの]に 行[い]ってこよう。
\\	がてら
\\	"お寺[てら]にお 参[まい]りし
\\	がてら
\\	、 叔父[おじ]さんのうちに 寄[よ]ってきた。
\\	"お寺[てら]にお 参[まい]りし
\\	、 叔父[おじ]さんのうちに 寄[よ]ってきた。
\\	がてら
\\	"あの 人達[ひとたち]は、もうそこまでいってしまっていては、 離婚[りこん]する
\\	ほか
\\	仕方[しかた]がない。
\\	"あの 人達[ひとたち]は、もうそこまでいってしまっていては、 離婚[りこん]する
\\	仕方[しかた]がない。
\\	しか	
\\	ほか
\\	"こんなひどい 雨[あめ]では、 今日[きょう]の 試合[しあい]は 中止[ちゅうし]する
\\	ほか
\\	ないね。
\\	"こんなひどい 雨[あめ]では、 今日[きょう]の 試合[しあい]は 中止[ちゅうし]する
\\	ないね。
\\	しか	
\\	ほか
\\	"そのおもちゃ、 僕[ぼく]のだ
\\	い
\\	"そのおもちゃ、 僕[ぼく]のだ
\\	い
\\	"わしも 年[とし]をとったわ
\\	い
\\	"わしも 年[とし]をとったわ
\\	い
\\	"いかは 食[た]べたか
\\	い
\\	"いかは 食[た]べたか
\\	い
\\	"今晩[こんばん] 麻雀[マージャン]やらないか
\\	い
\\	"今晩[こんばん] 麻雀[マージャン]やらないか
\\	い
\\	"今[いま]すぐ 来[こ]られないか
\\	い
\\	"今[いま]すぐ 来[こ]られないか
\\	い
\\	"そりゃ、 嘘[うそ]じゃない
\\	かい
\\	"そりゃ、 嘘[うそ]じゃない
\\	かい
\\	"こんないい 天気[てんき]に、 雨[あめ]なんか 降[ふ]るもん
\\	かい
\\	"こんないい 天気[てんき]に、 雨[あめ]なんか 降[ふ]るもん
\\	かい
\\	"怪談[かいだん]?そんなもん 信[しん]じる
\\	かい
\\	"怪談[かいだん]?そんなもん 信[しん]じる
\\	かい
\\	"大学[だいがく] 入試[にゅうし]、 今年[ことし]こそはうまくいく
\\	かな(かなあ)
\\	"大学[だいがく] 入試[にゅうし]、 今年[ことし]こそはうまくいく
\\	かな(かなあ)
\\	"どうして 一人[ひとり]で 出[で]かけたの
\\	かな(かなあ)
\\	"どうして 一人[ひとり]で 出[で]かけたの
\\	かな(かなあ)
\\	"コーヒーでもいれよう
\\	かな(かなあ)
\\	"コーヒーでもいれよう
\\	かな(かなあ)
\\	"そろそろ 帰[かえ]ろう
\\	かな
\\	"そろそろ 帰[かえ]ろう
\\	かな
\\	"早[はや]く 春[はる]にならない
\\	かな(かなあ)
\\	"早[はや]く 春[はる]にならない
\\	~ないかな	
\\	かな(かなあ)
\\	"お 父[とう]さん 早[はや]く 出張[しゅっちょう]から 帰[かえ]ってこない
\\	かな(かなあ)
\\	"お 父[とう]さん 早[はや]く 出張[しゅっちょう]から 帰[かえ]ってこない
\\	~ないかな	
\\	かな(かなあ)
\\	"あの 男[おとこ]の 人[ひと]は 顔付[かおつ]き
\\	からして
\\	ちょっと 変[へん]だ。
\\	"あの 男[おとこ]の 人[ひと]は 顔付[かおつ]き
\\	ちょっと 変[へん]だ。
\\	からして
\\	"彼女[かのじょ]は 持[も]ち 物[もの]
\\	からして
\\	ぜいたくだ。
\\	"彼女[かのじょ]は 持[も]ち 物[もの]
\\	ぜいたくだ。
\\	からして
\\	から 
\\	ので	"今年[ことし]の 収支[しゅうし]は 赤字[あかじ]である
\\	からして
\\	、 来年[らいねん]はなんとか 黒字[くろじ]にしよう。
\\	"今年[ことし]の 収支[しゅうし]は 赤字[あかじ]である
\\	、 来年[らいねん]はなんとか 黒字[くろじ]にしよう。
\\	からして
\\	から 
\\	ので	"そういう 噂[うわさ]だ
\\	からして
\\	、あの 二人[ふたり]は 近々[ちかぢか] 結婚[けっこん]するだろう。
\\	"そういう 噂[うわさ]だ
\\	、あの 二人[ふたり]は 近々[ちかぢか] 結婚[けっこん]するだろう。
\\	からして
\\	"誰[だれ]
\\	かしら
\\	、 今頃[いまごろ] 戸[と]をたたくのは。
\\	"誰[だれ]
\\	、 今頃[いまごろ] 戸[と]をたたくのは。
\\	かしら
\\	"確[たし]かにここに 置[お]いたはずだけど、ないわ。どこ
\\	かしら
\\	"確[たし]かにここに 置[お]いたはずだけど、ないわ。どこ
\\	かしら
\\	"大[おお]きな 包[つつみ]ね。なに
\\	かしら
\\	"大[おお]きな 包[つつみ]ね。なに
\\	かしら
\\	"この 本[ほん]を 買[か]ったの、いつだった
\\	かしら
\\	"この 本[ほん]を 買[か]ったの、いつだった
\\	かしら
\\	"私[わたし]、そんなこと 言[い]った
\\	かしら
\\	。 覚[おぼ]えていないけど。
\\	"私[わたし]、そんなこと 言[い]った
\\	。 覚[おぼ]えていないけど。
\\	かしら
\\	"今度[こんど]はいついらっしゃいます
\\	かしら
\\	"今度[こんど]はいついらっしゃいます
\\	かしら
\\	"今日[きょう] 店[みせ]は 開[あ]いている
\\	かしら
\\	"今日[きょう] 店[みせ]は 開[あ]いている
\\	かしら
\\	"今晩[こんばん] 映画[えいが]に 行[い]こう
\\	かしら
\\	"今晩[こんばん] 映画[えいが]に 行[い]こう
\\	かしら
\\	"ケーキを 買[か]って 帰[かえ]ろう
\\	かしら
\\	"ケーキを 買[か]って 帰[かえ]ろう
\\	かしら
\\	"早[はや]く 合格[ごうかく] 通知[つうち]が 来[こ]ない
\\	かしら
\\	"早[はや]く 合格[ごうかく] 通知[つうち]が 来[こ]ない
\\	かしら
\\	"今[いま]すぐ 来[き]てきれない
\\	かしら
\\	"今[いま]すぐ 来[き]てきれない
\\	かしら
\\	"今[いま] 出[で]かけても、 十時[じゅうじ]の 汽車[きしゃ]に 間[ま]に 合[あ]う
\\	かしら
\\	"今[いま] 出[で]かけても、 十時[じゅうじ]の 汽車[きしゃ]に 間[ま]に 合[あ]う
\\	かしら
\\	"そのスーツケース、 重[おも]すぎない
\\	かしら
\\	"そのスーツケース、 重[おも]すぎない
\\	かしら
\\	だ. 
\\	"小[ちい]さいとき、よく 木登[きのぼ]りをしたもの(もん)だっ
\\	け
\\	"小[ちい]さいとき、よく 木登[きのぼ]りをしたもの(もん)だっ
\\	け
\\	だ. 
\\	"そういえば、そんな 所[ところ]へ 行[い]ったっ
\\	け
\\	ねえ。
\\	"そういえば、そんな 所[ところ]へ 行[い]ったっ
\\	ねえ。
\\	け
\\	"いつのことだったっ
\\	け
\\	、 君[きみ]と 音楽会[おんがくかい]に 行[い]ったのは。
\\	"いつのことだったっ
\\	、 君[きみ]と 音楽会[おんがくかい]に 行[い]ったのは。
\\	け
\\	"あの 人[ひと]のところの 電話[でんわ] 番号[ばんごう]、 何番[なんばん]だっ
\\	け
\\	"あの 人[ひと]のところの 電話[でんわ] 番号[ばんごう]、 何番[なんばん]だっ
\\	け
\\	"この 料理[りょうり]は 美味[おい]しい
\\	けれども
\\	、 作[つく]るのに 手間[てま]がかかる。
\\	"この 料理[りょうり]は 美味[おい]しい
\\	、 作[つく]るのに 手間[てま]がかかる。
\\	けれど、けど、けども	
\\	けれども
\\	"ちょっと 地味[じみ]だ
\\	けれども
\\	、 今夜[こんや]のパーティーにはこれを 着[き]て行[い]こう。
\\	"ちょっと 地味[じみ]だ
\\	、 今夜[こんや]のパーティーにはこれを 着[き]て行[い]こう。
\\	けれど、けど、けども	
\\	けれども
\\	"これ、ほんの 少[すこ]しです
\\	けれども
\\	、 召[め]し 上[あが]って 下[くだ]さい。
\\	"これ、ほんの 少[すこ]しです
\\	、 召[め]し 上[あが]って 下[くだ]さい。
\\	けれども
\\	"突然[とつぜん]です
\\	けれども
\\	、 会社[かいしゃ]を 辞[や]めさせていただきます。
\\	"突然[とつぜん]です
\\	、 会社[かいしゃ]を 辞[や]めさせていただきます。
\\	けれども
\\	"犬[いぬ]が 吠[ほ]えている
\\	けれども
\\	、 誰[だれ]か 外[そと]にいるんしゃない?
\\	"犬[いぬ]が 吠[ほ]えている
\\	、 誰[だれ]か 外[そと]にいるんしゃない?
\\	けれども
\\	"この 地方[ちほう]は 寒[さむ]いと 聞[き]いた
\\	けれども
\\	、 本当[ほんとう]に 毎日[まいにち] 冷[ひ]え 込[こ]むね。
\\	"この 地方[ちほう]は 寒[さむ]いと 聞[き]いた
\\	、 本当[ほんとう]に 毎日[まいにち] 冷[ひ]え 込[こ]むね。
\\	けれども
\\	"もう 少[すこ]し 日本語[にほんご]がよく 分[わ]かるといいのだ
\\	けれども
\\	"もう 少[すこ]し 日本語[にほんご]がよく 分[わ]かるといいのだ
\\	けれども
\\	"もう 少[すこ]しお 金[かね]があれば、あのステレオが 買[か]えるのだ
\\	けれども
\\	"もう 少[すこ]しお 金[かね]があれば、あのステレオが 買[か]えるのだ
\\	けれども
\\	"走[はし]ったってどうせ 間[ま]に 合[あ]わない
\\	けれでも
\\	"走[はし]ったってどうせ 間[ま]に 合[あ]わない
\\	どうせ~けれども	
\\	けれども
\\	"こんな 安物[やすもの]、 買[か]ってもどうせすぐ 壊[こわ]れるだろう
\\	けれども
\\	"こんな 安物[やすもの]、 買[か]ってもどうせすぐ 壊[こわ]れるだろう
\\	どうせ~けれども	
\\	けれども
\\	"十時[じゅうじ]の 汽車[きしゃ]でしたね。そろそろ 出[で]かける 時間[じかん]です
\\	けれども
\\	"十時[じゅうじ]の 汽車[きしゃ]でしたね。そろそろ 出[で]かける 時間[じかん]です
\\	けれども
\\	"勃勃[ぼつぼつ] 講義[こうぎ]が 始[はじ]まります
\\	けれども
\\	"勃勃[ぼつぼつ] 講義[こうぎ]が 始[はじ]まります
\\	けれども
\\	"日本語[にほんご]は 難[むずか]しい。
\\	けれども
\\	、 一生懸命[いっしょうけんめい] 勉強[べんきょう]すれば、 必[かなら]ず 分[わ]かるようになる。
\\	"日本語[にほんご]は 難[むずか]しい。
\\	、 一生懸命[いっしょうけんめい] 勉強[べんきょう]すれば、 必[かなら]ず 分[わ]かるようになる。
\\	けれども
\\	"猫[ねこ]は、 確[たし]かに 可愛[かわ]らしい 動物[どうぶつ]です。
\\	けれども
\\	、 私[わたし]はアレルギーですから、 飼[か]えません。
\\	"猫[ねこ]は、 確[たし]かに 可愛[かわ]らしい 動物[どうぶつ]です。
\\	、 私[わたし]はアレルギーですから、 飼[か]えません。
\\	けれども
\\	"夏休[なつやす]みには、 山[やま]に 行[い]くのもいい。
\\	けれども
\\	、 海[うみ]もまたいいね。
\\	"夏休[なつやす]みには、 山[やま]に 行[い]くのもいい。
\\	、 海[うみ]もまたいいね。
\\	けれども
\\	"二人[ふたり]
\\	きり
\\	でどこかえ 行[い]ってしまいたい。
\\	"二人[ふたり]
\\	でどこかえ 行[い]ってしまいたい。
\\	きり
\\	"私[わたし]はヨーロッパは、 一度[いちど] 行[い]った
\\	きり
\\	です。
\\	"私[わたし]はヨーロッパは、 一度[いちど] 行[い]った
\\	です。
\\	きり
\\	"年老[としお]いた 父[ちち]は、 寝[ね]た
\\	きり
\\	になった。
\\	"年老[としお]いた 父[ちち]は、 寝[ね]た
\\	になった。
\\	きり
\\	"座[すわ]りっ
\\	きり
\\	で 仕事[しごと]を 続[つづ]けたので、 肩[かた]が 凝[こ]った。
\\	"座[すわ]りっ
\\	で 仕事[しごと]を 続[つづ]けたので、 肩[かた]が 凝[こ]った。
\\	きり
\\	"彼[かれ]は 一度[いちど] 電話[でんわ]をかけてきた
\\	きり
\\	、 行方[ゆくえ]が 分[わ]かりません。
\\	"彼[かれ]は 一度[いちど] 電話[でんわ]をかけてきた
\\	、 行方[ゆくえ]が 分[わ]かりません。
\\	きり
\\	"彼女[かのじょ]とは、 去年[きょけん] 会[あ]った
\\	きり
\\	、 手紙[てがみ]も 出[だ]していない。
\\	"彼女[かのじょ]とは、 去年[きょけん] 会[あ]った
\\	、 手紙[てがみ]も 出[だ]していない。
\\	きり
\\	"僕[ぼく]は、お 酒[さけ]はまるっ
\\	きり
\\	駄目[だめ]だ。
\\	"僕[ぼく]は、お 酒[さけ]はまるっ
\\	駄目[だめ]だ。
\\	まるっきり	
\\	きり
\\	"私[わたし]は、まるっ
\\	きり
\\	泳[およ]げないんです。
\\	"私[わたし]は、まるっ
\\	泳[およ]げないんです。
\\	まるっきり	
\\	きり
\\	"あの 山[やま]
\\	こそ
\\	、 沢山[たくさん]の 登山者[とざんしゃ]が 命[いのち]を 落[お]とした 山[やま]だ。
\\	"あの 山[やま]
\\	、 沢山[たくさん]の 登山者[とざんしゃ]が 命[いのち]を 落[お]とした 山[やま]だ。
\\	こそ
\\	"来年[らいねん]
\\	こそ
\\	日本[にほん]へ 行[い]ってみよう。
\\	"来年[らいねん]
\\	日本[にほん]へ 行[い]ってみよう。
\\	こそ
\\	"こうなることが 分[わ]かっていたから
\\	こそ
\\	、 私[わたし]は 賛成[さんせい]しなかったのです。
\\	"こうなることが 分[わ]かっていたから
\\	、 私[わたし]は 賛成[さんせい]しなかったのです。
\\	こそ
\\	"あなたのことを 考[かんが]えれば
\\	こそ
\\	、 決断[けつだん]を 思[おも]いとどまったのよ。
\\	"あなたのことを 考[かんが]えれば
\\	、 決断[けつだん]を 思[おも]いとどまったのよ。
\\	こそ
\\	"あの 人[ひと]は、 人[ひと]を 褒[ほ]め
\\	こそ
\\	すれ、 貶[けな]したことはない。
\\	"あの 人[ひと]は、 人[ひと]を 褒[ほ]め
\\	すれ、 貶[けな]したことはない。
\\	こそ
\\	"テストの 前[まえ]にインフルエンザにでもかかったら、それ
\\	こそ
\\	大変[たいへん]だよ。
\\	"テストの 前[まえ]にインフルエンザにでもかかったら、それ
\\	大変[たいへん]だよ。
\\	それこそ	
\\	こそ
\\	"そんなことを 彼女[かのじょ]に 言[い]おうものなら、それ
\\	こそ
\\	ただでは 済[す]まないよ。
\\	"そんなことを 彼女[かのじょ]に 言[い]おうものなら、それ
\\	ただでは 済[す]まないよ。
\\	それこそ	
\\	こそ
\\	"彼[かれ]は 大学生[だいがくせい]の
\\	くせに
\\	、まだ 母親[ははおや]に 洗濯[せんたく]してもらっているんだって。
\\	"彼[かれ]は 大学生[だいがくせい]の
\\	、まだ 母親[ははおや]に 洗濯[せんたく]してもらっているんだって。
\\	くせ 
\\	に 
\\	くせに
\\	"いつも 寝坊[ねぼう]する
\\	くせに
\\	、 今日[きょう]はずいぶん 早[はや]く 起[お]きたね。
\\	"いつも 寝坊[ねぼう]する
\\	、 今日[きょう]はずいぶん 早[はや]く 起[お]きたね。
\\	くせ 
\\	に 
\\	くせに
\\	"ミルクまだ 残[のこ]したの。 子供[こども]の 時[とき]はあんなに 好[す]きだった
\\	くせに
\\	"ミルクまだ 残[のこ]したの。 子供[こども]の 時[とき]はあんなに 好[す]きだった
\\	くせに
\\	"やめなさい、そんなに 沢山[たくさん] 食[た]べるのは。この 間[あいだ]もお 腹[なか]をこわした
\\	くせに
\\	"やめなさい、そんなに 沢山[たくさん] 食[た]べるのは。この 間[あいだ]もお 腹[なか]をこわした
\\	くせに
\\	"今更[いまさら] 申[もう]し 上[あ]げる
\\	までも
\\	ありませんが、 私[わたくし]は 来月[らいげつ] 定年[ていねん] 退職[たいしょく]いたします。
\\	"今更[いまさら] 申[もう]し 上[あ]げる
\\	ありませんが、 私[わたくし]は 来月[らいげつ] 定年[ていねん] 退職[たいしょく]いたします。
\\	まで 
\\	も 
\\	までも
\\	"手紙[てがみ]に 書[か]く
\\	までも
\\	ないことですから、 電話[でんわ]でお 伝[つた]えいたします。
\\	"手紙[てがみ]に 書[か]く
\\	ないことですから、 電話[でんわ]でお 伝[つた]えいたします。
\\	まで 
\\	も 
\\	までも
\\	"はっきりと 覚[おぼ]えていない
\\	までも
\\	、 少[すこ]しは 思[おも]い 出[だ]せるでしょう?
\\	"はっきりと 覚[おぼ]えていない
\\	、 少[すこ]しは 思[おも]い 出[だ]せるでしょう?
\\	までも
\\	"入院[にゅういん]しない
\\	までも
\\	、 充分[じゅうぶん]な 教養[きゅうよう]が 必要[ひつよう]です。
\\	"入院[にゅういん]しない
\\	、 充分[じゅうぶん]な 教養[きゅうよう]が 必要[ひつよう]です。
\\	までも
\\	"あんまり 嬉[うれ]しかった
\\	もんで
\\	、ほかのこと 全部[ぜんぶ] 忘[わす]れちゃった。
\\	"あんまり 嬉[うれ]しかった
\\	、ほかのこと 全部[ぜんぶ] 忘[わす]れちゃった。
\\	もの 
\\	で 
\\	もんで 
\\	もんで
\\	"そのニュースを 今[いま] 聞[き]いた
\\	もので
\\	、びっくりしてお 電話[でんわ]をおかけしているところです。
\\	"そのニュースを 今[いま] 聞[き]いた
\\	、びっくりしてお 電話[でんわ]をおかけしているところです。
\\	もの 
\\	で 
\\	もんで 
\\	もので
\\	"僕[ぼく]の 気持[きも]ちが 君[きみ]に 分[わ]かってたまる
\\	ものか
\\	"僕[ぼく]の 気持[きも]ちが 君[きみ]に 分[わ]かってたまる
\\	ものですか 
\\	もんか 
\\	もんですか	
\\	ものか
\\	"彼[かれ]のほうが 悪[わる]いんですもの。 私[わたし] 絶対[ぜったい]に 謝[あやま]る
\\	もんですか
\\	"彼[かれ]のほうが 悪[わる]いんですもの。 私[わたし] 絶対[ぜったい]に 謝[あやま]る
\\	ものですか 
\\	もんか 
\\	もんですか	
\\	もんですか
\\	"タバコの 吸殻[すいがら]を 捨[す]ててはいけないよ。 火事[かじ]にでもなろう
\\	ものなら
\\	大変[たいへん]だから。
\\	"タバコの 吸殻[すいがら]を 捨[す]ててはいけないよ。 火事[かじ]にでもなろう
\\	大変[たいへん]だから。
\\	もんなら 
\\	ものなら
\\	"早[はや]く 寝[ね]なさいよ。 明日[あした]の 朝[あさ]、 寝坊[ねぼう]しよう
\\	ものなら
\\	、 遠足[えんそく]にいけなくなるよ。
\\	"早[はや]く 寝[ね]なさいよ。 明日[あした]の 朝[あさ]、 寝坊[ねぼう]しよう
\\	、 遠足[えんそく]にいけなくなるよ。
\\	もんなら 
\\	ものなら
\\	"あなたがどんなに 貧乏[びんぼう]になってもいい
\\	ものなら
\\	、 何[なに]も 言[い]いませんよ。
\\	"あなたがどんなに 貧乏[びんぼう]になってもいい
\\	、 何[なに]も 言[い]いませんよ。
\\	ものなら
\\	"そんなのでいい
\\	ものなら
\\	、どこにでも 売[う]っていますよ。
\\	"そんなのでいい
\\	、どこにでも 売[う]っていますよ。
\\	ものなら
\\	"このうなぎ、つかめる
\\	ものなら
\\	、つかんでごらん。
\\	"このうなぎ、つかめる
\\	、つかんでごらん。
\\	ものなら
\\	"飛[と]べる
\\	ものなら
\\	飛[と]んでごらんよ。
\\	"飛[と]べる
\\	飛[と]んでごらんよ。
\\	ものなら
\\	"いつもより 早[はや]くうちを 出[で]た
\\	ものの
\\	、バスがなかなか 来[こ]なくて、 遅刻[ちこく]してしまった。
\\	"いつもより 早[はや]くうちを 出[で]た
\\	、バスがなかなか 来[こ]なくて、 遅刻[ちこく]してしまった。
\\	ものの
\\	"車[くるま]は 買[か]った
\\	ものの
\\	、まだ 恐[こわ]くてハイウェイは 走[はし]れない。
\\	"車[くるま]は 買[か]った
\\	、まだ 恐[こわ]くてハイウェイは 走[はし]れない。
\\	ものの
\\	"間[ま]に 合[あ]ったからいいような
\\	ものの
\\	、うちを 出[で]るのが、もう 少[すこ]し 遅[おそ]かったら、 飛行機[ひこうき]に 乗[の]り 遅[おく]れるところだったよ。
\\	"間[ま]に 合[あ]ったからいいような
\\	、うちを 出[で]るのが、もう 少[すこ]し 遅[おそ]かったら、 飛行機[ひこうき]に 乗[の]り 遅[おく]れるところだったよ。
\\	ものの
\\	"安[やす]いからとはいう
\\	ものの
\\	、 沢山[たくさん] 買[か]うとかえって 損[そん]をすることがある。
\\	"安[やす]いからとはいう
\\	、 沢山[たくさん] 買[か]うとかえって 損[そん]をすることがある。
\\	ものの
\\	"早[はや]く 休[やす]めばよかった
\\	ものを
\\	、 無理[むり]をして 働[はたら]いたので、 病気[びょうき]になってしまった。
\\	"早[はや]く 休[やす]めばよかった
\\	、 無理[むり]をして 働[はたら]いたので、 病気[びょうき]になってしまった。
\\	ものを
\\	"こんなに 雨[あめ]でずぶぬれになって! 出[で]かける 前[まえ]に 天気[てんき] 予報[よほう]を 聞[き]いてくればよかった
\\	ものを
\\	"こんなに 雨[あめ]でずぶぬれになって! 出[で]かける 前[まえ]に 天気[てんき] 予報[よほう]を 聞[き]いてくればよかった
\\	ものを
\\	"外[そと]で 夕食[ゆうしょく]すませてきたの? 待[ま]っていたのに。 一言[ひとこと] 電話[でんわ]してくればいい
\\	ものを
\\	"外[そと]で 夕食[ゆうしょく]すませてきたの? 待[ま]っていたのに。 一言[ひとこと] 電話[でんわ]してくればいい
\\	ものを
\\	"少[すこ]し 会[あ]わないうちに 大[おお]きくなった
\\	な(なあ)
\\	"少[すこ]し 会[あ]わないうちに 大[おお]きくなった
\\	な 
\\	なあ 
\\	なあ 
\\	な(なあ)
\\	"野菜[やさい]の 値段[ねだん]が 随分[ずいぶん] 上[あ]がった
\\	な(なあ)
\\	"野菜[やさい]の 値段[ねだん]が 随分[ずいぶん] 上[あ]がった
\\	な 
\\	なあ 
\\	なあ 
\\	な(なあ)
\\	"ビール 三[さん]ダースか、ずいぶん 沢山[たくさん] 飲[の]んだ
\\	な(なあ)
\\	"ビール 三[さん]ダースか、ずいぶん 沢山[たくさん] 飲[の]んだ
\\	な 
\\	なあ 
\\	なあ 
\\	な(なあ)
\\	"明日[あした]も 試験[しけん]だ。いやだ
\\	な(なあ)
\\	"明日[あした]も 試験[しけん]だ。いやだ
\\	な(なあ)
\\	"僕[ぼく]は、その 噂[うわさ]は 本当[ほんとう]だと 思[おも]う
\\	な
\\	"僕[ぼく]は、その 噂[うわさ]は 本当[ほんとう]だと 思[おも]う
\\	な
\\	"お 借[か]りした 本[ほん]は、 確[たし]かにお 返[かえ]ししました
\\	な
\\	"お 借[か]りした 本[ほん]は、 確[たし]かにお 返[かえ]ししました
\\	な
\\	"去年[きょねん]の 春[はる]、 京都[きょうと]でお 目[め]にかかりました
\\	な
\\	"去年[きょねん]の 春[はる]、 京都[きょうと]でお 目[め]にかかりました
\\	な
\\	"毎日[まいにち] 忙[いそが]しすぎる。 少[すこ]しのんびりしたい
\\	なあ
\\	"毎日[まいにち] 忙[いそが]しすぎる。 少[すこ]しのんびりしたい
\\	なあ
\\	"南[みなみ] 十字[じゅうじ] 星[せい]が 見[み]えるといい
\\	な(なあ)
\\	"南[みなみ] 十字[じゅうじ] 星[せい]が 見[み]えるといい
\\	な(なあ)
\\	"明日[あした] 必[かなら]ずお 電話[でんわ] 下[くだ]さい
\\	な
\\	"明日[あした] 必[かなら]ずお 電話[でんわ] 下[くだ]さい
\\	~(て)下さい、~なさい	
\\	な
\\	"もう 十二時[じゅうにじ]すぎですよ。 早[はや]くおやすみなさい
\\	な
\\	"もう 十二時[じゅうにじ]すぎですよ。 早[はや]くおやすみなさい
\\	~(て)下さい、~なさい	
\\	な
\\	"あの
\\	な
\\	、このことは 内緒[ないしょ]だよ。
\\	"あの
\\	、このことは 内緒[ないしょ]だよ。
\\	な
\\	"ところで
\\	な
\\	、いいニュースがあるうんだよ。
\\	"ところで
\\	、いいニュースがあるうんだよ。
\\	な
\\	"いよいよマラソン 大会[たいかい]の 日[ひ]だね。 最後[さいご]まで 挫[くじ]ける
\\	な
\\	"いよいよマラソン 大会[たいかい]の 日[ひ]だね。 最後[さいご]まで 挫[くじ]ける
\\	な
\\	"昨日[きのう]の 雪[ゆき]が 氷[こおり]になったから 危[あぶ]ないよ。 転[ころ]ぶ
\\	な
\\	"昨日[きのう]の 雪[ゆき]が 氷[こおり]になったから 危[あぶ]ないよ。 転[ころ]ぶ
\\	な
\\	"さあ、 早[はや]く 見[み]せ
\\	な
\\	"さあ、 早[はや]く 見[み]せ
\\	な
\\	"とっとと 歩[ある]き
\\	な
\\	"とっとと 歩[ある]き
\\	な
\\	"小学校[しょうがっこう]や 中学校[ちゅうがっこう]
\\	なだ
\\	でインフルエンザがはやっています。
\\	"小学校[しょうがっこう]や 中学校[ちゅうがっこう]
\\	でインフルエンザがはやっています。
\\	~や~など 
\\	なんか	
\\	など
\\	"私[わたし]は 忙[いそが]しくて、 洗濯[せんたく]や 掃除[そうじ]
\\	など
\\	はつい 後回[あとまわ]しにしてしまいます。
\\	"私[わたし]は 忙[いそが]しくて、 洗濯[せんたく]や 掃除[そうじ]
\\	はつい 後回[あとまわ]しにしてしまいます。
\\	~や~など 
\\	なんか	
\\	など
\\	"あの 人[ひと]の 話[はなし]
\\	など
\\	信用[しんよう]できるもんですか。
\\	"あの 人[ひと]の 話[はなし]
\\	信用[しんよう]できるもんですか。
\\	など
\\	"誰[だれ]が 妬[ねた]み
\\	など
\\	するものですか。
\\	"誰[だれ]が 妬[ねた]み
\\	するものですか。
\\	など
\\	"勉強[べんきょう]が 忙[いそが]しくて、 買[か]い 物[もの]
\\	なんか
\\	行[い]けないよ。
\\	"勉強[べんきょう]が 忙[いそが]しくて、 買[か]い 物[もの]
\\	行[い]けないよ。
\\	なんか
\\	"暗[くら]くて、 新聞[しんぶん]
\\	なんか
\\	読[よ]めないわ。
\\	"暗[くら]くて、 新聞[しんぶん]
\\	読[よ]めないわ。
\\	なんか
\\	"あの 絵[え]
\\	など
\\	、なかなかよくかけているね。
\\	"あの 絵[え]
\\	、なかなかよくかけているね。
\\	など
\\	"北海道[ほっかいどう]
\\	など
\\	はもう 雪[ゆき]だってさ。
\\	"北海道[ほっかいどう]
\\	はもう 雪[ゆき]だってさ。
\\	など
\\	"この 代数[だいすう]の 問題[もんだい]
\\	なんか
\\	、 簡単[かんたん]に 解[と]けるよ。
\\	"この 代数[だいすう]の 問題[もんだい]
\\	、 簡単[かんたん]に 解[と]けるよ。
\\	なんか
\\	"あそこの 店[みせ]
\\	なんか
\\	、いいドレスがありそうよ。
\\	"あそこの 店[みせ]
\\	、いいドレスがありそうよ。
\\	なんか
\\	"息子[むすこ]からの 手紙[てがみ]に、 日本食[にほんしょく]が 恋[こい]しい
\\	など
\\	と 書[か]いてあったわ。
\\	"息子[むすこ]からの 手紙[てがみ]に、 日本食[にほんしょく]が 恋[こい]しい
\\	と 書[か]いてあったわ。
\\	など
\\	"もう 決[けっ]して 離[はな]れない
\\	など
\\	と 言[い]っていたのに、 半年[はんとし]したら 別[わか]れてしまったよ、あのカップル。
\\	"もう 決[けっ]して 離[はな]れない
\\	と 言[い]っていたのに、 半年[はんとし]したら 別[わか]れてしまったよ、あのカップル。
\\	など
\\	ながら 
\\	"私[わたし]は、 勉強[べんきょう]し
\\	ながらも
\\	、あの 事件[じけん]のことが 頭[あたま]から 離[はな]れない。
\\	"私[わたし]は、 勉強[べんきょう]し
\\	、あの 事件[じけん]のことが 頭[あたま]から 離[はな]れない。
\\	ながら 
\\	も	
\\	ながらも
\\	ながら 
\\	"彼女[かのじょ]は、 病[やまい]の 床[とこ]に 伏[ふ]し
\\	ながらも
\\	、 本[ほん]を 書[か]き 続[つづ]けた。
\\	"彼女[かのじょ]は、 病[やまい]の 床[とこ]に 伏[ふ]し
\\	、 本[ほん]を 書[か]き 続[つづ]けた。
\\	ながら 
\\	も	
\\	ながらも
\\	"この 料理[りょうり]は、 美味[おい]しくない
\\	ながらも
\\	苦心[くしん]の 作[さく]なんだよ。
\\	"この 料理[りょうり]は、 美味[おい]しくない
\\	苦心[くしん]の 作[さく]なんだよ。
\\	ながらも
\\	"あの 子[こ]は、 小[ちい]さい
\\	ながらも
\\	、 両親[りょうしん]の 気持[きも]ちがよく 分[わ]かっている。
\\	"あの 子[こ]は、 小[ちい]さい
\\	、 両親[りょうしん]の 気持[きも]ちがよく 分[わ]かっている。
\\	ながらも
\\	"来週[らいしゅう]は 必[かなら]ずお 邪魔[じゃま]します
\\	なんて
\\	言[い]っていたけど、あの 大工[だいく]さん、 本当[ほんとう]に 来[き]てくれるから。
\\	"来週[らいしゅう]は 必[かなら]ずお 邪魔[じゃま]します
\\	言[い]っていたけど、あの 大工[だいく]さん、 本当[ほんとう]に 来[き]てくれるから。
\\	なんて
\\	"試験[しけん]で 悪[わる]い 点[てん]を 取[と]ったぐらいで、この 世[よ]も 終[お]わりだ
\\	なんて
\\	考[かんが]えるなよ。
\\	"試験[しけん]で 悪[わる]い 点[てん]を 取[と]ったぐらいで、この 世[よ]も 終[お]わりだ
\\	考[かんが]えるなよ。
\\	なんて
\\	"彼[かれ]は、 宿題[しゅくだい]を 済[す]ませる
\\	なり
\\	遊[あそ]びに 行[い]ってしまった。
\\	"彼[かれ]は、 宿題[しゅくだい]を 済[す]ませる
\\	遊[あそ]びに 行[い]ってしまった。
\\	なり
\\	"僕[ぼく]は、 毎晩[まいばん] 頭[あたま]を 枕[まくら]につける
\\	なり
\\	眠[ねむ]ってしまう。
\\	"僕[ぼく]は、 毎晩[まいばん] 頭[あたま]を 枕[まくら]につける
\\	眠[ねむ]ってしまう。
\\	なり
\\	"彼[かれ]は、 靴[くつ]を 履[は]いた
\\	なり
\\	、 畳[たたみ]に 上[あ]がってしまった。
\\	"彼[かれ]は、 靴[くつ]を 履[は]いた
\\	、 畳[たたみ]に 上[あ]がってしまった。
\\	~たなり 
\\	なり
\\	"彼女[かのじょ]は、アルプス 登山[とざん]に 出[で]かけた
\\	なり
\\	、ついに 戻[もど]らなかった。
\\	"彼女[かのじょ]は、アルプス 登山[とざん]に 出[で]かけた
\\	、ついに 戻[もど]らなかった。
\\	~たなり 
\\	なり
\\	"たまには 温泉[おんせん]へ
\\	なり
\\	行[い]ってみましょうよ。
\\	"たまには 温泉[おんせん]へ
\\	行[い]ってみましょうよ。
\\	なり
\\	"そのこのならお 医者[いしゃ]さんに
\\	なり
\\	聞[き]いてみたほうがいいよ。
\\	"そのこのならお 医者[いしゃ]さんに
\\	聞[き]いてみたほうがいいよ。
\\	なり
\\	"行[い]く
\\	なり
\\	行[い]かない
\\	なり
\\	、 早[はや]く 決[き]めなさい。
\\	"行[い]く
\\	行[い]かない
\\	、 早[はや]く 決[き]めなさい。
\\	~なり~なり	
\\	なり
\\	"風邪[かぜ]を 引[ひ]いたときは、ジュース
\\	なり
\\	水[みず]
\\	なり
\\	沢山[たくさん] 飲[の]むことです。
\\	"風邪[かぜ]を 引[ひ]いたときは、ジュース
\\	水[みず]
\\	沢山[たくさん] 飲[の]むことです。
\\	~なり~なり	
\\	なり
\\	"町[まち]から 遠[とお]いところに 住[す]むのだから、 大[だい]
\\	なり
\\	小[しょう]
\\	なり
\\	の 不便[ふべん]はあるだろう。
\\	"町[まち]から 遠[とお]いところに 住[す]むのだから、 大[だい]
\\	小[しょう]
\\	の 不便[ふべん]はあるだろう。
\\	~なり~なり	
\\	なり
\\	"ご病気[びょうき]と 伺[うかが]いながらお 見舞[みま]いにも 上[あ]がりませ
\\	んで
\\	、 失礼[しつれい]いたしました。
\\	"ご病気[びょうき]と 伺[うかが]いながらお 見舞[みま]いにも 上[あ]がりませ
\\	、 失礼[しつれい]いたしました。
\\	(ませ)ん 
\\	ます 
\\	で	
\\	んで
\\	"お電話[でんわ]をいただきながらすぐにお 返事[へんじ]もしませ
\\	んで
\\	、 申[もう]し 訳[わけ]ありませんでした。
\\	"お電話[でんわ]をいただきながらすぐにお 返事[へんじ]もしませ
\\	、 申[もう]し 訳[わけ]ありませんでした。
\\	(ませ)ん 
\\	ます 
\\	で	
\\	んで
\\	"せっかくいらして 下[くだ]さいましたのに、お 茶[ちゃ]も 差[さ]し 上[あ]げませ
\\	んで
\\	"せっかくいらして 下[くだ]さいましたのに、お 茶[ちゃ]も 差[さ]し 上[あ]げませ
\\	んで
\\	"態々[わざわざ]お 出[い]で 下[くだ]さいましたのに、 主人[しゅじん]は 出張[しゅっちょう] 中[ちゅう]でお 目[め]にかかれませ
\\	んで
\\	"態々[わざわざ]お 出[い]で 下[くだ]さいましたのに、 主人[しゅじん]は 出張[しゅっちょう] 中[ちゅう]でお 目[め]にかかれませ
\\	んで
\\	"まあ、バイオリンが 上手[じょうず]に 弾[ひ]けるの
\\	ね(ねえ)
\\	"まあ、バイオリンが 上手[じょうず]に 弾[ひ]けるの
\\	ね 
\\	ねえ 
\\	ねえ 
\\	ね(ねえ)
\\	"立派[りっぱ]なお 宅[たく]にお 住[す]まいでいらっしゃいませ
\\	ね(ねえ)
\\	"立派[りっぱ]なお 宅[たく]にお 住[す]まいでいらっしゃいませ
\\	ね 
\\	ねえ 
\\	ねえ 
\\	ね(ねえ)
\\	"今[いま] 私[わたし]が 言[い]ったこと、よく 分[わ]からなかったみたい
\\	ね
\\	"今[いま] 私[わたし]が 言[い]ったこと、よく 分[わ]からなかったみたい
\\	ね
\\	"これはやはり
\\	本物[ほんもの]ではないようです
\\	ね
\\	"これはやはり 本物[ほんもの]ではないようです
\\	ね
\\	"分[わ]かったから、もう 泣[な]かないで
\\	ね
\\	"分[わ]かったから、もう 泣[な]かないで
\\	ね
\\	"許[ゆる]して 下[くだ]さい
\\	ね
\\	。もう 二度[にど]とご 心配[しんぱい]をおかけしませんから。
\\	"許[ゆる]して 下[くだ]さい
\\	。もう 二度[にど]とご 心配[しんぱい]をおかけしませんから。
\\	ね
\\	"アメリカに 旅行[りょこう]したのは、いつだったっけ
\\	ね(ねえ)
\\	"アメリカに 旅行[りょこう]したのは、いつだったっけ
\\	ね(ねえ)
\\	"君[きみ]の 故郷[こきょう]はどこか
\\	ね
\\	"君[きみ]の 故郷[こきょう]はどこか
\\	ね
\\	"五合[ごごう] 目[め]まで 車[くるま]で 行[い]って、それから
\\	ね(ねえ)
\\	、その 先[さき]は 歩[ある]いて 頂上[ちょうじょう]まで 登[のぼ]ったよ。
\\	"五合[ごごう] 目[め]まで 車[くるま]で 行[い]って、それから
\\	、その 先[さき]は 歩[ある]いて 頂上[ちょうじょう]まで 登[のぼ]ったよ。
\\	ね(ねえ)
\\	"あたし
\\	ね(ねえ)
\\	、 今日[きょう] 駆[か]けっこで 一等[いっとう]になったのよ。
\\	"あたし
\\	、 今日[きょう] 駆[か]けっこで 一等[いっとう]になったのよ。
\\	ね(ねえ)
\\	あの_、だ_/です_	"あの
\\	ね(ねえ)
\\	、 今晩[こんばん] 僕[ぼく]のほうからまた 電話[でんわ]するよ。
\\	"あの
\\	、 今晩[こんばん] 僕[ぼく]のほうからまた 電話[でんわ]するよ。
\\	ね(ねえ)
\\	あの_、だ_/です_	"そうだ
\\	ね(ねえ)
\\	、もう 一年[いちねん] 前[まえ]になるんだ、 卒業[そつぎょう]したのは。
\\	"そうだ
\\	、もう 一年[いちねん] 前[まえ]になるんだ、 卒業[そつぎょう]したのは。
\\	ね(ねえ)
\\	"彼女[かのじょ]は、 昼休[ひるやす]み
\\	にも
\\	仕事[しごと]をしている。
\\	"彼女[かのじょ]は、 昼休[ひるやす]み
\\	仕事[しごと]をしている。
\\	に 
\\	も 
\\	にも
\\	"私[わたし]は、アフリカ
\\	にも
\\	行[い]ったことがあります。
\\	"私[わたし]は、アフリカ
\\	行[い]ったことがあります。
\\	に 
\\	も 
\\	にも
\\	"そのお 菓子[かし]、 僕[ぼく]
\\	にも
\\	頂戴[ちょうだい]。
\\	"そのお 菓子[かし]、 僕[ぼく]
\\	頂戴[ちょうだい]。
\\	に 
\\	も 
\\	にも
\\	"先生[せんせい]
\\	にも
\\	お 元気[げんき]のことと 思[おも]います。
\\	"先生[せんせい]
\\	お 元気[げんき]のことと 思[おも]います。
\\	にも
\\	"お 宅[たく]の 皆様[みなさま]
\\	にも
\\	お 変[か]わりはございませんか。
\\	"お 宅[たく]の 皆様[みなさま]
\\	お 変[か]わりはございませんか。
\\	にも
\\	"慌[あわ]てて、 着物[きもの]を 着[き]たまま 水[みず]に 飛[と]び 込[こ]んでしまい、 泳[およ]ぐ
\\	にも
\\	泳[およ]げませんでした。
\\	"慌[あわ]てて、 着物[きもの]を 着[き]たまま 水[みず]に 飛[と]び 込[こ]んでしまい、 泳[およ]ぐ
\\	泳[およ]げませんでした。
\\	にも 
\\	ず/ない 
\\	にも
\\	"こんなに 部屋[へや]を 散[ち]らかして、 歩[ある]く
\\	にも
\\	歩[ある]けないじゃないの。
\\	"こんなに 部屋[へや]を 散[ち]らかして、 歩[ある]く
\\	歩[ある]けないじゃないの。
\\	にも 
\\	ず/ない 
\\	にも
\\	"間[ま]に 合[あ]うように 駅[えき]に 着[つ]いた
\\	にもかかわらず
\\	、 電車[でんしゃ]はもう 出[で]てしまった 後[あと]だった。
\\	"間[ま]に 合[あ]うように 駅[えき]に 着[つ]いた
\\	、 電車[でんしゃ]はもう 出[で]てしまった 後[あと]だった。
\\	に、も、ず 
\\	関わる 
\\	にもかかわらず
\\	"今日[きょう]は 晴[は]れている
\\	にもかかわらず
\\	、スモッグで 空[そら]は 灰色[はいいろ]です。
\\	"今日[きょう]は 晴[は]れている
\\	、スモッグで 空[そら]は 灰色[はいいろ]です。
\\	に、も、ず 
\\	関わる 
\\	にもかかわらず
\\	"彼[かれ]が 勉強[べんきょう]しているのを 見[み]たことがない。
\\	にかかわらず
\\	、 成績[せいせき]はいつもトップだ。
\\	"彼[かれ]が 勉強[べんきょう]しているのを 見[み]たことがない。
\\	、 成績[せいせき]はいつもトップだ。
\\	それにも関わらず 
\\	にかかわらず
\\	"今日[きょう]はひどい 暑[あつ]さだ。
\\	にかかわらず
\\	、あのおばあさんは 冬[ふゆ]のコートを 着[き]ている。
\\	"今日[きょう]はひどい 暑[あつ]さだ。
\\	、あのおばあさんは 冬[ふゆ]のコートを 着[き]ている。
\\	それにも関わらず 
\\	にかかわらず
\\	"一九九二[せんきゅうひゃくきゅうじゅうに] 年[ねん]のオリンピックはバルセロナ
\\	において
\\	行[おこな]われます。
\\	"一九九二[せんきゅうひゃくきゅうじゅうに] 年[ねん]のオリンピックはバルセロナ
\\	行[おこな]われます。
\\	に 
\\	おいて, 
\\	おく 
\\	において
\\	"十八世[じゅうはっせい] 紀[き]
\\	において
\\	活躍[かつやく]した 作曲家[さっきょくか]にモーツァルトがいます。
\\	"十八世[じゅうはっせい] 紀[き]
\\	活躍[かつやく]した 作曲家[さっきょくか]にモーツァルトがいます。
\\	に 
\\	おいて, 
\\	おく 
\\	において
\\	"彼[かれ]は、アメリカにいる 間[あいだ]に、 経済[せいざい] 学[がく]
\\	において
\\	優[すぐ]れた 研究[けんきゅう]をした。
\\	"彼[かれ]は、アメリカにいる 間[あいだ]に、 経済[せいざい] 学[がく]
\\	優[すぐ]れた 研究[けんきゅう]をした。
\\	において
\\	"絵画[かいが]
\\	において
\\	も、 音楽[おんがく]
\\	において
\\	も、 彼女[かのじょ]より 才能[さいのう]に 恵[めぐ]まれた 人[ひと]はあまりいない。
\\	"絵画[かいが]
\\	も、 音楽[おんがく]
\\	も、 彼女[かのじょ]より 才能[さいのう]に 恵[めぐ]まれた 人[ひと]はあまりいない。
\\	において
\\	"彼[かれ]は 旅行[りょこう] 中[ちゅう]
\\	にして
\\	、 病[やまい]に 倒[たお]れた。
\\	"彼[かれ]は 旅行[りょこう] 中[ちゅう]
\\	、 病[やまい]に 倒[たお]れた。
\\	に 
\\	して, 
\\	する 
\\	にして
\\	"あのピアニストは 百歳[ひゃくさい]
\\	にして
\\	、 来月[らいげつ]またリサイタルをするそうです。
\\	"あのピアニストは 百歳[ひゃくさい]
\\	、 来月[らいげつ]またリサイタルをするそうです。
\\	に 
\\	して, 
\\	する 
\\	にして
\\	"子供[こども]を 叱[しか]る
\\	にしても
\\	、 自分[じぶん]の 感情[かんじょう]だけでしかってはいけない。
\\	"子供[こども]を 叱[しか]る
\\	、 自分[じぶん]の 感情[かんじょう]だけでしかってはいけない。
\\	に、して 
\\	する 
\\	も	
\\	にしても
\\	"今[いま]まで 充分[じゅうぶん] 練習[れんしゅう]したと 思[おも]っている
\\	にしても
\\	、 試合[しあい]の 日[ひ]まで 練習[れんしゅう]を 怠[なま]けていけない。
\\	"今[いま]まで 充分[じゅうぶん] 練習[れんしゅう]したと 思[おも]っている
\\	、 試合[しあい]の 日[ひ]まで 練習[れんしゅう]を 怠[なま]けていけない。
\\	に、して 
\\	する 
\\	も	
\\	にしても
\\	"君[きみ]
\\	にしては
\\	まあよく 歌[うた]えたよ。
\\	"君[きみ]
\\	まあよく 歌[うた]えたよ。
\\	に、して 
\\	する 
\\	は 
\\	にしては
\\	"冬[ふゆ]
\\	にしては
\\	、この 二三日[にさんにち] 割合[わりあい]に 暖[あたた]かいね。
\\	"冬[ふゆ]
\\	、この 二三日[にさんにち] 割合[わりあい]に 暖[あたた]かいね。
\\	に、して 
\\	する 
\\	は 
\\	にしては
\\	"次[つぎ] 物理[ぶつり] 学会[がっかい]は、 広島[ひろしま]
\\	にて
\\	開催[かいさい]されます。
\\	"次[つぎ] 物理[ぶつり] 学会[がっかい]は、 広島[ひろしま]
\\	開催[かいさい]されます。
\\	で. 
\\	これにて 
\\	ここにて 
\\	にて
\\	"今日[きょう]の 遠足[えんそく]はここ
\\	にて
\\	解散[かいさん]します。
\\	"今日[きょう]の 遠足[えんそく]はここ
\\	解散[かいさん]します。
\\	で. 
\\	これにて 
\\	ここにて 
\\	にて
\\	"今学期[こんがっき]はこれ
\\	にて
\\	おしまいにします。
\\	"今学期[こんがっき]はこれ
\\	おしまいにします。
\\	で. 
\\	これにて 
\\	ここにて 
\\	にて
\\	"昨日[きのう]は、 風邪[かぜ]
\\	にて
\\	欠席[けっせき]いたしました。
\\	"昨日[きのう]は、 風邪[かぜ]
\\	欠席[けっせき]いたしました。
\\	にて
\\	"私[わたくし]の 不注意[ふちゅうい]
\\	にて
\\	申[もう]し 訳[わけ]ありませんでした。
\\	"私[わたくし]の 不注意[ふちゅうい]
\\	申[もう]し 訳[わけ]ありませんでした。
\\	にて
\\	"あの 大雪[おおゆき]で、 私[わたし] 達[たち]は 歩[ある]くことさえ 出来[でき]なかった
\\	のだ
\\	"あの 大雪[おおゆき]で、 私[わたし] 達[たち]は 歩[ある]くことさえ 出来[でき]なかった
\\	の 
\\	だ. 
\\	んだ. 
\\	のです. 
\\	のだった	
\\	のだ
\\	"長年[ながねん] 外国[がいこく]に 暮[く]らしていると、 益々[ますます] 日本食[にほんしょく]が 恋[こい]しくなる
\\	のだ
\\	"長年[ながねん] 外国[がいこく]に 暮[く]らしていると、 益々[ますます] 日本食[にほんしょく]が 恋[こい]しくなる
\\	の 
\\	だ. 
\\	んだ. 
\\	のです. 
\\	のだった	
\\	のだ
\\	"僕[ぼく]は 今度[こんど]こそ 弁論[べんろん] 大会[たいかい]で 優勝[ゆうしょう]する
\\	のだ
\\	"僕[ぼく]は 今度[こんど]こそ 弁論[べんろん] 大会[たいかい]で 優勝[ゆうしょう]する
\\	のだ
\\	"この 美術[びじゅつ] 展[てん]は 見逃[みのが]せない。どんな 事[こと]があっても 絶対[ぜったい]に 行[い]く
\\	んだ
\\	"この 美術[びじゅつ] 展[てん]は 見逃[みのが]せない。どんな 事[こと]があっても 絶対[ぜったい]に 行[い]く
\\	んだ
\\	"勉強[べんきょう]しないで 遊[あそ]んでばかりいたから、 行[い]きたい 学校[がっこう]に 入[はい]れなかった
\\	のだ
\\	"勉強[べんきょう]しないで 遊[あそ]んでばかりいたから、 行[い]きたい 学校[がっこう]に 入[はい]れなかった
\\	のだ
\\	"彼[かれ]は 忙[いそが]しいのに、 長[なが]い 手紙[てがみ]をくれた
\\	んだ
\\	。すぐに 返事[へんじ]を 出[だ]さなくては。
\\	"彼[かれ]は 忙[いそが]しいのに、 長[なが]い 手紙[てがみ]をくれた
\\	。すぐに 返事[へんじ]を 出[だ]さなくては。
\\	んだ
\\	"今[いま] 我々[われわれ]にとって 一番[いちばん] 必要[ひつよう]なのは、 公害[こうがい] 問題[もんだい]の 解決[かいげつ]な
\\	のだろう
\\	"今[いま] 我々[われわれ]にとって 一番[いちばん] 必要[ひつよう]なのは、 公害[こうがい] 問題[もんだい]の 解決[かいげつ]な
\\	の 
\\	だろう. 
\\	んだろう	
\\	のだろう
\\	"どうした
\\	んだろう
\\	、もう 三十分[さんじゅっぷん]にもなるのに、 彼女[かのじょ]はまだ 戻[もど]ってこない。
\\	"どうした
\\	、もう 三十分[さんじゅっぷん]にもなるのに、 彼女[かのじょ]はまだ 戻[もど]ってこない。
\\	の 
\\	だろう. 
\\	んだろう	
\\	んだろう
\\	から	"赤[あか]ちゃんが 生[う]まれた
\\	ので(んで)
\\	、みんなで 大喜[おおよろ]びしています。
\\	"赤[あか]ちゃんが 生[う]まれた
\\	、みんなで 大喜[おおよろ]びしています。
\\	ので 
\\	から 
\\	ので 
\\	から 
\\	ので(んで)
\\	から	"その 料理[りょうり]は、 思[おも]ったより 美味[おい]しくなかった
\\	ので(んで)
\\	、 一口[ひとくち]しか 食[た]べませんでした。
\\	"その 料理[りょうり]は、 思[おも]ったより 美味[おい]しくなかった
\\	、 一口[ひとくち]しか 食[た]べませんでした。
\\	ので 
\\	から 
\\	ので 
\\	から 
\\	ので(んで)
\\	から	"彼[かれ]が 美術館[びじゅつかん]に 行[い]きたい
\\	ので(んで)
\\	、 案内[あんない]してあげました。
\\	"彼[かれ]が 美術館[びじゅつかん]に 行[い]きたい
\\	、 案内[あんない]してあげました。
\\	と言うので 
\\	あ(ん)まり~ので	
\\	ので(んで)
\\	から	"あんまり 寒[さむ]い
\\	ので(んで)
\\	、ストーブを 点[つ]けました。
\\	"あんまり 寒[さむ]い
\\	、ストーブを 点[つ]けました。
\\	と言うので 
\\	あ(ん)まり~ので	
\\	ので(んで)
\\	から	"彼[かれ]のアパートへ 行[い]ったの、どうしても 来[き]てもらいたいと 言[い]う
\\	ので(んで)
\\	"彼[かれ]のアパートへ 行[い]ったの、どうしても 来[き]てもらいたいと 言[い]う
\\	ので(んで)
\\	から	"もう 寝[ね]るよ、 明日[あした]の 朝[あさ]が 早[はや]い
\\	ので(んで)
\\	"もう 寝[ね]るよ、 明日[あした]の 朝[あさ]が 早[はや]い
\\	ので(んで)
\\	"物[もの]が 落[お]ちるのは、 地球[ちきゅう]に 引力[いんりょく]があるからな
\\	のである
\\	"物[もの]が 落[お]ちるのは、 地球[ちきゅう]に 引力[いんりょく]があるからな
\\	のである
\\	"昔[むかし]、 恐竜[きょうりゅう]が 住[す]んでいたが 分[わ]かったのは、 化石[かせき]が 発見[はっけん]されたからな
\\	のである
\\	"昔[むかし]、 恐竜[きょうりゅう]が 住[す]んでいたが 分[わ]かったのは、 化石[かせき]が 発見[はっけん]されたからな
\\	のである
\\	"今日[きょう]はお 盆[ぼん]だから、 店[みせ]が 開[あ]いていない
\\	のでしょう
\\	か。
\\	"今日[きょう]はお 盆[ぼん]だから、 店[みせ]が 開[あ]いていない
\\	か。
\\	の 
\\	でしょう 
\\	んでしょう	
\\	のでしょう
\\	"あなたは、 私[わたし]に 会[あ]いたくて 来[き]た
\\	んでしょう
\\	"あなたは、 私[わたし]に 会[あ]いたくて 来[き]た
\\	の 
\\	でしょう 
\\	んでしょう	
\\	のでしょう
\\	"子供[こども] 達[たち]を 守[まも]るのは、 私[わたし] 達[たち] 大人[おとな]の 責任[せきにん]な
\\	の(ん)です
\\	"子供[こども] 達[たち]を 守[まも]るのは、 私[わたし] 達[たち] 大人[おとな]の 責任[せきにん]な
\\	の 
\\	です. 
\\	のだ 
\\	んです	
\\	の(ん)です
\\	"昨夜[ゆうべ]お 酒[さけ]を 飲[の]みすぎたせいか 頭[あたま]が 痛[いた]い
\\	の(ん)です
\\	"昨夜[ゆうべ]お 酒[さけ]を 飲[の]みすぎたせいか 頭[あたま]が 痛[いた]い
\\	の 
\\	です. 
\\	のだ 
\\	んです	
\\	の(ん)です
\\	"あなたは、いつになったら 貸[か]したお 金[かね]を 返[かえ]してくれる
\\	の(ん)です
\\	"あなたは、いつになったら 貸[か]したお 金[かね]を 返[かえ]してくれる
\\	の(ん)です
\\	"大事[だいじ]な 壷[つぼ]を 壊[こわ]してしまって、どうしてくれる
\\	んです
\\	か。
\\	"大事[だいじ]な 壷[つぼ]を 壊[こわ]してしまって、どうしてくれる
\\	か。
\\	んです
\\	"感動的[かんどうてき]なスピーチに、 聴衆[ちょうしゅう]は 心[こころ]から 拍手[はくしゅ]を 送[おく]った
\\	のでした
\\	"感動的[かんどうてき]なスピーチに、 聴衆[ちょうしゅう]は 心[こころ]から 拍手[はくしゅ]を 送[おく]った
\\	のでした
\\	"かわいそうに、 主人[しゅじん]が 死[し]んだことも 知[し]らずに、その 犬[いぬ]は 毎日[まいにち] 駅[えき]まで 迎[むか]えに 行[い]っていた
\\	のでした
\\	"かわいそうに、 主人[しゅじん]が 死[し]んだことも 知[し]らずに、その 犬[いぬ]は 毎日[まいにち] 駅[えき]まで 迎[むか]えに 行[い]っていた
\\	のでした
\\	だけ 
\\	ばかり 
\\	"この 村[むら]
\\	のみ
\\	が、 水害[すいがい]に 遭[あ]わなかった。
\\	"この 村[むら]
\\	が、 水害[すいがい]に 遭[あ]わなかった。
\\	のみ
\\	だけ 
\\	ばかり 
\\	"一度[いちど]や 二度[にど]のテスト
\\	のみ
\\	で、 学生[がくせい]を 評価[ひょうか]してはいけない。
\\	"一度[いちど]や 二度[にど]のテスト
\\	で、 学生[がくせい]を 評価[ひょうか]してはいけない。
\\	のみ
\\	だけ 
\\	ばかり 
\\	"この 水族館[すいぞくかん]には、 深海[しんかい] 魚[ぎょ]
\\	のみ
\\	ならず、 熱帯[ねったい] 魚[ぎょ]もいる。
\\	"この 水族館[すいぞくかん]には、 深海[しんかい] 魚[ぎょ]
\\	ならず、 熱帯[ねったい] 魚[ぎょ]もいる。
\\	のみならず	
\\	のみ
\\	だけ 
\\	ばかり 
\\	"彼女[かのじょ]は 教授[きょうじゅ]として
\\	のみ
\\	ならず、 作家[さっか]としても 活躍[かつやく]している。
\\	"彼女[かのじょ]は 教授[きょうじゅ]として
\\	ならず、 作家[さっか]としても 活躍[かつやく]している。
\\	のみならず	
\\	のみ
\\	"どうか 父[ちち]が 生[い]きて 帰[かえ]って 来[く]るようにとただ 祈[いの]る
\\	のみ
\\	"どうか 父[ちち]が 生[い]きて 帰[かえ]って 来[く]るようにとただ 祈[いの]る
\\	のみ
\\	"水不足[みずぶそく]はいよいよ 深刻[しんこく]だ。あとはただ 雨[あめ]が 降[ふ]るのを 待[ま]つ
\\	のみ
\\	だ。
\\	"水不足[みずぶそく]はいよいよ 深刻[しんこく]だ。あとはただ 雨[あめ]が 降[ふ]るのを 待[ま]つ
\\	だ。
\\	のみ
\\	"日本語[にほんご]がもっと 上手[じょうず]に 話[はな]せるようになる
\\	のに
\\	は、どうしたらいいですか。
\\	"日本語[にほんご]がもっと 上手[じょうず]に 話[はな]せるようになる
\\	は、どうしたらいいですか。
\\	の 
\\	に	
\\	のに
\\	"山歩[やまある]きをする
\\	のに
\\	は、 虫除[むしよ]けを 持[も]って 行[い]ったほうがいいよ。
\\	"山歩[やまある]きをする
\\	は、 虫除[むしよ]けを 持[も]って 行[い]ったほうがいいよ。
\\	の 
\\	に	
\\	のに
\\	"この 箱[はこ]、 本[ほん]を 入[い]れる
\\	のに
\\	丁度[ちょうど]いいね。
\\	"この 箱[はこ]、 本[ほん]を 入[い]れる
\\	丁度[ちょうど]いいね。
\\	のに
\\	"この 貯金[ちょきん]、 旅行[りょこう]する
\\	のに
\\	少[すこ]し 足[た]りないわね。
\\	"この 貯金[ちょきん]、 旅行[りょこう]する
\\	少[すこ]し 足[た]りないわね。
\\	のに
\\	"昨夜[ゆうべ] 徹夜[てつや]したんだもの、 眠[ねむ]いのは 当[あ]たり 前[まえ]
\\	さ
\\	"昨夜[ゆうべ] 徹夜[てつや]したんだもの、 眠[ねむ]いのは 当[あ]たり 前[まえ]
\\	さ
\\	"頭痛[ずつう]がひどいって? 昨夜[ゆうべ]あんなに 飲[の]んだんだもの。 罰[ばち]が 当[あ]たったの
\\	さ
\\	"頭痛[ずつう]がひどいって? 昨夜[ゆうべ]あんなに 飲[の]んだんだもの。 罰[ばち]が 当[あ]たったの
\\	さ
\\	"じゃあ、 誰[だれ]に 聞[き]いたらいいの
\\	さ
\\	"じゃあ、 誰[だれ]に 聞[き]いたらいいの
\\	さ
\\	"なに
\\	さ
\\	、 返事[へんじ]もしないで。
\\	"なに
\\	、 返事[へんじ]もしないで。
\\	さ
\\	"太田[おおた] 君[くん]、 来月[らいげつ]ロサンゼルスに 転勤[てんきん]だって
\\	さ
\\	"太田[おおた] 君[くん]、 来月[らいげつ]ロサンゼルスに 転勤[てんきん]だって
\\	てさ 
\\	とさ	
\\	さ
\\	"それから 王子[おおじ] 様[さま]と 王女[おおじょ] 様[さま]は 幸[しあわ]せに 暮[く]らしましたと
\\	さ
\\	"それから 王子[おおじ] 様[さま]と 王女[おおじょ] 様[さま]は 幸[しあわ]せに 暮[く]らしましたと
\\	てさ 
\\	とさ	
\\	さ
\\	"この 魚[さかな]は、 猫[ねこ]
\\	さえ
\\	見向[みむ]きもしないよ。
\\	"この 魚[さかな]は、 猫[ねこ]
\\	見向[みむ]きもしないよ。
\\	さえ~(だ)から 
\\	さえ
\\	"たった 一[ひと]つの 間違[まちが]いで
\\	さえ
\\	ずいぶん 差[さ]が 出[で]てしまう。
\\	"たった 一[ひと]つの 間違[まちが]いで
\\	ずいぶん 差[さ]が 出[で]てしまう。
\\	さえ~(だ)から 
\\	さえ
\\	"時間[じかん]
\\	さえ
\\	あれば、 歌舞伎[かぶき]も 見[み]たいのですが。
\\	"時間[じかん]
\\	あれば、 歌舞伎[かぶき]も 見[み]たいのですが。
\\	さえ
\\	"お宅[たく] 様[さま]
\\	さえ
\\	宜[よろ]しければ、 今晩[こんばん] 家内[かない]とお 邪魔[じゃま]したいと 思[おも]うのですが。
\\	"お宅[たく] 様[さま]
\\	宜[よろ]しければ、 今晩[こんばん] 家内[かない]とお 邪魔[じゃま]したいと 思[おも]うのですが。
\\	さえ
\\	"どうしたのかしら、この 猫[ねこ]。お 魚[さかな]ばかりじゃなく、 大好物[だいこうぶつ]のミルク
\\	さえ
\\	飲[の]まないのよ。
\\	"どうしたのかしら、この 猫[ねこ]。お 魚[さかな]ばかりじゃなく、 大好物[だいこうぶつ]のミルク
\\	飲[の]まないのよ。
\\	~ばかりでなく~さえ	
\\	さえ
\\	"昔[むかし]と 違[ちが]って、 今[いま]は、 夏[なつ]ばかりでなく、 冬[ふゆ]で
\\	さえ
\\	茄子[なす]や 胡瓜[きゅうり]が 食[た]べられます。
\\	"昔[むかし]と 違[ちが]って、 今[いま]は、 夏[なつ]ばかりでなく、 冬[ふゆ]で
\\	茄子[なす]や 胡瓜[きゅうり]が 食[た]べられます。
\\	~ばかりでなく~さえ	
\\	さえ
\\	"職[しょく]は 失[うしな]う
\\	し
\\	、 妻[つま]とは 別[わか]れる
\\	し
\\	、 彼[かれ]はひどく 元気[げんき]がない。
\\	"職[しょく]は 失[うしな]う
\\	、 妻[つま]とは 別[わか]れる
\\	、 彼[かれ]はひどく 元気[げんき]がない。
\\	~し~し 
\\	し
\\	"地震[じしん]でうちは 壊[こわ]れる
\\	し
\\	、 洪水[こいずい]で 畑[はたけ]は 駄目[だめ]になる
\\	し
\\	、どうしたらいいのか 分[わ]からない。
\\	"地震[じしん]でうちは 壊[こわ]れる
\\	、 洪水[こいずい]で 畑[はたけ]は 駄目[だめ]になる
\\	、どうしたらいいのか 分[わ]からない。
\\	~し~し 
\\	し
\\	"風[かぜ]は 強[つよ]い
\\	し
\\	、 雪[ゆき]は 降[ふ]り 出[だ]した
\\	し
\\	、 今日[きょう]は 出[で]かけるのはやめましょうよ。
\\	"風[かぜ]は 強[つよ]い
\\	、 雪[ゆき]は 降[ふ]り 出[だ]した
\\	、 今日[きょう]は 出[で]かけるのはやめましょうよ。
\\	し
\\	"もうこんな 時間[じかん]だ
\\	し
\\	、 明日[あした]にしようよ。
\\	"もうこんな 時間[じかん]だ
\\	、 明日[あした]にしようよ。
\\	し
\\	"新[あたら]しい 洋服[ようふく]は 欲[ほ]しい
\\	し
\\	、お 金[かね]はない
\\	し
\\	、 学生[がくせい]の 懐[ふところ]は 寂[さび]しいなあ。
\\	"新[あたら]しい 洋服[ようふく]は 欲[ほ]しい
\\	、お 金[かね]はない
\\	、 学生[がくせい]の 懐[ふところ]は 寂[さび]しいなあ。
\\	し
\\	"歯[は]は 痛[いた]い
\\	し
\\	、 歯医者[はいしゃ]には 行[い]きたくない
\\	し
\\	、 困[こま]ったよ。
\\	"歯[は]は 痛[いた]い
\\	、 歯医者[はいしゃ]には 行[い]きたくない
\\	、 困[こま]ったよ。
\\	し
\\	あるまい_	"夏[なつ]でもあるまい
\\	し
\\	、 薄[うす]いシャツ 一枚[いちまい]で。 寒[さむ]くないの?
\\	"夏[なつ]でもあるまい
\\	、 薄[うす]いシャツ 一枚[いちまい]で。 寒[さむ]くないの?
\\	し
\\	あるまい_	"大金持[おおがねも]ちでもあるまい
\\	し
\\	、あんなに 贅沢[ぜいたく]な 生活[せいかつ]をしていてお 金[かね]が 続[つづ]くのかしら?
\\	"大金持[おおがねも]ちでもあるまい
\\	、あんなに 贅沢[ぜいたく]な 生活[せいかつ]をしていてお 金[かね]が 続[つづ]くのかしら?
\\	し
\\	まだ_、おり_; 
\\	"英語[えいご]ならまだ
\\	しも
\\	、ドイツ 語[ご]なんて 全然[ぜんぜん] 分[わ]からないよ。
\\	"英語[えいご]ならまだ
\\	、ドイツ 語[ご]なんて 全然[ぜんぜん] 分[わ]からないよ。
\\	しも
\\	まだ_、おり_; 
\\	"夕立[ゆうだち]がやんで、 折[おり]
\\	しも
\\	空[そら]に 美[うつく]しい 虹[にじ]がかかった。
\\	"夕立[ゆうだち]がやんで、 折[おり]
\\	空[そら]に 美[うつく]しい 虹[にじ]がかかった。
\\	しも
\\	誰_ 
\\	"それは 誰[だれ]
\\	しも
\\	考[かんが]え 付[つ]くことだ。
\\	"それは 誰[だれ]
\\	考[かんが]え 付[つ]くことだ。
\\	しも
\\	誰_ 
\\	"誰[だれ]
\\	しも
\\	死[し]ぬのは 恐[こわ]い。
\\	"誰[だれ]
\\	死[し]ぬのは 恐[こわ]い。
\\	しも
\\	必ずし_, 
\\	"いい 大学[だいがく]に 入[はい]ったからといって、 必[かなら]ず
\\	しも
\\	いい 職[しょく]が 見[み]つかるとは 限[かぎ]らない。
\\	"いい 大学[だいがく]に 入[はい]ったからといって、 必[かなら]ず
\\	いい 職[しょく]が 見[み]つかるとは 限[かぎ]らない。
\\	しも
\\	必ずし_, 
\\	"昨日[きのう]まで 暖[あたた]かい 日[ひ]が 何日[なんにち] も続[つづ]いたから、 今日[きょう]も 暖[あたた]かいとは 必[かなら]ず
\\	しも
\\	言[い]えないよ。
\\	"昨日[きのう]まで 暖[あたた]かい 日[ひ]が 何日[なんにち] も続[つづ]いたから、 今日[きょう]も 暖[あたた]かいとは 必[かなら]ず
\\	言[い]えないよ。
\\	しも
\\	"世界[せかい]には、 三度[さんど]の 食事[しょくじ]
\\	すら
\\	満足[まんぞく]に 食[た]べられない 人[ひと]が 沢山[たくさん]いるのだよ。
\\	"世界[せかい]には、 三度[さんど]の 食事[しょくじ]
\\	満足[まんぞく]に 食[た]べられない 人[ひと]が 沢山[たくさん]いるのだよ。
\\	すら
\\	"あの 老人[ろうじん]は、もう 自分[じぶん]の 娘[むすめ]
\\	すら
\\	見分[みわ]けがつかなくなってしまった。
\\	"あの 老人[ろうじん]は、もう 自分[じぶん]の 娘[むすめ]
\\	見分[みわ]けがつかなくなってしまった。
\\	すら
\\	"熊[くま]はおりの 中[なか]で、 行[い]っ
\\	たり
\\	来[き]
\\	たり
\\	していた。
\\	"熊[くま]はおりの 中[なか]で、 行[い]っ
\\	来[き]
\\	していた。
\\	~たり~たり 
\\	たり
\\	"少女[しょうじょ] 達[たち]が、 飛[と]ん
\\	だり
\\	跳[は]ね
\\	たり
\\	して 遊[あそ]んでいる。
\\	"少女[しょうじょ] 達[たち]が、 飛[と]ん
\\	跳[は]ね
\\	して 遊[あそ]んでいる。
\\	~たり~たり 
\\	たり
\\	"こんなに 暗[くら]くなるまで 外[そと]で 遊[あそ]んでい
\\	たり
\\	しては、 危[あぶ]ないよ。
\\	"こんなに 暗[くら]くなるまで 外[そと]で 遊[あそ]んでい
\\	しては、 危[あぶ]ないよ。
\\	たり
\\	"そんな 所[ところ]にゴミを 捨[す]て
\\	たり
\\	してはいけません。
\\	"そんな 所[ところ]にゴミを 捨[す]て
\\	してはいけません。
\\	たり
\\	"今頃[いまごろ] 旅行[りょこう]に 行[い]こうっ
\\	たって
\\	、どこも 満員[まんいん]だよ。
\\	"今頃[いまごろ] 旅行[りょこう]に 行[い]こうっ
\\	、どこも 満員[まんいん]だよ。
\\	たって
\\	"帰[かえ]りたいと 思[おも]っ
\\	たって
\\	、もうバスも 電車[でんしゃ]もないよ。
\\	"帰[かえ]りたいと 思[おも]っ
\\	、もうバスも 電車[でんしゃ]もないよ。
\\	たって
\\	"少[すこ]しぐらい 貯金[ちょきん]があるっ
\\	たって
\\	、いざという 時[とき]には 役[やく]に 立[た]たないね。
\\	"少[すこ]しぐらい 貯金[ちょきん]があるっ
\\	、いざという 時[とき]には 役[やく]に 立[た]たないね。
\\	ったって	
\\	たって
\\	"飲[の]むっ
\\	たって
\\	、 杯[さかづき]にほんの 四杯[よんはい]か 五杯[ごはい]だけだよ。
\\	"飲[の]むっ
\\	、 杯[さかづき]にほんの 四杯[よんはい]か 五杯[ごはい]だけだよ。
\\	ったって	
\\	たって
\\	"お 客様[きゃくさま]ですよ。きちんと 座[すわ]っ
\\	て
\\	ご 挨拶[あいさつ]なさい。
\\	"お 客様[きゃくさま]ですよ。きちんと 座[すわ]っ
\\	ご 挨拶[あいさつ]なさい。
\\	って 
\\	て
\\	"桜[さくら]の 花[はな]が 散[ち]っ
\\	て
\\	、 若葉[わかば]が 出始[ではじ]めた。
\\	"桜[さくら]の 花[はな]が 散[ち]っ
\\	、 若葉[わかば]が 出始[ではじ]めた。
\\	って 
\\	て
\\	"あの 子[こ]は、 先生[せんせい]に 叱[しか]られ
\\	て
\\	泣[な]いている。
\\	"あの 子[こ]は、 先生[せんせい]に 叱[しか]られ
\\	泣[な]いている。
\\	て
\\	"初[はじ]めて 五十[ごじゅう]メートル 泳[およ]げた。お 母[さあ]さんに 褒[ほ]められ
\\	て
\\	、とても 嬉[うれ]しかった。
\\	"初[はじ]めて 五十[ごじゅう]メートル 泳[およ]げた。お 母[さあ]さんに 褒[ほ]められ
\\	、とても 嬉[うれ]しかった。
\\	て
\\	"私[わたし]は、コーヒーにお 砂糖[さとう]とミルクを 入[い]れ
\\	て
\\	飲[の]みます。
\\	"私[わたし]は、コーヒーにお 砂糖[さとう]とミルクを 入[い]れ
\\	飲[の]みます。
\\	て
\\	"遠回[とおまわ]りをし
\\	て
\\	行[い]こう。
\\	"遠回[とおまわ]りをし
\\	行[い]こう。
\\	て
\\	"彼女[かのじょ]は、 優[やさ]しく
\\	て
\\	気[き]さくで、 誰[だれ]にでも 好[す]かれる。
\\	"彼女[かのじょ]は、 優[やさ]しく
\\	気[き]さくで、 誰[だれ]にでも 好[す]かれる。
\\	て
\\	"このみかんは、 新[あたら]しく
\\	て
\\	安[やす]く
\\	て
\\	美味[おい]しい。
\\	"このみかんは、 新[あたら]しく
\\	安[やす]く
\\	美味[おい]しい。
\\	て
\\	"原稿[げんこう]の 締[し]め 切[き]り 日[ひ]が 近付[ちかづ]いて、 彼[かれ]は 書[か]い
\\	て
\\	書[か]い
\\	て
\\	書[か]きまくった。
\\	"原稿[げんこう]の 締[し]め 切[き]り 日[ひ]が 近付[ちかづ]いて、 彼[かれ]は 書[か]い
\\	書[か]い
\\	書[か]きまくった。
\\	~(っ)て~(っ)て~まくる	
\\	て
\\	"引[ひき] 出[だ]しに 入[い]れておいた 一万[いちまん] 円[えん]が 見[み]つからない。 僕[ぼく]は 探[さが]し
\\	て
\\	探[さが]し
\\	て
\\	探[さが]しまくった。
\\	"引[ひき] 出[だ]しに 入[い]れておいた 一万[いちまん] 円[えん]が 見[み]つからない。 僕[ぼく]は 探[さが]し
\\	探[さが]し
\\	探[さが]しまくった。
\\	~(っ)て~(っ)て~まくる	
\\	て
\\	"君[きみ]、その 噂[うわさ]を 知[し]ってい
\\	て
\\	知[し]らないふりをするなんて、なぜなんだい?
\\	"君[きみ]、その 噂[うわさ]を 知[し]ってい
\\	知[し]らないふりをするなんて、なぜなんだい?
\\	て
\\	"結婚[けっこん]の 約束[よくそく]までしておい
\\	て
\\	、 彼女[かのじょ]を 見捨[みす]てるとは、あんまりじゃないか。
\\	"結婚[けっこん]の 約束[よくそく]までしておい
\\	、 彼女[かのじょ]を 見捨[みす]てるとは、あんまりじゃないか。
\\	て
\\	"本日[ほんじつ]は、 日米[にちべい] 貿易[ぼうえき]につい
\\	て
\\	討論[とうろん]をいたします。
\\	"本日[ほんじつ]は、 日米[にちべい] 貿易[ぼうえき]につい
\\	討論[とうろん]をいたします。
\\	について	
\\	て
\\	"火山[かざん]がまた 活動[かつどう]を 始[はじ]めました。 道路[どうろ] 情報[じょうほう]につい
\\	て
\\	はラジオのニュースをお 聞[き]き 下[くだ]さい。
\\	"火山[かざん]がまた 活動[かつどう]を 始[はじ]めました。 道路[どうろ] 情報[じょうほう]につい
\\	はラジオのニュースをお 聞[き]き 下[くだ]さい。
\\	について	
\\	て
\\	"私[わたし]は 今[いま] 顔[かお]を 洗[あら]っ
\\	て
\\	います。
\\	"私[わたし]は 今[いま] 顔[かお]を 洗[あら]っ
\\	います。
\\	て 
\\	て
\\	"私[わたし]は 宿題[しゅくだい]の 論文[ろんぶん]を 書[か]い
\\	て
\\	しまいました。
\\	"私[わたし]は 宿題[しゅくだい]の 論文[ろんぶん]を 書[か]い
\\	しまいました。
\\	て 
\\	て
\\	"あら、このスカート、ぴったりあっ
\\	て
\\	よ。
\\	"あら、このスカート、ぴったりあっ
\\	よ。
\\	~(っ)てよ	
\\	て
\\	"あの 方[かた]、そのお 話[はなし]ならもう 知[し]っていらし
\\	て
\\	よ。
\\	"あの 方[かた]、そのお 話[はなし]ならもう 知[し]っていらし
\\	よ。
\\	~(っ)てよ	
\\	て
\\	"あなた、 私[わたし]の 言[い]うこと 分[わ]かっ
\\	て
\\	"あなた、 私[わたし]の 言[い]うこと 分[わ]かっ
\\	て
\\	"そんな 高[たか]いとこに 登[のぼ]れ
\\	て
\\	"そんな 高[たか]いとこに 登[のぼ]れ
\\	て
\\	"お 母[かあ]さん、 珍[めずら]しいチョウチョがいるよ。 早[はや]く 来[き]
\\	て
\\	"お 母[かあ]さん、 珍[めずら]しいチョウチョがいるよ。 早[はや]く 来[き]
\\	てよ 
\\	てね 
\\	て
\\	"早[はや]く 食[た]べ
\\	て
\\	よ。 急[いそ]いでるんだから。
\\	"早[はや]く 食[た]べ
\\	よ。 急[いそ]いでるんだから。
\\	てよ 
\\	てね 
\\	て
\\	"君[きみ]の 病気[びょうき]は 必[かなら]ず 治[なお]るっ
\\	て
\\	"君[きみ]の 病気[びょうき]は 必[かなら]ず 治[なお]るっ
\\	て
\\	"あなたの 優[やさ]しさを、あいつもきっと 分[わ]かるっ
\\	て
\\	"あなたの 優[やさ]しさを、あいつもきっと 分[わ]かるっ
\\	て
\\	と, 
\\	"ここはなん
\\	て
\\	いう 町[まち]だろう。
\\	"ここはなん
\\	いう 町[まち]だろう。
\\	て
\\	と, 
\\	"お 兄[にい]ちゃんは、そんなことは 出来[でき]ん
\\	て
\\	言[い]ってたよ。
\\	"お 兄[にい]ちゃんは、そんなことは 出来[でき]ん
\\	言[い]ってたよ。
\\	て
\\	"木村[きむら] 君[くん]
\\	でば
\\	、インフルエンザでまだ 学校[がっこう] 休[やす]んでるんだよ。
\\	"木村[きむら] 君[くん]
\\	、インフルエンザでまだ 学校[がっこう] 休[やす]んでるんだよ。
\\	てば
\\	"おじいちゃん
\\	てば
\\	、またテレビの 前[まえ]で 居眠[いねむ]りしているよ。
\\	"おじいちゃん
\\	、またテレビの 前[まえ]で 居眠[いねむ]りしているよ。
\\	てば
\\	"お 夕飯[ゆうはん]できたわよ。 熱[あつ]いうちに 食[た]べないと 冷[さ]めちゃうっ
\\	てば
\\	"お 夕飯[ゆうはん]できたわよ。 熱[あつ]いうちに 食[た]べないと 冷[さ]めちゃうっ
\\	てば
\\	"分[わ]かったよ。すぐ 行[い]くっ
\\	てば
\\	"分[わ]かったよ。すぐ 行[い]くっ
\\	てば
\\	"どんなに 高[たか]く
\\	ても
\\	その 絵[え]を 買[か]いたい。
\\	"どんなに 高[たか]く
\\	その 絵[え]を 買[か]いたい。
\\	ても 
\\	ん 
\\	でも 
\\	ても
\\	"今[いま]はどんなに 悲[かな]しく
\\	ても
\\	、いつかは 乗[の]り 越[こ]えられるよ。
\\	"今[いま]はどんなに 悲[かな]しく
\\	、いつかは 乗[の]り 越[こ]えられるよ。
\\	ても 
\\	ん 
\\	でも 
\\	ても
\\	"この 単語[たんご]、なんべん 辞書[じしょ]をひい
\\	ても
\\	すぐ 意味[いま]を 忘[わす]れてしまう。
\\	"この 単語[たんご]、なんべん 辞書[じしょ]をひい
\\	すぐ 意味[いま]を 忘[わす]れてしまう。
\\	ても
\\	"じっと 寝[ね]てい
\\	ても
\\	、 腰[こし]の 痛[いた]いのが 治[なお]らない。
\\	"じっと 寝[ね]てい
\\	、 腰[こし]の 痛[いた]いのが 治[なお]らない。
\\	ても
\\	"この 小包[こづつみ]は 重[おも]く
\\	ても
\\	五[ご]キロはないでしょう。
\\	"この 小包[こづつみ]は 重[おも]く
\\	五[ご]キロはないでしょう。
\\	ても
\\	"その 本[ほん]は 安[やす]く
\\	ても
\\	五千[ごせん] 円[えん]はするでしょう。
\\	"その 本[ほん]は 安[やす]く
\\	五千[ごせん] 円[えん]はするでしょう。
\\	ても
\\	"もうお 乳[ちち]を 飲[の]ませたから、 赤[あか]ちゃんを 抱[だ]いて 下[くだ]さっ
\\	ても
\\	いいですよ。
\\	"もうお 乳[ちち]を 飲[の]ませたから、 赤[あか]ちゃんを 抱[だ]いて 下[くだ]さっ
\\	いいですよ。
\\	ても
\\	"この 薬[くすり]なら、ミルクを 一緒[いっしょ]に 飲[の]ん
\\	でも
\\	構[かま]いませんよ。
\\	"この 薬[くすり]なら、ミルクを 一緒[いっしょ]に 飲[の]ん
\\	構[かま]いませんよ。
\\	でも
\\	下さい	"お 母[かあ]さん、 明日[あした]の 遠足[えんそく]、お 握[にぎ]り 作[つく]っ
\\	てね
\\	"お 母[かあ]さん、 明日[あした]の 遠足[えんそく]、お 握[にぎ]り 作[つく]っ
\\	てね
\\	下さい	"今夜[こんや]は 早[はや]く 帰[かえ]ってき
\\	てね
\\	、 真理子[まりこ]の 誕生日[たんじょうび]だから。
\\	"今夜[こんや]は 早[はや]く 帰[かえ]ってき
\\	、 真理子[まりこ]の 誕生日[たんじょうび]だから。
\\	てね
\\	"ああ、 三井[みつい] 君[くん]
\\	といえば
\\	、さっき 廊下[ろうか]で 会[あ]ったよ。
\\	"ああ、 三井[みつい] 君[くん]
\\	、さっき 廊下[ろうか]で 会[あ]ったよ。
\\	と、ば、
\\	いえ, 
\\	いう	
\\	といえば
\\	"滝[たき]
\\	といえば
\\	、なんといってもナイアガラだね。
\\	"滝[たき]
\\	、なんといってもナイアガラだね。
\\	と、ば、
\\	いえ, 
\\	いう	
\\	といえば
\\	"あの 店[みせ]のステーキは、 味[あじ]
\\	といい
\\	値段[ねだん]
\\	といい
\\	いうことないね。
\\	"あの 店[みせ]のステーキは、 味[あじ]
\\	値段[ねだん]
\\	いうことないね。
\\	と 
\\	いい, 
\\	いう. 
\\	~といい~といい	
\\	といい
\\	"長谷川[はせがわ]さんは、 勉強[べんきょう]
\\	といい
\\	スポーツ
\\	といい
\\	いつでもトップなのよ。
\\	"長谷川[はせがわ]さんは、 勉強[べんきょう]
\\	スポーツ
\\	いつでもトップなのよ。
\\	と 
\\	いい, 
\\	いう. 
\\	~といい~といい	
\\	といい
\\	"日本[にほん]のお 土産[みやげ]んら、 真珠[しんじゅ]や 七宝[しっぽう]
\\	といった
\\	アクセサリーもいいでしょう。
\\	"日本[にほん]のお 土産[みやげ]んら、 真珠[しんじゅ]や 七宝[しっぽう]
\\	アクセサリーもいいでしょう。
\\	と 
\\	いった 
\\	いう	
\\	といった
\\	"うちの 娘[むすめ]は、 数学[すうがく]や 理科[りか]
\\	といった
\\	科目[かもく]が 得意[とくい]です。
\\	"うちの 娘[むすめ]は、 数学[すうがく]や 理科[りか]
\\	科目[かもく]が 得意[とくい]です。
\\	と 
\\	いった 
\\	いう	
\\	といった
\\	"急[いそ]げば 十時[じゅうじ]のバスに 間[ま]に 合[あ]いそうだ。
\\	といって
\\	走[はし]るのはいやだ。 次[つき]のバスにしよう。
\\	"急[いそ]げば 十時[じゅうじ]のバスに 間[ま]に 合[あ]いそうだ。
\\	走[はし]るのはいやだ。 次[つき]のバスにしよう。
\\	と 
\\	いって 
\\	いう	
\\	といって
\\	"日本[にほん]は、 確[たし]かに 経済[けいざい] 大国[たいこく]になった。
\\	といって
\\	、 国民[こくみん] 一人[ひとり] 一人[ひとり]の 暮[く]らしが 豊[ゆた]かになったわけではない。
\\	"日本[にほん]は、 確[たし]かに 経済[けいざい] 大国[たいこく]になった。
\\	、 国民[こくみん] 一人[ひとり] 一人[ひとり]の 暮[く]らしが 豊[ゆた]かになったわけではない。
\\	と 
\\	いって 
\\	いう	
\\	といって
\\	"もうヘトヘトだなど
\\	といわず
\\	に 頑張[がんば]って。ほら、もうすぐだよ。
\\	"もうヘトヘトだなど
\\	に 頑張[がんば]って。ほら、もうすぐだよ。
\\	と 
\\	ず 
\\	いわ(ない), 
\\	いう	
\\	といわず
\\	"諦[あきら]めた
\\	といわず
\\	、もう 一度[いちど] 先生[せんせい]に 相談[そうだん]してみたらいいんじゃない?
\\	"諦[あきら]めた
\\	、もう 一度[いちど] 先生[せんせい]に 相談[そうだん]してみたらいいんじゃない?
\\	と 
\\	ず 
\\	いわ(ない), 
\\	いう	
\\	といわず
\\	"手
\\	といわず
\\	足[あし]
\\	といわず
\\	、 体中[からだじゅう] 蚊[か]にさされた。
\\	"手
\\	足[あし]
\\	、 体中[からだじゅう] 蚊[か]にさされた。
\\	~といわず~といわず	
\\	といわず
\\	"彼[かれ]は、 朝[あさ]
\\	といわず
\\	夜[よる]
\\	といわず
\\	、 暇[ひま]さえあれば 漫画[まんが]を 読[よ]んでいる。
\\	"彼[かれ]は、 朝[あさ]
\\	夜[よる]
\\	、 暇[ひま]さえあれば 漫画[まんが]を 読[よ]んでいる。
\\	~といわず~といわず	
\\	といわず
\\	"うちへ 帰[かえ]ってみた
\\	ところ
\\	、 合格[ごうかく] 通知[つうち]が 届[とど]いていた。
\\	"うちへ 帰[かえ]ってみた
\\	、 合格[ごうかく] 通知[つうち]が 届[とど]いていた。
\\	ところ
\\	"人事部[じんじぶ]に 聞[き]いてみた
\\	ところ
\\	、 部長[ぶちょう]のロンドン 転勤[てんきん]は 本当[ほんとう]だと 分[わ]かった。
\\	"人事部[じんじぶ]に 聞[き]いてみた
\\	、 部長[ぶちょう]のロンドン 転勤[てんきん]は 本当[ほんとう]だと 分[わ]かった。
\\	ところ
\\	"嘆[なげ]いてみた
\\	ところで
\\	、なくしたビデオカメラは 戻[もど]っては 来[こ]ないさ。
\\	"嘆[なげ]いてみた
\\	、なくしたビデオカメラは 戻[もど]っては 来[こ]ないさ。
\\	ところで
\\	"今[いま] 電話[でんわ]した
\\	ところで
\\	、 彼[かれ]はまだ 留守[るす]でしょう。
\\	"今[いま] 電話[でんわ]した
\\	、 彼[かれ]はまだ 留守[るす]でしょう。
\\	ところで
\\	"私[わたし]が 宿題[しゅくだい]を 始[はじ]めようとしている
\\	ところへ
\\	、 村田[むらた]さんが 遊[あそ]びに 来[き]た。
\\	"私[わたし]が 宿題[しゅくだい]を 始[はじ]めようとしている
\\	、 村田[むらた]さんが 遊[あそ]びに 来[き]た。
\\	ところ 
\\	へ 
\\	ことろへ
\\	"大雨[おおあめ]がやんだ
\\	ことろへ
\\	、 山崩[やまくず]れが 起[お]きた。
\\	"大雨[おおあめ]がやんだ
\\	、 山崩[やまくず]れが 起[お]きた。
\\	ところ 
\\	へ 
\\	ことろへ
\\	"美味[おい]しそうな 苺[いちご]なので、 沢山[たくさん] 買[か]った
\\	ところが
\\	、 箱[はこ]の 下[した]の 方[ほう]のは 大分[だいぶ] 腐[くさ]っていた。
\\	"美味[おい]しそうな 苺[いちご]なので、 沢山[たくさん] 買[か]った
\\	、 箱[はこ]の 下[した]の 方[ほう]のは 大分[だいぶ] 腐[くさ]っていた。
\\	ところが
\\	"朝[あさ]お 天気[てんき]だったから 傘[かさ]を 持[も]たずに 出[で]かけた
\\	ところが
\\	、 午後[ごご]になって 土砂[どしゃ] 降[ぶ]りになった。
\\	"朝[あさ]お 天気[てんき]だったから 傘[かさ]を 持[も]たずに 出[で]かけた
\\	、 午後[ごご]になって 土砂[どしゃ] 降[ぶ]りになった。
\\	ところが
\\	"主人[しゅじん]は、いつもなら 十時[じゅうじ]には 帰[かえ]って 来[く]る
\\	ところを
\\	、 今夜[こんや]はまだ 帰[かえ]らない。
\\	"主人[しゅじん]は、いつもなら 十時[じゅうじ]には 帰[かえ]って 来[く]る
\\	、 今夜[こんや]はまだ 帰[かえ]らない。
\\	とこを 
\\	ところを
\\	"いつもの 年[とし]なら、 今頃[いまごろ]は 山[やま]は 雪[ゆき]で 白[しろ]くなる
\\	ところを
\\	、 今年[ことし]は 暖[あたた]かくてまだ 全然[ぜんぜん] 雪[ゆき]がありません。
\\	"いつもの 年[とし]なら、 今頃[いまごろ]は 山[やま]は 雪[ゆき]で 白[しろ]くなる
\\	、 今年[ことし]は 暖[あたた]かくてまだ 全然[ぜんぜん] 雪[ゆき]がありません。
\\	とこを 
\\	ところを
\\	"いつもなら 七時[しちじ]には 起[お]きる
\\	とこを
\\	、 今朝[けさ]は 目覚[めざま]し 時計[どけい]が 故障[こしょう]して、 寝坊[ねぼう]してしまった。
\\	"いつもなら 七時[しちじ]には 起[お]きる
\\	、 今朝[けさ]は 目覚[めざま]し 時計[どけい]が 故障[こしょう]して、 寝坊[ねぼう]してしまった。
\\	とこを 
\\	とこを
\\	"ゴールまであと 一回[ひとまわ]りだ。どんなに 苦[くる]しく
\\	とも
\\	頑張[がんば]って 走[はし]るぞ。
\\	"ゴールまであと 一回[ひとまわ]りだ。どんなに 苦[くる]しく
\\	頑張[がんば]って 走[はし]るぞ。
\\	どんなに~とも、どれだけ~とも	
\\	とも
\\	"どれだけ 時間[じかん]がかかろう
\\	とも
\\	、この 仕事[しごと]だけはやってしまわなければならない。
\\	"どれだけ 時間[じかん]がかかろう
\\	、この 仕事[しごと]だけはやってしまわなければならない。
\\	どんなに~とも、どれだけ~とも	
\\	とも
\\	"遅[おそ]く
\\	とも
\\	来週[らいしゅう]までにはお 返事[へんじ]を 下[くだ]さい。
\\	"遅[おそ]く
\\	来週[らいしゅう]までにはお 返事[へんじ]を 下[くだ]さい。
\\	とも
\\	"部屋[へや]を 探[さが]しているんですが、 少[すく]なく
\\	とも
\\	六畳[ろくじょう]は 欲[ほ]しいと 思[おも]います。
\\	"部屋[へや]を 探[さが]しているんですが、 少[すく]なく
\\	六畳[ろくじょう]は 欲[ほ]しいと 思[おも]います。
\\	とも
\\	"「君[きみ]、 本当[ほんとう]に
\\	を 見[み]たのか?」
\\	見[み]た
\\	とも
\\	"「君[きみ]、 本当[ほんとう]に
\\	を 見[み]たのか?」
\\	見[み]た
\\	とも
\\	"君[きみ]の 聞[き]き 違[ちが]いだ
\\	とも
\\	。 僕[ぼく]がそんなこと 言[い]うわけないだろ?
\\	"君[きみ]の 聞[き]き 違[ちが]いだ
\\	。 僕[ぼく]がそんなこと 言[い]うわけないだろ?
\\	とも
\\	"明日[あす]アメリカにたちます。あなた
\\	と-も
\\	当分[とうぶん]お 目[め]にかかれませんが、どうかお 元気[げんき]で。
\\	"明日[あす]アメリカにたちます。あなた
\\	当分[とうぶん]お 目[め]にかかれませんが、どうかお 元気[げんき]で。
\\	と 
\\	も	
\\	と-も
\\	"警官[けいかん]
\\	と-も
\\	あろう 人[ひと]が、 盗[ぬす]みを 働[はたら]いたとはねえ。
\\	"警官[けいかん]
\\	あろう 人[ひと]が、 盗[ぬす]みを 働[はたら]いたとはねえ。
\\	と 
\\	も	
\\	と-も
\\	は, 
\\	"行[い]き 先[さき]
\\	ともあれ
\\	、 今年[ことし]こそは 海外[かいがい] 旅行[りょこう]をしようよ。
\\	"行[い]き 先[さき]
\\	、 今年[ことし]こそは 海外[かいがい] 旅行[りょこう]をしようよ。
\\	ともあれ
\\	は, 
\\	"言[い]い 訳[わけ]は
\\	ともあれ
\\	、もっと 勉強[べんきょう]しないと 希望[きんぼう]する 学校[がっこう]に 入[はい]れないよ。
\\	"言[い]い 訳[わけ]は
\\	、もっと 勉強[べんきょう]しないと 希望[きんぼう]する 学校[がっこう]に 入[はい]れないよ。
\\	ともあれ
\\	"「うわぁ、 美味[おい]しそうなケーキ!」
\\	今[いま]いただいたのよ。 何[なに]は
\\	ともあれ
\\	、 早速[さっそく]いただきましょうよ。」
\\	"「うわぁ、 美味[おい]しそうなケーキ!」
\\	今[いま]いただいたのよ。 何[なに]は
\\	、 早速[さっそく]いただきましょうよ。」
\\	何はともあれ	
\\	ともあれ
\\	"何[なに]は
\\	ともあれ
\\	、 全[すべ]ては 明日[あす]のことにして、 今日[きょう]のところは 寝[ね]ることにしましょう。
\\	"何[なに]は
\\	、 全[すべ]ては 明日[あす]のことにして、 今日[きょう]のところは 寝[ね]ることにしましょう。
\\	何はともあれ	
\\	ともあれ
\\	"今度[こんど]の 水泳[すいえい] 大会[かいたい]には、 早川[はやかわ]さんが 学校[がっこう] 代表[たいひょう] 選手[せんしゅ]
\\	として
\\	選[えら]ばれました。
\\	"今度[こんど]の 水泳[すいえい] 大会[かいたい]には、 早川[はやかわ]さんが 学校[がっこう] 代表[たいひょう] 選手[せんしゅ]
\\	選[えら]ばれました。
\\	と 
\\	て, 
\\	し 
\\	する. 
\\	として
\\	"じょん 万次郎[まんじろう]は、 日本人[にほんじん]
\\	として
\\	初[はじ]めて、アメリカで 教育[きょういく]を 受[う]けた。
\\	"じょん 万次郎[まんじろう]は、 日本人[にほんじん]
\\	初[はじ]めて、アメリカで 教育[きょういく]を 受[う]けた。
\\	と 
\\	て, 
\\	し 
\\	する. 
\\	として
\\	"話[はなし]は 後[あと]でする
\\	として
\\	、まず 食事[しょくじ]にしましょう。
\\	"話[はなし]は 後[あと]でする
\\	、まず 食事[しょくじ]にしましょう。
\\	として
\\	"その 噂[うわさ]はそれ
\\	として
\\	、 彼[かれ]についてはまだ 色々[いろいろ]と 話[はなし]があるんだよ。
\\	"その 噂[うわさ]はそれ
\\	、 彼[かれ]についてはまだ 色々[いろいろ]と 話[はなし]があるんだよ。
\\	として
\\	"一人[ひとり]
\\	として
\\	この 問題[もんだい]が 解[と]けた 学生[がくせい]はいませんでした。
\\	"一人[ひとり]
\\	この 問題[もんだい]が 解[と]けた 学生[がくせい]はいませんでした。
\\	として
\\	"僕[ぼく]は、 一日[いちにち]
\\	として
\\	彼女[かのじょ]のことを 思[おも]わない 日[ひ]はない。
\\	"僕[ぼく]は、 一日[いちにち]
\\	彼女[かのじょ]のことを 思[おも]わない 日[ひ]はない。
\\	として
\\	"私[わたし]
\\	としても
\\	、あなたの 悩[なや]みはよく 分[わ]かります。 同[おな]じような 経験[けいけん]ありますから。
\\	"私[わたし]
\\	、あなたの 悩[なや]みはよく 分[わ]かります。 同[おな]じような 経験[けいけん]ありますから。
\\	と 
\\	も, 
\\	して 
\\	する 
\\	としても
\\	としても
\\	、 授業[じゅぎょう] 料[りょう]の 値上[ねあ]げは、 問題[もんだい]だと 考[かんがえ]えます。
\\	、 授業[じゅぎょう] 料[りょう]の 値上[ねあ]げは、 問題[もんだい]だと 考[かんがえ]えます。
\\	と 
\\	も, 
\\	して 
\\	する 
\\	としても
\\	"私[わたし]はその 案[あん]には 賛成[さんせい]する
\\	としても
\\	、お 金[かね]は 今[いま]は 出[だ]せません。
\\	"私[わたし]はその 案[あん]には 賛成[さんせい]する
\\	、お 金[かね]は 今[いま]は 出[だ]せません。
\\	としても
\\	"車[くるま]を 買[か]い 替[か]える
\\	としても
\\	、 今度[こんど]もやっぱり 白[しろ]いのにしましょうね。
\\	"車[くるま]を 買[か]い 替[か]える
\\	、 今度[こんど]もやっぱり 白[しろ]いのにしましょうね。
\\	としても
\\	"彼女[かのじょ]
\\	としては
\\	、 彼[かれ]のプロポーズ 断[ことわ]るほかなかったのだろう。
\\	"彼女[かのじょ]
\\	、 彼[かれ]のプロポーズ 断[ことわ]るほかなかったのだろう。
\\	と 
\\	は, 
\\	して 
\\	する	
\\	としては
\\	"私[わたし]
\\	としては
\\	、 彼[かれ]の 意見[いけん]に 反対[はんたい]です。
\\	"私[わたし]
\\	、 彼[かれ]の 意見[いけん]に 反対[はんたい]です。
\\	と 
\\	は, 
\\	して 
\\	する	
\\	としては
\\	とはいいながら 
\\	とはいうものの	"梅雨[つゆ]
\\	とはいえ
\\	、 今年[ことし]は 晴[は]れた 日[ひ]が 多[おお]い。
\\	"梅雨[つゆ]
\\	、 今年[ことし]は 晴[は]れた 日[ひ]が 多[おお]い。
\\	と 
\\	は, 
\\	いえ 
\\	いう 
\\	とはいえ
\\	とはいいながら 
\\	とはいうものの	"この 松茸[まつたけ]は 美味[おい]しい
\\	とはいいながら
\\	、ずいぶん 値段[ねだん]が 高[たか]いね。
\\	"この 松茸[まつたけ]は 美味[おい]しい
\\	、ずいぶん 値段[ねだん]が 高[たか]いね。
\\	と 
\\	は, 
\\	いえ 
\\	いう 
\\	とはいいながら
\\	"富士山[ふじさん]
\\	という
\\	山[やま]は、 日本[にほん]で 一番[いちばん] 高[たか]い 山[やま]です。
\\	"富士山[ふじさん]
\\	山[やま]は、 日本[にほん]で 一番[いちばん] 高[たか]い 山[やま]です。
\\	と 
\\	いう	
\\	という
\\	"ナンシー
\\	という
\\	女性[じょせい]はどの 人[ひと]ですか。
\\	"ナンシー
\\	女性[じょせい]はどの 人[ひと]ですか。
\\	と 
\\	いう	
\\	という
\\	"プロ 野球[やきゅう]の 開幕[かいまく] 試合[じあい]に、 五万[ごまん] 人[にん]
\\	という
\\	人[ひと]がスタジアムに 押[お]しかけた。
\\	"プロ 野球[やきゅう]の 開幕[かいまく] 試合[じあい]に、 五万[ごまん] 人[にん]
\\	人[ひと]がスタジアムに 押[お]しかけた。
\\	という
\\	"何千枚[なんぜんまい]
\\	という
\\	宝[たから]くじが、 一日[いちにち]で 売[う]り 切[き]れた。
\\	"何千枚[なんぜんまい]
\\	宝[たから]くじが、 一日[いちにち]で 売[う]り 切[き]れた。
\\	という
\\	"林[はやし]さんの 息子[むすこ]さんがバスケットボールの 選手[せんしゅ]だ
\\	という
\\	ことを 聞[き]きました。
\\	"林[はやし]さんの 息子[むすこ]さんがバスケットボールの 選手[せんしゅ]だ
\\	ことを 聞[き]きました。
\\	という
\\	"お 化[ば]けがでる
\\	という
\\	話[はなし]は 信[しん]じられません。
\\	"お 化[ば]けがでる
\\	話[はなし]は 信[しん]じられません。
\\	という
\\	"彼[かれ]の 祖父[そふ]は 大学[だいがく] 者[しゃ]だった
\\	という
\\	"彼[かれ]の 祖父[そふ]は 大学[だいがく] 者[しゃ]だった
\\	という
\\	"昔[むかし]この 町[まち]は 山[やま]だった
\\	という
\\	"昔[むかし]この 町[まち]は 山[やま]だった
\\	という
\\	"濃[こ]い 霧[きり]のため、 車[くるま]
\\	という
\\	車[くるま]はみんなライトを 点[つ]けて 走[はし]っている。
\\	"濃[こ]い 霧[きり]のため、 車[くるま]
\\	車[くるま]はみんなライトを 点[つ]けて 走[はし]っている。
\\	という
\\	"お 祭[まつ]りで、 村中[むらじゅう]の 男[おとこ]
\\	という
\\	男[おとこ]がお 神輿[みこし]を 担[かつ]いでいる。
\\	"お 祭[まつ]りで、 村中[むらじゅう]の 男[おとこ]
\\	男[おとこ]がお 神輿[みこし]を 担[かつ]いでいる。
\\	という
\\	"彼[かれ]からまだ 電話[でんわ]がかかってこない。
\\	ということは
\\	、まだ 出張[しゅっちょう]から 帰[かえ]ってないのだろう。
\\	"彼[かれ]からまだ 電話[でんわ]がかかってこない。
\\	、まだ 出張[しゅっちょう]から 帰[かえ]ってないのだろう。
\\	と 
\\	は, 
\\	いう 
\\	こと	
\\	ということは
\\	"彼[かれ]は 他人[たにん]を 信[しん]じない。
\\	というのは
\\	、 友達[ともだち]にだまされたことがあるからです。
\\	"彼[かれ]は 他人[たにん]を 信[しん]じない。
\\	、 友達[ともだち]にだまされたことがあるからです。
\\	と 
\\	の 
\\	は(も), 
\\	いう	
\\	というのは
\\	"彼[かれ]は 最近[さいきん] 会社[かいしゃ]をやめました。
\\	というのも
\\	、 新[あたら]しい 事業[じぎょう]を 始[はじ]めたからです。
\\	"彼[かれ]は 最近[さいきん] 会社[かいしゃ]をやめました。
\\	、 新[あたら]しい 事業[じぎょう]を 始[はじ]めたからです。
\\	と 
\\	の 
\\	は(も), 
\\	いう	
\\	というのも
\\	"この 大学[だいがく]では、 前日[ぜんじつ]まで 雨[あめ]でも、 卒業[そつぎょう] 式[しき]
\\	というと
\\	必[かなら]ず 晴[は]れるのだよ。
\\	"この 大学[だいがく]では、 前日[ぜんじつ]まで 雨[あめ]でも、 卒業[そつぎょう] 式[しき]
\\	必[かなら]ず 晴[は]れるのだよ。
\\	と 
\\	いう. 
\\	というと
\\	"学生[がくせい]は 遊[あそ]ぶところを 沢山[たくさん] 知[し]ってるからね。
\\	というと
\\	、 学生[がくせい]はあまり 勉強[べんきょう]していないよに 聞[き]こえるだど。
\\	"学生[がくせい]は 遊[あそ]ぶところを 沢山[たくさん] 知[し]ってるからね。
\\	、 学生[がくせい]はあまり 勉強[べんきょう]していないよに 聞[き]こえるだど。
\\	というと
\\	"あら、 変[へん]だ
\\	わ
\\	。 車[くるま]の 鍵[かぎ]、 確[たし]かにここに 置[お]いておいたんだけど。
\\	"あら、 変[へん]だ
\\	。 車[くるま]の 鍵[かぎ]、 確[たし]かにここに 置[お]いておいたんだけど。
\\	わ
\\	"あら、 素敵[すてき]だ
\\	わ
\\	ね、あなたのドレス。
\\	"あら、 素敵[すてき]だ
\\	ね、あなたのドレス。
\\	わ
\\	"そうね、いますぐ 彼[かれ]に 電話[でんわ]したほうがいいと 思[おも]う
\\	わ
\\	"そうね、いますぐ 彼[かれ]に 電話[でんわ]したほうがいいと 思[おも]う
\\	わ
\\	"上手だ
\\	わ
\\	、この 絵[え]。 幸子[さちこ]さんが 描[か]いたんですって。
\\	"上手だ
\\	、この 絵[え]。 幸子[さちこ]さんが 描[か]いたんですって。
\\	わ
\\	"驚[おどろ]いた
\\	わ
\\	、 恐竜[きょうりゅう]ってあんなに 大[おお]きいとは 思[おも]わなかったもの。
\\	"驚[おどろ]いた
\\	、 恐竜[きょうりゅう]ってあんなに 大[おお]きいとは 思[おも]わなかったもの。
\\	わ
\\	"熱[ねつ]はでる
\\	わ
\\	、 喉[のど]は 痛[いた]い
\\	わ
\\	、 今度[どんど]の 風邪[かぜ]はひどかった。
\\	"熱[ねつ]はでる
\\	、 喉[のど]は 痛[いた]い
\\	、 今度[どんど]の 風邪[かぜ]はひどかった。
\\	わ
\\	"アリの 巣[す]をつついたら、 出[で]る
\\	わ
\\	出[で]る
\\	わ
\\	、 何百[なんひゃく]というアリが 這[は]い 出[だ]してきた。
\\	"アリの 巣[す]をつついたら、 出[で]る
\\	出[で]る
\\	、 何百[なんひゃく]というアリが 這[は]い 出[だ]してきた。
\\	わ
\\	"秋雄[あきお] 君[くん]
\\	やい
\\	、ちょっとおじさんに 手[て]を 貸[か]してくれ。
\\	"秋雄[あきお] 君[くん]
\\	、ちょっとおじさんに 手[て]を 貸[か]してくれ。
\\	やい
\\	"毛虫[けむし]が 恐[こわ]いの?いくじなし
\\	やい
\\	"毛虫[けむし]が 恐[こわ]いの?いくじなし
\\	やい
\\	"その 蜂[はち] 壊[こわ]したの 僕[ぼく]じゃない
\\	やい
\\	"その 蜂[はち] 壊[こわ]したの 僕[ぼく]じゃない
\\	やい
\\	"もう 君[きみ]になんか 頼[たの]まない
\\	やい
\\	"もう 君[きみ]になんか 頼[たの]まない
\\	やい
\\	"主人[しゅじん]は 出張[しゅっちょう]から 帰[かえ]る
\\	やいなや
\\	、 風邪[かぜ]で 寝込[ねこ]んでしまった。
\\	"主人[しゅじん]は 出張[しゅっちょう]から 帰[かえ]る
\\	、 風邪[かぜ]で 寝込[ねこ]んでしまった。
\\	や, 
\\	いな 
\\	や.	
\\	やいなや
\\	"若者[わかもの]は 馬[うま]に 跨[またが]る
\\	やいなや
\\	、 駆[か]け 去[さ]った。
\\	"若者[わかもの]は 馬[うま]に 跨[またが]る
\\	、 駆[か]け 去[さ]った。
\\	や, 
\\	いな 
\\	や.	
\\	やいなや
\\	"僕[ぼく]のおじいさんは 歯[は]が 沢山[たくさん] 欠[か]けているので、 何[なに]を 言[い]っている(の)
\\	やら
\\	さっぱり 分[わ]からない。
\\	"僕[ぼく]のおじいさんは 歯[は]が 沢山[たくさん] 欠[か]けているので、 何[なに]を 言[い]っている(の)
\\	さっぱり 分[わ]からない。
\\	やら
\\	"ひいおばあさんが 亡[な]くなったのは、いつのことだった
\\	やら
\\	忘[わす]れてしまったよ。
\\	"ひいおばあさんが 亡[な]くなったのは、いつのことだった
\\	忘[わす]れてしまったよ。
\\	やら
\\	"高見[たかみ]と
\\	やら
\\	いう 人[ひと]が 目[め]にかかりたいそうです。
\\	"高見[たかみ]と
\\	いう 人[ひと]が 目[め]にかかりたいそうです。
\\	とやら	
\\	やら
\\	"坂本[さかもと]さんは、どこと
\\	やら
\\	私[わたし]の 姉[あね]に 似[に]ている。
\\	"坂本[さかもと]さんは、どこと
\\	私[わたし]の 姉[あね]に 似[に]ている。
\\	とやら	
\\	やら
\\	"息子[むすこ]も 大[おお]きくなって、どっちが 主人[しゅじん]のシャツ
\\	やら
\\	息子[みすこ]のシャツ
\\	やら
\\	、 見分[みわ]けがつかない。
\\	"息子[むすこ]も 大[おお]きくなって、どっちが 主人[しゅじん]のシャツ
\\	息子[みすこ]のシャツ
\\	、 見分[みわ]けがつかない。
\\	~やら~やら 
\\	やら
\\	"今日[きょう]は 雨[あめ]が 降[ふ]るの
\\	やら
\\	降[ふ]らないの
\\	やら
\\	。まあ 傘[かさ]だけは 持[も]っていこう。
\\	"今日[きょう]は 雨[あめ]が 降[ふ]るの
\\	降[ふ]らないの
\\	。まあ 傘[かさ]だけは 持[も]っていこう。
\\	~やら~やら 
\\	やら
\\	"明日[あした]はクリスマス、 美味[おい]しいごちそう
\\	やら
\\	ケーキ
\\	やら
\\	沢山[たくさん] 作[つく]りましょうね。
\\	"明日[あした]はクリスマス、 美味[おい]しいごちそう
\\	ケーキ
\\	沢山[たくさん] 作[つく]りましょうね。
\\	やら
\\	"上[うえ]の 階[かい]でまた 夫婦[ふうふ]げんかが 始[はじ]まったよ。ものを 投[な]げる
\\	やら
\\	わめく
\\	やら
\\	、 大変[たいへん]だなあ。
\\	"上[うえ]の 階[かい]でまた 夫婦[ふうふ]げんかが 始[はじ]まったよ。ものを 投[な]げる
\\	わめく
\\	、 大変[たいへん]だなあ。
\\	やら
\\	"今日[きょう]も 主人[しゅじん]は 魚[さかな] 釣[つ]りに 出[で]かけた。 何匹[なんびき] 釣[つ]ってくる(の)
\\	やら
\\	"今日[きょう]も 主人[しゅじん]は 魚[さかな] 釣[つ]りに 出[で]かけた。 何匹[なんびき] 釣[つ]ってくる(の)
\\	やら
\\	"約束[やくそく]の 時間[じかん]を 一時[いちじ] 十五分[じゅうごふん]も 過[す]ぎたのに、 彼女[かのじょ]はまだ 来[こ]ない。どれだけ 待[ま]たされる(の)
\\	やら
\\	"約束[やくそく]の 時間[じかん]を 一時[いちじ] 十五分[じゅうごふん]も 過[す]ぎたのに、 彼女[かのじょ]はまだ 来[こ]ない。どれだけ 待[ま]たされる(の)
\\	やら
\\	"彼女[かのじょ]の 花嫁[はなよめ] 姿[すがた]、 本当[ほんとう]に 綺麗[きれい]だった
\\	よ
\\	"彼女[かのじょ]の 花嫁[はなよめ] 姿[すがた]、 本当[ほんとう]に 綺麗[きれい]だった
\\	よ
\\	"あなたの 演奏[えんそう] 素敵[すてき]だったわ
\\	よ
\\	"あなたの 演奏[えんそう] 素敵[すてき]だったわ
\\	よ
\\	"その 恰好[かっこう]は 何[なん]だ
\\	よ
\\	。ズボンからシャツがはみ 出[だ]しているじゃないか。
\\	"その 恰好[かっこう]は 何[なん]だ
\\	。ズボンからシャツがはみ 出[だ]しているじゃないか。
\\	よ
\\	"急[きゅう]に 泣[な]き 出[だ]したりして、 一体[いったい]どうしたの
\\	よ
\\	"急[きゅう]に 泣[な]き 出[だ]したりして、 一体[いったい]どうしたの
\\	よ
\\	"早[はや]く 食[た]べろ
\\	よ
\\	。 学校[がっこう]に 送[おく]れるぞ。
\\	"早[はや]く 食[た]べろ
\\	。 学校[がっこう]に 送[おく]れるぞ。
\\	よ
\\	"黙[だま]って 辞書[じしょ]を 使[つか]ったぐらいで、そんなに 怒[おこ]るな
\\	よ
\\	"黙[だま]って 辞書[じしょ]を 使[つか]ったぐらいで、そんなに 怒[おこ]るな
\\	よ
\\	"いいお 天気[てんき]だ。 散歩[さんぽ]にでも 行[い]こう
\\	よ
\\	"いいお 天気[てんき]だ。 散歩[さんぽ]にでも 行[い]こう
\\	おお(う) 
\\	よう	
\\	よ
\\	"ねえ、お 父[とう]さん、お 寿司[すし] 食[た]べよう
\\	よ
\\	"ねえ、お 父[とう]さん、お 寿司[すし] 食[た]べよう
\\	おお(う) 
\\	よう	
\\	よ
\\	"なんて 暗[くら]い 道[みち]なんでしょう
\\	よ
\\	"なんて 暗[くら]い 道[みち]なんでしょう
\\	おお(う) 
\\	よう	
\\	よ
\\	"彼女[かのじょ]には 新[あたら]しいボーイフレンドが 出来[でき]たらしい。 僕[ぼく]のことなんかすぐ 忘[わす]れるだろう
\\	よ
\\	"彼女[かのじょ]には 新[あたら]しいボーイフレンドが 出来[でき]たらしい。 僕[ぼく]のことなんかすぐ 忘[わす]れるだろう
\\	おお(う) 
\\	よう	
\\	よ
\\	"神[かみ]
\\	よ
\\	、お 守[まも]り 下[くだ]さい。
\\	"神[かみ]
\\	、お 守[まも]り 下[くだ]さい。
\\	よ
\\	"おーい! 平岩[ひらいわ] 君[くん]
\\	よ
\\	、 早[はや]く 登[のぼ]ってこいよ。 素晴[すば]らしい 景色[けしき]だぞ。
\\	"おーい! 平岩[ひらいわ] 君[くん]
\\	、 早[はや]く 登[のぼ]ってこいよ。 素晴[すば]らしい 景色[けしき]だぞ。
\\	よ
\\	"お 父[とう]さんまだ 帰[かえ]ってこないの?ひょっとするとだ
\\	よ
\\	、おじさんの 家[うち]に 寄[よ]っているのかもしれないね。
\\	"お 父[とう]さんまだ 帰[かえ]ってこないの?ひょっとするとだ
\\	、おじさんの 家[うち]に 寄[よ]っているのかもしれないね。
\\	よ
\\	"万一[まんいち]だ
\\	よ
\\	、 宝[たから]くじの 一等[いっとう]が 当[あ]たったらどうする?
\\	"万一[まんいち]だ
\\	、 宝[たから]くじの 一等[いっとう]が 当[あ]たったらどうする?
\\	よ
\\	"吉岡[よしおか] 君[くん]が 係長[かかりちょう]になるうんだ。みんなでお 祝[いわ]いしよう
\\	ぜ
\\	"吉岡[よしおか] 君[くん]が 係長[かかりちょう]になるうんだ。みんなでお 祝[いわ]いしよう
\\	ぜ
\\	"今[いま]までは 三日[みっか] 坊主[ぼうず]だった。だが、 今度[こんど]はきっとタバコをやめる
\\	ぜ
\\	"今[いま]までは 三日[みっか] 坊主[ぼうず]だった。だが、 今度[こんど]はきっとタバコをやめる
\\	ぜ
\\	"君[きみ]より 僕[ぼく]のほうが 背[せ]が 高[たか]い
\\	ぞ
\\	"君[きみ]より 僕[ぼく]のほうが 背[せ]が 高[たか]い
\\	ぞ
\\	"ほら、 真一[しんいち]、 投[な]げる
\\	ぞ
\\	"ほら、 真一[しんいち]、 投[な]げる
\\	ぞ
\\	"歯[は]が 悪[わる]いので、 固[かた]いものは 少[すこ]し
\\	ずつ
\\	しかかめません。
\\	"歯[は]が 悪[わる]いので、 固[かた]いものは 少[すこ]し
\\	しかかめません。
\\	ずつ
\\	"漫画[まんが]の 本[ほん]、 一冊[いっさつ]
\\	ずつ
\\	じゃなくて、もっと 沢山[たくさん] 貸[か]してよ。
\\	"漫画[まんが]の 本[ほん]、 一冊[いっさつ]
\\	じゃなくて、もっと 沢山[たくさん] 貸[か]してよ。
\\	ずつ
\\	"どのクラスにも、 学級[がっきゅう] 委員[いいん]が 二人[ふたり]
\\	ずつ
\\	います。
\\	"どのクラスにも、 学級[がっきゅう] 委員[いいん]が 二人[ふたり]
\\	います。
\\	ずつ
\end{CJK}
\end{document}