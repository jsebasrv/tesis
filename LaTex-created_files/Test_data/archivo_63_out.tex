\documentclass[8pt]{extreport} 
\usepackage{hyperref}
\usepackage{CJKutf8}
\begin{document}
\begin{CJK}{UTF8}{min}
\\	「有る(ある)」のアクセントは?0	ある{有;在} 仮名 
\\	ある 漢字 
\\	有る、在る 発音図:	ある ある		ある ある あります あります あって あって あった あった ない ない なかった なかった あれば あれば あれ あれ あられる あられる あろう あろう				動詞-一般 
\\	「言う(いう)」のアクセントは?1		言う いう		言う いう 言います いいます 言って いって 言った いった 言わない いわない 言わなかった いわなかった 言えば いえば 言わせる いわせる 言われる いわれる 言え いえ 言える いえる 言おう いおう				動詞-一般 
\\	「思う(おもう)」のアクセントは?2	思う 仮名 
\\	おもう 漢字 
\\	思う 発音図:	思う おもう		思う おもう 思います おもいます 思って おもって 思った おもった 思わない おもわない 思わなかった おもわなかった 思えば おもえば 思わせる おもわせる 思われる おもわれる 思え おもえ 思える おもえる 思おう おもおう				動詞-一般 
\\	「等(など)」のアクセントは?3	など{等} 仮名 
\\	など 漢字 
\\	等 発音図: 助詞: オ	等 など		等 など				助詞-副助詞 
\\	「私(わたくし)」のアクセントは?4	私 仮名 
\\	わたくし 漢字 
\\	私 発音図: 助詞: オ	私 わたくし		私 わたくし				代名詞 
\\	「見る(みる)」のアクセントは?5	見る 仮名 
\\	みる 漢字 
\\	見る 発音図:	見る みる		見る みる 見ます みます 見て みて 見た みた 見ない みない 見なかった みなかった 見れば みれば 見させる みさせる 見られる みられる 見ろ みろ 見られる みられる 見よう みよう				動詞-一般 
\\	「来る(くる)」のアクセントは?6	来る 仮名 
\\	くる 漢字 
\\	来る 発音図:	来る くる		来る くる 来ます きます 来て きて 来た きた 来ない こない 来なかった こなかった 来れば くれば 来させる こさせる 来られる こられる 来い こい 来られる こられる 来よう こよう				動詞-一般 
\\	「行く(いく)」のアクセントは?7	行く 仮名 
\\	いく、ゆく 漢字 
\\	行く 発音図: 発音図:	行く いく		行く いく 行きます いきます 行って いって 行った いった 行かない いかない 行かなかった いかなかった 行けば いけば 行かせる いかせる 行かれる いかれる 行け いけ 行ける いける 行こう いこう				動詞-一般 
\\	「自分(じぶん)」のアクセントは?8	自分 仮名 
\\	じぶん 漢字 
\\	自分 発音図: 助詞: オ	自分 じぶん		自分 じぶん				名詞-普通名詞-一般 
\\	「如何(どう)《〜する》」のアクセントは?9	どう{如何}(〜する) 仮名 
\\	どう、どー 漢字 
\\	如何 発音図: 例文: スる							副詞 
\\	「方(ほう)」のアクセントは?10	方 仮名 
\\	ほう、ほー 漢字 
\\	方 発音図:	方 ほう		方 ほう				名詞-普通名詞-一般 
\\	「出来る(できる)」のアクセントは?11	出来る 仮名 
\\	できる 漢字 
\\	出来る 発音図:	出来る できる		出来る できる 出来ます できます 出来て できて 出来た できた 出来ない できない 出来なかった できなかった 出来れば できれば 出来させる できさせる 出来ろ できろ 出来よう できよう				動詞-一般 
\\	「一(いち)」のアクセントは?12	
\\	仮名 
\\	いち 漢字 
\\	一、1 発音図: 助詞: オ	一 いち		一 いち				名詞-数詞 
\\	「考える(かんがえる)」のアクセントは?13	考える 仮名 
\\	かんがえる 漢字 
\\	考える 発音図: 発音図:	考える かんがえる かんがえる		考える かんがえる かんがえる 考えます かんがえます 考えて かんがえて 考えた かんがえた 考えない かんがえない 考えなかった かんがえなかった 考えれば かんがえれば かんがえれば 考えさせる かんがえさせる 考えられる かんがえられる 考えろ かんがえろ かんがえろ 考えられる かんがえられる 考えよう かんがえよう				動詞-一般 
\\	「良い(いい)」のアクセントは?14	いい{好;良;善} 仮名 
\\	いい 漢字 
\\	良い、善い、好い 発音図: 例文: 夜になって、ホう告が届いた							形容詞-一般 
\\	「持つ(もつ)」のアクセントは?15	持つ 仮名 
\\	もつ 漢字 
\\	持つ 発音図:	持つ もつ		持つ もつ 持ちます もちます 持って もって 持った もった 持たない もたない 持たなかった もたなかった 持てば もてば 持たせる もたせる 持たれる もたれる 持て もて 持てる もてる 持とう もとう				動詞-一般 
\\	「分かる(わかる)」のアクセントは?16	わかる{分} 仮名 
\\	わかる 漢字 
\\	分かる 発音図:	分かる わかる		分かる わかる 分かります わかります 分かって わかって 分かった わかった 分からない わからない 分からなかった わからなかった 分かれば わかれば 分からせる わからせる 分かられる わかられる 分かれ わかれ 分かろう わかろう				動詞-一般 
\\	「出る(でる)」のアクセントは?17	出る 仮名 
\\	でる 漢字 
\\	出る 発音図:	出る でる		出る でる 出ます でます 出て でて 出た でた 出ない でない 出なかった でなかった 出れば でれば 出させる でさせる 出られる でられる 出ろ でろ 出られる でられる 出よう でよう				動詞-一般 
\\	「行う(おこなう)」のアクセントは?18	行う 仮名 
\\	おこなう 漢字 
\\	行う 発音図:	行う おこなう		行う おこなう 行います おこないます 行って おこなって 行った おこなった 行わない おこなわない 行わなかった おこなわなかった 行えば おこなえば 行わせる おこなわせる 行われる おこなわれる 行え おこなえ 行える おこなえる 行おう おこなおう				動詞-一般 
\\	「遣る(やる)」のアクセントは?19	やる{遣} 仮名 
\\	やる 漢字 
\\	遣る 発音図:	やる やる		やる やる やります やります やって やって やった やった やらない やらない やらなかった やらなかった やれば やれば やらせる やらせる やられる やられる やれ やれ やれる やれる やろう やろう				動詞-一般 
\\	「訳(わけ)」のアクセントは?20	訳 仮名 
\\	やく、わけ 漢字 
\\	訳 発音図: 助詞: オ 例文: 英語のテストはナら自信がある 発音図:	訳 わけ		訳 わけ				名詞-普通名詞-一般 
\\	「彼(かれ)」のアクセントは?21	彼 仮名 
\\	かれ 漢字 
\\	彼 発音図: 助詞: オ	彼 かれ		彼 かれ				代名詞 
\\	「然て(そして)」のアクセントは?22	そして{而;然} 仮名 
\\	そして 漢字 
\\	然て、而て 発音図:							接続詞 
\\	「聞く(きく)」のアクセントは?23	聞く 仮名 
\\	きく 漢字 
\\	聞く 発音図: 例文: 話をコとから信頼が生まれる	聞く きく		聞く きく 聞きます ききます 聞いて きいて 聞いた きいた 聞かない きかない 聞かなかった きかなかった 聞けば きけば 聞かせる きかせる 聞かれる きかれる 聞け きけ 聞ける きける 聞こう きこう				動詞-一般 
\\	「知る(しる)」のアクセントは?24	知る 仮名 
\\	しる 漢字 
\\	知る 発音図:	知る しる		知る しる 知ります しります 知って しって 知った しった 知らない しらない 知らなかった しらなかった 知れば しれば 知らせる しらせる 知られる しられる 知れ しれ 知れる しれる 知ろう しろう				動詞-一般 
\\	「入る(はいる)」のアクセントは?25	入る 仮名 
\\	いる、はいる 漢字 
\\	入る 発音図: 発音図:	入る はいる		入る はいる 入ります はいります 入って はいって 入った はいった 入らない はいらない 入らなかった はいらなかった 入れば はいれば 入らせる はいらせる 入られる はいられる 入れ はいれ 入れる はいれる 入ろう はいろう				動詞-一般 
\\	「日本(にっぽん)」のアクセントは?26	日本 仮名 
\\	にっぽん、にほん 漢字 
\\	日本 発音図: 助詞: オ 発音図: 助詞: オ	日本 にっぽん		日本 にっぽん				名詞-固有名詞-地名-国 
\\	「此処(ここ)」のアクセントは?27	ここ{此処} 仮名 
\\	ここ 漢字 
\\	此処 発音図:	ここ ここ		ここ ここ				代名詞 
\\	「使う(つかう)」のアクセントは?28	使う 仮名 
\\	つかう 漢字 
\\	使う 発音図:	使う つかう		使う つかう 使います つかいます 使って つかって 使った つかった 使わない つかわない 使わなかった つかわなかった 使えば つかえば 使わせる つかわせる 使われる つかわれる 使え つかえ 使える つかえる 使おう つかおう				動詞-一般 
\\	「御座る(ござる)」のアクセントは?29	ござる{御座} 仮名 
\\	ござる 漢字 
\\	御座る 発音図:	ござる ござる		ござる ござる ござります ござります ござって ござって ござった ござった ござらない ござらない ござらなかった ござらなかった ござれば ござれば ござらせる ござらせる ござられる ござられる ござろう ござろう				動詞-一般 
\\	「気(き)」のアクセントは?30	気 仮名 
\\	き、け 漢字 
\\	気 発音図: 例文: 勝負を前にオ集中する 発音図: 例文: やや早産のガある 発音図: 例文: やや早産のガある	気 き		気 き				名詞-普通名詞-一般 
\\	「一つ(ひとつ)」のアクセントは?31	一つ 仮名 
\\	ひとつ 漢字 
\\	一つ 発音図: 助詞: オ	一つ ひとつ		一つ ひとつ				名詞-数詞 
\\	「其処(そこ)」のアクセントは?32	そこ{其処} 仮名 
\\	そこ 漢字 
\\	其処 発音図: 助詞: オ	そこ そこ		そこ そこ				代名詞 
\\	「多い(おおい)」のアクセントは?33	多い 仮名 
\\	おーい、おおい 漢字 
\\	多い 発音図: 例文: 少ないよりニ越したことはない 発音図: 例文: 少ないよりニ越したことはない	多い おおい		多い おおい 多いです おおいです 多くて おおくて 多かった おおかった 多くない おおくない 多くなかった おおくなかった 多ければ おおければ 多い おおい 多く おおく				形容詞-一般 
\\	「取る(とる)」のアクセントは?34	取る 仮名 
\\	とる 漢字 
\\	取る 発音図:	取る とる		取る とる 取ります とります 取って とって 取った とった 取らない とらない 取らなかった とらなかった 取れば とれば 取らせる とらせる 取られる とられる 取れ とれ 取れる とれる 取ろう とろう				動詞-一般 
\\	「作る(つくる)」のアクセントは?35	作る 仮名 
\\	つくる 漢字 
\\	作る 発音図:	作る つくる		作る つくる 作ります つくります 作って つくって 作った つくった 作らない つくらない 作らなかった つくらなかった 作れば つくれば 作らせる つくらせる 作られる つくられる 作れ つくれ 作れる つくれる 作ろう つくろう				動詞-一般 
\\	「問題(もんだい)」のアクセントは?36	問題 仮名 
\\	もんだい 漢字 
\\	問題 発音図: 助詞: オ	問題 もんだい		問題 もんだい				名詞-普通名詞-一般 
\\	「及び(および)」のアクセントは?37	および{及} 仮名 
\\	および 漢字 
\\	及び 発音図: 発音図:							接続詞 
\\	「手(て)」のアクセントは?38	手 仮名 
\\	て 漢字 
\\	手 発音図: 助詞: オ	手 て		手 て				名詞-普通名詞-一般 
\\	「僕(ぼく)」のアクセントは?39	僕 仮名 
\\	ぼく 漢字 
\\	僕 発音図: 助詞: オ 発音図: 助詞: オ	僕 ぼく		僕 ぼく				代名詞 
\\	「次(つぎ)」のアクセントは?40	次 仮名 
\\	つぎ 漢字 
\\	次 発音図: 例文: お客がカら次へとやってくる	次 つぎ		次 つぎ				名詞-普通名詞-一般 
\\	「受ける(うける)」のアクセントは?41	受ける 仮名 
\\	うける 漢字 
\\	受ける 発音図:	受ける うける		受ける うける 受けます うけます 受けて うけて 受けた うけた 受けない うけない 受けなかった うけなかった 受ければ うければ 受けさせる うけさせる 受けられる うけられる 受けろ うけろ 受けられる うけられる 受けよう うけよう				動詞-一般 
\\	「子供(こども)」のアクセントは?42		子供 こども		子供 こども				名詞-普通名詞-一般 
\\	「良く(よく)」のアクセントは?43	よく{良;善;能} 仮名 
\\	よく 漢字 
\\	良く、能く、善く 発音図: 例文: もっとカんがえたほうがいい							副詞 
\\	「話(はなし)」のアクセントは?44	話 仮名 
\\	はなし 漢字 
\\	話 発音図: 助詞: オ	話 はなし		話 はなし				名詞-普通名詞-一般 
\\	「付く(つく)」のアクセントは?45	付く 仮名 
\\	つく 漢字 
\\	付く 発音図: 例文: ズボンにドろをはらう 発音図: 例文: ズボンにドろをはらう	付く つく		付く つく 付きます つきます 付いて ついて 付いた ついた 付かない つかない 付かなかった つかなかった 付けば つけば 付かせる つかせる 付かれる つかれる 付け つけ 付こう つこう				動詞-一般 
\\	「貴方(あなた)《〜様》」のアクセントは?46	あなた{貴方}(〜様) 仮名 
\\	あなた 漢字 
\\	貴方 発音図: 助詞: オ	あなた あなた		あなた あなた				代名詞 
\\	「書く(かく)」のアクセントは?47	書く 仮名 
\\	かく 漢字 
\\	書く 発音図: 例文: 書をトころはいつも座敷です	書く かく		書く かく 書きます かきます 書いて かいて 書いた かいた 書かない かかない 書かなかった かかなかった 書けば かけば 書かせる かかせる 書かれる かかれる 書け かけ 書ける かける 書こう かこう				動詞-一般 
\\	「何時(いつ)」のアクセントは?48	いつ{何時} 仮名 
\\	いつ 漢字 
\\	何時 発音図:	いつ いつ		いつ いつ				代名詞 
\\	「入れる(いれる)」のアクセントは?49	入れる 仮名 
\\	いれる 漢字 
\\	入れる 発音図:	入れる いれる		入れる いれる 入れます いれます 入れて いれて 入れた いれた 入れない いれない 入れなかった いれなかった 入れれば いれれば 入れさせる いれさせる 入れられる いれられる 入れろ いれろ 入れられる いれられる 入れよう いれよう				動詞-一般 
\\	「四(よん)」のアクセントは?50	
\\	仮名 
\\	し、よん 漢字 
\\	四、4 発音図: 助詞: オ 例文: 3の次はデす 発音図:	四 よん		四 よん				名詞-数詞 
\\	「顔(かお)」のアクセントは?51	顔 仮名 
\\	かお 漢字 
\\	顔 発音図: 助詞: オ	顔 かお		顔 かお				名詞-普通名詞-一般 
\\	「彼女(かのじょ)」のアクセントは?52	彼女 仮名 
\\	かのじょ 漢字 
\\	彼女 発音図: 助詞: オ	彼女 かのじょ		彼女 かのじょ				代名詞 
\\	「言葉(ことば)《お〜》」のアクセントは?53	ことば{言葉}(お〜) 仮名 
\\	ことば 漢字 
\\	言葉 発音図: 助詞: オ	言葉 ことば		言葉 ことば				名詞-普通名詞-一般 
\\	「でも(でも)《それでも》」のアクセントは?54	でも(それでも) 仮名 
\\	でも 漢字 
\\	発音図:							接続詞 
\\	「掛ける(かける)」のアクセントは?55	掛ける 仮名 
\\	かける 漢字 
\\	掛ける 発音図:	掛ける かける		掛ける かける 掛けます かけます 掛けて かけて 掛けた かけた 掛けない かけない 掛けなかった かけなかった 掛ければ かければ 掛けさせる かけさせる 掛けられる かけられる 掛けろ かけろ 掛けられる かけられる 掛けよう かけよう				動詞-一般 
\\	「上げる(あげる)」のアクセントは?56	上げる 仮名 
\\	あげる 漢字 
\\	上げる 発音図:	上げる あげる		上げる あげる 上げます あげます 上げて あげて 上げた あげた 上げない あげない 上げなかった あげなかった 上げれば あげれば 上げさせる あげさせる 上げられる あげられる 上げろ あげろ 上げられる あげられる 上げよう あげよう				動詞-一般 
\\	「誰(だれ)」のアクセントは?57	だれ{誰} 仮名 
\\	だれ 漢字 
\\	誰 発音図: 助詞: オ	誰 だれ		誰 だれ				代名詞 
\\	「付ける(つける)」のアクセントは?58	付ける 仮名 
\\	つける 漢字 
\\	付ける 発音図:	付ける つける		付ける つける 付けます つけます 付けて つけて 付けた つけた 付けない つけない 付けなかった つけなかった 付ければ つければ 付けさせる つけさせる 付けられる つけられる 付けろ つけろ 付けられる つけられる 付けよう つけよう				動詞-一般 
\\	「為に(ために)」のアクセントは?59	ために{為} 仮名 
\\	ために 漢字 
\\	為に 発音図:							助詞-接続助詞 
\\	「国(くに)」のアクセントは?60	国 仮名 
\\	くに 漢字 
\\	国 発音図: 助詞: オ	国 くに		国 くに				名詞-普通名詞-一般 
\\	「少し(すこし)」のアクセントは?61	少し 仮名 
\\	すこし 漢字 
\\	少し 発音図:	少し すこし		少し すこし				副詞 
\\	「何処(どこ)」のアクセントは?62	どこ{何処} 仮名 
\\	どこ 漢字 
\\	何処 発音図: 助詞: オ	どこ どこ		どこ どこ				代名詞 
\\	「高い(たかい)」のアクセントは?63	高い 仮名 
\\	たかい 漢字 
\\	高い 発音図:	高い たかい		高い たかい 高いです たかいです 高くて たかくて たかくて 高かった たかかった たかかった 高くない たかくない たかくない 高くなかった たかくなかった たかくなかった 高ければ たかければ たかければ 高い たかい 高く たかく たかく				形容詞-一般 
\\	「見える(みえる)」のアクセントは?64	見える 仮名 
\\	みえる 漢字 
\\	見える 発音図:	見える みえる		見える みえる 見えます みえます 見えて みえて 見えた みえた 見えない みえない 見えなかった みえなかった 見えれば みえれば 見えさせる みえさせる 見えられる みえられる 見えろ みえろ 見えよう みえよう				動詞-一般 
\\	「男(おとこ)」のアクセントは?65	男 仮名 
\\	おとこ 漢字 
\\	男 発音図: 助詞: オ	男 おとこ		男 おとこ				名詞-普通名詞-一般 
\\	「出す(だす)」のアクセントは?66	出す 仮名 
\\	だす 漢字 
\\	出す 発音図:	出す だす		出す だす 出します だします 出して だして 出した だした 出さない ださない 出さなかった ださなかった 出せば だせば 出させる ださせる 出される だされる 出せ だせ 出せる だせる 出そう だそう				動詞-一般 
\\	「未だ(まだ)」のアクセントは?67	まだ{未} 仮名 
\\	まだ 漢字 
\\	未だ 発音図:							副詞 
\\	「人間(にんげん)」のアクセントは?68	人間 仮名 
\\	にんげん 漢字 
\\	人間 発音図: 助詞: オ	人間 にんげん		人間 にんげん				名詞-普通名詞-一般 
\\	「仕事(しごと)《お〜》」のアクセントは?69	仕事(お〜) 仮名 
\\	しごと 漢字 
\\	仕事 発音図: 助詞: オ	仕事 しごと		仕事 しごと				名詞-普通名詞-一般 
\\	「掛かる(かかる)」のアクセントは?70	掛かる 仮名 
\\	かかる 漢字 
\\	掛かる 発音図:							動詞-一般 
\\	「者(もの)」のアクセントは?71	者 仮名 
\\	もの 漢字 
\\	者 発音図: 例文: 怪しいガうろうろしている	者 もの		者 もの				名詞-普通名詞-一般 
\\	「家(いえ)」のアクセントは?72	家 仮名 
\\	いえ、や 漢字 
\\	家 発音図: 助詞: オ 発音図: 助詞: オ	家 いえ		家 いえ				名詞-普通名詞-一般 
\\	「教える(おしえる)」のアクセントは?74	教える 仮名 
\\	おしえる 漢字 
\\	教える 発音図:	教える おしえる		教える おしえる 教えます おしえます 教えて おしえて 教えた おしえた 教えない おしえない 教えなかった おしえなかった 教えれば おしえれば 教えさせる おしえさせる 教えられる おしえられる 教えろ おしえろ 教えられる おしえられる 教えよう おしえよう				動詞-一般 
\\	「必要な(ひつような)」のアクセントは?75		必要な ひつような		必要な ひつような 必要です ひつようです 必要で ひつようで 必要だった ひつようだった 必要じゃない ひつようじゃない 必要じゃなかった ひつようじゃなかった 必要ならば ひつようならば 必要だ ひつようだ 必要に ひつように				名詞-普通名詞-一般 
\\	「必要(ひつよう)」のアクセントは?76	必要 仮名 
\\	ひつよう、ひつよー 漢字 
\\	必要 発音図: 助詞: オ	必要 ひつよう		必要 ひつよう				名詞-普通名詞-一般 
\\	「私(わたし)」のアクセントは?78	わたし{私} 仮名 
\\	わたし 漢字 
\\	私 発音図: 助詞: オ	私 わたし		私 わたし				代名詞 
\\	「違う(ちがう)」のアクセントは?79	違う 仮名 
\\	ちがう 漢字 
\\	違う 発音図:	違う ちがう		違う ちがう 違います ちがいます 違って ちがって 違った ちがった 違わない ちがわない 違わなかった ちがわなかった 違えば ちがえば 違わせる ちがわせる 違われる ちがわれる 違え ちがえ 違おう ちがおう				動詞-一般 
\\	「大きな(おーきな)」のアクセントは?80	大きな 仮名 
\\	おーきな、おおきな 漢字 
\\	大きな 発音図:							連体詞 
\\	「大きい(おおきい)」のアクセントは?81	大きい 仮名 
\\	おーきい、おおきい 漢字 
\\	大きい 発音図:	大きい おおきい		大きい おおきい 大きいです おおきいです 大きくて おおきくて おおきくて 大きかった おおきかった おおきかった 大きくない おおきくない おおきくない 大きくなかった おおきくなかった おおきくなかった 大きければ おおきければ おおきければ 大きい おおきい 大きく おおきく おおきく				形容詞-一般 
\\	「心(こころ)」のアクセントは?82	心 仮名 
\\	こころ、しん 漢字 
\\	心 発音図: 助詞: オ 発音図: 助詞: オ 発音図: 助詞: オ	心 こころ		心 こころ				名詞-普通名詞-一般 
\\	「時間(じかん)」のアクセントは?83	時間 仮名 
\\	じかん 漢字 
\\	時間 発音図: 助詞: オ	時間 じかん		時間 じかん				名詞-普通名詞-一般 
\\	「食べる(たべる)」のアクセントは?84	食べる 仮名 
\\	たべる 漢字 
\\	食べる 発音図:	食べる たべる		食べる たべる 食べます たべます 食べて たべて 食べた たべた 食べない たべない 食べなかった たべなかった 食べれば たべれば 食べさせる たべさせる 食べられる たべられる 食べろ たべろ 食べられる たべられる 食べよう たべよう				動詞-一般 
\\	「又(また)」のアクセントは?85	また{復;又;亦} 仮名 
\\	また 漢字 
\\	又、亦、復 発音図: 例文: 台風がヤってきた							副詞 
\\	「又(また)《〜の》」のアクセントは?86	また{又}(〜の) 仮名 
\\	また 漢字 
\\	又 発音図: 例文: オ約束して別れた							副詞 
\\	「本当(ほんとう)」のアクセントは?87	ほんとう{本当} 仮名 
\\	ほんとう、ほんとー 漢字 
\\	本当 発音図: 助詞: オ	本当 ほんとう		本当 ほんとう				名詞-普通名詞-一般 
\\	「置く(おく)」のアクセントは?88	置く 仮名 
\\	おく 漢字 
\\	置く 発音図:	置く おく		置く おく 置きます おきます 置いて おいて 置いた おいた 置かない おかない 置かなかった おかなかった 置けば おけば 置かせる おかせる 置かれる おかれる 置け おけ 置ける おける 置こう おこう				動詞-一般 
\\	「呼ぶ(よぶ)」のアクセントは?89	呼ぶ 仮名 
\\	よぶ 漢字 
\\	呼ぶ 発音図:	呼ぶ よぶ		呼ぶ よぶ 呼びます よびます 呼んで よんで 呼んだ よんだ 呼ばない よばない 呼ばなかった よばなかった 呼べば よべば 呼ばせる よばせる 呼ばれる よばれる 呼べ よべ 呼べる よべる 呼ぼう よぼう				動詞-一般 
\\	「直ぐ(すぐ)」のアクセントは?90	すぐ{直} 仮名 
\\	すぐ 漢字 
\\	直ぐ 発音図:							副詞 
\\	「俺(おれ)」のアクセントは?91	おれ{俺} 仮名 
\\	おれ 漢字 
\\	俺 発音図: 例文: 皆がノことを嫌う	俺 おれ		俺 おれ				代名詞 
\\	「特に(とくに)」のアクセントは?92	特に 仮名 
\\	とくに 漢字 
\\	特に 発音図:							副詞 
\\	「何の(どの)」のアクセントは?93	どの{何} 仮名 
\\	どの 漢字 
\\	何の 発音図:							連体詞 
\\	「変わる(かわる)」のアクセントは?94	変わる 仮名 
\\	かわる 漢字 
\\	変わる 発音図:	変わる かわる		変わる かわる 変わります かわります 変わって かわって 変わった かわった 変わらない かわらない 変わらなかった かわらなかった 変われば かわれば 変わらせる かわらせる 変わられる かわられる 変われ かわれ 変われる かわれる 変わろう かわろう				動詞-一般 
\\	「悪い(わるい)」のアクセントは?95	悪い 仮名 
\\	わるい 漢字 
\\	悪い 発音図:	悪い わるい		悪い わるい 悪いです わるいです 悪くて わるくて わるくて 悪かった わるかった わるかった 悪くない わるくない わるくない 悪くなかった わるくなかった わるくなかった 悪ければ わるければ わるければ 悪い わるい 悪く わるく わるく				形容詞-一般 
\\	「強い(つよい)」のアクセントは?96	強い 仮名 
\\	つよい 漢字 
\\	強い 発音図:	強い つよい		強い つよい 強いです つよいです 強くて つよくて つよくて 強かった つよかった つよかった 強くない つよくない つよくない 強くなかった つよくなかった つよくなかった 強ければ つよければ つよければ 強い つよい 強く つよく つよく				形容詞-一般 
\\	「筈(はず)」のアクセントは?97	はず{筈} 仮名 
\\	はず 漢字 
\\	筈 発音図: 助詞: オ	筈 はず		筈 はず				名詞-普通名詞-一般 
\\	「例えば(たとえば)」のアクセントは?98	例えば 仮名 
\\	たとえば 漢字 
\\	例えば 発音図:							副詞 
\\	「五(ご)」のアクセントは?99	
\\	仮名 
\\	ご 漢字 
\\	五、5 発音図: 例文: 漢数字は五、算用数字はト書く	五 ご		五 ご				名詞-数詞 
\\	「買う(かう)」のアクセントは?100	買う 仮名 
\\	かう 漢字 
\\	買う 発音図:	買う かう		買う かう 買います かいます 買って かって 買った かった 買わない かわない 買わなかった かわなかった 買えば かえば 買わせる かわせる 買われる かわれる 買え かえ 買える かえる 買おう かおう				動詞-一般 
\\	「示す(しめす)」のアクセントは?101	示す 仮名 
\\	しめす 漢字 
\\	示す 発音図: 発音図:	示す しめす		示す しめす 示します しめします 示して しめして 示した しめした 示さない しめさない 示さなかった しめさなかった 示せば しめせば 示させる しめさせる 示される しめされる 示せ しめせ 示せる しめせる 示そう しめそう				動詞-一般 
\\	「女性(じょせい)」のアクセントは?102	女性 仮名 
\\	じょせい、にょしょう、にょしょー 漢字 
\\	女性 発音図: 助詞: オ 発音図: 助詞: オ	女性 じょせい		女性 じょせい				名詞-普通名詞-一般 
\\	「相手(あいて)《お〜》」のアクセントは?103	相手(お〜) 仮名 
\\	あいて 漢字 
\\	相手 発音図: 助詞: オ	相手 あいて		相手 あいて				名詞-普通名詞-一般 
\\	「世界(せかい)」のアクセントは?104	世界 仮名 
\\	せかい 漢字 
\\	世界 発音図: 助詞: オ 発音図: 助詞: オ	世界 せかい		世界 せかい				名詞-普通名詞-一般 
\\	「女(おんな)」のアクセントは?105	女 仮名 
\\	おんな 漢字 
\\	女 発音図: 助詞: オ	女 おんな		女 おんな				名詞-普通名詞-一般 
\\	「何故(なぜ)」のアクセントは?106	なぜ{何故} 仮名 
\\	なぜ 漢字 
\\	何故 発音図:							副詞 
\\	「気持ち(きもち)」のアクセントは?107		気持ち きもち		気持ち きもち				名詞-普通名詞-一般 
\\	「得る(える)」のアクセントは?108	得る 仮名 
\\	うる、える 漢字 
\\	得る 発音図: 発音図:	得る える		得る える 得ます えます 得て えて 得た えた 得ない えない 得なかった えなかった 得れば えれば 得させる えさせる 得られる えられる 得ろ えろ 得る える 得よう えよう				動詞-一般 
\\	「図(ず)」のアクセントは?109	図 仮名 
\\	ず 漢字 
\\	図 発音図: 助詞: オ	図 ず		図 ず				名詞-普通名詞-一般 
\\	「返る(かえる)」のアクセントは?110	返る 仮名 
\\	かえる 漢字 
\\	返る 発音図:	返る かえる		返る かえる 返ります かえります 返って かえって 返った かえった 返らない かえらない 返らなかった かえらなかった 返れば かえれば 返らせる かえらせる 返られる かえられる 返れ かえれ 返れる かえれる 返ろう かえろう				動詞-一般 
\\	「平成(へいせい)」のアクセントは?111	平成 仮名 
\\	へいせい 漢字 
\\	平成 発音図: 助詞: オ	平成 へいせい		平成 へいせい				名詞-固有名詞-一般 
\\	「新しい(あたらしい)」のアクセントは?112	新しい 仮名 
\\	あたらしい 漢字 
\\	新しい 発音図:	新しい あたらしい		新しい あたらしい 新しいです あたらしいです 新しくて あたらしくて あたらしくて 新しかった あたらしかった あたらしかった 新しくない あたらしくない あたらしくない 新しくなかった あたらしくなかった あたらしくなかった 新しければ あたらしければ あたらしければ 新しい あたらしい 新しく あたらしく あたらしく				形容詞-一般 
\\	「頭(あたま)」のアクセントは?113	頭 仮名 
\\	あたま 漢字 
\\	頭 発音図: 助詞: オ 発音図: 助詞: オ	頭 あたま		頭 あたま				名詞-普通名詞-一般 
\\	「方法(ほうほう)」のアクセントは?115	方法 仮名 
\\	ほうほう、ほーほー 漢字 
\\	方法 発音図: 助詞: オ	方法 ほうほう		方法 ほうほう				名詞-普通名詞-一般 
\\	「感ずる(かんずる)」のアクセントは?116	感ずる 仮名 
\\	かんずる 漢字 
\\	感ずる 発音図:							動詞-一般 
\\	「体(からだ)《お〜》」のアクセントは?117	体(お〜) 仮名 
\\	からだ 漢字 
\\	体 発音図: 助詞: オ	体 からだ		体 からだ				名詞-普通名詞-一般 
\\	「早い(はやい)」のアクセントは?118	早い 仮名 
\\	はやい 漢字 
\\	早い 発音図:	早い はやい		早い はやい 早いです はやいです 早くて はやくて はやくて 早かった はやかった はやかった 早くない はやくない はやくない 早くなかった はやくなかった はやくなかった 早ければ はやければ はやければ 早い はやい 早く はやく はやく				形容詞-一般 
\\	「好きな(すきな)《お〜》」のアクセントは?119	好き(お〜) 仮名 
\\	すき 漢字 
\\	好き 発音図: 例文: 君のナ映画はどんなのかな	好きな すきな		好きな すきな 好きです すきです 好きで すきで 好きだった すきだった 好きじゃない すきじゃない 好きじゃなかった すきじゃなかった 好きならば すきならば 好きだ すきだ 好きに すきに				形状詞-一般 
\\	「点(てん)」のアクセントは?120	点 仮名 
\\	てん 漢字 
\\	点 発音図: 助詞: オ	点 てん		点 てん				名詞-普通名詞-一般 
\\	「一緒(いっしょ)《〜に》」のアクセントは?121	いっしょ{一緒}(〜に) 仮名 
\\	いっしょ 漢字 
\\	一緒 発音図: 助詞: ニ	一緒 いっしょ		一緒 いっしょ				名詞-普通名詞-一般 
\\	「子(こ)」のアクセントは?122	子 仮名 
\\	こ 漢字 
\\	子 発音図: 例文: あの母親はニべったりだ	子 こ		子 こ				名詞-普通名詞-一般 
\\	「形(かたち)」のアクセントは?124	形 仮名 
\\	かた、かたち 漢字 
\\	形 発音図: 助詞: オ 発音図: 助詞: オ	形 かたち		形 かたち				名詞-普通名詞-一般 
\\	「だから(だから)《であるから》」のアクセントは?125	だから(であるから) 仮名 
\\	だから 漢字 
\\	発音図:							接続詞 
\\	「勿論(もちろん)」のアクセントは?126	もちろん{勿論} 仮名 
\\	もちろん 漢字 
\\	勿論 発音図:							副詞 
\\	「場所(ばしょ)」のアクセントは?127	場所 仮名 
\\	ばしょ 漢字 
\\	場所 発音図: 助詞: オ	場所 ばしょ		場所 ばしょ				名詞-普通名詞-一般 
\\	「与える(あたえる)」のアクセントは?128	与える 仮名 
\\	あたえる 漢字 
\\	与える 発音図:	与える あたえる		与える あたえる 与えます あたえます 与えて あたえて 与えた あたえた 与えない あたえない 与えなかった あたえなかった 与えれば あたえれば 与えさせる あたえさせる 与えられる あたえられる 与えろ あたえろ 与えられる あたえられる 与えよう あたえよう				動詞-一般 
\\	「求める(もとめる)」のアクセントは?129	求める 仮名 
\\	もとめる 漢字 
\\	求める 発音図:	求める もとめる		求める もとめる 求めます もとめます 求めて もとめて 求めた もとめた 求めない もとめない 求めなかった もとめなかった 求めれば もとめれば 求めさせる もとめさせる 求められる もとめられる 求めろ もとめろ 求められる もとめられる 求めよう もとめよう				動詞-一般 
\\	「感じ(かんじ)」のアクセントは?130	感じ 仮名 
\\	かんじ 漢字 
\\	感じ 発音図: 助詞: オ	感じ かんじ		感じ かんじ				名詞-普通名詞-一般 
\\	「既に(すでに)」のアクセントは?131	すでに{已;既} 仮名 
\\	すでに 漢字 
\\	既に、已に 発音図:							副詞 
\\	「内容(ないよう)」のアクセントは?132	内容 仮名 
\\	ないよう、ないよー 漢字 
\\	内容 発音図: 助詞: オ	内容 ないよう		内容 ないよう				名詞-普通名詞-一般 
\\	「車(くるま)《お〜》」のアクセントは?133	車(お〜) 仮名 
\\	くるま 漢字 
\\	車 発音図: 助詞: オ	車 くるま		車 くるま				名詞-普通名詞-一般 
\\	「続く(つづく)」のアクセントは?134	続く 仮名 
\\	つずく、つづく 漢字 
\\	続く 発音図:	続く つづく		続く つづく 続きます つづきます 続いて つづいて 続いた つづいた 続かない つづかない 続かなかった つづかなかった 続けば つづけば 続かせる つづかせる 続かれる つづかれる 続け つづけ 続ける つづける 続こう つづこう				動詞-一般 
\\	「認める(みとめる)」のアクセントは?135	認める 仮名 
\\	みとめる 漢字 
\\	認める 発音図:	認める みとめる		認める みとめる 認めます みとめます 認めて みとめて 認めた みとめた 認めない みとめない 認めなかった みとめなかった 認めれば みとめれば 認めさせる みとめさせる 認められる みとめられる 認めろ みとめろ 認められる みとめられる 認めよう みとめよう				動詞-一般 
\\	「長い(ながい)」のアクセントは?136	長い{永} 仮名 
\\	ながい 漢字 
\\	長い、永い 発音図: 例文: 実にジ間だ	長い ながい		長い ながい 長いです ながいです 長くて ながくて ながくて 長かった ながかった ながかった 長くない ながくない ながくない 長くなかった ながくなかった ながくなかった 長ければ ながければ ながければ 長い ながい 長く ながく ながく				形容詞-一般 
\\	「1度(いちど)《温度;角度など》」のアクセントは?137	1度(温度;角度など) 仮名 
\\	いちど 漢字 
\\	一度、1度 発音図: 例文: 平均気温がタかくなった	一度 いちど		一度 いちど				名詞-数詞 
\\	「1度(いちど)《1回》」のアクセントは?138	1度(1回) 仮名 
\\	いちど 漢字 
\\	一度、1度 発音図: 例文: 彼と会うのはニしたい	一度 いちど		一度 いちど				名詞-数詞 
\\	「一度(いちど)」のアクセントは?139	一度 仮名 
\\	いちど 漢字 
\\	一度 発音図: 例文: 彼と、ユっくり話したい	一度 いちど		一度 いちど				名詞-数詞 
\\	「読む(よむ)」のアクセントは?140	読む 仮名 
\\	よむ 漢字 
\\	読む 発音図:	読む よむ		読む よむ 読みます よみます 読んで よんで 読んだ よんだ 読まない よまない 読まなかった よまなかった 読めば よめば 読ませる よませる 読まれる よまれる 読め よめ 読める よめる 読もう よもう				動詞-一般 
\\	「笑い(わらい)」のアクセントは?141	笑い 仮名 
\\	わらい 漢字 
\\	笑い 発音図: 助詞: オ							名詞-普通名詞-一般 
\\	「初めて(はじめて)」のアクセントは?142	初めて 仮名 
\\	はじめて 漢字 
\\	初めて 発音図:	初めて はじめて		初めて はじめて				副詞 
\\	「水(みず)」のアクセントは?143	水 仮名 
\\	みず 漢字 
\\	水 発音図: 助詞: オ	水 みず		水 みず				名詞-普通名詞-一般 
\\	「開く(ひらく)」のアクセントは?144	開く 仮名 
\\	あく、ひらく 漢字 
\\	開く 発音図: 発音図:	開く ひらく		開く ひらく 開きます ひらきます 開いて ひらいて 開いた ひらいた 開かない ひらかない 開かなかった ひらかなかった 開けば ひらけば 開かせる ひらかせる 開かれる ひらかれる 開け ひらけ 開ける ひらける 開こう ひらこう				動詞-一般 
\\	「含む(ふくむ)」のアクセントは?145	含む 仮名 
\\	ふくむ 漢字 
\\	含む 発音図:	含む ふくむ		含む ふくむ 含みます ふくみます 含んで ふくんで 含んだ ふくんだ 含まない ふくまない 含まなかった ふくまなかった 含めば ふくめば 含ませる ふくませる 含まれる ふくまれる 含め ふくめ 含もう ふくもう				動詞-一般 
\\	「更に(さらに)」のアクセントは?146	さらに{更} 仮名 
\\	さらに 漢字 
\\	更に 発音図:							接続詞 
\\	「力(ちから)」のアクセントは?147	力 仮名 
\\	ちから 漢字 
\\	力 発音図: 助詞: オ	力 ちから		力 ちから				名詞-普通名詞-一般 
\\	「行ける(いける)」のアクセントは?148	行ける 仮名 
\\	いける 漢字 
\\	行ける 発音図:							動詞-一般 
\\	「人々(ひとびと)」のアクセントは?149	人々 仮名 
\\	ひとびと 漢字 
\\	人々 発音図: 助詞: オ	人々 ひとびと		人々 ひとびと				名詞-普通名詞-一般 
\\	「写真(しゃしん)」のアクセントは?150	写真 仮名 
\\	しゃしん 漢字 
\\	写真 発音図: 助詞: オ	写真 しゃしん		写真 しゃしん				名詞-普通名詞-一般 
\\	「少ない(すくない)」のアクセントは?151	少ない 仮名 
\\	すくない 漢字 
\\	少ない 発音図:	少ない すくない		少ない すくない 少ないです すくないです 少なくて すくなくて すくなくて 少なかった すくなかった すくなかった 少なくない すくなくない すくなくない 少なくなかった すくなくなかった すくなくなかった 少なければ すくなければ すくなければ 少ない すくない 少なく すくなく すくなく				形容詞-一般 
\\	「最後(さいご)」のアクセントは?152	最後 仮名 
\\	さいご 漢字 
\\	最後 発音図: 助詞: オ	最後 さいご		最後 さいご				名詞-普通名詞-一般 
\\	「口(くち)」のアクセントは?153	口 仮名 
\\	くち 漢字 
\\	口 発音図: 助詞: オ	口 くち		口 くち				名詞-普通名詞-一般 
\\	「状況(じょうきょう)」のアクセントは?154	状況 仮名 
\\	じょうきょう、じょーきょー 漢字 
\\	状況 発音図: 助詞: オ	状況 じょうきょう		状況 じょうきょう				名詞-普通名詞-一般 
\\	「アメリカ(あめりか)」のアクセントは?155		アメリカ あめりか		アメリカ あめりか				名詞-固有名詞-地名-国 
\\	「理由(りゆう)」のアクセントは?156	理由 仮名 
\\	りゆー、りゆう 漢字 
\\	理由 発音図: 助詞: オ	理由 りゆう		理由 りゆう				名詞-普通名詞-一般 
\\	「乗る(のる)」のアクセントは?157	乗る 仮名 
\\	のる 漢字 
\\	乗る 発音図:	乗る のる		乗る のる 乗ります のります 乗って のって 乗った のった 乗らない のらない 乗らなかった のらなかった 乗れば のれば 乗らせる のらせる 乗られる のられる 乗れ のれ 乗れる のれる 乗ろう のろう				動詞-一般 
\\	「姿(すがた)」のアクセントは?158	姿 仮名 
\\	すがた 漢字 
\\	姿 発音図: 助詞: オ	姿 すがた		姿 すがた				名詞-普通名詞-一般 
\\	「最も(もっとも)」のアクセントは?159	最も 仮名 
\\	もっとも 漢字 
\\	最も 発音図: 発音図:							副詞 
\\	「会社(かいしゃ)」のアクセントは?160	会社 仮名 
\\	かいしゃ 漢字 
\\	会社 発音図: 助詞: オ	会社 かいしゃ		会社 かいしゃ				名詞-普通名詞-一般 
\\	「戻る(もどる)」のアクセントは?161	戻る 仮名 
\\	もどる 漢字 
\\	戻る 発音図:	戻る もどる		戻る もどる 戻ります もどります 戻って もどって 戻った もどった 戻らない もどらない 戻らなかった もどらなかった 戻れば もどれば 戻らせる もどらせる 戻られる もどられる 戻れ もどれ 戻れる もどれる 戻ろう もどろう				動詞-一般 
\\	「重要な(じゅうような)」のアクセントは?162		重要な じゅうような		重要な じゅうような 重要です じゅうようです 重要で じゅうようで 重要だった じゅうようだった 重要じゃない じゅうようじゃない 重要じゃなかった じゅうようじゃなかった 重要ならば じゅうようならば 重要だ じゅうようだ 重要に じゅうように				形状詞-一般 
\\	「重要(じゅうよう)」のアクセントは?163	重要 仮名 
\\	じゅうよう、じゅーよー 漢字 
\\	重要 発音図: 助詞: オ	重要 じゅうよう		重要 じゅうよう				形状詞-一般 
\\	「言う(ゆう)」のアクセントは?164	言う 仮名 
\\	ゆう 漢字 
\\	言う 発音図:							動詞-一般 
\\	「飲む(のむ)」のアクセントは?165	飲む 仮名 
\\	のむ 漢字 
\\	飲む 発音図:	飲む のむ		飲む のむ 飲みます のみます 飲んで のんで 飲んだ のんだ 飲まない のまない 飲まなかった のまなかった 飲めば のめば 飲ませる のませる 飲まれる のまれる 飲め のめ 飲める のめる 飲もう のもう				動詞-一般 
\\	「下(した)《…の〜》」のアクセントは?166	下(…の〜) 仮名 
\\	した 漢字 
\\	下 発音図: 例文: 机のニかくれる							名詞-普通名詞-一般 
\\	「下(した)」のアクセントは?167		下 した した		下 した した				名詞-普通名詞-一般 
\\	「何(なに)」のアクセントは?168	何 仮名 
\\	なに 漢字 
\\	何 発音図: 助詞: オ	何 なに		何 なに				代名詞 
\\	「何(なに)《〜構うものか》」のアクセントは?169	なに{何}(〜構うものか) 仮名 
\\	なに 漢字 
\\	何 発音図: 例文: カまうものか	何 なに		何 なに				代名詞 
\\	「部屋(へや)《お〜》」のアクセントは?170	部屋(お〜) 仮名 
\\	へや 漢字 
\\	部屋 発音図: 助詞: オ	部屋 へや		部屋 へや				名詞-普通名詞-一般 
\\	「向かう(むかう)」のアクセントは?171	向かう 仮名 
\\	むかう 漢字 
\\	向かう 発音図:	向かう むかう		向かう むかう 向かいます むかいます 向かって むかって 向かった むかった 向かわない むかわない 向かわなかった むかわなかった 向かえば むかえば 向かわせる むかわせる 向かわれる むかわれる 向かえ むかえ 向かえる むかえる 向かおう むかおう				動詞-一般 
\\	「図る(はかる)」のアクセントは?172	図る 仮名 
\\	はかる 漢字 
\\	図る 発音図:	図る はかる		図る はかる 図ります はかります 図って はかって 図った はかった 図らない はからない 図らなかった はからなかった 図れば はかれば 図らせる はからせる 図られる はかられる 図れ はかれ 図れる はかれる 図ろう はかろう				動詞-一般 
\\	「生きる(いきる)」のアクセントは?173	生きる 仮名 
\\	いきる 漢字 
\\	生きる 発音図:	生きる いきる		生きる いきる 生きます いきます 生きて いきて 生きた いきた 生きない いきない 生きなかった いきなかった 生きれば いきれば 生きさせる いきさせる 生きられる いきられる 生きろ いきろ 生きられる いきられる 生きよう いきよう				動詞-一般 
\\	「通り(とおり)《道》」のアクセントは?174	通り(道) 仮名 
\\	とーり、とおり 漢字 
\\	通り 発音図: 助詞: オ	通り とおり		通り とおり				名詞-普通名詞-一般 
\\	「終わる(おわる)」のアクセントは?175	終わる 仮名 
\\	おわる 漢字 
\\	終わる 発音図:	終わる おわる		終わる おわる 終わります おわります 終わって おわって 終わった おわった 終わらない おわらない 終わらなかった おわらなかった 終われば おわれば 終わらせる おわらせる 終わられる おわられる 終われ おわれ 終われる おわれる 終わろう おわろう				動詞-一般 
\\	「待つ(まつ)」のアクセントは?176	待つ 仮名 
\\	まつ 漢字 
\\	待つ 発音図:	待つ まつ		待つ まつ 待ちます まちます 待って まって 待った まった 待たない またない 待たなかった またなかった 待てば まてば 待たせる またせる 待たれる またれる 待て まて 待てる まてる 待とう まとう				動詞-一般 
\\	「話す(はなす)」のアクセントは?177	話す 仮名 
\\	はなす 漢字 
\\	話す 発音図:	話す はなす		話す はなす 話します はなします 話して はなして 話した はなした 話さない はなさない 話さなかった はなさなかった 話せば はなせば 話させる はなさせる 話される はなされる 話せ はなせ 話せる はなせる 話そう はなそう				動詞-一般 
\\	「もっと(もっと)《〜ください》」のアクセントは?178	もっと(〜ください) 仮名 
\\	もっと 漢字 
\\	発音図: 例文: クださい							副詞 
\\	「状態(じょうたい)」のアクセントは?179	状態 仮名 
\\	じょうたい、じょーたい 漢字 
\\	状態 発音図: 助詞: オ	状態 じょうたい		状態 じょうたい				名詞-普通名詞-一般 
\\	「六(ろく)」のアクセントは?180	
\\	仮名 
\\	ろく 漢字 
\\	六、6 発音図: 例文: 数字のワ5のあと7の前だ	六 ろく		六 ろく				名詞-数詞 
\\	「非常な(ひじょうな)」のアクセントは?181		非常な ひじょうな		非常な ひじょうな 非常です ひじょうです 非常で ひじょうで 非常だった ひじょうだった 非常じゃない ひじょうじゃない 非常じゃなかった ひじょうじゃなかった 非常ならば ひじょうならば 非常だ ひじょうだ 非常に ひじょうに				形状詞-一般 
\\	「非常(ひじょう)」のアクセントは?182	非常 仮名 
\\	ひじょう、ひじょー 漢字 
\\	非常 発音図: 助詞: オ	非常 ひじょう		非常 ひじょう				形状詞-一般 
\\	「答える(こたえる)」のアクセントは?183	答える 仮名 
\\	こたえる 漢字 
\\	答える 発音図: 発音図:	答える こたえる こたえる		答える こたえる こたえる 答えます こたえます 答えて こたえて 答えた こたえた 答えない こたえない 答えなかった こたえなかった 答えれば こたえれば こたえれば 答えさせる こたえさせる 答えられる こたえられる 答えろ こたえろ こたえろ 答えられる こたえられる 答えよう こたえよう				動詞-一般 
\\	「身(み)」のアクセントは?184	身 仮名 
\\	み 漢字 
\\	身 発音図: 例文: このままではガもたない	身 み		身 み				名詞-普通名詞-一般 
\\	「然う(そう)《〜いう》」のアクセントは?185	そう{然}(〜いう) 仮名 
\\	そう、そー 漢字 
\\	然う 発音図: 例文: イう							名詞-助動詞語幹 
\\	「然(そう)《〜だ》」のアクセントは?186	そう{然}(〜だ) 仮名 
\\	そう、そー 漢字 
\\	然 発音図: 助詞: ダ							名詞-助動詞語幹 
\\	「部分(ぶぶん)」のアクセントは?187	部分 仮名 
\\	ぶぶん 漢字 
\\	部分 発音図: 助詞: オ	部分 ぶぶん		部分 ぶぶん				名詞-普通名詞-一般 
\\	「会う(あう)」のアクセントは?188	会う 仮名 
\\	あう 漢字 
\\	会う 発音図:	会う あう		会う あう 会います あいます 会って あって 会った あった 会わない あわない 会わなかった あわなかった 会えば あえば 会わせる あわせる 会われる あわれる 会え あえ 会える あえる 会おう あおう				動詞-一般 
\\	「情報(じょうほう)」のアクセントは?189	情報 仮名 
\\	じょうほう、じょーほー 漢字 
\\	情報 発音図: 助詞: オ	情報 じょうほう		情報 じょうほう				名詞-普通名詞-一般 
\\	「対象(たいしょう)」のアクセントは?190	対象 仮名 
\\	たいしょう、たいしょー 漢字 
\\	対象 発音図: 助詞: オ	対象 たいしょう		対象 たいしょう				名詞-普通名詞-一般 
\\	「見せる(みせる)」のアクセントは?191	見せる 仮名 
\\	みせる 漢字 
\\	見せる 発音図:	見せる みせる		見せる みせる 見せます みせます 見せて みせて 見せた みせた 見せない みせない 見せなかった みせなかった 見せれば みせれば 見せさせる みせさせる 見せられる みせられる 見せろ みせろ 見せられる みせられる 見せよう みせよう				動詞-一般 
\\	「残る(のこる)」のアクセントは?192	残る 仮名 
\\	のこる 漢字 
\\	残る 発音図:	残る のこる		残る のこる 残ります のこります 残って のこって 残った のこった 残らない のこらない 残らなかった のこらなかった 残れば のこれば 残らせる のこらせる 残られる のこられる 残れ のこれ 残ろう のころう				動詞-一般 
\\	「変える(かえる)」のアクセントは?193	変える 仮名 
\\	かえる 漢字 
\\	変える 発音図:	変える かえる		変える かえる 変えます かえます 変えて かえて 変えた かえた 変えない かえない 変えなかった かえなかった 変えれば かえれば 変えさせる かえさせる 変えられる かえられる 変えろ かえろ 変えられる かえられる 変えよう かえよう				動詞-一般 
\\	「時代(じだい)」のアクセントは?194	時代 仮名 
\\	じだい 漢字 
\\	時代 発音図: 助詞: オ	時代 じだい		時代 じだい				名詞-普通名詞-一般 
\\	「地域(ちいき)」のアクセントは?195	地域 仮名 
\\	ちいき 漢字 
\\	地域 発音図: 助詞: オ	地域 ちいき		地域 ちいき				名詞-普通名詞-一般 
\\	「名前(なまえ)」のアクセントは?196	名前 仮名 
\\	なまえ 漢字 
\\	名前 発音図: 助詞: オ	名前 なまえ		名前 なまえ				名詞-普通名詞-一般 
\\	「蚤(のみ)」のアクセントは?197	のみ{蚤} 仮名 
\\	のみ 漢字 
\\	蚤 発音図: 助詞: オ							助詞-副助詞 
\\	「鑿(のみ)」のアクセントは?198	のみ{鑿} 仮名 
\\	のみ 漢字 
\\	鑿 発音図: 助詞: オ							助詞-副助詞 
\\	「死ぬ(しぬ)」のアクセントは?199	死ぬ 仮名 
\\	しぬ 漢字 
\\	死ぬ 発音図:	死ぬ しぬ		死ぬ しぬ 死にます しにます 死んで しんで 死んだ しんだ 死なない しなない 死ななかった しななかった 死ねば しねば 死なせる しなせる 死なれる しなれる 死ね しね 死ねる しねる 死のう しのう				動詞-一般 
\\	「先生(せんせい)」のアクセントは?200	先生 仮名 
\\	せんせい 漢字 
\\	先生 発音図: 助詞: オ	先生 せんせい		先生 せんせい				名詞-普通名詞-一般 
\\	「中心(ちゅうしん)」のアクセントは?201	中心 仮名 
\\	ちゅーしん、ちゅうしん 漢字 
\\	中心 発音図: 助詞: オ	中心 ちゅうしん		中心 ちゅうしん				名詞-普通名詞-一般 
\\	「目的(もくてき)」のアクセントは?202	目的 仮名 
\\	もくてき 漢字 
\\	目的 発音図: 助詞: オ	目的 もくてき		目的 もくてき				名詞-普通名詞-一般 
\\	「合わせる(あわせる)」のアクセントは?203	合わせる 仮名 
\\	あわせる 漢字 
\\	合わせる 発音図:	合わせる あわせる		合わせる あわせる 合わせます あわせます 合わせて あわせて 合わせた あわせた 合わせない あわせない 合わせなかった あわせなかった 合わせれば あわせれば 合わせさせる あわせさせる 合わせられる あわせられる 合わせろ あわせろ 合わせられる あわせられる 合わせよう あわせよう				動詞-一般 
\\	「巧い(うまい)」のアクセントは?204	うまい{上手;巧;旨;甘;美味} 仮名 
\\	うまい 漢字 
\\	旨い、甘い、美味い、上手い、巧い 発音図:	上手い うまい		上手い うまい 上手いです うまいです 上手くて うまくて うまくて 上手かった うまかった うまかった 上手くない うまくない うまくない 上手くなかった うまくなかった うまくなかった 上手ければ うまければ うまければ 上手い うまい 上手く うまく うまく				形容詞-一般 
\\	「生まれる(うまれる)」のアクセントは?205	生まれる 仮名 
\\	うまれる 漢字 
\\	生まれる 発音図:	生まれる うまれる		生まれる うまれる 生まれます うまれます 生まれて うまれて 生まれた うまれた 生まれない うまれない 生まれなかった うまれなかった 生まれれば うまれれば 生まれさせる うまれさせる 生まれられる うまれられる 生まれろ うまれろ 生まれよう うまれよう				動詞-一般 
\\	「定める(さだめる)」のアクセントは?206	定める 仮名 
\\	さだめる 漢字 
\\	定める 発音図:	定める さだめる		定める さだめる 定めます さだめます 定めて さだめて 定めた さだめた 定めない さだめない 定めなかった さだめなかった 定めれば さだめれば 定めさせる さだめさせる 定められる さだめられる 定めろ さだめろ 定められる さだめられる 定めよう さだめよう				動詞-一般 
\\	「始まる(はじまる)」のアクセントは?207	始まる 仮名 
\\	はじまる 漢字 
\\	始まる 発音図:	始まる はじまる		始まる はじまる 始まります はじまります 始まって はじまって 始まった はじまった 始まらない はじまらない 始まらなかった はじまらなかった 始まれば はじまれば 始まらせる はじまらせる 始まられる はじまられる 始まれ はじまれ 始まろう はじまろう				動詞-一般 
\\	「吐く(つく)《うそを〜》」のアクセントは?208	つく{吐}(うそを〜) 仮名 
\\	つく 漢字 
\\	吐く 発音図: 例文: 嘘をコとはいけません 発音図: 例文: 嘘をコとはいけません							動詞-一般 
\\	「搗く(つく)」のアクセントは?209	つく{搗} 仮名 
\\	つく 漢字 
\\	搗く 発音図: 例文: 最近はもちをイえは少ない 発音図: 例文: 最近はもちをイえは少ない							動詞-一般 
\\	「撞く(つく)」のアクセントは?210	つく{撞} 仮名 
\\	つく 漢字 
\\	撞く 発音図: 例文: 今度は君が球をバんだ 発音図: 例文: 今度は君が球をバんだ							動詞-一般 
\\	「点く(つく)《明かりが〜》」のアクセントは?211	つく{点}(明かりが〜) 仮名 
\\	つく 漢字 
\\	点く 発音図: 例文: 明かりがコろに帰ってくる 発音図: 例文: 明かりがコろに帰ってくる	点く つく		点く つく 点きます つきます 点いて ついて 点いた ついた 点かない つかない 点かなかった つかなかった 点けば つけば 点かせる つかせる 点かれる つかれる 点け つけ 点こう つこう				動詞-一般 
\\	「即く(つく)《位に〜》」のアクセントは?212	つく{即}(位に〜) 仮名 
\\	つく 漢字 
\\	即く 発音図: 例文: 社長のポストにヒとを選ぶ 発音図: 例文: 社長のポストにヒとを選ぶ							動詞-一般 
\\	「憑く(つく)」のアクセントは?213	つく{憑} 仮名 
\\	つく 漢字 
\\	憑く 発音図: 例文: 昔はもののけがトいった							動詞-一般 
\\	「然も(しかも)」のアクセントは?214	しかも{然;而} 仮名 
\\	しかも 漢字 
\\	然も、而も 発音図:							接続詞 
\\	「母(はは)」のアクセントは?215	母 仮名 
\\	はは 漢字 
\\	母 発音図: 助詞: オ	母 はは		母 はは				名詞-普通名詞-一般 
\\	「加える(くわえる)」のアクセントは?216	加える 仮名 
\\	くわえる 漢字 
\\	加える 発音図: 発音図:	加える くわえる		加える くわえる 加えます くわえます 加えて くわえて 加えた くわえた 加えない くわえない 加えなかった くわえなかった 加えれば くわえれば 加えさせる くわえさせる 加えられる くわえられる 加えろ くわえろ 加えられる くわえられる 加えよう くわえよう				動詞-一般 
\\	「因る(よる)」のアクセントは?217	よる{由;因;拠} 仮名 
\\	よる 漢字 
\\	因る、由る、拠る 発音図:	因る よる		因る よる 因ります よります 因って よって 因った よった 因らない よらない 因らなかった よらなかった 因れば よれば 因らせる よらせる 因られる よられる 因れ よれ 因ろう よろう				動詞-一般 
\\	「当たる(あたる)」のアクセントは?218	当たる 仮名 
\\	あたる 漢字 
\\	当たる 発音図:	当たる あたる		当たる あたる 当たります あたります 当たって あたって 当たった あたった 当たらない あたらない 当たらなかった あたらなかった 当たれば あたれば 当たらせる あたらせる 当たられる あたられる 当たれ あたれ 当たれる あたれる 当たろう あたろう				動詞-一般 
\\	「進む(すすむ)」のアクセントは?219	進む 仮名 
\\	すすむ 漢字 
\\	進む 発音図:	進む すすむ		進む すすむ 進みます すすみます 進んで すすんで 進んだ すすんだ 進まない すすまない 進まなかった すすまなかった 進めば すすめば 進ませる すすませる 進まれる すすまれる 進め すすめ 進める すすめる 進もう すすもう				動詞-一般 
\\	「皆さん(みなさん)」のアクセントは?220	皆さん 仮名 
\\	みなさん 漢字 
\\	皆さん 発音図: 助詞: オ	皆さん みなさん		皆さん みなさん				名詞-普通名詞-一般 
\\	「切る(きる)」のアクセントは?221	切る 仮名 
\\	きる 漢字 
\\	切る 発音図:	切る きる		切る きる 切ります きります 切って きって 切った きった 切らない きらない 切らなかった きらなかった 切れば きれば 切らせる きらせる 切られる きられる 切れ きれ 切れる きれる 切ろう きろう				動詞-一般 
\\	「道(みち)」のアクセントは?222	道 仮名 
\\	どう、どー、みち 漢字 
\\	道 発音図: 助詞: オ 発音図: 助詞: オ	道 みち		道 みち				名詞-普通名詞-一般 
\\	「花(はな)《お〜》」のアクセントは?223	花(お〜) 仮名 
\\	はな 漢字 
\\	花 発音図: 例文: きれいなオ摘んでください	花 はな		花 はな				名詞-普通名詞-一般 
\\	「選ぶ(えらぶ)」のアクセントは?224	選ぶ 仮名 
\\	えらぶ 漢字 
\\	選ぶ 発音図:	選ぶ えらぶ		選ぶ えらぶ 選びます えらびます 選んで えらんで 選んだ えらんだ 選ばない えらばない 選ばなかった えらばなかった 選べば えらべば 選ばせる えらばせる 選ばれる えらばれる 選べ えらべ 選べる えらべる 選ぼう えらぼう				動詞-一般 
\\	「利用(りようする)」のアクセントは?225	利用 仮名 
\\	りよう、りよー 漢字 
\\	利用 発音図: 助詞: オ	利用する りようする		利用する りようする 利用します りようします 利用して りようして 利用した りようした 利用しない りようしない 利用しなかった りようしなかった 利用すれば りようすれば 利用させる りようさせる 利用される りようされる 利用しろ りようしろ 利用できる りようできる 利用しよう りようしよう				動詞-一般 
\\	「我々(われわれ)」のアクセントは?226	われわれ{我々} 仮名 
\\	われわれ 漢字 
\\	我々 発音図: 助詞: オ	我々 われわれ		我々 われわれ				代名詞 
\\	「基づく(もとづく)」のアクセントは?227	基づく 仮名 
\\	もとずく、もとづく 漢字 
\\	基づく 発音図:	基づく もとづく		基づく もとづく 基づきます もとづきます 基づいて もとづいて 基づいた もとづいた 基づかない もとづかない 基づかなかった もとづかなかった 基づけば もとづけば 基づかせる もとづかせる 基づかれる もとづかれる 基づけ もとづけ 基づこう もとづこう				動詞-一般 
\\	「足(あし)」のアクセントは?228	足 仮名 
\\	あし 漢字 
\\	足 発音図: 助詞: オ	足 あし		足 あし				名詞-普通名詞-一般 
\\	「述べる(のべる)」のアクセントは?229	述べる 仮名 
\\	のべる 漢字 
\\	述べる 発音図:	述べる のべる		述べる のべる 述べます のべます 述べて のべて 述べた のべた 述べない のべない 述べなかった のべなかった 述べれば のべれば 述べさせる のべさせる 述べられる のべられる 述べろ のべろ 述べられる のべられる 述べよう のべよう				動詞-一般 
\\	「決める(きめる)」のアクセントは?230	決める 仮名 
\\	きめる 漢字 
\\	決める 発音図:	決める きめる		決める きめる 決めます きめます 決めて きめて 決めた きめた 決めない きめない 決めなかった きめなかった 決めれば きめれば 決めさせる きめさせる 決められる きめられる 決めろ きめろ 決められる きめられる 決めよう きめよう				動詞-一般 
\\	「越える(こえる)」のアクセントは?231	越える 仮名 
\\	こえる 漢字 
\\	越える 発音図:	越える こえる		越える こえる 越えます こえます 越えて こえて 越えた こえた 越えない こえない 越えなかった こえなかった 越えれば こえれば 越えさせる こえさせる 越えられる こえられる 越えよ こえよ 越えられる こえられる 越えよう こえよう				動詞-一般 
\\	「可能性(かのうせい)」のアクセントは?232	可能性 仮名 
\\	かのうせい、かのーせい 漢字 
\\	可能性 発音図: 助詞: オ	可能性 かのうせい		可能性 かのうせい				名詞-普通名詞-一般 
\\	「続ける(つづける)」のアクセントは?233	続ける 仮名 
\\	つずける、つづける 漢字 
\\	続ける 発音図:	続ける つづける		続ける つづける 続けます つづけます 続けて つづけて 続けた つづけた 続けない つづけない 続けなかった つづけなかった 続ければ つづければ 続けさせる つづけさせる 続けられる つづけられる 続けろ つづけろ 続けられる つづけられる 続けよう つづけよう				動詞-一般 
\\	「若い(わかい)」のアクセントは?234	若い 仮名 
\\	わかい 漢字 
\\	若い 発音図:	若い わかい		若い わかい 若いです わかいです 若くて わかくて わかくて 若かった わかかった わかかった 若くない わかくない わかくない 若くなかった わかくなかった わかくなかった 若ければ わかければ わかければ 若い わかい 若く わかく わかく				形容詞-一般 
\\	「例(れい)」のアクセントは?236	例 仮名 
\\	れい 漢字 
\\	例 発音図: 助詞: オ	例 れい		例 れい				名詞-普通名詞-一般 
\\	「比べる(くらべる)」のアクセントは?237	比べる 仮名 
\\	くらべる 漢字 
\\	比べる 発音図:	比べる くらべる		比べる くらべる 比べます くらべます 比べて くらべて 比べた くらべた 比べない くらべない 比べなかった くらべなかった 比べれば くらべれば 比べさせる くらべさせる 比べられる くらべられる 比べろ くらべろ 比べられる くらべられる 比べよう くらべよう				動詞-一般 
\\	「凄い(すごい)」のアクセントは?238	すごい{凄} 仮名 
\\	すごい 漢字 
\\	凄い 発音図:	凄い すごい		凄い すごい 凄いです すごいです 凄くて すごくて すごくて 凄かった すごかった すごかった 凄くない すごくない すごくない 凄くなかった すごくなかった すごくなかった 凄ければ すごければ すごければ 凄い すごい 凄く すごく すごく				形容詞-一般 
\\	「向ける(むける)」のアクセントは?239	向ける 仮名 
\\	むける 漢字 
\\	向ける 発音図:	向ける むける		向ける むける 向けます むけます 向けて むけて 向けた むけた 向けない むけない 向けなかった むけなかった 向ければ むければ 向けさせる むけさせる 向けられる むけられる 向けろ むけろ 向けられる むけられる 向けよう むけよう				動詞-一般 
\\	「七(なな)」のアクセントは?240	
\\	仮名 
\\	しち、なな 漢字 
\\	七、7 発音図: 助詞: オ 発音図:	七 なな		七 なな				名詞-数詞 
\\	「実施(じっし)」のアクセントは?241	実施 仮名 
\\	じっし 漢字 
\\	実施 発音図: 助詞: オ	実施 じっし		実施 じっし 実施する じっしする 実施します じっしします 実施して じっしして 実施した じっしした 実施しない じっししない 実施しなかった じっししなかった 実施すれば じっしすれば 実施させる じっしさせる 実施される じっしされる 実施しろ じっししろ 実施できる じっしできる 実施しよう じっししよう				動詞-一般 
\\	「可能な(かのうな)」のアクセントは?242		可能な かのうな		可能な かのうな 可能です かのうです 可能で かのうで 可能だった かのうだった 可能じゃない かのうじゃない 可能じゃなかった かのうじゃなかった 可能ならば かのうならば 可能だ かのうだ 可能に かのうに				形状詞-一般 
\\	「可能(かのう)」のアクセントは?243	可能 仮名 
\\	かのう、かのー 漢字 
\\	可能 発音図: 助詞: オ	可能 かのう		可能 かのう				形状詞-一般 
\\	「簡単な(かんたんな)」のアクセントは?244		簡単な かんたんな		簡単な かんたんな 簡単です かんたんです 簡単で かんたんで 簡単だった かんたんだった 簡単じゃない かんたんじゃない 簡単じゃなかった かんたんじゃなかった 簡単ならば かんたんならば 簡単だ かんたんだ 簡単に かんたんに				形状詞-一般 
\\	「簡単(かんたん)」のアクセントは?245	簡単 仮名 
\\	かんたん 漢字 
\\	簡単 発音図: 助詞: オ	簡単 かんたん		簡単 かんたん				形状詞-一般 
\\	「立てる(たてる)」のアクセントは?246	立てる 仮名 
\\	たてる 漢字 
\\	立てる 発音図:	立てる たてる		立てる たてる 立てます たてます 立てて たてて 立てた たてた 立てない たてない 立てなかった たてなかった 立てれば たてれば 立てさせる たてさせる 立てられる たてられる 立てろ たてろ 立てられる たてられる 立てよう たてよう				動詞-一般 
\\	「起こる(おこる)」のアクセントは?247	起こる 仮名 
\\	おこる 漢字 
\\	起こる 発音図:	起こる おこる		起こる おこる 起こります おこります 起こって おこって 起こった おこった 起こらない おこらない 起こらなかった おこらなかった 起これば おこれば 起こらせる おこらせる 起こられる おこられる 起これ おこれ 起ころう おころう				動詞-一般 
\\	「学校(がっこう)」のアクセントは?248	学校 仮名 
\\	がっこう、がっこー 漢字 
\\	学校 発音図: 助詞: オ	学校 がっこう		学校 がっこう				名詞-普通名詞-一般 
\\	「此方(こちら)」のアクセントは?249	こちら{此方} 仮名 
\\	こちら 漢字 
\\	此方 発音図: 助詞: オ	こちら こちら		こちら こちら				代名詞 
\\	「明らかな(あきらかな)」のアクセントは?250	明らか 仮名 
\\	あきらか 漢字 
\\	明らか 発音図:	明らかな あきらかな		明らかな あきらかな 明らかです あきらかです 明らかで あきらかで 明らかだった あきらかだった 明らかじゃない あきらかじゃない 明らかじゃなかった あきらかじゃなかった 明らかならば あきらかならば 明らかだ あきらかだ 明らかに あきらかに				形状詞-一般 
\\	「小さな(ちーさな)」のアクセントは?251	小さな 仮名 
\\	ちーさな、ちいさな 漢字 
\\	小さな 発音図:							連体詞 
\\	「現在(げんざい)」のアクセントは?252	現在 仮名 
\\	げんざい 漢字 
\\	現在 発音図: 助詞: オ	現在 げんざい		現在 げんざい				副詞 
\\	「送る(おくる)」のアクセントは?253	送る 仮名 
\\	おくる 漢字 
\\	送る 発音図:	送る おくる		送る おくる 送ります おくります 送って おくって 送った おくった 送らない おくらない 送らなかった おくらなかった 送れば おくれば 送らせる おくらせる 送られる おくられる 送れ おくれ 送れる おくれる 送ろう おくろう				動詞-一般 
\\	「働く(はたらく)」のアクセントは?254	働く 仮名 
\\	はたらく 漢字 
\\	働く 発音図:	働く はたらく		働く はたらく 働きます はたらきます 働いて はたらいて 働いた はたらいた 働かない はたらかない 働かなかった はたらかなかった 働けば はたらけば 働かせる はたらかせる 働かれる はたらかれる 働け はたらけ 働ける はたらける 働こう はたらこう				動詞-一般 
\\	「進める(すすめる)」のアクセントは?255	進める 仮名 
\\	すすめる 漢字 
\\	進める 発音図:	進める すすめる		進める すすめる 進めます すすめます 進めて すすめて 進めた すすめた 進めない すすめない 進めなかった すすめなかった 進めれば すすめれば 進めさせる すすめさせる 進められる すすめられる 進めろ すすめろ 進められる すすめられる 進めよう すすめよう				動詞-一般 
\\	「用いる(もちいる)」のアクセントは?256	用いる 仮名 
\\	もちーる、もちいる 漢字 
\\	用いる 発音図: 発音図:	用いる もちいる		用いる もちいる 用います もちいます 用いて もちいて 用いた もちいた 用いない もちいない 用いなかった もちいなかった 用いれば もちいれば 用いさせる もちいさせる 用いられる もちいられる 用いろ もちいろ 用いられる もちいられる 用いよう もちいよう				動詞-一般 
\\	「全て(すべて)」のアクセントは?257		全て すべて		全て すべて				副詞 
\\	「企業(きぎょう)」のアクセントは?258	企業 仮名 
\\	きぎょう、きぎょー 漢字 
\\	企業 発音図: 助詞: オ	企業 きぎょう		企業 きぎょう				名詞-普通名詞-一般 
\\	「父(ちち)」のアクセントは?259	父 仮名 
\\	ちち 漢字 
\\	父 発音図: 助詞: オ 発音図: 助詞: オ	父 ちち ちち		父 ちち ちち				名詞-普通名詞-一般 
\\	「始める(はじめる)」のアクセントは?260	始める 仮名 
\\	はじめる 漢字 
\\	始める 発音図:	始める はじめる		始める はじめる 始めます はじめます 始めて はじめて 始めた はじめた 始めない はじめない 始めなかった はじめなかった 始めれば はじめれば 始めさせる はじめさせる 始められる はじめられる 始めろ はじめろ 始められる はじめられる 始めよう はじめよう				動詞-一般 
\\	「走る(はしる)」のアクセントは?261	走る 仮名 
\\	はしる 漢字 
\\	走る 発音図:	走る はしる		走る はしる 走ります はしります 走って はしって 走った はしった 走らない はしらない 走らなかった はしらなかった 走れば はしれば 走らせる はしらせる 走られる はしられる 走れ はしれ 走れる はしれる 走ろう はしろう				動詞-一般 
\\	「小さい(ちいさい)」のアクセントは?262	小さい 仮名 
\\	ちーさい、ちいさい 漢字 
\\	小さい 発音図:	小さい ちいさい		小さい ちいさい 小さいです ちいさいです 小さくて ちいさくて ちいさくて 小さかった ちいさかった ちいさかった 小さくない ちいさくない ちいさくない 小さくなかった ちいさくなかった ちいさくなかった 小さければ ちいさければ ちいさければ 小さい ちいさい 小さく ちいさく ちいさく				形容詞-一般 
\\	「3月(さんがつ)」のアクセントは?263	3月 仮名 
\\	さんがつ 漢字 
\\	三月、3月 発音図: 助詞: オ	三月 さんがつ		三月 さんがつ				名詞-数詞 
\\	「家族(かぞく)」のアクセントは?264	家族 仮名 
\\	かぞく 漢字 
\\	家族 発音図: 助詞: オ	家族 かぞく		家族 かぞく				名詞-普通名詞-一般 
\\	「大切(たいせつ)」のアクセントは?265	大切 仮名 
\\	たいせつ 漢字 
\\	大切 発音図: 助詞: オ	大切な たいせつな		大切な たいせつな 大切です たいせつです 大切で たいせつで 大切だった たいせつだった 大切じゃない たいせつじゃない 大切じゃなかった たいせつじゃなかった 大切ならば たいせつならば 大切だ たいせつだ 大切に たいせつに				形状詞-一般 
\\	「外(そと)」のアクセントは?266	外 仮名 
\\	そと 漢字 
\\	外 発音図: 助詞: オ	外 そと		外 そと				名詞-普通名詞-一般 
\\	「描く(えがく)」のアクセントは?267	描く 仮名 
\\	えがく 漢字 
\\	描く 発音図:	描く えがく		描く えがく 描きます えがきます 描いて えがいて 描いた えがいた 描かない えがかない 描かなかった えがかなかった 描けば えがけば 描かせる えがかせる 描かれる えがかれる 描け えがけ 描ける えがける 描こう えがこう				動詞-一般 
\\	「色(いろ)」のアクセントは?268	色 仮名 
\\	いろ 漢字 
\\	色 発音図: 助詞: オ	色 いろ		色 いろ				名詞-普通名詞-一般 
\\	「止める(やめる)」のアクセントは?269	やめる{止;罷} 仮名 
\\	やめる 漢字 
\\	止める、罷める 発音図:	止める やめる		止める やめる 止めます やめます 止めて やめて 止めた やめた 止めない やめない 止めなかった やめなかった 止めれば やめれば 止めさせる やめさせる 止められる やめられる 止めろ やめろ 止められる やめられる 止めよう やめよう				動詞-一般 
\\	「但し(ただし)」のアクセントは?270	ただし{但} 仮名 
\\	ただし 漢字 
\\	但し 発音図:							接続詞 
\\	「笑う(わらう)」のアクセントは?271	笑う 仮名 
\\	わらう 漢字 
\\	笑う 発音図:	笑う わらう		笑う わらう 笑います わらいます 笑って わらって 笑った わらった 笑わない わらわない 笑わなかった わらわなかった 笑えば わらえば 笑わせる わらわせる 笑われる わらわれる 笑え わらえ 笑える わらえる 笑おう わらおう				動詞-一般 
\\	「二つ(ふたつ)《副》」のアクセントは?272	二つ(副) 仮名 
\\	ふたつ 漢字 
\\	二つ 発音図: 例文: その花瓶はトも偽物だ	二つ ふたつ		二つ ふたつ				名詞-数詞 
\\	「二つ(ふたつ)《名》」のアクセントは?273	二つ(名) 仮名 
\\	ふたつ 漢字 
\\	二つ 発音図: 例文: 花びんガある	二つ ふたつ		二つ ふたつ				名詞-数詞 
\\	「引く(ひく)」のアクセントは?274	引く{退} 仮名 
\\	ひく 漢字 
\\	引く、退く 発音図:	引く ひく		引く ひく 引きます ひきます 引いて ひいて 引いた ひいた 引かない ひかない 引かなかった ひかなかった 引けば ひけば 引かせる ひかせる 引かれる ひかれる 引け ひけ 引ける ひける 引こう ひこう				動詞-一般 
\\	「難しい(むずかしい)」のアクセントは?275	難しい 仮名 
\\	むずかしい 漢字 
\\	難しい 発音図: 発音図:	難しい むずかしい むずかしい		難しい むずかしい むずかしい 難しいです むずかしいです 難しくて むずかしくて むずかしくて 難しかった むずかしかった むずかしかった 難しくない むずかしくない むずかしくない むずかしくない 難しくなかった むずかしくなかった むずかしくなかった むずかしくなかった 難しければ むずかしければ むずかしければ 難しい むずかしい むずかしい 難しく むずかしく むずかしく むずかしく				形容詞-一般 
\\	「使用(しよう)」のアクセントは?276	使用 仮名 
\\	しよう、しよー 漢字 
\\	使用 発音図: 助詞: オ	使用 しよう		使用 しよう 使用する しようする 使用します しようします 使用して しようして 使用した しようした 使用しない しようしない 使用しなかった しようしなかった 使用すれば しようすれば 使用させる しようさせる 使用される しようされる 使用しろ しようしろ 使用できる しようできる 使用しよう しようしよう				動詞-一般 
\\	「楽しい(たのしい)」のアクセントは?277	楽しい 仮名 
\\	たのしい 漢字 
\\	楽しい 発音図:	楽しい たのしい		楽しい たのしい 楽しいです たのしいです 楽しくて たのしくて たのしくて 楽しかった たのしかった たのしかった 楽しくない たのしくない たのしくない 楽しくなかった たのしくなかった たのしくなかった 楽しければ たのしければ たのしければ 楽しい たのしい 楽しく たのしく たのしく				形容詞-一般 
\\	「同時(どうじ)」のアクセントは?278	同時 仮名 
\\	どうじ、どーじ 漢字 
\\	同時 発音図: 助詞: オ 発音図: 助詞: オ	同時 どうじ		同時 どうじ				名詞-普通名詞-一般 
\\	「ずっと(ずっと)」のアクセントは?279	ずっと 仮名 
\\	ずっと 漢字 
\\	発音図:							副詞 
\\	「美味しい(おいしい)」のアクセントは?280	おいしい{美味} 仮名 
\\	おいしい 漢字 
\\	美味しい 発音図: 発音図:	美味しい おいしい おいしい		美味しい おいしい おいしい 美味しいです おいしいです 美味しくて おいしくて おいしくて 美味しかった おいしかった おいしかった 美味しくない おいしくない おいしくない おいしくない 美味しくなかった おいしくなかった おいしくなかった おいしくなかった 美味しければ おいしければ おいしければ 美味しい おいしい おいしい 美味しく おいしく おいしく おいしく				形容詞-一般 
\\	「表(ひょう)」のアクセントは?281		表 ひょう		表 ひょう				名詞-普通名詞-一般 
\\	「親(おや)」のアクセントは?282	親 仮名 
\\	おや、しん 漢字 
\\	親 発音図: 助詞: オ 発音図: 助詞: オ	親 おや		親 おや				名詞-普通名詞-一般 
\\	「程度(ていど)」のアクセントは?283	程度 仮名 
\\	ていど 漢字 
\\	程度 発音図: 助詞: オ 発音図: 助詞: オ	程度 ていど ていど		程度 ていど ていど				名詞-普通名詞-一般 
\\	「異なる(ことなる)」のアクセントは?284	異なる 仮名 
\\	ことなる 漢字 
\\	異なる 発音図:	異なる ことなる		異なる ことなる 異なります ことなります 異なって ことなって 異なった ことなった 異ならない ことならない 異ならなかった ことならなかった 異なれば ことなれば 異ならせる ことならせる 異なられる ことなられる 異なれ ことなれ 異なろう ことなろう				動詞-一般 
\\	「頑張る(がんばる)」のアクセントは?285	頑張る 仮名 
\\	がんばる 漢字 
\\	頑張る 発音図:	頑張る がんばる		頑張る がんばる 頑張ります がんばります 頑張って がんばって 頑張った がんばった 頑張らない がんばらない 頑張らなかった がんばらなかった 頑張れば がんばれば 頑張らせる がんばらせる 頑張られる がんばられる 頑張れ がんばれ 頑張れる がんばれる 頑張ろう がんばろう				動詞-一般 
\\	「貰う(もらう)」のアクセントは?286	もらう{貰} 仮名 
\\	もらう 漢字 
\\	貰う 発音図:	もらう もらう		もらう もらう もらいます もらいます もらって もらって もらった もらった もらわない もらわない もらわなかった もらわなかった もらえば もらえば もらわせる もらわせる もらわれる もらわれる もらえ もらえ もらえる もらえる もらおう もらおう				動詞-一般 
\\	「必ず(かならず)」のアクセントは?287	必ず 仮名 
\\	かならず 漢字 
\\	必ず 発音図:							副詞 
\\	「気付く(きづく)」のアクセントは?288		気付く きづく		気付く きづく 気付きます きづきます 気付いて きづいて 気付いた きづいた 気付かない きづかない 気付かなかった きづかなかった 気付けば きづけば 気付かせる きづかせる 気付かれる きづかれる 気付け きづけ 気付ける きづける 気付こう きづこう				動詞-一般 
\\	「原因(げんいん)」のアクセントは?289	原因 仮名 
\\	げんいん 漢字 
\\	原因 発音図: 助詞: オ	原因 げんいん		原因 げんいん				名詞-普通名詞-一般 
\\	「作品(さくひん)」のアクセントは?290	作品 仮名 
\\	さくひん 漢字 
\\	作品 発音図: 助詞: オ	作品 さくひん		作品 さくひん				名詞-普通名詞-一般 
\\	「守る(まもる)」のアクセントは?291	守る 仮名 
\\	まもる 漢字 
\\	守る 発音図:	守る まもる		守る まもる 守ります まもります 守って まもって 守った まもった 守らない まもらない 守らなかった まもらなかった 守れば まもれば 守らせる まもらせる 守られる まもられる 守れ まもれ 守れる まもれる 守ろう まもろう				動詞-一般 
\\	「伝える(つたえる)」のアクセントは?292	伝える 仮名 
\\	つたえる 漢字 
\\	伝える 発音図: 発音図:	伝える つたえる		伝える つたえる 伝えます つたえます 伝えて つたえて 伝えた つたえた 伝えない つたえない 伝えなかった つたえなかった 伝えれば つたえれば 伝えさせる つたえさせる 伝えられる つたえられる 伝えろ つたえろ 伝えられる つたえられる 伝えよう つたえよう				動詞-一般 
\\	「近い(ちかい)」のアクセントは?293	近い 仮名 
\\	ちかい 漢字 
\\	近い 発音図:	近い ちかい		近い ちかい 近いです ちかいです 近くて ちかくて ちかくて 近かった ちかかった ちかかった 近くない ちかくない ちかくない 近くなかった ちかくなかった ちかくなかった 近ければ ちかければ ちかければ 近い ちかい 近く ちかく ちかく				形容詞-一般 
\\	「娘(むすめ)」のアクセントは?294	娘 仮名 
\\	むすめ 漢字 
\\	娘 発音図: 助詞: オ	娘 むすめ		娘 むすめ				名詞-普通名詞-一般 
\\	「覚える(おぼえる)」のアクセントは?295	覚える 仮名 
\\	おぼえる 漢字 
\\	覚える 発音図:	覚える おぼえる		覚える おぼえる 覚えます おぼえます 覚えて おぼえて 覚えた おぼえた 覚えない おぼえない 覚えなかった おぼえなかった 覚えれば おぼえれば 覚えさせる おぼえさせる 覚えられる おぼえられる 覚えろ おぼえろ 覚えられる おぼえられる 覚えよう おぼえよう				動詞-一般 
\\	「低い(ひくい)」のアクセントは?296	低い 仮名 
\\	ひくい 漢字 
\\	低い 発音図:	低い ひくい		低い ひくい 低いです ひくいです 低くて ひくくて ひくくて 低かった ひくかった ひくかった 低くない ひくくない ひくくない 低くなかった ひくくなかった ひくくなかった 低ければ ひくければ ひくければ 低い ひくい 低く ひくく ひくく				形容詞-一般 
\\	「法律(ほうりつ)」のアクセントは?297	法律 仮名 
\\	ほうりつ、ほーりつ 漢字 
\\	法律 発音図: 助詞: オ	法律 ほうりつ		法律 ほうりつ				名詞-普通名詞-一般 
\\	「増える(ふえる)」のアクセントは?298	増える 仮名 
\\	ふえる 漢字 
\\	増える 発音図:	増える ふえる		増える ふえる 増えます ふえます 増えて ふえて 増えた ふえた 増えない ふえない 増えなかった ふえなかった 増えれば ふえれば 増えさせる ふえさせる 増えられる ふえられる 増えろ ふえろ 増えよう ふえよう				動詞-一般 
\\	「住む(すむ)」のアクセントは?299	住む 仮名 
\\	すむ 漢字 
\\	住む 発音図:	住む すむ		住む すむ 住みます すみます 住んで すんで 住んだ すんだ 住まない すまない 住まなかった すまなかった 住めば すめば 住ませる すませる 住まれる すまれる 住め すめ 住める すめる 住もう すもう				動詞-一般 
\\	「東京(とうきょう)」のアクセントは?300	東京 仮名 
\\	とうきょう、とーきょー 漢字 
\\	東京 発音図: 助詞: オ	東京 とうきょう		東京 とうきょう				名詞-固有名詞-地名-一般 
\\	「より(より)《〜早く》」のアクセントは?301	より(〜早く) 仮名 
\\	より 漢字 
\\	発音図: 例文: 列車はハやく走る							副詞 
\\	「縒り(より)《〜を戻す》」のアクセントは?302	より{縒}(〜を戻す) 仮名 
\\	より 漢字 
\\	縒り 発音図: 例文: 離婚した妻とオ戻した							副詞 
\\	「夢(ゆめ)」のアクセントは?303	夢 仮名 
\\	ゆめ 漢字 
\\	夢 発音図: 例文: その夜は、一晩中デうなされた	夢 ゆめ		夢 ゆめ				名詞-普通名詞-一般 
\\	「伴う(ともなう)」のアクセントは?304	伴う 仮名 
\\	ともなう 漢字 
\\	伴う 発音図:	伴う ともなう		伴う ともなう 伴います ともないます 伴って ともなって 伴った ともなった 伴わない ともなわない 伴わなかった ともなわなかった 伴えば ともなえば 伴わせる ともなわせる 伴われる ともなわれる 伴え ともなえ 伴える ともなえる 伴おう ともなおう				動詞-一般 
\\	「様子(ようす)」のアクセントは?305	様子 仮名 
\\	ようす、よーす 漢字 
\\	様子 発音図: 助詞: オ	様子 ようす		様子 ようす				名詞-普通名詞-一般 
\\	「10月(じゅうがつ)」のアクセントは?306	10月 仮名 
\\	じゅーがつ、じゅうがつ 漢字 
\\	十月、10月 発音図: 例文: もうすぐガ来る 発音図: 例文: もうすぐガ来る	十月 じゅうがつ		十月 じゅうがつ				名詞-数詞 
\\	「上がる(あがる)」のアクセントは?307	上がる 仮名 
\\	あがる 漢字 
\\	上がる 発音図:	上がる あがる		上がる あがる 上がります あがります 上がって あがって 上がった あがった 上がらない あがらない 上がらなかった あがらなかった 上がれば あがれば 上がらせる あがらせる 上がられる あがられる 上がれ あがれ 上がれる あがれる 上がろう あがろう				動詞-一般 
\\	「即ち(すなわち)」のアクセントは?308	すなわち{即} 仮名 
\\	すなわち 漢字 
\\	即ち 発音図:							接続詞 
\\	「八(はち)」のアクセントは?309	
\\	仮名 
\\	はち 漢字 
\\	八、8 発音図: 例文: 大きなオ書く	八 はち		八 はち				名詞-数詞 
\\	「頂く(いただく)」のアクセントは?310	頂く 仮名 
\\	いただく 漢字 
\\	頂く 発音図:	頂く いただく		頂く いただく 頂きます いただきます 頂いて いただいて 頂いた いただいた 頂かない いただかない 頂かなかった いただかなかった 頂けば いただけば 頂かせる いただかせる 頂かれる いただかれる 頂け いただけ 頂ける いただける 頂こう いただこう				動詞-一般 
\\	「調べる(しらべる)」のアクセントは?311	調べる 仮名 
\\	しらべる 漢字 
\\	調べる 発音図:	調べる しらべる		調べる しらべる 調べます しらべます 調べて しらべて 調べた しらべた 調べない しらべない 調べなかった しらべなかった 調べれば しらべれば 調べさせる しらべさせる 調べられる しらべられる 調べろ しらべろ 調べられる しらべられる 調べよう しらべよう				動詞-一般 
\\	「打つ(うつ)」のアクセントは?313	打つ 仮名 
\\	うつ 漢字 
\\	打つ 発音図:	打つ うつ		打つ うつ 打ちます うちます 打って うって 打った うった 打たない うたない 打たなかった うたなかった 打てば うてば 打たせる うたせる 打たれる うたれる 打て うて 打てる うてる 打とう うとう				動詞-一般 
\\	「何方(どちら)」のアクセントは?314	どちら{何方} 仮名 
\\	どちら 漢字 
\\	何方 発音図: 助詞: オ	どちら どちら		どちら どちら				代名詞 
\\	「神(かみ)《〜様》」のアクセントは?315	神(〜様) 仮名 
\\	かみ 漢字 
\\	神 発音図: 助詞: オ	神 かみ		神 かみ				名詞-普通名詞-一般 
\\	「店(みせ)《お〜》」のアクセントは?316	店(お〜) 仮名 
\\	みせ 漢字 
\\	店 発音図: 助詞: オ	店 みせ		店 みせ				名詞-普通名詞-一般 
\\	「嬉しい(うれしい)」のアクセントは?317	うれしい{嬉} 仮名 
\\	うれしい 漢字 
\\	嬉しい 発音図:	嬉しい うれしい		嬉しい うれしい 嬉しいです うれしいです 嬉しくて うれしくて うれしくて 嬉しかった うれしかった うれしかった 嬉しくない うれしくない うれしくない 嬉しくなかった うれしくなかった うれしくなかった 嬉しければ うれしければ うれしければ 嬉しい うれしい 嬉しく うれしく うれしく				形容詞-一般 
\\	「4月(しがつ)」のアクセントは?318	4月 仮名 
\\	しがつ 漢字 
\\	四月、4月 発音図: 例文: 3月が終わるとガ来ます 発音図: 例文: 3月が終わるとガ来ます	四月 しがつ		四月 しがつ				名詞-数詞 
\\	「開ける(あける)」のアクセントは?319	開ける 仮名 
\\	あける、ひらける 漢字 
\\	開ける 発音図: 発音図:	開ける あける		開ける あける 開けます あけます 開けて あけて 開けた あけた 開けない あけない 開けなかった あけなかった 開ければ あければ 開けさせる あけさせる 開けられる あけられる 開けろ あけろ 開けられる あけられる 開けよう あけよう				動詞-一般 
\\	「面白い(おもしろい)」のアクセントは?320	おもしろい{面白} 仮名 
\\	おもしろい 漢字 
\\	面白い 発音図:	面白い おもしろい		面白い おもしろい 面白いです おもしろいです 面白くて おもしろくて おもしろくて 面白かった おもしろかった おもしろかった 面白くない おもしろくない おもしろくない 面白くなかった おもしろくなかった おもしろくなかった 面白ければ おもしろければ おもしろければ 面白い おもしろい 面白く おもしろく おもしろく				形容詞-一般 
\\	「残す(のこす)」のアクセントは?321	残す 仮名 
\\	のこす 漢字 
\\	残す 発音図:	残す のこす		残す のこす 残します のこします 残して のこして 残した のこした 残さない のこさない 残さなかった のこさなかった 残せば のこせば 残させる のこさせる 残される のこされる 残せ のこせ 残せる のこせる 残そう のこそう				動詞-一般 
\\	「以下(いか)」のアクセントは?322	以下 仮名 
\\	いか 漢字 
\\	以下 発音図: 助詞: オ	以下 いか		以下 いか				名詞-普通名詞-一般 
\\	「存在(そんざい)」のアクセントは?323	存在 仮名 
\\	そんざい 漢字 
\\	存在 発音図: 助詞: オ	存在 そんざい		存在 そんざい 存在する そんざいする 存在します そんざいします 存在して そんざいして 存在した そんざいした 存在しない そんざいしない 存在しなかった そんざいしなかった 存在すれば そんざいすれば 存在させる そんざいさせる 存在される そんざいされる 存在しろ そんざいしろ 存在できる そんざいできる 存在しよう そんざいしよう				動詞-一般 
\\	「起きる(おきる)」のアクセントは?324	起きる 仮名 
\\	おきる 漢字 
\\	起きる 発音図:	起きる おきる		起きる おきる 起きます おきます 起きて おきて 起きた おきた 起きない おきない 起きなかった おきなかった 起きれば おきれば 起きさせる おきさせる 起きられる おきられる 起きろ おきろ 起きられる おきられる 起きよう おきよう				動詞-一般 
\\	「同様な(どうような)」のアクセントは?325		同様な どうような		同様な どうような 同様です どうようです 同様で どうようで 同様だった どうようだった 同様じゃない どうようじゃない 同様じゃなかった どうようじゃなかった 同様ならば どうようならば 同様だ どうようだ 同様に どうように				形状詞-一般 
\\	「同様(どうよう)」のアクセントは?326	同様 仮名 
\\	どうよう、どーよー 漢字 
\\	同様 発音図: 助詞: オ	同様 どうよう		同様 どうよう				形状詞-一般 
\\	「成す(なす)」のアクセントは?327	成す 仮名 
\\	なす 漢字 
\\	成す 発音図:	成す なす		成す なす 成します なします 成して なして 成した なした 成さない なさない 成さなかった なさなかった 成せば なせば 成させる なさせる 成される なされる 成せ なせ 成せる なせる 成そう なそう				動詞-一般 
\\	「12月(じゅうにがつ)」のアクセントは?328	12月 仮名 
\\	じゅーにがつ、じゅうにがつ 漢字 
\\	十二月、12月 発音図: 例文: もうすぐガ来る 発音図: 例文: もうすぐガ来る	十二月 じゅうにがつ		十二月 じゅうにがつ				名詞-数詞 
\\	「日本人(にほんじん)」のアクセントは?329	日本人 仮名 
\\	にっぽんじん、にほんじん 漢字 
\\	日本人 発音図: 助詞: オ 発音図: 助詞: オ	日本人 にほんじん		日本人 にほんじん				名詞-普通名詞-一般 
\\	「新た(あらた)」のアクセントは?330	新た 仮名 
\\	あらた 漢字 
\\	新た 発音図:	新た あらた		新た あらた				形状詞-一般 
\\	「新たな(あらたな)」のアクセントは?331		新たな あらたな		新たな あらたな 新たです あらたです 新たで あらたで 新ただった あらただった 新たじゃない あらたじゃない 新たじゃなかった あらたじゃなかった 新たならば あらたならば 新ただ あらただ 新たに あらたに				形状詞-一般 
\\	「色々(いろいろ)」のアクセントは?332		色々 いろいろ		色々 いろいろ				形状詞-一般 
\\	「深い(ふかい)」のアクセントは?333	深い 仮名 
\\	ふかい 漢字 
\\	深い 発音図:	深い ふかい		深い ふかい 深いです ふかいです 深くて ふかくて ふかくて 深かった ふかかった ふかかった 深くない ふかくない ふかくない 深くなかった ふかくなかった ふかくなかった 深ければ ふかければ ふかければ 深い ふかい 深く ふかく ふかく				形容詞-一般 
\\	「山(やま)《お〜》」のアクセントは?334	山(お〜) 仮名 
\\	やま 漢字 
\\	山 発音図: 助詞: オ	山 やま		山 やま				名詞-普通名詞-一般 
\\	「動き(うごき)」のアクセントは?335	動き 仮名 
\\	うごき 漢字 
\\	動き 発音図: 助詞: オ	動き うごき		動き うごき				名詞-普通名詞-一般 
\\	「探す(さがす)」のアクセントは?336	探す 仮名 
\\	さがす 漢字 
\\	探す 発音図:	探す さがす		探す さがす 探します さがします 探して さがして 探した さがした 探さない さがさない 探さなかった さがさなかった 探せば さがせば 探させる さがさせる 探される さがされる 探せ さがせ 探せる さがせる 探そう さがそう				動詞-一般 
\\	「夫(おっと)」のアクセントは?337	夫 仮名 
\\	おっと 漢字 
\\	夫 発音図: 助詞: オ	夫 おっと		夫 おっと				名詞-普通名詞-一般 
\\	「妻(つま)」のアクセントは?338	妻 仮名 
\\	さい、つま 漢字 
\\	妻 発音図: 助詞: オ 発音図: 助詞: オ	妻 つま		妻 つま				名詞-普通名詞-一般 
\\	「場(ば)」のアクセントは?339	場 仮名 
\\	じょう、じょー、ば 漢字 
\\	場 発音図: 助詞: オ 発音図:	場 ば		場 ば				名詞-普通名詞-一般 
\\	「奴(やつ)」のアクセントは?340	やつ{奴} 仮名 
\\	やつ 漢字 
\\	奴 発音図: 助詞: オ	奴 やつ		奴 やつ				名詞-普通名詞-一般 
\\	「為せる(させる)」のアクセントは?341	させる{為} 仮名 
\\	させる 漢字 
\\	為せる 発音図:							助動詞 
\\	「男性(だんせい)」のアクセントは?342	男性 仮名 
\\	だんせい 漢字 
\\	男性 発音図: 助詞: オ	男性 だんせい		男性 だんせい				名詞-普通名詞-一般 
\\	「嫌なな(いやな)《〜だ》」のアクセントは?343	嫌(〜だ) 仮名 
\\	いや 漢字 
\\	嫌 発音図: 助詞: ダ	嫌な いやな		嫌な いやな 嫌です いやです 嫌で いやで 嫌だった いやだった 嫌じゃない いやじゃない 嫌じゃなかった いやじゃなかった 嫌ならば いやならば 嫌だ いやだ 嫌に いやに				形状詞-一般 
\\	「数(かず)」のアクセントは?344		数 かず		数 かず				名詞-普通名詞-一般 
\\	「寝る(ねる)」のアクセントは?345	寝る 仮名 
\\	ねる 漢字 
\\	寝る 発音図:	寝る ねる		寝る ねる 寝ます ねます 寝て ねて 寝た ねた 寝ない ねない 寝なかった ねなかった 寝れば ねれば 寝させる ねさせる 寝られる ねられる 寝ろ ねろ 寝られる ねられる 寝よう ねよう				動詞-一般 
\\	「海(うみ)」のアクセントは?346	海 仮名 
\\	うみ 漢字 
\\	海 発音図: 例文: 風の日はガ荒れる	海 うみ		海 うみ				名詞-普通名詞-一般 
\\	「動く(うごく)」のアクセントは?347	動く 仮名 
\\	うごく 漢字 
\\	動く 発音図:	動く うごく		動く うごく 動きます うごきます 動いて うごいて 動いた うごいた 動かない うごかない 動かなかった うごかなかった 動けば うごけば 動かせる うごかせる 動かれる うごかれる 動け うごけ 動ける うごける 動こう うごこう				動詞-一般 
\\	「無くなる(なくなる)」のアクセントは?348	無くなる 仮名 
\\	なくなる 漢字 
\\	無くなる 発音図:	無くなる なくなる		無くなる なくなる 無くなります なくなります 無くなって なくなって 無くなった なくなった 無くならない なくならない 無くならなかった なくならなかった 無くなれば なくなれば 無くならせる なくならせる 無くなられる なくなられる 無くなれ なくなれ 無くなろう なくなろう				動詞-一般 
\\	「1月(いちがつ)」のアクセントは?349	1月 仮名 
\\	いちがつ 漢字 
\\	一月、1月 発音図: 例文: 1年は、カら始まる 発音図: 例文: 1年は、カら始まる							名詞-数詞 
\\	「一月(いちがつ)」のアクセントは?350		一月 いちがつ		一月 いちがつ				名詞-数詞 
\\	「どうして(どうして)」のアクセントは?351	どうして 仮名 
\\	どうして、どーして 漢字 
\\	発音図:							副詞 
\\	「社会(しゃかい)」のアクセントは?352	社会 仮名 
\\	しゃかい 漢字 
\\	社会 発音図: 助詞: オ	社会 しゃかい		社会 しゃかい				名詞-普通名詞-一般 
\\	「6月(ろくがつ)」のアクセントは?353	6月 仮名 
\\	ろくがつ 漢字 
\\	六月、6月 発音図: 例文: 欧米ではワ幸せの月と言う 発音図: 例文: 欧米ではワ幸せの月と言う	六月 ろくがつ		六月 ろくがつ				名詞-数詞 
\\	「意見(いけん)」のアクセントは?354	意見 仮名 
\\	いけん 漢字 
\\	意見 発音図: 助詞: オ	意見 いけん		意見 いけん				名詞-普通名詞-一般 
\\	「語る(かたる)」のアクセントは?355	語る 仮名 
\\	かたる 漢字 
\\	語る 発音図:	語る かたる		語る かたる 語ります かたります 語って かたって 語った かたった 語らない かたらない 語らなかった かたらなかった 語れば かたれば 語らせる かたらせる 語られる かたられる 語れ かたれ 語れる かたれる 語ろう かたろう				動詞-一般 
\\	「従って(したがって)」のアクセントは?356	したがって{従} 仮名 
\\	したがって 漢字 
\\	従って 発音図: 発音図:							接続詞 
\\	「最近(さいきん)」のアクセントは?357	最近 仮名 
\\	さいきん 漢字 
\\	最近 発音図: 助詞: オ	最近 さいきん		最近 さいきん				副詞 
\\	「三十分(さんじっぷん)」のアクセントは?358		三十分 さんじっぷん		三十分 さんじっぷん				名詞-数詞 
\\	「所謂(いわゆる)」のアクセントは?359	いわゆる{所謂} 仮名 
\\	いわゆる 漢字 
\\	所謂 発音図: 発音図:							連体詞 
\\	「光(ひかり)」のアクセントは?360	光 仮名 
\\	ひかり 漢字 
\\	光 発音図: 助詞: オ	光 ひかり		光 ひかり				名詞-普通名詞-一般 
\\	「5月(ごがつ)」のアクセントは?361	5月 仮名 
\\	ごがつ 漢字 
\\	五月、5月 発音図: 助詞: オ	五月 ごがつ		五月 ごがつ				名詞-数詞 
\\	「広い(ひろい)」のアクセントは?362	広い 仮名 
\\	ひろい 漢字 
\\	広い 発音図:	広い ひろい		広い ひろい 広いです ひろいです 広くて ひろくて ひろくて 広かった ひろかった ひろかった 広くない ひろくない ひろくない 広くなかった ひろくなかった ひろくなかった 広ければ ひろければ ひろければ 広い ひろい 広く ひろく ひろく				形容詞-一般 
\\	「8月(はちがつ)」のアクセントは?363	8月 仮名 
\\	はちがつ 漢字 
\\	八月、8月 発音図: 例文: 今年もワ暑くなるだろう 発音図: 例文: 今年もワ暑くなるだろう	八月 はちがつ		八月 はちがつ				名詞-数詞 
\\	「国民(こくみん)」のアクセントは?364	国民 仮名 
\\	くにたみ、こくみん 漢字 
\\	国民 発音図: 助詞: オ 発音図: 助詞: オ	国民 こくみん		国民 こくみん				名詞-普通名詞-一般 
\\	「11月(じゅういちがつ)」のアクセントは?365	11月 仮名 
\\	じゅーいちがつ、じゅういちがつ 漢字 
\\	十一月、11月 発音図: 例文: もうすぐガ来る 発音図: 例文: もうすぐガ来る	十一月 じゅういちがつ		十一月 じゅういちがつ				名詞-数詞 
\\	「町(まち)」のアクセントは?366	町 仮名 
\\	ちょう、ちょー、まち 漢字 
\\	町 発音図: 助詞: オ 発音図:	町 まち		町 まち				名詞-普通名詞-一般 
\\	「離れる(はなれる)」のアクセントは?367	離れる 仮名 
\\	はなれる 漢字 
\\	離れる 発音図:	離れる はなれる		離れる はなれる 離れます はなれます 離れて はなれて 離れた はなれた 離れない はなれない 離れなかった はなれなかった 離れれば はなれれば 離れさせる はなれさせる 離れられる はなれられる 離れろ はなれろ 離れられる はなれられる 離れよう はなれよう				動詞-一般 
\\	「暫く(しばらく)」のアクセントは?368	しばらく{暫} 仮名 
\\	しばらく 漢字 
\\	暫く 発音図:							副詞 
\\	「従う(したがう)」のアクセントは?369	従う 仮名 
\\	したがう 漢字 
\\	従う 発音図: 発音図:	従う したがう		従う したがう 従います したがいます 従って したがって 従った したがった 従わない したがわない 従わなかった したがわなかった 従えば したがえば 従わせる したがわせる 従われる したがわれる 従え したがえ 従える したがえる 従おう したがおう				動詞-一般 
\\	「欲しい(ほしい)」のアクセントは?370	欲しい 仮名 
\\	ほしい 漢字 
\\	欲しい 発音図:	欲しい ほしい		欲しい ほしい 欲しいです ほしいです 欲しくて ほしくて ほしくて 欲しかった ほしかった ほしかった 欲しくない ほしくない ほしくない 欲しくなかった ほしくなかった ほしくなかった 欲しければ ほしければ ほしければ 欲しい ほしい 欲しく ほしく ほしく				形容詞-一般 
\\	「首(くび)」のアクセントは?371	首 仮名 
\\	くび 漢字 
\\	首 発音図: 助詞: オ	首 くび		首 くび				名詞-普通名詞-一般 
\\	「友達(ともだち)」のアクセントは?372	友達 仮名 
\\	ともだち 漢字 
\\	友達 発音図: 助詞: オ	友達 ともだち		友達 ともだち				名詞-普通名詞-一般 
\\	「事件(じけん)」のアクセントは?373	事件 仮名 
\\	じけん 漢字 
\\	事件 発音図: 助詞: オ	事件 じけん		事件 じけん				名詞-普通名詞-一般 
\\	「人生(じんせい)」のアクセントは?374	人生 仮名 
\\	じんせい 漢字 
\\	人生 発音図: 助詞: オ	人生 じんせい		人生 じんせい				名詞-普通名詞-一般 
\\	「占める(しめる)」のアクセントは?375	占める 仮名 
\\	しめる 漢字 
\\	占める 発音図:	占める しめる		占める しめる 占めます しめます 占めて しめて 占めた しめた 占めない しめない 占めなかった しめなかった 占めれば しめれば 占めさせる しめさせる 占められる しめられる 占めろ しめろ 占めよう しめよう				動詞-一般 
\\	「幾つ(いくつ)《お〜》」のアクセントは?376	幾つ(お〜) 仮名 
\\	いくつ 漢字 
\\	幾つ 発音図:	幾つ いくつ		幾つ いくつ				名詞-数詞 
\\	「驚く(おどろく)」のアクセントは?377	驚く 仮名 
\\	おどろく 漢字 
\\	驚く 発音図:	驚く おどろく		驚く おどろく 驚きます おどろきます 驚いて おどろいて 驚いた おどろいた 驚かない おどろかない 驚かなかった おどろかなかった 驚けば おどろけば 驚かせる おどろかせる 驚かれる おどろかれる 驚け おどろけ 驚ける おどろける 驚こう おどろこう				動詞-一般 
\\	「金(かね)」のアクセントは?378		金 かね		金 かね				名詞-普通名詞-一般 
\\	「具体的な(ぐたいてきな)」のアクセントは?379	具体的 仮名 
\\	ぐたいてき 漢字 
\\	具体的 発音図: 助詞: オ	具体的な ぐたいてきな		具体的な ぐたいてきな 具体的です ぐたいてきです 具体的で ぐたいてきで 具体的だった ぐたいてきだった 具体的じゃない ぐたいてきじゃない 具体的じゃなかった ぐたいてきじゃなかった 具体的ならば ぐたいてきならば 具体的だ ぐたいてきだ 具体的に ぐたいてきに				形状詞-一般 
\\	「思える(おもえる)」のアクセントは?380		思える おもえる		思える おもえる 思えます おもえます 思えて おもえて 思えた おもえた 思えない おもえない 思えなかった おもえなかった 思えれば おもえれば 思えよう おもえよう				動詞-一般 
\\	「止める(とめる)」のアクセントは?382	止める 仮名 
\\	とめる 漢字 
\\	止める 発音図:	止める とめる		止める とめる 止めます とめます 止めて とめて 止めた とめた 止めない とめない 止めなかった とめなかった 止めれば とめれば 止めさせる とめさせる 止められる とめられる 止めろ とめろ 止められる とめられる 止めよう とめよう				動詞-一般 
\\	「名(な)」のアクセントは?383	名 仮名 
\\	な 漢字 
\\	名 発音図: 例文: そのワ広く知られている	名 な		名 な				名詞-普通名詞-一般 
\\	「紹介(しょうかい)」のアクセントは?384	紹介 仮名 
\\	しょうかい、しょーかい 漢字 
\\	紹介 発音図: 助詞: オ	紹介 しょうかい		紹介 しょうかい 紹介する しょうかいする 紹介します しょうかいします 紹介して しょうかいして 紹介した しょうかいした 紹介しない しょうかいしない 紹介しなかった しょうかいしなかった 紹介すれば しょうかいすれば 紹介させる しょうかいさせる 紹介される しょうかいされる 紹介しろ しょうかいしろ 紹介できる しょうかいできる 紹介しよう しょうかいしよう				動詞-一般 
\\	「除く(のぞく)」のアクセントは?385	除く 仮名 
\\	のぞく 漢字 
\\	除く 発音図:	除く のぞく		除く のぞく 除きます のぞきます 除いて のぞいて 除いた のぞいた 除かない のぞかない 除かなかった のぞかなかった 除けば のぞけば 除かせる のぞかせる 除かれる のぞかれる 除け のぞけ 除ける のぞける 除こう のぞこう				動詞-一般 
\\	「説明(せつめい)」のアクセントは?386	説明 仮名 
\\	せつめい 漢字 
\\	説明 発音図: 助詞: オ	説明 せつめい		説明 せつめい 説明する せつめいする 説明します せつめいします 説明して せつめいして 説明した せつめいした 説明しない せつめいしない 説明しなかった せつめいしなかった 説明すれば せつめいすれば 説明させる せつめいさせる 説明される せつめいされる 説明しろ せつめいしろ 説明できる せつめいできる 説明しよう せつめいしよう				動詞-一般 
\\	「効果(こうか)」のアクセントは?387	効果 仮名 
\\	こうか、こーか 漢字 
\\	効果 発音図: 助詞: オ	効果 こうか		効果 こうか				名詞-普通名詞-一般 
\\	「胸(むね)」のアクセントは?388	胸 仮名 
\\	むね 漢字 
\\	胸 発音図: 例文: 堂々とオ張って行進した	胸 むね		胸 むね				名詞-普通名詞-一般 
\\	「似る(にる)」のアクセントは?389	似る 仮名 
\\	にる 漢字 
\\	似る 発音図:	似る にる		似る にる 似ます にます 似て にて 似た にた 似ない にない 似なかった になかった 似れば にれば 似させる にさせる 似られる にられる 似ろ にろ 似よう によう				動詞-一般 
\\	「2月(にがつ)」のアクセントは?390	2月 仮名 
\\	にがつ 漢字 
\\	二月、2月 発音図: 例文: きさらぎとワ同じである 発音図: 例文: きさらぎとワ同じである	二月 にがつ		二月 にがつ				名詞-数詞 
\\	「殺す(ころす)」のアクセントは?391	殺す 仮名 
\\	ころす 漢字 
\\	殺す 発音図:	殺す ころす		殺す ころす 殺します ころします 殺して ころして 殺した ころした 殺さない ころさない 殺さなかった ころさなかった 殺せば ころせば 殺させる ころさせる 殺される ころされる 殺せ ころせ 殺せる ころせる 殺そう ころそう				動詞-一般 
\\	「限る(かぎる)」のアクセントは?392	限る 仮名 
\\	かぎる 漢字 
\\	限る 発音図:	限る かぎる		限る かぎる 限ります かぎります 限って かぎって 限った かぎった 限らない かぎらない 限らなかった かぎらなかった 限れば かぎれば 限らせる かぎらせる 限られる かぎられる 限れ かぎれ 限れる かぎれる 限ろう かぎろう				動詞-一般 
\\	「確認(かくにん)」のアクセントは?393	確認 仮名 
\\	かくにん 漢字 
\\	確認 発音図: 助詞: オ	確認 かくにん		確認 かくにん 確認する かくにんする 確認します かくにんします 確認して かくにんして 確認した かくにんした 確認しない かくにんしない 確認しなかった かくにんしなかった 確認すれば かくにんすれば 確認させる かくにんさせる 確認される かくにんされる 確認しろ かくにんしろ 確認できる かくにんできる 確認しよう かくにんしよう				動詞-一般 
\\	「詳しい(くわしい)」のアクセントは?394	詳しい 仮名 
\\	くわしい 漢字 
\\	詳しい 発音図:	詳しい くわしい		詳しい くわしい 詳しいです くわしいです 詳しくて くわしくて くわしくて 詳しかった くわしかった くわしかった 詳しくない くわしくない くわしくない 詳しくなかった くわしくなかった くわしくなかった 詳しければ くわしければ くわしければ 詳しい くわしい 詳しく くわしく くわしく				形容詞-一般 
\\	「立場(たちば)」のアクセントは?395	立場 仮名 
\\	たちば 漢字 
\\	立場 発音図: 助詞: オ 発音図: 助詞: オ	立場 たちば たちば		立場 たちば たちば				名詞-普通名詞-一般 
\\	「決まる(きまる)」のアクセントは?396	決まる 仮名 
\\	きまる 漢字 
\\	決まる 発音図:	決まる きまる		決まる きまる 決まります きまります 決まって きまって 決まった きまった 決まらない きまらない 決まらなかった きまらなかった 決まれば きまれば 決まらせる きまらせる 決まられる きまられる 決まれ きまれ 決まろう きまろう				動詞-一般 
\\	「大丈夫な(だいじょうぶな)《〜だ》」のアクセントは?397	大丈夫(〜だ) 仮名 
\\	だいじょうぶ、だいじょーぶ 漢字 
\\	大丈夫 発音図: 助詞: ダ	大丈夫な だいじょうぶな		大丈夫な だいじょうぶな 大丈夫です だいじょうぶです 大丈夫で だいじょうぶで 大丈夫だった だいじょうぶだった 大丈夫じゃない だいじょうぶじゃない 大丈夫じゃなかった だいじょうぶじゃなかった 大丈夫ならば だいじょうぶならば 大丈夫だ だいじょうぶだ 大丈夫に だいじょうぶに				形状詞-一般 
\\	「落ちる(おちる)」のアクセントは?398	落ちる 仮名 
\\	おちる 漢字 
\\	落ちる 発音図:	落ちる おちる		落ちる おちる 落ちます おちます 落ちて おちて 落ちた おちた 落ちない おちない 落ちなかった おちなかった 落ちれば おちれば 落ちさせる おちさせる 落ちられる おちられる 落ちろ おちろ 落ちよう おちよう				動詞-一般 
\\	「取れる(とれる)」のアクセントは?399	取れる 仮名 
\\	とれる 漢字 
\\	取れる 発音図:	取れる とれる		取れる とれる 取れます とれます 取れて とれて 取れた とれた 取れない とれない 取れなかった とれなかった 取れれば とれれば 取れさせる とれさせる 取れられる とれられる 取れろ とれろ 取れよう とれよう				動詞-一般 
\\	「困る(こまる)」のアクセントは?400	困る 仮名 
\\	こまる 漢字 
\\	困る 発音図:	困る こまる		困る こまる 困ります こまります 困って こまって 困った こまった 困らない こまらない 困らなかった こまらなかった 困れば こまれば 困らせる こまらせる 困られる こまられる 困れ こまれ 困ろう こまろう				動詞-一般 
\\	「済む(すむ)」のアクセントは?401	済む 仮名 
\\	すむ 漢字 
\\	済む 発音図:	済む すむ		済む すむ 済みます すみます 済んで すんで 済んだ すんだ 済まない すまない 済まなかった すまなかった 済めば すめば 済ませる すませる 済まれる すまれる 済め すめ 済もう すもう				動詞-一般 
\\	「病院(びょういん)」のアクセントは?402	病院 仮名 
\\	びょういん、びょーいん 漢字 
\\	病院 発音図: 助詞: オ	病院 びょういん		病院 びょういん				名詞-普通名詞-一般 
\\	「失う(うしなう)」のアクセントは?403	失う 仮名 
\\	うしなう 漢字 
\\	失う 発音図:	失う うしなう		失う うしなう 失います うしないます 失って うしなって 失った うしなった 失わない うしなわない 失わなかった うしなわなかった 失えば うしなえば 失わせる うしなわせる 失われる うしなわれる 失え うしなえ 失える うしなえる 失おう うしなおう				動詞-一般 
\\	「土地(とち)」のアクセントは?404	土地 仮名 
\\	とち 漢字 
\\	土地 発音図: 助詞: オ	土地 とち		土地 とち				名詞-普通名詞-一般 
\\	「元(もと)《〜が取れる》」のアクセントは?405	元(〜が取れる) 仮名 
\\	もと 漢字 
\\	元 発音図: 例文: これでやっとガ取れる	元 もと		元 もと				名詞-普通名詞-一般 
\\	「元(もと)《〜大臣》」のアクセントは?406	元(〜大臣) 仮名 
\\	もと 漢字 
\\	元 発音図: 例文: 仲人はジョう司にお願いした	元 もと		元 もと				名詞-普通名詞-一般 
\\	「何度(なんど)」のアクセントは?407	何度 仮名 
\\	なんど 漢字 
\\	何度 発音図: 助詞: オ							名詞-数詞 
\\	「考え方(かんがえかた)」のアクセントは?408	考え方 仮名 
\\	かんがえかた 漢字 
\\	考え方 発音図: 助詞: オ							名詞-普通名詞-一般 
\\	「座る(すわる)」のアクセントは?409	座る 仮名 
\\	すわる 漢字 
\\	座る 発音図:	座る すわる		座る すわる 座ります すわります 座って すわって 座った すわった 座らない すわらない 座らなかった すわらなかった 座れば すわれば 座らせる すわらせる 座られる すわられる 座れ すわれ 座れる すわれる 座ろう すわろう				動詞-一般 
\\	「起こす(おこす)」のアクセントは?410	起こす 仮名 
\\	おこす 漢字 
\\	起こす 発音図:	起こす おこす		起こす おこす 起こします おこします 起こして おこして 起こした おこした 起こさない おこさない 起こさなかった おこさなかった 起こせば おこせば 起こさせる おこさせる 起こされる おこされる 起こせ おこせ 起こせる おこせる 起こそう おこそう				動詞-一般 
\\	「聞こえる(きこえる)」のアクセントは?411	聞こえる 仮名 
\\	きこえる 漢字 
\\	聞こえる 発音図:	聞こえる きこえる		聞こえる きこえる 聞こえます きこえます 聞こえて きこえて 聞こえた きこえた 聞こえない きこえない 聞こえなかった きこえなかった 聞こえれば きこえれば 聞こえさせる きこえさせる 聞こえられる きこえられる 聞こえろ きこえろ 聞こえよう きこえよう				動詞-一般 
\\	「含める(ふくめる)」のアクセントは?412	含める 仮名 
\\	ふくめる 漢字 
\\	含める 発音図:	含める ふくめる		含める ふくめる 含めます ふくめます 含めて ふくめて 含めた ふくめた 含めない ふくめない 含めなかった ふくめなかった 含めれば ふくめれば 含めさせる ふくめさせる 含められる ふくめられる 含めろ ふくめろ 含められる ふくめられる 含めよう ふくめよう				動詞-一般 
\\	「現実(げんじつ)」のアクセントは?413	現実 仮名 
\\	げんじつ 漢字 
\\	現実 発音図: 助詞: オ	現実 げんじつ		現実 げんじつ				名詞-普通名詞-一般 
\\	「歴史(れきし)」のアクセントは?414	歴史 仮名 
\\	れきし 漢字 
\\	歴史 発音図: 助詞: オ	歴史 れきし		歴史 れきし				名詞-普通名詞-一般 
\\	「目指す(めざす)」のアクセントは?415	目指す 仮名 
\\	めざす 漢字 
\\	目指す 発音図:	目指す めざす		目指す めざす 目指します めざします 目指して めざして 目指した めざした 目指さない めざさない 目指さなかった めざさなかった 目指せば めざせば 目指させる めざさせる 目指される めざされる 目指せ めざせ 目指せる めざせる 目指そう めざそう				動詞-一般 
\\	「至る(いたる)」のアクセントは?416	至る 仮名 
\\	いたる 漢字 
\\	至る 発音図: 発音図:	至る いたる		至る いたる 至ります いたります 至って いたって 至った いたった 至らない いたらない 至らなかった いたらなかった 至れば いたれば 至らせる いたらせる 至られる いたられる 至れ いたれ 至れる いたれる 至ろう いたろう				動詞-一般 
\\	「再び(ふたたび)」のアクセントは?417	再び 仮名 
\\	ふたたび 漢字 
\\	再び 発音図:							副詞 
\\	「過ぎる(すぎる)」のアクセントは?418	過ぎる 仮名 
\\	すぎる 漢字 
\\	過ぎる 発音図:	過ぎる すぎる		過ぎる すぎる 過ぎます すぎます 過ぎて すぎて 過ぎた すぎた 過ぎない すぎない 過ぎなかった すぎなかった 過ぎれば すぎれば 過ぎさせる すぎさせる 過ぎられる すぎられる 過ぎろ すぎろ 過ぎられる すぎられる 過ぎよう すぎよう				動詞-一般 
\\	「繰り返す(くりかえす)」のアクセントは?419	繰り返す 仮名 
\\	くりかえす 漢字 
\\	繰り返す 発音図: 発音図:	繰り返す くりかえす		繰り返す くりかえす 繰り返します くりかえします 繰り返して くりかえして 繰り返した くりかえした 繰り返さない くりかえさない 繰り返さなかった くりかえさなかった 繰り返せば くりかえせば 繰り返させる くりかえさせる 繰り返される くりかえされる 繰り返せ くりかえせ 繰り返せる くりかえせる 繰り返そう くりかえそう				動詞-一般 
\\	「環境(かんきょう)」のアクセントは?420	環境 仮名 
\\	かんきょう、かんきょー 漢字 
\\	環境 発音図: 助詞: オ	環境 かんきょう		環境 かんきょう				名詞-普通名詞-一般 
\\	「耳(みみ)《お〜》」のアクセントは?421	耳(お〜) 仮名 
\\	みみ 漢字 
\\	耳 発音図: 助詞: オ	耳 みみ		耳 みみ				名詞-普通名詞-一般 
\\	「白い(しろい)」のアクセントは?422	白い 仮名 
\\	しろい 漢字 
\\	白い 発音図:	白い しろい		白い しろい 白いです しろいです 白くて しろくて しろくて 白かった しろかった しろかった 白くない しろくない しろくない 白くなかった しろくなかった しろくなかった 白ければ しろければ しろければ 白い しろい 白く しろく しろく				形容詞-一般 
\\	「母親(ははおや)」のアクセントは?423	母親 仮名 
\\	ははおや 漢字 
\\	母親 発音図: 助詞: オ	母親 ははおや		母親 ははおや				名詞-普通名詞-一般 
\\	「売る(うる)」のアクセントは?424	売る 仮名 
\\	うる 漢字 
\\	売る 発音図:	売る うる		売る うる 売ります うります 売って うって 売った うった 売らない うらない 売らなかった うらなかった 売れば うれば 売らせる うらせる 売られる うられる 売れ うれ 売れる うれる 売ろう うろう				動詞-一般 
\\	「応ずる(おうずる)」のアクセントは?425	応ずる 仮名 
\\	おうずる、おーずる 漢字 
\\	応ずる 発音図: 発音図:							動詞-一般 
\\	「病気(びょうき)」のアクセントは?426	病気 仮名 
\\	びょうき、びょーき 漢字 
\\	病気 発音図: 助詞: オ	病気 びょうき		病気 びょうき				名詞-普通名詞-一般 
\\	「因みに(ちなみに)」のアクセントは?427	ちなみに{因} 仮名 
\\	ちなみに 漢字 
\\	因みに 発音図: 発音図:							接続詞 
\\	「参加(さんか)」のアクセントは?428	参加 仮名 
\\	さんか 漢字 
\\	参加 発音図: 助詞: オ	参加 さんか		参加 さんか 参加する さんかする 参加します さんかします 参加して さんかして 参加した さんかした 参加しない さんかしない 参加しなかった さんかしなかった 参加すれば さんかすれば 参加させる さんかさせる 参加される さんかされる 参加しろ さんかしろ 参加できる さんかできる 参加しよう さんかしよう				動詞-一般 
\\	「厳しい(きびしい)」のアクセントは?429	厳しい 仮名 
\\	きびしい 漢字 
\\	厳しい 発音図:	厳しい きびしい		厳しい きびしい 厳しいです きびしいです 厳しくて きびしくて きびしくて 厳しかった きびしかった きびしかった 厳しくない きびしくない きびしくない 厳しくなかった きびしくなかった きびしくなかった 厳しければ きびしければ きびしければ 厳しい きびしい 厳しく きびしく きびしく				形容詞-一般 
\\	「差す(さす)」のアクセントは?430	差す 仮名 
\\	さす 漢字 
\\	差す 発音図:	差す さす		差す さす 差します さします 差して さして 差した さした 差さない ささない 差さなかった ささなかった 差せば させば 差させる ささせる 差される さされる 差せ させ 差せる させる 差そう さそう				動詞-一般 
\\	「知れる(しれる)」のアクセントは?431	知れる 仮名 
\\	しれる 漢字 
\\	知れる 発音図:							動詞-一般 
\\	「乗せる(のせる)」のアクセントは?432	乗せる 仮名 
\\	のせる 漢字 
\\	乗せる 発音図:	乗せる のせる		乗せる のせる 乗せます のせます 乗せて のせて 乗せた のせた 乗せない のせない 乗せなかった のせなかった 乗せれば のせれば 乗せさせる のせさせる 乗せられる のせられる 乗せろ のせろ 乗せられる のせられる 乗せよう のせよう				動詞-一般 
\\	「信ずる(しんずる)」のアクセントは?433	信ずる 仮名 
\\	しんずる 漢字 
\\	信ずる 発音図:							動詞-一般 
\\	「遊ぶ(あそぶ)」のアクセントは?434	遊ぶ 仮名 
\\	あそぶ 漢字 
\\	遊ぶ 発音図:	遊ぶ あそぶ		遊ぶ あそぶ 遊びます あそびます 遊んで あそんで 遊んだ あそんだ 遊ばない あそばない 遊ばなかった あそばなかった 遊べば あそべば 遊ばせる あそばせる 遊ばれる あそばれる 遊べ あそべ 遊べる あそべる 遊ぼう あそぼう				動詞-一般 
\\	「楽しむ(たのしむ)」のアクセントは?435	楽しむ 仮名 
\\	たのしむ 漢字 
\\	楽しむ 発音図:	楽しむ たのしむ		楽しむ たのしむ 楽しみます たのしみます 楽しんで たのしんで 楽しんだ たのしんだ 楽しまない たのしまない 楽しまなかった たのしまなかった 楽しめば たのしめば 楽しませる たのしませる 楽しまれる たのしまれる 楽しめ たのしめ 楽しめる たのしめる 楽しもう たのしもう				動詞-一般 
\\	「思い出す(おもいだす)」のアクセントは?436	思い出す 仮名 
\\	おもいだす 漢字 
\\	思い出す 発音図: 発音図: 発音図:	思い出す おもいだす		思い出す おもいだす 思い出します おもいだします 思い出して おもいだして 思い出した おもいだした 思い出さない おもいださない 思い出さなかった おもいださなかった 思い出せば おもいだせば 思い出させる おもいださせる 思い出される おもいだされる 思い出せ おもいだせ 思い出せる おもいだせる 思い出そう おもいだそう				動詞-一般 
\\	「美しい(うつくしい)」のアクセントは?437	美しい 仮名 
\\	うつくしい 漢字 
\\	美しい 発音図:	美しい うつくしい		美しい うつくしい 美しいです うつくしいです 美しくて うつくしくて うつくしくて 美しかった うつくしかった うつくしかった 美しくない うつくしくない うつくしくない 美しくなかった うつくしくなかった うつくしくなかった 美しければ うつくしければ うつくしければ 美しい うつくしい 美しく うつくしく うつくしく				形容詞-一般 
\\	「政府(せいふ)」のアクセントは?438	政府 仮名 
\\	せいふ 漢字 
\\	政府 発音図: 助詞: オ	政府 せいふ		政府 せいふ				名詞-普通名詞-一般 
\\	「資料(しりょう)」のアクセントは?439	資料 仮名 
\\	しりょう、しりょー 漢字 
\\	資料 発音図: 助詞: オ	資料 しりょう		資料 しりょう				名詞-普通名詞-一般 
\\	「決して(けっして)」のアクセントは?440	決して 仮名 
\\	けっして 漢字 
\\	決して 発音図:							副詞 
\\	「居る(おる)」のアクセントは?442	おる{居} 仮名 
\\	おる 漢字 
\\	居る 発音図:	居る おる		居る おる 居ります おります 居って おって 居った おった 居らない おらない 居らなかった おらなかった 居れば おれば 居らせる おらせる 居られる おられる 居れ おれ 居れる おれる 居ろう おろう				動詞-一般 
\\	「通す(とおす)」のアクセントは?443	通す 仮名 
\\	とーす、とおす 漢字 
\\	通す 発音図:	通す とおす		通す とおす 通します とおします 通して とおして 通した とおした 通さない とおさない 通さなかった とおさなかった 通せば とおせば 通させる とおさせる 通される とおされる 通せ とおせ 通せる とおせる 通そう とおそう				動詞-一般 
\\	「役割(やくわり)」のアクセントは?444	役割 仮名 
\\	やくわり 漢字 
\\	役割 発音図: 例文: きちんとオ決めてかかる 発音図: 例文: きちんとオ決めてかかる 発音図: 例文: きちんとオ決めてかかる	役割 やくわり やくわり		役割 やくわり やくわり				名詞-普通名詞-一般 
\\	「連れる(つれる)」のアクセントは?445	連れる 仮名 
\\	つれる 漢字 
\\	連れる 発音図:	連れる つれる		連れる つれる 連れます つれます 連れて つれて 連れた つれた 連れない つれない 連れなかった つれなかった 連れれば つれれば 連れさせる つれさせる 連れられる つれられる 連れろ つれろ 連れられる つれられる 連れよう つれよう				動詞-一般 
\\	「常(つね)」のアクセントは?446	常 仮名 
\\	つね 漢字 
\\	常 発音図: 助詞: オ							名詞-普通名詞-一般 
\\	「非道い(ひどい)」のアクセントは?447	ひどい{非道;酷} 仮名 
\\	ひどい 漢字 
\\	酷い、非道い 発音図:	酷い ひどい		酷い ひどい 酷いです ひどいです 酷くて ひどくて ひどくて 酷かった ひどかった ひどかった 酷くない ひどくない ひどくない 酷くなかった ひどくなかった ひどくなかった 酷ければ ひどければ ひどければ 酷い ひどい 酷く ひどく ひどく				形容詞-一般 
\\	「歌(うた)」のアクセントは?448	歌 仮名 
\\	うた 漢字 
\\	歌 発音図: 助詞: オ	歌 うた		歌 うた				名詞-普通名詞-一般 
\\	「生ずる(しょうずる)」のアクセントは?449	生ずる 仮名 
\\	しょうずる、しょーずる 漢字 
\\	生ずる 発音図: 発音図:							動詞-一般 
\\	「上る(のぼる)」のアクセントは?450	上る 仮名 
\\	のぼる 漢字 
\\	上る 発音図:	上る のぼる		上る のぼる 上ります のぼります 上って のぼって 上った のぼった 上らない のぼらない 上らなかった のぼらなかった 上れば のぼれば 上らせる のぼらせる 上られる のぼられる 上れ のぼれ 上れる のぼれる 上ろう のぼろう				動詞-一般 
\\	「責任(せきにん)」のアクセントは?451	責任 仮名 
\\	せきにん 漢字 
\\	責任 発音図: 助詞: オ	責任 せきにん		責任 せきにん				名詞-普通名詞-一般 
\\	「商品(しょうひん)」のアクセントは?452	商品 仮名 
\\	しょうひん、しょーひん 漢字 
\\	商品 発音図: 助詞: オ	商品 しょうひん		商品 しょうひん				名詞-普通名詞-一般 
\\	「風(かぜ)」のアクセントは?453	風 仮名 
\\	かぜ、ふー、ふう 漢字 
\\	風 発音図: 助詞: オ 発音図: 助詞: オ	風 かぜ		風 かぜ				名詞-普通名詞-一般 
\\	「実(じつ)」のアクセントは?454	実 仮名 
\\	じつ、み 漢字 
\\	実 発音図: 助詞: オ 例文: 秋になると柿のガ赤くなる 発音図:	実 じつ		実 じつ				名詞-普通名詞-一般 
\\	「唯(ただ)」のアクセントは?455	ただ{唯;只} 仮名 
\\	ただ 漢字 
\\	唯、只 発音図: 助詞: オ	ただ ただ		ただ ただ				副詞 
\\	「確り(しっかり)」のアクセントは?456	しっかり{確} 仮名 
\\	しっかり 漢字 
\\	確り 発音図:							副詞 
\\	「正しい(ただしい)」のアクセントは?457	正しい 仮名 
\\	ただしい 漢字 
\\	正しい 発音図:	正しい ただしい		正しい ただしい 正しいです ただしいです 正しくて ただしくて ただしくて 正しかった ただしかった ただしかった 正しくない ただしくない ただしくない 正しくなかった ただしくなかった ただしくなかった 正しければ ただしければ ただしければ 正しい ただしい 正しく ただしく ただしく				形容詞-一般 
\\	「発生(はっせいする)」のアクセントは?458	発生 仮名 
\\	はっせい 漢字 
\\	発生 発音図: 助詞: オ	発生する はっせいする		発生する はっせいする 発生します はっせいします 発生して はっせいして 発生した はっせいした 発生しない はっせいしない 発生しなかった はっせいしなかった 発生すれば はっせいすれば 発生させる はっせいさせる 発生される はっせいされる 発生しろ はっせいしろ 発生できる はっせいできる 発生しよう はっせいしよう				動詞-一般 
\\	「触れる(ふれる)」のアクセントは?459	触れる 仮名 
\\	ふれる 漢字 
\\	触れる 発音図:	触れる ふれる		触れる ふれる 触れます ふれます 触れて ふれて 触れた ふれた 触れない ふれない 触れなかった ふれなかった 触れれば ふれれば 触れさせる ふれさせる 触れられる ふれられる 触れろ ふれろ 触れられる ふれられる 触れよう ふれよう				動詞-一般 
\\	「死(し)」のアクセントは?460	死 仮名 
\\	し 漢字 
\\	死 発音図: 例文: だれにでもワ訪れる	死 し		死 し				名詞-普通名詞-一般 
\\	「迎える(むかえる)」のアクセントは?461	迎える 仮名 
\\	むかえる 漢字 
\\	迎える 発音図:	迎える むかえる		迎える むかえる 迎えます むかえます 迎えて むかえて 迎えた むかえた 迎えない むかえない 迎えなかった むかえなかった 迎えれば むかえれば 迎えさせる むかえさせる 迎えられる むかえられる 迎えろ むかえろ 迎えられる むかえられる 迎えよう むかえよう				動詞-一般 
\\	「雨(あめ)」のアクセントは?462	雨 仮名 
\\	あめ、う 漢字 
\\	雨 発音図: 助詞: オ 例文: 雨の音(オン)はダ 発音図:	雨 あめ		雨 あめ				名詞-普通名詞-一般 
\\	「九(きゅう)」のアクセントは?463	
\\	仮名 
\\	きゅー、きゅう、く 漢字 
\\	九、9 発音図: 助詞: オ 発音図: 助詞: オ	九 きゅう		九 きゅう				名詞-数詞 
\\	「申し上げる(もうしあげる)」のアクセントは?464	申し上げる 仮名 
\\	もうしあげる、もーしあげる 漢字 
\\	申し上げる 発音図: 発音図:	申し上げる もうしあげる		申し上げる もうしあげる 申し上げます もうしあげます 申し上げて もうしあげて 申し上げた もうしあげた 申し上げない もうしあげない 申し上げなかった もうしあげなかった 申し上げれば もうしあげれば 申し上げさせる もうしあげさせる 申し上げられる もうしあげられる 申し上げろ もうしあげろ 申し上げられる もうしあげられる 申し上げよう もうしあげよう				動詞-一般 
\\	「軽い(かるい)」のアクセントは?465	軽い 仮名 
\\	かるい 漢字 
\\	軽い 発音図:	軽い かるい		軽い かるい 軽いです かるいです 軽くて かるくて 軽かった かるかった 軽くない かるくない 軽くなかった かるくなかった 軽ければ かるければ 軽い かるい かるい 軽く かるく				形容詞-一般 
\\	「事業(じぎょう)」のアクセントは?466	事業 仮名 
\\	じぎょう、じぎょー 漢字 
\\	事業 発音図: 助詞: オ	事業 じぎょう		事業 じぎょう				名詞-普通名詞-一般 
\\	「着る(きる)」のアクセントは?467	着る 仮名 
\\	きる 漢字 
\\	着る 発音図:	着る きる		着る きる 着ます きます 着て きて 着た きた 着ない きない 着なかった きなかった 着れば きれば 着させる きさせる 着られる きられる 着ろ きろ 着られる きられる 着よう きよう				動詞-一般 
\\	「正に(まさに)」のアクセントは?468	正に 仮名 
\\	まさに 漢字 
\\	正に 発音図:							副詞 
\\	「極めて(きわめて)」のアクセントは?469	きわめて{極} 仮名 
\\	きわめて 漢字 
\\	極めて 発音図:							副詞 
\\	「恐らく(おそらく)」のアクセントは?470	恐らく 仮名 
\\	おそらく 漢字 
\\	恐らく 発音図:							副詞 
\\	「安い(やすい)」のアクセントは?471	安い 仮名 
\\	やすい 漢字 
\\	安い 発音図:	安い やすい		安い やすい 安いです やすいです 安くて やすくて やすくて 安かった やすかった やすかった 安くない やすくない やすくない 安くなかった やすくなかった やすくなかった 安ければ やすければ やすければ 安い やすい 安く やすく やすく				形容詞-一般 
\\	「十(じゅう)」のアクセントは?472		十 じゅう		十 じゅう				名詞-数詞 
\\	「十(じゅー)」のアクセントは?473	
\\	仮名 
\\	じゅー、じゅう 漢字 
\\	十、10 発音図: 助詞: オ							名詞-数詞 
\\	「理解(りかい)」のアクセントは?474	理解 仮名 
\\	りかい 漢字 
\\	理解 発音図: 助詞: オ	理解 りかい		理解 りかい 理解する りかいする 理解します りかいします 理解して りかいして 理解した りかいした 理解しない りかいしない 理解しなかった りかいしなかった 理解すれば りかいすれば 理解させる りかいさせる 理解される りかいされる 理解しろ りかいしろ 理解できる りかいできる 理解しよう りかいしよう				動詞-一般 
\\	「参る(まいる)」のアクセントは?475	参る 仮名 
\\	まいる 漢字 
\\	参る 発音図:	参る まいる		参る まいる 参ります まいります 参って まいって 参った まいった 参らない まいらない 参らなかった まいらなかった 参れば まいれば 参らせる まいらせる 参られる まいられる 参れ まいれ 参れる まいれる 参ろう まいろう				動詞-一般 
\\	「テレビ(てれび)」のアクセントは?476	テレビ 仮名 
\\	テレビ、てれび 漢字 
\\	発音図: 助詞: オ	テレビ てれび		テレビ てれび				名詞-普通名詞-一般 
\\	「表示(ひょうじ)」のアクセントは?477	表示 仮名 
\\	ひょうじ、ひょーじ 漢字 
\\	表示 発音図: 助詞: オ 発音図: 助詞: オ	表示 ひょうじ		表示 ひょうじ 表示する ひょうじする 表示します ひょうじします 表示して ひょうじして 表示した ひょうじした 表示しない ひょうじしない 表示しなかった ひょうじしなかった 表示すれば ひょうじすれば 表示させる ひょうじさせる 表示される ひょうじされる 表示しろ ひょうじしろ 表示できる ひょうじできる 表示しよう ひょうじしよう				動詞-一般 
\\	「兎に角(とにかく)」のアクセントは?478	とにかく{兎角} 仮名 
\\	とにかく 漢字 
\\	兎に角 発音図:							副詞 
\\	「略(ほぼ)《〜等しい》」のアクセントは?479	ほぼ{略}(〜等しい) 仮名 
\\	ほぼ 漢字 
\\	略 発音図: 例文: ヒとしい							副詞 
\\	「消える(きえる)」のアクセントは?480	消える 仮名 
\\	きえる 漢字 
\\	消える 発音図:	消える きえる		消える きえる 消えます きえます 消えて きえて 消えた きえた 消えない きえない 消えなかった きえなかった 消えれば きえれば 消えさせる きえさせる 消えられる きえられる 消えろ きえろ 消えられる きえられる 消えよう きえよう				動詞-一般 
\\	「完全な(かんぜんな)」のアクセントは?481		完全な かんぜんな		完全な かんぜんな 完全です かんぜんです 完全で かんぜんで 完全だった かんぜんだった 完全じゃない かんぜんじゃない 完全じゃなかった かんぜんじゃなかった 完全ならば かんぜんならば 完全だ かんぜんだ 完全に かんぜんに				形状詞-一般 
\\	「完全(かんぜん)」のアクセントは?482	完全 仮名 
\\	かんぜん 漢字 
\\	完全 発音図: 助詞: オ	完全 かんぜん		完全 かんぜん				形状詞-一般 
\\	「考え(かんがえ)」のアクセントは?483	考え 仮名 
\\	かんがえ 漢字 
\\	考え 発音図: 助詞: オ	考え かんがえ		考え かんがえ				名詞-普通名詞-一般 
\\	「大学(だいがく)」のアクセントは?484	大学 仮名 
\\	だいがく 漢字 
\\	大学 発音図: 助詞: オ	大学 だいがく		大学 だいがく				名詞-普通名詞-一般 
\\	「条件(じょうけん)」のアクセントは?485	条件 仮名 
\\	じょうけん、じょーけん 漢字 
\\	条件 発音図: 助詞: オ 発音図: 助詞: オ	条件 じょうけん		条件 じょうけん				名詞-普通名詞-一般 
\\	「振る(ふる)」のアクセントは?486	振る 仮名 
\\	ふる 漢字 
\\	振る 発音図:	振る ふる		振る ふる 振ります ふります 振って ふって 振った ふった 振らない ふらない 振らなかった ふらなかった 振れば ふれば 振らせる ふらせる 振られる ふられる 振れ ふれ 振れる ふれる 振ろう ふろう				動詞-一般 
\\	「押す(おす)」のアクセントは?487	押す 仮名 
\\	おす 漢字 
\\	押す 発音図: 例文: ベルをト彼が出てきた	押す おす		押す おす 押します おします 押して おして 押した おした 押さない おさない 押さなかった おさなかった 押せば おせば 押させる おさせる 押される おされる 押せ おせ 押せる おせる 押そう おそう				動詞-一般 
\\	「有り難う(ありがとう)」のアクセントは?488	ありがとう{有難} 仮名 
\\	ありがとう、ありがとー 漢字 
\\	有り難う 発音図:							感動詞-一般 
\\	「注(ちゅう)」のアクセントは?489	注 仮名 
\\	ちゅー、ちゅう 漢字 
\\	注 発音図: 助詞: オ	注 ちゅう		注 ちゅう				名詞-普通名詞-一般 
\\	「面(めん)《表面;方面》」のアクセントは?490	面(表面;方面) 仮名 
\\	めん 漢字 
\\	面 発音図: 助詞: オ	面 めん		面 めん				名詞-普通名詞-一般 
\\	「面(めん)《仮面;剣道;木工》」のアクセントは?491	面(仮面;剣道;木工) 仮名 
\\	めん 漢字 
\\	面 発音図: 助詞: オ	面 めん		面 めん				名詞-普通名詞-一般 
\\	「流れる(ながれる)」のアクセントは?492	流れる 仮名 
\\	ながれる 漢字 
\\	流れる 発音図:	流れる ながれる		流れる ながれる 流れます ながれます 流れて ながれて 流れた ながれた 流れない ながれない 流れなかった ながれなかった 流れれば ながれれば 流れさせる ながれさせる 流れられる ながれられる 流れろ ながれろ 流れられる ながれられる 流れよう ながれよう				動詞-一般 
\\	「対応(たいおう)」のアクセントは?493	対応 仮名 
\\	たいおう、たいおー 漢字 
\\	対応 発音図: 助詞: オ	対応 たいおう		対応 たいおう 対応する たいおうする 対応します たいおうします 対応して たいおうして 対応した たいおうした 対応しない たいおうしない 対応しなかった たいおうしなかった 対応すれば たいおうすれば 対応させる たいおうさせる 対応される たいおうされる 対応しろ たいおうしろ 対応できる たいおうできる 対応しよう たいおうしよう				動詞-一般 
\\	「特徴(とくちょう)」のアクセントは?494	特徴 仮名 
\\	とくちょう、とくちょー 漢字 
\\	特徴 発音図: 助詞: オ	特徴 とくちょう		特徴 とくちょう				名詞-普通名詞-一般 
\\	「歌う(うたう)」のアクセントは?495	歌う 仮名 
\\	うたう 漢字 
\\	歌う 発音図:	歌う うたう		歌う うたう 歌います うたいます 歌って うたって 歌った うたった 歌わない うたわない 歌わなかった うたわなかった 歌えば うたえば 歌わせる うたわせる 歌われる うたわれる 歌え うたえ 歌える うたえる 歌おう うたおう				動詞-一般 
\\	「可愛い(かわいい)」のアクセントは?496	かわいい{可愛} 仮名 
\\	かわいい 漢字 
\\	可愛い 発音図:	可愛い かわいい		可愛い かわいい 可愛いです かわいいです 可愛くて かわいくて かわいくて 可愛かった かわいかった かわいかった 可愛くない かわいくない かわいくない 可愛くなかった かわいくなかった かわいくなかった 可愛ければ かわいければ かわいければ 可愛い かわいい 可愛く かわいく かわいく				形容詞-一般 
\\	「映画(えいが)」のアクセントは?497	映画 仮名 
\\	えいが 漢字 
\\	映画 発音図: 助詞: オ 発音図: 助詞: オ	映画 えいが えいが		映画 えいが えいが				名詞-普通名詞-一般 
\\	「今日(きょう)」のアクセントは?499	きょう{今日} 仮名 
\\	きょう、きょー 漢字 
\\	今日 発音図: 助詞: オ	今日 きょう		今日 きょう				副詞 
\\	「友人(ゆうじん)」のアクセントは?500	友人 仮名 
\\	ゆーじん、ゆうじん 漢字 
\\	友人 発音図: 助詞: オ	友人 ゆうじん		友人 ゆうじん				名詞-普通名詞-一般 
\\	「7月(しちがつ)」のアクセントは?501	7月 仮名 
\\	しちがつ 漢字 
\\	七月、7月 発音図: 例文: 6月が過ぎてニなった 発音図: 例文: 6月が過ぎてニなった	七月 しちがつ		七月 しちがつ				名詞-数詞 
\\	「評定(ひょうじょう)《小田原〜》」のアクセントは?502	評定(小田原〜) 仮名 
\\	ひょうじょう、ひょーじょー 漢字 
\\	表情、評定 発音図: 助詞: オ	表情 ひょうじょう		表情 ひょうじょう				名詞-普通名詞-一般 
\\	「表情(ひょうじょう)」のアクセントは?503	表情 仮名 
\\	ひょうじょう、ひょーじょー 漢字 
\\	表情 発音図: 助詞: オ	表情 ひょうじょう		表情 ひょうじょう				名詞-普通名詞-一般 
\\	「空(そら)」のアクセントは?504	空 仮名 
\\	から、そら 漢字 
\\	空 発音図: 助詞: オ 例文: ニする 発音図:	空 そら		空 そら				名詞-普通名詞-一般 
\\	「蠅(はい)」のアクセントは?505	はい{蠅} 仮名 
\\	はい 漢字 
\\	蠅 発音図: 例文: 食べ物にたかるオ防ごう							感動詞-一般 
\\	「胚(はい)」のアクセントは?506	はい{胚;牌} 仮名 
\\	はい 漢字 
\\	胚、牌 発音図: 助詞: オ							感動詞-一般 
\\	「並ぶ(ならぶ)」のアクセントは?507	並ぶ 仮名 
\\	ならぶ 漢字 
\\	並ぶ 発音図:	並ぶ ならぶ		並ぶ ならぶ 並びます ならびます 並んで ならんで 並んだ ならんだ 並ばない ならばない 並ばなかった ならばなかった 並べば ならべば 並ばせる ならばせる 並ばれる ならばれる 並べ ならべ 並べる ならべる 並ぼう ならぼう				動詞-一般 
\\	「回る(まわる)」のアクセントは?508	回る 仮名 
\\	まわる 漢字 
\\	回る 発音図:	回る まわる		回る まわる 回ります まわります 回って まわって 回った まわった 回らない まわらない 回らなかった まわらなかった 回れば まわれば 回らせる まわらせる 回られる まわられる 回れ まわれ 回れる まわれる 回ろう まわろう				動詞-一般 
\\	「古い(ふるい)」のアクセントは?509	古い 仮名 
\\	ふるい 漢字 
\\	古い 発音図:	古い ふるい		古い ふるい 古いです ふるいです 古くて ふるくて ふるくて 古かった ふるかった ふるかった 古くない ふるくない ふるくない 古くなかった ふるくなかった ふるくなかった 古ければ ふるければ ふるければ 古い ふるい 古く ふるく ふるく				形容詞-一般 
\\	「横(よこ)」のアクセントは?510	横 仮名 
\\	よこ 漢字 
\\	横 発音図: 助詞: オ	横 よこ		横 よこ				名詞-普通名詞-一般 
\\	「激しい(はげしい)」のアクセントは?511	激しい 仮名 
\\	はげしい 漢字 
\\	激しい 発音図:	激しい はげしい		激しい はげしい 激しいです はげしいです 激しくて はげしくて はげしくて 激しかった はげしかった はげしかった 激しくない はげしくない はげしくない 激しくなかった はげしくなかった はげしくなかった 激しければ はげしければ はげしければ 激しい はげしい 激しく はげしく はげしく				形容詞-一般 
\\	「犬(いぬ)」のアクセントは?512	犬 仮名 
\\	いぬ 漢字 
\\	犬 発音図: 助詞: オ	犬 いぬ		犬 いぬ				名詞-普通名詞-一般 
\\	「繋がる(つながる)」のアクセントは?513	つながる{繋} 仮名 
\\	つながる 漢字 
\\	繋がる 発音図:	繋がる つながる		繋がる つながる 繋がります つながります 繋がって つながって 繋がった つながった 繋がらない つながらない 繋がらなかった つながらなかった 繋がれば つながれば 繋がらせる つながらせる 繋がられる つながられる 繋がれ つながれ 繋がれる つながれる 繋がろう つながろう				動詞-一般 
\\	「許す(ゆるす)」のアクセントは?514	許す 仮名 
\\	ゆるす 漢字 
\\	許す 発音図:	許す ゆるす		許す ゆるす 許します ゆるします 許して ゆるして 許した ゆるした 許さない ゆるさない 許さなかった ゆるさなかった 許せば ゆるせば 許させる ゆるさせる 許される ゆるされる 許せ ゆるせ 許せる ゆるせる 許そう ゆるそう				動詞-一般 
\\	「学ぶ(まなぶ)」のアクセントは?515	学ぶ 仮名 
\\	まなぶ 漢字 
\\	学ぶ 発音図:	学ぶ まなぶ		学ぶ まなぶ 学びます まなびます 学んで まなんで 学んだ まなんだ 学ばない まなばない 学ばなかった まなばなかった 学べば まなべば 学ばせる まなばせる 学ばれる まなばれる 学べ まなべ 学べる まなべる 学ぼう まなぼう				動詞-一般 
\\	「合う(あう)」のアクセントは?516	合う 仮名 
\\	あう 漢字 
\\	合う 発音図:	合う あう		合う あう 合います あいます 合って あって 合った あった 合わない あわない 合わなかった あわなかった 合えば あえば 合わせる あわせる 合われる あわれる 合え あえ 合おう あおう				動詞-一般 
\\	「一般(いっぱん)」のアクセントは?517	一般 仮名 
\\	いっぱん 漢字 
\\	一般 発音図: 助詞: オ	一般 いっぱん		一般 いっぱん				名詞-普通名詞-一般 
\\	「味(あじ)」のアクセントは?518	味 仮名 
\\	あじ 漢字 
\\	味 発音図: 助詞: オ	味 あじ		味 あじ				名詞-普通名詞-一般 
\\	「優しい(やさしい)」のアクセントは?519	優しい 仮名 
\\	やさしい 漢字 
\\	優しい 発音図:	優しい やさしい		優しい やさしい 優しいです やさしいです 優しくて やさしくて 優しかった やさしかった 優しくない やさしくない 優しくなかった やさしくなかった 優しければ やさしければ 優しい やさしい やさしい 優しく やさしく				形容詞-一般 
\\	「データ(でーた)」のアクセントは?520	データ 仮名 
\\	データ、でーた 漢字 
\\	発音図: 助詞: オ 発音図: 助詞: オ	データ でーた でーた		データ でーた でーた				名詞-普通名詞-一般 
\\	「行為(こうい)」のアクセントは?521	行為 仮名 
\\	こうい、こーい 漢字 
\\	行為 発音図: 助詞: オ	行為 こうい		行為 こうい				名詞-普通名詞-一般 
\\	「払う(はらう)」のアクセントは?522	払う 仮名 
\\	はらう 漢字 
\\	払う 発音図:	払う はらう		払う はらう 払います はらいます 払って はらって 払った はらった 払わない はらわない 払わなかった はらわなかった 払えば はらえば 払わせる はらわせる 払われる はらわれる 払え はらえ 払える はらえる 払おう はらおう				動詞-一般 
\\	「積極的な(せっきょくてきな)」のアクセントは?523	積極的 仮名 
\\	せっきょくてき 漢字 
\\	積極的 発音図:	積極的な せっきょくてきな		積極的な せっきょくてきな 積極的です せっきょくてきです 積極的で せっきょくてきで 積極的だった せっきょくてきだった 積極的じゃない せっきょくてきじゃない 積極的じゃなかった せっきょくてきじゃなかった 積極的ならば せっきょくてきならば 積極的だ せっきょくてきだ 積極的に せっきょくてきに				形状詞-一般 
\\	「頼む(たのむ)」のアクセントは?524	頼む 仮名 
\\	たのむ 漢字 
\\	頼む 発音図:	頼む たのむ		頼む たのむ 頼みます たのみます 頼んで たのんで 頼んだ たのんだ 頼まない たのまない 頼まなかった たのまなかった 頼めば たのめば 頼ませる たのませる 頼まれる たのまれる 頼め たのめ 頼める たのめる 頼もう たのもう				動詞-一般 
\\	「文字(もじ)」のアクセントは?525	文字 仮名 
\\	もじ、もんじ 漢字 
\\	文字 発音図: 助詞: オ 発音図: 助詞: オ	文字 もじ		文字 もじ				名詞-普通名詞-一般 
\\	「増加(ぞうか)」のアクセントは?526	増加 仮名 
\\	ぞうか、ぞーか 漢字 
\\	増加 発音図: 助詞: オ	増加 ぞうか		増加 ぞうか 増加する ぞうかする 増加します ぞうかします 増加して ぞうかして 増加した ぞうかした 増加しない ぞうかしない 増加しなかった ぞうかしなかった 増加すれば ぞうかすれば 増加させる ぞうかさせる 増加される ぞうかされる 増加しろ ぞうかしろ 増加できる ぞうかできる 増加しよう ぞうかしよう				動詞-一般 
\\	「全体(ぜんたい)《副》」のアクセントは?527	全体(副) 仮名 
\\	ぜんたい 漢字 
\\	全体 発音図: 助詞: オ 発音図: 助詞: オ	全体 ぜんたい		全体 ぜんたい				名詞-普通名詞-一般 
\\	「全体(ぜんたい)《名》」のアクセントは?528	全体(名) 仮名 
\\	ぜんたい 漢字 
\\	全体 発音図: 助詞: オ	全体 ぜんたい		全体 ぜんたい				名詞-普通名詞-一般 
\\	「息子(むすこ)」のアクセントは?529	息子 仮名 
\\	むすこ 漢字 
\\	息子 発音図: 助詞: オ	息子 むすこ		息子 むすこ				名詞-普通名詞-一般 
\\	「気分(きぶん)」のアクセントは?530	気分 仮名 
\\	きぶん 漢字 
\\	気分 発音図: 助詞: オ	気分 きぶん		気分 きぶん				名詞-普通名詞-一般 
\\	「右(みぎ)《〜のとおり》」のアクセントは?531	右(〜のとおり) 仮名 
\\	みぎ 漢字 
\\	右 発音図: 例文: ノとおり	右 みぎ		右 みぎ				名詞-普通名詞-一般 
\\	「右(みぎ)《〜左》」のアクセントは?532	右(〜左) 仮名 
\\	みぎ 漢字 
\\	右 発音図: 例文: ヒだり	右 みぎ		右 みぎ				名詞-普通名詞-一般 
\\	「届く(とどく)」のアクセントは?533	届く 仮名 
\\	とどく 漢字 
\\	届く 発音図:	届く とどく		届く とどく 届きます とどきます 届いて とどいて 届いた とどいた 届かない とどかない 届かなかった とどかなかった 届けば とどけば 届かせる とどかせる 届かれる とどかれる 届け とどけ 届こう とどこう				動詞-一般 
\\	「腰(こし)」のアクセントは?534	腰 仮名 
\\	こし 漢字 
\\	腰 発音図: 助詞: オ	腰 こし		腰 こし				名詞-普通名詞-一般 
\\	「下りる(おりる)」のアクセントは?535	下りる 仮名 
\\	おりる 漢字 
\\	下りる 発音図:	下りる おりる		下りる おりる 下ります おります 下りて おりて 下りた おりた 下りない おりない 下りなかった おりなかった 下りれば おりれば 下りさせる おりさせる 下りられる おりられる 下りろ おりろ 下りられる おりられる 下りよう おりよう				動詞-一般 
\\	「大人(おとな)」のアクセントは?536	大人 仮名 
\\	おとな、たいじん、だいにん 漢字 
\\	大人 発音図: 助詞: オ 発音図: 助詞: オ 発音図: 助詞: オ	大人 おとな		大人 おとな				名詞-普通名詞-一般 
\\	「肩(かた)」のアクセントは?537	肩 仮名 
\\	かた 漢字 
\\	肩 発音図: 助詞: オ	肩 かた		肩 かた				名詞-普通名詞-一般 
\\	「機会(きかい)」のアクセントは?538	機会 仮名 
\\	きかい 漢字 
\\	機会 発音図: 助詞: オ 発音図: 助詞: オ	機会 きかい		機会 きかい				名詞-普通名詞-一般 
\\	「三つ(みっつ)《〜の年》」のアクセントは?539	三つ(〜の年) 仮名 
\\	みっつ 漢字 
\\	三つ 発音図: 例文: 年はオ数える	三つ みっつ		三つ みっつ				名詞-数詞 
\\	「三つ(みっつ)《〜ある》」のアクセントは?540	三つ(〜ある) 仮名 
\\	みっつ 漢字 
\\	三つ 発音図: 例文: 問題の要点はアる	三つ みっつ		三つ みっつ				名詞-数詞 
\\	「広がる(ひろがる)」のアクセントは?541	広がる 仮名 
\\	ひろがる 漢字 
\\	広がる 発音図:	広がる ひろがる		広がる ひろがる 広がります ひろがります 広がって ひろがって 広がった ひろがった 広がらない ひろがらない 広がらなかった ひろがらなかった 広がれば ひろがれば 広がらせる ひろがらせる 広がられる ひろがられる 広がれ ひろがれ 広がろう ひろがろう				動詞-一般 
\\	「予定(よてい)」のアクセントは?542	予定 仮名 
\\	よてい 漢字 
\\	予定 発音図: 助詞: オ	予定 よてい		予定 よてい				名詞-普通名詞-一般 
\\	「月(つき)」のアクセントは?543	月 仮名 
\\	つき 漢字 
\\	月 発音図: 例文: 今夜のワ白く輝いて見える	月 つき		月 つき				名詞-普通名詞-一般 
\\	「方向(ほうこう)」のアクセントは?544	方向 仮名 
\\	ほうこう、ほーこー 漢字 
\\	方向 発音図: 助詞: オ	方向 ほうこう		方向 ほうこう				名詞-普通名詞-一般 
\\	「絵(え)」のアクセントは?545	絵 仮名 
\\	え 漢字 
\\	絵 発音図: 例文: 富士のオかく	絵 え		絵 え				名詞-普通名詞-一般 
\\	「火(ひ)」のアクセントは?546	火 仮名 
\\	ひ 漢字 
\\	火 発音図: 例文: 炭でオ起こす	火 ひ		火 ひ				名詞-普通名詞-一般 
\\	「パソコン(ぱそこん)」のアクセントは?547	パソコン 仮名 
\\	パソコン、ぱそこん 漢字 
\\	発音図: 助詞: オ	パソコン ぱそこん		パソコン ぱそこん				名詞-普通名詞-一般 
\\	「怖い(こわい)」のアクセントは?548	怖い{恐;強} 仮名 
\\	こわい 漢字 
\\	強い、怖い、恐い 発音図:	怖い こわい		怖い こわい 怖いです こわいです 怖くて こわくて こわくて 怖かった こわかった こわかった 怖くない こわくない こわくない 怖くなかった こわくなかった こわくなかった 怖ければ こわければ こわければ 怖い こわい 怖く こわく こわく				形容詞-一般 
\\	「捕らえる(とらえる)」のアクセントは?549	捕らえる 仮名 
\\	とらえる 漢字 
\\	捕らえる 発音図: 発音図:	捕らえる とらえる		捕らえる とらえる 捕らえます とらえます 捕らえて とらえて 捕らえた とらえた 捕らえない とらえない 捕らえなかった とらえなかった 捕らえれば とらえれば 捕らえさせる とらえさせる 捕らえられる とらえられる 捕らえろ とらえろ 捕らえられる とらえられる 捕らえよう とらえよう				動詞-一般 
\\	「突然(とつぜん)」のアクセントは?550	突然 仮名 
\\	とつぜん 漢字 
\\	突然 発音図:							副詞 
\\	「流れ(ながれ)」のアクセントは?551	流れ 仮名 
\\	ながれ 漢字 
\\	流れ 発音図: 助詞: オ	流れ ながれ		流れ ながれ				名詞-普通名詞-一般 
\\	「建物(たてもの)」のアクセントは?552	建物 仮名 
\\	たてもの 漢字 
\\	建物 発音図: 助詞: オ 発音図: 助詞: オ	建物 たてもの たてもの		建物 たてもの たてもの				名詞-普通名詞-一般 
\\	「有する(ゆうする)」のアクセントは?553	有する 仮名 
\\	ゆーする、ゆうする 漢字 
\\	有する 発音図:	有する ゆうする		有する ゆうする 有します ゆうします 有して ゆうして 有した ゆうした 有さない ゆうさない 有さなかった ゆうさなかった 有せば ゆうせば 有させる ゆうさせる 有される ゆうされる 有せ ゆうせ 有せる ゆうせる 有そう ゆうそう				動詞-一般 
\\	「尋ねる(たずねる)」のアクセントは?554	尋ねる 仮名 
\\	たずねる 漢字 
\\	尋ねる 発音図:	尋ねる たずねる		尋ねる たずねる 尋ねます たずねます 尋ねて たずねて 尋ねた たずねた 尋ねない たずねない 尋ねなかった たずねなかった 尋ねれば たずねれば 尋ねさせる たずねさせる 尋ねられる たずねられる 尋ねろ たずねろ 尋ねられる たずねられる 尋ねよう たずねよう				動詞-一般 
\\	「結婚(けっこん)」のアクセントは?555	結婚 仮名 
\\	けっこん 漢字 
\\	結婚 発音図: 助詞: オ	結婚 けっこん		結婚 けっこん 結婚する けっこんする 結婚します けっこんします 結婚して けっこんして 結婚した けっこんした 結婚しない けっこんしない 結婚しなかった けっこんしなかった 結婚すれば けっこんすれば 結婚させる けっこんさせる 結婚される けっこんされる 結婚しろ けっこんしろ 結婚できる けっこんできる 結婚しよう けっこんしよう				動詞-一般 
\\	「集める(あつめる)」のアクセントは?556	集める 仮名 
\\	あつめる 漢字 
\\	集める 発音図:	集める あつめる		集める あつめる 集めます あつめます 集めて あつめて 集めた あつめた 集めない あつめない 集めなかった あつめなかった 集めれば あつめれば 集めさせる あつめさせる 集められる あつめられる 集めろ あつめろ 集められる あつめられる 集めよう あつめよう				動詞-一般 
\\	「集まる(あつまる)」のアクセントは?557	集まる 仮名 
\\	あつまる 漢字 
\\	集まる 発音図:	集まる あつまる		集まる あつまる 集まります あつまります 集まって あつまって 集まった あつまった 集まらない あつまらない 集まらなかった あつまらなかった 集まれば あつまれば 集まらせる あつまらせる 集まられる あつまられる 集まれ あつまれ 集まれる あつまれる 集まろう あつまろう				動詞-一般 
\\	「では(では)《〜さようなら》」のアクセントは?558	では(〜さようなら) 仮名 
\\	では、でわ 漢字 
\\	発音図: 例文: サようなら							接続詞 
\\	「腕(うで)」のアクセントは?559	腕 仮名 
\\	うで 漢字 
\\	腕 発音図: 助詞: オ	腕 うで		腕 うで				名詞-普通名詞-一般 
\\	「規定(きてい)」のアクセントは?560	規定 仮名 
\\	きてい 漢字 
\\	規定 発音図: 助詞: オ	規定 きてい		規定 きてい 規定する きていする 規定します きていします 規定して きていして 規定した きていした 規定しない きていしない 規定しなかった きていしなかった 規定すれば きていすれば 規定させる きていさせる 規定される きていされる 規定しろ きていしろ 規定できる きていできる 規定しよう きていしよう				動詞-一般 
\\	「近付く(ちかづく)」のアクセントは?561		近付く ちかづく		近付く ちかづく 近付きます ちかづきます 近付いて ちかづいて 近付いた ちかづいた 近付かない ちかづかない 近付かなかった ちかづかなかった 近付けば ちかづけば 近付かせる ちかづかせる 近付かれる ちかづかれる 近付け ちかづけ 近付ける ちかづける 近付こう ちかづこう				動詞-一般 
\\	「1人(ひとり)」のアクセントは?562	1人 仮名 
\\	いちにん、ひとり 漢字 
\\	一人、1人 発音図: 助詞: オ 発音図: 助詞: オ	一人 ひとり		一人 ひとり				副詞 
\\	「壁(かべ)」のアクセントは?563	壁 仮名 
\\	かべ 漢字 
\\	壁 発音図: 助詞: オ	壁 かべ		壁 かべ				名詞-普通名詞-一般 
\\	「止まる(とまる)」のアクセントは?564	止まる 仮名 
\\	とまる 漢字 
\\	止まる 発音図:	止まる とまる		止まる とまる 止まります とまります 止まって とまって 止まった とまった 止まらない とまらない 止まらなかった とまらなかった 止まれば とまれば 止まらせる とまらせる 止まられる とまられる 止まれ とまれ 止まれる とまれる 止まろう とまろう				動詞-一般 
\\	「父親(ちちおや)」のアクセントは?565	父親 仮名 
\\	ちちおや 漢字 
\\	父親 発音図: 助詞: オ	父親 ちちおや		父親 ちちおや				名詞-普通名詞-一般 
\\	「後ろ(うしろ)」のアクセントは?566	後ろ 仮名 
\\	うしろ 漢字 
\\	後ろ 発音図: 助詞: オ	後ろ うしろ		後ろ うしろ				名詞-普通名詞-一般 
\\	「身体(しんたい)」のアクセントは?567	身体 仮名 
\\	しんたい 漢字 
\\	身体 発音図: 助詞: オ	身体 しんたい		身体 しんたい				名詞-普通名詞-一般 
\\	「技術(ぎじゅつ)」のアクセントは?568	技術 仮名 
\\	ぎじゅつ 漢字 
\\	技術 発音図: 助詞: オ	技術 ぎじゅつ		技術 ぎじゅつ				名詞-普通名詞-一般 
\\	「押さえる(おさえる)」のアクセントは?569	押さえる 仮名 
\\	おさえる 漢字 
\\	押さえる 発音図: 発音図:	押さえる おさえる		押さえる おさえる 押さえます おさえます 押さえて おさえて 押さえた おさえた 押さえない おさえない 押さえなかった おさえなかった 押さえれば おさえれば 押さえさせる おさえさせる 押さえられる おさえられる 押さえろ おさえろ 押さえられる おさえられる 押さえよう おさえよう				動詞-一般 
\\	「そんな(そんな)《そのような》」のアクセントは?570	そんな(そのような) 仮名 
\\	そんな 漢字 
\\	発音図:							形状詞-一般 
\\	「手紙(てがみ)《お〜》」のアクセントは?571	手紙(お〜) 仮名 
\\	てがみ 漢字 
\\	手紙 発音図: 助詞: オ	手紙 てがみ		手紙 てがみ				名詞-普通名詞-一般 
\\	「種類(しゅるい)」のアクセントは?572	種類 仮名 
\\	しゅるい 漢字 
\\	種類 発音図: 助詞: オ	種類 しゅるい		種類 しゅるい				名詞-普通名詞-一般 
\\	「丁度(ちょうど)」のアクセントは?573	ちょうど{丁度} 仮名 
\\	ちょうど、ちょーど 漢字 
\\	丁度 発音図:							副詞 
\\	「額(がく)」のアクセントは?574		額 がく		額 がく				名詞-普通名詞-一般 
\\	「窓(まど)」のアクセントは?575	窓 仮名 
\\	まど 漢字 
\\	窓 発音図: 助詞: オ	窓 まど		窓 まど				名詞-普通名詞-一般 
\\	「喜ぶ(よろこぶ)」のアクセントは?576	喜ぶ 仮名 
\\	よろこぶ 漢字 
\\	喜ぶ 発音図:	喜ぶ よろこぶ		喜ぶ よろこぶ 喜びます よろこびます 喜んで よろこんで 喜んだ よろこんだ 喜ばない よろこばない 喜ばなかった よろこばなかった 喜べば よろこべば 喜ばせる よろこばせる 喜ばれる よろこばれる 喜べ よろこべ 喜べる よろこべる 喜ぼう よろこぼう				動詞-一般 
\\	「今後(こんご)」のアクセントは?577	今後 仮名 
\\	こんご 漢字 
\\	今後 発音図: 助詞: オ 発音図: 助詞: オ	今後 こんご		今後 こんご				名詞-普通名詞-一般 
\\	「権利(けんり)」のアクセントは?578	権利 仮名 
\\	けんり 漢字 
\\	権利 発音図: 助詞: オ	権利 けんり		権利 けんり				名詞-普通名詞-一般 
\\	「ポイント(ぽいんと)」のアクセントは?579	ポイント 仮名 
\\	ポイント、ぽいんと 漢字 
\\	発音図: 助詞: オ	ポイント ぽいんと		ポイント ぽいんと				名詞-普通名詞-一般 
\\	「制度(せいど)」のアクセントは?580	制度 仮名 
\\	せいど 漢字 
\\	制度 発音図: 助詞: オ	制度 せいど		制度 せいど				名詞-普通名詞-一般 
\\	「課題(かだい)」のアクセントは?581	課題 仮名 
\\	かだい 漢字 
\\	課題 発音図: 助詞: オ	課題 かだい		課題 かだい				名詞-普通名詞-一般 
\\	「掲げる(かかげる)」のアクセントは?582	掲げる 仮名 
\\	かかげる 漢字 
\\	掲げる 発音図: 発音図:	掲げる かかげる		掲げる かかげる 掲げます かかげます 掲げて かかげて 掲げた かかげた 掲げない かかげない 掲げなかった かかげなかった 掲げれば かかげれば 掲げさせる かかげさせる 掲げられる かかげられる 掲げろ かかげろ 掲げられる かかげられる 掲げよう かかげよう				動詞-一般 
\\	「見詰める(みつめる)」のアクセントは?583		見詰める みつめる		見詰める みつめる 見詰めます みつめます 見詰めて みつめて 見詰めた みつめた 見詰めない みつめない 見詰めなかった みつめなかった 見詰めれば みつめれば 見詰めさせる みつめさせる 見詰められる みつめられる 見詰めろ みつめろ 見詰められる みつめられる 見詰めよう みつめよう				動詞-一般 
\\	「命(いのち)」のアクセントは?584	命 仮名 
\\	いのち、めい 漢字 
\\	命 発音図: 助詞: オ 発音図: 助詞: オ	命 いのち		命 いのち				名詞-普通名詞-一般 
\\	「ドア(どあ)」のアクセントは?585	ドア 仮名 
\\	ドア、どあ 漢字 
\\	発音図: 助詞: オ	ドア どあ		ドア どあ				名詞-普通名詞-一般 
\\	「個人(こじん)」のアクセントは?586	個人 仮名 
\\	こじん 漢字 
\\	個人 発音図: 助詞: オ	個人 こじん		個人 こじん				名詞-普通名詞-一般 
\\	「飛ぶ(とぶ)」のアクセントは?587	飛ぶ 仮名 
\\	とぶ 漢字 
\\	飛ぶ 発音図:	飛ぶ とぶ		飛ぶ とぶ 飛びます とびます 飛んで とんで 飛んだ とんだ 飛ばない とばない 飛ばなかった とばなかった 飛べば とべば 飛ばせる とばせる 飛ばれる とばれる 飛べ とべ 飛べる とべる 飛ぼう とぼう				動詞-一般 
\\	「他人(たにん)」のアクセントは?588	他人 仮名 
\\	たにん 漢字 
\\	他人 発音図: 助詞: オ	他人 たにん		他人 たにん				名詞-普通名詞-一般 
\\	「イメージ(いめーじ)」のアクセントは?589	イメージ 仮名 
\\	イメージ、いめーじ 漢字 
\\	発音図: 助詞: オ 発音図: 助詞: オ	イメージ いめーじ いめーじ		イメージ いめーじ いめーじ				名詞-普通名詞-一般 
\\	「大変な(たいへんな)」のアクセントは?590		大変な たいへんな		大変な たいへんな 大変です たいへんです 大変で たいへんで 大変だった たいへんだった 大変じゃない たいへんじゃない 大変じゃなかった たいへんじゃなかった 大変ならば たいへんならば 大変だ たいへんだ 大変に たいへんに				副詞 
\\	「クリック(くりっく)」のアクセントは?591	クリック 仮名 
\\	クリック、くりっく 漢字 
\\	発音図: 助詞: オ	クリック くりっく		クリック くりっく クリックする くりっくする クリックします くりっくします クリックして くりっくして クリックした くりっくした クリックしない くりっくしない クリックしなかった くりっくしなかった クリックすれば くりっくすれば クリックさせる くりっくさせる クリックされる くりっくされる クリックしろ くりっくしろ クリックできる くりっくできる クリックしよう くりっくしよう				動詞-一般 
\\	「大変(たいへん)」のアクセントは?592	たいへん{大変} 仮名 
\\	たいへん 漢字 
\\	大変 発音図: 助詞: オ	大変 たいへん		大変 たいへん				副詞 
\\	「呉れる(くれる)」のアクセントは?593	くれる{呉} 仮名 
\\	くれる 漢字 
\\	呉れる 発音図:	くれる くれる		くれる くれる くれます くれます くれて くれて くれた くれた くれない くれない くれなかった くれなかった くれれば くれれば くれさせる くれさせる くれ くれ くれよう くれよう				動詞-一般 
\\	「張る(はる)」のアクセントは?594	張る 仮名 
\\	はる 漢字 
\\	張る 発音図:							動詞-一般 
\\	「果たす(はたす)」のアクセントは?595	果たす 仮名 
\\	はたす 漢字 
\\	果たす 発音図:	果たす はたす		果たす はたす 果たします はたします 果たして はたして 果たした はたした 果たさない はたさない 果たさなかった はたさなかった 果たせば はたせば 果たさせる はたさせる 果たされる はたされる 果たせ はたせ 果たせる はたせる 果たそう はたそう				動詞-一般 
\\	「家庭(かてい)」のアクセントは?596	家庭 仮名 
\\	かてい 漢字 
\\	家庭 発音図: 助詞: オ	家庭 かてい		家庭 かてい				名詞-普通名詞-一般 
\\	「運ぶ(はこぶ)」のアクセントは?597	運ぶ 仮名 
\\	はこぶ 漢字 
\\	運ぶ 発音図:	運ぶ はこぶ		運ぶ はこぶ 運びます はこびます 運んで はこんで 運んだ はこんだ 運ばない はこばない 運ばなかった はこばなかった 運べば はこべば 運ばせる はこばせる 運ばれる はこばれる 運べ はこべ 運べる はこべる 運ぼう はこぼう				動詞-一般 
\\	「曲(きょく)《〜がない》」のアクセントは?598	曲(〜がない) 仮名 
\\	きょく 漢字 
\\	曲 発音図: 例文: ガない	曲 きょく		曲 きょく				名詞-普通名詞-一般 
\\	「曲(きょく)」のアクセントは?599	曲 仮名 
\\	きょく 漢字 
\\	曲 発音図: 助詞: オ 発音図: 助詞: オ	曲 きょく		曲 きょく				名詞-普通名詞-一般 
\\	「有名な(ゆうめいな)」のアクセントは?600	有名 仮名 
\\	ゆーめい、ゆうめい 漢字 
\\	有名 発音図: 助詞: オ	有名な ゆうめいな		有名な ゆうめいな 有名です ゆうめいです 有名で ゆうめいで 有名だった ゆうめいだった 有名じゃない ゆうめいじゃない 有名じゃなかった ゆうめいじゃなかった 有名ならば ゆうめいならば 有名だ ゆうめいだ 有名に ゆうめいに				形状詞-一般 
\\	「英語(えいご)」のアクセントは?601	英語 仮名 
\\	えいご 漢字 
\\	英語 発音図: 助詞: オ	英語 えいご		英語 えいご				名詞-普通名詞-一般 
\\	「黒(くろ)」のアクセントは?602	黒 仮名 
\\	くろ 漢字 
\\	黒 発音図: 助詞: オ	黒 くろ		黒 くろ				名詞-普通名詞-一般 
\\	「十時(じゅうじ)」のアクセントは?603		十時 じゅうじ		十時 じゅうじ				名詞-数詞 
\\	「渡る(わたる)」のアクセントは?604	渡る 仮名 
\\	わたる 漢字 
\\	渡る 発音図:	渡る わたる		渡る わたる 渡ります わたります 渡って わたって 渡った わたった 渡らない わたらない 渡らなかった わたらなかった 渡れば わたれば 渡らせる わたらせる 渡られる わたられる 渡れ わたれ 渡れる わたれる 渡ろう わたろう				動詞-一般 
\\	「作成(さくせいする)」のアクセントは?605	作成 仮名 
\\	さくせい 漢字 
\\	作成 発音図: 助詞: オ	作成する さくせいする		作成する さくせいする 作成します さくせいします 作成して さくせいして 作成した さくせいした 作成しない さくせいしない 作成しなかった さくせいしなかった 作成すれば さくせいすれば 作成させる さくせいさせる 作成される さくせいされる 作成しろ さくせいしろ 作成できる さくせいできる 作成しよう さくせいしよう				動詞-一般 
\\	「記事(きじ)」のアクセントは?606	記事 仮名 
\\	きじ 漢字 
\\	記事 発音図: 助詞: オ	記事 きじ		記事 きじ				名詞-普通名詞-一般 
\\	「人物(じんぶつ)」のアクセントは?607	人物 仮名 
\\	じんぶつ 漢字 
\\	人物 発音図: 助詞: オ	人物 じんぶつ		人物 じんぶつ				名詞-普通名詞-一般 
\\	「可笑しい(おかしい)」のアクセントは?608	おかしい{可笑} 仮名 
\\	おかしい 漢字 
\\	可笑しい 発音図:	おかしい おかしい		おかしい おかしい おかしいです おかしいです おかしくて おかしくて おかしくて おかしかった おかしかった おかしかった おかしくない おかしくない おかしくない おかしくなかった おかしくなかった おかしくなかった おかしければ おかしければ おかしければ おかしい おかしい おかしく おかしく おかしく				形容詞-一般 
\\	「感じる(かんじる)」のアクセントは?609	感じる 仮名 
\\	かんじる 漢字 
\\	感じる 発音図:	感じる かんじる		感じる かんじる 感じます かんじます 感じて かんじて 感じた かんじた 感じない かんじない 感じなかった かんじなかった 感じれば かんじれば 感じさせる かんじさせる 感じられる かんじられる 感じろ かんじろ 感じられる かんじられる 感じよう かんじよう				動詞-一般 
\\	「金額(きんがく)」のアクセントは?610	金額 仮名 
\\	きんがく 漢字 
\\	金額 発音図: 助詞: オ	金額 きんがく		金額 きんがく				名詞-普通名詞-一般 
\\	「患者(かんじゃ)」のアクセントは?611	患者 仮名 
\\	かんじゃ 漢字 
\\	患者 発音図: 助詞: オ	患者 かんじゃ		患者 かんじゃ				名詞-普通名詞-一般 
\\	「避ける(さける)」のアクセントは?613	避ける 仮名 
\\	さける 漢字 
\\	避ける 発音図:	避ける さける		避ける さける 避けます さけます 避けて さけて 避けた さけた 避けない さけない 避けなかった さけなかった 避ければ さければ 避けさせる さけさせる 避けられる さけられる 避けろ さけろ 避けられる さけられる 避けよう さけよう				動詞-一般 
\\	「警察(けいさつ)」のアクセントは?614	警察 仮名 
\\	けいさつ 漢字 
\\	警察 発音図: 助詞: オ	警察 けいさつ		警察 けいさつ				名詞-普通名詞-一般 
\\	「ホテル(ほてる)」のアクセントは?615	ホテル 仮名 
\\	ホテル、ほてる 漢字 
\\	発音図: 助詞: オ	ホテル ほてる		ホテル ほてる				名詞-普通名詞-一般 
\\	「抱く(いだく)」のアクセントは?616	抱く{懐} 仮名 
\\	いだく 漢字 
\\	抱く 発音図:	抱く いだく		抱く いだく 抱きます いだきます 抱いて いだいて 抱いた いだいた 抱かない いだかない 抱かなかった いだかなかった 抱けば いだけば 抱かせる いだかせる 抱かれる いだかれる 抱け いだけ 抱ける いだける 抱こう いだこう				動詞-一般 
\\	「後(あと)」のアクセントは?617	あと{後} 仮名 
\\	あと 漢字 
\\	後 発音図: 助詞: オ	後 あと		後 あと				副詞 
\\	「仕方(しかた)」のアクセントは?619	しかた{仕方} 仮名 
\\	しかた 漢字 
\\	仕方 発音図: 助詞: オ	仕方 しかた		仕方 しかた				名詞-普通名詞-一般 
\\	「興味(きょうみ)《〜津々》」のアクセントは?620	興味(〜津々) 仮名 
\\	きょうみ、きょーみ 漢字 
\\	興味 発音図: 例文: ガある 発音図: 例文: ガある	興味 きょうみ		興味 きょうみ				名詞-普通名詞-一般 
\\	「当然な(とうぜんな)」のアクセントは?621		当然な とうぜんな		当然な とうぜんな 当然です とうぜんです 当然で とうぜんで 当然だった とうぜんだった 当然じゃない とうぜんじゃない 当然じゃなかった とうぜんじゃなかった 当然ならば とうぜんならば 当然だ とうぜんだ 当然に とうぜんに				副詞 
\\	「当然(とうぜん)」のアクセントは?622	当然 仮名 
\\	とうぜん、とーぜん 漢字 
\\	当然 発音図:	当然 とうぜん		当然 とうぜん				副詞 
\\	「酒(さけ)」のアクセントは?623	酒 仮名 
\\	さけ 漢字 
\\	酒 発音図: 助詞: オ							名詞-普通名詞-一般 
\\	「纏める(まとめる)」のアクセントは?624	まとめる{纏} 仮名 
\\	まとめる 漢字 
\\	纏める 発音図:	纏める まとめる		纏める まとめる 纏めます まとめます 纏めて まとめて 纏めた まとめた 纏めない まとめない 纏めなかった まとめなかった 纏めれば まとめれば 纏めさせる まとめさせる 纏められる まとめられる 纏めろ まとめろ 纏められる まとめられる 纏めよう まとめよう				動詞-一般 
\\	「血(ち)」のアクセントは?625	血 仮名 
\\	ち 漢字 
\\	血 発音図: 例文: おびただしいガ流された	血 ち		血 ち				名詞-普通名詞-一般 
\\	「捨てる(すてる)」のアクセントは?626	捨てる 仮名 
\\	すてる 漢字 
\\	捨てる 発音図:	捨てる すてる		捨てる すてる 捨てます すてます 捨てて すてて 捨てた すてた 捨てない すてない 捨てなかった すてなかった 捨てれば すてれば 捨てさせる すてさせる 捨てられる すてられる 捨てろ すてろ 捨てられる すてられる 捨てよう すてよう				動詞-一般 
\\	「推進(すいしんする)」のアクセントは?627	推進 仮名 
\\	すいしん 漢字 
\\	推進 発音図: 助詞: オ	推進する すいしんする		推進する すいしんする 推進します すいしんします 推進して すいしんして 推進した すいしんした 推進しない すいしんしない 推進しなかった すいしんしなかった 推進すれば すいしんすれば 推進させる すいしんさせる 推進される すいしんされる 推進しろ すいしんしろ 推進できる すいしんできる 推進しよう すいしんしよう				動詞-一般 
\\	「船(ふね)」のアクセントは?628	船 仮名 
\\	ふね 漢字 
\\	船 発音図: 助詞: オ	船 ふね		船 ふね				名詞-普通名詞-一般 
\\	「性格(せいかく)」のアクセントは?629	性格 仮名 
\\	せいかく 漢字 
\\	性格 発音図: 助詞: オ	性格 せいかく		性格 せいかく				名詞-普通名詞-一般 
\\	「明るい(あかるい)」のアクセントは?630	明るい 仮名 
\\	あかるい 漢字 
\\	明るい 発音図:	明るい あかるい		明るい あかるい 明るいです あかるいです 明るくて あかるくて 明るかった あかるかった 明るくない あかるくない 明るくなかった あかるくなかった 明るければ あかるければ 明るい あかるい あかるい 明るく あかるく				形容詞-一般 
\\	「目標(もくひょう)」のアクセントは?631	目標 仮名 
\\	もくひょう、もくひょー 漢字 
\\	目標 発音図: 助詞: オ	目標 もくひょう		目標 もくひょう				名詞-普通名詞-一般 
\\	「音楽(おんがく)」のアクセントは?632	音楽 仮名 
\\	おんがく 漢字 
\\	音楽 発音図: 助詞: オ 発音図: 助詞: オ	音楽 おんがく		音楽 おんがく				名詞-普通名詞-一般 
\\	「本当な(ほんとな)」のアクセントは?633	ほんと{本当} 仮名 
\\	ほんと 漢字 
\\	本当 発音図: 助詞: オ	ほんとな ほんとな		ほんとな ほんとな ほんとです ほんとです ほんとで ほんとで ほんとだった ほんとだった ほんとじゃない ほんとじゃない ほんとじゃなかった ほんとじゃなかった ほんとならば ほんとならば ほんとだ ほんとだ ほんとに ほんとに				名詞-普通名詞-一般 
\\	「ケース(けーす)」のアクセントは?634	ケース 仮名 
\\	ケース、けーす 漢字 
\\	発音図: 助詞: オ	ケース けーす		ケース けーす				名詞-普通名詞-一般 
\\	「過ごす(すごす)」のアクセントは?635	過ごす 仮名 
\\	すごす 漢字 
\\	過ごす 発音図:	過ごす すごす		過ごす すごす 過ごします すごします 過ごして すごして 過ごした すごした 過ごさない すごさない 過ごさなかった すごさなかった 過ごせば すごせば 過ごさせる すごさせる 過ごされる すごされる 過ごせ すごせ 過ごせる すごせる 過ごそう すごそう				動詞-一般 
\\	「本人(ほんにん)」のアクセントは?636	本人 仮名 
\\	ほんにん 漢字 
\\	本人 発音図: 助詞: オ	本人 ほんにん		本人 ほんにん				名詞-普通名詞-一般 
\\	「当てる(あてる)」のアクセントは?637	当てる 仮名 
\\	あてる 漢字 
\\	当てる 発音図:	当てる あてる		当てる あてる 当てます あてます 当てて あてて 当てた あてた 当てない あてない 当てなかった あてなかった 当てれば あてれば 当てさせる あてさせる 当てられる あてられる 当てろ あてろ 当てられる あてられる 当てよう あてよう				動詞-一般 
\\	「大きさ(おおきさ)」のアクセントは?638	大きさ 仮名 
\\	おーきさ、おおきさ 漢字 
\\	大きさ 発音図: 助詞: オ	大きさ おおきさ		大きさ おおきさ				名詞-普通名詞-一般 
\\	「伸ばす(のばす)」のアクセントは?639	伸ばす 仮名 
\\	のばす 漢字 
\\	伸ばす 発音図:	伸ばす のばす		伸ばす のばす 伸ばします のばします 伸ばして のばして 伸ばした のばした 伸ばさない のばさない 伸ばさなかった のばさなかった 伸ばせば のばせば 伸ばさせる のばさせる 伸ばされる のばされる 伸ばせ のばせ 伸ばせる のばせる 伸ばそう のばそう				動詞-一般 
\\	「期待(きたい)」のアクセントは?640	期待 仮名 
\\	きたい 漢字 
\\	期待 発音図: 助詞: オ	期待 きたい		期待 きたい 期待する きたいする 期待します きたいします 期待して きたいして 期待した きたいした 期待しない きたいしない 期待しなかった きたいしなかった 期待すれば きたいすれば 期待させる きたいさせる 期待される きたいされる 期待しろ きたいしろ 期待できる きたいできる 期待しよう きたいしよう				動詞-一般 
\\	「短い(みじかい)」のアクセントは?641	短い 仮名 
\\	みじかい 漢字 
\\	短い 発音図:	短い みじかい		短い みじかい 短いです みじかいです 短くて みじかくて みじかくて 短かった みじかかった みじかかった 短くない みじかくない みじかくない 短くなかった みじかくなかった みじかくなかった 短ければ みじかければ みじかければ 短い みじかい 短く みじかく みじかく				形容詞-一般 
\\	「白(しろ)」のアクセントは?642	白 仮名 
\\	しろ、はく 漢字 
\\	白 発音図: 例文: 好きな色はダと聞いている 発音図: 例文: 雪が降って辺りはイっ色になった	白 しろ		白 しろ				名詞-普通名詞-一般 
\\	「赤い(あかい)」のアクセントは?643	赤い 仮名 
\\	あかい 漢字 
\\	赤い 発音図:	赤い あかい		赤い あかい 赤いです あかいです 赤くて あかくて 赤かった あかかった 赤くない あかくない 赤くなかった あかくなかった 赤ければ あかければ 赤い あかい あかい 赤く あかく				形容詞-一般 
\\	「検討(けんとう)」のアクセントは?644	検討 仮名 
\\	けんとう、けんとー 漢字 
\\	検討 発音図: 助詞: オ	検討 けんとう		検討 けんとう 検討する けんとうする 検討します けんとうします 検討して けんとうして 検討した けんとうした 検討しない けんとうしない 検討しなかった けんとうしなかった 検討すれば けんとうすれば 検討させる けんとうさせる 検討される けんとうされる 検討しろ けんとうしろ 検討できる けんとうできる 検討しよう けんとうしよう				動詞-一般 
\\	「素晴らしい(すばらしい)」のアクセントは?645		素晴らしい すばらしい		素晴らしい すばらしい 素晴らしいです すばらしいです 素晴らしくて すばらしくて すばらしくて 素晴らしかった すばらしかった すばらしかった 素晴らしくない すばらしくない すばらしくない 素晴らしくなかった すばらしくなかった すばらしくなかった 素晴らしければ すばらしければ すばらしければ 素晴らしい すばらしい 素晴らしく すばらしく すばらしく				形容詞-一般 
\\	「左(ひだり)」のアクセントは?646	左 仮名 
\\	さ、ひだり 漢字 
\\	左 発音図: 助詞: オ 例文: 詳細はオ参照 発音図:	左 ひだり		左 ひだり				名詞-普通名詞-一般 
\\	「開催(かいさい)」のアクセントは?647	開催 仮名 
\\	かいさい 漢字 
\\	開催 発音図: 助詞: オ	開催 かいさい		開催 かいさい 開催する かいさいする 開催します かいさいします 開催して かいさいして 開催した かいさいした 開催しない かいさいしない 開催しなかった かいさいしなかった 開催すれば かいさいすれば 開催させる かいさいさせる 開催される かいさいされる 開催しろ かいさいしろ 開催できる かいさいできる 開催しよう かいさいしよう				動詞-一般 
\\	「痛い(いたい)」のアクセントは?648	痛い 仮名 
\\	いたい 漢字 
\\	痛い 発音図:	痛い いたい		痛い いたい 痛いです いたいです 痛くて いたくて いたくて 痛かった いたかった いたかった 痛くない いたくない いたくない 痛くなかった いたくなかった いたくなかった 痛ければ いたければ いたければ 痛い いたい 痛く いたく いたく				形容詞-一般 
\\	「落とす(おとす)」のアクセントは?649	落とす 仮名 
\\	おとす 漢字 
\\	落とす 発音図:	落とす おとす		落とす おとす 落とします おとします 落として おとして 落とした おとした 落とさない おとさない 落とさなかった おとさなかった 落とせば おとせば 落とさせる おとさせる 落とされる おとされる 落とせ おとせ 落とせる おとせる 落とそう おとそう				動詞-一般 
\\	「指(ゆび)」のアクセントは?650	指 仮名 
\\	ゆび 漢字 
\\	指 発音図: 助詞: オ	指 ゆび		指 ゆび				名詞-普通名詞-一般 
\\	「匂い(におい)」のアクセントは?651	におい{匂;香;臭} 仮名 
\\	におい 漢字 
\\	匂い、臭い、香い 発音図: 助詞: オ	匂い におい		匂い におい				名詞-普通名詞-一般 
\\	「結ぶ(むすぶ)」のアクセントは?652	結ぶ 仮名 
\\	むすぶ 漢字 
\\	結ぶ 発音図:	結ぶ むすぶ		結ぶ むすぶ 結びます むすびます 結んで むすんで 結んだ むすんだ 結ばない むすばない 結ばなかった むすばなかった 結べば むすべば 結ばせる むすばせる 結ばれる むすばれる 結べ むすべ 結べる むすべる 結ぼう むすぼう				動詞-一般 
\\	「すると(すると)《そうすると》」のアクセントは?653	すると(そうすると) 仮名 
\\	すると 漢字 
\\	発音図:							接続詞 
\\	「息(いき)」のアクセントは?654	息 仮名 
\\	いき 漢字 
\\	息 発音図: 助詞: オ	息 いき		息 いき				名詞-普通名詞-一般 
\\	「全然(ぜんぜん)」のアクセントは?655	全然 仮名 
\\	ぜんぜん 漢字 
\\	全然 発音図:							副詞 
\\	「眺める(ながめる)」のアクセントは?656	眺める 仮名 
\\	ながめる 漢字 
\\	眺める 発音図:	眺める ながめる		眺める ながめる 眺めます ながめます 眺めて ながめて 眺めた ながめた 眺めない ながめない 眺めなかった ながめなかった 眺めれば ながめれば 眺めさせる ながめさせる 眺められる ながめられる 眺めろ ながめろ 眺められる ながめられる 眺めよう ながめよう				動詞-一般 
\\	「あらゆる(あらゆる)」のアクセントは?657	あらゆる 仮名 
\\	あらゆる 漢字 
\\	発音図:							連体詞 
\\	「仰る(おっしゃる)」のアクセントは?658	おっしゃる{仰} 仮名 
\\	おっしゃる 漢字 
\\	仰る 発音図:	おっしゃる おっしゃる		おっしゃる おっしゃる おっしゃいます おっしゃいます おっしゃって おっしゃって おっしゃった おっしゃった おっしゃらない おっしゃらない おっしゃらなかった おっしゃらなかった おっしゃれば おっしゃれば おっしゃらせる おっしゃらせる おっしゃられる おっしゃられる おっしゃい おっしゃい おっしゃれる おっしゃれる おっしゃろう おっしゃろう				動詞-一般 
\\	「態度(たいど)」のアクセントは?659	態度 仮名 
\\	たいど 漢字 
\\	態度 発音図: 助詞: オ	態度 たいど		態度 たいど				名詞-普通名詞-一般 
\\	「材料(ざいりょう)」のアクセントは?660	材料 仮名 
\\	ざいりょう、ざいりょー 漢字 
\\	材料 発音図: 助詞: オ	材料 ざいりょう		材料 ざいりょう				名詞-普通名詞-一般 
\\	「漸く(ようやく)」のアクセントは?661	ようやく{漸} 仮名 
\\	ようやく、よーやく 漢字 
\\	漸く 発音図:							副詞 
\\	「頷く(うなずく)」のアクセントは?662	うなずく{頷} 仮名 
\\	うなずく 漢字 
\\	頷く 発音図: 発音図:	うなずく うなずく うなずく		うなずく うなずく うなずく うなずきます うなずきます うなずいて うなずいて うなずいて うなずいた うなずいた うなずいた うなずかない うなずかない うなずかない うなずかなかった うなずかなかった うなずかなかった うなずけば うなずけば うなずけば うなずかせる うなずかせる うなずかせる うなずかれる うなずかれる うなずかれる うなずけ うなずけ うなずけ うなずける うなずける うなずける うなずこう うなずこう				動詞-一般 
\\	「静かな(しずかな)」のアクセントは?663	静か 仮名 
\\	しずか 漢字 
\\	静か 発音図:	静かな しずかな		静かな しずかな 静かです しずかです 静かで しずかで 静かだった しずかだった 静かじゃない しずかじゃない 静かじゃなかった しずかじゃなかった 静かならば しずかならば 静かだ しずかだ 静かに しずかに				形状詞-一般 
\\	「設ける(もうける)」のアクセントは?664	設ける 仮名 
\\	もうける、もーける 漢字 
\\	設ける 発音図:	設ける もうける		設ける もうける 設けます もうけます 設けて もうけて 設けた もうけた 設けない もうけない 設けなかった もうけなかった 設ければ もうければ 設けさせる もうけさせる 設けられる もうけられる 設けろ もうけろ 設けられる もうけられる 設けよう もうけよう				動詞-一般 
\\	「傾向(けいこう)」のアクセントは?665	傾向 仮名 
\\	けいこう、けいこー 漢字 
\\	傾向 発音図: 助詞: オ	傾向 けいこう		傾向 けいこう				名詞-普通名詞-一般 
\\	「女の子(おんなのこ)」のアクセントは?666	女の子 仮名 
\\	おんなのこ 漢字 
\\	女の子 発音図: 助詞: オ	女の子 おんなのこ		女の子 おんなのこ				名詞-普通名詞-一般 
\\	「能力(のうりょく)」のアクセントは?667	能力 仮名 
\\	のうりょく、のーりょく 漢字 
\\	能力 発音図: 助詞: オ	能力 のうりょく		能力 のうりょく				名詞-普通名詞-一般 
\\	「知識(ちしき)」のアクセントは?668	知識 仮名 
\\	ちしき 漢字 
\\	知識 発音図: 助詞: オ	知識 ちしき		知識 ちしき				名詞-普通名詞-一般 
\\	「分ける(わける)」のアクセントは?669	分ける 仮名 
\\	わける 漢字 
\\	分ける 発音図:	分ける わける		分ける わける 分けます わけます 分けて わけて 分けた わけた 分けない わけない 分けなかった わけなかった 分ければ わければ 分けさせる わけさせる 分けられる わけられる 分けろ わけろ 分けられる わけられる 分けよう わけよう				動詞-一般 
\\	「泣く(なく)」のアクセントは?670	泣く 仮名 
\\	なく 漢字 
\\	泣く 発音図:	泣く なく		泣く なく 泣きます なきます 泣いて ないて 泣いた ないた 泣かない なかない 泣かなかった なかなかった 泣けば なけば 泣かせる なかせる 泣かれる なかれる 泣け なけ 泣ける なける 泣こう なこう				動詞-一般 
\\	「奥(おく)」のアクセントは?671	奥 仮名 
\\	おく 漢字 
\\	奥 発音図: 助詞: オ	奥 おく		奥 おく				名詞-普通名詞-一般 
\\	「雰囲気(ふんいき)」のアクセントは?672	雰囲気 仮名 
\\	ふんいき 漢字 
\\	雰囲気 発音図: 助詞: オ	雰囲気 ふんいき		雰囲気 ふんいき				名詞-普通名詞-一般 
\\	「1年(いちねん)」のアクセントは?673	1年 仮名 
\\	いちねん 漢字 
\\	一年、1年 発音図: 助詞: オ	一年 いちねん		一年 いちねん				名詞-数詞 
\\	「一年(いちねん)」のアクセントは?674		一年 いちねん		一年 いちねん				名詞-数詞 
\\	「辛い(つらい)」のアクセントは?675	つらい{辛} 仮名 
\\	つらい 漢字 
\\	辛い 発音図:	辛い つらい		辛い つらい 辛いです つらいです 辛くて つらくて 辛かった つらかった 辛くない つらくない 辛くなかった つらくなかった 辛ければ つらければ 辛い つらい つらい 辛く つらく				形容詞-一般 
\\	「全国(ぜんこく)」のアクセントは?676	全国 仮名 
\\	ぜんこく 漢字 
\\	全国 発音図: 助詞: オ	全国 ぜんこく		全国 ぜんこく				名詞-普通名詞-一般 
\\	「提供(ていきょう)」のアクセントは?677	提供 仮名 
\\	ていきょう、ていきょー 漢字 
\\	提供 発音図: 助詞: オ	提供 ていきょう		提供 ていきょう 提供する ていきょうする 提供します ていきょうします 提供して ていきょうして 提供した ていきょうした 提供しない ていきょうしない 提供しなかった ていきょうしなかった 提供すれば ていきょうすれば 提供させる ていきょうさせる 提供される ていきょうされる 提供しろ ていきょうしろ 提供できる ていきょうできる 提供しよう ていきょうしよう				動詞-一般 
\\	「地(ち)」のアクセントは?678	地 仮名 
\\	じ、ち 漢字 
\\	地 発音図: 例文: うっかりガ出てしまう 発音図: 例文: うっかりガ出てしまう 発音図: 例文: 足をニつけた考え方を望む	地 ち		地 ち				名詞-普通名詞-一般 
\\	「米国(べいこく)」のアクセントは?679	米国 仮名 
\\	べいこく 漢字 
\\	米国 発音図: 助詞: オ	米国 べいこく		米国 べいこく				名詞-固有名詞-地名-国 
\\	「設定(せってい)」のアクセントは?680	設定 仮名 
\\	せってい 漢字 
\\	設定 発音図: 助詞: オ	設定 せってい		設定 せってい 設定する せっていする 設定します せっていします 設定して せっていして 設定した せっていした 設定しない せっていしない 設定しなかった せっていしなかった 設定すれば せっていすれば 設定させる せっていさせる 設定される せっていされる 設定しろ せっていしろ 設定できる せっていできる 設定しよう せっていしよう				動詞-一般 
\\	「設置(せっちする)」のアクセントは?681	設置 仮名 
\\	せっち 漢字 
\\	設置 発音図: 助詞: オ 発音図: 助詞: オ	設置する せっちする		設置する せっちする 設置します せっちします 設置して せっちして 設置した せっちした 設置しない せっちしない 設置しなかった せっちしなかった 設置すれば せっちすれば 設置させる せっちさせる 設置される せっちされる 設置しろ せっちしろ 設置できる せっちできる 設置しよう せっちしよう				動詞-一般 
\\	「馬(うま)」のアクセントは?682	馬 仮名 
\\	うま 漢字 
\\	馬 発音図: 助詞: オ	馬 うま		馬 うま				名詞-普通名詞-一般 
\\	「訪れる(おとずれる)」のアクセントは?683	訪れる 仮名 
\\	おとずれる 漢字 
\\	訪れる 発音図:	訪れる おとずれる		訪れる おとずれる 訪れます おとずれます 訪れて おとずれて 訪れた おとずれた 訪れない おとずれない 訪れなかった おとずれなかった 訪れれば おとずれれば 訪れさせる おとずれさせる 訪れられる おとずれられる 訪れろ おとずれろ 訪れられる おとずれられる 訪れよう おとずれよう				動詞-一般 
\\	「達する(たっする)」のアクセントは?684	達する 仮名 
\\	たっする 漢字 
\\	達する 発音図:	達する たっする		達する たっする 達します たっします 達して たっして 達した たっした 達しない たっしない 達しなかった たっしなかった 達すれば たっすれば 達しさせる たっしさせる 達せられる たっせられる 達しろ たっしろ 達せられる たっせられる 達しよう たっしよう				動詞-一般 
\\	「涙(なみだ)」のアクセントは?685	涙 仮名 
\\	なみだ 漢字 
\\	涙 発音図: 助詞: オ	涙 なみだ		涙 なみだ				名詞-普通名詞-一般 
\\	「返す(かえす)」のアクセントは?686	返す 仮名 
\\	かえす 漢字 
\\	返す 発音図:	返す かえす		返す かえす 返します かえします 返して かえして 返した かえした 返さない かえさない 返さなかった かえさなかった 返せば かえせば 返させる かえさせる 返される かえされる 返せ かえせ 返せる かえせる 返そう かえそう				動詞-一般 
\\	「残念な(ざんねんな)」のアクセントは?687	残念 仮名 
\\	ざんねん 漢字 
\\	残念 発音図: 助詞: オ	残念な ざんねんな		残念な ざんねんな 残念です ざんねんです 残念で ざんねんで 残念だった ざんねんだった 残念じゃない ざんねんじゃない 残念じゃなかった ざんねんじゃなかった 残念ならば ざんねんならば 残念だ ざんねんだ 残念に ざんねんに				形状詞-一般 
\\	「通る(とおる)」のアクセントは?688	通る 仮名 
\\	とーる、とおる 漢字 
\\	通る 発音図:	通る とおる		通る とおる 通ります とおります 通って とおって 通った とおった 通らない とおらない 通らなかった とおらなかった 通れば とおれば 通らせる とおらせる 通られる とおられる 通れ とおれ 通れる とおれる 通ろう とおろう				動詞-一般 
\\	「イギリス(いぎりす)」のアクセントは?689		イギリス いぎりす		イギリス いぎりす				名詞-固有名詞-地名-国 
\\	「敵(てき)」のアクセントは?690	敵 仮名 
\\	かたき、てき 漢字 
\\	敵 発音図: 助詞: オ 発音図: 助詞: オ	敵 てき		敵 てき				名詞-普通名詞-一般 
\\	「周囲(しゅうい)」のアクセントは?691	周囲 仮名 
\\	しゅーい、しゅうい 漢字 
\\	周囲 発音図: 助詞: オ	周囲 しゅうい		周囲 しゅうい				名詞-普通名詞-一般 
\\	「施設(しせつ)」のアクセントは?692	施設 仮名 
\\	しせつ 漢字 
\\	施設 発音図: 助詞: オ 発音図: 助詞: オ	施設 しせつ しせつ		施設 しせつ しせつ				名詞-普通名詞-一般 
\\	「利益(りえき)」のアクセントは?693	利益 仮名 
\\	りえき 漢字 
\\	利益 発音図: 助詞: オ	利益 りえき		利益 りえき				名詞-普通名詞-一般 
\\	「薬(くすり)《お〜》」のアクセントは?694	薬(お〜) 仮名 
\\	くすり 漢字 
\\	薬 発音図: 助詞: オ	薬 くすり		薬 くすり				名詞-普通名詞-一般 
\\	「いらっしゃる(いらっしゃる)」のアクセントは?695	いらっしゃる 仮名 
\\	いらっしゃる 漢字 
\\	発音図:	いらっしゃる いらっしゃる		いらっしゃる いらっしゃる いらっしゃいます いらっしゃいます いらっしゃって いらっしゃって いらっしゃった いらっしゃった いらっしゃらない いらっしゃらない いらっしゃらなかった いらっしゃらなかった いらっしゃれば いらっしゃれば いらっしゃらせる いらっしゃらせる いらっしゃられる いらっしゃられる いらっしゃい いらっしゃい いらっしゃれる いらっしゃれる いらっしゃろう いらっしゃろう				動詞-一般 
\\	「支える(ささえる)」のアクセントは?696	支える 仮名 
\\	ささえる 漢字 
\\	支える 発音図: 発音図:	支える ささえる		支える ささえる 支えます ささえます 支えて ささえて 支えた ささえた 支えない ささえない 支えなかった ささえなかった 支えれば ささえれば 支えさせる ささえさせる 支えられる ささえられる 支えろ ささえろ 支えられる ささえられる 支えよう ささえよう				動詞-一般 
\\	「事情(じじょう)」のアクセントは?697	事情 仮名 
\\	じじょう、じじょー 漢字 
\\	事情 発音図: 助詞: オ	事情 じじょう		事情 じじょう				名詞-普通名詞-一般 
\\	「叫ぶ(さけぶ)」のアクセントは?698	叫ぶ 仮名 
\\	さけぶ 漢字 
\\	叫ぶ 発音図:	叫ぶ さけぶ		叫ぶ さけぶ 叫びます さけびます 叫んで さけんで 叫んだ さけんだ 叫ばない さけばない 叫ばなかった さけばなかった 叫べば さけべば 叫ばせる さけばせる 叫ばれる さけばれる 叫べ さけべ 叫べる さけべる 叫ぼう さけぼう				動詞-一般 
\\	「無料(むりょう)」のアクセントは?699	無料 仮名 
\\	むりょう、むりょー 漢字 
\\	無料 発音図: 助詞: オ 発音図: 助詞: オ	無料 むりょう		無料 むりょう				名詞-普通名詞-一般 
\\	「選択(せんたくする)」のアクセントは?700	選択 仮名 
\\	せんたく 漢字 
\\	選択 発音図: 助詞: オ	選択する せんたくする		選択する せんたくする 選択します せんたくします 選択して せんたくして 選択した せんたくした 選択しない せんたくしない 選択しなかった せんたくしなかった 選択すれば せんたくすれば 選択させる せんたくさせる 選択される せんたくされる 選択しろ せんたくしろ 選択できる せんたくできる 選択しよう せんたくしよう				動詞-一般 
\\	「量(りょう)」のアクセントは?701	量 仮名 
\\	りょう、りょー 漢字 
\\	量 発音図: 助詞: オ	量 りょう		量 りょう				名詞-普通名詞-一般 
\\	「文化(ぶんか)」のアクセントは?702	文化 仮名 
\\	ぶんか 漢字 
\\	文化 発音図: 助詞: オ	文化 ぶんか		文化 ぶんか				名詞-普通名詞-一般 
\\	「柝(き)《さえた〜の音》」のアクセントは?703	き{柝}(さえた〜の音) 仮名 
\\	き 漢字 
\\	柝 発音図: 例文: 芝居小屋に開幕のガ響く							助動詞 
\\	「取り組む(とりくむ)」のアクセントは?704	取り組む 仮名 
\\	とりくむ 漢字 
\\	取り組む 発音図: 発音図:	取り組む とりくむ		取り組む とりくむ 取り組みます とりくみます 取り組んで とりくんで 取り組んだ とりくんだ 取り組まない とりくまない 取り組まなかった とりくまなかった 取り組めば とりくめば 取り組ませる とりくませる 取り組まれる とりくまれる 取り組め とりくめ 取り組める とりくめる 取り組もう とりくもう				動詞-一般 
\\	「猫(ねこ)」のアクセントは?705	猫 仮名 
\\	ねこ 漢字 
\\	猫 発音図: 助詞: オ	猫 ねこ		猫 ねこ				名詞-普通名詞-一般 
\\	「段階(だんかい)」のアクセントは?706	段階 仮名 
\\	だんかい 漢字 
\\	段階 発音図: 助詞: オ	段階 だんかい		段階 だんかい				名詞-普通名詞-一般 
\\	「少なくとも(すくなくとも)」のアクセントは?707	少なくとも 仮名 
\\	すくなくとも 漢字 
\\	少なくとも 発音図: 発音図:							副詞 
\\	「発表(はっぴょう)」のアクセントは?708	発表 仮名 
\\	はっぴょう、はっぴょー 漢字 
\\	発表 発音図: 助詞: オ	発表 はっぴょう		発表 はっぴょう 発表する はっぴょうする 発表します はっぴょうします 発表して はっぴょうして 発表した はっぴょうした 発表しない はっぴょうしない 発表しなかった はっぴょうしなかった 発表すれば はっぴょうすれば 発表させる はっぴょうさせる 発表される はっぴょうされる 発表しろ はっぴょうしろ 発表できる はっぴょうできる 発表しよう はっぴょうしよう				動詞-一般 
\\	「重い(おもい)」のアクセントは?709	重い 仮名 
\\	おもい 漢字 
\\	重い 発音図:	重い おもい		重い おもい 重いです おもいです 重くて おもくて 重かった おもかった 重くない おもくない 重くなかった おもくなかった 重ければ おもければ 重い おもい おもい 重く おもく				形容詞-一般 
\\	「人気(にんき)」のアクセントは?710	人気 仮名 
\\	にんき 漢字 
\\	人気 発音図: 助詞: オ	人気 にんき		人気 にんき				名詞-普通名詞-一般 
\\	「逃げる(にげる)」のアクセントは?712	逃げる 仮名 
\\	にげる 漢字 
\\	逃げる 発音図:	逃げる にげる		逃げる にげる 逃げます にげます 逃げて にげて 逃げた にげた 逃げない にげない 逃げなかった にげなかった 逃げれば にげれば 逃げさせる にげさせる 逃げられる にげられる 逃げろ にげろ 逃げられる にげられる 逃げよう にげよう				動詞-一般 
\\	「及ぶ(およぶ)」のアクセントは?713	及ぶ 仮名 
\\	およぶ 漢字 
\\	及ぶ 発音図:	及ぶ およぶ		及ぶ およぶ 及びます およびます 及んで およんで 及んだ およんだ 及ばない およばない 及ばなかった およばなかった 及べば およべば 及ばせる およばせる 及ばれる およばれる 及べ およべ 及ぼう およぼう				動詞-一般 
\\	「適切な(てきせつな)」のアクセントは?714		適切な てきせつな		適切な てきせつな 適切です てきせつです 適切で てきせつで 適切だった てきせつだった 適切じゃない てきせつじゃない 適切じゃなかった てきせつじゃなかった 適切ならば てきせつならば 適切だ てきせつだ 適切に てきせつに				形状詞-一般 
\\	「適切(てきせつ)」のアクセントは?715	適切 仮名 
\\	てきせつ 漢字 
\\	適切 発音図: 助詞: オ	適切 てきせつ		適切 てきせつ				形状詞-一般 
\\	「発見(はっけん)」のアクセントは?716	発見 仮名 
\\	はっけん 漢字 
\\	発見 発音図: 助詞: オ	発見 はっけん		発見 はっけん 発見する はっけんする 発見します はっけんします 発見して はっけんして 発見した はっけんした 発見しない はっけんしない 発見しなかった はっけんしなかった 発見すれば はっけんすれば 発見させる はっけんさせる 発見される はっけんされる 発見しろ はっけんしろ 発見できる はっけんできる 発見しよう はっけんしよう				動詞-一般 
\\	「追う(おう)」のアクセントは?717	追う 仮名 
\\	おう 漢字 
\\	追う 発音図:	追う おう		追う おう 追います おいます 追って おって 追った おった 追わない おわない 追わなかった おわなかった 追えば おえば 追わせる おわせる 追われる おわれる 追え おえ 追える おえる 追おう おおう				動詞-一般 
\\	「フランス(ふらんす)」のアクセントは?718		フランス ふらんす		フランス ふらんす				名詞-固有名詞-地名-国 
\\	「新聞(しんぶん)」のアクセントは?719	新聞 仮名 
\\	しんぶん 漢字 
\\	新聞 発音図: 助詞: オ	新聞 しんぶん		新聞 しんぶん				名詞-普通名詞-一般 
\\	「遅い(おそい)」のアクセントは?720	遅い 仮名 
\\	おそい 漢字 
\\	遅い 発音図:	遅い おそい		遅い おそい 遅いです おそいです 遅くて おそくて 遅かった おそかった 遅くない おそくない 遅くなかった おそくなかった 遅ければ おそければ 遅い おそい おそい 遅く おそく				形容詞-一般 
\\	「購入(こうにゅう)」のアクセントは?721	購入 仮名 
\\	こうにゅう、こーにゅー 漢字 
\\	購入 発音図: 助詞: オ	購入 こうにゅう		購入 こうにゅう 購入する こうにゅうする 購入します こうにゅうします 購入して こうにゅうして 購入した こうにゅうした 購入しない こうにゅうしない 購入しなかった こうにゅうしなかった 購入すれば こうにゅうすれば 購入させる こうにゅうさせる 購入される こうにゅうされる 購入しろ こうにゅうしろ 購入できる こうにゅうできる 購入しよう こうにゅうしよう				動詞-一般 
\\	「問う(とう)」のアクセントは?722	問う 仮名 
\\	とう 漢字 
\\	問う 発音図: 発音図:	問う とう		問う とう 問います といます 問うて とうて 問うた とうた 問わない とわない 問わなかった とわなかった 問えば とえば 問わせる とわせる 問われる とわれる 問え とえ 問える とえる 問おう とおう				動詞-一般 
\\	「参考(さんこう)」のアクセントは?723	参考 仮名 
\\	さんこう、さんこー 漢字 
\\	参考 発音図: 助詞: オ	参考 さんこう		参考 さんこう				名詞-普通名詞-一般 
\\	「関心(かんしん)」のアクセントは?724	関心 仮名 
\\	かんしん 漢字 
\\	関心 発音図: 助詞: オ	関心 かんしん		関心 かんしん				名詞-普通名詞-一般 
\\	「代わり(かわり)《お〜》」のアクセントは?725	代わり(お〜) 仮名 
\\	かわり 漢字 
\\	代わり 発音図: 助詞: オ	代わり かわり		代わり かわり				名詞-普通名詞-一般 
\\	「遂に(ついに)」のアクセントは?726	ついに{遂} 仮名 
\\	ついに 漢字 
\\	遂に 発音図:							副詞 
\\	「同(どう)」のアクセントは?727	同 仮名 
\\	どう、どー 漢字 
\\	同 発音図:	同 どう		同 どう				名詞-普通名詞-一般 
\\	「差(さ)」のアクセントは?728	差 仮名 
\\	さ 漢字 
\\	差 発音図: 例文: 実力のワない	差 さ		差 さ				名詞-普通名詞-一般 
\\	「抱える(かかえる)」のアクセントは?729	抱える 仮名 
\\	かかえる 漢字 
\\	抱える 発音図:	抱える かかえる		抱える かかえる 抱えます かかえます 抱えて かかえて 抱えた かかえた 抱えない かかえない 抱えなかった かかえなかった 抱えれば かかえれば 抱えさせる かかえさせる 抱えられる かかえられる 抱えろ かかえろ 抱えられる かかえられる 抱えよう かかえよう				動詞-一般 
\\	「指摘(してき)」のアクセントは?730	指摘 仮名 
\\	してき 漢字 
\\	指摘 発音図: 助詞: オ	指摘 してき		指摘 してき 指摘する してきする 指摘します してきします 指摘して してきして 指摘した してきした 指摘しない してきしない 指摘しなかった してきしなかった 指摘すれば してきすれば 指摘させる してきさせる 指摘される してきされる 指摘しろ してきしろ 指摘できる してきできる 指摘しよう してきしよう				動詞-一般 
\\	「嘘(うそ)」のアクセントは?731	うそ{嘘} 仮名 
\\	うそ 漢字 
\\	嘘 発音図: 助詞: オ	嘘 うそ		嘘 うそ				名詞-普通名詞-一般 
\\	「期間(きかん)」のアクセントは?732	期間 仮名 
\\	きかん 漢字 
\\	期間 発音図: 助詞: オ 発音図: 助詞: オ	期間 きかん きかん		期間 きかん きかん				名詞-普通名詞-一般 
\\	「向こう(むこう)《お〜》」のアクセントは?733	向こう(お〜) 仮名 
\\	むこう、むこー 漢字 
\\	向こう 発音図: 助詞: オ 発音図: 助詞: オ	向こう むこう		向こう むこう				名詞-普通名詞-一般 
\\	「基準(きじゅん)」のアクセントは?734	基準 仮名 
\\	きじゅん 漢字 
\\	基準 発音図: 助詞: オ	基準 きじゅん		基準 きじゅん				名詞-普通名詞-一般 
\\	「利く(きく)」のアクセントは?735	利く 仮名 
\\	きく 漢字 
\\	利く 発音図: 例文: 上には見通しがマどがある							動詞-一般 
\\	「判断(はんだん)」のアクセントは?736	判断 仮名 
\\	はんだん 漢字 
\\	判断 発音図: 助詞: オ 発音図: 助詞: オ	判断 はんだん		判断 はんだん 判断する はんだんする 判断します はんだんします 判断して はんだんして 判断した はんだんした 判断しない はんだんしない 判断しなかった はんだんしなかった 判断すれば はんだんすれば 判断させる はんだんさせる 判断される はんだんされる 判断しろ はんだんしろ 判断できる はんだんできる 判断しよう はんだんしよう				動詞-一般 
\\	「年齢(ねんれい)」のアクセントは?737	年齢 仮名 
\\	ねんれい 漢字 
\\	年齢 発音図: 助詞: オ	年齢 ねんれい		年齢 ねんれい				名詞-普通名詞-一般 
\\	「明治(めいじ)」のアクセントは?738	明治 仮名 
\\	めいじ 漢字 
\\	明治 発音図: 助詞: オ	明治 めいじ		明治 めいじ				名詞-固有名詞-一般 
\\	「用意(ようい)」のアクセントは?739	用意 仮名 
\\	ようい、よーい 漢字 
\\	用意 発音図: 助詞: オ	用意 ようい		用意 ようい 用意する よういする 用意します よういします 用意して よういして 用意した よういした 用意しない よういしない 用意しなかった よういしなかった 用意すれば よういすれば 用意させる よういさせる 用意される よういされる 用意しろ よういしろ 用意できる よういできる 用意しよう よういしよう				動詞-一般 
\\	「価格(かかく)」のアクセントは?740	価格 仮名 
\\	かかく 漢字 
\\	価格 発音図: 助詞: オ 発音図: 助詞: オ	価格 かかく		価格 かかく				名詞-普通名詞-一般 
\\	「伸びる(のびる)」のアクセントは?741	伸びる 仮名 
\\	のびる 漢字 
\\	伸びる 発音図:	伸びる のびる		伸びる のびる 伸びます のびます 伸びて のびて 伸びた のびた 伸びない のびない 伸びなかった のびなかった 伸びれば のびれば 伸びさせる のびさせる 伸びられる のびられる 伸びろ のびろ 伸びよう のびよう				動詞-一般 
\\	「腹(はら)」のアクセントは?742	腹 仮名 
\\	はら 漢字 
\\	腹 発音図: 助詞: オ	腹 はら		腹 はら				名詞-普通名詞-一般 
\\	「システム(しすてむ)」のアクセントは?743	システム 仮名 
\\	システム、しすてむ 漢字 
\\	発音図: 助詞: オ	システム しすてむ		システム しすてむ				名詞-普通名詞-一般 
\\	「旅(たび)」のアクセントは?744	旅 仮名 
\\	たび、りょ 漢字 
\\	旅 発音図: 助詞: オ 発音図: 助詞: オ	旅 たび		旅 たび				名詞-普通名詞-一般 
\\	「髪(かみ)」のアクセントは?745	髪 仮名 
\\	かみ 漢字 
\\	髪 発音図: 助詞: オ	髪 かみ		髪 かみ				名詞-普通名詞-一般 
\\	「勤める(つとめる)」のアクセントは?746	勤める 仮名 
\\	つとめる 漢字 
\\	勤める 発音図:	勤める つとめる		勤める つとめる 勤めます つとめます 勤めて つとめて 勤めた つとめた 勤めない つとめない 勤めなかった つとめなかった 勤めれば つとめれば 勤めさせる つとめさせる 勤められる つとめられる 勤めろ つとめろ 勤められる つとめられる 勤めよう つとめよう				動詞-一般 
\\	「費用(ひよう)」のアクセントは?747	費用 仮名 
\\	ひよう、ひよー 漢字 
\\	費用 発音図: 助詞: オ	費用 ひよう		費用 ひよう				名詞-普通名詞-一般 
\\	「やや(やや)」のアクセントは?748	やや 仮名 
\\	やや 漢字 
\\	発音図:							副詞 
\\	「育てる(そだてる)」のアクセントは?749	育てる 仮名 
\\	そだてる 漢字 
\\	育てる 発音図:	育てる そだてる		育てる そだてる 育てます そだてます 育てて そだてて 育てた そだてた 育てない そだてない 育てなかった そだてなかった 育てれば そだてれば 育てさせる そだてさせる 育てられる そだてられる 育てろ そだてろ 育てられる そだてられる 育てよう そだてよう				動詞-一般 
\\	「印象(いんしょう)」のアクセントは?750	印象 仮名 
\\	いんしょう、いんしょー 漢字 
\\	印象 発音図: 助詞: オ	印象 いんしょう		印象 いんしょう				名詞-普通名詞-一般 
\\	「魚(さかな)」のアクセントは?751	魚 仮名 
\\	うお、さかな 漢字 
\\	魚 発音図: 助詞: オ 発音図: 助詞: オ	魚 さかな		魚 さかな				名詞-普通名詞-一般 
\\	「実現(じつげん)」のアクセントは?752	実現 仮名 
\\	じつげん 漢字 
\\	実現 発音図: 助詞: オ	実現 じつげん		実現 じつげん 実現する じつげんする 実現します じつげんします 実現して じつげんして 実現した じつげんした 実現しない じつげんしない 実現しなかった じつげんしなかった 実現すれば じつげんすれば 実現させる じつげんさせる 実現される じつげんされる 実現しろ じつげんしろ 実現できる じつげんできる 実現しよう じつげんしよう				動詞-一般 
\\	「どんどん(どんどん)《〜走る》」のアクセントは?753	どんどん(〜走る) 仮名 
\\	どんどん 漢字 
\\	発音図: 例文: ハしる							副詞 
\\	「ファックス(ふぁっくす)」のアクセントは?754	ファックス 仮名 
\\	ファックス、ふぁっくす 漢字 
\\	発音図: 助詞: オ	ファックス ふぁっくす		ファックス ふぁっくす				名詞-普通名詞-一般 
\\	「怒る(おこる)」のアクセントは?755	怒る 仮名 
\\	いかる、おこる 漢字 
\\	怒る 発音図: 発音図:	怒る おこる		怒る おこる 怒ります おこります 怒って おこって 怒った おこった 怒らない おこらない 怒らなかった おこらなかった 怒れば おこれば 怒らせる おこらせる 怒られる おこられる 怒れ おこれ 怒れる おこれる 怒ろう おころう				動詞-一般 
\\	「背景(はいけい)」のアクセントは?756	背景 仮名 
\\	はいけい 漢字 
\\	背景 発音図: 助詞: オ	背景 はいけい はいけい		背景 はいけい はいけい				名詞-普通名詞-一般 
\\	「如何に(いかに)」のアクセントは?757	いかに{如何} 仮名 
\\	いかに 漢字 
\\	如何に 発音図:							副詞 
\\	「減る(へる)」のアクセントは?758	減る 仮名 
\\	へる 漢字 
\\	減る 発音図:	減る へる		減る へる 減ります へります 減って へって 減った へった 減らない へらない 減らなかった へらなかった 減れば へれば 減らせる へらせる 減られる へられる 減れ へれ 減ろう へろう				動詞-一般 
\\	「韓国(かんこく)」のアクセントは?759		韓国 かんこく		韓国 かんこく				名詞-固有名詞-地名-国 
\\	「注意(ちゅうい)」のアクセントは?760	注意 仮名 
\\	ちゅーい、ちゅうい 漢字 
\\	注意 発音図: 助詞: オ	注意 ちゅうい		注意 ちゅうい 注意する ちゅういする 注意します ちゅういします 注意して ちゅういして 注意した ちゅういした 注意しない ちゅういしない 注意しなかった ちゅういしなかった 注意すれば ちゅういすれば 注意させる ちゅういさせる 注意される ちゅういされる 注意しろ ちゅういしろ 注意できる ちゅういできる 注意しよう ちゅういしよう				動詞-一般 
\\	「隠す(かくす)」のアクセントは?761	隠す 仮名 
\\	かくす 漢字 
\\	隠す 発音図:	隠す かくす		隠す かくす 隠します かくします 隠して かくして 隠した かくした 隠さない かくさない 隠さなかった かくさなかった 隠せば かくせば 隠させる かくさせる 隠される かくされる 隠せ かくせ 隠せる かくせる 隠そう かくそう				動詞-一般 
\\	「黒い(くろい)」のアクセントは?762	黒い 仮名 
\\	くろい 漢字 
\\	黒い 発音図:	黒い くろい		黒い くろい 黒いです くろいです 黒くて くろくて くろくて 黒かった くろかった くろかった 黒くない くろくない くろくない 黒くなかった くろくなかった くろくなかった 黒ければ くろければ くろければ 黒い くろい 黒く くろく くろく				形容詞-一般 
\\	「固い(かたい)」のアクセントは?763	固い 仮名 
\\	かたい 漢字 
\\	固い 発音図:	固い かたい		固い かたい 固いです かたいです 固くて かたくて 固かった かたかった 固くない かたくない 固くなかった かたくなかった 固ければ かたければ 固い かたい かたい 固く かたく				形容詞-一般 
\\	「村(むら)」のアクセントは?764	村 仮名 
\\	そん、むら 漢字 
\\	村 発音図: 助詞: オ 例文: 山奥のニやっと医者が来た 発音図:	村 むら		村 むら				名詞-普通名詞-一般 
\\	「焼く(やく)」のアクセントは?765	焼く 仮名 
\\	やく 漢字 
\\	焼く 発音図: 例文: せんべいをショく人さん	焼く やく		焼く やく 焼きます やきます 焼いて やいて 焼いた やいた 焼かない やかない 焼かなかった やかなかった 焼けば やけば 焼かせる やかせる 焼かれる やかれる 焼け やけ 焼ける やける 焼こう やこう				動詞-一般 
\\	「姿勢(しせい)」のアクセントは?766	姿勢 仮名 
\\	しせい 漢字 
\\	姿勢 発音図: 助詞: オ	姿勢 しせい		姿勢 しせい				名詞-普通名詞-一般 
\\	「弱い(よわい)」のアクセントは?767	弱い 仮名 
\\	よわい 漢字 
\\	弱い 発音図:	弱い よわい		弱い よわい 弱いです よわいです 弱くて よわくて よわくて 弱かった よわかった よわかった 弱くない よわくない よわくない 弱くなかった よわくなかった よわくなかった 弱ければ よわければ よわければ 弱い よわい 弱く よわく よわく				形容詞-一般 
\\	「側(そば)《お〜》」のアクセントは?768	そば{傍;側}(お〜) 仮名 
\\	そば 漢字 
\\	側、傍 発音図: 助詞: オ	側 そば		側 そば				名詞-普通名詞-一般 
\\	「何だか(なんだか)」のアクセントは?769	何だか 仮名 
\\	なんだか 漢字 
\\	何だか 発音図:							副詞 
\\	「客(きゃく)」のアクセントは?770	客 仮名 
\\	きゃく 漢字 
\\	客 発音図: 助詞: オ	客 きゃく		客 きゃく				名詞-普通名詞-一般 
\\	「負ける(まける)」のアクセントは?771	負ける 仮名 
\\	まける 漢字 
\\	負ける 発音図:	負ける まける		負ける まける 負けます まけます 負けて まけて 負けた まけた 負けない まけない 負けなかった まけなかった 負ければ まければ 負けさせる まけさせる 負けられる まけられる 負けろ まけろ 負けられる まけられる 負けよう まけよう				動詞-一般 
\\	「対する(たいする)」のアクセントは?772	対する 仮名 
\\	たいする 漢字 
\\	対する 発音図:	対する たいする		対する たいする 対します たいします 対して たいして 対した たいした 対さない たいさない 対さなかった たいさなかった 対すれば たいすれば 対させる たいさせる 対される たいされる 対せ たいせ 対せる たいせる 対しよう たいしよう				動詞-一般 
\\	「範囲(はんい)」のアクセントは?773	範囲 仮名 
\\	はんい 漢字 
\\	範囲 発音図: 助詞: オ	範囲 はんい		範囲 はんい				名詞-普通名詞-一般 
\\	「薄い(うすい)」のアクセントは?774	薄い 仮名 
\\	うすい 漢字 
\\	薄い 発音図:	薄い うすい		薄い うすい 薄いです うすいです 薄くて うすくて 薄かった うすかった 薄くない うすくない 薄くなかった うすくなかった 薄ければ うすければ 薄い うすい うすい 薄く うすく				形容詞-一般 
\\	「少年(しょうねん)」のアクセントは?775	少年 仮名 
\\	しょうねん、しょーねん 漢字 
\\	少年 発音図: 助詞: オ	少年 しょうねん		少年 しょうねん				名詞-普通名詞-一般 
\\	「もの(もの)」のアクセントは?776		もの もの		もの もの				助詞-終助詞 
\\	「動かす(うごかす)」のアクセントは?777	動かす 仮名 
\\	うごかす 漢字 
\\	動かす 発音図:	動かす うごかす		動かす うごかす 動かします うごかします 動かして うごかして 動かした うごかした 動かさない うごかさない 動かさなかった うごかさなかった 動かせば うごかせば 動かさせる うごかさせる 動かされる うごかされる 動かせ うごかせ 動かせる うごかせる 動かそう うごかそう				動詞-一般 
\\	「優れる(すぐれる)」のアクセントは?778	すぐれる{勝;優} 仮名 
\\	すぐれる 漢字 
\\	優れる、勝れる 発音図:	優れる すぐれる		優れる すぐれる 優れます すぐれます 優れて すぐれて 優れた すぐれた 優れない すぐれない 優れなかった すぐれなかった 優れれば すぐれれば 優れさせる すぐれさせる 優れられる すぐれられる 優れろ すぐれろ 優れよう すぐれよう				動詞-一般 
\\	「登場(とうじょう)」のアクセントは?779	登場 仮名 
\\	とうじょう、とーじょー 漢字 
\\	登場 発音図: 助詞: オ	登場 とうじょう		登場 とうじょう 登場する とうじょうする 登場します とうじょうします 登場して とうじょうして 登場した とうじょうした 登場しない とうじょうしない 登場しなかった とうじょうしなかった 登場すれば とうじょうすれば 登場させる とうじょうさせる 登場される とうじょうされる 登場しろ とうじょうしろ 登場できる とうじょうできる 登場しよう とうじょうしよう				動詞-一般 
\\	「暮らす(くらす)」のアクセントは?780	暮らす 仮名 
\\	くらす 漢字 
\\	暮らす 発音図:	暮らす くらす		暮らす くらす 暮らします くらします 暮らして くらして 暮らした くらした 暮らさない くらさない 暮らさなかった くらさなかった 暮らせば くらせば 暮らさせる くらさせる 暮らされる くらされる 暮らせ くらせ 暮らせる くらせる 暮らそう くらそう				動詞-一般 
\\	「価値(かち)」のアクセントは?781	価値 仮名 
\\	かち 漢字 
\\	価値 発音図: 助詞: オ	価値 かち		価値 かち				名詞-普通名詞-一般 
\\	「変化(へんか)」のアクセントは?782	変化 仮名 
\\	へんか 漢字 
\\	変化 発音図: 助詞: オ	変化 へんか		変化 へんか 変化する へんかする 変化します へんかします 変化して へんかして 変化した へんかした 変化しない へんかしない 変化しなかった へんかしなかった 変化すれば へんかすれば 変化させる へんかさせる 変化される へんかされる 変化しろ へんかしろ 変化できる へんかできる 変化しよう へんかしよう				動詞-一般 
\\	「此方(こっち)」のアクセントは?783	こっち{此方} 仮名 
\\	こっち 漢字 
\\	此方 発音図: 助詞: オ	こっち こっち		こっち こっち				代名詞 
\\	「数字(すうじ)」のアクセントは?784	数字 仮名 
\\	すーじ、すうじ 漢字 
\\	数字 発音図: 助詞: オ	数字 すうじ		数字 すうじ				名詞-普通名詞-一般 
\\	「借りる(かりる)」のアクセントは?785	借りる 仮名 
\\	かりる 漢字 
\\	借りる 発音図:	借りる かりる		借りる かりる 借ります かります 借りて かりて 借りた かりた 借りない かりない 借りなかった かりなかった 借りれば かりれば 借りさせる かりさせる 借りられる かりられる 借りろ かりろ 借りられる かりられる 借りよう かりよう				動詞-一般 
\\	「文章(ぶんしょう)」のアクセントは?786	文章 仮名 
\\	ぶんしょう、ぶんしょー 漢字 
\\	文章 発音図: 助詞: オ	文章 ぶんしょう		文章 ぶんしょう				名詞-普通名詞-一般 
\\	「明確(めいかく)」のアクセントは?787	明確 仮名 
\\	めいかく 漢字 
\\	明確 発音図: 助詞: オ	明確 めいかく		明確 めいかく				形状詞-一般 
\\	「明確な(めいかくな)」のアクセントは?788		明確な めいかくな		明確な めいかくな 明確です めいかくです 明確で めいかくで 明確だった めいかくだった 明確じゃない めいかくじゃない 明確じゃなかった めいかくじゃなかった 明確ならば めいかくならば 明確だ めいかくだ 明確に めいかくに				形状詞-一般 
\\	「愛する(あいする)」のアクセントは?789	愛する 仮名 
\\	あいする 漢字 
\\	愛する 発音図:	愛する あいする		愛する あいする 愛します あいします 愛して あいして 愛した あいした 愛さない あいさない 愛さなかった あいさなかった 愛すれば あいすれば 愛させる あいさせる 愛される あいされる 愛せ あいせ 愛せる あいせる 愛そう あいそう				動詞-一般 
\\	「こんな(こんな)《〜に》」のアクセントは?790	こんな(〜に) 仮名 
\\	こんな 漢字 
\\	発音図: 助詞: ニ							形状詞-一般 
\\	「銀行(ぎんこう)」のアクセントは?791	銀行 仮名 
\\	ぎんこう、ぎんこー 漢字 
\\	銀行 発音図: 助詞: オ	銀行 ぎんこう		銀行 ぎんこう				名詞-普通名詞-一般 
\\	「街(まち)」のアクセントは?792	街 仮名 
\\	まち 漢字 
\\	街 発音図: 助詞: オ	街 まち		街 まち				名詞-普通名詞-一般 
\\	「空気(くうき)」のアクセントは?793	空気 仮名 
\\	くーき、くうき 漢字 
\\	空気 発音図: 助詞: オ	空気 くうき		空気 くうき				名詞-普通名詞-一般 
\\	「仲間(なかま)《お〜》」のアクセントは?794	仲間(お〜) 仮名 
\\	なかま 漢字 
\\	仲間 発音図: 助詞: オ	仲間 なかま		仲間 なかま				名詞-普通名詞-一般 
\\	「通う(かよう)」のアクセントは?795	通う 仮名 
\\	かよう 漢字 
\\	通う 発音図:	通う かよう		通う かよう 通います かよいます 通って かよって 通った かよった 通わない かよわない 通わなかった かよわなかった 通えば かよえば 通わせる かよわせる 通われる かよわれる 通え かよえ 通える かよえる 通おう かよおう				動詞-一般 
\\	「叩く(たたく)」のアクセントは?797	たたく{叩;敲} 仮名 
\\	たたく 漢字 
\\	叩く、敲く 発音図:	叩く たたく		叩く たたく 叩きます たたきます 叩いて たたいて 叩いた たたいた 叩かない たたかない 叩かなかった たたかなかった 叩けば たたけば 叩かせる たたかせる 叩かれる たたかれる 叩け たたけ 叩ける たたける 叩こう たたこう				動詞-一般 
\\	「着く(つく)」のアクセントは?798	着く 仮名 
\\	つく 漢字 
\\	着く 発音図: 例文: 彼は明日成田へハずです 発音図: 例文: 彼は明日成田へハずです	着く つく		着く つく 着きます つきます 着いて ついて 着いた ついた 着かない つかない 着かなかった つかなかった 着けば つけば 着かせる つかせる 着かれる つかれる 着け つけ 着ける つける 着こう つこう				動詞-一般 
\\	「扱う(あつかう)」のアクセントは?799	扱う 仮名 
\\	あつかう 漢字 
\\	扱う 発音図: 発音図:	扱う あつかう		扱う あつかう 扱います あつかいます 扱って あつかって 扱った あつかった 扱わない あつかわない 扱わなかった あつかわなかった 扱えば あつかえば 扱わせる あつかわせる 扱われる あつかわれる 扱え あつかえ 扱える あつかえる 扱おう あつかおう				動詞-一般 
\\	「落ち着く(おちつく)」のアクセントは?800	落ち着く 仮名 
\\	おちつく 漢字 
\\	落ち着く 発音図:	落ち着く おちつく		落ち着く おちつく 落ち着きます おちつきます 落ち着いて おちついて 落ち着いた おちついた 落ち着かない おちつかない 落ち着かなかった おちつかなかった 落ち着けば おちつけば 落ち着かせる おちつかせる 落ち着かれる おちつかれる 落ち着け おちつけ 落ち着ける おちつける 落ち着こう おちつこう				動詞-一般 
\\	「きちんと(きちんと)《〜する》」のアクセントは?801	きちんと(〜する) 仮名 
\\	きちんと 漢字 
\\	発音図: 例文: スる							副詞 
\\	「取り上げる(とりあげる)」のアクセントは?802	取り上げる 仮名 
\\	とりあげる 漢字 
\\	取り上げる 発音図: 発音図:	取り上げる とりあげる とりあげる		取り上げる とりあげる とりあげる 取り上げます とりあげます 取り上げて とりあげて とりあげて 取り上げた とりあげた とりあげた 取り上げない とりあげない とりあげない 取り上げなかった とりあげなかった とりあげなかった 取り上げれば とりあげれば とりあげれば 取り上げさせる とりあげさせる とりあげさせる 取り上げられる とりあげられる とりあげられる 取り上げろ とりあげろ とりあげろ 取り上げられる とりあげられる とりあげられる 取り上げよう とりあげよう				動詞-一般 
\\	「望む(のぞむ)」のアクセントは?803	望む 仮名 
\\	のぞむ 漢字 
\\	望む 発音図:	望む のぞむ		望む のぞむ 望みます のぞみます 望んで のぞんで 望んだ のぞんだ 望まない のぞまない 望まなかった のぞまなかった 望めば のぞめば 望ませる のぞませる 望まれる のぞまれる 望め のぞめ 望める のぞめる 望もう のぞもう				動詞-一般 
\\	「付き合う(つきあう)」のアクセントは?804	つきあう{付合} 仮名 
\\	つきあう 漢字 
\\	付き合う 発音図:	付き合う つきあう		付き合う つきあう 付き合います つきあいます 付き合って つきあって 付き合った つきあった 付き合わない つきあわない 付き合わなかった つきあわなかった 付き合えば つきあえば 付き合わせる つきあわせる 付き合われる つきあわれる 付き合え つきあえ 付き合える つきあえる 付き合おう つきあおう				動詞-一般 
\\	「減少(げんしょうする)」のアクセントは?805	減少 仮名 
\\	げんしょう、げんしょー 漢字 
\\	減少 発音図: 助詞: オ	減少する げんしょうする		減少する げんしょうする 減少します げんしょうします 減少して げんしょうして 減少した げんしょうした 減少しない げんしょうしない 減少しなかった げんしょうしなかった 減少すれば げんしょうすれば 減少させる げんしょうさせる 減少される げんしょうされる 減少しろ げんしょうしろ 減少できる げんしょうできる 減少しよう げんしょうしよう				動詞-一般 
\\	「市(し)」のアクセントは?806		市 し		市 し				名詞-普通名詞-一般 
\\	「感情(かんじょう)」のアクセントは?807	感情 仮名 
\\	かんじょう、かんじょー 漢字 
\\	感情 発音図: 助詞: オ	感情 かんじょう		感情 かんじょう				名詞-普通名詞-一般 
\\	「疑問(ぎもん)」のアクセントは?808	疑問 仮名 
\\	ぎもん 漢字 
\\	疑問 発音図: 助詞: オ	疑問 ぎもん		疑問 ぎもん				名詞-普通名詞-一般 
\\	「戦う(たたかう)」のアクセントは?809	戦う 仮名 
\\	たたかう 漢字 
\\	戦う 発音図:	戦う たたかう		戦う たたかう 戦います たたかいます 戦って たたかって 戦った たたかった 戦わない たたかわない 戦わなかった たたかわなかった 戦えば たたかえば 戦わせる たたかわせる 戦われる たたかわれる 戦え たたかえ 戦える たたかえる 戦おう たたかおう				動詞-一般 
\\	「すっかり(すっかり)《〜なくなる》」のアクセントは?810	すっかり(〜なくなる) 仮名 
\\	すっかり 漢字 
\\	発音図: 例文: ナくなる							副詞 
\\	「努める(つとめる)」のアクセントは?811	努める 仮名 
\\	つとめる 漢字 
\\	努める 発音図:	努める つとめる		努める つとめる 努めます つとめます 努めて つとめて 努めた つとめた 努めない つとめない 努めなかった つとめなかった 努めれば つとめれば 努めさせる つとめさせる 努められる つとめられる 努めろ つとめろ 努められる つとめられる 努めよう つとめよう				動詞-一般 
\\	「下さる(くださる)《褒美を〜》」のアクセントは?812	下さる(褒美を〜) 仮名 
\\	くださる 漢字 
\\	下さる 発音図:							動詞-一般 
\\	「生む(うむ)」のアクセントは?813	生む 仮名 
\\	うむ 漢字 
\\	生む 発音図:							動詞-一般 
\\	「一枚(いちまい)《〜加わる》」のアクセントは?814	一枚(〜加わる) 仮名 
\\	いちまい 漢字 
\\	一枚 発音図: 例文: クわわる							名詞-数詞 
\\	「自信(じしん)」のアクセントは?815	自信 仮名 
\\	じしん 漢字 
\\	自信 発音図: 助詞: オ	自信 じしん		自信 じしん				名詞-普通名詞-一般 
\\	「1枚(いちまい)」のアクセントは?816	1枚 仮名 
\\	いちまい 漢字 
\\	一枚、1枚 発音図: 助詞: オ							名詞-数詞 
\\	「流す(ながす)」のアクセントは?817	流す 仮名 
\\	ながす 漢字 
\\	流す 発音図:	流す ながす		流す ながす 流します ながします 流して ながして 流した ながした 流さない ながさない 流さなかった ながさなかった 流せば ながせば 流させる ながさせる 流される ながされる 流せ ながせ 流せる ながせる 流そう ながそう				動詞-一般 
\\	「生かす(いかす)」のアクセントは?818	生かす 仮名 
\\	いかす 漢字 
\\	生かす 発音図:	生かす いかす		生かす いかす 生かします いかします 生かして いかして 生かした いかした 生かさない いかさない 生かさなかった いかさなかった 生かせば いかせば 生かさせる いかさせる 生かされる いかされる 生かせ いかせ 生かせる いかせる 生かそう いかそう				動詞-一般 
\\	「基本(きほん)」のアクセントは?819	基本 仮名 
\\	きほん 漢字 
\\	基本 発音図: 助詞: オ	基本 きほん		基本 きほん				名詞-普通名詞-一般 
\\	「主人(しゅじん)」のアクセントは?820	主人 仮名 
\\	しゅじん 漢字 
\\	主人 発音図: 助詞: オ	主人 しゅじん		主人 しゅじん				名詞-普通名詞-一般 
\\	「掴む(つかむ)」のアクセントは?821	つかむ{掴} 仮名 
\\	つかむ 漢字 
\\	掴む 発音図:	掴む つかむ		掴む つかむ 掴みます つかみます 掴んで つかんで 掴んだ つかんだ 掴まない つかまない 掴まなかった つかまなかった 掴めば つかめば 掴ませる つかませる 掴まれる つかまれる 掴め つかめ 掴める つかめる 掴もう つかもう				動詞-一般 
\\	「決定(けってい)」のアクセントは?822	決定 仮名 
\\	けってい 漢字 
\\	決定 発音図: 助詞: オ	決定 けってい		決定 けってい 決定する けっていする 決定します けっていします 決定して けっていして 決定した けっていした 決定しない けっていしない 決定しなかった けっていしなかった 決定すれば けっていすれば 決定させる けっていさせる 決定される けっていされる 決定しろ けっていしろ 決定できる けっていできる 決定しよう けっていしよう				動詞-一般 
\\	「記す(しるす)」のアクセントは?823	記す 仮名 
\\	しるす 漢字 
\\	記す 発音図: 発音図:	記す しるす		記す しるす 記します しるします 記して しるして 記した しるした 記さない しるさない 記さなかった しるさなかった 記せば しるせば 記させる しるさせる 記される しるされる 記せ しるせ 記せる しるせる 記そう しるそう				動詞-一般 
\\	「日時(にちじ)」のアクセントは?824	日時 仮名 
\\	にちじ 漢字 
\\	日時 発音図: 助詞: オ 発音図: 助詞: オ	日時 にちじ		日時 にちじ				名詞-普通名詞-一般 
\\	「暗い(くらい)」のアクセントは?825	暗い 仮名 
\\	くらい 漢字 
\\	暗い 発音図:	暗い くらい		暗い くらい 暗いです くらいです 暗くて くらくて 暗かった くらかった 暗くない くらくない 暗くなかった くらくなかった 暗ければ くらければ 暗い くらい くらい 暗く くらく				形容詞-一般 
\\	「床(ゆか)」のアクセントは?826	床 仮名 
\\	とこ、ゆか 漢字 
\\	床 発音図: 助詞: オ 発音図: 助詞: オ	床 ゆか		床 ゆか				名詞-普通名詞-一般 
\\	「了う(しまう)」のアクセントは?827	しまう{仕舞;了} 仮名 
\\	しまう 漢字 
\\	仕舞う、了う 発音図:	しまう しまう		しまう しまう しまいます しまいます しまって しまって しまった しまった しまわない しまわない しまわなかった しまわなかった しまえば しまえば しまわせる しまわせる しまわれる しまわれる しまえ しまえ しまえる しまえる しまおう しまおう				動詞-一般 
\\	「ドイツ(どいつ)」のアクセントは?828		ドイツ どいつ		ドイツ どいつ				名詞-固有名詞-地名-国 
\\	「現状(げんじょう)」のアクセントは?829	現状 仮名 
\\	げんじょう、げんじょー 漢字 
\\	現状 発音図: 助詞: オ	現状 げんじょう		現状 げんじょう				名詞-普通名詞-一般 
\\	「提出(ていしゅつする)」のアクセントは?830	提出 仮名 
\\	ていしゅつ 漢字 
\\	提出 発音図: 助詞: オ	提出する ていしゅつする		提出する ていしゅつする 提出します ていしゅつします 提出して ていしゅつして 提出した ていしゅつした 提出しない ていしゅつしない 提出しなかった ていしゅつしなかった 提出すれば ていしゅつすれば 提出させる ていしゅつさせる 提出される ていしゅつされる 提出しろ ていしゅつしろ 提出できる ていしゅつできる 提出しよう ていしゅつしよう				動詞-一般 
\\	「3日(みっか)」のアクセントは?831	3日 仮名 
\\	みっか 漢字 
\\	三日、3日 発音図: 助詞: オ	三日 みっか		三日 みっか				名詞-数詞 
\\	「話題(わだい)」のアクセントは?832	話題 仮名 
\\	わだい 漢字 
\\	話題 発音図: 助詞: オ	話題 わだい		話題 わだい				名詞-普通名詞-一般 
\\	「最大(さいだい)」のアクセントは?833	最大 仮名 
\\	さいだい 漢字 
\\	最大 発音図: 助詞: オ	最大 さいだい		最大 さいだい				名詞-普通名詞-一般 
\\	「受け取る(うけとる)」のアクセントは?834	受け取る 仮名 
\\	うけとる 漢字 
\\	受け取る 発音図: 発音図:	受け取る うけとる うけとる		受け取る うけとる うけとる 受け取ります うけとります 受け取って うけとって うけとって 受け取った うけとった うけとった 受け取らない うけとらない うけとらない 受け取らなかった うけとらなかった うけとらなかった 受け取れば うけとれば うけとれば 受け取らせる うけとらせる うけとらせる 受け取られる うけとられる うけとられる 受け取れ うけとれ うけとれ 受け取れる うけとれる うけとれる 受け取ろう うけとろう				動詞-一般 
\\	「楽しみ(たのしみ)」のアクセントは?835	楽しみ 仮名 
\\	たのしみ 漢字 
\\	楽しみ 発音図: 例文: 人生には多くのガある 発音図: 例文: 人生には多くのガある 発音図: 例文: 人生には多くのガある	楽しみ たのしみ		楽しみ たのしみ				名詞-普通名詞-一般 
\\	「楽しみな(たのしみな)」のアクセントは?836		楽しみな たのしみな		楽しみな たのしみな 楽しみです たのしみです 楽しみで たのしみで 楽しみだった たのしみだった 楽しみじゃない たのしみじゃない 楽しみじゃなかった たのしみじゃなかった 楽しみならば たのしみならば 楽しみだ たのしみだ 楽しみに たのしみに				名詞-普通名詞-一般 
\\	「受け入れる(うけいれる)」のアクセントは?837	受け入れる 仮名 
\\	うけいれる 漢字 
\\	受け入れる 発音図: 発音図:	受け入れる うけいれる		受け入れる うけいれる 受け入れます うけいれます 受け入れて うけいれて 受け入れた うけいれた 受け入れない うけいれない 受け入れなかった うけいれなかった 受け入れれば うけいれれば 受け入れさせる うけいれさせる 受け入れられる うけいれられる 受け入れろ うけいれろ 受け入れられる うけいれられる 受け入れよう うけいれよう				動詞-一般 
\\	「痛み(いたみ)」のアクセントは?838	痛み 仮名 
\\	いたみ 漢字 
\\	痛み 発音図: 助詞: オ	痛み いたみ		痛み いたみ				名詞-普通名詞-一般 
\\	「悩む(なやむ)」のアクセントは?839	悩む 仮名 
\\	なやむ 漢字 
\\	悩む 発音図:	悩む なやむ		悩む なやむ 悩みます なやみます 悩んで なやんで 悩んだ なやんだ 悩まない なやまない 悩まなかった なやまなかった 悩めば なやめば 悩ませる なやませる 悩まれる なやまれる 悩め なやめ 悩める なやめる 悩もう なやもう				動詞-一般 
\\	「出会う(であう)」のアクセントは?840	出会う 仮名 
\\	であう 漢字 
\\	出会う 発音図: 発音図:	出会う であう		出会う であう 出会います であいます 出会って であって 出会った であった 出会わない であわない 出会わなかった であわなかった 出会えば であえば 出会わせる であわせる 出会われる であわれる 出会え であえ 出会える であえる 出会おう であおう				動詞-一般 
\\	「観点(かんてん)」のアクセントは?841	観点 仮名 
\\	かんてん 漢字 
\\	観点 発音図: 助詞: オ 発音図: 助詞: オ	観点 かんてん		観点 かんてん				名詞-普通名詞-一般 
\\	「問い合わせ(といあわせ)」のアクセントは?842	問い合わせ 仮名 
\\	といあわせ 漢字 
\\	問い合わせ 発音図: 助詞: オ	問い合わせ といあわせ		問い合わせ といあわせ				名詞-普通名詞-一般 
\\	「渡す(わたす)」のアクセントは?843	渡す 仮名 
\\	わたす 漢字 
\\	渡す 発音図:	渡す わたす		渡す わたす 渡します わたします 渡して わたして 渡した わたした 渡さない わたさない 渡さなかった わたさなかった 渡せば わたせば 渡させる わたさせる 渡される わたされる 渡せ わたせ 渡せる わたせる 渡そう わたそう				動詞-一般 
\\	「当時(とうじ)」のアクセントは?844	当時 仮名 
\\	とうじ、とーじ 漢字 
\\	当時 発音図: 助詞: オ	当時 とうじ		当時 とうじ				副詞 
\\	「画像(がぞう)」のアクセントは?845	画像 仮名 
\\	がぞう、がぞー 漢字 
\\	画像 発音図: 助詞: オ							名詞-普通名詞-一般 
\\	「措置(そち)」のアクセントは?846	措置 仮名 
\\	そち 漢字 
\\	措置 発音図: 助詞: オ	措置 そち		措置 そち				名詞-普通名詞-一般 
\\	「下げる(さげる)」のアクセントは?847	下げる 仮名 
\\	さげる 漢字 
\\	下げる 発音図:	下げる さげる		下げる さげる 下げます さげます 下げて さげて 下げた さげた 下げない さげない 下げなかった さげなかった 下げれば さげれば 下げさせる さげさせる 下げられる さげられる 下げろ さげろ 下げられる さげられる 下げよう さげよう				動詞-一般 
\\	「改めて(あらためて)」のアクセントは?848	改めて 仮名 
\\	あらためて 漢字 
\\	改めて 発音図:							副詞 
\\	「食う(くう)」のアクセントは?849	食う 仮名 
\\	くう 漢字 
\\	食う 発音図:	食う くう		食う くう 食います くいます 食って くって 食った くった 食わない くわない 食わなかった くわなかった 食えば くえば 食わせる くわせる 食われる くわれる 食え くえ 食える くえる 食おう くおう				動詞-一般 
\\	「側(がわ)《〜がうるさい》」のアクセントは?850	側(〜がうるさい) 仮名 
\\	がわ 漢字 
\\	側 発音図: 例文: ガうるさい							名詞-普通名詞-一般 
\\	「実際(じっさい)」のアクセントは?851	実際 仮名 
\\	じっさい 漢字 
\\	実際 発音図: 助詞: オ	実際 じっさい		実際 じっさい				副詞 
\\	「穴(あな)」のアクセントは?852	穴{孔} 仮名 
\\	あな 漢字 
\\	穴 発音図: 助詞: オ	穴 あな		穴 あな				名詞-普通名詞-一般 
\\	「医師(いし)」のアクセントは?853	医師 仮名 
\\	いし 漢字 
\\	医師 発音図: 助詞: オ	医師 いし		医師 いし				名詞-普通名詞-一般 
\\	「遠い(とおい)」のアクセントは?854	遠い 仮名 
\\	とーい、とおい 漢字 
\\	遠い 発音図:	遠い とおい		遠い とおい 遠いです とおいです とおいです とおいです 遠くて とおくて とおくて 遠かった とおかった とおかった 遠くない とおくない 遠くなかった とおくなかった 遠ければ とおければ とおければ 遠い とおい とおい とおい 遠く とおく				形容詞-一般 
\\	「種(しゅ)」のアクセントは?855		種 しゅ		種 しゅ				名詞-普通名詞-一般 
\\	「高さ(たかさ)」のアクセントは?856	高さ 仮名 
\\	たかさ 漢字 
\\	高さ 発音図: 助詞: オ	高さ たかさ		高さ たかさ				名詞-普通名詞-一般 
\\	「テーマ(てーま)」のアクセントは?857	テーマ 仮名 
\\	テーマ、てーま 漢字 
\\	発音図: 助詞: オ	テーマ てーま		テーマ てーま				名詞-普通名詞-一般 
\\	「会場(かいじょう)」のアクセントは?858	会場 仮名 
\\	かいじょう、かいじょー 漢字 
\\	会場 発音図: 助詞: オ	会場 かいじょう		会場 かいじょう				名詞-普通名詞-一般 
\\	「珍しい(めずらしい)」のアクセントは?859	珍しい 仮名 
\\	めずらしい 漢字 
\\	珍しい 発音図:	珍しい めずらしい		珍しい めずらしい 珍しいです めずらしいです 珍しくて めずらしくて めずらしくて 珍しかった めずらしかった めずらしかった 珍しくない めずらしくない めずらしくない 珍しくなかった めずらしくなかった めずらしくなかった 珍しければ めずらしければ めずらしければ 珍しい めずらしい 珍しく めずらしく めずらしく				形容詞-一般 
\\	「意味(いみ)」のアクセントは?860	意味 仮名 
\\	いみ 漢字 
\\	意味 発音図: 助詞: オ	意味 いみ		意味 いみ 意味する いみする 意味します いみします 意味して いみして 意味した いみした 意味しない いみしない 意味しなかった いみしなかった 意味すれば いみすれば 意味させる いみさせる 意味される いみされる 意味しろ いみしろ 意味しよう いみしよう				動詞-一般 
\\	「黙る(だまる)」のアクセントは?861	黙る 仮名 
\\	だまる 漢字 
\\	黙る 発音図:	黙る だまる		黙る だまる 黙ります だまります 黙って だまって 黙った だまった 黙らない だまらない 黙らなかった だまらなかった 黙れば だまれば 黙らせる だまらせる 黙られる だまられる 黙れ だまれ 黙れる だまれる 黙ろう だまろう				動詞-一般 
\\	「鬆(す)」のアクセントは?862	す{簀;鬆} 仮名 
\\	す 漢字 
\\	鬆、簀 発音図: 例文: 割り竹で作ったワあまり見ない 発音図: 例文: 割り竹で作ったワあまり見ない							動詞-一般 
\\	「業務(ぎょうむ)」のアクセントは?863	業務 仮名 
\\	ぎょうむ、ぎょーむ 漢字 
\\	業務 発音図: 助詞: オ							名詞-普通名詞-一般 
\\	「汗(あせ)」のアクセントは?864	汗 仮名 
\\	あせ 漢字 
\\	汗 発音図: 助詞: オ	汗 あせ		汗 あせ				名詞-普通名詞-一般 
\\	「外す(はずす)」のアクセントは?865	外す 仮名 
\\	はずす 漢字 
\\	外す 発音図:	外す はずす		外す はずす 外します はずします 外して はずして 外した はずした 外さない はずさない 外さなかった はずさなかった 外せば はずせば 外させる はずさせる 外される はずされる 外せ はずせ 外せる はずせる 外そう はずそう				動詞-一般 
\\	「タバコ(たばこ)」のアクセントは?866		タバコ たばこ		タバコ たばこ				名詞-普通名詞-一般 
\\	「動物(どうぶつ)」のアクセントは?867	動物 仮名 
\\	どうぶつ、どーぶつ 漢字 
\\	動物 発音図: 助詞: オ	動物 どうぶつ		動物 どうぶつ				名詞-普通名詞-一般 
\\	「仮令(たとえ)」のアクセントは?868	たとえ{仮令} 仮名 
\\	たとえ 漢字 
\\	仮令 発音図: 例文: 君がシん友でも金は貸せない 発音図: 例文: 君がシん友でも金は貸せない							副詞 
\\	「エネルギー(えねるぎー)」のアクセントは?869	エネルギー 仮名 
\\	エネルギー、えねるぎー 漢字 
\\	発音図: 助詞: オ 発音図: 助詞: オ	エネルギー えねるぎー えねるぎー		エネルギー えねるぎー えねるぎー				名詞-普通名詞-一般 
\\	「分野(ぶんや)」のアクセントは?870	分野 仮名 
\\	ぶんや 漢字 
\\	分野 発音図: 助詞: オ	分野 ぶんや		分野 ぶんや				名詞-普通名詞-一般 
\\	「収める(おさめる)」のアクセントは?871	収める 仮名 
\\	おさめる 漢字 
\\	収める 発音図:	収める おさめる		収める おさめる 収めます おさめます 収めて おさめて 収めた おさめた 収めない おさめない 収めなかった おさめなかった 収めれば おさめれば 収めさせる おさめさせる 収められる おさめられる 収めろ おさめろ 収められる おさめられる 収めよう おさめよう				動詞-一般 
\\	「関する(かんする)」のアクセントは?872	関する 仮名 
\\	かんする 漢字 
\\	関する 発音図:	関する かんする		関する かんする 関します かんします 関して かんして 関した かんした				動詞-一般 
\\	「指定(してい)」のアクセントは?873	指定 仮名 
\\	してい 漢字 
\\	指定 発音図: 助詞: オ	指定 してい		指定 してい 指定する していする 指定します していします 指定して していして 指定した していした 指定しない していしない 指定しなかった していしなかった 指定すれば していすれば 指定させる していさせる 指定される していされる 指定しろ していしろ 指定できる していできる 指定しよう していしよう				動詞-一般 
\\	「単に(たんに)」のアクセントは?874	単に 仮名 
\\	たんに 漢字 
\\	単に 発音図:							副詞 
\\	「高める(たかめる)」のアクセントは?875	高める 仮名 
\\	たかめる 漢字 
\\	高める 発音図:	高める たかめる		高める たかめる 高めます たかめます 高めて たかめて 高めた たかめた 高めない たかめない 高めなかった たかめなかった 高めれば たかめれば 高めさせる たかめさせる 高められる たかめられる 高めろ たかめろ 高められる たかめられる 高めよう たかめよう				動詞-一般 
\\	「事項(じこう)」のアクセントは?876	事項 仮名 
\\	じこう、じこー 漢字 
\\	事項 発音図: 助詞: オ	事項 じこう		事項 じこう				名詞-普通名詞-一般 
\\	「事故(じこ)」のアクセントは?877	事故 仮名 
\\	じこ 漢字 
\\	事故 発音図: 助詞: オ	事故 じこ		事故 じこ				名詞-普通名詞-一般 
\\	「事態(じたい)」のアクセントは?878	事態 仮名 
\\	じたい 漢字 
\\	事態 発音図: 助詞: オ	事態 じたい		事態 じたい				名詞-普通名詞-一般 
\\	「疲れる(つかれる)」のアクセントは?879	疲れる 仮名 
\\	つかれる 漢字 
\\	疲れる 発音図:	疲れる つかれる		疲れる つかれる 疲れます つかれます 疲れて つかれて 疲れた つかれた 疲れない つかれない 疲れなかった つかれなかった 疲れれば つかれれば 疲れさせる つかれさせる 疲れられる つかれられる 疲れろ つかれろ 疲れよう つかれよう				動詞-一般 
\\	「展開(てんかいする)」のアクセントは?880	展開 仮名 
\\	てんかい 漢字 
\\	展開 発音図: 助詞: オ	展開する てんかいする		展開する てんかいする 展開します てんかいします 展開して てんかいして 展開した てんかいした 展開しない てんかいしない 展開しなかった てんかいしなかった 展開すれば てんかいすれば 展開させる てんかいさせる 展開される てんかいされる 展開しろ てんかいしろ 展開できる てんかいできる 展開しよう てんかいしよう				動詞-一般 
\\	「視線(しせん)」のアクセントは?881	視線 仮名 
\\	しせん 漢字 
\\	視線 発音図: 助詞: オ	視線 しせん		視線 しせん				名詞-普通名詞-一般 
\\	「役(やく)」のアクセントは?882	役 仮名 
\\	やく 漢字 
\\	役 発音図: 例文: 次の舞台でいいガついた	役 やく		役 やく				名詞-普通名詞-一般 
\\	「ベッド(べっど)」のアクセントは?883	ベッド 仮名 
\\	ベッド、べっど 漢字 
\\	発音図: 助詞: オ	ベッド べっど		ベッド べっど				名詞-普通名詞-一般 
\\	「申し込み(もうしこみ)」のアクセントは?884	申し込み 仮名 
\\	もうしこみ、もーしこみ 漢字 
\\	申し込み 発音図: 助詞: オ	申し込み もうしこみ		申し込み もうしこみ				名詞-普通名詞-一般 
\\	「確実な(かくじつな)」のアクセントは?885		確実な かくじつな		確実な かくじつな 確実です かくじつです 確実で かくじつで 確実だった かくじつだった 確実じゃない かくじつじゃない 確実じゃなかった かくじつじゃなかった 確実ならば かくじつならば 確実だ かくじつだ 確実に かくじつに				形状詞-一般 
\\	「症状(しょうじょう)」のアクセントは?886	症状 仮名 
\\	しょうじょう、しょーじょー 漢字 
\\	症状 発音図: 助詞: オ 発音図: 助詞: オ	症状 しょうじょう		症状 しょうじょう				名詞-普通名詞-一般 
\\	「確実(かくじつ)」のアクセントは?887	確実 仮名 
\\	かくじつ 漢字 
\\	確実 発音図: 助詞: オ	確実 かくじつ		確実 かくじつ				形状詞-一般 
\\	「庭(にわ)」のアクセントは?888	庭 仮名 
\\	にわ 漢字 
\\	庭 発音図: 助詞: オ	庭 にわ		庭 にわ				名詞-普通名詞-一般 
\\	「戦い(たたかい)」のアクセントは?889	戦い 仮名 
\\	たたかい 漢字 
\\	戦い 発音図: 助詞: オ	戦い たたかい		戦い たたかい				名詞-普通名詞-一般 
\\	「障害(しょうがい)」のアクセントは?890	障害 仮名 
\\	しょうがい、しょーがい 漢字 
\\	障害 発音図: 助詞: オ	障害 しょうがい		障害 しょうがい				名詞-普通名詞-一般 
\\	「有効な(ゆうこうな)」のアクセントは?891	有効 仮名 
\\	ゆうこう、ゆーこー 漢字 
\\	有効 発音図: 助詞: オ	有効な ゆうこうな		有効な ゆうこうな 有効です ゆうこうです 有効で ゆうこうで 有効だった ゆうこうだった 有効じゃない ゆうこうじゃない 有効じゃなかった ゆうこうじゃなかった 有効ならば ゆうこうならば 有効だ ゆうこうだ 有効に ゆうこうに				形状詞-一般 
\\	「狭い(せまい)」のアクセントは?892	狭い 仮名 
\\	せまい 漢字 
\\	狭い 発音図:	狭い せまい		狭い せまい 狭いです せまいです 狭くて せまくて せまくて 狭かった せまかった せまかった 狭くない せまくない せまくない 狭くなかった せまくなかった せまくなかった 狭ければ せまければ せまければ 狭い せまい 狭く せまく せまく				形容詞-一般 
\\	「テーブル(てーぶる)」のアクセントは?893	テーブル 仮名 
\\	テーブル、てーぶる 漢字 
\\	発音図: 助詞: オ	テーブル てーぶる		テーブル てーぶる				名詞-普通名詞-一般 
\\	「被害(ひがい)」のアクセントは?894	被害 仮名 
\\	ひがい 漢字 
\\	被害 発音図: 助詞: オ	被害 ひがい		被害 ひがい				名詞-普通名詞-一般 
\\	「道路(どうろ)」のアクセントは?895	道路 仮名 
\\	どうろ、どーろ 漢字 
\\	道路 発音図: 助詞: オ	道路 どうろ		道路 どうろ				名詞-普通名詞-一般 
\\	「振り返る(ふりかえる)」のアクセントは?896	振り返る 仮名 
\\	ふりかえる 漢字 
\\	振り返る 発音図:	振り返る ふりかえる		振り返る ふりかえる 振り返ります ふりかえります 振り返って ふりかえって 振り返った ふりかえった 振り返らない ふりかえらない 振り返らなかった ふりかえらなかった 振り返れば ふりかえれば 振り返らせる ふりかえらせる 振り返られる ふりかえられる 振り返れ ふりかえれ 振り返れる ふりかえれる 振り返ろう ふりかえろう				動詞-一般 
\\	「御飯(ごはん)」のアクセントは?897		御飯 ごはん		御飯 ごはん				名詞-普通名詞-一般 
\\	「意外な(いがいな)」のアクセントは?898		意外な いがいな		意外な いがいな 意外です いがいです 意外で いがいで 意外だった いがいだった 意外じゃない いがいじゃない 意外じゃなかった いがいじゃなかった 意外ならば いがいならば 意外だ いがいだ 意外に いがいに				形状詞-一般 
\\	「意外(いがい)」のアクセントは?899	意外 仮名 
\\	いがい 漢字 
\\	意外 発音図: 助詞: オ 発音図: 助詞: オ	意外 いがい		意外 いがい				形状詞-一般 
\\	「距離(きょり)」のアクセントは?900	距離 仮名 
\\	きょり 漢字 
\\	距離 発音図: 助詞: オ	距離 きょり		距離 きょり				名詞-普通名詞-一般 
\\	「豊かな(ゆたかな)」のアクセントは?901	豊か 仮名 
\\	ゆたか 漢字 
\\	豊か 発音図:	豊かな ゆたかな		豊かな ゆたかな 豊かです ゆたかです 豊かで ゆたかで 豊かだった ゆたかだった 豊かじゃない ゆたかじゃない 豊かじゃなかった ゆたかじゃなかった 豊かならば ゆたかならば 豊かだ ゆたかだ 豊かに ゆたかに				形状詞-一般 
\\	「香り(かおり)」のアクセントは?902	香り 仮名 
\\	かおり 漢字 
\\	香り 発音図: 助詞: オ	香り かおり		香り かおり				名詞-普通名詞-一般 
\\	「評価(ひょうか)」のアクセントは?903	評価 仮名 
\\	ひょうか、ひょーか 漢字 
\\	評価 発音図: 助詞: オ	評価 ひょうか		評価 ひょうか 評価する ひょうかする 評価します ひょうかします 評価して ひょうかして 評価した ひょうかした 評価しない ひょうかしない 評価しなかった ひょうかしなかった 評価すれば ひょうかすれば 評価させる ひょうかさせる 評価される ひょうかされる 評価しろ ひょうかしろ 評価できる ひょうかできる 評価しよう ひょうかしよう				動詞-一般 
\\	「経る(へる)」のアクセントは?904	経る 仮名 
\\	へる 漢字 
\\	経る 発音図:	経る へる		経る へる 経ます へます 経て へて 経た へた 経ない へない 経なかった へなかった 経れば へれば 経させる へさせる 経られる へられる 経ろ へろ 経よう へよう				動詞-一般 
\\	「成功(せいこう)」のアクセントは?905	成功 仮名 
\\	せいこう、せいこー 漢字 
\\	成功 発音図: 助詞: オ	成功 せいこう		成功 せいこう 成功する せいこうする 成功します せいこうします 成功して せいこうして 成功した せいこうした 成功しない せいこうしない 成功しなかった せいこうしなかった 成功すれば せいこうすれば 成功させる せいこうさせる 成功される せいこうされる 成功しろ せいこうしろ 成功できる せいこうできる 成功しよう せいこうしよう				動詞-一般 
\\	「重ねる(かさねる)」のアクセントは?906	重ねる 仮名 
\\	かさねる 漢字 
\\	重ねる 発音図:	重ねる かさねる		重ねる かさねる 重ねます かさねます 重ねて かさねて 重ねた かさねた 重ねない かさねない 重ねなかった かさねなかった 重ねれば かさねれば 重ねさせる かさねさせる 重ねられる かさねられる 重ねろ かさねろ 重ねられる かさねられる 重ねよう かさねよう				動詞-一般 
\\	「勝つ(かつ)」のアクセントは?907	勝つ 仮名 
\\	かつ 漢字 
\\	勝つ 発音図:	勝つ かつ		勝つ かつ 勝ちます かちます 勝って かって 勝った かった 勝たない かたない 勝たなかった かたなかった 勝てば かてば 勝たせる かたせる 勝たれる かたれる 勝て かて 勝てる かてる 勝とう かとう				動詞-一般 
\\	「主張(しゅちょう)」のアクセントは?908	主張 仮名 
\\	しゅちょう、しゅちょー 漢字 
\\	主張 発音図: 助詞: オ	主張 しゅちょう		主張 しゅちょう 主張する しゅちょうする 主張します しゅちょうします 主張して しゅちょうして 主張した しゅちょうした 主張しない しゅちょうしない 主張しなかった しゅちょうしなかった 主張すれば しゅちょうすれば 主張させる しゅちょうさせる 主張される しゅちょうされる 主張しろ しゅちょうしろ 主張できる しゅちょうできる 主張しよう しゅちょうしよう				動詞-一般 
\\	「昨日(きのう)」のアクセントは?909		昨日 きのう きのう		昨日 きのう きのう				副詞 
\\	「昨日(きのう)《副》」のアクセントは?910	きのう{昨日}(副) 仮名 
\\	きのう、きのー 漢字 
\\	昨日 発音図: 助詞: オ							副詞 
\\	「昨日(きのう)《名》」のアクセントは?911	きのう{昨日}(名) 仮名 
\\	きのう、きのー 漢字 
\\	昨日 発音図: 助詞: オ							副詞 
\\	「下がる(さがる)」のアクセントは?912	下がる 仮名 
\\	さがる 漢字 
\\	下がる 発音図:	下がる さがる		下がる さがる 下がります さがります 下がって さがって 下がった さがった 下がらない さがらない 下がらなかった さがらなかった 下がれば さがれば 下がらせる さがらせる 下がられる さがられる 下がれ さがれ 下がれる さがれる 下がろう さがろう				動詞-一般 
\\	「席(せき)」のアクセントは?913	席 仮名 
\\	せき 漢字 
\\	席 発音図: 助詞: オ	席 せき		席 せき				名詞-普通名詞-一般 
\\	「肌(はだ)」のアクセントは?914	肌 仮名 
\\	はだ 漢字 
\\	肌 発音図: 助詞: オ	肌 はだ		肌 はだ				名詞-普通名詞-一般 
\\	「両親(りょうしん)」のアクセントは?915	両親 仮名 
\\	りょうしん、りょーしん 漢字 
\\	両親 発音図: 助詞: オ	両親 りょうしん		両親 りょうしん				名詞-普通名詞-一般 
\\	「学生(がくせい)」のアクセントは?916	学生 仮名 
\\	がくせい 漢字 
\\	学生 発音図: 助詞: オ	学生 がくせい		学生 がくせい				名詞-普通名詞-一般 
\\	「選手(せんしゅ)」のアクセントは?917	選手 仮名 
\\	せんしゅ 漢字 
\\	選手 発音図: 助詞: オ	選手 せんしゅ		選手 せんしゅ				名詞-普通名詞-一般 
\\	「大好きな(だいすきな)」のアクセントは?918	大好き 仮名 
\\	だいすき 漢字 
\\	大好き 発音図:	大好きな だいすきな		大好きな だいすきな 大好きです だいすきです 大好きで だいすきで 大好きだった だいすきだった 大好きじゃない だいすきじゃない 大好きじゃなかった だいすきじゃなかった 大好きならば だいすきならば 大好きだ だいすきだ 大好きに だいすきに				形状詞-一般 
\\	「抜く(ぬく)」のアクセントは?919	抜く 仮名 
\\	ぬく 漢字 
\\	抜く 発音図:	抜く ぬく		抜く ぬく 抜きます ぬきます 抜いて ぬいて 抜いた ぬいた 抜かない ぬかない 抜かなかった ぬかなかった 抜けば ぬけば 抜かせる ぬかせる 抜かれる ぬかれる 抜け ぬけ 抜ける ぬける 抜こう ぬこう				動詞-一般 
\\	「洗う(あらう)」のアクセントは?920	洗う 仮名 
\\	あらう 漢字 
\\	洗う 発音図:	洗う あらう		洗う あらう 洗います あらいます 洗って あらって 洗った あらった 洗わない あらわない 洗わなかった あらわなかった 洗えば あらえば 洗わせる あらわせる 洗われる あらわれる 洗え あらえ 洗える あらえる 洗おう あらおう				動詞-一般 
\\	「一言(ひとこと)」のアクセントは?921		一言 ひとこと		一言 ひとこと				名詞-普通名詞-一般 
\\	「構う(かまう)」のアクセントは?922	構う 仮名 
\\	かまう 漢字 
\\	構う 発音図:	構う かまう		構う かまう 構います かまいます 構って かまって 構った かまった 構わない かまわない 構わなかった かまわなかった 構えば かまえば 構わせる かまわせる 構われる かまわれる 構え かまえ 構える かまえる 構おう かまおう				動詞-一般 
\\	「葉(は)」のアクセントは?923	葉 仮名 
\\	は、よう、よー 漢字 
\\	葉 発音図: 助詞: オ 例文: 街路樹のガ黄色く色づくころだ 発音図:	葉 は		葉 は				名詞-普通名詞-一般 
\\	「魅力(みりょく)」のアクセントは?924	魅力 仮名 
\\	みりょく 漢字 
\\	魅力 発音図: 助詞: オ	魅力 みりょく		魅力 みりょく				名詞-普通名詞-一般 
\\	「包む(つつむ)」のアクセントは?925	包む 仮名 
\\	つつむ 漢字 
\\	包む 発音図:	包む つつむ		包む つつむ 包みます つつみます 包んで つつんで 包んだ つつんだ 包まない つつまない 包まなかった つつまなかった 包めば つつめば 包ませる つつませる 包まれる つつまれる 包め つつめ 包める つつめる 包もう つつもう				動詞-一般 
\\	「地方(ちほう)」のアクセントは?926	地方 仮名 
\\	じかた、ちほう、ちほー 漢字 
\\	地方 発音図: 助詞: オ 発音図: 助詞: オ 発音図: 助詞: オ	地方 ちほう ちほう		地方 ちほう ちほう				名詞-普通名詞-一般 
\\	「鞆(とも)」のアクセントは?927	とも{鞆} 仮名 
\\	とも 漢字 
\\	鞆 発音図: 例文: 展示館で昔のオ見た 発音図: 例文: 展示館で昔のオ見た							助詞-接続助詞 
\\	「艫(とも)」のアクセントは?928	とも{艫} 仮名 
\\	とも 漢字 
\\	艫 発音図: 例文: 船の後部をトいう 発音図: 例文: 船の後部をトいう							助詞-接続助詞 
\\	「辺(へん)」のアクセントは?929	辺 仮名 
\\	へん 漢字 
\\	辺 発音図: 助詞: オ	辺 へん		辺 へん				名詞-普通名詞-一般 
\\	「辺(へん)《数》」のアクセントは?930	辺(数) 仮名 
\\	へん 漢字 
\\	辺 発音図: 助詞: オ	辺 へん		辺 へん				名詞-普通名詞-一般 
\\	「下ろす(おろす)」のアクセントは?931	下ろす 仮名 
\\	おろす 漢字 
\\	下ろす 発音図:	下ろす おろす		下ろす おろす 下ろします おろします 下ろして おろして 下ろした おろした 下ろさない おろさない 下ろさなかった おろさなかった 下ろせば おろせば 下ろさせる おろさせる 下ろされる おろされる 下ろせ おろせ 下ろせる おろせる 下ろそう おろそう				動詞-一般 
\\	「野菜(やさい)」のアクセントは?932	野菜 仮名 
\\	やさい 漢字 
\\	野菜 発音図: 助詞: オ	野菜 やさい		野菜 やさい				名詞-普通名詞-一般 
\\	「緑(みどり)《〜の黒髪》」のアクセントは?933	緑(〜の黒髪) 仮名 
\\	みどり 漢字 
\\	緑 発音図: 例文: ノ黒髪	緑 みどり		緑 みどり				名詞-普通名詞-一般 
\\	「質問(しつもん)」のアクセントは?934	質問 仮名 
\\	しつもん 漢字 
\\	質問 発音図: 助詞: オ	質問 しつもん		質問 しつもん 質問する しつもんする 質問します しつもんします 質問して しつもんして 質問した しつもんした 質問しない しつもんしない 質問しなかった しつもんしなかった 質問すれば しつもんすれば 質問させる しつもんさせる 質問される しつもんされる 質問しろ しつもんしろ 質問できる しつもんできる 質問しよう しつもんしよう				動詞-一般 
\\	「寒い(さむい)」のアクセントは?935	寒い 仮名 
\\	さむい 漢字 
\\	寒い 発音図:	寒い さむい		寒い さむい 寒いです さむいです 寒くて さむくて さむくて 寒かった さむかった さむかった 寒くない さむくない さむくない 寒くなかった さむくなかった さむくなかった 寒ければ さむければ さむければ 寒い さむい 寒く さむく さむく				形容詞-一般 
\\	「助ける(たすける)」のアクセントは?936	助ける 仮名 
\\	たすける 漢字 
\\	助ける 発音図:	助ける たすける		助ける たすける 助けます たすけます 助けて たすけて 助けた たすけた 助けない たすけない 助けなかった たすけなかった 助ければ たすければ 助けさせる たすけさせる 助けられる たすけられる 助けろ たすけろ 助けられる たすけられる 助けよう たすけよう				動詞-一般 
\\	「感覚(かんかく)」のアクセントは?937	感覚 仮名 
\\	かんかく 漢字 
\\	感覚 発音図: 助詞: オ	感覚 かんかく		感覚 かんかく				名詞-普通名詞-一般 
\\	「どうぞ(どうぞ)《何卒》」のアクセントは?938	どうぞ(何卒) 仮名 
\\	どうぞ、どーぞ 漢字 
\\	発音図:							副詞 
\\	「成立(せいりつする)」のアクセントは?939	成立 仮名 
\\	せいりつ 漢字 
\\	成立 発音図: 助詞: オ	成立する せいりつする		成立する せいりつする 成立します せいりつします 成立して せいりつして 成立した せいりつした 成立しない せいりつしない 成立しなかった せいりつしなかった 成立すれば せいりつすれば 成立させる せいりつさせる 成立される せいりつされる 成立しろ せいりつしろ 成立できる せいりつできる 成立しよう せいりつしよう				動詞-一般 
\\	「降る(ふる)」のアクセントは?940	降る 仮名 
\\	ふる 漢字 
\\	降る 発音図:	降る ふる		降る ふる 降ります ふります 降って ふって 降った ふった 降らない ふらない 降らなかった ふらなかった 降れば ふれば 降らせる ふらせる 降られる ふられる 降れ ふれ 降ろう ふろう				動詞-一般 
\\	「自由(じゆう)」のアクセントは?941	自由 仮名 
\\	じゆー、じゆう 漢字 
\\	自由 発音図: 助詞: オ	自由 じゆう		自由 じゆう				名詞-普通名詞-一般 
\\	「自由な(じゆうな)」のアクセントは?942		自由な じゆうな		自由な じゆうな 自由です じゆうです 自由で じゆうで 自由だった じゆうだった 自由じゃない じゆうじゃない 自由じゃなかった じゆうじゃなかった 自由ならば じゆうならば 自由だ じゆうだ 自由に じゆうに				名詞-普通名詞-一般 
\\	「確保(かくほ)」のアクセントは?943	確保 仮名 
\\	かくほ 漢字 
\\	確保 発音図: 助詞: オ	確保 かくほ		確保 かくほ 確保する かくほする 確保します かくほします 確保して かくほして 確保した かくほした 確保しない かくほしない 確保しなかった かくほしなかった 確保すれば かくほすれば 確保させる かくほさせる 確保される かくほされる 確保しろ かくほしろ 確保できる かくほできる 確保しよう かくほしよう				動詞-一般 
\\	「回す(まわす)」のアクセントは?944	回す 仮名 
\\	まわす 漢字 
\\	回す 発音図:	回す まわす		回す まわす 回します まわします 回して まわして 回した まわした 回さない まわさない 回さなかった まわさなかった 回せば まわせば 回させる まわさせる 回される まわされる 回せ まわせ 回せる まわせる 回そう まわそう				動詞-一般 
\\	「咲く(さく)」のアクセントは?945	咲く 仮名 
\\	さく 漢字 
\\	咲く 発音図: 例文: 花がヒも近い	咲く さく		咲く さく 咲きます さきます 咲いて さいて 咲いた さいた 咲かない さかない 咲かなかった さかなかった 咲けば さけば 咲かせる さかせる 咲かれる さかれる 咲け さけ 咲こう さこう				動詞-一般 
\\	「自ら(みずから)」のアクセントは?946		自ら みずから		自ら みずから				副詞 
\\	「海外(かいがい)」のアクセントは?947	海外 仮名 
\\	かいがい 漢字 
\\	海外 発音図: 助詞: オ	海外 かいがい		海外 かいがい				名詞-普通名詞-一般 
\\	「広げる(ひろげる)」のアクセントは?948	広げる 仮名 
\\	ひろげる 漢字 
\\	広げる 発音図:	広げる ひろげる		広げる ひろげる 広げます ひろげます 広げて ひろげて 広げた ひろげた 広げない ひろげない 広げなかった ひろげなかった 広げれば ひろげれば 広げさせる ひろげさせる 広げられる ひろげられる 広げろ ひろげろ 広げられる ひろげられる 広げよう ひろげよう				動詞-一般 
\\	「表現(ひょうげん)」のアクセントは?949	表現 仮名 
\\	ひょうげん、ひょーげん 漢字 
\\	表現 発音図: 助詞: オ 発音図: 助詞: オ	表現 ひょうげん		表現 ひょうげん 表現する ひょうげんする 表現します ひょうげんします 表現して ひょうげんして 表現した ひょうげんした 表現しない ひょうげんしない 表現しなかった ひょうげんしなかった 表現すれば ひょうげんすれば 表現させる ひょうげんさせる 表現される ひょうげんされる 表現しろ ひょうげんしろ 表現できる ひょうげんできる 表現しよう ひょうげんしよう				動詞-一般 
\\	「支払う(しはらう)」のアクセントは?950	支払う 仮名 
\\	しはらう 漢字 
\\	支払う 発音図:	支払う しはらう		支払う しはらう 支払います しはらいます 支払って しはらって 支払った しはらった 支払わない しはらわない 支払わなかった しはらわなかった 支払えば しはらえば 支払わせる しはらわせる 支払われる しはらわれる 支払え しはらえ 支払える しはらえる 支払おう しはらおう				動詞-一般 
\\	「国家(こっか)」のアクセントは?951	国家 仮名 
\\	こっか 漢字 
\\	国家 発音図: 助詞: オ	国家 こっか		国家 こっか				名詞-普通名詞-一般 
\\	「時点(じてん)」のアクセントは?952	時点 仮名 
\\	じてん 漢字 
\\	時点 発音図: 助詞: オ 発音図: 助詞: オ	時点 じてん じてん		時点 じてん じてん				名詞-普通名詞-一般 
\\	「抜ける(ぬける)」のアクセントは?953	抜ける 仮名 
\\	ぬける 漢字 
\\	抜ける 発音図:	抜ける ぬける		抜ける ぬける 抜けます ぬけます 抜けて ぬけて 抜けた ぬけた 抜けない ぬけない 抜けなかった ぬけなかった 抜ければ ぬければ 抜けさせる ぬけさせる 抜けられる ぬけられる 抜けろ ぬけろ 抜けられる ぬけられる 抜けよう ぬけよう				動詞-一般 
\\	「背(せ)《〜を向ける》」のアクセントは?954	背(〜を向ける) 仮名 
\\	せ 漢字 
\\	背 発音図: 例文: 何を言ってもオ向けたままだ 発音図: 例文: 何を言ってもオ向けたままだ							名詞-普通名詞-一般 
\\	「背(せ)」のアクセントは?955		背 せ せ		背 せ せ				名詞-普通名詞-一般 
\\	「開始(かいし)」のアクセントは?956	開始 仮名 
\\	かいし 漢字 
\\	開始 発音図: 助詞: オ	開始 かいし		開始 かいし 開始する かいしする 開始します かいしします 開始して かいしして 開始した かいしした 開始しない かいししない 開始しなかった かいししなかった 開始すれば かいしすれば 開始させる かいしさせる 開始される かいしされる 開始しろ かいししろ 開始できる かいしできる 開始しよう かいししよう				動詞-一般 
\\	「伝わる(つたわる)」のアクセントは?957	伝わる 仮名 
\\	つたわる 漢字 
\\	伝わる 発音図:	伝わる つたわる		伝わる つたわる 伝わります つたわります 伝わって つたわって 伝わった つたわった 伝わらない つたわらない 伝わらなかった つたわらなかった 伝われば つたわれば 伝わらせる つたわらせる 伝わられる つたわられる 伝われ つたわれ 伝わろう つたわろう				動詞-一般 
\\	「鷲(わし)」のアクセントは?958	わし{鷲;儂} 仮名 
\\	わし 漢字 
\\	鷲、儂 発音図: 助詞: オ							代名詞 
\\	「市民(しみん)」のアクセントは?959	市民 仮名 
\\	しみん 漢字 
\\	市民 発音図: 助詞: オ	市民 しみん		市民 しみん				名詞-普通名詞-一般 
\\	「引き続く(ひきつずく)」のアクセントは?960	引き続く 仮名 
\\	ひきつずく、ひきつづく 漢字 
\\	引き続く 発音図:							動詞-一般 
\\	「生徒(せいと)」のアクセントは?961	生徒 仮名 
\\	せいと 漢字 
\\	生徒 発音図: 助詞: オ	生徒 せいと		生徒 せいと				名詞-普通名詞-一般 
\\	「向く(むく)」のアクセントは?962	向く 仮名 
\\	むく 漢字 
\\	向く 発音図:	向く むく		向く むく 向きます むきます 向いて むいて 向いた むいた 向かない むかない 向かなかった むかなかった 向けば むけば 向かせる むかせる 向かれる むかれる 向け むけ 向ける むける 向こう むこう				動詞-一般 
\\	「終わり(おわり)」のアクセントは?963	終わり 仮名 
\\	おわり 漢字 
\\	終わり 発音図: 助詞: オ	終わり おわり		終わり おわり				名詞-普通名詞-一般 
\\	「石(いし)」のアクセントは?964	石 仮名 
\\	いし 漢字 
\\	石 発音図: 助詞: オ	石 いし		石 いし				名詞-普通名詞-一般 
\\	「手続き(てつずき)」のアクセントは?965	手続き 仮名 
\\	てつずき、てつづき 漢字 
\\	手続き 発音図: 助詞: オ							名詞-普通名詞-一般 
\\	「ヨーロッパ(よーろっぱ)」のアクセントは?966	ヨーロッパ 仮名 
\\	ヨーロッパ、よーろっぱ 漢字 
\\	発音図: 助詞: オ	ヨーロッパ よーろっぱ		ヨーロッパ よーろっぱ				名詞-固有名詞-地名-一般 
\\	「現場(げんば)」のアクセントは?967	現場 仮名 
\\	げんじょう、げんじょー、げんば 漢字 
\\	現場 発音図: 助詞: オ 発音図: 助詞: オ	現場 げんば		現場 げんば				名詞-普通名詞-一般 
\\	「鼻(はな)」のアクセントは?968	鼻 仮名 
\\	はな 漢字 
\\	鼻 発音図: 例文: 良いできでガ高い	鼻 はな		鼻 はな				名詞-普通名詞-一般 
\\	「目立つ(めだつ)」のアクセントは?969	目立つ 仮名 
\\	めだつ 漢字 
\\	目立つ 発音図:	目立つ めだつ		目立つ めだつ 目立ちます めだちます 目立って めだって 目立った めだった 目立たない めだたない 目立たなかった めだたなかった 目立てば めだてば 目立たせる めだたせる 目立たれる めだたれる 目立て めだて 目立てる めだてる 目立とう めだとう				動詞-一般 
\\	「取り出す(とりだす)」のアクセントは?970	取り出す 仮名 
\\	とりだす 漢字 
\\	取り出す 発音図: 発音図:	取り出す とりだす		取り出す とりだす 取り出します とりだします 取り出して とりだして 取り出した とりだした 取り出さない とりださない 取り出さなかった とりださなかった 取り出せば とりだせば 取り出させる とりださせる 取り出される とりだされる 取り出せ とりだせ 取り出せる とりだせる 取り出そう とりだそう				動詞-一般 
\\	「詰まる(つまる)」のアクセントは?971	詰まる 仮名 
\\	つまる 漢字 
\\	詰まる 発音図:	詰まる つまる		詰まる つまる 詰まります つまります 詰まって つまって 詰まった つまった 詰まらない つまらない 詰まらなかった つまらなかった 詰まれば つまれば 詰まらせる つまらせる 詰まられる つまられる 詰まれ つまれ 詰まろう つまろう				動詞-一般 
\\	「変更(へんこうする)」のアクセントは?972	変更 仮名 
\\	へんこう、へんこー 漢字 
\\	変更 発音図: 助詞: オ	変更する へんこうする		変更する へんこうする 変更します へんこうします 変更して へんこうして 変更した へんこうした 変更しない へんこうしない 変更しなかった へんこうしなかった 変更すれば へんこうすれば 変更させる へんこうさせる 変更される へんこうされる 変更しろ へんこうしろ 変更できる へんこうできる 変更しよう へんこうしよう				動詞-一般 
\\	「駅(えき)」のアクセントは?973	駅 仮名 
\\	えき 漢字 
\\	駅 発音図: 助詞: オ	駅 えき		駅 えき				名詞-普通名詞-一般 
\\	「要る(いる)」のアクセントは?974	要る 仮名 
\\	いる 漢字 
\\	要る 発音図:	要る いる		要る いる 要ります いります 要って いって 要った いった 要らない いらない 要らなかった いらなかった 要れば いれば 要らせる いらせる 要られる いられる 要ろう いろう				動詞-一般 
\\	「浮かぶ(うかぶ)」のアクセントは?975	浮かぶ 仮名 
\\	うかぶ 漢字 
\\	浮かぶ 発音図:	浮かぶ うかぶ		浮かぶ うかぶ 浮かびます うかびます 浮かんで うかんで 浮かんだ うかんだ 浮かばない うかばない 浮かばなかった うかばなかった 浮かべば うかべば 浮かばせる うかばせる 浮かばれる うかばれる 浮かべ うかべ 浮かべる うかべる 浮かぼう うかぼう				動詞-一般 
\\	「次々(つぎつぎ)」のアクセントは?976	次々 仮名 
\\	つぎつぎ 漢字 
\\	次々 発音図: 助詞: ニ	次々 つぎつぎ		次々 つぎつぎ				副詞 
\\	「市場(しじょう)」のアクセントは?977	市場 仮名 
\\	しじょう、しじょー 漢字 
\\	市場 発音図: 助詞: オ	市場 しじょう		市場 しじょう				名詞-普通名詞-一般 
\\	「未だ(いまだ)」のアクセントは?978	いまだ{未} 仮名 
\\	いまだ 漢字 
\\	未だ 発音図: 発音図:							副詞 
\\	「眠る(ねむる)」のアクセントは?979	眠る 仮名 
\\	ねむる 漢字 
\\	眠る 発音図:	眠る ねむる		眠る ねむる 眠ります ねむります 眠って ねむって 眠った ねむった 眠らない ねむらない 眠らなかった ねむらなかった 眠れば ねむれば 眠らせる ねむらせる 眠られる ねむられる 眠れ ねむれ 眠れる ねむれる 眠ろう ねむろう				動詞-一般 
\\	「記載(きさいする)」のアクセントは?980	記載 仮名 
\\	きさい 漢字 
\\	記載 発音図: 助詞: オ	記載する きさいする		記載する きさいする 記載します きさいします 記載して きさいして 記載した きさいした 記載しない きさいしない 記載しなかった きさいしなかった 記載すれば きさいすれば 記載させる きさいさせる 記載される きさいされる 記載しろ きさいしろ 記載できる きさいできる 記載しよう きさいしよう				動詞-一般 
\\	「服(ふく)」のアクセントは?981	服 仮名 
\\	ふく 漢字 
\\	服 発音図: 例文: 好みのオ着る	服 ふく		服 ふく				名詞-普通名詞-一般 
\\	「握る(にぎる)」のアクセントは?982	握る 仮名 
\\	にぎる 漢字 
\\	握る 発音図:	握る にぎる		握る にぎる 握ります にぎります 握って にぎって 握った にぎった 握らない にぎらない 握らなかった にぎらなかった 握れば にぎれば 握らせる にぎらせる 握られる にぎられる 握れ にぎれ 握れる にぎれる 握ろう にぎろう				動詞-一般 
\\	「レベル(れべる)」のアクセントは?983	レベル 仮名 
\\	レベル、れべる 漢字 
\\	発音図: 助詞: オ 発音図: 助詞: オ	レベル れべる		レベル れべる				名詞-普通名詞-一般 
\\	「笑顔(えがお)」のアクセントは?984	笑顔 仮名 
\\	えがお 漢字 
\\	笑顔 発音図: 助詞: オ	笑顔 えがお		笑顔 えがお				名詞-普通名詞-一般 
\\	「狙う(ねらう)」のアクセントは?985	狙う 仮名 
\\	ねらう 漢字 
\\	狙う 発音図:	狙う ねらう		狙う ねらう 狙います ねらいます 狙って ねらって 狙った ねらった 狙わない ねらわない 狙わなかった ねらわなかった 狙えば ねらえば 狙わせる ねらわせる 狙われる ねらわれる 狙え ねらえ 狙える ねらえる 狙おう ねらおう				動詞-一般 
\\	「10日(とおか)」のアクセントは?986	10日 仮名 
\\	とーか、とおか 漢字 
\\	十日、10日 発音図: 助詞: オ	十日 とおか		十日 とおか				名詞-数詞 
\\	「迫る(せまる)」のアクセントは?987	迫る 仮名 
\\	せまる 漢字 
\\	迫る 発音図:	迫る せまる		迫る せまる 迫ります せまります 迫って せまって 迫った せまった 迫らない せまらない 迫らなかった せまらなかった 迫れば せまれば 迫らせる せまらせる 迫られる せまられる 迫れ せまれ 迫れる せまれる 迫ろう せまろう				動詞-一般 
\\	「京都(きょうと)」のアクセントは?988	京都 仮名 
\\	きょうと、きょーと 漢字 
\\	京都 発音図: 助詞: オ	京都 きょうと		京都 きょうと				名詞-固有名詞-地名-一般 
\\	「要する(ようする)」のアクセントは?989	要する 仮名 
\\	ようする、よーする 漢字 
\\	要する 発音図:	要する ようする		要する ようする 要します ようします 要して ようして 要した ようした 要さない ようさない 要さなかった ようさなかった 要せば ようせば 要させる ようさせる 要される ようされる 要そう ようそう				動詞-一般 
\\	「両手(りょうて)」のアクセントは?990	両手 仮名 
\\	りょうて、りょーて 漢字 
\\	両手 発音図: 助詞: オ	両手 りょうて		両手 りょうて				名詞-普通名詞-一般 
\\	「方々(かたがた)」のアクセントは?991	かたがた{方;旁} 仮名 
\\	かたがた 漢字 
\\	方々、旁々 発音図: 助詞: オ	方々 かたがた		方々 かたがた				名詞-普通名詞-一般 
\\	「2度(にど)《温度;角度など》」のアクセントは?993	2度(温度;角度など) 仮名 
\\	にど 漢字 
\\	二度、2度 発音図: 例文: 気温がサがると涼しく感じる							名詞-数詞 
\\	「2度(にど)《2回》《副》」のアクセントは?994	2度(2回)(副) 仮名 
\\	にど 漢字 
\\	二度、2度 発音図: 例文: もう彼はトしないだろう							名詞-数詞 
\\	「2度(にど)《2回》《名》」のアクセントは?995	2度(2回)(名) 仮名 
\\	にど 漢字 
\\	二度、2度 発音図: 例文: もう彼はトしないだろう							名詞-数詞 
\\	「都市(とし)」のアクセントは?996	都市 仮名 
\\	とし 漢字 
\\	都市 発音図: 助詞: オ	都市 とし		都市 とし				名詞-普通名詞-一般 
\\	「戻す(もどす)」のアクセントは?997	戻す 仮名 
\\	もどす 漢字 
\\	戻す 発音図:	戻す もどす		戻す もどす 戻します もどします 戻して もどして 戻した もどした 戻さない もどさない 戻さなかった もどさなかった 戻せば もどせば 戻させる もどさせる 戻される もどされる 戻せ もどせ 戻せる もどせる 戻そう もどそう				動詞-一般 
\\	「不安(ふあん)」のアクセントは?998	不安 仮名 
\\	ふあん 漢字 
\\	不安 発音図: 助詞: オ	不安 ふあん		不安 ふあん				形状詞-一般 
\\	「不安な(ふあんな)」のアクセントは?999		不安な ふあんな		不安な ふあんな 不安です ふあんです 不安で ふあんで 不安だった ふあんだった 不安じゃない ふあんじゃない 不安じゃなかった ふあんじゃなかった 不安ならば ふあんならば 不安だ ふあんだ 不安に ふあんに				形状詞-一般 
\\	「背中(せなか)」のアクセントは?1000	背中 仮名 
\\	せなか 漢字 
\\	背中 発音図: 助詞: オ	背中 せなか		背中 せなか				名詞-普通名詞-一般 
\\	「急ぐ(いそぐ)」のアクセントは?1001	急ぐ 仮名 
\\	いそぐ 漢字 
\\	急ぐ 発音図:	急ぐ いそぐ		急ぐ いそぐ 急ぎます いそぎます 急いで いそいで 急いだ いそいだ 急がない いそがない 急がなかった いそがなかった 急げば いそげば 急がせる いそがせる 急がれる いそがれる 急げ いそげ 急げる いそげる 急ごう いそごう				動詞-一般 
\\	「冷たい(つめたい)」のアクセントは?1002	冷たい 仮名 
\\	つめたい 漢字 
\\	冷たい 発音図:	冷たい つめたい		冷たい つめたい 冷たいです つめたいです 冷たくて つめたくて 冷たかった つめたかった 冷たくない つめたくない 冷たくなかった つめたくなかった 冷たければ つめたければ 冷たい つめたい つめたい 冷たく つめたく				形容詞-一般 
\\	「比較(ひかく)」のアクセントは?1003	比較 仮名 
\\	ひかく 漢字 
\\	比較 発音図: 助詞: オ	比較 ひかく		比較 ひかく 比較する ひかくする 比較します ひかくします 比較して ひかくして 比較した ひかくした 比較しない ひかくしない 比較しなかった ひかくしなかった 比較すれば ひかくすれば 比較させる ひかくさせる 比較される ひかくされる 比較しろ ひかくしろ 比較できる ひかくできる 比較しよう ひかくしよう				動詞-一般 
\\	「通ずる(つーずる)」のアクセントは?1004	通ずる 仮名 
\\	つーずる、つうずる 漢字 
\\	通ずる 発音図:							動詞-一般 
\\	「椅子(いす)」のアクセントは?1005	いす{椅子} 仮名 
\\	いす 漢字 
\\	椅子 発音図: 助詞: オ	椅子 いす		椅子 いす				名詞-普通名詞-一般 
\\	「仕組み(しくみ)」のアクセントは?1006	仕組み 仮名 
\\	しくみ 漢字 
\\	仕組み 発音図: 助詞: オ	仕組み しくみ		仕組み しくみ				名詞-普通名詞-一般 
\\	「推移(すいい)」のアクセントは?1007	推移 仮名 
\\	すいい 漢字 
\\	推移 発音図: 助詞: オ							名詞-普通名詞-一般 
\\	「甘い(あまい)」のアクセントは?1008	甘い 仮名 
\\	あまい 漢字 
\\	甘い 発音図:	甘い あまい		甘い あまい 甘いです あまいです 甘くて あまくて 甘かった あまかった 甘くない あまくない 甘くなかった あまくなかった 甘ければ あまければ 甘い あまい あまい 甘く あまく				形容詞-一般 
\\	「旨(むね)《その〜》」のアクセントは?1009	旨(その〜) 仮名 
\\	むね 漢字 
\\	旨 発音図: 例文: 全員にそのオ伝えてください							名詞-普通名詞-一般 
\\	「旨(むね)《〜とする》」のアクセントは?1010	旨(〜とする) 仮名 
\\	むね 漢字 
\\	旨 発音図: 例文: 私は腹八分目をトしている 発音図: 例文: 私は腹八分目をトしている							名詞-普通名詞-一般 
\\	「ファイル(ふぁいる)」のアクセントは?1011	ファイル 仮名 
\\	ファイル、ふぁいる 漢字 
\\	発音図: 助詞: オ	ファイル ふぁいる ふぁいる		ファイル ふぁいる ふぁいる				名詞-普通名詞-一般 
\\	「インターネット(いんたーねっと)」のアクセントは?1012	インターネット 仮名 
\\	インターネット、いんたーねっと 漢字 
\\	発音図: 助詞: オ	インターネット いんたーねっと		インターネット いんたーねっと				名詞-普通名詞-一般 
\\	「15日(じゅーごにち)」のアクセントは?1013	15日 仮名 
\\	じゅーごにち、じゅうごにち 漢字 
\\	十五日、15日 発音図: 助詞: オ							名詞-数詞 
\\	「温かい(あたたかい)」のアクセントは?1014	温かい 仮名 
\\	あたたかい、あったかい 漢字 
\\	温かい 発音図: 発音図:	温かい あたたかい		温かい あたたかい 温かいです あたたかいです 温かくて あたたかくて あたたかくて 温かかった あたたかかった あたたかかった 温かくない あたたかくない あたたかくない 温かくなかった あたたかくなかった あたたかくなかった 温かければ あたたかければ あたたかければ 温かい あたたかい 温かく あたたかく あたたかく				形容詞-一般 
\\	「構成(こうせい)」のアクセントは?1015	構成 仮名 
\\	こうせい、こーせい 漢字 
\\	構成 発音図: 助詞: オ	構成 こうせい		構成 こうせい 構成する こうせいする 構成します こうせいします 構成して こうせいして 構成した こうせいした 構成しない こうせいしない 構成しなかった こうせいしなかった 構成すれば こうせいすれば 構成させる こうせいさせる 構成される こうせいされる 構成しろ こうせいしろ 構成できる こうせいできる 構成しよう こうせいしよう				動詞-一般 
\\	「恐れ(おそれ)」のアクセントは?1016	恐れ 仮名 
\\	おそれ 漢字 
\\	恐れ 発音図: 助詞: オ	恐れ おそれ		恐れ おそれ				名詞-普通名詞-一般 
\\	「覗く(のぞく)」のアクセントは?1017	のぞく{覗} 仮名 
\\	のぞく 漢字 
\\	覗く 発音図:	覗く のぞく		覗く のぞく 覗きます のぞきます 覗いて のぞいて 覗いた のぞいた 覗かない のぞかない 覗かなかった のぞかなかった 覗けば のぞけば 覗かせる のぞかせる 覗かれる のぞかれる 覗け のぞけ 覗ける のぞける 覗こう のぞこう				動詞-一般 
\\	「唇(くちびる)」のアクセントは?1018	唇 仮名 
\\	くちびる 漢字 
\\	唇 発音図: 助詞: オ	唇 くちびる		唇 くちびる				名詞-普通名詞-一般 
\\	「地元(じもと)」のアクセントは?1019	地元 仮名 
\\	じもと 漢字 
\\	地元 発音図: 例文: この土地は彼のデす 発音図: 例文: この土地は彼のデす	地元 じもと		地元 じもと				名詞-普通名詞-一般 
\\	「人口(じんこう)」のアクセントは?1020	人口 仮名 
\\	じんこう、じんこー 漢字 
\\	人口 発音図: 助詞: オ	人口 じんこう		人口 じんこう				名詞-普通名詞-一般 
\\	「バランス(ばらんす)」のアクセントは?1021	バランス 仮名 
\\	バランス、ばらんす 漢字 
\\	発音図: 助詞: オ	バランス ばらんす		バランス ばらんす				名詞-普通名詞-一般 
\\	「経済(けいざい)」のアクセントは?1022	経済 仮名 
\\	けいざい 漢字 
\\	経済 発音図: 助詞: オ	経済 けいざい		経済 けいざい				名詞-普通名詞-一般 
\\	「心配(しんぱい)」のアクセントは?1023	心配 仮名 
\\	しんぱい 漢字 
\\	心配 発音図: 助詞: オ	心配 しんぱい		心配 しんぱい 心配する しんぱいする 心配します しんぱいします 心配して しんぱいして 心配した しんぱいした 心配しない しんぱいしない 心配しなかった しんぱいしなかった 心配すれば しんぱいすれば 心配させる しんぱいさせる 心配される しんぱいされる 心配しろ しんぱいしろ 心配できる しんぱいできる 心配しよう しんぱいしよう				動詞-一般 
\\	「兄(あに)」のアクセントは?1024	兄 仮名 
\\	あに、けい 漢字 
\\	兄 発音図: 助詞: オ 発音図: 助詞: オ	兄 あに		兄 あに				名詞-普通名詞-一般 
\\	「デザイン(でざいん)」のアクセントは?1026	デザイン 仮名 
\\	デザイン、でざいん 漢字 
\\	発音図: 助詞: オ	デザイン でざいん		デザイン でざいん				名詞-普通名詞-一般 
\\	「育つ(そだつ)」のアクセントは?1027	育つ 仮名 
\\	そだつ 漢字 
\\	育つ 発音図:	育つ そだつ		育つ そだつ 育ちます そだちます 育って そだって 育った そだった 育たない そだたない 育たなかった そだたなかった 育てば そだてば 育たせる そだたせる 育たれる そだたれる 育て そだて 育とう そだとう				動詞-一般 
\\	「細かい(こまかい)」のアクセントは?1028	細かい 仮名 
\\	こまかい 漢字 
\\	細かい 発音図:	細かい こまかい		細かい こまかい 細かいです こまかいです 細かくて こまかくて こまかくて 細かかった こまかかった こまかかった 細かくない こまかくない こまかくない 細かくなかった こまかくなかった こまかくなかった 細かければ こまかければ こまかければ 細かい こまかい 細かく こまかく こまかく				形容詞-一般 
\\	「前提(ぜんてい)」のアクセントは?1029	前提 仮名 
\\	ぜんてい 漢字 
\\	前提 発音図: 助詞: オ	前提 ぜんてい		前提 ぜんてい				名詞-普通名詞-一般 
\\	「犯人(はんにん)」のアクセントは?1030	犯人 仮名 
\\	はんにん 漢字 
\\	犯人 発音図: 助詞: オ	犯人 はんにん		犯人 はんにん				名詞-普通名詞-一般 
\\	「高齢者(こうれいしゃ)」のアクセントは?1031	高齢者 仮名 
\\	こうれいしゃ、こーれいしゃ 漢字 
\\	高齢者 発音図: 助詞: オ	高齢者 こうれいしゃ		高齢者 こうれいしゃ				名詞-普通名詞-一般 
\\	「長さ(ながさ)」のアクセントは?1032	長さ 仮名 
\\	ながさ 漢字 
\\	長さ 発音図: 助詞: オ	長さ ながさ		長さ ながさ				名詞-普通名詞-一般 
\\	「遅れる(おくれる)」のアクセントは?1033	遅れる 仮名 
\\	おくれる 漢字 
\\	遅れる 発音図:	遅れる おくれる		遅れる おくれる 遅れます おくれます 遅れて おくれて 遅れた おくれた 遅れない おくれない 遅れなかった おくれなかった 遅れれば おくれれば 遅れさせる おくれさせる 遅れられる おくれられる 遅れろ おくれろ 遅れよう おくれよう				動詞-一般 
\\	「裏(うら)」のアクセントは?1034	裏 仮名 
\\	うら 漢字 
\\	裏 発音図: 助詞: オ	裏 うら		裏 うら				名詞-普通名詞-一般 
\\	「ボール(ぼーる)」のアクセントは?1035	ボール 仮名 
\\	ボール、ぼーる 漢字 
\\	発音図: 助詞: オ 発音図: 助詞: オ	ボール ぼーる		ボール ぼーる				名詞-普通名詞-一般 
\\	「消す(けす)」のアクセントは?1036	消す 仮名 
\\	けす 漢字 
\\	消す 発音図:	消す けす		消す けす 消します けします 消して けして 消した けした 消さない けさない 消さなかった けさなかった 消せば けせば 消させる けさせる 消される けされる 消せ けせ 消せる けせる 消そう けそう				動詞-一般 
\\	「忙しい(いそがしい)《お〜》」のアクセントは?1037	忙しい(お〜) 仮名 
\\	いそがしい 漢字 
\\	忙しい 発音図:	忙しい いそがしい		忙しい いそがしい 忙しいです いそがしいです 忙しくて いそがしくて いそがしくて 忙しかった いそがしかった いそがしかった 忙しくない いそがしくない いそがしくない 忙しくなかった いそがしくなかった いそがしくなかった 忙しければ いそがしければ いそがしければ 忙しい いそがしい 忙しく いそがしく いそがしく				形容詞-一般 
\\	「亡くなる(なくなる)」のアクセントは?1038	亡くなる 仮名 
\\	なくなる 漢字 
\\	亡くなる 発音図:	亡くなる なくなる		亡くなる なくなる 亡くなります なくなります 亡くなって なくなって 亡くなった なくなった 亡くならない なくならない 亡くならなかった なくならなかった 亡くなれば なくなれば 亡くならせる なくならせる 亡くなられる なくなられる 亡くなれ なくなれ 亡くなろう なくなろう				動詞-一般 
\\	「踏まえる(ふまえる)」のアクセントは?1039	踏まえる 仮名 
\\	ふまえる 漢字 
\\	踏まえる 発音図: 発音図:	踏まえる ふまえる ふまえる		踏まえる ふまえる ふまえる 踏まえます ふまえます 踏まえて ふまえて 踏まえた ふまえた 踏まえない ふまえない 踏まえなかった ふまえなかった 踏まえれば ふまえれば ふまえれば 踏まえさせる ふまえさせる 踏まえられる ふまえられる 踏まえろ ふまえろ ふまえろ 踏まえられる ふまえられる 踏まえよう ふまえよう				動詞-一般 
\\	「慌てる(あわてる)」のアクセントは?1040	慌てる 仮名 
\\	あわてる 漢字 
\\	慌てる 発音図:	慌てる あわてる		慌てる あわてる 慌てます あわてます 慌てて あわてて 慌てた あわてた 慌てない あわてない 慌てなかった あわてなかった 慌てれば あわてれば 慌てさせる あわてさせる 慌てられる あわてられる 慌てろ あわてろ 慌てよう あわてよう				動詞-一般 
\\	「大阪(おおさか)」のアクセントは?1041	大阪 仮名 
\\	おーさか、おおさか 漢字 
\\	大阪 発音図: 助詞: オ	大阪 おおさか		大阪 おおさか				名詞-固有名詞-地名-一般 
\\	「高まる(たかまる)」のアクセントは?1042	高まる 仮名 
\\	たかまる 漢字 
\\	高まる 発音図:	高まる たかまる		高まる たかまる 高まります たかまります 高まって たかまって 高まった たかまった 高まらない たかまらない 高まらなかった たかまらなかった 高まれば たかまれば 高まらせる たかまらせる 高まられる たかまられる 高まれ たかまれ 高まれる たかまれる 高まろう たかまろう				動詞-一般 
\\	「活用(かつよう)」のアクセントは?1043	活用 仮名 
\\	かつよう、かつよー 漢字 
\\	活用 発音図: 助詞: オ	活用 かつよう		活用 かつよう 活用する かつようする 活用します かつようします 活用して かつようして 活用した かつようした 活用しない かつようしない 活用しなかった かつようしなかった 活用すれば かつようすれば 活用させる かつようさせる 活用される かつようされる 活用しろ かつようしろ 活用できる かつようできる 活用しよう かつようしよう				動詞-一般 
\\	「舞台(ぶたい)」のアクセントは?1044	舞台 仮名 
\\	ぶたい 漢字 
\\	舞台 発音図: 助詞: オ	舞台 ぶたい		舞台 ぶたい				名詞-普通名詞-一般 
\\	「精神(せいしん)」のアクセントは?1045	精神 仮名 
\\	せいしん 漢字 
\\	精神 発音図: 助詞: オ	精神 せいしん		精神 せいしん				名詞-普通名詞-一般 
\\	「見なす(みなす)」のアクセントは?1046	見なす 仮名 
\\	みなす 漢字 
\\	見なす 発音図: 発音図:	見なす みなす みなす		見なす みなす みなす 見なします みなします 見なして みなして みなして 見なした みなした みなした 見なさない みなさない みなさない 見なさなかった みなさなかった みなさなかった 見なせば みなせば みなせば 見なさせる みなさせる みなさせる 見なされる みなされる みなされる 見なせ みなせ みなせ 見なせる みなせる みなせる 見なそう みなそう				動詞-一般 
\\	「当たり前(あたりまえ)」のアクセントは?1047		当たり前 あたりまえ		当たり前 あたりまえ				形状詞-一般 
\\	「拡大(かくだい)」のアクセントは?1048	拡大 仮名 
\\	かくだい 漢字 
\\	拡大 発音図: 助詞: オ	拡大 かくだい		拡大 かくだい 拡大する かくだいする 拡大します かくだいします 拡大して かくだいして 拡大した かくだいした 拡大しない かくだいしない 拡大しなかった かくだいしなかった 拡大すれば かくだいすれば 拡大させる かくだいさせる 拡大される かくだいされる 拡大しろ かくだいしろ 拡大できる かくだいできる 拡大しよう かくだいしよう				動詞-一般 
\\	「呟く(つぶやく)」のアクセントは?1049	つぶやく{呟} 仮名 
\\	つぶやく 漢字 
\\	呟く 発音図:	呟く つぶやく		呟く つぶやく 呟きます つぶやきます 呟いて つぶやいて 呟いた つぶやいた 呟かない つぶやかない 呟かなかった つぶやかなかった 呟けば つぶやけば 呟かせる つぶやかせる 呟かれる つぶやかれる 呟け つぶやけ 呟ける つぶやける 呟こう つぶやこう				動詞-一般 
\\	「卵(たまご)」のアクセントは?1050	卵 仮名 
\\	たまご、らん 漢字 
\\	卵 発音図: 助詞: オ 発音図: 助詞: オ 発音図: 助詞: オ	卵 たまご		卵 たまご				名詞-普通名詞-一般 
\\	「成果(せいか)」のアクセントは?1051	成果 仮名 
\\	せいか 漢字 
\\	成果 発音図: 助詞: オ	成果 せいか		成果 せいか				名詞-普通名詞-一般 
\\	「住民(じゅうみん)」のアクセントは?1052	住民 仮名 
\\	じゅーみん、じゅうみん 漢字 
\\	住民 発音図: 助詞: オ	住民 じゅうみん		住民 じゅうみん				名詞-普通名詞-一般 
\\	「勧める(すすめる)」のアクセントは?1053	勧める 仮名 
\\	すすめる 漢字 
\\	勧める 発音図:	勧める すすめる		勧める すすめる 勧めます すすめます 勧めて すすめて 勧めた すすめた 勧めない すすめない 勧めなかった すすめなかった 勧めれば すすめれば 勧めさせる すすめさせる 勧められる すすめられる 勧めろ すすめろ 勧められる すすめられる 勧めよう すすめよう				動詞-一般 
\\	「今回(こんかい)」のアクセントは?1054	今回 仮名 
\\	こんかい 漢字 
\\	今回 発音図: 助詞: オ	今回 こんかい		今回 こんかい				副詞 
\\	「政治(せいじ)」のアクセントは?1055	政治 仮名 
\\	せいじ 漢字 
\\	政治 発音図: 助詞: オ	政治 せいじ		政治 せいじ				名詞-普通名詞-一般 
\\	「訴える(うったえる)」のアクセントは?1056	訴える 仮名 
\\	うったえる 漢字 
\\	訴える 発音図: 発音図:	訴える うったえる うったえる		訴える うったえる うったえる 訴えます うったえます 訴えて うったえて 訴えた うったえた 訴えない うったえない うったえない 訴えなかった うったえなかった うったえなかった 訴えれば うったえれば うったえれば 訴えさせる うったえさせる 訴えられる うったえられる 訴えろ うったえろ うったえろ 訴えられる うったえられる 訴えよう うったえよう				動詞-一般 
\\	「満たす(みたす)」のアクセントは?1057	満たす 仮名 
\\	みたす 漢字 
\\	満たす 発音図:	満たす みたす		満たす みたす 満たします みたします 満たして みたして 満たした みたした 満たさない みたさない 満たさなかった みたさなかった 満たせば みたせば 満たさせる みたさせる 満たされる みたされる 満たせ みたせ 満たせる みたせる 満たそう みたそう				動詞-一般 
\\	「対策(たいさく)」のアクセントは?1058	対策 仮名 
\\	たいさく 漢字 
\\	対策 発音図: 助詞: オ	対策 たいさく		対策 たいさく				名詞-普通名詞-一般 
\\	「保つ(たもつ)」のアクセントは?1059	保つ 仮名 
\\	たもつ 漢字 
\\	保つ 発音図:	保つ たもつ		保つ たもつ 保ちます たもちます 保って たもって 保った たもった 保たない たもたない 保たなかった たもたなかった 保てば たもてば 保たせる たもたせる 保たれる たもたれる 保て たもて 保てる たもてる 保とう たもとう				動詞-一般 
\\	「慣れる(なれる)」のアクセントは?1060	慣れる 仮名 
\\	なれる 漢字 
\\	慣れる 発音図:	慣れる なれる		慣れる なれる 慣れます なれます 慣れて なれて 慣れた なれた 慣れない なれない 慣れなかった なれなかった 慣れれば なれれば 慣れさせる なれさせる 慣れられる なれられる 慣れろ なれろ 慣れられる なれられる 慣れよう なれよう				動詞-一般 
\\	「歯(は)」のアクセントは?1061	歯 仮名 
\\	は 漢字 
\\	歯 発音図: 例文: 良いオ保つには手入れが大事	歯 は		歯 は				名詞-普通名詞-一般 
\\	「構造(こうぞう)」のアクセントは?1062	構造 仮名 
\\	こうぞう、こーぞー 漢字 
\\	構造 発音図: 助詞: オ	構造 こうぞう		構造 こうぞう				名詞-普通名詞-一般 
\\	「罪(つみ)」のアクセントは?1063	罪 仮名 
\\	つみ 漢字 
\\	罪 発音図: 助詞: オ	罪 つみ		罪 つみ				名詞-普通名詞-一般 
\\	「素敵な(すてきな)《〜な》」のアクセントは?1064	すてき{素敵}(〜な) 仮名 
\\	すてき 漢字 
\\	素敵 発音図: 助詞: ナ	素敵な すてきな		素敵な すてきな 素敵です すてきです 素敵で すてきで 素敵だった すてきだった 素敵じゃない すてきじゃない 素敵じゃなかった すてきじゃなかった 素敵ならば すてきならば 素敵だ すてきだ 素敵に すてきに				形状詞-一般 
\\	「手段(しゅだん)」のアクセントは?1065	手段 仮名 
\\	しゅだん 漢字 
\\	手段 発音図: 助詞: オ	手段 しゅだん		手段 しゅだん				名詞-普通名詞-一般 
\\	「特定(とくてい)」のアクセントは?1066	特定 仮名 
\\	とくてい 漢字 
\\	特定 発音図: 助詞: オ	特定 とくてい		特定 とくてい				名詞-普通名詞-一般 
\\	「願う(ねがう)」のアクセントは?1067	願う 仮名 
\\	ねがう 漢字 
\\	願う 発音図:	願う ねがう		願う ねがう 願います ねがいます 願って ねがって 願った ねがった 願わない ねがわない 願わなかった ねがわなかった 願えば ねがえば 願わせる ねがわせる 願われる ねがわれる 願え ねがえ 願える ねがえる 願おう ねがおう				動詞-一般 
\\	「太陽(たいよう)」のアクセントは?1068	太陽 仮名 
\\	たいよう、たいよー 漢字 
\\	太陽 発音図: 助詞: オ	太陽 たいよう		太陽 たいよう				名詞-普通名詞-一般 
\\	「溢れる(あふれる)」のアクセントは?1069	あふれる{溢} 仮名 
\\	あふれる 漢字 
\\	溢れる 発音図:	溢れる あふれる		溢れる あふれる 溢れます あふれます 溢れて あふれて 溢れた あふれた 溢れない あふれない 溢れなかった あふれなかった 溢れれば あふれれば 溢れさせる あふれさせる 溢れられる あふれられる 溢れろ あふれろ 溢れよう あふれよう				動詞-一般 
\\	「若者(わかもの)」のアクセントは?1070	若者 仮名 
\\	わかもの 漢字 
\\	若者 発音図: 助詞: オ	若者 わかもの		若者 わかもの				名詞-普通名詞-一般 
\\	「つい(つい)《〜うっかりと》」のアクセントは?1071	つい(〜うっかりと) 仮名 
\\	つい 漢字 
\\	発音図: 例文: ウっかりと							副詞 
\\	「終(つい)《〜の住みか》」のアクセントは?1072	つい{終}(〜の住みか) 仮名 
\\	つい 漢字 
\\	終 発音図: 例文: ノ住みか							副詞 
\\	「ママ(まま)」のアクセントは?1073	ママ 仮名 
\\	ママ、まま 漢字 
\\	発音図: 例文: 花を贈ってオ喜ばせる	ママ まま		ママ まま				名詞-普通名詞-一般 
\\	「職員(しょくいん)」のアクセントは?1074	職員 仮名 
\\	しょくいん 漢字 
\\	職員 発音図: 助詞: オ	職員 しょくいん		職員 しょくいん				名詞-普通名詞-一般 
\\	「絶対(ぜったい)」のアクセントは?1075	絶対 仮名 
\\	ぜったい 漢字 
\\	絶対 発音図: 助詞: オ							副詞 
\\	「相談(そうだん)」のアクセントは?1076	相談 仮名 
\\	そうだん、そーだん 漢字 
\\	相談 発音図: 助詞: オ	相談 そうだん		相談 そうだん 相談する そうだんする 相談します そうだんします 相談して そうだんして 相談した そうだんした 相談しない そうだんしない 相談しなかった そうだんしなかった 相談すれば そうだんすれば 相談させる そうだんさせる 相談される そうだんされる 相談しろ そうだんしろ 相談できる そうだんできる 相談しよう そうだんしよう				動詞-一般 
\\	「画面(がめん)」のアクセントは?1077	画面 仮名 
\\	がめん 漢字 
\\	画面 発音図: 助詞: オ 発音図: 助詞: オ	画面 がめん		画面 がめん				名詞-普通名詞-一般 
\\	「オーケー(オーケー)」のアクセントは?1078	オーケー 仮名 
\\	オーケー、おーけー 漢字 
\\	発音図: 助詞: オ							名詞-普通名詞-一般 
\\	「間違う(まちがう)」のアクセントは?1079		間違う まちがう		間違う まちがう 間違います まちがいます 間違って まちがって 間違った まちがった 間違わない まちがわない 間違わなかった まちがわなかった 間違えば まちがえば 間違わせる まちがわせる 間違われる まちがわれる 間違え まちがえ 間違おう まちがおう				動詞-一般 
\\	「調子(ちょうし)」のアクセントは?1080	調子 仮名 
\\	ちょうし、ちょーし 漢字 
\\	調子 発音図: 助詞: オ	調子 ちょうし		調子 ちょうし				名詞-普通名詞-一般 
\\	「場面(ばめん)」のアクセントは?1081	場面 仮名 
\\	ばめん 漢字 
\\	場面 発音図: 助詞: オ 発音図: 助詞: オ	場面 ばめん ばめん		場面 ばめん ばめん				名詞-普通名詞-一般 
\\	「立派な(りっぱな)」のアクセントは?1082	立派 仮名 
\\	りっぱ 漢字 
\\	立派 発音図: 助詞: オ	立派な りっぱな		立派な りっぱな 立派です りっぱです 立派で りっぱで 立派だった りっぱだった 立派じゃない りっぱじゃない 立派じゃなかった りっぱじゃなかった 立派ならば りっぱならば 立派だ りっぱだ 立派に りっぱに				形状詞-一般 
\\	「定員(ていいん)」のアクセントは?1083	定員 仮名 
\\	ていいん 漢字 
\\	定員 発音図: 助詞: オ	定員 ていいん		定員 ていいん				名詞-普通名詞-一般 
\\	「告げる(つげる)」のアクセントは?1084	告げる 仮名 
\\	つげる 漢字 
\\	告げる 発音図:	告げる つげる		告げる つげる 告げます つげます 告げて つげて 告げた つげた 告げない つげない 告げなかった つげなかった 告げれば つげれば 告げさせる つげさせる 告げられる つげられる 告げろ つげろ 告げられる つげられる 告げよう つげよう				動詞-一般 
\\	「容易(ようい)」のアクセントは?1085	容易 仮名 
\\	ようい、よーい 漢字 
\\	容易 発音図: 助詞: オ	容易 ようい		容易 ようい				形状詞-一般 
\\	「容易な(よういな)」のアクセントは?1086		容易な よういな		容易な よういな 容易です よういです 容易で よういで 容易だった よういだった 容易じゃない よういじゃない 容易じゃなかった よういじゃなかった 容易ならば よういならば 容易だ よういだ 容易に よういに				形状詞-一般 
\\	「塩(しお)」のアクセントは?1087	塩 仮名 
\\	しお 漢字 
\\	塩 発音図: 助詞: オ	塩 しお		塩 しお				名詞-普通名詞-一般 
\\	「注目(ちゅうもく)」のアクセントは?1088	注目 仮名 
\\	ちゅーもく、ちゅうもく 漢字 
\\	注目 発音図: 助詞: オ	注目 ちゅうもく		注目 ちゅうもく 注目する ちゅうもくする 注目します ちゅうもくします 注目して ちゅうもくして 注目した ちゅうもくした 注目しない ちゅうもくしない 注目しなかった ちゅうもくしなかった 注目すれば ちゅうもくすれば 注目させる ちゅうもくさせる 注目される ちゅうもくされる 注目しろ ちゅうもくしろ 注目できる ちゅうもくできる 注目しよう ちゅうもくしよう				動詞-一般 
\\	「閉じる(とじる)」のアクセントは?1089	閉じる 仮名 
\\	とじる 漢字 
\\	閉じる 発音図:	閉じる とじる		閉じる とじる 閉じます とじます 閉じて とじて 閉じた とじた 閉じない とじない 閉じなかった とじなかった 閉じれば とじれば 閉じさせる とじさせる 閉じられる とじられる 閉じろ とじろ 閉じられる とじられる 閉じよう とじよう				動詞-一般 
\\	「熱い(あつい)」のアクセントは?1090	熱い 仮名 
\\	あつい 漢字 
\\	熱い 発音図:	熱い あつい		熱い あつい 熱いです あついです 熱くて あつくて あつくて 熱かった あつかった あつかった 熱くない あつくない あつくない 熱くなかった あつくなかった あつくなかった 熱ければ あつければ あつければ 熱い あつい 熱く あつく あつく				形容詞-一般 
\\	「老人(ろうじん)」のアクセントは?1091	老人 仮名 
\\	ろうじん、ろーじん 漢字 
\\	老人 発音図: 助詞: オ	老人 ろうじん		老人 ろうじん				名詞-普通名詞-一般 
\\	「肉(にく)」のアクセントは?1092	肉 仮名 
\\	にく 漢字 
\\	肉 発音図: 助詞: オ	肉 にく		肉 にく				名詞-普通名詞-一般 
\\	「製品(せいひん)」のアクセントは?1093	製品 仮名 
\\	せいひん 漢字 
\\	製品 発音図: 助詞: オ	製品 せいひん		製品 せいひん				名詞-普通名詞-一般 
\\	「導入(どうにゅう)」のアクセントは?1094	導入 仮名 
\\	どうにゅう、どーにゅー 漢字 
\\	導入 発音図: 助詞: オ	導入 どうにゅう		導入 どうにゅう 導入する どうにゅうする 導入します どうにゅうします 導入して どうにゅうして 導入した どうにゅうした 導入しない どうにゅうしない 導入しなかった どうにゅうしなかった 導入すれば どうにゅうすれば 導入させる どうにゅうさせる 導入される どうにゅうされる 導入しろ どうにゅうしろ 導入できる どうにゅうできる 導入しよう どうにゅうしよう				動詞-一般 
\\	「奥さん(おくさん)」のアクセントは?1095	奥さん 仮名 
\\	おくさん 漢字 
\\	奥さん 発音図: 助詞: オ	奥さん おくさん		奥さん おくさん				名詞-普通名詞-一般 
\\	「採用(さいよう)」のアクセントは?1096	採用 仮名 
\\	さいよう、さいよー 漢字 
\\	採用 発音図: 助詞: オ	採用 さいよう		採用 さいよう 採用する さいようする 採用します さいようします 採用して さいようして 採用した さいようした 採用しない さいようしない 採用しなかった さいようしなかった 採用すれば さいようすれば 採用させる さいようさせる 採用される さいようされる 採用しろ さいようしろ 採用できる さいようできる 採用しよう さいようしよう				動詞-一般 
\\	「輝く(かがやく)」のアクセントは?1097	輝く 仮名 
\\	かがやく 漢字 
\\	輝く 発音図:	輝く かがやく		輝く かがやく 輝きます かがやきます 輝いて かがやいて 輝いた かがやいた 輝かない かがやかない 輝かなかった かがやかなかった 輝けば かがやけば 輝かせる かがやかせる 輝かれる かがやかれる 輝け かがやけ 輝ける かがやける 輝こう かがやこう				動詞-一般 
\\	「並べる(ならべる)」のアクセントは?1098	並べる 仮名 
\\	ならべる 漢字 
\\	並べる 発音図:	並べる ならべる		並べる ならべる 並べます ならべます 並べて ならべて 並べた ならべた 並べない ならべない 並べなかった ならべなかった 並べれば ならべれば 並べさせる ならべさせる 並べられる ならべられる 並べろ ならべろ 並べられる ならべられる 並べよう ならべよう				動詞-一般 
\\	「必ずしも(かならずしも)」のアクセントは?1099	必ずしも 仮名 
\\	かならずしも 漢字 
\\	必ずしも 発音図: 発音図:							副詞 
\\	「自宅(じたく)」のアクセントは?1100	自宅 仮名 
\\	じたく 漢字 
\\	自宅 発音図: 助詞: オ	自宅 じたく		自宅 じたく				名詞-普通名詞-一般 
\\	「沿う(そう)」のアクセントは?1101	沿う 仮名 
\\	そう 漢字 
\\	沿う 発音図: 発音図:	沿う そう そう		沿う そう そう 沿います そいます 沿って そって そって 沿った そった そった 沿わない そわない そわない 沿わなかった そわなかった そわなかった 沿えば そえば そえば 沿わせる そわせる そわせる 沿われる そわれる そわれる 沿え そえ そえ 沿える そえる そえる 沿おう そおう				動詞-一般 
\\	「空間(くうかん)」のアクセントは?1102	空間 仮名 
\\	くーかん、くうかん 漢字 
\\	空間 発音図: 助詞: オ	空間 くうかん		空間 くうかん				名詞-普通名詞-一般 
\\	「移動(いどうする)」のアクセントは?1103	移動 仮名 
\\	いどう、いどー 漢字 
\\	移動 発音図: 助詞: オ	移動する いどうする		移動する いどうする 移動します いどうします 移動して いどうして 移動した いどうした 移動しない いどうしない 移動しなかった いどうしなかった 移動すれば いどうすれば 移動させる いどうさせる 移動される いどうされる 移動しろ いどうしろ 移動できる いどうできる 移動しよう いどうしよう				動詞-一般 
\\	「諦める(あきらめる)」のアクセントは?1104		諦める あきらめる		諦める あきらめる 諦めます あきらめます 諦めて あきらめて 諦めた あきらめた 諦めない あきらめない 諦めなかった あきらめなかった 諦めれば あきらめれば 諦めさせる あきらめさせる 諦められる あきらめられる 諦めろ あきらめろ 諦められる あきらめられる 諦めよう あきらめよう				動詞-一般 
\\	「備える(そなえる)」のアクセントは?1105	備える 仮名 
\\	そなえる 漢字 
\\	備える 発音図: 発音図:	備える そなえる		備える そなえる 備えます そなえます 備えて そなえて 備えた そなえた 備えない そなえない 備えなかった そなえなかった 備えれば そなえれば 備えさせる そなえさせる 備えられる そなえられる 備えろ そなえろ 備えられる そなえられる 備えよう そなえよう				動詞-一般 
\\	「法(ほう)」のアクセントは?1106	法 仮名 
\\	ほう、ほー 漢字 
\\	法 発音図: 助詞: オ	法 ほう		法 ほう				名詞-普通名詞-一般 
\\	「未だ未だ(まだまだ)」のアクセントは?1107	まだまだ{未} 仮名 
\\	まだまだ 漢字 
\\	未だ未だ 発音図:							副詞 
\\	「地球(ちきゅう)」のアクセントは?1108	地球 仮名 
\\	ちきゅー、ちきゅう 漢字 
\\	地球 発音図: 助詞: オ	地球 ちきゅう		地球 ちきゅう				名詞-普通名詞-一般 
\\	「世の中(よのなか)」のアクセントは?1109	世の中 仮名 
\\	よのなか 漢字 
\\	世の中 発音図: 助詞: オ	世の中 よのなか		世の中 よのなか				名詞-普通名詞-一般 
\\	「低下(ていか)」のアクセントは?1110	低下 仮名 
\\	ていか 漢字 
\\	低下 発音図: 助詞: オ 発音図: 助詞: オ	低下 ていか		低下 ていか 低下する ていかする 低下します ていかします 低下して ていかして 低下した ていかした 低下しない ていかしない 低下しなかった ていかしなかった 低下すれば ていかすれば 低下させる ていかさせる 低下される ていかされる 低下しろ ていかしろ 低下できる ていかできる 低下しよう ていかしよう				動詞-一般 
\\	「要素(ようそ)」のアクセントは?1111	要素 仮名 
\\	ようそ、よーそ 漢字 
\\	要素 発音図: 助詞: オ	要素 ようそ		要素 ようそ				名詞-普通名詞-一般 
\\	「20日(はつか)」のアクセントは?1112	20日 仮名 
\\	はつか 漢字 
\\	二十日、20日 発音図: 助詞: オ	二十日 はつか		二十日 はつか				名詞-数詞 
\\	「二十日(はつか)」のアクセントは?1113	二十日 仮名 
\\	はつか 漢字 
\\	二十日 発音図: 助詞: オ	二十日 はつか		二十日 はつか				名詞-数詞 
\\	「たって(たって)」のアクセントは?1114	たって 仮名 
\\	たって 漢字 
\\	発音図: 発音図:							助詞-接続助詞 
\\	「移る(うつる)」のアクセントは?1115	移る 仮名 
\\	うつる 漢字 
\\	移る 発音図:	移る うつる		移る うつる 移ります うつります 移って うつって 移った うつった 移らない うつらない 移らなかった うつらなかった 移れば うつれば 移らせる うつらせる 移られる うつられる 移れ うつれ 移れる うつれる 移ろう うつろう				動詞-一般 
\\	「値段(ねだん)《お〜》」のアクセントは?1116	値段(お〜) 仮名 
\\	ねだん 漢字 
\\	値段 発音図: 助詞: オ	値段 ねだん		値段 ねだん				名詞-普通名詞-一般 
\\	「赤ちゃん(あかちゃん)」のアクセントは?1117	赤ちゃん 仮名 
\\	あかちゃん 漢字 
\\	赤ちゃん 発音図: 助詞: オ	赤ちゃん あかちゃん		赤ちゃん あかちゃん				名詞-普通名詞-一般 
\\	「誘う(さそう)」のアクセントは?1118	誘う 仮名 
\\	さそう 漢字 
\\	誘う 発音図:	誘う さそう		誘う さそう 誘います さそいます 誘って さそって 誘った さそった 誘わない さそわない 誘わなかった さそわなかった 誘えば さそえば 誘わせる さそわせる 誘われる さそわれる 誘え さそえ 誘える さそえる 誘おう さそおう				動詞-一般 
\\	「報告(ほうこく)」のアクセントは?1119	報告 仮名 
\\	ほうこく、ほーこく 漢字 
\\	報告 発音図: 助詞: オ	報告 ほうこく		報告 ほうこく 報告する ほうこくする 報告します ほうこくします 報告して ほうこくして 報告した ほうこくした 報告しない ほうこくしない 報告しなかった ほうこくしなかった 報告すれば ほうこくすれば 報告させる ほうこくさせる 報告される ほうこくされる 報告しろ ほうこくしろ 報告できる ほうこくできる 報告しよう ほうこくしよう				動詞-一般 
\\	「最早(もはや)」のアクセントは?1120	もはや{最早} 仮名 
\\	もはや 漢字 
\\	最早 発音図:							副詞 
\\	「足りる(たりる)」のアクセントは?1121	足りる 仮名 
\\	たりる 漢字 
\\	足りる 発音図:	足りる たりる		足りる たりる 足ります たります 足りて たりて 足りた たりた 足りない たりない 足りなかった たりなかった 足りれば たりれば 足りさせる たりさせる 足りられる たりられる 足りろ たりろ 足りよう たりよう				動詞-一般 
\\	「過程(かてい)」のアクセントは?1122	過程 仮名 
\\	かてい 漢字 
\\	過程 発音図: 助詞: オ	過程 かてい		過程 かてい				名詞-普通名詞-一般 
\\	「要求(ようきゅう)」のアクセントは?1123	要求 仮名 
\\	ようきゅう、よーきゅー 漢字 
\\	要求 発音図: 助詞: オ	要求 ようきゅう		要求 ようきゅう 要求する ようきゅうする 要求します ようきゅうします 要求して ようきゅうして 要求した ようきゅうした 要求しない ようきゅうしない 要求しなかった ようきゅうしなかった 要求すれば ようきゅうすれば 要求させる ようきゅうさせる 要求される ようきゅうされる 要求しろ ようきゅうしろ 要求できる ようきゅうできる 要求しよう ようきゅうしよう				動詞-一般 
\\	「取り組み(とりくみ)《問題に〜》」のアクセントは?1124	取り組み(問題に〜) 仮名 
\\	とりくみ 漢字 
\\	取り組み 発音図: 助詞: オ	取り組み とりくみ		取り組み とりくみ				名詞-普通名詞-一般 
\\	「影(かげ)」のアクセントは?1125	影 仮名 
\\	かげ 漢字 
\\	影 発音図: 助詞: オ	影 かげ		影 かげ				名詞-普通名詞-一般 
\\	「弟(おとうと)」のアクセントは?1126	弟 仮名 
\\	おとうと、おとーと 漢字 
\\	弟 発音図: 助詞: オ	弟 おとうと		弟 おとうと				名詞-普通名詞-一般 
\\	「夫婦(ふうふ)」のアクセントは?1127	夫婦 仮名 
\\	ふーふ、ふうふ 漢字 
\\	夫婦 発音図: 助詞: オ	夫婦 ふうふ		夫婦 ふうふ				名詞-普通名詞-一般 
\\	「グループ(ぐるーぷ)」のアクセントは?1128	グループ 仮名 
\\	グループ、ぐるーぷ 漢字 
\\	発音図: 助詞: オ	グループ ぐるーぷ		グループ ぐるーぷ				名詞-普通名詞-一般 
\\	「膝(ひざ)」のアクセントは?1129	ひざ{膝} 仮名 
\\	ひざ 漢字 
\\	膝 発音図: 助詞: オ	膝 ひざ		膝 ひざ				名詞-普通名詞-一般 
\\	「去る(さる)」のアクセントは?1130	去る 仮名 
\\	さる 漢字 
\\	去る 発音図: 助詞: オ	去る さる		去る さる 去ります さります 去って さって 去った さった 去らない さらない 去らなかった さらなかった 去れば されば 去らせる さらせる 去られる さられる 去れ され 去れる される 去ろう さろう				動詞-一般 
\\	「去る(さる)《〜一日》」のアクセントは?1131	去る(〜一日) 仮名 
\\	さる 漢字 
\\	去る 発音図: 例文: ツいたち	去る さる		去る さる 去ります さります 去って さって 去った さった 去らない さらない 去らなかった さらなかった 去れば されば 去らせる さらせる 去られる さられる 去れ され 去れる される 去ろう さろう				動詞-一般 
\\	「何方(どっち)」のアクセントは?1132	どっち{何方} 仮名 
\\	どっち 漢字 
\\	何方 発音図: 助詞: オ							代名詞 
\\	「一旦(いったん)」のアクセントは?1133	いったん{一旦} 仮名 
\\	いったん 漢字 
\\	一旦 発音図:							副詞 
\\	「バス(バス)
\\	のアクセントは?1134	バス
\\	仮名 
\\	バス、ばす 漢字 
\\	発音図: 助詞: オ							名詞-普通名詞-一般 
\\	「バス(ばす)
\\	のアクセントは?1135	バス
\\	仮名 
\\	バス、ばす 漢字 
\\	発音図: 助詞: オ	バス ばす		バス ばす				名詞-普通名詞-一般 
\\	「バス(バス)
\\	のアクセントは?1136	バス
\\	仮名 
\\	バス、ばす 漢字 
\\	発音図: 助詞: オ							名詞-普通名詞-一般 
\\	「何より(なにより)《それは〜だ》」のアクセントは?1137	何より(それは〜だ) 仮名 
\\	なにより 漢字 
\\	何より 発音図: 発音図:							副詞 
\\	「証拠(しょうこ)」のアクセントは?1138	証拠 仮名 
\\	しょうこ、しょーこ 漢字 
\\	証拠 発音図: 助詞: オ	証拠 しょうこ		証拠 しょうこ				名詞-普通名詞-一般 
\\	「南(みなみ)」のアクセントは?1139	南 仮名 
\\	みなみ 漢字 
\\	南 発音図: 助詞: オ	南 みなみ		南 みなみ				名詞-普通名詞-一般 
\\	「大幅な(おおはばな)」のアクセントは?1140	大幅 仮名 
\\	おーはば、おおはば 漢字 
\\	大幅 発音図: 例文: 利益がニ減った 発音図: 例文: 利益がニ減った	大幅な おおはばな		大幅な おおはばな 大幅です おおはばです 大幅で おおはばで 大幅だった おおはばだった 大幅じゃない おおはばじゃない 大幅じゃなかった おおはばじゃなかった 大幅ならば おおはばならば 大幅だ おおはばだ 大幅に おおはばに				形状詞-一般 
\\	「錐(きり)」のアクセントは?1141	きり{錐} 仮名 
\\	きり 漢字 
\\	錐 発音図: 例文: 胃がデさされたように痛い							助詞-副助詞 
\\	「桐(きり)」のアクセントは?1142	きり{桐} 仮名 
\\	きり 漢字 
\\	桐 発音図: 例文: 娘が生まれたのでオ植えた							助詞-副助詞 
\\	「底(そこ)」のアクセントは?1143	底 仮名 
\\	そこ、てい 漢字 
\\	底 発音図: 助詞: オ 発音図: 助詞: オ	底 そこ		底 そこ				名詞-普通名詞-一般 
\\	「会(かい)」のアクセントは?1144	会 仮名 
\\	かい 漢字 
\\	会 発音図: 助詞: オ	会 かい		会 かい				名詞-普通名詞-一般 
\\	「物語(ものがたり)」のアクセントは?1145	物語 仮名 
\\	ものがたり 漢字 
\\	物語 発音図: 助詞: オ	物語 ものがたり		物語 ものがたり				名詞-普通名詞-一般 
\\	「メンバー(めんばー)」のアクセントは?1146	メンバー 仮名 
\\	メンバー、めんばー 漢字 
\\	発音図: 助詞: オ	メンバー めんばー		メンバー めんばー				名詞-普通名詞-一般 
\\	「維持(いじ)」のアクセントは?1147	維持 仮名 
\\	いじ 漢字 
\\	維持 発音図: 助詞: オ	維持 いじ		維持 いじ 維持する いじする 維持します いじします 維持して いじして 維持した いじした 維持しない いじしない 維持しなかった いじしなかった 維持すれば いじすれば 維持させる いじさせる 維持される いじされる 維持しろ いじしろ 維持できる いじできる 維持しよう いじしよう				動詞-一般 
\\	「猶(なお)」のアクセントは?1148	なお{尚;猶} 仮名 
\\	なお 漢字 
\\	猶、尚 発音図:							副詞 
\\	「外国(がいこく)」のアクセントは?1149	外国 仮名 
\\	がいこく 漢字 
\\	外国 発音図: 助詞: オ	外国 がいこく		外国 がいこく				名詞-普通名詞-一般 
\\	「どんな(どんな)」のアクセントは?1150	どんな 仮名 
\\	どんな 漢字 
\\	発音図:							形状詞-一般 
\\	「出来事(できごと)」のアクセントは?1151	出来事 仮名 
\\	できごと 漢字 
\\	出来事 発音図: 助詞: オ	出来事 できごと		出来事 できごと				名詞-普通名詞-一般 
\\	「チーム(ちーむ)」のアクセントは?1152	チーム 仮名 
\\	チーム、ちーむ 漢字 
\\	発音図: 助詞: オ	チーム ちーむ		チーム ちーむ				名詞-普通名詞-一般 
\\	「適用(てきようする)」のアクセントは?1153	適用 仮名 
\\	てきよう、てきよー 漢字 
\\	適用 発音図: 助詞: オ	適用する てきようする		適用する てきようする 適用します てきようします 適用して てきようして 適用した てきようした 適用しない てきようしない 適用しなかった てきようしなかった 適用すれば てきようすれば 適用させる てきようさせる 適用される てきようされる 適用しろ てきようしろ 適用できる てきようできる 適用しよう てきようしよう				動詞-一般 
\\	「教師(きょうし)」のアクセントは?1154	教師 仮名 
\\	きょうし、きょーし 漢字 
\\	教師 発音図: 助詞: オ	教師 きょうし		教師 きょうし				名詞-普通名詞-一般 
\\	「委員(いいん)」のアクセントは?1155	委員 仮名 
\\	いいん 漢字 
\\	委員 発音図: 助詞: オ	委員 いいん		委員 いいん				名詞-普通名詞-一般 
\\	「びっくり(びっくりする)」のアクセントは?1156	びっくり 仮名 
\\	びっくり 漢字 
\\	発音図:	びっくりする びっくりする		びっくりする びっくりする びっくりします びっくりします びっくりして びっくりして びっくりした びっくりした びっくりしない びっくりしない びっくりしなかった びっくりしなかった びっくりすれば びっくりすれば びっくりさせる びっくりさせる びっくりされる びっくりされる びっくりしろ びっくりしろ びっくりできる びっくりできる びっくりしよう びっくりしよう				動詞-一般 
\\	「細い(ほそい)」のアクセントは?1157	細い 仮名 
\\	ほそい 漢字 
\\	細い 発音図:	細い ほそい		細い ほそい 細いです ほそいです 細くて ほそくて ほそくて 細かった ほそかった ほそかった 細くない ほそくない ほそくない 細くなかった ほそくなかった ほそくなかった 細ければ ほそければ ほそければ 細い ほそい 細く ほそく ほそく				形容詞-一般 
\\	「5日(いつか)」のアクセントは?1158	5日 仮名 
\\	いつか 漢字 
\\	五日、5日 発音図: 助詞: オ	五日 いつか		五日 いつか				名詞-数詞 
\\	「癖(くせ)」のアクセントは?1159	癖 仮名 
\\	くせ、へき 漢字 
\\	癖 発音図: 助詞: オ 発音図: 助詞: オ							名詞-普通名詞-一般 
\\	「階段(かいだん)」のアクセントは?1160	階段 仮名 
\\	かいだん 漢字 
\\	階段 発音図: 助詞: オ	階段 かいだん		階段 かいだん				名詞-普通名詞-一般 
\\	「反映(はんえい)」のアクセントは?1161	反映 仮名 
\\	はんえい 漢字 
\\	反映 発音図: 助詞: オ	反映 はんえい		反映 はんえい 反映する はんえいする 反映します はんえいします 反映して はんえいして 反映した はんえいした 反映しない はんえいしない 反映しなかった はんえいしなかった 反映すれば はんえいすれば 反映させる はんえいさせる 反映される はんえいされる 反映しろ はんえいしろ 反映できる はんえいできる 反映しよう はんえいしよう				動詞-一般 
\\	「脳(のう)」のアクセントは?1162	脳 仮名 
\\	のう、のー 漢字 
\\	脳 発音図: 助詞: オ	脳 のう		脳 のう				名詞-普通名詞-一般 
\\	「鍵(かぎ)」のアクセントは?1163	かぎ{鍵;鉤} 仮名 
\\	かぎ 漢字 
\\	鍵、鉤 発音図: 助詞: オ	鍵 かぎ		鍵 かぎ				名詞-普通名詞-一般 
\\	「雑誌(ざっし)」のアクセントは?1164	雑誌 仮名 
\\	ざっし 漢字 
\\	雑誌 発音図: 助詞: オ	雑誌 ざっし		雑誌 ざっし				名詞-普通名詞-一般 
\\	「著しい(いちじるしい)」のアクセントは?1165	著しい 仮名 
\\	いちじるしい 漢字 
\\	著しい 発音図:	著しい いちじるしい		著しい いちじるしい 著しいです いちじるしいです 著しくて いちじるしくて いちじるしくて 著しかった いちじるしかった いちじるしかった 著しくない いちじるしくない いちじるしくない 著しくなかった いちじるしくなかった いちじるしくなかった 著しければ いちじるしければ いちじるしければ 著しい いちじるしい 著しく いちじるしく いちじるしく				形容詞-一般 
\\	「僅かな(わずかな)」のアクセントは?1166		僅かな わずかな		僅かな わずかな 僅かです わずかです 僅かで わずかで 僅かだった わずかだった 僅かじゃない わずかじゃない 僅かじゃなかった わずかじゃなかった 僅かならば わずかならば 僅かだ わずかだ 僅かに わずかに				副詞 
\\	「僅か(わずか)」のアクセントは?1167		僅か わずか		僅か わずか				副詞 
\\	「繋ぐ(つなぐ)」のアクセントは?1168	つなぐ{繋} 仮名 
\\	つなぐ 漢字 
\\	繋ぐ 発音図:	繋ぐ つなぐ		繋ぐ つなぐ 繋ぎます つなぎます 繋いで つないで 繋いだ つないだ 繋がない つながない 繋がなかった つながなかった 繋げば つなげば 繋がせる つながせる 繋がれる つながれる 繋げ つなげ 繋げる つなげる 繋ごう つなごう				動詞-一般 
\\	「季節(きせつ)」のアクセントは?1169	季節 仮名 
\\	きせつ 漢字 
\\	季節 発音図: 助詞: オ 発音図: 助詞: オ	季節 きせつ きせつ		季節 きせつ きせつ				名詞-普通名詞-一般 
\\	「モデル(もでる)」のアクセントは?1170	モデル 仮名 
\\	モデル、もでる 漢字 
\\	発音図: 助詞: オ 発音図: 助詞: オ	モデル もでる もでる		モデル もでる もでる				名詞-普通名詞-一般 
\\	「互い(たがい)《〜に》」のアクセントは?1171	互い(〜に) 仮名 
\\	たがい 漢字 
\\	互い 発音図: 助詞: ニ	互い たがい		互い たがい				名詞-普通名詞-一般 
\\	「解決(かいけつ)」のアクセントは?1172	解決 仮名 
\\	かいけつ 漢字 
\\	解決 発音図: 助詞: オ	解決 かいけつ		解決 かいけつ 解決する かいけつする 解決します かいけつします 解決して かいけつして 解決した かいけつした 解決しない かいけつしない 解決しなかった かいけつしなかった 解決すれば かいけつすれば 解決させる かいけつさせる 解決される かいけつされる 解決しろ かいけつしろ 解決できる かいけつできる 解決しよう かいけつしよう				動詞-一般 
\\	「迷う(まよう)」のアクセントは?1173	迷う 仮名 
\\	まよう 漢字 
\\	迷う 発音図:	迷う まよう		迷う まよう 迷います まよいます 迷って まよって 迷った まよった 迷わない まよわない 迷わなかった まよわなかった 迷えば まよえば 迷わせる まよわせる 迷われる まよわれる 迷え まよえ 迷える まよえる 迷おう まよおう				動詞-一般 
\\	「真(しん)」のアクセントは?1174	真 仮名 
\\	しん 漢字 
\\	真 発音図: 助詞: オ							名詞-普通名詞-一般 
\\	「奪う(うばう)」のアクセントは?1175	奪う 仮名 
\\	うばう 漢字 
\\	奪う 発音図:	奪う うばう		奪う うばう 奪います うばいます 奪って うばって 奪った うばった 奪わない うばわない 奪わなかった うばわなかった 奪えば うばえば 奪わせる うばわせる 奪われる うばわれる 奪え うばえ 奪える うばえる 奪おう うばおう				動詞-一般 
\\	「実態(じったい)」のアクセントは?1176	実態 仮名 
\\	じったい 漢字 
\\	実態 発音図: 助詞: オ	実態 じったい		実態 じったい				名詞-普通名詞-一般 
\\	「入る(いる)」のアクセントは?1177		入る いる		入る いる 入ります いります 入って いって 入った いった 入らない いらない 入らなかった いらなかった 入れば いれば 入らせる いらせる 入られる いられる 入れ いれ 入られる いられる 入ろう いろう				動詞-一般 
\\	「直る(なおる)」のアクセントは?1178	直る 仮名 
\\	なおる 漢字 
\\	直る 発音図:	直る なおる		直る なおる 直ります なおります 直って なおって 直った なおった 直らない なおらない 直らなかった なおらなかった 直れば なおれば 直らせる なおらせる 直られる なおられる 直れ なおれ 直ろう なおろう				動詞-一般 
\\	「余裕(よゆう)」のアクセントは?1179	余裕 仮名 
\\	よゆー、よゆう 漢字 
\\	余裕 発音図: 助詞: オ	余裕 よゆう		余裕 よゆう				名詞-普通名詞-一般 
\\	「小説(しょうせつ)」のアクセントは?1180	小説 仮名 
\\	しょうせつ、しょーせつ 漢字 
\\	小説 発音図: 助詞: オ	小説 しょうせつ		小説 しょうせつ				名詞-普通名詞-一般 
\\	「別れる(わかれる)」のアクセントは?1181	別れる 仮名 
\\	わかれる 漢字 
\\	別れる 発音図:	別れる わかれる		別れる わかれる 別れます わかれます 別れて わかれて 別れた わかれた 別れない わかれない 別れなかった わかれなかった 別れれば わかれれば 別れさせる わかれさせる 別れられる わかれられる 別れろ わかれろ 別れられる わかれられる 別れよう わかれよう				動詞-一般 
\\	「安心(あんしん)」のアクセントは?1182	安心 仮名 
\\	あんしん 漢字 
\\	安心 発音図: 助詞: オ	安心 あんしん		安心 あんしん 安心する あんしんする 安心します あんしんします 安心して あんしんして 安心した あんしんした 安心しない あんしんしない 安心しなかった あんしんしなかった 安心すれば あんしんすれば 安心させる あんしんさせる 安心される あんしんされる 安心しろ あんしんしろ 安心できる あんしんできる 安心しよう あんしんしよう				動詞-一般 
\\	「怒り(いかり)」のアクセントは?1183	怒り 仮名 
\\	いかり 漢字 
\\	怒り 発音図: 例文: 供物を供えて、神のオ解いた 発音図: 例文: 供物を供えて、神のオ解いた	怒り いかり		怒り いかり				名詞-普通名詞-一般 
\\	「終える(おえる)」のアクセントは?1184	終える 仮名 
\\	おえる 漢字 
\\	終える 発音図:	終える おえる		終える おえる 終えます おえます 終えて おえて 終えた おえた 終えない おえない 終えなかった おえなかった 終えれば おえれば 終えさせる おえさせる 終えられる おえられる 終えろ おえろ 終えられる おえられる 終えよう おえよう				動詞-一般 
\\	「社長(しゃちょう)」のアクセントは?1185	社長 仮名 
\\	しゃちょう、しゃちょー 漢字 
\\	社長 発音図: 助詞: オ	社長 しゃちょう		社長 しゃちょう				名詞-普通名詞-一般 
\\	「作り方(つくりかた)」のアクセントは?1186	作り方 仮名 
\\	つくりかた 漢字 
\\	作り方 発音図: 助詞: オ 発音図: 助詞: オ							名詞-普通名詞-一般 
\\	「自己(じこ)」のアクセントは?1187	自己 仮名 
\\	じこ 漢字 
\\	自己 発音図: 助詞: オ	自己 じこ		自己 じこ				名詞-普通名詞-一般 
\\	「視点(してん)」のアクセントは?1188	視点 仮名 
\\	してん 漢字 
\\	視点 発音図: 助詞: オ	視点 してん		視点 してん				名詞-普通名詞-一般 
\\	「移す(うつす)」のアクセントは?1189	移す 仮名 
\\	うつす 漢字 
\\	移す 発音図:	移す うつす		移す うつす 移します うつします 移して うつして 移した うつした 移さない うつさない 移さなかった うつさなかった 移せば うつせば 移させる うつさせる 移される うつされる 移せ うつせ 移せる うつせる 移そう うつそう				動詞-一般 
\\	「職場(しょくば)」のアクセントは?1190	職場 仮名 
\\	しょくば 漢字 
\\	職場 発音図: 例文: 彼はオ追われた 発音図: 例文: 彼はオ追われた	職場 しょくば		職場 しょくば				名詞-普通名詞-一般 
\\	「妹(いもうと)」のアクセントは?1191	妹 仮名 
\\	いもうと、いもーと 漢字 
\\	妹 発音図: 助詞: オ	妹 いもうと		妹 いもうと				名詞-普通名詞-一般 
\\	「喋る(しゃべる)」のアクセントは?1192	しゃべる{喋} 仮名 
\\	しゃべる 漢字 
\\	喋る 発音図:	しゃべる しゃべる		しゃべる しゃべる しゃべります しゃべります しゃべって しゃべって しゃべった しゃべった しゃべらない しゃべらない しゃべらなかった しゃべらなかった しゃべれば しゃべれば しゃべらせる しゃべらせる しゃべられる しゃべられる しゃべれ しゃべれ しゃべれる しゃべれる しゃべろう しゃべろう				動詞-一般 
\\	「残り(のこり)」のアクセントは?1193	残り 仮名 
\\	のこり 漢字 
\\	残り 発音図: 助詞: オ	残り のこり		残り のこり				名詞-普通名詞-一般 
\\	「市町村(しちょうそん)」のアクセントは?1194	市町村 仮名 
\\	しちょうそん、しちょーそん 漢字 
\\	市町村 発音図: 助詞: オ	市町村 しちょうそん		市町村 しちょうそん				名詞-普通名詞-一般 
\\	「住所(じゅうしょ)」のアクセントは?1195	住所 仮名 
\\	じゅーしょ、じゅうしょ 漢字 
\\	住所 発音図: 助詞: オ	住所 じゅうしょ		住所 じゅうしょ				名詞-普通名詞-一般 
\\	「切れる(きれる)」のアクセントは?1196	切れる 仮名 
\\	きれる 漢字 
\\	切れる 発音図:	切れる きれる		切れる きれる 切れます きれます 切れて きれて 切れた きれた 切れない きれない 切れなかった きれなかった 切れれば きれれば 切れさせる きれさせる 切れられる きれられる 切れろ きれろ 切れよう きれよう				動詞-一般 
\\	「防ぐ(ふせぐ)」のアクセントは?1197	防ぐ 仮名 
\\	ふせぐ 漢字 
\\	防ぐ 発音図:	防ぐ ふせぐ		防ぐ ふせぐ 防ぎます ふせぎます 防いで ふせいで 防いだ ふせいだ 防がない ふせがない 防がなかった ふせがなかった 防げば ふせげば 防がせる ふせがせる 防がれる ふせがれる 防げ ふせげ 防げる ふせげる 防ごう ふせごう				動詞-一般 
\\	「交ぜる(まぜる)」のアクセントは?1198	交ぜる 仮名 
\\	まぜる 漢字 
\\	交ぜる 発音図:							動詞-一般 
\\	「挟む(はさむ)」のアクセントは?1199	挟む{剪;鋏} 仮名 
\\	はさむ 漢字 
\\	挟む、鋏む、剪む 発音図:	挟む はさむ		挟む はさむ 挟みます はさみます 挟んで はさんで 挟んだ はさんだ 挟まない はさまない 挟まなかった はさまなかった 挟めば はさめば 挟ませる はさませる 挟まれる はさまれる 挟め はさめ 挟める はさめる 挟もう はさもう				動詞-一般 
\\	「喜び(よろこび)」のアクセントは?1200	喜び 仮名 
\\	よろこび 漢字 
\\	喜び 発音図: 例文: 生きているオ感じる 発音図: 例文: 生きているオ感じる 発音図: 例文: 生きているオ感じる							名詞-普通名詞-一般 
\\	「確かめる(たしかめる)」のアクセントは?1201	確かめる 仮名 
\\	たしかめる 漢字 
\\	確かめる 発音図:	確かめる たしかめる		確かめる たしかめる 確かめます たしかめます 確かめて たしかめて 確かめた たしかめた 確かめない たしかめない 確かめなかった たしかめなかった 確かめれば たしかめれば 確かめさせる たしかめさせる 確かめられる たしかめられる 確かめろ たしかめろ 確かめられる たしかめられる 確かめよう たしかめよう				動詞-一般 
\\	「天皇(てんのう)」のアクセントは?1202	天皇 仮名 
\\	てんのう、てんのー 漢字 
\\	天皇 発音図: 助詞: オ	天皇 てんのう		天皇 てんのう				名詞-普通名詞-一般 
\\	「折角(せっかく)」のアクセントは?1203	せっかく{折角} 仮名 
\\	せっかく 漢字 
\\	折角 発音図:							副詞 
\\	「未来(みらい)」のアクセントは?1204	未来 仮名 
\\	みらい 漢字 
\\	未来 発音図: 助詞: オ	未来 みらい		未来 みらい				名詞-普通名詞-一般 
\\	「土(つち)」のアクセントは?1205	土 仮名 
\\	つち 漢字 
\\	土 発音図: 助詞: オ	土 つち		土 つち				名詞-普通名詞-一般 
\\	「働き(はたらき)」のアクセントは?1206	働き 仮名 
\\	はたらき 漢字 
\\	働き 発音図: 助詞: オ	働き はたらき		働き はたらき				名詞-普通名詞-一般 
\\	「中身(なかみ)」のアクセントは?1207	中身 仮名 
\\	なかみ 漢字 
\\	中身 発音図: 助詞: オ	中身 なかみ		中身 なかみ				名詞-普通名詞-一般 
\\	「囲む(かこむ)」のアクセントは?1208	囲む 仮名 
\\	かこむ 漢字 
\\	囲む 発音図:	囲む かこむ		囲む かこむ 囲みます かこみます 囲んで かこんで 囲んだ かこんだ 囲まない かこまない 囲まなかった かこまなかった 囲めば かこめば 囲ませる かこませる 囲まれる かこまれる 囲め かこめ 囲める かこめる 囲もう かこもう				動詞-一般 
\\	「じっと(じっと)」のアクセントは?1209	じっと 仮名 
\\	じっと 漢字 
\\	発音図:							副詞 
\\	「連中(れんちゅう)」のアクセントは?1210	連中 仮名 
\\	れんじゅー、れんじゅう、れんちゅー、れんちゅう 漢字 
\\	連中 発音図: 助詞: オ 発音図: 助詞: オ	連中 れんちゅう		連中 れんちゅう				名詞-普通名詞-一般 
\\	「2階(にかい)《お〜》」のアクセントは?1211	2階(お〜) 仮名 
\\	にかい 漢字 
\\	二階、2階 発音図: 助詞: オ							名詞-数詞 
\\	「危険(きけん)」のアクセントは?1212	危険 仮名 
\\	きけん 漢字 
\\	危険 発音図: 助詞: オ	危険 きけん		危険 きけん				名詞-普通名詞-一般 
\\	「危険な(きけんな)」のアクセントは?1213		危険な きけんな		危険な きけんな 危険です きけんです 危険で きけんで 危険だった きけんだった 危険じゃない きけんじゃない 危険じゃなかった きけんじゃなかった 危険ならば きけんならば 危険だ きけんだ 危険に きけんに				名詞-普通名詞-一般 
\\	「要因(よういん)」のアクセントは?1214	要因 仮名 
\\	よういん、よーいん 漢字 
\\	要因 発音図: 助詞: オ	要因 よういん		要因 よういん				名詞-普通名詞-一般 
\\	「勉強(べんきょう)」のアクセントは?1215	勉強 仮名 
\\	べんきょう、べんきょー 漢字 
\\	勉強 発音図: 助詞: オ	勉強 べんきょう		勉強 べんきょう 勉強する べんきょうする 勉強します べんきょうします 勉強して べんきょうして 勉強した べんきょうした 勉強しない べんきょうしない 勉強しなかった べんきょうしなかった 勉強すれば べんきょうすれば 勉強させる べんきょうさせる 勉強される べんきょうされる 勉強しろ べんきょうしろ 勉強できる べんきょうできる 勉強しよう べんきょうしよう				動詞-一般 
\\	「属する(ぞくする)」のアクセントは?1216	属する 仮名 
\\	ぞくする 漢字 
\\	属する 発音図:	属する ぞくする		属する ぞくする 属します ぞくします 属して ぞくして 属した ぞくした 属さない ぞくさない 属さなかった ぞくさなかった 属せば ぞくせば 属させる ぞくさせる 属される ぞくされる 属せ ぞくせ 属せる ぞくせる 属そう ぞくそう				動詞-一般 
\\	「組む(くむ)」のアクセントは?1217	組む 仮名 
\\	くむ 漢字 
\\	組む 発音図:	組む くむ		組む くむ 組みます くみます 組んで くんで 組んだ くんだ 組まない くまない 組まなかった くまなかった 組めば くめば 組ませる くませる 組まれる くまれる 組め くめ 組める くめる 組もう くもう				動詞-一般 
\\	「生み出す(うみだす)」のアクセントは?1218	生み出す 仮名 
\\	うみだす 漢字 
\\	生み出す 発音図:	生み出す うみだす		生み出す うみだす 生み出します うみだします 生み出して うみだして 生み出した うみだした 生み出さない うみださない 生み出さなかった うみださなかった 生み出せば うみだせば 生み出させる うみださせる 生み出される うみだされる 生み出せ うみだせ 生み出せる うみだせる 生み出そう うみだそう				動詞-一般 
\\	「ゲーム(げーむ)」のアクセントは?1219	ゲーム 仮名 
\\	ゲーム、げーむ 漢字 
\\	発音図: 助詞: オ	ゲーム げーむ		ゲーム げーむ				名詞-普通名詞-一般 
\\	「主に(おもに)」のアクセントは?1220	主に 仮名 
\\	おもに 漢字 
\\	主に 発音図:							副詞 
\\	「無視(むし)」のアクセントは?1221	無視 仮名 
\\	むし 漢字 
\\	無視 発音図: 例文: 私の発言はサれた	無視 むし		無視 むし 無視する むしする 無視します むしします 無視して むしして 無視した むしした 無視しない むししない 無視しなかった むししなかった 無視すれば むしすれば 無視させる むしさせる 無視される むしされる 無視しろ むししろ 無視できる むしできる 無視しよう むししよう				動詞-一般 
\\	「はっきり(はっきりする)」のアクセントは?1222	はっきり 仮名 
\\	はっきり 漢字 
\\	発音図:	はっきりする はっきりする		はっきりする はっきりする はっきりします はっきりします はっきりして はっきりして はっきりした はっきりした はっきりしない はっきりしない はっきりしなかった はっきりしなかった はっきりすれば はっきりすれば はっきりさせる はっきりさせる はっきりされる はっきりされる はっきりしろ はっきりしろ はっきりできる はっきりできる はっきりしよう はっきりしよう				動詞-一般 
\\	「込める(こめる)」のアクセントは?1223	込める 仮名 
\\	こめる 漢字 
\\	込める 発音図:	込める こめる		込める こめる 込めます こめます 込めて こめて 込めた こめた 込めない こめない 込めなかった こめなかった 込めれば こめれば 込めさせる こめさせる 込められる こめられる 込めろ こめろ 込められる こめられる 込めよう こめよう				動詞-一般 
\\	「生命(せいめい)」のアクセントは?1224	生命 仮名 
\\	せいめい 漢字 
\\	生命 発音図: 助詞: オ 発音図: 助詞: オ	生命 せいめい		生命 せいめい				名詞-普通名詞-一般 
\\	「以前(いぜん)」のアクセントは?1225	以前 仮名 
\\	いぜん 漢字 
\\	以前 発音図: 助詞: オ	以前 いぜん		以前 いぜん				副詞 
\\	「少女(しょうじょ)」のアクセントは?1226	少女 仮名 
\\	しょうじょ、しょーじょ 漢字 
\\	少女 発音図: 助詞: オ	少女 しょうじょ		少女 しょうじょ				名詞-普通名詞-一般 
\\	「資金(しきん)」のアクセントは?1227	資金 仮名 
\\	しきん 漢字 
\\	資金 発音図: 助詞: オ 発音図: 助詞: オ	資金 しきん しきん		資金 しきん しきん				名詞-普通名詞-一般 
\\	「直す(なおす)」のアクセントは?1228	直す 仮名 
\\	なおす 漢字 
\\	直す 発音図:	直す なおす		直す なおす 直します なおします 直して なおして 直した なおした 直さない なおさない 直さなかった なおさなかった 直せば なおせば 直させる なおさせる 直される なおされる 直せ なおせ 直せる なおせる 直そう なおそう				動詞-一般 
\\	「不可能な(ふかのうな)」のアクセントは?1229		不可能な ふかのうな		不可能な ふかのうな 不可能です ふかのうです 不可能で ふかのうで 不可能だった ふかのうだった 不可能じゃない ふかのうじゃない 不可能じゃなかった ふかのうじゃなかった 不可能ならば ふかのうならば 不可能だ ふかのうだ 不可能に ふかのうに				形状詞-一般 
\\	「不可能(ふかのう)」のアクセントは?1230	不可能 仮名 
\\	ふかのう、ふかのー 漢字 
\\	不可能 発音図: 助詞: オ	不可能 ふかのう		不可能 ふかのう				形状詞-一般 
\\	「まさか(まさか)」のアクセントは?1231	まさか 仮名 
\\	まさか 漢字 
\\	発音図:							副詞 
\\	「消費者(しょうひしゃ)」のアクセントは?1232	消費者 仮名 
\\	しょうひしゃ、しょーひしゃ 漢字 
\\	消費者 発音図: 助詞: オ	消費者 しょうひしゃ		消費者 しょうひしゃ				名詞-普通名詞-一般 
\\	「認識(にんしき)」のアクセントは?1233	認識 仮名 
\\	にんしき 漢字 
\\	認識 発音図: 助詞: オ	認識 にんしき		認識 にんしき 認識する にんしきする 認識します にんしきします 認識して にんしきして 認識した にんしきした 認識しない にんしきしない 認識しなかった にんしきしなかった 認識すれば にんしきすれば 認識させる にんしきさせる 認識される にんしきされる 認識しろ にんしきしろ 認識できる にんしきできる 認識しよう にんしきしよう				動詞-一般 
\\	「ブログ(ぶろぐ)」のアクセントは?1234		ブログ ぶろぐ		ブログ ぶろぐ				名詞-普通名詞-一般 
\\	「加わる(くわわる)」のアクセントは?1235	加わる 仮名 
\\	くわわる 漢字 
\\	加わる 発音図: 発音図:	加わる くわわる くわわる		加わる くわわる くわわる 加わります くわわります 加わって くわわって くわわって 加わった くわわった くわわった 加わらない くわわらない くわわらない 加わらなかった くわわらなかった くわわらなかった 加われば くわわれば くわわれば 加わらせる くわわらせる くわわらせる 加わられる くわわられる くわわられる 加われ くわわれ くわわれ 加われる くわわれる くわわれる 加わろう くわわろう				動詞-一般 
\\	「単純(たんじゅん)」のアクセントは?1236	単純 仮名 
\\	たんじゅん 漢字 
\\	単純 発音図: 助詞: オ	単純 たんじゅん		単純 たんじゅん				形状詞-一般 
\\	「早速(さっそく)」のアクセントは?1237	早速 仮名 
\\	さっそく 漢字 
\\	早速 発音図: 助詞: オ							副詞 
\\	「単純な(たんじゅんな)」のアクセントは?1238		単純な たんじゅんな		単純な たんじゅんな 単純です たんじゅんです 単純で たんじゅんで 単純だった たんじゅんだった 単純じゃない たんじゅんじゃない 単純じゃなかった たんじゅんじゃなかった 単純ならば たんじゅんならば 単純だ たんじゅんだ 単純に たんじゅんに				形状詞-一般 
\\	「見掛ける(みかける)」のアクセントは?1239		見掛ける みかける みかける		見掛ける みかける みかける 見掛けます みかけます 見掛けて みかけて みかけて 見掛けた みかけた みかけた 見掛けない みかけない みかけない 見掛けなかった みかけなかった みかけなかった 見掛ければ みかければ みかければ 見掛けさせる みかけさせる みかけさせる 見掛けられる みかけられる みかけられる 見掛けろ みかけろ みかけろ 見掛けられる みかけられる みかけられる 見掛けよう みかけよう				動詞-一般 
\\	「負う(おう)」のアクセントは?1240	負う 仮名 
\\	おう 漢字 
\\	負う 発音図:	負う おう		負う おう 負います おいます 負って おって 負った おった 負わない おわない 負わなかった おわなかった 負えば おえば 負わせる おわせる 負われる おわれる 負え おえ 負える おえる 負おう おおう				動詞-一般 
\\	「大臣(だいじん)」のアクセントは?1242	大臣 仮名 
\\	おとど、だいじん 漢字 
\\	大臣 発音図: 助詞: オ 発音図: 助詞: オ 発音図: 助詞: オ	大臣 だいじん		大臣 だいじん				名詞-普通名詞-一般 
\\	「たっぷり(たっぷり)」のアクセントは?1243	たっぷり 仮名 
\\	たっぷり 漢字 
\\	発音図:	たっぷり たっぷり		たっぷり たっぷり				副詞 
\\	「たっぷりな(たっぷりな)」のアクセントは?1244		たっぷりな たっぷりな		たっぷりな たっぷりな たっぷりです たっぷりです たっぷりで たっぷりで たっぷりだった たっぷりだった たっぷりじゃない たっぷりじゃない たっぷりじゃなかった たっぷりじゃなかった たっぷりならば たっぷりならば たっぷりだ たっぷりだ たっぷりに たっぷりに				副詞 
\\	「詩(し)」のアクセントは?1245	詩 仮名 
\\	し 漢字 
\\	詩 発音図: 例文: 美しいオつくる	詩 し		詩 し				名詞-普通名詞-一般 
\\	「増す(ます)」のアクセントは?1246	増す 仮名 
\\	ます 漢字 
\\	増す 発音図: 例文: 水かさがノが確認された	増す ます		増す ます 増します まします 増して まして 増した ました 増さない まさない 増さなかった まさなかった 増せば ませば 増させる まさせる 増される まされる 増せ ませ 増そう まそう				動詞-一般 
\\	「自転車(じてんしゃ)」のアクセントは?1247	自転車 仮名 
\\	じてんしゃ 漢字 
\\	自転車 発音図: 助詞: オ 発音図: 助詞: オ	自転車 じてんしゃ		自転車 じてんしゃ				名詞-普通名詞-一般 
\\	「労働者(ろうどうしゃ)」のアクセントは?1248	労働者 仮名 
\\	ろうどうしゃ、ろーどーしゃ 漢字 
\\	労働者 発音図: 助詞: オ							名詞-普通名詞-一般 
\\	「1時間(いちじかん)」のアクセントは?1249	1時間 仮名 
\\	いちじかん 漢字 
\\	一時間、1時間 発音図: 助詞: オ	一時間 いちじかん		一時間 いちじかん				名詞-数詞 
\\	「ふと(ふと)《〜考える》」のアクセントは?1250	ふと(〜考える) 仮名 
\\	ふと 漢字 
\\	発音図: 例文: カんがえる 発音図: 例文: カんがえる							副詞 
\\	「断る(ことわる)」のアクセントは?1251	断る 仮名 
\\	ことわる 漢字 
\\	断る 発音図:	断る ことわる		断る ことわる 断ります ことわります 断って ことわって 断った ことわった 断らない ことわらない 断らなかった ことわらなかった 断れば ことわれば 断らせる ことわらせる 断られる ことわられる 断れ ことわれ 断れる ことわれる 断ろう ことわろう				動詞-一般 
\\	「命ずる(めいずる)」のアクセントは?1252	命ずる 仮名 
\\	めいずる 漢字 
\\	命ずる 発音図: 発音図:							動詞-一般 
\\	「授業(じゅぎょう)」のアクセントは?1253	授業 仮名 
\\	じゅぎょう、じゅぎょー 漢字 
\\	授業 発音図: 助詞: オ	授業 じゅぎょう		授業 じゅぎょう				名詞-普通名詞-一般 
\\	「3時(さんじ)」のアクセントは?1254	3時 仮名 
\\	さんじ 漢字 
\\	三時、3時 発音図: 助詞: オ	三時 さんじ		三時 さんじ				名詞-数詞 
\\	「噂(うわさ)」のアクセントは?1255	うわさ{噂} 仮名 
\\	うわさ 漢字 
\\	噂 発音図: 助詞: オ	噂 うわさ		噂 うわさ				名詞-普通名詞-一般 
\\	「吸う(すう)」のアクセントは?1256	吸う 仮名 
\\	すう 漢字 
\\	吸う 発音図:	吸う すう		吸う すう 吸います すいます 吸って すって 吸った すった 吸わない すわない 吸わなかった すわなかった 吸えば すえば 吸わせる すわせる 吸われる すわれる 吸え すえ 吸える すえる 吸おう すおう				動詞-一般 
\\	「倒れる(たおれる)」のアクセントは?1257	倒れる 仮名 
\\	たおれる 漢字 
\\	倒れる 発音図:	倒れる たおれる		倒れる たおれる 倒れます たおれます 倒れて たおれて 倒れた たおれた 倒れない たおれない 倒れなかった たおれなかった 倒れれば たおれれば 倒れさせる たおれさせる 倒れられる たおれられる 倒れろ たおれろ 倒れよう たおれよう				動詞-一般 
\\	「発する(はっする)」のアクセントは?1258	発する 仮名 
\\	はっする 漢字 
\\	発する 発音図:							動詞-一般 
\\	「我が家(わがや)」のアクセントは?1259		我が家 わがや		我が家 わがや				名詞-普通名詞-一般 
\\	「飾る(かざる)」のアクセントは?1260	飾る 仮名 
\\	かざる 漢字 
\\	飾る 発音図:	飾る かざる		飾る かざる 飾ります かざります 飾って かざって 飾った かざった 飾らない かざらない 飾らなかった かざらなかった 飾れば かざれば 飾らせる かざらせる 飾られる かざられる 飾れ かざれ 飾れる かざれる 飾ろう かざろう				動詞-一般 
\\	「紙(かみ)」のアクセントは?1261	紙 仮名 
\\	かみ 漢字 
\\	紙 発音図: 助詞: オ	紙 かみ		紙 かみ				名詞-普通名詞-一般 
\\	「現象(げんしょう)」のアクセントは?1262	現象 仮名 
\\	げんしょう、げんしょー 漢字 
\\	現象 発音図: 助詞: オ	現象 げんしょう		現象 げんしょう				名詞-普通名詞-一般 
\\	「休む(やすむ)」のアクセントは?1263	休む 仮名 
\\	やすむ 漢字 
\\	休む 発音図:	休む やすむ		休む やすむ 休みます やすみます 休んで やすんで 休んだ やすんだ 休まない やすまない 休まなかった やすまなかった 休めば やすめば 休ませる やすませる 休まれる やすまれる 休め やすめ 休める やすめる 休もう やすもう				動詞-一般 
\\	「寂しい(さびしい)」のアクセントは?1264	寂しい 仮名 
\\	さびしい 漢字 
\\	寂しい 発音図:	寂しい さびしい		寂しい さびしい 寂しいです さびしいです 寂しくて さびしくて さびしくて 寂しかった さびしかった さびしかった 寂しくない さびしくない さびしくない 寂しくなかった さびしくなかった さびしくなかった 寂しければ さびしければ さびしければ 寂しい さびしい 寂しく さびしく さびしく				形容詞-一般 
\\	「考慮(こうりょ)」のアクセントは?1265	考慮 仮名 
\\	こうりょ、こーりょ 漢字 
\\	考慮 発音図: 助詞: オ	考慮 こうりょ		考慮 こうりょ 考慮する こうりょする 考慮します こうりょします 考慮して こうりょして 考慮した こうりょした 考慮しない こうりょしない 考慮しなかった こうりょしなかった 考慮すれば こうりょすれば 考慮させる こうりょさせる 考慮される こうりょされる 考慮しろ こうりょしろ 考慮できる こうりょできる 考慮しよう こうりょしよう				動詞-一般 
\end{CJK}
\end{document}