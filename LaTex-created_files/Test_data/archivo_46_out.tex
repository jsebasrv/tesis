\documentclass[8pt]{extreport} 
\usepackage{hyperref}
\usepackage{CJKutf8}
\begin{document}
\begin{CJK}{UTF8}{min}
\\	それ	それ	それ	
\\	それはとってもいい話だ。	それはとってもいい 話[はなし]だ。	それ は とっても いい はなし だ	
\\	はとってもいい 話[はなし]だ。		
\\	一つ	一[ひと]つ	ひとつ	
\\	それを一つください。	それを 一[ひと]つください。	それ を ひとつ ください	
\\	それを
\\	ください。		
\\	一	一[いち]	いち	
\\	一から始めましょう。	一[いち]から 始[はじ]めましょう。	いち から はじめましょう	
\\	から 始[はじ]めましょう。		
\\	二	二[に]	に	
\\	その人には二回会った。	その 人[ひと]には 二[に] 回会[かい あ]った。	その ひと に は に かい あった	
\\	その 人[ひと]には
\\	回会[かい あ]った。		
\\	二つ	二[ふた]つ	ふたつ	
\\	ソフトクリームを二つください。	ソフトクリームを 二[ふた]つください。	そふとくりーむ を ふたつ ください	
\\	ソフトクリームを
\\	ください。		
\\	三	三[さん]	さん	
\\	彼女は三人の子供の母親だ。	彼女[かのじょ]は 三[さん] 人[にん]の 子供[こども]の 母親[ははおや]だ。	かのじょ は さんにん の こども の ははおや だ	
\\	彼女[かのじょ]は
\\	人[にん]の 子供[こども]の 母親[ははおや]だ。		
\\	三つ	三[みっ]つ	みっつ	
\\	コップを三つ買いました。	コップを 三[みっ]つ 買[か]いました。	こっぷ を みっつ かいました	
\\	コップを
\\	買[か]いました。		
\\	四つ	四[よっ]つ	よっつ	
\\	私は腕時計を四つ持っています。	私[わたし]は 腕時計[うでどけい]を 四[よっ]つ 持[も]っています。	わたし は うでどけい を よっつ もって います	
\\	私[わたし]は 腕時計[うでどけい]を
\\	持[も]っています。		
\\	四	四[し]	し	
\\	四月に大学に入学しました。	四[し] 月[がつ]に 大学[だいがく]に 入学[にゅうがく]しました。	しがつ に だいがく に にゅうがく しました	
\\	月[がつ]に 大学[だいがく]に 入学[にゅうがく]しました。		
\\	これ	これ	これ	
\\	これをください。	これをください。	これをください。	
\\	をください。		
\\	四	四[よん]	よん	
\\	ハワイは四回目です。	ハワイは 四[よん] 回目[かいめ]です。	はわい は よんかいめ です	
\\	ハワイは
\\	回目[かいめ]です。		
\\	五つ	五[いつ]つ	いつつ	
\\	桃を五つください。	桃[もも]を 五[いつ]つください。	もも を いつつ ください	
\\	桃[もも]を
\\	ください。		
\\	五	五[ご]	ご	
\\	五人で旅行に行きました。	五[ご] 人[にん]で 旅行[りょこう]に 行[い]きました。	ごにん で りょこう に いきました	
\\	人[にん]で 旅行[りょこう]に 行[い]きました。		
\\	六つ	六[むっ]つ	むっつ	
\\	息子は六つになりました。	息子[むすこ]は 六[むっ]つになりました。	むすこ は むっつ に なりました	
\\	息子[むすこ]は
\\	になりました。		
\\	六	六[ろく]	ろく	
\\	彼には子供が六人います。	彼[かれ]には 子供[こども]が 六[ろく] 人[にん]います。	かれ に は こども が ろくにん います	
\\	彼[かれ]には 子供[こども]が
\\	人[にん]います。		
\\	七	七[なな]	なな	
\\	バナナが七本あります。	バナナが 七[なな] 本[ほん]あります。	ばなな が ななほん あります	
\\	バナナが
\\	本[ほん]あります。		
\\	七つ	七[なな]つ	ななつ	
\\	この子は今年七つになります。	この 子[こ]は 今年[ことし] 七[なな]つになります。	この こ は ことし ななつ に なります	
\\	この 子[こ]は 今年[ことし]
\\	になります。		
\\	八	八[はち]	はち	
\\	りんごを八個ください。	りんごを 八[はち] 個[こ]ください。	りんご を はちこ ください	
\\	りんごを
\\	個[こ]ください。		
\\	八つ	八[やっ]つ	やっつ	
\\	あの家には時計が八つあります。	あの 家[いえ]には 時計[とけい]が 八[やっ]つあります。	あの いえ に は とけい が やっつ あります	
\\	あの 家[いえ]には 時計[とけい]が
\\	あります。		
\\	なる	なる	なる	
\\	彼は医者になりました。	彼[かれ]は 医者[いしゃ]になりました。	かれ は いしゃ に なりました	
\\	彼[かれ]は 医者[いしゃ]に
\\	九	九[きゅう]	きゅう	
\\	野球は九人で1チームです。	野球[やきゅう]は 九[きゅう] 人[にん]で 1[ひと]チームです。	やきゅう は きゅうにん で ひとちーむ です	
\\	野球[やきゅう]は
\\	人[にん]で 1[ひと]チームです。		
\\	九つ	九[ここの]つ	ここのつ	
\\	娘は明日九つになります。	娘[むすめ]は 明日[あす] 九[ここの]つになります。	むすめ は あす ここのつ に なります	
\\	娘[むすめ]は 明日[あす]
\\	になります。		
\\	十	十[じゅう]	じゅう	
\\	その子は指で十数えました。	その 子[こ]は 指[ゆび]で 十[じゅう] 数[かぞ]えました。	その こ は ゆび で じゅう かぞえました	
\\	その 子[こ]は 指[ゆび]で
\\	数[かぞ]えました。		
\\	百	百[ひゃく]	ひゃく	
\\	私の祖母は百才です。	私[わたし]の 祖母[そぼ]は 百[ひゃく] 才[さい]です。	わたし の そぼ は ひゃくさい です	
\\	私[わたし]の 祖母[そぼ]は
\\	才[さい]です。		
\\	千	千[せん]	せん	
\\	千円貸してください。	千[せん] 円貸[えん か]してください。	せんえん かして ください	
\\	円貸[えん か]してください。		
\\	万	万[まん]	まん	
\\	この靴は1万円です。	この 靴[くつ]は 1[いち] 万[まん] 円[えん]です。	この くつ は いちまんえん です	
\\	この 靴[くつ]は 1[いち]
\\	円[えん]です。		
\\	円	円[えん]	えん	
\\	そこに大きな円を描いて。	そこに 大[おお]きな 円[えん]を 描[か]いて。	そこ に おおき な えん を かいて	
\\	そこに 大[おお]きな
\\	を 描[か]いて。		
\\	円	円[えん]	えん	
\\	カレーライスは700円です。	カレーライスは 700[ななひゃく] 円[えん]です。	かれーらいす は ななひゃくえん です	
\\	カレーライスは 700[ななひゃく]
\\	です。		
\\	時	時[とき]	とき	
\\	時の経つのは早い。	時[とき]の 経[た]つのは 早[はや]い。	とき の たつ の は はやい	
\\	の 経[た]つのは 早[はや]い。		
\\	する	する	する	
\\	友達と一緒に宿題をした。	友達[ともだち]と 一緒[いっしょ]に 宿題[しゅくだい]をした。	ともだち と いっしょ に しゅくだい を した	
\\	友達[ともだち]と 一緒[いっしょ]に 宿題[しゅくだい]を
\\	時々	時々[ときどき]	ときどき	
\\	彼は時々遅刻します。	彼[かれ]は 時々[ときどき] 遅刻[ちこく]します。	かれ は ときどき ちこく します	
\\	彼[かれ]は
\\	遅刻[ちこく]します。		
\\	日	日[にち]	にち	
\\	私たちは先月11日に結婚しました。	私[わたし]たちは 先月11[せんげつ じゅういち] 日[にち]に 結婚[けっこん]しました。	わたしたち は せんげつ じゅういち にち に けっこん しました	
\\	私[わたし]たちは 先月11[せんげつ じゅういち]
\\	に 結婚[けっこん]しました。		
\\	六日	六日[むいか]	むいか	
\\	六日前に日本に帰ってきました。	六日[むいか] 前[まえ]に 日本[にほん]に 帰[かえ]ってきました。	むいかまえ に にほん に かえって きました	
\\	前[まえ]に 日本[にほん]に 帰[かえ]ってきました。		
\\	三日	三日[みっか]	みっか	
\\	手紙が届くのに三日かかりました。	手紙[てがみ]が 届[とど]くのに 三日[みっか]かかりました。	てがみ が とどく の に みっか かかりました	
\\	手紙[てがみ]が 届[とど]くのに
\\	かかりました。		
\\	五日	五日[いつか]	いつか	
\\	五月五日は祝日です	五月[ごがつ] 五日[いつか]は 祝日[しゅくじつ]です	ごがつ いつか は しゅくじつ です	
\\	五月[ごがつ]
\\	は 祝日[しゅくじつ]です		
\\	八日	八日[ようか]	ようか	
\\	八日からイギリスに行きます。	八日[ようか]からイギリスに 行[い]きます。	ようか から いぎりす に いきます	
\\	からイギリスに 行[い]きます。		
\\	二十日	二十日[はつか]	はつか	
\\	来月の二十日は弟の誕生日です。	来月[らいげつ]の 二十日[はつか]は 弟[おとうと]の 誕生日[たんじょうび]です。	らいげつ の はつか は おとうと の たんじょうび です	
\\	来月[らいげつ]の
\\	は 弟[おとうと]の 誕生日[たんじょうび]です。		
\\	二日	二日[ふつか]	ふつか	
\\	私は二日待った。	私[わたし]は 二日[ふつか] 待[ま]った。	わたし は ふつか まった	
\\	私[わたし]は
\\	待[ま]った。		
\\	九日	九日[ここのか]	ここのか	
\\	九日に荷物が届きます。	九日[ここのか]に 荷物[にもつ]が 届[とど]きます。	ここのか に にもつ が とどきます	
\\	に 荷物[にもつ]が 届[とど]きます。		
\\	ところ	ところ	ところ	
\\	私は友達のところに泊まった。	私[わたし]は 友達[ともだち]のところに 泊[と]まった。	わたし は ともだち の ところ に とまった	
\\	私[わたし]は 友達[ともだち]の
\\	に 泊[と]まった。		
\\	一日	一日[ついたち]	ついたち	
\\	来月の一日は空いていますか。	来月[らいげつ]の 一日[ついたち]は 空[あ]いていますか。	らいげつ の ついたち は あいています か	
\\	来月[らいげつ]の
\\	は 空[あ]いていますか。		
\\	十日	十日[とおか]	とおか	
\\	十日後に帰ります。	十日[とおか] 後[ご]に 帰[かえ]ります。	とおかご に かえります	
\\	後[ご]に 帰[かえ]ります。		
\\	七日	七日[なのか]	なのか	
\\	先月の七日に孫が生まれました。	先月[せんげつ]の 七日[なのか]に 孫[まご]が 生[う]まれました。	せんげつ の なのか に まご が うまれました	
\\	先月[せんげつ]の
\\	に 孫[まご]が 生[う]まれました。		
\\	四日	四日[よっか]	よっか	
\\	新学期は来月の四日からです。	新学期[しんがっき]は 来月[らいげつ]の 四日[よっか]からです。	しんがっき は らいげつ の よっか から です	
\\	新学期[しんがっき]は 来月[らいげつ]の
\\	からです。		
\\	月	月[つき]	つき	
\\	今夜は月がとてもきれいです。	今夜[こんや]は 月[つき]がとてもきれいです。	こんや は つき が とても きれい です	
\\	今夜[こんや]は
\\	がとてもきれいです。		
\\	水	水[みず]	みず	
\\	水を一杯ください。	水[みず]を 一杯[いっぱい]ください。	みず を いっぱい ください	
\\	を 一杯[いっぱい]ください。		
\\	金	金[かね]	かね	
\\	これはかなり金がかかった。	これはかなり 金[かね]がかかった。	これ は かなり かね が かかった	
\\	これはかなり
\\	がかかった。		
\\	日曜日	日曜日[にちようび]	にちようび	
\\	日曜日は海に行きました。	日曜日[にちようび]は 海[うみ]に 行[い]きました。	にちようび は うみ に いきました	
\\	は 海[うみ]に 行[い]きました。		
\\	土曜日	土曜日[どようび]	どようび	
\\	土曜日の夜はクラブに行きます。	土曜日[どようび]の 夜[よる]はクラブに 行[い]きます。	どようび の よる は くらぶ に いきます	
\\	の 夜[よる]はクラブに 行[い]きます。		
\\	やる	やる	やる	
\\	一緒に宿題をやろう。	一緒[いっしょ]に 宿題[しゅくだい]をやろう。	いっしょ に しゅくだい を やろう	
\\	一緒[いっしょ]に 宿題[しゅくだい]を
\\	金曜日	金曜日[きんようび]	きんようび	
\\	金曜日の夜は友達と出かけます。	金曜日[きんようび]の 夜[よる]は 友達[ともだち]と 出[で]かけます。	きんようび の よる は ともだち と でかけます	
\\	の 夜[よる]は 友達[ともだち]と 出[で]かけます。		
\\	月曜日	月曜日[げつようび]	げつようび	
\\	月曜日に会いましょう。	月曜日[げつようび]に 会[あ]いましょう。	げつようび に あいましょう	
\\	に 会[あ]いましょう。		
\\	木曜日	木曜日[もくようび]	もくようび	
\\	木曜日は仕事が休みです。	木曜日[もくようび]は 仕事[しごと]が 休[やす]みです。	もくようび は しごと が やすみ です	
\\	は 仕事[しごと]が 休[やす]みです。		
\\	曜日	曜日[ようび]	ようび	
\\	曜日を間違えました。	曜日[ようび]を 間違[まちが]えました。	ようび を まちがえました	
\\	を 間違[まちが]えました。		
\\	火曜日	火曜日[かようび]	かようび	
\\	火曜日に会議があります。	火曜日[かようび]に 会議[かいぎ]があります。	かようび に かいぎ が あります	
\\	に 会議[かいぎ]があります。		
\\	水曜日	水曜日[すいようび]	すいようび	
\\	水曜日はバイトがあります。	水曜日[すいようび]はバイトがあります。	すいようび は ばいと が あります	
\\	はバイトがあります。		
\\	週	週[しゅう]	しゅう	
\\	その次の週は空いてますか。	その 次[つぎ]の 週[しゅう]は 空[あ]いてますか。	その つぎ の しゅう は あいてます か	
\\	その 次[つぎ]の
\\	は 空[あ]いてますか。		
\\	年	年[とし]	とし	
\\	新しい年が始まりました。	新[あたら]しい 年[とし]が 始[はじ]まりました。	あたらしい とし が はじまりました	
\\	新[あたら]しい
\\	が 始[はじ]まりました。		
\\	分かる	分[わ]かる	わかる	
\\	質問の意味は分かりましたか。	質問[しつもん]の 意味[いみ]は 分[わ]かりましたか。	しつもん の いみ は わかりました か	
\\	質問[しつもん]の 意味[いみ]は
\\	か。		
\\	そう	そう	そう	
\\	私もそう思います。	私[わたし]もそう 思[おも]います。	わたし も そう おもいます	
\\	私[わたし]も
\\	思[おも]います。		
\\	何	何[なに]	なに	
\\	夕食には何を食べたいですか。	夕食[ゆうしょく]には 何[なに]を 食[た]べたいですか。	ゆうしょく に は なに を たべたい です か	
\\	夕食[ゆうしょく]には
\\	を 食[た]べたいですか。		
\\	先	先[さき]	さき	
\\	お先にどうぞ。	お 先[さき]にどうぞ。	おさきに どうぞ	
\\	お
\\	にどうぞ。		
\\	今年	今年[ことし]	ことし	
\\	今年はイタリアに旅行したい。	今年[ことし]はイタリアに 旅行[りょこう]したい。	ことし は いたりあ に りょこう したい	
\\	はイタリアに 旅行[りょこう]したい。		
\\	今	今[いま]	いま	
\\	彼は今、勉強しています。	彼[かれ]は 今[いま]、 勉強[べんきょう]しています。	かれ は いま べんきょう して います	
\\	彼[かれ]は
\\	、 勉強[べんきょう]しています。		
\\	今日	今日[きょう]	きょう	
\\	今日は仕事がありません。	今日[きょう]は 仕事[しごと]がありません。	きょう は しごと が ありません	
\\	は 仕事[しごと]がありません。		
\\	今月	今月[こんげつ]	こんげつ	
\\	今月はとても忙しい。	今月[こんげつ]はとても 忙[いそが]しい。	こんげつ は とても いそがしい	
\\	はとても 忙[いそが]しい。		
\\	今週	今週[こんしゅう]	こんしゅう	
\\	今週は日本語のテストがあります。	今週[こんしゅう]は 日本語[にほんご]のテストがあります。	こんしゅう は にほんご の てすと が あります	
\\	は 日本語[にほんご]のテストがあります。		
\\	来る	来[く]る	くる	
\\	彼は昼過ぎに来ます。	彼[かれ]は 昼過[ひるす]ぎに 来[き]ます。	かれ は ひるすぎ に きます	
\\	彼[かれ]は 昼過[ひるす]ぎに
\\	来年	来年[らいねん]	らいねん	
\\	来年一緒に旅行しましょう。	来年[らいねん] 一緒[いっしょ]に 旅行[りょこう]しましょう。	らいねん いっしょ に りょこう しましょう	
\\	一緒[いっしょ]に 旅行[りょこう]しましょう。		
\\	もう	もう	もう	
\\	彼はもう帰りました。	彼[かれ]はもう 帰[かえ]りました。	かれ は もう かえりました 。	
\\	彼[かれ]は
\\	帰[かえ]りました。		
\\	行く	行[い]く	いく	
\\	日曜日は図書館に行きます。	日曜日[にちようび]は 図書館[としょかん]に 行[い]きます。	にちようび は としょかん に いきます	
\\	日曜日[にちようび]は 図書館[としょかん]に
\\	帰る	帰[かえ]る	かえる	
\\	家に帰ろう。	家[うち]に 帰[かえ]ろう。	うち に かえろう	
\\	家[うち]に
\\	大きい	大[おお]きい	おおきい	
\\	あの大きい建物は何ですか。	あの 大[おお]きい 建物[たてもの]は 何[なん]ですか。	あの おおきい たてもの は なん です か	
\\	あの
\\	建物[たてもの]は 何[なん]ですか。		
\\	小さい	小[ちい]さい	ちいさい	
\\	小さい花が咲いています。	小[ちい]さい 花[はな]が 咲[さ]いています。	ちいさい はな が さいて います	
\\	花[はな]が 咲[さ]いています。		
\\	少ない	少[すく]ない	すくない	
\\	今年は雨が少ないです。	今年[ことし]は 雨[あめ]が 少[すく]ないです。	ことし は あめ が すくない です 。	
\\	今年[ことし]は 雨[あめ]が
\\	です。		
\\	少し	少[すこ]し	すこし	
\\	少し疲れました。	少[すこ]し 疲[つか]れました。	すこし つかれました	
\\	疲[つか]れました。		
\\	多い	多[おお]い	おおい	
\\	京都にはお寺が多い。	京都[きょうと]にはお 寺[てら]が 多[おお]い。	きょうと に は おてら が おおい	
\\	京都[きょうと]にはお 寺[てら]が
\\	多分	多分[たぶん]	たぶん	
\\	彼女は多分家で寝ています。	彼女[かのじょ]は 多分[たぶん] 家[いえ]で 寝[ね]ています。	かのじょ は たぶん いえ で ねて います	
\\	彼女[かのじょ]は
\\	家[いえ]で 寝[ね]ています。		
\\	上げる	上[あ]げる	あげる	
\\	彼は荷物をあみだなに上げた。	彼[かれ]は 荷物[にもつ]をあみだなに 上[あ]げた。	かれ は にもつ を あみだな に あげた	
\\	彼[かれ]は 荷物[にもつ]をあみだなに
\\	よく	よく	よく	
\\	彼女はよく旅行に行きます。	彼女[かのじょ]はよく 旅行[りょこう]に 行[い]きます。	かのじょ は よく りょこう に いきます	
\\	彼女[かのじょ]は
\\	旅行[りょこう]に 行[い]きます。		
\\	上る	上[のぼ]る	のぼる	
\\	猫が屋根に上っている。	猫[ねこ]が 屋根[やね]に 上[のぼ]っている。	ねこ が やね に のぼって いる	
\\	猫[ねこ]が 屋根[やね]に
\\	上	上[うえ]	うえ	
\\	上を向いて。	上[うえ]を 向[む]いて。	うえ を むいて	
\\	を 向[む]いて。		
\\	下	下[した]	した	
\\	財布は机の下にあった。	財布[さいふ]は 机[つくえ]の 下[した]にあった。	さいふ は つくえ の した に あった	
\\	財布[さいふ]は 机[つくえ]の
\\	にあった。		
\\	右	右[みぎ]	みぎ	
\\	右のポケットにハンカチが入っています。	右[みぎ]のポケットにハンカチが 入[はい]っています。	みぎ の ぽけっと に はんかち が はいって います	
\\	のポケットにハンカチが 入[はい]っています。		
\\	左	左[ひだり]	ひだり	
\\	そこを左に曲がってください。	そこを 左[ひだり]に 曲[ま]がってください。	そこ を ひだり に まがって ください	
\\	そこを
\\	に 曲[ま]がってください。		
\\	方	方[ほう]	ほう	
\\	彼は私の方を見ました。	彼[かれ]は 私[わたし]の 方[ほう]を 見[み]ました。	かれ は わたし の ほう を みました	
\\	彼[かれ]は 私[わたし]の
\\	を 見[み]ました。		
\\	大人	大人[おとな]	おとな	
\\	お酒は大人になってから。	お 酒[さけ]は 大人[おとな]になってから。	おさけ は おとな に なって から	
\\	お 酒[さけ]は
\\	になってから。		
\\	人	人[ひと]	ひと	
\\	彼は優しい人です。	彼[かれ]は 優[やさ]しい 人[ひと]です。	かれ は やさしい ひと です	
\\	彼[かれ]は 優[やさ]しい
\\	です。		
\\	一人	一人[ひとり]	ひとり	
\\	そこには私一人しかいなかった。	そこには 私[わたし] 一人[ひとり]しかいなかった。	そこ に は わたし ひとり しか いなかった	
\\	そこには 私[わたし]
\\	しかいなかった。		
\\	どう	どう	どう	
\\	あなたはどう思いますか。	あなたはどう 思[おも]いますか。	あなた は どう おもいます か	
\\	あなたは
\\	思[おも]いますか。		
\\	入れる	入[い]れる	いれる	
\\	彼はかばんに手帳を入れた。	彼[かれ]はかばんに 手帳[てちょう]を 入[い]れた。	かれ は かばん に てちょう を いれた	
\\	彼[かれ]はかばんに 手帳[てちょう]を
\\	入る	入[はい]る	はいる	
\\	寒いので中に入ってください。	寒[さむ]いので 中[なか]に 入[はい]ってください。	さむい の で なか に はいって ください	
\\	寒[さむ]いので 中[なか]に
\\	ください。		
\\	出る	出[で]る	でる	
\\	今朝は早く家を出ました。	今朝[けさ]は 早[はや]く 家[いえ]を 出[で]ました。	けさ は はやく いえ を でました	
\\	今朝[けさ]は 早[はや]く 家[いえ]を
\\	できる	できる	できる	
\\	彼女はイタリア語が出来ます。	彼女[かのじょ]はイタリア 語[ご]が できます。	かのじょ は いたりあご が できます	
\\	彼女[かのじょ]はイタリア 語[ご]が
\\	出す	出[だ]す	だす	
\\	彼はかばんから教科書を出した。	彼[かれ]はかばんから 教科書[きょうかしょ]を 出[だ]した。	かれ は かばん から きょうかしょ を だした	
\\	彼[かれ]はかばんから 教科書[きょうかしょ]を
\\	本	本[ほん]	ほん	
\\	本を1冊買いました。	本[ほん]を 1冊買[いっさつ か]いました。	ほん を いっさつ かいました	
\\	を 1冊買[いっさつ か]いました。		
\\	休む	休[やす]む	やすむ	
\\	明日、会社を休みます。	明日[あした]、 会社[かいしゃ]を 休[やす]みます。	あした かいしゃ を やすみます	
\\	明日[あした]、 会社[かいしゃ]を
\\	体	体[からだ]	からだ	
\\	私は体が丈夫だ。	私[わたし]は 体[からだ]が 丈夫[じょうぶ]だ。	わたし は からだ が じょうぶ だ	
\\	私[わたし]は
\\	が 丈夫[じょうぶ]だ。		
\\	目	目[め]	め	
\\	彼女は青い目をしています。	彼女[かのじょ]は 青[あお]い 目[め]をしています。	かのじょ は あおい め を して います	
\\	彼女[かのじょ]は 青[あお]い
\\	をしています。		
\\	どこ	どこ	どこ	
\\	あの本をどこに置きましたか。	あの 本[ほん]をどこに 置[お]きましたか。	あの ほん を どこ に おきました か	
\\	あの 本[ほん]を
\\	に 置[お]きましたか。		
\\	口	口[くち]	くち	
\\	口を大きく開けてください。	口[くち]を 大[おお]きく 開[あ]けてください。	くち を おおきく あけて ください	
\\	を 大[おお]きく 開[あ]けてください。		
\\	耳	耳[みみ]	みみ	
\\	彼女は耳がよく聞こえません。	彼女[かのじょ]は 耳[みみ]がよく 聞[き]こえません。	かのじょ は みみ が よく きこえません	
\\	彼女[かのじょ]は
\\	がよく 聞[き]こえません。		
\\	上手	上手[じょうず]	じょうず	
\\	妹は歌が上手です。	妹[いもうと]は 歌[うた]が 上手[じょうず]です。	いもうと は うた が じょうず です	
\\	妹[いもうと]は 歌[うた]が
\\	です。		
\\	手	手[て]	て	
\\	分かった人は手を上げてください。	分[わ]かった 人[ひと]は 手[て]を 上[あ]げてください。	わかった ひと は て を あげて ください	
\\	分[わ]かった 人[ひと]は
\\	を 上[あ]げてください。		
\\	足	足[あし]	あし	
\\	彼は足が長い。	彼[かれ]は 足[あし]が 長[なが]い。	かれ は あし が ながい	
\\	彼[かれ]は
\\	が 長[なが]い。		
\\	空く	空[す]く	すく	
\\	レストランは空いていました。	レストランは 空[す]いていました。	れすとらん は すいて いました	
\\	レストランは
\\	男	男[おとこ]	おとこ	
\\	男の人が私たちに話しかけた。	男[おとこ]の 人[ひと]が 私[わたし]たちに 話[はな]しかけた。	おとこ の ひと が わたしたち に はなしかけた	
\\	の 人[ひと]が 私[わたし]たちに 話[はな]しかけた。		
\\	女	女[おんな]	おんな	
\\	店員は若い女の人でした。	店員[てんいん]は 若[わか]い 女[おんな]の 人[ひと]でした。	てんいん は わかい おんな の ひと でした	
\\	店員[てんいん]は 若[わか]い
\\	の 人[ひと]でした。		
\\	子供	子供[こども]	こども	
\\	電車で子供が騒いでいた。	電車[でんしゃ]で 子供[こども]が 騒[さわ]いでいた。	でんしゃ で こども が さわいで いた	
\\	電車[でんしゃ]で
\\	が 騒[さわ]いでいた。		
\\	あげる	あげる	あげる	
\\	この本、あなたにあげます。	この 本[ほん]、あなたにあげます。	この ほん あなた に あげます	
\\	この 本[ほん]、あなたに
\\	好き	好[す]き	すき	
\\	私はワインが好きです。	私[わたし]はワインが 好[す]きです。	わたし は わいん が すき です	
\\	私[わたし]はワインが
\\	です。		
\\	大好き	大好[だいす]き	だいすき	
\\	私は犬が大好きだ。	私[わたし]は 犬[いぬ]が 大好[だいす]きだ。	わたし は いぬ が だいすき だ	
\\	私[わたし]は 犬[いぬ]が
\\	だ。		
\\	私	私[わたし]	わたし	
\\	私が行きましょう。	私[わたし]が 行[い]きましょう。	わたし が いきましょう	
\\	が 行[い]きましょう。		
\\	友達	友達[ともだち]	ともだち	
\\	私には友達がたくさんいます。	私[わたし]には 友達[ともだち]がたくさんいます。	わたし に は ともだち が たくさん います	
\\	私[わたし]には
\\	がたくさんいます。		
\\	家	家[うち]	うち	
\\	家に遊びに来てください。	家[うち]に 遊[あそ]びに 来[き]てください。	うち に あそび に きて ください	
\\	に 遊[あそ]びに 来[き]てください。		
\\	気	気[き]	き	
\\	彼は意外に気が小さい。	彼[かれ]は 意外[いがい]に 気[き]が 小[ちい]さい。	かれ は いがい に き が ちいさい	
\\	彼[かれ]は 意外[いがい]に
\\	が 小[ちい]さい。		
\\	元気	元気[げんき]	げんき	
\\	おかげさまで元気です。	おかげさまで 元気[げんき]です。	おかげさま で げんき です	
\\	おかげさまで
\\	です。		
\\	天気	天気[てんき]	てんき	
\\	今日はいい天気ですね。	今日[きょう]はいい 天気[てんき]ですね。	きょう は いい てんき です ね	
\\	今日[きょう]はいい
\\	ですね。		
\\	晴れる	晴[は]れる	はれる	
\\	明日は晴れるといいですね。	明日[あした]は 晴[は]れるといいですね。	あした は はれる と いい です ね	
\\	明日[あした]は
\\	といいですね。		
\\	こう	こう	こう	
\\	こう小さい字は読めない。	こう 小[ちい]さい 字[じ]は 読[よ]めない。	こう ちいさい じ は よめない	
\\	小[ちい]さい 字[じ]は 読[よ]めない。		
\\	昨日	昨日[きのう]	きのう	
\\	昨日、友達に会った。	昨日[きのう]、 友達[ともだち]に 会[あ]った。	きのう ともだち に あった	
\\	、 友達[ともだち]に 会[あ]った。		
\\	開ける	開[あ]ける	あける	
\\	窓を開けてください。	窓[まど]を 開[あ]けてください。	まど を あけて ください	
\\	窓[まど]を
\\	ください。		
\\	開く	開[ひら]く	ひらく	
\\	を/が		
\\	32ページを開いてください。	32[さんじゅうに]ページを 開[ひら]いてください。	さんじゅうにぺーじ を ひらいて ください	
\\	32[さんじゅうに]ページを
\\	ください。		
\\	閉じる	閉[と]じる	とじる	
\\	教科書を閉じてください。	教科書[きょうかしょ]を 閉[と]じてください。	きょうかしょ を とじて ください	
\\	教科書[きょうかしょ]を
\\	ください。		
\\	閉める	閉[し]める	しめる	
\\	ちゃんとドアを閉めてよ。	ちゃんとドアを 閉[し]めてよ。	ちゃんと どあ を しめてよ	
\\	ちゃんとドアを
\\	よ。		
\\	閉まる	閉[し]まる	しまる	
\\	お店はもう閉まっていました。	お 店[みせ]はもう 閉[し]まっていました。	おみせ は もう しまって いました	
\\	お 店[みせ]はもう
\\	聞く	聞[き]く	きく	
\\	彼女はラジオを聞いています。	彼女[かのじょ]はラジオを 聞[き]いています。	かのじょ は らじお を きいて います	
\\	彼女[かのじょ]はラジオを
\\	時間	時間[じかん]	じかん	
\\	今は時間がありません。	今[いま]は 時間[じかん]がありません。	いま は じかん が ありません	
\\	今[いま]は
\\	がありません。		
\\	高い	高[たか]い	たかい	
\\	これがこの町で一番高いビルです。	これがこの 町[まち]で 一番[いちばん] 高[たか]いビルです。	これ が この まち で いちばん たかい びる です	
\\	これがこの 町[まち]で 一番[いちばん]
\\	ビルです。		
\\	くれる	くれる	くれる	
\\	友達が誕生日プレゼントをくれた。	友達[ともだち]が 誕生日[たんじょうび]プレゼントをくれた。	ともだち が たんじょうび ぷれぜんと を くれた	
\\	友達[ともだち]が 誕生日[たんじょうび]プレゼントを
\\	安い	安[やす]い	やすい	
\\	この服はとても安かった。	この 服[ふく]はとても 安[やす]かった。	この ふく は とても やすかった	
\\	この 服[ふく]はとても
\\	低い	低[ひく]い	ひくい	
\\	彼は背が低い。	彼[かれ]は 背[せ]が 低[ひく]い。	かれ は せ が ひくい	
\\	彼[かれ]は 背[せ]が
\\	前	前[まえ]	まえ	
\\	その店の前で会いましょう。	その 店[みせ]の 前[まえ]で 会[あ]いましょう。	その みせ の まえ で あいましょう	
\\	その 店[みせ]の
\\	で 会[あ]いましょう。		
\\	後	後[あと]	あと	
\\	仕事の後、映画を見た。	仕事[しごと]の 後[あと]、 映画[えいが]を 見[み]た。	しごと の あと えいが を みた	
\\	仕事[しごと]の
\\	、 映画[えいが]を 見[み]た。		
\\	後ろ	後[うし]ろ	うしろ	
\\	後ろを向いて。	後[うし]ろを 向[む]いて。	うしろ を むいて	
\\	を 向[む]いて。		
\\	午後	午後[ごご]	ごご	
\\	明日の午後、お客様が来る。	明日[あす]の 午後[ごご]、お 客様[きゃくさま]が 来[く]る。	あす の ごご おきゃくさま が くる	
\\	明日[あす]の
\\	、お 客様[きゃくさま]が 来[く]る。		
\\	午前	午前[ごぜん]	ごぜん	
\\	午前9時のニュースです。	午前[ごぜん] 9時[くじ]のニュースです。	ごぜん くじ の にゅーす です	
\\	9時[くじ]のニュースです。		
\\	朝	朝[あさ]	あさ	
\\	気持ちのいい朝です。	気持[きも]ちのいい 朝[あさ]です。	きもち の いい あさ です	
\\	気持[きも]ちのいい
\\	です。		
\\	昼	昼[ひる]	ひる	
\\	私は昼のドラマを毎日見ます。	私[わたし]は 昼[ひる]のドラマを 毎日見[まいにち み]ます。	わたし は ひる の どらま を まいにち みます	
\\	私[わたし]は
\\	のドラマを 毎日見[まいにち み]ます。		
\\	かなり	かなり	かなり	
\\	彼はかなり英語が上手です。	彼[かれ]はかなり 英語[えいご]が 上手[じょうず]です。	かれ は かなり えいご が じょうず です	
\\	彼[かれ]は
\\	英語[えいご]が 上手[じょうず]です。		
\\	晩	晩[ばん]	ばん	
\\	晩ご飯は食べましたか。	晩[ばん]ご 飯[はん]は 食[た]べましたか。	ばんごはん は たべました か	
\\	ご 飯[はん]は 食[た]べましたか。		
\\	今晩	今晩[こんばん]	こんばん	
\\	今晩のパーティーは何時からですか。	今晩[こんばん]のパーティーは 何時[なんじ]からですか。	こんばん の ぱーてぃー は なんじ から です か	
\\	のパーティーは 何時[なんじ]からですか。		
\\	夜	夜[よる]	よる	
\\	きのうの夜は家にいました。	きのうの 夜[よる]は 家[いえ]にいました。	きのう の よる は いえ に いました	
\\	きのうの
\\	は 家[いえ]にいました。		
\\	食べる	食[た]べる	たべる	
\\	昨日タイカレーを食べました。	昨日[きのう]タイカレーを 食[た]べました。	きのう たいかれー を たべました	
\\	昨日[きのう]タイカレーを
\\	飲む	飲[の]む	のむ	
\\	友達とお酒を飲んでいます。	友達[ともだち]とお 酒[さけ]を 飲[の]んでいます。	ともだち と おさけ を のんで います	
\\	友達[ともだち]とお 酒[さけ]を
\\	ご飯	ご 飯[はん]	ごはん	
\\	私はパンよりご飯が好きだ。	私[わたし]はパンよりご 飯[はん]が 好[す]きだ。	わたし は ぱん より ごはん が すき だ	
\\	私[わたし]はパンより
\\	が 好[す]きだ。		
\\	買う	買[か]う	かう	
\\	郵便局で切手を買いました。	郵便局[ゆうびんきょく]で 切手[きって]を 買[か]いました。	ゆうびんきょく で きって を かいました	
\\	郵便局[ゆうびんきょく]で 切手[きって]を
\\	見る	見[み]る	みる	
\\	私は絵を見るのが好きです。	私[わたし]は 絵[え]を 見[み]るのが 好[す]きです。	わたし は え を みる の が すき です	
\\	私[わたし]は 絵[え]を
\\	のが 好[す]きです。		
\\	見せる	見[み]せる	みせる	
\\	その写真を見せてください。	その 写真[しゃしん]を 見[み]せてください。	その しゃしん を みせて ください	
\\	その 写真[しゃしん]を
\\	ください。		
\\	もっと	もっと	もっと	
\\	もっと近くに来てください。	もっと 近[ちか]くに 来[き]てください。	もっと ちかく に きて ください	
\\	近[ちか]くに 来[き]てください。		
\\	見つける	見[み]つける	みつける	
\\	新しい仕事を見つけました。	新[あたら]しい 仕事[しごと]を 見[み]つけました。	あたらしい しごと を みつけました。	
\\	新[あたら]しい 仕事[しごと]を
\\	見える	見[み]える	みえる	
\\	ここから富士山がよく見えます。	ここから 富士山[ふじさん]がよく 見[み]えます。	ここ から ふじさん が よく みえます	
\\	ここから 富士山[ふじさん]がよく
\\	見つかる	見[み]つかる	みつかる	
\\	メガネが見つかりません。	メガネが 見[み]つかりません。	めがね が みつかりません	
\\	メガネが
\\	言う	言[い]う	いう	
\\	上司が「一杯、飲もう。」と言った。	上司[じょうし]が
\\	一杯[いっぱい]、 飲[の]もう。」と 言[い]った。	じょうし が いっぱい のもう と いった	
\\	上司[じょうし]が
\\	一杯[いっぱい]、 飲[の]もう。」と
\\	話す	話[はな]す	はなす	
\\	母と電話で話しました。	母[はは]と 電話[でんわ]で 話[はな]しました。	はは と でんわ で はなしました	
\\	母[はは]と 電話[でんわ]で
\\	読む	読[よ]む	よむ	
\\	彼女は雑誌を読んでいます。	彼女[かのじょ]は 雑誌[ざっし]を 読[よ]んでいます。	かのじょ は ざっし を よんで います	
\\	彼女[かのじょ]は 雑誌[ざっし]を
\\	漢字	漢字[かんじ]	かんじ	
\\	漢字は中国から来ました。	漢字[かんじ]は 中国[ちゅうごく]から 来[き]ました。	かんじ は ちゅうごく から きました	
\\	は 中国[ちゅうごく]から 来[き]ました。		
\\	書く	書[か]く	かく	
\\	彼に手紙を書きました。	彼[かれ]に 手紙[てがみ]を 書[か]きました。	かれ に てがみ を かきました	
\\	彼[かれ]に 手紙[てがみ]を
\\	覚える	覚[おぼ]える	おぼえる	
\\	妹は平仮名を全部覚えました。	妹[いもうと]は 平仮名[ひらがな]を 全部[ぜんぶ] 覚[おぼ]えました。	いもうと は ひらがな を ぜんぶ おぼえました	
\\	妹[いもうと]は 平仮名[ひらがな]を 全部[ぜんぶ]
\\	そこ	そこ	そこ	
\\	そこに座ってください。	そこに 座[すわ]ってください。	そこ に すわって ください	
\\	に 座[すわ]ってください。		
\\	会う	会[あ]う	あう	
\\	また会いましょう。	また 会[あ]いましょう。	また あいましょう	
\\	また
\\	仕事	仕事[しごと]	しごと	
\\	3月は仕事が忙しい。	3月[さんがつ]は 仕事[しごと]が 忙[いそが]しい。	さんがつ は しごと が いそがしい	
\\	3月[さんがつ]は
\\	が 忙[いそが]しい。		
\\	場合	場合[ばあい]	ばあい	
\\	分からない場合は私に聞いてください。	分[わ]からない 場合[ばあい]は 私[わたし]に 聞[き]いてください。	わからない ばあい は わたし に きいて ください	
\\	分[わ]からない
\\	は 私[わたし]に 聞[き]いてください。		
\\	車	車[くるま]	くるま	
\\	弟が車を買った。	弟[おとうと]が 車[くるま]を 買[か]った。	おとうと が くるま を かった	
\\	弟[おとうと]が
\\	を 買[か]った。		
\\	電車	電車[でんしゃ]	でんしゃ	
\\	私は電車で通学しています。	私[わたし]は 電車[でんしゃ]で 通学[つうがく]しています。	わたし は でんしゃ で つうがく して います	
\\	私[わたし]は
\\	で 通学[つうがく]しています。		
\\	駅	駅[えき]	えき	
\\	駅はどこですか。	駅[えき]はどこですか。	えき は どこ です か	
\\	はどこですか。		
\\	道	道[みち]	みち	
\\	この道を真っ直ぐ行くと駅です。	この 道[みち]を 真[ま]っ 直[す]ぐ 行[い]くと 駅[えき]です。	この みち を まっすぐ いく と えき です	
\\	この
\\	を 真[ま]っ 直[す]ぐ 行[い]くと 駅[えき]です。		
\\	他	他[ほか]	ほか	
\\	他に方法がありません。	他[ほか]に 方法[ほうほう]がありません。	ほか に ほうほう が ありません	
\\	に 方法[ほうほう]がありません。		
\\	止める	止[や]める	やめる	
\\	話すのを止めてください。	話[はな]すのを 止[や]めてください。	はなす の を やめて ください	
\\	話[はな]すのを
\\	ください。		
\\	ここ	ここ	ここ	
\\	ここに本があります。	ここに 本[ほん]があります。	ここ に ほん が あります	
\\	に 本[ほん]があります。		
\\	歩く	歩[ある]く	あるく	
\\	駅まで歩きましょう。	駅[えき]まで 歩[ある]きましょう。	えき まで あるきましょう	
\\	駅[えき]まで
\\	走る	走[はし]る	はしる	
\\	彼は毎晩3キロ走っています。	彼[かれ]は 毎晩3[まいばん さん]キロ 走[はし]っています。	かれ は まいばん さんきろ はしって います	
\\	彼[かれ]は 毎晩3[まいばん さん]キロ
\\	近く	近[ちか]く	ちかく	
\\	駅の近くで食事をした。	駅[えき]の 近[ちか]くで 食事[しょくじ]をした。	えき の ちかく で しょくじ を した	
\\	駅[えき]の
\\	で 食事[しょくじ]をした。		
\\	近い	近[ちか]い	ちかい	
\\	駅はここから近いです。	駅[えき]はここから 近[ちか]いです。	えき は ここ から ちかい です	
\\	駅[えき]はここから
\\	です。		
\\	近く	近[ちか]く	ちかく	
\\	私の家は駅の近くです。	私[わたし]の 家[いえ]は 駅[えき]の 近[ちか]くです。	わたし の いえ は えき の ちかく です	
\\	私[わたし]の 家[いえ]は 駅[えき]の
\\	です。		
\\	遠い	遠[とお]い	とおい	
\\	家から学校までは遠いです。	家[いえ]から 学校[がっこう]までは 遠[とお]いです。	いえ から がっこう まで は とおい です	
\\	家[いえ]から 学校[がっこう]までは
\\	です。		
\\	長い	長[なが]い	ながい	
\\	彼女の髪はとても長い。	彼女[かのじょ]の 髪[かみ]はとても 長[なが]い。	かのじょ の かみ は とても ながい	
\\	彼女[かのじょ]の 髪[かみ]はとても
\\	短い	短[みじか]い	みじかい	
\\	彼は足が短い。	彼[かれ]は 足[あし]が 短[みじか]い。	かれ は あし が みじかい	
\\	彼[かれ]は 足[あし]が
\\	広い	広[ひろ]い	ひろい	
\\	彼の家はとても広い。	彼[かれ]の 家[いえ]はとても 広[ひろ]い。	かれ の いえ は とても ひろい	
\\	彼[かれ]の 家[いえ]はとても
\\	もらう	もらう	もらう	
\\	彼女からプレゼントをもらいました。	彼女[かのじょ]からプレゼントをもらいました。	かのじょ から ぷれぜんと を もらいました	
\\	彼女[かのじょ]からプレゼントを
\\	全部	全部[ぜんぶ]	ぜんぶ	
\\	それ、全部ください。	それ、 全部[ぜんぶ]ください。	それ ぜんぶ ください	
\\	それ、
\\	ください。		
\\	国	国[くに]	くに	
\\	私の国について少しお話しましょう。	私[わたし]の 国[くに]について 少[すこ]しお 話[はなし]しましょう。	わたし の くに に ついて すこし おはなし しましょう	
\\	私[わたし]の
\\	について 少[すこ]しお 話[はなし]しましょう。		
\\	白い	白[しろ]い	しろい	
\\	彼は歯が白い。	彼[かれ]は 歯[は]が 白[しろ]い。	かれ は は が しろい	
\\	彼[かれ]は 歯[は]が
\\	赤い	赤[あか]い	あかい	
\\	赤いバラを買いました。	赤[あか]いバラを 買[か]いました。	あかい ばら を かいました	
\\	バラを 買[か]いました。		
\\	部屋	部屋[へや]	へや	
\\	私の部屋は2階にあります。	私[わたし]の 部屋[へや]は 2階[にかい]にあります。	わたし の へや は にかい に あります	
\\	私[わたし]の
\\	は 2階[にかい]にあります。		
\\	米	米[こめ]	こめ	
\\	日本人はお米が大好きです。	日本人[にほんじん]はお 米[こめ]が 大好[だいす]きです。	にほんじん は おこめ が だいすき です	
\\	日本人[にほんじん]はお
\\	が 大好[だいす]きです。		
\\	まだ	まだ	まだ	
\\	宿題はまだ終わっていません。	宿題[しゅくだい]はまだ 終[お]わっていません。	しゅくだい は まだ おわって いません	
\\	宿題[しゅくだい]は
\\	終[お]わっていません。		
\\	ある	ある	ある	
\\	私の机の上に書類がたくさんあります。	私[わたし]の 机[つくえ]の 上[うえ]に 書類[しょるい]がたくさん あります。	わたし の つくえ の うえ に しょるい が たくさんあります	
\\	私[わたし]の 机[つくえ]の 上[うえ]に 書類[しょるい]がたくさん
\\	無い	無[な]い	ない	
\\	ここには何も無い。	ここには 何[なに]も 無[な]い。	ここ に は なに も ない	
\\	ここには 何[なに]も
\\	とても	とても	とても	
\\	この本はとてもおもしろい。	この 本[ほん]はとてもおもしろい。	この ほん は とても おもしろい	
\\	この 本[ほん]は
\\	おもしろい。		
\\	作る	作[つく]る	つくる	
\\	今、朝ご飯を作っています。	今[いま]、 朝[あさ]ご 飯[はん]を 作[つく]っています。	いま あさごはん を つくって います	
\\	今[いま]、 朝[あさ]ご 飯[はん]を
\\	使う	使[つか]う	つかう	
\\	このパソコンを使ってください。	このパソコンを 使[つか]ってください。	この ぱそこん を つかって ください	
\\	このパソコンを
\\	ください。		
\\	消す	消[け]す	けす	
\\	昼間は電気を消してください。	昼間[ひるま]は 電気[でんき]を 消[け]してください。	ひるま は でんき を けして ください	
\\	昼間[ひるま]は 電気[でんき]を
\\	ください。		
\\	売る	売[う]る	うる	
\\	を		
\\	彼は家を売った。	彼[かれ]は 家[いえ]を 売[う]った。	かれ は いえ を うった	
\\	彼[かれ]は 家[いえ]を
\\	店	店[みせ]	みせ	
\\	私はこの店によく来ます。	私[わたし]はこの 店[みせ]によく 来[き]ます。	わたし は この みせ に よく きます	
\\	私[わたし]はこの
\\	によく 来[き]ます。		
\\	春	春[はる]	はる	
\\	今年の春は暖かいね。	今年[ことし]の 春[はる]は 暖[あたた]かいね。	ことし の はる は あたたかい ね	
\\	今年[ことし]の
\\	は 暖[あたた]かいね。		
\\	夏	夏[なつ]	なつ	
\\	私は夏が大好き。	私[わたし]は 夏[なつ]が 大好[だいす]き。	わたし は なつ が だいすき	
\\	私[わたし]は
\\	が 大好[だいす]き。		
\\	秋	秋[あき]	あき	
\\	彼女は秋に結婚します。	彼女[かのじょ]は 秋[あき]に 結婚[けっこん]します。	かのじょ は あき に けっこん します	
\\	彼女[かのじょ]は
\\	に 結婚[けっこん]します。		
\\	冬	冬[ふゆ]	ふゆ	
\\	カナダの冬はとても寒いです。	カナダの 冬[ふゆ]はとても 寒[さむ]いです。	かなだ の ふゆ は とても さむい です	
\\	カナダの
\\	はとても 寒[さむ]いです。		
\\	みんな	みんな	みんな	
\\	みんなにお菓子をあげましょう。	みんなにお 菓子[かし]をあげましょう。	みんな に おかし を あげましょう	
\\	にお 菓子[かし]をあげましょう。		
\\	暑い	暑[あつ]い	あつい	
\\	今日はとても暑い。	今日[きょう]はとても 暑[あつ]い。	きょう は とても あつい	
\\	今日[きょう]はとても
\\	熱い	熱[あつ]い	あつい	
\\	このスープはとても熱い。	このスープはとても 熱[あつ]い。	この すーぷ は とても あつい	
\\	このスープはとても
\\	寒い	寒[さむ]い	さむい	
\\	この部屋は寒いです。	この 部屋[へや]は 寒[さむ]いです。	この へや は さむい です	
\\	この 部屋[へや]は
\\	です。		
\\	暖かい	暖[あたた]かい	あたたかい	
\\	このコートはとても暖かい。	このコートはとても 暖[あたた]かい。	この こーと は とても あたたかい。	
\\	このコートはとても
\\	新しい	新[あたら]しい	あたらしい	
\\	彼の車は新しい。	彼[かれ]の 車[くるま]は 新[あたら]しい。	かれ の くるま は あたらしい	
\\	彼[かれ]の 車[くるま]は
\\	古い	古[ふる]い	ふるい	
\\	私は古い車が好きです。	私[わたし]は 古[ふる]い 車[くるま]が 好[す]きです。	わたし は ふるい くるま が すき です	
\\	私[わたし]は
\\	車[くるま]が 好[す]きです。		
\\	良い	良[い]い	いい	
\\	彼女は良い友達です。	彼女[かのじょ]は 良[い]い 友達[ともだち]です。	かのじょ は いい ともだち です	
\\	彼女[かのじょ]は
\\	友達[ともだち]です。		
\\	悪い	悪[わる]い	わるい	
\\	たばこは体に悪い。	たばこは 体[からだ]に 悪[わる]い。	たばこ は からだ に わるい	
\\	たばこは 体[からだ]に
\\	思う	思[おも]う	おもう	
\\	私もそう思います。	私[わたし]もそう 思[おも]います。	わたし も そう おもいます	
\\	私[わたし]もそう
\\	いつも	いつも	いつも	
\\	彼女はいつも元気だ。	彼女[かのじょ]はいつも 元気[げんき]だ。	かのじょ は いつも げんき だ	
\\	彼女[かのじょ]は
\\	元気[げんき]だ。		
\\	忘れる	忘[わす]れる	わすれる	
\\	約束を忘れないでください。	約束[やくそく]を 忘[わす]れないでください。	やくそく を わすれない で ください	
\\	約束[やくそく]を
\\	ください。		
\\	考える	考[かんが]える	かんがえる	
\\	よく考えてください。	よく 考[かんが]えてください。	よく かんがえて ください	
\\	よく
\\	ください。		
\\	決める	決[き]める	きめる	
\\	帰国することに決めました。	帰国[きこく]することに 決[き]めました。	きこく する こと に きめました	
\\	帰国[きこく]することに
\\	決まる	決[き]まる	きまる	
\\	旅行の日程が決まりました。	旅行[りょこう]の 日程[にってい]が 決[き]まりました。	りょこう の にってい が きまりました	
\\	旅行[りょこう]の 日程[にってい]が
\\	知る	知[し]る	しる	
\\	誰もその話を知らない。	誰[だれ]もその 話[はなし]を 知[し]らない。	だれ も その はなし を しらない	
\\	誰[だれ]もその 話[はなし]を
\\	一番	一番[いちばん]	いちばん	
\\	彼はクラスで一番背が高い。	彼[かれ]はクラスで 一番[いちばん] 背[せ]が 高[たか]い。	かれ は くらす で いちばん せ が たかい	
\\	彼[かれ]はクラスで
\\	背[せ]が 高[たか]い。		
\\	住む	住[す]む	すむ	
\\	彼は会社の近くに住んでいる。	彼[かれ]は 会社[かいしゃ]の 近[ちか]くに 住[す]んでいる。	かれ は かいしゃ の ちかく に すんで いる	
\\	彼[かれ]は 会社[かいしゃ]の 近[ちか]くに
\\	名前	名前[なまえ]	なまえ	
\\	あなたの名前を教えてください。	あなたの 名前[なまえ]を 教[おし]えてください。	あなた の なまえ を おしえて ください	
\\	あなたの
\\	を 教[おし]えてください。		
\\	食べ物	食[た]べ 物[もの]	たべもの	
\\	日本の食べ物はとても美味しいです。	日本[にっぽん]の 食[た]べ 物[もの]はとても 美味[おい]しいです。	にっぽん の たべもの は とても おいしい です	
\\	日本[にっぽん]の
\\	はとても 美味[おい]しいです。		
\\	いつ	いつ	いつ	
\\	彼女はいつ来ますか。	彼女[かのじょ]はいつ 来[き]ますか。	かのじょ は いつ きます か	
\\	彼女[かのじょ]は
\\	来[き]ますか。		
\\	飲み物	飲[の]み 物[もの]	のみもの	
\\	何か飲み物が欲しいな。	何[なに]か 飲[の]み 物[もの]が 欲[ほ]しいな。	なに か のみもの が ほしい な 。	
\\	何[なに]か
\\	が 欲[ほ]しいな。		
\\	重い	重[おも]い	おもい	
\\	このかばんは重いです。	このかばんは 重[おも]いです。	この かばん は おもい です	
\\	このかばんは
\\	です。		
\\	軽い	軽[かる]い	かるい	
\\	この靴はとても軽い。	この 靴[くつ]はとても 軽[かる]い。	この くつ は とても かるい	
\\	この 靴[くつ]はとても
\\	送る	送[おく]る	おくる	
\\	彼の家に荷物を送りました。	彼[かれ]の 家[いえ]に 荷物[にもつ]を 送[おく]りました。	かれ の いえ に にもつ を おくりました	
\\	彼[かれ]の 家[いえ]に 荷物[にもつ]を
\\	取る	取[と]る	とる	
\\	テストでいい点を取った。	テストでいい 点[てん]を 取[と]った。	てすと で いい てん を とった	
\\	テストでいい 点[てん]を
\\	待つ	待[ま]つ	まつ	
\\	あなたが来るのを待っています。	あなたが 来[く]るのを 待[ま]っています。	あなた が くる の を まって います	
\\	あなたが 来[く]るのを
\\	持つ	持[も]つ	もつ	
\\	私は車を持っています。	私[わたし]は 車[くるま]を 持[も]っています。	わたし は くるま を もって います	
\\	私[わたし]は 車[くるま]を
\\	気持ち	気持[きも]ち	きもち	
\\	彼の気持ちが分からない。	彼[かれ]の 気持[きも]ちが 分[わ]からない。	かれ の きもち が わからない	
\\	彼[かれ]の
\\	が 分[わ]からない。		
\\	生きる	生[い]きる	いきる	
\\	皆一生懸命生きている。	皆一生懸命[みんな いっしょうけんめい] 生[い]きている。	みんな いっしょうけんめい いきて いる	
\\	皆一生懸命[みんな いっしょうけんめい]
\\	どちら	どちら	どちら	
\\	肉と魚とどちらが好きですか。	肉[にく]と 魚[さかな]とどちらが 好[す]きですか。	にく と さかな と どちら が すき です か	
\\	肉[にく]と 魚[さかな]と
\\	が 好[す]きですか。		
\\	先生	先生[せんせい]	せんせい	
\\	私は日本語の先生になりたいです。	私[わたし]は 日本語[にほんご]の 先生[せんせい]になりたいです。	わたし は にほんご の せんせい に なりたい です	
\\	私[わたし]は 日本語[にほんご]の
\\	になりたいです。		
\\	大学	大学[だいがく]	だいがく	
\\	大学に行ってもっと勉強したいです。	大学[だいがく]に 行[い]ってもっと 勉強[べんきょう]したいです。	だいがく に いって もっと べんきょう したい です	
\\	に 行[い]ってもっと 勉強[べんきょう]したいです。		
\\	学生	学生[がくせい]	がくせい	
\\	彼は真面目な学生です。	彼[かれ]は 真面目[まじめ]な 学生[がくせい]です。	かれ は まじめ な がくせい です	
\\	彼[かれ]は 真面目[まじめ]な
\\	です。		
\\	大学生	大学生[だいがくせい]	だいがくせい	
\\	姉は大学生です。	姉[あね]は 大学生[だいがくせい]です。	あね は だいがくせい です	
\\	姉[あね]は
\\	です。		
\\	学校	学校[がっこう]	がっこう	
\\	学校は8時半に始まります。	学校[がっこう]は 8時半[はちじはん]に 始[はじ]まります。	がっこう は はちじはん に はじまります	
\\	は 8時半[はちじはん]に 始[はじ]まります。		
\\	高校生	高校生[こうこうせい]	こうこうせい	
\\	私の弟は高校生です。	私[わたし]の 弟[おとうと]は 高校生[こうこうせい]です。	わたし の おとうと は こうこうせい です	
\\	私[わたし]の 弟[おとうと]は
\\	です。		
\\	教える	教[おし]える	おしえる	
\\	彼は数学を教えています。	彼[かれ]は 数学[すうがく]を 教[おし]えています。	かれ は すうがく を おしえて います	
\\	彼[かれ]は 数学[すうがく]を
\\	勉強	勉強[べんきょう]	べんきょう	
\\	私は日本語を勉強しています。	私[わたし]は 日本語[にほんご]を 勉強[べんきょう]しています。	わたし は にほんご を べんきょう して います	
\\	私[わたし]は 日本語[にほんご]を
\\	しています。		
\\	強い	強[つよ]い	つよい	
\\	今日は風が強い。	今日[きょう]は 風[かぜ]が 強[つよ]い。	きょう は かぜ が つよい	
\\	今日[きょう]は 風[かぜ]が
\\	どれ	どれ	どれ	
\\	この中でどれが好きですか。	この 中[なか]でどれが 好[す]きですか。	この なか で どれ が すき です か	
\\	この 中[なか]で
\\	が 好[す]きですか。		
\\	弱い	弱[よわ]い	よわい	
\\	その子は体が少し弱い。	その 子[こ]は 体[からだ]が 少[すこ]し 弱[よわ]い。	その こ は からだ が すこし よわい	
\\	その 子[こ]は 体[からだ]が 少[すこ]し
\\	引く	引[ひ]く	ひく	
\\	このドアは引いてください。	このドアは 引[ひ]いてください。	この どあ は ひいて ください	
\\	このドアは
\\	ください。		
\\	質問	質問[しつもん]	しつもん	
\\	質問のある方はどうぞ。	質問[しつもん]のある 方[かた]はどうぞ。	しつもん の ある かた は どうぞ	
\\	のある 方[かた]はどうぞ。		
\\	難しい	難[むずか]しい	むずかしい	
\\	この本は難しいですね。	この 本[ほん]は 難[むずか]しいですね。	この ほん は むずかしい です ね	
\\	この 本[ほん]は
\\	ですね。		
\\	数	数[かず]	かず	
\\	グラスの数が足りません。	グラスの 数[かず]が 足[た]りません。	ぐらす の かず が たりません	
\\	グラスの
\\	が 足[た]りません。		
\\	勝つ	勝[か]つ	かつ	
\\	今日はヤンキースが勝った。	今日[きょう]はヤンキースが 勝[か]った。	きょう は やんきーす が かった	
\\	今日[きょう]はヤンキースが
\\	負ける	負[ま]ける	まける	
\\	私たちのチームはその試合で負けた。	私[わたし]たちのチームはその 試合[しあい]で 負[ま]けた。	わたしたち の ちーむ は その しあい で まけた	
\\	私[わたし]たちのチームはその 試合[しあい]で
\\	本当に	本当[ほんとう]に	ほんとうに	
\\	あなたが本当に好きです。	あなたが 本当[ほんとう]に 好[す]きです。	あなた が ほんとうに すき です	
\\	あなたが
\\	好[す]きです。		
\\	要る	要[い]る	いる	
\\	予約は要りません。	予約[よやく]は 要[い]りません。	よやく は いりません	
\\	予約[よやく]は
\\	ドア	ドア	ドア	
\\	ドアを開けてください。	ドアを 開[あ]けてください。	どあ を あけて ください	
\\	を 開[あ]けてください。		
\\	時計	時計[とけい]	とけい	
\\	時計を見たらちょうど3時だった。	時計[とけい]を 見[み]たらちょうど 3時[さんじ]だった。	とけい を みたら ちょうど さんじ だった	
\\	を 見[み]たらちょうど 3時[さんじ]だった。		
\\	払う	払[はら]う	はらう	
\\	私が払いましょう。	私[わたし]が 払[はら]いましょう。	わたし が はらいましょう	
\\	私[わたし]が
\\	切る	切[き]る	きる	
\\	この紙を半分に切ってください。	この 紙[かみ]を 半分[はんぶん]に 切[き]ってください。	この かみ を はんぶん に きって ください	
\\	この 紙[かみ]を 半分[はんぶん]に
\\	ください。		
\\	変える	変[か]える	かえる	
\\	旅行の日程を変えました。	旅行[りょこう]の 日程[にってい]を 変[か]えました。	りょこう の にってい を かえました	
\\	旅行[りょこう]の 日程[にってい]を
\\	乗る	乗[の]る	のる	
\\	駅からはタクシーに乗ってください。	駅[えき]からはタクシーに 乗[の]ってください。	えき からは たくしー に のって ください	
\\	駅[えき]からはタクシーに
\\	ください。		
\\	着る	着[き]る	きる	
\\	今日はスーツを着ています。	今日[きょう]はスーツを 着[き]ています。	きょう は すーつ を きて います	
\\	今日[きょう]はスーツを
\\	立つ	立[た]つ	たつ	
\\	彼はステージに立った。	彼[かれ]はステージに 立[た]った。	かれ は すてーじ に たった	
\\	彼[かれ]はステージに
\\	座る	座[すわ]る	すわる	
\\	私は窓側の席に座った。	私[わたし]は 窓側[まどがわ]の 席[せき]に 座[すわ]った。	わたし は まどがわ の せき に すわった	
\\	私[わたし]は 窓側[まどがわ]の 席[せき]に
\\	次	次[つぎ]	つぎ	
\\	次はいつ会いましょうか。	次[つぎ]はいつ 会[あ]いましょうか。	つぎ は いつ あいましょう か	
\\	はいつ 会[あ]いましょうか。		
\\	喋る	喋る[しゃべる]	しゃべる	
\\	彼女はよく喋るね。	彼女[かのじょ]はよく喋る[しゃべる]ね。	かのじょ は よく しゃべる ね	
\\	彼女[かのじょ]はよく
\\	ね。		
\\	動く	動[うご]く	うごく	
\\	(が)		
\\	動かないで。	動[うご]かないで 。	うごかない で	
\\	働く	働[はたら]く	はたらく	
\\	姉は銀行で働いています。	姉[あね]は 銀行[ぎんこう]で 働[はたら]いています。	あね は ぎんこう で はたらいています。	
\\	姉[あね]は 銀行[ぎんこう]で
\\	早い	早[はや]い	はやい	
\\	まだ学校へ行くには早い時間です。	まだ 学校[がっこう]へ 行[い]くには 早[はや]い 時間[じかん]です。	まだ がっこう へ いく に は はやい じかん です	
\\	まだ 学校[がっこう]へ 行[い]くには
\\	時間[じかん]です。		
\\	速い	速[はや]い	はやい	
\\	彼は走るのが速い。	彼[かれ]は 走[はし]るのが 速[はや]い。	かれ は はしる の が はやい	
\\	彼[かれ]は 走[はし]るのが
\\	遅い	遅[おそ]い	おそい	
\\	前の車はとても遅い。	前[まえ]の 車[くるま]はとても 遅[おそ]い。	まえ の くるま は とても おそい	
\\	前[まえ]の 車[くるま]はとても
\\	始める	始[はじ]める	はじめる	
\\	を		
\\	テストを始めてください。	テストを 始[はじ]めてください。	てすと を はじめて ください	
\\	テストを
\\	ください。		
\\	始まる	始[はじ]まる	はじまる	
\\	が		
\\	新しい仕事が始まりました。	新[あたら]しい 仕事[しごと]が 始[はじ]まりました。	あたらしい しごと が はじまりました	
\\	新[あたら]しい 仕事[しごと]が
\\	終わる	終[お]わる	おわる	
\\	会議は4時に終わります。	会議[かいぎ]は 4時[よじ]に 終[お]わります。	かいぎ は よじ に おわります	
\\	会議[かいぎ]は 4時[よじ]に
\\	終わり	終[お]わり	おわり	
\\	夏休みももう終わりだ。	夏休[なつやす]みももう 終[お]わりだ。	なつやすみ も もう おわり だ	
\\	夏休[なつやす]みももう
\\	だ。		
\\	テーブル	テーブル	テーブル	
\\	新しいテーブルを買いました。	新[あたら]しいテーブルを 買[か]いました。	あたらしい てーぶる を かいました	
\\	新[あたら]しい
\\	を 買[か]いました。		
\\	去年	去年[きょねん]	きょねん	
\\	私は去年フランスへ行った。	私[わたし]は 去年[きょねん]フランスへ 行[い]った。	わたし は きょねん ふらんす へ いった	
\\	私[わたし]は
\\	フランスへ 行[い]った。		
\\	紙	紙[かみ]	かみ	
\\	紙と鉛筆はありますか。	紙[かみ]と 鉛筆[えんぴつ]はありますか。	かみ と えんぴつ は あります か	
\\	と 鉛筆[えんぴつ]はありますか。		
\\	楽しむ	楽[たの]しむ	たのしむ	
\\	今日は一人の時間を楽しみたい。	今日[きょう]は 一人[ひとり]の 時間[じかん]を 楽[たの]しみたい。	きょう は ひとり の じかん を たのしみたい	
\\	今日[きょう]は 一人[ひとり]の 時間[じかん]を
\\	楽しい	楽[たの]しい	たのしい	
\\	彼はとても楽しい人です。	彼[かれ]はとても 楽[たの]しい 人[ひと]です。	かれ は とても たのしい ひと です	
\\	彼[かれ]はとても
\\	人[ひと]です。		
\\	歌う	歌[うた]う	うたう	
\\	私たちは大きな声で歌いました。	私[わたし]たちは 大[おお]きな 声[こえ]で 歌[うた]いました。	わたしたち は おおき な こえ で うたいました	
\\	私[わたし]たちは 大[おお]きな 声[こえ]で
\\	欲しい	欲[ほ]しい	ほしい	
\\	僕は新しい靴が欲しいです。	僕[ぼく]は 新[あたら]しい 靴[くつ]が 欲[ほ]しいです。	ぼく は あたらしい くつ が ほしい です	
\\	僕[ぼく]は 新[あたら]しい 靴[くつ]が
\\	です。		
\\	色	色[いろ]	いろ	
\\	すてきな色のセーターですね。	すてきな 色[いろ]のセーターですね。	すてき な いろ の せーたー です ね	
\\	すてきな
\\	のセーターですね。		
\\	茶色	茶色[ちゃいろ]	ちゃいろ	
\\	彼女は茶色の靴を履いています。	彼女[かのじょ]は 茶色[ちゃいろ]の 靴[くつ]を 履[は]いています。	かのじょ は ちゃいろ の くつ を はいて います	
\\	彼女[かのじょ]は
\\	の 靴[くつ]を 履[は]いています。		
\\	すぐ	すぐ	すぐ	
\\	すぐ行きます。	すぐ 行[い]きます。	すぐ いきます	
\\	行[い]きます。		
\\	ゲーム	ゲーム	ゲーム	
\\	私たちはビデオゲームをした。	私[わたし]たちはビデオゲームをした。	わたしたち は びでおげーむ を した	
\\	私[わたし]たちはビデオ
\\	をした。		
\\	書き直す	書[か]き 直[なお]す	かきなおす	
\\	この書類を書き直してください。	この 書類[しょるい]を 書[か]き 直[なお]してください。	この しょるい を かきなおして ください	
\\	この 書類[しょるい]を
\\	ください。		
\\	直る	直[なお]る	なおる	
\\	クーラーはまだ直りません。	クーラーはまだ 直[なお]りません。	くーらー は まだ なおりません	
\\	クーラーはまだ
\\	線	線[せん]	せん	
\\	など)		
\\	赤い線を2本引いて下さい。	赤[あか]い 線[せん]を 2本引[に ほん ひ]いて 下[くだ]さい。	あかい せん を に ほん ひいて ください	
\\	赤[あか]い
\\	を 2本引[に ほん ひ]いて 下[くだ]さい。		
\\	曲がる	曲[ま]がる	まがる	
\\	そこを左に曲がってください。	そこを 左[ひだり]に 曲[ま]がってください。	そこ を ひだり に まがって ください	
\\	そこを 左[ひだり]に
\\	ください。		
\\	同じ	同[おな]じ	おなじ	
\\	彼の日本語のレベルは私と同じ位だ。	彼[かれ]の 日本語[にほんご]のレベルは 私[わたし]と 同[おな]じ 位[くらい]だ。	かれ の にほんご の れべる は わたし と おなじ くらい だ	
\\	彼[かれ]の 日本語[にほんご]のレベルは 私[わたし]と
\\	位[くらい]だ。		
\\	違う	違[ちが]う	ちがう	
\\	答えが違います。	答[こた]えが 違[ちが]います。	こたえ が ちがいます	
\\	答[こた]えが
\\	図書館	図書館[としょかん]	としょかん	
\\	図書館で料理の本を借りた。	図書館[としょかん]で 料理[りょうり]の 本[ほん]を 借[か]りた。	としょかん で りょうり の ほん を かりた	
\\	で 料理[りょうり]の 本[ほん]を 借[か]りた。		
\\	泊まる	泊[と]まる	とまる	
\\	今日はこのホテルに泊まります。	今日[きょう]はこのホテルに 泊[と]まります。	きょう は この ほてる に とまります	
\\	今日[きょう]はこのホテルに
\\	遊ぶ	遊[あそ]ぶ	あそぶ	
\\	子供たちが公園で遊んでいる。	子供[こども]たちが 公園[こうえん]で 遊[あそ]んでいる。	こどもたち が こうえん で あそんで いる	
\\	子供[こども]たちが 公園[こうえん]で
\\	どうして	どうして	どうして	
\\	どうして泣いているの。	どうして 泣[な]いているの。	どうして ないて いる の	
\\	泣[な]いているの。		
\\	服	服[ふく]	ふく	
\\	昨日、新しい服を買った。	昨日[きのう]、 新[あたら]しい 服[ふく]を 買[か]った。	きのう あたらしい ふく を かった	
\\	昨日[きのう]、 新[あたら]しい
\\	を 買[か]った。		
\\	お父さん	お 父[とう]さん	おとうさん	
\\	お父さんは会社員です。	お 父[とう]さんは 会社員[かいしゃいん]です。	おとうさん は かいしゃいん です	
\\	は 会社員[かいしゃいん]です。		
\\	父	父[ちち]	ちち	
\\	私は父が大好きです。	私[わたし]は 父[ちち]が 大好[だいす]きです。	わたし は ちち が だいすき です	
\\	私[わたし]は
\\	が 大好[だいす]きです。		
\\	お母さん	お 母[かあ]さん	おかあさん	
\\	お母さんによろしくお伝えください。	お 母[かあ]さんによろしくお 伝[つた]えください。	おかあさん に よろしく おつたえ ください	
\\	によろしくお 伝[つた]えください。		
\\	母	母[はは]	はは	
\\	昨日、母と話をしました。	昨日[きのう]、 母[はは]と 話[はなし]をしました。	きのう はは と はなし を しました	
\\	昨日[きのう]、
\\	と 話[はなし]をしました。		
\\	親	親[おや]	おや	
\\	親の愛はありがたい。	親[おや]の 愛[あい]は ありがたい。	おや の あい は ありがたい	
\\	の 愛[あい]は ありがたい。		
\\	姉	姉[あね]	あね	
\\	姉は大学生です。	姉[あね]は 大学生[だいがくせい]です。	あね は だいがくせい です	
\\	は 大学生[だいがくせい]です。		
\\	お姉さん	お 姉[ねえ]さん	おねえさん	
\\	昨日、あなたのお姉さんに会ったよ。	昨日[きのう]、あなたのお 姉[ねえ]さんに 会[あ]ったよ。	きのう あなた の おねえさん に あった よ	
\\	昨日[きのう]、あなたの
\\	に 会[あ]ったよ。		
\\	妹	妹[いもうと]	いもうと	
\\	私の妹は小学生です。	私[わたし]の 妹[いもうと]は 小学生[しょうがくせい]です。	わたし の いもうと は しょうがくせい です	
\\	私[わたし]の
\\	は 小学生[しょうがくせい]です。		
\\	おなか	おなか	おなか	
\\	おなかが空きました。	おなかが 空[す]きました。	おなか が すきました	
\\	が 空[す]きました。		
\\	兄	兄[あに]	あに	
\\	兄は水泳が得意です。	兄[あに]は 水泳[すいえい]が 得意[とくい]です。	あに は すいえい が とくい です	
\\	は 水泳[すいえい]が 得意[とくい]です。		
\\	お兄さん	お 兄[にい]さん	おにいさん	
\\	あなたのお兄さんは何歳?	あなたのお 兄[にい]さんは 何歳?[なんさい]	あなた の おにいさん は なんさい	
\\	あなたの
\\	は 何歳?[なんさい]		
\\	弟	弟[おとうと]	おとうと	
\\	弟は野球が好きです。	弟[おとうと]は 野球[やきゅう]が 好[す]きです。	おとうと は やきゅう が すき です	
\\	は 野球[やきゅう]が 好[す]きです。		
\\	娘	娘[むすめ]	むすめ	
\\	私の娘はアメリカにいます。	私[わたし]の 娘[むすめ]はアメリカにいます。	わたし の むすめ は あめりか に います	
\\	私[わたし]の
\\	はアメリカにいます。		
\\	息子	息子[むすこ]	むすこ	
\\	うちの息子は大学1年生です。	うちの 息子[むすこ]は 大学1年生[だいがく いちねんせい]です。	うち の むすこ は だいがく いちねんせい です	
\\	うちの
\\	は 大学1年生[だいがく いちねんせい]です。		
\\	若い	若[わか]い	わかい	
\\	彼はまだ若いです。	彼[かれ]はまだ 若[わか]いです。	かれ は まだ わかい です	
\\	彼[かれ]はまだ
\\	です。		
\\	彼女	彼女[かのじょ]	かのじょ	
\\	彼女は
\\	です。	彼女[かのじょ]は 
\\	[おーえる]です。	かのじょ は おーえる です	
\\	は 
\\	[おーえる]です。		
\\	彼	彼[かれ]	かれ	
\\	彼は私の上司です。	彼[かれ]は 私[わたし]の 上司[じょうし]です。	かれ は わたし の じょうし です	
\\	は 私[わたし]の 上司[じょうし]です。		
\\	結婚	結婚[けっこん]	けっこん	
\\	彼女は来月結婚します。	彼女[かのじょ]は 来月[らいげつ] 結婚[けっこん]します。	かのじょ は らいげつ けっこん します	
\\	彼女[かのじょ]は 来月[らいげつ]
\\	します。		
\\	ミーティング	ミーティング	ミーティング	
\\	朝9時からミーティングが始まった。	朝9時[あさ 
\\	じ]からミーティングが 始[はじ]まった。	あさ 
\\	じ から みーてぃんぐ が はじまった	
\\	朝9時[あさ 
\\	じ]から
\\	が 始[はじ]まった。		
\\	自転車	自転車[じてんしゃ]	じてんしゃ	
\\	毎日、駅まで自転車で行きます。	毎日[まいにち]、 駅[えき]まで 自転車[じてんしゃ]で 行[い]きます。	まいにち えき まで じてんしゃ で いきます	
\\	毎日[まいにち]、 駅[えき]まで
\\	で 行[い]きます。		
\\	自分	自分[じぶん]	じぶん	
\\	宿題は自分でやりなさい。	宿題[しゅくだい]は 自分[じぶん]でやりなさい。	しゅくだい は じぶん で やりなさい	
\\	宿題[しゅくだい]は
\\	でやりなさい。		
\\	ご主人	ご 主人[しゅじん]	ごしゅじん	
\\	ご主人はお元気ですか。	ご 主人[しゅじん]はお 元気[げんき]ですか。	ごしゅじん は おげんき です か	
\\	はお 元気[げんき]ですか。		
\\	答える	答[こた]える	こたえる	
\\	私の質問に答えてください。	私[わたし]の 質問[しつもん]に 答[こた]えてください。	わたし の しつもん に こたえて ください	
\\	私[わたし]の 質問[しつもん]に
\\	ください。		
\\	別	別[べつ]	べつ	
\\	別の本も見せてください。	別[べつ]の 本[ほん]も 見[み]せてください。	べつ の ほん も みせて ください	
\\	の 本[ほん]も 見[み]せてください。		
\\	病気	病気[びょうき]	びょうき	
\\	祖父が病気になった。	祖父[そふ]が 病気[びょうき]になった。	そふ が びょうき に なった	
\\	祖父[そふ]が
\\	になった。		
\\	死ぬ	死[し]ぬ	しぬ	
\\	犬が病気で死にました。	犬[いぬ]が 病気[びょうき]で 死[し]にました。	いぬ が びょうき で しにました。	
\\	犬[いぬ]が 病気[びょうき]で
\\	痛い	痛[いた]い	いたい	
\\	今日は頭が痛いです。	今日[きょう]は 頭[あたま]が 痛[いた]いです。	きょう は あたま が いたい です	
\\	今日[きょう]は 頭[あたま]が
\\	です。		
\\	酒	酒[さけ]	さけ	
\\	彼女は酒に強い。	彼女[かのじょ]は 酒[さけ]に 強[つよ]い。	かのじょ は さけ に つよい	
\\	彼女[かのじょ]は
\\	に 強[つよ]い。		
\\	つける	つける	つける	
\\	電気をつけてください。	電気[でんき]をつけてください。	でんき を つけて ください	
\\	電気[でんき]を
\\	ください。		
\\	一杯	一杯[いっぱい]	いっぱい	
\\	プールは人で一杯です。	プールは 人[ひと]で 一杯[いっぱい]です。	ぷーる は ひと で いっぱい です	
\\	プールは 人[ひと]で
\\	です。		
\\	飛ぶ	飛[と]ぶ	とぶ	
\\	鳥が飛んでいます。	鳥[とり]が 飛[と]んでいます。	とり が とんで います	
\\	鳥[とり]が
\\	飛行機	飛行機[ひこうき]	ひこうき	
\\	息子は飛行機のおもちゃが好きです。	息子[むすこ]は 飛行機[ひこうき]のおもちゃが 好[す]きです。	むすこ は ひこうき の おもちゃ が すき です	
\\	息子[むすこ]は
\\	のおもちゃが 好[す]きです。		
\\	お願い	お 願[ねが]い	おねがい	
\\	お願いがあります。	お 願[ねが]いがあります。	おねがい が あります	
\\	があります。		
\\	続く	続[つづ]く	つづく	
\\	が		
\\	工事は3月まで続きます。	工事[こうじ]は 3月[さんがつ]まで 続[つづ]きます。	こうじ は さんがつ まで つづきます	
\\	工事[こうじ]は 3月[さんがつ]まで
\\	日記	日記[にっき]	にっき	
\\	私は毎日、日記を付けています。	私[わたし]は 毎日[まいにち]、 日記[にっき]を 付[つ]けています。	わたし は まいにち にっき を つけて います	
\\	私[わたし]は 毎日[まいにち]、
\\	を 付[つ]けています。		
\\	首	首[くび]	くび	
\\	きりんの首は長い。	きりんの 首[くび]は 長[なが]い。	きりん の くび は ながい	
\\	きりんの
\\	は 長[なが]い。		
\\	頭	頭[あたま]	あたま	
\\	今朝から頭が痛い。	今朝[けさ]から 頭[あたま]が 痛[いた]い。	けさ から あたま が いたい	
\\	今朝[けさ]から
\\	が 痛[いた]い。		
\\	顔	顔[かお]	かお	
\\	彼はタオルで顔を拭きました。	彼[かれ]はタオルで 顔[かお]を 拭[ふ]きました。	かれ は たおる で かお を ふきました	
\\	彼[かれ]はタオルで
\\	を 拭[ふ]きました。		
\\	あれ	あれ	あれ	
\\	あれは何ですか。	あれは 何[なん]ですか。	あれ は なん です か	
\\	は 何[なん]ですか。		
\\	感じる	感[かん]じる	かんじる	
\\	膝に痛みを感じます。	膝[ひざ]に 痛[いた]みを 感[かん]じます。	ひざ に いたみ を かんじます	
\\	膝[ひざ]に 痛[いた]みを
\\	探す	探[さが]す	さがす	
\\	彼は郵便局を探していました。	彼[かれ]は 郵便局[ゆうびんきょく]を 探[さが]していました。	かれ は ゆうびんきょく を さがして いました	
\\	彼[かれ]は 郵便局[ゆうびんきょく]を
\\	落ちる	落[お]ちる	おちる	
\\	猿も木から落ちる。	猿[さる]も 木[き]から 落[お]ちる。	さる も き から おちる	
\\	猿[さる]も 木[き]から
\\	お手洗い	お 手洗[てあら]い	おてあらい	
\\	お手洗いはどこですか。	お 手洗[てあら]いはどこですか。	おてあらい は どこ です か	
\\	はどこですか。		
\\	冷たい	冷[つめ]たい	つめたい	
\\	冷たい飲み物をください。	冷[つめ]たい 飲[の]み 物[もの]をください。	つめたい のみもの を ください	
\\	飲[の]み 物[もの]をください。		
\\	汚い	汚[きたな]い	きたない	
\\	彼の部屋はとても汚い。	彼[かれ]の 部屋[へや]はとても 汚[きたな]い。	かれ の へや は とても きたない	
\\	彼[かれ]の 部屋[へや]はとても
\\	太い	太[ふと]い	ふとい	
\\	彼女は足が太い。	彼女[かのじょ]は 足[あし]が 太[ふと]い。	かのじょ は あし が ふとい	
\\	彼女[かのじょ]は 足[あし]が
\\	曇る	曇[くも]る	くもる	
\\	明日は昼頃から曇るでしょう。	明日[あす]は 昼頃[ひるごろ]から 曇[くも]るでしょう。	あす は ひるごろ から くもる でしょう	
\\	明日[あす]は 昼頃[ひるごろ]から
\\	でしょう。		
\\	神	神[かみ]	かみ	
\\	彼は神を信じている。	彼[かれ]は 神[かみ]を 信[しん]じている。	かれ は かみ を しんじて いる	
\\	彼[かれ]は
\\	を 信[しん]じている。		
\\	まずい	まずい	まずい	
\\	ここの料理はまずい。	ここの 料理[りょうり]はまずい。	ここ の りょうり は まずい	
\\	ここの 料理[りょうり]は
\\	建てる	建[た]てる	たてる	
\\	私たちは来年、家を建てます。	私[わたし]たちは 来年[らいねん]、 家[いえ]を 建[た]てます。	わたしたち は らいねん いえ を たてます	
\\	私[わたし]たちは 来年[らいねん]、 家[いえ]を
\\	置く	置[お]く	おく	
\\	彼はかばんをいすの上に置きました。	彼[かれ]はかばんをいすの 上[うえ]に 置[お]きました。	かれ は かばん を いす の うえ に おきました	
\\	彼[かれ]はかばんをいすの 上[うえ]に
\\	辺	辺[へん]	へん	
\\	彼はこの辺に住んでいます。	彼[かれ]はこの 辺[へん]に 住[す]んでいます。	かれ は このへん に すんで います	
\\	彼[かれ]はこの
\\	に 住[す]んでいます。		
\\	黄色い	黄色[きいろ]い	きいろい	
\\	彼女に黄色いバラを買いました。	彼女[かのじょ]に 黄色[きいろ]いバラを 買[か]いました。	かのじょ に きいろい ばら を かいました	
\\	彼女[かのじょ]に
\\	バラを 買[か]いました。		
\\	一緒に	一緒[いっしょ]に	いっしょに	
\\	一緒に帰りましょう。	一緒[いっしょ]に 帰[かえ]りましょう。	いっしょに かえりましょう	
\\	帰[かえ]りましょう。		
\\	緑	緑[みどり]	みどり	
\\	この町には緑がたくさんあります。	この 町[まち]には 緑[みどり]がたくさんあります。	この まち に は みどり が たくさん あります	
\\	この 町[まち]には
\\	がたくさんあります。		
\\	易しい	易[やさ]しい	やさしい	
\\	この問題はかなり易しいです。	この 問題[もんだい]はかなり 易[やさ]しいです。	この もんだい は かなり やさしい です	
\\	この 問題[もんだい]はかなり
\\	です。		
\\	留学生	留学生[りゅうがくせい]	りゅうがくせい	
\\	彼は留学生です。	彼[かれ]は 留学生[りゅうがくせい]です。	かれ は りゅうがくせい です	
\\	彼[かれ]は
\\	です。		
\\	戻る	戻[もど]る	もどる	
\\	今、会社に戻ります。	今[いま]、 会社[かいしゃ]に 戻[もど]ります。	いま かいしゃ に もどります	
\\	今[いま]、 会社[かいしゃ]に
\\	そば	そば	そば	
\\	そばにいて下さい。	そばにいて 下[くだ]さい。	そば に いて ください	
\\	にいて 下[くだ]さい。		
\\	起きる	起[お]きる	おきる	
\\	私は毎朝6時に起きます。	私[わたし]は 毎朝6時[まいあさ ろくじ]に 起[お]きます。	わたし は まいあさ ろくじ に おきます	
\\	私[わたし]は 毎朝6時[まいあさ ろくじ]に
\\	起こる	起[お]こる	おこる	
\\	町で大事件が起こりました。	町[まち]で 大事件[だいじけん]が 起[お]こりました。	まち で だいじけん が おこりました	
\\	町[まち]で 大事件[だいじけん]が
\\	起こす	起[お]こす	おこす	
\\	明日、6時に起こしてください。	明日[あす]、 6時[ろくじ]に 起[お]こしてください。	あす ろくじ に おこして ください	
\\	明日[あす]、 6時[ろくじ]に
\\	ください。		
\\	起きる	起[お]きる	おきる	
\\	近所で盗難事件が起きました。	近所[きんじょ]で 盗難事件[とうなん じけん]が 起[お]きました。	きんじょ で とうなん じけん が おきました	
\\	近所[きんじょ]で 盗難事件[とうなん じけん]が
\\	寝る	寝[ね]る	ねる	
\\	もう寝よう。	もう 寝[ね]よう。	もう ねよう	
\\	もう
\\	細い	細[ほそ]い	ほそい	
\\	彼女は指が細いですね。	彼女[かのじょ]は 指[ゆび]が 細[ほそ]いですね。	かのじょ は ゆび が ほそい です ね	
\\	彼女[かのじょ]は 指[ゆび]が
\\	ですね。		
\\	載せる	載[の]せる	のせる	
\\	皿にケーキを載せました。	皿[さら]にケーキを 載[の]せました。	さら に けーき を のせました。	
\\	皿[さら]にケーキを
\\	締める	締[し]める	しめる	
\\	彼はシートベルトを締めた。	彼[かれ]はシートベルトを 締[し]めた。	かれ は しーとべると を しめた	
\\	彼[かれ]はシートベルトを
\\	甘い	甘[あま]い	あまい	
\\	このイチゴは甘い。	このイチゴは 甘[あま]い。	この いちご は あまい	
\\	このイチゴは
\\	こっち	こっち	こっち	
\\	こっちに来て下さい。	こっちに 来[き]て 下[くだ]さい。	こっち に きて ください	
\\	に 来[き]て 下[くだ]さい。		
\\	辛い	辛[から]い	からい	
\\	彼は辛いものが好きです。	彼[かれ]は 辛[から]いものが 好[す]きです。	かれ は からい もの が すき です	
\\	彼[かれ]は
\\	ものが 好[す]きです。		
\\	優しい	優[やさ]しい	やさしい	
\\	彼はとても優しい人です。	彼[かれ]はとても 優[やさ]しい 人[ひと]です。	かれ は とても やさしい ひと です	
\\	彼[かれ]はとても
\\	人[ひと]です。		
\\	夫	夫[おっと]	おっと	
\\	私の夫はサラリーマンです。	私[わたし]の 夫[おっと]はサラリーマンです。	わたし の おっと は さらりーまん です	
\\	私[わたし]の
\\	はサラリーマンです。		
\\	妻	妻[つま]	つま	
\\	今日は妻の誕生日だ。	今日[きょう]は 妻[つま]の 誕生日[たんじょうび]だ。	きょう は つま の たんじょうび だ	
\\	今日[きょう]は
\\	の 誕生日[たんじょうび]だ。		
\\	誰	誰[だれ]	だれ	
\\	誰と会ってみたいですか。	誰[だれ]と 会[あ]ってみたいですか。	だれ と あって みたい です か	
\\	と 会[あ]ってみたいですか。		
\\	愛する	愛[あい]する	あいする	
\\	私は家族を愛しています。	私[わたし]は 家族[かぞく]を 愛[あい]しています。	わたし は かぞく を あいして います	
\\	私[わたし]は 家族[かぞく]を
\\	笑う	笑[わら]う	わらう	
\\	赤ん坊が笑っています。	赤[あか]ん 坊[ぼう]が 笑[わら]っています。	あかんぼう が わらって います	
\\	赤[あか]ん 坊[ぼう]が
\\	酸っぱい	酸[す]っぱい	すっぱい	
\\	このぶどうは酸っぱいです。	このぶどうは 酸[す]っぱいです。	この ぶどう は すっぱい です	
\\	このぶどうは
\\	です。		
\\	言葉	言葉[ことば]	ことば	
\\	この言葉の意味が分かりません。	この 言葉[ことば]の 意味[いみ]が 分[わ]かりません。	この ことば の いみ が わかりません	
\\	この
\\	の 意味[いみ]が 分[わ]かりません。		
\\	よう	よう	よう	
\\	彼女は眠いようです。	彼女[かのじょ]は 眠[ねむ]いようです。	かのじょ は ねむい よう です	
\\	彼女[かのじょ]は 眠[ねむ]い
\\	です。		
\\	呼ぶ	呼[よ]ぶ	よぶ	
\\	ウェイターを呼びましょう。	ウェイターを 呼[よ]びましょう。	うぇいたー を よびましょう	
\\	ウェイターを
\\	胸	胸[むね]	むね	
\\	胸に少し痛みがあります。	胸[むね]に 少[すこ]し 痛[いた]みがあります。	むね に すこし いたみ が あります	
\\	に 少[すこ]し 痛[いた]みがあります。		
\\	腰	腰[こし]	こし	
\\	昨日から腰が痛い。	昨日[きのう]から 腰[こし]が 痛[いた]い。	きのう から こし が いたい	
\\	昨日[きのう]から
\\	が 痛[いた]い。		
\\	背	背[せ]	せ	
\\	彼女は背が高い。	彼女[かのじょ]は 背[せ]が 高[たか]い。	かのじょ は せ が たかい	
\\	彼女[かのじょ]は
\\	が 高[たか]い。		
\\	片仮名	片仮名[かたかな]	かたかな	
\\	お名前を片仮名で書いてください。	お 名前[なまえ]を 片仮名[かたかな]で 書[か]いてください。	おなまえ を かたかな で かいて ください	
\\	お 名前[なまえ]を
\\	で 書[か]いてください。		
\\	平仮名	平仮名[ひらがな]	ひらがな	
\\	私は平仮名を全部読めます。	私[わたし]は 平仮名[ひらがな]を 全部読[ぜんぶ よ]めます。	わたし は ひらがな を ぜんぶ よめます	
\\	私[わたし]は
\\	を 全部読[ぜんぶ よ]めます。		
\\	悲しい	悲[かな]しい	かなしい	
\\	その映画はとても悲しかった。	その 映画[えいが]はとても 悲[かな]しかった。	その えいが は とても かなしかった	
\\	その 映画[えいが]はとても
\\	美しい	美[うつく]しい	うつくしい	
\\	この絵は美しいです。	この 絵[え]は 美[うつく]しいです。	この え は うつくしい です	
\\	この 絵[え]は
\\	です。		
\\	授業	授業[じゅぎょう]	じゅぎょう	
\\	今日は日本語の授業があります。	今日[きょう]は 日本語[にほんご]の 授業[じゅぎょう]があります。	きょう は にほんご の じゅぎょう が あります	
\\	今日[きょう]は 日本語[にほんご]の
\\	があります。		
\\	あそこ	あそこ	あそこ	
\\	あそこにバス停があります。	あそこにバス 停[てい]があります。	あそこ に ばすてい が あります	
\\	にバス 停[てい]があります。		
\\	手伝う	手伝[てつだ]う	てつだう	
\\	私が手伝いましょう。	私[わたし]が 手伝[てつだ]いましょう。	わたし が てつだいましょう	
\\	私[わたし]が
\\	狭い	狭[せま]い	せまい	
\\	私の部屋は狭いです。	私[わたし]の 部屋[へや]は 狭[せま]いです。	わたし の へや は せまい です	
\\	私[わたし]の 部屋[へや]は
\\	です。		
\\	触る	触[さわ]る	さわる	
\\	絵に触らないでください。	絵[え]に 触[さわ]らないでください。	え に さわらない で ください	
\\	絵[え]に
\\	ください。		
\\	嫌い	嫌[きら]い	きらい	
\\	私はタバコが嫌いです。	私[わたし]はタバコが 嫌[きら]いです。	わたし は たばこ が きらい です	
\\	私[わたし]はタバコが
\\	です。		
\\	浴びる	浴[あ]びる	あびる	
\\	私は朝、シャワーを浴びます。	私[わたし]は 朝[あさ]、シャワーを 浴[あ]びます。	わたし は あさ しゃわー を あびます	
\\	私[わたし]は 朝[あさ]、シャワーを
\\	渇く	渇[かわ]く	かわく	
\\	のどが渇きました。	のどが 渇[かわ]きました。	のどが かわきました	
\\	喉[のど]が
\\	髪の毛	髪[かみ]の 毛[け]	かみのけ	
\\	髪の毛が伸びたね。	髪[かみ]の 毛[け]が 伸[の]びたね。	かみのけ が のびた ね	
\\	が 伸[の]びたね。		
\\	肩	肩[かた]	かた	
\\	肩が凝りました。	肩[かた]が 凝[こ]りました。	かた が こりました	
\\	が 凝[こ]りました。		
\\	鼻	鼻[はな]	はな	
\\	鼻がかゆいです。	鼻[はな]がかゆいです。	はな が かゆい です	
\\	がかゆいです。		
\\	おいしい	おいしい	おいしい	
\\	このケーキはおいしいね。	このケーキはおいしいね。	この けーき は おいしい ね	
\\	このケーキは
\\	ね。		
\\	腕	腕[うで]	うで	
\\	彼の腕は太い。	彼[かれ]の 腕[うで]は 太[ふと]い。	かれ の うで は ふとい	
\\	彼[かれ]の
\\	は 太[ふと]い。		
\\	掛ける	掛[か]ける	かける	
\\	夫の服をハンガーに掛けた。	夫[おっと]の 服[ふく]をハンガーに 掛[か]けた。	おっと の ふく を はんがー に かけた	
\\	夫[おっと]の 服[ふく]をハンガーに
\\	僕	僕[ぼく]	ぼく	
\\	僕は学生です。	僕[ぼく]は 学生[がくせい]です。	ぼく は がくせい です	
\\	は 学生[がくせい]です。		
\\	だめ	だめ	だめ	
\\	彼はだめな男だ。	彼[かれ]はだめな 男[おとこ]だ。	かれ は だめ な おとこ だ	
\\	彼[かれ]は
\\	な 男[おとこ]だ。		
\\	大丈夫	大丈夫[だいじょうぶ]	だいじょうぶ	
\\	大丈夫ですか。	大丈夫[だいじょうぶ]ですか。	だいじょうぶ です か	
\\	ですか。		
\\	風邪	風邪[かぜ]	かぜ	
\\	私は風邪を引きました。	私[わたし]は 風邪[かぜ]を 引[ひ]きました。	わたし は かぜ を ひきました 。	
\\	私[わたし]は
\\	を 引[ひ]きました。		
\\	きれい	きれい	きれい	
\\	彼女はとてもきれいだ。	彼女[かのじょ]はとてもきれいだ。	かのじょ は とても きれい だ	
\\	彼女[かのじょ]はとても
\\	だ。		
\\	嬉しい	嬉[うれ]しい	うれしい	
\\	彼に会えて嬉しかった。	彼[かれ]に 会[あ]えて 嬉[うれ]しかった。	かれ に あえて うれしかった	
\\	彼[かれ]に 会[あ]えて
\\	なる (生る)	なる	なる	
\\	柿の木に実がたくさんなっています。	柿[かき]の 木[き]に 実[み]がたくさんなっています。	かきのき に み が たくさん なって います	
\\	柿[かき]の 木[き]に 実[み]がたくさん
\\	ため	ため	ため	
\\	これは医者のためのサイトです。	これは 医者[いしゃ]のためのサイトです。	これ は いしゃ の ため の さいと です	
\\	これは 医者[いしゃ]の
\\	のサイトです。		
\\	より	より	より	
\\	これから、より一層努力します。	これから、より 一層努力[いっそう どりょく]します。	これから より いっそう どりょく します	
\\	これから、
\\	一層努力[いっそう どりょく]します。		
\\	七	七[しち]	しち	
\\	そのグループのメンバーは全部で七人だ。	そのグループのメンバーは 全部[ぜんぶ]で 七[しち] 人[にん]だ。	その ぐるーぷ の めんばー は ぜんぶ で しちにん だ	
\\	そのグループのメンバーは 全部[ぜんぶ]で
\\	人[にん]だ。		
\\	九	九[く]	く	
\\	私は九月に行く予定です。	私[わたし]は 九[く] 月[がつ]に 行[い]く 予定[よてい]です。	わたし は くがつ に いく よてい です	
\\	私[わたし]は
\\	月[がつ]に 行[い]く 予定[よてい]です。		
\\	億	億[おく]	おく	
\\	世界人口は65億人だ。	世界人口[せかい じんこう]は 65[ろくじゅうご] 億[おく] 人[にん]だ。	せかい じんこう は ろくじゅうごおくにん だ	
\\	世界人口[せかい じんこう]は 65[ろくじゅうご]
\\	人[にん]だ。		
\\	寺	寺[てら]	てら	
\\	あそこに古いお寺があります。	あそこに 古[ふる]いお 寺[てら]があります。	あそこ に ふるい おてら が あります	
\\	あそこに 古[ふる]いお
\\	があります。		
\\	ドル	ドル	ドル	
\\	この服は300ドルしました。	この 服[ふく]は 300[さんびゃく]ドルしました。	この ふく は さんびゃくどる しました	
\\	この 服[ふく]は 300[さんびゃく]
\\	しました。		
\\	日	日[ひ]	ひ	
\\	夏は日が長い。	夏[なつ]は 日[ひ]が 長[なが]い。	なつ は ひ が ながい	
\\	夏[なつ]は
\\	が 長[なが]い。		
\\	火	火[ひ]	ひ	
\\	火を消して。	火[ひ]を 消[け]して。	ひ を けして	
\\	を 消[け]して。		
\\	木	木[き]	き	
\\	台風で木が倒れた。	台風[たいふう]で 木[き]が 倒[たお]れた。	たいふう で き が たおれた	
\\	台風[たいふう]で
\\	が 倒[たお]れた。		
\\	金	金[きん]	きん	
\\	彼女は金のネックレスをしています。	彼女[かのじょ]は 金[きん]のネックレスをしています。	かのじょ は きん の ねっくれす を して います	
\\	彼女[かのじょ]は
\\	のネックレスをしています。		
\\	システム	システム	システム	
\\	全てのシステムが停止した。	全[すべ]てのシステムが 停止[ていし]した。	すべて の しすてむ が ていし した	
\\	全[すべ]ての
\\	が 停止[ていし]した。		
\\	十分	十分[じゅうぶん]	じゅうぶん	
\\	お金はまだ十分あります。	お 金[かね]はまだ 十分[じゅうぶん]あります。	おかね は まだ じゅうぶん あります	
\\	お 金[かね]はまだ
\\	あります。		
\\	分	分[ぶん]	ぶん	
\\	このケーキはあなたの分です。	このケーキはあなたの 分[ぶん]です。	この けーき は あなた の ぶん です	
\\	このケーキはあなたの
\\	です。		
\\	分ける	分[わ]ける	わける	
\\	お菓子をみんなで分けました。	お 菓子[かし]をみんなで 分[わ]けました。	おかし を みんな で わけました	
\\	お 菓子[かし]をみんなで
\\	分かれる	分[わ]かれる	わかれる	
\\	グループの中で意見が分かれました。	グループの 中[なか]で 意見[いけん]が 分[わ]かれました。	ぐるーぷ の なか で いけん が わかれました	
\\	グループの 中[なか]で 意見[いけん]が
\\	コンピュータ	コンピュータ	コンピュータ	
\\	新しいコンピュータを買った。	新[あたら]しいコンピュータを 買[か]った。	あたらしい こんぴゅーた を かった	
\\	新[あたら]しい
\\	を 買[か]った。		
\\	何か	何[なに]か	なにか	
\\	道に何か落ちています。	道[みち]に 何[なに]か 落[お]ちています。	みち に なにか おちて います	
\\	道[みち]に
\\	落[お]ちています。		
\\	何	何[なん]	なん	
\\	質問は何ですか。	質問[しつもん]は 何[なん]ですか。	しつもん は なん です か	
\\	質問[しつもん]は
\\	ですか。		
\\	まず	まず	まず	
\\	帰ったらまず手を洗いましょう。	帰[かえ]ったらまず 手[て]を 洗[あら]いましょう。	かえったら まず て を あらいましょう 。	
\\	帰[かえ]ったら
\\	手[て]を 洗[あら]いましょう。		
\\	先月	先月[せんげつ]	せんげつ	
\\	先月、友達の結婚式があった。	先月[せんげつ]、 友達[ともだち]の 結婚式[けっこんしき]があった。	せんげつ ともだち の けっこんしき が あった	
\\	、 友達[ともだち]の 結婚式[けっこんしき]があった。		
\\	やる	やる	やる	
\\	犬にえさをやった。	犬[いぬ]にえさをやった。	いぬ に えさ を やった	
\\	犬[いぬ]にえさを
\\	先週	先週[せんしゅう]	せんしゅう	
\\	先週は海に行った。	先週[せんしゅう]は 海[うみ]に 行[い]った。	せんしゅう は うみ に いった	
\\	は 海[うみ]に 行[い]った。		
\\	今まで	今[いま]まで	いままで	
\\	今までどこにいたのですか。	今[いま]までどこにいたのですか。	いままで どこ に いた の です か	
\\	どこにいたのですか。		
\\	来月	来月[らいげつ]	らいげつ	
\\	来月から大学生になります。	来月[らいげつ]から 大学生[だいがくせい]になります。	らいげつ から だいがくせい に なります	
\\	から 大学生[だいがくせい]になります。		
\\	来週	来週[らいしゅう]	らいしゅう	
\\	続きは来週やりましょう。	続[つづ]きは 来週[らいしゅう]やりましょう。	つづき は らいしゅう やりましょう	
\\	続[つづ]きは
\\	やりましょう。		
\\	ロボット	ロボット	ロボット	
\\	彼はロボットの研究をしています。	彼[かれ]はロボットの 研究[けんきゅう]をしています。	かれ は ろぼっと の けんきゅう を しています 。	
\\	彼[かれ]は
\\	の 研究[けんきゅう]をしています。		
\\	行う	行[おこな]う	おこなう	
\\	その会社は来月、キャンペーンを行う。	その 会社[かいしゃ]は 来月[らいげつ]、キャンペーンを 行[おこな]う。	その かいしゃ は らいげつ きゃんぺーん を おこなう	
\\	その 会社[かいしゃ]は 来月[らいげつ]、キャンペーンを
\\	行き	行[い]き	いき	
\\	行きは新幹線で行った。	行[い]きは 新幹線[しんかんせん]で 行[い]った。	いき は しんかんせん で いった	
\\	は 新幹線[しんかんせん]で 行[い]った。		
\\	行き	行[ゆ]き	ゆき	
\\	東京行きの列車に乗った。	東京[とうきょう] 行[ゆ]きの 列車[れっしゃ]に 乗[の]った。	とうきょうゆき の れっしゃ に のった	
\\	東京[とうきょう]
\\	の 列車[れっしゃ]に 乗[の]った。		
\\	帰り	帰[かえ]り	かえり	
\\	仕事の帰りにビールを飲んだ。	仕事[しごと]の 帰[かえ]りにビールを 飲[の]んだ。	しごと の かえり に びーる を のんだ	
\\	仕事[しごと]の
\\	にビールを 飲[の]んだ。		
\\	ほとんど	ほとんど	ほとんど	
\\	お金がほとんどありません。	お 金[かね]がほとんどありません。	おかね が ほとんど ありません	
\\	お 金[かね]が
\\	ありません。		
\\	大きさ	大[おお]きさ	おおきさ	
\\	この大きさの封筒が欲しいのですが。	この 大[おお]きさの 封筒[ふうとう]が 欲[ほ]しいのですが。	この おおきさ の ふうとう が ほしい の です が	
\\	この
\\	の 封筒[ふうとう]が 欲[ほ]しいのですが。		
\\	大分	大分[だいぶ]	だいぶ	
\\	大分ピアノが上手くなりました。	大分[だいぶ]ピアノが 上手[うま]くなりました。	だいぶ ぴあの が うまく なりました	
\\	ピアノが 上手[うま]くなりました。		
\\	中	中[なか]	なか	
\\	財布は引き出しの中にあります。	財布[さいふ]は 引[ひ]き 出[だ]しの 中[なか]にあります。	さいふ は ひきだし の なか に あります	
\\	財布[さいふ]は 引[ひ]き 出[だ]しの
\\	にあります。		
\\	少年	少年[しょうねん]	しょうねん	
\\	少年たちがサッカーをしている。	少年[しょうねん]たちがサッカーをしている。	しょうねんたち が さっかー を して いる	
\\	たちがサッカーをしている。		
\\	パソコン	パソコン	パソコン	
\\	彼はパソコンを2台持っています。	彼[かれ]はパソコンを 2台持[にだい も]っています。	かれ は ぱそこん を にだい もって います	
\\	彼[かれ]は
\\	を 2台持[にだい も]っています。		
\\	少しも	少[すこ]しも	すこしも	
\\	全然	あなたは少しも悪くない。	あなたは 少[すこ]しも 悪[わる]くない。	あなた は すこしも わるく ない	
\\	あなたは
\\	悪[わる]くない。		
\\	少々	少々[しょうしょう]	しょうしょう	
\\	塩を少々入れてください。	塩[しお]を 少々[しょうしょう] 入[い]れてください。	しお を しょうしょう いれて ください	
\\	塩[しお]を
\\	入[い]れてください。		
\\	多く	多[おお]く	おおく	
\\	毎年多くの人が海外へ旅行する。	毎年[まいとし] 多[おお]くの 人[ひと]が 海外[かいがい]へ 旅行[りょこう]する。	まいとし おおく の ひと が かいがい へ りょこう する	
\\	毎年[まいとし]
\\	の 人[ひと]が 海外[かいがい]へ 旅行[りょこう]する。		
\\	上がる	上[あ]がる	あがる	
\\	私たちは2階に上がった。	私[わたし]たちは 2階[にかい]に 上[あ]がった。	わたしたち は にかい に あがった	
\\	私[わたし]たちは 2階[にかい]に
\\	もう	もう	もう	
\\	コーヒーをもう一杯ください。	コーヒーをもう 一杯[いっぱい]ください。	こーひー を もう いっぱい ください	
\\	コーヒーを
\\	一杯[いっぱい]ください。		
\\	上がる	上[あ]がる	あがる	
\\	彼は人前だと上がってしまう。	彼[かれ]は 人前[ひとまえ]だと 上[あ]がってしまう。	かれ は ひとまえ だ と あがって しまう	
\\	彼[かれ]は 人前[ひとまえ]だと
\\	年上	年上[としうえ]	としうえ	
\\	彼は私より年上です。	彼[かれ]は 私[わたし]より 年上[としうえ]です。	かれ は わたし より としうえ です	
\\	彼[かれ]は 私[わたし]より
\\	です。		
\\	上り	上[のぼ]り	のぼり	
\\	これは上り電車です。	これは 上[のぼ]り 電車[でんしゃ]です。	これ は のぼり でんしゃ です	
\\	これは
\\	電車[でんしゃ]です。		
\\	下げる	下[さ]げる	さげる	
\\	少し音量を下げてください。	少[すこ]し 音量[おんりょう]を 下[さ]げてください。	すこし おんりょう を さげて ください	
\\	少[すこ]し 音量[おんりょう]を
\\	ください。		
\\	プログラム	プログラム	プログラム	
\\	受付でプログラムを受け取った。	受付[うけつけ]でプログラムを 受[う]け 取[と]った。	うけつけ で ぷろぐらむ を うけとった	
\\	受付[うけつけ]で
\\	を 受[う]け 取[と]った。		
\\	下がる	下[さ]がる	さがる	
\\	やっと熱が下がった。	やっと 熱[ねつ]が 下[さ]がった。	やっと ねつ が さがった	
\\	やっと 熱[ねつ]が
\\	下りる	下[お]りる	おりる	
\\	そこの階段を下りてください。	そこの 階段[かいだん]を 下[お]りてください。	そこ の かいだん を おりて ください	
\\	そこの 階段[かいだん]を
\\	ください。		
\\	下ろす	下[お]ろす	おろす	
\\	棚からその箱を下ろしてください。	棚[たな]からその 箱[はこ]を 下[お]ろしてください。	たな から その はこ を おろして ください	
\\	棚[たな]からその 箱[はこ]を
\\	ください。		
\\	下り	下[くだ]り	くだり	
\\	もうすぐ下りの電車が発車します。	もうすぐ 下[くだ]りの 電車[でんしゃ]が 発車[はっしゃ]します。	もうすぐ くだり の でんしゃ が はっしゃ します	
\\	もうすぐ
\\	の 電車[でんしゃ]が 発車[はっしゃ]します。		
\\	よく	よく	よく	
\\	よく答えが分かりましたね。	よく 答[こた]えが 分[わ]かりましたね。	よく こたえ が わかりました ね	
\\	答[こた]えが 分[わ]かりましたね。		
\\	年下	年下[としした]	としした	
\\	彼は奥さんより年下です。	彼[かれ]は 奥[おく]さんより 年下[としした]です。	かれ は おくさん より としした です	
\\	彼[かれ]は 奥[おく]さんより
\\	です。		
\\	下る	下[くだ]る	くだる	
\\	小さな船が川を下っています。	小[ちい]さな 船[ふね]が 川[かわ]を 下[くだ]っています。	ちいさ な ふね が かわ を くだって います	
\\	小[ちい]さな 船[ふね]が 川[かわ]を
\\	一方	一方[いっぽう]	いっぽう	
\\	ここは一方通行です。	ここは 一方[いっぽう] 通行[つうこう]です。	ここ は いっぽう つうこう です	
\\	ここは
\\	通行[つうこう]です。		
\\	方	方[かた]	かた	
\\	次の方、どうぞ。	次[つぎ]の 方[かた]、どうぞ。	つぎ の かた どうぞ	
\\	次[つぎ]の
\\	、どうぞ。		
\\	まま	まま	まま	
\\	電気がついたままですよ。	電気[でんき]がついたままですよ。	でんき が ついた まま です よ	
\\	電気[でんき]がついた
\\	ですよ。		
\\	二人	二人[ふたり]	ふたり	
\\	今日は妻と二人で食事をします。	今日[きょう]は 妻[つま]と 二人[ふたり]で 食事[しょくじ]をします。	きょう は つま と ふたり で しょくじ を します	
\\	今日[きょう]は 妻[つま]と
\\	で 食事[しょくじ]をします。		
\\	大人しい	大人[おとな]しい	おとなしい	
\\	私の彼女はとても大人しいです。	私[わたし]の 彼女[かのじょ]はとても 大人[おとな]しいです。	わたし の かのじょ は とても おとなしい です	
\\	私[わたし]の 彼女[かのじょ]はとても
\\	です。		
\\	人々	人々[ひとびと]	ひとびと	
\\	あの村の人々はとても親切です。	あの 村[むら]の 人々[ひとびと]はとても 親切[しんせつ]です。	あの むら の ひとびと は とても しんせつ です	
\\	あの 村[むら]の
\\	はとても 親切[しんせつ]です。		
\\	一人で	一人[ひとり]で	ひとりで	
\\	今日は一人で映画を見ます。	今日[きょう]は 一人[ひとり]で 映画[えいが]を 見[み]ます。	きょう は ひとりで えいが を みます	
\\	今日[きょう]は
\\	映画[えいが]を 見[み]ます。		
\\	テレビ	テレビ	テレビ	
\\	私はテレビをあまり見ません。	私[わたし]はテレビをあまり 見[み]ません。	わたし は てれび を あまり みません	
\\	私[わたし]は
\\	をあまり 見[み]ません。		
\\	外人	外人[がいじん]	がいじん	
\\	この町には外人が少ない。	この 町[まち]には 外人[がいじん]が 少[すく]ない。	この まち に は がいじん が すくない	
\\	この 町[まち]には
\\	が 少[すく]ない。		
\\	外	外[そと]	そと	
\\	外は暑いよ。	外[そと]は 暑[あつ]いよ。	そと は あついよ	
\\	は 暑[あつ]いよ。		
\\	休日	休日[きゅうじつ]	きゅうじつ	
\\	休日は家でよくテレビを見ます。	休日[きゅうじつ]は 家[いえ]でよくテレビを 見[み]ます。	きゅうじつ は いえ で よく てれび を みます	
\\	は 家[いえ]でよくテレビを 見[み]ます。		
\\	休み	休[やす]み	やすみ	
\\	木曜日は仕事が休みです。	木曜日[もくようび]は 仕事[しごと]が 休[やす]みです。	もくようび は しごと が やすみ です	
\\	木曜日[もくようび]は 仕事[しごと]が
\\	です。		
\\	ソフト	ソフト	ソフト	
\\	このソフトで日本語を勉強することができます。	このソフトで 日本語[にほんご]を 勉強[べんきょう]することができます。	この そふと で にほんご を べんきょう する こと が できます	
\\	この
\\	で 日本語[にほんご]を 勉強[べんきょう]することができます。		
\\	力	力[ちから]	ちから	
\\	お相撲さんは、みんな力持ちだ。	お 相撲[すもう]さんは、みんな 力[ちから] 持[も]ちだ。	おすもうさん は みんな ちからもち だ	
\\	お 相撲[すもう]さんは、みんな
\\	持[も]ちだ。		
\\	協力	協力[きょうりょく]	きょうりょく	
\\	このプロジェクトにはみんなの協力が必要です。	このプロジェクトにはみんなの 協力[きょうりょく]が 必要[ひつよう]です。	この ぷろじぇくと に は みんな の きょうりょく が ひつよう です	
\\	このプロジェクトにはみんなの
\\	が 必要[ひつよう]です。		
\\	人口	人口[じんこう]	じんこう	
\\	その国の人口はどのくらいですか。	その 国[くに]の 人口[じんこう]はどのくらいですか。	その くに の じんこう は どの くらい です か	
\\	その 国[くに]の
\\	はどのくらいですか。		
\\	出口	出口[でぐち]	でぐち	
\\	出口はあそこです。	出口[でぐち]はあそこです。	でぐち は あそこ です	
\\	はあそこです。		
\\	ただ	ただ	ただ	
\\	この温泉はただです。	この 温泉[おんせん]はただです。	この おんせん は ただ です	
\\	この 温泉[おんせん]は
\\	です。		
\\	入り口	入[い]り 口[ぐち]	いりぐち	
\\	入り口は向こうです。	入[い]り 口[ぐち]は 向[む]こうです。	いりぐち は むこう です	
\\	は 向[む]こうです。		
\\	右手	右手[みぎて]	みぎて	
\\	私は右手で字を書きます。	私[わたし]は 右手[みぎて]で 字[じ]を 書[か]きます。	わたし は みぎて で じ を かきます	
\\	私[わたし]は
\\	で 字[じ]を 書[か]きます。		
\\	左手	左手[ひだりて]	ひだりて	
\\	彼女は左手で字を書く。	彼女[かのじょ]は 左手[ひだりて]で 字[じ]を 書[か]く。	かのじょ は ひだりて で じ を かく	
\\	彼女[かのじょ]は
\\	で 字[じ]を 書[か]く。		
\\	下手	下手[へた]	へた	
\\	私は歌が下手だ。	私[わたし]は 歌[うた]が 下手[へた]だ。	わたし は うた が へた だ	
\\	私[わたし]は 歌[うた]が
\\	だ。		
\\	これら	これら	これら	
\\	今日はこれらの問題について話し合います。	今日[きょう]はこれらの 問題[もんだい]について 話[はな]し 合[あ]います。	きょう は これら の もんだい に ついて はなしあいます	
\\	今日[きょう]は
\\	の 問題[もんだい]について 話[はな]し 合[あ]います。		
\\	足りる	足[た]りる	たりる	
\\	お金が足りなくて買えなかった。	お 金[かね]が 足[た]りなくて 買[か]えなかった。	おかね が たりなくて かえなかった	
\\	お 金[かね]が
\\	買[か]えなかった。		
\\	足す	足[た]す	たす	
\\	母は味噌汁に水を足した。	母[はは]は 味噌汁[みそしる]に 水[みず]を 足[た]した。	はは は みそしる に みず を たした	
\\	母[はは]は 味噌汁[みそしる]に 水[みず]を
\\	山	山[やま]	やま	
\\	山の空気はきれいだ。	山[やま]の 空気[くうき]はきれいだ。	やま の くうき は きれい だ	
\\	の 空気[くうき]はきれいだ。		
\\	川	川[かわ]	かわ	
\\	小さな川を渡りました。	小[ちい]さな 川[かわ]を 渡[わた]りました。	ちいさ な かわ を わたりました	
\\	小[ちい]さな
\\	を 渡[わた]りました。		
\\	いずれ	いずれ	いずれ	
\\	いずれまたお会いしましょう。	いずれまたお 会[あ]いしましょう。	いずれ また おあい しましょう	
\\	またお 会[あ]いしましょう。		
\\	空く	空[あ]く	あく	
\\	後ろの席が空いています。	後[うし]ろの 席[せき]が 空[あ]いています。	うしろ の せき が あいて います	
\\	後[うし]ろの 席[せき]が
\\	空手	空手[からて]	からて	
\\	彼は空手を習っています。	彼[かれ]は 空手[からて]を 習[なら]っています。	かれ は からて を ならって います	
\\	彼[かれ]は
\\	を 習[なら]っています。		
\\	空	空[そら]	そら	
\\	空が真っ青です。	空[そら]が 真[ま]っ 青[さお]です。	そら が まっさお です	
\\	が 真[ま]っ 青[さお]です。		
\\	海外	海外[かいがい]	かいがい	
\\	彼は海外での生活が長いです。	彼[かれ]は 海外[かいがい]での 生活[せいかつ]が 長[なが]いです。	かれ は かいがい で の せいかつ が ながい です	
\\	彼[かれ]は
\\	での 生活[せいかつ]が 長[なが]いです。		
\\	あまり	あまり	あまり	
\\	このビールはあまり美味しくありません。	このビールはあまり 美味[おい]しくありません。	この びーる は あまり おいしく ありません	
\\	このビールは
\\	美味[おい]しくありません。		
\\	海	海[うみ]	うみ	
\\	海は広くて大きい。	海[うみ]は 広[ひろ]くて 大[おお]きい。	うみ は ひろく て おおきい	
\\	は 広[ひろ]くて 大[おお]きい。		
\\	毎日	毎日[まいにち]	まいにち	
\\	私たちは毎日散歩をします。	私[わたし]たちは 毎日[まいにち] 散歩[さんぽ]をします。	わたしたち は まいにち さんぽ を します	
\\	私[わたし]たちは
\\	散歩[さんぽ]をします。		
\\	毎年	毎年[まいとし]	まいとし	
\\	私は毎年、海外旅行に行きます。	私[わたし]は 毎年[まいとし]、 海外旅行[かいがい りょこう]に 行[い]きます。	わたし は まいとし かいがい りょこう に いきます	
\\	私[わたし]は
\\	、 海外旅行[かいがい りょこう]に 行[い]きます。		
\\	毎年	毎年[まいねん]	まいねん	
\\	毎年給料が上がる。	毎年[まいねん] 給料[きゅうりょう]が 上[あ]がる。	まいねん きゅうりょう が あがる	
\\	給料[きゅうりょう]が 上[あ]がる。		
\\	なお	なお	なお	
\\	なお、雨の場合は中止です。	なお、 雨[あめ]の 場合[ばあい]は 中止[ちゅうし]です。	なお あめ の ばあい は ちゅうし です	
\\	、 雨[あめ]の 場合[ばあい]は 中止[ちゅうし]です。		
\\	毎週	毎週[まいしゅう]	まいしゅう	
\\	私は毎週母に電話をします。	私[わたし]は 毎週[まいしゅう] 母[はは]に 電話[でんわ]をします。	わたし は まいしゅう はは に でんわ を します	
\\	私[わたし]は
\\	母[はは]に 電話[でんわ]をします。		
\\	毎月	毎月[まいつき]	まいつき	
\\	私は毎月貯金をしています。	私[わたし]は 毎月[まいつき] 貯金[ちょきん]をしています。	わたし は まいつき ちょきん を して います	
\\	私[わたし]は
\\	貯金[ちょきん]をしています。		
\\	石	石[いし]	いし	
\\	私は石につまづいた。	私[わたし]は 石[いし]につまづいた。	わたし は いし に つまづいた	
\\	私[わたし]は
\\	につまづいた。		
\\	田んぼ	田[た]んぼ	たんぼ	
\\	この辺は田んぼがたくさんあります。	この 辺[へん]は 田[た]んぼがたくさんあります。	このへん は たんぼ が たくさん あります	
\\	この 辺[へん]は
\\	がたくさんあります。		
\\	ほぼ	ほぼ	ほぼ	
\\	仕事がほぼ終わりました。	仕事[しごと]がほぼ 終[お]わりました。	しごと が ほぼ おわりました	
\\	仕事[しごと]が
\\	終[お]わりました。		
\\	花	花[はな]	はな	
\\	きれいな花が咲きました。	きれいな 花[はな]が 咲[さ]きました。	きれい な はな が さきました	
\\	きれいな
\\	が 咲[さ]きました。		
\\	林	林[はやし]	はやし	
\\	私たちは林の中に入っていった。	私[わたし]たちは 林[はやし]の 中[なか]に 入[はい]っていった。	わたしたち は はやし の なか に はいって いった	
\\	私[わたし]たちは
\\	の 中[なか]に 入[はい]っていった。		
\\	森	森[もり]	もり	
\\	私は森を歩くのが好きです。	私[わたし]は 森[もり]を 歩[ある]くのが 好[す]きです。	わたし は もり を あるく の が すき です	
\\	私[わたし]は
\\	を 歩[ある]くのが 好[す]きです。		
\\	子	子[こ]	こ	
\\	その子は日本語が分からない。	その 子[こ]は 日本語[にほんご]が 分[わ]からない。	その こ は にほんご が わからない	
\\	その
\\	は 日本語[にほんご]が 分[わ]からない。		
\\	サービス	サービス	サービス	
\\	この店はサービスがいい。	この 店[みせ]はサービスがいい。	この みせ は さーびす が いい 。	
\\	この 店[みせ]は
\\	がいい。		
\\	女の子	女[おんな]の 子[こ]	おんなのこ	
\\	あの女の子を知っていますか。	あの 女[おんな]の 子[こ]を 知[し]っていますか。	あの おんなのこ を しって います か	
\\	あの
\\	を 知[し]っていますか。		
\\	男の子	男[おとこ]の 子[こ]	おとこのこ	
\\	男の子たちがサッカーをしている。	男[おとこ]の 子[こ]たちがサッカーをしている。	おとこのこたち が さっかー を して いる	
\\	たちがサッカーをしている。		
\\	私たち	私[わたし]たち	わたしたち	
\\	私たちは来月結婚します。	私[わたし]たちは 来月結婚[らいげつ けっこん]します。	わたしたち は らいげつ けっこん します 。	
\\	は 来月結婚[らいげつ けっこん]します。		
\\	達する	達[たっ]する	たっする	
\\	気温は35度に達した。	気温[きおん]は 35度[さんじゅうごど]に 達[たっ]した。	きおん は さんじゅうごど に たっした	
\\	気温[きおん]は 35度[さんじゅうごど]に
\\	グループ	グループ	グループ	
\\	店に学生のグループが来た。	店[みせ]に 学生[がくせい]のグループが 来[き]た。	みせ に がくせい の ぐるーぷ が きた	
\\	店[みせ]に 学生[がくせい]の
\\	が 来[き]た。		
\\	家	家[いえ]	いえ	
\\	ここが私の家です。	ここが 私[わたし]の 家[いえ]です。	ここ が わたし の いえ です	
\\	ここが 私[わたし]の
\\	です。		
\\	家内	家内[かない]	かない	
\\	家内は九州出身です。	家内[かない]は 九州出身[きゅうしゅう しゅっしん]です。	かない は きゅうしゅう しゅっしん です	
\\	は 九州出身[きゅうしゅう しゅっしん]です。		
\\	客	客[きゃく]	きゃく	
\\	その店は若い客が多いです。	その 店[みせ]は 若[わか]い 客[きゃく]が 多[おお]いです。	その みせ は わかい きゃく が おおい です	
\\	その 店[みせ]は 若[わか]い
\\	が 多[おお]いです。		
\\	空気	空気[くうき]	くうき	
\\	ここは空気がきれいです。	ここは 空気[くうき]がきれいです。	ここ は くうき が きれい です	
\\	ここは
\\	がきれいです。		
\\	ホテル	ホテル	ホテル	
\\	今ホテルに着きました。	今[いま]ホテルに 着[つ]きました。	いま ほてる に つきました	
\\	今[いま]
\\	に 着[つ]きました。		
\\	気に入る	気[き]に 入[い]る	きにいる	
\\	新しい靴がとても気に入りました。	新[あたら]しい 靴[くつ]がとても 気[き]に 入[い]りました。	あたらしい くつ が とても きにいりました	
\\	新[あたら]しい 靴[くつ]がとても
\\	人気	人気[にんき]	にんき	
\\	このバンドはとても人気があるよ。	このバンドはとても 人気[にんき]があるよ。	この ばんど は とても にんき が ある よ	
\\	このバンドはとても
\\	があるよ。		
\\	雨	雨[あめ]	あめ	
\\	雨が降っています。	雨[あめ]が 降[ふ]っています。	あめ が ふって います	
\\	が 降[ふ]っています。		
\\	雪	雪[ゆき]	ゆき	
\\	クリスマスに雪が降りました。	クリスマスに 雪[ゆき]が 降[ふ]りました。	くりすます に ゆき が ふりました	
\\	クリスマスに
\\	が 降[ふ]りました。		
\\	まとめる	まとめる	まとめる	
\\	彼女は荷物をまとめて出て行った。	彼女[かのじょ]は 荷物[にもつ]をまとめて 出[で]て 行[い]った。	かのじょ は にもつ を まとめて でて いった	
\\	彼女[かのじょ]は 荷物[にもつ]を
\\	出[で]て 行[い]った。		
\\	青い	青[あお]い	あおい	
\\	ここの海はとても青い。	ここの 海[うみ]はとても 青[あお]い。	ここ の うみ は とても あおい	
\\	ここの 海[うみ]はとても
\\	青	青[あお]	あお	
\\	私の好きな色は青です。	私[わたし]の 好[す]きな 色[いろ]は 青[あお]です。	わたし の すき な いろ は あお です	
\\	私[わたし]の 好[す]きな 色[いろ]は
\\	です。		
\\	晴れ	晴[は]れ	はれ	
\\	明日の天気は晴れです。	明日[あす]の 天気[てんき]は 晴[は]れです。	あす の てんき は はれ です	
\\	明日[あす]の 天気[てんき]は
\\	です。		
\\	明らか	明[あき]らか	あきらか	
\\	明らかに彼が悪い。	明[あき]らかに 彼[かれ]が 悪[わる]い。	あきらか に かれ が わるい	
\\	に 彼[かれ]が 悪[わる]い。		
\\	やはり	やはり	やはり	
\\	彼はやはり遅刻しました。	彼[かれ]はやはり 遅刻[ちこく]しました。	かれ は やはり ちこく しました	
\\	彼[かれ]は
\\	遅刻[ちこく]しました。		
\\	明るい	明[あか]るい	あかるい	
\\	彼女は明るい性格です。	彼女[かのじょ]は 明[あか]るい 性格[せいかく]です。	かのじょ は あかるい せいかく です	
\\	彼女[かのじょ]は
\\	性格[せいかく]です。		
\\	明日	明日[あした]	あした	
\\	明日、会社を休みます。	明日[あした]、 会社[かいしゃ]を 休[やす]みます。	あした かいしゃ を やすみます	
\\	、 会社[かいしゃ]を 休[やす]みます。		
\\	暗い	暗[くら]い	くらい	
\\	東の空が暗いです。	東[ひがし]の 空[そら]が 暗[くら]いです。	ひがし の そら が くらい です	
\\	東[ひがし]の 空[そら]が
\\	です。		
\\	昨年	昨年[さくねん]	さくねん	
\\	昨年は地震が多い年でした。	昨年[さくねん]は 地震[じしん]が 多[おお]い 年[とし]でした。	さくねん は じしん が おおい とし でした	
\\	は 地震[じしん]が 多[おお]い 年[とし]でした。		
\\	はっきり	はっきり	はっきり	
\\	今日は山がはっきり見える。	今日[きょう]は 山[やま]がはっきり 見[み]える。	きょう は やま が はっきり みえる	
\\	今日[きょう]は 山[やま]が
\\	見[み]える。		
\\	一昨年	一昨年[おととし]	おととし	
\\	一昨年初めて京都へ旅行しました。	一昨年[おととし] 初[はじ]めて 京都[きょうと]へ 旅行[りょこう]しました。	おととし はじめて きょうと へ りょこう しました	
\\	初[はじ]めて 京都[きょうと]へ 旅行[りょこう]しました。		
\\	一昨日	一昨日[おととい]	おととい	
\\	一昨日彼から電話がありました。	一昨日[おととい] 彼[かれ]から 電話[でんわ]がありました。	おととい かれ から でんわ が ありました	
\\	彼[かれ]から 電話[でんわ]がありました。		
\\	東	東[ひがし]	ひがし	
\\	東の空が暗いです。	東[ひがし]の 空[そら]が 暗[くら]いです。	ひがし の そら が くらい です	
\\	の 空[そら]が 暗[くら]いです。		
\\	西	西[にし]	にし	
\\	太陽は西に沈みます。	太陽[たいよう]は 西[にし]に 沈[しず]みます。	たいよう は にし に しずみます	
\\	太陽[たいよう]は
\\	に 沈[しず]みます。		
\\	つまり	つまり	つまり	
\\	つまり、あなたは何も知らないのですね。	つまり、あなたは 何[なに]も 知[し]らないのですね。	つまり あなた は なに も しらない の です ね	
\\	、あなたは 何[なに]も 知[し]らないのですね。		
\\	南	南[みなみ]	みなみ	
\\	私の家は町の南にあります。	私[わたし]の 家[いえ]は 町[まち]の 南[みなみ]にあります。	わたし の いえ は まち の みなみ に あります	
\\	私[わたし]の 家[いえ]は 町[まち]の
\\	にあります。		
\\	北	北[きた]	きた	
\\	ロシアは日本の北にあります。	ロシアは 日本[にっぽん]の 北[きた]にあります。	ろしあ は にっぽん の きた に あります	
\\	ロシアは 日本[にっぽん]の
\\	にあります。		
\\	方向	方向[ほうこう]	ほうこう	
\\	あの人たちは皆、同じ方向を見ている。	あの 人[ひと]たちは 皆[みな]、 同[おな]じ 方向[ほうこう]を 見[み]ている。	あの ひとたち は みな おなじ ほうこう を みて いる	
\\	あの 人[ひと]たちは 皆[みな]、 同[おな]じ
\\	を 見[み]ている。		
\\	向かう	向[む]かう	むかう	
\\	今、会社に向かっています。	今[いま]、 会社[かいしゃ]に 向[む]かっています。	いま かいしゃ に むかって います	
\\	今[いま]、 会社[かいしゃ]に
\\	ビル	ビル	ビル	
\\	(ローマ字)		
\\	私の会社はあのビルの8階です。	私[わたし]の 会社[かいしゃ]はあのビルの 8階[はちかい]です。	わたし の かいしゃ は あの びる の はちかい です	
\\	私[わたし]の 会社[かいしゃ]はあの
\\	の 8階[はちかい]です。		
\\	向こう	向[む]こう	むこう	
\\	友達は向こうにいます。	友達[ともだち]は 向[む]こうにいます。	ともだち は むこう に います	
\\	友達[ともだち]は
\\	にいます。		
\\	向く	向[む]く	むく	
\\	こっちを向いてください。	こっちを 向[む]いてください。	こっち を むいて ください	
\\	こっちを
\\	ください。		
\\	開く	開[あ]く	あく	
\\	が		
\\	電車のドアが開きました。	電車[でんしゃ]のドアが 開[あ]きました。	でんしゃ の どあ が あきました	
\\	電車[でんしゃ]のドアが
\\	聞こえる	聞[き]こえる	きこえる	
\\	隣の部屋からテレビの音が聞こえる。	隣[となり]の 部屋[へや]からテレビの 音[おと]が 聞[き]こえる。	となり の へや から てれび の おと が きこえる	
\\	隣[となり]の 部屋[へや]からテレビの 音[おと]が
\\	もちろん	もちろん	もちろん	
\\	もちろん一緒に行きます。	もちろん 一緒[いっしょ]に 行[い]きます。	もちろん いっしょ に いきます	
\\	一緒[いっしょ]に 行[い]きます。		
\\	年間	年間[ねんかん]	ねんかん	
\\	年間5万人がここを訪れます。	年間[ねんかん] 5万人[ごまんにん]がここを 訪[おとず]れます。	ねんかん ごまんにん が ここ を おとずれます	
\\	5万人[ごまんにん]がここを 訪[おとず]れます。		
\\	この間	この 間[あいだ]	このあいだ	
\\	この間彼女に会った。	この 間[あいだ] 彼女[かのじょ]に 会[あ]った。	このあいだ かのじょ に あった	
\\	彼女[かのじょ]に 会[あ]った。		
\\	間	間[あいだ]	あいだ	
\\	雲の間から月が出た。	雲[くも]の 間[あいだ]から 月[つき]が 出[で]た。	くも の あいだ から つき が でた	
\\	雲[くも]の
\\	から 月[つき]が 出[で]た。		
\\	人間	人間[にんげん]	にんげん	
\\	人間の心は複雑です。	人間[にんげん]の 心[こころ]は 複雑[ふくざつ]です。	にんげん の こころ は ふくざつ です	
\\	の 心[こころ]は 複雑[ふくざつ]です。		
\\	かつて	かつて	かつて	
\\	かつて私が学生だった頃のことです。	かつて 私[わたし]が 学生[がくせい]だった 頃[ころ]のことです。	かつて わたし が がくせい だった ころ の こと です	
\\	私[わたし]が 学生[がくせい]だった 頃[ころ]のことです。		
\\	高さ	高[たか]さ	たかさ	
\\	富士山の高さは3,776メートルです。	富士山[ふじさん]の 高[たか]さは 
\\	776[さんぜんななひゃく-ななじゅうろく]メートルです。	ふじさん の たかさ は さんぜんななひゃく-ななじゅうろくめーとる です	
\\	富士山[ふじさん]の
\\	は 
\\	776[さんぜんななひゃく-ななじゅうろく]メートルです。		
\\	最大	最大[さいだい]	さいだい	
\\	これは世界最大の船です。	これは 世界[せかい] 最大[さいだい]の 船[ふね]です。	これ は せかい さいだい の ふね です	
\\	これは 世界[せかい]
\\	の 船[ふね]です。		
\\	初めて	初[はじ]めて	はじめて	
\\	東京に来るのは初めてです。	東京[とうきょう]に 来[く]るのは 初[はじ]めてです。	とうきょう に くる の は はじめて です 。	
\\	東京[とうきょう]に 来[く]るのは
\\	です。		
\\	最初	最初[さいしょ]	さいしょ	
\\	5ページの最初を見てください。	5[ご]ページの 最初[さいしょ]を 見[み]てください。	ごぺーじ の さいしょ を みて ください	
\\	5[ご]ページの
\\	を 見[み]てください。		
\\	スポーツ	スポーツ	スポーツ	
\\	あなたは何かスポーツをしていますか。	あなたは 何[なに]かスポーツをしていますか。	あなた は なにか すぽーつ を して います か	
\\	あなたは 何[なに]か
\\	をしていますか。		
\\	初め	初[はじ]め	はじめ	
\\	初めは上手くできませんでした。	初[はじ]めは 上手[うま]くできませんでした。	はじめ は うまく できません でした 。	
\\	は 上手[うま]くできませんでした。		
\\	今後	今後[こんご]	こんご	
\\	今後ともよろしくお願いします。	今後[こんご]ともよろしくお 願[ねが]いします。	こんご とも よろしく おねがい します	
\\	ともよろしくお 願[ねが]いします。		
\\	後[のち]	後[のち]	のち	
\\	後[のち]に彼は総理大臣になりました。	後[のち]に 彼[かれ]は 総理大臣[そうり だいじん]になりました。	のち に かれ は そうり だいじん に なりました	
\\	に 彼[かれ]は 総理大臣[そうり だいじん]になりました。		
\\	最後	最後[さいご]	さいご	
\\	今日が夏休み最後の日だ。	今日[きょう]が 夏休[なつやす]み 最後[さいご]の 日[ひ]だ。	きょう が なつやすみ さいご の ひ だ	
\\	今日[きょう]が 夏休[なつやす]み
\\	の 日[ひ]だ。		
\\	なぜ	なぜ	なぜ	
\\	なぜ来なかったの。	なぜ 来[こ]なかったの。	なぜ こなかった の	
\\	来[こ]なかったの。		
\\	明後日	明後日[あさって]	あさって	
\\	明後日は休日です。	明後日[あさって]は 休日[きゅうじつ]です。	あさって は きゅうじつ です 。	
\\	は 休日[きゅうじつ]です。		
\\	牛	牛[うし]	うし	
\\	牛が草を食べています。	牛[うし]が 草[くさ]を 食[た]べています。	うし が くさ を たべて います	
\\	が 草[くさ]を 食[た]べています。		
\\	半分	半分[はんぶん]	はんぶん	
\\	お菓子を友達に半分あげた。	お 菓子[かし]を 友達[ともだち]に 半分[はんぶん]あげた。	おかし を ともだち に はんぶん あげた	
\\	お 菓子[かし]を 友達[ともだち]に
\\	あげた。		
\\	半年	半年[はんとし]	はんとし	
\\	日本に来て半年になります。	日本[にっぽん]に 来[き]て 半年[はんとし]になります。	にっぽん に きて はんとし に なります	
\\	日本[にっぽん]に 来[き]て
\\	になります。		
\\	そのまま	そのまま	そのまま	
\\	そのままお待ちください。	そのままお 待[ま]ちください。	そのまま おまち ください	
\\	お 待[ま]ちください。		
\\	半月	半月[はんつき]	はんつき	
\\	半月前に日本に来ました。	半月[はんつき] 前[まえ]に 日本[にほん]に 来[き]ました。	はんつきまえ に にほん に きました	
\\	前[まえ]に 日本[にほん]に 来[き]ました。		
\\	半日	半日[はんにち]	はんにち	
\\	今日は半日だけ仕事だ。	今日[きょう]は 半日[はんにち]だけ 仕事[しごと]だ。	きょう は はんにち だけ しごと だ	
\\	今日[きょう]は
\\	だけ 仕事[しごと]だ。		
\\	毎朝	毎朝[まいあさ]	まいあさ	
\\	私は毎朝ジョギングをします。	私[わたし]は 毎朝[まいあさ]ジョギングをします。	わたし は まいあさ じょぎんぐ を します	
\\	私[わたし]は
\\	ジョギングをします。		
\\	今朝	今朝[けさ]	けさ	
\\	今朝から頭が痛い。	今朝[けさ]から 頭[あたま]が 痛[いた]い。	けさ から あたま が いたい	
\\	から 頭[あたま]が 痛[いた]い。		
\\	もし	もし	もし	
\\	もし雨が降ったら、行きません。	もし 雨[あめ]が 降[ふ]ったら、 行[い]きません。	もし あめ が ふったら いきません	
\\	雨[あめ]が 降[ふ]ったら、 行[い]きません。		
\\	昼休み	昼休[ひるやす]み	ひるやすみ	
\\	昼休みに公園に行った。	昼休[ひるやす]みに 公園[こうえん]に 行[い]った。	ひるやすみ に こうえん に いった	
\\	に 公園[こうえん]に 行[い]った。		
\\	昼前	昼前[ひるまえ]	ひるまえ	
\\	昼前に会議があった。	昼前[ひるまえ]に 会議[かいぎ]があった。	ひるまえ に かいぎ が あった	
\\	に 会議[かいぎ]があった。		
\\	昼間	昼間[ひるま]	ひるま	
\\	昼間は仕事で忙しいです。	昼間[ひるま]は 仕事[しごと]で 忙[いそが]しいです。	ひるま は しごと で いそがしい です	
\\	は 仕事[しごと]で 忙[いそが]しいです。		
\\	毎晩	毎晩[まいばん]	まいばん	
\\	姉は毎晩日記を書いています。	姉[あね]は 毎晩[まいばん] 日記[にっき]を 書[か]いています。	あね は まいばん にっき を かいて います	
\\	姉[あね]は
\\	日記[にっき]を 書[か]いています。		
\\	つもり	つもり	つもり	
\\	明日からタバコを止めるつもりです。	明日[あした]からタバコを 止[や]めるつもりです。	あした から たばこ を やめる つもり です	
\\	明日[あした]からタバコを 止[や]める
\\	です。		
\\	今夜	今夜[こんや]	こんや	
\\	今夜は月がとてもきれいです。	今夜[こんや]は 月[つき]がとてもきれいです。	こんや は つき が とても きれい です	
\\	は 月[つき]がとてもきれいです。		
\\	昨夜	昨夜[ゆうべ]	ゆうべ	
\\	昨夜、流れ星を見ました。	昨夜[ゆうべ]、 流[なが]れ 星[ぼし]を 見[み]ました。	ゆうべ ながれぼし を みました	
\\	、 流[なが]れ 星[ぼし]を 見[み]ました。		
\\	夜中	夜中[よなか]	よなか	
\\	夜中に電話がありました。	夜中[よなか]に 電話[でんわ]がありました。	よなか に でんわ が ありました	
\\	に 電話[でんわ]がありました。		
\\	夕方	夕方[ゆうがた]	ゆうがた	
\\	夕方そちらに着きます。	夕方[ゆうがた]そちらに 着[つ]きます。	ゆうがた そちら に つきます	
\\	そちらに 着[つ]きます。		
\\	やっと	やっと	やっと	
\\	やっと仕事が終わりました。	やっと 仕事[しごと]が 終[お]わりました。	やっと しごと が おわりました	
\\	仕事[しごと]が 終[お]わりました。		
\\	昼食	昼食[ちゅうしょく]	ちゅうしょく	
\\	昼食に寿司を食べました。	昼食[ちゅうしょく]に 寿司[すし]を 食[た]べました。	ちゅうしょく に すし を たべました	
\\	に 寿司[すし]を 食[た]べました。		
\\	朝食	朝食[ちょうしょく]	ちょうしょく	
\\	朝食に納豆を食べました。	朝食[ちょうしょく]に 納豆[なっとう]を 食[た]べました。	ちょうしょく に なっとう を たべました	
\\	に 納豆[なっとう]を 食[た]べました。		
\\	夕食	夕食[ゆうしょく]	ゆうしょく	
\\	夕食は7時です。	夕食[ゆうしょく]は 7時[しちじ]です。	ゆうしょく は しちじ です	
\\	は 7時[しちじ]です。		
\\	夕飯	夕飯[ゆうはん]	ゆうはん	
\\	(かたい)		
\\	夕飯は寿司でした。	夕飯[ゆうはん]は 寿司[すし]でした。	ゆうはん は すし でした	
\\	は 寿司[すし]でした。		
\\	ニュース	ニュース	ニュース	
\\	夜のニュースを見ましたか。	夜[よる]のニュースを 見[み]ましたか。	よる の にゅーす を みました か	
\\	夜[よる]の
\\	を 見[み]ましたか。		
\\	見方	見方[みかた]	みかた	
\\	彼に対する見方が変わりました。	彼[かれ]に 対[たい]する 見方[みかた]が 変[か]わりました。	かれ に たいする みかた が かわりました	
\\	彼[かれ]に 対[たい]する
\\	が 変[か]わりました。		
\\	花見	花見[はなみ]	はなみ	
\\	友達と花見をしました。	友達[ともだち]と 花見[はなみ]をしました。	ともだち と はなみ を しました	
\\	友達[ともだち]と
\\	をしました。		
\\	言い方	言[い]い 方[かた]	いいかた	
\\	そんな言い方をしてはいけません。	そんな 言[い]い 方[かた]をしてはいけません。	そんな いいかた を して は いけません	
\\	そんな
\\	をしてはいけません。		
\\	話	話[はなし]	はなし	
\\	あなたの話は面白いね。	あなたの 話[はなし]は 面白[おもしろ]いね。	あなた の はなし は おもしろい ね	
\\	あなたの
\\	は 面白[おもしろ]いね。		
\\	ずっと	ずっと	ずっと	
\\	父は休みの日はずっとテレビを見ている。	父[ちち]は 休[やす]みの 日[ひ]はずっとテレビを 見[み]ている。	ちち は やすみ の ひ は ずっと てれび を みて いる	
\\	父[ちち]は 休[やす]みの 日[ひ]は
\\	テレビを 見[み]ている。		
\\	読み	読[よ]み	よみ	
\\	母に読み書きを習いました。	母[はは]に 読[よ]み 書[か]きを 習[なら]いました。	はは に よみかき を ならいました	
\\	母[はは]に
\\	書[か]きを 習[なら]いました。		
\\	読み方	読[よ]み 方[かた]	よみかた	
\\	この漢字の読み方を教えてください。	この 漢字[かんじ]の 読[よ]み 方[かた]を 教[おし]えてください。	この かんじ の よみかた を おしえて ください	
\\	この 漢字[かんじ]の
\\	を 教[おし]えてください。		
\\	語る	語[かた]る	かたる	
\\	今夜は大いに語りましょう。	今夜[こんや]は 大[おお]いに 語[かた]りましょう。	こんや は おおいに かたりましょう	
\\	今夜[こんや]は 大[おお]いに
\\	言語	言語[げんご]	げんご	
\\	彼はアジアの言語を研究している。	彼[かれ]はアジアの 言語[げんご]を 研究[けんきゅう]している。	かれ は あじあ の げんご を けんきゅう して いる	
\\	彼[かれ]はアジアの
\\	を 研究[けんきゅう]している。		
\\	ビデオ	ビデオ	ビデオ	
\\	私はその番組をビデオに撮った。	私[わたし]はその 番組[ばんぐみ]をビデオに 撮[と]った。	わたし は その ばんぐみ を びでお に とった	
\\	私[わたし]はその 番組[ばんぐみ]を
\\	に 撮[と]った。		
\\	英語	英語[えいご]	えいご	
\\	あなたは英語が話せますか。	あなたは 英語[えいご]が 話[はな]せますか。	あなた は えいご が はなせます か	
\\	あなたは
\\	が 話[はな]せますか。		
\\	文字	文字[もじ]	もじ	
\\	壁に文字が書いてあった。	壁[かべ]に 文字[もじ]が 書[か]いてあった。	かべ に もじ が かいて あった	
\\	壁[かべ]に
\\	が 書[か]いてあった。		
\\	ローマ字	ローマ 字[じ]	ろーまじ	
\\	ローマ字で名前を書いてください。	ローマ 字[じ]で 名前[なまえ]を 書[か]いてください。	ろーまじ で なまえ を かいて ください	
\\	で 名前[なまえ]を 書[か]いてください。		
\\	字	字[じ]	じ	
\\	もっと大きく字を書いてください。	もっと 大[おお]きく 字[じ]を 書[か]いてください。	もっと おおきく じ を かいて ください	
\\	もっと 大[おお]きく
\\	を 書[か]いてください。		
\\	マンション	マンション	マンション	
\\	彼はマンションに住んでいます。	彼[かれ]はマンションに 住[す]んでいます。	かれ は まんしょん に すんで います	
\\	彼[かれ]は
\\	に 住[す]んでいます。		
\\	書き方	書[か]き 方[かた]	かきかた	
\\	彼はその漢字の書き方が分からない。	彼[かれ]はその 漢字[かんじ]の 書[か]き 方[かた]が 分[わ]からない。	かれ は その かんじ の かきかた が わからない	
\\	彼[かれ]はその 漢字[かんじ]の
\\	が 分[わ]からない。		
\\	覚める	覚[さ]める	さめる	
\\	(目が)		
\\	今朝は6時に目が覚めた。	今朝[けさ]は 6時[ろくじ]に 目[め]が 覚[さ]めた。	けさ は ろくじ に め が さめた	
\\	今朝[けさ]は 6時[ろくじ]に 目[め]が
\\	覚ます	覚[さ]ます	さます	
\\	目を		
\\	子供が目を覚ました。	子供[こども]が 目[め]を 覚[さ]ました。	こども が め を さました	
\\	子供[こども]が 目[め]を
\\	大会	大会[たいかい]	たいかい	
\\	夏には川辺で花火大会があります。	夏[なつ]には 川辺[かわべ]で 花火[はなび] 大会[たいかい]があります。	なつ に は かわべ で はなびたいかい が あります	
\\	夏[なつ]には 川辺[かわべ]で 花火[はなび]
\\	があります。		
\\	しばらく	しばらく	しばらく	
\\	そこでしばらく休んでいます。	そこでしばらく 休[やす]んでいます。	そこで しばらく やすんで います	
\\	そこで
\\	休[やす]んでいます。		
\\	会話	会話[かいわ]	かいわ	
\\	親子の会話は大切です。	親子[おやこ]の 会話[かいわ]は 大切[たいせつ]です。	おやこ の かいわ は たいせつ です	
\\	親子[おやこ]の
\\	は 大切[たいせつ]です。		
\\	話し合う	話[はな]し 合[あ]う	はなしあう	
\\	私たちはよく話し合いました。	私[わたし]たちはよく 話[はな]し 合[あ]いました。	わたしたち は よく はなしあいました	
\\	私[わたし]たちはよく
\\	合う	合[あ]う	あう	
\\	この靴は私の足に合っている。	この 靴[くつ]は 私[わたし]の 足[あし]に 合[あ]っている。	この くつ は わたし の あし に あって いる	
\\	この 靴[くつ]は 私[わたし]の 足[あし]に
\\	間に合う	間[ま]に 合[あ]う	まにあう	
\\	なくても〜 
\\	授業に間に合いました。	授業[じゅぎょう]に 間[ま]に 合[あ]いました。	じゅぎょう に まにあいました	
\\	授業[じゅぎょう]に
\\	ガス	ガス	ガス	
\\	地震でガスが止まった。	地震[じしん]でガスが 止[と]まった。	じしん で がす が とまった	
\\	地震[じしん]で
\\	が 止[と]まった。		
\\	会社	会社[かいしゃ]	かいしゃ	
\\	彼は小さな会社に勤めています。	彼[かれ]は 小[ちい]さな 会社[かいしゃ]に 勤[つと]めています。	かれ は ちいさ な かいしゃ に つとめて います	
\\	彼[かれ]は 小[ちい]さな
\\	に 勤[つと]めています。		
\\	社会	社会[しゃかい]	しゃかい	
\\	これは大きな社会問題になっている。	これは 大[おお]きな 社会[しゃかい] 問題[もんだい]になっている。	これ は おおき な しゃかい もんだい に なって いる	
\\	これは 大[おお]きな
\\	問題[もんだい]になっている。		
\\	社員	社員[しゃいん]	しゃいん	
\\	彼は優秀な社員です。	彼[かれ]は 優秀[ゆうしゅう]な 社員[しゃいん]です。	かれ は ゆうしゅう な しゃいん です	
\\	彼[かれ]は 優秀[ゆうしゅう]な
\\	です。		
\\	仕方	仕方[しかた]	しかた	
\\	ファイルのダウンロードの仕方が分かりません。	ファイルのダウンロードの 仕方[しかた]が 分[わ]かりません。	ふぁいる の だうんろーど の しかた が わかりません	
\\	ファイルのダウンロードの
\\	が 分[わ]かりません。		
\\	うまい	うまい	うまい	
\\	彼は野球がうまい。	彼[かれ]は 野球[やきゅう]がうまい。	かれ は やきゅう が うまい	
\\	彼[かれ]は 野球[やきゅう]が
\\	食事	食事[しょくじ]	しょくじ	
\\	今日は上司と食事をする。	今日[きょう]は 上司[じょうし]と 食事[しょくじ]をする。	きょう は じょうし と しょくじ を する	
\\	今日[きょう]は 上司[じょうし]と
\\	をする。		
\\	火事	火事[かじ]	かじ	
\\	火事です。119番に電話してください。	火事[かじ]です。 119番[ひゃくじゅうきゅうばん]に 電話[でんわ]してください。	かじ です ひゃくじゅうきゅうばん に でんわ して ください	
\\	です。 119番[ひゃくじゅうきゅうばん]に 電話[でんわ]してください。		
\\	大事	大事[だいじ]	だいじ	
\\	お体をお大事に。	お 体[からだ]をお 大事[だいじ]に。	おからだ を おだいじ に	
\\	お 体[からだ]をお
\\	に。		
\\	事故	事故[じこ]	じこ	
\\	彼は事故で怪我をしました。	彼[かれ]は 事故[じこ]で 怪我[けが]をしました。	かれ は じこ で けが を しました	
\\	彼[かれ]は
\\	で 怪我[けが]をしました。		
\\	サラリーマン	サラリーマン	サラリーマン	
\\	父はサラリーマンです。	父[ちち]はサラリーマンです。	ちち は さらりーまん です	
\\	父[ちち]は
\\	です。		
\\	工事	工事[こうじ]	こうじ	
\\	工事の音がうるさい。	工事[こうじ]の 音[おと]がうるさい。	こうじ の おと が うるさい	
\\	の 音[おと]がうるさい。		
\\	工場	工場[こうじょう]	こうじょう	
\\	彼は食品工場で働いています。	彼[かれ]は 食品[しょくひん] 工場[こうじょう]で 働[はたら]いています。	かれ は しょくひん こうじょう で はたらいて います	
\\	彼[かれ]は 食品[しょくひん]
\\	で 働[はたら]いています。		
\\	電話	電話[でんわ]	でんわ	
\\	あとで電話します。	あとで 電話[でんわ]します。	あとで でんわ します	
\\	あとで
\\	します。		
\\	電気	電気[でんき]	でんき	
\\	電気をつけてください。	電気[でんき]をつけてください。	でんき をつけてください	
\\	をつけてください。		
\\	バス	バス	バス	
\\	バスで行こう。	バスで 行[い]こう。	ばす で いこう	
\\	で 行[い]こう。		
\\	車	車[くるま]	くるま	
\\	この道は車が多い。	この 道[みち]は 車[くるま]が 多[おお]い。	この みち は くるま が おおい	
\\	この 道[みち]は
\\	が 多[おお]い。		
\\	駅員	駅員[えきいん]	えきいん	
\\	駅員に聞きましょう。	駅員[えきいん]に 聞[き]きましょう。	えきいん に ききましょう	
\\	に 聞[き]きましょう。		
\\	通り	通[とお]り	とおり	
\\	この通りはにぎやかですね。	この 通[とお]りはにぎやかですね。	この とおり は にぎやか です ね	
\\	この
\\	はにぎやかですね。		
\\	通る	通[とお]る	とおる	
\\	毎日、この道を通ります。	毎日[まいにち]、この 道[みち]を 通[とお]ります。	まいにち この みち を とおります	
\\	毎日[まいにち]、この 道[みち]を
\\	クラス	クラス	クラス	
\\	(ローマ字)		
\\	この学校は1クラス30人です。	この 学校[がっこう]は1クラス30 人[にん]です。	この がっこう は 
\\	くらす 
\\	にん です 。	
\\	この 学校[がっこう]は1
\\	人[にん]です。		
\\	通う	通[かよ]う	かよう	
\\	私はジムに通っています。	私[わたし]はジムに 通[かよ]っています。	わたし は じむ に かよって います	
\\	私[わたし]はジムに
\\	交通事故	交通事故[こうつうじこ]	こうつうじこ	
\\	彼は交通事故を起こした。	彼[かれ]は 交通事故[こうつうじこ]を 起[お]こした。	かれ は こうつうじこ を おこした	
\\	彼[かれ]は
\\	を 起[お]こした。		
\\	水道	水道[すいどう]	すいどう	
\\	東京は水道の水が不味い。	東京[とうきょう]は 水道[すいどう]の 水[みず]が 不味[まず]い。	とうきょう は すいどう の みず が まずい	
\\	東京[とうきょう]は
\\	の 水[みず]が 不味[まず]い。		
\\	車道	車道[しゃどう]	しゃどう	
\\	車道の工事が始まりました。	車道[しゃどう]の 工事[こうじ]が 始[はじ]まりました。	しゃどう の こうじ が はじまりました	
\\	の 工事[こうじ]が 始[はじ]まりました。		
\\	トラック	トラック	トラック	
\\	トラックを運転できますか。	トラックを 運転[うんてん]できますか。	とらっく を うんてん できます か	
\\	を 運転[うんてん]できますか。		
\\	道路	道路[どうろ]	どうろ	
\\	この道路は3年前にできました。	この 道路[どうろ]は 3年前[さんねんまえ]にできました。	この どうろ は さんねんまえ に できました	
\\	この
\\	は 3年前[さんねんまえ]にできました。		
\\	土地	土地[とち]	とち	
\\	ここは父の土地です。	ここは 父[ちち]の 土地[とち]です。	ここ は ちち の とち です	
\\	ここは 父[ちち]の
\\	です。		
\\	地図	地図[ちず]	ちず	
\\	地図を見て来てください。	地図[ちず]を 見[み]て 来[き]てください。	ちず を みて きて ください	
\\	を 見[み]て 来[き]てください。		
\\	他	他[た]	た	
\\	留学生はアジア人が多く、その他は3割です。	留学生[りゅうがくせい]はアジア 人[じん]が 多[おお]く、その 他[た]は 3割[さんわり]です。	りゅうがくせい は あじあじん が おおく そのた は さんわり です	
\\	留学生[りゅうがくせい]はアジア 人[じん]が 多[おお]く、その
\\	は 3割[さんわり]です。		
\\	パーティー	パーティー	パーティー	
\\	明日、うちでパーティーを開きます。	明日[あした]、うちでパーティーを 開[ひら]きます。	あした うち で ぱーてぃー を ひらきます	
\\	明日[あした]、うちで
\\	を 開[ひら]きます。		
\\	止める	止[と]める	とめる	
\\	車を止めて。	車[くるま]を 止[と]めて。	くるま を とめて	
\\	車[くるま]を
\\	止まる	止[と]まる	とまる	
\\	今朝、事故で電車が止まりました。	今朝[けさ]、 事故[じこ]で 電車[でんしゃ]が 止[と]まりました。	けさ じこ で でんしゃ が とまりました	
\\	今朝[けさ]、 事故[じこ]で 電車[でんしゃ]が
\\	止む	止[や]む	やむ	
\\	雨が止みました。	雨[あめ]が 止[や]みました。	あめ が やみました	
\\	雨[あめ]が
\\	歩道	歩道[ほどう]	ほどう	
\\	歩道を歩きましょう。	歩道[ほどう]を 歩[ある]きましょう。	ほどう を あるきましょう	
\\	を 歩[ある]きましょう。		
\\	うまい	うまい	うまい	
\\	うまい寿司屋を見つけたよ。	うまい 寿司屋[すしや]を 見[み]つけたよ。	うまい すしや を みつけた よ	
\\	寿司屋[すしや]を 見[み]つけたよ。		
\\	渡す	渡[わた]す	わたす	
\\	彼に手紙を渡しました。	彼[かれ]に 手紙[てがみ]を 渡[わた]しました。	かれ に てがみ を わたしました	
\\	彼[かれ]に 手紙[てがみ]を
\\	渡る	渡[わた]る	わたる	
\\	私たちは歩いて橋を渡った。	私[わたし]たちは 歩[ある]いて 橋[はし]を 渡[わた]った。	わたしたち は あるいて はし を わたった	
\\	私[わたし]たちは 歩[ある]いて 橋[はし]を
\\	年度	年度[ねんど]	ねんど	
\\	売り上げは年度によって違います。	売[う]り 上[あ]げは 年度[ねんど]によって 違[ちが]います。	うりあげ は ねんど に よって ちがいます	
\\	売[う]り 上[あ]げは
\\	によって 違[ちが]います。		
\\	今度	今度[こんど]	こんど	
\\	今度はどこに行きたいですか。	今度[こんど]はどこに 行[い]きたいですか。	こんど は どこ に いきたい です か	
\\	はどこに 行[い]きたいですか。		
\\	ガラス	ガラス	ガラス	
\\	クリスタルガラスの花瓶を買いました。	クリスタルガラスの 花瓶[かびん]を 買[か]いました。	くりすたるがらす の かびん を かいました	
\\	クリスタル
\\	の 花瓶[かびん]を 買[か]いました。		
\\	何度	何度[なんど]	なんど	
\\	あの店には何度も行きました。	あの 店[みせ]には 何度[なんど]も 行[い]きました。	あの みせ に は なんど も いきました	
\\	あの 店[みせ]には
\\	も 行[い]きました。		
\\	最近	最近[さいきん]	さいきん	
\\	それは最近話題の本ですね。	それは 最近[さいきん] 話題[わだい]の 本[ほん]ですね。	それ は さいきん わだい の ほん です ね	
\\	それは
\\	話題[わだい]の 本[ほん]ですね。		
\\	遠く	遠[とお]く	とおく	
\\	遠くに船が見えます。	遠[とお]くに 船[ふね]が 見[み]えます。	とおく に ふね が みえます	
\\	に 船[ふね]が 見[み]えます。		
\\	社長	社長[しゃちょう]	しゃちょう	
\\	おばは小さな会社の社長です。	おばは 小[ちい]さな 会社[かいしゃ]の 社長[しゃちょう]です。	おば は ちいさ な かいしゃ の しゃちょう です	
\\	おばは 小[ちい]さな 会社[かいしゃ]の
\\	です。		
\\	コース	コース	コース	
\\	私は日本語コースを取っています。	私[わたし]は 日本語[にほんご]コースを 取[と]っています。	わたし は にほんご こーす を とって います	
\\	私[わたし]は 日本語[にほんご]
\\	を 取[と]っています。		
\\	会長	会長[かいちょう]	かいちょう	
\\	初めに会長が挨拶した。	初[はじ]めに 会長[かいちょう]が 挨拶[あいさつ]した。	はじめ に かいちょう が あいさつ した	
\\	初[はじ]めに
\\	が 挨拶[あいさつ]した。		
\\	長さ	長[なが]さ	ながさ	
\\	このケーブルの長さは1メートルです。	このケーブルの 長[なが]さは 1[いち]メートルです。	この けーぶる の ながさ は いちめーとる です	
\\	このケーブルの
\\	は 1[いち]メートルです。		
\\	長男	長男[ちょうなん]	ちょうなん	
\\	彼の長男は8才です。	彼[かれ]の 長男[ちょうなん]は 8才[はっさい]です。	かれ の ちょうなん は はっさい です	
\\	彼[かれ]の
\\	は 8才[はっさい]です。		
\\	長女	長女[ちょうじょ]	ちょうじょ	
\\	うちの長女は10歳です。	うちの 長女[ちょうじょ]は 10歳[じゅっさい]です。	うち の ちょうじょ は じゅっさい です	
\\	うちの
\\	は 10歳[じゅっさい]です。		
\\	アパート	アパート	アパート	
\\	彼はアパートに住んでいます。	彼[かれ]はアパートに 住[す]んでいます。	かれ は あぱーと に すんで います	
\\	彼[かれ]は
\\	に 住[す]んでいます。		
\\	広がる	広[ひろ]がる	ひろがる	
\\	が		
\\	留学してから私の世界が広がった。	留学[りゅうがく]してから 私[わたし]の 世界[せかい]が 広[ひろ]がった。	りゅうがく して から わたし の せかい が ひろがった	
\\	留学[りゅうがく]してから 私[わたし]の 世界[せかい]が
\\	広さ	広[ひろ]さ	ひろさ	
\\	その家の広さはどれぐらいですか。	その 家[いえ]の 広[ひろ]さはどれぐらいですか。	その いえ の ひろさ は どれ ぐらい です か	
\\	その 家[いえ]の
\\	はどれ 位[くらい]ですか。		
\\	全体	全体[ぜんたい]	ぜんたい	
\\	全体の80パーセントが完成しました。	全体[ぜんたい]の 80[はちじゅっ]パーセントが 完成[かんせい]しました。	ぜんたい の はちじゅっぱーせんと が かんせい しました	
\\	の 80[はちじゅっ]パーセントが 完成[かんせい]しました。		
\\	全く	全[まった]く	まったく	
\\	私には全く分かりません。	私[わたし]には 全[まった]く 分[わ]かりません。	わたし に は まったく わかりません	
\\	私[わたし]には
\\	分[わ]かりません。		
\\	レコード	レコード	レコード	
\\	ジャズのレコードをかけました。	ジャズのレコードをかけました。	じゃず の れこーど を かけました	
\\	ジャズの
\\	をかけました。		
\\	安全	安全[あんぜん]	あんぜん	
\\	安全が第一です。	安全[あんぜん]が 第一[だいいち]です。	あんぜん が だいいち です	
\\	が 第一[だいいち]です。		
\\	一部	一部[いちぶ]	いちぶ	
\\	計画を一部変更しましょう。	計画[けいかく]を 一部[いちぶ] 変更[へんこう]しましょう。	けいかく を いちぶ へんこう しましょう	
\\	計画[けいかく]を
\\	変更[へんこう]しましょう。		
\\	部分	部分[ぶぶん]	ぶぶん	
\\	この部分は問題ないです。	この 部分[ぶぶん]は 問題[もんだい]ないです。	この ぶぶん は もんだい ない です	
\\	この
\\	は 問題[もんだい]ないです。		
\\	国内	国内[こくない]	こくない	
\\	この携帯電話が使えるのは国内だけです。	この 携帯電話[けいたい でんわ]が 使[つか]えるのは 国内[こくない]だけです。	この けいたい でんわ が つかえる の は こくない だけ です	
\\	この 携帯電話[けいたい でんわ]が 使[つか]えるのは
\\	だけです。		
\\	どんどん	どんどん	どんどん	
\\	ドアをどんどんとたたいた。	ドアをどんどんとたたいた。	どあ を どんどん と たたいた	
\\	ドアを
\\	とたたいた。		
\\	全国	全国[ぜんこく]	ぜんこく	
\\	次は全国のお天気です。	次[つぎ]は 全国[ぜんこく]のお 天気[てんき]です。	つぎ は ぜんこく の おてんき です	
\\	次[つぎ]は
\\	のお 天気[てんき]です。		
\\	外国	外国[がいこく]	がいこく	
\\	母はまだ外国に行ったことがありません。	母[はは]はまだ 外国[がいこく]に 行[い]ったことがありません。	はは は まだ がいこく に いった こと が ありません	
\\	母[はは]はまだ
\\	に 行[い]ったことがありません。		
\\	国会	国会[こっかい]	こっかい	
\\	国会が再開した。	国会[こっかい]が 再開[さいかい]した。	こっかい が さいかい した	
\\	が 再開[さいかい]した。		
\\	帰国	帰国[きこく]	きこく	
\\	彼は帰国しました。	彼[かれ]は 帰国[きこく]しました。	かれ は きこく しました	
\\	彼[かれ]は
\\	しました。		
\\	カメラ	カメラ	カメラ	
\\	カメラが壊れた。	カメラが 壊[こわ]れた。	かめら が こわれた	
\\	が 壊[こわ]れた。		
\\	外国人	外国人[がいこくじん]	がいこくじん	
\\	日本に住む外国人が増えています。	日本[にほん]に 住[す]む 外国人[がいこくじん]が 増[ふ]えています。	にほん に すむ がいこくじん が ふえて います	
\\	日本[にほん]に 住[す]む
\\	が 増[ふ]えています。		
\\	外国語	外国語[がいこくご]	がいこくご	
\\	外国語を習うのは難しい。	外国語[がいこくご]を 習[なら]うのは 難[むずか]しい。	がいこくご を ならう の は むずかしい	
\\	を 習[なら]うのは 難[むずか]しい。		
\\	世界	世界[せかい]	せかい	
\\	私は世界旅行をしたい。	私[わたし]は 世界[せかい] 旅行[りょこう]をしたい。	わたし は せかい りょこう を したい	
\\	私[わたし]は
\\	旅行[りょこう]をしたい。		
\\	白	白[しろ]	しろ	
\\	白は雪の色です。	白[しろ]は 雪[ゆき]の 色[いろ]です。	しろ は ゆき の いろ です	
\\	は 雪[ゆき]の 色[いろ]です。		
\\	テープ	テープ	テープ	
\\	彼女はその会話をテープに録音した。	彼女[かのじょ]はその 会話[かいわ]をテープに 録音[ろくおん]した。	かのじょ は その かいわ を てーぷ に ろくおん した	
\\	彼女[かのじょ]はその 会話[かいわ]を
\\	に 録音[ろくおん]した。		
\\	黒い	黒[くろ]い	くろい	
\\	彼女は黒いドレスを着ています。	彼女[かのじょ]は 黒[くろ]いドレスを 着[き]ています。	かのじょ は くろい どれす を きて います	
\\	彼女[かのじょ]は
\\	ドレスを 着[き]ています。		
\\	黒	黒[くろ]	くろ	
\\	黒のボールペンはありますか。	黒[くろ]のボールペンはありますか。	くろ の ぼーるぺん は あります か	
\\	のボールペンはありますか。		
\\	赤ちゃん	赤[あか]ちゃん	あかちゃん	
\\	ベッドで赤ちゃんが眠っています。	ベッドで 赤[あか]ちゃんが 眠[ねむ]っています。	べっど で あかちゃん が ねむって います	
\\	ベッドで
\\	が 眠[ねむ]っています。		
\\	赤	赤[あか]	あか	
\\	信号が赤に変わりました。	信号[しんごう]が 赤[あか]に 変[か]わりました。	しんごう が あか に かわりました	
\\	信号[しんごう]が
\\	に 変[か]わりました。		
\\	ビール	ビール	ビール	
\\	夏はビールがとても美味しい。	夏[なつ]はビールがとても 美味[おい]しい。	なつ は びーる が とても おいしい	
\\	夏[なつ]は
\\	がとても 美味[おい]しい。		
\\	銀行	銀行[ぎんこう]	ぎんこう	
\\	銀行は3時まで開いています。	銀行[ぎんこう]は 3時[さんじ]まで 開[あ]いています。	ぎんこう は さんじ まで あいて います	
\\	は 3時[さんじ]まで 開[あ]いています。		
\\	銀	銀[ぎん]	ぎん	
\\	彼は銀メダルを取った。	彼[かれ]は 銀[ぎん]メダルを 取[と]った。	かれ は ぎんめだる を とった	
\\	彼[かれ]は
\\	メダルを 取[と]った。		
\\	地下鉄	地下鉄[ちかてつ]	ちかてつ	
\\	私は地下鉄で通勤しています。	私[わたし]は 地下鉄[ちかてつ]で 通勤[つうきん]しています。	わたし は ちかてつ で つうきん して います	
\\	私[わたし]は
\\	で 通勤[つうきん]しています。		
\\	牛肉	牛肉[ぎゅうにく]	ぎゅうにく	
\\	夕食に牛肉を買った。	夕食[ゆうしょく]に 牛肉[ぎゅうにく]を 買[か]った。	ゆうしょく に ぎゅうにく を かった	
\\	夕食[ゆうしょく]に
\\	を 買[か]った。		
\\	ページ	ページ	ページ	
\\	32ページを開いてください。	32[さんじゅうに]ページを 開[ひら]いてください。	さんじゅうにぺーじ を ひらいて ください	
\\	32[さんじゅうに]
\\	を 開[ひら]いてください。		
\\	肉	肉[にく]	にく	
\\	肉が焼けました。	肉[にく]が 焼[や]けました。	にく が やけました	
\\	が 焼[や]けました。		
\\	魚	魚[さかな]	さかな	
\\	肉と魚とどちらが好きですか。	肉[にく]と 魚[さかな]とどちらが 好[す]きですか。	にく と さかな と どちら が すき です か	
\\	肉[にく]と
\\	とどちらが 好[す]きですか。		
\\	分野	分野[ぶんや]	ぶんや	
\\	音楽は彼の得意な分野です。	音楽[おんがく]は 彼[かれ]の 得意[とくい]な 分野[ぶんや]です。	おんがく は かれ の とくい な ぶんや です	
\\	音楽[おんがく]は 彼[かれ]の 得意[とくい]な
\\	です。		
\\	野菜	野菜[やさい]	やさい	
\\	私は毎日たくさん野菜を食べます。	私[わたし]は 毎日[まいにち]たくさん 野菜[やさい]を 食[た]べます。	わたし は まいにち たくさん やさい を たべます	
\\	私[わたし]は 毎日[まいにち]たくさん
\\	を 食[た]べます。		
\\	グラフ	グラフ	グラフ	
\\	彼は売上をグラフにした。	彼[かれ]は 売上[うりあげ]をグラフにした。	かれ は うりあげ を ぐらふ に した	
\\	彼[かれ]は 売上[うりあげ]を
\\	にした。		
\\	本屋	本屋[ほんや]	ほんや	
\\	駅前に本屋があります。	駅前[えきまえ]に 本屋[ほんや]があります。	えきまえ に ほんや が あります	
\\	駅前[えきまえ]に
\\	があります。		
\\	八百屋	八百屋[やおや]	やおや	
\\	八百屋でみかんを買いました。	八百屋[やおや]でみかんを 買[か]いました。	やおや で みかん を かいました	
\\	でみかんを 買[か]いました。		
\\	そば屋	そば 屋[や]	そばや	
\\	昼はそば屋に行きました。	昼[ひる]はそば 屋[や]に 行[い]きました。	ひる は そばや に いきました 。	
\\	昼[ひる]は
\\	に 行[い]きました。		
\\	たばこ屋	たばこ 屋[や]	たばこや	
\\	私はたばこ屋でライターを買った。	私[わたし]はたばこ 屋[や]でライターを 買[か]った。	わたし は たばこや で らいたー を かった	
\\	私[わたし]は
\\	でライターを 買[か]った。		
\\	ポスト	ポスト	ポスト	
\\	手紙をポストに入れました。	手紙[てがみ]をポストに 入[い]れました。	てがみ を ぽすと に いれました	
\\	手紙[てがみ]を
\\	に 入[い]れました。		
\\	茶	茶[ちゃ]	ちゃ	
\\	私たちは毎日お茶を飲みます。	私[わたし]たちは 毎日[まいにち]お 茶[ちゃ]を 飲[の]みます。	わたしたち は まいにち おちゃ を のみます	
\\	私[わたし]たちは 毎日[まいにち]お
\\	を 飲[の]みます。		
\\	お茶	お 茶[ちゃ]	おちゃ	
\\	お茶を入れましょうか。	お 茶[ちゃ]を 入[い]れましょうか。	おちゃ を いれましょう か	
\\	を 入[い]れましょうか。		
\\	茶わん	茶[ちゃ]わん	ちゃわん	
\\	茶わんにご飯をよそいました。	茶[ちゃ]わんにご 飯[はん]をよそいました。	ちゃわん に ごはん を よそいました	
\\	にご 飯[はん]をよそいました。		
\\	味	味[あじ]	あじ	
\\	この料理は味が薄い。	この 料理[りょうり]は 味[あじ]が 薄[うす]い。	この りょうり は あじ が うすい	
\\	この 料理[りょうり]は
\\	が 薄[うす]い。		
\\	テスト	テスト	テスト	
\\	テストを始めてください。	テストを 始[はじ]めてください。	てすと を はじめて ください	
\\	を 始[はじ]めてください。		
\\	未来	未来[みらい]	みらい	
\\	未来は誰にも分からない。	未来[みらい]は 誰[だれ]にも 分[わ]からない。	みらい は だれ に も わからない	
\\	は 誰[だれ]にも 分[わ]からない。		
\\	週末	週末[しゅうまつ]	しゅうまつ	
\\	週末は家でゆっくりします。	週末[しゅうまつ]は 家[うち]でゆっくりします。	しゅうまつ は うち で ゆっくり します	
\\	は 家[うち]でゆっくりします。		
\\	料理	料理[りょうり]	りょうり	
\\	母は料理が得意です。	母[はは]は 料理[りょうり]が 得意[とくい]です。	はは は りょうり が とくい です	
\\	母[はは]は
\\	が 得意[とくい]です。		
\\	無理	無理[むり]	むり	
\\	5時までに家に帰るのは無理です。	5時[ごじ]までに 家[いえ]に 帰[かえ]るのは 無理[むり]です。	ごじ まで に いえ に かえる の は むり です	
\\	5時[ごじ]までに 家[いえ]に 帰[かえ]るのは
\\	です。		
\\	あちこち	あちこち	あちこち	
\\	私たちは朝からあちこち散歩しました。	私[わたし]たちは 朝[あさ]からあちこち 散歩[さんぽ]しました。	わたしたち は あさ から あちこち さんぽ しました	
\\	私[わたし]たちは 朝[あさ]から
\\	散歩[さんぽ]しました。		
\\	なくす	なくす	なくす	
\\	今日、鍵をなくしました。	今日[きょう]、 鍵[かぎ]をなくしました。	きょう かぎ を なくしました	
\\	今日[きょう]、 鍵[かぎ]を
\\	無くなる	無[な]くなる	なくなる	
\\	もうお金が無くなりました。	もうお 金[かね]が 無[な]くなりました。	もう おかね が なくなりました	
\\	もうお 金[かね]が
\\	作文	作文[さくぶん]	さくぶん	
\\	日本語で作文を書きました。	日本語[にほんご]で 作文[さくぶん]を 書[か]きました。	にほんご で さくぶん を かきました	
\\	日本語[にほんご]で
\\	を 書[か]きました。		
\\	用いる	用[もち]いる	もちいる	
\\	彼はその詩を用いて自分の気持ちを伝えた。	彼[かれ]はその 詩[し]を 用[もち]いて 自分[じぶん]の 気持[きも]ちを 伝[つた]えた。	かれ は その し を もちいて じぶん の きもち を つたえた	
\\	彼[かれ]はその 詩[し]を
\\	自分[じぶん]の 気持[きも]ちを 伝[つた]えた。		
\\	ゴルフ	ゴルフ	ゴルフ	
\\	兄はゴルフを始めました。	兄[あに]はゴルフを 始[はじ]めました。	あに は ごるふ を はじめました	
\\	兄[あに]は
\\	を 始[はじ]めました。		
\\	用事	用事[ようじ]	ようじ	
\\	父は用事で出掛けています。	父[ちち]は 用事[ようじ]で 出掛[でか]けています。	ちち は ようじ で でかけて います	
\\	父[ちち]は
\\	で 出掛[でか]けています。		
\\	交通費	交通費[こうつうひ]	こうつうひ	
\\	会社までの交通費は一ヶ月8,000円です。	会社[かいしゃ]までの 交通費[こうつうひ]は 一ヶ月8,000円[いっかげつ はっせんえん]です。	かいしゃ まで の こうつうひ は いっかげつ はっせんえん です	
\\	会社[かいしゃ]までの
\\	は 一ヶ月8,000円[いっかげつ はっせんえん]です。		
\\	消える	消[き]える	きえる	
\\	突然、電気が消えた。	突然[とつぜん]、 電気[でんき]が 消[き]えた。	とつぜん でんき が きえた	
\\	突然[とつぜん]、 電気[でんき]が
\\	消しゴム	消[け]しゴム	けしごむ	
\\	消しゴムを貸して下さい。	消[け]しゴムを 貸[か]して 下[くだ]さい。	けしごむ を かして ください	
\\	を 貸[か]して 下[くだ]さい。		
\\	ラジオ	ラジオ	ラジオ	
\\	彼女はラジオを聞いています。	彼女[かのじょ]はラジオを 聞[き]いています。	かのじょ は らじお を きいて います	
\\	彼女[かのじょ]は
\\	を 聞[き]いています。		
\\	売れる	売[う]れる	うれる	
\\	今年の夏はクーラーがよく売れた。	今年[ことし]の 夏[なつ]はクーラーがよく 売[う]れた。	ことし の なつ は くーらー が よく うれた	
\\	今年[ことし]の 夏[なつ]はクーラーがよく
\\	売り場	売[う]り 場[ば]	うりば	
\\	くつ売り場はどこですか。	くつ 売[う]り 場[ば]はどこですか。	くつうりば は どこ です か	
\\	くつ
\\	はどこですか。		
\\	店員	店員[てんいん]	てんいん	
\\	あの店員はとても親切です。	あの 店員[てんいん]はとても 親切[しんせつ]です。	あの てんいん は とても しんせつ です	
\\	あの
\\	はとても 親切[しんせつ]です。		
\\	売店	売店[ばいてん]	ばいてん	
\\	駅の売店で雑誌を買った。	駅[えき]の 売店[ばいてん]で 雑誌[ざっし]を 買[か]った。	えき の ばいてん で ざっし を かった	
\\	駅[えき]の
\\	で 雑誌[ざっし]を 買[か]った。		
\\	タクシー	タクシー	タクシー	
\\	タクシーを呼んでください。	タクシーを 呼[よ]んでください。	たくしー を よんで ください	
\\	を 呼[よ]んでください。		
\\	商品	商品[しょうひん]	しょうひん	
\\	この商品はよく売れている。	この 商品[しょうひん]はよく 売[う]れている。	この しょうひん は よく うれて いる	
\\	この
\\	はよく 売[う]れている。		
\\	作品	作品[さくひん]	さくひん	
\\	この絵はゴッホの作品だ。	この 絵[え]はゴッホの 作品[さくひん]だ。	この え は ごっほ の さくひん だ	
\\	この 絵[え]はゴッホの
\\	だ。		
\\	販売	販売[はんばい]	はんばい	
\\	前売券は窓口で販売しています。	前売券[まえうりけん]は 窓口[まどぐち]で 販売[はんばい]しています。	まえうりけん は まどぐち で はんばい して います	
\\	前売券[まえうりけん]は 窓口[まどぐち]で
\\	しています。		
\\	二階	二階[にかい]	にかい	
\\	兄は二階にいます。	兄[あに]は 二階[にかい]にいます。	あに は にかい に います	
\\	兄[あに]は
\\	にいます。		
\\	ゆっくり	ゆっくり	ゆっくり	
\\	もっとゆっくり話してください。	もっとゆっくり 話[はな]してください。	もっと ゆっくり はなして ください	
\\	もっと
\\	話[はな]してください。		
\\	段階	段階[だんかい]	だんかい	
\\	この段階では、決断するのはまだ早い。	この 段階[だんかい]では、 決断[けつだん]するのはまだ 早[はや]い。	この だんかい で は けつだん する の は まだ はやい	
\\	この
\\	では、 決断[けつだん]するのはまだ 早[はや]い。		
\\	階段	階段[かいだん]	かいだん	
\\	私たちは駅の階段をかけ上がった。	私[わたし]たちは 駅[えき]の 階段[かいだん]をかけ 上[あ]がった。	わたしたち は えき の かいだん を かけあがった	
\\	私[わたし]たちは 駅[えき]の
\\	をかけ 上[あ]がった。		
\\	段々	段々[だんだん]	だんだん	
\\	段々仕事が楽しくなってきました。	段々[だんだん] 仕事[しごと]が 楽[たの]しくなってきました。	だんだん しごと が たのしく なって きました	
\\	仕事[しごと]が 楽[たの]しくなってきました。		
\\	値段	値段[ねだん]	ねだん	
\\	このベルトの値段は2500円でした。	このベルトの 値段[ねだん]は 2500円[にせんごひゃくえん]でした。	この べると の ねだん は にせんごひゃくえん でした	
\\	このベルトの
\\	は 2500円[にせんごひゃくえん]でした。		
\\	レストラン	レストラン	レストラン	
\\	レストランでインド料理を食べました。	レストランでインド 料理[りょうり]を 食[た]べました。	れすとらん で いんど りょうり を たべました	
\\	でインド 料理[りょうり]を 食[た]べました。		
\\	価格	価格[かかく]	かかく	
\\	ガソリンの価格がどんどん上がっている。	ガソリンの 価格[かかく]がどんどん 上[あ]がっている。	がそりん の かかく が どんどん あがって いる	
\\	ガソリンの
\\	がどんどん 上[あ]がっている。		
\\	合格	合格[ごうかく]	ごうかく	
\\	娘が入学試験に合格しました。	娘[むすめ]が 入学試験[にゅうがく しけん]に 合格[ごうかく]しました。	むすめ が にゅうがく しけん に ごうかく しました	
\\	娘[むすめ]が 入学試験[にゅうがく しけん]に
\\	しました。		
\\	夏休み	夏休[なつやす]み	なつやすみ	
\\	今日が夏休み最後の日だ。	今日[きょう]が 夏休[なつやす]み 最後[さいご]の 日[ひ]だ。	きょう が なつやすみ さいご の ひ だ	
\\	今日[きょう]が
\\	最後[さいご]の 日[ひ]だ。		
\\	冬休み	冬休[ふゆやす]み	ふゆやすみ	
\\	冬休みにお祖父ちゃんの家に行きます。	冬休[ふゆやす]みにお 祖父[じい]ちゃんの 家[うち]に 行[い]きます。	ふゆやすみ に おじいちゃん の うち に いきます	
\\	にお 祖父[じい]ちゃんの 家[うち]に 行[い]きます。		
\\	カード	カード	カード	
\\	支払いはカードでお願いします。	支払[しはら]いはカードでお 願[ねが]いします。	しはらい は かーど で おねがい します	
\\	支払[しはら]いは
\\	でお 願[ねが]いします。		
\\	四季	四季[しき]	しき	
\\	日本には四季がある。	日本[にほん]には 四季[しき]がある。	にほん に は しき が ある	
\\	日本[にほん]には
\\	がある。		
\\	暑さ	暑[あつ]さ	あつさ	
\\	今年の夏は暑さが厳しい。	今年[ことし]の 夏[なつ]は 暑[あつ]さが 厳[きび]しい。	ことし の なつ は あつさ が きびしい	
\\	今年[ことし]の 夏[なつ]は
\\	が 厳[きび]しい。		
\\	熱	熱[ねつ]	ねつ	
\\	昨日の夜、熱が出ました。	昨日[きのう]の 夜[よる]、 熱[ねつ]が 出[で]ました。	きのう の よる ねつ が でました	
\\	昨日[きのう]の 夜[よる]、
\\	が 出[で]ました。		
\\	寒さ	寒[さむ]さ	さむさ	
\\	今日は厳しい寒さになるでしょう。	今日[きょう]は 厳[きび]しい 寒[さむ]さになるでしょう。	きょう は きびしい さむさ に なる でしょう	
\\	今日[きょう]は 厳[きび]しい
\\	になるでしょう。		
\\	アルバイト	アルバイト	アルバイト	
\\	兄はアルバイトをしています。	兄[あに]はアルバイトをしています。	あに は あるばいと を して います	
\\	兄[あに]は
\\	をしています。		
\\	暖める	暖[あたた]める	あたためる	
\\	を		
\\	今、車を暖めています。	今[いま]、 車[くるま]を 暖[あたた]めています。	いま くるま を あたためています	
\\	今[いま]、 車[くるま]を
\\	暖まる	暖[あたた]まる	あたたまる	
\\	が		
\\	まだ部屋が暖まらない。	まだ 部屋[へや]が 暖[あたた]まらない。	まだ へや が あたたまらない。	
\\	まだ 部屋[へや]が
\\	温度	温度[おんど]	おんど	
\\	今、部屋の温度は25度だ。	今[いま]、 部屋[へや]の 温度[おんど]は 25度[にじゅうごど]だ。	いま へや の おんど は にじゅうごど だ	
\\	今[いま]、 部屋[へや]の
\\	は 25度[にじゅうごど]だ。		
\\	気温	気温[きおん]	きおん	
\\	今日の気温は26度です。	今日[きょう]の 気温[きおん]は 26度[にじゅうろくど]です。	きょう の きおん は にじゅうろくど です	
\\	今日[きょう]の
\\	は 26度[にじゅうろくど]です。		
\\	コピー	コピー	コピー	
\\	会議で書類のコピーを配った。	会議[かいぎ]で 書類[しょるい]のコピーを 配[くば]った。	かいぎ で しょるい の こぴー を くばった	
\\	会議[かいぎ]で 書類[しょるい]の
\\	を 配[くば]った。		
\\	台	台[だい]	だい	
\\	そこにちょうど良い台がある。	そこにちょうど 良[い]い 台[だい]がある。	そこ に ちょうど いい だい が ある	
\\	そこにちょうど 良[い]い
\\	がある。		
\\	風	風[かぜ]	かぜ	
\\	今日は風が強いです。	今日[きょう]は 風[かぜ]が 強[つよ]いです。	きょう は かぜ が つよい です	
\\	今日[きょう]は
\\	が 強[つよ]いです。		
\\	台風	台風[たいふう]	たいふう	
\\	台風が近づいている。	台風[たいふう]が 近[ちか]づいている。	たいふう が ちかづいて いる	
\\	が 近[ちか]づいている。		
\\	事情	事情[じじょう]	じじょう	
\\	あなたの事情はよく分かりました。	あなたの 事情[じじょう]はよく 分[わ]かりました。	あなた の じじょう は よく わかりました	
\\	あなたの
\\	はよく 分[わ]かりました。		
\\	ぶつかる	ぶつかる	ぶつかる	
\\	に		
\\	車が電柱にぶつかった。	車[くるま]が 電柱[でんちゅう]にぶつかった。	くるま が でんちゅう に ぶつかった	
\\	車[くるま]が 電柱[でんちゅう]に
\\	情報	情報[じょうほう]	じょうほう	
\\	学生たちはインターネットでいろいろな情報を集めた。	学生[がくせい]たちはインターネットでいろいろな 情報[じょうほう]を 集[あつ]めた。	がくせいたち は いんたーねっと で いろいろな じょうほう を あつめた	
\\	学生[がくせい]たちはインターネットでいろいろな
\\	を 集[あつ]めた。		
\\	報告	報告[ほうこく]	ほうこく	
\\	昨日の会議について報告があります。	昨日[きのう]の 会議[かいぎ]について 報告[ほうこく]があります。	きのう の かいぎ に ついて ほうこく が あります	
\\	昨日[きのう]の 会議[かいぎ]について
\\	があります。		
\\	新聞	新聞[しんぶん]	しんぶん	
\\	今日の新聞、どこに置いた?	今日[きょう]の 新聞[しんぶん]、どこに 置[お]いた?	きょう の しんぶん どこ に おいた	
\\	今日[きょう]の
\\	、どこに 置[お]いた?		
\\	新年	新年[しんねん]	しんねん	
\\	新年明けましておめでとうございます。	新年[しんねん] 明[あ]けましておめでとうございます。	しんねん あけまして おめでとう ございます	
\\	明[あ]けましておめでとうございます。		
\\	フィルム	フィルム	フィルム	
\\	(外来語)		
\\	旅行のためにたくさんフィルムを買った。	旅行[りょこう]のためにたくさんフィルムを 買[か]った。	りょこう の ため に たくさん ふぃるむ を かった	
\\	旅行[りょこう]のためにたくさん
\\	を 買[か]った。		
\\	良い	良[よ]い	よい	
\\	彼は良いところだけを強調した。	彼[かれ]は 良[よ]いところだけを 強調[きょうちょう]した。	かれ は よい ところ だけ を きょうちょう した	
\\	彼[かれ]は
\\	ところだけを 強調[きょうちょう]した。		
\\	中心	中心[ちゅうしん]	ちゅうしん	
\\	ここが建物の中心です。	ここが 建物[たてもの]の 中心[ちゅうしん]です。	ここ が たてもの の ちゅうしん です	
\\	ここが 建物[たてもの]の
\\	です。		
\\	安心	安心[あんしん]	あんしん	
\\	それを聞いて安心しました。	それを 聞[き]いて 安心[あんしん]しました。	それ を きいて あんしん しました	
\\	それを 聞[き]いて
\\	しました。		
\\	思い出す	思[おも]い 出[だ]す	おもいだす	
\\	大切な用事を思い出しました。	大切[たいせつ]な 用事[ようじ]を 思[おも]い 出[だ]しました。	たいせつ な ようじ を おもいだしました	
\\	大切[たいせつ]な 用事[ようじ]を
\\	デパート	デパート	デパート	
\\	私はデパートで靴を買った。	私[わたし]はデパートで 靴[くつ]を 買[か]った。	わたし は でぱーと で くつ を かった	
\\	私[わたし]は
\\	で 靴[くつ]を 買[か]った。		
\\	思い出	思[おも]い 出[で]	おもいで	
\\	旅行で楽しい思い出ができました。	旅行[りょこう]で 楽[たの]しい 思[おも]い 出[で]ができました。	りょこう で たのしい おもいで が できました	
\\	旅行[りょこう]で 楽[たの]しい
\\	ができました。		
\\	考え	考[かんが]え	かんがえ	
\\	それは良い考えです。	それは 良[い]い 考[かんが]えです。	それ は いい かんがえ です	
\\	それは 良[い]い
\\	です。		
\\	解決	解決[かいけつ]	かいけつ	
\\	トラブルがやっと解決した。	トラブルがやっと 解決[かいけつ]した。	とらぶる が やっと かいけつ した	
\\	トラブルがやっと
\\	した。		
\\	知らせる	知[し]らせる	しらせる	
\\	皆に会議の日にちを知らせた。	皆[みんな]に 会議[かいぎ]の 日[ひ]にちを 知[し]らせた。	みんな に かいぎ の ひにち を しらせた	
\\	皆[みんな]に 会議[かいぎ]の 日[ひ]にちを
\\	ベッド	ベッド	ベッド	
\\	彼はベッドで寝ています。	彼[かれ]はベッドで 寝[ね]ています。	かれ は べっど で ねて います	
\\	彼[かれ]は
\\	で 寝[ね]ています。		
\\	能力	能力[のうりょく]	のうりょく	
\\	彼は能力のある社員です。	彼[かれ]は 能力[のうりょく]のある 社員[しゃいん]です。	かれ は のうりょく の ある しゃいん です	
\\	彼[かれ]は
\\	のある 社員[しゃいん]です。		
\\	可能	可能[かのう]	かのう	
\\	20キロのダイエットは可能だと思いますか。	20[にじゅっ]キロのダイエットは 可能[かのう]だと 思[おも]いますか。	にじゅっきろ の だいえっと は かのう だ と おもいます か	
\\	20[にじゅっ]キロのダイエットは
\\	だと 思[おも]いますか。		
\\	可	可[か]	か	
\\	このアルバイトは「学生可」ですね。	このアルバイトは
\\	学生[がくせい] 可[か]」ですね。	この あるばいと は がくせい か です ね	
\\	このアルバイトは
\\	学生[がくせい]
\\	ですね。		
\\	郵便	郵便[ゆうびん]	ゆうびん	
\\	さっき郵便が届きました。	さっき 郵便[ゆうびん]が 届[とど]きました。	さっき ゆうびん が とどきました	
\\	さっき
\\	が 届[とど]きました。		
\\	コート	コート	コート	
\\	寒かったのでコートを着た。	寒[さむ]かったのでコートを 着[き]た。	さむかった の で こーと を きた	
\\	寒[さむ]かったので
\\	を 着[き]た。		
\\	不便	不便[ふべん]	ふべん	
\\	私の家は駅から遠くて不便です。	私[わたし]の 家[いえ]は 駅[えき]から 遠[とお]くて 不便[ふべん]です。	わたし の いえ は えき から とおく て ふべん です	
\\	私[わたし]の 家[いえ]は 駅[えき]から 遠[とお]くて
\\	です。		
\\	郵便屋さん	郵便屋[ゆうびんや]さん	ゆうびんやさん	
\\	郵便屋さんはもう来ましたか。	郵便屋[ゆうびんや]さんはもう 来[き]ましたか。	ゆうびんやさん は もう きました か	
\\	はもう 来[き]ましたか。		
\\	郵便局	郵便局[ゆうびんきょく]	ゆうびんきょく	
\\	郵便局で切手を買いました。	郵便局[ゆうびんきょく]で 切手[きって]を 買[か]いました。	ゆうびんきょくで きって を かいました	
\\	で 切手[きって]を 買[か]いました。		
\\	交番	交番[こうばん]	こうばん	
\\	あそこの交番で道を聞きましょう。	あそこの 交番[こうばん]で 道[みち]を 聞[き]きましょう。	あそこ の こうばん で みち を ききましょう	
\\	あそこの
\\	で 道[みち]を 聞[き]きましょう。		
\\	ノート	ノート	ノート	
\\	ノートを開いてください。	ノートを 開[ひら]いてください。	のーと を ひらいて ください	
\\	を 開[ひら]いてください。		
\\	番地	番地[ばんち]	ばんち	
\\	その建物の番地は分かりますか。	その 建物[たてもの]の 番地[ばんち]は 分[わ]かりますか。	その たてもの の ばんち は わかります か	
\\	その 建物[たてもの]の
\\	は 分[わ]かりますか。		
\\	番号	番号[ばんごう]	ばんごう	
\\	この番号に電話してください。	この 番号[ばんごう]に 電話[でんわ]してください。	この ばんごう に でんわ して ください	
\\	この
\\	に 電話[でんわ]してください。		
\\	場所	場所[ばしょ]	ばしょ	
\\	会社の場所を教えてください。	会社[かいしゃ]の 場所[ばしょ]を 教[おし]えてください。	かいしゃ の ばしょ を おしえて ください	
\\	会社[かいしゃ]の
\\	を 教[おし]えてください。		
\\	近所	近所[きんじょ]	きんじょ	
\\	近所にカナダ人が住んでいる。	近所[きんじょ]にカナダ 人[じん]が 住[す]んでいる。	きんじょ に かなだじん が すんで いる	
\\	にカナダ 人[じん]が 住[す]んでいる。		
\\	ワイン	ワイン	ワイン	
\\	ワインを少し飲みました。	ワインを 少[すこ]し 飲[の]みました。	わいん を すこし のみました	
\\	を 少[すこ]し 飲[の]みました。		
\\	台所	台所[だいどころ]	だいどころ	
\\	お母さんは台所にいます。	お 母[かあ]さんは 台所[だいどころ]にいます。	おかあさん は だいどころ に います	
\\	お 母[かあ]さんは
\\	にいます。		
\\	住所	住所[じゅうしょ]	じゅうしょ	
\\	この住所に行ってください。	この 住所[じゅうしょ]に 行[い]ってください。	この じゅうしょ に いって ください	
\\	この
\\	に 行[い]ってください。		
\\	便所	便所[べんじょ]	べんじょ	
\\	便所はそこです。	便所[べんじょ]はそこです。	べんじょ は そこ です	
\\	はそこです。		
\\	有名	有名[ゆうめい]	ゆうめい	
\\	ボルドーはワインの生産で有名だ。	ボルドーはワインの 生産[せいさん]で 有名[ゆうめい]だ。	ぼるどー は わいん の せいさん で ゆうめい だ	
\\	ボルドーはワインの 生産[せいさん]で
\\	だ。		
\\	おかしい	おかしい	おかしい	
\\	彼の様子がおかしい。	彼[かれ]の 様子[ようす]がおかしい。	かれ の ようす が おかしい	
\\	彼[かれ]の 様子[ようす]が
\\	名字	名字[みょうじ]	みょうじ	
\\	あなたの名字は何ですか。	あなたの 名字[みょうじ]は 何[なん]ですか。	あなた の みょうじ は なん です か	
\\	あなたの
\\	は 何[なん]ですか。		
\\	氏名	氏名[しめい]	しめい	
\\	ここに住所と氏名を書いてください。	ここに 住所[じゅうしょ]と 氏名[しめい]を 書[か]いてください。	ここ に じゅうしょ と しめい を かいて ください	
\\	ここに 住所[じゅうしょ]と
\\	を 書[か]いてください。		
\\	各国	各国[かっこく]	かっこく	
\\	各国の代表がニューヨークに集まった。	各国[かっこく]の 代表[だいひょう]がニューヨークに 集[あつ]まった。	かっこく の だいひょう が にゅーよーく に あつまった	
\\	の 代表[だいひょう]がニューヨークに 集[あつ]まった。		
\\	町	町[まち]	まち	
\\	あの町は緑を増やしています。	あの 町[まち]は 緑[みどり]を 増[ふ]やしています。	あの まち は みどり を ふやして います	
\\	あの
\\	は 緑[みどり]を 増[ふ]やしています。		
\\	トイレ	トイレ	トイレ	
\\	トイレを掃除しましたか。	トイレを 掃除[そうじ]しましたか。	といれ を そうじ しました か	
\\	を 掃除[そうじ]しましたか。		
\\	都市	都市[とし]	とし	
\\	東京は日本一大きな都市です。	東京[とうきょう]は 日本一大[にほんいち おお]きな 都市[とし]です。	とうきょう は にほんいち おおき な とし です	
\\	東京[とうきょう]は 日本一大[にほんいち おお]きな
\\	です。		
\\	都合	都合[つごう]	つごう	
\\	今日は都合が悪くて行けません。	今日[きょう]は 都合[つごう]が 悪[わる]くて 行[い]けません。	きょう は つごう が わるくて いけません	
\\	今日[きょう]は
\\	が 悪[わる]くて 行[い]けません。		
\\	朝ご飯	朝[あさ]ご 飯[はん]	あさごはん	
\\	七時に朝ご飯を食べました。	七時[しちじ]に 朝[あさ]ご 飯[はん]を 食[た]べました。	しちじ に あさごはん を たべました	
\\	七時[しちじ]に
\\	を 食[た]べました。		
\\	買い物	買[か]い 物[もの]	かいもの	
\\	母は買い物に出かけています。	母[はは]は 買[か]い 物[もの]に 出[で]かけています。	はは は かいもの に でかけて います	
\\	母[はは]は
\\	に 出[で]かけています。		
\\	キャンプ	キャンプ	キャンプ	
\\	友達とキャンプに行った。	友達[ともだち]とキャンプに 行[い]った。	ともだち と きゃんぷ に いった	
\\	友達[ともだち]と
\\	に 行[い]った。		
\\	荷物	荷物[にもつ]	にもつ	
\\	彼の家に荷物を送りました。	彼[かれ]の 家[いえ]に 荷物[にもつ]を 送[おく]りました。	かれ の いえ に にもつ を おくりました	
\\	彼[かれ]の 家[いえ]に
\\	を 送[おく]りました。		
\\	品物	品物[しなもの]	しなもの	
\\	その店は色々な品物を売っている。	その 店[みせ]は 色々[いろいろ]な 品物[しなもの]を 売[う]っている。	その みせ は いろいろ な しなもの を うって いる	
\\	その 店[みせ]は 色々[いろいろ]な
\\	を 売[う]っている。		
\\	見物	見物[けんぶつ]	けんぶつ	
\\	皆で東京見物をした。	皆[みんな]で 東京[とうきょう] 見物[けんぶつ]をした。	みんな で とうきょう けんぶつ を した	
\\	皆[みんな]で 東京[とうきょう]
\\	をした。		
\\	物	物[もの]	もの	
\\	彼女の家には物がたくさんあります。	彼女[かのじょ]の 家[いえ]には 物[もの]がたくさんあります。	かのじょ の いえ に は もの が たくさん あります	
\\	彼女[かのじょ]の 家[いえ]には
\\	がたくさんあります。		
\\	プラスチック	プラスチック	プラスチック	
\\	このカップはプラスチックです。	このカップはプラスチックです。	この かっぷ は ぷらすちっく です	
\\	このカップは
\\	です。		
\\	忘れ物	忘[わす]れ 物[もの]	わすれもの	
\\	学校に忘れ物をしました。	学校[がっこう]に 忘[わす]れ 物[もの]をしました。	がっこう に わすれもの を しました	
\\	学校[がっこう]に
\\	をしました。		
\\	重さ	重[おも]さ	おもさ	
\\	この荷物の重さを計ってください。	この 荷物[にもつ]の 重[おも]さを 計[はか]ってください。	この にもつ の おもさ を はかって ください	
\\	この 荷物[にもつ]の
\\	を 計[はか]ってください。		
\\	配る	配[くば]る	くばる	
\\	会議で書類を配った。	会議[かいぎ]で 書類[しょるい]を 配[くば]った。	かいぎ で しょるい を くばった	
\\	会議[かいぎ]で 書類[しょるい]を
\\	配達	配達[はいたつ]	はいたつ	
\\	彼は新聞配達をしている。	彼[かれ]は 新聞[しんぶん] 配達[はいたつ]をしている。	かれ は しんぶん はいたつ を して いる	
\\	彼[かれ]は 新聞[しんぶん]
\\	をしている。		
\\	カラー	カラー	カラー	
\\	カラーコピーは一枚幾らですか。	カラーコピーは 一枚幾[いちまい いく]らですか。	からーこぴー は いちまい いくら です か	
\\	コピーは 一枚幾[いちまい いく]らですか。		
\\	心配	心配[しんぱい]	しんぱい	
\\	明日のプレゼンテーションが心配だ。	明日[あす]のプレゼンテーションが 心配[しんぱい]だ。	あす の ぷれぜんてーしょん が しんぱい だ	
\\	明日[あす]のプレゼンテーションが
\\	だ。		
\\	見送る	見送[みおく]る	みおくる	
\\	彼が外国に行くのを見送りました。	彼[かれ]が 外国[がいこく]に 行[い]くのを 見送[みおく]りました。	かれ が がいこく に いく の を みおくりました	
\\	彼[かれ]が 外国[がいこく]に 行[い]くのを
\\	見送り	見送[みおく]り	みおくり	
\\	駅に友人の見送りに行きました。	駅[えき]に 友人[ゆうじん]の 見送[みおく]りに 行[い]きました。	えき に ゆうじん の みおくり に いきました	
\\	駅[えき]に 友人[ゆうじん]の
\\	に 行[い]きました。		
\\	受ける	受[う]ける	うける	
\\	彼は就職試験を受けた。	彼[かれ]は 就職試験[しゅうしょく しけん]を 受[う]けた。	かれ は しゅうしょく しけん を うけた	
\\	彼[かれ]は 就職試験[しゅうしょく しけん]を
\\	ピアノ	ピアノ	ピアノ	
\\	昔、ピアノを習っていました。	昔[むかし]、ピアノを 習[なら]っていました。	むかし ぴあの を ならって いました	
\\	昔[むかし]、
\\	を 習[なら]っていました。		
\\	受け取る	受[う]け 取[と]る	うけとる	
\\	彼からメールを受け取りました。	彼[かれ]からメールを 受[う]け 取[と]りました。	かれ から めーる を うけとりました	
\\	彼[かれ]からメールを
\\	取れる	取[と]れる	とれる	
\\	シャツのボタンが取れた。	シャツのボタンが 取[と]れた。	しゃつ の ぼたん が とれた	
\\	シャツのボタンが
\\	書き取る	書[か]き 取[と]る	かきとる	
\\	話しのポイントを書き取った。	話[はな]しのポイントを 書[か]き 取[と]った。	はなし の ぽいんと を かきとった	
\\	話[はな]しのポイントを
\\	届く	届[とど]く	とどく	
\\	昨日、父から手紙が届いた。	昨日[きのう]、 父[ちち]から 手紙[てがみ]が 届[とど]いた。	きのう ちち から てがみ が とどいた	
\\	昨日[きのう]、 父[ちち]から 手紙[てがみ]が
\\	スキー	スキー	スキー	
\\	冬はよくスキーに行きます。	冬[ふゆ]はよくスキーに 行[い]きます。	ふゆ は よく すきー に いきます	
\\	冬[ふゆ]はよく
\\	に 行[い]きます。		
\\	届ける	届[とど]ける	とどける	
\\	これを彼に届けてください。	これを 彼[かれ]に 届[とど]けてください。	これ を かれ に とどけて ください	
\\	これを 彼[かれ]に
\\	ください。		
\\	持つ	持[も]つ	もつ	
\\	この車はよく持っているね。	この 車[くるま]はよく 持[も]っているね。	この くるま は よく もって いる ね	
\\	この 車[くるま]はよく
\\	ね。		
\\	金持ち	金持[かねも]ち	かねもち	
\\	彼は金持ちです。	彼[かれ]は 金持[かねも]ちです。	かれ は かねもち です	
\\	彼[かれ]は
\\	です。		
\\	持って行く	持[も]って 行[い]く	もっていく	
\\	水を持って行きましょう。	水[みず]を 持[も]って 行[い]きましょう。	みず を もっていきましょう	
\\	水[みず]を
\\	なかなか	なかなか	なかなか	
\\	荷物がなかなか届きません。	荷物[にもつ]がなかなか 届[とど]きません。	にもつ が なかなか とどきません	
\\	荷物[にもつ]が
\\	届[とど]きません。		
\\	持って来る	持[も]って 来[く]る	もってくる	
\\	そのいすを持って来てください。	そのいすを 持[も]って 来[き]てください。	その いす を もって きて ください	
\\	そのいすを
\\	ください。		
\\	打つ	打[う]つ	うつ	
\\	転んでひざを打ちました。	転[ころ]んでひざを 打[う]ちました。	ころんで ひざ を うちました	
\\	転[ころ]んでひざを
\\	投げる	投[な]げる	なげる	
\\	ボールをこっちに投げてください。	ボールをこっちに 投[な]げてください。	ぼーる を こっち に なげて ください	
\\	ボールをこっちに
\\	ください。		
\\	生まれる	生[う]まれる	うまれる	
\\	姉夫婦に男の子が生まれました。	姉夫婦[あね ふうふ]に 男[おとこ]の 子[こ]が 生[う]まれました。	あね ふうふ に おとこのこ が うまれました	
\\	姉夫婦[あね ふうふ]に 男[おとこ]の 子[こ]が
\\	プール	プール	プール	
\\	私は夏休みにプールに行った。	私[わたし]は 夏休[なつやす]みにプールに 行[い]った。	わたし は なつやすみ に ぷーる に いった	
\\	私[わたし]は 夏休[なつやす]みに
\\	に 行[い]った。		
\\	生む	生[う]む	うむ	
\\	うちのネコが子猫を生みました。	うちのネコが 子猫[こねこ]を 生[う]みました。	うち の ねこ が こねこ を うみました	
\\	うちのネコが 子猫[こねこ]を
\\	女性	女性[じょせい]	じょせい	
\\	そのパーティーに女性は何人来ますか。	そのパーティーに 女性[じょせい]は 何人来[なんにん き]ますか。	その ぱーてぃー に じょせい は なんにん きます か	
\\	そのパーティーに
\\	は 何人来[なんにん き]ますか。		
\\	生産	生産[せいさん]	せいさん	
\\	ボルドーはワインの生産で有名だ。	ボルドーはワインの 生産[せいさん]で 有名[ゆうめい]だ。	ぼるどー は わいん の せいさん で ゆうめい だ	
\\	ボルドーはワインの
\\	で 有名[ゆうめい]だ。		
\\	お土産	お 土産[みやげ]	おみやげ	
\\	父はお土産にお菓子を買ってきた。	父[ちち]はお 土産[みやげ]にお 菓子[かし]を 買[か]ってきた。	ちち は おみやげ に おかし を かって きた	
\\	父[ちち]は
\\	にお 菓子[かし]を 買[か]ってきた。		
\\	ホーム	ホーム	ホーム	
\\	もうすぐこのホームに電車が来ます。	もうすぐこのホームに 電車[でんしゃ]が 来[き]ます。	もうすぐ この ほーむ に でんしゃ が きます	
\\	もうすぐこの
\\	に 電車[でんしゃ]が 来[き]ます。		
\\	生活	生活[せいかつ]	せいかつ	
\\	日本での生活は楽しいです。	日本[にほん]での 生活[せいかつ]は 楽[たの]しいです。	にほん で の せいかつ は たのしい です	
\\	日本[にほん]での
\\	は 楽[たの]しいです。		
\\	生徒	生徒[せいと]	せいと	
\\	このクラスの生徒は30人です。	このクラスの 生徒[せいと]は 30人[さんじゅうにん]です。	この くらす の せいと は さんじゅうにん です	
\\	このクラスの
\\	は 30人[さんじゅうにん]です。		
\\	中学	中学[ちゅうがく]	ちゅうがく	
\\	息子は中学に通っています。	息子[むすこ]は 中学[ちゅうがく]に 通[かよ]っています。	むすこ は ちゅうがく に かよって います	
\\	息子[むすこ]は
\\	に 通[かよ]っています。		
\\	入学	入学[にゅうがく]	にゅうがく	
\\	妹は九月にアメリカの大学に入学します。	妹[いもうと]は 九月[くがつ]にアメリカの 大学[だいがく]に 入学[にゅうがく]します。	いもうと は くがつ に あめりか の だいがく に にゅうがく します	
\\	妹[いもうと]は 九月[くがつ]にアメリカの 大学[だいがく]に
\\	します。		
\\	エレベーター	エレベーター	エレベーター	
\\	エレベーターで下に降りましょう。	エレベーターで 下[した]に 降[お]りましょう。	えれべーたー で した に おりましょう	
\\	で 下[した]に 降[お]りましょう。		
\\	中学生	中学生[ちゅうがくせい]	ちゅうがくせい	
\\	息子は中学生です。	息子[むすこ]は 中学生[ちゅうがくせい]です。	むすこ は ちゅうがくせい です	
\\	息子[むすこ]は
\\	です。		
\\	小学生	小学生[しょうがくせい]	しょうがくせい	
\\	うちの息子は来年、小学生になります。	うちの 息子[むすこ]は 来年[らいねん]、 小学生[しょうがくせい]になります。	うち の むすこ は らいねん しょうがくせい に なります	
\\	うちの 息子[むすこ]は 来年[らいねん]、
\\	になります。		
\\	見学	見学[けんがく]	けんがく	
\\	今日、工場の見学に行きました。	今日[きょう]、 工場[こうじょう]の 見学[けんがく]に 行[い]きました。	きょう こうじょう の けんがく に いきました	
\\	今日[きょう]、 工場[こうじょう]の
\\	に 行[い]きました。		
\\	通学	通学[つうがく]	つうがく	
\\	毎朝、通学に1時間かかる。	毎朝[まいあさ]、 通学[つうがく]に 1時間[いちじかん]かかる。	まいあさ つうがく に いちじかん かかる	
\\	毎朝[まいあさ]、
\\	に 1時間[いちじかん]かかる。		
\\	メモ	メモ	メモ	
\\	メモを取ってください。	メモを 取[と]ってください。	めも を とって ください	
\\	を 取[と]ってください。		
\\	高校	高校[こうこう]	こうこう	
\\	妹は高校に通っています。	妹[いもうと]は 高校[こうこう]に 通[かよ]っています。	いもうと は こうこう に かよって います	
\\	妹[いもうと]は
\\	に 通[かよ]っています。		
\\	小学校	小学校[しょうがっこう]	しょうがっこう	
\\	家の近くに小学校があります。	家[いえ]の 近[ちか]くに 小学校[しょうがっこう]があります。	いえ の ちかく に しょうがっこう が あります	
\\	家[いえ]の 近[ちか]くに
\\	があります。		
\\	中学校	中学校[ちゅうがっこう]	ちゅうがっこう	
\\	息子の中学校は家から5分です。	息子[むすこ]の 中学校[ちゅうがっこう]は 家[いえ]から 5分[ごふん]です。	むすこ の ちゅうがっこう は いえ から ごふん です	
\\	息子[むすこ]の
\\	は 家[いえ]から 5分[ごふん]です。		
\\	校長	校長[こうちょう]	こうちょう	
\\	あの人は高校の校長だ。	あの 人[ひと]は 高校[こうこう]の 校長[こうちょう]だ。	あの ひと は こうこう の こうちょう だ	
\\	あの 人[ひと]は 高校[こうこう]の
\\	だ。		
\\	パン	パン	パン	
\\	朝ご飯にはいつもパンを食べる。	朝[あさ]ご 飯[はん]にはいつもパンを 食[た]べる。	あさごはん に は いつも ぱん を たべる	
\\	朝[あさ]ご 飯[はん]にはいつも
\\	を 食[た]べる。		
\\	休校	休校[きゅうこう]	きゅうこう	
\\	学校は今週は休校です。	学校[がっこう]は 今週[こんしゅう]は 休校[きゅうこう]です。	がっこう は こんしゅう は きゅうこう です	
\\	学校[がっこう]は 今週[こんしゅう]は
\\	です。		
\\	教会	教会[きょうかい]	きょうかい	
\\	私たちは教会で結婚式をしました。	私[わたし]たちは 教会[きょうかい]で 結婚式[けっこんしき]をしました。	わたしたち は きょうかい で けっこんしき を しました	
\\	私[わたし]たちは
\\	で 結婚式[けっこんしき]をしました。		
\\	教育	教育[きょういく]	きょういく	
\\	彼は海外で教育を受けました。	彼[かれ]は 海外[かいがい]で 教育[きょういく]を 受[う]けました。	かれ は かいがい で きょういく を うけました	
\\	彼[かれ]は 海外[かいがい]で
\\	を 受[う]けました。		
\\	育てる	育[そだ]てる	そだてる	
\\	彼女は三人の子を育てました。	彼女[かのじょ]は 三人[さんにん]の 子[こ]を 育[そだ]てました。	かのじょ は さんにん の こ を そだてました	
\\	彼女[かのじょ]は 三人[さんにん]の 子[こ]を
\\	びっくりする	びっくりする	びっくりする	
\\	大きな音にびっくりしました。	大[おお]きな 音[おと]にびっくりしました。	おおきな おと に びっくり しました	
\\	大[おお]きな 音[おと]に
\\	育つ	育[そだ]つ	そだつ	
\\	野菜がよく育っている。	野菜[やさい]がよく 育[そだ]っている。	やさい が よく そだって いる	
\\	野菜[やさい]がよく
\\	制度	制度[せいど]	せいど	
\\	来年から新しい制度が始まります。	来年[らいねん]から 新[あたら]しい 制度[せいど]が 始[はじ]まります。	らいねん から あたらしい せいど が はじまります	
\\	来年[らいねん]から 新[あたら]しい
\\	が 始[はじ]まります。		
\\	強さ	強[つよ]さ	つよさ	
\\	風の強さに驚きました。	風[かぜ]の 強[つよ]さに 驚[おどろ]きました。	かぜ の つよさ に おどろきました	
\\	風[かぜ]の
\\	に 驚[おどろ]きました。		
\\	取引	取引[とりひき]	とりひき	
\\	私たちは中国の会社と取引しています。	私[わたし]たちは 中国[ちゅうごく]の 会社[かいしゃ]と 取引[とりひき]しています。	わたしたち は ちゅうごく の かいしゃ と とりひき して います	
\\	私[わたし]たちは 中国[ちゅうごく]の 会社[かいしゃ]と
\\	しています。		
\\	ズボン	ズボン	ズボン	
\\	ズボンが汚れた。	ズボンが 汚[よご]れた。	ずぼん が よごれた	
\\	が 汚[よご]れた。		
\\	引き出し	引[ひ]き 出[だ]し	ひきだし	
\\	財布は引き出しの中にあります。	財布[さいふ]は 引[ひ]き 出[だ]しの 中[なか]にあります。	さいふ は ひきだし の なか に あります	
\\	財布[さいふ]は
\\	の 中[なか]にあります。		
\\	押さえる	押[お]さえる	おさえる	
\\	ドアを押さえてください。	ドアを 押[お]さえてください。	どあ を おさえて ください	
\\	ドアを
\\	ください。		
\\	押し入れ	押[お]し 入[い]れ	おしいれ	
\\	布団を押し入れにしまいました。	布団[ふとん]を 押[お]し 入[い]れにしまいました。	ふとん を おしいれ に しまいました	
\\	布団[ふとん]を
\\	にしまいました。		
\\	練習	練習[れんしゅう]	れんしゅう	
\\	娘は今、バイオリンを練習しています。	娘[むすめ]は 今[いま]、バイオリンを 練習[れんしゅう]しています。	むすめ は いま ばいおりん を れんしゅう して います	
\\	娘[むすめ]は 今[いま]、バイオリンを
\\	しています。		
\\	おもちゃ	おもちゃ	おもちゃ	
\\	赤ちゃんが自動車のおもちゃで遊んでいる。	赤[あか]ちゃんが 自動車[じどうしゃ]のおもちゃで 遊[あそ]んでいる。	あかちゃん が じどうしゃ の おもちゃ で あそんで いる	
\\	赤[あか]ちゃんが 自動車[じどうしゃ]の
\\	で 遊[あそ]んでいる。		
\\	習う	習[なら]う	ならう	
\\	彼は空手を習っています。	彼[かれ]は 空手[からて]を 習[なら]っています。	かれ は からて を ならって います	
\\	彼[かれ]は 空手[からて]を
\\	慣れる	慣[な]れる	なれる	
\\	新しい家にはもう慣れましたか。	新[あたら]しい 家[いえ]にはもう 慣[な]れましたか。	あたらしい いえ に は もう なれました か	
\\	新[あたら]しい 家[いえ]にはもう
\\	か。		
\\	習慣	習慣[しゅうかん]	しゅうかん	
\\	毎朝コーヒーを飲むのが習慣です。	毎朝[まいあさ]コーヒーを 飲[の]むのが 習慣[しゅうかん]です。	まいあさ こーひー を のむ の が しゅうかん です	
\\	毎朝[まいあさ]コーヒーを 飲[の]むのが
\\	です。		
\\	研究	研究[けんきゅう]	けんきゅう	
\\	彼は何年も地震の研究をしている。	彼[かれ]は 何年[なんねん]も 地震[じしん]の 研究[けんきゅう]をしている。	かれ は なんねん も じしん の けんきゅう を して いる	
\\	彼[かれ]は 何年[なんねん]も 地震[じしん]の
\\	をしている。		
\\	グラム	グラム	グラム	
\\	ひき肉を200グラムください。	ひき 肉[にく]を 200[にひゃく]グラムください。	ひきにく を にひゃくぐらむ ください	
\\	ひき 肉[にく]を 200[にひゃく]
\\	ください。		
\\	試験	試験[しけん]	しけん	
\\	明日の試験、頑張ってね。	明日[あした]の 試験[しけん]、 頑張[がんば]ってね。	あした の しけん がんばって ね	
\\	明日[あした]の
\\	、 頑張[がんば]ってね。		
\\	問題	問題[もんだい]	もんだい	
\\	問題が一つあります。	問題[もんだい]が 一[ひと]つあります。	もんだい が ひとつ あります	
\\	が 一[ひと]つあります。		
\\	簡単	簡単[かんたん]	かんたん	
\\	この料理はとても簡単です。	この 料理[りょうり]はとても 簡単[かんたん]です。	この りょうり は とても かんたん です	
\\	この 料理[りょうり]はとても
\\	です。		
\\	複雑	複雑[ふくざつ]	ふくざつ	
\\	このプログラムはとても複雑です。	このプログラムはとても 複雑[ふくざつ]です。	この ぷろぐらむ は とても ふくざつ です	
\\	このプログラムはとても
\\	です。		
\\	コーヒー	コーヒー	コーヒー	
\\	私は毎朝コーヒーを飲みます。	私[わたし]は 毎朝[まいあさ]コーヒーを 飲[の]みます。	わたし は まいあさ こーひー を のみます	
\\	私[わたし]は 毎朝[まいあさ]
\\	を 飲[の]みます。		
\\	数字	数字[すうじ]	すうじ	
\\	数字は苦手です。	数字[すうじ]は 苦手[にがて]です。	すうじ は にがて です	
\\	は 苦手[にがて]です。		
\\	数学	数学[すうがく]	すうがく	
\\	兄は数学の先生です。	兄[あに]は 数学[すうがく]の 先生[せんせい]です。	あに は すうがく の せんせい です	
\\	兄[あに]は
\\	の 先生[せんせい]です。		
\\	数える	数[かぞ]える	かぞえる	
\\	いすの数を数えてください。	いすの 数[かず]を 数[かぞ]えてください。	いす の かず を かぞえて ください	
\\	いすの 数[かず]を
\\	ください。		
\\	今回	今回[こんかい]	こんかい	
\\	まあ今回は許してあげよう。	まあ 今回[こんかい]は 許[ゆる]してあげよう。	まあ こんかい は ゆるして あげよう	
\\	まあ
\\	は 許[ゆる]してあげよう。		
\\	テント	テント	テント	
\\	みんなでテントを張りました。	みんなでテントを 張[は]りました。	みんな で てんと を はりました	
\\	みんなで
\\	を 張[は]りました。		
\\	回る	回[まわ]る	まわる	
\\	月は地球のまわりを回っています。	月[つき]は 地球[ちきゅう]のまわりを 回[まわ]っています。	つき は ちきゅう の まわり を まわって います	
\\	月[つき]は 地球[ちきゅう]のまわりを
\\	回す	回[まわ]す	まわす	
\\	ねじは左に回すと外れます。	ねじは 左[ひだり]に 回[まわ]すと 外[はず]れます。	ねじ は ひだり に まわす と はずれます	
\\	ねじは 左[ひだり]に
\\	と 外[はず]れます。		
\\	個人	個人[こじん]	こじん	
\\	これは私個人の意見です。	これは 私[わたし] 個人[こじん]の 意見[いけん]です。	これ は わたし こじん の いけん です	
\\	これは 私[わたし]
\\	の 意見[いけん]です。		
\\	担当	担当[たんとう]	たんとう	
\\	私はセールスを担当しています。	私[わたし]はセールスを 担当[たんとう]しています。	わたし は せーるす を たんとう して います	
\\	私[わたし]はセールスを
\\	しています。		
\\	ボート	ボート	ボート	
\\	船	池でボートに乗りました。	池[いけ]でボートに 乗[の]りました。	いけ で ぼーと に のりました	
\\	池[いけ]で
\\	に 乗[の]りました。		
\\	当たる	当[あ]たる	あたる	
\\	ボールが彼の頭に当たった。	ボールが 彼[かれ]の 頭[あたま]に 当[あ]たった。	ぼーる が かれ の あたま に あたった	
\\	ボールが 彼[かれ]の 頭[あたま]に
\\	当時	当時[とうじ]	とうじ	
\\	彼女は当時、まだ3才だった。	彼女[かのじょ]は 当時[とうじ]、まだ 3才[さんさい]だった。	かのじょ は とうじ まだ さんさい だった	
\\	彼女[かのじょ]は
\\	、まだ 3才[さんさい]だった。		
\\	本当	本当[ほんとう]	ほんとう	
\\	その話は本当ですか。	その 話[はなし]は 本当[ほんとう]ですか。	その はなし は ほんとう です か	
\\	その 話[はなし]は
\\	ですか。		
\\	当然	当然[とうぜん]	とうぜん	
\\	彼女が怒るのも当然だ。	彼女[かのじょ]が 怒[おこ]るのも 当然[とうぜん]だ。	かのじょ が おこる の も とうぜん だ	
\\	彼女[かのじょ]が 怒[おこ]るのも
\\	だ。		
\\	ボール	ボール	ボール	
\\	彼はボールを投げた。	彼[かれ]はボールを 投[な]げた。	かれ は ぼーる を なげた	
\\	彼[かれ]は
\\	を 投[な]げた。		
\\	全然	全然[ぜんぜん]	ぜんぜん	
\\	この本は全然面白くなかった。	この 本[ほん]は 全然[ぜんぜん] 面白[おもしろ]くなかった。	この ほん は ぜんぜん おもしろく なかった	
\\	この 本[ほん]は
\\	面白[おもしろ]くなかった。		
\\	方法	方法[ほうほう]	ほうほう	
\\	いい方法を思いつきました。	いい 方法[ほうほう]を 思[おも]いつきました。	いい ほうほう を おもいつきました	
\\	いい
\\	を 思[おも]いつきました。		
\\	法律	法律[ほうりつ]	ほうりつ	
\\	新しい法律ができた。	新[あたら]しい 法律[ほうりつ]ができた。	あたらしい ほうりつ が できた	
\\	新[あたら]しい
\\	ができた。		
\\	規則	規則[きそく]	きそく	
\\	あの会社の規則は厳しいです。	あの 会社[かいしゃ]の 規則[きそく]は 厳[きび]しいです。	あの かいしゃ の きそく は きびしい です	
\\	あの 会社[かいしゃ]の
\\	は 厳[きび]しいです。		
\\	オートバイ	オートバイ	オートバイ	
\\	彼はオートバイに乗っている。	彼[かれ]はオートバイに 乗[の]っている。	かれ は おーとばい に のって いる	
\\	彼[かれ]は
\\	に 乗[の]っている。		
\\	経験	経験[けいけん]	けいけん	
\\	今日の試合はいい経験になりました。	今日[きょう]の 試合[しあい]はいい 経験[けいけん]になりました。	きょう の しあい は いい けいけん に なりました	
\\	今日[きょう]の 試合[しあい]はいい
\\	になりました。		
\\	経つ	経[た]つ	たつ	
\\	あれから14年が経ちました。	あれから 14年[じゅうよねん]が 経[た]ちました。	あれ から じゅうよねん が たちました	
\\	あれから 14年[じゅうよねん]が
\\	経済	経済[けいざい]	けいざい	
\\	大学で経済を勉強しました。	大学[だいがく]で 経済[けいざい]を 勉強[べんきょう]しました。	だいがく で けいざい を べんきょう しました	
\\	大学[だいがく]で
\\	を 勉強[べんきょう]しました。		
\\	経営	経営[けいえい]	けいえい	
\\	我が社の経営はうまくいっています。	我[わ]が 社[しゃ]の 経営[けいえい]はうまくいっています。	わがしゃ の けいえい は うまく いって います	
\\	我[わ]が 社[しゃ]の
\\	はうまくいっています。		
\\	ひどい	ひどい	ひどい	
\\	妹とひどい喧嘩をした。	妹[いもうと]とひどい 喧嘩[けんか]をした。	いもうと と ひどい けんか を した	
\\	妹[いもうと]と
\\	喧嘩[けんか]をした。		
\\	株	株[かぶ]	かぶ	
\\	最近株を始めました。	最近[さいきん] 株[かぶ]を 始[はじ]めました。	さいきん かぶ を はじめました	
\\	最近[さいきん]
\\	を 始[はじ]めました。		
\\	企業	企業[きぎょう]	きぎょう	
\\	彼女はアメリカの企業で働いています。	彼女[かのじょ]はアメリカの 企業[きぎょう]で 働[はたら]いています。	かのじょ は あめりか の きぎょう で はたらいて います	
\\	彼女[かのじょ]はアメリカの
\\	で 働[はたら]いています。		
\\	作業	作業[さぎょう]	さぎょう	
\\	作業するにはもっと広いスペースが必要だ。	作業[さぎょう]するにはもっと 広[ひろ]いスペースが 必要[ひつよう]だ。	さぎょう する に は もっと ひろい すぺーす が ひつよう だ	
\\	するにはもっと 広[ひろ]いスペースが 必要[ひつよう]だ。		
\\	産業	産業[さんぎょう]	さんぎょう	
\\	日本の自動車産業は世界的に有名だ。	日本[にほん]の 自動車[じどうしゃ] 産業[さんぎょう]は 世界的[せかいてき]に 有名[ゆうめい]だ。	にほん の じどうしゃ さんぎょう は せかいてき に ゆうめい だ	
\\	日本[にほん]の 自動車[じどうしゃ]
\\	は 世界的[せかいてき]に 有名[ゆうめい]だ。		
\\	あなた	あなた	あなた	
\\	この本、あなたにあげます。	この 本[ほん]、あなたにあげます。	この ほん あなた に あげます	
\\	この 本[ほん]、
\\	にあげます。		
\\	工業	工業[こうぎょう]	こうぎょう	
\\	そこは工業都市だ。	そこは 工業[こうぎょう] 都市[とし]だ。	そこ は こうぎょう とし だ	
\\	そこは
\\	都市[とし]だ。		
\\	商業	商業[しょうぎょう]	しょうぎょう	
\\	この町では商業が盛んだ。	この 町[まち]では 商業[しょうぎょう]が 盛[さか]んだ。	この まち で は しょうぎょう が さかん だ	
\\	この 町[まち]では
\\	が 盛[さか]んだ。		
\\	利用	利用[りよう]	りよう	
\\	私はよく図書館を利用します。	私[わたし]はよく 図書館[としょかん]を 利用[りよう]します。	わたし は よく としょかん を りよう します	
\\	私[わたし]はよく 図書館[としょかん]を
\\	します。		
\\	便利	便利[べんり]	べんり	
\\	インターネットはとても便利です。	インターネットはとても 便利[べんり]です。	いんたーねっと は とても べんり です	
\\	インターネットはとても
\\	です。		
\\	スイッチ	スイッチ	スイッチ	
\\	彼はカーラジオのスイッチを入れた。	彼[かれ]はカーラジオのスイッチを 入[い]れた。	かれ は かーらじお の すいっち を いれた	
\\	彼[かれ]はカーラジオの
\\	を 入[い]れた。		
\\	技術	技術[ぎじゅつ]	ぎじゅつ	
\\	彼は非常に高い技術を持っている。	彼[かれ]は 非常[ひじょう]に 高[たか]い 技術[ぎじゅつ]を 持[も]っている。	かれ は ひじょう に たかい ぎじゅつ を もって いる	
\\	彼[かれ]は 非常[ひじょう]に 高[たか]い
\\	を 持[も]っている。		
\\	手術	手術[しゅじゅつ]	しゅじゅつ	
\\	父は胸の手術をした。	父[ちち]は 胸[むね]の 手術[しゅじゅつ]をした。	ちち は むね の しゅじゅつ を した	
\\	父[ちち]は 胸[むね]の
\\	をした。		
\\	製造	製造[せいぞう]	せいぞう	
\\	この工場ではエアコンを製造しています。	この 工場[こうじょう]ではエアコンを 製造[せいぞう]しています。	この こうじょう で は えあこん を せいぞう して います	
\\	この 工場[こうじょう]ではエアコンを
\\	しています。		
\\	必ず	必[かなら]ず	かならず	
\\	必ずシートベルトを着けて下さい。	必[かなら]ずシートベルトを 着[つ]けて 下[くだ]さい。	かならず しーとべると を つけて ください	
\\	シートベルトを 着[つ]けて 下[くだ]さい。		
\\	プレゼント	プレゼント	プレゼント	
\\	誕生日にプレゼントをもらいました。	誕生日[たんじょうび]にプレゼントをもらいました。	たんじょうび に ぷれぜんと を もらいました	
\\	誕生日[たんじょうび]に
\\	をもらいました。		
\\	必要	必要[ひつよう]	ひつよう	
\\	私にはたくさんのお金が必要だ。	私[わたし]にはたくさんのお 金[かね]が 必要[ひつよう]だ。	わたし に は たくさん の おかね が ひつよう だ	
\\	私[わたし]にはたくさんのお 金[かね]が
\\	だ。		
\\	重要	重要[じゅうよう]	じゅうよう	
\\	これは重要な書類です。	これは 重要[じゅうよう]な 書類[しょるい]です。	これ は じゅうよう な しょるい です	
\\	これは
\\	な 書類[しょるい]です。		
\\	要求	要求[ようきゅう]	ようきゅう	
\\	彼は私の要求にこたえた。	彼[かれ]は 私[わたし]の 要求[ようきゅう]にこたえた。	かれ は わたし の ようきゅう に こたえた	
\\	彼[かれ]は 私[わたし]の
\\	にこたえた。		
\\	目覚まし時計	目覚[めざ]まし 時計[どけい]	めざましどけい	
\\	7時に目覚まし時計が鳴りました。	7時[しちじ]に 目覚[めざ]まし 時計[どけい]が 鳴[な]りました。	しちじ に めざましどけい が なりました	
\\	7時[しちじ]に
\\	が 鳴[な]りました。		
\\	いつでも	いつでも	いつでも	
\\	いつでもうちに来てください。	いつでもうちに 来[き]てください。	いつでも うち に きて ください	
\\	うちに 来[き]てください。		
\\	計算	計算[けいさん]	けいさん	
\\	その計算は間違っている。	その 計算[けいさん]は 間違[まちが]っている。	その けいさん は まちがって いる	
\\	その
\\	は 間違[まちが]っている。		
\\	引き算	引[ひ]き 算[ざん]	ひきざん	
\\	娘は学校で引き算を習っている。	娘[むすめ]は 学校[がっこう]で 引[ひ]き 算[ざん]を 習[なら]っている。	むすめ は がっこう で ひきざん を ならって いる	
\\	娘[むすめ]は 学校[がっこう]で
\\	を 習[なら]っている。		
\\	足し算	足[た]し 算[ざん]	たしざん	
\\	娘は学校で足し算を習った。	娘[むすめ]は 学校[がっこう]で 足[た]し 算[ざん]を 習[なら]った。	むすめ は がっこう で たしざん を ならった	
\\	娘[むすめ]は 学校[がっこう]で
\\	を 習[なら]った。		
\\	交差点	交差点[こうさてん]	こうさてん	
\\	次の交差点を左に曲がってください。	次[つぎ]の 交差点[こうさてん]を 左[ひだり]に 曲[ま]がってください。	つぎ の こうさてん を ひだり に まがって ください	
\\	次[つぎ]の
\\	を 左[ひだり]に 曲[ま]がってください。		
\\	テニス	テニス	テニス	
\\	彼らはよくテニスをしています。	彼[かれ]らはよくテニスをしています。	かれら は よく てにす を して います	
\\	彼[かれ]らはよく
\\	をしています。		
\\	割る	割[わ]る	わる	
\\	皿を落として割った。	皿[さら]を 落[お]として 割[わ]った。	さら を おとして わった	
\\	皿[さら]を 落[お]として
\\	割れる	割[わ]れる	われる	
\\	(が)		
\\	コップが落ちて割れた。	コップが 落[お]ちて 割[わ]れた。	こっぷ が おちて われた	
\\	コップが 落[お]ちて
\\	割り算	割[わ]り 算[ざん]	わりざん	
\\	割り算は小学校で習います。	割[わ]り 算[ざん]は 小学校[しょうがっこう]で 習[なら]います。	わりざん は しょうがっこう で ならいます	
\\	は 小学校[しょうがっこう]で 習[なら]います。		
\\	残る	残[のこ]る	のこる	
\\	料理がたくさん残りました。	料理[りょうり]がたくさん 残[のこ]りました。	りょうり が たくさん のこりました	
\\	料理[りょうり]がたくさん
\\	こちら	こちら	こちら	
\\	受付はこちらです。	受付[うけつけ]はこちらです。	うけつけ は こちら です	
\\	受付[うけつけ]は
\\	です。		
\\	残す	残[のこ]す	のこす	
\\	彼女はメッセージを残しました。	彼女[かのじょ]はメッセージを 残[のこ]しました。	かのじょ は めっせーじ を のこしました	
\\	彼女[かのじょ]はメッセージを
\\	返す	返[かえ]す	かえす	
\\	図書館に本を返した。	図書館[としょかん]に 本[ほん]を 返[かえ]した。	としょかん に ほん を かえした	
\\	図書館[としょかん]に 本[ほん]を
\\	返事	返事[へんじ]	へんじ	
\\	手紙の返事を出しました。	手紙[てがみ]の 返事[へんじ]を 出[だ]しました。	てがみ の へんじ を だしました	
\\	手紙[てがみ]の
\\	を 出[だ]しました。		
\\	借りる	借[か]りる	かりる	
\\	彼にビデオを借りました。	彼[かれ]にビデオを 借[か]りました。	かれ に びでお を かりました	
\\	彼[かれ]にビデオを
\\	ボタン	ボタン	ボタン	
\\	ボタンを押してください。	ボタンを 押[お]してください。	ぼたん を おして ください	
\\	を 押[お]してください。		
\\	貸す	貸[か]す	かす	
\\	私は彼に本を貸しています。	私[わたし]は 彼[かれ]に 本[ほん]を 貸[か]しています。	わたし は かれ に ほん を かして います	
\\	私[わたし]は 彼[かれ]に 本[ほん]を
\\	貸し出す	貸[か]し 出[だ]す	かしだす	
\\	その本は貸し出し中です。	その 本[ほん]は 貸[か]し 出[だ]し 中[ちゅう]です。	その ほん は かしだしちゅう です	
\\	その 本[ほん]は
\\	中[ちゅう]です。		
\\	申し込む	申[もう]し 込[こ]む	もうしこむ	
\\	彼女はそのセミナーに申し込んだ。	彼女[かのじょ]はそのセミナーに 申[もう]し 込[こ]んだ。	かのじょ は その せみなー に もうしこんだ	
\\	彼女[かのじょ]はそのセミナーに
\\	期待	期待[きたい]	きたい	
\\	みんな私たちに期待しています。	みんな 私[わたし]たちに 期待[きたい]しています。	みんな わたしたち に きたい して います	
\\	みんな 私[わたし]たちに
\\	しています。		
\\	おかしい	おかしい	おかしい	
\\	彼の話はおかしかった。	彼[かれ]の 話[はなし]はおかしかった。	かれ の はなし は おかしかった	
\\	彼[かれ]の 話[はなし]は
\\	期間	期間[きかん]	きかん	
\\	テスト期間は10日から15日までだ。	テスト 期間[きかん]は 10日[とおか]から 15日[じゅうごにち]までだ。	てすと きかん は とおか から じゅうごにち まで だ	
\\	テスト
\\	は 10日[とおか]から 15日[じゅうごにち]までだ。		
\\	時期	時期[じき]	じき	
\\	今はあなたにとって大事な時期です。	今[いま]はあなたにとって 大事[だいじ]な 時期[じき]です。	いま は あなた に とって だいじ な じき です	
\\	今[いま]はあなたにとって 大事[だいじ]な
\\	です。		
\\	限る	限[かぎ]る	かぎる	
\\	このサービスは週末に限ります	このサービスは 週末[しゅうまつ]に 限[かぎ]ります	この さーびす は しゅうまつ に かぎります	
\\	このサービスは 週末[しゅうまつ]に
\\	急ぐ	急[いそ]ぐ	いそぐ	
\\	私たちは駅へ急ぎました。	私[わたし]たちは 駅[えき]へ 急[いそ]ぎました。	わたしたち は えき へ いそぎました	
\\	私[わたし]たちは 駅[えき]へ
\\	コート	コート	コート	
\\	新しいコートでテニスをしました。	新[あたら]しいコートでテニスをしました。	あたらしい こーと で てにす を しました	
\\	新[あたら]しい
\\	でテニスをしました。		
\\	急に	急[きゅう]に	きゅうに	
\\	急に用事を思い出した。	急[きゅう]に 用事[ようじ]を 思[おも]い 出[だ]した。	きゅうに ようじ を おもいだした	
\\	用事[ようじ]を 思[おも]い 出[だ]した。		
\\	急	急[きゅう]	きゅう	
\\	急な坂道を上った。	急[きゅう]な 坂道[さかみち]を 上[のぼ]った。	きゅう な さかみち を のぼった	
\\	な 坂道[さかみち]を 上[のぼ]った。		
\\	急行	急行[きゅうこう]	きゅうこう	
\\	ちょうど急行電車が来た。	ちょうど 急行[きゅうこう] 電車[でんしゃ]が 来[き]た。	ちょうど きゅうこう でんしゃ が きた	
\\	ちょうど
\\	電車[でんしゃ]が 来[き]た。		
\\	切れる	切[き]れる	きれる	
\\	このはさみはよく切れますね。	このはさみはよく 切[き]れますね。	この はさみ は よく きれます ね	
\\	このはさみはよく
\\	ね。		
\\	ドラマ	ドラマ	ドラマ	
\\	私はドラマを見るのが好きです。	私[わたし]はドラマを 見[み]るのが 好[す]きです。	わたし は どらま を みる の が すき です	
\\	私[わたし]は
\\	を 見[み]るのが 好[す]きです。		
\\	切手	切手[きって]	きって	
\\	郵便局で切手を買いました。	郵便局[ゆうびんきょく]で 切手[きって]を 買[か]いました。	ゆうびんきょく で きって を かいました	
\\	郵便局[ゆうびんきょく]で
\\	を 買[か]いました。		
\\	売り切れる	売[う]り 切[き]れる	うりきれる	
\\	その本は直ぐ売り切れた。	その 本[ほん]は 直[す]ぐ 売[う]り 切[き]れた。	その ほん は すぐ うりきれた	
\\	その 本[ほん]は 直[す]ぐ
\\	売り切れ	売[う]り 切[き]れ	うりきれ	
\\	チケットはもう売り切れだって。	チケットはもう 売[う]り 切[き]れだって。	ちけっと は もう うりきれ だって	
\\	チケットはもう
\\	だって。		
\\	大切	大切[たいせつ]	たいせつ	
\\	これは母が大切にしていた指輪です。	これは 母[はは]が 大切[たいせつ]にしていた 指輪[ゆびわ]です。	これ は はは が たいせつ に して いた ゆびわ です	
\\	これは 母[はは]が
\\	にしていた 指輪[ゆびわ]です。		
\\	ビザ	ビザ	ビザ	
\\	学生ビザを持っています。	学生[がくせい]ビザを 持[も]っています。	がくせい びざ を もって います	
\\	学生[がくせい]
\\	を 持[も]っています。		
\\	切符	切符[きっぷ]	きっぷ	
\\	東京までの切符を買った。	東京[とうきょう]までの 切符[きっぷ]を 買[か]った。	とうきょう まで の きっぷ を かった	
\\	東京[とうきょう]までの
\\	を 買[か]った。		
\\	入場券	入場券[にゅうじょうけん]	にゅうじょうけん	
\\	入場券は3000円です。	入場券[にゅうじょうけん]は 3000円[さんぜんえん]です。	にゅうじょうけん は さんぜんえん です	
\\	は 3000円[さんぜんえん]です。		
\\	家賃	家賃[やちん]	やちん	
\\	ここの家賃は12万円です。	ここの 家賃[やちん]は 12万円[じゅうにまんえん]です。	ここ の やちん は じゅうにまんえん です	
\\	ここの
\\	は 12万円[じゅうにまんえん]です。		
\\	時代	時代[じだい]	じだい	
\\	今は便利さとスピードの時代だ。	今[いま]は 便利[べんり]さとスピードの 時代[じだい]だ。	いま は べんりさ と すぴーど の じだい だ	
\\	今[いま]は 便利[べんり]さとスピードの
\\	だ。		
\\	ポケット	ポケット	ポケット	
\\	財布をポケットにしまった。	財布[さいふ]をポケットにしまった。	さいふ を ぽけっと に しまった	
\\	財布[さいふ]を
\\	にしまった。		
\\	代わる	代[か]わる	かわる	
\\	上司に代わって会議に出た。	上司[じょうし]に 代[か]わって 会議[かいぎ]に 出[で]た。	じょうし に かわって かいぎ に でた	
\\	上司[じょうし]に
\\	会議[かいぎ]に 出[で]た。		
\\	代える	代[か]える	かえる	
\\	1を) 
\\	2に), 
\\	社長に代えて部長を出席させます。	社長[しゃちょう]に 代[か]えて 部長[ぶちょう]を 出席[しゅっせき]させます。	しゃちょう に かえて ぶちょう を しゅっせき させます	
\\	社長[しゃちょう]に
\\	部長[ぶちょう]を 出席[しゅっせき]させます。		
\\	指	指[ゆび]	ゆび	
\\	彼は指が太い。	彼[かれ]は 指[ゆび]が 太[ふと]い。	かれ は ゆび が ふとい	
\\	彼[かれ]は
\\	が 太[ふと]い。		
\\	決定	決定[けってい]	けってい	
\\	会議で重要な決定がありました。	会議[かいぎ]で 重要[じゅうよう]な 決定[けってい]がありました。	かいぎ で じゅうよう な けってい が ありました	
\\	会議[かいぎ]で 重要[じゅうよう]な
\\	がありました。		
\\	そろそろ	そろそろ	そろそろ	
\\	そろそろ始めましょうか。	そろそろ 始[はじ]めましょうか。	そろそろ はじめましょう か	
\\	始[はじ]めましょうか。		
\\	一定	一定[いってい]	いってい	
\\	申し込むには一定の資格が要る。	申[もう]し 込[こ]むには 一定[いってい]の 資格[しかく]が 要[い]る。	もうしこむ に は いってい の しかく が いる	
\\	申[もう]し 込[こ]むには
\\	の 資格[しかく]が 要[い]る。		
\\	定期券	定期券[ていきけん]	ていきけん	
\\	定期券は1万2千円でした。	定期券[ていきけん]は 1万2千円[いちまんにせんえん]でした。	ていきけん は いちまんにせんえん でした	
\\	は 1万2千円[いちまんにせんえん]でした。		
\\	予定	予定[よてい]	よてい	
\\	今日の予定を教えてください。	今日[きょう]の 予定[よてい]を 教[おし]えてください。	きょう の よてい を おしえて ください	
\\	今日[きょう]の
\\	を 教[おし]えてください。		
\\	天気予報	天気予報[てんきよほう]	てんきよほう	
\\	明日の天気予報は雨です。	明日[あした]の 天気予報[てんきよほう]は 雨[あめ]です。	あした の てんきよほう は あめ です	
\\	明日[あした]の
\\	は 雨[あめ]です。		
\\	ぶどう	ぶどう	ぶどう	
\\	私はぶどうが好きです。	私[わたし]はぶどうが 好[す]きです。	わたし は ぶどう が すき です	
\\	私[わたし]は
\\	が 好[す]きです。		
\\	予習	予習[よしゅう]	よしゅう	
\\	明日の予習をしましょう。	明日[あした]の 予習[よしゅう]をしましょう。	あした の よしゅう を しましょう	
\\	明日[あした]の
\\	をしましょう。		
\\	予約	予約[よやく]	よやく	
\\	レストランを予約しました。	レストランを 予約[よやく]しました。	れすとらん を よやく しました	
\\	レストランを
\\	しました。		
\\	約束	約束[やくそく]	やくそく	
\\	約束は守ります。	約束[やくそく]は 守[まも]ります。	やくそく は まもります	
\\	は 守[まも]ります。		
\\	変わる	変[か]わる	かわる	
\\	が		
\\	信号が青に変わりました。	信号[しんごう]が 青[あお]に 変[か]わりました。	しんごう が あお に かわりました。	
\\	信号[しんごう]が 青[あお]に
\\	めったに	めったに	めったに	
\\	彼女はめったに怒りません。	彼女[かのじょ]はめったに 怒[おこ]りません。	かのじょ は めったに おこりません 。	
\\	彼女[かのじょ]は
\\	怒[おこ]りません。		
\\	大変	大変[たいへん]	たいへん	
\\	大変なことが起こりました。	大変[たいへん]なことが 起[お]こりました。	たいへん な こと が おこりました	
\\	なことが 起[お]こりました。		
\\	変	変[へん]	へん	
\\	変な音が聞こえます。	変[へん]な 音[おと]が 聞[き]こえます。	へん な おと が きこえます	
\\	な 音[おと]が 聞[き]こえます。		
\\	変化	変化[へんか]	へんか	
\\	今年は変化の多い年でした。	今年[ことし]は 変化[へんか]の 多[おお]い 年[とし]でした。	ことし は へんか の おおい とし でした	
\\	今年[ことし]は
\\	の 多[おお]い 年[とし]でした。		
\\	強化	強化[きょうか]	きょうか	
\\	国は国語教育を強化しています。	国[くに]は 国語教育[こくご きょういく]を 強化[きょうか]しています。	くに は こくご きょういく を きょうか して います	
\\	国[くに]は 国語教育[こくご きょういく]を
\\	しています。		
\\	クリスマス	クリスマス	クリスマス	
\\	クリスマスにはケーキを食べます。	クリスマスにはケーキを 食[た]べます。	くりすます に は けーき を たべます	
\\	にはケーキを 食[た]べます。		
\\	文化	文化[ぶんか]	ぶんか	
\\	私はこの国の文化を勉強しています。	私[わたし]はこの 国[くに]の 文化[ぶんか]を 勉強[べんきょう]しています。	わたし は この くに の ぶんか を べんきょう して います	
\\	私[わたし]はこの 国[くに]の
\\	を 勉強[べんきょう]しています。		
\\	増える	増[ふ]える	ふえる	
\\	が		
\\	この町は人口が増えた。	この 町[まち]は 人口[じんこう]が 増[ふ]えた。	この まち は じんこう が ふえた	
\\	この 町[まち]は 人口[じんこう]が
\\	増やす	増[ふ]やす	ふやす	
\\	を		
\\	あの町は緑を増やしています。	あの 町[まち]は 緑[みどり]を 増[ふ]やしています。	あの まち は みどり を ふやして います	
\\	あの 町[まち]は 緑[みどり]を
\\	減る	減[へ]る	へる	
\\	が		
\\	体重がかなり減りました。	体重[たいじゅう]がかなり 減[へ]りました。	たいじゅう が かなり へりました	
\\	体重[たいじゅう]がかなり
\\	ネクタイ	ネクタイ	ネクタイ	
\\	父の日にネクタイをプレゼントした。	父[ちち]の 日[ひ]にネクタイをプレゼントした。	ちちのひ に ねくたい を ぷれぜんと した	
\\	父[ちち]の 日[ひ]に
\\	をプレゼントした。		
\\	減らす	減[へ]らす	へらす	
\\	を		
\\	最近、食事を減らしています。	最近[さいきん]、 食事[しょくじ]を 減[へ]らしています。	さいきん しょくじ を へらして います	
\\	最近[さいきん]、 食事[しょくじ]を
\\	乗り物	乗[の]り 物[もの]	のりもの	
\\	自転車は便利な乗り物です。	自転車[じてんしゃ]は 便利[べんり]な 乗[の]り 物[もの]です。	じてんしゃ は べんり な のりもの です	
\\	自転車[じてんしゃ]は 便利[べんり]な
\\	です。		
\\	降りる	降[お]りる	おりる	
\\	次の駅で降ります。	次[つぎ]の 駅[えき]で 降[お]ります。	つぎ の えき で おります。	
\\	次[つぎ]の 駅[えき]で
\\	降ろす	降[お]ろす	おろす	
\\	彼は車から荷物を降ろした。	彼[かれ]は 車[くるま]から 荷物[にもつ]を 降[お]ろした。	かれ は くるま から にもつ を おろした	
\\	彼[かれ]は 車[くるま]から 荷物[にもつ]を
\\	バイオリン	バイオリン	バイオリン	
\\	彼女はバイオリンを習っています。	彼女[かのじょ]はバイオリンを 習[なら]っています。	かのじょ は ばいおりん を ならって います	
\\	彼女[かのじょ]は
\\	を 習[なら]っています。		
\\	降る	降[ふ]る	ふる	
\\	激しい雨が降っています。	激[はげ]しい 雨[あめ]が 降[ふ]っています。	はげしい あめ が ふって います	
\\	激[はげ]しい 雨[あめ]が
\\	着く	着[つ]く	つく	
\\	午後8時に大阪に着きます。	午後8時[ごご はちじ]に 大阪[おおさか]に 着[つ]きます。	ごご はちじ に おおさか に つきます	
\\	午後8時[ごご はちじ]に 大阪[おおさか]に
\\	着物	着物[きもの]	きもの	
\\	彼女は着物がよく似合います。	彼女[かのじょ]は 着物[きもの]がよく 似合[にあ]います。	かのじょ は きもの が よく にあいます	
\\	彼女[かのじょ]は
\\	がよく 似合[にあ]います。		
\\	下着	下着[したぎ]	したぎ	
\\	私は下着を手で洗う。	私[わたし]は 下着[したぎ]を 手[て]で 洗[あら]う。	わたし は したぎ を て で あらう	
\\	私[わたし]は
\\	を 手[て]で 洗[あら]う。		
\\	パスポート	パスポート	パスポート	
\\	パスポートが見つからない。	パスポートが 見[み]つからない。	ぱすぽーと が みつからない	
\\	が 見[み]つからない。		
\\	上着	上着[うわぎ]	うわぎ	
\\	暑いので上着を脱ぎました。	暑[あつ]いので 上着[うわぎ]を 脱[ぬ]ぎました。	あつい の で うわぎ を ぬぎました	
\\	暑[あつ]いので
\\	を 脱[ぬ]ぎました。		
\\	着せる	着[き]せる	きせる	
\\	娘にかわいいドレスを着せた。	娘[むすめ]にかわいいドレスを 着[き]せた。	むすめ に かわいい どれす を きせた	
\\	娘[むすめ]にかわいいドレスを
\\	脱ぐ	脱[ぬ]ぐ	ぬぐ	
\\	靴を脱いでください。	靴[くつ]を 脱[ぬ]いでください。	くつ を ぬいで ください	
\\	靴[くつ]を
\\	ください。		
\\	立場	立場[たちば]	たちば	
\\	彼女は自分の立場を分かっていない。	彼女[かのじょ]は 自分[じぶん]の 立場[たちば]を 分[わ]かっていない。	かのじょ は じぶん の たちば を わかって いない	
\\	彼女[かのじょ]は 自分[じぶん]の
\\	を 分[わ]かっていない。		
\\	バッグ	バッグ	バッグ	
\\	私は黒いバッグを持っています。	私[わたし]は 黒[くろ]いバッグを 持[も]っています。	わたし は くろい ばっぐ を もって います	
\\	私[わたし]は 黒[くろ]い
\\	を 持[も]っています。		
\\	目立つ	目立[めだ]つ	めだつ	
\\	彼女の大きな帽子はとても目立つ。	彼女[かのじょ]の 大[おお]きな 帽子[ぼうし]はとても 目立[めだ]つ。	かのじょ の おおき な ぼうし は とても めだつ	
\\	彼女[かのじょ]の 大[おお]きな 帽子[ぼうし]はとても
\\	立てる	立[た]てる	たてる	
\\	彼はケーキにろうそくを立てた。	彼[かれ]はケーキにろうそくを 立[た]てた。	かれ は けーき に ろうそく を たてた	
\\	彼[かれ]はケーキにろうそくを
\\	役に立つ	役[やく]に 立[た]つ	やくにたつ	
\\	私は人々の役に立ちたいと思っています。	私[わたし]は 人々[ひとびと]の 役[やく]に 立[た]ちたいと 思[おも]っています。	わたし は ひとびと の やくにたちたい と おもって います	
\\	私[わたし]は 人々[ひとびと]の
\\	と 思[おも]っています。		
\\	出席	出席[しゅっせき]	しゅっせき	
\\	午後は会議に出席します。	午後[ごご]は 会議[かいぎ]に 出席[しゅっせき]します。	ごご は かいぎ に しゅっせき します	
\\	午後[ごご]は 会議[かいぎ]に
\\	します。		
\\	ゴム	ゴム	ゴム	
\\	ゴムが伸びてしまった。	ゴムが 伸[の]びてしまった。	ごむ が のびて しまった	
\\	が 伸[の]びてしまった。		
\\	席	席[せき]	せき	
\\	この席、空いてますか。	この 席[せき]、 空[あ]いてますか。	この せき あいてます か	
\\	この
\\	、 空[あ]いてますか。		
\\	欠席	欠席[けっせき]	けっせき	
\\	風邪のため、今日は欠席します。	風邪[かぜ]のため、 今日[きょう]は 欠席[けっせき]します。	かぜ の ため、 きょう は けっせき します	
\\	風邪[かぜ]のため、 今日[きょう]は
\\	します。		
\\	次男	次男[じなん]	じなん	
\\	次男は今、海外にいます。	次男[じなん]は 今[いま]、 海外[かいがい]にいます。	じなん は いま かいがい に います	
\\	は 今[いま]、 海外[かいがい]にいます。		
\\	次女	次女[じじょ]	じじょ	
\\	うちの次女は春から中学生です。	うちの 次女[じじょ]は 春[はる]から 中学生[ちゅうがくせい]です。	うち の じじょ は はる から ちゅうがくせい です	
\\	うちの
\\	は 春[はる]から 中学生[ちゅうがくせい]です。		
\\	ギター	ギター	ギター	
\\	彼女はギターが得意です。	彼女[かのじょ]はギターが 得意[とくい]です。	かのじょ は ぎたー が とくい です	
\\	彼女[かのじょ]は
\\	が 得意[とくい]です。		
\\	運ぶ	運[はこ]ぶ	はこぶ	
\\	いすを二階に運んでください。	いすを 二階[にかい]に 運[はこ]んでください。	いす を にかい に はこんで ください	
\\	いすを 二階[にかい]に
\\	ください。		
\\	運転	運転[うんてん]	うんてん	
\\	父は安全運転だ。	父[ちち]は 安全[あんぜん] 運転[うんてん]だ。	ちち は あんぜん うんてん だ	
\\	父[ちち]は 安全[あんぜん]
\\	だ。		
\\	運転手	運転手[うんてんしゅ]	うんてんしゅ	
\\	彼はタクシーの運転手です。	彼[かれ]はタクシーの 運転手[うんてんしゅ]です。	かれ は たくしー の うんてんしゅ です	
\\	彼[かれ]はタクシーの
\\	です。		
\\	転ぶ	転[ころ]ぶ	ころぶ	
\\	お祖母ちゃんがお風呂場で転んだ。	お 祖母[ばあ]ちゃんがお 風呂場[ふろば]で 転[ころ]んだ。	おばあちゃん が おふろば で ころんだ	
\\	お 祖母[ばあ]ちゃんがお 風呂場[ふろば]で
\\	セーター	セーター	セーター	
\\	このセーターはウールだ。	このセーターはウールだ。	この せーたー は うーる だ	
\\	この
\\	はウールだ。		
\\	移る	移[うつ]る	うつる	
\\	が		
\\	彼女は新しい会社に移った。	彼女[かのじょ]は 新[あたら]しい 会社[かいしゃ]に 移[うつ]った。	かのじょ は あたらしい かいしゃ に うつった	
\\	彼女[かのじょ]は 新[あたら]しい 会社[かいしゃ]に
\\	移す	移[うつ]す	うつす	
\\	机を窓のそばに移しました。	机[つくえ]を 窓[まど]のそばに 移[うつ]しました。	つくえ を まど のそばに うつしました	
\\	机[つくえ]を 窓[まど]の 傍[そば]に
\\	動き	動[うご]き	うごき	
\\	まず相手の動きを見ましょう。	まず 相手[あいて]の 動[うご]きを 見[み]ましょう。	まず あいて の うごき を みましょう	
\\	まず 相手[あいて]の
\\	を 見[み]ましょう。		
\\	運動	運動[うんどう]	うんどう	
\\	彼はもっと運動した方がいい。	彼[かれ]はもっと 運動[うんどう]した 方[ほう]がいい。	かれ は もっと うんどう した ほう が いい	
\\	彼[かれ]はもっと
\\	した 方[ほう]がいい。		
\\	やっぱり	やっぱり	やっぱり	
\\	やっぱり旅が大好きだ。	やっぱり 旅[たび]が 大好[だいす]きだ。	やっぱり たび が だいすき だ	
\\	旅[たび]が 大好[だいす]きだ。		
\\	活動	活動[かつどう]	かつどう	
\\	彼は地域の活動に参加した。	彼[かれ]は 地域[ちいき]の 活動[かつどう]に 参加[さんか]した。	かれ は ちいき の かつどう に さんか した	
\\	彼[かれ]は 地域[ちいき]の
\\	に 参加[さんか]した。		
\\	動物	動物[どうぶつ]	どうぶつ	
\\	私は動物が大好きです。	私[わたし]は 動物[どうぶつ]が 大好[だいす]きです。	わたし は どうぶつ が だいすき です	
\\	私[わたし]は
\\	が 大好[だいす]きです。		
\\	動かす	動[うご]かす	うごかす	
\\	部屋の家具を動かしました。	部屋[へや]の 家具[かぐ]を 動[うご]かしました。	へや の かぐ を うごかしました	
\\	部屋[へや]の 家具[かぐ]を
\\	不動産屋	不動産屋[ふどうさんや]	ふどうさんや	
\\	私の父は不動産屋です。	私[わたし]の 父[ちち]は 不動産屋[ふどうさんや]です。	わたし の ちち は ふどうさんや です	
\\	私[わたし]の 父[ちち]は
\\	です。		
\\	カーテン	カーテン	カーテン	
\\	朝、カーテンを開けた。	朝[あさ]、カーテンを 開[あ]けた。	あさ かーてん を あけた	
\\	朝[あさ]、
\\	を 開[あ]けた。		
\\	早く	早[はや]く	はやく	
\\	なるべく早く来て下さい。	なるべく 早[はや]く 来[き]て 下[くだ]さい。	なるべく はやく きて ください	
\\	なるべく
\\	来[き]て 下[くだ]さい。		
\\	早口	早口[はやくち]	はやくち	
\\	彼女は早口だ。	彼女[かのじょ]は 早口[はやくち]だ。	かのじょ は はやくち だ	
\\	彼女[かのじょ]は
\\	だ。		
\\	速さ	速[はや]さ	はやさ	
\\	新幹線の速さはどれぐらいですか。	新幹線[しんかんせん]の 速[はや]さはどれぐらいですか。	しんかんせん の はやさ は どれぐらいです か	
\\	新幹線[しんかんせん]の
\\	はどれ 位[くらい]ですか。		
\\	速達	速達[そくたつ]	そくたつ	
\\	これを速達で送りたいのですが。	これを 速達[そくたつ]で 送[おく]りたいのですが。	これ を そくたつ で おくりたい の です が	
\\	これを
\\	で 送[おく]りたいのですが。		
\\	シャツ	シャツ	シャツ	
\\	このシャツはアイロンが必要だ。	このシャツはアイロンが 必要[ひつよう]だ。	この しゃつ は あいろん が ひつよう だ	
\\	この
\\	はアイロンが 必要[ひつよう]だ。		
\\	遅れる	遅[おく]れる	おくれる	
\\	今朝彼女は学校に遅れました。	今朝彼女[けさ かのじょ]は 学校[がっこう]に 遅[おく]れました。	けさ かのじょ は がっこう に おくれました	
\\	今朝彼女[けさ かのじょ]は 学校[がっこう]に
\\	開始	開始[かいし]	かいし	
\\	運動会は9時開始です。	運動会[うんどうかい]は 9時[くじ] 開始[かいし]です。	うんどうかい は くじ かいし です	
\\	運動会[うんどうかい]は 9時[くじ]
\\	です。		
\\	始めに	始[はじ]めに	はじめに	
\\	始めにスープが出ます。	始[はじ]めにスープが 出[で]ます。	はじめに すーぷ が でます	
\\	スープが 出[で]ます。		
\\	現在	現在[げんざい]	げんざい	
\\	現在の気温は30度です。	現在[げんざい]の 気温[きおん]は 30度[さんじゅうど]です。	げんざい の きおん は さんじゅうど です	
\\	の 気温[きおん]は 30度[さんじゅうど]です。		
\\	ナイフ	ナイフ	ナイフ	
\\	ナイフで手を切った。	ナイフで 手[て]を 切[き]った。	ないふ で て を きった	
\\	で 手[て]を 切[き]った。		
\\	実現	実現[じつげん]	じつげん	
\\	夢を実現するには努力が必要です。	夢[ゆめ]を 実現[じつげん]するには 努力[どりょく]が 必要[ひつよう]です。	ゆめ を じつげん する に は どりょく が ひつよう です	
\\	夢[ゆめ]を
\\	するには 努力[どりょく]が 必要[ひつよう]です。		
\\	実施	実施[じっし]	じっし	
\\	現在、スペシャルキャンペーンを実施中です。	現在[げんざい]、スペシャルキャンペーンを 実施[じっし] 中[ちゅう]です。	げんざい すぺしゃるきゃんぺーん を じっしちゅう です	
\\	現在[げんざい]、スペシャルキャンペーンを
\\	中[ちゅう]です。		
\\	事実	事実[じじつ]	じじつ	
\\	それは全て事実ですか。	それは 全[すべ]て 事実[じじつ]ですか。	それ は すべて じじつ です か	
\\	それは 全[すべ]て
\\	ですか。		
\\	実行	実行[じっこう]	じっこう	
\\	彼はその計画を実行した。	彼[かれ]はその 計画[けいかく]を 実行[じっこう]した。	かれ は その けいかく を じっこう した	
\\	彼[かれ]はその 計画[けいかく]を
\\	した。		
\\	バイク	バイク	バイク	
\\	兄はバイクが大好きです。	兄[あに]はバイクが 大好[だいす]きです。	あに は ばいく が だいすき です	
\\	兄[あに]は
\\	が 大好[だいす]きです。		
\\	実験	実験[じっけん]	じっけん	
\\	科学の授業で実験をした。	科学[かがく]の 授業[じゅぎょう]で 実験[じっけん]をした。	かがく の じゅぎょう で じっけん を した	
\\	科学[かがく]の 授業[じゅぎょう]で
\\	をした。		
\\	通過	通過[つうか]	つうか	
\\	次の駅は通過します。	次[つぎ]の 駅[えき]は 通過[つうか]します。	つぎ の えき は つうか します	
\\	次[つぎ]の 駅[えき]は
\\	します。		
\\	過ぎる	過[す]ぎる	すぎる	
\\	時が過ぎるのは速い。	時[とき]が 過[す]ぎるのは 速[はや]い。	とき が すぎる の は はやい 。	
\\	時[とき]が
\\	のは 速[はや]い。		
\\	昼過ぎ	昼過[ひるす]ぎ	ひるすぎ	
\\	彼は昼過ぎに来ます。	彼[かれ]は 昼過[ひるす]ぎに 来[き]ます。	かれ は ひるすぎ に きます	
\\	彼[かれ]は
\\	に 来[き]ます。		
\\	いつか	いつか	いつか	
\\	僕はいつかアフリカに行きたい。	僕[ぼく]はいつかアフリカに 行[い]きたい。	ぼく は いつか あふりか に いきたい	
\\	僕[ぼく]は
\\	アフリカに 行[い]きたい。		
\\	過去	過去[かこ]	かこ	
\\	それは過去の話だ。	それは 過去[かこ]の 話[はなし]だ。	それ は かこ の はなし だ	
\\	それは
\\	の 話[はなし]だ。		
\\	開発	開発[かいはつ]	かいはつ	
\\	ダムの開発に住民は反対しています。	ダムの 開発[かいはつ]に 住民[じゅうみん]は 反対[はんたい]しています。	だむ の かいはつ に じゅうみん は はんたい して います	
\\	ダムの
\\	に 住民[じゅうみん]は 反対[はんたい]しています。		
\\	発生	発生[はっせい]	はっせい	
\\	交差点で事故が発生した。	交差点[こうさてん]で 事故[じこ]が 発生[はっせい]した。	こうさてん で じこ が はっせい した	
\\	交差点[こうさてん]で 事故[じこ]が
\\	した。		
\\	発言	発言[はつげん]	はつげん	
\\	会議で全員が発言した。	会議[かいぎ]で 全員[ぜんいん]が 発言[はつげん]した。	かいぎ で ぜんいん が はつげん した	
\\	会議[かいぎ]で 全員[ぜんいん]が
\\	した。		
\\	オレンジ	オレンジ	オレンジ	
\\	私はオレンジが好きです。	私[わたし]はオレンジが 好[す]きです。	わたし は おれんじ が すき です	
\\	私[わたし]は
\\	が 好[す]きです。		
\\	発見	発見[はっけん]	はっけん	
\\	新しい星が発見された。	新[あたら]しい 星[ほし]が 発見[はっけん]された。	あたらしい ほし が はっけん された	
\\	新[あたら]しい 星[ほし]が
\\	された。		
\\	出発	出発[しゅっぱつ]	しゅっぱつ	
\\	あと15分で出発です。	あと 15分[じゅうごふん]で 出発[しゅっぱつ]です。	あと じゅうごふん で しゅっぱつ です	
\\	あと 15分[じゅうごふん]で
\\	です。		
\\	発車	発車[はっしゃ]	はっしゃ	
\\	バスが発車します。	バスが 発車[はっしゃ]します。	ばす が はっしゃ します	
\\	バスが
\\	します。		
\\	発表	発表[はっぴょう]	はっぴょう	
\\	合格者が発表された。	合格者[ごうかくしゃ]が 発表[はっぴょう]された。	ごうかくしゃ が はっぴょう された	
\\	合格者[ごうかくしゃ]が
\\	された。		
\\	きつい	きつい	きつい	
\\	このシャツは少しきついです。	このシャツは 少[すこ]しきついです。	この しゃつ は すこし きつい です	
\\	このシャツは 少[すこ]し
\\	です。		
\\	表現	表現[ひょうげん]	ひょうげん	
\\	彼は歌で自分の気持ちを表現した。	彼[かれ]は 歌[うた]で 自分[じぶん]の 気持[きも]ちを 表現[ひょうげん]した。	かれ は うた で じぶん の きもち を ひょうげん した	
\\	彼[かれ]は 歌[うた]で 自分[じぶん]の 気持[きも]ちを
\\	した。		
\\	代表	代表[だいひょう]	だいひょう	
\\	彼がクラスの代表だ。	彼[かれ]がクラスの 代表[だいひょう]だ。	かれ が くらす の だいひょう だ	
\\	彼[かれ]がクラスの
\\	だ。		
\\	表	表[おもて]	おもて	
\\	表に人が来ています。	表[おもて]に 人[ひと]が 来[き]ています。	おもてに ひと が きて います	
\\	に 人[ひと]が 来[き]ています。		
\\	表	表[ひょう]	ひょう	
\\	この表を見てください。	この 表[ひょう]を 見[み]てください。	この ひょう を みて ください	
\\	この
\\	を 見[み]てください。		
\\	さっき	さっき	さっき	
\\	さっきの話を続けましょう。	さっきの 話[はなし]を 続[つづ]けましょう。	さっき の はなし を つづけましょう	
\\	の 話[はなし]を 続[つづ]けましょう。		
\\	手紙	手紙[てがみ]	てがみ	
\\	友人から手紙をもらいました。	友人[ゆうじん]から 手紙[てがみ]をもらいました。	ゆうじん から てがみ を もらいました	
\\	友人[ゆうじん]から
\\	をもらいました。		
\\	絵	絵[え]	え	
\\	これは有名な画家の絵です。	これは 有名[ゆうめい]な 画家[がか]の 絵[え]です。	これ は ゆうめい な がか の え です	
\\	これは 有名[ゆうめい]な 画家[がか]の
\\	です。		
\\	雑誌	雑誌[ざっし]	ざっし	
\\	この雑誌はよく売れています。	この 雑誌[ざっし]はよく 売[う]れています。	この ざっし は よく うれて います	
\\	この
\\	はよく 売[う]れています。		
\\	音	音[おと]	おと	
\\	雨の音が聞こえる。	雨[あめ]の 音[おと]が 聞[き]こえる。	あめ の おと が きこえる	
\\	雨[あめ]の
\\	が 聞[き]こえる。		
\\	テキスト	テキスト	テキスト	
\\	テキストを読んでください。	テキストを 読[よ]んでください。	てきすと を よんで ください	
\\	を 読[よ]んでください。		
\\	音	音[おと]	おと	
\\	このピアノは音があまり良くない。	このピアノは 音[おと]があまり 良[よ]くない。	この ぴあの は おと が あまり よく ない	
\\	このピアノは
\\	があまり 良[よ]くない。		
\\	音楽	音楽[おんがく]	おんがく	
\\	私は音楽を聞くのが好きだ。	私[わたし]は 音楽[おんがく]を 聞[き]くのが 好[す]きだ。	わたし は おんがく を きく の が すき だ	
\\	私[わたし]は
\\	を 聞[き]くのが 好[す]きだ。		
\\	薬	薬[くすり]	くすり	
\\	この薬を必ず飲んでください。	この 薬[くすり]を 必[かなら]ず 飲[の]んでください。	この くすり を かならず のんで ください	
\\	この
\\	を 必[かなら]ず 飲[の]んでください。		
\\	歌	歌[うた]	うた	
\\	私はその歌を知らなかった。	私[わたし]はその 歌[うた]を 知[し]らなかった。	わたし は その うた を しらなかった 。	
\\	私[わたし]はその
\\	を 知[し]らなかった。		
\\	ビニール袋	ビニール 袋[ぶくろ]	ビニールぶくろ	
\\	ゴミはそのビニール袋に入れてください。	ゴミはそのビニール 袋[ぶくろ]に 入[い]れてください。	ごみ は その びにーるぶくろ に いれて ください	
\\	ゴミはその
\\	袋[ぶくろ]に 入[い]れてください。		
\\	歌手	歌手[かしゅ]	かしゅ	
\\	その歌手は歌が下手だ。	その 歌手[かしゅ]は 歌[うた]が 下手[へた]だ。	その かしゅ は うた が へた だ	
\\	その
\\	は 歌[うた]が 下手[へた]だ。		
\\	欲しがる	欲[ほ]しがる	ほしがる	
\\	子供がジュースを欲しがっています。	子供[こども]がジュースを 欲[ほ]しがっています。	こども が じゅーす を ほしがって います	
\\	子供[こども]がジュースを
\\	計画	計画[けいかく]	けいかく	
\\	彼は一人旅の計画を立てた。	彼[かれ]は 一人旅[ひとりたび]の 計画[けいかく]を 立[た]てた。	かれ は ひとりたび の けいかく を たてた	
\\	彼[かれ]は 一人旅[ひとりたび]の
\\	を 立[た]てた。		
\\	映画	映画[えいが]	えいが	
\\	彼はよく映画を見ます。	彼[かれ]はよく 映画[えいが]を 見[み]ます。	かれ は よく えいが を みます	
\\	彼[かれ]はよく
\\	を 見[み]ます。		
\\	りんご	りんご	りんご	
\\	りんごを一つください。	りんごを 一[ひと]つください。	りんご を ひとつ ください	
\\	を 一[ひと]つください。		
\\	面白い	面白[おもしろ]い	おもしろい	
\\	この本は全然面白くなかった。	この 本[ほん]は全然[ぜんぜん] 面白[おもしろ]くなかった。	この ほん は ぜんぜん おもしろく なかった	
\\	この 本[ほん]は 全然[ぜんぜん]
\\	写る	写[うつ]る	うつる	
\\	このカメラはよく写りますよ。	このカメラはよく 写[うつ]りますよ。	この かめら は よく うつります よ	
\\	このカメラはよく
\\	よ。		
\\	写す	写[うつ]す	うつす	
\\	彼は友達の答えを写した。	彼[かれ]は 友達[ともだち]の 答[こた]えを 写[うつ]した。	かれ は ともだち の こたえ を うつした	
\\	彼[かれ]は 友達[ともだち]の 答[こた]えを
\\	写真	写真[しゃしん]	しゃしん	
\\	写真は良い思い出になります。	写真[しゃしん]は 良[い]い 思[おも]い 出[で]になります。	しゃしん は いい おもいで に なります	
\\	は 良[い]い 思[おも]い 出[で]になります。		
\\	アルバム	アルバム	アルバム	
\\	私は彼のアルバムを見た。	私[わたし]は 彼[かれ]のアルバムを 見[み]た。	わたし は かれ の あるばむ を みた	
\\	私[わたし]は 彼[かれ]の
\\	を 見[み]た。		
\\	真っ赤	真[ま]っ 赤[か]	まっか	
\\	彼の顔は真っ赤でした。	彼[かれ]の 顔[かお]は 真[ま]っ 赤[か]でした。	かれ の かお は まっか でした	
\\	彼[かれ]の 顔[かお]は
\\	でした。		
\\	真面目	真面目[まじめ]	まじめ	
\\	彼は真面目な人です。	彼[かれ]は 真面目[まじめ]な 人[ひと]です。	かれ は まじめ な ひと です	
\\	彼[かれ]は
\\	な 人[ひと]です。		
\\	真ん中	真[ま]ん 中[なか]	まんなか	
\\	道の真ん中に人が立っている。	道[みち]の 真[ま]ん 中[なか]に 人[ひと]が 立[た]っている。	みち の まんなか に ひと が たって いる	
\\	道[みち]の
\\	に 人[ひと]が 立[た]っている。		
\\	真っ白	真[ま]っ 白[しろ]	まっしろ	
\\	外は雪で真っ白だった。	外[そと]は 雪[ゆき]で 真[ま]っ 白[しろ]だった。	そと は ゆき で まっしろ だった	
\\	外[そと]は 雪[ゆき]で
\\	だった。		
\\	スカート	スカート	スカート	
\\	彼女はあまりスカートははかない。	彼女[かのじょ]はあまりスカートははかない。	かのじょ は あまり すかーと は はかない	
\\	彼女[かのじょ]はあまり
\\	ははかない。		
\\	真っ暗	真[ま]っ 暗[くら]	まっくら	
\\	外は真っ暗です。	外[そと]は 真[ま]っ 暗[くら]です。	そと は まっくら です	
\\	外[そと]は
\\	です。		
\\	真っ黒	真[ま]っ 黒[くろ]	まっくろ	
\\	インクで手が真っ黒になった	インクで 手[て]が 真[ま]っ 黒[くろ]になった	いんく で て が まっくろ に なった	
\\	インクで 手[て]が
\\	になった		
\\	真っ青	真[ま]っ 青[さお]	まっさお	
\\	空が真っ青です。	空[そら]が 真[ま]っ 青[さお]です。	そら が まっさお です	
\\	空[そら]が
\\	です。		
\\	色々	色々[いろいろ]	いろいろ	
\\	彼は色々なことを知っている。	彼[かれ]は 色々[いろいろ]なことを 知[し]っている。	かれ は いろいろ な こと を しって いる	
\\	彼[かれ]は
\\	なことを 知[し]っている。		
\\	ペン	ペン	ペン	
\\	ペンを貸してください。	ペンを 貸[か]してください。	ぺん を かして ください	
\\	を 貸[か]してください。		
\\	人形	人形[にんぎょう]	にんぎょう	
\\	彼女は人形をたくさん持っています。	彼女[かのじょ]は 人形[にんぎょう]をたくさん 持[も]っています。	かのじょ は にんぎょう を たくさん もって います	
\\	彼女[かのじょ]は
\\	をたくさん 持[も]っています。		
\\	形	形[かたち]	かたち	
\\	その椅子は変わった形をしている。	その 椅子[いす]は 変[か]わった 形[かたち]をしている。	その いす は かわった かたち を して いる	
\\	その 椅子[いす]は 変[か]わった
\\	をしている。		
\\	大型	大型[おおがた]	おおがた	
\\	大型のテレビを買った。	大型[おおがた]のテレビを 買[か]った。	おおがた の てれび を かった	
\\	のテレビを 買[か]った。		
\\	種類	種類[しゅるい]	しゅるい	
\\	バラには色々な種類があります。	バラには 色々[いろいろ]な 種類[しゅるい]があります。	ばら に は いろいろ な しゅるい が あります	
\\	バラには 色々[いろいろ]な
\\	があります。		
\\	おしゃべり	おしゃべり	おしゃべり	
\\	妹はとてもおしゃべりです。	妹[いもうと]はとてもおしゃべりです。	いもうと は とても おしゃべり です	
\\	妹[いもうと]はとても
\\	です。		
\\	直す	直[なお]す	なおす	
\\	私がそれを直しました。	私[わたし]がそれを 直[なお]しました。	わたし が それ を なおしました	
\\	私[わたし]がそれを
\\	真っ直ぐ	真[ま]っ 直[す]ぐ	まっすぐ	
\\	この道を真っ直ぐ行ってください。	この 道[みち]を 真[ま]っ 直[す]ぐ 行[い]ってください。	この みち を まっすぐ いって ください	
\\	この 道[みち]を
\\	行[い]ってください。		
\\	直接	直接[ちょくせつ]	ちょくせつ	
\\	彼に直接お願いしなさい。	彼[かれ]に 直接[ちょくせつ]お 願[ねが]いしなさい。	かれ に ちょくせつ おねがい しなさい	
\\	彼[かれ]に
\\	お 願[ねが]いしなさい。		
\\	角	角[かど]	かど	
\\	次の角で左に曲がってください。	次[つぎ]の 角[かど]で 左[ひだり]に 曲[ま]がってください。	つぎ の かど で ひだり に まがって ください	
\\	次[つぎ]の
\\	で 左[ひだり]に 曲[ま]がってください。		
\\	きっと	きっと	きっと	
\\	明日はきっと雨が降ります。	明日[あした]はきっと 雨[あめ]が 降[ふ]ります。	あした は きっと あめ が ふります	
\\	明日[あした]は
\\	雨[あめ]が 降[ふ]ります。		
\\	三角	三角[さんかく]	さんかく	
\\	紙を三角に切りました。	紙[かみ]を 三角[さんかく]に 切[き]りました。	かみ を さんかく に きりました	
\\	紙[かみ]を
\\	に 切[き]りました。		
\\	四角	四角[しかく]	しかく	
\\	紙を四角に切ってください。	紙[かみ]を 四角[しかく]に 切[き]ってください。	かみ を しかく に きって ください	
\\	紙[かみ]を
\\	に 切[き]ってください。		
\\	四角い	四角[しかく]い	しかくい	
\\	こっちの四角いテーブルを買おうよ。	こっちの 四角[しかく]いテーブルを 買[か]おうよ。	こっち の しかくい てーぶる を かおうよ	
\\	こっちの
\\	テーブルを 買[か]おうよ。		
\\	四つ角	四[よ]つ 角[かど]	よつかど	
\\	あそこの四つ角を左に曲がってください。	あそこの 四[よ]つ 角[かど]を 左[ひだり]に 曲[ま]がってください。	あそこ の よつかど を ひだり に まがって ください	
\\	あそこの
\\	を 左[ひだり]に 曲[ま]がってください。		
\\	メニュー	メニュー	メニュー	
\\	メニューをください。	メニューをください。	めにゅー を ください	
\\	をください。		
\\	曲	曲[きょく]	きょく	
\\	私はこの曲が大好きです。	私[わたし]はこの 曲[きょく]が 大好[だいす]きです。	わたし は この きょく が だいすき です	
\\	私[わたし]はこの
\\	が 大好[だいす]きです。		
\\	曲げる	曲[ま]げる	まげる	
\\	ひざを曲げてください。	ひざを 曲[ま]げてください。	ひざ を まげて ください	
\\	ひざを
\\	ください。		
\\	曲がり角	曲[ま]がり 角[かど]	まがりかど	
\\	ポストはそこの曲がり角にあります。	ポストはそこの 曲[ま]がり 角[かど]にあります。	ぽすと は そこ の まがりかど に あります	
\\	ポストはそこの
\\	にあります。		
\\	同様	同様[どうよう]	どうよう	
\\	私たちは彼を家族同様に思っている。	私[わたし]たちは 彼[かれ]を 家族[かぞく] 同様[どうよう]に 思[おも]っている。	わたしたち は かれ を かぞく どうよう に おもって いる	
\\	私[わたし]たちは 彼[かれ]を 家族[かぞく]
\\	に 思[おも]っている。		
\\	アイスクリーム	アイスクリーム	アイスクリーム	
\\	弟はアイスクリームが大好きです。	弟[おとうと]はアイスクリームが 大好[だいす]きです。	おとうと は あいすくりーむ が だいすき です	
\\	弟[おとうと]は
\\	が 大好[だいす]きです。		
\\	間違い	間違[まちが]い	まちがい	
\\	この文には間違いがあります。	この 文[ぶん]には 間違[まちが]いがあります。	この ぶん に は まちがい が あります	
\\	この 文[ぶん]には
\\	があります。		
\\	間違える	間違[まちが]える	まちがえる	
\\	(を)		
\\	電話番号を間違えました。	電話番号[でんわ ばんごう]を 間違[まちが]えました。	でんわ ばんごう を まちがえました	
\\	電話番号[でんわ ばんごう]を
\\	間違う	間違[まちが]う	まちがう	
\\	あなたは間違っている。	あなたは 間違[まちが]っている。	あなた は まちがって いる	
\\	あなたは
\\	似ている	似[に]ている	にている	
\\	私は母に似ています。	私[わたし]は 母[はは]に 似[に]ています。	わたし は はは に にて います	
\\	私[わたし]は 母[はは]に
\\	おしゃれ	おしゃれ	おしゃれ	
\\	あの子はおしゃれだね。	あの 子[こ]はおしゃれだね。	あの こ は おしゃれ だ ね	
\\	あの 子[こ]は
\\	だね。		
\\	以上	以上[いじょう]	いじょう	
\\	飛行機が1時間以上遅れた。	飛行機[ひこうき]が 1時間[いちじかん] 以上[いじょう] 遅[おく]れた。	ひこうき が いちじかん いじょう おくれた	
\\	飛行機[ひこうき]が 1時間[いちじかん]
\\	遅[おく]れた。		
\\	旅行	旅行[りょこう]	りょこう	
\\	彼女は旅行が好きです。	彼女[かのじょ]は 旅行[りょこう]が 好[す]きです。	かのじょ は りょこう が すき です	
\\	彼女[かのじょ]は
\\	が 好[す]きです。		
\\	大使館	大使館[たいしかん]	たいしかん	
\\	彼は大使館に勤めています。	彼[かれ]は 大使館[たいしかん]に 勤[つと]めています。	かれ は たいしかん に つとめて います	
\\	彼[かれ]は
\\	に 勤[つと]めています。		
\\	旅館	旅館[りょかん]	りょかん	
\\	京都では旅館に泊まりました。	京都[きょうと]では 旅館[りょかん]に 泊[と]まりました。	きょうと で は りょかん に とまりました	
\\	京都[きょうと]では
\\	に 泊[と]まりました。		
\\	ジュース	ジュース	ジュース	
\\	このジュースは甘すぎる。	このジュースは 甘[あま]すぎる。	この じゅーす は あま すぎる	
\\	この
\\	は 甘[あま]すぎる。		
\\	映画館	映画館[えいがかん]	えいがかん	
\\	彼と近くの映画館に行きました。	彼[かれ]と 近[ちか]くの 映画館[えいがかん]に 行[い]きました。	かれ と ちかく の えいがかん に いきました	
\\	彼[かれ]と 近[ちか]くの
\\	に 行[い]きました。		
\\	宿題	宿題[しゅくだい]	しゅくだい	
\\	友達と一緒に宿題をした。	友達[ともだち]と 一緒[いっしょ]に 宿題[しゅくだい]をした。	ともだち と いっしょ に しゅくだい を した	
\\	友達[ともだち]と 一緒[いっしょ]に
\\	をした。		
\\	泊める	泊[と]める	とめる	
\\	友達をうちに泊めてあげました。	友達[ともだち]をうちに 泊[と]めてあげました。	ともだち を うち に とめて あげました	
\\	友達[ともだち]をうちに
\\	遊び	遊[あそ]び	あそび	
\\	お正月には色々な遊びをします。	お 正月[しょうがつ]には 色々[いろいろ]な 遊[あそ]びをします。	おしょうがつ に は いろいろ な あそび を します	
\\	お 正月[しょうがつ]には 色々[いろいろ]な
\\	をします。		
\\	ゼロ	ゼロ	ゼロ	
\\	今日の交通事故はゼロです。	今日[きょう]の 交通事故[こうつう じこ]はゼロです。	きょう の こうつう じこ は ぜろ です	
\\	今日[きょう]の 交通事故[こうつう じこ]は
\\	です。		
\\	洋服	洋服[ようふく]	ようふく	
\\	今日は洋服を買いに行きます。	今日[きょう]は 洋服[ようふく]を 買[か]いに 行[い]きます。	きょう は ようふく を かい に いきます	
\\	今日[きょう]は
\\	を 買[か]いに 行[い]きます。		
\\	教室	教室[きょうしつ]	きょうしつ	
\\	私の教室は3階にあります。	私[わたし]の 教室[きょうしつ]は 3階[さんがい]にあります。	わたし の きょうしつ は さんがい に あります	
\\	私[わたし]の
\\	は 3階[さんがい]にあります。		
\\	図書室	図書室[としょしつ]	としょしつ	
\\	図書室で勉強した。	図書室[としょしつ]で 勉強[べんきょう]した。	としょしつ で べんきょう した	
\\	で 勉強[べんきょう]した。		
\\	窓	窓[まど]	まど	
\\	窓を開けてください。	窓[まど]を 開[あ]けてください。	まど を あけて ください	
\\	を 開[あ]けてください。		
\\	ピンク	ピンク	ピンク	
\\	娘がピンクのドレスを着ている。	娘[むすめ]がピンクのドレスを 着[き]ている。	むすめ が ぴんく の どれす を きて いる	
\\	娘[むすめ]が
\\	のドレスを 着[き]ている。		
\\	親しい	親[した]しい	したしい	
\\	週末、親しい友達を家に呼んだ。	週末[しゅうまつ]、 親[した]しい 友達[ともだち]を 家[いえ]に 呼[よ]んだ。	しゅうまつ したしい ともだち を いえ に よんだ	
\\	週末[しゅうまつ]、
\\	友達[ともだち]を 家[いえ]に 呼[よ]んだ。		
\\	親切	親切[しんせつ]	しんせつ	
\\	親切にしてくださってどうもありがとうございます。	親切[しんせつ]にしてくださってどうもありがとうございます。	しんせつ に して くださって どうも ありがとう ございます	
\\	にしてくださってどうもありがとうございます。		
\\	不親切	不親切[ふしんせつ]	ふしんせつ	
\\	その店員は不親切だった。	その 店員[てんいん]は 不親切[ふしんせつ]だった。	その てんいん は ふしんせつ だった	
\\	その 店員[てんいん]は
\\	だった。		
\\	家族	家族[かぞく]	かぞく	
\\	うちは五人家族です。	うちは 五人[ごにん] 家族[かぞく]です。	うち は ごにん かぞく です	
\\	うちは 五人[ごにん]
\\	です。		
\\	グラス	グラス	グラス	
\\	これはきれいなグラスですね。	これはきれいなグラスですね。	これ は きれい な ぐらす です ね	
\\	これはきれいな
\\	ですね。		
\\	万歳	万歳[ばんざい]	ばんざい	
\\	勝った、万歳!	勝[か]った、 万歳[ばんざい]!	かった ばんざい	
\\	勝[か]った、
\\	二十歳	二十歳[はたち]	はたち	
\\	妹は来年、二十歳になります。	妹[いもうと]は 来年[らいねん]、 二十歳[はたち]になります。	いもうと は らいねん はたち に なります	
\\	妹[いもうと]は 来年[らいねん]、
\\	になります。		
\\	姉さん	姉[ねえ]さん	ねえさん	
\\	姉さん、ごめんね。	姉[ねえ]さん、ごめんね。	ねえさん ごめん ね	
\\	、ごめんね。		
\\	兄さん	兄[にい]さん	にいさん	
\\	兄さん、おめでとう。	兄[にい]さん、おめでとう。	にいさん おめでとう	
\\	、おめでとう。		
\\	ダンス	ダンス	ダンス	
\\	彼女はダンスが大好きです。	彼女[かのじょ]はダンスが 大好[だいす]きです。	かのじょ は だんす が だいすき です	
\\	彼女[かのじょ]は
\\	が 大好[だいす]きです。		
\\	兄弟	兄弟[きょうだい]	きょうだい	
\\	彼は3人兄弟です。	彼[かれ]は 3人[さんにん] 兄弟[きょうだい]です。	かれ は さんにん きょうだい です	
\\	彼[かれ]は 3人[さんにん]
\\	です。		
\\	業者	業者[ぎょうしゃ]	ぎょうしゃ	
\\	引っ越しを業者に頼んだ。	引[ひ]っ 越[こ]しを 業者[ぎょうしゃ]に 頼[たの]んだ。	ひっこし を ぎょうしゃ に たのんだ	
\\	引[ひ]っ 越[こ]しを
\\	に 頼[たの]んだ。		
\\	若者	若者[わかもの]	わかもの	
\\	最近の若者は本を読まない。	最近[さいきん]の 若者[わかもの]は 本[ほん]を 読[よ]まない。	さいきん の わかもの は ほん を よまない	
\\	最近[さいきん]の
\\	は 本[ほん]を 読[よ]まない。		
\\	彼ら	彼[かれ]ら	かれら	
\\	彼らはバスケットの選手です。	彼[かれ]らはバスケットの 選手[せんしゅ]です。	かれら は ばすけっと の せんしゅ です	
\\	はバスケットの 選手[せんしゅ]です。		
\\	おじ	おじ	おじ	
\\	おじは銀行に勤めています。	おじは 銀行[ぎんこう]に 勤[つと]めています。	おじ は ぎんこう に つとめて います	
\\	は 銀行[ぎんこう]に 勤[つと]めています。		
\\	結合	結合[けつごう]	けつごう	
\\	ファイルを結合して1つにしました。	ファイルを 結合[けつごう]して 1[ひと]つにしました。	ふぁいる を けつごう して ひとつ に しました	
\\	ファイルを
\\	して 1[ひと]つにしました。		
\\	結果	結果[けっか]	けっか	
\\	試合の結果を早く知りたい。	試合[しあい]の 結果[けっか]を 早[はや]く 知[し]りたい。	しあい の けっか を はやく しりたい	
\\	試合[しあい]の
\\	を 早[はや]く 知[し]りたい。		
\\	果物	果物[くだもの]	くだもの	
\\	デザートに果物を食べましょう。	デザートに 果物[くだもの]を 食[た]べましょう。	でざーと に くだもの を たべましょう	
\\	デザートに
\\	を 食[た]べましょう。		
\\	課題	課題[かだい]	かだい	
\\	夏休みの課題は何ですか。	夏休[なつやす]みの 課題[かだい]は 何[なん]ですか。	なつやすみ の かだい は なん です か	
\\	夏休[なつやす]みの
\\	は 何[なん]ですか。		
\\	サッカー	サッカー	サッカー	
\\	彼はサッカーの選手です。	彼[かれ]はサッカーの 選手[せんしゅ]です。	かれ は さっかー の せんしゅ です	
\\	彼[かれ]は
\\	の 選手[せんしゅ]です。		
\\	課	課[か]	か	
\\	今から課のミーティングがある。	今[いま]から 課[か]のミーティングがある。	いま から か の みーてぃんぐ が ある	
\\	今[いま]から
\\	のミーティングがある。		
\\	効果	効果[こうか]	こうか	
\\	この薬には胃を守る効果がある。	この 薬[くすり]には 胃[い]を 守[まも]る 効果[こうか]がある。	この くすり に は い を まもる こうか が ある	
\\	この 薬[くすり]には 胃[い]を 守[まも]る
\\	がある。		
\\	自動車	自動車[じどうしゃ]	じどうしゃ	
\\	彼は自動車会社に就職した。	彼[かれ]は 自動車[じどうしゃ] 会社[がいしゃ]に 就職[しゅうしょく]した。	かれ は じどうしゃ がいしゃ に しゅうしょく した	
\\	彼[かれ]は
\\	会社[がいしゃ]に 就職[しゅうしょく]した。		
\\	自然	自然[しぜん]	しぜん	
\\	みんなで自然を守りましょう。	みんなで 自然[しぜん]を 守[まも]りましょう。	みんな で しぜん を まもりましょう	
\\	みんなで
\\	を 守[まも]りましょう。		
\\	スープ	スープ	スープ	
\\	母がコーンスープを作っている。	母[はは]がコーンスープを 作[つく]っている。	はは が こーんすーぷ を つくって いる	
\\	母[はは]がコーン
\\	を 作[つく]っている。		
\\	自然	自然[しぜん]	しぜん	
\\	親が子供を守るのは自然なことだ。	親[おや]が 子供[こども]を 守[まも]るのは 自然[しぜん]なことだ。	おや が こども を まもる の は しぜん な こと だ	
\\	親[おや]が 子供[こども]を 守[まも]るのは
\\	なことだ。		
\\	自習	自習[じしゅう]	じしゅう	
\\	明日は自習の時間がある。	明日[あす]は 自習[じしゅう]の 時間[じかん]がある。	あす は じしゅう の じかん が ある	
\\	明日[あす]は
\\	の 時間[じかん]がある。		
\\	理由	理由[りゆう]	りゆう	
\\	遅れた理由を教えてください。	遅[おく]れた 理由[りゆう]を 教[おし]えてください。	おくれた りゆう を おしえて ください	
\\	遅[おく]れた
\\	を 教[おし]えてください。		
\\	自由	自由[じゆう]	じゆう	
\\	今日は自由な時間が多い。	今日[きょう]は 自由[じゆう]な 時間[じかん]が 多[おお]い。	きょう は じゆう な じかん が おおい	
\\	今日[きょう]は
\\	な 時間[じかん]が 多[おお]い。		
\\	バナナ	バナナ	バナナ	
\\	私は毎朝バナナを食べます。	私[わたし]は 毎朝[まいあさ]バナナを 食[た]べます。	わたし は まいあさ ばなな を たべます	
\\	私[わたし]は 毎朝[まいあさ]
\\	を 食[た]べます。		
\\	信じる	信[しん]じる	しんじる	
\\	彼はキリストを信じている。	彼[かれ]はキリストを 信[しん]じている。	かれ は きりすと を しんじて いる	
\\	彼[かれ]はキリストを
\\	信号	信号[しんごう]	しんごう	
\\	信号が青になった。	信号[しんごう]が 青[あお]になった。	しんごう が あお に なった	
\\	が 青[あお]になった。		
\\	頼む	頼[たの]む	たのむ	
\\	私はハンバーガーを頼みました。	私[わたし]はハンバーガーを 頼[たの]みました。	わたし は はんばーがー を たのみました	
\\	私[わたし]はハンバーガーを
\\	市民	市民[しみん]	しみん	
\\	市民の安全は大切だ。	市民[しみん]の 安全[あんぜん]は 大切[たいせつ]だ。	しみん の あんぜん は たいせつ だ	
\\	の 安全[あんぜん]は 大切[たいせつ]だ。		
\\	うるさい	うるさい	うるさい	
\\	この通りは車の音がうるさい。	この 通[とお]りは 車[くるま]の 音[おと]がうるさい。	この とおり は くるま の おと が うるさい	
\\	この 通[とお]りは 車[くるま]の 音[おと]が
\\	住民	住民[じゅうみん]	じゅうみん	
\\	地域の住民が集まって話合いをした。	地域[ちいき]の 住民[じゅうみん]が 集[あつ]まって 話合[はなしあ]いをした。	ちいき の じゅうみん が あつまって はなしあい を した	
\\	地域[ちいき]の
\\	が 集[あつ]まって 話合[はなしあ]いをした。		
\\	主人	主人[しゅじん]	しゅじん	
\\	主人は今、留守です。	主人[しゅじん]は 今[いま]、 留守[るす]です。	しゅじん は いま るす です	
\\	は 今[いま]、 留守[るす]です。		
\\	議員	議員[ぎいん]	ぎいん	
\\	彼は国会議員です。	彼[かれ]は 国会[こっかい] 議員[ぎいん]です。	かれ は こっかい ぎいん です	
\\	彼[かれ]は 国会[こっかい]
\\	です。		
\\	会議	会議[かいぎ]	かいぎ	
\\	今日の午後、大事な会議があります。	今日[きょう]の 午後[ごご]、 大事[だいじ]な 会議[かいぎ]があります。	きょう の ごご だいじ な かいぎ が あります	
\\	今日[きょう]の 午後[ごご]、 大事[だいじ]な
\\	があります。		
\\	パトカー	パトカー	パトカー	
\\	あそこにパトカーがいる。	あそこにパトカーがいる。	あそこ に ぱとかー が いる	
\\	あそこに
\\	がいる。		
\\	対する	対[たい]する	たいする	
\\	その質問に対する答えが見つからなかった。	その 質問[しつもん]に 対[たい]する 答[こた]えが 見[み]つからなかった。	その しつもん に たいする こたえ が みつからなかった	
\\	その 質問[しつもん]に
\\	答[こた]えが 見[み]つからなかった。		
\\	対立	対立[たいりつ]	たいりつ	
\\	その2社は対立しています。	その 2社[にしゃ]は 対立[たいりつ]しています。	その にしゃ は たいりつ して います	
\\	その 2社[にしゃ]は
\\	しています。		
\\	反対	反対[はんたい]	はんたい	
\\	私は反対です。	私[わたし]は 反対[はんたい]です。	わたし は はんたい です	
\\	私[わたし]は
\\	です。		
\\	答え	答[こた]え	こたえ	
\\	彼はその問題の答えが分からない。	彼[かれ]はその 問題[もんだい]の 答[こた]えが 分[わ]からない。	かれ は その もんだい の こたえ が わからない	
\\	彼[かれ]はその 問題[もんだい]の
\\	が 分[わ]からない。		
\\	ハンバーガー	ハンバーガー	ハンバーガー	
\\	今日の昼ご飯はハンバーガーでした。	今日[きょう]の 昼[ひる]ご 飯[はん]はハンバーガーでした。	きょう の ひるごはん は はんばーがー でした	
\\	今日[きょう]の 昼[ひる]ご 飯[はん]は
\\	でした。		
\\	特に	特[とく]に	とくに	
\\	特に質問はありません。	特[とく]に 質問[しつもん]はありません。	とくに しつもん は ありません	
\\	質問[しつもん]はありません。		
\\	特急	特急[とっきゅう]	とっきゅう	
\\	東京まで特急で3時間かかります。	東京[とうきょう]まで 特急[とっきゅう]で 3時間[さんじかん]かかります。	とうきょう まで とっきゅう で さんじかん かかります	
\\	東京[とうきょう]まで
\\	で 3時間[さんじかん]かかります。		
\\	特別	特別[とくべつ]	とくべつ	
\\	あなたは私にとって特別な人です。	あなたは 私[わたし]にとって 特別[とくべつ]な 人[ひと]です。	あなた は わたし に とって とくべつ な ひと です	
\\	あなたは 私[わたし]にとって
\\	な 人[ひと]です。		
\\	別々	別々[べつべつ]	べつべつ	
\\	別々に払いましょう。	別々[べつべつ]に 払[はら]いましょう。	べつべつ に はらいましょう	
\\	に 払[はら]いましょう。		
\\	エスカレーター	エスカレーター	エスカレーター	
\\	3階までエスカレーターで行きましょう。	3階[さんかい]までエスカレーターで 行[い]きましょう。	さんかい まで えすかれーたー で いきましょう	
\\	3階[さんかい]まで
\\	で 行[い]きましょう。		
\\	別れる	別[わか]れる	わかれる	
\\	駅で友だちと別れました。	駅[えき]で 友[とも]だちと 別[わか]れました。	えき で ともだち と わかれました	
\\	駅[えき]で 友[とも]だちと
\\	一般	一般[いっぱん]	いっぱん	
\\	一般の方はこちらの席へどうぞ。	一般[いっぱん]の 方[かた]はこちらの 席[せき]へどうぞ。	いっぱん の かた は こちら の せき へ どうぞ	
\\	の 方[かた]はこちらの 席[せき]へどうぞ。		
\\	目的	目的[もくてき]	もくてき	
\\	彼が来た目的が分かりません。	彼[かれ]が 来[き]た 目的[もくてき]が 分[わ]かりません。	かれ が きた もくてき が わかりません	
\\	彼[かれ]が 来[き]た
\\	が 分[わ]かりません。		
\\	普通	普通[ふつう]	ふつう	
\\	彼女は普通の女の子だ。	彼女[かのじょ]は 普通[ふつう]の 女[おんな]の 子[こ]だ。	かのじょ は ふつう の おんな の こ だ	
\\	彼女[かのじょ]は
\\	の 女[おんな]の 子[こ]だ。		
\\	タオル	タオル	タオル	
\\	私はタオルで顔をふいた。	私[わたし]はタオルで 顔[かお]をふいた。	わたし は たおる で かお を ふいた	
\\	私[わたし]は
\\	で 顔[かお]をふいた。		
\\	並ぶ	並[なら]ぶ	ならぶ	
\\	ここに並んでください。	ここに 並[なら]んでください。	ここ に ならんで ください	
\\	ここに
\\	ください。		
\\	並べる	並[なら]べる	ならべる	
\\	私は料理をテーブルに並べた。	私[わたし]は 料理[りょうり]をテーブルに 並[なら]べた。	わたし は りょうり を てーぶる に ならべた	
\\	私[わたし]は 料理[りょうり]をテーブルに
\\	平和	平和[へいわ]	へいわ	
\\	この国は平和です。	この 国[くに]は 平和[へいわ]です。	この くに は へいわ です	
\\	この 国[くに]は
\\	です。		
\\	高等学校	高等学校[こうとうがっこう]	こうとうがっこう	
\\	弟が高等学校を卒業しました。	弟[おとうと]が 高等学校[こうとうがっこう]を 卒業[そつぎょう]しました。	おとうと が こうとうがっこう を そつぎょう しました	
\\	弟[おとうと]が
\\	を 卒業[そつぎょう]しました。		
\\	パチンコ	パチンコ	パチンコ	
\\	彼は毎日パチンコをしています。	彼[かれ]は 毎日[まいにち]パチンコをしています。	かれ は まいにち ぱちんこ を して います	
\\	彼[かれ]は 毎日[まいにち]
\\	をしています。		
\\	病院	病院[びょういん]	びょういん	
\\	病院はどこですか。	病院[びょういん]はどこですか。	びょういん は どこ です か	
\\	はどこですか。		
\\	入院	入院[にゅういん]	にゅういん	
\\	昨日、母が入院しました。	昨日[きのう]、 母[はは]が 入院[にゅういん]しました。	きのう はは が にゅういん しました	
\\	昨日[きのう]、 母[はは]が
\\	しました。		
\\	大学院	大学院[だいがくいん]	だいがくいん	
\\	彼は大学院に進みました。	彼[かれ]は 大学院[だいがくいん]に 進[すす]みました。	かれ は だいがくいん に すすみました	
\\	彼[かれ]は
\\	に 進[すす]みました。		
\\	医者	医者[いしゃ]	いしゃ	
\\	私は医者に相談した。	私[わたし]は 医者[いしゃ]に 相談[そうだん]した。	わたし は いしゃ に そうだん した	
\\	私[わたし]は
\\	に 相談[そうだん]した。		
\\	みかん	みかん	みかん	
\\	みかんを1つください。	みかんを 1[ひと]つください。	みかん を ひとつ ください	
\\	を 1[ひと]つください。		
\\	お医者さん	お 医者[いしゃ]さん	おいしゃさん	
\\	熱があるのでお医者さんに行った。	熱[ねつ]があるのでお 医者[いしゃ]さんに 行[い]った。	ねつ が ある の で おいしゃさん に いった	
\\	熱[ねつ]があるので
\\	に 行[い]った。		
\\	歯医者	歯医者[はいしゃ]	はいしゃ	
\\	私は歯医者が嫌いです。	私[わたし]は 歯医者[はいしゃ]が 嫌[きら]いです。	わたし は はいしゃ が きらい です	
\\	私[わたし]は
\\	が 嫌[きら]いです。		
\\	歯	歯[は]	は	
\\	私の歯は丈夫です。	私[わたし]の 歯[は]は 丈夫[じょうぶ]です。	わたし の は は じょうぶ です	
\\	私[わたし]の
\\	は 丈夫[じょうぶ]です。		
\\	歯ブラシ	歯[は]ブラシ	はぶらし	
\\	新しい歯ブラシが必要だ。	新[あたら]しい 歯[は]ブラシが 必要[ひつよう]だ。	あたらしい はぶらし が ひつよう だ	
\\	新[あたら]しい
\\	が 必要[ひつよう]だ。		
\\	ケーキ	ケーキ	ケーキ	
\\	誕生日にケーキを食べました。	誕生日[たんじょうび]にケーキを 食[た]べました。	たんじょうび に けーき を たべました	
\\	誕生日[たんじょうび]に
\\	を 食[た]べました。		
\\	科学	科学[かがく]	かがく	
\\	科学は常に進歩している。	科学[かがく]は 常[つね]に 進歩[しんぽ]している。	かがく は つねに しんぽ して いる	
\\	は 常[つね]に 進歩[しんぽ]している。		
\\	教科書	教科書[きょうかしょ]	きょうかしょ	
\\	日本語の教科書を忘れた。	日本語[にほんご]の 教科書[きょうかしょ]を 忘[わす]れた。	にほんご の きょうかしょ を わすれた	
\\	日本語[にほんご]の
\\	を 忘[わす]れた。		
\\	理科	理科[りか]	りか	
\\	私は理科が得意です。	私[わたし]は 理科[りか]が 得意[とくい]です。	わたし は りか が とくい です	
\\	私[わたし]は
\\	が 得意[とくい]です。		
\\	亡くなる	亡[な]くなる	なくなる	
\\	おととい、昔の友人が亡くなった。	おととい、 昔[むかし]の 友人[ゆうじん]が 亡[な]くなった。	おととい むかし の ゆうじん が なくなった	
\\	おととい、 昔[むかし]の 友人[ゆうじん]が
\\	コップ	コップ	コップ	
\\	私は毎朝コップ一杯の水を飲む。	私[わたし]は 毎朝[まいあさ]コップ 一杯[いっぱい]の 水[みず]を 飲[の]む。	わたし は まいあさ こっぷ いっぱい の みず を のむ	
\\	私[わたし]は 毎朝[まいあさ]
\\	一杯[いっぱい]の 水[みず]を 飲[の]む。		
\\	忙しい	忙[いそが]しい	いそがしい	
\\	忙しいので手伝ってください。	忙[いそが]しいので 手伝[てつだ]ってください。	いそがしい の で てつだって ください	
\\	ので 手伝[てつだ]ってください。		
\\	疲れる	疲[つか]れる	つかれる	
\\	私は疲れてふらふらです。	私[わたし]は 疲[つか]れてふらふらです。	わたし は つかれて ふらふら です	
\\	私[わたし]は
\\	ふらふらです。		
\\	禁煙	禁煙[きんえん]	きんえん	
\\	彼は今、禁煙しています。	彼[かれ]は 今[いま]、 禁煙[きんえん]しています。	かれ は いま きんえん して います	
\\	彼[かれ]は 今[いま]、
\\	しています。		
\\	酔っ払い	酔[よ]っ 払[ぱら]い	よっぱらい	
\\	彼はただの酔っ払いです。	彼[かれ]はただの 酔[よ]っ 払[ぱら]いです。	かれ は ただ の よっぱらい です	
\\	彼[かれ]はただの
\\	です。		
\\	ナイロン	ナイロン	ナイロン	
\\	このジャケットはナイロンでできています。	このジャケットはナイロンでできています。	この じゃけっと は ないろん で できて います	
\\	このジャケットは
\\	でできています。		
\\	酔っ払う	酔[よ]っ 払[ぱら]う	よっぱらう	
\\	昨夜は酔っ払いました。	昨夜[ゆうべ]は 酔[よ]っ 払[ぱら]いました。	ゆうべ は よっぱらいました	
\\	昨夜[ゆうべ]は
\\	危ない	危[あぶ]ない	あぶない	
\\	その道は車が多くて危ない。	その 道[みち]は 車[くるま]が 多[おお]くて 危[あぶ]ない。	その みち は くるま が おおくて あぶない	
\\	その 道[みち]は 車[くるま]が 多[おお]くて
\\	危険	危険[きけん]	きけん	
\\	その地域は今、危険だ。	その 地域[ちいき]は 今[いま]、 危険[きけん]だ。	その ちいき は いま きけん だ	
\\	その 地域[ちいき]は 今[いま]、
\\	だ。		
\\	存在	存在[そんざい]	そんざい	
\\	宇宙人は存在すると思いますか。	宇宙人[うちゅうじん]は 存在[そんざい]すると 思[おも]いますか。	うちゅうじん は そんざい する と おもいます か	
\\	宇宙人[うちゅうじん]は
\\	すると 思[おも]いますか。		
\\	スーツ	スーツ	スーツ	
\\	あのスーツはそんなに高くない。	あのスーツはそんなに 高[たか]くない。	あの すーつ は そんなに たかく ない	
\\	あの
\\	はそんなに 高[たか]くない。		
\\	注目	注目[ちゅうもく]	ちゅうもく	
\\	私たちはその会社に注目している。	私[わたし]たちはその 会社[かいしゃ]に 注目[ちゅうもく]している。	わたしたち は その かいしゃ に ちゅうもく して いる	
\\	私[わたし]たちはその 会社[かいしゃ]に
\\	している。		
\\	注文	注文[ちゅうもん]	ちゅうもん	
\\	レストランでピザを注文しました。	レストランでピザを 注文[ちゅうもん]しました。	れすとらん で ぴざ を ちゅうもん しました	
\\	レストランでピザを
\\	しました。		
\\	意味	意味[いみ]	いみ	
\\	それはどういう意味ですか。	それはどういう 意味[いみ]ですか。	それ は どういう いみ です か	
\\	それはどういう
\\	ですか。		
\\	意見	意見[いけん]	いけん	
\\	あなたの意見が聞きたいです。	あなたの 意見[いけん]が 聞[き]きたいです。	あなた の いけん が ききたい です	
\\	あなたの
\\	が 聞[き]きたいです。		
\\	チケット	チケット	チケット	
\\	この遊園地のチケットは3000円です。	この 遊園地[ゆうえんち]のチケットは 3000円[さんぜんえん]です。	この ゆうえんち の ちけっと は さんぜんえん です	
\\	この 遊園地[ゆうえんち]の
\\	は 3000円[さんぜんえん]です。		
\\	注意	注意[ちゅうい]	ちゅうい	
\\	車に注意してください。	車[くるま]に 注意[ちゅうい]してください。	くるま に ちゅうい して ください	
\\	車[くるま]に
\\	してください。		
\\	用意	用意[ようい]	ようい	
\\	食事の用意ができました。	食事[しょくじ]の 用意[ようい]ができました。	しょくじ の ようい が できました	
\\	食事[しょくじ]の
\\	ができました。		
\\	確か	確[たし]か	たしか	
\\	彼の昇進は確かだ。	彼[かれ]の 昇進[しょうしん]は 確[たし]かだ。	かれ の しょうしん は たしか だ	
\\	彼[かれ]の 昇進[しょうしん]は
\\	だ。		
\\	確認	確認[かくにん]	かくにん	
\\	もう一度、予約を確認した。	もう 一度[いちど]、 予約[よやく]を 確認[かくにん]した。	もう いちど よやく を かくにん した	
\\	もう 一度[いちど]、 予約[よやく]を
\\	した。		
\\	チョコレート	チョコレート	チョコレート	
\\	妹はチョコレートが大好きです。	妹[いもうと]はチョコレートが 大好[だいす]きです。	いもうと は ちょこれーと が だいすき です	
\\	妹[いもうと]は
\\	が 大好[だいす]きです。		
\\	機能	機能[きのう]	きのう	
\\	このソフトにはいろいろな機能があります。	このソフトにはいろいろな 機能[きのう]があります。	この そふと に は いろいろな きのう が あります	
\\	このソフトにはいろいろな
\\	があります。		
\\	ジェット機	ジェット 機[き]	ジェットき	
\\	ジェット機が飛んでいる。	ジェット 機[き]が 飛[と]んでいる。	じぇっとき が とんで いる	
\\	が 飛[と]んでいる。		
\\	機械	機械[きかい]	きかい	
\\	新しい機械が壊れた。	新[あたら]しい 機械[きかい]が 壊[こわ]れた。	あたらしい きかい が こわれた	
\\	新[あたら]しい
\\	が 壊[こわ]れた。		
\\	材料	材料[ざいりょう]	ざいりょう	
\\	サラダの材料をそろえました。	サラダの 材料[ざいりょう]をそろえました。	さらだ の ざいりょう を そろえました	
\\	サラダの
\\	をそろえました。		
\\	チャンネル	チャンネル	チャンネル	
\\	テレビのチャンネルを変えてください。	テレビのチャンネルを 変[か]えてください。	てれび の ちゃんねる を かえて ください	
\\	テレビの
\\	を 変[か]えてください。		
\\	具体的	具体的[ぐたいてき]	ぐたいてき	
\\	具体的な例をいくつか見せてください。	具体的[ぐたいてき]な 例[れい]をいくつか 見[み]せてください。	ぐたいてき な れい を いくつか みせて ください	
\\	な 例[れい]をいくつか 見[み]せてください。		
\\	基づく	基[もと]づく	もとづく	
\\	この話は真実に基づいています。	この 話[はなし]は 真実[しんじつ]に 基[もと]づいています。	この はなし は しんじつ に もとづいて います	
\\	この 話[はなし]は 真実[しんじつ]に
\\	基本	基本[きほん]	きほん	
\\	今、ジャズダンスの基本を習っています。	今[いま]、ジャズダンスの 基本[きほん]を 習[なら]っています。	いま じゃず だんす の きほん を ならって います	
\\	今[いま]、ジャズダンスの
\\	を 習[なら]っています。		
\\	基準	基準[きじゅん]	きじゅん	
\\	判断の基準が示された。	判断[はんだん]の 基準[きじゅん]が 示[しめ]された。	はんだん の きじゅん が しめされた	
\\	判断[はんだん]の
\\	が 示[しめ]された。		
\\	つまらない	つまらない	つまらない	
\\	彼の話はつまらないですね。	彼[かれ]の 話[はなし]はつまらないですね。	かれ の はなし は つまらない です ね	
\\	彼[かれ]の 話[はなし]は
\\	ですね。		
\\	備える	備[そな]える	そなえる	
\\	災害に備えて大量の水を買い込んだ。	災害[さいがい]に 備[そな]えて 大量[たいりょう]の 水[みず]を 買[か]い 込[こ]んだ。	さいがい に そなえて たいりょう の みず を かいこんだ	
\\	災害[さいがい]に
\\	大量[たいりょう]の 水[みず]を 買[か]い 込[こ]んだ。		
\\	準備	準備[じゅんび]	じゅんび	
\\	明日の会議の準備をした。	明日[あす]の 会議[かいぎ]の 準備[じゅんび]をした。	あす の かいぎ の じゅんび を した	
\\	明日[あす]の 会議[かいぎ]の
\\	をした。		
\\	設計	設計[せっけい]	せっけい	
\\	兄は船の設計をしています。	兄[あに]は 船[ふね]の 設計[せっけい]をしています。	あに は ふね の せっけい を して います	
\\	兄[あに]は 船[ふね]の
\\	をしています。		
\\	施設	施設[しせつ]	しせつ	
\\	そのホテルにはレジャー施設がたくさんある。	そのホテルにはレジャー 施設[しせつ]がたくさんある。	その ほてる に は れじゃー しせつ が たくさん ある	
\\	そのホテルにはレジャー
\\	がたくさんある。		
\\	マッチ	マッチ	マッチ	
\\	マッチを持っていますか。	マッチを 持[も]っていますか。	まっち を もって います か	
\\	を 持[も]っていますか。		
\\	設ける	設[もう]ける	もうける	
\\	授業の終わりに復習の時間を設けた。	授業[じゅぎょう]の 終[お]わりに 復習[ふくしゅう]の 時間[じかん]を 設[もう]けた。	じゅぎょう の おわり に ふくしゅう の じかん を もうけた	
\\	授業[じゅぎょう]の 終[お]わりに 復習[ふくしゅう]の 時間[じかん]を
\\	説明	説明[せつめい]	せつめい	
\\	この単語の意味を説明してください。	この 単語[たんご]の 意味[いみ]を 説明[せつめい]してください。	この たんご の いみ を せつめい して ください	
\\	この 単語[たんご]の 意味[いみ]を
\\	してください。		
\\	小説	小説[しょうせつ]	しょうせつ	
\\	私は月に3冊くらい小説を読みます。	私[わたし]は 月[つき]に 3冊[さんさつ]くらい 小説[しょうせつ]を 読[よ]みます。	わたし は つき に さんさつ くらい しょうせつ を よみます	
\\	私[わたし]は 月[つき]に 3冊[さんさつ]くらい
\\	を 読[よ]みます。		
\\	公開	公開[こうかい]	こうかい	
\\	その映画は今日、公開されます。	その 映画[えいが]は 今日[きょう]、 公開[こうかい]されます。	その えいが は きょう こうかい されます	
\\	その 映画[えいが]は 今日[きょう]、
\\	されます。		
\\	いとこ	いとこ	いとこ	
\\	いとこと私は同じ年です。	いとこと 私[わたし]は 同[おな]じ 年[とし]です。	いとこ と わたし は おなじ とし です	
\\	と 私[わたし]は 同[おな]じ 年[とし]です。		
\\	公園	公園[こうえん]	こうえん	
\\	子供たちが公園で遊んでいる。	子供[こども]たちが 公園[こうえん]で 遊[あそ]んでいる。	こどもたち が こうえん で あそんで いる	
\\	子供[こども]たちが
\\	で 遊[あそ]んでいる。		
\\	動物園	動物園[どうぶつえん]	どうぶつえん	
\\	昨日子供たちと動物園に行きました。	昨日子供[きのう こども]たちと 動物園[どうぶつえん]に 行[い]きました。	きのう こどもたち と どうぶつえん に いきました	
\\	昨日子供[きのう こども]たちと
\\	に 行[い]きました。		
\\	祭り	祭[まつ]り	まつり	
\\	彼女は祭りが大好きです。	彼女[かのじょ]は 祭[まつ]りが 大好[だいす]きです。	かのじょ は まつり が だいすき です 。	
\\	彼女[かのじょ]は
\\	が 大好[だいす]きです。		
\\	国際	国際[こくさい]	こくさい	
\\	ここで国際会議が開かれます。	ここで 国際[こくさい] 会議[かいぎ]が 開[ひら]かれます。	ここ で こくさい かいぎ が ひらかれます	
\\	ここで
\\	会議[かいぎ]が 開[ひら]かれます。		
\\	ストーブ	ストーブ	ストーブ	
\\	うちではまだストーブを使っています。	うちではまだストーブを 使[つか]っています。	うち で は まだ すとーぶ を つかって います	
\\	うちではまだ
\\	を 使[つか]っています。		
\\	実際	実際[じっさい]	じっさい	
\\	彼は実際にはあまり背が高くない。	彼[かれ]は 実際[じっさい]にはあまり 背[せ]が 高[たか]くない。	かれ は じっさい に は あまり せ が たかく ない	
\\	彼[かれ]は
\\	にはあまり 背[せ]が 高[たか]くない。		
\\	飛行場	飛行場[ひこうじょう]	ひこうじょう	
\\	バスが飛行場に着きました。	バスが 飛行場[ひこうじょう]に 着[つ]きました。	ばす が ひこうじょう に つきました	
\\	バスが
\\	に 着[つ]きました。		
\\	航空便	航空便[こうくうびん]	こうくうびん	
\\	航空便で書類が届きました。	航空便[こうくうびん]で 書類[しょるい]が 届[とど]きました。	こうくうびん で しょるい が とどきました	
\\	で 書類[しょるい]が 届[とど]きました。		
\\	船便	船便[ふなびん]	ふなびん	
\\	アメリカの友人から船便が届いた。	アメリカの 友人[ゆうじん]から 船便[ふなびん]が 届[とど]いた。	あめりか の ゆうじん から ふなびん が とどいた	
\\	アメリカの 友人[ゆうじん]から
\\	が 届[とど]いた。		
\\	ガソリンスタンド	ガソリンスタンド	ガソリンスタンド	
\\	この近くにガソリンスタンドはありますか。	この 近[ちか]くにガソリンスタンドはありますか。	この ちかく に がそりんすたんど は あります か	
\\	この 近[ちか]くに
\\	はありますか。		
\\	船	船[ふね]	ふね	
\\	私たちは船に乗った。	私[わたし]たちは 船[ふね]に 乗[の]った。	わたしたち は ふね に のった	
\\	私[わたし]たちは
\\	に 乗[の]った。		
\\	空港	空港[くうこう]	くうこう	
\\	空港までリムジンバスで行った。	空港[くうこう]までリムジンバスで 行[い]った。	くうこう まで りむじん ばす で いった	
\\	までリムジンバスで 行[い]った。		
\\	港	港[みなと]	みなと	
\\	港に船が着きました。	港[みなと]に 船[ふね]が 着[つ]きました。	みなと に ふね が つきました	
\\	に 船[ふね]が 着[つ]きました。		
\\	島	島[しま]	しま	
\\	日本は島国です。	日本[にっぽん]は 島[しま] 国[ぐに]です。	にっぽん は しまぐに です	
\\	日本[にっぽん]は
\\	国[ぐに]です。		
\\	デート	デート	デート	
\\	デートで遊園地に行きました。	デートで 遊園地[ゆうえんち]に 行[い]きました。	でーと で ゆうえんち に いきました	
\\	で 遊園地[ゆうえんち]に 行[い]きました。		
\\	完成	完成[かんせい]	かんせい	
\\	新しいホームページが完成した。	新[あたら]しいホームページが 完成[かんせい]した。	あたらしい ほーむぺーじ が かんせい した	
\\	新[あたら]しいホームページが
\\	した。		
\\	平成	平成[へいせい]	へいせい	
\\	彼女は平成3年生まれです。	彼女[かのじょ]は 平成[へいせい] 3年生[さんねんう]まれです。	かのじょ は へいせい さんねん うまれ です	
\\	彼女[かのじょ]は
\\	3年生[さんねんう]まれです。		
\\	成功	成功[せいこう]	せいこう	
\\	ついに実験が成功した。	ついに 実験[じっけん]が 成功[せいこう]した。	ついに じっけん が せいこう した	
\\	ついに 実験[じっけん]が
\\	した。		
\\	原因	原因[げんいん]	げんいん	
\\	この事故の原因は何ですか。	この 事故[じこ]の 原因[げんいん]は 何[なん]ですか。	このじこ の げんいん は なん です か	
\\	この 事故[じこ]の
\\	は 何[なん]ですか。		
\\	ふすま	ふすま	ふすま	
\\	ふすまを閉めてください。	ふすまを 閉[し]めてください。	ふすま を しめて ください	
\\	を 閉[し]めてください。		
\\	資金	資金[しきん]	しきん	
\\	私たちは今、結婚資金を貯めています。	私[わたし]たちは 今[いま]、 結婚[けっこん] 資金[しきん]を 貯[た]めています。	わたしたち は いま けっこん しきん を ためて います	
\\	私[わたし]たちは 今[いま]、 結婚[けっこん]
\\	を 貯[た]めています。		
\\	投資	投資[とうし]	とうし	
\\	私は4つの会社に投資しています。	私[わたし]は 4[よっ]つの 会社[かいしゃ]に 投資[とうし]しています。	わたし は よっつ の かいしゃ に とうし して います	
\\	私[わたし]は 4[よっ]つの 会社[かいしゃ]に
\\	しています。		
\\	願う	願[ねが]う	ねがう	
\\	彼が元気になるよう願っています。	彼[かれ]が 元気[げんき]になるよう 願[ねが]っています。	かれ が げんき に なる よう ねがって います	
\\	彼[かれ]が 元気[げんき]になるよう
\\	正確	正確[せいかく]	せいかく	
\\	彼の計算は正確です。	彼[かれ]の 計算[けいさん]は 正確[せいかく]です。	かれ の けいさん は せいかく です	
\\	彼[かれ]の 計算[けいさん]は
\\	です。		
\\	レモン	レモン	レモン	
\\	紅茶にレモンを入れて飲んだ。	紅茶[こうちゃ]にレモンを 入[い]れて 飲[の]んだ。	こうちゃ に れもん を いれて のんだ	
\\	紅茶[こうちゃ]に
\\	を 入[い]れて 飲[の]んだ。		
\\	正しい	正[ただ]しい	ただしい	
\\	それは正しい答えです。	それは 正[ただ]しい 答[こた]えです。	それ は ただしい こたえ です	
\\	それは
\\	答[こた]えです。		
\\	正月	正月[しょうがつ]	しょうがつ	
\\	お正月にはたいてい、家族が集まる。	お 正月[しょうがつ]にはたいてい、 家族[かぞく]が 集[あつ]まる。	おしょうがつ に は たいてい かぞく が あつまる	
\\	お
\\	にはたいてい、 家族[かぞく]が 集[あつ]まる。		
\\	正直	正直[しょうじき]	しょうじき	
\\	彼女はとても正直だ。	彼女[かのじょ]はとても 正直[しょうじき]だ。	かのじょ は とても しょうじき だ	
\\	彼女[かのじょ]はとても
\\	だ。		
\\	異なる	異[こと]なる	ことなる	
\\	彼と私はいつも意見が異なる。	彼[かれ]と 私[わたし]はいつも 意見[いけん]が 異[こと]なる。	かれ と わたし は いつも いけん が ことなる	
\\	彼[かれ]と 私[わたし]はいつも 意見[いけん]が
\\	チーズ	チーズ	チーズ	
\\	チーズを一切れ食べました。	チーズを 一切[ひとき]れ 食[た]べました。	ちーず を ひときれ たべました	
\\	を 一切[ひとき]れ 食[た]べました。		
\\	通常	通常[つうじょう]	つうじょう	
\\	通常は夜8時まで営業しています。	通常[つうじょう]は 夜8時[よる はちじ]まで 営業[えいぎょう]しています。	つうじょう は よる はちじ まで えいぎょう して います	
\\	は 夜8時[よる はちじ]まで 営業[えいぎょう]しています。		
\\	非常に	非常[ひじょう]に	ひじょうに	
\\	これは非常に重要です。	これは 非常[ひじょう]に 重要[じゅうよう]です。	これ は ひじょう に じゅうよう です	
\\	これは
\\	重要[じゅうよう]です。		
\\	調べる	調[しら]べる	しらべる	
\\	この単語の意味を辞書で調べましょう。	この 単語[たんご]の 意味[いみ]を 辞書[じしょ]で 調[しら]べましょう。	この たんご の いみ を じしょ で しらべましょう	
\\	この 単語[たんご]の 意味[いみ]を 辞書[じしょ]で
\\	強調	強調[きょうちょう]	きょうちょう	
\\	彼は良いところだけを強調した。	彼[かれ]は 良[よ]いところだけを 強調[きょうちょう]した。	かれ は よい ところ だけ を きょうちょう した	
\\	彼[かれ]は 良[よ]いところだけを
\\	した。		
\\	ドライブ	ドライブ	ドライブ	
\\	今日は群馬までドライブしました。	今日[きょう]は 群馬[ぐんま]までドライブしました。	きょう は ぐんま まで どらいぶ しました	
\\	今日[きょう]は 群馬[ぐんま]まで
\\	しました。		
\\	季節	季節[きせつ]	きせつ	
\\	私の一番好きな季節は春です。	私[わたし]の 一番好[いちばん す]きな 季節[きせつ]は 春[はる]です。	わたし の いちばん すき な きせつ は はる です	
\\	私[わたし]の 一番好[いちばん す]きな
\\	は 春[はる]です。		
\\	調査	調査[ちょうさ]	ちょうさ	
\\	私たちがその問題を調査しています。	私[わたし]たちがその 問題[もんだい]を 調査[ちょうさ]しています。	わたしたち が その もんだい を ちょうさ して います	
\\	私[わたし]たちがその 問題[もんだい]を
\\	しています。		
\\	提供	提供[ていきょう]	ていきょう	
\\	彼がパーティー会場を提供してくれました。	彼[かれ]がパーティー 会場[かいじょう]を 提供[ていきょう]してくれました。	かれ が ぱーてぃー かいじょう を ていきょう して くれました	
\\	彼[かれ]がパーティー 会場[かいじょう]を
\\	してくれました。		
\\	提案	提案[ていあん]	ていあん	
\\	そのアイデアは彼の提案です。	そのアイデアは 彼[かれ]の 提案[ていあん]です。	その あいであ は かれ の ていあん です	
\\	そのアイデアは 彼[かれ]の
\\	です。		
\\	ラケット	ラケット	ラケット	
\\	テニスのラケットを買いました。	テニスのラケットを 買[か]いました。	てにす の らけっと を かいました	
\\	テニスの
\\	を 買[か]いました。		
\\	案内	案内[あんない]	あんない	
\\	私が中をご案内します。	私[わたし]が 中[なか]をご 案内[あんない]します。	わたし が なか を ごあんない します	
\\	私[わたし]が 中[なか]をご
\\	します。		
\\	示す	示[しめ]す	しめす	
\\	彼は新製品に興味を示している。	彼[かれ]は 新製品[しんせいひん]に 興味[きょうみ]を 示[しめ]している。	かれ は しんせいひん に きょうみ を しめして いる	
\\	彼[かれ]は 新製品[しんせいひん]に 興味[きょうみ]を
\\	連れて行く	連[つ]れて 行[い]く	つれていく	
\\	私も連れて行ってください。	私[わたし]も 連[つ]れて 行[い]ってください。	わたし も つれて いって ください	
\\	私[わたし]も
\\	ください。		
\\	連れて来る	連[つ]れて 来[く]る	つれてくる	
\\	息子が友達を連れて来ました。	息子[むすこ]が 友達[ともだち]を 連[つ]れて 来[き]ました。	むすこ が ともだち を つれて きました	
\\	息子[むすこ]が 友達[ともだち]を
\\	スチュワーデス	スチュワーデス	スチュワーデス	
\\	彼女はスチュワーデスになった。	彼女[かのじょ]はスチュワーデスになった。	かのじょ は すちゅわーです に なった	
\\	彼女[かのじょ]は
\\	になった。		
\\	続ける	続[つづ]ける	つづける	
\\	を		
\\	仕事を続けてください。	仕事[しごと]を 続[つづ]けてください。	しごと を つづけて ください	
\\	仕事[しごと]を
\\	ください。		
\\	相手	相手[あいて]	あいて	
\\	試合の相手は誰ですか。	試合[しあい]の 相手[あいて]は 誰[だれ]ですか。	しあい の あいて は だれ です か	
\\	試合[しあい]の
\\	は 誰[だれ]ですか。		
\\	会談	会談[かいだん]	かいだん	
\\	会談の内容が発表されました。	会談[かいだん]の 内容[ないよう]が 発表[はっぴょう]されました。	かいだん の ないよう が はっぴょう されました	
\\	の 内容[ないよう]が 発表[はっぴょう]されました。		
\\	相談	相談[そうだん]	そうだん	
\\	相談したいことがあります。	相談[そうだん]したいことがあります。	そうだん したい こと が あります	
\\	したいことがあります。		
\\	テープレコーダー	テープレコーダー	テープレコーダー	
\\	テープレコーダーで自分の声を録音しました。	テープレコーダーで 自分[じぶん]の 声[こえ]を 録音[ろくおん]しました。	てーぷれこーだー で じぶん の こえ を ろくおん しました	
\\	で 自分[じぶん]の 声[こえ]を 録音[ろくおん]しました。		
\\	記者	記者[きしゃ]	きしゃ	
\\	彼は新聞記者です。	彼[かれ]は 新聞[しんぶん] 記者[きしゃ]です。	かれ は しんぶん きしゃ です	
\\	彼[かれ]は 新聞[しんぶん]
\\	です。		
\\	記録	記録[きろく]	きろく	
\\	マラソンで世界記録が出た。	マラソンで 世界[せかい] 記録[きろく]が 出[で]た。	まらそん で せかい きろく が でた	
\\	マラソンで 世界[せかい]
\\	が 出[で]た。		
\\	録音	録音[ろくおん]	ろくおん	
\\	そのラジオ番組はもう録音しました。	そのラジオ 番組[ばんぐみ]はもう 録音[ろくおん]しました。	その らじお ばんぐみ は もう ろくおん しました	
\\	そのラジオ 番組[ばんぐみ]はもう
\\	しました。		
\\	登る	登[のぼ]る	のぼる	
\\	私たちは昨年、富士山に登りました。	私[わたし]たちは 昨年[さくねん]、 富士山[ふじさん]に 登[のぼ]りました。	わたしたち は さくねん ふじさん に のぼりました	
\\	私[わたし]たちは 昨年[さくねん]、 富士山[ふじさん]に
\\	ひげ	ひげ	ひげ	
\\	父はひげをはやしています。	父[ちち]はひげをはやしています。	ちち は ひげ を はやして います	
\\	父[ちち]は
\\	をはやしています。		
\\	関する	関[かん]する	かんする	
\\	その問題に関する記事を読みました。	その 問題[もんだい]に 関[かん]する 記事[きじ]を 読[よ]みました。	その もんだい に かんする きじ を よみました	
\\	その 問題[もんだい]に
\\	記事[きじ]を 読[よ]みました。		
\\	関連	関連[かんれん]	かんれん	
\\	関連のセクションへ連絡した。	関連[かんれん]のセクションへ 連絡[れんらく]した。	かんれん の せくしょん へ れんらく した	
\\	のセクションへ 連絡[れんらく]した。		
\\	関係	関係[かんけい]	かんけい	
\\	彼はその事件に関係がない。	彼[かれ]はその 事件[じけん]に 関係[かんけい]がない。	かれ は その じけん に かんけい が ない	
\\	彼[かれ]はその 事件[じけん]に
\\	がない。		
\\	状況	状況[じょうきょう]	じょうきょう	
\\	この状況では出発は難しいです。	この 状況[じょうきょう]では 出発[しゅっぱつ]は 難[むずか]しいです。	この じょうきょう で は しゅっぱつ は むずかしい です	
\\	この
\\	では 出発[しゅっぱつ]は 難[むずか]しいです。		
\\	ぐっすり	ぐっすり	ぐっすり	
\\	赤ちゃんがぐっすり寝ている。	赤[あか]ちゃんがぐっすり 寝[ね]ている。	あかちゃん が ぐっすり ねて いる	
\\	赤[あか]ちゃんが
\\	寝[ね]ている。		
\\	状態	状態[じょうたい]	じょうたい	
\\	ここは道の状態がとても悪いです。	ここは 道[みち]の 状態[じょうたい]がとても 悪[わる]いです。	ここ は みち の じょうたい が とても わるい です	
\\	ここは 道[みち]の
\\	がとても 悪[わる]いです。		
\\	治る	治[なお]る	なおる	
\\	けがはもう治りましたか。	けがはもう 治[なお]りましたか。	けが は もう なおりました か	
\\	けがはもう
\\	か。		
\\	政治	政治[せいじ]	せいじ	
\\	私は政治に関心がある。	私[わたし]は 政治[せいじ]に 関心[かんしん]がある。	わたし は せいじ に かんしん が ある	
\\	私[わたし]は
\\	に 関心[かんしん]がある。		
\\	治す	治[なお]す	なおす	
\\	早く風邪を治してください。	早[はや]く 風邪[かぜ]を 治[なお]してください。	はやく かぜ を なおして ください	
\\	早[はや]く 風邪[かぜ]を
\\	ください。		
\\	ソース	ソース	ソース	
\\	ソースはどれですか。	ソースはどれですか。	そーす は どれ です か	
\\	はどれですか。		
\\	政府	政府[せいふ]	せいふ	
\\	そのデモについて政府は何もしなかった。	そのデモについて 政府[せいふ]は 何[なに]もしなかった。	その でも に ついて せいふ は なにも しなかった	
\\	そのデモについて
\\	は 何[なに]もしなかった。		
\\	党	党[とう]	とう	
\\	党の代表が質問に答えました。	党[とう]の 代表[だいひょう]が 質問[しつもん]に 答[こた]えました。	とう の だいひょう が しつもん に こたえました	
\\	の 代表[だいひょう]が 質問[しつもん]に 答[こた]えました。		
\\	対策	対策[たいさく]	たいさく	
\\	一緒に対策を考えましょう。	一緒[いっしょ]に 対策[たいさく]を 考[かんが]えましょう。	いっしょ に たいさく を かんがえましょう	
\\	一緒[いっしょ]に
\\	を 考[かんが]えましょう。		
\\	政策	政策[せいさく]	せいさく	
\\	新しい政策はあまり良いとは思えません。	新[あたら]しい 政策[せいさく]はあまり 良[い]いとは 思[おも]えません。	あたらしい せいさく は あまり いい と は おもえません	
\\	新[あたら]しい
\\	はあまり 良[い]いとは 思[おも]えません。		
\\	タイプライター	タイプライター	タイプライター	
\\	母は古いタイプライターを持っています。	母[はは]は 古[ふる]いタイプライターを 持[も]っています。	はは は ふるい たいぷらいたー を もって います	
\\	母[はは]は 古[ふる]い
\\	を 持[も]っています。		
\\	選ぶ	選[えら]ぶ	えらぶ	
\\	良い家を選ぶのは難しい。	良[よ]い 家[いえ]を 選[えら]ぶのは 難[むずか]しい。	よい いえ を えらぶ の は むずかしい	
\\	良[よ]い 家[いえ]を
\\	のは 難[むずか]しい。		
\\	選手	選手[せんしゅ]	せんしゅ	
\\	彼はプロのサッカー選手だ。	彼[かれ]はプロのサッカー 選手[せんしゅ]だ。	かれ は ぷろ の さっかー せんしゅ だ	
\\	彼[かれ]はプロのサッカー
\\	だ。		
\\	選挙	選挙[せんきょ]	せんきょ	
\\	彼は選挙に出るつもりだ。	彼[かれ]は 選挙[せんきょ]に 出[で]るつもりだ。	かれ は せんきょ に でる つもり だ	
\\	彼[かれ]は
\\	に 出[で]るつもりだ。		
\\	候補	候補[こうほ]	こうほ	
\\	会長の候補は3人います。	会長[かいちょう]の 候補[こうほ]は 3人[さんにん]います。	かいちょう の こうほ は さんにん います	
\\	会長[かいちょう]の
\\	は 3人[さんにん]います。		
\\	トマト	トマト	トマト	
\\	私はトマトが大好きです。	私[わたし]はトマトが 大好[だいす]きです。	わたし は とまと が だいすき です	
\\	私[わたし]は
\\	が 大好[だいす]きです。		
\\	首相	首相[しゅしょう]	しゅしょう	
\\	今の首相はあまり力がない。	今[いま]の 首相[しゅしょう]はあまり 力[ちから]がない。	いま の しゅしょう は あまり ちから が ない	
\\	今[いま]の
\\	はあまり 力[ちから]がない。		
\\	首都	首都[しゅと]	しゅと	
\\	東京は日本の首都です。	東京[とうきょう]は 日本[にっぽん]の 首都[しゅと]です。	とうきょう は にっぽん の しゅと です	
\\	東京[とうきょう]は 日本[にっぽん]の
\\	です。		
\\	改革	改革[かいかく]	かいかく	
\\	彼は行政を改革したいと思っている。	彼[かれ]は 行政[ぎょうせい]を 改革[かいかく]したいと 思[おも]っている。	かれ は ぎょうせい を かいかく したい と おもって いる	
\\	彼[かれ]は 行政[ぎょうせい]を
\\	したいと 思[おも]っている。		
\\	革	革[かわ]	かわ	
\\	革のベルトを買いました。	革[かわ]のベルトを 買[か]いました。	かわ の べると を かいました	
\\	のベルトを 買[か]いました。		
\\	バター	バター	バター	
\\	パンにバターをぬって食べました。	パンにバターをぬって 食[た]べました。	ぱん に ばたー を ぬって たべました	
\\	パンに
\\	をぬって 食[た]べました。		
\\	命令	命令[めいれい]	めいれい	
\\	彼女は命令に従わなかった。	彼女[かのじょ]は 命令[めいれい]に 従[したが]わなかった。	かのじょ は めいれい に したがわなかった	
\\	彼女[かのじょ]は
\\	に 従[したが]わなかった。		
\\	番組	番組[ばんぐみ]	ばんぐみ	
\\	私はこの番組が好きです。	私[わたし]はこの 番組[ばんぐみ]が 好[す]きです。	わたし は この ばんぐみ が すき です	
\\	私[わたし]はこの
\\	が 好[す]きです。		
\\	組み立てる	組[く]み 立[た]てる	くみたてる	
\\	日曜日に本棚を組み立てます。	日曜日[にちようび]に 本棚[ほんだな]を 組[く]み 立[た]てます。	にちようび に ほんだな を くみたてます	
\\	日曜日[にちようび]に 本棚[ほんだな]を
\\	組織	組織[そしき]	そしき	
\\	彼はある組織のリーダーだ。	彼[かれ]はある 組織[そしき]のリーダーだ。	かれ は ある そしき の りーだー だ	
\\	彼[かれ]はある
\\	のリーダーだ。		
\\	バレーボール	バレーボール	バレーボール	
\\	妹はバレーボールが得意です。	妹[いもうと]はバレーボールが 得意[とくい]です。	いもうと は ばれーぼーる が とくい です	
\\	妹[いもうと]は
\\	が 得意[とくい]です。		
\\	進める	進[すす]める	すすめる	
\\	早く授業を進めましょう。	早[はや]く 授業[じゅぎょう]を 進[すす]めましょう。	はやく じゅぎょう を すすめましょう	
\\	早[はや]く 授業[じゅぎょう]を
\\	進む	進[すす]む	すすむ	
\\	前に進んでください。	前[まえ]に 進[すす]んでください。	まえ に すすんで ください	
\\	前[まえ]に
\\	ください。		
\\	進学	進学[しんがく]	しんがく	
\\	彼女は来年、大学に進学する。	彼女[かのじょ]は 来年[らいねん]、 大学[だいがく]に 進学[しんがく]する。	かのじょ は らいねん だいがく に しんがく する	
\\	彼女[かのじょ]は 来年[らいねん]、 大学[だいがく]に
\\	する。		
\\	拡大	拡大[かくだい]	かくだい	
\\	この図を拡大コピーしてください。	この 図[ず]を 拡大[かくだい]コピーしてください。	この ず を かくだい こぴー して ください	
\\	この 図[ず]を
\\	コピーしてください。		
\\	おかず	おかず	おかず	
\\	晩ご飯のおかずは何?	晩[ばん]ご 飯[はん]のおかずは 何?[なに]	ばんごはん の おかず は なに	
\\	晩[ばん]ご 飯[はん]の
\\	は 何?[なに]		
\\	責任	責任[せきにん]	せきにん	
\\	彼は失敗の責任を取って、会社を辞めた。	彼[かれ]は 失敗[しっぱい]の 責任[せきにん]を 取[と]って、 会社[かいしゃ]を 辞[や]めた。	かれ は しっぱい の せきにん を とって かいしゃ を やめた	
\\	彼[かれ]は 失敗[しっぱい]の
\\	を 取[と]って、 会社[かいしゃ]を 辞[や]めた。		
\\	辞める	辞[や]める	やめる	
\\	彼は会社を辞めます。	彼[かれ]は 会社[かいしゃ]を 辞[や]めます。	かれ は かいしゃ を やめます	
\\	彼[かれ]は 会社[かいしゃ]を
\\	辞書	辞書[じしょ]	じしょ	
\\	彼女は辞書をよく使います。	彼女[かのじょ]は 辞書[じしょ]をよく 使[つか]います。	かのじょ は じしょ を よく つかいます	
\\	彼女[かのじょ]は
\\	をよく 使[つか]います。		
\\	通勤	通勤[つうきん]	つうきん	
\\	毎朝、通勤に30分かかります。	毎朝[まいあさ]、 通勤[つうきん]に 30分[さんじゅっぷん]かかります。	まいあさ つうきん に さんじゅっぷん かかります	
\\	毎朝[まいあさ]、
\\	に 30分[さんじゅっぷん]かかります。		
\\	カレンダー	カレンダー	カレンダー	
\\	カレンダーに予定を書いた。	カレンダーに 予定[よてい]を 書[か]いた。	かれんだー に よてい を かいた	
\\	に 予定[よてい]を 書[か]いた。		
\\	勤める	勤[つと]める	つとめる	
\\	(~に)		
\\	私は銀行に勤めています。	私[わたし]は 銀行[ぎんこう]に 勤[つと]めています。	わたし は ぎんこう に つとめて います	
\\	私[わたし]は 銀行[ぎんこう]に
\\	事務所	事務所[じむしょ]	じむしょ	
\\	後で事務所に来てください。	後[あと]で 事務所[じむしょ]に 来[き]てください。	あとで じむしょ に きて ください	
\\	後[あと]で
\\	に 来[き]てください。		
\\	事務室	事務室[じむしつ]	じむしつ	
\\	事務室でコピーを取って来ます。	事務室[じむしつ]でコピーを 取[と]って 来[き]ます。	じむしつ で こぴー を とって きます	
\\	でコピーを 取[と]って 来[き]ます。		
\\	従来	従来[じゅうらい]	じゅうらい	
\\	このプリンターは従来のものより速い。	このプリンターは 従来[じゅうらい]のものより 速[はや]い。	この ぷりんたー は じゅうらい の もの より はやい	
\\	このプリンターは
\\	のものより 速[はや]い。		
\\	ハンドバッグ	ハンドバッグ	ハンドバッグ	
\\	ハンドバッグを忘れました。	ハンドバッグを 忘[わす]れました。	はんどばっぐ を わすれました	
\\	を 忘[わす]れました。		
\\	成績	成績[せいせき]	せいせき	
\\	成績が上がりました。	成績[せいせき]が 上[あ]がりました。	せいせき が あがりました	
\\	が 上[あ]がりました。		
\\	集める	集[あつ]める	あつめる	
\\	弟は切手を集めています。	弟[おとうと]は 切手[きって]を 集[あつ]めています。	おとうと は きって を あつめて います	
\\	弟[おとうと]は 切手[きって]を
\\	集まる	集[あつ]まる	あつまる	
\\	駅前に人が集まっています。	駅前[えきまえ]に 人[ひと]が 集[あつ]まっています。	えきまえ に ひと が あつまって います	
\\	駅前[えきまえ]に 人[ひと]が
\\	採用	採用[さいよう]	さいよう	
\\	その会社は女性を多く採用している。	その 会社[かいしゃ]は 女性[じょせい]を 多[おお]く 採用[さいよう]している。	その かいしゃ は じょせい を おおく さいよう して いる	
\\	その 会社[かいしゃ]は 女性[じょせい]を 多[おお]く
\\	している。		
\\	ベル	ベル	ベル	
\\	玄関のベルが鳴った。	玄関[げんかん]のベルが 鳴[な]った。	げんかん の べる が なった	
\\	玄関[げんかん]の
\\	が 鳴[な]った。		
\\	給料	給料[きゅうりょう]	きゅうりょう	
\\	来年から給料が上がります。	来年[らいねん]から 給料[きゅうりょう]が 上[あ]がります。	らいねん から きゅうりょう が あがります	
\\	来年[らいねん]から
\\	が 上[あ]がります。		
\\	卒業	卒業[そつぎょう]	そつぎょう	
\\	私は去年、大学を卒業した。	私[わたし]は 去年[きょねん]、 大学[だいがく]を 卒業[そつぎょう]した。	わたし は きょねん だいがく を そつぎょう した	
\\	私[わたし]は 去年[きょねん]、 大学[だいがく]を
\\	した。		
\\	就職	就職[しゅうしょく]	しゅうしょく	
\\	最近、若い人たちの就職が難しくなっています。	最近[さいきん]、 若[わか]い 人[ひと]たちの 就職[しゅうしょく]が 難[むずか]しくなっています。	さいきん わかい ひとたち の しゅうしょく が むずかしく なって います	
\\	最近[さいきん]、 若[わか]い 人[ひと]たちの
\\	が 難[むずか]しくなっています。		
\\	退院	退院[たいいん]	たいいん	
\\	母が今日退院します。	母[はは]が 今日[きょう] 退院[たいいん]します。	はは が きょう たいいん します	
\\	母[はは]が 今日[きょう]
\\	します。		
\\	インク	インク	インク	
\\	プリンターのインクを買った。	プリンターのインクを 買[か]った。	ぷりんたー の いんく を かった	
\\	プリンターの
\\	を 買[か]った。		
\\	契約	契約[けいやく]	けいやく	
\\	その選手は新しいチームと契約した。	その 選手[せんしゅ]は 新[あたら]しいチームと 契約[けいやく]した。	その せんしゅ は あたらしい ちーむ と けいやく した	
\\	その 選手[せんしゅ]は 新[あたら]しいチームと
\\	した。		
\\	交渉	交渉[こうしょう]	こうしょう	
\\	今、値段を交渉しています。	今[いま]、 値段[ねだん]を 交渉[こうしょう]しています。	いま ねだん を こうしょう して います	
\\	今[いま]、 値段[ねだん]を
\\	しています。		
\\	事件	事件[じけん]	じけん	
\\	その事件の犯人はまだ捕まっていない。	その 事件[じけん]の 犯人[はんにん]はまだ 捕[つか]まっていない。	その じけん の はんにん は まだ つかまって いない	
\\	その
\\	の 犯人[はんにん]はまだ 捕[つか]まっていない。		
\\	条件	条件[じょうけん]	じょうけん	
\\	この条件では厳し過ぎます。	この 条件[じょうけん]では 厳[きび]し 過[す]ぎます。	この じょうけん で は きびし すぎます	
\\	この
\\	では 厳[きび]し 過[す]ぎます。		
\\	クーラー	クーラー	クーラー	
\\	暑いのでクーラーをつけました。	暑[あつ]いのでクーラーをつけました。	あつい の で くーらー を つけました	
\\	暑[あつ]いので
\\	をつけました。		
\\	参加	参加[さんか]	さんか	
\\	明日は市民マラソンに参加します。	明日[あす]は 市民[しみん]マラソンに 参加[さんか]します。	あす は しみん まらそん に さんか します	
\\	明日[あす]は 市民[しみん]マラソンに
\\	します。		
\\	増加	増加[ぞうか]	ぞうか	
\\	島の人口は年々増加しています。	島[しま]の 人口[じんこう]は 年々[ねんねん] 増加[ぞうか]しています。	しま の じんこう は ねんねん ぞうか して います	
\\	島[しま]の 人口[じんこう]は 年々[ねんねん]
\\	しています。		
\\	加える	加[くわ]える	くわえる	
\\	塩、コショウを加えてください。	塩[しお]、コショウを 加[くわ]えてください。	しお こしょう を くわえて ください	
\\	塩[しお]、コショウを
\\	ください。		
\\	加工	加工[かこう]	かこう	
\\	この工場では魚を加工している。	この 工場[こうじょう]では 魚[さかな]を 加工[かこう]している。	この こうじょう で は さかな を かこう して いる	
\\	この 工場[こうじょう]では 魚[さかな]を
\\	している。		
\\	ミルク	ミルク	ミルク	
\\	赤ちゃんにミルクをあげた。	赤[あか]ちゃんにミルクをあげた。	あかちゃん に みるく を あげた	
\\	赤[あか]ちゃんに
\\	をあげた。		
\\	比べる	比[くら]べる	くらべる	
\\	今月と先月の売上を比べた。	今月[こんげつ]と 先月[せんげつ]の 売上[うりあげ]を 比[くら]べた。	こんげつ と せんげつ の うりあげ を くらべた	
\\	今月[こんげつ]と 先月[せんげつ]の 売上[うりあげ]を
\\	批判	批判[ひはん]	ひはん	
\\	彼は同僚を批判した。	彼[かれ]は 同僚[どうりょう]を 批判[ひはん]した。	かれ は どうりょう を ひはん した	
\\	彼[かれ]は 同僚[どうりょう]を
\\	した。		
\\	評価	評価[ひょうか]	ひょうか	
\\	最近、彼の評価が上がった。	最近[さいきん]、 彼[かれ]の 評価[ひょうか]が 上[あ]がった。	さいきん かれ の ひょうか が あがった	
\\	最近[さいきん]、 彼[かれ]の
\\	が 上[あ]がった。		
\\	対象	対象[たいしょう]	たいしょう	
\\	このアンケートは大学生が対象です。	このアンケートは 大学生[だいがくせい]が 対象[たいしょう]です。	この あんけーと は だいがくせい が たいしょう です	
\\	このアンケートは 大学生[だいがくせい]が
\\	です。		
\\	サラダ	サラダ	サラダ	
\\	サラダをたくさん食べました。	サラダをたくさん 食[た]べました。	さらだ を たくさん たべました	
\\	をたくさん 食[た]べました。		
\\	故障	故障[こしょう]	こしょう	
\\	冷蔵庫が故障しました。	冷蔵庫[れいぞうこ]が 故障[こしょう]しました。	れいぞうこ が こしょう しました	
\\	冷蔵庫[れいぞうこ]が
\\	しました。		
\\	修理	修理[しゅうり]	しゅうり	
\\	車を修理に出した。	車[くるま]を 修理[しゅうり]に 出[だ]した。	くるま を しゅうり に だした	
\\	車[くるま]を
\\	に 出[だ]した。		
\\	乗り換える	乗[の]り 換[か]える	のりかえる	
\\	など)		
\\	次の駅で地下鉄に乗り換えます。	次[つぎ]の 駅[えき]で 地下鉄[ちかてつ]に 乗[の]り 換[か]えます。	つぎ の えき で ちかてつ に のりかえます	
\\	次[つぎ]の 駅[えき]で 地下鉄[ちかてつ]に
\\	乗り換え	乗[の]り 換[か]え	のりかえ	
\\	など), 
\\	次の駅で乗り換えです。	次[つぎ]の 駅[えき]で 乗[の]り 換[か]えです。	つぎ の えき で のりかえ です	
\\	次[つぎ]の 駅[えき]で
\\	です。		
\\	おじさん	おじさん	おじさん	
\\	昨日おじさんに会いました。	昨日[きのう]おじさんに 会[あ]いました。	きのう おじさん に あいました	
\\	昨日[きのう]
\\	に 会[あ]いました。		
\\	換える	換[か]える	かえる	
\\	車のタイヤを換えた。	車[くるま]のタイヤを 換[か]えた。	くるま の たいや を かえた	
\\	車[くるま]のタイヤを
\\	着替える	着替[きが]える	きがえる	
\\	彼はきれいな服に着替えた。	彼[かれ]はきれいな 服[ふく]に 着替[きが]えた。	かれ は きれい な ふく に きがえた	
\\	彼[かれ]はきれいな 服[ふく]に
\\	被る	被[かぶ]る	かぶる	
\\	帽子を被って外出した。	帽子[ぼうし]を 被[かぶ]って 外出[がいしゅつ]した。	ぼうし を かぶって がいしゅつ した	
\\	帽子[ぼうし]を
\\	外出[がいしゅつ]した。		
\\	破る	破[やぶ]る	やぶる	
\\	(~を)		
\\	彼は約束を破った。	彼[かれ]は 約束[やくそく]を 破[やぶ]った。	かれ は やくそく を やぶった	
\\	彼[かれ]は 約束[やくそく]を
\\	ベルト	ベルト	ベルト	
\\	ベルトがきつくなりました。	ベルトがきつくなりました。	べると が きつく なりました	
\\	がきつくなりました。		
\\	破れる	破[やぶ]れる	やぶれる	
\\	シャツが破れている。	シャツが 破[やぶ]れている。	しゃつ が やぶれて いる	
\\	シャツが
\\	壊す	壊[こわ]す	こわす	
\\	彼女が私のケータイを壊した。	彼女[かのじょ]が 私[わたし]のケータイを 壊[こわ]した。	かのじょ が わたし の けーたい を こわした	
\\	彼女[かのじょ]が 私[わたし]のケータイを
\\	壊れる	壊[こわ]れる	こわれる	
\\	会社のパソコンが壊れた。	会社[かいしゃ]のパソコンが 壊[こわ]れた。	かいしゃ の ぱそこん が こわれた	
\\	会社[かいしゃ]のパソコンが
\\	救急車	救急車[きゅうきゅうしゃ]	きゅうきゅうしゃ	
\\	誰か救急車を呼んでください。	誰[だれ]か 救急車[きゅうきゅうしゃ]を 呼[よ]んでください。	だれか きゅうきゅうしゃ を よんで ください	
\\	誰[だれ]か
\\	を 呼[よ]んでください。		
\\	ラーメン	ラーメン	ラーメン	
\\	夕食にラーメンを食べました。	夕食[ゆうしょく]にラーメンを 食[た]べました。	ゆうしょく に らーめん を たべました	
\\	夕食[ゆうしょく]に
\\	を 食[た]べました。		
\\	助ける	助[たす]ける	たすける	
\\	彼女は病気の犬を助けた。	彼女[かのじょ]は 病気[びょうき]の 犬[いぬ]を 助[たす]けた。	かのじょ は びょうき の いぬ を たすけた	
\\	彼女[かのじょ]は 病気[びょうき]の 犬[いぬ]を
\\	立派	立派[りっぱ]	りっぱ	
\\	彼は立派な人です。	彼[かれ]は 立派[りっぱ]な 人[ひと]です。	かれ は りっぱ な ひと です	
\\	彼[かれ]は
\\	な 人[ひと]です。		
\\	警察	警察[けいさつ]	けいさつ	
\\	警察を呼んでください。	警察[けいさつ]を 呼[よ]んでください。	けいさつ を よんで ください	
\\	を 呼[よ]んでください。		
\\	管理	管理[かんり]	かんり	
\\	彼女が私のスケジュールを管理している。	彼女[かのじょ]が 私[わたし]のスケジュールを 管理[かんり]している。	かのじょ が わたし の すけじゅーる を かんり して いる	
\\	彼女[かのじょ]が 私[わたし]のスケジュールを
\\	している。		
\\	ライター	ライター	ライター	
\\	店にライターを忘れました。	店[みせ]にライターを 忘[わす]れました。	みせ に らいたー を わすれました	
\\	店[みせ]に
\\	を 忘[わす]れました。		
\\	盗む	盗[ぬす]む	ぬすむ	
\\	誰かが私のカバンを盗みました。	誰[だれ]かが 私[わたし]のカバンを 盗[ぬす]みました。	だれか が わたし の かばん を ぬすみました	
\\	誰[だれ]かが 私[わたし]のカバンを
\\	殺す	殺[ころ]す	ころす	
\\	私は生き物を殺すのが嫌いだ。	私[わたし]は 生[い]き 物[もの]を 殺[ころ]すのが 嫌[きら]いだ。	わたし は いきもの を ころす の が きらい だ	
\\	私[わたし]は 生[い]き 物[もの]を
\\	のが 嫌[きら]いだ。		
\\	逮捕	逮捕[たいほ]	たいほ	
\\	逃げていた犯人が逮捕されました。	逃[に]げていた 犯人[はんにん]が 逮捕[たいほ]されました。	にげて いた はんにん が たいほ されました	
\\	逃[に]げていた 犯人[はんにん]が
\\	されました。		
\\	逃げる	逃[に]げる	にげる	
\\	鳥が窓から逃げたよ。	鳥[とり]が 窓[まど]から 逃[に]げたよ。	とり が まど から にげた よ	
\\	鳥[とり]が 窓[まど]から
\\	よ。		
\\	ワイシャツ	ワイシャツ	ワイシャツ	
\\	彼は白いワイシャツを着ている。	彼[かれ]は 白[しろ]いワイシャツを 着[き]ている。	かれ は しろい わいしゃつ を きて いる	
\\	彼[かれ]は 白[しろ]い
\\	を 着[き]ている。		
\\	戦争	戦争[せんそう]	せんそう	
\\	2003年にイラクで戦争があった。	2003年[にせんさんねん]にイラクで 戦争[せんそう]があった。	にせんさんねん に いらく で せんそう が あった	
\\	2003年[にせんさんねん]にイラクで
\\	があった。		
\\	競争	競争[きょうそう]	きょうそう	
\\	2社は互いに競争している。	2社[にしゃ]は 互[たが]いに 競争[きょうそう]している。	にしゃ は たがい に きょうそう して いる	
\\	2社[にしゃ]は 互[たが]いに
\\	している。		
\\	混む	混[こ]む	こむ	
\\	電車が混んでいる。	電車[でんしゃ]が 混[こ]んでいる。	でんしゃ が こんで いる	
\\	電車[でんしゃ]が
\\	大統領	大統領[だいとうりょう]	だいとうりょう	
\\	フランスの大統領は誰ですか。	フランスの 大統領[だいとうりょう]は 誰[だれ]ですか。	ふらんす の だいとうりょう は だれ です か	
\\	フランスの
\\	は 誰[だれ]ですか。		
\\	シャワー	シャワー	シャワー	
\\	朝、急いでシャワーを浴びました。	朝[あさ]、 急[いそ]いでシャワーを 浴[あ]びました。	あさ いそいで しゃわー を あびました	
\\	朝[あさ]、 急[いそ]いで
\\	を 浴[あ]びました。		
\\	捨てる	捨[す]てる	すてる	
\\	ゴミを捨ててください。	ゴミを 捨[す]ててください。	ごみ を すてて ください	
\\	ゴミを
\\	ください。		
\\	拾う	拾[ひろ]う	ひろう	
\\	道で財布を拾った。	道[みち]で 財布[さいふ]を 拾[ひろ]った。	みち で さいふ を ひろった	
\\	道[みち]で 財布[さいふ]を
\\	池	池[いけ]	いけ	
\\	池に鯉がいます。	池[いけ]に 鯉[こい]がいます。	いけ に こい が います	
\\	に 鯉[こい]がいます。		
\\	電池	電池[でんち]	でんち	
\\	新しい電池を入れましょう。	新[あたら]しい 電池[でんち]を 入[い]れましょう。	あたらしい でんち を いれましょう	
\\	新[あたら]しい
\\	を 入[い]れましょう。		
\\	うどん	うどん	うどん	
\\	私はお昼にうどんを食べました。	私[わたし]はお 昼[ひる]にうどんを 食[た]べました。	わたし は おひる に うどん を たべました	
\\	私[わたし]はお 昼[ひる]に
\\	を 食[た]べました。		
\\	深い	深[ふか]い	ふかい	
\\	あの池はとても深い。	あの 池[いけ]はとても 深[ふか]い。	あの いけ は とても ふかい	
\\	あの 池[いけ]はとても
\\	深さ	深[ふか]さ	ふかさ	
\\	このプールの深さは2メートルです。	このプールの 深[ふか]さは 2[に]メートルです。	この ぷーる の ふかさ は にめーとる です	
\\	このプールの
\\	は 2[に]メートルです。		
\\	浅い	浅[あさ]い	あさい	
\\	この川は浅いです。	この 川[かわ]は 浅[あさ]いです。	この かわ は あさい です	
\\	この 川[かわ]は
\\	です。		
\\	落とす	落[お]とす	おとす	
\\	途中で財布を落としました。	途中[とちゅう]で 財布[さいふ]を 落[お]としました。	とちゅう で さいふ を おとしました	
\\	途中[とちゅう]で 財布[さいふ]を
\\	ごみ	ごみ	ごみ	
\\	今日はごみの日だ。	今日[きょう]はごみの 日[ひ]だ。	きょう は ごみ の ひ だ	
\\	今日[きょう]は
\\	の 日[ひ]だ。		
\\	泳ぐ	泳[およ]ぐ	およぐ	
\\	彼女はダイエットのために泳いでいる。	彼女[かのじょ]はダイエットのために 泳[およ]いでいる。	かのじょ は だいえっと の ため に およいで いる	
\\	彼女[かのじょ]はダイエットのために
\\	水泳	水泳[すいえい]	すいえい	
\\	母は健康のために水泳をしている。	母[はは]は 健康[けんこう]のために 水泳[すいえい]をしている。	はは は けんこう の ため に すいえい を して いる	
\\	母[はは]は 健康[けんこう]のために
\\	をしている。		
\\	流れる	流[なが]れる	ながれる	
\\	ラジオから美しい音楽が流れています。	ラジオから 美[うつく]しい 音楽[おんがく]が 流[なが]れています。	らじお から うつくしい おんがく が ながれて います	
\\	ラジオから 美[うつく]しい 音楽[おんがく]が
\\	流行る	流行[はや]る	はやる	
\\	去年はスニーカーが流行りました。	去年[きょねん]はスニーカーが 流行[はや]りました。	きょねん は すにーかー が はやりました	
\\	去年[きょねん]はスニーカーが
\\	スーパーマーケット	スーパーマーケット	スーパーマーケット	
\\	スーパーマーケットで人参を買った。	スーパーマーケットで 人参[にんじん]を 買[か]った。	すーぱーまーけっと で にんじん を かった	
\\	で 人参[にんじん]を 買[か]った。		
\\	洗う	洗[あら]う	あらう	
\\	早く顔を洗いなさい。	早[はや]く 顔[かお]を 洗[あら]いなさい。	はやく かお を あらいなさい	
\\	早[はや]く 顔[かお]を
\\	洗面所	洗面所[せんめんじょ]	せんめんじょ	
\\	洗面所で顔を洗った。	洗面所[せんめんじょ]で 顔[かお]を 洗[あら]った。	せんめんじょ で かお を あらった	
\\	で 顔[かお]を 洗[あら]った。		
\\	油	油[あぶら]	あぶら	
\\	水と油は混ざらない。	水[みず]と 油[あぶら]は 混[ま]ざらない。	みず と あぶら は まざらない	
\\	水[みず]と
\\	は 混[ま]ざらない。		
\\	沈む	沈[しず]む	しずむ	
\\	ボートが川に沈んだ。	ボートが 川[かわ]に 沈[しず]んだ。	ぼーと が かわ に しずんだ	
\\	ボートが 川[かわ]に
\\	ハンカチ	ハンカチ	ハンカチ	
\\	ハンカチで手をふきました。	ハンカチで 手[て]をふきました。	はんかち で て を ふきました	
\\	で 手[て]をふきました。		
\\	久しぶり	久[ひさ]しぶり	ひさしぶり	
\\	明日、久しぶりに友達に会います。	明日[あした]、 久[ひさ]しぶりに 友達[ともだち]に 会[あ]います。	あした ひさしぶり に ともだち に あいます	
\\	明日[あした]、
\\	に 友達[ともだち]に 会[あ]います。		
\\	氷	氷[こおり]	こおり	
\\	グラスに氷を入れてください。	グラスに 氷[こおり]を 入[い]れてください。	ぐらす に こおり を いれて ください	
\\	グラスに
\\	を 入[い]れてください。		
\\	冷える	冷[ひ]える	ひえる	
\\	クーラーで体が冷えた。	クーラーで 体[からだ]が 冷[ひ]えた。	くーらー で からだ が ひえた	
\\	クーラーで 体[からだ]が
\\	冷やす	冷[ひ]やす	ひやす	
\\	(を)		
\\	頭を冷やしなさい。	頭[あたま]を 冷[ひ]やしなさい。	あたま を ひやしなさい	
\\	頭[あたま]を
\\	ビデオテープ	ビデオテープ	ビデオテープ	
\\	私はビデオテープを30本持っています。	私[わたし]はビデオテープを 30本持[さんじゅっぽん も]っています。	わたし は びでおてーぷ を さんじゅっぽん もって います	
\\	私[わたし]は
\\	を 30本持[さんじゅっぽん も]っています。		
\\	冷める	冷[さ]める	さめる	
\\	彼への気持ちが冷めた。	彼[かれ]への 気持[きも]ちが 冷[さ]めた。	かれ へ の きもち が さめた	
\\	彼[かれ]への 気持[きも]ちが
\\	凍る	凍[こお]る	こおる	
\\	寒い朝は道路が凍ります。	寒[さむ]い 朝[あさ]は 道路[どうろ]が 凍[こお]ります。	さむい あさ は どうろ が こおります	
\\	寒[さむ]い 朝[あさ]は 道路[どうろ]が
\\	冷蔵庫	冷蔵庫[れいぞうこ]	れいぞうこ	
\\	飲み物は冷蔵庫にあります。	飲[の]み 物[もの]は 冷蔵庫[れいぞうこ]にあります。	のみもの は れいぞうこ に あります	
\\	飲[の]み 物[もの]は
\\	にあります。		
\\	涼しい	涼[すず]しい	すずしい	
\\	夕方は涼しくなりますよ。	夕方[ゆうがた]は 涼[すず]しくなりますよ。	ゆうがた は すずしく なります よ	
\\	夕方[ゆうがた]は
\\	なりますよ。		
\\	ぶらぶら	ぶらぶら	ぶらぶら	
\\	朝、近所をぶらぶらした。	朝[あさ]、 近所[きんじょ]をぶらぶらした。	あさ きんじょ を ぶらぶら した	
\\	朝[あさ]、 近所[きんじょ]を
\\	した。		
\\	汚す	汚[よご]す	よごす	
\\	彼は服を汚した。	彼[かれ]は 服[ふく]を 汚[よご]した。	かれ は ふく を よごした	
\\	彼[かれ]は 服[ふく]を
\\	汚れ	汚[よご]れ	よごれ	
\\	靴の汚れを落としました。	靴[くつ]の 汚[よご]れを 落[お]としました。	くつ の よごれ を おとしました	
\\	靴[くつ]の
\\	を 落[お]としました。		
\\	汚れる	汚[よご]れる	よごれる	
\\	エプロンをしないと服が汚れます。	エプロンをしないと 服[ふく]が 汚[よご]れます。	えぷろん を しない と ふく が よごれます	
\\	エプロンをしないと 服[ふく]が
\\	景色	景色[けしき]	けしき	
\\	ここは景色がきれいですね。	ここは 景色[けしき]がきれいですね。	ここ は けしき が きれい です ね	
\\	ここは
\\	がきれいですね。		
\\	アナウンサー	アナウンサー	アナウンサー	
\\	私はアナウンサーになりたい。	私[わたし]はアナウンサーになりたい。	わたし は あなうんさー に なりたい	
\\	私[わたし]は
\\	になりたい。		
\\	影響	影響[えいきょう]	えいきょう	
\\	私は彼から大きな影響を受けました。	私[わたし]は 彼[かれ]から 大[おお]きな 影響[えいきょう]を 受[う]けました。	わたし は かれ から おおき な えいきょう を うけました	
\\	私[わたし]は 彼[かれ]から 大[おお]きな
\\	を 受[う]けました。		
\\	光る	光[ひか]る	ひかる	
\\	波がきらきら光っていました。	波[なみ]がきらきら 光[ひか]っていました。	なみ が きらきら ひかって いました	
\\	波[なみ]がきらきら
\\	太る	太[ふと]る	ふとる	
\\	私の姉はすぐ太ります。	私[わたし]の 姉[あね]はすぐ 太[ふと]ります。	わたし の あね は すぐ ふとります	
\\	私[わたし]の 姉[あね]はすぐ
\\	太陽	太陽[たいよう]	たいよう	
\\	太陽が雲に隠れた。	太陽[たいよう]が 雲[くも]に 隠[かく]れた。	たいよう が くも に かくれた	
\\	が 雲[くも]に 隠[かく]れた。		
\\	スプーン	スプーン	スプーン	
\\	カレーライスはスプーンで食べます。	カレーライスはスプーンで 食[た]べます。	かれーらいす は すぷーん で たべます	
\\	カレーライスは
\\	で 食[た]べます。		
\\	星	星[ほし]	ほし	
\\	今夜は星がよく見えます。	今夜[こんや]は 星[ほし]がよく 見[み]えます。	こんや は ほし が よく みえます	
\\	今夜[こんや]は
\\	がよく 見[み]えます。		
\\	地球	地球[ちきゅう]	ちきゅう	
\\	地球は丸い。	地球[ちきゅう]は 丸[まる]い。	ちきゅう は まるい	
\\	は 丸[まる]い。		
\\	野球	野球[やきゅう]	やきゅう	
\\	友達と野球をしました。	友達[ともだち]と 野球[やきゅう]をしました。	ともだち と やきゅう を しました	
\\	友達[ともだち]と
\\	をしました。		
\\	雲	雲[くも]	くも	
\\	今日は雲が多い。	今日[きょう]は 雲[くも]が 多[おお]い。	きょう は くも が おおい	
\\	今日[きょう]は
\\	が 多[おお]い。		
\\	ブラウス	ブラウス	ブラウス	
\\	彼女は白いブラウスを着ている。	彼女[かのじょ]は 白[しろ]いブラウスを 着[き]ている。	かのじょ は しろい ぶらうす を きて いる	
\\	彼女[かのじょ]は 白[しろ]い
\\	を 着[き]ている。		
\\	曇り	曇[くも]り	くもり	
\\	今日は一日曇りでした。	今日[きょう]は 一日[いちにち] 曇[くも]りでした。	きょう は いちにち くもり でした	
\\	今日[きょう]は 一日[いちにち]
\\	でした。		
\\	地震	地震[じしん]	じしん	
\\	日本は地震が多いです。	日本[にほん]は 地震[じしん]が 多[おお]いです。	にほん は じしん が おおい です	
\\	日本[にほん]は
\\	が 多[おお]いです。		
\\	震える	震[ふる]える	ふるえる	
\\	彼は寒くて震えていました。	彼[かれ]は 寒[さむ]くて 震[ふる]えていました。	かれ は さむくて ふるえて いました	
\\	彼[かれ]は 寒[さむ]くて
\\	振る	振[ふ]る	ふる	
\\	犬がしっぽを振っている。	犬[いぬ]がしっぽを 振[ふ]っている。	いぬ が しっぽ を ふって いる	
\\	犬[いぬ]がしっぽを
\\	リボン	リボン	リボン	
\\	プレゼントにリボンを付けた。	プレゼントにリボンを 付[つ]けた。	ぷれぜんと に りぼん を つけた	
\\	プレゼントに
\\	を 付[つ]けた。		
\\	揺れる	揺[ゆ]れる	ゆれる	
\\	風で木が揺れています。	風[かぜ]で 木[き]が 揺[ゆ]れています。	かぜ で き が ゆれて います	
\\	風[かぜ]で 木[き]が
\\	います。		
\\	神社	神社[じんじゃ]	じんじゃ	
\\	京都には神社がたくさんある。	京都[きょうと]には 神社[じんじゃ]がたくさんある。	きょうと に は じんじゃ が たくさん ある	
\\	京都[きょうと]には
\\	がたくさんある。		
\\	秘密	秘密[ひみつ]	ひみつ	
\\	これは秘密です。	これは 秘密[ひみつ]です。	これ は ひみつ です	
\\	これは
\\	です。		
\\	厳しい	厳[きび]しい	きびしい	
\\	私の上司はとても厳しい。	私[わたし]の 上司[じょうし]はとても 厳[きび]しい。	わたし の じょうし は とても きびしい	
\\	私[わたし]の 上司[じょうし]はとても
\\	いかが	いかが	いかが	
\\	お加減はいかがですか。	お 加減[かげん]はいかがですか。	おかげん は いかが です か	
\\	お 加減[かげん]は
\\	ですか。		
\\	年寄り	年寄[としよ]り	としより	
\\	あの村にはお年寄りがたくさん住んでいます。	あの 村[むら]にはお 年寄[としよ]りがたくさん 住[す]んでいます。	あの むら に は おとしより が たくさん すんで います	
\\	あの 村[むら]にはお
\\	がたくさん 住[す]んでいます。		
\\	歴史	歴史[れきし]	れきし	
\\	私は歴史に興味があります。	私[わたし]は 歴史[れきし]に 興味[きょうみ]があります。	わたし は れきし に きょうみ が あります	
\\	私[わたし]は
\\	に 興味[きょうみ]があります。		
\\	世紀	世紀[せいき]	せいき	
\\	新しい世紀の始まりです。	新[あたら]しい 世紀[せいき]の 始[はじ]まりです。	あたらしい せいき の はじまり です	
\\	新[あたら]しい
\\	の 始[はじ]まりです。		
\\	建設	建設[けんせつ]	けんせつ	
\\	新しいビルの建設が始まった。	新[あたら]しいビルの 建設[けんせつ]が 始[はじ]まった。	あたらしい びる の けんせつ が はじまった	
\\	新[あたら]しいビルの
\\	が 始[はじ]まった。		
\\	ステレオ	ステレオ	ステレオ	
\\	彼はステレオで音楽を聞いた。	彼[かれ]はステレオで 音楽[おんがく]を 聞[き]いた。	かれ は すてれお で おんがく を きいた	
\\	彼[かれ]は
\\	で 音楽[おんがく]を 聞[き]いた。		
\\	建物	建物[たてもの]	たてもの	
\\	これは日本一古い建物です。	これは 日本一古[にほんいち ふる]い 建物[たてもの]です。	これ は にほんいち ふるい たてもの です	
\\	これは 日本一古[にほんいち ふる]い
\\	です。		
\\	建つ	建[た]つ	たつ	
\\	ここに来年、家が建ちます。	ここに 来年[らいねん]、 家[いえ]が 建[た]ちます。	ここ に らいねん いえ が たちます	
\\	ここに 来年[らいねん]、 家[いえ]が
\\	構成	構成[こうせい]	こうせい	
\\	システムの構成を変えてみました。	システムの 構成[こうせい]を 変[か]えてみました。	しすてむ の こうせい を かえて みました	
\\	システムの
\\	を 変[か]えてみました。		
\\	構造	構造[こうぞう]	こうぞう	
\\	この建物の構造は複雑です。	この 建物[たてもの]の 構造[こうぞう]は 複雑[ふくざつ]です。	この たてもの の こうぞう は ふくざつ です	
\\	この 建物[たてもの]の
\\	は 複雑[ふくざつ]です。		
\\	フォーク	フォーク	フォーク	
\\	フォークをもらえますか。	フォークをもらえますか。	ふぉーく を もらえます か	
\\	をもらえますか。		
\\	橋	橋[はし]	はし	
\\	あの橋は日本で一番長い。	あの 橋[はし]は 日本[にほん]で 一番長[いちばん なが]い。	あの はし は にほん で いちばん ながい	
\\	あの
\\	は 日本[にほん]で 一番長[いちばん なが]い。		
\\	柱	柱[はしら]	はしら	
\\	この家の柱は太い。	この 家[いえ]の 柱[はしら]は 太[ふと]い。	この いえ の はしら は ふとい	
\\	この 家[いえ]の
\\	は 太[ふと]い。		
\\	位置	位置[いち]	いち	
\\	私の町は東京の北に位置します。	私[わたし]の 町[まち]は 東京[とうきょう]の 北[きた]に 位置[いち]します。	わたし の まち は とうきょう の きた に いち します	
\\	私[わたし]の 町[まち]は 東京[とうきょう]の 北[きた]に
\\	します。		
\\	離婚	離婚[りこん]	りこん	
\\	友人が離婚しました。	友人[ゆうじん]が 離婚[りこん]しました。	ゆうじん が りこん しました	
\\	友人[ゆうじん]が
\\	しました。		
\\	おば	おば	おば	
\\	おばは大阪に住んでいます。	おばは 大阪[おおさか]に 住[す]んでいます。	おば は おおさか に すんで います	
\\	は 大阪[おおさか]に 住[す]んでいます。		
\\	停車	停車[ていしゃ]	ていしゃ	
\\	この電車は東京まで停車致しません。	この 電車[でんしゃ]は 東京[とうきょう]まで 停車[ていしゃ] 致[いた]しません。	この でんしゃ は とうきょう まで ていしゃ いたしません	
\\	この 電車[でんしゃ]は 東京[とうきょう]まで
\\	致[いた]しません。		
\\	バス停	バス 停[てい]	ばすてい	
\\	次のバス停で降ります。	次[つぎ]のバス 停[てい]で 降[お]ります。	つぎ の ばすてい で おります	
\\	次[つぎ]の
\\	で 降[お]ります。		
\\	周辺	周辺[しゅうへん]	しゅうへん	
\\	この周辺には大学が多い。	この 周辺[しゅうへん]には 大学[だいがく]が 多[おお]い。	この しゅうへん に は だいがく が おおい	
\\	この
\\	には 大学[だいがく]が 多[おお]い。		
\\	隣	隣[となり]	となり	
\\	隣の家には犬がいます。	隣[となり]の 家[いえ]には 犬[いぬ]がいます。	となり の いえ に は いぬ が います	
\\	の 家[いえ]には 犬[いぬ]がいます。		
\\	きらきら	きらきら	きらきら	
\\	星がきらきら光っている。	星[ほし]がきらきら 光[ひか]っている。	ほし が きらきら ひかって いる	
\\	星[ほし]が
\\	光[ひか]っている。		
\\	黄色	黄色[きいろ]	きいろ	
\\	信号は黄色でした。	信号[しんごう]は 黄色[きいろ]でした。	しんごう は きいろ でした	
\\	信号[しんごう]は
\\	でした。		
\\	横	横[よこ]	よこ	
\\	横の長さは1メートルです。	横[よこ]の 長[なが]さは 1[いち]メートルです。	よこ の ながさ は いちめーとる です	
\\	の 長[なが]さは 1[いち]メートルです。		
\\	横書き	横書[よこが]き	よこがき	
\\	この本は横書きです。	この 本[ほん]は 横書[よこが]きです。	この ほん は よこがき です	
\\	この 本[ほん]は
\\	です。		
\\	判断	判断[はんだん]	はんだん	
\\	彼の判断は正しい。	彼[かれ]の 判断[はんだん]は 正[ただ]しい。	かれ の はんだん は ただしい	
\\	彼[かれ]の
\\	は 正[ただ]しい。		
\\	どうぞ	どうぞ	どうぞ	
\\	こちらへどうぞ。	こちらへどうぞ。	こちらへどうぞ。	
\\	こちらへ
\\	断る	断[ことわ]る	ことわる	
\\	私は彼のプロポーズを断った。	私[わたし]は 彼[かれ]のプロポーズを 断[ことわ]った。	わたし は かれ の ぷろぽーず を ことわった	
\\	私[わたし]は 彼[かれ]のプロポーズを
\\	横断歩道	横断歩道[おうだんほどう]	おうだんほどう	
\\	あそこに横断歩道があります。	あそこに 横断歩道[おうだんほどう]があります。	あそこ に おうだんほどう が あります	
\\	あそこに
\\	があります。		
\\	大幅	大幅[おおはば]	おおはば	
\\	計画を大幅に変更した。	計画[けいかく]を 大幅[おおはば]に 変更[へんこう]した。	けいかく を おおはば に へんこう した	
\\	計画[けいかく]を
\\	に 変更[へんこう]した。		
\\	訪れる	訪[おとず]れる	おとずれる	
\\	私は夏に京都を訪れました。	私[わたし]は 夏[なつ]に 京都[きょうと]を 訪[おとず]れました。	わたし は なつ に きょうと を おとずれました	
\\	私[わたし]は 夏[なつ]に 京都[きょうと]を
\\	ネックレス	ネックレス	ネックレス	
\\	妻にネックレスをプレゼントしました。	妻[つま]にネックレスをプレゼントしました。	つま に ねっくれす を ぷれぜんと しました	
\\	妻[つま]に
\\	をプレゼントしました。		
\\	訪ねる	訪[たず]ねる	たずねる	
\\	彼はニューヨークの友達を訪ねた。	彼[かれ]はニューヨークの 友達[ともだち]を 訪[たず]ねた。	かれ は にゅーよーく の ともだち を たずねた	
\\	彼[かれ]はニューヨークの 友達[ともだち]を
\\	冷房	冷房[れいぼう]	れいぼう	
\\	冷房を入れてください。	冷房[れいぼう]を 入[い]れてください。	れいぼう を いれて ください	
\\	を 入[い]れてください。		
\\	暖房	暖房[だんぼう]	だんぼう	
\\	冬は暖房が必要です。	冬[ふゆ]は 暖房[だんぼう]が 必要[ひつよう]です。	ふゆ は だんぼう が ひつよう です	
\\	冬[ふゆ]は
\\	が 必要[ひつよう]です。		
\\	文房具	文房具[ぶんぼうぐ]	ぶんぼうぐ	
\\	新しい文房具を買いました。	新[あたら]しい 文房具[ぶんぼうぐ]を 買[か]いました。	あたらしい ぶんぼうぐ を かいました	
\\	新[あたら]しい
\\	を 買[か]いました。		
\\	パジャマ	パジャマ	パジャマ	
\\	このパジャマを着て。	このパジャマを 着[き]て。	この ぱじゃま を きて	
\\	この
\\	を 着[き]て。		
\\	諸国	諸国[しょこく]	しょこく	
\\	彼はヨーロッパ諸国を旅行した。	彼[かれ]はヨーロッパ 諸国[しょこく]を 旅行[りょこう]した。	かれ は よーろっぱ しょこく を りょこう した	
\\	彼[かれ]はヨーロッパ
\\	を 旅行[りょこう]した。		
\\	緑色	緑色[みどりいろ]	みどりいろ	
\\	彼は緑色のジャケットを着ています。	彼[かれ]は 緑色[みどりいろ]のジャケットを 着[き]ています。	かれ は みどりいろ の じゃけっと を きて います	
\\	彼[かれ]は
\\	のジャケットを 着[き]ています。		
\\	貿易	貿易[ぼうえき]	ぼうえき	
\\	父は貿易の仕事をしています。	父[ちち]は 貿易[ぼうえき]の 仕事[しごと]をしています。	ちち は ぼうえき の しごと を して います	
\\	父[ちち]は
\\	の 仕事[しごと]をしています。		
\\	輸入	輸入[ゆにゅう]	ゆにゅう	
\\	これはイタリアから輸入した服です。	これはイタリアから 輸入[ゆにゅう]した 服[ふく]です。	これ は いたりあ から ゆにゅう した ふく です	
\\	これはイタリアから
\\	した 服[ふく]です。		
\\	ボールペン	ボールペン	ボールペン	
\\	ボールペンで名前を書いてください。	ボールペンで 名前[なまえ]を 書[か]いてください。	ぼーるぺん で なまえ を かいて ください	
\\	で 名前[なまえ]を 書[か]いてください。		
\\	輸出	輸出[ゆしゅつ]	ゆしゅつ	
\\	彼の会社は車を輸出しています。	彼[かれ]の 会社[かいしゃ]は 車[くるま]を 輸出[ゆしゅつ]しています。	かれ の かいしゃ は くるま を ゆしゅつ して います	
\\	彼[かれ]の 会社[かいしゃ]は 車[くるま]を
\\	しています。		
\\	指輪	指輪[ゆびわ]	ゆびわ	
\\	彼女に指輪をプレゼントしました。	彼女[かのじょ]に 指輪[ゆびわ]をプレゼントしました。	かのじょ に ゆびわ を ぷれぜんと しました	
\\	彼女[かのじょ]に
\\	をプレゼントしました。		
\\	往復	往復[おうふく]	おうふく	
\\	往復切符をください。	往復[おうふく] 切符[きっぷ]をください。	おうふく きっぷ を ください	
\\	切符[きっぷ]をください。		
\\	復習	復習[ふくしゅう]	ふくしゅう	
\\	昨日の復習をしましたか。	昨日[きのう]の 復習[ふくしゅう]をしましたか。	きのう の ふくしゅう を しました か	
\\	昨日[きのう]の
\\	をしましたか。		
\\	すいか	すいか	すいか	
\\	夏はすいかが美味しい。	夏[なつ]はすいかが 美味[おい]しい。	なつ は すいか が おいしい	
\\	夏[なつ]は
\\	が 美味[おい]しい。		
\\	繰り返す	繰[く]り 返[かえ]す	くりかえす	
\\	を		
\\	彼女は同じ間違いを繰り返した。	彼女[かのじょ]は 同[おな]じ 間違[まちが]いを 繰[く]り 返[かえ]した。	かのじょ は おなじ まちがい を くりかえした	
\\	彼女[かのじょ]は 同[おな]じ 間違[まちが]いを
\\	留学	留学[りゅうがく]	りゅうがく	
\\	1年間、アメリカに留学しました。	1年間[いちねんかん]、アメリカに 留学[りゅうがく]しました。	いちねんかん あめりか に りゅうがく しました	
\\	1年間[いちねんかん]、アメリカに
\\	しました。		
\\	停留所	停留所[ていりゅうじょ]	ていりゅうじょ	
\\	バスの停留所で10分待ちました。	バスの 停留所[ていりゅうじょ]で 10分[じゅっぷん]待[ま]ちました。	ばす の ていりゅうじょ で じゅっぷん まちました	
\\	バスの
\\	で 10分[じゅっぷん]待[ま]ちました。		
\\	書留	書留[かきとめ]	かきとめ	
\\	これを書留で送りたいのですが。	これを 書留[かきとめ]で 送[おく]りたいのですが。	これ を かきとめ で おくりたい の です が	
\\	これを
\\	で 送[おく]りたいのですが。		
\\	そちら	そちら	そちら	
\\	夕方そちらに着きます。	夕方[ゆうがた]そちらに 着[つ]きます。	ゆうがた そちら に つきます	
\\	夕方[ゆうがた]
\\	に 着[つ]きます。		
\\	守る	守[まも]る	まもる	
\\	彼は約束を守る人です。	彼[かれ]は 約束[やくそく]を 守[まも]る 人[ひと]です。	かれ は やくそく を まもる ひと です	
\\	彼[かれ]は 約束[やくそく]を
\\	人[ひと]です。		
\\	留守	留守[るす]	るす	
\\	父は今、留守です。	父[ちち]は 今[いま]、 留守[るす]です。	ちち は いま るす です	
\\	父[ちち]は 今[いま]、
\\	です。		
\\	住宅	住宅[じゅうたく]	じゅうたく	
\\	ここは静かな住宅地だ。	ここは 静[しず]かな 住宅[じゅうたく] 地[ち]だ。	ここ は しずかな じゅうたくち だ	
\\	ここは 静[しず]かな
\\	地[ち]だ。		
\\	自宅	自宅[じたく]	じたく	
\\	自宅に電話を下さい。	自宅[じたく]に 電話[でんわ]を 下[くだ]さい。	じたく に でんわ を ください	
\\	に 電話[でんわ]を 下[くだ]さい。		
\\	そば	そば	そば	
\\	お昼にそばを食べた。	お 昼[ひる]にそばを 食[た]べた。	おひる に そば を たべた	
\\	お 昼[ひる]に
\\	を 食[た]べた。		
\\	お宅	お 宅[たく]	おたく	
\\	(敬語)		
\\	先生のお宅はどちらですか。	先生[せんせい]のお 宅[たく]はどちらですか。	せんせい の おたく は どちら です か	
\\	先生[せんせい]の
\\	はどちらですか。		
\\	早起き	早起[はやお]き	はやおき	
\\	祖父は早起きです。	祖父[そふ]は 早起[はやお]きです。	そふ は はやおき です	
\\	祖父[そふ]は
\\	です。		
\\	昼寝	昼寝[ひるね]	ひるね	
\\	私の子供は毎日昼寝をします。	私[わたし]の 子供[こども]は 毎日[まいにち] 昼寝[ひるね]をします。	わたし の こども は まいにち ひるね を します	
\\	私[わたし]の 子供[こども]は 毎日[まいにち]
\\	をします。		
\\	静か	静[しず]か	しずか	
\\	今年の夏休みには静かな所へ行きたい。	今年[ことし]の 夏休[なつやす]みには 静[しず]かな 所[ところ]へ 行[い]きたい。	ことし の なつやすみ に は しずか な ところ へ いきたい	
\\	今年[ことし]の 夏休[なつやす]みには
\\	な 所[ところ]へ 行[い]きたい。		
\\	どっち	どっち	どっち	
\\	どっちの色が好きですか。	どっちの 色[いろ]が 好[す]きですか。	どっち の いろ が すき です か 。	
\\	の 色[いろ]が 好[す]きですか。		
\\	暇	暇[ひま]	ひま	
\\	明日は暇ですか。	明日[あした]は 暇[ひま]ですか。	あした は ひま です か	
\\	明日[あした]は
\\	ですか。		
\\	趣味	趣味[しゅみ]	しゅみ	
\\	私の趣味は映画とテニスです。	私[わたし]の 趣味[しゅみ]は 映画[えいが]とテニスです。	わたし の しゅみ は えいが と てにす です	
\\	私[わたし]の
\\	は 映画[えいが]とテニスです。		
\\	両方	両方[りょうほう]	りょうほう	
\\	チョコレートとケーキを両方ください。	チョコレートとケーキを 両方[りょうほう]ください。	ちょこれーと と けーき を りょうほう ください	
\\	チョコレートとケーキを
\\	ください。		
\\	両替	両替[りょうがえ]	りょうがえ	
\\	1万 
\\	1万円を両替してください。	1万円[いちまんえん]を 両替[りょうがえ]してください。	いちまんえん を りょうがえ して ください	
\\	1万円[いちまんえん]を
\\	してください。		
\\	バスケットボール	バスケットボール	バスケットボール	
\\	姉はバスケットボールの選手です。	姉[あね]はバスケットボールの 選手[せんしゅ]です。	あね は ばすけっとぼーる の せんしゅ です	
\\	姉[あね]は
\\	の 選手[せんしゅ]です。		
\\	両親	両親[りょうしん]	りょうしん	
\\	私の両親は大阪に住んでいます。	私[わたし]の 両親[りょうしん]は 大阪[おおさか]に 住[す]んでいます。	わたし の りょうしん は おおさか に すんで います	
\\	私[わたし]の
\\	は 大阪[おおさか]に 住[す]んでいます。		
\\	片道	片道[かたみち]	かたみち	
\\	東京まで片道切符を買った。	東京[とうきょう]まで 片道[かたみち] 切符[きっぷ]を 買[か]った。	とうきょう まで かたみち きっぷ を かった	
\\	東京[とうきょう]まで
\\	切符[きっぷ]を 買[か]った。		
\\	内側	内側[うちがわ]	うちがわ	
\\	白線の内側に下がってください。	白線[はくせん]の 内側[うちがわ]に 下[さ]がってください。	はくせん の うちがわ に さがって ください	
\\	白線[はくせん]の
\\	に 下[さ]がってください。		
\\	向こう側	向[む]こう 側[がわ]	むこうがわ	
\\	私の家は川の向こう側にあります。	私[わたし]の 家[いえ]は 川[かわ]の 向[む]こう 側[がわ]にあります。	わたし の いえ は かわ の むこうがわ に あります	
\\	私[わたし]の 家[いえ]は 川[かわ]の
\\	にあります。		
\\	サンドイッチ	サンドイッチ	サンドイッチ	
\\	今日のお昼はサンドイッチです。	今日[きょう]のお 昼[ひる]はサンドイッチです。	きょう の おひる は さんどいっち です	
\\	今日[きょう]のお 昼[ひる]は
\\	です。		
\\	外側	外側[そとがわ]	そとがわ	
\\	白線の外側を歩かないでください。	白線[はくせん]の 外側[そとがわ]を 歩[ある]かないでください。	はくせん の そとがわ を あるかない で ください	
\\	白線[はくせん]の
\\	を 歩[ある]かないでください。		
\\	左側	左側[ひだりがわ]	ひだりがわ	
\\	画面の左側を見てください。	画面[がめん]の 左側[ひだりがわ]を 見[み]てください。	がめん の ひだりがわ を みて ください	
\\	画面[がめん]の
\\	を 見[み]てください。		
\\	右側	右側[みぎがわ]	みぎがわ	
\\	彼女はいつも私の右側を歩きます。	彼女[かのじょ]はいつも 私[わたし]の 右側[みぎがわ]を 歩[ある]きます。	かのじょ は いつも わたし の みぎがわ を あるきます	
\\	彼女[かのじょ]はいつも 私[わたし]の
\\	を 歩[ある]きます。		
\\	裏	裏[うら]	うら	
\\	会社の裏に公園があります。	会社[かいしゃ]の 裏[うら]に 公園[こうえん]があります。	かいしゃ の うら に こうえん が あります	
\\	会社[かいしゃ]の
\\	に 公園[こうえん]があります。		
\\	にこにこ	にこにこ	にこにこ	
\\	彼女はいつもにこにこしています。	彼女[かのじょ]はいつもにこにこしています。	かのじょ は いつも にこにこ して います	
\\	彼女[かのじょ]はいつも
\\	しています。		
\\	裏返す	裏返[うらがえ]す	うらがえす	
\\	彼は紙を裏返しました。	彼[かれ]は 紙[かみ]を 裏返[うらがえ]しました。	かれ は かみ を うらがえしました	
\\	彼[かれ]は 紙[かみ]を
\\	週刊誌	週刊誌[しゅうかんし]	しゅうかんし	
\\	その週刊誌はゴシップばかりだ。	その 週刊誌[しゅうかんし]はゴシップばかりだ。	その しゅうかんし は ごしっぷ ばかり だ	
\\	その
\\	はゴシップばかりだ。		
\\	朝刊	朝刊[ちょうかん]	ちょうかん	
\\	今日の朝刊に面白い記事があった。	今日[きょう]の 朝刊[ちょうかん]に 面白[おもしろ]い 記事[きじ]があった。	きょう の ちょうかん に おもしろい きじ が あった	
\\	今日[きょう]の
\\	に 面白[おもしろ]い 記事[きじ]があった。		
\\	夕刊	夕刊[ゆうかん]	ゆうかん	
\\	そのニュースは夕刊で見ました。	そのニュースは 夕刊[ゆうかん]で 見[み]ました。	その にゅーす は ゆうかん で みました	
\\	そのニュースは
\\	で 見[み]ました。		
\\	ウール	ウール	ウール	
\\	このウールのセーターは暖かい。	このウールのセーターは 暖[あたた]かい。	この うーる の せーたー は あたたかい	
\\	この
\\	のセーターは 暖[あたた]かい。		
\\	詳しい	詳[くわ]しい	くわしい	
\\	もっと詳しく説明してください。	もっと 詳[くわ]しく 説明[せつめい]してください。	もっと くわしく せつめい して ください	
\\	もっと
\\	説明[せつめい]してください。		
\\	細かい	細[こま]かい	こまかい	
\\	彼女は細かいことにうるさい。	彼女[かのじょ]は 細[こま]かいことにうるさい。	かのじょ は こまかい こと に うるさい	
\\	彼女[かのじょ]は
\\	ことにうるさい。		
\\	細か	細[こま]か	こまか	
\\	それは細かな問題です。	それは 細[こま]かな 問題[もんだい]です。	それ は こまか な もんだい です	
\\	それは
\\	な 問題[もんだい]です。		
\\	積もる	積[つ]もる	つもる	
\\	など)		
\\	雪が積もっています。	雪[ゆき]が 積[つ]もっています。	ゆき が つもって います	
\\	雪[ゆき]が
\\	コーラ	コーラ	コーラ	
\\	私の兄はコーラが好きだ。	私[わたし]の 兄[あに]はコーラが 好[す]きだ。	わたし の あに は こーら が すき だ	
\\	私[わたし]の 兄[あに]は
\\	が 好[す]きだ。		
\\	訳	訳[やく]	やく	
\\	その文の訳を読みました。	その 文[ぶん]の 訳[やく]を 読[よ]みました。	その ぶん の やく を よみました	
\\	その 文[ぶん]の
\\	を 読[よ]みました。		
\\	訳す	訳[やく]す	やくす	
\\	この文を訳してください。	この 文[ぶん]を 訳[やく]してください。	この ぶん を やくして ください	
\\	この 文[ぶん]を
\\	ください。		
\\	検討	検討[けんとう]	けんとう	
\\	今日中にこの問題を検討してください。	今日中[きょう じゅう]にこの 問題[もんだい]を 検討[けんとう]してください。	きょう じゅう に この もんだい を けんとう して ください	
\\	今日中[きょう じゅう]にこの 問題[もんだい]を
\\	してください。		
\\	塗る	塗[ぬ]る	ぬる	
\\	壁にペンキを塗っています。	壁[かべ]にペンキを 塗[ぬ]っています。	かべ に ぺんき を ぬって います	
\\	壁[かべ]にペンキを
\\	トイレットペーパー	トイレットペーパー	トイレットペーパー	
\\	トイレットペーパーを交換しました。	トイレットペーパーを 交換[こうかん]しました。	といれっとぺーぱー を こうかん しました	
\\	を 交換[こうかん]しました。		
\\	付く	付[つ]く	つく	
\\	靴に泥が付いています。	靴[くつ]に 泥[どろ]が 付[つ]いています。	くつ に どろ が ついて います	
\\	靴[くつ]に 泥[どろ]が
\\	付ける	付[つ]ける	つける	
\\	おまけを付けました。	おまけを 付[つ]けました。	おまけ を つけました	
\\	おまけを
\\	受け付ける	受[う]け 付[つ]ける	うけつける	
\\	郵便物は5時まで受け付けています。	郵便物[ゆうびんぶつ]は 5時[ごじ]まで 受[う]け 付[つ]けています。	ゆうびんぶつ は ごじ まで うけつけて います	
\\	郵便物[ゆうびんぶつ]は 5時[ごじ]まで
\\	片付ける	片付[かたづ]ける	かたづける	
\\	早く部屋を片付けなさい。	早[はや]く 部屋[へや]を 片付[かたづ]けなさい。	はやく へや を かたづけなさい	
\\	早[はや]く 部屋[へや]を
\\	どなた	どなた	どなた	
\\	失礼ですが、どなたですか。	失礼[しつれい]ですが、どなたですか。	しつれい です が 、 どなた です か 。	
\\	失礼[しつれい]ですが、
\\	ですか。		
\\	受付	受付[うけつけ]	うけつけ	
\\	受付は9時からです。	受付[うけつけ]は 9時[くじ]からです。	うけつけ は くじ から です	
\\	は 9時[くじ]からです。		
\\	近付く	近付[ちかづ]く	ちかづく	
\\	女性が近付いてきました。	女性[じょせい]が 近付[ちかづ]いてきました。	じょせい が ちかづいて きました	
\\	女性[じょせい]が
\\	きました。		
\\	片付く	片付[かたづ]く	かたづく	
\\	仕事が大分片付いた。	仕事[しごと]が 大分[だいぶ] 片付[かたづ]いた。	しごと が だいぶ かたづいた	
\\	仕事[しごと]が 大分[だいぶ]
\\	気を付ける	気[き]を 付[つ]ける	きをつける	
\\	体に気を付けてください。	体[からだ]に 気[き]を 付[つ]けてください。	からだ に き を つけて ください	
\\	体[からだ]に
\\	ください。		
\\	ハイキング	ハイキング	ハイキング	
\\	昨日は友人とハイキングに行きました。	昨日[きのう]は 友人[ゆうじん]とハイキングに 行[い]きました。	きのう は ゆうじん と はいきんぐ に いきました	
\\	昨日[きのう]は 友人[ゆうじん]と
\\	に 行[い]きました。		
\\	貼る	貼[は]る	はる	
\\	机にシールを貼った。	机[つくえ]にシールを 貼[は]った。	つくえ に しーる を はった	
\\	机[つくえ]にシールを
\\	記念日	記念日[きねんび]	きねんび	
\\	今日は両親の結婚記念日です。	今日[きょう]は 両親[りょうしん]の 結婚[けっこん] 記念日[きねんび]です。	きょう は りょうしん の けっこん きねんび です	
\\	今日[きょう]は 両親[りょうしん]の 結婚[けっこん]
\\	です。		
\\	残念	残念[ざんねん]	ざんねん	
\\	その試合は残念な結果になった。	その 試合[しあい]は 残念[ざんねん]な 結果[けっか]になった。	その しあい は ざんねん な けっか に なった	
\\	その 試合[しあい]は
\\	な 結果[けっか]になった。		
\\	例えば	例[たと]えば	たとえば	
\\	例えば、このソフトで日本語を勉強することができます。	例[たと]えば、このソフトで 日本語[にほんご]を 勉強[べんきょう]することができます。	たとえば この そふと で にほんご を べんきょう する こと が できます	
\\	、このソフトで 日本語[にほんご]を 勉強[べんきょう]することができます。		
\\	ハム	ハム	ハム	
\\	ハムサンドをください。	ハムサンドをください。	はむ さんど を ください	
\\	サンドをください。		
\\	例文	例文[れいぶん]	れいぶん	
\\	例文を3つ作ってください。	例文[れいぶん]を 3[みっ]つ 作[つく]ってください。	れいぶん を みっつ つくって ください	
\\	を 3[みっ]つ 作[つく]ってください。		
\\	例	例[れい]	れい	
\\	一つ例をあげてください。	一[ひと]つ 例[れい]をあげてください。	ひとつ れい を あげて ください	
\\	一[ひと]つ
\\	をあげてください。		
\\	余る	余[あま]る	あまる	
\\	夕食の料理がたくさん余った。	夕食[ゆうしょく]の 料理[りょうり]がたくさん 余[あま]った。	ゆうしょく の りょうり が たくさん あまった	
\\	夕食[ゆうしょく]の 料理[りょうり]がたくさん
\\	除く	除[のぞ]く	のぞく	
\\	部長を除く全員が土曜日も働いた。	部長[ぶちょう]を 除[のぞ]く 全員[ぜんいん]が 土曜日[どようび]も 働[はたら]いた。	ぶちょう を のぞく ぜんいん が どようび も はたらいた	
\\	部長[ぶちょう]を
\\	全員[ぜんいん]が 土曜日[どようび]も 働[はたら]いた。		
\\	ボーイフレンド	ボーイフレンド	ボーイフレンド	
\\	私のボーイフレンドはハンサムな方でした。	私[わたし]のボーイフレンドはハンサムな 方[かた]でした。	わたし の ぼーいふれんど は はんさむな かた でした	
\\	私[わたし]の
\\	はハンサムな 方[かた]でした。		
\\	削る	削[けず]る	けずる	
\\	彼は家族のために仕事の時間を削った。	彼[かれ]は 家族[かぞく]のために 仕事[しごと]の 時間[じかん]を 削[けず]った。	かれ は かぞく の ため に しごと の じかん を けずった	
\\	彼[かれ]は 家族[かぞく]のために 仕事[しごと]の 時間[じかん]を
\\	遅刻	遅刻[ちこく]	ちこく	
\\	遅刻しないでください。	遅刻[ちこく]しないでください。	ちこく しない で ください	
\\	しないでください。		
\\	緩い	緩[ゆる]い	ゆるい	
\\	このズボンは緩いです。	このズボンは 緩[ゆる]いです。	この ずぼん は ゆるい です	
\\	このズボンは
\\	です。		
\\	苦しい	苦[くる]しい	くるしい	
\\	食べ過ぎておなかが苦しい。	食[た]べ 過[す]ぎておなかが 苦[くる]しい。	たべすぎて おなか が くるしい	
\\	食[た]べ 過[す]ぎておなかが
\\	アイロン	アイロン	アイロン	
\\	彼女はシャツにアイロンをかけた。	彼女[かのじょ]はシャツにアイロンをかけた。	かのじょ は しゃつ に あいろん を かけた	
\\	彼女[かのじょ]はシャツに
\\	をかけた。		
\\	苦い	苦[にが]い	にがい	
\\	私は苦いコーヒーが好きです。	私[わたし]は 苦[にが]いコーヒーが 好[す]きです。	わたし は にがい こーひー が すき です	
\\	私[わたし]は
\\	コーヒーが 好[す]きです。		
\\	苦手	苦手[にがて]	にがて	
\\	私は料理が苦手です。	私[わたし]は 料理[りょうり]が 苦手[にがて]です。	わたし は りょうり が にがて です	
\\	私[わたし]は 料理[りょうり]が
\\	です。		
\\	困る	困[こま]る	こまる	
\\	ケータイをなくして困っています。	ケータイをなくして 困[こま]っています。	けーたい を なくして こまって います	
\\	ケータイをなくして
\\	貧乏	貧乏[びんぼう]	びんぼう	
\\	彼は昔は貧乏だった。	彼[かれ]は 昔[むかし]は 貧乏[びんぼう]だった。	かれ は むかし は びんぼう だった	
\\	彼[かれ]は 昔[むかし]は
\\	だった。		
\\	カレーライス	カレーライス	カレーライス	
\\	子供はカレーライスが好きです。	子供[こども]はカレーライスが 好[す]きです。	こども は かれーらいす が すき です	
\\	子供[こども]は
\\	が 好[す]きです。		
\\	不幸	不幸[ふこう]	ふこう	
\\	彼女の家族に不幸があった。	彼女[かのじょ]の 家族[かぞく]に 不幸[ふこう]があった。	かのじょ の かぞく に ふこう が あった	
\\	彼女[かのじょ]の 家族[かぞく]に
\\	があった。		
\\	幸せ	幸[しあわ]せ	しあわせ	
\\	良い友達がいて私は幸せだ。	良[い]い 友達[ともだち]がいて 私[わたし]は 幸[しあわ]せだ。	いい ともだち が いて わたし は しあわせ だ	
\\	良[い]い 友達[ともだち]がいて 私[わたし]は
\\	だ。		
\\	塩	塩[しお]	しお	
\\	もうちょっと塩を入れて。	もうちょっと 塩[しお]を 入[い]れて。	もう ちょっと しお を いれて	
\\	もうちょっと
\\	を 入[い]れて。		
\\	塩辛い	塩辛[しおから]い	しおからい	
\\	海の水は塩辛い。	海[うみ]の 水[みず]は 塩辛[しおから]い。	うみ の みず は しおからい	
\\	海[うみ]の 水[みず]は
\\	キャッシュカード	キャッシュカード	キャッシュカード	
\\	キャッシュカードでお金を下ろした。	キャッシュカードでお 金[かね]を 下[お]ろした。	きゃっしゅかーど で おかね を おろした	
\\	でお 金[かね]を 下[お]ろした。		
\\	砂糖	砂糖[さとう]	さとう	
\\	コーヒーに砂糖は入れますか。	コーヒーに 砂糖[さとう]は 入[い]れますか。	こーひー に さとう は いれます か	
\\	コーヒーに
\\	は 入[い]れますか。		
\\	規模	規模[きぼ]	きぼ	
\\	など), 
\\	この動物園は日本一の規模です。	この 動物園[どうぶつえん]は 日本一[にっぽんいち]の 規模[きぼ]です。	この どうぶつえん は にっぽんいち の きぼ です	
\\	この 動物園[どうぶつえん]は 日本一[にっぽんいち]の
\\	です。		
\\	農業	農業[のうぎょう]	のうぎょう	
\\	私は農業を勉強しています。	私[わたし]は 農業[のうぎょう]を 勉強[べんきょう]しています。	わたし は のうぎょう を べんきょう して います	
\\	私[わたし]は
\\	を 勉強[べんきょう]しています。		
\\	濃い	濃[こ]い	こい	
\\	私は濃い味が好きだ。	私[わたし]は 濃[こ]い 味[あじ]が 好[す]きだ。	わたし は こい あじ が すき だ	
\\	私[わたし]は
\\	味[あじ]が 好[す]きだ。		
\\	イヤリング	イヤリング	イヤリング	
\\	彼女はすてきなイヤリングをしている。	彼女[かのじょ]はすてきなイヤリングをしている。	かのじょ は すてき な いやりんぐ を して いる	
\\	彼女[かのじょ]はすてきな
\\	をしている。		
\\	薄い	薄[うす]い	うすい	
\\	この電子辞書はとても薄い。	この 電子辞書[でんし じしょ]はとても 薄[うす]い。	この でんし じしょ は とても うすい	
\\	この 電子辞書[でんし じしょ]はとても
\\	厚い	厚[あつ]い	あつい	
\\	その辞書はとても厚い。	その 辞書[じしょ]はとても 厚[あつ]い。	その じしょ は とても あつい	
\\	その 辞書[じしょ]はとても
\\	厚さ	厚[あつ]さ	あつさ	
\\	私は板の厚さを測った。	私[わたし]は 板[いた]の 厚[あつ]さを 測[はか]った。	わたし は いた の あつさ を はかった	
\\	私[わたし]は 板[いた]の
\\	を 測[はか]った。		
\\	迫る	迫[せま]る	せまる	
\\	締め切りが迫っています。	締[し]め 切[き]りが 迫[せま]っています。	しめきり が せまって います	
\\	締[し]め 切[き]りが
\\	ガールフレンド	ガールフレンド	ガールフレンド	
\\	昔は彼女のことをガールフレンドと言っていました。	昔[むかし]は 彼女[かのじょ]のことをガールフレンドと 言[い]っていました。	むかし は かのじょ の こと を がーるふれんど と いって いました	
\\	昔[むかし]は 彼女[かのじょ]のことを
\\	と 言[い]っていました。		
\\	伸びる	伸[の]びる	のびる	
\\	髪がだいぶ伸びたね。	髪[かみ]がだいぶ 伸[の]びたね。	かみ が だいぶ のびた ね	
\\	髪[かみ]がだいぶ
\\	ね。		
\\	引っ越す	引[ひ]っ 越[こ]す	ひっこす	
\\	来月、大阪に引っ越します。	来月[らいげつ]、 大阪[おおさか]に 引[ひ]っ 越[こ]します。	らいげつ おおさか に ひっこします	
\\	来月[らいげつ]、 大阪[おおさか]に
\\	越える	越[こ]える	こえる	
\\	私たちは高い山を越えました。	私[わたし]たちは 高[たか]い 山[やま]を 越[こ]えました。	わたしたち は たかい やま を こえました	
\\	私[わたし]たちは 高[たか]い 山[やま]を
\\	引っ越し	引[ひ]っ 越[こ]し	ひっこし	
\\	去年、引っ越ししました。	去年[きょねん]、 引[ひ]っ 越[こ]ししました。	きょねん ひっこし しました	
\\	去年[きょねん]、
\\	しました。		
\\	カセットテープ	カセットテープ	カセットテープ	
\\	昔はカセットテープを使っていました。	昔[むかし]はカセットテープを 使[つか]っていました。	むかし は かせっとてーぷ を つかっていました 。	
\\	昔[むかし]は
\\	を 使[つか]っていました。		
\\	追い越す	追[お]い 越[こ]す	おいこす	
\\	大きなトラックが私たちを追い越した。	大[おお]きなトラックが 私[わたし]たちを 追[お]い 越[こ]した。	おおき な とらっく が わたしたち を おいこした	
\\	大[おお]きなトラックが 私[わたし]たちを
\\	上昇	上昇[じょうしょう]	じょうしょう	
\\	地球の気温は上昇している。	地球[ちきゅう]の 気温[きおん]は 上昇[じょうしょう]している。	ちきゅう の きおん は じょうしょう して いる	
\\	地球[ちきゅう]の 気温[きおん]は
\\	している。		
\\	改札口	改札口[かいさつぐち]	かいさつぐち	
\\	改札口で会いましょう。	改札口[かいさつぐち]で 会[あ]いましょう。	かいさつぐち で あいましょう	
\\	で 会[あ]いましょう。		
\\	失礼	失礼[しつれい]	しつれい	
\\	ではそろそろ失礼します。	ではそろそろ 失礼[しつれい]します。	では そろそろ しつれい します	
\\	ではそろそろ
\\	します。		
\\	かゆい	かゆい	かゆい	
\\	背中がかゆいです。	背中[せなか]がかゆいです。	せなか が かゆい です	
\\	背中[せなか]が
\\	です。		
\\	お礼	お 礼[れい]	おれい	
\\	彼女にお礼の手紙を書きました。	彼女[かのじょ]にお 礼[れい]の 手紙[てがみ]を 書[か]きました。	かのじょ に おれい の てがみ を かきました	
\\	彼女[かのじょ]に
\\	の 手紙[てがみ]を 書[か]きました。		
\\	謝る	謝[あやま]る	あやまる	
\\	彼は直ぐに謝りました。	彼[かれ]は 直[す]ぐに 謝[あやま]りました。	かれ は すぐ に あやまりました	
\\	彼[かれ]は 直[す]ぐに
\\	注射	注射[ちゅうしゃ]	ちゅうしゃ	
\\	彼は注射があまり好きではありません。	彼[かれ]は 注射[ちゅうしゃ]があまり 好[す]きではありません。	かれ は ちゅうしゃ が あまり すき で は ありません	
\\	彼[かれ]は
\\	があまり 好[す]きではありません。		
\\	程度	程度[ていど]	ていど	
\\	この程度の怪我なら大丈夫です。	この 程度[ていど]の 怪我[けが]なら 大丈夫[だいじょうぶ]です。	この ていど の けが なら だいじょうぶ です	
\\	この
\\	の 怪我[けが]なら 大丈夫[だいじょうぶ]です。		
\\	ぐらぐら	ぐらぐら	ぐらぐら	
\\	地震で家がぐらぐらと揺れた。	地震[じしん]で 家[いえ]がぐらぐらと 揺[ゆ]れた。	じしん で いえ が ぐらぐら と ゆれた	
\\	地震[じしん]で 家[いえ]が
\\	と 揺[ゆ]れた。		
\\	誘う	誘[さそ]う	さそう	
\\	彼女をデートに誘った。	彼女[かのじょ]をデートに 誘[さそ]った。	かのじょ を でーと に さそった	
\\	彼女[かのじょ]をデートに
\\	導入	導入[どうにゅう]	どうにゅう	
\\	会社で新しいシステムを導入した。	会社[かいしゃ]で 新[あたら]しいシステムを 導入[どうにゅう]した。	かいしゃ で あたらしい しすてむ を どうにゅう した	
\\	会社[かいしゃ]で 新[あたら]しいシステムを
\\	した。		
\\	努力	努力[どりょく]	どりょく	
\\	もっと努力しよう。	もっと 努力[どりょく]しよう。	もっと どりょく しよう	
\\	もっと
\\	しよう。		
\\	怒る	怒[おこ]る	おこる	
\\	彼女が嘘をついたので、彼は怒った。	彼女[かのじょ]が 嘘[うそ]をついたので、 彼[かれ]は 怒[おこ]った。	かのじょ が うそ を ついた の で かれ は おこった	
\\	彼女[かのじょ]が 嘘[うそ]をついたので、 彼[かれ]は
\\	ジャム	ジャム	ジャム	
\\	このイチゴでジャムを作りましょう。	このイチゴでジャムを 作[つく]りましょう。	この いちご で じゃむ を つくりましょう	
\\	このイチゴで
\\	を 作[つく]りましょう。		
\\	独身	独身[どくしん]	どくしん	
\\	私の兄はまだ独身です。	私[わたし]の 兄[あに]はまだ 独身[どくしん]です。	わたし の あに は まだ どくしん です	
\\	私[わたし]の 兄[あに]はまだ
\\	です。		
\\	占める	占[し]める	しめる	
\\	私に届くメールのうち、迷惑メールが7割を占めている。	私[わたし]に 届[とど]くメールのうち、 迷惑[めいわく]メールが 7割[ななわり]を 占[し]めている。	わたし に とどく めーる の うち めいわく めーる が ななわり を しめて いる	
\\	私[わたし]に 届[とど]くメールのうち、 迷惑[めいわく]メールが 7割[ななわり]を
\\	処理	処理[しょり]	しょり	
\\	加工, 
\\	事務的な処理に1週間かかります。	事務的[じむてき]な 処理[しょり]に 1週間[いっしゅうかん]かかります。	じむてき な しょり に いっしゅうかん かかります	
\\	事務的[じむてき]な
\\	に 1週間[いっしゅうかん]かかります。		
\\	紹介	紹介[しょうかい]	しょうかい	
\\	両親に彼女を紹介した。	両親[りょうしん]に 彼女[かのじょ]を 紹介[しょうかい]した。	りょうしん に かのじょ を しょうかい した	
\\	両親[りょうしん]に 彼女[かのじょ]を
\\	した。		
\\	スリッパ	スリッパ	スリッパ	
\\	スリッパをはいてください。	スリッパをはいてください。	すりっぱ を はいて ください	
\\	をはいてください。		
\\	招く	招[まね]く	まねく	
\\	両親を食事に招いた。	両親[りょうしん]を 食事[しょくじ]に 招[まね]いた。	りょうしん を しょくじ に まねいた	
\\	両親[りょうしん]を 食事[しょくじ]に
\\	招待	招待[しょうたい]	しょうたい	
\\	高校の時の先生を結婚式に招待した。	高校[こうこう]の 時[とき]の 先生[せんせい]を 結婚式[けっこんしき]に 招待[しょうたい]した。	こうこう の とき の せんせい を けっこんしき に しょうたい した	
\\	高校[こうこう]の 時[とき]の 先生[せんせい]を 結婚式[けっこんしき]に
\\	した。		
\\	夫婦	夫婦[ふうふ]	ふうふ	
\\	その夫婦はとても仲がいい。	その 夫婦[ふうふ]はとても 仲[なか]がいい。	その ふうふ は とても なか が いい	
\\	その
\\	はとても 仲[なか]がいい。		
\\	奥	奥[おく]	おく	
\\	はさみは机の奥にあった。	はさみは 机[つくえ]の 奥[おく]にあった。	はさみ は つくえ の おく に あった	
\\	はさみは 机[つくえ]の
\\	にあった。		
\\	トランプ	トランプ	トランプ	
\\	友達とトランプをして遊びました。	友達[ともだち]とトランプをして 遊[あそ]びました。	ともだち と とらんぷ を して あそびました	
\\	友達[ともだち]と
\\	をして 遊[あそ]びました。		
\\	奥さん	奥[おく]さん	おくさん	
\\	彼の奥さんはきれいな方です。	彼[かれ]の 奥[おく]さんはきれいな 方[かた]です。	かれ の おくさん は きれい な かた です	
\\	彼[かれ]の
\\	はきれいな 方[かた]です。		
\\	皆さん	皆[みな]さん	みなさん	
\\	(丁寧語)		
\\	皆さんにお話があります。	皆[みな]さんにお 話[はなし]があります。	みなさん に おはなし が あります	
\\	にお 話[はなし]があります。		
\\	皆様	皆様[みなさま]	みなさま	
\\	(敬語)		
\\	皆様、こんにちは。	皆様[みなさま]、こんにちは。	みなさま こんにちは	
\\	、こんにちは。		
\\	誰か	誰[だれ]か	だれか	
\\	誰かに聞いてみてください。	誰[だれ]かに 聞[き]いてみてください。	だれか に きいて みて ください	
\\	に 聞[き]いてみてください。		
\\	よろしい	よろしい	よろしい	
\\	(いいの改まった言い方)		
\\	レポートはこれでよろしいですか。	レポートはこれでよろしいですか。	れぽーと は これ で よろしい です か	
\\	レポートはこれで
\\	ですか。		
\\	国籍	国籍[こくせき]	こくせき	
\\	私は日本国籍です。	私[わたし]は 日本[にほん] 国籍[こくせき]です。	わたし は にほん こくせき です	
\\	私[わたし]は 日本[にほん]
\\	です。		
\\	愛	愛[あい]	あい	
\\	彼女は愛をこめて手紙を書いた。	彼女[かのじょ]は 愛[あい]をこめて 手紙[てがみ]を 書[か]いた。	かのじょ は あい を こめて てがみ を かいた	
\\	彼女[かのじょ]は
\\	をこめて 手紙[てがみ]を 書[か]いた。		
\\	可愛い	可愛[かわい]い	かわいい	
\\	彼女の赤ちゃんは可愛いです。	彼女[かのじょ]の 赤[あか]ちゃんは 可愛[かわい]いです。	かのじょ の あかちゃん は かわいい です	
\\	彼女[かのじょ]の 赤[あか]ちゃんは
\\	です。		
\\	恋人	恋人[こいびと]	こいびと	
\\	彼は恋人を失った。	彼[かれ]は 恋人[こいびと]を 失[うしな]った。	かれ は こいびと を うしなった	
\\	彼[かれ]は
\\	を 失[うしな]った。		
\\	ラッシュアワー	ラッシュアワー	ラッシュアワー	
\\	私はラッシュアワーの電車が嫌いです。	私[わたし]はラッシュアワーの 電車[でんしゃ]が 嫌[きら]いです。	わたし は らっしゅあわー の でんしゃ が きらい です	
\\	私[わたし]は
\\	の 電車[でんしゃ]が 嫌[きら]いです。		
\\	誕生日	誕生日[たんじょうび]	たんじょうび	
\\	彼女の誕生日は7月16日です。	彼女[かのじょ]の 誕生日[たんじょうび]は 7月16日[しちがつ じゅうろくにち]です。	かのじょ の たんじょうび は しちがつ じゅうろくにち です	
\\	彼女[かのじょ]の
\\	は 7月16日[しちがつ じゅうろくにち]です。		
\\	祝日	祝日[しゅくじつ]	しゅくじつ	
\\	5月3日は祝日だ。	5月3日[ごがつ みっか]は 祝日[しゅくじつ]だ。	ごがつ みっか は しゅくじつ だ	
\\	5月3日[ごがつ みっか]は
\\	だ。		
\\	お祝い	お 祝[いわ]い	おいわい	
\\	彼の就職のお祝いをしよう。	彼[かれ]の 就職[しゅうしょく]のお 祝[いわ]いをしよう。	かれ の しゅうしょく の おいわい を しよう	
\\	彼[かれ]の 就職[しゅうしょく]の
\\	をしよう。		
\\	夢	夢[ゆめ]	ゆめ	
\\	昨夜恐ろしい夢を見た。	昨夜[ゆうべ] 恐[おそ]ろしい 夢[ゆめ]を 見[み]た。	ゆうべ おそろしい ゆめ を みた	
\\	昨夜[ゆうべ] 恐[おそ]ろしい
\\	を 見[み]た。		
\\	カップ	カップ	カップ	
\\	水を1カップ入れてください。	水[みず]を 1[いち]カップ 入[い]れてください。	みず を いちかっぷ いれて ください	
\\	水[みず]を 1[いち]
\\	入[い]れてください。		
\\	泣く	泣[な]く	なく	
\\	妹はすぐに泣く。	妹[いもうと]はすぐに 泣[な]く。	いもうと は すぐ に なく	
\\	妹[いもうと]はすぐに
\\	涙	涙[なみだ]	なみだ	
\\	彼女の目から涙がこぼれた。	彼女[かのじょ]の 目[め]から 涙[なみだ]がこぼれた。	かのじょ の め から なみだ が こぼれた	
\\	彼女[かのじょ]の 目[め]から
\\	がこぼれた。		
\\	喜ぶ	喜[よろこ]ぶ	よろこぶ	
\\	彼女はとても喜びました。	彼女[かのじょ]はとても 喜[よろこ]びました。	かのじょ は とても よろこびました	
\\	彼女[かのじょ]はとても
\\	恥ずかしい	恥[は]ずかしい	はずかしい	
\\	とても恥ずかしかった。	とても 恥[は]ずかしかった。	とても はずかしかった	
\\	とても
\\	スケート	スケート	スケート	
\\	湖でスケートをした。	湖[みずうみ]でスケートをした。	みずうみ で すけーと を した	
\\	湖[みずうみ]で
\\	をした。		
\\	弁当	弁当[べんとう]	べんとう	
\\	今日は弁当を持ってきました。	今日[きょう]は 弁当[べんとう]を 持[も]ってきました。	きょう は べんとう を もって きました	
\\	今日[きょう]は
\\	を 持[も]ってきました。		
\\	看護師	看護師[かんごし]	かんごし	
\\	母は看護師です。	母[はは]は 看護師[かんごし]です。	はは は かんごし です	
\\	母[はは]は
\\	です。		
\\	患者	患者[かんじゃ]	かんじゃ	
\\	患者は眠っています。	患者[かんじゃ]は 眠[ねむ]っています。	かんじゃ は ねむって います	
\\	は 眠[ねむ]っています。		
\\	述べる	述[の]べる	のべる	
\\	上司が意見を述べた。	上司[じょうし]が 意見[いけん]を 述[の]べた。	じょうし が いけん を のべた	
\\	上司[じょうし]が 意見[いけん]を
\\	たくさん	たくさん	たくさん	
\\	昨日はたくさん泳ぎました。	昨日[きのう]はたくさん 泳[およ]ぎました。	きのう は たくさん およぎました	
\\	昨日[きのう]は
\\	泳[およ]ぎました。		
\\	訴える	訴[うった]える	うったえる	
\\	彼女は会社を訴えた。	彼女[かのじょ]は 会社[かいしゃ]を 訴[うった]えた。	かのじょ は かいしゃ を うったえた	
\\	彼女[かのじょ]は 会社[かいしゃ]を
\\	迷う	迷[まよ]う	まよう	
\\	道に迷いました。	道[みち]に 迷[まよ]いました。	みち に まよいました	
\\	道[みち]に
\\	迷惑	迷惑[めいわく]	めいわく	
\\	人に迷惑をかけてはいけません。	人[ひと]に 迷惑[めいわく]をかけてはいけません。	ひと に めいわく を かけて は いけません	
\\	人[ひと]に
\\	をかけてはいけません。		
\\	地域	地域[ちいき]	ちいき	
\\	この地域は雨が多い。	この 地域[ちいき]は 雨[あめ]が 多[おお]い。	この ちいき は あめ が おおい	
\\	この
\\	は 雨[あめ]が 多[おお]い。		
\\	ピクニック	ピクニック	ピクニック	
\\	今日はピクニックに行きましょう。	今日[きょう]はピクニックに 行[い]きましょう。	きょう は ぴくにっく に いきましょう	
\\	今日[きょう]は
\\	に 行[い]きましょう。		
\\	政権	政権[せいけん]	せいけん	
\\	政権が交代した。	政権[せいけん]が 交代[こうたい]した。	せいけん が こうたい した	
\\	が 交代[こうたい]した。		
\\	贈る	贈[おく]る	おくる	
\\	とか)		
\\	母に花を贈った。	母[はは]に 花[はな]を 贈[おく]った。	はは に はな を おくった	
\\	母[はは]に 花[はな]を
\\	贈り物	贈[おく]り 物[もの]	おくりもの	
\\	(和語)		
\\	すてきな贈り物をどうもありがとう。	すてきな 贈[おく]り 物[もの]をどうもありがとう。	すてき な おくりもの を どうも ありがとう	
\\	すてきな
\\	をどうもありがとう。		
\\	与える	与[あた]える	あたえる	
\\	(かたい)		
\\	そのニュースは彼に大きなショックを与えた。	そのニュースは 彼[かれ]に 大[おお]きなショックを 与[あた]えた。	その にゅーす は かれ に おおき な しょっく を あたえた	
\\	そのニュースは 彼[かれ]に 大[おお]きなショックを
\\	あちら	あちら	あちら	
\\	あちらに行ってみよう。	あちらに 行[い]ってみよう。	あちら に いって みよう	
\\	に 行[い]ってみよう。		
\\	貯金	貯金[ちょきん]	ちょきん	
\\	貯金は十分にあります。	貯金[ちょきん]は 十分[じゅうぶん]にあります。	ちょきん は じゅうぶん に あります	
\\	は 十分[じゅうぶん]にあります。		
\\	預ける	預[あず]ける	あずける	
\\	鍵を彼に預けた。	鍵[かぎ]を 彼[かれ]に 預[あず]けた。	かぎ を かれ に あずけた	
\\	鍵[かぎ]を 彼[かれ]に
\\	預かる	預[あず]かる	あずかる	
\\	荷物を預かってください。	荷物[にもつ]を 預[あず]かってください。	にもつ を あずかって ください	
\\	荷物[にもつ]を
\\	ください。		
\\	得意	得意[とくい]	とくい	
\\	彼は歌が得意です。	彼[かれ]は 歌[うた]が 得意[とくい]です。	かれ は うた が とくい です	
\\	彼[かれ]は 歌[うた]が
\\	です。		
\\	どきどき	どきどき	どきどき	
\\	彼女に会うとどきどきします。	彼女[かのじょ]に 会[あ]うとどきどきします。	かのじょ に あう と どきどき します	
\\	彼女[かのじょ]に 会[あ]うと
\\	します。		
\\	燃える	燃[も]える	もえる	
\\	が		
\\	山が燃えています。	山[やま]が 燃[も]えています。	やま が もえて います	
\\	山[やま]が
\\	焼ける	焼[や]ける	やける	
\\	が		
\\	肉が焼けました。	肉[にく]が 焼[や]けました。	にく が やけました	
\\	肉[にく]が
\\	すき焼き	すき 焼[や]き	すきやき	
\\	昨日の夜はすき焼きを食べた。	昨日[きのう]の 夜[よる]はすき 焼[や]きを 食[た]べた。	きのう の よる は すきやき を たべた	
\\	昨日[きのう]の 夜[よる]は
\\	を 食[た]べた。		
\\	焼く	焼[や]く	やく	
\\	を		
\\	今、魚を焼いています。	今[いま]、 魚[さかな]を 焼[や]いています。	いま さかな を やいて います	
\\	今[いま]、 魚[さかな]を
\\	まあまあ	まあまあ	まあまあ	
\\	彼の成績はまあまあです。	彼[かれ]の 成績[せいせき]はまあまあです。	かれ の せいせき は まあまあ です	
\\	彼[かれ]の 成績[せいせき]は
\\	です。		
\\	乾く	乾[かわ]く	かわく	
\\	(が)		
\\	夏は洗濯物がすぐ乾く。	夏[なつ]は 洗濯物[せんたくもの]がすぐ 乾[かわ]く。	なつ は せんたくもの が すぐ かわく	
\\	夏[なつ]は 洗濯物[せんたくもの]がすぐ
\\	乾杯	乾杯[かんぱい]	かんぱい	
\\	お二人の未来に乾杯しましょう。	お 二人[ふたり]の 未来[みらい]に 乾杯[かんぱい]しましょう。	おふたり の みらい に かんぱい しましょう	
\\	お 二人[ふたり]の 未来[みらい]に
\\	しましょう。		
\\	乾かす	乾[かわ]かす	かわかす	
\\	を		
\\	ぬれた服を乾かした。	ぬれた 服[ふく]を 乾[かわ]かした。	ぬれた ふく を かわかした	
\\	ぬれた 服[ふく]を
\\	新幹線	新幹線[しんかんせん]	しんかんせん	
\\	新幹線で京都に行きました。	新幹線[しんかんせん]で 京都[きょうと]に 行[い]きました。	しんかんせん で きょうと に いきました	
\\	で 京都[きょうと]に 行[い]きました。		
\\	マフラー	マフラー	マフラー	
\\	彼女は首にマフラーをまいていた。	彼女[かのじょ]は 首[くび]にマフラーをまいていた。	かのじょ は くび に まふらー を まいて いた	
\\	彼女[かのじょ]は 首[くび]に
\\	をまいていた。		
\\	素晴らしい	素晴[すば]らしい	すばらしい	
\\	素晴らしい景色ですね。	素晴[すば]らしい 景色[けしき]ですね。	すばらしい けしき です ね	
\\	景色[けしき]ですね。		
\\	海岸	海岸[かいがん]	かいがん	
\\	海岸を散歩しましょう。	海岸[かいがん]を 散歩[さんぽ]しましょう。	かいがん を さんぽ しましょう	
\\	を 散歩[さんぽ]しましょう。		
\\	家庭	家庭[かてい]	かてい	
\\	彼は家庭を大切にしている。	彼[かれ]は 家庭[かてい]を 大切[たいせつ]にしている。	かれ は かてい を たいせつ に して いる	
\\	彼[かれ]は
\\	を 大切[たいせつ]にしている。		
\\	庭	庭[にわ]	にわ	
\\	庭にバラを植えました。	庭[にわ]にバラを 植[う]えました。	にわ に ばら を うえました	
\\	にバラを 植[う]えました。		
\\	アクセサリー	アクセサリー	アクセサリー	
\\	このアクセサリーは素敵ね。	このアクセサリーは 素敵[すてき]ね。	この あくせさりー は すてき ね	
\\	この
\\	は 素敵[すてき]ね。		
\\	桜	桜[さくら]	さくら	
\\	桜は三月か四月に咲きます。	桜[さくら]は 三月[さんがつ]か 四月[しがつ]に 咲[さ]きます。	さくら は さんがつ か しがつ に さきます	
\\	は 三月[さんがつ]か 四月[しがつ]に 咲[さ]きます。		
\\	咲く	咲[さ]く	さく	
\\	桜の花が咲きました。	桜[さくら]の 花[はな]が 咲[さ]きました。	さくら の はな が さきました	
\\	桜[さくら]の 花[はな]が
\\	吹く	吹[ふ]く	ふく	
\\	今日は北風が吹いている。	今日[きょう]は 北風[きたかぜ]が 吹[ふ]いている。	きょう は きたかぜ が ふいて いる	
\\	今日[きょう]は 北風[きたかぜ]が
\\	散歩	散歩[さんぽ]	さんぽ	
\\	私のお祖父さんは毎日散歩します。	私[わたし]のお 祖父[じい]さんは 毎日[まいにち] 散歩[さんぽ]します。	わたし の おじいさん は まいにち さんぽ します 。	
\\	私[わたし]のお 祖父[じい]さんは
\\	します。		
\\	あくび	あくび	あくび	
\\	父があくびをした。	父[ちち]があくびをした。	ちち が あくび を した	
\\	父[ちち]が
\\	をした。		
\\	植える	植[う]える	うえる	
\\	庭にバラを植えました。	庭[にわ]にバラを 植[う]えました。	にわ に ばら を うえました	
\\	庭[にわ]にバラを
\\	屋根	屋根[やね]	やね	
\\	屋根にカラスが止まっています。	屋根[やね]にカラスが 止[と]まっています。	やね に からす が とまって います	
\\	にカラスが 止[と]まっています。		
\\	黒板	黒板[こくばん]	こくばん	
\\	答えを黒板に書いてください。	答[こた]えを 黒板[こくばん]に 書[か]いてください。	こたえ を こくばん に かいて ください	
\\	答[こた]えを
\\	に 書[か]いてください。		
\\	掲示板	掲示板[けいじばん]	けいじばん	
\\	掲示板のお知らせを見ましたか。	掲示板[けいじばん]のお 知[し]らせを 見[み]ましたか。	けいじばん の おしらせ を みました か	
\\	のお 知[し]らせを 見[み]ましたか。		
\\	からから	からから	からから	
\\	おしゃべりして喉がからからになりました。	おしゃべりして 喉[のど]がからからになりました。	おしゃべり して のど が からから に なりました	
\\	おしゃべりして 喉[のど]が
\\	になりました。		
\\	草	草[くさ]	くさ	
\\	庭に草が生えている。	庭[にわ]に 草[くさ]が 生[は]えている。	にわ に くさ が はえて いる	
\\	庭[にわ]に
\\	が 生[は]えている。		
\\	葉	葉[は]	は	
\\	これは桜の葉です。	これは 桜[さくら]の 葉[は]です。	これ は さくら の は です	
\\	これは 桜[さくら]の
\\	です。		
\\	絵葉書	絵葉書[えはがき]	えはがき	
\\	友達から絵葉書が届きました。	友達[ともだち]から 絵葉書[えはがき]が 届[とど]きました。	ともだち から えはがき が とどきました	
\\	友達[ともだち]から
\\	が 届[とど]きました。		
\\	はがき	はがき	はがき	
\\	家族にはがきを書いています。	家族[かぞく]にはがきを 書[か]いています。	かぞく に はがき を かいています 。	
\\	家族[かぞく]に
\\	を 書[か]いています。		
\\	ふらふら	ふらふら	ふらふら	
\\	私は疲れてふらふらです。	私[わたし]は 疲[つか]れてふらふらです。	わたし は つかれて ふらふら です	
\\	私[わたし]は 疲[つか]れて
\\	です。		
\\	吸う	吸[す]う	すう	
\\	彼は大きく息を吸った。	彼[かれ]は 大[おお]きく 息[いき]を 吸[す]った。	かれ は おおきく いき を すった	
\\	彼[かれ]は 大[おお]きく 息[いき]を
\\	普及	普及[ふきゅう]	ふきゅう	
\\	ゴミのリサイクルが普及している。	ゴミのリサイクルが 普及[ふきゅう]している。	ごみ の りさいくる が ふきゅう して いる	
\\	ゴミのリサイクルが
\\	している。		
\\	胃	胃[い]	い	
\\	昨日から胃が痛い。	昨日[きのう]から 胃[い]が 痛[いた]い。	きのう から い が いたい	
\\	昨日[きのう]から
\\	が 痛[いた]い。		
\\	皿	皿[さら]	さら	
\\	皿にケーキを載せました。	皿[さら]にケーキを 載[の]せました。	さら に けーき を のせました。	
\\	にケーキを 載[の]せました。		
\\	こぼす	こぼす	こぼす	
\\	ソファにワインをこぼした。	ソファにワインをこぼした。	そふぁ に わいん を こぼした	
\\	ソファにワインを
\\	血	血[ち]	ち	
\\	血が出ていますよ。	血[ち]が 出[で]ていますよ。	ち が でて います よ	
\\	が 出[で]ていますよ。		
\\	内容	内容[ないよう]	ないよう	
\\	この本の内容を説明してください。	この 本[ほん]の 内容[ないよう]を 説明[せつめい]してください。	この ほん の ないよう を せつめい して ください	
\\	この 本[ほん]の
\\	を 説明[せつめい]してください。		
\\	背景	背景[はいけい]	はいけい	
\\	事件の背景に何があったのだろう。	事件[じけん]の 背景[はいけい]に 何[なに]があったのだろう。	じけん の はいけい に なに が あった の だろう	
\\	事件[じけん]の
\\	に 何[なに]があったのだろう。		
\\	骨	骨[ほね]	ほね	
\\	彼は足の骨を折りました。	彼[かれ]は 足[あし]の 骨[ほね]を 折[お]りました。	かれ は あし の ほね を おりました	
\\	彼[かれ]は 足[あし]の
\\	を 折[お]りました。		
\\	ランチ	ランチ	ランチ	
\\	一緒にランチに行きませんか。	一緒[いっしょ]にランチに 行[い]きませんか。	いっしょ に らんち に いきません か	
\\	一緒[いっしょ]に
\\	に 行[い]きませんか。		
\\	滑る	滑[すべ]る	すべる	
\\	彼は雪の上を滑った。	彼[かれ]は 雪[ゆき]の 上[うえ]を 滑[すべ]った。	かれ は ゆき の うえ を すべった	
\\	彼[かれ]は 雪[ゆき]の 上[うえ]を
\\	折れる	折[お]れる	おれる	
\\	強風で木の枝が折れた。	強風[きょうふう]で 木[き]の 枝[えだ]が 折[お]れた。	きょうふう で き の えだ が おれた	
\\	強風[きょうふう]で 木[き]の 枝[えだ]が
\\	折る	折[お]る	おる	
\\	祖父は足の骨を折りました。	祖父[そふ]は 足[あし]の 骨[ほね]を 折[お]りました。	そふ は あし の ほね を おりました	
\\	祖父[そふ]は 足[あし]の 骨[ほね]を
\\	健康	健康[けんこう]	けんこう	
\\	私は健康な生活を送っています。	私[わたし]は 健康[けんこう]な 生活[せいかつ]を 送[おく]っています。	わたし は けんこう な せいかつ を おくって います	
\\	私[わたし]は
\\	な 生活[せいかつ]を 送[おく]っています。		
\\	レシート	レシート	レシート	
\\	レシートを財布に入れました。	レシートを 財布[さいふ]に 入[い]れました。	れしーと を さいふ に いれました	
\\	を 財布[さいふ]に 入[い]れました。		
\\	珍しい	珍[めずら]しい	めずらしい	
\\	昨日珍しい果物を食べました。	昨日[きのう] 珍[めずら]しい 果物[くだもの]を 食[た]べました。	きのう めずらしい くだもの を たべました	
\\	昨日[きのう]
\\	果物[くだもの]を 食[た]べました。		
\\	撮る	撮[と]る	とる	
\\	写真をたくさん撮りました。	写真[しゃしん]をたくさん 撮[と]りました。	しゃしん を たくさん とりました	
\\	写真[しゃしん]をたくさん
\\	再び	再[ふたた]び	ふたたび	
\\	彼は再びここに戻ってきた。	彼[かれ]は 再[ふたた]びここに 戻[もど]ってきた。	かれ は ふたたび ここ に もどって きた	
\\	彼[かれ]は
\\	ここに 戻[もど]ってきた。		
\\	再来年	再来年[さらいねん]	さらいねん	
\\	次のオリンピックは再来年です。	次[つぎ]のオリンピックは 再来年[さらいねん]です。	つぎ の おりんぴっく は さらいねん です	
\\	次[つぎ]のオリンピックは
\\	です。		
\\	チョーク	チョーク	チョーク	
\\	チョークで黒板に字を書きました。	チョークで 黒板[こくばん]に 字[じ]を 書[か]きました。	ちょーく で こくばん に じ を かきました	
\\	で 黒板[こくばん]に 字[じ]を 書[か]きました。		
\\	再来月	再来月[さらいげつ]	さらいげつ	
\\	再来月まで予約で一杯です。	再来月[さらいげつ]まで 予約[よやく]で 一杯[いっぱい]です。	さらいげつ まで よやく で いっぱい です	
\\	まで 予約[よやく]で 一杯[いっぱい]です。		
\\	再来週	再来週[さらいしゅう]	さらいしゅう	
\\	再来週は忙しいです。	再来週[さらいしゅう]は 忙[いそが]しいです。	さらいしゅう は いそがしい です	
\\	は 忙[いそが]しいです。		
\\	放送	放送[ほうそう]	ほうそう	
\\	その番組は来週放送されます。	その 番組[ばんぐみ]は 来週[らいしゅう] 放送[ほうそう]されます。	その ばんぐみ は らいしゅう ほうそう されます	
\\	その 番組[ばんぐみ]は 来週[らいしゅう]
\\	されます。		
\\	装置	装置[そうち]	そうち	
\\	実験にはこの装置を使います。	実験[じっけん]にはこの 装置[そうち]を 使[つか]います。	じっけん に は この そうち を つかいます	
\\	実験[じっけん]にはこの
\\	を 使[つか]います。		
\\	ティッシュ	ティッシュ	ティッシュ	
\\	ティッシュを取ってください。	ティッシュを 取[と]ってください。	てぃっしゅ を とって ください	
\\	を 取[と]ってください。		
\\	仮名	仮名[かな]	かな	
\\	お名前に仮名を振ってください。	お 名前[なまえ]に 仮名[かな]を 振[ふ]ってください。	おなまえ に かな を ふって ください	
\\	お 名前[なまえ]に
\\	を 振[ふ]ってください。		
\\	送り仮名	送[おく]り 仮名[がな]	おくりがな	
\\	その送り仮名は間違っています。	その 送[おく]り 仮名[がな]は 間違[まちが]っています。	その おくりがな は まちがって います	
\\	その
\\	は 間違[まちが]っています。		
\\	鏡	鏡[かがみ]	かがみ	
\\	彼女は鏡を見て髪を直した。	彼女[かのじょ]は 鏡[かがみ]を 見[み]て 髪[かみ]を 直[なお]した。	かのじょ は かがみ を みて かみ を なおした	
\\	彼女[かのじょ]は
\\	を 見[み]て 髪[かみ]を 直[なお]した。		
\\	悲しむ	悲[かな]しむ	かなしむ	
\\	父は友だちの死を悲しんでいます。	父[ちち]は 友[とも]だちの 死[し]を 悲[かな]しんでいます。	ちち は ともだち の し を かなしんで います	
\\	父[ちち]は 友[とも]だちの 死[し]を
\\	ノック	ノック	ノック	
\\	入る時はドアをノックしてください。	入[はい]る 時[とき]はドアをノックしてください。	はいる とき は どあ を のっく して ください	
\\	入[はい]る 時[とき]はドアを
\\	してください。		
\\	固い	固[かた]い	かたい	
\\	私の上司は頭が固い。	私[わたし]の 上司[じょうし]は 頭[あたま]が 固[かた]い。	わたし の じょうし は あたま が かたい	
\\	私[わたし]の 上司[じょうし]は 頭[あたま]が
\\	美術館	美術館[びじゅつかん]	びじゅつかん	
\\	昨日、美術館に行きました。	昨日[きのう]、 美術館[びじゅつかん]に 行[い]きました。	きのう びじゅつかん に いきました	
\\	昨日[きのう]、
\\	に 行[い]きました。		
\\	美人	美人[びじん]	びじん	
\\	彼のお母さんは美人です。	彼[かれ]のお 母[かあ]さんは 美人[びじん]です。	かれ の おかあさん は びじん です	
\\	彼[かれ]のお 母[かあ]さんは
\\	です。		
\\	美容院	美容院[びよういん]	びよういん	
\\	父は美容院で髪を切ります。	父[ちち]は 美容院[びよういん]で 髪[かみ]を 切[き]ります。	ちち は びよういん で かみ を きります	
\\	父[ちち]は
\\	で 髪[かみ]を 切[き]ります。		
\\	よろしく	よろしく	よろしく	
\\	よろしくお願いします。	よろしくお 願[ねが]いします。	よろしく おねがい します	
\\	お 願[ねが]いします。		
\\	博物館	博物館[はくぶつかん]	はくぶつかん	
\\	昨日、車の博物館に行った。	昨日[きのう]、 車[くるま]の 博物館[はくぶつかん]に 行[い]った。	きのう くるま の はくぶつかん に いった	
\\	昨日[きのう]、 車[くるま]の
\\	に 行[い]った。		
\\	開催	開催[かいさい]	かいさい	
\\	京都で国際会議が開催された。	京都[きょうと]で 国際会議[こくさいかいぎ]が 開催[かいさい]された。	きょうと で こくさいかいぎ が かいさい された	
\\	京都[きょうと]で 国際会議[こくさいかいぎ]が
\\	された。		
\\	特徴	特徴[とくちょう]	とくちょう	
\\	この曲線がこの車の特徴です。	この 曲線[きょくせん]がこの 車[くるま]の 特徴[とくちょう]です。	この きょくせん が この くるま の とくちょう です	
\\	この 曲線[きょくせん]がこの 車[くるま]の
\\	です。		
\\	許す	許[ゆる]す	ゆるす	
\\	私は彼を許しました。	私[わたし]は 彼[かれ]を 許[ゆる]しました。	わたし は かれ を ゆるしました	
\\	私[わたし]は 彼[かれ]を
\\	がらがら	がらがら	がらがら	
\\	映画館はがらがらでした。	映画館[えいがかん]はがらがらでした。	えいがかん は がらがら でした	
\\	映画館[えいがかん]は
\\	でした。		
\\	免許証	免許証[めんきょしょう]	めんきょしょう	
\\	免許証を見せてください。	免許証[めんきょしょう]を 見[み]せてください。	めんきょしょう を みせて ください	
\\	を 見[み]せてください。		
\\	教師	教師[きょうし]	きょうし	
\\	彼は高校教師だ。	彼[かれ]は 高校[こうこう] 教師[きょうし]だ。	かれ は こうこう きょうし だ	
\\	彼[かれ]は 高校[こうこう]
\\	だ。		
\\	教授	教授[きょうじゅ]	きょうじゅ	
\\	彼は化学の教授です。	彼[かれ]は 化学[かがく]の 教授[きょうじゅ]です。	かれ は かがく の きょうじゅ です	
\\	彼[かれ]は 化学[かがく]の
\\	です。		
\\	伝える	伝[つた]える	つたえる	
\\	みんなにこのことを伝えてください。	みんなにこのことを 伝[つた]えてください。	みんな に この こと を つたえて ください	
\\	みんなにこのことを
\\	ください。		
\\	ぎらぎら	ぎらぎら	ぎらぎら	
\\	太陽がぎらぎらしている。	太陽[たいよう]がぎらぎらしている。	たいよう が ぎらぎら して いる	
\\	太陽[たいよう]が
\\	している。		
\\	鳥	鳥[とり]	とり	
\\	鳥が飛んでいます。	鳥[とり]が 飛[と]んでいます。	とり が とんで います	
\\	が 飛[と]んでいます。		
\\	鳴る	鳴[な]る	なる	
\\	今朝、5時に電話が鳴った。	今朝[けさ]、 5時[ごじ]に 電話[でんわ]が 鳴[な]った。	けさ ごじ に でんわ が なった	
\\	今朝[けさ]、 5時[ごじ]に 電話[でんわ]が
\\	鳴く	鳴[な]く	なく	
\\	どこかでネコが鳴いている。	どこかでネコが 鳴[な]いている。	どこか で ねこ が ないて いる	
\\	どこかでネコが
\\	声	声[こえ]	こえ	
\\	彼は大きな声で話した。	彼[かれ]は 大[おお]きな 声[こえ]で 話[はな]した。	かれ は おおき な こえ で はなした	
\\	彼[かれ]は 大[おお]きな
\\	で 話[はな]した。		
\\	そっち	そっち	そっち	
\\	そっちが私の部屋です。	そっちが 私[わたし]の 部屋[へや]です。	そっち が わたし の へや です	
\\	が 私[わたし]の 部屋[へや]です。		
\\	卵	卵[たまご]	たまご	
\\	ニワトリは卵を産みます。	ニワトリは 卵[たまご]を 産[う]みます。	にわとり は たまご を うみます	
\\	ニワトリは
\\	を 産[う]みます。		
\\	犬	犬[いぬ]	いぬ	
\\	この犬はとても賢い。	この 犬[いぬ]はとても 賢[かしこ]い。	この いぬ は とても かしこい	
\\	この
\\	はとても 賢[かしこ]い。		
\\	馬	馬[うま]	うま	
\\	彼は牧場で馬に乗った。	彼[かれ]は 牧場[ぼくじょう]で 馬[うま]に 乗[の]った。	かれ は ぼくじょう で うま に のった	
\\	彼[かれ]は 牧場[ぼくじょう]で
\\	に 乗[の]った。		
\\	駐車場	駐車場[ちゅうしゃじょう]	ちゅうしゃじょう	
\\	車は駐車場に止めてください。	車[くるま]は 駐車場[ちゅうしゃじょう]に 止[と]めてください。	くるま は ちゅうしゃじょう に とめて ください	
\\	車[くるま]は
\\	に 止[と]めてください。		
\\	ピンポン	ピンポン	ピンポン	
\\	昨日の夕方、友達とピンポンをした。	昨日[きのう]の 夕方[ゆうがた]、 友達[ともだち]とピンポンをした。	きのう の ゆうがた ともだち と ぴんぽん を した	
\\	昨日[きのう]の 夕方[ゆうがた]、 友達[ともだち]と
\\	をした。		
\\	騒ぐ	騒[さわ]ぐ	さわぐ	
\\	電車の中で騒がないでください。	電車[でんしゃ]の 中[なか]で 騒[さわ]がないでください。	でんしゃ の なか で さわがない で ください	
\\	電車[でんしゃ]の 中[なか]で
\\	ください。		
\\	刺す	刺[さ]す	さす	
\\	虫に腕を刺されました。	虫[むし]に 腕[うで]を 刺[さ]されました。	むし に うで を さされました	
\\	虫[むし]に 腕[うで]を
\\	刺身	刺身[さしみ]	さしみ	
\\	私は刺身は食べません。	私[わたし]は 刺身[さしみ]は 食[た]べません。	わたし は さしみ は たべません	
\\	私[わたし]は
\\	は 食[た]べません。		
\\	激しい	激[はげ]しい	はげしい	
\\	激しい雨が降っています。	激[はげ]しい 雨[あめ]が 降[ふ]っています。	はげしい あめ が ふって います	
\\	雨[あめ]が 降[ふ]っています。		
\\	ぺこぺこ	ぺこぺこ	ぺこぺこ	
\\	私はおなかがぺこぺこです。	私[わたし]はおなかがぺこぺこです。	わたし は おなか が ぺこぺこ です	
\\	私[わたし]はおなかが
\\	です。		
\\	驚く	驚[おどろ]く	おどろく	
\\	彼は血を見て驚いた。	彼[かれ]は 血[ち]を 見[み]て 驚[おどろ]いた。	かれ は ち を みて おどろいた	
\\	彼[かれ]は 血[ち]を 見[み]て
\\	倒れる	倒[たお]れる	たおれる	
\\	強風で木が倒れた。	強風[きょうふう]で 木[き]が 倒[たお]れた。	きょうふう で き が たおれた	
\\	強風[きょうふう]で 木[き]が
\\	倒す	倒[たお]す	たおす	
\\	そのスキーヤーはポールを倒した。	そのスキーヤーはポールを 倒[たお]した。	その すきーやー は ぽーる を たおした	
\\	そのスキーヤーはポールを
\\	傾向	傾向[けいこう]	けいこう	
\\	最近の若者は難しい本を読まない傾向がある。	最近[さいきん]の 若者[わかもの]は 難[むずか]しい 本[ほん]を 読[よ]まない 傾向[けいこう]がある。	さいきん の わかもの は むずかしい ほん を よまない けいこう が ある	
\\	最近[さいきん]の 若者[わかもの]は 難[むずか]しい 本[ほん]を 読[よ]まない
\\	がある。		
\\	ぺらぺら	ぺらぺら	ぺらぺら	
\\	彼女はドイツ語がぺらぺらです。	彼女[かのじょ]はドイツ 語[ご]がぺらぺらです。	かのじょ は どいつご が ぺらぺら です	
\\	彼女[かのじょ]はドイツ 語[ご]が
\\	です。		
\\	柔道	柔道[じゅうどう]	じゅうどう	
\\	私は柔道を習っています。	私[わたし]は 柔道[じゅうどう]を 習[なら]っています。	わたし は じゅうどう を ならって います	
\\	私[わたし]は
\\	を 習[なら]っています。		
\\	柔らかい	柔[やわ]らかい	やわらかい	
\\	布団がとても柔らかい。	布団[ふとん]がとても 柔[やわ]らかい。	ふとん が とても やわらかい	
\\	布団[ふとん]がとても
\\	柔らか	柔[やわ]らか	やわらか	
\\	彼の声は柔らかだ。	彼[かれ]の 声[こえ]は 柔[やわ]らかだ。	かれ の こえ は やわらか だ	
\\	彼[かれ]の 声[こえ]は
\\	だ。		
\\	主張	主張[しゅちょう]	しゅちょう	
\\	上司は私の主張を受け入れた。	上司[じょうし]は 私[わたし]の 主張[しゅちょう]を 受[う]け 入[い]れた。	じょうし は わたし の しゅちょう を うけいれた	
\\	上司[じょうし]は 私[わたし]の
\\	を 受[う]け 入[い]れた。		
\\	レインコート	レインコート	レインコート	
\\	雨なのでレインコートを着ました。	雨[あめ]なのでレインコートを 着[き]ました。	あめ な の で れいんこーと を きました	
\\	雨[あめ]なので
\\	を 着[き]ました。		
\\	引っ張る	引[ひ]っ 張[ぱ]る	ひっぱる	
\\	娘が私の手を引っ張った。	娘[むすめ]が 私[わたし]の 手[て]を 引[ひ]っ 張[ぱ]った。	むすめ が わたし の て を ひっぱった	
\\	娘[むすめ]が 私[わたし]の 手[て]を
\\	突き当たる	突[つ]き 当[あ]たる	つきあたる	
\\	突き当たったら右に曲がってください。	突[つ]き 当[あ]たったら 右[みぎ]に 曲[ま]がってください。	つきあたったら みぎ に まがって ください	
\\	右[みぎ]に 曲[ま]がってください。		
\\	突き当たり	突[つ]き 当[あ]たり	つきあたり	
\\	この先は突き当たりです。	この 先[さき]は 突[つ]き 当[あ]たりです。	この さき は つきあたり です	
\\	この 先[さき]は
\\	です。		
\\	壁	壁[かべ]	かべ	
\\	壁に絵が掛かっている。	壁[かべ]に 絵[え]が 掛[か]かっている。	かべ に え が かかって いる	
\\	に 絵[え]が 掛[か]かっている。		
\\	アイス	アイス	アイス	
\\	暑いのでアイスを食べました。	暑[あつ]いのでアイスを 食[た]べました。	あつい の で あいす を たべました	
\\	暑[あつ]いので
\\	を 食[た]べました。		
\\	弾く	弾[ひ]く	ひく	
\\	彼はギターを弾きます。	彼[かれ]はギターを 弾[ひ]きます。	かれ は ぎたー を ひきます	
\\	彼[かれ]はギターを
\\	丸い	丸[まる]い	まるい	
\\	地球は丸い。	地球[ちきゅう]は 丸[まる]い。	ちきゅう は まるい。	
\\	地球[ちきゅう]は
\\	丸	丸[まる]	まる	
\\	紙に大きな丸を書きました。	紙[かみ]に 大[おお]きな 丸[まる]を 書[か]きました。	かみ に おおき な まる を かきました	
\\	紙[かみ]に 大[おお]きな
\\	を 書[か]きました。		
\\	攻撃	攻撃[こうげき]	こうげき	
\\	2003年にアメリカはイラクを攻撃した。	2003年[にせんさんねん]にアメリカはイラクを 攻撃[こうげき]した。	にせんさんねん に あめりか は いらく を こうげき した	
\\	2003年[にせんさんねん]にアメリカはイラクを
\\	した。		
\\	いつごろ	いつごろ	いつごろ	
\\	いつごろ夏休みを取りますか。	いつごろ 夏休[なつやす]みを 取[と]りますか。	いつごろ なつやすみ を とります か	
\\	夏休[なつやす]みを 取[と]りますか。		
\\	絶対に	絶対[ぜったい]に	ぜったいに	
\\	絶対に駄目です。	絶対[ぜったい]に 駄目[だめ]です。	ぜったいに だめ です	
\\	駄目[だめ]です。		
\\	消防車	消防車[しょうぼうしゃ]	しょうぼうしゃ	
\\	消防車が4台も来た。	消防車[しょうぼうしゃ]が 4台[よんだい]も 来[き]た。	しょうぼうしゃ が よんだい も きた	
\\	が 4台[よんだい]も 来[き]た。		
\\	嫌	嫌[いや]	いや	
\\	私は待つのが嫌だ。	私[わたし]は 待[ま]つのが 嫌[いや]だ。	わたし は まつ の が いや だ	
\\	私[わたし]は 待[ま]つのが
\\	だ。		
\\	大嫌い	大嫌[だいきら]い	だいきらい	
\\	私はテストが大嫌い。	私[わたし]はテストが 大嫌[だいきら]い。	わたし は てすと が だいきらい	
\\	私[わたし]はテストが
\\	ウィスキー	ウィスキー	ウィスキー	
\\	このウィスキーは強いよ。	このウィスキーは 強[つよ]いよ。	この うぃすきー は つよい よ	
\\	この
\\	は 強[つよ]いよ。		
\\	大抵	大抵[たいてい]	たいてい	
\\	朝食は大抵7時頃に食べます。	朝食[ちょうしょく]は 大抵[たいてい] 7時頃[しちじごろ]に 食[た]べます。	ちょうしょく は たいてい しちじごろ に たべます	
\\	朝食[ちょうしょく]は
\\	7時頃[しちじごろ]に 食[た]べます。		
\\	大勢	大勢[おおぜい]	おおぜい	
\\	大勢で食事に出かけました。	大勢[おおぜい]で 食事[しょくじ]に 出[で]かけました。	おおぜい で しょくじ に でかけました	
\\	で 食事[しょくじ]に 出[で]かけました。		
\\	姿	姿[すがた]	すがた	
\\	遠くに彼女の姿が見えた。	遠[とお]くに 彼女[かのじょ]の 姿[すがた]が 見[み]えた。	とおく に かのじょ の すがた が みえた	
\\	遠[とお]くに 彼女[かのじょ]の
\\	が 見[み]えた。		
\\	姿勢	姿勢[しせい]	しせい	
\\	あの子はいつも姿勢が悪い。	あの 子[こ]はいつも 姿勢[しせい]が 悪[わる]い。	あの こ は いつも しせい が わるい	
\\	あの 子[こ]はいつも
\\	が 悪[わる]い。		
\\	エアメール	エアメール	エアメール	
\\	彼にエアメールを送りました。	彼[かれ]にエアメールを 送[おく]りました。	かれ に えあめーる を おくりました	
\\	彼[かれ]に
\\	を 送[おく]りました。		
\\	恐ろしい	恐[おそ]ろしい	おそろしい	
\\	昨夜恐ろしい夢を見た。	昨夜[ゆうべ] 恐[おそ]ろしい 夢[ゆめ]を 見[み]た。	ゆうべ おそろしい ゆめ を みた	
\\	昨夜[ゆうべ]
\\	夢[ゆめ]を 見[み]た。		
\\	怖い	怖[こわ]い	こわい	
\\	私は犬が怖いです。	私[わたし]は 犬[いぬ]が 怖[こわ]いです。	わたし は いぬ が こわい です	
\\	私[わたし]は 犬[いぬ]が
\\	です。		
\\	糸	糸[いと]	いと	
\\	母は糸を針に通した。	母[はは]は 糸[いと]を 針[はり]に 通[とお]した。	はは は いと を はり に とおした	
\\	母[はは]は
\\	を 針[はり]に 通[とお]した。		
\\	孫	孫[まご]	まご	
\\	昨日、孫が生まれました。	昨日[きのう]、 孫[まご]が 生[う]まれました。	きのう まご が うまれました	
\\	昨日[きのう]、
\\	が 生[う]まれました。		
\\	シャープペンシル	シャープペンシル	シャープペンシル	
\\	私のシャープペンシルがありません。	私[わたし]のシャープペンシルがありません。	わたし の しゃーぷぺんしる が ありません	
\\	私[わたし]の
\\	がありません。		
\\	木綿	木綿[もめん]	もめん	
\\	彼女は木綿のシャツを着ています。	彼女[かのじょ]は 木綿[もめん]のシャツを 着[き]ています。	かのじょ は もめん の しゃつ を きて います	
\\	彼女[かのじょ]は
\\	のシャツを 着[き]ています。		
\\	偉い	偉[えら]い	えらい	
\\	彼は偉い学者です。	彼[かれ]は 偉[えら]い 学者[がくしゃ]です。	かれ は えらい がくしゃ です	
\\	彼[かれ]は
\\	学者[がくしゃ]です。		
\\	爪	爪[つめ]	つめ	
\\	爪が伸びています。	爪[つめ]が 伸[の]びています。	つめ が のびて います	
\\	が 伸[の]びています。		
\\	机	机[つくえ]	つくえ	
\\	新しい机を買ってもらいました。	新[あたら]しい 机[つくえ]を 買[か]ってもらいました。	あたらしい つくえ を かって もらいました	
\\	新[あたら]しい
\\	を 買[か]ってもらいました。		
\\	セロテープ	セロテープ	セロテープ	
\\	セロテープはありますか。	セロテープはありますか。	せろてーぷ は あります か	
\\	はありますか。		
\\	棚	棚[たな]	たな	
\\	大きな棚はとても便利です。	大[おお]きな 棚[たな]はとても 便利[べんり]です。	おおき な たな は とても べんり です	
\\	大[おお]きな
\\	はとても 便利[べんり]です。		
\\	本棚	本棚[ほんだな]	ほんだな	
\\	これはとても大きな本棚ですね。	これはとても 大[おお]きな 本棚[ほんだな]ですね。	これ は とても おおき な ほんだな です ね	
\\	これはとても 大[おお]きな
\\	ですね。		
\\	方針	方針[ほうしん]	ほうしん	
\\	今後の方針が決まった。	今後[こんご]の 方針[ほうしん]が 決[き]まった。	こんご の ほうしん が きまった	
\\	今後[こんご]の
\\	が 決[き]まった。		
\\	釣る	釣[つ]る	つる	
\\	昨日大きな魚を釣りました。	昨日大[きのう おお]きな 魚[さかな]を 釣[つ]りました。	きのう おおき な さかな を つりました	
\\	昨日大[きのう おお]きな 魚[さかな]を
\\	やけど	やけど	やけど	
\\	彼は手にやけどをしました。	彼[かれ]は 手[て]にやけどをしました。	かれ は て に やけど を しました	
\\	彼[かれ]は 手[て]に
\\	をしました。		
\\	お釣り	お 釣[つ]り	おつり	
\\	母はお釣りを確かめた。	母[はは]はお 釣[つ]りを 確[たし]かめた。	はは は おつり を たしかめた	
\\	母[はは]は
\\	を 確[たし]かめた。		
\\	鍵	鍵[かぎ]	かぎ	
\\	出かける時は鍵を掛けてください。	出[で]かける 時[とき]は 鍵[かぎ]を 掛[か]けてください。	でかける とき は かぎ を かけて ください	
\\	出[で]かける 時[とき]は
\\	を 掛[か]けてください。		
\\	鍋	鍋[なべ]	なべ	
\\	鍋にスープが入っているよ。	鍋[なべ]にスープが 入[はい]っているよ。	なべ に すーぷ が はいって いる よ	
\\	にスープが 入[はい]っているよ。		
\\	寿司	寿司[すし]	すし	
\\	彼女は寿司を初めて食べました。	彼女[かのじょ]は 寿司[すし]を 初[はじ]めて 食[た]べました。	かのじょ は すし を はじめて たべました 。	
\\	彼女[かのじょ]は
\\	を 初[はじ]めて 食[た]べました。		
\\	アイスコーヒー	アイスコーヒー	アイスコーヒー	
\\	アイスコーヒーをください。	アイスコーヒーをください。	あいすこーひー を ください	
\\	をください。		
\\	泥棒	泥棒[どろぼう]	どろぼう	
\\	近所に泥棒が入った。	近所[きんじょ]に 泥棒[どろぼう]が 入[はい]った。	きんじょ に どろぼう が はいった	
\\	近所[きんじょ]に
\\	が 入[はい]った。		
\\	湯	湯[ゆ]	ゆ	
\\	お風呂のお湯が一杯です。	お 風呂[ふろ]のお 湯[ゆ]が 一杯[いっぱい]です。	おふろ の おゆ が いっぱい です	
\\	お 風呂[ふろ]のお
\\	が 一杯[いっぱい]です。		
\\	沸く	沸[わ]く	わく	
\\	お風呂が沸きました。	お 風呂[ふろ]が 沸[わ]きました。	お ふろ が わきました。	
\\	お 風呂[ふろ]が
\\	沸かす	沸[わ]かす	わかす	
\\	お湯を沸かしてください。	お 湯[ゆ]を 沸[わ]かしてください。	おゆ を わかして ください	
\\	お 湯[ゆ]を
\\	ください。		
\\	あっち	あっち	あっち	
\\	郵便局はあっちです。	郵便局[ゆうびんきょく]はあっちです。	ゆうびんきょく は あっち です	
\\	郵便局[ゆうびんきょく]は
\\	です。		
\\	洗濯機	洗濯機[せんたくき]	せんたくき	
\\	新しい洗濯機を買いました。	新[あたら]しい 洗濯機[せんたっき]を 買[か]いました。	あたらしい せんたっき を かいました	
\\	新[あたら]しい
\\	を 買[か]いました。		
\\	洗濯	洗濯[せんたく]	せんたく	
\\	一週間、洗濯をしていない。	一週間[いっしゅうかん]、 洗濯[せんたく]をしていない。	いっしゅうかん せんたく を して いない	
\\	一週間[いっしゅうかん]、
\\	をしていない。		
\\	濡れる	濡[ぬ]れる	ぬれる	
\\	雨で濡れてしまった。	雨[あめ]で 濡[ぬ]れてしまった。	あめ で ぬれて しまった	
\\	雨[あめ]で
\\	喫茶店	喫茶店[きっさてん]	きっさてん	
\\	喫茶店でコーヒーを飲んだ。	喫茶店[きっさてん]でコーヒーを 飲[の]んだ。	きっさてん で こーひー を のんだ	
\\	でコーヒーを 飲[の]んだ。		
\\	あんなに	あんなに	あんなに	
\\	あんなにいい人はいません。	あんなにいい 人[ひと]はいません。	あんなに いい ひと は いません	
\\	いい 人[ひと]はいません。		
\\	怠ける	怠[なま]ける	なまける	
\\	(が), 
\\	(を)		
\\	怠けていないで、手伝って。	怠[なま]けていないで、 手伝[てつだ]って。	なまけて いない で てつだって	
\\	、 手伝[てつだ]って。		
\\	一生懸命	一生懸命[いっしょうけんめい]	いっしょうけんめい	
\\	彼は毎日一生懸命働いている。	彼[かれ]は 毎日[まいにち] 一生懸命[いっしょうけんめい] 働[はたら]いている。	かれ は まいにち いっしょうけんめい はたらいて いる	
\\	彼[かれ]は 毎日[まいにち]
\\	働[はたら]いている。		
\\	休憩	休憩[きゅうけい]	きゅうけい	
\\	少し休憩しましょう。	少[すこ]し 休憩[きゅうけい]しましょう。	すこし きゅうけい しましょう	
\\	少[すこ]し
\\	しましょう。		
\\	天井	天井[てんじょう]	てんじょう	
\\	この部屋は天井が高いですね。	この 部屋[へや]は 天井[てんじょう]が 高[たか]いですね。	この へや は てんじょう が たかい です ね	
\\	この 部屋[へや]は
\\	が 高[たか]いですね。		
\\	いたずら	いたずら	いたずら	
\\	その子はいたずらが大好きだ。	その 子[こ]はいたずらが 大好[だいす]きだ。	その こ は いたずら が だいすき だ	
\\	その 子[こ]は
\\	が 大好[だいす]きだ。		
\\	納豆	納豆[なっとう]	なっとう	
\\	私は納豆をよく食べます。	私[わたし]は 納豆[なっとう]をよく 食[た]べます。	わたし は なっとう を よく たべます	
\\	私[わたし]は
\\	をよく 食[た]べます。		
\\	化粧	化粧[けしょう]	けしょう	
\\	彼女は化粧が上手い。	彼女[かのじょ]は 化粧[けしょう]が 上手[うま]い。	かのじょ は けしょう が うまい	
\\	彼女[かのじょ]は
\\	が 上手[うま]い。		
\\	畑	畑[はたけ]	はたけ	
\\	母は畑仕事が好きだ。	母[はは]は 畑[はたけ] 仕事[しごと]が 好[す]きだ。	はは は はたけ しごと が すき だ	
\\	母[はは]は
\\	仕事[しごと]が 好[す]きだ。		
\\	炊く	炊[た]く	たく	
\\	母は毎朝ご飯を炊く。	母[はは]は 毎朝[まいあさ]ご 飯[はん]を 炊[た]く。	はは は まいあさ ごはん を たく	
\\	母[はは]は 毎朝[まいあさ]ご 飯[はん]を
\\	ウェートレス	ウェートレス	ウェートレス	
\\	ウェートレスを呼んだ。	ウェートレスを 呼[よ]んだ。	うぇーとれす を よんだ	
\\	を 呼[よ]んだ。		
\\	自炊	自炊[じすい]	じすい	
\\	私は毎日、自炊している。	私[わたし]は 毎日[まいにち]、 自炊[じすい]している。	わたし は まいにち じすい して いる	
\\	私[わたし]は 毎日[まいにち]、
\\	している。		
\\	灰皿	灰皿[はいざら]	はいざら	
\\	灰皿をください。	灰皿[はいざら]をください。	はいざら を ください	
\\	をください。		
\\	灰	灰[はい]	はい	
\\	火事ですべて灰になった。	火事[かじ]ですべて 灰[はい]になった。	かじ で すべて はい に なった	
\\	火事[かじ]ですべて
\\	になった。		
\\	灰色	灰色[はいいろ]	はいいろ	
\\	今日の空は灰色だ。	今日[きょう]の 空[そら]は 灰色[はいいろ]だ。	きょう の そら は はいいろ だ	
\\	今日[きょう]の 空[そら]は
\\	だ。		
\\	おばさん	おばさん	おばさん	
\\	おばさん一家が遊びに来ました。	おばさん 一家[いっか]が 遊[あそ]びに 来[き]ました。	おばさん いっか が あそび に きました	
\\	一家[いっか]が 遊[あそ]びに 来[き]ました。		
\\	西暦	西暦[せいれき]	せいれき	
\\	西暦1964年に東京でオリンピックがあった。	西暦[せいれき] 
\\	年[ねん]に 東京[とうきょう]でオリンピックがあった。	せいれき せんきゅうひゃくろくじゅうよねん に とうきょう で おりんぴっく が あった	
\\	1964年[せんきゅうひゃくろくじゅうよねん]に 東京[とうきょう]でオリンピックがあった。		
\\	毛	毛[け]	け	
\\	猫の毛がセーターに付いた。	猫[ねこ]の 毛[け]がセーターに 付[つ]いた。	ねこ の け が せーたー に ついた	
\\	猫[ねこ]の
\\	がセーターに 付[つ]いた。		
\\	髪	髪[かみ]	かみ	
\\	昨日、髪を切りました。	昨日[きのう]、 髪[かみ]を 切[き]りました。	きのう かみ を きりました	
\\	昨日[きのう]、
\\	を 切[き]りました。		
\\	舌	舌[した]	した	
\\	舌を出してください。	舌[した]を 出[だ]してください。	した を だして ください	
\\	を 出[だ]してください。		
\\	くし	くし	くし	
\\	彼はくしで髪をとかした。	彼[かれ]はくしで 髪[かみ]をとかした。	かれ は くし で かみ を とかした	
\\	彼[かれ]は
\\	で 髪[かみ]をとかした。		
\\	臭い	臭[くさ]い	くさい	
\\	納豆は臭い。	納豆[なっとう]は 臭[くさ]い。	なっとう は くさい	
\\	納豆[なっとう]は
\\	匂い	匂[にお]い	におい	
\\	いい匂いがする。	いい 匂[にお]いがする。	いい におい が する	
\\	いい
\\	がする。		
\\	居る	居[い]る	いる	
\\	今日は一日中家にいました。	今日[きょう]は 一日中家[いちにちじゅう うち]にいました。	きょう は いちにちじゅう うち に いました	
\\	今日[きょう]は 一日中家[いちにちじゅう うち]に
\\	履く	履[は]く	はく	
\\	彼女はブーツを履いています。	彼女[かのじょ]はブーツを 履[は]いています。	かのじょ は ぶーつ を はいて います	
\\	彼女[かのじょ]はブーツを
\\	クリーニング	クリーニング	クリーニング	
\\	スーツをクリーニングに出しました。	スーツをクリーニングに 出[だ]しました。	すーつ を くりーにんぐ に だしました	
\\	スーツを
\\	に 出[だ]しました。		
\\	戸	戸[と]	と	
\\	部屋の戸が開いています。	部屋[へや]の 戸[と]が 開[あ]いています。	へや の と が あいて います	
\\	部屋[へや]の
\\	が 開[あ]いています。		
\\	扇風機	扇風機[せんぷうき]	せんぷうき	
\\	暑いから扇風機をつけよう。	暑[あつ]いから 扇風機[せんぷうき]をつけよう。	あつい から せんぷうき を つけよう	
\\	暑[あつ]いから
\\	をつけよう。		
\\	豚	豚[ぶた]	ぶた	
\\	豚はすごく鼻が良い。	豚[ぶた]はすごく 鼻[はな]が 良[い]い。	ぶた は すごく はな が いい	
\\	はすごく 鼻[はな]が 良[い]い。		
\\	豚肉	豚肉[ぶたにく]	ぶたにく	
\\	豚肉は美味しいです。	豚肉[ぶたにく]は 美味[おい]しいです。	ぶたにく は おいしい です	
\\	は 美味[おい]しいです。		
\\	ごちそう	ごちそう	ごちそう	
\\	テーブルの上にごちそうが並んでいる。	テーブルの 上[うえ]にごちそうが 並[なら]んでいる。	てーぶる の うえ に ごちそう が ならんで いる	
\\	テーブルの 上[うえ]に
\\	が 並[なら]んでいる。		
\\	鶏	鶏[にわとり]	にわとり	
\\	その鶏は毎朝鳴きます。	その 鶏[にわとり]は 毎朝鳴[まいあさ な]きます。	その にわとり は まいあさ なきます	
\\	その
\\	は 毎朝鳴[まいあさ な]きます。		
\\	腕時計	腕時計[うでどけい]	うでどけい	
\\	私の腕時計は遅れている。	私[わたし]の 腕時計[うでどけい]は 遅[おく]れている。	わたし の うでどけい は おくれて いる	
\\	私[わたし]の
\\	は 遅[おく]れている。		
\\	お菓子	お 菓子[かし]	おかし	
\\	みんなにお菓子をあげましょう。	みんなにお 菓子[かし]をあげましょう。	みんな に おかし を あげましょう	
\\	みんなに
\\	をあげましょう。		
\\	お辞儀	お 辞儀[じぎ]	おじぎ	
\\	皆、社長にお辞儀をした。	皆[みんな]、 社長[しゃちょう]にお 辞儀[じぎ]をした。	みんな しゃちょう に おじぎ を した	
\\	皆[みんな]、 社長[しゃちょう]に
\\	をした。		
\\	ごちそうする	ごちそうする	ごちそうする	
\\	今日の夕食は私がごちそうします。	今日[きょう]の 夕食[ゆうしょく]は 私[わたし]がごちそうします。	きょう の ゆうしょく は わたし が ごちそう します	
\\	今日[きょう]の 夕食[ゆうしょく]は 私[わたし]が
\\	寝坊	寝坊[ねぼう]	ねぼう	
\\	今朝は寝坊しました。	今朝[けさ]は 寝坊[ねぼう]しました。	けさ は ねぼう しました	
\\	今朝[けさ]は
\\	しました。		
\\	旗	旗[はた]	はた	
\\	旗が風に揺れている。	旗[はた]が 風[かぜ]に 揺[ゆ]れている。	はた が かぜ に ゆれて いる	
\\	が 風[かぜ]に 揺[ゆ]れている。		
\\	鉛筆	鉛筆[えんぴつ]	えんぴつ	
\\	鉛筆を貸して下さい。	鉛筆[えんぴつ]を 貸[か]して 下[くだ]さい。	えんぴつ を かして ください	
\\	を 貸[か]して 下[くだ]さい。		
\\	万年筆	万年筆[まんねんひつ]	まんねんひつ	
\\	父に万年筆をプレゼントしました。	父[ちち]に 万年筆[まんねんひつ]をプレゼントしました。	ちち に まんねんひつ を ぷれぜんと しました	
\\	父[ちち]に
\\	をプレゼントしました。		
\\	こぼれる	こぼれる	こぼれる	
\\	彼女の目から涙がこぼれた。	彼女[かのじょ]の 目[め]から 涙[なみだ]がこぼれた。	かのじょ の め から なみだ が こぼれた	
\\	彼女[かのじょ]の 目[め]から 涙[なみだ]が
\\	箱	箱[はこ]	はこ	
\\	この箱は重い。	この 箱[はこ]は 重[おも]い。	この はこ は おもい	
\\	この
\\	は 重[おも]い。		
\\	本箱	本箱[ほんばこ]	ほんばこ	
\\	雑誌を本箱に入れました。	雑誌[ざっし]を 本箱[ほんばこ]に 入[い]れました。	ざっし を ほんばこ に いれました	
\\	雑誌[ざっし]を
\\	に 入[い]れました。		
\\	手袋	手袋[てぶくろ]	てぶくろ	
\\	寒いので手袋をしました。	寒[さむ]いので 手袋[てぶくろ]をしました。	さむい の で てぶくろ を しました	
\\	寒[さむ]いので
\\	をしました。		
\\	袋	袋[ふくろ]	ふくろ	
\\	袋はいりません。	袋[ふくろ]はいりません。	ふくろ は いりません	
\\	はいりません。		
\\	こんなに	こんなに	こんなに	
\\	こんなに面白い本は初めて読んだ。	こんなに 面白[おもしろ]い 本[ほん]は 初[はじ]めて 読[よ]んだ。	こんなに おもしろい ほん は はじめて よんだ	
\\	面白[おもしろ]い 本[ほん]は 初[はじ]めて 読[よ]んだ。		
\\	財布	財布[さいふ]	さいふ	
\\	会社に財布を忘れた。	会社[かいしゃ]に 財布[さいふ]を 忘[わす]れた。	かいしゃ に さいふ を わすれた	
\\	会社[かいしゃ]に
\\	を 忘[わす]れた。		
\\	毛布	毛布[もうふ]	もうふ	
\\	この毛布は暖かい。	この 毛布[もうふ]は 暖[あたた]かい。	この もうふ は あたたかい	
\\	この
\\	は 暖[あたた]かい。		
\\	布団	布団[ふとん]	ふとん	
\\	母が布団を干している。	母[はは]が 布団[ふとん]を 干[ほ]している。	はは が ふとん を ほして いる	
\\	母[はは]が
\\	を 干[ほ]している。		
\\	小包	小包[こづつみ]	こづつみ	
\\	フランスの友達から小包が届いた。	フランスの 友達[ともだち]から 小包[こづつみ]が 届[とど]いた。	ふらんす の ともだち から こづつみ が とどいた	
\\	フランスの 友達[ともだち]から
\\	が 届[とど]いた。		
\\	ざあざあ	ざあざあ	ざあざあ	
\\	雨がざあざあ降っている。	雨[あめ]がざあざあ 降[ふ]っている。	あめ が ざあざあ ふって いる	
\\	雨[あめ]が
\\	降[ふ]っている。		
\\	包む	包[つつ]む	つつむ	
\\	プレゼントをきれいな紙で包みました。	プレゼントをきれいな 紙[かみ]で 包[つつ]みました。	ぷれぜんと を きれい な かみ で つつみました	
\\	プレゼントをきれいな 紙[かみ]で
\\	飾る	飾[かざ]る	かざる	
\\	テーブルの上に花を飾りました。	テーブルの 上[うえ]に 花[はな]を 飾[かざ]りました。	てーぶる の うえ に はな を かざりました	
\\	テーブルの 上[うえ]に 花[はな]を
\\	手帳	手帳[てちょう]	てちょう	
\\	新しい手帳を買いました。	新[あたら]しい 手帳[てちょう]を 買[か]いました。	あたらしい てちょう を かいました	
\\	新[あたら]しい
\\	を 買[か]いました。		
\\	電話帳	電話帳[でんわちょう]	でんわちょう	
\\	電話帳を見せてください。	電話帳[でんわちょう]を 見[み]せてください。	でんわちょう を みせて ください	
\\	を 見[み]せてください。		
\\	ジャガイモ	ジャガイモ	ジャガイモ	
\\	ポテトサラダを作るから、ジャガイモを買ってきて。	ポテトサラダを 作[つく]るから、ジャガイモを 買[か]ってきて。	ぽてと さらだ を つくる から じゃがいも を かって きて	
\\	ポテトサラダを 作[つく]るから、
\\	を 買[か]ってきて。		
\\	牛乳	牛乳[ぎゅうにゅう]	ぎゅうにゅう	
\\	私は毎朝、牛乳を飲む。	私[わたし]は 毎朝[まいあさ]、 牛乳[ぎゅうにゅう]を 飲[の]む。	わたし は まいあさ ぎゅうにゅう を のむ	
\\	私[わたし]は 毎朝[まいあさ]、
\\	を 飲[の]む。		
\\	玄関	玄関[げんかん]	げんかん	
\\	玄関に花を飾りました。	玄関[げんかん]に 花[はな]を 飾[かざ]りました。	げんかん に はな を かざりました	
\\	に 花[はな]を 飾[かざ]りました。		
\\	眼鏡	眼鏡[めがね]	めがね	
\\	彼は眼鏡をかけています。	彼[かれ]は 眼鏡[めがね]をかけています。	かれ は めがね を かけて います	
\\	彼[かれ]は
\\	をかけています。		
\\	眠い	眠[ねむ]い	ねむい	
\\	今日はとても眠いです。	今日[きょう]はとても 眠[ねむ]いです。	きょう は とても ねむい です	
\\	今日[きょう]はとても
\\	です。		
\\	ソファ	ソファ	ソファ	
\\	このソファは気持ちがいい。	このソファは 気持[きも]ちがいい。	この そふぁ は きもち が いい	
\\	この
\\	は 気持[きも]ちがいい。		
\\	眠る	眠[ねむ]る	ねむる	
\\	昨日は8時間眠りました。	昨日[きのう]は 8時間[はちじかん] 眠[ねむ]りました。	きのう は はちじかん ねむりました	
\\	昨日[きのう]は 8時間[はちじかん]
\\	封筒	封筒[ふうとう]	ふうとう	
\\	その手紙を封筒に入れた。	その 手紙[てがみ]を 封筒[ふうとう]に 入[い]れた。	その てがみ を ふうとう に いれた	
\\	その 手紙[てがみ]を
\\	に 入[い]れた。		
\\	出かける	出[で]かける	でかける	
\\	主人はもう出かけました。	主人[しゅじん]はもう 出[で]かけました。	しゅじん は もう でかけました	
\\	主人[しゅじん]はもう
\\	かかる	かかる	かかる	
\\	など) が		
\\	壁に大きな時計がかかっています。	壁[かべ]に 大[おお]きな 時計[とけい]がかかっています。	かべ に おおき な とけい が かかって います	
\\	壁[かべ]に 大[おお]きな 時計[とけい]が
\\	煙草	煙草[たばこ]	たばこ	
\\	彼は煙草を吸いません。	彼[かれ]は 煙草[たばこ]を 吸[す]いません。	かれ は たばこ を すいません	
\\	彼[かれ]は
\\	を 吸[す]いません。		
\\	掛け算	掛[か]け 算[ざん]	かけざん	
\\	弟は掛け算を習っている。	弟[おとうと]は 掛[か]け 算[ざん]を 習[なら]っている。	おとうと は かけざん を ならって いる	
\\	弟[おとうと]は
\\	を 習[なら]っている。		
\\	拍手	拍手[はくしゅ]	はくしゅ	
\\	大きな拍手が上がった。	大[おお]きな 拍手[はくしゅ]が 上[あ]がった。	おおき な はくしゅ が あがった	
\\	大[おお]きな
\\	が 上[あ]がった。		
\\	掃除	掃除[そうじ]	そうじ	
\\	週末は部屋の掃除をしました。	週末[しゅうまつ]は 部屋[へや]の 掃除[そうじ]をしました。	しゅうまつ は へや の そうじ を しました	
\\	週末[しゅうまつ]は 部屋[へや]の
\\	をしました。		
\\	掃く	掃[は]く	はく	
\\	床をほうきで掃きました。	床[ゆか]をほうきで 掃[は]きました。	ゆか を ほうき で はきました	
\\	床[ゆか]をほうきで
\\	ちょうど	ちょうど	ちょうど	
\\	値段はちょうど3万円です。	値段[ねだん]はちょうど 3万円[さんまんえん]です。	ねだん は ちょうど さんまんえん です	
\\	値段[ねだん]は
\\	3万円[さんまんえん]です。		
\\	掃除機	掃除機[そうじき]	そうじき	
\\	掃除機が壊れた。	掃除機[そうじき]が 壊[こわ]れた。	そうじき が こわれた	
\\	が 壊[こわ]れた。		
\\	握る	握[にぎ]る	にぎる	
\\	少女は母親の手を握った。	少女[しょうじょ]は 母親[ははおや]の 手[て]を 握[にぎ]った。	しょうじょ は ははおや の て を にぎった	
\\	少女[しょうじょ]は 母親[ははおや]の 手[て]を
\\	握手	握手[あくしゅ]	あくしゅ	
\\	彼らは握手をした。	彼[かれ]らは 握手[あくしゅ]をした。	かれら は あくしゅ を した	
\\	彼[かれ]らは
\\	をした。		
\\	迎える	迎[むか]える	むかえる	
\\	姉が空港まで迎えに来てくれます。	姉[あね]が 空港[くうこう]まで 迎[むか]えに 来[き]てくれます。	あね が くうこう まで むかえに きて くれます	
\\	姉[あね]が 空港[くうこう]まで
\\	に 来[き]てくれます。		
\\	ちょっと	ちょっと	ちょっと	
\\	もうちょっと塩を入れて。	もうちょっと 塩[しお]を 入[い]れて。	もう ちょっと しお を いれて	
\\	もう
\\	塩[しお]を 入[い]れて。		
\\	お巡りさん	お 巡[まわ]りさん	おまわりさん	
\\	あのお巡りさんに道を聞きましょう。	あのお 巡[まわ]りさんに 道[みち]を 聞[き]きましょう。	あの おまわりさん に みち を ききましょう	
\\	あの
\\	に 道[みち]を 聞[き]きましょう。		
\\	いくつ	いくつ	いくつ	
\\	娘さんはいくつになりましたか。	娘[むすめ]さんはいくつになりましたか。	むすめさん は いくつ に なりました か	
\\	娘[むすめ]さんは
\\	になりましたか。		
\\	幾ら	幾[いく]ら	いくら	
\\	この靴は幾らですか。	この 靴[くつ]は 幾[いく]らですか。	この くつ は いくら です か	
\\	この 靴[くつ]は
\\	ですか。		
\\	冗談	冗談[じょうだん]	じょうだん	
\\	冗談は止めてください。	冗談[じょうだん]は 止[や]めてください。	じょうだん は やめて ください	
\\	は 止[や]めてください。		
\\	つく	つく	つく	
\\	部屋の電気がついています。	部屋[へや]の 電気[でんき]がついています。	へや の でんき が ついて います	
\\	部屋[へや]の 電気[でんき]が
\\	穴	穴[あな]	あな	
\\	靴下に穴が開いている。	靴下[くつした]に 穴[あな]が 開[あ]いている。	くつした に あな が あいて いる	
\\	靴下[くつした]に
\\	が 開[あ]いている。		
\\	寂しい	寂[さび]しい	さびしい	
\\	これは寂しい曲ですね。	これは 寂[さび]しい 曲[きょく]ですね。	これ は さびしい きょく です ね	
\\	これは
\\	曲[きょく]ですね。		
\\	丁寧	丁寧[ていねい]	ていねい	
\\	彼女はいつも丁寧に仕事をします。	彼女[かのじょ]はいつも 丁寧[ていねい]に 仕事[しごと]をします。	かのじょ は いつも ていねい に しごと を します	
\\	彼女[かのじょ]はいつも
\\	に 仕事[しごと]をします。		
\\	かわいそう	可哀相[かわいそう]	かわいそう	
\\	そのかわいそうな子供たちは食べるものがない。	そのかわいそうな 子供[こども]たちは 食[た]べるものがない。	そのかわいそう な こどもたち は たべる もの が ない	
\\	その
\\	な 子供[こども]たちは 食[た]べるものがない。		
\\	てんぷら	てんぷら	てんぷら	
\\	私はえびのてんぷらが好きです。	私[わたし]はえびのてんぷらが 好[す]きです。	わたし は えび の てんぷら が すき です	
\\	私[わたし]はえびの
\\	が 好[す]きです。		
\\	怪我	怪我[けが]	けが	
\\	彼女は腕を怪我した。	彼女[かのじょ]は 腕[うで]を 怪我[けが]した。	かのじょ は うで を けがした	
\\	彼女[かのじょ]は 腕[うで]を
\\	した。		
\\	我慢	我慢[がまん]	がまん	
\\	彼のわがままには我慢できません。	彼[かれ]のわがままには 我慢[がまん]できません。	かれ の わがまま に は がまん できません	
\\	彼[かれ]のわがままには
\\	できません。		
\\	幼稚園	幼稚園[ようちえん]	ようちえん	
\\	娘は幼稚園に通っています。	娘[むすめ]は 幼稚園[ようちえん]に 通[かよ]っています。	むすめ は ようちえん に かよって います	
\\	娘[むすめ]は
\\	に 通[かよ]っています。		
\\	隠れる	隠[かく]れる	かくれる	
\\	太陽が雲に隠れた。	太陽[たいよう]が 雲[くも]に 隠[かく]れた。	たいよう が くも に かくれた	
\\	太陽[たいよう]が 雲[くも]に
\\	とんとん	とんとん	とんとん	
\\	母の肩をとんとんたたいた。	母[はは]の 肩[かた]をとんとんたたいた。	はは の かた を とんとん たたいた	
\\	母[はは]の 肩[かた]を
\\	たたいた。		
\\	日陰	日陰[ひかげ]	ひかげ	
\\	暑いので日陰で休みましょう。	暑[あつ]いので 日陰[ひかげ]で 休[やす]みましょう。	あつい の で ひかげ で やすみましょう	
\\	暑[あつ]いので
\\	で 休[やす]みましょう。		
\\	随分	随分[ずいぶん]	ずいぶん	
\\	随分[ずいぶん]早く終わったね。	随分[ずいぶん] 早[はや]く 終[お]わったね。	ずいぶん はやく おわった ね	
\\	早[はや]く 終[お]わったね。		
\\	頑張る	頑張[がんば]る	がんばる	
\\	明日の試験、頑張ってね。	明日[あした]の 試験[しけん]、 頑張[がんば]ってね。	あした の しけん がんばって ね	
\\	明日[あした]の 試験[しけん]、
\\	ね。		
\\	頃	頃[ころ]	ころ	
\\	子供の頃、よくその公園で遊んだ。	子供[こども]の 頃[ころ]、よくその 公園[こうえん]で 遊[あそ]んだ。	こども の ころ よく その こうえん で あそんだ	
\\	子供[こども]の
\\	、よくその 公園[こうえん]で 遊[あそ]んだ。		
\\	人参	人参[にんじん]	にんじん	
\\	私は人参が嫌いだ。	私[わたし]は人参[にんじん]が 嫌[きら]いだ。	わたし は にんじん が きらい だ	
\\	私[わたし]は
\\	が 嫌[きら]いだ。		
\\	この頃	この 頃[ごろ]	このごろ	
\\	この頃、よく肩が凝る。	この 頃[ごろ]、よく 肩[かた]が 凝[こ]る。	このごろ よく かた が こる	
\\	、よく 肩[かた]が 凝[こ]る。		
\\	消防署	消防署[しょうぼうしょ]	しょうぼうしょ	
\\	この町には消防署が1つしかありません。	この 町[まち]には 消防署[しょうぼうしょ]が 1[ひと]つしかありません。	この まち に は しょうぼうしょ が ひとつ しか ありません	
\\	この 町[まち]には
\\	が 1[ひと]つしかありません。		
\\	尋ねる	尋[たず]ねる	たずねる	
\\	彼女は道を尋ねた。	彼女[かのじょ]は 道[みち]を 尋[たず]ねた。	かのじょ は みち を たずねた	
\\	彼女[かのじょ]は 道[みち]を
\\	缶詰	缶詰[かんづめ]	かんづめ	
\\	缶詰のフルーツはとても甘い。	缶詰[かんづめ]のフルーツはとても 甘[あま]い。	かんづめ の ふるーつ は とても あまい	
\\	のフルーツはとても 甘[あま]い。		
\\	ぬるい	ぬるい	ぬるい	
\\	風呂がぬるかった。	風呂[ふろ]がぬるかった。	ふろ が ぬるかった	
\\	風呂[ふろ]が
\\	缶	缶[かん]	かん	
\\	コーヒーはその缶に入っています。	コーヒーはその 缶[かん]に 入[はい]っています。	こーひー は その かん に はいって います	
\\	コーヒーはその
\\	に 入[はい]っています。		
\\	腐る	腐[くさ]る	くさる	
\\	リンゴが腐っている。	リンゴが 腐[くさ]っている。	りんご が くさって いる	
\\	リンゴが
\\	豆腐	豆腐[とうふ]	とうふ	
\\	私は毎日豆腐を食べます。	私[わたし]は 毎日[まいにち] 豆腐[とうふ]を 食[た]べます。	わたし は まいにち とうふ を たべます	
\\	私[わたし]は 毎日[まいにち]
\\	を 食[た]べます。		
\\	床	床[ゆか]	ゆか	
\\	床が濡れている。	床[ゆか]が 濡[ぬ]れている。	ゆか が ぬれて いる	
\\	がぬれている。		
\\	パーセント	パーセント	パーセント	
\\	この村の80パーセントは老人です。	この 村[むら]の 80[はちじゅっ]パーセントは 老人[ろうじん]です。	この むら の はちじゅっぱーせんと は ろうじん です	
\\	この 村[むら]の 80[はちじゅっ]
\\	は 老人[ろうじん]です。		
\\	床屋	床屋[とこや]	とこや	
\\	昨日床屋で髪を切った。	昨日[きのう] 床屋[とこや]で 髪[かみ]を 切[き]った。	きのう とこや で かみ を きった	
\\	昨日[きのう]
\\	で 髪[かみ]を 切[き]った。		
\\	畳	畳[たたみ]	たたみ	
\\	そのホテルには畳の部屋がある。	そのホテルには 畳[たたみ]の 部屋[へや]がある。	その ほてる に は たたみ の へや が ある	
\\	そのホテルには
\\	の 部屋[へや]がある。		
\\	畳む	畳[たた]む	たたむ	
\\	布団を畳んでください。	布団[ふとん]を 畳[たた]んでください。	ふとん を たたんで ください	
\\	布団[ふとん]を
\\	ください。		
\\	干す	干[ほ]す	ほす	
\\	母は洗濯物を干しています。	母[はは]は 洗濯物[せんたくもの]を 干[ほ]しています。	はは は せんたくもの を ほして います	
\\	母[はは]は 洗濯物[せんたくもの]を
\\	います。		
\\	ひざ	ひざ	ひざ	
\\	スキーでひざを怪我しました。	スキーでひざを 怪我[けが]しました。	すきー で ひざ を けがしました	
\\	スキーで
\\	を 怪我[けが]しました。		
\\	帽子	帽子[ぼうし]	ぼうし	
\\	暑いので帽子を被りましょう。	暑[あつ]いので 帽子[ぼうし]を 被[かぶ]りましょう。	あつい の で ぼうし を かぶりましょう	
\\	暑[あつ]いので
\\	を 被[かぶ]りましょう。		
\\	ぜひ	ぜひ	ぜひ	
\\	ぜひ、うちに来てください。	是非[ぜひ]、うちに 来[き]てください。	ぜひ うち に きて ください	
\\	、うちに 来[き]てください。		
\\	敬語	敬語[けいご]	けいご	
\\	お客様には敬語を使いなさい。	お 客様[きゃくさま]には 敬語[けいご]を 使[つか]いなさい。	おきゃくさま に は けいご を つかいなさい	
\\	お 客様[きゃくさま]には
\\	を 使[つか]いなさい。		
\\	尊敬	尊敬[そんけい]	そんけい	
\\	祖父は家族みんなに尊敬されています。	祖父[そふ]は 家族[かぞく]みんなに 尊敬[そんけい]されています。	そふ は かぞく みんな に そんけい されて います	
\\	祖父[そふ]は 家族[かぞく]みんなに
\\	されています。		
\\	ひじ	ひじ	ひじ	
\\	彼はひじに怪我をした。	彼[かれ]はひじに 怪我[けが]をした。	かれ は ひじ に けが を した	
\\	彼[かれ]は
\\	に 怪我[けが]をした。		
\\	敷く	敷[し]く	しく	
\\	生まれて初めて布団を敷いた。	生[う]まれて 初[はじ]めて 布団[ふとん]を 敷[し]いた。	うまれて はじめて ふとん を しいた	
\\	生[う]まれて 初[はじ]めて 布団[ふとん]を
\\	雷	雷[かみなり]	かみなり	
\\	雷が鳴っています。	雷[かみなり]が 鳴[な]っています。	かみなり が なって います	
\\	が 鳴[な]っています。		
\\	零	零[れい]	れい	
\\	今ちょうど零時です。	今[いま]ちょうど 零[れい] 時[じ]です。	いま ちょうど れいじ です	
\\	今[いま]ちょうど
\\	時[じ]です。		
\\	仕舞う	仕舞[しま]う	しまう	
\\	彼女は大切な書類を机に仕舞った。	彼女[かのじょ]は 大切[たいせつ]な 書類[しょるい]を 机[つくえ]に 仕舞[しま]った。	かのじょ は たいせつ な しょるい を つくえ に しまった。	
\\	彼女[かのじょ]は 大切[たいせつ]な 書類[しょるい]を 机[つくえ]に
\\	ひも	ひも	ひも	
\\	靴のひもが切れた。	靴[くつ]のひもが 切[き]れた。	くつ の ひも が きれた	
\\	靴[くつ]の
\\	が 切[き]れた。		
\\	踊る	踊[おど]る	おどる	
\\	彼女はクラブで踊るのが好きです。	彼女[かのじょ]はクラブで 踊[おど]るのが 好[す]きです。	かのじょ は くらぶ で おどる の が すき です	
\\	彼女[かのじょ]はクラブで
\\	のが 好[す]きです。		
\\	踊り	踊[おど]り	おどり	
\\	この踊りは易しいですよ。	この 踊[おど]りは 易[やさ]しいですよ。	この おどり は やさしい です よ	
\\	この
\\	は 易[やさ]しいですよ。		
\\	踏む	踏[ふ]む	ふむ	
\\	運転手がブレーキを踏んだ。	運転手[うんてんしゅ]がブレーキを 踏[ふ]んだ。	うんてんしゅ が ぶれーき を ふんだ	
\\	運転手[うんてんしゅ]がブレーキを
\\	踏切	踏切[ふみきり]	ふみきり	
\\	その踏切は長い。	その 踏切[ふみきり]は 長[なが]い。	その ふみきり は ながい	
\\	その
\\	は 長[なが]い。		
\\	まく	まく	まく	
\\	日本では、二月に豆をまく行事がある。	日本[にほん]では、 二月[にがつ]に 豆[まめ]をまく 行事[ぎょうじ]がある。	にほん で は にがつ に まめ を まく ぎょうじ が ある	
\\	日本[にほん]では、 二月[にがつ]に 豆[まめ]を
\\	行事[ぎょうじ]がある。		
\\	蹴る	蹴[け]る	ける	
\\	ゴールキーパーがボールを蹴った。	ゴールキーパーがボールを 蹴[け]った。	ごーるきーぱー が ぼーる を けった。	
\\	ゴールキーパーがボールを
\\	食堂	食堂[しょくどう]	しょくどう	
\\	大学の食堂は安い。	大学[だいがく]の 食堂[しょくどう]は 安[やす]い。	だいがく の しょくどう は やすい	
\\	大学[だいがく]の
\\	は 安[やす]い。		
\\	猫	猫[ねこ]	ねこ	
\\	私は猫が大好きです。	私[わたし]は 猫[ねこ]が 大好[だいす]きです。	わたし は ねこ が だいすき です	
\\	私[わたし]は
\\	が 大好[だいす]きです。		
\\	文章	文章[ぶんしょう]	ぶんしょう	
\\	彼は文章がとてもうまい。	彼[かれ]は 文章[ぶんしょう]がとてもうまい。	かれ は ぶんしょう が とても うまい	
\\	彼[かれ]は
\\	がとてもうまい。		
\\	やかん	やかん	やかん	
\\	やかんでお湯を沸かしました。	やかんでお 湯[ゆ]を 沸[わ]かしました。	やかん で おゆ を わかしました	
\\	でお 湯[ゆ]を 沸[わ]かしました。		
\\	丈夫	丈夫[じょうぶ]	じょうぶ	
\\	祖母は身体が丈夫だ。	祖母[そぼ]は 身体[からだ]が 丈夫[じょうぶ]だ。	そぼ は からだ が じょうぶ だ	
\\	祖母[そぼ]は 身体[からだ]が
\\	だ。		
\\	また	また	また	
\\	明日、また来ます。	明日[あした]、また 来[き]ます。	あした また きます	
\\	明日[あした]、
\\	来[き]ます。		
\\	お祖父さん	お 祖父[じい]さん	おじいさん	
\\	私のお祖父さんは毎日散歩します。	私[わたし]のお 祖父[じい]さんは 毎日散歩[まいにち さんぽ]します。	わたし の おじいさん は まいにち さんぽ します 。	
\\	私[わたし]の
\\	は 毎日散歩[まいにち さんぽ]します。		
\\	お祖母さん	お 祖母[ばあ]さん	おばあさん	
\\	彼女はお祖母さんと住んでいる。	彼女[かのじょ]はお 祖母[ばあ]さんと 住[す]んでいる。	かのじょ は おばあさん と すん でいる 。	
\\	彼女[かのじょ]は
\\	と 住[す]んでいる。		
\\	ようこそ	ようこそ	ようこそ	
\\	日本へようこそ。	日本[にっぽん]へようこそ。	にっぽん へ ようこそ	
\\	日本[にっぽん]へ
\\	祖父	祖父[そふ]	そふ	
\\	祖父は元気です。	祖父[そふ]は 元気[げんき]です。	そふ は げんき です	
\\	は 元気[げんき]です。		
\\	祖母	祖母[そぼ]	そぼ	
\\	祖母は京都で生まれました。	祖母[そぼ]は 京都[きょうと]で 生[う]まれました。	そぼ は きょうと で うまれました	
\\	は 京都[きょうと]で 生[う]まれました。		
\\	邪魔	邪魔[じゃま]	じゃま	
\\	邪魔です、どいてください。	邪魔[じゃま]です、どいてください。	じゃま です どいて ください	
\\	です、どいてください。		
\\	風邪薬	風邪薬[かぜぐすり]	かぜぐすり	
\\	この風邪薬を飲みなさい。	この 風邪薬[かぜぐすり]を 飲[の]みなさい。	この かぜぐすり を のみなさい	
\\	この
\\	を 飲[の]みなさい。		
\\	ウェーター	ウェーター	ウェーター	
\\	ウェーターが水を運んできました。	ウェーターが 水[みず]を 運[はこ]んできました。	うぇーたー が みず を はこんで きました	
\\	が 水[みず]を 運[はこ]んできました。		
\\	風呂屋	風呂屋[ふろや]	ふろや	
\\	昨日、友達とお風呂屋さんに行った。	昨日[きのう]、 友達[ともだち]とお 風呂屋[ふろや]さんに 行[い]った。	きのう ともだち と おふろやさん に いった	
\\	昨日[きのう]、 友達[ともだち]とお
\\	さんに 行[い]った。		
\\	風呂	風呂[ふろ]	ふろ	
\\	父は今お風呂に入っています。	父[ちち]は 今[いま]お 風呂[ふろ]に 入[はい]っています。	ちち は いま お ふろ に はいって います	
\\	父[ちち]は 今[いま]お
\\	に 入[はい]っています。		
\\	昭和	昭和[しょうわ]	しょうわ	
\\	私の両親は昭和生まれです。	私[わたし]の 両親[りょうしん]は 昭和[しょうわ] 生[う]まれです。	わたし の りょうしん は しょうわうまれ です	
\\	私[わたし]の 両親[りょうしん]は
\\	生[う]まれです。		
\\	紫	紫[むらさき]	むらさき	
\\	彼女は紫のドレスを着ていた。	彼女[かのじょ]は 紫[むらさき]のドレスを 着[き]ていた。	かのじょ は むらさき の どれす を きて いた	
\\	彼女[かのじょ]は
\\	のドレスを 着[き]ていた。		
\\	そんなに	そんなに	そんなに	
\\	一度にそんなにたくさんはできない。	一度[いちど]にそんなにたくさんはできない。	いちど に そんなに たくさん は できない	
\\	一度[いちど]に
\\	たくさんはできない。		
\\	紅茶	紅茶[こうちゃ]	こうちゃ	
\\	温かい紅茶が飲みたい。	温[あたた]かい 紅茶[こうちゃ]が 飲[の]みたい。	あたたかい こうちゃ が のみたい	
\\	温[あたた]かい
\\	が 飲[の]みたい。		
\\	梅雨	梅雨[つゆ]	つゆ	
\\	梅雨は6月頃です。	梅雨[つゆ]は 6月頃[ろくがつごろ]です。	つゆ は ろくがつごろ です	
\\	は 6月頃[ろくがつごろ]です。		
\\	桃	桃[もも]	もも	
\\	私の一番好きな果物は桃です。	私[わたし]の 一番好[いちばん す]きな 果物[くだもの]は 桃[もも]です。	わたし の いちばん すき な くだもの は もも です	
\\	私[わたし]の 一番好[いちばん す]きな 果物[くだもの]は
\\	です。		
\\	遭う	遭[あ]う	あう	
\\	彼は交通事故に遭った。	彼[かれ]は 交通事故[こうつう じこ]に 遭[あ]った。	かれ は こうつう じこ に あった	
\\	彼[かれ]は 交通事故[こうつう じこ]に
\\	枕	枕[まくら]	まくら	
\\	私は低い枕が好きです。	私[わたし]は 低[ひく]い 枕[まくら]が 好[す]きです。	わたし は ひくい まくら が すき です	
\\	私[わたし]は 低[ひく]い
\\	が 好[す]きです。		
\\	嘘	嘘[うそ]	うそ	
\\	嘘をついてはいけません。	嘘[うそ]をついてはいけません。	うそ を ついて は いけません	
\\	をついてはいけません。		
\\	遠慮	遠慮[えんりょ]	えんりょ	
\\	私は遠慮します。	私[わたし]は 遠慮[えんりょ]します。	わたし は えんりょ します	
\\	私[わたし]は
\\	します。		
\\	叱る	叱[しか]る	しかる	
\\	父親が子供を叱っている。	父親[ちちおや]が 子供[こども]を 叱[しか]っている。	ちちおや が こども を しかって いる	
\\	父親[ちちおや]が 子供[こども]を
\\	傘	傘[かさ]	かさ	
\\	電車に傘を忘れた。	電車[でんしゃ]に 傘[かさ]を 忘[わす]れた。	でんしゃ に かさ を わすれた	
\\	電車[でんしゃ]に
\\	を 忘[わす]れた。		
\\	おじょうさん	おじょうさん	おじょうさん	
\\	おじょうさんはおいくつですか。	おじょうさんはおいくつですか。	おじょうさん は おいくつ です か	
\\	はおいくつですか。		
\\	年賀状	年賀状[ねんがじょう]	ねんがじょう	
\\	昨日、年賀状を出しました。	昨日[きのう]、 年賀状[ねんがじょう]を 出[だ]しました。	きのう ねんがじょう を だしました	
\\	昨日[きのう]、
\\	を 出[だ]しました。		
\\	賑やか	賑[にぎ]やか	にぎやか	
\\	浅草は賑やかな街です。	浅草[あさくさ]は 賑[にぎ]やかな 街[まち]です。	あさくさ は にぎやか な まち です	
\\	浅草[あさくさ]は
\\	な 街[まち]です。		
\\	蚊	蚊[か]	か	
\\	蚊に足を刺された。	蚊[か]に 足[あし]を 刺[さ]された。	か に あし を さされた	
\\	に 足[あし]を 刺[さ]された。		
\\	拭く	拭[ふ]く	ふく	
\\	タオルで体を拭きました。	タオルで 体[からだ]を 拭[ふ]きました。	たおる で からだ を ふきました	
\\	タオルで 体[からだ]を
\\	挨拶	挨拶[あいさつ]	あいさつ	
\\	彼女は笑顔で挨拶した。	彼女[かのじょ]は 笑顔[えがお]で 挨拶[あいさつ]した。	かのじょ は えがお で あいさつ した	
\\	彼女[かのじょ]は 笑顔[えがお]で
\\	した。		
\\	伴う	伴[ともな]う	ともなう	
\\	その仕事は危険を伴う。	その 仕事[しごと]は 危険[きけん]を 伴[ともな]う。	その しごと は きけん を ともなう	
\\	その 仕事[しごと]は 危険[きけん]を
\\	巻く	巻[ま]く	まく	
\\	彼は頭にタオルを巻いていた。	彼[かれ]は 頭[あたま]にタオルを 巻[ま]いていた。	かれ は あたま に たおる を まいて いた	
\\	彼[かれ]は 頭[あたま]にタオルを
\\	靴	靴[くつ]	くつ	
\\	靴が汚れた。	靴[くつ]が 汚[よご]れた。	くつ が よごれた	
\\	が 汚[よご]れた。		
\\	靴下	靴下[くつした]	くつした	
\\	この靴下は3足で1000円です。	この 靴下[くつした]は 3足[さんそく]で 1000円[せんえん]です。	この くつした は さんそく で せんえん です	
\\	この
\\	は 3足[さんそく]で 1000円[せんえん]です。		
\\	磨く	磨[みが]く	みがく	
\\	靴を磨いてください。	靴[くつ]を 磨[みが]いてください。	くつ を みがいて ください	
\\	靴[くつ]を
\\	ください。		
\\	歯磨き	歯磨[はみが]き	はみがき	
\\	歯磨きはしましたか。	歯磨[はみが]きはしましたか。	はみがき は しました か	
\\	はしましたか。		
\\	廊下	廊下[ろうか]	ろうか	
\\	廊下は走らないでください。	廊下[ろうか]は 走[はし]らないでください。	ろうか は はしらない で ください	
\\	は 走[はし]らないでください。		
\\	瓶	瓶[びん]	びん	
\\	瓶ビールを注文した。	瓶[びん]ビールを 注文[ちゅうもん]した。	びんびーる を ちゅうもん した	
\\	ビールを 注文[ちゅうもん]した。		
\\	褒める	褒[ほ]める	ほめる	
\\	頑張ったので褒められました。	頑張[がんば]ったので 褒[ほ]められました。	がんばった の で ほめられました	
\\	頑張[がんば]ったので
\\	元旦	元旦[がんたん]	がんたん	
\\	元旦に彼から年賀状が来た。	元旦[がんたん]に 彼[かれ]から 年賀状[ねんがじょう]が 来[き]た。	がんたん に かれ から ねんがじょう が きた 。	
\\	に 彼[かれ]から 年賀状[ねんがじょう]が 来[き]た。		
\\	袖	袖[そで]	そで	
\\	このシャツは袖が短い。	このシャツは 袖[そで]が 短[みじか]い。	この しゃつ は そで が みじかい	
\\	このシャツは
\\	が 短[みじか]い。		
\\	長袖	長袖[ながそで]	ながそで	
\\	今日は寒いので長袖を着ました。	今日[きょう]は 寒[さむ]いので 長袖[ながそで]を 着[き]ました。	きょう は さむい ので ながそで を きました 。	
\\	今日[きょう]は 寒[さむ]いので
\\	を 着[き]ました。		
\\	半袖	半袖[はんそで]	はんそで	
\\	今日は、半袖のシャツを着よう。	今日[きょう]は、 半袖[はんそで]のシャツを 着[き]よう。	きょう は 、 はんそで の しゃつ を きよう 。	
\\	今日[きょう]は、
\\	のシャツを 着[き]よう。		
\\	馬鹿	馬鹿[ばか]	ばか	
\\	私は馬鹿だった。	私[わたし]は 馬鹿[ばか]だった。	わたし は ばか だった	
\\	私[わたし]は
\\	だった。		
\\	凄い	凄[すご]い	すごい	
\\	凄い雨になった。	凄[すご]い 雨[あめ]になった。	すごい あめ に なった	
\\	雨[あめ]になった。		
\\	剃る	剃[そ]る	そる	
\\	父は毎日髭を剃ります。	父[ちち]は 毎日髭[まいにち ひげ]を 剃[そ]ります。	ちち は まいにち ひげ を そります	
\\	父[ちち]は 毎日髭[まいにち ひげ]を
\\	喧嘩	喧嘩[けんか]	けんか	
\\	喧嘩はやめて。	喧嘩[けんか]はやめて。	けんか は やめて	
\\	はやめて。		
\\	叩く	叩[たた]く	たたく	
\\	彼は子供のおしりを叩いた。	彼[かれ]は 子供[こども]のおしりを 叩[たた]いた。	かれ は こども の おしり を たたいた	
\\	彼[かれ]は 子供[こども]のおしりを
\\	噛む	噛[か]む	かむ	
\\	もっとよく噛みなさい。	もっとよく 噛[か]みなさい。	もっと よく かみなさい	
\\	もっとよく
\\	味噌汁	味噌汁[みそしる]	みそしる	
\\	私は毎日味噌汁を飲みます。	私[わたし]は 毎日[まいにち] 味噌汁[みそしる]を 飲[の]みます。	わたし は まいにち みそしる を のみます 。	
\\	私[わたし]は 毎日[まいにち]
\\	を 飲[の]みます。		
\\	姪	姪[めい]	めい	
\\	私の姪は3才です。	私[わたし]の 姪[めい]は 3才[さんさい]です。	わたし の めい は さんさい です	
\\	私[わたし]の
\\	は 3才[さんさい]です。		
\\	苺	苺[いちご]	いちご	
\\	この苺はとても甘い。	この 苺[いちご]はとても 甘[あま]い。	この いちご は とても あまい	
\\	この
\\	はとても 甘[あま]い。		
\\	茄子	茄子[なす]	なす	
\\	夕飯に茄子の天ぷらを食べました。	夕飯[ゆうはん]に 茄子[なす]の 天[てん]ぷらを 食[た]べました。	ゆうはん に なす の てんぷら を たべました 。	
\\	夕飯[ゆうはん]に
\\	の 天[てん]ぷらを 食[た]べました。		
\\	逢う	逢[あ]う	あう	
\\	ついに素晴らしい女性に逢えた。	ついに 素晴[すば]らしい 女性[じょせい]に 逢[あ]えた。	ついに すばらしい じょせい に あえた 。	
\\	ついに 素晴[すば]らしい 女性[じょせい]に
\\	椅子	椅子[いす]	いす	
\\	そのお年寄りは椅子に座った。	そのお 年寄[としよ]りは 椅子[いす]に 座[すわ]った。	その お としより は いす に すわった 。	
\\	そのお 年寄[としよ]りは
\\	に 座[すわ]った。		
\\	痩せる	痩[や]せる	やせる	
\\	私は少し痩せました。	私[わたし]は 少[すこ]し 痩[や]せました。	わたし は すこし やせました	
\\	私[わたし]は 少[すこ]し
\\	箸	箸[はし]	はし	
\\	箸を上手に使えるよ。	箸[はし]を 上手[じょうず]に 使[つか]えるよ。	はし を じょうず に つかえる よ	
\\	を 上手[じょうず]に 使[つか]えるよ。		
\\	糊	糊[のり]	のり	
\\	糊で2枚の紙を貼り合わせた。	糊[のり]で 2枚[にまい]の 紙[かみ]を 貼[は]り 合[あ]わせた。	のり で にまい の かみ を はり あわせた	
\\	で 2枚[にまい]の 紙[かみ]を 貼[は]り 合[あ]わせた。		
\\	醤油	醤油[しょうゆ]	しょうゆ	
\\	もう少し醤油を足してください。	もう 少[すこ]し 醤油[しょうゆ]を 足[た]してください。	もうすこし しょうゆ を たして ください 。	
\\	もう 少[すこ]し
\\	を 足[た]してください。		
\\	鋏	鋏[はさみ]	はさみ	
\\	この鋏はよく切れる。	この 鋏[はさみ]はよく 切[き]れる。	この はさみ は よく きれる	
\\	この
\\	はよく 切[き]れる。		
\\	鞄	鞄[かばん]	かばん	
\\	その黒い鞄は僕のです。	その 黒[くろ]い 鞄[かばん]は 僕[ぼく]のです。	その くろい かばん は ぼく の です	
\\	その 黒[くろ]い
\\	は 僕[ぼく]のです。		
\\	顎	顎[あご]	あご	
\\	顎が痛い。	顎[あご]が 痛[いた]い。	あご が いたい	
\\	が 痛[いた]い。		
\\	飴	飴[あめ]	あめ	
\\	缶に飴が入っています。	缶[かん]に 飴[あめ]が 入[はい]っています。	かん に あめ が はいって います	
\\	缶[かん]に
\\	が 入[はい]っています。		
\\	石鹸	石鹸[せっけん]	せっけん	
\\	石鹸で手を洗ってください。	石鹸[せっけん]で 手[て]を 洗[あら]ってください。	せっけん で て を あらって ください	
\\	で 手[て]を 洗[あら]ってください。		
\\	差	差[さ]	さ	
\\	都心と地方では収入に大きな差があるね。	都心[としん]と 地方[ちほう]では 収入[しゅうにゅう]に 大[おお]きな 差[さ]があるね。	としん と ちほう で は しゅうにゅう に おおき な さ が ある ね	
\\	都心[としん]と 地方[ちほう]では 収入[しゅうにゅう]に 大[おお]きな
\\	があるね。		
\\	失う	失[うしな]う	うしなう	
\\	彼は地震で親を失いました。	彼[かれ]は 地震[じしん]で 親[おや]を 失[うしな]いました。	かれ は じしん で おや を うしないました	
\\	彼[かれ]は 地震[じしん]で 親[おや]を
\\	外れる	外[はず]れる	はずれる	
\\	びんのふたが外れません。	びんのふたが 外[はず]れません。	びん の ふた が はずれません	
\\	びんのふたが
\\	測る	測[はか]る	はかる	
\\	このドアの高さを測ってください。	このドアの 高[たか]さを 測[はか]ってください。	この どあ の たかさ を はかって ください	
\\	このドアの 高[たか]さを
\\	ください。		
\\	含める	含[ふく]める	ふくめる	
\\	私を含めて10人が参加しました。	私[わたし]を 含[ふく]めて10 人[にん]が 参加[さんか]しました。	わたし を ふくめて 
\\	にん が さんか しました	
\\	私[わたし]を
\\	人[にん]が 参加[さんか]しました。		
\\	含む	含[ふく]む	ふくむ	
\\	その食品は有害物質を含んでいるぞ。	その 食品[しょくひん]は 有害物質[ゆうがい ぶっしつ]を 含[ふく]んでいるぞ。	その しょくひん は ゆうがい ぶっしつ を ふくんで いる ぞ	
\\	その 食品[しょくひん]は 有害物質[ゆうがい ぶっしつ]を
\\	ぞ。		
\\	頂戴	頂戴[ちょうだい]	ちょうだい	
\\	それ、ひとつ頂戴。	それ、ひとつ 頂戴[ちょうだい]。	それ ひとつ ちょうだい	
\\	それ、ひとつ
\end{CJK}
\end{document}