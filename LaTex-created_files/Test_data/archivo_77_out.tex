\documentclass[8pt]{extreport} 
\usepackage{hyperref}
\usepackage{CJKutf8}
\begin{document}
\begin{CJK}{UTF8}{min}
\\	髭・鬚・髯	ひげ	
\\	私[わたし]はひげの 伸[の]びが 速[はや]い。	
\\	〔口髭〕 
\\	〔顎(あご)鬚〕 
\\	〔頬(ほお)髯〕 
\\	〔猫などの〕 
\\	〔ヤギの〕 
\\	〔ナマズなど魚の〕 
\\	〔昆虫の〕 (口器) 
\\	(触角) 
\\	〔ひげ状のもの〕 (豆のさやの) 
\\	(大根の) 
\\	(時計などの) 
\\	【株式】 〔ローソク足チャートで, 上下に伸びる細い線; 高値と安値を表す〕 
\\	髭	ひげ	
\\	高校	こうこう	
\\	彼[かれ]は 高校[こうこう] 出[で]だ。	
\\	〔高等学校〕 
\\	*《口》 
\\	[⇒こうとうがっこう]	高校	こうこう	
\\	騒ぐ	さわぐ	▲こら 教室[きょうしつ]で 騒[さわ]ぐな。	
\\	〔大声を出してやかましくする〕 
\\	《口》 
\\	〔ざわざわと音を立てる〕 
\\	〔不満を訴える〕 
\\	〔抗議する〕 
\\	《口》 
\\	〔大げさに話題にする〕 
\\	〔冷静さをなくす〕 
\\	〔夢中になる〕 
\\	〔抵抗して叫ぶ〕 
\\	〔楽しく時を過ごす〕 
\\	〔波立つ〕 
\\	騒ぐ	さわぐ	
\\	駐車場	ちゅうしゃじょう	
\\	駐車場[ちゅうしゃじょう]に 車[くるま]を 入[い]れる。	
\\	駐車場	ちゅうしゃじょう	
\\	駅	えき	
\\	秋葉原[あきはばら]の 次[つぎ]の 駅[えき]。	
\\	〔鉄道の〕 
\\	〔宿場〕 
\\	駅	えき	
\\	駄目	だめ	
\\	自分[じぶん]の 意見[いけん]をきちんと 言[い]えなければだめだ。	
\\	【囲碁】 〔無駄目〕 
\\	~な 〔無益な・役に立たない〕 
\\	〔無駄な〕 
\\	〔望みがない〕 
\\	〔無能な〕 
\\	〔不可能〕 
\\	〔禁止〕	駄目	だめ	
\\	頭・首	かしら	
\\	頭[あたま]を 上[あ]げて 胸[むね]を 張[は]れ!	
\\	〔頭部〕 
\\	[=あたま 
\\	〔頭髪〕 
\\	〔先頭・最上位〕 
\\	首	くび	
\\	食料品	しょくりょうひん	▲この 代用[だいよう] 貨幣[かへい]では、 会社[かいしゃ]の 直営店[ちょくえいてん]に 行[い]き、その 店[みせ]で 食料品[しょくりょうひん]や 衣料[いりょう]を 買[か]うことしかできませんでした。	
\\	〔一品〕 
\\	〈集合的に〉 
\\	《文》 
\\	食料品	しょくりょうひん	
\\	食事	しょくじ	
\\	食事[しょくじ]の 前[まえ]に
\\	後[ご]で}。	
\\	(規定食) 
\\	(まかない) 
\\	~する =~をする.	食事	しょくじ	
\\	飛行場	ひこうじょう	
\\	私[わたし]はその 飛行機[ひこうき]に 乗り遅[のりおく]れないように 飛行場[ひこうじょう]へ 急[いそ]いだ。	
\\	〔陸地の〕 
\\	〔水上に浮かぶ〕 
\\	〔空軍の〕 
\\	飛行場	ひこうじょう	
\\	風邪	かぜ	
\\	風邪[かぜ]が ぶり返[ぶりかえ]すといけないから 早[はや]く 寝[ね]よう。	
\\	風邪	かぜ	
\\	風	ふう	
\\	知[し]らない 風[かぜ]を 装[よそお]う。	
\\	〔様子・ふり〕 
\\	〔身なり〕 
\\	〔態度〕 
\\	〔風習〕 
\\	〔習慣〕 
\\	〔風儀〕 
\\	〔風潮〕 
\\	〔傾向〕 
\\	〔ぐあい・やり方〕 
\\	〔(主に名詞について)流・式・型〕 
\\	〔流派〕 
\\	〔気質〕 
\\	〔種類〕 
\\	風	かぜ	
\\	顔	かお	
\\	顔[かお]が 太[ふと]りぎみである。	
\\	〔顔面〕 
\\	〔顔の色〕 
\\	〔目鼻だち〕 
\\	〔顔つき〕 
\\	(表情) 
\\	〔代表・役割・出現・出席・参加〕 
\\	〔面目〕 
\\	〔知名度〕 
\\	〔信用〕 
\\	〔影響力〕 
\\	顔	かお	
\\	頭	あたま	
\\	小野[おの] 選手[せんしゅ]からのパスを 彼[かれ]が 頭[あたま]で 合[あ]わせて 決勝点[けっしょうてん]を 挙[あ]げた。	
\\	〔頭部〕 
\\	〔脳天〕 
\\	【解】 
\\	〔頭髪〕 
\\	〔髪型〕 
\\	〔頭の働き・頭脳〕 
\\	〔知力〕 
\\	〔考慮〕 
\\	〔考え〕 
\\	〔見方〕 
\\	〔首領〕 
\\	〔最初〕 
\\	〔上方〕 
\\	〔先端〕 
\\	〔人数・メンバー〕 ⇒あたまかず.	頭	あたま	
\\	音楽	おんがく	
\\	彼女[かのじょ]の 踊[おど]りは 音楽[おんがく]に 合[あ]っていた。	▲〔リズムが〕 
\\	〔雰囲気が〕 
\\	音楽	おんがく	
\\	鞄	かばん	
\\	鞄[かばん]を 肩[かた]に 掛[か]けていた。	
\\	〔袋状の〕 
\\	(旅行用) 
\\	(小型の手提げ) 
\\	(折り鞄) 
\\	(ランドセル) 
\\	(学生鞄) 
\\	〔箱状の〕 
\\	(旅行用の大型) 
\\	(小型) 
\\	(書類用) 
\\	(紙ばさみ式の) 
\\	鞄	かばん	
\\	面白い	おもしろい	▲ピクニックはとても 面白[おもしろ]かった。	
\\	〔愉快な・楽しい〕 
\\	〔滑稽(こっけい)な〕 
\\	〔奇妙な〕 
\\	〔興趣がある・心が引かれる〕 
\\	〔好ましい・望ましい〕 (通例は「面白くない」の形で) 
\\	〔(「面白くない」の形で) 不愉快な・不満な〕 
\\	面白い	おもしろい	
\\	非常に		▲しかし 物々交換[ぶつぶつこうかん] 制[せい]は、 人々[ひとびと]の 要求[ようきゅう]が 一致[いっち]することは 希[まれ]であったので、 非常[ひじょう]に 不満足[ふまんぞく]な 制度[せいど]だった。	
\\	非常に	ひじょうに	
\\	青い	あおい	
\\	顔[かお]が 青[あお]い 青[あお]い 顔[かお]をしている。	
\\	〔青色の〕 
\\	〔緑の〕 
\\	〔青白い〕 
\\	〔未熟な〕 
\\	〔経験不足な〕 
\\	[⇒あおにさい]	青い	あおい	
\\	電灯	でんとう	
\\	電灯[でんとう]をつけるひも。	
\\	電灯	でんとう	
\\	電報	でんぽう	
\\	電報[でんぽう]の 配達[はいたつ]。	
\\	電報	でんぽう	
\\	雲	くも	
\\	雲[くも]の 切れ目[きれめ]
\\	切れ間[きれま]}。	
\\	〔空の〕 
\\	〔気分をすっきりさせないもの〕	雲	くも	
\\	階段	かいだん	
\\	階段[かいだん]を2 段[だん] 上[あ]がる。	
\\	【鉱山】 〔露天掘などの〕 
\\	〔段階・等級〕 
\\	階段	かいだん	
\\	降り出す	ふりだす	
\\	今[いま]にも 降り出[ふりだ]しそうな 空[そら]。	
\\	降り出す	ふりだす	
\\	下りる・降りる	おりる	
\\	木[き]から 下[お]りる。	
\\	【下りる】 〔高い所から下方へ移動する・移動して下に着く〕 
\\	【下りる】 〔飛んでいるものが下方へ移動する・着地や着陸する〕 (鳥が) 
\\	(滑空して) 
\\	(着陸する) 
\\	【下りる】 〔物が操作されて上から下へ動く〕 
\\	【降りる】 〔霜などが地上にできる〕 
\\	【下りる】 〔支給・発給される・申し渡される〕 
\\	【降りる】 〔乗り物から出る〕 
\\	《文》 
\\	【下りる】 〔体外に出る〕 
\\	〔流産する〕 
\\	【下りる】 〔負担がなくなる〕 
\\	【降りる】 〔地位・役・勝負などから退く〕 
\\	降りる	おりる	
\\	間違える	まちがえる	
\\	計算[けいさん]を 間違[まちが]える。	
\\	〔誤る〕 
\\	〔取り違える〕 
\\	間違える	まちがえる	
\\	間に合う	まにあう	
\\	時[じ]の 札幌[さっぽろ] 行[い]きに 間に合[まにあ]うように 空港[くうこう]に 着[つ]いた。	
\\	〔時間に遅れない〕 
\\	〔役に立つ〕 
\\	〔十分である〕 
\\	〔用意ができる〕 
\\	〔必要をみたす〕 
\\	間に合う	まにあう	
\\	間	あいだ	
\\	間[かん]を 詰[つ]めて 書[か]く。	
\\	〔行と行との〕 
\\	〔空間的間隔〕 
\\	〔時間的間隔〕 
\\	〔間(ま)〕 
\\	〔空間〕 
\\	〔すきま〕 
\\	〔継続した状態や時間・間中〕 
\\	〔ひと続きの空間・途中・距離〕 
\\	〔どちらにも属さない中間・真ん中〕 
\\	〔関係者の中・仲間うち〕 
\\	〔関係・間柄〕 
\\	間	あいだ	
\\	開ける・明ける・空ける	あける	
\\	裏口[うらぐち]のドアを 開[あ]けるとすぐ 狭[せま]い 路地[ろじ]だ。	
\\	〔とじていたものをひらく〕 
\\	〔解きあける〕 
\\	〔空間を作る・あきを設ける〕 
\\	〔中身をだして空にする〕 
\\	〔未使用の状態にする〕 (場所を) 
\\	(時間を) 
\\	(地位を) 
\\	開ける	あける	
\\	開く	ひらく	
\\	世界[せかい]に 目[め]を 開[ひら]く。	
\\	〔開け放つ〕 
\\	〔開け放たれる〕 
\\	〔通じさせる〕 
\\	〔開口する〕 
\\	[⇒ひらいた, ひらかれる] 
\\	〔見えるようにする〕 
\\	【電算】 
\\	〔見えるようになる〕 
\\	〔広げる〕 
\\	〔広がる〕 
\\	〔咲く〕 
\\	〔開催する〕 
\\	〔開業する〕 
\\	〔創立する〕 
\\	〔切り開く〕 
\\	〔隔たりができる〕 
\\	〔競技などで〕 
\\	【数】 〔累乗根を求める〕 
\\	〔括弧を外した式にする〕 
\\	〔身をかわす〕 
\\	〔体勢が甘くなる〕 
\\	〔魚を〕 
\\	【印刷】 〔漢字をひらがなに〕 
\\	開く	あく	
\\	鍵	かぎ	
\\	故障[こしょう]したらしく 鍵[かぎ]がうまくかからない。	
\\	〔錠の穴に入れて開閉する器具〕 
\\	〔錠〕 
\\	〔解決・成功などへの手がかり〕 
\\	鍵	かぎ	
\\	釣る	つる	▲この 魚[さかな]はよく 釣[つ]れる。	
\\	〔魚を〕 
\\	(釣り上げる) 
\\	〔トンボを〕 
\\	〔気を引く〕 
\\	[⇒つられる]	釣る	つる	
\\	重い	おもい	
\\	疲[つか]れて 靴[くつ]が 重[おも]く 感[かん]じる。	
\\	〔目方が多い・比重が大きい〕 
\\	〔動きが鈍い・容易に動かない〕 
\\	〔気分などが重苦しい〕 
\\	〔安定感や重量感がある〕 
\\	〔重大・重要な〕 
\\	〔深刻な〕 
\\	〔程度がはなはだしい〕 (罰などが) 
\\	(病気が) 
\\	【電算】 〔システムが鈍重な〕 
\\	重い	おもい	
\\	酷い	ひどい	▲ひどいことを 言[い]う。	
\\	〔むごい〕 
\\	(乱暴な) 
\\	(不当な) 
\\	〔極端に激しい〕 
\\	〔大変な〕 
\\	〔法外な〕 
\\	《口》 
\\	[⇒ひどく2] 
\\	〔極端に悪い〕 
\\	《文》 
\\	《口》 
\\	〔お粗末な〕 
\\	酷い	ひどい	
\\	都合	つごう	
\\	家[いえ]の 都合[つごう]で 退職[たいしょく]した。	
\\	〔諸事のかねあい〕 
\\	〔さしさわり〕 
\\	〔思惑〕 
\\	〔見通し〕 
\\	〔調子〕 
\\	〔なりゆき〕 
\\	〔折〕 
\\	〔やりくり〕 
\\	~する 
\\	〔全部で〕 
\\	[⇒ごうけい]	都合	つごう	
\\	都	みやこ	
\\	水[みず]の 都[と]ヴェニス。	
\\	〔首府〕 
\\	〔中心地〕 
\\	~する 〔王朝が〕 
\\	〔天皇が〕 
\\	都	と	
\\	部長	ぶちょう	
\\	部長[ぶちょう]は、その 地位[ちい]に 適[てき]しているには 控え目[ひかえめ]な 人[ひと]だとほのめかした。	
\\	〔大学学部の〕 
\\	〔クラブの〕 
\\	部長	ぶちょう	
\\	郊外	こうがい	
\\	東京[とうきょう]の 郊外[こうがい]に 住[す]む。	
\\	〔住宅地〕 
\\	郊外	こうがい	
\\	適当	てきとう	
\\	適当[てきとう]な 家[いえ]。	
\\	~な 
\\	〔条件に合う〕 
\\	~する =てきする1.	適当	てきとう	
\\	道具	どうぐ	
\\	道具[どうぐ]を 備[そな]えつける。	
\\	〔用具〕 
\\	(工具) 
\\	(小形で便利な) 
\\	(一式) 
\\	〈集合的に〉 
\\	〔家具〕 
\\	〔舞台用の〕 
\\	(大道具) 
\\	(小道具) 
\\	〔古道具〕 ⇒ふるどうぐ. 
\\	〔手段・方便〕 
\\	道具	どうぐ	
\\	過ぎる	すぎる	
\\	森[もり]を 過[す]ぎる。	
\\	〔通過する〕 
\\	〔経過する〕 
\\	〔期限がきれる〕 
\\	〔超過する〕 
\\	〔度を越える〕 
\\	[⇒-すぎる] 
\\	〔過分である〕	過ぎる	すぎる	
\\	運ぶ	はこぶ	
\\	遺体[いたい]を 運[はこ]んで 捨[す]てる。	
\\	〔(物を)運搬する〕 
\\	《口》 
\\	〔風などが〕 
\\	〔(事が)はかどる〕 
\\	〔(事を)進める〕 
\\	〔行く・来る〕 
\\	〔道具を動かす〕 
\\	運ぶ	はこぶ	
\\	遊ぶ	あそぶ	
\\	私[わたし]は 子供[こども]と 遊[あそ]ぶのが 大好[だいす]きで 将来[しょうらい]は 保母[ほぼ]になりたいと 思[おも]っています。	
\\	〔愉快なことをして楽しく過ごす〕 
\\	〔なんの束縛もなく自由に過ごす〕 
\\	〔仕事など有意義なことを何もしないで過ごす〕 
\\	(定職につかないで) 
\\	(職場などでまともに仕事をせずに) 
\\	〔遊興する〕 
\\	〔遊学する〕 
\\	〔有効に利用されずにある〕 
\\	[⇒あそばせる 
\\	〔もてあそぶ〕	遊ぶ	あそぶ	
\\	遅い	おそい	
\\	帰[かえ]りの 遅[おそ]いときは 電話[でんわ]しなさい。	
\\	〔時刻・時期が〕 
\\	(遅刻して) 
\\	(遅まきの) 
\\	〔速度・テンポ・反応が〕 
\\	遅い	おそい	
\\	進む	すすむ	
\\	進[すす]め。	▲【号令】 
\\	〔前方へ移動する〕 
\\	【海】 
\\	〔先行する〕 
\\	〔進歩する〕 
\\	〔目標に近づく〕 
\\	〔進学する〕 
\\	〔昇進する〕 
\\	〔程度が高まる〕 
\\	〔深刻になる〕 
\\	〔時計の表示が〕 
\\	〔積極的になる〕 ⇒すすんで.	進む	すすむ	
\\	連れる	つれる	
\\	子供[こども]を 歯医者[はいしゃ]に 連[つ]れて 行[い]く。	
\\	連れる	つれる	
\\	通る	とおる	
\\	検査[けんさ]に 通[とお]る。	
\\	〔通過する〕 
\\	[《文》 
\\	(経由する) 
\\	〔合格する〕 
\\	〔行き来する〕 
\\	〔突き抜ける〕 
\\	(尖ったものが) 
\\	〔反対側へ抜ける〕 
\\	〔開通する〕 
\\	〔運行する〕 
\\	〔論理が成り立つ〕 
\\	〔意味が明確である〕 
\\	〔取り次がれる〕 
\\	(厨房に) 
\\	〔受け入れられる〕 
\\	《文》 
\\	〔入室する〕 
\\	〔通用する〕 
\\	〔知れわたる〕 
\\	〔全体にゆきわたる〕 
\\	(液体が) 
\\	〔遠くまで届く〕 
\\	通る	とおる	
\\	送る	おくる	
\\	メールを 送[おく]る。	
\\	〔物や情報などを届ける〕 
\\	〔発送する〕 
\\	(商品を) 
\\	(船・車で) 
\\	(送信する) 
\\	(送金する) 
\\	〔人を派遣する・送り込む〕 
\\	〔人を送り出す・見送る〕 
\\	〔死者を見送る・死別する〕 
\\	〔去る[行く]人にある所まで付き添って行く〕 
\\	〔月日を過ごす〕 
\\	〔位置を次へ移す・順に受け渡す〕 
\\	〔送り仮名を付ける〕 
\\	送る	おくる	
\\	返事	へんじ	
\\	返事[へんじ]がある。	
\\	返事	へんじ	
\\	返す・反す	かえす	
\\	土[ど]
\\	田[た]}を 返[かえ]す。	
\\	【反す・返す】 〔ひっくり返す〕 
\\	〔裏返す〕 
\\	〔逆にする〕 
\\	【返す】 〔元の場所・状態へ戻す〕 
\\	【返す】 〔元の所有者に戻す〕 
\\	〔返却する〕 
\\	〔送り返す〕 
\\	【返す】 〔同じことを相手に対してする〕 
\\	〔返礼する〕 
\\	〔報復する〕 
\\	〔補助動詞として〕 
\\	返す	かえす	
\\	近所	きんじょ	
\\	近所[きんじょ] 中[ちゅう]の 笑[わら]いもの。	
\\	近所	きんじょ	
\\	迎える	むかえる	
\\	駅[えき]で 迎[むか]える。	
\\	〔出迎える〕 
\\	〔接待する〕 
\\	〔歓迎する〕 
\\	〔招く〕 
\\	〔仲間として受け入れる〕 
\\	〔ある時期・状況となる〕 
\\	〔(敵などに)相対する〕 
\\	迎える	むかえる	
\\	込む	こむ	
\\	込[こ]むといけないから 早[はや]く 行[い]こう。	
\\	〔混雑する〕 
\\	(満員) 
\\	[⇒こみあう] 
\\	〔数多く入りこむ〕 
\\	〔(「手が込む」の形で)細かく入り組む〕	込む	こむ	
\\	辞典	じてん	
\\	私[わたし]はいつも 辞典[じてん]を3 冊[さつ] 手元[てもと]に 置[お]いています。	
\\	[=じしょ5]	辞典	じてん	
\\	辛い	からい	▲このキムチは 私[わたし]には 辛[つら]くて 食[た]べられない。	
\\	〔ぴりっと感じる〕 
\\	〔塩辛い〕 
\\	[=しおからい] 
\\	〔酒が〕 
\\	〔評価などが厳しい〕 
\\	《口》 
\\	辛い	からい	
\\	輸出	ゆしゅつ	
\\	公害[こうがい]の 輸出[ゆしゅつ]。	
\\	~する 
\\	輸出	ゆしゅつ	
\\	輸入	ゆにゅう	
\\	海外[かいがい]からの 物資[ぶっし]の 輸入[ゆにゅう]。	
\\	〔文物の〕 
\\	~する 
\\	輸入	ゆにゅう	
\\	軽い	かるい	
\\	軽[かる]そうに 運[はこ]ぶ。	
\\	〔目方が少ない〕 
\\	〔軽快だ〕 
\\	〔軽薄だ〕 
\\	〔重量感がない〕 
\\	〔重大でない・軽微な〕 
\\	〔容易な〕 
\\	〔気軽な〕 
\\	〔ランクの低い〕 
\\	〔本格的でない・あっさりした〕 
\\	軽い	かるい	
\\	踏む	ふむ	
\\	人[ひと]の 足[あし]
\\	足の指[あしのゆび]}を 踏[ふ]む。	
\\	〔足の下にする・足で強く押す〕 
\\	〔その場に立つ・訪れる〕 
\\	〔経験する〕 
\\	〔(手続きなどを)経る〕 
\\	〔評価する・見積もる〕 
\\	〔押韻する〕 
\\	踏む	ふむ	
\\	足りる	たりる	
\\	必要[ひつよう]なものは 足[た]りています。	
\\	〔十分である; 必要量がある〕 
\\	[=たる2 
\\	⇒たりない 
\\	〔値打ちがある〕 
\\	[=たる2 
\\	〔用がすむ〕 
\\	[=たる2 
\\	足りる	たりる	
\\	趣味	しゅみ	
\\	私[わたし]の 趣味[しゅみ]は 切手収集[きってしゅうしゅう]です。	
\\	〔道楽〕 
\\	〔感性〕 
\\	〔好み〕 
\\	趣味	しゅみ	
\\	起きる	おきる	
\\	病気[びょうき]の) 母[はは]はいつになったら 起[お]きられますか」
\\	間[ま]もなくですよ」。	
\\	〔起床する〕 
\\	《文》 
\\	《口》 
\\	〔病床を離れる〕 
\\	〔目覚める〕 
\\	《文》 
\\	〔眠らずにいる〕 
\\	〔体を起こす〕 
\\	(倒れた人が) 
\\	(傾いていた船などが) 
\\	(倒れ伏していた植物が) 
\\	〔発生する〕 (事故・症状などが) 
\\	(気持ちが) 
\\	《文》 
\\	起きる	おきる	
\\	赤ん坊	あかんぼ, あかんぼう	▲あれは 体[からだ]ばかり 大[おお]きくてまるで 赤ん坊[あかんぼう]です。	
\\	[⇒あかちゃん]	赤ん坊	あかんぼう	
\\	赤ちゃん	あかちゃん	▲クジラの 赤[あか]ちゃん。	
\\	赤ちゃん	あかちゃん	
\\	赤い	あかい	
\\	古[ふる]いマンションの 水道[すいどう]では 鉄[てつ]さびのせいで 赤[あか]い 水[みず]が 出[で]ることがある。	
\\	〔色〕 
\\	(深紅) 
\\	(緋) 
\\	〔共産主義の〕 
\\	赤い	あかい	
\\	貿易	ぼうえき	
\\	日本[にっぽん]の 対[たい] 米[べい] 貿易[ぼうえき]。	
\\	~する 
\\	貿易	ぼうえき	
\\	貸す	かす	▲きのうの 授業[じゅぎょう]のノートを 貸[か]してくれ。	
\\	〔無償で〕 (金品を) 
\\	(労力を) 
\\	〔担保を取って〕 
\\	〔賃貸しする〕 
\\	(家屋や部屋を) 
\\	【法】 (土地を) 
\\	〔掛け売りする〕 
\\	貸す	かす	
\\	買う	かう	
\\	物[もの]を 買[か]いに 行[い]く。	
\\	〔買い求める〕 
\\	〔招く〕 
\\	〔引き受ける〕 
\\	〔価値を認める〕 
\\	〔金を払って性的交渉をする〕 
\\	買う	かう	
\\	貰う	もらう	
\\	ノーベル賞[のーべるしょう]をもらう。	
\\	〔与えられて自分の物とする〕 
\\	〔受納する〕 
\\	〔授与される〕 
\\	《口》 
\\	〔申請して得る・受け取る〕 
\\	〔人を家族の一員として迎える〕 
\\	〔買う〕 
\\	〔獲得して自分の自由にする〕 
\\	〔嫌な物を身に引き受ける〕 
\\	〔(「…してもらう」の形で) 相手に行為を頼む; 行為が利益になる意味を添える〕 
\\	〔(「…してもらう」の形で) 行為が迷惑になる意味を添える〕 
\\	貰う	もらう	
\\	負ける	まける	▲マージャンで 友[とも]だちに5 千[せん] 円[えん] 負[ま]けた。	
\\	〔敗北する〕 
\\	(試合に) 
\\	(戦争に) 
\\	(競争に) 
\\	《口》 
\\	(選挙に) 
\\	〔敗訴する〕 
\\	〔賭け事に〕 
\\	〔譲歩する〕 
\\	〔免除する〕 
\\	〔屈服する〕 
\\	〔劣位である〕 
\\	〔圧倒される〕 
\\	〔値引きする〕 
\\	〔おまけを付ける〕 
\\	〔皮膚がかぶれる〕 
\\	負ける	まける	
\\	警察	けいさつ	▲とうとう 警察[けいさつ]の 手[て]が 回[まわ]った。	
\\	〔警察署〕 
\\	〔警察官〕 
\\	(女性警官) 
\\	警察	けいさつ	
\\	講義	こうぎ	
\\	講義[こうぎ]を 始[はじ]める。	▲〔先生が〕 
\\	~する 
\\	〔連続の〕 
\\	講義	こうぎ	
\\	講堂	こうどう	
\\	講堂[こうどう]が 暑[あつ]くなると、いつも 私[わたし]はファンを 相撲[すもう]に 連[つ]れていく。	
\\	〔階段式の〕 
\\	〔学校の〕 
\\	(大学の教室) 
\\	〔寺院の〕 
\\	講堂	こうどう	
\\	調べる	しらべる	
\\	自分[じぶん]の 入[はい]りたい 大学[だいがく]についてよく 調[しら]べておく。	
\\	〔調査する〕 
\\	〔研究する〕 
\\	〔辞書などを参照する〕 
\\	〔下調べする〕 
\\	〔点検する〕 
\\	〔確認する〕 
\\	〔試験する〕 
\\	〔さがす〕 
\\	〔とり調べる〕 
\\	調べる	しらべる	
\\	説明	せつめい	
\\	説明[せつめい]を 要[よう]しない。	
\\	〔さし絵や写真の〕 
\\	〔図表などの〕 
\\	~する 
\\	〔大体を〕 
\\	説明	せつめい	
\\	試験	しけん	
\\	試験[しけん] 慣[な]れしている。	
\\	〔学業・技術などの能力判定〕 
\\	《口》 
\\	~する 
\\	〔試行テスト〕 (実験) 
\\	(実演) 
\\	~する 
\\	試験	しけん	
\\	試合	しあい	
\\	試合[しあい] 中[ちゅう]のけが。	
\\	〔一試合〕 
\\	〔球技の〕 
\\	〔組み打ち・殴り合いの〕 
\\	〔相撲などの〕 
\\	〔ボクシングなどの〕 
\\	〔勝ち抜きの〕 
\\	〔一連の〕 
\\	〔試合すること〕 
\\	~する =~をする.	試合	しあい	
\\	計画	けいかく	
\\	計画[けいかく]を 延期[えんき]する。	
\\	~する 
\\	計画	けいかく	
\\	触る	さわる	
\\	鬼[おに]ごっこで) 鬼[おに]に 体[からだ]を 触[さわ]られたら 今度[こんど]は 君[きみ]が 鬼[おに]になるんだよ。	
\\	触る	さわる	
\\	角	つの	
\\	角[かく]が 落[お]ちる。	
\\	〔牛・羊・ヤギ・サイなどの〕 
\\	〔鹿の枝角〕 
\\	〔ナイフの柄などにする鹿の角〕 
\\	〔カブトムシなどの〕 
\\	〔悪魔の〕 
\\	〔カタツムリなどの〕 
\\	〔触角〕 
\\	〔角状突起〕 
\\	〔突起物〕	角	かど	
\\	親切	しんせつ	
\\	人[ひと]の 親切[しんせつ]を 踏[ふ]みにじってはいけません。	
\\	~な 
\\	親切	しんせつ	
\\	覚える	おぼえる	
\\	月[つき]10 日[にち]の 大空襲[だいくうしゅう]を 彼[かれ]はよく 覚[おぼ]えていた。	
\\	〔記憶する〕 
\\	〔暗記する〕 
\\	〔習得する〕 
\\	〔自覚する〕 
\\	〔思われる〕 
\\	覚える	おぼえる	
\\	規則	きそく	
\\	規則[きそく]に 従[したが]わないで…する。	
\\	規則	きそく	
\\	見付ける	みつける	
\\	誤植[ごしょく]を 見[み]つける。	
\\	〔発見する〕 
\\	〔気がつく〕 
\\	〔捜し出す〕 
\\	[⇒みつけだす] 
\\	〔探す〕 
\\	〔それと確認する〕 
\\	見つける	みつける	
\\	見える	みえる	
\\	波間[なみま]に 陸地[りくち]が 遠[とお]ざかりやがて 見[み]えなくなった。	
\\	〔形や色が目に届く〕 〈人が主語〉 
\\	(目に入る) 
\\	〈事物が主語〉 
\\	〔発見できる〕 
\\	〔形や色が見分けられる〕 
\\	〔解釈される・推測される〕 
\\	〔わかる〕 
\\	〔確認できる〕 (痕跡が) 
\\	(兆候が) 
\\	(存在が) 
\\	〔来る〕 
\\	〔到着する〕 
\\	見える	みえる	
\\	西洋	せいようとねりこ	
\\	彼[かれ]は 西洋[せいよう] 文化[ぶんか]が 好[す]きだ。	
\\	【植】 〔モクセイ科の落葉高木; ヨーロッパ・北アジア産〕 
\\	西洋	せいよう	
\\	褒める	ほめる	
\\	手放[てばな]しでほめるほめまくる。	
\\	褒める	ほめる	
\\	複雑	ふくざつ	
\\	複雑[ふくざつ]な 味[あじ]。	
\\	~な 
\\	〔曲折した〕 
\\	複雑	ふくざつ	
\\	行く	いく	▲その 店[みせ]はこの 道[みち]をもうちょっと 行[おこな]ったところにある。	
\\	〔こちらから向こうへ移動する〕 
\\	《文》 
\\	《文》 
\\	〔去る〕 
\\	〔こちらから向こうへ届く〕 
\\	〔出かける・訪問する・出席する〕 
\\	(相手のいるところへ) 
\\	〔学校などに入る・通う〕 
\\	〔嫁になる〕 
\\	〔養子になる〕 
\\	〔時が過ぎる〕 
\\	〔数量が達する〕 
\\	〔物事が進む〕 
\\	〔気持ちが進む〕 
\\	〔物事をする〕 
\\	〔セックスでオルガスムスに達する〕 
\\	〔現実から遊離している〕 
\\	〔(「…ていく」の形で)物事の継続・進行を表す〕 
\\	〔(「…にいく」の形で) …しようとする・物事を志す〕	行く	いく	
\\	虫	むし	
\\	幹[みき]に 虫[むし]がぞろぞろ 這[は]っている。	
\\	〔昆虫・羽虫〕 
\\	〔コオロギなど〕 
\\	〔脚のない虫; 蠕虫(ぜんちゅう)〕 
\\	〔毛虫〕 
\\	〔蛾〕 
\\	〔害虫〕 〈集合的に〉 
\\	〔回虫〕 
\\	〔心の中にあって感情に影響を及ぼすもの〕 
\\	〔虫気〕 ⇒むしけ. 
\\	〔一つのことに熱中する人〕 
\\	【電】 〔弁の〕 
\\	虫	むし	
\\	薄い	うすい	
\\	本棚[ほんだな]の 下[した]に 薄い板[うすいいた]をちょっと 挟[はさ]んだだけで 安定[あんてい]がよくなった。	
\\	〔厚みが少ない〕 
\\	〔濃度が低い〕 (液体・味などが) 
\\	(印字・印画が) 
\\	〔淡い〕 
\\	〔少ない〕 
\\	薄い	うすい	
\\	葡萄	ぶどう	▲ぶどうのつる。	
\\	【植】 〔ブドウ科ブドウ属 
\\	の栽培品種の総称; 落葉つる性木本〕 
\\	〔果実〕 
\\	葡萄	ぶどう	
\\	葉	は	
\\	葉[は]の 多[おお]い
\\	茂[しげ]った}。	
\\	【植】 
\\	(集合的) 
\\	(草の細長い葉) 
\\	(シダなどの切込んだ葉) 
\\	(針葉) 
\\	葉	は	
\\	草	くさ	▲すぐに 草[くさ]が 生[は]えてくる。	
\\	〔芝生や野原の〕 
\\	〔ハーブ〕 
\\	〔雑草〕 
\\	〈集合的に〉 
\\	草	くさ	
\\	英語	えいご	
\\	花[はな]」は 英語[えいご]では 何[なに]と 言[い]いますか。	
\\	〔単語〕 
\\	英語	えいご	
\\	苦い	にがい	
\\	苦[にが]い 味[あじ]がする。	
\\	〔味が〕 
\\	〔不興げな〕 
\\	〔つらい〕 
\\	苦い	にがい	
\\	花見	はなみ	
\\	花見[はなみ]に 行[い]く。	
\\	〔行楽〕 
\\	花見	はなみ	
\\	花瓶	かびん	
\\	花瓶[かびん]にさす。	
\\	花瓶	かびん	
\\	色々	いろいろ	
\\	心配性[しんぱいしょう]な 私[わたし]は 物事[ものごと]をいろいろ 様々[さまざま]に 考[かんが]えすぎる 傾向[けいこう]がある。	
\\	~な 〔種々な〕 
\\	色々	いろいろ	
\\	色	いろ	
\\	ペンキ塗り立[ぺんきぬりた]てだから 触[さわ]ると 色[いろ]が 付[つ]くよ。	
\\	〔色彩〕 
\\	〔色合い〕 
\\	〔肌の色〕 
\\	〔顔つき・表情〕 
\\	〔そぶり・態度〕 
\\	〔おもむき・様子〕 
\\	〔気配・きざし〕 
\\	〔風情〕 
\\	〔特色・カラー〕 
\\	〔調子〕 
\\	[⇒ねいろ, こわいろ] 
\\	〔情事〕 
\\	〔情欲〕 
\\	〔情人〕 
\\	〔密通の相手〕 
\\	〔容色〕 
\\	〔種類〕 
\\	【物】 〔クォークの〕 
\\	色	いろ	
\\	船・舟	ふね	
\\	船[ふね]を 軽[かる]くする。	
\\	〔船舶〕 (小舟・汽船) 
\\	(大型の) 
\\	(かい・ろを用いない大型の) 
\\	(汽船) 
\\	(定期航海の) 
\\	(河・港に用いる平底の) 
\\	〈集合的に〉 
\\	〔液体を入れる容器〕 
\\	(細長くてふたのない) 
\\	(水槽) 
\\	(工場用の) 
\\	〔西洋風の浴槽〕 
\\	〔船の形をしたもの〕 
\\	船	ふね	
\\	興味	きょうみ	
\\	興味[きょうみ]のない。	
\\	興味	きょうみ	
\\	背中	せなか	
\\	鏡[きょう]などで) 自分[じぶん]の 背中[せなか]を 見[み]る。	
\\	〔身体の〕 
\\	〔うしろ〕 
\\	背中	せなか	
\\	育てる	そだてる	▲シャーレで 菌[きん]を 育[そだ]てる。	
\\	〔発育させる〕 
\\	(動・植物を) 
\\	〔培養する〕 
\\	〔教育する〕 
\\	〔発展させる〕 
\\	〔完成に近づける〕 
\\	育てる	そだてる	
\\	聞こえる	きこえる	
\\	聞[き]こえないんですが。	▲〔電話口で〕 
\\	〔発表者などに〕 
\\	〔音声が耳に入る〕 〈人が主語〉 
\\	〈音が主語〉 
\\	〔受信できる〕 
\\	〔情報が届く〕 
\\	〔解釈される〕 
\\	(音として) 
\\	〔知れわたる〕 
\\	《文》 
\\	〔納得できる〕	聞こえる	きこえる	
\\	翻訳	ほんやく	
\\	詩[し]の 微妙[びみょう]なニュアンスを 翻訳[ほんやく]するのは 難[むずか]しい。	
\\	~する 
\\	〔暗号などを〕 
\\	翻訳	ほんやく	
\\	習慣	しゅうかん	▲…する 習慣[しゅうかん]をつける。	
\\	〔個人の習性・癖〕 
\\	〔社会のしきたり〕 
\\	習慣	しゅうかん	
\\	美術館	びじゅつかん	
\\	美術館[びじゅつかん]には 絵[え]がたくさんかかっている。	
\\	美術館	びじゅつかん	
\\	線	せん	
\\	線[せん]の 外側[そとがわ]に 出[で]る。	
\\	〈物事が主語〉 
\\	〔すじ〕 
\\	【鉱・解】 
\\	〔鉄道線路・航路〕 
\\	〔軌道〕 
\\	〔路線〕 
\\	〔電線〕 
\\	〔電話線〕 
\\	〔物の輪郭〕 
\\	〔その人の印象〕 
\\	〔区切り〕 
\\	〔限度〕 
\\	〔水準〕 
\\	〔物事の方向性〕 
\\	線	せん	
\\	続く	つづく	
\\	首相[しゅしょう]の 演説[えんぜつ]はまだ 続[つづ]いている。	
\\	〔終わらない〕 
\\	〔変わらない〕 
\\	〔途切れない〕 
\\	〔繰り返される〕 
\\	〔つながる〕 
\\	〔通じる〕 
\\	〔次々にある〕 
\\	〔その次の〕 ⇒つづいて. 
\\	〔後に従う〕 
\\	〔次位を占める〕 =つぐ1. 
\\	〔もちこたえる〕 
\\	続く	つづく	
\\	絹	きぬ	▲これには 絹[きぬ]が 混[ま]じっている。	
\\	〔絹〕 
\\	〔絹織物〕 =きぬおり(もの).	絹	きぬ	
\\	絵	え	▲この 雑誌[ざっし]にはきれいな 絵[え]が 入[はい]っている。	
\\	〔絵画〕 
\\	(彩色画) 
\\	(写生画・下画) 
\\	(挿絵) 
\\	(カット・挿絵) 
\\	(版画) 
\\	〔画法〕 
\\	絵	え	
\\	経験	けいけん	
\\	彼[かれ]は 色々[いろいろ]な 経験[けいけん]をしてきた 人[ひと]だ。	
\\	~する 
\\	経験	けいけん	
\\	経済	けいざい	▲アジア 経済[けいざい]の 崩壊[ほうかい]。	
\\	【経】 
\\	〔経済状態〕 
\\	〔財政〕 
\\	〔経費がかからないこと〕 
\\	経済	けいざい	
\\	終わる	おわる	
\\	料理[りょうり]の 修業[しゅうぎょう]は 卵[たまご]に 始[はじ]まって 卵[たまご]に 終[お]わるといわれるほど 卵料理[たまごりょうり]は 難[むずか]しい。	
\\	〔物事が終了する・おしまいになる〕 
\\	〔完了する〕 
\\	〔(「…に[で]~」の形で) …の結果となる, …の状態で終わりを迎える〕 
\\	〔物事を終了する・終える〕 
\\	〔(動詞の連用形に付いて) 動作などを終了する, …てしまう〕 
\\	終わる	おわる	
\\	紹介	しょうかい	
\\	新製品[しんせいひん]を 紹介[しょうかい]する。	
\\	~する 
\\	紹介	しょうかい	
\\	細かい	こまかい	
\\	細[こま]かい 編み目[あみめ]
\\	縫い目[ぬいめ]}。	
\\	〔形が小さい〕 
\\	〔動きが小さい〕 
\\	〔少額の〕 
\\	〔詳しい〕 
\\	〔精密な〕 
\\	〔厳密な〕 
\\	〔細部の〕 
\\	〔行き届いている〕 
\\	〔取るに足りない〕 
\\	〔金にうるさい〕 
\\	〔物事にこだわる〕 
\\	細かい	こまかい	
\\	素晴らしい	すばらしい	
\\	今年[ことし]の 米作[べいさく]は 素晴[すば]らしい 出来[でき]だ。	
\\	《口》 
\\	《俗》 
\\	素晴らしい	すばらしい	
\\	紙	かみ	
\\	紙[し]をすく。	
\\	〔薄いもの〕	紙	かみ	
\\	約束	やくそく	
\\	約束[やくそく]を 果[は]たす
\\	履行[りこう]する}。	
\\	〔あらかじめ取り決めたこと〕 
\\	〔会合の〕 
\\	〔契約〕 
\\	〔協定〕 
\\	〔条件〕 
\\	~する 
\\	〔しきたり〕 
\\	〔規則〕 
\\	〔宿命〕 
\\	約束	やくそく	
\\	米	べい	
\\	米[べい] 日程[にってい] 発表[はっぴょう]。	▲〔見出し〕 
\\	米	こめ	
\\	笑う	わらう	▲つらい 経験[けいけん]もいつか 笑[わら]って 話[はな]せる 日[ひ]が 来[く]る。	
\\	〔喜びやおかしさで〕 
\\	(微笑する) 
\\	(大笑する) 
\\	(爆笑する) 
\\	(くすくす) 
\\	[⇒わらえる] 
\\	〔笑い顔を作る〕 
\\	〔嘲笑する〕 
\\	〔追従(ついしょう)して〕 
\\	〔無力感で〕 
\\	〔状態がゆるむ〕 
\\	笑う	わらう	
\\	空港	くうこう	
\\	羽田空港[はねだくうこう]。	
\\	空港	くうこう	
\\	空気	くうき	
\\	空気[くうき]が 汚[よご]れている。	
\\	〔気体〕 
\\	〔雰囲気〕 
\\	空気	くうき	
\\	開く・明く・空く	あく	
\\	栓[せん]が 開[ひら]かない。	
\\	〔閉じていたものがひらく〕 
\\	〔始まる〕 
\\	〔営業などが続く〕 
\\	〔空間ができる〕 
\\	〔からになる・欠員が生じる・用済みになる〕 〈物が主語〉 
\\	〈物・人が主語〉 
\\	空く	あく	
\\	秋	あき	
\\	秋[あき]の 虫[むし]。	
\\	秋	あき	
\\	神社	じんじゃ	
\\	神社[じんじゃ]に 詣[まい](もう)でる。	
\\	神社	じんじゃ	
\\	祖父	そふ	
\\	祖父[そふ]の{らしい}。	
\\	祖父	そふ	
\\	祖母	そぼ	
\\	祖母[そぼ]のような。	
\\	祖母	そぼ	
\\	研究室	けんきゅうしつ	
\\	青木[あおき] 研究[けんきゅう] 室[しつ]。	
\\	〔ゼミナールの〕 
\\	〔化学などの〕 
\\	《口》 
\\	研究室	けんきゅうしつ	
\\	研究	けんきゅう	▲…を 研究[けんきゅう]で 明[あき]らかにする。	
\\	〔学術の〕 
\\	〔探求〕 
\\	~する 
\\	〔探求する〕 
\\	〔考究する〕 
\\	研究	けんきゅう	
\\	砂	いさご	
\\	浜[はま]の 砂[すな]。	
\\	砂	すな	
\\	着物	きもの	
\\	借[か]りた 着物[きもの]。	
\\	〔和服〕 
\\	〔衣服・衣類〕 
\\	〈集合的〉 
\\	[⇒ふく1]	着物	きもの	
\\	眠る	ねむる	
\\	半分[はんぶん] 眠[ねむ]りながら 書[か]いたような 字[じ]。	
\\	〔寝入る〕 
\\	〔ちょっと眠る〕 
\\	〔居眠りする〕 
\\	〔活動を休止する〕 
\\	〔永眠する〕 
\\	〔活用されないでいる〕 
\\	眠る	ねむる	
\\	眠い	ねむい	
\\	眠[ねむ]い 時[とき]は 我慢[がまん]せずに 寝[ね]たほうがいい。	
\\	〔病的に〕 
\\	眠い	ねむい	
\\	真面目	まじめ	▲まじめな 声[こえ]で。	
\\	~な 〔真剣な〕 
\\	【劇】 
\\	〔誠実な〕 
\\	〔堅実な〕 
\\	真面目	まじめ	
\\	真ん中	まんなか	▲ステッキの 真ん中[まんなか]を 持[も]って 振り回[ふりまわ]す。	
\\	〔中央・中心〕 (線上) 
\\	真ん中	まんなか	
\\	直る	なおる	
\\	直[なお]るものなら 直[なお]して 使[つか]いたい。	
\\	〔修理される〕 
\\	(状態) 
\\	〔修理可能である〕 
\\	〔復旧する〕 
\\	(服装の乱れなどが) 
\\	〔姿勢を戻す〕 
\\	〔訂正される〕 
\\	〔訂正可能である〕 
\\	〔矯正(きょうせい)される〕 
\\	〔矯正可能である〕 
\\	直る	なおる	
\\	盛ん	さかん	
\\	彼[かれ]の 盛[さか]んな 時代[じだい]もあった。	
\\	~な 〔繁栄している〕 
\\	〔盛大である〕 
\\	~な 〔普及している〕 
\\	~な 〔熱心な〕 
\\	〔頻繁な〕 
\\	~な 〔勢いのある〕 
\\	~な 〔充実している〕 
\\	〔力強い〕 
\\	(精力的な) 
\\	盛ん	さかん	
\\	盗む	ぬすむ	
\\	金庫[きんこ]
\\	自動販売機[じどうはんばいき]}から 金[きん]を 盗[ぬす]む。	
\\	〔他人の物を密かに取って自分の物とする〕 
\\	《文》 
\\	《口》 
\\	*《俗》 
\\	(婉曲的に) 
\\	〔強奪する〕 
\\	〔万引きする〕 
\\	〔他人の考え・芸・文などを自分のものとする〕 
\\	〔相手に気付かれないよう行動する〕 
\\	〔合間の時間を利用する〕 ⇒ひま1 
\\	【野球】 〔盗塁する〕 
\\	盗む	ぬすむ	
\\	皆	みな, みんな	
\\	君[くん]たち 皆[みな]で 寄[よ]ってたかってこの 計画[けいかく]をぶちこわしたのだ。	
\\	〔全員〕 
\\	〔全部; すべてのもの〕 
\\	皆	みな	
\\	発音	はつおん	
\\	と
\\	の 発音[はつおん]をし 分[わ]ける。	
\\	【言語】 
\\	〔音節に分けての〕 
\\	【物・生物】 
\\	~する 
\\	《文》 
\\	〔音節に分けて〕 
\\	発音	はつおん	
\\	痩せる	やせる	
\\	顔[かお]がやせている。	
\\	〔体の肉づきが悪くなる〕 
\\	(病気で) 
\\	(心配で) 
\\	(ダイエットして) 
\\	[⇒やせた] 
\\	〔地味が乏しくなる〕 
\\	痩せる	やせる	
\\	痛い	いたい	
\\	注射[ちゅうしゃ]をした 看護婦[かんごふ]が 患者[かんじゃ]に〕 痛[いた]くはなかったですか。	
\\	〔痛む〕 
\\	[=いたむ2 
\\	〔つらい〕 
\\	〔困る〕 
\\	痛い	いたい	
\\	番組	ばんぐみ	
\\	番組[ばんぐみ] 放送[ほうそう] 中[ちゅう]。	
\\	〔テレビ・ラジオなどの放送単位〕 
\\	番組	ばんぐみ	
\\	留守	るす	
\\	散歩[さんぽ]に 行[い]って 留守[るす]だ。	
\\	〔不在〕 
\\	〔留守番〕 ⇒るすばん. 
\\	〔おろそかになること〕	留守	るす	
\\	男性	だんせい	
\\	男性[だんせい]の 魅力[みりょく]。	
\\	〔男〕 
\\	〔男に特徴的とされる気質〕 
\\	【文法】 
\\	男性	だんせい	
\\	男の子	おとこのこ	
\\	男の子[おとこのこ]らしいいたずら。	
\\	《文・古》 
\\	〔男の赤ん坊〕 
\\	男の子	おとこのこ	
\\	男	おとこ	
\\	男[おとこ]の 心[こころ]をつかむのは 手料理[てりょうり]よ。	
\\	〔男子〕 
\\	《口》 
\\	《口》 
\\	〔一人前の〕 
\\	〔男子の面目〕 
\\	〔男としての評価〕 
\\	〔情夫〕 
\\	男	おとこ	
\\	申す	もうす	
\\	私[わたし]は 山田[やまだ] 太郎[たろう]と 申[もう]します。	
\\	〔言う〕 ⇒いう1. 
\\	〔…する〕 
\\	申す	もうす	
\\	申し上げる	もうしあげる	
\\	過日[かじつ] 申し上[もうしあ]げたとおり。	
\\	申し上げる	もうしあげる	
\\	用意	ようい	▲さあと 言[い]われればすぐにでも 出発[しゅっぱつ]できるように 用意[ようい]している。	
\\	〔準備〕 
\\	〔手はず〕 
\\	〔旅行などの支度〕 
\\	〔貯え〕 
\\	〔備え〕 
\\	~する 
\\	用意	ようい	
\\	用	よう	▲「だれに 用[よう]ですか」
\\	川本[かわもと]さんに 用[よう]なんですが」。	
\\	〔用事〕 
\\	(命ぜられた) 
\\	[⇒ようじ4, ごよう1 
\\	〔役・働き〕 
\\	〔(名詞について)そのための〕 
\\	〔用便〕	用	よう	
\\	生活	せいかつ	▲アリの 生活[せいかつ]。	
\\	〔生物の暮らし・活動〕 
\\	~する 
\\	〔人の暮らし〕 
\\	〔生計〕 
\\	[⇒くらし] ~する 
\\	〔生計を立てる〕 
\\	生活	せいかつ	
\\	甘い	あまい	▲サトウキビは 甘[あま]い。	
\\	〔甘味の〕 
\\	〔塩加減が〕 
\\	[《文》 
\\	〔うっとりと快い〕 
\\	〔おだてるような・誠意のない〕 
\\	《文》 
\\	《文》 
\\	〔言いくるめるような〕 
\\	〔魅力的な〕 
\\	〔甘やかす〕 
\\	〔厳しくない〕 
\\	〔浅薄な〕 
\\	〔のん気な〕 
\\	〔与(くみ)しやすい〕 
\\	〔(野球などで)打ちやすい〕 
\\	〔鋭くない〕 
\\	〔(栓などが)ゆるい〕 
\\	【相場】 
\\	甘い	あまい	
\\	理由	りゆう	
\\	理由[りゆう]なく。	
\\	〔不当に〕 
\\	〔わけ・原因もなく〕 
\\	〔わけ〕 
\\	〔根拠〕 
\\	〔口実〕 
\\	理由	りゆう	
\\	珍しい	めずらしい	
\\	珍[めずら]しい 物好[ものず]き。	
\\	〔目新しい〕 
\\	〔まれな〕 
\\	〔珍奇な〕 
\\	〔常でない〕 
\\	〔結構な〕 
\\	珍しい	めずらしい	
\\	特急	とっきゅう	
\\	特急[とっきゅう]に 乗[の]る。	
\\	〔特別急行列車〕 
\\	《俗》 
\\	〔急ぎ〕	特急	とっきゅう	
\\	特に	とくに	▲これは 特[とく]に 本紙[ほんし]のために 寄[よ]せられた 受賞者[じゅしょうしゃ]の 言葉[ことば]である。	
\\	〔殊更に〕 
\\	[⇒とくべつ]	特に	とくに	
\\	片仮名	かたかな	
\\	片仮名[かたかな]で 外来語[がいらいご]を 表記[ひょうき]する。	
\\	片仮名	かたかな	
\\	熱心	ねっしん	
\\	非常[ひじょう]に 熱心[ねっしん]である。	
\\	《口》 
\\	~な 
\\	熱心	ねっしん	
\\	熱	ねつ	
\\	熱[ねつ]を 通[とお]さない 建材[けんざい]。	
\\	〔高い温度・それを生む運動エネルギー〕 
\\	〔体温〕 
\\	〔病気などによる平常より高い体温〕 
\\	【医】 〔発熱〕 
\\	〔熱意〕 
\\	〔熱中・流行〕 
\\	熱	ねつ	
\\	焼ける	やける	▲こんがり 焼[や]けた。	
\\	〔食物が〕 (肉が) 
\\	(魚肉が) 
\\	(パンを作るときに) 
\\	(パンをトーストにするときに) 
\\	(火にあぶって) 
\\	(オーブンで) 
\\	〔焼失する〕 
\\	〔肌が日に〕 
\\	〔灼熱する〕 
\\	〔地面が〕 
\\	[⇒ひやけ] 
\\	〔変色する〕 
\\	〔日ざしで損なわれる〕 
\\	〔空が〕 
\\	[⇒あさやけ, ゆうやけ] 
\\	〔胸が〕 
\\	[⇒むねやけ] 
\\	〔嫉妬する〕 
\\	(羨む) 
\\	焼ける	やける	
\\	焼く	やく	
\\	鶏肉[けいにく]は 焼[や]きすぎると 固[かた]くなる。	
\\	〔食物を〕 
\\	(肉や魚を焼き網で) 
\\	(パンを作るとき) 
\\	(パン切れや海苔を) 
\\	〔燃やしてなくす〕 
\\	〔火葬に付す〕 
\\	〔火で熱する〕 
\\	〔日光で肌を〕 
\\	(薬品などで) 
\\	〔熱で損なう〕 
\\	【医】 〔焼灼する〕 
\\	〔火を使って製品を作る〕 
\\	(陶磁器を) 
\\	(炭を) 
\\	(塩を) 
\\	【写真】 
\\	【電算】 
\\	などを〕 
\\	〔ねたむ〕 
\\	〔羨む〕 
\\	[⇒やきもち]	焼く	やく	
\\	無理	むり	
\\	無理[むり]を 言[い]う。	
\\	〔理不尽・不自然〕 
\\	~な 〔理に合わない〕 
\\	〔不当な〕 
\\	〔不自然な〕 
\\	~な 〔不可能な〕 
\\	〔困難な〕 
\\	〔強引にやること〕 
\\	〔過度にやること〕 
\\	〔過労〕 
\\	~な 
\\	~する =~をする. 
\\	〔金銭的負担の大きいこと〕 
\\	【数】	無理	むり	
\\	無くなる	なくなる	
\\	私[わたし]のハンカチがなくなった。	
\\	〔紛失する〕 
\\	〔消える・消失する〕 
\\	〔廃止になる〕 
\\	〔消滅する・撲滅される〕 
\\	〔ゼロになる〕 
\\	《口》 
\\	(使い果たして) 
\\	無くなる	なくなる	
\\	為	ため	
\\	人[ひと]のためを 思[おも]ってする。	
\\	〔利益〕 
\\	〔目的〕 
\\	〔原因・理由〕	為	ため	
\\	点	てん	
\\	番号[ばんごう] 順[じゅん]に 点[てん]をつなぐとウサギの 絵[え]になった。	
\\	[⇒点つなぎ].	
\\	〔微小な円形〕 
\\	〔記号としての〕 【数】 
\\	〔小数点〕 
\\	〔読点〕 ⇒とうてん6. 
\\	〔評価点〕 
\\	〔競技の得点〕 
\\	〔論点〕 
\\	〔ことがら〕 
\\	〔見地〕 
\\	〔箇所〕 
\\	【-点】 〔点数〕 
\\	【-点】 〔競技の得点数〕 
\\	【-点】 〔品物の個数〕 
\\	点	てん	
\\	火曜日	かようび	
\\	火曜日[かようび]に 国会[こっかい]が 開[ひら]かれる。	
\\	(略: 
\\	火曜日	かようび	
\\	火	ひ	
\\	火の気[ひのけ]のない。	
\\	〔ともったもの〕 
\\	〔火炎〕 
\\	《文》 
\\	〔火花〕 
\\	〔灼熱した状態〕 
\\	〔暖炉・火鉢などの〕 
\\	〔燠(おき)〕 
\\	〔炭火〕 
\\	〔暖房の熱〕 
\\	〔火事〕 
\\	〔調理の熱〕 
\\	〔調理の熱源〕 
\\	〔灯火〕 ⇒ひ3	火	ひ	
\\	濡れる	ぬれる	▲ぬれぬ 先[さき]の 傘[かさ]。	
\\	【諺】.	
\\	〔水滴がつく〕 
\\	〔しみ込む〕 
\\	〔ずぶぬれになる〕 
\\	〔情事を行う〕 
\\	濡れる	ぬれる	
\\	浸ける・漬ける	つける	▲ブリの 切り身[きりみ]をたれにつけておく。	
\\	【浸ける】 〔ひたす〕 
\\	《口》 
\\	【漬ける】 〔漬け物にする〕 
\\	漬ける	つける	
\\	漫画	まんが	▲これは 漫画[まんが]になる。	
\\	〔新聞雑誌の(通例 
\\	コマの)続き漫画〕 
\\	《口》 
\\	*《口》 
\\	〔風刺漫画〕 
\\	漫画	まんが	
\\	漢字	かんじ	
\\	漢字[かんじ]で 書[か]く。	
\\	漢字	かんじ	
\\	滑る	すべる	
\\	子供[こども]が) 階段[かいだん]のてすりを 滑[すべ]って 降[お]りる。	
\\	〔滑走する〕 
\\	(スケートで) 
\\	(スキーで) 
\\	〔登山・バレエで〕 
\\	〔つるつるする〕 
\\	〔足や手などが物をしっかりとらえられない〕 
\\	〔入試に失敗する, 不合格になる〕 
\\	*《口》 
\\	《口》 
\\	滑る	すべる	
\\	準備	じゅんび	
\\	準備[じゅんび]はすっかりできた。	
\\	〔用意〕 
\\	[=ようい1, したく1] ~する 
\\	〔備える〕 
\\	準備	じゅんび	
\\	湯	ゆ	
\\	湯[ゆ]も 水[みず]も 出[で]るシャワー。	
\\	〔熱せられた水〕 
\\	〔風呂〕 
\\	(銭湯) 
\\	〔温泉〕 
\\	【製鉄】 〔溶けた鉄〕 
\\	湯	ゆ	
\\	湖	こ	
\\	琵琶湖[びわこ]。	
\\	[=こすい1]	湖	みずうみ	
\\	港	みなと	
\\	港[みなと]に 出迎[でむか]える。	
\\	〔停泊地〕 
\\	港	みなと	
\\	暖かい・温かい	あたたかい	
\\	暖[あたた]かい 日[ひ]が 続[つづ]くとなだれが 起[お]きやすい。	
\\	〔気温や物の温度が適度に高い〕 
\\	〔心のこもった・思いやりのある・友好的な〕 
\\	《文》 
\\	〔色が暖色系である〕 
\\	〔経済状態がゆたかだ〕	温かい	あたたかい	
\\	済む	すむ	
\\	試験[しけん]が 済[す]んだ。	
\\	〔終了する〕 
\\	〔用が足りる〕 〈事が主語〉 
\\	〈人が主語〉 
\\	〔解決する〕 〈事が主語〉 
\\	〈人が主語〉 
\\	済む	すむ	
\\	深い	ふかい	
\\	深[ふか]い 傷[きず]。	
\\	〔底までの距離が長い〕 
\\	〔奥までの距離が長い〕 
\\	〔密度が濃い〕 
\\	〔色が濃い〕 
\\	〔関係が濃い〕 
\\	《口》 
\\	〔たけなわである〕 
\\	〔深遠な〕 
\\	〔程度が甚だしい〕 
\\	深い	ふかい	
\\	消しゴム	けしゴム	
\\	消[け]しゴムで 消[け]せない 落書[らくが]き。	
\\	消しゴム	けしごむ	
\\	海	うみ	
\\	地球[ちきゅう]の 温暖化[おんだんか]が 進[すす]むと この辺[このあた]りは 海[うみ]に 沈[しず]んでしまう。	
\\	〔海洋〕 
\\	〔海域〕 
\\	〔一面の広がり〕 
\\	【天】 〔月面の〕 
\\	〔硯(すずり)の〕 
\\	海	うみ	
\\	泳ぐ	およぐ	
\\	月[つき]の 空[そら]にこいのぼりが 元気[げんき]よく 泳[およ]いでいた。	
\\	〔水中を〕 
\\	〔水面上を〕 
\\	〔人込みなどを〕 
\\	〔世間を〕 
\\	〔体をふらつかせる〕 
\\	〔衣服・靴などが大きすぎて〕 
\\	泳ぐ	およぐ	
\\	注意	ちゅうい	
\\	注意深[ちゅういぶか]く 観察[かんさつ]する。	
\\	〔留意・注目〕 
\\	〔警戒・用心〕 
\\	~する 〔心に留める〕 
\\	〔用心する〕 
\\	〔忠告〕 
\\	〔警告〕 
\\	〔示唆〕 
\\	〔説諭〕 
\\	〔心得〕 
\\	~する 〔忠告する〕 
\\	〔警告する〕 
\\	〔説諭する〕 
\\	〔思い出させる〕 
\\	注意	ちゅうい	
\\	法律	ほうりつ	
\\	法律[ほうりつ]をかいくぐる。	
\\	〈集合的に〉 
\\	〔総称〕 
\\	法律	ほうりつ	
\\	治る	なおる	
\\	今年[ことし]の 風邪[かぜ]は 治[なお]りにくい。	
\\	〔病気が〕 (人が) 
\\	(病気が) 
\\	〔傷が〕 
\\	〔骨折が〕 
\\	治る	なおる	
\\	沸く	わく	
\\	風呂[ふろ]が 沸[わ]いているから 入[はい]りなさい。	
\\	〔熱くなる〕 
\\	〔沸騰する〕 
\\	〔声が高まる〕 
\\	〔興奮状態になる〕 
\\	[⇒わきかえる] 
\\	〔発酵する〕 
\\	〔金属が溶ける〕 
\\	沸く	わく	
\\	沸かす	わかす	
\\	風呂[ふろ]を 沸[わ]かしなおす。	
\\	〔熱くする〕 
\\	〔沸騰させる〕 
\\	〔興奮させる〕 =わかせる. 
\\	〔金属を〕 
\\	[=とかす2 
\\	沸かす	わかす	
\\	汽車	きしゃ	
\\	汽車[きしゃ]が 駅[えき]からシュッシュッポッポと 出[で]ていった。	
\\	〔列車〕 
\\	(電車に対し) 
\\	〔客車 
\\	両〕 
\\	[⇒れっしゃ1]	汽車	きしゃ	
\\	決める	きめる	
\\	来年[らいねん]の 予定[よてい]を 決[き]めかねている。	
\\	〔確定させる〕 
\\	(選んで) 
\\	(契約して) 
\\	(限定して) 
\\	(日時などを) 
\\	(結論を出す) 
\\	(決心する) 
\\	〔決定づける〕 
\\	〔話をつける〕 
\\	〔決着をつける〕 
\\	[⇒きめている] 
\\	〔勝負をつける〕 
\\	〔仕掛けを成功させる〕 
\\	〔そうすることに決めて押し通す〕 =きめこむ 
\\	〔型どおり整える〕 
\\	《口》 
\\	【相撲・柔道】 
\\	【演劇】 
\\	決める	きめる	
\\	決まる	きまる	
\\	開催[かいさい]の 場所[ばしょ]はまだ 決[き]まっていない。	
\\	〔確定する〕 
\\	〔話がつく〕 
\\	[⇒きまっている, きまった, きまって, さだまる] 
\\	〔勝負がつく〕 
\\	〔仕掛けが成功する〕 
\\	〔うまく固定される〕 
\\	〔歌舞伎などで〕 
\\	決まる	きまる	
\\	決して	けっして	
\\	彼[かれ]は 決[けっ]してそんな 人[ひと]ではない。	
\\	〔決して…ない〕 
\\	《口》 
\\	決して	けっして	
\\	池	いけ	
\\	池[いけ]で 泳[およ]いではいけません。	
\\	(小さな) 
\\	(庭園・公園などの) 
\\	(川・湖水などに通じる) 
\\	〔貯水池〕 
\\	池	いけ	
\\	汚れる	よごれる	
\\	汚れた手[けがれたて]で。	
\\	〔きたなくなる〕 
\\	〔けがれる〕 
\\	汚れる	よごれる	
\\	水泳	すいえい	
\\	水泳[すいえい]の 先生[せんせい]{コーチ}。	
\\	~する 
\\	水泳	すいえい	
\\	気持ち	きもち	▲がんばらなくていいと 言[い]われて 気持[きも]ちがより 楽[らく]になった。	
\\	〔心の持ち方〕 
\\	〔感じ〕 
\\	〔気分〕 
\\	〔印象〕 
\\	〔生理的な気分〕 
\\	〔真心〕 
\\	〔少しばかり〕 
\\	気持ち	きもち	
\\	気分	きぶん	▲この 香[かお]りはいらいらした 気分[きぶん]をやわらげてくれます。	
\\	〔気持ち〕 
\\	〔体の調子〕 
\\	〔雰囲気〕 
\\	〔風潮〕 
\\	気分	きぶん	
\\	気	き	
\\	気[き]を 強[つよ]く 持[も]つ。	
\\	〔精神・心・意識〕 
\\	〔気質・性格〕 
\\	〔意向〕 
\\	〔意志〕 
\\	〔意欲〕 
\\	〔意気・気力〕 
\\	〔気持ち・気分〕 
\\	〔関心〕 
\\	〔注意力・心遣い〕 
\\	〔空気〕 
\\	〔大気〕 
\\	〔ガス〕 
\\	〔蒸発気〕 
\\	〔雰囲気〕 
\\	〔精気〕 
\\	【中国哲学】 〔生命の根元・生命力〕 
\\	〔香気〕 
\\	〔風味〕 
\\	〔燻気〕 
\\	気	き	
\\	毛	け	▲ウェーブのかかった 毛[もう]。	
\\	〔頭髪〕 
\\	〔体毛〕 
\\	[⇒かみのけ] 
\\	〔動物の体毛〕 
\\	(羊の) 
\\	(豚の) 
\\	【動】 
\\	〔鳥類の羽毛〕 
\\	〔ペンギンの〕 
\\	〔繊維素材〕 (羊の) 
\\	(子羊の) 
\\	(ウサギの) 
\\	(ラクダの) 
\\	〔製品の羽毛〕 
\\	【植】 〔葉・茎の表面の〕 
\\	〔わずかなこと〕	毛	け	
\\	比べる	くらべる	
\\	去年[きょねん]に 比[くら]べて 今年[ことし]は 収穫[しゅうかく] 量[りょう]が 少[すく]ない。	
\\	〔比較する〕 
\\	〔対照する〕 
\\	〔優劣を競う〕 
\\	比べる	くらべる	
\\	残る	のこる	
\\	ガラス窓[がらすまど]に 指紋[しもん]が 残[のこ]っていた。	
\\	〔消えずにある〕 
\\	(後世まで) 
\\	〔記録される〕 
\\	〔使われずにある〕 
\\	〔処理されずにある〕 
\\	〔放置される〕 
\\	〔居続ける〕 
\\	〔選抜される〕 
\\	〔死なずにいる〕 
\\	〔もちこたえる〕 
\\	《口》 
\\	【相撲】 
\\	残る	のこる	
\\	歴史	れきし	
\\	歴史[れきし]を 作[つく]る。	
\\	〔変遷の跡〕 
\\	〔変遷の記録〕 
\\	〔来歴・由緒〕 
\\	〔伝統〕 
\\	歴史	れきし	
\\	歯医者	はいしゃ	
\\	歯医者[はいしゃ]にかかる。	
\\	歯医者	はいしゃ	
\\	歩く	あるく	▲スキーをつけたままだとうまく 歩[ある]けない。	
\\	《口》 
\\	【野球】 
\\	歩く	ありく	
\\	正しい	ただしい	▲どちらが 正[ただ]しいか 調[しら]べる。	
\\	〔道理にかなっている〕 
\\	〔道徳・正義にかなっている〕 
\\	〔誤りのない〕 
\\	〔正確な〕 
\\	〔本物の〕 
\\	〔純粋な〕 
\\	〔本当の・本来の〕 
\\	〔合法の〕 
\\	正しい	ただしい	
\\	止める・停める	とめる	
\\	供給[きょうきゅう]を 止[と]める。	
\\	〔稼働を停止する〕 
\\	(電源を切る) 
\\	〔途絶えさせる〕 
\\	〔通過を阻止する〕 
\\	〔停車させる〕 
\\	〔停船させる〕 
\\	〔やめさせる〕 
\\	〔禁止する〕 
\\	〔静める〕 
\\	〔制御する・進行を抑える〕 
\\	止める	とめる	
\\	止む・已む	やむ	
\\	雨[あめ]がやんだ。	
\\	〔雨が〕 
\\	〔止まる〕 
\\	〔風などがおさまる〕 
\\	〔終わる〕 
\\	〔静まる・済む〕 
\\	〔(音などが)消える〕 
\\	止む	やむ	
\\	歌う	うたう	▲ピアノに 合[あ]わせて 歌[うた]う。	
\\	〔節をつけて発声する〕 
\\	〔詩歌を作る〕 
\\	〔詩歌を朗読する〕 
\\	歌う	うたう	
\\	歌	うた	
\\	海[うみ]が 見[み]えたときこの 歌[うた]が 口[くち]をついて 出[で]た。	
\\	〔歌謡〕 
\\	〔歌うこと〕 
\\	〔詩歌〕 
\\	〈総称〉 
\\	〔和歌〕 
\\	〔歌詞〕 
\\	歌	うた	
\\	機会	きかい	▲この 機会[きかい]を 逃[のが]すな。	
\\	機会	きかい	
\\	楽しむ	たのしむ	▲たっぷり
\\	心[こころ]ゆくまで} 楽[たの]しむ。	
\\	〔楽しい気持ちをもつ〕 
\\	〔娯楽にする〕 
\\	楽しむ	たのしむ	
\\	楽しみ	たのしみ	
\\	あの人[あのひと]は 何[なに]が 楽[たの]しみで 生[い]きているのだろう。	
\\	〔楽しさ〕 
\\	〔楽しいこと〕 (娯楽) 
\\	(気晴らし) 
\\	(慰安) 
\\	(趣味) 
\\	〔期待〕 
\\	[⇒おたのしみ]	楽しみ	たのしみ	
\\	椅子	いす	
\\	椅子[いす]を 並[なら]べる。	
\\	〔腰掛け〕 
\\	(背もたれのない) 
\\	(ピアノの) 
\\	(オーケストラ奏者の) 
\\	(ベンチ) 
\\	〔地位〕 
\\	[=ちい2]	椅子	いす	
\\	棚	たな	
\\	棚[たな]には 本[ほん]や
\\	が 置[お]いてある。	
\\	〔物を置くための〕 
\\	〈集合的〉 
\\	(網・格子状の) 
\\	(壁から出張ったもの) 
\\	(炉の側面の) 
\\	(炉の上の) 
\\	〔フジ・ブドウなどの〕 
\\	〔海中で魚が群れている層〕 
\\	棚	たな	
\\	校長	こうちょう	▲マイクは 校長先生[こうちょうせんせい]である 大垣[おおがき] 先生[せんせい]を 訪[たず]ねました。	
\\	(女性) 
\\	は米国では私立学校の校長の意に用いる.	校長	こうちょう	
\\	柔道	じゅうどう	
\\	柔道[じゅうどう]の 有段者[ゆうだんしゃ]。	
\\	[⇒柔道の技]	柔道	じゅうどう	
\\	柔らかい	やわらかい	
\\	柔[やわ]らかくなるまで 煮[に]る。	
\\	〔柔軟な〕 
\\	(ぐにゃぐにゃした) 
\\	(展性のある) 
\\	(果物など熟れすぎて) 
\\	〔穏和な〕 
\\	(光などが) 
\\	〔堅苦しくない〕 
\\	〔融通の利く〕 
\\	柔らかい	やわらかい	
\\	木綿	もめん	
\\	木綿[もめん]のような。	
\\	木綿	もめん	
\\	朝寝坊	あさねぼう	
\\	朝寝坊[あさねぼう]の 人[ひと]には 朝顔[あさがお]の 花[はな]が 咲[さ]くところは 見[み]られない。	
\\	〔事〕 
\\	〔人〕 
\\	~する 
\\	朝寝坊	あさねぼう	
\\	朝	ちょう	
\\	仁徳天皇[にんとくてんのう]の 朝[あさ]。	
\\	〔王朝〕 
\\	〔治世〕 
\\	〔時代〕 
\\	〔朝廷〕 
\\	朝	あさ	
\\	最近	さいきん	
\\	最近[さいきん]の 流行[りゅうこう]。	
\\	〔このごろ〕 
\\	最近	さいきん	
\\	最初	さいしょ	
\\	わが社[わがしゃ]の 今世紀[こんせいき] 最初[さいしょ]の 出版[しゅっぱん]はこの 本[ほん]であった。	
\\	〔冒頭〕 
\\	〔一番目〕 
\\	~の 〔冒頭の〕 
\\	〔一番目の〕 
\\	〔初回〕 
\\	~の 
\\	〔当初〕 ~の 
\\	最初	さいしょ	
\\	書く・描く	かく	▲この 切符[きっぷ]にはシベリア 経由[けいゆ]パリ 行[い]きと 書[か]いてある。	
\\	【書く】 
\\	〔書き記す〕 
\\	〔記述する〕 
\\	【描く】 (線で) 
\\	(彩色して) 
\\	〔眉を〕 
\\	書く	かく	
\\	暮れる	くれる	
\\	日[にち]が 暮[く]れかかっていた。	
\\	〔夜になる〕 
\\	〔過ぎ去る〕 
\\	〔終わる〕 
\\	〔暗い気持ちが続く〕 
\\	暮れる	くれる	
\\	暑い	あつい	
\\	焼[や]けつくように 暑[あつ]い。	
\\	《口》 
\\	暑い	あつい	
\\	景色	けしき	▲すばらしい 景色[けしき]。	
\\	〔眺め〕 
\\	(茶陶の) 
\\	景色	けしき	
\\	普通	ふつう	
\\	普通[ふつう]の 場合[ばあい]に(は)。	
\\	普通	ふつう	
\\	時代	じだい	▲…と 時代[じだい]を 同[おな]じくする。	
\\	〔時期〕 
\\	〔年代〕 
\\	〔世代; 一代〕 
\\	〔当世・当代・現代〕 
\\	〔ある面から特徴づけられた時代・時勢・時流〕 
\\	〔古色〕 
\\	時代	じだい	
\\	昼間	ひるま	▲コウモリは 昼間[ひるま]は 活動[かつどう]しない。	
\\	〔日の出から日の入りまで〕 
\\	昼間	ひるま	
\\	昼休み	ひるやすみ	
\\	昼休[ひるやす]みを 取[と]る。	
\\	昼休み	ひるやすみ	
\\	是非	ぜひ	
\\	是非[ぜひ]に 及[およ]ばぬ。	
\\	〔善悪〕 
\\	〔当否〕 
\\	《文》 
\\	~する 
\\	〔必ず〕 
\\	(《文》 
\\	是非	ぜひ	
\\	映画館	えいがかん	
\\	突然[とつぜん]、 映画館[えいがかん]で 火事[かじ]が 起[お]きた。	
\\	映画館	えいがかん	
\\	映画	えいが	
\\	映画[えいが]を 見[み]せる。	
\\	《口》 
\\	〈総称〉 
\\	映画	えいが	
\\	星	ほし	
\\	星[ほし]の 観察[かんさつ]。	
\\	〔天体〕 
\\	〔総称〕 
\\	〔模様としての〕 
\\	〔斑点〕 
\\	〔目の〕 
\\	〔馬などの顔の〕 
\\	〔(的の)中心〕 
\\	〔碁盤の〕 
\\	〔運勢〕 
\\	[⇒ほしうらない] 
\\	〔勝負の成績〕 
\\	〔格付けの〕 
\\	〔犯人〕 
\\	《俗》 
\\	〔注目される人物・花形〕 
\\	〔花火の〕 
\\	星	ほし	
\\	昔	むかし	
\\	昔[むかし]に 立[た]ち 帰[かえ]る。	
\\	〔遠い過去〕 
\\	〔往時〕 
\\	昔	むかし	
\\	明日	あした	
\\	君[くん]あしたの 予定[よてい]は?	
\\	〔今日の次の日〕 
\\	[=あす] 
\\	〔将来〕 
\\	[=あす]	明日	あした	
\\	日記	にっき	
\\	鍵[かぎ] 付[つ]きの 日記[にっき]。	
\\	日記	にっき	
\\	日	ひ	
\\	真っ黒[まっくろ]に 日[ひ]に 焼[や]けた 子供[こども]たち。	
\\	〔太陽〕 
\\	〔日光〕 
\\	〔日差し〕 
\\	〔昼間の時間〕 
\\	時間〕 
\\	〔ある一日〕 
\\	〔日々; 日数〕 
\\	〔日付〕 
\\	〔日取り〕 
\\	〔日がら〕 
\\	日	ひ	
\\	旅館	りょかん	
\\	旅館[りょかん]の 主人[しゅじん]。	
\\	〔和式の〕 
\\	旅館	りょかん	
\\	-方	-かた	
\\	歩き方[あるきかた]。	
\\	〔その方法〕 
\\	〔一方の側〕 
\\	〔係〕 
\\	〔それをすること〕 
\\	〔寄留先の家を示す語〕 
\\	(略: 
\\	〔敬称で人数〕 
\\	方	かた	
\\	文法	ぶんぽう	▲それは 文法[ぶんぽう]が 間違[まちが]っている。	
\\	文法	ぶんぽう	
\\	文学	ぶんがく	
\\	不健全[ふけんぜん]な 文学[ぶんがく]。	
\\	【図書】 
\\	文学	ぶんがく	
\\	文化	ぶんか	
\\	文化[ぶんか]の 違[ちが]いに 驚[おどろ]く。	
\\	〔文明〕 
\\	文化	ぶんか	
\\	数学	すうがく	
\\	数学的[すうがくてき] 正確[せいかく]さで。	
\\	*《口》 
\\	《口》 
\\	数学	すうがく	
\\	教育	きょういく	
\\	私[わたし]は 入社[にゅうしゃ]して 電話[でんわ]の 受け方[うけかた]をまず 教育[きょういく]された。	
\\	〔教えること〕 
\\	〔教授〕 
\\	〔訓練〕 
\\	〔養育〕 
\\	~する 
\\	〔教える〕 
\\	〔仕込む〕 
\\	〔教育で身についたもの〕 
\\	教育	きょういく	
\\	教会	きょうかい	
\\	教会[きょうかい]の。	
\\	〔キリスト教の団体・宗派〕 
\\	〔教会堂〕 
\\	教会	きょうかい	
\\	放送	ほうそう	
\\	放送[ほうそう]を 終了[しゅうりょう]する。	▲〈局が〉 
\\	〔一回の放送(物)〕 
\\	(テレビの) 
\\	[⇒ラジオ, テレビ] ~する 〈局が主語〉 
\\	〈人が主語〉 
\\	放送	ほうそう	
\\	支度・仕度	したく	
\\	食事[しょくじ]のしたくをする。	
\\	《口》 
\\	〔用意〕 
\\	〔身仕度〕 
\\	支度	したく	
\\	揺れる	ゆれる	
\\	前後左右[ぜんごさゆう]に 揺[ゆ]れる。	
\\	(震える) 
\\	(震動する) 
\\	〔船などが〕 (縦に) 
\\	(横に) 
\\	(前後に) 
\\	(上下に) 
\\	〔天秤などが〕 
\\	〔電車などが〕 
\\	(ぐいと急に) 
\\	〔つるしたものが〕 
\\	〔火が〕 
\\	〔炎が〕 
\\	〔心が〕 
\\	〔社会などが〕 
\\	揺れる	ゆれる	
\\	捜す・探す	さがす	▲もっとよく 捜[さが]せよ。	
\\	【捜す】 〔なくした物を〕 
\\	〔犯人などを〕 
\\	【探す】 〔目的の物を〕 
\\	〔職や人材を〕 
\\	探す	さがす	
\\	捨てる	すてる	
\\	産業廃棄物[さんぎょうはいきぶつ]を 不法[ふほう]に 捨[す]てる。	
\\	〔投げすてる〕 
\\	〔ごみなどを〕 
\\	《口》 
\\	〔緊急時に船外・機外に積み荷などを〕 
\\	〔トランプで手札を〕 
\\	〔考えなどをやめる〕 
\\	〔手離す・見捨てる〕 
\\	《口》 
\\	*《口》 
\\	〔犠牲にする〕 
\\	〔交渉を絶つ〕 
\\	〔放置する・棄却する〕 
\\	〔あきらめる・放棄する〕 
\\	〔乗ることをやめる〕	捨てる	すてる	
\\	捕まえる・掴まえる・捉まえる	つかまえる	▲つかまえられたサル。	
\\	〔拘束する・捕獲する〕 
\\	〔自分のものにする〕 
\\	〔逮捕する〕 
\\	*《口》 
\\	《口》 
\\	〔呼び止める〕 
\\	〔引き留める〕 
\\	〔放さないようにする〕 
\\	捕まえる	つかまえる	
\\	挨拶	あいさつ	
\\	彼[かれ]をそんなによく 知[し]っているわけではありません. 朝夕[あさゆう]の 挨拶[あいさつ]を 交[か]わす 程度[ていど]ですから。	
\\	〔出会いや別れのときなどにする会釈などの動作や言葉〕 
\\	~する 
\\	〔軍人が〕 
\\	〔式や会合での儀礼的演説〕 
\\	《文》 
\\	~する 
\\	《文》 
\\	〔ひとこと〕 
\\	〔社交上の儀礼〕 
\\	〔返事〕 
\\	〔詫び〕 
\\	〔(予告の)通知〕 
\\	~する 〔返事する〕 
\\	〔詫びる〕 
\\	〔承諾を求める〕 
\\	〔通知する〕 
\\	〔(やくざなどの世界での)仕返し〕 
\\	〔(相手の失礼な言葉や態度に対する)皮肉をこめた応酬〕	挨拶	あいさつ	
\\	指輪・指環	ゆびわ	
\\	金の指輪[きんのゆびわ]。	
\\	指輪	ゆびわ	
\\	指	ゆび	
\\	本[ほん]の 指[ゆび]。	▲〔手の〕 
\\	〔足の〕 
\\	"〔手の〕 
\\	〔足の〕 
\\	➡通常 親指 
\\	は 
\\	から除外し, 人差し指から 
\\	と数えるが, 
\\	本の指を使うピアノなどの鍵盤楽器の指示では親指を 
\\	と数える.
\\	指	ゆび	
\\	拾う	ひろう	
\\	息子[むすこ]が 犬[いぬ]を 拾[ひろ]ってきた。	
\\	〔足元から取り上げる〕 
\\	〔拾得する〕 
\\	〔のら犬・のら猫を〕 
\\	〔獲得する〕 
\\	〔失わずに済む〕 
\\	〔選び取る〕 
\\	(ひとつひとつ) 
\\	〔見つけ出す〕 
\\	〔見つけて集める〕 
\\	〔音や電波をとらえる〕 
\\	〔職を与える〕 
\\	〔途上で乗車させる〕 
\\	〔(球技で)かろうじて打ち返す〕 
\\	〔徒歩で行く〕 
\\	拾う	ひろう	
\\	押す	おす	
\\	私[わたし]は 後[うし]ろから 押[お]されて 前[まえ]につんのめった。	
\\	〔前方などに力を加える・力を加えて動かす〕 
\\	(頭や角で) 
\\	(ひじなどでちょっと押す) 
\\	〔指などで上から押しつける・機械のボタンなどを操作する〕 
\\	〔圧迫する〕 
\\	(強く) 
\\	〔ミカンなどを汁を搾るために〕 
\\	〔空気などを圧縮する〕 
\\	〔印を押す〕 
\\	《文》 
\\	〔箔(はく)を押す〕 
\\	〔優勢である〕 
\\	〔相手に対し強引に働きかける・無理やり押し通す〕 
\\	〔無理をする・我慢する〕 
\\	〔確かめる〕	押す	おす	
\\	投げる	なげる	
\\	池[いけ]の 鯉[こい](こい)にえさを 投[な]げる。	
\\	〔ほうる〕 
\\	(軽く) 
\\	〔たおす〕 
\\	〔あきらめてやめる〕 
\\	《口》 
\\	[⇒なげだす 
\\	〔ある方向に向ける〕 
\\	〔届かせる〕 
\\	投げる	なげる	
\\	技術	ぎじゅつ	▲さらに 技術[ぎじゅつ]を 磨[みが]く。	
\\	〔わざ〕 
\\	〔科学・工業の〕 
\\	〔科目〕 
\\	[⇒ぎじゅつかていか]	技術	ぎじゅつ	
\\	承知	しょうち	
\\	百[ひゃく]も 承知[しょうち]している。	
\\	〔知っていること〕 
\\	〔承諾〕 
\\	〔承認〕 
\\	〔容認〕 
\\	~する 〔承諾する〕 
\\	〔承認する〕 
\\	〔容赦〕 ~する 〔通例否定の形で〕 
\\	承知	しょうち	
\\	払う	はらう	
\\	煤[すす](すす)を 払[はら]う。	
\\	〔きれいにする〕 
\\	〔除去する〕 
\\	〔切り払う〕 
\\	〔追い払う〕 
\\	〔追放する〕 
\\	〔よける〕 
\\	〔横になぐ〕 
\\	〔支払う〕 
\\	《口》 
\\	〔目的のために提供する〕 
\\	〔心を注ぐ〕 
\\	〔字画の末を筆を払うようにして書く〕 
\\	払う	はらう	
\\	手袋	てぶくろ	
\\	手袋[てぶくろ]が 大[おお]きすぎて 指先[ゆびさき]が 余[あま]る。	
\\	〔各指が分かれた〕 
\\	〔親指だけ分かれた〕 
\\	〔腕まで覆う〕 
\\	手袋	てぶくろ	
\\	戻る	もどる	▲またすっかり 夏[なつ]に 戻[もど]ったみたいだ。	
\\	〔引き返す〕 
\\	(船が) 
\\	〔帰る〕 
\\	〔帰宅する〕 
\\	〔元へかえる〕 
\\	(ばねが) 
\\	(ゴムなどが) 
\\	(よりが) 
\\	〔持ち主に返される〕 
\\	(盗品などが) 
\\	戻る	もどる	
\\	意味	いみ	
\\	真[しん]の 意味[いみ]で。	
\\	〔表す内容〕 
\\	(単語や文の) 
\\	(記号の) 
\\	(態度や行為の) 
\\	(事実の) 
\\	〔主旨〕 
\\	《文》 
\\	(条文などの) 《文》 
\\	〔含み〕 
\\	~する (単語や文が) 
\\	(記号が) 
\\	(態度・行為や事実が) 
\\	〔含意する〕 
\\	〔価値〕 
\\	《文》 
\\	〔重要性〕 
\\	意味	いみ	
\\	恥ずかしい	はずかしい	
\\	君[くん]はそんなことをして 恥[は]ずかしくないか。	
\\	〔面目ない〕 
\\	〔照れくさい・きまりが悪い〕 
\\	恥ずかしい	はずかしい	
\\	怪我	けが	▲けがしなかったかい?	
\\	〔負傷〕 
\\	〔失策〕 
\\	~する 
\\	〔不測のこと〕 
\\	怪我	けが	
\\	急行	きゅうこう	
\\	急行[きゅうこう]させる。	
\\	〔急いで行くこと〕 ~する 
\\	〔急行列車〕 
\\	[⇒きゅうこうれっしゃ]	急行	きゅうこう	
\\	急	きゅう	
\\	急[きゅう]な 用事[ようじ]で。	
\\	〔突然なさま〕 ~な 
\\	〔緊急の〕 
\\	〔傾きが大きいさま〕 ~な 
\\	(角度が鋭い) 
\\	〔速度が早いさま〕 ~な 〔流れが〕 
\\	〔調子が〕 
\\	急	きゅう	
\\	怖い	こわい	
\\	怖[こわ]くなって 逃[に]げる。	
\\	〔おそろしい〕 
\\	〔悪い結果を生む〕 
\\	〔不気味に思う〕 
\\	怖い	こわい	
\\	忙しい	いそがしい	
\\	年の瀬[としのせ]はなんとなく 忙[いそが]しい。	
\\	《口》 
\\	忙しい	いそがしい	
\\	忘れ物	わすれもの	
\\	列車[れっしゃ]などで〕お 降[ふ]りの 際[さい]はお 忘れ物[わすれもの]のないよう ご注意[ごちゅうい]ください。	
\\	〔遺失物〕 
\\	忘れ物	わすれもの	
\\	必要	ひつよう	▲…が 必要[ひつよう]になる。	
\\	~な 〔無いと困る〕 
\\	〔不可欠の〕 
\\	〔しなければならない〕 
\\	必要	ひつよう	
\\	心配	しんぱい	
\\	心配[しんぱい]の 種[たね]。	
\\	〔気がかり〕 
\\	《文》 
\\	〔不安〕 
\\	〔心づかい〕 
\\	~な 
\\	~する 〔憂慮する〕 
\\	〔不安を感じる〕 
\\	〔心を遣う〕 
\\	心配	しんぱい	
\\	心	しん	
\\	慈悲心[じひしん]。	
\\	〔こころ〕 
\\	[=こころ] 
\\	〔心底〕 =しんそこ. 
\\	〔心身の深部〕 
\\	[⇒しん2 
\\	心	こころ	
\\	復習	ふくしゅう	
\\	復習[ふくしゅう] 用[よう]の 練習[れんしゅう]
\\	問[とい]) 題[だい]。	
\\	(仕事関係の) 
\\	~する 
\\	復習	ふくしゅう	
\\	後ろ	うしろ	
\\	私[わたし]の 後[うし]ろにきちんと 並[なら]んでください。	
\\	[=あと1 
\\	後ろ	うしろ	
\\	後	あと	
\\	後[ご]に ついて行[ついてい]く。	
\\	〔後方〕 
\\	[⇒うしろ] 
\\	〔これ[それ]以後〕 
\\	〔次〕 
\\	〔今後〕 
\\	〔最後〕 
\\	〔結果〕 
\\	〔結末〕 
\\	〔残り〕 
\\	〔なごり〕 
\\	〔子孫〕 
\\	《文》 
\\	〔後継者〕 
\\	[⇒あと2 
\\	〔後妻〕 
\\	〔追加〕 
\\	〔(接続詞的に)それから〕 
\\	後	あと	
\\	弟	てい	
\\	両者[りょうしゃ]は) 兄[けい]たり 難[かた]く 弟[おとうと]たり 難[むずか]し。	▲⇒けい1 
\\	〔兄(けい)に対する〕 
\\	[=おとうと]	弟	おとうと	
\\	引っ越す	ひっこす	▲これまでもう6 回[かい]も 引っ越[ひっこ]した。	
\\	引っ越す	ひっこす	
\\	引き出す	ひきだす	
\\	厩[うまや](うまや)から 馬[うま]を 引き出[ひきだ]す。	
\\	〔外へ出す〕 
\\	(引きずって) 
\\	〔顕在化させる〕 (能力を) 
\\	(味や色を) 
\\	〔金を出させる〕 
\\	〔回答を得る〕 
\\	〔預貯金を〕 
\\	〔手に入れる〕 
\\	引き出す	ひきだす	
\\	引き出し	ひきだし	
\\	私[わたし]の 頭[あたま]の 中[なか]にはアイデアの 引き出[ひきだ]しがたくさんある。	
\\	〔机などの〕 
\\	〔預貯金の〕 
\\	引き出し	ひきだし	
\\	床屋	とこや	
\\	床屋[とこや]のいす。	
\\	〔人〕 
\\	〔店〕 
\\	《戯言》 
\\	床屋	とこや	
\\	帰る	かえる	
\\	泣[な]いて 帰[かえ]る 泣[な]き 帰[かえ]る。	
\\	〔元の場所に戻る〕 
\\	〔以前の状態に戻る〕 
\\	〔去る〕 
\\	帰る	かえる	
\\	席	せき	▲(よく 見[み]えるように) 席[せき]を 移動[いどう]する。	▲〔動かす〕 
\\	〔移る〕 
\\	〔座席〕 
\\	区画の) 
\\	〔すわる余地〕 
\\	[⇒ざせき] 
\\	〔場所〕 
\\	〔室〕 
\\	〔場合〕 
\\	〔地位・身分〕 
\\	席	せき	
\\	布団・蒲団	ふとん	
\\	自分[じぶん]の 手[て]で) 布団[ふとん]を 作[つく]る。	
\\	〔夜具〕 
\\	〔座布団〕 
\\	布団	ふとん	
\\	市民	しみん	
\\	京都[きょうと] 市民[しみん]。	
\\	〔個人〕 
\\	〈集合的に〉 
\\	市民	しみん	
\\	市	し	
\\	市[し]の 花[はな]
\\	鳥[とり]}。	
\\	〔都市〕 
\\	〔行政区画〕 
\\	市	し	
\\	差し上げる	さしあげる	
\\	子供[こども]を 高[たか]く 差し上[さしあ]げる。	
\\	〔持ち上げる〕 
\\	〔進呈する〕 
\\	[=あげる1 
\\	〔(動詞の連用形に「て(で)」を伴い)敬意や丁寧な気持ちを添える〕 
\\	差し上げる	さしあげる	
\\	工業	こうぎょう	
\\	工業[こうぎょう]の。	
\\	〈総称〉 
\\	工業	こうぎょう	
\\	工場	こうじょう	
\\	工場[こうじょう]を 閉鎖[へいさ]する。	
\\	〔ストのため〕 
\\	[=こうば1]	工場	こうじょう	
\\	川・河	かわ	
\\	神社[じんじゃ]の 脇[わき]を 川[かわ]が 流[なが]れている。	
\\	〔流れ〕 
\\	〔小川〕 
\\	〔大河〕 
\\	川	かわ	
\\	島	しま	
\\	島伝[しまづた]いに。	
\\	《雅》 
\\	〔小島〕 
\\	〔やくざなどの縄張り〕 
\\	島	しま	
\\	展覧会	てんらんかい	
\\	展覧会[てんらんかい]を 開[ひら]く
\\	見[み]に 行[い]く}。	
\\	〔小規模の〕 
\\	展覧会	てんらんかい	
\\	届ける	とどける	
\\	拾[ひろ]った 書類[しょるい]を 交番[こうばん]に 届[とど]けた。	
\\	〔持っていく〕 
\\	〔書類や口頭で報告する〕 
\\	届ける	とどける	
\\	小鳥	ことり	
\\	小鳥[ことり]に 餌[えさ]をやる。	
\\	〔鳴禽〕 
\\	《口・小児》 
\\	小鳥	ことり	
\\	小説	しょうせつ	
\\	漱石[そうせき]の 小説[しょうせつ]に 描[えが]かれた 明治時代[めいじじだい]の 生活[せいかつ]。	
\\	小説	しょうせつ	
\\	尋ねる	たずねる	
\\	先方[せんぽう]
\\	相手[あいて]}の 意向[いこう]を 尋[たず]ねる。	
\\	〔問う〕 
\\	〔捜す〕 
\\	〔捜し求める〕 
\\	尋ねる	たずねる	
\\	将来	しょうらい	▲この 子[こ]の 将来[しょうらい]が 楽[たの]しみだ。	
\\	〔未来〕 
\\	〔将来の見込み〕 
\\	〔未来において〕 
\\	《文》 
\\	《口》 
\\	〔今後〕 ⇒こんご; 〔いつか〕 
\\	〔持ってくること〕 ~する 
\\	《文》 
\\	〔ある結果を招くこと〕 ⇒しょうらい1 
\\	将来	しょうらい	
\\	寺	てら	
\\	寺[てら]に 参[まい]る{もうでる}。	
\\	寺	てら	
\\	寝坊	ねぼう	
\\	今朝[けさ]はついうっかり 寝坊[ねぼう]してしまった。	
\\	〔寝過ごすこと〕 
\\	〔その癖のある人〕 
\\	《文》 
\\	~する 
\\	寝坊	ねぼう	
\\	寄る	よる	
\\	少[すこ]し 北[きた]へ 寄[よ]る。	
\\	【海】 
\\	〔近づく〕 
\\	〔にじりよる〕 
\\	【相撲】 
\\	〔もたれる〕 
\\	〔集まる〕 
\\	〔一方の側に集まる〕 
\\	〔立ち寄る〕 
\\	〔寄港する〕 
\\	〔ついでに〕 *《口》 
\\	【取引】 〔取り引きが成立する〕 
\\	寄る	よる	
\\	寂しい	さびしい	▲ふところが 寂[さび]しい。	
\\	〔にぎやかでない〕 
\\	〔荒涼たる〕 
\\	〔孤独な〕 
\\	寂しい	さびしい	
\\	家族	かぞく	
\\	私[わたし]の 立候補[りっこうほ]を 家族[かぞく]みんなが 応援[おうえん]してくれた。	
\\	〔縁・婚姻などによって結ばれ, 共同生活を行っている人々〕 
\\	〔本人以外の家族構成員〕 
\\	家族	かぞく	
\\	家内	かない	
\\	家内安全[かないあんぜん]を 祈[いの]る。	
\\	〔家の内・家族〕 
\\	《口》 
\\	〔自分の妻〕 
\\	家内	かない	
\\	家	いえ	
\\	住[す]む 家[いえ]がない。	
\\	〔家屋〕 
\\	〔住宅〕 
\\	〔自宅〕 
\\	[⇒うち2] 
\\	〔家庭〕 
\\	〔所帯〕 
\\	〔家名〕 
\\	〔家督〕 
\\	〔家業〕 
\\	〔家柄〕 
\\	〔家系〕 
\\	〔由緒ある家柄〕 
\\	〔親族団体〕 
\\	〔家制度での一団〕 
\\	家	いえ	
\\	宜しい	よろしい	▲あなたは 頭[あたま]がおよろしいから…。	
\\	〔よい・結構だ・好ましい〕 =よい3. 
\\	〔承認・許可〕 
\\	〔意にかなう〕 
\\	〔不要〕 
\\	宜しい	よろしい	
\\	学生	がくせい	
\\	学生[がくせい]が 集[あつ]まらない 私学[しがく]。	
\\	〔一校全体の〕 
\\	学生	がくせい	
\\	学校	がっこう	
\\	今度[こんど]の 土曜日[どようび]は 学校[がっこう]がある。	
\\	〔専門・大学程度の〕 
\\	〔中学以上高等学校程度の〕 
\\	〔神学などの〕 
\\	〔総称的に〕 
\\	学校	がっこう	
\\	季節	きせつ	
\\	新[しん]じゃがの 出回[でまわ]る 季節[きせつ]。	
\\	〔気候区分〕 
\\	〔ある特徴をもった時期〕 
\\	季節	きせつ	
\\	子	こ	
\\	金持[かねも]ちの 子[こ]に 生[う]まれたかった。	
\\	〔子女〕 
\\	《文》 
\\	《口》 
\\	(息子) 
\\	(娘) 
\\	(跡取り); 
\\	(男) 
\\	〈集合的に〉 【主に法】 
\\	(一般に) 
\\	[=こども 
\\	〔幼少者〕 
\\	[⇒こ7] 
\\	〔人間〕 
\\	〔動物の仔〕 
\\	匹) 
\\	《文》 
\\	〈集合的に〉 
\\	〔稚魚〕 
\\	〔魚卵〕 〈集合的に〉 
\\	【植】 〔植物の分生〕 
\\	〔トランプや花札遊びの〕 
\\	子	こ	
\\	娘	むすめ	
\\	娘[むすめ]の 婚約者[こんやくしゃ]に)ふつつかな
\\	至[いた]らぬ} 娘[むすめ]ですがよろしく お願[おねが]いします。	
\\	➡この日本語のようなへり下った言い方は英米ではしない.	
\\	〔親にとって, 女の子〕 
\\	〔未婚の若い女性〕 
\\	〔少女〕 
\\	娘	むすめ	
\\	姉	あね	
\\	実[み]の 姉[あね]。	
\\	➡年長であることを特に表現しない最後の言い方が英語としては普通.	姉	あね	
\\	具・妻	つま	
\\	話[はなし]のつま。	
\\	〔添え物〕 
\\	【料理】 
\\	妻	つま	
\\	女の子	おんなのこ	
\\	女の子[おんなのこ] 向[む]きの 雑誌[ざっし]。	
\\	〔少女〕 
\\	〔娘〕 
\\	〔女の赤ちゃん〕 
\\	〔若い女性〕 
\\	女の子	おんなのこ	
\\	女	おんな	▲かわいい 女[おんな]。	
\\	〔ヒトの性のうち女性・女子〕 
\\	〔成熟した女性, 女性としての特質を備えた女性〕 
\\	〔愛人・情婦〕 
\\	女	おんな	
\\	奥さん	おくさん	
\\	林[はやし]さんの 今[いま]の 奥[おく]さん。	
\\	[⇒おくさま]	奥さん	おくさん	
\\	失敗	しっぱい	
\\	失敗[しっぱい]して 目[め]が 覚[さ]める。	
\\	《口》 
\\	〔失策〕 
\\	~する 〈人が主語〉 
\\	〔失策する〕 
\\	〔落ちぶれる〕 
\\	〈事が主語〉 
\\	失敗	しっぱい	
\\	太る	ふとる	
\\	彼[かれ]は 丸々[まるまる]と 太[ふと]って お相撲[おすもう]さんのようだ。	
\\	〔体格などが〕 
\\	〔財産などがふえる〕 
\\	太る	ふとる	
\\	大分	だいぶ	
\\	今度[こんど]はだいぶ 儲[もう]かったみたいだね。	
\\	大分	だいぶ	
\\	大体	だいたい	▲だいたいでいいから 費用[ひよう]を 教[おし]えてください。	
\\	〔概要・趣旨〕 
\\	《文》 
\\	《文》 
\\	《文》 
\\	《文》 
\\	〔大部分〕 
\\	〔概して〕 
\\	〔およそ〕 
\\	〔大部分・ほとんど〕 
\\	〔そもそも〕 
\\	大体	だいたい	
\\	大人	たいじん	
\\	大人[おとな]は 赤子[あかご]の 心[こころ]を 失[うしな]わず。	
\\	〔大きな人〕 =きょじん 
\\	〔成人〕 
\\	[=おとな] 
\\	〔大人物〕 
\\	〔高位の人〕 
\\	大人	おとな	
\\	大事	だいじ	
\\	大事[だいじ]を 企[くわだ]てる。	
\\	〔大事業〕 
\\	〔大きなこと〕 
\\	〔重大事〕 
\\	〔深刻な事態〕 
\\	[=おおごと]	大事	だいじ	
\\	夢	ゆめ	
\\	夢[ゆめ]をさます。	
\\	〔覚醒させる〕 
\\	〔睡眠中の〕 
\\	〔幻想〕 
\\	〔(実現できそうもない)高い願い・理想・期待〕 
\\	〔白昼夢〕 
\\	夢	ゆめ	
\\	外国人	がいこくじん	
\\	外国[がいこく]へ 行[い]けば 私[わたし]のほうが 外国人[がいこくじん]だ。	
\\	(外国からのお客さま) 
\\	〔主に法律で〕 
\\	〔日本人以外の人〕 
\\	外国人	がいこくじん	
\\	外国	がいこく	
\\	拳銃[けんじゅう]が 外国[がいこく]から 持ち込[もちこ]まれた。	
\\	外国	がいこく	
\\	変	へん	
\\	変[へん]な 癖[くせ]。	
\\	~な 〔ほかと異なった〕 
\\	〔普段と異なった〕 
\\	〔異様な〕 
\\	〔動乱〕 
\\	〔変事〕 
\\	〔災禍〕 
\\	〔非常事〕 
\\	【音楽】 
\\	(記号: 
\\	変	へん	
\\	売る	うる	
\\	定価[ていか]より 高[たか]く 売[う]る。	
\\	〔販売する〕 
\\	〔(所持品など)売り払う〕 
\\	〔広める〕 
\\	〔裏切る・そむく〕 
\\	[=うらぎる] 
\\	〔しかける〕	売る	うる	
\\	壊れる	こわれる	
\\	箱[はこ]の 中[なか]のものが 壊[こわ]れないように 静[しず]かに 運[はこ]んでください。	
\\	〔ものの形が損なわれる〕 
\\	〔成分が失われる〕 
\\	〔機能しなくなる〕 
\\	《文》 
\\	《口》 
\\	〔うまくいかなくなる〕 
\\	〔破綻する〕 
\\	〔台なしになる〕 
\\	〔両替できる〕 ⇒くずれる 
\\	壊れる	こわれる	
\\	壊す	こわす	
\\	賊[ぞく]はかぎを 壊[こわ]して 侵入[しんにゅう]した。	
\\	〔ものの形を損なう〕 
\\	〔傷害を起こさせる〕 
\\	〔機能を損なう〕 
\\	〔破綻させる〕 
\\	〔台なしにする〕 
\\	〔両替する〕 ⇒くずす 
\\	壊す	こわす	
\\	増える	ふえる	
\\	借金[しゃっきん]は 増[ふ]えるばかりだ。	
\\	〔増加する〕 
\\	(倍加する) 
\\	(増大する) 
\\	(かさむ) 
\\	〔増量する〕 
\\	[⇒ます3 
\\	増える	ふえる	
\\	塗る	ぬる	▲さっとなでるようにペンキを 塗[ぬ]る。	
\\	〔絵の具・ペンキを〕 
\\	〔しっくいなどを〕 
\\	〔ニスを〕 
\\	〔漆を〕 
\\	〔絵の具・漆などを〕 
\\	〔塗り薬・ペンキなどを〕 
\\	〔船底などにタールを〕 
\\	【薬】 〔外用剤などを皮膚上に〕 
\\	塗る	ぬる	
\\	場所	ばしょ	▲うちの 猫[ねこ]はいつもこの 場所[ばしょ]にいる.この 場所[ばしょ]が 好[す]きなんだな。	
\\	〔ところ〕 
\\	〔土地〕 
\\	〔位置〕 
\\	〔所在地〕 
\\	〔敷地・跡〕 
\\	〔現場〕 
\\	〔いる場所・ある場所〕 
\\	〔座席〕 
\\	〔相撲の興行〕 
\\	〔経験〕 
\\	[⇒ばかず]	場所	ばしょ	
\\	場合	ばあい	▲いざ 戦争[せんそう]となった 場合[ばあい](に)は。	
\\	〔そのような時〕 
\\	〔事例〕 
\\	〔事情・状況〕 
\\	〔…に関して〕 
\\	場合	ばあい	
\\	坂	さか	
\\	長崎[ながさき]は 坂[さか]の 町[まち]だ。	
\\	〔傾斜した道〕 
\\	[⇒さかみち] 
\\	〔年齢の区切り〕 
\\	坂	さか	
\\	地震	じしん	▲この 地震[じしん]のエネルギーを 示[しめ]すマグニチュードは3.7でした。	
\\	地震	じしん	
\\	国際	こくさい	
\\	国際[こくさい] テロ支援国家[てろしえんこっか]。	
\\	国際	こくさい	
\\	回る	まわる	
\\	扇風機[せんぷうき]が 回[まわ]っている。	
\\	〔輪を描く〕 
\\	〔回転する〕 
\\	〔縁辺に沿う〕 
\\	〔順路を伝う〕 
\\	〔あちこち移動する〕 
\\	〔順に訪問する〕 
\\	〔経由する〕 
\\	〔寄り道する〕 
\\	〔遠回りする〕 
\\	〔場所を移る〕 
\\	〔立場を変える〕 
\\	〔順番が来る〕 
\\	〔稼働する〕 
\\	〔機能する〕 
\\	〔行き渡る〕 
\\	〔届く〕 
\\	〔ある時刻を過ぎる〕 
\\	〔利息がつく〕 
\\	回る	まわる	
\\	四月	よつき	
\\	来年[らいねん]の 4月[しがつ]で10 年[ねん]ここで 働[はたら]いたことになる。	
\\	四月	しがつ	
\\	喧嘩	けんか	▲けんかの 種[たね]。	
\\	〔口論〕 
\\	〔論争〕 
\\	〔仲たがい〕 
\\	〔不和〕 
\\	〔殴り合い〕 
\\	〔乱闘〕 
\\	〔小さな〕 
\\	~する 〔口論する〕 
\\	〔仲たがいする〕 
\\	〔殴り合う〕 
\\	〔規模の大小〕 (小さなけんかをする) 
\\	(大きなけんかをする) 
\\	喧嘩	けんか	
\\	喜ぶ	よろこぶ	
\\	心[こころ]から 喜[よろこ]ぶ。	
\\	喜ぶ	よろこぶ	
\\	品物	しなもの	
\\	小倉[おぐら]さんから(あなたへの) 品物[しなもの]を 預[あず]かっています。	
\\	〔物品〕 
\\	〔商品〕 
\\	〔在庫品〕 
\\	〔品質〕 
\\	[=しな1 
\\	品物	しなもの	
\\	味噌	みそ	
\\	味噌[みそ] 味[み]
\\	仕立[した]て}の。	
\\	〔調味料〕 
\\	〔味噌状のもの〕 
\\	〔誇れる特徴〕	味噌	みそ	
\\	味	あじ	▲この 鍋[なべ]はぽん 酢[す]でもごまだれでも お好[おこの]みの 味[あじ]で お召[おめ]しあがりください。	
\\	〔味覚に感じるもの〕 
\\	〔妙味・趣〕 
\\	〔経験によって得るそのものの感じ〕 
\\	味	あじ	
\\	周り	まわり	▲テーブルの 周[まわ]りにすわる。	
\\	〔周囲〕 
\\	〔ふち・へり〕 
\\	〔周辺〕 
\\	〔周囲の人々・状況〕 
\\	周り	まわり	
\\	君	きみ	
\\	君[くん] 君[くん]たらずとも 臣[しん] 臣[しん]たらざるべからず。	
\\	〔君主〕 
\\	〔主君〕 
\\	〔あなた〕 
\\	〔呼び掛け〕 
\\	〔恋人に〕 
\\	君	きみ	
\\	向かう	むかう	
\\	机[つくえ]に 向[む]かう。	
\\	〔面する〕 
\\	[=めんする] 
\\	〔めざして行く〕 
\\	〔近づく〕 (時が) 
\\	(ある状態に) 
\\	〔逆らう〕 
\\	〔はむかう〕 
\\	向かう	むかう	
\\	同じ	おなじ	
\\	平和[へいわ]を 望[のぞ]む 気持[きも]ちは 皆[みな] 同[おな]じだ。	
\\	〔ひとつのものである〕 
\\	(寸分たがわぬ) 《文》 
\\	〔(複数のものが)互いに等しい〕 
\\	(特に数値が) 
\\	(特に型などが) 
\\	〔(複数の考え方・条件などが)共通している〕 
\\	〔酷似している〕 
\\	〔差がない〕 
\\	〔どうせ・どっちみち〕	同じ	おなじ	
\\	合う	あう	
\\	視線[しせん]が 合[あ]ったので
\\	私[わたし]は 彼[かれ]に) 会釈[えしゃく]した。	
\\	〔二つ以上のものが出会う〕 
\\	〔二つ以上のものがうまく一致する〕 
\\	〔他のものに〕 
\\	〔基準などに合致する・適合する〕 
\\	〔正しい〕 
\\	(時計が) 
\\	(計算が) 
\\	〔うまく調和する・釣り合う〕 
\\	〔ひきあう〕 
\\	[=ひきあう 
\\	合う	あう	
\\	吃驚	びっくり	▲びっくりするほど。	
\\	~する 〔物音や景色に〕 
\\	〔不意打ちされて〕 
\\	〔意外に感じて〕 
\\	吃驚	びっくり	
\\	叱る	しかる	▲そんなことをするとしかられるだけじゃ 済[す]まないぞ。	
\\	〔非難する〕 
\\	〔叱責・譴責(けんせき)する〕 
\\	《口》 
\\	〔説教する〕 
\\	〔きつく言う〕 
\\	叱る	しかる	
\\	可愛い	かわいい	▲スズランがかわいい 花[はな]をつけた。	
\\	〔愛情の対象として〕 
\\	〔魅力的な〕 
\\	〔可憐な〕 
\\	〔憎めない〕 
\\	〔好ましい〕 
\\	〔小さい〕 
\\	可愛い	かわいい	
\\	召し上がる	めしあがる	
\\	何[なに]か 召し上[めしあ]がりませんか。	
\\	=たべる.	召し上がる	めしあがる	
\\	取り替える・取り換える	とりかえる	▲それと 取り替[とりか]えてください。	
\\	〔交換する〕 
\\	〔代える〕 
\\	〔新たにする〕 
\\	〔代わりを入れる〕 
\\	取り替える	とりかえる	
\\	参る・詣る	まいる	
\\	明日[あした]また 参[まい]ります。	
\\	〔行く; 来る〕 
\\	〔詣でる〕 
\\	〔負ける〕 
\\	〔降参する〕 
\\	〔物が圧力などのために〕 
\\	〔困惑する〕 
\\	《口》 
\\	〔悩まされる〕 
\\	〔疲れ切る〕 
\\	《口》 
\\	[⇒へいこう4] 
\\	〔心を奪われる〕 
\\	《俗》 
\\	参る	まいる	
\\	厳しい	きびしい	
\\	厳[きび]しい 尋問[じんもん]。	
\\	〔厳格な〕 
\\	〔容赦のない〕 
\\	(尋問の) 
\\	〔困難が多く過酷な〕 
\\	〔気象などが激烈な〕 
\\	〔緊張している〕 
\\	厳しい	きびしい	
\\	原因	げんいん	
\\	主[おも]な 原因[げんいん]。	
\\	〔契機〕 
\\	〔動機〕 
\\	〔要因〕 
\\	〔根源〕 
\\	~する 〈結果が主語〉 
\\	〈原因が主語〉 
\\	原因	げんいん	
\\	厚い	あつい	
\\	厚[あつ]い 壁[かべ]。	
\\	〔物の表裏・上端下端のへだたりが大きい〕 
\\	〔(比喩的に)幅が広い〕 
\\	〔心入れの程度が大きい〕 
\\	〔手厚い〕 
\\	《文》 
\\	〔囲碁で〕 
\\	厚い	あつい	
\\	危険	きけん	
\\	危険[きけん]は 知[し]れたものだ。	
\\	〔あぶないこと〕 
\\	~な 
\\	(冒険的な) 
\\	〔悪い結果になるおそれ〕 =きけんせい.	危険	きけん	
\\	危ない	あぶない	▲その 仕事[しごと]は 少[すこ]しも 危[あぶ]ないことはない。	
\\	〔危険な〕 
\\	《口》 
\\	〔警告として〕 
\\	〔生命・容態などが〕 
\\	〔負けそう〕 
\\	〔疑わしい〕 
\\	〔心もとない〕 
\\	〔当てにならない〕 
\\	〔しっかりしない〕 
\\	[⇒あやしい 
\\	〔間一髪の〕 
\\	〔(人物が)正常ではない; 通例アブナイと表記〕 
\\	危ない	あぶない	
\\	卒業	そつぎょう	
\\	大学[だいがく]を 卒業[そつぎょう]して 間[ま]もない 青年[せいねん]。	
\\	〔教育課程を終えること〕 
\\	は通例高校レベル以上の卒業に限って使う. ~する 
\\	〔ある段階を通り過ぎること〕 ~する 
\\	卒業	そつぎょう	
\\	十月	じゅうがつ	▲ニューヨークの1987 年[ねん] 10月[じゅうがつ]の 株式[かぶしき] 大[だい] 暴落[ぼうらく]はまだ 生々[なまなま]しく 記憶[きおく]されている。	
\\	(略: 
\\	十月	じゅうがつ	
\\	十分・充分	じゅうぶん	
\\	十分[じゅうぶん]な 経験[けいけん]。	
\\	~な 〔必要を満たす〕 
\\	〔満足な〕 
\\	〔不足・欠点のない〕 
\\	〔必要を満たす程度に〕 
\\	〔欠点なく〕 
\\	〔存分に〕 
\\	十分	じゅうぶん	
\\	十二月	じゅうにがつ	
\\	応募[おうぼ] 書類[しょるい]を 記入[きにゅう]して 十二月[じゅうにがつ]の 二[に] 日[にち]までに 送り返[おくりかえ]してください。	
\\	(略: 
\\	十二月	じゅうにがつ	
\\	十一月	じゅういちがつ	
\\	11月[じゅういちがつ]に 入[はい]ると 日増[ひま]しに 寒[さむ]くなる。	
\\	(略: 
\\	十一月	じゅういちがつ	
\\	医者	いしゃ	
\\	今後[こんご]の 治療[ちりょう]については 医者[いしゃ]の 判断[はんだん]に 任[まか]せることにする。	
\\	〔開業医〕 
\\	《口》 
\\	医者	いしゃ	
\\	包む	つつむ	
\\	物[もの]を 一[ひと]つに 包[つつ]む。	
\\	〔包装する〕 
\\	〔おおう〕 
\\	〔着用する〕 
\\	[⇒つつまれる] 
\\	〔隠す〕 
\\	[⇒つつみかくす] 
\\	〔(現金を) 差し出す〕 
\\	包む	つつむ	
\\	匂い・臭い	におい	▲においがいいいいにおいがする。	
\\	〔嗅覚に感じるもの〕 
\\	〔悪臭〕 
\\	〔趣き・気分〕 
\\	匂い	におい	
\\	力	ちから	
\\	力[ちから]を 入[い]れて
\\	込[こ]めて}。	
\\	〔体力・筋力〕 
\\	〔精力〕 
\\	〔元気〕 
\\	*《俗》 
\\	〔勢い・迫力〕 
\\	【物】 〔物理的な〕 
\\	〔人知を超えた働き・作用〕 
\\	〔権力・威力〕 
\\	〔努力・骨折り〕 
\\	〔効力・効果〕 
\\	〔貢献・助力・尽力〕 
\\	〔能力〕 
\\	〔学力〕 
\\	〔知識〕 
\\	〔資力〕 
\\	力	ちから	
\\	割合	わりあい	
\\	乗客[じょうきゃく]は 日本人[にっぽんじん]10 に対[にたい]して 外国人[がいこくじん]は3の 割合[わりあい]だった。	
\\	〔比率〕 
\\	(百分比) 
\\	〔割り当て〕 
\\	〔比較的・思った以上に〕 =わりに.	割合	わりあい	
\\	割れる	われる	▲こなごなに 割[わ]れる。	
\\	〔原形がなくなる〕 
\\	〔複数になる〕 
\\	〔分裂する〕 
\\	(関係が絶たれる) 
\\	〔裂け目ができる〕 
\\	〔傷口が開く〕 
\\	〔切れ目がある〕 
\\	〔明らかになる〕 
\\	〔限度を超す〕 
\\	〔最低基準を下回る〕 
\\	割れる	われる	
\\	利用	りよう	
\\	人脈[じんみゃく]を 利用[りよう]する。	
\\	〔役立たせること〕 
\\	〔生かして使うこと〕 
\\	〔利己的に方便とすること〕 
\\	~する 〔役立たせる〕 
\\	〔活用する〕 
\\	〔利己目的の方便とする〕 
\\	利用	りよう	
\\	別れる	わかれる	
\\	手[て]を 振[ふ]って 別[わか]れる。	
\\	〔別行動になる〕 
\\	〔関係を解消する〕 (夫婦が) 
\\	(恋人同士が) 
\\	(コンビが) 
\\	別れる	わかれる	
\\	別	べつ	▲たいていの 靴下[くつした]には 左右[さゆう]の 別[べつ]はない。	
\\	〔区別〕 
\\	〔違い〕 
\\	~の[な] 〔異なる〕 
\\	〔ほかの〕 
\\	別	べつ	
\\	出席	しゅっせき	
\\	折り返[おりかえ]しご 出席[しゅっせき]の 有無[うむ]をご 一報[いっぽう]ください。	
\\	~する 
\\	〔顔を出す〕 
\\	《口》 
\\	出席	しゅっせき	
\\	冷房	れいぼう	
\\	ただ今[ただいま] 冷房[れいぼう] 中[ちゅう]です.ドアを 開け放[あけはな]しにしないでください。	
\\	~する 
\\	冷房	れいぼう	
\\	冷える	ひえる	▲ラードが 冷[ひ]えて 固[かた]まった。	
\\	〔温度が下がる〕 
\\	〔寒く感じる〕 
\\	〔冷え性で〕 
\\	〔熱意がなくなる〕 
\\	〔関係が悪化する〕 
\\	冷える	ひえる	
\\	再来週	さらいしゅう	
\\	彼女[かのじょ]は 再来週[さらいしゅう]80 歳[さい]になる。	
\\	再来週	さらいしゅう	
\\	再来月	さらいげつ	
\\	来月[らいげつ]か 再来月[さらいげつ]。	
\\	再来月	さらいげつ	
\\	円	えん	▲この 円[えん]の 面積[めんせき]を 求[もと]めよ。	
\\	〔輪・円形〕 
\\	(小型の) 
\\	〔日本の通貨単位〕 
\\	(記号 
\\	円	えん	
\\	内	うち	
\\	雨[あめ]に 濡[ぬ]れるからもっと 内[うち]へ 入[はい]ってください。	
\\	〔ある空間の範囲内〕 
\\	[⇒うちがわ, なか1] 
\\	〔心の奥〕 
\\	〔本心〕 
\\	〔内実〕 
\\	〔自分の所属団体〕 
\\	(わが社) 
\\	[⇒うち2] 
\\	〔ある数量の範囲内〕 [⇒なか1] 
\\	〔ある部類の範囲内〕 
\\	〔ある時間・状況の範囲内〕 
\\	〔代筆の妻の署名〕	内	うち	
\\	具合	ぐあい	
\\	機械[きかい]などの) 具合[ぐあい]がよい。	
\\	〔物事の状態・調子〕 
\\	〔健康状態〕 
\\	〔都合〕 
\\	〔体裁〕 
\\	〔方法〕 
\\	具合	ぐあい	
\\	六月	ろくがつ	
\\	彼[かれ]は 6月[ろくがつ]に 東京[とうきょう]を 離[はな]れて 関西[かんさい]に 来[く]る。	
\\	(略: 
\\	六月	ろくがつ	
\\	公務員	こうむいん	
\\	公務員[こうむいん]になる。	
\\	〔各省の〕 
\\	〈集合的に〉 
\\	[⇒こっかこうむいん, ちほうこうむいん, かんり2]	公務員	こうむいん	
\\	八月	はちがつ	▲ジョンテル 社[しゃ]はチップ・セットの 発売[はつばい]を 8月[はちがつ]まで 延期[えんき]する 見込[みこ]みだ。	
\\	(略: 
\\	八月	はちがつ	
\\	全然	ぜんぜん	
\\	英語[えいご]が 全然[ぜんぜん]だめな 人[ひと]。	
\\	〔否定的に〕 
\\	〔肯定で強調して〕 
\\	〔非常に〕 
\\	全然	ぜんぜん	
\\	入院	にゅういん	
\\	入院[にゅういん]の 必要[ひつよう]はありません。	
\\	~する 
\\	入院	にゅういん	
\\	入学	にゅうがく	
\\	無試験[むしけん]で
\\	試験[しけん]の 上[うえ]} 入学[にゅうがく]を 許可[きょか]する。	
\\	〔大学への〕 
\\	~する 
\\	〔大学に〕 
\\	入学	にゅうがく	
\\	入れる	いれる	
\\	洗濯機[せんたくき]で 洗濯[せんたく]する 時[とき]は 水[すい] 洗剤[せんざい] 洗濯物[せんたくもの]の 順[じゅん]に 入[い]れます。	
\\	〔物を外から中へ移す〕 
\\	〔詰める〕 
\\	〔注入する〕 
\\	〔つっこむ〕 
\\	【電算】 
\\	〔中に移したままにする〕 
\\	〔収納する・しまう〕 
\\	〔収録する〕 
\\	〔はめこむ〕 
\\	〔挿入する〕 
\\	〔送り込む〕 
\\	〔加える〕 
\\	〔添加する〕 
\\	〔書き込む〕 
\\	〔導入する〕 
\\	〔仕入れる〕 
\\	〔納入する〕 
\\	〔入金する〕 
\\	〔ある場所に人や物を導き入れる; 通す〕 
\\	〔仲間に加える〕 
\\	〔所属させる〕 
\\	〔雇用する〕 
\\	〔学校や病院などに入らせる〕 
\\	〔収容する〕 
\\	〔間や途中に差し挟む〕 
\\	〔意思を伝える〕 
\\	〔修正を加える〕 
\\	〔気持ちや力を加える〕 
\\	〔含める〕 
\\	〔聞き入れる〕 
\\	〔受け入れる〕 
\\	〔応じる〕 
\\	〔従う〕 
\\	〔点数をつける〕 
\\	〔得点する〕 
\\	〔投票する〕 
\\	〔お茶などを淹れる〕 
\\	〔機械や道具を作動させる〕 
\\	入れる	いれる	
\\	光る	ひかる	
\\	西[にし]の 空[そら]が 光[ひか]って 雷鳴[らいめい]が 聞[き]こえた。	
\\	〔発光する〕 
\\	(暗中に) 
\\	(かすかに) 
\\	(ぴかっと) 
\\	(星などが) 
\\	〔反射する〕 
\\	(鋭く) 
\\	(宝石などが) 
\\	(金銀などが) 
\\	〔光沢がある〕 
\\	〔目を引く〕 
\\	光る	ひかる	
\\	光	ひかり	▲ギリシャは 夏[なつ]の 光[ひかり]があふれていた。	
\\	〔明るさ〕 
\\	〔照らすもの〕 
\\	〔可視光線など〕 
\\	(光束) 
\\	〔閃光〕 
\\	(暗中での) 
\\	〔微光〕 
\\	〔輝き〕 
\\	〔社会・人生・希望・幸福などの明るさ〕 
\\	〔威光〕 
\\	[⇒ななひかり] 
\\	〔視覚〕	光	ひかり	
\\	先輩	せんぱい	
\\	話[はな]してみたらその 人[ひと]は 自分[じぶん]の 大学[だいがく]の 先輩[せんぱい]にあたることがわかった。	
\\	《口》 
\\	先輩	せんぱい	
\\	兄	けい	
\\	両者[りょうしゃ]は) 兄[けい]たり 難[かた]く 弟[おとうと]たり 難[むずか]し。	
\\	〔弟(てい)に対して〕 
\\	[=あに1] 
\\	〔敬称〕 
\\	〔貴君〕 
\\	兄	あに	
\\	僕	しもべ	
\\	公務員[こうむいん]は 国民[こくみん]の 僕[ぼく]だ。	
\\	[⇒げなん, げじょ]	僕	ぼく	
\\	傘	かさ	
\\	晴雨[せいう] 兼用[けんよう]の 傘[かさ]。	
\\	〔雨傘〕 
\\	〔日傘〕 
\\	傘	かさ	
\\	値段	ねだん	
\\	値段[ねだん]を 安[やす]くつける。	
\\	[⇒ね3, かかく2, そうば 
\\	値段	ねだん	
\\	借りる	かりる	
\\	忘れ物[わすれもの]をして〕 友人[ゆうじん]に 消[け]しゴムを 借[か]りる。	
\\	〔金品を一時使わせてもらう〕 
\\	〔賃借りする〕 
\\	〔貸し切り契約で乗り物を〕 
\\	〔他人の助力・協力などを得る〕 
\\	〔臨時に別の用途に使う〕 
\\	借りる	かりる	
\\	倍	ばい	
\\	旧[きゅう]に 倍[ばい]してご 愛顧[あいこ]のほどよろしく お願[おねが]いいたします。	
\\	~する 
\\	倍	ばい	
\\	例えば	たとえば	
\\	例[たと]えばタカやワシのような 猛禽[もうきん]。	
\\	〔…のような〕 
\\	例えば	たとえば	
\\	体	たい	
\\	体[からだ]をかわす。	
\\	〔身体〕 
\\	〔体格〕 
\\	〔体勢〕 
\\	〔本質・実体〕 
\\	〔書体・字体〕 
\\	【解】 
\\	【物】 
\\	【数】 
\\	【-体】	体	からだ	
\\	似る	にる	
\\	多少[たしょう] 似[に]たところがある。	
\\	〔共通点がある〕 
\\	〔同じようである〕 
\\	〔印象どおりである・いつも通りである〕	似る	にる	
\\	伯母さん・叔母さん・小母さん	おばさん	▲まさよおばさん。	
\\	【伯母さん・叔母さん】 〔伯母・叔母〕 
\\	《小児語》 
\\	【小母さん】 〔年長の女性〕 
\\	伯母さん	おばさん	
\\	伝える	つたえる	
\\	彼[かれ]らは 満足[まんぞく]の 意[い]を 表[あらわ]したと 伝[つた]えられている。	
\\	〔知らせる〕 
\\	〔報道する〕 
\\	〔届かせる〕 
\\	〔発信する〕 
\\	〔感じさせる〕 
\\	〔もたらす〕 
\\	〔伝授する〕 
\\	〔伝承する〕 
\\	《文》 
\\	伝える	つたえる	
\\	会社	かいしゃ	▲きょうは 祝日[しゅくじつ]だから 会社[かいしゃ]がない。	
\\	(略: 
\\	(略: 
\\	〔職場〕 
\\	〔仕事〕 
\\	会社	かいしゃ	
\\	会う・逢う・遭う	あう	
\\	僕[ぼく]に 会[あ]いたくなったらこの 浜辺[はまべ]においで. 僕[ぼく]は 天気[てんき]さえよければいつもここでサーフィンやってるから。	
\\	〔約束して対面する・面談する〕 
\\	〔偶然出会う〕 
\\	《口》 
\\	《文》 
\\	*《口》 
\\	〔好ましくない事態に遭遇する〕 
\\	会う	あう	
\\	付く	つく	▲たばこのにおいは 服[ふく]につくんですよ。	
\\	〔とれなくなる〕 
\\	(残る) 
\\	〔離れずにある〕 
\\	〔定着する〕 
\\	〔固定する〕 
\\	〔付随する〕 
\\	〔接続できる〕 
\\	〔そなわる〕 
\\	〔自分のものになる〕 
\\	〔程度・数量が増す〕 
\\	〔(遅れずに)従う〕 
\\	〔付いて離れない〕 
\\	[⇒ついていく] 
\\	〔付き添う〕 
\\	〔味方になる〕 
\\	〔後ろ盾になる〕 
\\	〔記される〕 
\\	〔割り振られる〕 
\\	〔めざす状態になる〕 
\\	〔可能となる・及ぶ〕 
\\	〔決定する・決着する・つながる〕 
\\	〔ある値段になる〕 
\\	〔運が向く〕 
\\	[⇒ついて(い)る]	付く	つく	
\\	仕方	しかた	
\\	料理[りょうり]の 仕方[しかた]が 悪[わる]かった。	
\\	〔やり方〕 
\\	〔方策〕 
\\	〔手段〕 
\\	仕方	しかた	
\\	今度	こんど	
\\	今度[こんど]の 校長[こうちょう]。	
\\	〔少し過去のこと〕 
\\	〔最新のこと・このたび〕 
\\	~の 〔少し過去の〕 
\\	〔最新の・このたびの〕 
\\	〔次回・この次〕 
\\	~の 
\\	今度	こんど	
\\	今夜	こんや	
\\	今夜[こんや] 中[なか]。	
\\	[=こんばん]	今夜	こんや	
\\	今	いま	
\\	今[いま]のうちだ 先生[せんせい]が 来[く]る 前[まえ]に 早[はや]く 僕[ぼく]の 宿題[しゅくだい]を 写[うつ]せ。	
\\	〔現在〕 
\\	(この瞬間) 
\\	(目下) 
\\	[⇒いまごろ 
\\	〔現代〕 
\\	〔当節〕 
\\	〔いましがた〕 
\\	〔さっき〕 
\\	〔直ちに〕 
\\	〔じきに〕 
\\	〔もうすぐ〕 
\\	〔さらに〕 
\\	〔もう〕 
\\	〔別の〕 
\\	[⇒いまひとつ] 
\\	〔この場で〕 
\\	今	いま	
\\	人形	にんぎょう	
\\	人形[にんぎょう]のようにかわいい
\\	表情[ひょうじょう]に 乏[とぼ]しい}。	
\\	〔玩具の〕 
\\	〔自由意志のない人・お飾り〕 
\\	人形	にんぎょう	
\\	人口	じんこう	
\\	人口[じんこう]5 万[まん]を 数[かぞ]える。	
\\	〔人の数〕 
\\	〔人の話題〕 
\\	人口	じんこう	
\\	交通	こうつう	
\\	交通[こうつう]を 取り締[とりし]まる。	
\\	〔人の行き来〕 
\\	〔運輸・通信〕 
\\	〔意思の疎通〕 
\\	《文》 
\\	~する 
\\	交通	こうつう	
\\	亡くなる	なくなる	
\\	父[ちち]が 昨晩[さくばん] 亡[な]くなりました。	
\\	亡くなる	なくなる	
\\	五月	ごがつ	
\\	五月の節句[ごがつのせっく]。	
\\	五月	ごがつ	
\\	五日	いつか	
\\	今月[こんげつ]5 日[にち]から 時間表[じかんひょう]が 改正[かいせい]される。	
\\	〔日付〕 (書き言葉で) 
\\	(話し言葉で) 
\\	〔日数〕 
\\	五日	いつか	
\\	五つ	いつつ	
\\	私[わたし]は5つの 職[しょく]に 申し込[もうしこ]みをしたが、 全部[ぜんぶ]だめだった。	
\\	〔数〕 
\\	〔年齢〕 
\\	五つ	いつつ	
\\	二月	ふたつき	
\\	予期[よき]していなかったことが 起[お]こり、 残念[ざんねん]ながら 2月[にがつ]27 日[にち]にお 会[あ]いできなくなりました。	
\\	二月	にがつ	
\\	事故	じこ	
\\	事故[じこ]にあう。	
\\	〔故障〕 
\\	事故	じこ	
\\	事務所	じむしょ	
\\	事務所[じむしょ]を 移転[いてん]する。	
\\	事務所	じむしょ	
\\	予習	よしゅう	
\\	明日[あした]の 予習[よしゅう]をする。	
\\	~する 
\\	《口》 
\\	予習	よしゅう	
\\	予約	よやく	▲ホテルの 予約[よやく]が 取[と]れなかった。	
\\	〔あらかじめ約束すること〕 
\\	〔出版物の〕 
\\	〔船・座席の〕 
\\	〔商品の〕 
\\	~する 
\\	〔あらかじめ機械をセットすること〕 
\\	~する 
\\	予約	よやく	
\\	予定	よてい	
\\	予定[よてい]を 早[はや]める。	
\\	〔計画〕 
\\	〔手配〕 
\\	〔予想〕 
\\	〔価額などの〕 
\\	~する 
\\	〔予定される〕 
\\	〔時日に関して〕 
\\	予定	よてい	
\\	九月	くがつ	
\\	九月[くがつ]10 日[にち]。	
\\	(略: 
\\	九月	くがつ	
\\	乗り物	のりもの	
\\	乗り物[のりもの]の 便[びん]がいい。	
\\	〔交通手段としての〕 
\\	《文》 
\\	〔遊戯施設〕 
\\	乗り物	のりもの	
\\	乗り換える	のりかえる	
\\	別[べつ]の 船[ふね]に 乗り換[のりか]える。	
\\	〔交通機関を〕 
\\	(船を) 
\\	〔主義・主張を〕 
\\	〔方式を〕 
\\	【電算】 〔ソフトを〕 
\\	〔株式を〕 
\\	乗り換える	のりかえる	
\\	中学校	ちゅうがっこう	▲この 歌[うた]を 聞[き]くと 私[わたし]の 中学校[ちゅうがっこう] 時代[じだい]を 思い出[おもいだ]します。	
\\	*《口》 
\\	[⇒ちゅうがく]	中学校	ちゅうがっこう	
\\	両方	りょうほう	
\\	彼女[かのじょ]は 甘い物[あまいもの]も 酒[さけ]も 両方[りょうほう] いける口[いけるくち]だ。	
\\	〔否定〕 
\\	両方	りょうほう	
\\	世話	せわ	▲あんな 施設[しせつ]に 老父[ろうふ]の 世話[せわ]を 頼[たの]んだらどんなことになるかわからない。	
\\	〔面倒を見ること〕 
\\	~する 〔面倒を見る〕 
\\	〔推薦・斡旋(あっせん)〕 
\\	《文》 
\\	~する 
\\	〔斡旋する〕 
\\	〔厄介〕 
\\	[⇒おせわ] 
\\	〔世俗〕 
\\	〔日常語〕 
\\	世話	せわ	
\\	世界	せかい	
\\	世界[せかい]に 呼[よ]びかける。	
\\	〔地球上のすべての国〕 
\\	〔地球〕 
\\	〔その人の認識する社会・世間〕 
\\	〔特定の社会〕 
\\	〔領域〕 
\\	〔仏教の〕 
\\	世界	せかい	
\\	不便	ふべん	
\\	携帯[けいたい]に 不便[ふべん]である。	
\\	~な 
\\	〔扱いにくい〕 
\\	〔厄介な〕 
\\	不便	ふべん	
\\	下着	したぎ	
\\	下着[したぎ]を 取り替[とりか]える。	
\\	〔女性の〕 
\\	《口》 
\\	下着	したぎ	
\\	下宿	げしゅく	
\\	監督[かんとく]の 自宅[じたく]に 下宿[げしゅく]してトレーニングに 励[はげ]んでいる。	
\\	〔下宿すること〕 
\\	〔下宿屋〕 ⇒下宿屋. ~する 
\\	下宿	げしゅく	
\\	下げる	さげる	
\\	軒[のき]に 看板[かんばん]を 下[さ]げる。	
\\	〔ぶらさげる〕 
\\	〔垂らす〕 
\\	〔吊す〕 
\\	〔下に向ける, 低所へ移す〕 
\\	〔数値を小さくする, 値下げする〕 
\\	〔降格させる〕 
\\	〔評価を落とす〕 
\\	〔後退させる〕 
\\	〔他の選手と交代させる〕 
\\	〔食器や料理を片付ける〕 
\\	(祭壇などから) 
\\	【音楽】 〔低音にする〕 
\\	〔けなす〕 
\\	〔下品にする〕 
\\	〔現代に近づける〕 
\\	下げる	さげる	
\\	下がる	さがる	
\\	店[みせ]の 前[まえ]には
\\	営業中[えいぎょうちゅう]」の 札[さつ]が 下[さ]がっていた。	
\\	〔ぶらさがる〕 
\\	《文》 
\\	〔たれさがる〕 
\\	〔ずり落ちる〕 
\\	〔低所へ移る〕 
\\	〔下を向く〕 
\\	〔数値が小さくなる〕 
\\	〔下落する〕 
\\	〔技量や評価が低下する〕 
\\	[⇒おちる 
\\	〔降格する〕 
\\	〔退く〕 
\\	〔退出する〕 
\\	〔現代に近づく〕 
\\	〔(京都で)南へ行く〕 
\\	〔下付される〕 =おりる 
\\	下がる	さがる	
\\	上着	うわぎ	
\\	上着[うわぎ]を 着[き]る
\\	脱[ぬ]ぐ}。	
\\	〔婦人・子供服のゆるい〕 
\\	〔詰襟の軍服の〕 
\\	上着	うわぎ	
\\	上げる・挙げる	あげる	
\\	布団[ふとん]を 上[あ]げる。	
\\	〔上方へ移動させる〕 
\\	〔程度を高くする〕 
\\	〔示す〕 
\\	〔自分の気持ちや考えなどを表す手段として声を出す〕 
\\	〔儀式などを取り行う〕 
\\	〔よい結果を生み出す〕 
\\	〔終える・仕上げる・済ます〕 
\\	〔能力など発揮する〕 
\\	[⇒あげて] 
\\	〔潮が満ちてくる〕 
\\	〔与える・やる・渡す〕 
\\	〔神仏に供える〕 
\\	〔食べた物を戻す〕 ⇒もどす, はく5. 
\\	〔他のためにする動作のていねい表現〕	上げる	あげる	
\\	上がる	あがる	
\\	花火[はなび]が 上[あ]がった。	
\\	〔上方へ移動する〕 
\\	〔程度が高くなる・盛んになる〕 
\\	〔のぼせて落ち着きを失う〕 
\\	〔取りだされて目につくところに出る〕 
\\	〔一斉に声が発せられる〕 
\\	〔収益などが生じる〕 
\\	〔終わる・完了する・出尽くす〕 
\\	〔召しあがる〕 
\\	〔参る・伺う〕 
\\	〔供物などがそなえられている〕	上がる	あがる	
\\	三月	みつき	
\\	梅[うめ]の 花[はな]は 3月[さんがつ]にさく。	
\\	三月	さんがつ	
\\	七月	しちがつ	▲さくらんぼは 六月[ろくがつ]か 七月[しちがつ]に 熟[じゅく]す。	
\\	(略: 
\\	七月	しちがつ	
\\	丁寧	ていねい	
\\	丁寧[ていねい]なあいさつ。	
\\	〔配慮が行き届き, 礼儀正しいさま〕 ~な 
\\	〔細かいところまで注意が行き届くさま〕 ~な 
\\	〔綿密な〕 
\\	丁寧	ていねい	
\\	一番	いちばん	
\\	いの一番[いのいちばん]に。	
\\	〔第一番〕 
\\	〔(勝負の)1 回〕 
\\	〔(謡曲などの)1 曲〕 
\\	〔もっとも〕 
\\	〔試しに〕 ⇒ためし (試しに).	一番	いちばん	
\\	一月	ひとつき	
\\	一月[いちがつ]の。	
\\	一月	いちがつ	
\\	一昨日	いっさくじつ	
\\	一昨日[おととい]の 朝[あさ]。	
\\	[=おととい]	一昨日	おととい	
\\	一昨年	おととし	▲おととしの 夏[なつ]。	
\\	一昨年	おととし	
\\	一	いち	
\\	分[ぶん]の1。	
\\	つ〕 
\\	〔全体の中の 
\\	つ〕 
\\	〔ローマ数字〕 
\\	【数】 
\\	〔最初〕 
\\	〔最も〕 
\\	〔すぐれた・重要な〕 
\\	〔或る〕 
\\	〔単なる・つまらない〕 
\\	〔さいころやトランプの〕 
\\	一	いち	
\\	ワープロ	ワープロ	▲ワープロで 手紙[てがみ]を 打[う]つ。	
\\	【電算】 =ワード・プロセッサー.	ワープロ	わーぷろ	
\\	レポート	レポート	
\\	被災地[ひさいち]の 状況[じょうきょう]をレポートする。	
\\	〔報告〕 
\\	〔学校で課する〕 
\\	~する 
\\	レポート	れぽーと	
\\	ベル	ベル	▲ベルの 音[おと]。	
\\	ベル	べる	
\\	プレゼント	プレゼント	
\\	放送局[ほうそうきょく]にリクエストして) 曲[きょく]をプレゼントする。	
\\	[=おくりもの] ~する 
\\	プレゼント	ぷれぜんと	
\\	ファックス	ファックス	
\\	注文書[ちゅうもんしょ]をファックスする。	
\\	〔文字・画像などの電送システム〕 
\\	〔機械〕 
\\	〔送信メッセージ〕 
\\	~する 
\\	ファックス	ふぁっくす	
\\	ビル	ビル	
\\	階建[かいだ]てのビル。	
\\	〔建物〕 
\\	ビル	びる	
\\	ピアノ	ピアノ	
\\	彼女[かのじょ]はピアノが 上手[じょうず]だ。	
\\	【楽器】 
\\	【音楽】 〔強弱記号〕 
\\	(略: 
\\	ピアノ	ぴあの	
\\	パソコン	パソコン	
\\	彼女[かのじょ]は 娘[むすめ]のためにパソコンを 買[か]ってやった。	
\\	【電算】 〔パーソナル・コンピューター〕 
\\	パソコン	ぱそこん	
\\	テニス	テニス	▲テニスの 花形[はながた] 選手[せんしゅ]。	
\\	〔第一人者〕 
\\	テニス	てにす	
\\	タイプ	タイプ	
\\	部屋[へや]のタイプに 合[あ]わせて 家具[かぐ]を 選[えら]ぶ。	
\\	〔型〕 
\\	〔種類〕 
\\	〔(人間の)類型〕 
\\	〔タイプライター〕 
\\	〔タイプライターを打つこと〕 
\\	~する 
\\	タイプ	たいぷ	
\\	ソフト	ソフト	▲ソフトな 語り口[かたりくち]。	
\\	~な 〔柔らかい〕 
\\	〔ソフト帽〕 
\\	【電算】 =ソフトウェア. 
\\	〔映画などの販売パッケージ〕 
\\	〔非物的側面〕 
\\	=ソフト・クリーム.	ソフト	そふと	
\\	ステーキ	ステーキ	
\\	厚[あつ]いステーキ。	
\\	ステーキ	すてーき	
\\	スクリーン	スクリーン	▲その 映画[えいが]は500 劇場[げきじょう]の1200スクリーンで 公開[こうかい]された。	
\\	〔映写幕〕 
\\	〔映画界〕 
\\	〔コンピューター・レーダーなどの画面〕 
\\	スクリーン	すくりーん	
\\	スーツケース	スーツケース	▲このスーツケースを 運[はこ]ぶのを 手伝[てつだ]ってもらえませんか。	
\\	スーツケース	すーつけーす	
\\	スーツ	スーツ	
\\	新調[しんちょう]したスーツに 身[み]を 固[かた]める。	
\\	スーツ	すーつ	
\\	サンドイッチ	サンドイッチ	
\\	母[はは]はサンドイッチを 紙[かみ]に 包[つつ]んでくれた。	
\\	サンドイッチ	さんどいっち	
\\	サンダル	サンダル	▲サンダルをはいた 男[おとこ]。	
\\	サンダル	さんだる	
\\	サラダ	サラダ	▲サラダを 作[つく]る。	
\\	サラダ	さらだ	
\\	カレンダー	カレンダー	▲カレンダーに 予定[よてい]を 書き込[かきこ]む。	
\\	〔暦〕 
\\	カレンダー	カレンダー	
\\	カメラ	カメラ	▲カメラのフラッシュを 浴[あ]びる。	
\\	カメラ	かめら	
\\	お茶	おちゃ	
\\	出[で]がらしの お茶[おちゃ]。	
\\	〔飲みもの〕 
\\	(緑茶) 
\\	〔休憩〕 
\\	[⇒ちゃ 
\\	〔茶の湯〕 
\\	お茶	おちゃ	
\\	お父さん	おとうさん	
\\	子[こ]が 呼[よ]びかけて〕 お父[おとう]さん。	
\\	《口》 
\\	お父さん	おとうさん	
\\	お手洗い	おてあらい	
\\	お手洗[おてあら]いに 行[い]く。	
\\	〔便所〕 
\\	《口》 
\\	(公共の建物の) 
\\	お手洗い	おてあらい	
\\	お弁当	おべんとう	▲ほっぺに お弁当[おべんとう]がついてるよ。	
\\	=べんとう. 
\\	〔口元についた飯粒など〕	お弁当	おべんとう	
\\	お姉さん	おねえさん	
\\	料理屋[りょうりや]で〕 お姉[おねえ]さんビール3 本[ほん] 追加[ついか]!	
\\	〔姉〕 
\\	(呼びかけて) 
\\	(➡英語では家族間での呼びかけには名前を言うのが普通); 〔年上の女性になれなれしく呼びかけて〕 
\\	(若い女性に対して) 
\\	(ウエートレス・仲居などに) 
\\	お姉さん	おねえさん	
\\	お兄さん	おにいさん	▲「ちょっとそこの お兄[おにい]さん」
\\	私[わたし]のことですか」。	
\\	〔兄〕 
\\	(呼びかけて) 
\\	(➡英語では家族間での呼びかけには名前を言うのが普通); 〔他人になれなれしく呼びかけて〕 
\\	(若い男性に対して) 
\\	(飲食店などの男性従業員に) 
\\	お兄さん	おにいさん	
\\	お祖父さん・お爺さん	おじいさん	
\\	昔[むかし]々あるところにおじいさんと お婆[おばあ]さんが 住[す]んでおりました。	
\\	【お祖父さん】 〔祖父〕 
\\	《口》 
\\	《口》 
\\	〔呼びかけ〕 
\\	【お爺さん】 〔男の老人〕 
\\	おじいさん	おじいさん	
\\	嫌	いや	▲どうして 食[た]べないの 何[なに]がいやなの。	
\\	〔欲しない〕 
\\	〔気が進まない〕 
\\	〔許容できない〕 
\\	〔気に入らない〕 
\\	〔好かない〕 
\\	[⇒いやでも, いやに] 
\\	~な 〔不快な〕 
\\	〔不審な〕 
\\	〔変な〕 
\\	〔(困惑して)迷惑な〕 
\\	〔心外な〕	いや	いや	
\\	何時も	いつも	
\\	彼[かれ]はいつもそう 言[い]っていた。	
\\	〔常に〕 
\\	(平常) 
\\	(常習的に) 
\\	(必ず) 
\\	(絶えず) 
\\	(いつでも) 
\\	〈否定の構文で〉 
\\	[⇒いつでも] 
\\	〔普段〕	いつも	いつも	
\\	何時	いつ	▲いつからピアノを 習[なら]っていますか。	
\\	[⇒いつなんどき]	いつ	いつ	
\\	幾ら	いくら	▲いくらで 売[う]るとは 申[もう]しません.いくらでも 結構[けっこう]です. お客[おきゃく]さま 次第[しだい]で。	
\\	〔不定の値段・数量〕 
\\	〔値段・数量についての疑問〕 
\\	〔(否定文で)わずか・少し〕 
\\	〔(「~でも」「~も」の形で肯定文で)たくさん・限りなく〕 
\\	〔たとえどんなに〕 
\\	いくら	いくら	
\\	いいえ	いいえ	▲「これは 失礼[しつれい]しました」「いいえどういたしまして」。	
\\	〔答が否定のとき〕 
\\	〔答が肯定のとき〕 
\\	いいえ	いいえ	
\\	あれ	あれ	▲あれは 何[なに]の 音[おと]だろう。	
\\	〔離れた所にあるもの・人を指して〕 
\\	〔妻・子供など親しい人・目下の者を指して〕 
\\	〔聞き手も知っている時・人・事・物・場所を指して〕 
\\	〔言いにくいことを漠然と指して〕	あれ	あれ	
\\	有る・在る	ある	▲あるいはそんなことがあったかもしれない。	
\\	〔存在する〕 
\\	〔…に存する〕 
\\	〔…による〕 
\\	〔位置する〕 
\\	(山・建物などが) 
\\	(都市・国などが) 
\\	(道・川が) 
\\	〔持っている〕 
\\	《文》 
\\	(よいものを) 
\\	(悪いものを) 
\\	〔授かっている〕 
\\	〔付属してある〕 
\\	〔設備がある〕 
\\	〔含まれている〕 
\\	〔数量がある〕 
\\	〔売っている・手に入る〕 
\\	〔経験がある〕 
\\	〔起こる〕 
\\	《文》 
\\	〔開かれる・行われる〕 
\\	(会議などが) 
\\	〔…と書かれている〕 
\\	〔…であるので〕 
\\	〔動作・作用の結果が続いている; 準備されている〕	ある	ある	
\\	余り	あまり	
\\	準備[じゅんび]した 材料[ざいりょう]から 余[あま]りが 出[で]た。	
\\	〔残余〕 
\\	〔余分〕 
\\	〔残り物〕 
\\	〔残高〕 
\\	〔余地〕 
\\	〔普通の程度を越えた状態〕 
\\	〔副詞〕 
\\	〔否定〕 
\\	あまり	あまり	
\\	アパート	アパート	▲アパートを 借[か]りる。	
\\	〔建物〕 
\\	〔名前の一部として〕 
\\	〔部屋〕 
\\	アパート	アパート	
\\	あの, あのう	あの, あのう	▲あの 課長[かちょう]3 番[ばん]に お電話[おでんわ]です。	
\\	〔人に呼びかけて〕 
\\	〔考え込んで〕 
\\	〔口ごもって〕 
\\	あの	あの	
\\	あそこ	あそこ	▲あそこに{で}。	
\\	〔向こうに見えるあの場所〕 
\\	〔既知のあの場所・組織・集団など〕 
\\	〔あのような程度・あれほど〕	あそこ	あそこ	
\\	明後日	あさって	▲あさってなら 仕事[しごと]が 休[やす]みだから 私[わたし]も 一緒[いっしょ]に 連[つ]れていってください。	
\\	あさって	あさって	
\\	ああ	ああ	▲ああそうですか。	
\\	〔感動したときなどの発声〕 
\\	《文》 
\\	〔呼びかけ〕 
\\	ああ	ああ	
\\	階	かい	
\\	階建[かいだ]てのビル。	
\\	〔建物の層〕 
\\	【地質】 〔地質時代のそれぞれの期 
\\	に堆積した地層〕 
\\	【数】 〔微分方程式の〕 
\\	〔テンソルの〕 
\\	〔行列・群の〕 
\\	階	〜かい	
\\	回	かい	
\\	回戦[かいせん]
\\	試合[しあい]}。	▲〔ボクシングなどの〕 
\\	〔回数〕 
\\	〔競技の〕 
\\	【野球】 
\\	【クリケット】 
\\	【スヌーカーなど】 
\\	【音楽】 
\\	【解】 
\\	回	〜かい	
\\	鳴る	なる	
\\	風[かぜ]で 電線[でんせん]が 鳴[な]っている。	
\\	〔音を立てる〕 
\\	(鈴・鐘が) 
\\	(雷が) 
\\	(きしんで) 
\\	〔音が感じられる〕 
\\	〔反響する〕 
\\	〔知れわたる〕 
\\	鳴る	なる	
\\	高等学校	こうとうがっこう	
\\	彼[かれ]は 高等学校[こうとうがっこう]の 生徒[せいと]ではありません。	
\\	〔新制の〕 
\\	〔旧制の〕 
\\	[⇒こうこう12]	高等学校	こうとうがっこう	
\\	高校生		▲もう 君[きみ]は 高校生[こうこうせい]なのだから 自分[じぶん]のすることに 責任[せきにん]を 持[も]つべきである。	
\\	*《口》 
\\	*《口》 
\\	高校生	こうこうせい	
\\	雨	あめ	
\\	恵みの雨[めぐみのあめ]。	
\\	回の降雨〕 
\\	(通り雨) 
\\	雨	あめ	
\\	隅	すみ	▲…の 右[みぎ] 上[じょう]
\\	左下[ひだりした]} 隅[すみ]に。	
\\	隅	すみ	
\\	鉛筆	えんぴつ	▲この 鉛筆[えんぴつ]の 字[じ]は 消[け]しゴムで 消[き]えない。	
\\	➡鉛筆の芯は, 柔らかいほうから 
\\	鉛筆	えんぴつ	
\\	遠く	とおく	
\\	遠[とお]くに 住[す]む。	
\\	〔遠方〕 
\\	〔遠い所〕 
\\	〔隔たりが大きいさま〕 (時間的に) 
\\	(距離的に) 
\\	(程度の) 
\\	遠く	とおく	
\\	連絡	れんらく	▲…と 連絡[れんらく]を 取[と]る。	
\\	〔つながり〕 
\\	〔接触〕 
\\	〔通信〕 
\\	【軍】 〔根拠地との〕 
\\	~する 
\\	〔知らせる〕 
\\	〔無線などで〕 
\\	〔交通機関の接続〕 
\\	~する 
\\	連絡	れんらく	
\\	途中	とちゅう	
\\	旅[たび]の 途中[とちゅう]はずっと 快晴[かいせい]だった。	
\\	〔行き来の途上〕 
\\	〔物事が終わらないうち〕	途中	とちゅう	
\\	逃げる	にげる	
\\	警官[けいかん]が 来[き]たのは 泥棒[どろぼう]が 逃[に]げた 後[のち]だった。	
\\	〔避けて立ち去る〕 
\\	(逃走する) 
\\	〔避難する〕 
\\	〔抜け出す〕 
\\	〔脱走する〕 
\\	〔かかわりを避ける〕 
\\	〔消極的になる〕 
\\	〔放散する〕 
\\	〔よそへ流れる〕 
\\	逃げる	にげる	
\\	退院	たいいん	
\\	患者[かんじゃ]さんはあと23 か月[かげつ]で 退院[たいいん]するでしょう。	
\\	〔少年院などからの〕 
\\	~する 
\\	退院	たいいん	
\\	足す	たす	▲きょうの 出費[しゅっぴ]を 全部[ぜんぶ] 足[た]す。	
\\	〔足し算をする〕 
\\	〔補う〕 
\\	〔用を済ます〕 
\\	[⇒よう2]	足す	たす	
\\	足・脚	あし	
\\	川岸[かわぎし]で 遊[あそ]んでいた 男の子[おとこのこ]が 足[あし]を 滑[すべ]らせて 川[かわ]に 落[お]ちた。	
\\	〔足首から先〕 
\\	足	あし	
\\	貴方	あなた	▲あなたたち
\\	方[かた]}。	
\\	〔呼びかけ〕 
\\	(妻が夫に) 
\\	(若夫婦など) 
\\	貴方	あなた	
\\	訳	わけ	
\\	何[なに]か 深[ふか]い 訳[わけ]があるのだろう。	
\\	〔事情・理由〕 
\\	[⇒わけあり] 
\\	〔内容の示すこと・意味〕 
\\	〔道理〕 
\\	〔当然のこと〕 
\\	〔なりゆきの次第や結果〕 
\\	〔意味合い〕 
\\	〔手間〕 
\\	[⇒わけない] 
\\	〔男女の関係〕 
\\	[⇒わけあり]	訳	わけ	
\\	訪ねる	たずねる	
\\	思い出[おもいで]の 場所[ばしょ]を 再[ふたた]び 訪[たず]ねる。	
\\	訪ねる	たずねる	
\\	言う	いう	
\\	彼[かれ]も 言[い]うねえ。	
\\	〔口をきく〕 
\\	〔口に出す〕 
\\	〔言葉で表す〕 
\\	〔告げる〕 
\\	〔主張する〕 
\\	〔断言する〕 
\\	〔説明する〕 
\\	〔評する〕 
\\	〔うわさする〕 
\\	〔報道する〕 
\\	〔称する〕 
\\	〔…と呼ぶ〕 
\\	〔動物などが声を出す〕 
\\	〔物が音をたてる〕 
\\	言う	いう	
\\	見物	けんぶつ	▲…に 見物[けんぶつ]
\\	人[じん])がどっと 押[お]しかける。	
\\	〔名所の〕 
\\	〔催しの〕 
\\	〔事故などの〕 
\\	《口》 
\\	〔人〕 =見物客, 見物人. ~する 〔見る〕 
\\	〔観光する〕 
\\	《口》 
\\	〔傍観する〕 
\\	見物	けんぶつ	
\\	血	ち	
\\	血[ち]を 洗い落[あらいお]とす。	
\\	〔血液〕 
\\	《俗》 
\\	〔凝固した〕 
\\	〔血統〕 
\\	〔人間らしい感情・活力〕	血	ち	
\\	良い	よい	
\\	彼[かれ]は 昔[むかし]はよい 暮[く]らしをしていたのだ。	
\\	〔悪・誤・劣・醜・凶に対して善・正・優・美・吉〕 (上等の) 
\\	(優秀な) 
\\	(良好な) 
\\	(見事な) 
\\	(正しい) 
\\	(上手な) 
\\	(美しい) 
\\	(社会的に上位の) 
\\	(裕福な) 
\\	(値が高い) 
\\	(親しい) 
\\	(吉の) 
\\	(幸運な) 
\\	[⇒いい2, よからぬ, よき, よかれ] 
\\	〔適当〕 (適当な) 
\\	(効果的な) 
\\	(好適な) 
\\	(快適な) 
\\	(好都合な) 
\\	(有益な) 
\\	(望ましい) 
\\	(好ましい) 
\\	[⇒いい2 
\\	〔(「…でよい」 「よかったら」などの形で) 許容・満足〕 [⇒いい2 
\\	よし5, よろしい] 
\\	〔(「…してよい」などの形で) 承認・許可〕 [⇒いい2 
\\	よし5] 
\\	〔(「…だとよい・すればよい・した方がよい」などの形で) 願望・勧誘〕 [⇒いい2 
\\	〔(「…してよかった」の形で) 安心・歓喜〕 
\\	〔(「…よりよい・…の方がよい」などの形で) 比較的まし〕 [⇒いい2 
\\	〔念押し〕 ⇒いい2 
\\	いいか. 
\\	〔(動詞について) …しやすい〕 
\\	[⇒いい2 
\\	良い	いい	
\\	自由	じゆう	
\\	自由[じゆう]な 発想[はっそう]。	
\\	~な 
\\	自由	じゆう	
\\	続ける	つづける	▲こんな 生活[せいかつ]をいつまで 続[つづ]けるつもりなんだ。	
\\	〔終わらせない〕 
\\	〔変えない〕 
\\	〔途切れさせない〕 
\\	〔繰り返す〕 
\\	〔つなげる〕 
\\	続ける	つづける	
\\	競争	きょうそう	
\\	競争[きょうそう]から 抜[ぬ]ける。	
\\	《口》 
\\	~する 
\\	競争	きょうそう	
\\	立てる	たてる	
\\	卵[たまご]を 立[た]てる。	
\\	〔直立させる〕 
\\	《文》 
\\	〔起こす〕 
\\	〔突き刺す〕 
\\	〔擁立する〕 
\\	〔地位につかせる〕 
\\	〔役目を割り当てる〕 
\\	〔それなりに遇する〕 
\\	〔人の面目を保たせる〕 
\\	〔定める〕 
\\	〔表明する〕 
\\	〔発生させる〕 
\\	〔はっきりさせる〕 
\\	〔閉める〕 
\\	[⇒あけたて] 
\\	〔割り算で〕	立てる	たてる	
\\	社長	しゃちょう	
\\	社長[しゃちょう]になる。	
\\	[⇒かいしゃ1 (会社の組織と肩書き)]	社長	しゃちょう	
\\	社会	しゃかい	
\\	社会[しゃかい]を 毒[どく]する。	
\\	〔共同社会〕 
\\	社会	しゃかい	
\\	確か	たしか	
\\	確[たし]かな 証拠[しょうこ]。	
\\	~な 〔間違いのない〕 
\\	《口》 
\\	〔疑いのない〕 
\\	〔確定した〕 
\\	〔明白な〕 
\\	~な 〔確実な〕 
\\	〔信頼しうる〕 
\\	〔偽りのない〕 
\\	〔危なげのない〕 
\\	〔強固な〕 
\\	〔正確な〕 
\\	〈副詞的に〉 〔たぶん〕 
\\	確か	たしか	
\\	知らせる	しらせる	
\\	警察[けいさつ]に 知[し]らせる。	
\\	〔予告する〕 
\\	〔公表する〕 
\\	知らせる	しらせる	
\\	相談	そうだん	▲その 縁談[えんだん]は 本人[ほんにん]たちには 何[なに]の 相談[そうだん]もなく 親[おや] 同士[どうし]の 間[ま]でまとめられた。	
\\	〔話し合い〕 
\\	《口》 
\\	〔申し合わせ〕 
\\	~する 〔話し合う〕 
\\	〔申し合わせる〕 
\\	〔助言を求めること〕 
\\	~する 
\\	〔交渉〕 
\\	相談	そうだん	
\\	直す	なおす	▲このエアコンを 直[なお]してください。	
\\	〔修理[修繕]する〕 
\\	〔復旧させる〕 
\\	〔服装の乱れなどを〕 
\\	〔修復する〕 
\\	〔変換する〕 
\\	〔換算する〕 
\\	《文》 
\\	〔訂正する〕 
\\	〔やめる〕 
\\	〔矯正する〕 
\\	《文》 
\\	《文》 
\\	〔変更する〕 
\\	〔改善する〕 
\\	《文》 
\\	〔格上げする〕 
\\	直す	なおす	
\\	畳	たたみ	
\\	畳[たたみ]の 部屋[へや]。	
\\	〔総称〕 
\\	畳	たたみ	
\\	用事	ようじ	
\\	町[まち]に 行[い]く 用事[ようじ]の ある人[あるひと]にこれを 頼[たの]もう。	
\\	〔使い走りの〕 
\\	[=よう2 
\\	用事	ようじ	
\\	産業	さんぎょう	
\\	産業[さんぎょう]の 発達[はったつ]。	
\\	産業	さんぎょう	
\\	生産	せいさん	▲すでに 来年[らいねん]のカレンダーの 生産[せいさん]が 始[はじ]まった。	
\\	~する 
\\	生産	せいさん	
\\	生まれる・産まれる	うまれる	▲おれなんか 生[う]まれてこなければよかったのだ。	
\\	〔生じる〕 
\\	生まれる	うまれる	
\\	特別	とくべつ	
\\	特別職[とくべつしょく]に 属[ぞく]する 地方公務員[ちほうこうむいん]。	
\\	~な 
\\	特別	とくべつ	
\\	浴びる	あびる	
\\	浴[あ]びるように 飲[の]む。	
\\	《口》 
\\	〔かぶる〕 
\\	〔浸る〕 
\\	〔光などを〕 
\\	〔非難などを〕 
\\	浴びる	あびる	
\\	洗う	あらう	
\\	足[あし]を 洗[あら]う。	▲〔悪事・よくない生活をやめる〕 
\\	〔職業をやめる〕 
\\	〔水などで汚れを落とす〕 
\\	《文》 
\\	〔水などが物にかかる〕 
\\	〔調べ上げる〕 
\\	洗う	あらう	
\\	注射	ちゅうしゃ	
\\	皮下[ひか]にコカインを 注射[ちゅうしゃ]する。	
\\	《口》 
\\	〔接種〕 
\\	~する 
\\	注射	ちゅうしゃ	
\\	泥棒	どろぼう	▲それでは 泥棒[どろぼう]に 追い銭[おいせん]だ。	
\\	〔盗人〕 
\\	(夜盗) 
\\	〔盗み〕 
\\	(くすねること) 
\\	~する 
\\	《口》 
\\	泥棒	どろぼう	
\\	泣く	なく	▲おいおい 泣[な]く。	
\\	〔声を出して〕 
\\	〔涙を流して〕 
\\	《口》 
\\	〔すすり泣く〕 
\\	〔(主に子供が)しくしく泣く〕 
\\	〔嘆き悲しむ〕 
\\	〔苦しむ・苦労する〕 
\\	〔嫌々ながら引き受ける〕 
\\	〔名前や名声にふさわしくない〕 
\\	泣く	なく	
\\	泊まる	とまる	
\\	今晩[こんばん] 泊[と]まってもいいですか。	
\\	〔他家に〕 
\\	〔宿泊施設に〕 
\\	〔病院などに〕 
\\	〔宿直で〕 
\\	〔停泊する〕 
\\	泊まる	とまる	
\\	水道	すいどう	▲お 宅[たく]は 水道[すいどう]ですか 井戸[いど]ですか。	
\\	〔設備〕 
\\	〔導水路〕 
\\	〔用水〕 
\\	〔水路〕 
\\	〔海峡〕 
\\	水道	すいどう	
\\	残念	ざんねん	
\\	残念[ざんねん]そうに。	
\\	~な 〔口惜しい〕 
\\	〔くやしい〕 
\\	残念	ざんねん	
\\	森	もり	
\\	森の奥[もりのおく] 深[ふか]く。	
\\	〔大きな〕 
\\	〔小さな〕 
\\	〔森林地帯〕 
\\	森	もり	
\\	林	はやし	
\\	松[まつ]の 林[はやし]。	
\\	〔小森〕 
\\	〔木立ち〕 
\\	(雑木林) 
\\	(茂み) 
\\	林	はやし	
\\	有る・在る	ある	
\\	伊豆半島[いずはんとう]には 温泉[おんせん]が30ほどある。	
\\	〔存在する〕 
\\	〔…に存する〕 
\\	〔…による〕 
\\	〔位置する〕 
\\	(山・建物などが) 
\\	(都市・国などが) 
\\	(道・川が) 
\\	〔持っている〕 
\\	《文》 
\\	(よいものを) 
\\	(悪いものを) 
\\	〔授かっている〕 
\\	〔付属してある〕 
\\	〔設備がある〕 
\\	〔含まれている〕 
\\	〔数量がある〕 
\\	〔売っている・手に入る〕 
\\	〔経験がある〕 
\\	〔起こる〕 
\\	《文》 
\\	〔開かれる・行われる〕 
\\	(会議などが) 
\\	〔…と書かれている〕 
\\	〔…であるので〕 
\\	〔動作・作用の結果が続いている; 準備されている〕	有る	ある	
\\	月	つき	
\\	月[つき]の 石[いし]。	
\\	〔地球の衛星〕 
\\	〔衛星〕 
\\	~の 
\\	〔時間の単位〕 
\\	《詩・雅》 
\\	月	つき	
\\	最後	さいご	
\\	最後の晩餐[さいごのばんさん]。	▲【キリスト教】 
\\	[⇒「最後の晩餐」].	
\\	〔おしまい〕 
\\	〔結末〕 
\\	~の 
\\	〔いったん…したら, それっきり〕	最後	さいご	
\\	暖房	だんぼう	▲この 部屋[へや]は 暖房[だんぼう]がよくきいている{あまりきいていない}。	
\\	暖房	だんぼう	
\\	明るい	あかるい	▲まだ 外[そと]は 明[あか]るい。	
\\	〔暗いに対して〕 
\\	〔明朗だ〕 
\\	〔公明な〕 
\\	〔物事に通じている〕 
\\	《文》 
\\	《文》 
\\	明るい	あかるい	
\\	新聞社	しんぶんしゃ	
\\	新聞社[しんぶんしゃ]に 勤[つと]める。	
\\	新聞社	しんぶんしゃ	
\\	新しい	あたらしい	
\\	俳句[はいく]を 始[はじ]めたら 平凡[へいぼん]な 日常[にちじょう]の 中[なか]に 新[あたら]しいものが 見[み]えてきた。	
\\	〔新奇な〕 
\\	〔新鮮な〕 
\\	〔最近の〕 
\\	〔現代的〕 
\\	新しい	あたらしい	
\\	故障	こしょう	▲エンジンに 何[なに]か 故障[こしょう]がある。	
\\	〔機械の異常〕 
\\	(破損) 
\\	【電】 
\\	~する 
\\	(コンピューターが) 
\\	《口》 
\\	〔列車などが〕 
\\	〔けがや病気〕 
\\	~する 
\\	〔さしつかえ・障害〕 
\\	〔異議・苦情〕 
\\	故障	こしょう	
\\	政治	せいじ	
\\	政治[せいじ]の 透明性[とうめいせい]。	
\\	〔施政〕 
\\	〔政務〕 
\\	《文》 
\\	政治	政治	
\\	拝見	はいけん	
\\	脈[みゃく]を 拝見[はいけん]しましょう。	
\\	〔見ること〕 
\\	~する 
\\	拝見	はいけん	
\\	招待	しょうたい	
\\	招待[しょうたい] 攻[ぜ]めにあう。	
\\	《口》 
\\	~する 
\\	招待	しょうたい	
\\	手伝う	てつだう	▲ここはぜひ 君[きみ]に 手伝[てつだ]ってもらいたい。	
\\	〔手助けをする〕 
\\	〔ある原因にさらに加わって影響する〕 
\\	手伝う	てつだう	
\\	戦争	せんそう	
\\	私[わたし]は 戦争[せんそう]で 父[ちち]を 失[うしな]った。	
\\	〔戦役〕 
\\	〔戦闘〕 
\\	《文》 
\\	(小規模の) 
\\	(交戦) 
\\	〔トランプ遊び〕 
\\	~する 
\\	〔交戦〕 
\\	〔激しい競争・混乱〕 
\\	戦争	せんそう	
\\	慣れる・馴れる	なれる	
\\	私[わたし]はこんなことには 慣[な]れている。	
\\	【慣れる】 〔日常になる〕 
\\	〔違和感がなくなる〕 
\\	【慣れる】 〔使えるようになる〕 
\\	〔物などになじむ〕 
\\	【慣れる】 〔仕事などを熟知する〕 
\\	〔熟練する〕 
\\	【馴れる】 〔動物や鳥が人になつく〕 
\\	慣れる	なれる	
\\	息子	むすこ	
\\	自慢[じまん]の 息子[むすこ]。	
\\	息子	むすこ	
\\	役に立つ		▲もしもし、 何[なに]かお 役に立[やくにた]ちましょうか。	
\\	〔腕がある〕 
\\	役に立つ	やくにたつ	
\\	建てる	たてる	
\\	山[やま]の 中腹[ちゅうふく]に 建[た]てられたホテル。	
\\	〔建築する〕 
\\	《文》 
\\	建てる	たてる	
\\	店員	てんいん	
\\	店員[てんいん]として 勤[つと]める。	
\\	〔店頭の販売員〕 
\\	(男性) 
\\	(女性) 
\\	〔従業員〕 
\\	店員	てんいん	
\\	小学校	しょうがっこう	
\\	小学校[しょうがっこう]を 出[で]ただけだ。	
\\	小学校	しょうがっこう	
\\	客	きゃく	
\\	不意[ふい]の 客[きゃく]。	
\\	〔訪問客〕 
\\	《口》 
\\	《単複両義》; 〔招かれて来た人〕 
\\	〔滞在客〕 
\\	〔商売の相手〕 
\\	〈集合的に〉 
\\	〔一般の商店の〕 
\\	〔金融・サービス業の〕 
\\	〔ホテルなどの〕 
\\	(短期間の) 
\\	(一泊の) 
\\	〔劇場などの〕 
\\	〈集合的に〉 
\\	〔施設などの〕 
\\	〔美術館など〕 
\\	〔弁護士・医者などの〕 
\\	〈集合的に〉 
\\	〔乗客〕 
\\	客	きゃく	
\\	字	じ	
\\	日本語[にほんご]では 字[じ]の 書き順[かきじゅん]が 決[き]まっている。	
\\	〔文字〕 
\\	など) 
\\	(音節) 
\\	(漢字など) 
\\	(書かれている文) 
\\	〔筆跡〕 
\\	〔助数詞として, 文字を数える語〕	字	じ	
\\	始める	はじめる	
\\	辻[つじ]さんは 最近[さいきん] ロシア語[ろしあご]を 始[はじ]めたらしい。	
\\	〔動作を開始する〕 
\\	〔新たに行動を起こす〕 
\\	〔開業する〕 
\\	始める	はじめる	
\\	妹	いもうと	▲いちばん 下[か]の 妹[いもうと]。	
\\	➡年少であることを特に表現せず単に 
\\	と言うのが英米では普通.	妹	いもうと	
\\	女性	じょせい	
\\	女性[じょせい]の 魅力[みりょく]。	
\\	〔女〕 
\\	〈集合的に〉 
\\	〔女の気質〕 
\\	【文法】 
\\	女性	じょせい	
\\	天気予報	てんきよほう	
\\	天気予報[てんきよほう]の 時間[じかん]。	▲〔ラジオ・テレビの〕 
\\	天気予報	てんきよほう	
\\	大学生	だいがくせい	
\\	大学生[だいがくせい]にもなってこんなことも 知[し]らないのか。	
\\	大学生	だいがくせい	
\\	大きい, 大きな	おおきい, おおきな	
\\	大[おお]きな 平原[へいげん]。	
\\	〔形が〕 
\\	〔容積が〕 
\\	〔面積が〕 
\\	〔数量が〕 
\\	〔音量が〕 
\\	〔程度・規模が〕 
\\	(ひどい) 
\\	〔成長した〕 
\\	〔年長の〕 
\\	〔重要[重大]な意味がある〕 
\\	〔誇大だ〕 
\\	〔尊大だ〕	大きい	おおきい	
\\	地理	ちり	
\\	地理[ちり] 上[じょう] 地理的[ちりてき]に。	
\\	〔地勢〕 
\\	〔地理学〕 
\\	地理	ちり	
\\	台風・颱風	たいふう	
\\	速[はや]い
\\	遅[おそ]い} 台風[たいふう]の 動[うご]き。	
\\	➡「風力 
\\	の風」の意では 
\\	台風	たいふう	
\\	反対	はんたい	
\\	核実験[かくじっけん]に 反対[はんたい]して 座り込[すわりこ]みをする。	
\\	〔同意しないこと〕 
\\	〔異論〕 
\\	~する 
\\	〔逆〕 
\\	〔対の一方〕 
\\	反対	はんたい	
\\	動物園	どうぶつえん	
\\	動物園[どうぶつえん]の 園長[えんちょう]。	
\\	動物園	どうぶつえん	
\\	出発	しゅっぱつ	
\\	早[はや]めに
\\	遅[おそ]く} 出発[しゅっぱつ]する。	
\\	~する 
\\	〔飛行機が〕 
\\	〔船が〕 
\\	出発	しゅっぱつ	
\\	優しい	やさしい	
\\	優[やさ]しいまなざしの。	
\\	〔柔和な〕 
\\	〔優雅な〕 
\\	〔愛情ある〕 
\\	〔温和な〕 
\\	〔親切な〕 
\\	〔温順な〕 
\\	〔人当たりのよい〕 
\\	優しい	やさしい	
\\	倒れる	たおれる	
\\	台風[たいふう]で 木[き]が 倒[たお]れた。	
\\	〔横転する〕 
\\	(人や動物が) 
\\	(建物が) 
\\	(船などが) 
\\	〔病気・疲労などで〕 
\\	〔病床につく〕 
\\	【倒れる・斃れる】 〔死ぬ〕 
\\	《口》 
\\	〔勝負に敗れる〕 
\\	〔くつがえる〕 
\\	〔滅びる〕 
\\	〔倒産する; 破産する〕 
\\	*《口》 
\\	倒れる	たおれる	
\\	住所	じゅうしょ	
\\	住所[じゅうしょ]を 書き取[かきと]る。	
\\	〔住んでいる所番地〕 
\\	〔住んでいる場所〕 
\\	【法】 
\\	住所	じゅうしょ	
\\	伯父さん・叔父さん・小父さん	おじさん	
\\	和夫[かずお]おじさん。	
\\	〔伯父・叔父〕 
\\	《小児語》 
\\	〔年長の男性〕 
\\	伯父さん	おじさん	
\\	二階建て	にかいだて	
\\	二階建[にかいだ]ての 建物[たてもの]。	
\\	二階建て	にかいだて	
\\	上	うえ	
\\	直線[ちょくせん]
\\	はその 平面[へいめん]の 上[うえ]にある。	▲【数】 
\\	〔上部〕 
\\	〔表面〕 
\\	〔頂上; ものの上部〕 
\\	〔上の階〕 
\\	〔比較〕 
\\	〔身分・地位など〕 
\\	〔…に関しては〕 
\\	[=-じょう1] 
\\	〔さらにまた〕 
\\	〔…の後〕 
\\	〔…の結果〕 
\\	〔…した時〕 
\\	〔主人・貴人〕 
\\	上	うえ	
\\	一緒	いっしょ	
\\	一緒[いっしょ]になる。	▲〔ひとまとまりになる〕 
\\	〔結婚する〕 
\\	〔ひとまとめ〕 
\\	〔同一視〕 
\\	〔無差別〕 
\\	[⇒いっしょくた] 
\\	〔同一〕 ⇒おなじ 
\\	〔共に〕 
\\	〔同じ時〕 ⇒どうじ3 
\\	一緒	いっしょ	
\\	一日	いちにち	
\\	一[いち] 日[にち]か2 日[にち]で。	
\\	〔日数〕 
\\	〔終日〕 
\\	〔ある日〕 
\\	〔ついたち〕 
\\	一日	いちにち	
\\	レジ	レジ	▲レジを 打[う]つ。	▲〔レジ係をする〕 
\\	〔レジスターに打ち込む〕 
\\	=レジスター.	レジ	れじ	
\\	もうすぐ	もうすぐ	▲この 画家[がか]はヤマガタのようにもうすぐ 有名[ゆうめい]になりますよ。	
\\	〔もうじき〕 ⇒もうじき, すぐ 
\\	もうすぐ	もうすぐ	
\\	ハンドバッグ	ハンドバッグ	
\\	男[おとこ]は、 彼女[かのじょ]からハンドバッグを 奪[うば]った。	
\\	ハンドバッグ	はんどばっぐ	
\\	パート	パート	▲パート2はたいてい 本作[ほんさく]を 超[こ]えられない。	
\\	〔部分〕 
\\	〔章・編〕 
\\	【音楽】 〔音部・声部〕 
\\	〔パートタイム〕 
\\	〔パートタイム就労者〕 
\\	パート	ぱーと	
\\	テキスト	テキスト	
\\	英語[えいご]のテキスト。	
\\	〔教科書〕 
\\	【'テクスト
\\	とも】 〔原文・本文〕 
\\	【電算】 〔可読文字データ〕 
\\	テキスト	てきすと	
\\	チェック	チェック	
\\	希望[きぼう]するものにチェックしてください。	
\\	〔小切手〕 
\\	〔格子縞〕 
\\	〔照合・検査〕 
\\	~する 
\\	〔しるしをつける〕 
\\	〔関心をもって追うこと・目をつけること〕 ~する 
\\	チェック	ちぇっく	
\\	ステレオ	ステレオ	▲ステレオ 用[よう]に 録音[ろくおん]する。	
\\	〔音響装置〕 
\\	〔方式〕 
\\	ステレオ	すてれお	
\\	ジャム	ジャム	▲ジャムを 作[つく]っておく。	
\\	〔食品〕 
\\	〔ジャズの〕 =ジャム・セッション.	ジャム	じゃむ	
\\	お金	おかね	
\\	お金[おかね]にして。	
\\	=かね1 
\\	お金	おかね	
\\	お母さん	おかあさん	
\\	お母[おかあ]さんになる。	
\\	お母さん	おかあさん	
\\	ええ	ええ	▲ええようございますとも。	
\\	〔肯定〕 
\\	《口》 
\\	〔言いよどみ・ためらい〕 
\\	[⇒ええっ, ええと]	ええ	ええ	
\\	幾つ	いくつ	▲リンゴはもういくつも 残[のこ]っていない。	
\\	〔疑問〕 
\\	〔不定の数〕 
\\	〔何歳〕 
\\	いくつ	いくつ	
\\	あちら	あちら	▲あちらとこちらとでどちらがお 好[す]きですか。	
\\	〔あれ〕 
\\	〔向こうのほう〕 
\\	《文》 
\\	[⇒あっち] 
\\	〔外国〕 
\\	〔あの人〕 
\\	〈複数〉 
\\	〔相手・先方〕 
\\	あちら	あちら	
\\	側	かわ, がわ	
\\	経営者[けいえいしゃ] 側[がわ]の 意向[いこう]。	
\\	〔方面〕 
\\	〔相対する二つの一方〕 
\\	〔取り囲むもの〕 
\\	(覆い) 
\\	(時計の) 
\\	側	〜がわ	
\\	朝御飯	あさごはん	
\\	痩[や]せたいからと 朝ご飯[あさごはん]を 抜[ぬ]くのはいけないことだ。	
\\	朝ご飯	あさごはん	
\\	入り口	いりぐち	
\\	入り口[いりくち]はこちらです。	
\\	〔はいり口〕 
\\	(業務用) 
\\	【機】 
\\	〔物事の始め〕 
\\	入口	いりぐち	
\\	エレベーター	エレベーター	
\\	高層[こうそう] 階[かい]
\\	低層[ていそう] 階[かい]} 行[い]きのエレベーター。	
\\	〔昇降機〕 
\\	【電算】 〔スクロール・ボタン〕 
\\	【空】 =昇降舵 (⇒しょうこう4).	エレベータ	エレベーター	
\\	碁	ご	
\\	碁[ご]の 名人[めいじん]。	
\\	御	ご	
\\	おいしい	おいしい	
\\	ご飯[ごはん]をおいしく 炊[た]くにはコツがある。	
\\	〔味がいい〕 
\\	〔好都合だ・得だ〕	美味しい	おいしい	
\\	酒	さけ	
\\	酒[さけ]が 入[はい]ると 別人[べつじん]のようになる。	
\\	〔アルコール飲料の総称〕 
\\	〔日本酒〕 
\\	〔酒を飲むこと〕 
\\	お酒	さけ	
\\	皿	さら	
\\	大切[たいせつ]な 皿[さら]を 割[わ]ってしまった。	
\\	〔平たい食器〕 
\\	(大皿) 
\\	(受け皿) 
\\	(スープなどを入れる) 
\\	〔皿に盛られた料理〕 
\\	皿分) 
\\	(コース料理の 
\\	皿) 
\\	〔皿の形をしたもの〕	お皿	さら	
\\	中	なか	
\\	中[ちゅう]からかぎをかける。	
\\	〔内部〕 
\\	〔間隔〕 
\\	〔順列の中心〕 
\\	〔範囲〕 
\\	〔最中〕 
\\	〔中道〕 
\\	〔平均・中間〕 
\\	〔仲介・仲裁〕 
\\	お腹	なか	
\\	-界	-かい	
\\	読書界[どくしょかい]。	
\\	〔社会〕 
\\	【博物】 〔自然物の分類〕 
\\	【地質】 〔地質時代のそれぞれの代 
\\	に堆積した地層〕 
\\	【電】 
\\	[⇒でんかい1, じかい8]	〜回	かい	
\\	買い物	かいもの	
\\	買い物[かいもの]から 戻[もど]る。	
\\	〔買うこと〕 
\\	~する =~をする. 
\\	〔買う予定の物・買った物〕 
\\	〔買い得品〕 
\\	買物	かいもの	
\\	-箇月	-かげつ	▲この 赤ん坊[あかんぼう]は8 か月[かげつ]で 生[う]まれてしまった。	
\\	〜か月	かげつ	
\\	側	かわ, がわ	▲ここからだと 富士山[ふじさん]は 北側[きたがわ]に 見[み]える。	
\\	〔方面〕 
\\	〔相対する二つの一方〕 
\\	〔取り囲むもの〕 
\\	(覆い) 
\\	(時計の) 
\\	〜側	がわ	
\\	御存じ	ごぞんじ	
\\	ご存[ごぞん]じのとおり 間[ま]もなく 海外[かいがい]に 出[で]かけます。	
\\	〔知っていること〕	ご存知	ごぞんじ	
\\	問う	とう	
\\	問[と]うは 一旦[いったん]の 恥[はじ] 問[と]わぬは 末代[まつだい]の 恥[はじ]。	▲=聞くは一時の恥, 聞かぬは末代の恥. [⇒きく5 
\\	〔尋ねる〕 
\\	〔質問する〕 
\\	〔調べる〕 
\\	[=きく5 
\\	〔問題にする〕 
\\	〔追及する〕 
\\	〔罪に問う〕 
\\	答	とう	
\\	すごい	すごい	▲すごい 目[め]つきで。	
\\	〔ぞっとするほどに恐ろしい〕 
\\	〔気味悪い〕 
\\	〔非常な〕 
\\	〔非常にすばらしい〕 
\\	《口》 
\\	凄い	すごい	
\\	なかなか	なかなか	▲なかなか 美[うつく]しい 娘[むすめ]。	
\\	〔非常に〕 
\\	〔かなり〕 
\\	〔思ったより〕 
\\	〔容易には〕 
\\	中々	なかなか	
\\	久し振り	ひさしぶり	
\\	久[ひさ]しぶりの 舞台劇[ぶたいげき]を 堪能[たんのう]した。	
\\	久しぶり	ひさしぶり	
\\	見付かる	みつかる	▲たばこを 吸[す]っているところを 親[おや]に 見[み]つかったら 大変[たいへん]だ。	
\\	〔目に留まる〕 
\\	〔見破られる〕 
\\	〔捜し出される〕 
\\	〔探せる〕 〈対象が主語〉 
\\	〈人が主語〉 
\\	〔それと確認できる〕 
\\	〔行き当たる〕 
\\	〔思い当たる〕 
\\	見つかる	みつかる	
\\	高校生		
\\	私[わたし]は 高校生[こうこうせい]です。	
\\	*《口》 
\\	*《口》 
\\	高校生	こうこうせい	
\\	上げる・挙げる	あげる	
\\	花火[はなび]を 上[あ]げる。	
\\	〔上方へ移動させる〕 
\\	〔程度を高くする〕 
\\	〔示す〕 
\\	〔自分の気持ちや考えなどを表す手段として声を出す〕 
\\	〔儀式などを取り行う〕 
\\	〔よい結果を生み出す〕 
\\	〔終える・仕上げる・済ます〕 
\\	〔能力など発揮する〕 
\\	[⇒あげて] 
\\	〔潮が満ちてくる〕 
\\	〔与える・やる・渡す〕 
\\	〔神仏に供える〕 
\\	〔食べた物を戻す〕 ⇒もどす, はく5. 
\\	〔他のためにする動作のていねい表現〕		あげる	
\\	居る	いる	
\\	犬[いぬ]なんかいなくたっていい。	
\\	〔存在する〕 
\\	〔棲息する〕 
\\	〔ある・所有する〕 
\\	〔飼っている〕 
\\	〔(構成員として)所属する〕 
\\	〔(ある場所に)在る・いる〕 
\\	(横たわって) 
\\	(立って) 
\\	(すわって) 
\\	(とどまって) 
\\	〔来ている・居合わせる〕 
\\	[⇒いつづける, いにくい] 
\\	〔居住する〕 
\\	〔滞在する〕 
\\	[⇒いつづける] 
\\	〔(「…ている」の形で)動作や状態の継続・繰り返しを表す〕(現に…している) 
\\	(し続けている) 
\\	〔(「…ている」の形で)完了や経験を表す〕 
\\	〔(「…ずに(は)いられない」「…ないで(は)いられない」の形で)どうしてもがまんできない〕 
\\	いる	
\\	大勢	おおぜい	
\\	大勢[たいせい]で 押[お]しかける。	
\\	〔群集〕 
\\	〔客などの〕 
\\	おおぜい	
\\	起きる	おきる	▲もう 起[お]きる 時間[じかん]ですよ。	
\\	〔起床する〕 
\\	《文》 
\\	《口》 
\\	〔病床を離れる〕 
\\	〔目覚める〕 
\\	《文》 
\\	〔眠らずにいる〕 
\\	〔体を起こす〕 
\\	(倒れた人が) 
\\	(傾いていた船などが) 
\\	(倒れ伏していた植物が) 
\\	〔発生する〕 (事故・症状などが) 
\\	(気持ちが) 
\\	《文》 
\\	おきる	
\\	お祖母さん・お婆さん	おばあさん	
\\	私[わたし]もとうとうおばあさんになりました。	
\\	【お祖母さん】 〔祖母〕 
\\	《口・小児語》 
\\	《小児語》 
\\	〔呼びかけ〕 
\\	【お婆さん】 〔老女〕 
\\	おばあさん	
\\	掛かる・懸かる	かかる	
\\	壁[かべ]に 掛[か]かっている 絵[え]。	
\\	〔固定されて垂れる〕 
\\	〔とめて固定される〕 
\\	[⇒ひっかかる 
\\	〔上に置かれる〕 
\\	〔表面に注がれる〕 (液体が) 
\\	(砂などが) 
\\	〔表面が覆われる〕 
\\	〔まわりに巻きつく〕 
\\	〔道具が作動する・道具の機能が働く・機能がセットされる・契約により掛け金が払われる〕 
\\	〔陥る〕 
\\	〔釣針や網につかまる〕 
\\	〔時間・労力などが費やされる〕 
\\	〔金が〕 
\\	〔着手する〕 
\\	〔取り組む〕 
\\	〔不利益を被る〕 
\\	〔課せられる〕 
\\	〔攻撃をしかける〕 
\\	〔(弱者が)挑む〕 
\\	〔働きかけられる〕 
\\	〔作用が及ぶ〕 
\\	〔ある時期・箇所に至る〕 
\\	〔託される・委ねられる・そのこと次第で物事が決まる〕 
\\	〔受診する〕 
\\	〔(心・目などに)留まる〕 
\\	〔扱われる・処理される〕 
\\	〔上演・上映される〕 
\\	〔兼ねる・関連をもつ〕		かかる	
\\	掛ける・懸ける	かける	▲こちらの 席[せき]におかけください。	
\\	〔固定させて垂らす〕 
\\	〔留めて固定させる〕 
\\	〔上に置く〕 
\\	〔表面に注ぐ〕 (液体を) 
\\	(塩などを) 
\\	〔表面を覆う〕 
\\	〔まわりに巻きつける〕 
\\	〔道具を機能させる・道具で処理する・機能をセットする・契約して掛け金を払う〕 
\\	〔仕組んで陥れる〕 
\\	〔費やす〕 
\\	(金を) 
\\	(時間を) 
\\	(労力を) 
\\	〔不利益・負担を被らせる〕 
\\	〔課する〕 
\\	〔働きかける〕 
\\	〔相手に作用を及ぼす〕 
\\	〔託す・委ねる〕 
\\	〔処理・対応する・心に留める〕 
\\	〔上演・上映する〕 
\\	〔関連づける・かけもちする〕 
\\	〔掛け算をする〕 
\\	〔交尾させる〕 =かけあわせる 
\\	かける	
\\	きっと	きっと	▲あいつが 犯人[はんにん]だよきっと。	
\\	〔間違いなく・必ず〕 
\\	〔自分の決意や相手への要望の強いさま〕 
\\	きっと	
\\	下さる	くださる	▲この お金[おかね]を 私[わたし]にくださるんですか。	
\\	〔お与えになる〕 
\\	〔その動作をしてくれる人を敬った言い方〕 
\\	くださる	
\\	呉れる	くれる	▲これ 私[わたし]にくれるの?	
\\	〔与える〕 
\\	〔贈る〕 
\\	〔給与する〕 
\\	〔人が自分のためになる動作をする〕 
\\	〔人に対して不利益になる動作をする〕		くれる	
\\	ケーキ	ケーキ	
\\	誕生祝[たんじょういわ]いの{バースデー}ケーキ。	
\\	ケーキ	
\\	けれど(も)	けれど(も)	▲いろいろ 忠告[ちゅうこく]したけれど(も)。	
\\	〔逆の関係を示す〕 
\\	[=-が1, かかわらず] 
\\	〔単に接続的に, あるいは特別の意味なしに〕		けれど	
\\	けれど(も)	けれど(も)	▲あらゆる 療法[りょうほう]を 試[ため]したけれど(も)。	
\\	〔逆の関係を示す〕 
\\	[=-が1, かかわらず] 
\\	〔単に接続的に, あるいは特別の意味なしに〕		けれども	
\\	請う・乞う	こう	
\\	出席[しゅっせき]を 乞[こ]う。	
\\	〔頼む〕 
\\	《文》 
\\	〔くれと頼む〕 
\\	こう	
\\	御馳走	ごちそう	▲たいしたごちそうでもなかったね。	
\\	〔もてなすこと〕 
\\	〔食事のもてなし〕 
\\	〔歓待〕 
\\	〔おいしい料理〕 
\\	~する 
\\	《口》 
\\	ごちそう	
\\	事	こと	
\\	事[こと]のなりゆきを 見守[みまも]る。	
\\	⇒-こととて, -ことなく, -ことなしに, -ことに(は), -ことにする, -ことになる, -ことやら, -のことだ(から), -ことだし, -ほどのこと, こととする, こともなげに. 
\\	〔事態・問題〕 
\\	〔重大事〕 
\\	〔事件〕 
\\	〔争いごと〕 
\\	〔計画〕 
\\	〔ことがら〕 
\\	〔事実〕 
\\	〔行為や行動〕 
\\	〔経験〕 
\\	〔場合〕 
\\	〔必要〕 
\\	〔うわさ〕 
\\	〔関心の対象〕 
\\	〔命令や進言の末尾に〕 
\\	〔できる内容〕 
\\	〔思う内容〕 
\\	〔想像の内容〕 
\\	〔時期を示す〕 
\\	〔同一や同義を示す〕 
\\	〔事実を認めて〕 
\\	〔謙遜して〕		こと	
\\	この間	このあいだ	
\\	ついこの間[ついこのあいだ]のことでした。	
\\	〔先日〕 
\\	このあいだ	
\\	この頃	このごろ	▲このごろはリュックを 背負[せお]って 通勤[つうきん]する 人[ひと]もいる。	
\\	〔最近〕 
\\	《文》 
\\	〔当今〕 
\\	このごろ	
\\	塵・芥	ごみ	▲ごみだらけの 部屋[へや]。	
\\	〔不要物〕 
\\	〔汚らしいもの〕 
\\	〔細かいもの〕 
\\	[⇒ちり1]		ごみ	
\\	コンサート	コンサート	▲コンサートを 開[ひら]く。	
\\	コンサート	
\\	コンピューター	コンピューター	▲コンピューターに 明[あか]るい
\\	暗[くら]い}。	
\\	〔電子計算機〕 
\\	コンピューター	
\\	さっき	さっき	▲さっきから。	
\\	さっき	
\\	しっかり	しっかり	▲しっかりつかまる。	
\\	〔物事が堅固なさま・固く〕 
\\	〔着実なさま・きちんと・間違いなく〕		しっかり	
\\	しばらく	しばらく	
\\	石油[せきゆ]はまだしばらくは 高値[たかね]が 続[つづ]くでしょう。	
\\	〔短時間〕 
\\	〔やや長時間〕 
\\	〔当分〕 
\\	〔かなり長時間〕 
\\	(誇張して) 
\\	〔久しぶりの再会の挨拶で〕		しばらく	
\\	邪魔	じゃま	
\\	前[まえ]の 人[ひと]の 頭[あたま]がじゃまで 舞台[ぶたい]がよく 見[み]えない。	
\\	〔妨げ・さしさわり〕 
\\	~な 
\\	〔厄介な〕 
\\	~する 〔妨げる〕 
\\	〔妨害する〕 
\\	じゃま	
\\	随分	ずいぶん	
\\	随分[ずいぶん] 遠[とお]くまで 歩[ある]いたんだねえ。	
\\	〔かなり〕 
\\	〔非常に〕 
\\	~な 〔かなりの〕 
\\	~な 〔ひどい〕 
\\	ずいぶん	
\\	すっかり	すっかり	▲あそこへは この頃[このごろ]すっかりごぶさたしている。	
\\	すっかり	
\\	すっと	すっと	
\\	彼女[かのじょ]はすっと 立[た]っているだけで 絵[え]になる 人[ひと]だ。	
\\	〔まっすぐに〕 
\\	〔静かに〕 
\\	〔すばやく〕 
\\	〔気持ちがすっきりして〕 ~する 〔さっぱりする〕 
\\	〔満足する〕 
\\	〔ほっとする〕 
\\	すっと	
\\	掏摸	すり	▲すりにあう。	
\\	〔人〕 
\\	*《口》 
\\	《戯》 
\\	〔行為〕 
\\	*《口》 
\\	すり	
\\	すると	すると	
\\	玄関[げんかん]のチャイムが 鳴[な]ったので 出[で]ていった.するとそこにはプレゼントを 抱[かか]えた 彼[かれ]が 立[た]っていた。	
\\	〔そうすると〕 
\\	〔それでは〕 
\\	[⇒それでは]		すると	
\\	添う・副う	そう	▲ご 要望[ようぼう]に 添[そ]えない 場合[ばあい]もございます。	
\\	〔離れずにいる〕 
\\	〔合致する〕 
\\	〔応じる〕 
\\	〔夫婦になる〕 
\\	〔交際する〕 
\\	[⇒そいとげる]		そう	
\\	それで	それで	▲それであの 計画[けいかく]はどうなったんですか。	
\\	〔それだから〕 ⇒それだから; 〔そして〕 
\\	それで	
\\	それに	それに	▲そのかばんには 手帳[てちょう]に 財布[さいふ]それに 家[いえ]の 鍵[かぎ]を 入[い]れていました。	
\\	〔そのうえ〕 ⇒そのうえ; 〔加えて〕 
\\	それに	
\\	それ程	それほど	
\\	君[きみ]は 声[こえ]がいいね」「いやそれほどでもないさ」。	
\\	〔(否定を伴って) 取り立てて言うほど〕 
\\	[⇒そんなに]		それほど	
\\	そろそろ	そろそろ	▲そろそろ 歩[ある]く。	
\\	〔ゆっくり〕 
\\	〔静かに〕 
\\	〔その時期や状態になりつつあるようす〕 
\\	そろそろ	
\\	そんな	そんな	▲そんなことだろうと 思[おも]った。	
\\	そんな	
\\	そんなに	そんなに	▲そんなにまで。	
\\	〔そのように(たくさん)〕 
\\	〔(否定を伴って) たいして〕 
\\	そんなに	
\\	大抵	たいてい	▲たいていのことは 彼女[かのじょ]に 聞[き]けばわかる。	
\\	~の 〔大部分の〕 
\\	〔ほとんどの〕 
\\	~の[な] 〔一通りの〕 
\\	〔通常の〕 
\\	〔(副詞的に) 大部分〕 
\\	〔通常〕 
\\	〔ほとんどいつも〕 
\\	〔恐らく〕 
\\	たいてい	
\\	だから	だから	▲「この 仕事[しごと]いやなんです」「だから 辞[や]めるというのか!」。	
\\	だから	
\\	たまに	たまに	▲たまには 頭[あたま]を 使[つか]えよ。	
\\	たまに	
\\	ちっとも	ちっとも	
\\	お腹[おなか]すかない?」「ううんちっとも」。	
\\	〔少しの間も〕 
\\	[=すこしも]		ちっとも	
\\	瀝青	チャン	▲チャンを 塗[ぬ]る。	
\\	ちゃん	
\\	積もり	つもり	
\\	慰[なぐさ]めるつもりでそう 言[い]ったのだ。	
\\	〔心組み〕 
\\	〔意図〕 
\\	(下心) 
\\	〔そうであるという仮定〕 
\\	〔思い込み〕 
\\	〔予算〕 
\\	[⇒みつもり]		つもり	
\\	できるだけ	できるだけ	
\\	彼[かれ]はできるだけたくさんの お金[おかね]を 稼[かせ]いだ。	
\\	=できる限り; 〔なるべく〕 =できることなら.		できるだけ	
\\	到頭	とうとう	▲とうとう 雨[あめ]が 降[ふ]りだした。	
\\	〔結局…しない〕 
\\	とうとう	
\\	どんどん	どんどん	
\\	太鼓[たいこ]がどんどんと 鳴[な]っていた。	
\\	〔強くたたく音〕 
\\	(太鼓などを) 
\\	〔銃声・砲声〕 
\\	〔勢いよく〕 
\\	〔急速に〕 
\\	〔盛んに〕 
\\	[=どしどし]		どんどん	
\\	なさる	
\\	なさる	▲どうなさったんですか. お顔[おかお]の 色[いろ]が 真っ青[まっさお]ですよ。	
\\	〔'する
\\	の尊敬語〕		なさる	
\\	なるべく	なるべく	▲なるべくなら 明日[あした]は 休[やす]みたいのですが。	
\\	〔できる限り〕 
\\	〔できれば〕 
\\	〔どちらかと言えば〕 
\\	〔否定の場合〕 
\\	なるべく	
\\	成程	なるほど	▲なるほど 君[きみ]の 言[い]うことはもっともだ。	
\\	〔確かに〕 
\\	なるほど	
\\	喉	のど	▲のどの 赤[あか]いツバメ。	
\\	〔咽喉〕 
\\	(食道) 
\\	(気管) 
\\	〔首の前面〕 
\\	〔声〕 
\\	〔声帯〕 
\\	【印】 〔本の綴じてある側〕 
\\	のど	
\\	はず	はず	▲…するはずがない。	
\\	〔当然の道理〕 
\\	〔予定〕 
\\	〔確信・確認〕		はず	
\\	はっきり	はっきり	
\\	照明[しょうめい]が 十分[じゅうぶん]でないので 彼[かれ]らの 顔[かお]がはっきりしなかった。	
\\	〔明瞭に〕 
\\	[⇒あきらか] 
\\	〔明確に・確定的に〕 
\\	《文》 
\\	~する 
\\	〔率直に・単刀直入に〕 
\\	~する 
\\	はっきり	
\\	程	ほど	
\\	彼女[かのじょ]の 実力[じつりょく]のほどはまだわからない。	
\\	〔程度〕 
\\	〔限度〕 
\\	〔様子・具合〕 
\\	〔距離〕 
\\	〔時間〕 
\\	ほど	
\\	殆ど	ほとんど	▲この 問題[もんだい]を 解[と]くなんて 私[わたし]にはほとんど 不可能[ふかのう]だ。	
\\	〔大体〕 
\\	(否定を伴って) 
\\	〔大部分〕 
\\	〔もう少しのところで〕		ほとんど	
\\	先ず	まず	▲まず…してから。	
\\	〔最初に〕 
\\	〔ともあれ〕 
\\	[⇒ひとまず] 
\\	〔おそらく〕 
\\	〔おそらく…ない〕 
\\	〔概して〕 
\\	まず	
\\	又は	または	
\\	手紙[てがみ]を 出[だ]すかまたは 電報[でんぽう]を 打[う]つかしなければならない。	
\\	〔言い換えれば〕 
\\	または	
\\	もし	もし	▲もしそうでなければ。	
\\	〔…の場合には〕 
\\	〔条件〕 
\\	もし	
\\	もちろん	もちろん	▲「あしたのパーティーに 出席[しゅっせき]しますか」「もちろんです」。	
\\	もちろん	
\\	尤も	もっとも	▲あなたがそう 言[い]うのももっともだ。	
\\	~な 〔道理のある〕 
\\	〔正しい〕 
\\	〔(人情として)当然の〕 
\\	〔とは言うものの〕		もっとも	
\\	やっと	やっと	▲「もう 読[よ]んじゃった?」「まだやっと3 分[ぶん]の1」。	
\\	〔ようやく〕 
\\	〔苦難の末に〕 
\\	〔かろうじて〕 
\\	〔ぎりぎりで〕 
\\	〔ひどい苦難の末にようやく〕 
\\	やっと	
\\	やはり, やっぱり	やはり, やっぱり	
\\	君[きみ]の 意見[いけん]は?」
\\	私[わたし]もやはり 賛成[さんせい]できません」。	
\\	〔同じく〕 
\\	〔相変わらず〕 
\\	〔つまるところ・しょせん〕 
\\	〔いずれにしても〕 
\\	〔案の定〕 
\\	〔予想どおり〕 
\\	〔再考してみるに〕 
\\	やっぱり	
\\	やはり, やっぱり	やはり, やっぱり	▲どんな 英雄[えいゆう]だって 人の子[ひとのこ]だ.やはり 死[し]ぬのは 恐[こわ]いに 違[ちが]いない。	
\\	〔同じく〕 
\\	〔相変わらず〕 
\\	〔つまるところ・しょせん〕 
\\	〔いずれにしても〕 
\\	〔案の定〕 
\\	〔予想どおり〕 
\\	〔再考してみるに〕 
\\	やはり	
\\	良い	よい	
\\	時々[ときどき] 昔[むかし]はよかったなあと 思[おも]う。	
\\	〔悪・誤・劣・醜・凶に対して善・正・優・美・吉〕 (上等の) 
\\	(優秀な) 
\\	(良好な) 
\\	(見事な) 
\\	(正しい) 
\\	(上手な) 
\\	(美しい) 
\\	(社会的に上位の) 
\\	(裕福な) 
\\	(値が高い) 
\\	(親しい) 
\\	(吉の) 
\\	(幸運な) 
\\	[⇒いい2, よからぬ, よき, よかれ] 
\\	〔適当〕 (適当な) 
\\	(効果的な) 
\\	(好適な) 
\\	(快適な) 
\\	(好都合な) 
\\	(有益な) 
\\	(望ましい) 
\\	(好ましい) 
\\	[⇒いい2 
\\	〔(「…でよい」 「よかったら」などの形で) 許容・満足〕 [⇒いい2 
\\	よし5, よろしい] 
\\	〔(「…してよい」などの形で) 承認・許可〕 [⇒いい2 
\\	よし5] 
\\	〔(「…だとよい・すればよい・した方がよい」などの形で) 願望・勧誘〕 [⇒いい2 
\\	〔(「…してよかった」の形で) 安心・歓喜〕 
\\	〔(「…よりよい・…の方がよい」などの形で) 比較的まし〕 [⇒いい2 
\\	〔念押し〕 ⇒いい2 
\\	いいか. 
\\	〔(動詞について) …しやすい〕 
\\	[⇒いい2 
\\	よい	
\end{CJK}
\end{document}