\documentclass[8pt]{extreport} 
\usepackage{hyperref}
\usepackage{CJKutf8}
\begin{document}
\begin{CJK}{UTF8}{min}
\\	魚。	さかな。	
\\	「だ」	
\\	魚だ。	さかなだ。	
\\	「だ」	
\\	魚じゃない。	さかなじゃない。	
\\	学生じゃない。	がくせいじゃない。	
\\	静かじゃない。	しずかじゃない。	
\\	魚だった。	さかなだった。	
\\	学生じゃなかった。	がくせいじゃなかった。	
\\	静かじゃなかった。	しぞかじゃなかった。	
\\	アリスは学生?/うん、学生。	アリスはがくせい。/うん、がくせい	
\\	「は」 
\\	ジムは明日?/明日じゃない。	ジムはあした?/あしたじゃない。	
\\	「は」 
\\	今日は試験だ。/ジムは?/ジムは明日。	あしたはしけんだ。/ジムは?/ジムはあした。	
\\	「は」 
\\	アリスは学生?/うん、トムも学生。	アリスはがくせい?/うん、トムもがくせい。	
\\	「も」 
\\	アリスは学生?/うん、でもトムは学生じゃない。	アリスはがくせい?/うん、でもトムはがくせいじゃない。	
\\	「も」 
\\	アリスは学生?/ううん、トムも学生じゃない。	アリスはがくせい?/ううん、トムもがくせいじゃない。	
\\	「も」 
\\	誰が学生?/ジムが学生。	だれががくせい?/ジムはがくせい。	
\\	「が」 
\\	誰が学生?	だれががくせい?	
\\	「が」 
\\	学生は誰?	がくせいはだれ?	
\\	「が」 
\\	静かな人。	しずかなひと。	
\\	友達は親切。	ともだちはしんせつ。	
\\	友達は親切な人。	ともだちはしんせつひと。	
\\	ボブは魚が好きだ。	ボブはさかながすきだ。	
\\	ボブは魚が好きじゃない。	ボブはさかながすきじゃない。	
\\	ボブは魚が好きだった。	ボブはさかながすきだった。	
\\	ボブは魚が好きじゃなかった。	ボブはさかながすきじゃなかった。	
\\	魚が好きなタイプ。	さかながすきなタイプ。	
\\	魚が好きじゃないタイプ。	さかながすきじゃないタイプ。	
\\	魚が好きだったタイプ。	さかながすきだったタイプ。	
\\	魚が好きじゃなかったタイプ。	さかながすきじゃなかったタイプ。	
\\	魚が好きじゃないタイプは、肉が好きだ。	さかながすきじゃないタイプは、にくがすきだ。	
\\	高いビル。	たかいビル。	
\\	高くないビル。	たかくないビル。	
\\	高かったビル。	たかかったビル。	
\\	高くなかったビル。	たかくなかったビル。	
\\	静かな高いビル。	すずかなたかいビル。	
\\	高くない静かなビル。	たかくないしずかなビル。	
\\	値段が高いレストランはあまり好きじゃない。	ねだんがたかいレストランはあまりすきじゃない。	
\\	値段があんまりよくない。	ねだんがあんまりよくない。	
\\	彼はかっこよかった!	かれはかっこよかった。	
\\	食べる。	たべる。	
\\	アリスは食べる。	アリスはたべる。	
\\	ジムが遊ぶ。	ジムはあそぶ。	
\\	ボブもする。	ボブもする。	
\\	お金がある。	おかねがある。	
\\	私は買う。	わたしはかう。	
\\	猫はいる。	ねこはいる。	
\\	アリスは食べない。	アリスはたべない。	
\\	ジムが遊ばない。	ジムはあそばない。	
\\	ボブもしない。	ボブもしない。	
\\	お金がない。	おかねがない。	
\\	私は買わない。	わたしはかわない。	
\\	猫はいない。	ねこはいない。	
\\	ご飯は、食べた。	ごはんは、たべた。	
\\	映画は、全部見た。	えいがは、ぜんぶみた。	
\\	今日は、走った。	きょうは、はしった。	
\\	友達が来た。	ともだちがきた。	
\\	私も遊んだ。	わたしもあそんだ。	
\\	勉強は、した。	べんきょうは、した。	
\\	アリスは食べなかった。	アリスはたべなかった。	
\\	ジムがしなかった。	ジムはしなかった。	
\\	ボブも行かなかった。	ボブもいかなかった。	
\\	お金がなかった。	おかねがなかった。	
\\	私は買わなかった。	わたしはかわなかった。	
\\	猫はいなかった。	ねこはいなかった。	
\\	魚を食べる。	さかなをたべる。	
\\	「を」 
\\	ジュースを飲んだ。	ジュースをのんだ。	
\\	「を」 
\\	街をぶらぶら歩く。	まちをぶらぶらあるく。	
\\	「を」 
\\	高速道路を走る。	こうそくどうろをはしる。	
\\	「を」 
\\	毎日、日本語を勉強する。	まいにち、にほんごをべんきょうする。	
\\	「を」 
\\	メールアドレスを登録した。	メールアドレスをとうろくした。	
\\	「を」 
\\	ボブは日本に行った。	ボブはにほんにいった。	
\\	「に」 
\\	家に帰らない。	{いえ
\\	うち}にかえらない。	
\\	「に」 
\\	部屋にくる。	へやにくる。	
\\	「に」 
\\	アリスは、アメリカからきた。	アリスは、アメリカからきた。	
\\	「に」 
\\	宿題を今日から明日までする。	しゅくだいをきょうからあしたまでする。	
\\	「に」 
\\	猫は部屋にいる。	ねこはへやにいる。	
\\	「に」 
\\	椅子が台所にあった。	いすがだいどころにあった。	
\\	「に」 
\\	いい友達に会った。	いいともだちにあった。	
\\	「に」 
\\	ジムは医者になる。	ジムはいしゃになる。	
\\	「に」 
\\	先週に図書館に行った。	せんしゅうにとしょかんにいった。	
\\	「に」 
\\	友達は、来年、日本に行く。	ともだちは、らいねん、にほんにいく。	
\\	「に」 
\\	友達は、来年に日本に行く。	ともだちは、らいねんににほんにいく。	
\\	「に」 
\\	ボブは日本へ行った。	ボブはにほんへいった。	
\\	「へ」 
\\	家へ帰らない。	{いえ
\\	うち}へかえらない。	
\\	「へ」 
\\	部屋へくる。	へやへくる。	
\\	「へ」 
\\	勝ちへ向かう。	かちへくかう。	
\\	「へ」 
\\	映画館で見た。	えいがかんでみた。	
\\	「で」 
\\	バスで帰る。	バスでかえる。	
\\	「で」 
\\	レストランで昼ご飯を食べた。	レストランでひるごはんをたべた。	
\\	「で」 
\\	何できた?/バスできた。	なにできた?/バスできた。	
\\	「で」 
\\	「何」	
\\	何できた?/暇だから。	なんできた?/ひまだから。	
\\	「で」 
\\	「何」	
\\	学校に行った?/行かなかった。/図書館には? /図書館にも行かなかった。	「がっこうにいった?/いかなかった。/としょかんは?/としょかんにもいかなかった。	
\\	どこで食べる? /イタリアレストランではどう?	どこでだべる?/イタリアレストランではどう?	
\\	日本語を習う。	にほんごをならう。	
\\	日本語は、習う。	にほんごは、ならう。	
\\	私が電気を付けた。	わたしがでんきをつけた。	
\\	電気が付いた。	でんきがついた。	
\\	電気を消す。	でんきをけす。	
\\	電気が消える。	でんきがきえる。	
\\	誰が窓を開けた?	だれがまどをあけた?	
\\	窓がどうして開いた?	まどがどうしてあいた?	
\\	部屋を出た。	へやをでた。	
\\	学生じゃない人は、学校に行かない。	がくせいじゃないひとは、がっこうにいかない。	
\\	子供だったアリスが立派な大人になった。	こどもだったアリスがりっぱなおとなになった。	
\\	友達じゃなかったアリスは、いい友達になった。	ともだちじゃなかったアリスは、いいともだちになった。	
\\	先週に医者だったボブは、仕事を辞めた。	せんしゅうにいしゃだったボブは、しごとをやめた。	
\\	先週に映画を見た人は誰?	せんしゅうにえいがをみたひとはだれ?	
\\	ボブは、いつも勉強する人だ。	ボブは、いつもべんきょうするひとだ。	
\\	赤いズボンを買う友達はボブだ。	あかいボンをかうともだちはボブだ。	
\\	晩ご飯を食べなかった人は、映画で見た銀行に行った。	ばんごはんをたべなかったひとは、えいがでみたぎんこうにいった。	
\\	スプーンとフォークで魚を食べた。	スプーンとフォークでさかなをたべた。	
\\	「と」 
\\	本と雑誌と葉書を買った。	ほんとざっしとはがきをかった。	
\\	「と」 
\\	友達と話した。	ともだちとはなした。	
\\	「と」 
\\	先生と会った。	せんせいとあった。	
\\	「と」 
\\	飲み物やカップやナプキンは、いらない?	のみものやカップやナプキンは、いらない?	
\\	「や」 
\\	「とか」 
\\	靴やシャツを買う。	くつやシャツをかう。	
\\	「や」 
\\	「とか」 
\\	飲み物とかカップとかナプキンは、いらない?	のみものとかカップとかナプキンは、いらない?	
\\	「や」 
\\	「とか」 
\\	靴とかシャツを買う。	くつとかシャツをかう。	
\\	「や」 
\\	「とか」 
\\	ボブの本。	ボブのほん。	
\\	「の」 
\\	ボブは、アメリカの大学の学生だ。	ボブは、アメリカのだいがくのがくせいだ。	
\\	「の」 
\\	そのシャツは誰の?/ボブのだ。	そのシャツはだれの?/ボブのだ。	
\\	「の」 
\\	白いのは、かわいい。	しろいのは、かわいい。	
\\	「の」 
\\	授業に行くのを忘れた。	じゅぎょうにいくのをわすれた。	
\\	「の」 
\\	白い物は、かわいい。	しろいものは、かわいい。	
\\	「の」 
\\	授業に行くことを忘れた。	じゅぎょうにいくことをわすれた。	
\\	「の」 
\\	毎日勉強するのは大変。	まいにちべんきょうのはたいへん。	
\\	「の」 
\\	毎日同じ物を食べるのは、面白くない。	まいにちおなじものをたべるのは、おもしろくない。	
\\	「の」 
\\	静かなのが、アリスの部屋だ。	しずかなのが、アリスのへやだ。	
\\	「の」 
\\	今は忙しいの。	いまはいそがしいの。	
\\	「の」 
\\	今は忙しいのだ。	いまはいそがしいのだ。	
\\	「の」 
\\	今は忙しいの?	いまはいそがしいの?	
\\	「の」 
\\	ジムのだ。	ジムのだ。	
\\	「の」 
\\	ジムなのだ。	ジムなのだ。	
\\	「の」 
\\	どこに行くの?/授業に行くんだ。	どこにいくの?/じゅぎょうにいくんだ。	
\\	「の」 
\\	今、授業があるんじゃない?/今は、ないんだ。	いま、じゅぎょうがあるんじゃない?/いまは、ないんだ。	
\\	「の」 
\\	今、授業がないんじゃない?/ううん、ある。	いま、じゅぎょうがないんじゃない?/ううん、ある。	
\\	「の」 
\\	その人が買うんじゃなかったの?/ううん、先生が買うんだ。	そのひとがかうんじゃなかったの?/ううん、せんせいがかうんだ。	
\\	「の」 
\\	朝ご飯を食べるんじゃなかった。/どうして?	あさごはんをたべるんじゃなかった。/どうして?	
\\	「の」 
\\	ボブは朝ご飯を早く食べた。	ボブはあさごはんをはやくたべた。	
\\	アリスは自分の部屋をきれいにした。	アリスはじぶんのへやをきれいにした。	
\\	映画をたくさん見た。	えいがをたくさんみた。	
\\	最近、全然食べない。	さいきん、ぜんぜんたべない。	
\\	ボブの声は、結構大きい。	ボブのこえは、けっこうおおきい。	
\\	この町は、最近大きく変わった。	このまちほ、さいきんおおきくかわった。	
\\	図書館の中では、静かにする。	としょかんのなかでは、しずかにする。	
\\	いい天気だね。/そうね。	いいてんきだね。/そうね。	
\\	「ね」 
\\	おもしろい映画だったね。/え?全然おもしろくなかった。	おもしろいえいがだったね。/え?ぜんぜんおもしろくなかった。	
\\	「ね」 
\\	時間がないよ。/大丈夫だよ。	じかんがないよ。/だいじょうぶだよ。	
\\	「よ」 
\\	今日はいい天気だね。/うん。でも、明日雨が降るよ。	きょうはいいてんきだね。/うん。でも、あしたあめがふるよ。	
\\	「よ」 
\\	ボブは、魚が好きなんだよね。/そうだね。	ボブは、さかながすきなんだよね。/そうだね。	
\\	「よね」	
\\	明日、映画を見に行く。	あした、えいがをみにいく。	
\\	昨日、友達が遊びにきた。	きのう、ともだちがあそびにきた。	
\\	明日、大学に行きます。	あした、だいがくにいきます。	
\\	「~ます」 
\\	先週、ボブに会いましたよ。	せんしゅう、ボブにあいましたよ。	
\\	「~ます」 
\\	晩ご飯を食べませんでしたね。	ばんごはんをたべませんでしたね。	
\\	「~ます」 
\\	面白くない映画は見ません。	おもしろくないえいがほみません。	
\\	「~ます」 
\\	子犬はとても好きです。	こいぬはとてもすきです。	
\\	「です」 
\\	昨日、時間がなかったんです。	きのう、じかんがなかったんです。	
\\	「です」 
\\	その部屋はあまり静かじゃないです。	そのへやはあまりしずかじゃないです。	
\\	「です」 
\\	先週に見た映画は、とても面白かったです。	せんしゅうにみたえいがは、とてもおもしろかったです。	
\\	「です」 
\\	その部屋はあまり静かじゃないですよ。	そのへやはあまりしずかじゃないですよ。	
\\	「です」 
\\	その部屋はあまり静かじゃありませんよ。	そのへやはあまりしずかじゃありませんよ。	
\\	「です」 
\\	私の名前はキムです。	わたしのなまえはキムです。	
\\	僕の名前はキムです。	ぼくのなまえはキムです。	
\\	僕の名前はボブだ。	ぼくのなまえはボブだ。	
\\	俺の名前はボブだ。	おれのなまえはボブだ。	
\\	あたしの名前はアリス。	あたしのなまえはアリス。	
\\	お母さんはどこですか。母は買い物に行きました。	おかあさんはどこですか。はははかいものにいきますた。	
\\	イタリア料理を食べに行きませんか。/すみません。ちょっと、お腹がいっぱいです。	イタリアりょうりをたべにいきませんか。/すみません。ちょっと、おなかがいっぱいです。	
\\	こんなのを本当に食べるか?	こんなのをほんとうにたべるか?	
\\	そんなのは、あるかよ!	そんなのは、あるかよ!	
\\	こんなのを本当に食べる?	こんなのをほんとうにたべる?	
\\	そんなのは、あるの?	そんなのは、あるの?	
\\	昨日何を食べたか忘れた。	きのうなにをたべたかわすれた。	
\\	「か」 
\\	彼は何を言ったかわからない。	かれはなにをいったかわからない。	
\\	「か」 
\\	先生が学校に行ったか教えない?	せんせいががっこうにいったかおしえない?	
\\	「か」 
\\	先生が学校に行ったかどうか知らない。	せんせいががっこうにいったかどうかしらない。	
\\	「か」 
\\	先生が学校に行ったか行かなかったか知らない。	せんせいががっこうにいったかいかなかったかしらない。	
\\	「か」 
\\	誰かがおいしいクッキーを全部食べた。	だれかがおいしいクッキーをぜんぶたべた。	
\\	誰が盗んだのか、誰か知りませんか。	だれがぬすんだのか、だれかしりませんか。	
\\	犯人をどこかで見ましたか。	はんにんをどこかでみましたか。	
\\	この中からどれかを選ぶの。	このなかからどれかをえらぶの。	
\\	この質問の答えは、誰も知らない。	このしつもんのこたえは、だれもしらない。	
\\	友達はいつも遅れる。	ともだちはいつもおくれる。	
\\	ここにあるレストランはどれもおいしくない。	ここにあるレストランはどれもおいしくない。	
\\	今週末は、どこにも行かなかった。	こんしゅうまつは、どこにもいかなかった。	
\\	この質問の答えは、誰でも分かる。	このしつもんのこたえは、だれでもわかる。	
\\	昼ご飯は、どこでもいいです。	ひるごはんは、どこでもいいです。	
\\	あの人は、本当に何でも食べる。	あのひとは、ほんとうになんでもたべる。	
\\	私の部屋は、きれいで、静かで、とても好き。	わたしのへやは、きれいで、しずかで、とてもすき。	
\\	彼女は、学生じゃなくて、先生だ。	彼女は、がくせいじゃなくて、せんせいだ。	
\\	田中さんは、お金持ちで、かっこよくて、魅力的ですね。	たなかさんは、おかねもちで、かっこよくて、みりょくてきですね。	
\\	食堂に行って、昼ご飯を食べて、昼寝をする。	しょくどうにいって、ひるごはんをたべて、ひるねをする。	
\\	食堂に行って、昼ご飯を食べて、昼寝をした。	しょくどうにいって、ひるごはんをたべて、ひるねをした。	
\\	時間がありまして、映画を見ました。	じかんがありまして、えいがをみました。	
\\	時間がなかったからパーティーに行きませんでした。	じかんがなかったからパーティーにいきませんでした。	
\\	「から」 
\\	「ので」	
\\	友達からプレゼントが来た。	ともだちからプレゼントがきた。	
\\	「から」 
\\	「ので」	
\\	友達だからプレゼントが来た。	ともだちだからプレゼントがきた。	
\\	「から」 
\\	「ので」	
\\	どうしてパーティーに行きませんでしたか。/時間がなかったからです。	どうしてパーティーにいきませんでしたか。/じかんがなかったからです。	
\\	「から」 
\\	「ので」	
\\	パーティーに行かなかったの?/うん、時間がなかったから。	パーティーにいかなかったの?/うん、じかんがなかったから。	
\\	「から」 
\\	「ので」	
\\	時間がなかった。/だからパーティーに行かなかったの?	じかんがなかった。/だからパーティーにいかなかったの?	
\\	「から」 
\\	「ので」	
\\	ちょっと忙しいので、そろそろ失礼します。	ちょっといそがしいので、そろそろしつれいします。	
\\	「から」 
\\	「ので」	
\\	私は学生なので、お金がないんです。	わたしはがくせいなので、おかながないんです。	
\\	「から」 
\\	「ので」	
\\	ここは静かなので、とても穏やかです。	ここはすずかなので、とてもおだやかです。	
\\	「から」 
\\	「ので」	
\\	なので、友達に会う時間がない。	なので、ともだちにあうじかんがない。	
\\	「から」 
\\	「ので」	
\\	時間がなかったんでパーティーに行かなかった。	じかんがなかったんでパーティーにいかなかった。	
\\	「から」 
\\	「ので」	
\\	ここは静かなんで、とても穏やかです。	ここはしずかなんで、とてもおだやかです。	
\\	「から」 
\\	「ので」	
\\	なんで、友達に会う時間がない。	なんで、ともだちにあうじかんがない。	
\\	「から」 
\\	「ので」	
\\	毎日運動したのに、全然痩せなかった。	まいにちうんどうしたのに、ぜんぜんやせなかった。	
\\	「のに」 
\\	学生なのに、彼女は勉強しない。	がくせいなのに、かのじょはべんきょうしない。	
\\	「のに」 
\\	デパートに行きましたが、何も欲しくなかったです。	デパートにいきましたが、なにもほしくなかったです。	
\\	「が」 
\\	「けど」	
\\	友達に聞いたけど、知らなかった。	ともだちにきいたけど、しらなかった。	
\\	「が」 
\\	「けど」	
\\	今日は暇だけど、明日は忙しい。	きょうはひまだけど、あしたはいそがしい。	
\\	「が」 
\\	「けど」	
\\	だけど、彼がまだ好きなの。	だけど、かれがまだすきなの。	
\\	「が」 
\\	「けど」	
\\	デパートに行きましたが、いい物がたくさんありました。	デパートにいきましたが、いいものがたくさんありました。	
\\	「が」 
\\	「けど」	
\\	マトリックスを見たけど、面白かった。	マトリックスをみたけど、おもしろかった。	
\\	「が」 
\\	「けど」	
\\	どうして友達じゃないんですか?/先生だし、年上だし・・・。	どうしてともだちじゃないんですか?/せんせいだし、ねんうえだし・・・。	
\\	「し」	
\\	どうして彼が好きなの?/優しいし、かっこいいし、面白いから。	どうしてかれがすきなの?/やさしいし、かっこいいし、おもしろいから。	
\\	「し」	
\\	映画を見たり、本を読んだり、昼寝したりする。	えいがをみたり、ほんをやんだり、ひるねしたりする。	
\\	「~たりする」	
\\	この大学の授業は簡単だったり、難しかったりする。	このだいがくのじゅぎょうはかんたんだったり、むずかしかったりする。	
\\	「~たりする」	
\\	映画を見たり、本を読んだりした。	えいがをみたり、ほんをよんだりした。	
\\	「~たりする」	
\\	映画を見たり、本を読んだりしない。	えいがをみたり、ほんをよんだりしない。	
\\	「~たりする」	
\\	映画を見たり、本を読んだりしなかった。	えいがをみたり、ほんをよんだりしなかった。	
\\	「~たりする」	
\\	友達は何をしているの?/昼ご飯を食べている。	ともだちはなにをしているの?/ひるごはんをたべている。	
\\	「~ている」 
\\	何を読んでいる?/教科書を読んでいます。	なにをよんでいる?/きょうかしょをよんでいます。	
\\	「~ている」 
\\	話を聞いていますか。/ううん、聞いていない。	はなしをきいていますか。/ううん、きいていない。	
\\	「~ている」 
\\	友達は何をしてるの?/昼ご飯を食べてる。	ともだちはなにをしてるの?/ひるごはんをたべている。	
\\	「~ている」 
\\	何を読んでる?/教科書を読んでいます。	なにをよんでる?/きょうかしょをよんでいます。	
\\	「~ている」 
\\	話を聞いていますか。/ううん、聞いてない。	はなしをきいていますか。/ううん、きいてない。	
\\	「~ている」 
\\	今日、知りました。	きょう、しりました。	
\\	この歌を知っていますか?	このうたをしっていますか。	
\\	道は分かりますか。	みちはわかりますか。	
\\	はい、はい、分かった、分かった。	はい、はい、わかった、わかった。	
\\	鈴木さんはどこですか。/もう、家に帰っている。	すずきさんはどこですか。/もう{うち
\\	えい}にかえっている。	
\\	先に行っているよ。/美恵ちゃんは、もう来ているよ。	さきにいっているよ。/みえちゃんは、もうきているよ。	
\\	準備はどうですか。準備は、もうしてあるよ。	じゅんびはどうですか。/じゅんびは、もうしてあるよ。	
\\	「~てある」 
\\	旅行の計画は終った?/うん、切符を買ったし、ホテルの予約もしてある。	りょこうのけいかくはおわった?/うん、きっぷをかったし、ホテルのよやくもしてある。	
\\	「~てある」 
\\	晩ご飯を作っておく。	ばんごはんをつくっておく。	
\\	「~ておく」 
\\	電池を買っておきます。	でんちをかっておきます。	
\\	「~ておく」 
\\	晩ご飯を作っとく。	ばんごはんをつくっとく。	
\\	「~ておく」 
\\	電池を買っときます。	だんちをかっときます。	
\\	「~ておく」 
\\	鉛筆を持っている?	えんぴつをもっている。	
\\	(行く、来る) 
\\	鉛筆を学校へ持っていく?	えんぴつをがっこうへもっていく?	
\\	(行く、来る) 
\\	鉛筆を家に持ってくる?	えんぴつを{うち
\\	いえ}にもってくる。	
\\	(行く、来る) 
\\	お父さんは、早く帰ってきました。	おとうさんは、はやくかえってきました。	
\\	(行く、来る) 
\\	駅の方へ走っていった。	えきのほうへはしっていった。	
\\	(行く、来る) 
\\	冬に入って、コートを着ている人が増えていきます。	ふゆにはいって、コートをきているひとがふえていきます。	
\\	(行く、来る) 
\\	一生懸命、頑張っていく!	いっしょうけんめい、がんばっていく!	
\\	(行く、来る) 
\\	色々な人と付き合ってきたけど、いい人はまだ見つからない。	いろいろなひととつきあってきたけど、いいひとはまだみつからない。	
\\	(行く、来る) 
\\	日本語をずっと前から勉強してきて、結局はやめた。	にほんをずっとまえからべんきょうしてきて、けっきょくはやめた。	
\\	(行く、来る) 
\\	漢字は書けますか?	かんじはかけますか。	
\\	残念だが、今週末は行けない。	ざんねんだが、こんしゅうまつはいけない。	
\\	もう信じられない。	もうしんじられない。	
\\	富士山が登れた。	ふじさんがのぼれた。	
\\	重い荷物が持てます。	おもいにもつがもてます。	
\\	今日は晴れて、富士山が見える。	きょうははれて、ふじさんがみえる。	
\\	「見える」 
\\	「聞こえる」 
\\	友達のおかげで、映画はただで見られた。	ともだちのあかげで、えいがはただでみられた。	
\\	「見える」 
\\	「聞こえる」 
\\	友達のおかげで、映画をただで見ることができた。	ともだちのおかげで、えいがをただでみることができた。	
\\	「見える」 
\\	「聞こえる」 
\\	久しぶりに彼の声が聞けた。	ひさしぶりにかれのこえがきけた。	
\\	「見える」 
\\	「聞こえる」 
\\	周りがうるさくて、彼が言っていることがあんまり聞こえなかった。	まわりがうるさくて、かれがいっていることがあんまりきこえなかった。	
\\	「見える」 
\\	「聞こえる」 
\\	そんなことはありうる。	そんなことはありうる。	
\\	「ある」, 
\\	そんなことはありえる	そんなことはありえる	
\\	「ある」, 
\\	そんなことはありえない。	そんなことはありえない。	
\\	「ある」, 
\\	彼が寝坊したこともありうるね。	かれがねぼうしたこともありうるね。	
\\	「ある」, 
\\	それは、ありえない話だよ。	それは、ありえないはなしだよ。	
\\	「ある」, 
\\	彼の日本語が上手になった。	かれのにほんががじょうずになった。	
\\	する 
\\	なる 
\\	に 
\\	「なる」 
\\	「する」 
\\	私は医者になった。	わたしはいしゃになった。	
\\	する 
\\	なる 
\\	に 
\\	「なる」 
\\	「する」 
\\	私は有名な人になる。	わたしはゆうめいなひとになる。	
\\	する 
\\	なる 
\\	に 
\\	「なる」 
\\	「する」 
\\	私は、ハンバーガーとサラダにします。	わたしは、ハンバーガーとサラダにします。	
\\	する 
\\	なる 
\\	に 
\\	「なる」 
\\	「する」 
\\	他にいいものがたくさんあるけど、やっぱりこれにする。	ほかにいいものがたくさんあるけど、やっぱりこれにする。	
\\	する 
\\	なる 
\\	に 
\\	「なる」 
\\	「する」 
\\	去年から背が高くなったね。	きょねんからせがたかくなったね。	
\\	する 
\\	なる 
\\	に 
\\	「なる」 
\\	運動しているから、強くなる。	うんどうしているから、つよくなる。	
\\	する 
\\	なる 
\\	に 
\\	「なる」 
\\	勉強をたくさんしたから、頭がよくなった。	べんきょうをたくさんしたから、あたまがよくなった。	
\\	する 
\\	なる 
\\	に 
\\	「なる」 
\\	海外に行くことになった。	かいがいにいくことになった。	
\\	する 
\\	なる 
\\	に 
\\	「なる」 
\\	「する」 
\\	毎日、肉を食べるようになった。	まいにち、にくをたべるようになった。	
\\	する 
\\	なる 
\\	に 
\\	「なる」 
\\	「する」 
\\	海外に行くことにした。	かいがいにいくことにした。	
\\	する 
\\	なる 
\\	に 
\\	「なる」 
\\	「する」 
\\	毎日、肉を食べるようにする。	まいにち、にくをたべるようにする。	
\\	する 
\\	なる 
\\	に 
\\	「なる」 
\\	「する」 
\\	日本に来て、寿司が食べられるようになった。	にほんにきて、すしがたべられるようになった。	
\\	する 
\\	なる 
\\	に 
\\	「なる」 
\\	「する」 
\\	一年間練習したから、ピアノが弾けるようになった。	いちねんかんれんしゅうしたから、ピアノがひけるようになった。	
\\	する 
\\	なる 
\\	に 
\\	「なる」 
\\	「する」 
\\	地下に入って、富士山が見えなくなった。	ちかにはいって、ふじさんがみえなくなった。	
\\	する 
\\	なる 
\\	に 
\\	「なる」 
\\	「する」 
\\	ボールを落すと落ちる。	ボールをおとすとおちる。	
\\	「と」	
\\	電気を消すと暗くなる。	でんきをけすとくらくなる。	
\\	「と」	
\\	学校に行かないと友達と会えないよ。	がっこうにいかないとともだちとあえないよ。	
\\	「と」	
\\	たくさん食べると太るよ。	たくさんたべるとふとるよ。	
\\	「と」	
\\	先生だと、きっと年上なんじゃないですか?	せんせいだと、きっととしうえなんじゃないですか?	
\\	「と」	
\\	みんなが行くなら私も行く。	みんながいくならわたしもいく。	
\\	「なら(ば)」	
\\	アリスさんが言うなら問題ないよ。	アリスさんがいうならもんだいないよ。	
\\	「なら(ば)」	
\\	図書館はどこですか。/図書館なら、あそこです。	としょかんはどこですか。/としょかんなら、あそこです。	
\\	「なら(ば)」	
\\	友達に会えれば、買い物に行きます。	ともだちにあえれば、かいものにいきます。	
\\	「ば」	
\\	お金があればいいね。	おかねがあればいいね。	
\\	「ば」	
\\	楽しければ、私も行く。	たのしければ、わたしもいく。	
\\	「ば」	
\\	楽しくなければ、私も行かない。	たのしくなければ、わたしもいかない。	
\\	「ば」	
\\	食べなければ病気になるよ。	たべなければびょうきになるよ。	
\\	「ば」	
\\	暇だったら、遊びに行くよ。	ひまだったら、あそびにいくよ。	
\\	「たら(ば)」	
\\	学生だったら、学生割引で買えます。	がくせいだったら、がくせいわりびきでかえます。	
\\	「たら(ば)」	
\\	友達に会えれば、買い物に行きます。	ともだちにあえれば、かいものにいきます。	
\\	「たら(ば)」	
\\	友達に会えたら、買い物に行きます。	ともだちにあえたら、かいものにいきます。	
\\	「たら(ば)」	
\\	お金があればいいね。	おかねがあればいいね。	
\\	「たら(ば)」	
\\	お金があったらいいね。	おかねがったらいいね。	
\\	「たら(ば)」	
\\	家に帰ったら、誰もいなかった。	いえにかえったら、だれもいなかった。	
\\	「たら(ば)」	
\\	アメリカに行ったら、たくさん太りました。	アメリカにいったら、たくさんふとりました。	
\\	「たら(ば)」	
\\	もしよかったら、映画を観に行きますか?	もしよかったら、えいがをみにいきますか?	
\\	「もし」 
\\	もし時間がないなら、明日でもいいよ。	もしじかんがないなら、あしたでもいいよ。	
\\	「もし」 
\\	ここに入ってはいけません。	ここにはいってはいけません。	
\\	「だめ」, 「いけない」, 
\\	「ならない」 
\\	それを食べてはだめ!	それをたべてはだめ!	
\\	「だめ」, 「いけない」, 
\\	「ならない」 
\\	夜、遅くまで電話してはならない。	よる、おそくまででんわしてはならない。	
\\	「だめ」, 「いけない」, 
\\	「ならない」 
\\	早く寝てはなりませんでした。	はやくねてはなりませんでした。	
\\	「だめ」, 「いけない」, 
\\	「ならない」 
\\	毎日学校に行かなくてはなりません。	まいにちがっこうにいかなくてはなりません。	
\\	宿題をしなくてはいけなかった。	しゅくだいをしなくてはいけなかった。	
\\	毎日学校に行かないとだめです。	まいにちがっこうにいかないとだめです。	
\\	宿題をしないといけない。	しゅくだいをしないといけない。	
\\	毎日学校に行かなければいけません。	まいにちがっこうにいかなければいけません。	
\\	宿題をしなければだめだった。	しゅくだいをしなければだめだった。	
\\	勉強しなくちゃ。	べんきょうしなくちゃ。	
\\	ご飯を食べなきゃ。	ごはんをたべなきゃ。	
\\	学校に行かないと。	がっこうにいかないと。	
\\	ここに入っちゃだめだよ。	ここにはいっちゃだめだよ。	
\\	死んじゃだめだよ!	しんじゃだめだよ。	
\\	全部食べてもいいよ。	ぜんぶたべてもいいよ。	
\\	全部食べなくてもいいよ。	ぜんぶたべなくてもいいよ。	
\\	全部飲んでも大丈夫だよ。	ぜんぶのんでもだいじょうぶだよ。	
\\	全部飲んでも構わないよ。	ぜんぶのんでもかまわないよ。	
\\	もう帰っていい?	もうかえっていい?	
\\	これ、ちょっと見ていい?	これ、ちょっとみていい?	
\\	何をしたいですか。	なにをしたいですか。	
\\	「たい」	
\\	温泉に行きたい。	おんせんにいきたい。	
\\	「たい」	
\\	ケーキ、食べたくないの?	ケーキ、たべたくないの?	
\\	「たい」	
\\	食べたくなかったけど食べたくなった。	たべたくなかったけどたべtくなった。	
\\	「たい」	
\\	ずっと一緒にいたい。	ずっといっしょにいたい。	
\\	「たい」	
\\	犬と遊びたいですか。	いぬとあそびたいですか。	
\\	「たい」	
\\	大きい縫いぐるみが欲しい!	おおきいぬいぐるみがほしい!	
\\	「欲しい」	
\\	全部食べてほしいんだけど・・・。	ぜんぶたべてほしいんだけど・・・。	
\\	「欲しい」	
\\	部屋をきれいにしてほしいのよ。	へやをきれいにしてほしいのよ。	
\\	「欲しい」	
\\	今日は何をしようか?/テーマパークに行こう!	あしたはなにをしようか?/テーマパークにいこう!	
\\	明日は何を食べようか?/カレーを食べよう!	あしたはなにをたべようか?/カレーをたべよう!	
\\	今日は何をしましょうか?/テーマパークに行きましょう!	あしたはなにをしましょうか?/テーマパークにいきましょう!	
\\	明日は何を食べましょうか?/カレーを食べましょう!	あしたはなにをたべましょうか?/カレーをたべましょう!	
\\	銀行に行ったらどうですか。	ぎんこうにいったらどうですか。	
\\	「ば」 
\\	「たら」 
\\	たまにご両親と話せばどう?	たまにごりょうしんとはなせばどう?	
\\	「ば」 
\\	「たら」 
\\	アリスが、「寒い」と言った。	アリスが、「さむい」といった。	
\\	「今日は授業がない」と先生から聞いたんだけど。	「きょうはじゅぎょうがない」とせんせいからきいたんだけど。	
\\	「寒い」とアリスが田中に言った。	「さむい」とアリスがたなかにいった。	
\\	先生から今日は授業がないと聞いたんだけど。	せんせいからきょうはじゅぎょうがないときいたんだけど。	
\\	これは、日本語で何と言いますか。	これは、にほんごでなんといいますか。	
\\	私は、アリスと言います。	わたしは、アリスといいます。	
\\	カレーを食べようと思ったけど、食べる時間がなかった。	カレーをたべようとおもったけど、たべるじかんがなかった。	
\\	今、どこに行こうかと考えている。	いま、どこにいこうかとかんがえている。	
\\	彼は、これは何だと言いましたか。	かれは、これはなんだといいましたか。	
\\	彼は高校生だと聞いたけど、信じられない。	かれはこうこうせいだときいたけど、しんじられない。	
\\	これは何だと言いましたか。	これはなんだといいましたか。	
\\	何と言いましたか。	なんといいましたか。	
\\	智子は来年、海外に行くんだって。	ともこはらいねん、かいがいにいくんだって。	
\\	「って」 
\\	「と」	
\\	もうお金がないって。	もうおかねがないって。	
\\	「って」 
\\	「と」	
\\	え?何だって?	え?なんだって?	
\\	「って」 
\\	「と」	
\\	今、時間がないって聞いたんだけど、本当?	いま、じかんがないってきいたんだけど、ほんとう?	
\\	「って」 
\\	「と」	
\\	今、時間がないって、本当?	いま、じかんがないって、ほんとう?	
\\	「って」 
\\	「と」	
\\	明日って、雨が降るんだって。	あしたって、あめがふるんだって。	
\\	「って」 
\\	「と」	
\\	アリスって、すごくいい人でしょ?	アリスって、すごくいいひとでしょ?	
\\	「って」 
\\	「と」	
\\	これは、なんという魚ですか。	これは、なんというさかなですか。	
\\	「いう」 
\\	この魚は、鯛といいます。	このさかなは、たいといいます。	
\\	「いう」 
\\	ルミネというデパートはどこにあるか、知っていますか?	ルミネというデパートはどこにあるか、しっていますか?	
\\	「いう」 
\\	「友達」は、英語で
\\	という意味です。	「ともだち」は、えいがで
\\	といういみです。	
\\	「いう」 
\\	主人公が犯人だったというのが一番面白かった。	しゅじんこうがはんにんだったというのがいちばんおもしろかった。	
\\	「いう」 
\\	日本人はお酒に弱いというのは本当?	にほんじんはおさけによわいというのはほんとう?	
\\	「いう」 
\\	独身だというのは、嘘だったの?	どくしんだというのは、うそだったの?	
\\	「いう」 
\\	リブートというのは、パソコンを再起動するということです。	リブートというのは、パソコンをさいきどうするということです。	
\\	「いう」 
\\	あんたは、いつもこういう時に来るんだから、困るんだよ。	あんたは、いつもこういうときにくるんだから、こまるんだよ。	
\\	「いう」 
\\	そういう人と一緒に仕事をするのは、嫌だよね。	そういうひとといっしょにしごとをするのは、いやだよね。	
\\	「いう」 
\\	ああいう人と結婚できたら、幸せになれると思います。	ああいうひととけっこんできたら、しあわせになれるとおもいます。	
\\	「いう」 
\\	大学に行かないって、どういう意味なの?	だいがくにいかないって、どういういみなの?	
\\	「いう」 
\\	みきちゃんは、あんたの彼女でしょう?/う~ん、彼女というか、友達というか、なんというか・・・	みきちゃんは、あんたのかのじょでしょう?/う~ん、かのじょというか、ともだちというか、なんというか・・・	
\\	「という」	
\\	お酒は好きというか、ないと生きていけない。	おさけはすきというか、ないといきていけない。	
\\	「という」	
\\	多分行かないと思う。というか、お金がないから、行けない。	たぶんいかないとおもう。というか、おかねがないから、いけない。	
\\	「という」	
\\	というか、もう帰らないとだめですけど。	というか、もうかえらないとだめですけど。	
\\	「という」	
\\	みきちゃんが洋介と別れたんだって。/ということは、みきちゃんは、今彼氏がいないということ?/ そう。そういうこと。	みきちゃんがようすけとわかれたんだって。/ということは、みきちゃんは、いまかれしがいないということ?/ そう。そういうこと。	
\\	「って」 
\\	「て」 
\\	「という」	
\\	来年留学するというのは、智子のこと?	らいねんりゅうがくするというのは、ともこのこと?	
\\	「って」 
\\	「て」 
\\	「という」	
\\	来年留学するって智子のこと?	らいねんりゅうがくするってともこのこと?	
\\	「って」 
\\	「て」 
\\	「という」	
\\	しないとだめだよ。/だって、時間がないからできないよ。	しないとだめだよ。/だって、じかんがないからできないよ。	
\\	「って」 
\\	「て」 
\\	「という」	
\\	行かなくてもいいよ。/だって、みんな行くって。私も行かないと。	いかなくてもいいよ。/だって、みんないくって。わたしもいかないと。	
\\	「って」 
\\	「て」 
\\	「という」	
\\	てことは、みきちゃんは、今彼氏がいないてこと?	てことは、みきちゃんは、いまかれしがいないてこと?	
\\	「って」 
\\	「て」 
\\	「という」	
\\	ていうか、もう帰らないとだめですけど。	ていうか、もうかえらないとだめですけど。	
\\	「って」 
\\	「て」 
\\	「という」	
\\	みきちゃんが、明日こないって。	みきちゃんが、あしたこないって。	
\\	「って」 
\\	「て」 
\\	「という」	
\\	てゆうか、もう帰らないとだめですけど。	てゆうか、もうかえらないとだめですけど。	
\\	「ゆう」 
\\	「いう」	
\\	そうゆうことじゃないって!	そうゆうことじゃないって!	
\\	「ゆう」 
\\	「いう」	
\\	お好み焼きを始めて食べてみたけど、とてもおいしかった!	おこのみやきをはじめてたべてみたけど、とてもおいしかった!	
\\	お酒を飲んでみましたが、すごく眠くなりました。	おさけをのんでみましたが、すごくねむくなりました。	
\\	新しいデパートに行ってみる。	あたらしいデパートにいってみる。	
\\	広島のお好み焼きを食べてみたい!	ひろしまのおこのみやきをたべてみたい!	
\\	毎日、勉強を避けようとする。	まいにち、べんきょうをさけようとする。	
\\	無理矢理に部屋に入ろうとしている。	むりやりにへやにはいろうとしている。	
\\	早く寝ようとしたけど、結局は徹夜した。	はやくねようとしたけど、けっきょくはてつやした。	
\\	お酒を飲もうとしたが、奥さんが止めた。	おさけをのもうとしたが、おくさんがとめた。	
\\	勉強をなるべく避けようと思った。	べんきょうをなるべくさけようとおもった。	
\\	毎日ジムに行こうと決めた。	まいにちジムにいこうときめた。	
\\	私が友達にプレゼントをあげた。	わたしがともだちにプレゼントをあげた。	
\\	「あげる」	
\\	これは先生にあげる。	これはせんせいにあげる。	
\\	「あげる」	
\\	車を買ってあげるよ。	くるまをかってあげるよ。	
\\	「あげる」	
\\	代わりに行ってあげる。	かわりにいってあげる。	
\\	「あげる」	
\\	学生がこれを先生にあげる。	がくせいがこれをせんせいにあげる。	
\\	「あげる」	
\\	友達が父にいいことを教えてあげた。	ともだちがちちにいいことをおしえてあげた。	
\\	「あげる」	
\\	犬に餌をやった?	いぬにえさをやった?	
\\	「やる」 
\\	「あげる」	
\\	友達が私にプレゼントをくれた。	ともだちがわたしにプレゼントをくれた。	
\\	「くれる」	
\\	これは、先生がくれた。	これは、せんせいがくれた。	
\\	「くれる」	
\\	車を買ってくれるの?	くるまをかってくれるの?	
\\	「くれる」	
\\	代わりに行ってくれる?	かわりにいってくれる?	
\\	「くれる」	
\\	先生がこれを学生にくれる。	せんせいがこれをがくせいにくれる。	
\\	「くれる」	
\\	友達が父にいいことを教えてくれた。	ともだちがちちにいいことをおしえてくれた。	
\\	「くれる」	
\\	先生が教えてあげるんですか。	せんせいがおしえてあげるんですか。	
\\	「くれる」	
\\	先生が教えてくれるんですか。	せんせいがおしえてくれるんですか。	
\\	「くれる」	
\\	私が全部食べてあげました。	わたしがぜんぶたべてあげました。	
\\	「くれる」	
\\	友達がプレゼントを私にくれた。	ともだちがプレゼントをわたしにくれた。	
\\	「くれる」	
\\	私が友達にプレゼントをもらった。	わたしがともだちにプレゼントをもらった。	
\\	「もらう」	
\\	友達からプレゼントをもらった。	ともだちからプレゼントをもらった。	
\\	「もらう」	
\\	これは友達に買ってもらった。	これはともだちにかってもらった。	
\\	「もらう」	
\\	宿題をチェックしてもらいたかったけど、時間がなくて無理だった。	しゅくだいをチェックしてもらいたかったけど、じかんがなくてむりだった。	
\\	「もらう」	
\\	その時計は私からもらったのよ。	そのとけいはわたしからもらったのよ。	
\\	「もらう」	
\\	千円を貸してくれる?	せんえんをかしてくれる。	
\\	「くれる」 
\\	「もらえる」	
\\	千円を貸してもらえる?	せんえんをかしてもらえる?	
\\	「くれる」 
\\	「もらえる」	
\\	ちょっと静かにしてくれない?	ちょっとしずかにしてくれない?	
\\	「くれる」 
\\	「もらえる」	
\\	漢字を書いてもらえませんか。	かんじをかいてもらえませんか。	
\\	「くれる」 
\\	「もらえる」	
\\	全部食べないでくれますか。	ぜんぶたべないでくれますか。	
\\	「くれる」 
\\	「もらえる」
\\	高い物を買わないでくれる?	たかいものをかわないでくれる?	
\\	「くれる」 
\\	「もらえる」
\\	それをください。	それをください。	
\\	「~ください」- 
\\	「くださる」	
\\	それをくれる?	それをくれる?	
\\	「~ください」- 
\\	「くださる」	
\\	漢字で書いてください。	かんじでかいてください。	
\\	「~ください」- 
\\	「くださる」	
\\	ゆっくり話してください。	ゆっくりはなしてください。	
\\	「~ください」- 
\\	「くださる」	
\\	落書きを書かないでください。	らくがきをかかないでください。	
\\	「~ください」- 
\\	「くださる」	
\\	ここにこないでください。	ここにこないでください。	
\\	「~ください」- 
\\	「くださる」	
\\	日本語で話して。	にほんごではなして。	
\\	「~ください」- 
\\	「くださる」	
\\	消しゴムを貸して。	けしゴムをかして。	
\\	「~ください」- 
\\	「くださる」	
\\	遠い所に行かないで。	とおいところにいかないで。	
\\	「~ください」- 
\\	「くださる」	
\\	日本語で話してくれ。	にほんごではなしてくれ。	
\\	「~ください」- 
\\	「くださる」	
\\	消しゴムを貸してくれ。	けしゴムをかしてくれ。	
\\	「~ください」- 
\\	「くださる」	
\\	遠い所に行かないでくれ。	とおいところにいかないでくれ。	
\\	「~ください」- 
\\	「くださる」	
\\	お父さんがくれた時計が壊れた。	おとうさんがくれたとけいがこわれた。	
\\	「~ください」- 
\\	「くださる」	
\\	「それをください」とお父さんが言った。	「それをください」とおとうさんがいった。	
\\	「~ください」- 
\\	「くださる」	
\\	スプーンをちょうだい。	スプーンをちょうだい。	
\\	「~ちょうだい」 
\\	ここに名前を書いてちょうだい。	ここになまえをかいてちょうだい。	
\\	「~ちょうだい」 
\\	よく聞きなさい!	よくききなさい!	
\\	「~なさい」 
\\	ここに座りなさい。	ここにすわりなさい。	
\\	「~なさい」 
\\	まだいっぱいあるから、たくさん食べな。	まだいっぱいあるから、たくさんたべな。	
\\	「~なさい」 
\\	それでいいと思うなら、そうしなよ。	それでいいとおもうなら、そうしなよ。	
\\	「~なさい」 
\\	好きにしろ。	すきにしろ。	
\\	あっち行け!	あっちいけ!	
\\	早く酒を持ってきてくれ。	はやくさけをもってきてくれ。	
\\	それを食べるな!	それをたべるな!	
\\	変なことを言うな!	へんなことをいうな!	
\\	今、図書館に行くんだよな。/うん、なんで?	いま、としょかんにいくんだよな。/うん、 なんで?	
\\	「な」 
\\	「さ」	
\\	日本語は、たくさん勉強したけどな。まだ全然わからない。/大丈夫よ。きっとわかるようになるからさ。/ならいいけどな。	にほんごは、たくさんべんきょうしたけどな。まだぜんぜんわからない。/だいじょうぶよ。きっとわかるようになるからさ。/ならいいけどな。	
\\	「な」 
\\	「さ」	
\\	今日は雨が降るかな?	きょうはあめがふるかな?	
\\	「な」 
\\	「さ」	
\\	いい大学に行けるかな?	いいだいがくにいけるかな?	
\\	「な」 
\\	「さ」	
\\	もう時間が ないわ。	もうじかんがないわ。	
\\	「わ」、「ぞ」、「ぜ」、「かしら」	
\\	おい、行くぞ!	おい、いくぞ!	
\\	「わ」、「ぞ」、「ぜ」、「かしら」	
\\	これで、もう終わりだぜ。	これで、もうおわりだぜ。	
\\	「わ」、「ぞ」、「ぜ」、「かしら」	
\\	いい大学に入れるかしら?	いいだいがくにはいれるかしら?	
\\	「わ」、「ぞ」、「ぜ」、「かしら」	
\\	全部食べさせた。	ぜんぶたべさせた。	
\\	全部食べさせてくれた。	せんぶたべさせてくれた。	
\\	先生が学生に宿題をたくさんさせた。	先生ががくせいにしゅくだいをたくさんさせた。	
\\	先生が質問をたくさん聞かせてくれた。	せんせいがしつもんをたくさんきかせてくれた。	
\\	今日は仕事を休ませてください。	きょうはしごとをやすませてください。	
\\	その部長は、よく長時間働かせる。	そのぶちょうは、よくちょうじかんはたらかせる。	
\\	トイレに行かせてくれますか。	トイレにいかせてくれますか。	
\\	トイレに行ってもいいですか。	といれにいってもいいですか。	
\\	同じことを何回も言わすな!	おなじことをなんかいもいわすな!	
\\	お腹空いているんだから、なんか食べさしてくれよ。	おなかあいているんだから、なんかたべさしてくれよ。	
\\	ポリッジが誰かに食べられた!	ポリッジがだれかにたべられた。	
\\	みんなに変だと言われます。	みんなにへんだといわれます。	
\\	光の速さを超えるのは、不可能だと思われる。	ひかりのはやさをこえるのは、ふかのうだとおもわれる。	
\\	この教科書は多くの人に読まれている。	このきょうかしょはおおくのひとによまれている。	
\\	外国人に質問を聞かれたが、答えられなかった。	がいこくじんにしつもんをきかれたが、こたえられなかった。	
\\	このパッケージには、あらゆるものが含まれている。	このパッケージには、あらゆるものがふくまれている。	
\\	レシートはどうされますか?	レシートはどうされますか?	
\\	明日の会議に行かれるんですか?	あしたのかいぎにいかれるんですか?	
\\	朝ご飯は食べたくなかったのに、食べさせられた。	あさごはんはたべたくなかったのに、たべさせられた。	
\\	日本では、お酒を飲ませられることが多い。	にほんでは、おさけをのませられることがおおい。	
\\	あいつに二時間も待たせられた。	あいつににじかんもまたせられた。	
\\	親に毎日宿題をさせられる。	おやにまいにちしょくだいをさせられる。	
\\	学生が廊下に立たされた。	がくせいがろうかにたたされた。	
\\	日本では、お酒を飲まされることが多い。	にほんでは、おさけをのまされることがおおい。	
\\	あいつに二時間も待たされた。	あいつににじかんもまたされた。	
\\	アリスさん、もう召し上がりましたか。	アリスさん、もうめしあがりましたか。	
\\	仕事で何をなさっているんですか。	しごとでなにをなさっているんですか。	
\\	推薦状を書いてくださるんですか。	すいせんじょうをかいてくださるんですか。	
\\	どちらからいらっしゃいましたか。	どちらからいらっしゃいましたか。	
\\	今日は、どちらへいらっしゃいますか。	きょうは、どちらへいらっしゃいますか。	
\\	私はキムと申します。	わたしはキムともうします。	
\\	私が書いたレポートを見ていただけますか。	わたしがかいたレポートをみていただけますか。	
\\	失礼致します。	しつれいします。	
\\	こちらは、私の部屋です。	こちらは、わたしのへやです。	
\\	こちらは、私の部屋でございます。	こちらは、わたしのへやでございます。	
\\	お手洗いはこのビルの二階にあります。	おてあらいはこのビルのにかいにあります。	
\\	お手洗いはこのビルの二階にございます。	おてあらいはこのビルのにかいにございます。	
\\	先生はお見えになりますか。	せんせいほかえになりますか。	
\\	もうお帰りですか。	もうおかえりですか。	
\\	店内でお召し上がりですか。	てんないでおめしあがりですか。	
\\	少々お待ちください。	しょうしょうおたちください。	
\\	「ください」 
\\	こちらにご覧下さい。	こちらにごらんください。	
\\	「ください」 
\\	閉まるドアにご注意下さい。	しまるドアにごちゅういください。	
\\	「ください」 
\\	よろしくお願いします。	よろしくおねがいします。	
\\	先生、お聞きしたいことがありますが。	せんせい、おききしたいことがありますが。	
\\	すみません、お待たせしました。	すみません、おまたせしました。	
\\	千円からお預かりいたします。	せんえんからおあずかりいたします。	
\\	いらっしゃいませ。	いらっしゃいませ。	
\\	いらっしゃい!	いらっしゃい!	
\\	ありがとうございました。またお越しくださいませ。	ありがとうございました。またおこしくださいませ。	
\\	どうぞ、ごゆっくりなさいませ。	どうぞ、ごゆっくりなさいませ。	
\\	宿題をやった?/しまった!	しゅくだいをやった?/しまった!	
\\	そのケーキを全部食べてしまった。	そのケーキをぜんぶたべてしまった。	
\\	「しまう」 
\\	毎日ケーキを食べて、2キロ太ってしまいました。	まいにちケーキをたべて、2キロふとってしまいました。	
\\	「しまう」 
\\	ちゃんと食べないと、痩せてしまいますよ。	ちゃんとたべないと、やせてしまいますよ。	
\\	「しまう」 
\\	結局、嫌なことをさせてしまった。	けっきょく、いやなことをさせてしまった。	
\\	「しまう」 
\\	ごめん、待たせてしまって!	ごめん、またせてしまって!	
\\	「しまう」 
\\	金魚がもう死んでしまった。	きんぎょもうしんでしまった。	
\\	「しまう」 
\\	金魚がもう死んじゃった。	きんぎょがもうしんじゃった。	
\\	「~てしまう」	
\\	もう帰っちゃっていい?	もうかえちゃっていい?	
\\	「~てしまう」	
\\	みんな、どっか行っちゃったよ。	みんな、どっかいっちゃったよ。	
\\	「~てしまう」	
\\	そろそろ遅くなっちゃうよ。	そろそろおそくなっちゃうよ。	
\\	「~てしまう」	
\\	また遅刻しちまったよ。	またちこくしちまったよ。	
\\	「~てしまう」	
\\	ごめん、ついお前を呼んじまった。	ごめん、ついおまえをよんじまった。	
\\	「~てしまう」	
\\	宿題をやってしまいなさい。	しゅくだいをやってしまいなさい。	
\\	「しまう」	
\\	徹夜して、宿題することはある。	てつやして、しゅくだいすることはある。	
\\	「こと」 
\\	一人で行くことはありません。	ひとりでいくことはありません。	
\\	「こと」 
\\	パリに行ったことはありますか。	バリにいったことはありますか。	
\\	「こと」 
\\	お寿司を食べたことがある。	おすしをたべたことがある。	
\\	「こと」 
\\	日本の映画を観たことないの?	にほんのえいがをみたことないの?	
\\	「こと」 
\\	ヨーロッパに行ったことがあったらいいな。	ヨーロッパにいったことがあったらいいな。	
\\	「こと」 
\\	そういうのを見たことがなかった。	そういうのをみたことがなかった。	
\\	「こと」 
\\	一度行ったこともないんです。	いちどいったこともないんです。	
\\	「こと」 
\\	早くきて。映画は、今ちょうどいいところだよ。	はやくきて。えいがは、いまちょうどいいところだよ。	
\\	「ところ」 
\\	彼は、優しいところもあるよ。	かれは、やさしいところもあるよ。	
\\	「ところ」 
\\	今は授業が終ったところです。	いまはじゅぎょうがおわったところです。	
\\	「ところ」 
\\	これから行くところでした。	これからいくところでした。	
\\	「ところ」 
\\	どうしてこなかったの?/授業があったの。	どうしてこなっかったの?/じゅぎょうがあったの。	
\\	「もの」 
\\	どうしてこなかったの?/授業があったもの。	どうしてこなっかったの?/じゅぎょうがあったもの。	
\\	「もの」 
\\	どうしてこなかったの?/授業があったもん。	どうしてこなっかったの?/じゅぎょうがあったもん。	
\\	「もの」 
\\	スミスさんは食堂に行ったかもしれません。	スミスさんはしょくどうにいったかもしれません。	
\\	「かもしれない」 
\\	雨で試合は中止になるかもしれないね。	あめでしあいはちゅうしになるかもしれないね。	
\\	「かもしれない」 
\\	この映画は一回見たことあるかも!	このえいがはいっかいみたことあるかも!	
\\	「かもしれない」 
\\	あそこが代々木公園かもしれない。	あそこがよよぎこうえんかもしれない。	
\\	「かもしれない」 
\\	もう逃げられないかもしれんぞ。	もうにげられないかもしれんぞ。	
\\	「かもしれない」 
\\	明日も雨でしょう。	あしたもあめでしょう。	
\\	「でしょう」 
\\	あなたは、学生さんでしょうか。	あなたは、がくせいさんでしょうか。	
\\	「でしょう」 
\\	これからどこへ行くんでしょうか?	これからどこへいくんでしょうか?	
\\	「でしょう」 
\\	休ませていただけますでしょうか。	やすませていただけますでしょうか。	
\\	「でしょう」 
\\	あっ!遅刻しちゃう!/だから、時間がないって言ったでしょう!	あっ!ちこくしちゃう!/だから、じかんがないっていったでしょう!	
\\	「でしょう」 
\\	「だろう」 
\\	これから食べに行くんでしょ。/だったら?	これからたべにいくんでしょ。/だったら?	
\\	「でしょう」 
\\	「だろう」 
\\	掃除、手伝ってくれるでしょう。/え?そうなの?	そうじ、てつだってくれるでしょう。/え?そうなの?	
\\	「でしょう」 
\\	「だろう」 
\\	アリスはどこだ?/もう寝ているだろう。	アリスはどこだ?もうねているだろう。	
\\	「でしょう」 
\\	「だろう」 
\\	もう家に帰るんだろう。/そうよ。	もう{うち
\\	いえ}にかえるんだろう。/そうよ。	
\\	「でしょう」 
\\	「だろう」 
\\	りんごだけ。	りんごだけ。	
\\	「だけ」	
\\	これとそれだけ。	これとそれだけ。	
\\	「だけ」	
\\	それだけは、食べないでください。	それだけは、たべないでください。	
\\	「だけ」	
\\	この歌だけを歌わなかった。	このうただけをうたわなかった。	
\\	「だけ」	
\\	その人だけが好きだったんだ。	そのひとだけがすきだったんだ。	
\\	「だけ」	
\\	この販売機だけでは、500円玉が使えない。	このはんばいきだけは、500えんだまがつかえない。	
\\	「だけ」	
\\	小林さんからだけには、返事が来なかった。	こばやしさんからだけには、へんじがこなかった。	
\\	「だけ」	
\\	準備が終わったから、これからは食べるだけだ。	じゅんびがおわったから、これからはたべるだけだ。	
\\	「だけ」	
\\	ここに名前を書くだけでいいですか?	ここになまえをかくだけでいいですか?	
\\	「だけ」	
\\	この乗車券は発売当日のみ有効です。	このじょうしゃけんははつばいとうじつのみゆうこうです。	
\\	「のみ」 
\\	「だけ」	
\\	アンケート対象は大学生のみです。	アンケートたいしょうはだいがくのみです。	
\\	「のみ」 
\\	「だけ」	
\\	これしかない。	これしかない。	
\\	「しか」	
\\	これだけ見る。	これだけみる。	
\\	「しか」	
\\	これだけ見ない。	これだけみない。	
\\	「しか」	
\\	これしか見ない。	これしかみない。	
\\	「しか」	
\\	今日は忙しくて、朝ご飯しか食べられなかった。	きょうはいそがしくて、あさごはんしかたべられなかった。	
\\	「しか」	
\\	全部買うの?/ううん、これだけ。	ぜんぶかうの?/ううん、これだけ。	
\\	「しか」	
\\	全部買うの?/ううん、これしか買わない。	ぜんぶかうの?/ううん、これしかかわない。	
\\	「しか」	
\\	アリスからしか何ももらってない。	アリスからしかなにももらってない。	
\\	「しか」	
\\	これから頑張るしかない!	これからがんばるしかない!	
\\	「しか」	
\\	こうなったら、逃げるしかない。	こうなったら、にげるしかない。	
\\	「しか」	
\\	もう腐っているから、捨てるしかないよ。	もうくさっているから、すてるしかないよ。	
\\	「しか」	
\\	これは買うっきゃない!	これはかうっきゃない!	
\\	「しか」- 「っきゃ」, 
\\	「しか」	
\\	こうなったら、もうやるっきゃない!	こうなったら、もうやるっきゃない!	
\\	「しか」- 「っきゃ」, 
\\	「しか」	
\\	何だよ!おばさんばっかりじゃないか?	なんだよ!おばさんばっかりじゃないか?	
\\	「だけ」 
\\	「ばかり」	
\\	いやだ。おばさんばっかり。	いやだ。おばさんばっかり。	
\\	「だけ」 
\\	「ばかり」	
\\	崇君は漫画ばっかり読んでてさ。かっこ悪い。	たかしくんはまんがばっかりよんでてさ。かっこわるい。	
\\	「だけ」 
\\	「ばかり」	
\\	彼は麻雀ばかりです。	かれはマージャンばかりです。	
\\	「だけ」 
\\	「ばかり」	
\\	直美ちゃんと遊ぶばっかりでしょう!	なおみちゃんとあそぶばっかりでしょう!	
\\	「だけ」 
\\	「ばかり」	
\\	最近は仕事ばっかだよ。	さいきんはしごとばっかだよ。	
\\	「だけ」 
\\	「ばかり」	
\\	佐藤さんは料理が上手で、また食べ過ぎました。	さとうさんはりょうりがじょうずで、またたべすぎました。	
\\	「すぎる」	
\\	お酒を飲みすぎないように気をつけてね。	おさけをのみすぎないようにきをつけてね。	
\\	「すぎる」	
\\	大きすぎるからトランクに入らないぞ。	おおきすぎるからトランクに入らないぞ。	
\\	「すぎる」	
\\	静かすぎる。罠かもしれないよ。	しずかすぎる。わなかもしれないよ。	
\\	「すぎる」	
\\	時間が足りなさすぎて、何もできなかった。	じかんがたりなさすぎて、なにもできなかった。	
\\	「すぎる」	
\\	彼には、彼女がもったいなさすぎるよ。	かれには、かのじょがもったいなさすぎるよ。	
\\	「すぎる」	
\\	昨晩のこと、全然覚えてないな。/それは飲みすぎだよ。	さくばんのこと、ぜんぜんおぼえてないな。/せれはのみすぎだよ。	
\\	「すぎる」	
\\	昨日、電話三回もしたよ!	きのう、でんわさんかいもしたよ!	
\\	「も」 
\\	試験のために三時間も勉強した。	しけんのためにさんじかんもべんきょうした。	
\\	「も」 
\\	今年、十キロも太っちゃった!	ことし、じゅっキロもふとちゃった!	
\\	「も」 
\\	今日の天気はそれほど寒くない。	きょうのてんきはそれほどさむくない。	
\\	「ほど」 
\\	寝る時間がないほど忙しい。	ねるじかんがないほどいそがしい。	
\\	「ほど」 
\\	韓国料理は食べれば食べるほど、おいしくなる。	かんこくりょうりはたべればたべるほど、おいしくなる。	
\\	「ほど」 
\\	歩いたら歩くほど、迷ってしまった。	あるいたらあるくほど、まよってしまった。	
\\	「ほど」 
\\	勉強をすればするほど、頭がよくなるよ。	べんきょうをすればするほど、あたまがよくなるよ。	
\\	「ほど」 
\\	は、ハードディスクの容量が大きければ大きいほどもっとたくさんの曲が保存できます。	
\\	は、ハードディスクのようりょうがおおきければおおきいほどもっとたくさんのきょくがほぞんできます。	
\\	「ほど」 
\\	航空券は安ければ安いほどいいとは限らない。	こうくうけんはやすければやすいほどいいとはかぎらない。	
\\	「ほど」 
\\	文章は、短ければ短いほど、簡単なら簡単なほどよいです。	ぶんしょうは、みじかければみじかいほど、かんたんならかんたんなほどよいです。	
\\	「ほど」 
\\	このビルの高さは何ですか?	このビルのたかさはなんですか?	
\\	「~さ」 
\\	犬の聴覚の敏感さを人間と比べると、はるかに上だ。	いぬのちょうかくのびんかんさをにんげんとくらべると、はるかにうえだ。	
\\	「~さ」 
\\	ここには、誰もいないようだ。	ここには、だれもいないようだ。	
\\	よう (様)	
\\	映画を観たようです。	えいがをみたようです。	
\\	よう (様)	
\\	学生のようだ。	がくせいのようだ。	
\\	よう (様)	
\\	ここは静かなようだ。	ここはしずかなようだ。	
\\	よう (様)	
\\	あの人を見たような気がした。	あのひとをみたようなきがした。	
\\	よう (様)	
\\	彼は学生のような雰囲気ですね。	かれはがくせいのようなふんいきですね。	
\\	よう (様)	
\\	ちょっと怒ったように聞こえた。	ちょっとおこったようにきこえた。	
\\	よう (様)	
\\	何も起こらなかったように言った。	なにもおこらなかったようにいった。	
\\	よう (様)	
\\	もう売り切れみたい。	もううりきれみたい。	
\\	「みたい」 
\\	制服を着ている姿をみると、学生みたいです。	せいふくをきているすがたをみると、がくせいみたいです。	
\\	「みたい」 
\\	このピザはお好み焼きみたいじゃない?	このピザはおこのみやきみたいじゃない?	
\\	「みたい」 
\\	喫茶店に行くみたいだった。	きっさてんにいくみたいだった。	
\\	「みたい」 
\\	秘密を教えてくれるみたいじゃなかった?	ひみつをおしえてくれるみたいじゃなかった?	
\\	「みたい」 
\\	もう売り切れのようだ。	もううりきれのようだ。	
\\	「みたい」 
\\	このピザはお好み焼きのように見える。	このピザはおこのみやきのようにみえる。	
\\	「みたい」 
\\	バランスが崩れて、一瞬倒れそうだった。	バランスがくずれて、いっしゅんたおれそうだった。	
\\	「~そう」- 
\\	この辺りにありそうだけどな。	このあたりにありそうだけどな。	
\\	「~そう」- 
\\	この漬物はおいしそう!	このつけものはおいしそう!	
\\	「~そう」- 
\\	「い」 
\\	これも結構よさそうだけど、やっぱり高いよね。	これもけっこうよさそうだけど、やっぱりたかいよね。	
\\	「~そう」- 
\\	「い」 
\\	お前なら、金髪の女が好きそうだな。	おまえなら、きんぱつのおんながすきそうだな。	
\\	「~そう」- 
\\	「い」 
\\	もう10時になったから、来なさそうだね。	もう10じになったから、こなさそうだね。	
\\	「~そう」 
\\	「い」 
\\	「さ」.	
\\	これはただの試合じゃなさそうだ。	これはただのしあいじゃなさそうだ。	
\\	「~そう」 
\\	「い」 
\\	「さ」.	
\\	その人は学生でしょう。	そのひとはがくせいでしょう。	
\\	「~そう」 
\\	その人は学生だろう。	そのひとはがくせいだろう。	
\\	「~そう」 
\\	この犬はかわいそう。	このいぬはかわいそう。	
\\	「~そう」 
\\	この犬はかわいい。	このいぬ犬はかわいい。	
\\	「~そう」 
\\	明日、雨が降るそうだ。	あした、あめがふるそうだ。	
\\	「~そうだ」	
\\	毎日会いに行ったそうです。	まいにちあいにいったそうです。	
\\	「~そうだ」	
\\	彼は、高校生だそうです。	かれは、こうこうせいだそうです。	
\\	「~そうだ」	
\\	今日、田中さんはこないの?/だそうです。	きょう、たなかさんはこないの?/だそうです。	
\\	「~そうだ」	
\\	今日、田中さんはこないの?/こないらしい。	きょう、たなかさんはこないの?/こないらしい。	
\\	「~らしい」	
\\	あの人は何なの?/美由紀さんの友達らしいですよ。	あのひとはなんなの?/みゆきさんのともだちらしいですよ。	
\\	「~らしい」	
\\	あの子は子供らしくない。	あのこはこどもらしくない。	
\\	「~らしい」	
\\	大人らしくするつもりだったのに、大騒ぎしてしまった。	おとならしくするつもりだったのに、おおさわぎしてしまった。	
\\	「~らしい」	
\\	あの人はちょっと韓国人っぽいよね。	あのひとはちょっとかんこくじんっぽいよね。	
\\	「っぽい」: 
\\	みんなで、もう全部食べてしまったっぽいよ。	みんなで、もうぜんぶたべてしまったっぽいよ。	
\\	「っぽい」: 
\\	恭子は全然女っぽくないね。	きょうこはぜんぜんおんなっぽくないね。	
\\	「っぽい」: 
\\	ご飯の方がおいしい。	ごはんのほうがおいしい。	
\\	方 
\\	よる 
\\	「方」 
\\	鈴木さんの方が若い。	すずきさんのほうがわかい。	
\\	方 
\\	よる 
\\	「方」 
\\	学生じゃない方がいいよ。	がくせいじゃないほうがいいよ。	
\\	方 
\\	よる 
\\	「方」 
\\	赤ちゃんは、静かな方が好き。	あかちゃんは、しずかなほうがすき。	
\\	方 
\\	よる 
\\	「方」 
\\	ゆっくり食べた方が健康にいいよ。	ゆっくりたべたほうがけんこうにいいよ。	
\\	方 
\\	よる 
\\	「方」 
\\	こちらから行った方が早かった。	こちらからいったほうがはやかった。	
\\	方 
\\	よる 
\\	「方」 
\\	マトリックス・レボリューションを観ない方がいいよ。	マトリックス・レボリューションをみないほうがいいよ。	
\\	方 
\\	よる 
\\	「方」 
\\	そんなに飲まなかった方がよかった。	そんなにのまなかったほうがよかった。	
\\	方 
\\	よる 
\\	「方」 
\\	花より団子。	はなよりだんご。	
\\	方 
\\	よる 
\\	「より」 
\\	ご飯の方が、パンよりおいしい。	ごはんのほうが、パンよりおいしい。	
\\	方 
\\	よる 
\\	「より」 
\\	キムさんより鈴木さんの方が若い。	キムさんよりすずきさんのほうがわかい。	
\\	方 
\\	よる 
\\	「より」 
\\	毎日仕事に行くのが嫌だ。/仕事がないよりましだよ。	まいにちしごとにいくのがいやだ。/しごとがないよりましだよ。	
\\	方 
\\	よる 
\\	「より」 
\\	ゆっくり食べた方が早く食べるよりいい。	ゆっくりたべたほうがはやくたべるよりいい。	
\\	方 
\\	よる 
\\	「より」 
\\	商品の品質を何より大切にしています。	しょうひんのひんしつをなによりたいせつにしています。	
\\	方 
\\	よる 
\\	「より」 
\\	「より」 
\\	この仕事は誰よりも早くできます。	このしごとはだれよりもはやくできます。	
\\	方 
\\	よる 
\\	「より」 
\\	「より」 
\\	新宿の行き方は分かりますか。	しんじゅくのいきかたはわかりますか。	
\\	方 
\\	よる 
\\	「方」 
\\	そういう食べ方は体によくないよ。	そういうたべかたはからだによくないよ。	
\\	方 
\\	よる 
\\	「方」 
\\	漢字の書き方を教えてくれますか?	かんじのかきかたをおしえてくれますか?	
\\	方 
\\	よる 
\\	「方」 
\\	パソコンの使い方は、みんな知っているでしょう。	パソコンのつかいかたは、みんなしっているでしょう。	
\\	方 
\\	よる 
\\	「方」 
\\	人によって話が違う。	ひとによってはなしがちがう。	
\\	方 
\\	よる 
\\	「によって」 
\\	季節によって果物はおいしくなったり、まずくなったりする。	きせつによってくだものはおいしくなったり、まずくなったりする。	
\\	方 
\\	よる 
\\	「によって」 
\\	今日は飲みに行こうか?/それは、裕子によるね。	きょうはのみにいこうか?/それは、ゆうこによるね。	
\\	方 
\\	よる 
\\	「によって」 
\\	天気予報によると、今日は雨だそうだ。	てんきよほうによると、きょうはあめだそうだ。	
\\	方 
\\	よる 
\\	「によると」	
\\	友達の話によると、朋子はやっとボーイフレンドを見つけたらしい。	ともだちのはなしによると、ともこはやっとボーイフレンドをみつけたらしい。	
\\	方 
\\	よる 
\\	「によると」	
\\	この字は読みにくい 。	このじはよみにくい。	
\\	「~やすい、~にくい」 
\\	カクテルはビールより飲みやすい。	カクテルはビールよりのみやすい。	
\\	「~やすい、~にくい」 
\\	部屋が暗かったので、見にくかった。	へやがくらかったので、みにくかった。	
\\	「~やすい、~にくい」 
\\	あの肉は食べにくい。	あのにくはたべにくい。	
\\	「~やすい、~にくい」 
\\	あの肉を食べるのは難しい。	あのにくをたべるのはむずかしい。	
\\	「~やすい、~にくい」 
\\	彼との忘れがたい思い出を大切にしている。	かれとのわすれがたいおもいでをたいせつにしている。	
\\	「~にくい」 
\\	「~がたい」 
\\	「~づらい」	
\\	とても信じがたい話だが、本当に起こったらしい。	とてもしんじがたいはなしだが、ほんとうにおこったらしい。	
\\	「~にくい」 
\\	「~がたい」 
\\	「~づらい」	
\\	日本語は読みづらいな。	にほんごはよみづらいな。	
\\	「~にくい」 
\\	「~がたい」 
\\	「~づらい」	
\\	待ち合わせは、分かりづらい場所にしないでね。	まちあわせは、わかりづらいばしょにしないでね。	
\\	「~にくい」 
\\	「~がたい」 
\\	「~づらい」	
\\	何も食べないで寝ました。	なにもたべないでねました。	
\\	歯を磨かないで、学校に行っちゃいました。	はをみがかないで、がっこうにいっちゃいました。	
\\	宿題をしないで、授業に行くのは、やめた方がいいよ。	しゅくだいをしないで、じゅぎょうにいくのは、やめたほうがいいよ。	
\\	先生と相談しないで、この授業を取ることは出来ない。	せんせいとそうだんしないで、このじゅぎょうをとることはできない。	
\\	彼は何も言わず、帰ってしまった。	かれはなにもいわず、かえってしまった。	
\\	何も食べずにそんなにお酒を飲むと当然酔っ払いますよ。	なにもたべずにそんなにおさけをのむととうぜんよっぱらいますよ。	
\\	勉強せずに東大に入れると思わないな。	べんきょうせずにとうだいにはいれるとおもわないな。	
\\	すまん。	すまん。	
\\	「ん」	
\\	韓国人と結婚しなくてはならん!	かんこくじんとけっこんしなくてはならん!	
\\	「ん」	
\\	そんなことはさせん!	そんなことはさせん!	
\\	「ん」	
\\	皆、今日行くって、知らんかったよ。	みんな、きょういくって、しらんかったよ。	
\\	「ん」	
\\	韓国人と結婚してはならぬ!	かんこくじんとけっこんしてはならぬ!	
\\	「ぬ」	
\\	模擬試験に何回も失敗して、実際に受けてみたら思わぬ結果が出た。	もぎしけんになんかいもしっぱいして、じっさいにうけてみたらおもわぬけっかがでた。	
\\	「ぬ」	
\\	いくら英語を勉強しても、うまくならないの。/つまり、語学には、能力がないという訳か。/失礼ね。	いくらえいごをべんきょうしても、うまくならないの。/つまり、ごがくには、のうりょくがないというわけか。しつれいね。	
\\	「わけ」	
\\	中国語が読めるわけがない。	ちゅうごくごがよめるわけがない。	
\\	「わけ」	
\\	広子の家に行ったことある?/あるわけないでしょう。	広子のいえにいったことある?/あるわけないでしょう。	
\\	「わけ」	
\\	微積分は分かる?/分かるわけないよ!	ひせきぶんはわかる?/わかるわけないよ。	
\\	「わけ」	
\\	ここの試験に合格するのはわけない。	ここのしあいにごうかくするのはわけない。	
\\	「わけ」	
\\	今度は負けるわけにはいかない。	こんどはまけるわけにはいかない。	
\\	「わけ」	
\\	ここまできて、あきらめるわけにはいかない。	ここまできて、あきらめるわけにはいかない。	
\\	「わけ」	
\\	明日に行くとする。	あしたにいくとする。	
\\	「とする」	
\\	今から行くとしたら、9時に着くと思います。	いまからいくとしたら、くじにつくとおもいます。	
\\	「とする」	
\\	観客として参加させてもらった。	かんきゃくとしてさんかさせてもらった。	
\\	「とする」	
\\	被害者としては、非常に幸いだった。	ひがいしゃとしては、ひじょうにさいわいだった。	
\\	「とする」	
\\	朝ご飯を食べたとしても、もう昼だからお腹が空いたでしょう。	あさごはんをたべたとしても、もうひるだからおなかがすいたでしょう。	
\\	「とする」	
\\	すみません、今食べたばかりなので、お腹がいっぱいです。	すみません、いまたべたばかりなので、おなかがいっぱいです。	
\\	「~ばかり」	
\\	10キロを走ったばかりで、凄く疲れた。	じゅっキロをはしったばかりで、すごくつかれた。	
\\	「~ばかり」	
\\	今、家に帰ったばかりです。	いま、{いえ;うち}にかえったばかりです。	
\\	「~ばかり」	
\\	昼ご飯を食べたばっかなのに、もうお腹が空いた。	ひるごはんをたべたばっかなのに、もうおなかがすいた。	
\\	「~ばかり」	
\\	まさか、今起きたばっかなの?	まさか、いまおきたばっかなの?	
\\	「~ばかり」	
\\	窓を開けたとたんに、猫が跳んでいった。	まどをあけたととんに、ねこがとんでいった。	
\\	「とたん」 
\\	テレビを観ながら、宿題をする。	テレビをみながら、しゅくだいをする。	
\\	「ながら」 
\\	音楽を聴きながら、学校へ歩くのが好き。	おんがくをききながら、がっこうへあるくのがすき。	
\\	「ながら」 
\\	相手に何も言わないながら、自分の気持ちをわかってほしいのは単なるわがままだと思わない?	あいてになにもいわないながら、じぶんのきもちをわかってほしいのはたんなるわがままだとおもわない?	
\\	「ながら」 
\\	ポップコーンを食べながら、映画を観る。	ポップコーンをたべながら、えいがをみる。	
\\	「ながら」 
\\	ポップコーンを食べながら、映画を観た。	ポップコーンをたべながら、えいがをみた。	
\\	「ながら」 
\\	口笛をしながら、手紙を書いていた。	くちぶえをしながら、てがみをかいていた。	
\\	「ながら」 
\\	仕事がいっぱい入って、残念ながら、今日は行けなくなりました。	しごとがいっぱいはいって、ざんねんながら、きょうはいけなくなりました。	
\\	「ながら」 
\\	貧乏ながらも、高級なバッグを買っちゃったよ。	びんぼうながらも、こうきゅうなバッグをかっちゃったよ。	
\\	「ながら」 
\\	彼は、初心者ながらも、実力はプロと同じだ。	かれは、しょしんしゃながらも、じつりょくはプロとおなじだ。	
\\	「ながら」 
\\	ゲームにはまっちゃって、最近パソコンを使いまくっているよ。	ゲームにはまっちゃって、さいきんパソコンをつかいまくっているよ。	
\\	「まくる」	
\\	アメリカにいた時はコーラを飲みまくっていた。	アメリカにいたときはコーラをのみまくっていた。	
\\	「まくる」	
\\	このままで宜しいですか?	このままでよろしいですか?	
\\	「まま」 
\\	半分しか食べてないままで捨てちゃダメ!	はんぶんしかたべてないままですてちゃダメ!	
\\	「まま」 
\\	今日だけは悲しいままでいさせてほしい。	きょうだけはかなしいままでいさせてほしい。	
\\	「まま」 
\\	その格好のままでクラブに入れないよ。	そのかっこうのままでクラブにはいれないよ。	
\\	「まま」 
\\	テレビを付けっぱなしにしなければ眠れない人は、結構いる。	テレビをつけっぱなしにしなければねむれないひとは、けっこういる。	
\\	「っぱなし」 
\\	窓が開けっ放しだったので、蚊がいっぱい入った。	まどがあけっばなしだったので、かがいっぱいはいった。	
\\	「っぱなし」 
\\	吾輩は猫である。	わがはいはねこである。	
\\	「である」 
\\	混合物とは、2種類以上の純物質が混じりあっている物質である。	こんごうぶつとは、2しゅるいいじょうのじゅんぶっしつがまじりあっているぶっしつである。	
\\	「である」 
\\	国土交通省は2年後に利用率を70%まで引き上げる考えで、買い替え時に利用する気になるかどうかがカギになりそうだ。	こくどこうつうしょうは2ねんごにりようりつを70%までひきあげるかんがえで、かいかえときにりようするきになるかどうかがカギになりそうだ。	
\\	「である」 
\\	これは不公平ではないでしょうか。	これはふこうへいではないでしょうか。	
\\	「である」	
\\	言語は簡単にマスターできることではない。	げんごはかんたんにマスターできることではない。	
\\	「である」	
\\	花火は、火薬と金属の粉末を混ぜたものに火を付け、燃焼時の火花を楽しむためのもの。	はなびは、かやくときんぞくのふんまつをまぜたものにひをづけ、ねんしょうときのひばなをたのしむためのもの。	
\\	企業内の顧客データを利用し、彼の行方を調べることが出来た。	きぎょうないのこきゃくデータをりようし、かのなめがたをしらべることができた。	
\\	封筒には写真が数枚入っており、手紙が添えられていた。	ふうとうにはしゃしんがすうまいいっっており、てがみがそえられていた。	
\\	この旅館は、様々な新しい設備が備えており、とても快適だった。	このりょかんは、さまざまなあたらしいせつびがそなえており、とてもかいてきだった。	
\\	彼は漫画マニアだから、これらをもう全部読んだはずだよ。	かれはまんがマニアだから、これらをもうぜんぶよんだはずだよ。	
\\	「はず」 
\\	この料理はおいしいはずだったが、焦げちゃって、まずくなった。	このりょうりはおいしいはずだったが、こげちゃって、まずくなった。	
\\	「はず」 
\\	色々予定してあるから、今年は楽しいクリスマスのはず。	いろいろよていしてあるから、こんねんはたのしいクリスマスのはず。	
\\	「はず」 
\\	そう簡単に直せるはずがないよ。	そうかんたんになおせるはずがないよ。	
\\	「はず」 
\\	打合せは毎週2時から始まるはずじゃないですか?	うちあわせはまいしゅう2ときからはじまるはずじゃないですか?	
\\	「はず」 
\\	何かを買う前に本当に必要かどうかをよく考えるべきだ。	なにかをかうまえにほんとうにひつようかどうかをよくかんがえるべきだ。	
\\	「べき」 
\\	例え国のためであっても、国民を騙すべきではないと思う。	たとえくにのためであっても、こくみんをだますべきではないとおもう。	
\\	「べき」 
\\	預金者が大手銀行を相手取って訴訟を起こすケースも出ており、金融庁は被害者の救済を優先させて、金融機関に犯罪防止対策の強化を促すべきだと判断。	よきんものがおおてぎんこうをあいてとってそしょうをおこすケースもでており、きんゆうちょうはひがいしゃのきゅうさいをゆうせんさせて、きんゆうきかんにはんざいぼうしたいさくのきょうかをうながすべきだとはんだん。	
\\	「べき」 
\\	早く帰るべき。	はやくかえるべき。	
\\	「べく」 
\\	早く帰るべく、準備をし始めた。	はやくかえるべく、じゅんびをしはじめた。	
\\	「べく」 
\\	試験に合格すべく、皆一生懸命に勉強している。	しけんにごうかくすべく、みないっしょうけんめいにべんきょうしている。	
\\	「べく」 
\\	今後もお客様との対話の窓口として、より充実していくべく努力してまいります。	こんごもおきゃくさまとのたいわのまどぐちとして、よりじゅうじつしていくべくどりょくしてまいります。	
\\	「べく」 
\\	ゴミ捨てるべからず。	ゴミすてるべからず。	
\\	「べからず」 
\\	安全措置を忘れるべからず。	あんぜんそちをわすれるべからず。	
\\	「べからず」 
\\	宿題が多すぎて、トイレに行く時間さえなかった。	しゅくだいがおおすぎて、トイレにいくじかんさえなかった。	
\\	「(で)さえ」 
\\	お金さえあれば、何でも出来るよ。	おきんさえあれば、なんでもできるよ。	
\\	「(で)さえ」 
\\	お弁当を買うお金さえなかった。	おべんとうをかうおきんさえなかった。	
\\	「(で)さえ」 
\\	私でさえ出来れば、あんたには楽ちんでしょう。	わたしでさえできれば、あんたにはらくちんでしょう。	
\\	「(で)さえ」 
\\	ビタミンを食べさえすれば、健康が保証されますよ。	ビタミンをたべさえすれば、けんこうがほしょうされますよ。	
\\	「(で)さえ」 
\\	自分の過ちを認めさえしなければ、問題は解決しないよ。	じぶんのあやまちをみとめさえしなければ、もんだいはかいけつしないよ。	
\\	「(で)さえ」 
\\	この天才の私ですらわからなかった。	このてんさいのわたしですらわからなかった。	
\\	「(で)すら」 
\\	「(で)さえ」 
\\	私は緊張しすぎて、ちらっと見ることすら出来ませんでした。	わたしはきんちょうしすぎて、ちらっとみることすらできませんでした。	
\\	「(で)すら」 
\\	「(で)さえ」 
\\	「人」の漢字すら知らない生徒は、いないでしょ!	「にん」のかんじすらしらないせいとは、いないでしょ!	
\\	「人」 
\\	「(で)すら」 
\\	「(で)さえ」 
\\	漢字はおろか、ひらがなさえ読めないよ!	かんじはおろか、ひらがなさえよめないよ!	
\\	「おろか」 
\\	結婚はおろか、2ヶ月付き合って、結局別れてしまった。	けっこんはおろか、2がつつきあって、けっきょくわかれてしまった。	
\\	「おろか」 
\\	大学はおろか、高校すら卒業しなかった。	だいがくはおろか、こうこうすらそつぎょうしなかった。	
\\	「おろか」 
\\	早くきてよ!何を恥ずかしがっているの?	はやくきてよ!なにをはずかしがっているの?	
\\	「~がる」	
\\	彼女は朝早く起こされるのを嫌がるタイプです。	かのじょはあさはやくおこされるのをいやがるタイプです。	
\\	「~がる」	
\\	うちの子供はプールに入るのを理由もなく怖がる。	うちのこどもはプールにいるのをりゆうもなくこわがる。	
\\	「~がる」	
\\	家に帰ったら、すぐパソコンを使いたがる。	いえにかえったら、すぐパソコンをつかいたがる。	
\\	「~がる」	
\\	みんなイタリアに行きたがってるんだけど、私の予算で行けるかどうかはとても怪しい。	みんなイタリアにいきたがってるんだけど、わたしのよさんでいけるかどうかはとてもあやしい。	
\\	「~がる」	
\\	妻はルイヴィトンのバッグを欲しがっているんだけど、そんなもん、買えるわけないでしょう!	つまはルイヴィトンのバッグをほしがっているんだけど、そんなもん、かえるわけないでしょう!	
\\	「~がる」	
\\	私は寒がり屋だから、ミネソタで暮らすのは辛かった。	わたしはさむがりやだから、ミネソタでくらすのはつらかった。	
\\	「~がる」	
\\	ボールは爆発せんばかりに、膨らんでいた。	ボールはばくはつせんばかりに、ふくらんでいた。	
\\	「ばかり」 
\\	「あんたと関係ない」と言わんばかりに彼女は彼を無視していた。	「あんたとかんけいない」といわんばかりにかのじょはかれをむししていた。	
\\	「ばかり」 
\\	昨日のケンカで何も言わなかったばかりに、平気な顔をしている。	きのうのケンカでなにもいわなかったばかりに、へいきなかおをしている。	
\\	「ばかり」 
\\	紅葉が始まり、すっかり秋めいた空気になってきた。	こうようがはじまり、すっかりあきめいたくうきになってきた。	
\\	「めく」 
\\	そんな謎めいた顔をされても、うまく説明できないよ。	そんななぞめいたかおをされても、うまくせつめいできないよ。	
\\	「めく」 
\\	いつも皮肉めいた言い方をしたら、みんなを嫌がらせるよ。	いつもひにくめいたいいかたをしたら、みんなをいやがらせるよ。	
\\	「めく」 
\\	このテレビがこれ以上壊れたら、新しいのを買わざるを得ないな。	このテレビがこれいじょうこわれたら、あたらしいのをかわざるをえないな。	
\\	「~ざるを得ない」	
\\	ずっと我慢してきたが、この状態だと歯医者さんに行かざるを得ない。	ずっとがまんしてきたが、このじょうたいだとはいしゃさんにいかざるをえない。	
\\	「~ざるを得ない」	
\\	上司の話を聞くと、どうしても海外に出張をせざるを得ないようです。	じょうしのはなしをきくと、どうしてもかいがいにしゅっちょうをせざるをえないようです。	
\\	「~ざるを得ない」	
\\	やむを得ない事由により手続が遅れた場合、必ずご連絡下さい。	やむをえないじゆうによりてつづきがおくれたばあい、かならずごれんらくください。	
\\	「やむを得ない」	
\\	この仕事は厳しいかもしれませんが、最近の不景気では新しい仕事が見つからないのでやむを得ない状態です。	このしごとはきびしいかもしれませんが、さいきんのふけいきではあたらしいしごとがみつからないのでやむをえないじょうたいです。	
\\	「やむを得ない」	
\\	この場ではちょっと決めかねますので、また別途会議を設けましょう。	このばではちょっときめかねますので、またべっとかいぎをもうけましょう。	
\\	「~かねる」	
\\	このままでは、個人情報が漏洩しかねないので、速やかに対応をお願い致します。	このままでは、こじんじょうほうがろうえいしかねないので、すみやかにたいおうをおねがいいたします。	
\\	「~かねる」	
\\	確定申告は忘れがちな手続のひとつだ。	かくていしんこくはわすれがちなてつづきのひとつだ。	
\\	「~がち」	
\\	留守がちなご家庭には、犬よりも、猫の方がおすすめです。	るすがちなごかていには、いぬよりも、ねこのほうがおすすめです。	
\\	「~がち」	
\\	父親は病気がちで、みんなが心配している。	ちちおやはびょうきがちで、みんながしんぱいしている。	
\\	「~がち」	
\\	テレビを見ながら、寝ちゃダメよ!	テレビをみながら、ねちゃダメよ!	
\\	「~つつ」	
\\	二日酔いで痛む頭を押さえつつ、トイレに入った。	ふつかよいでいたむあたまをおさえつつ、トイレにいっった。	
\\	「~つつ」	
\\	体によくないと思いつつ、最近は全然運動してない。	からだによくないとおもいつつ、さいきんはぜんぜんうんどうしてない。	
\\	「~つつ」	
\\	電気製品の発展につれて、ハードディスクの容量はますます大きくなりつつある。	でんきせいひんのはってんにつれて、ハードディスクのようりょうはますますおおきくなりつつある。	
\\	「~つつ」	
\\	今の日本では、終身雇用や年功序列という雇用慣行が崩れつつある。	いまのにっぽんでは、しゅうしんこようやねんこうじょれつというこようかんこうがくずれつつある。	
\\	「~つつ」	
\\	多くの大学生は、締切日ぎりぎりまで、宿題をやらないきらいがある。	おおくのだいがくせいは、しめきりびぎりぎりまで、しゅくだいをやらないきらいがある。	
\\	「きらいがある」	
\\	コーディングが好きな開発者は、ちゃんとしたドキュメント作成と十分なテストを怠るきらいがある。	コーディングがすきなかいはつしゃは、ちゃんとしたドキュメントさくせいとじゅうぶんなテストをおこたるきらいがある。	
\\	「きらいがある」	
\\	相手は剣の達人だ。そう簡単には勝てまい。	あいてはつるぎのたつじんだ。そうかんたんにはかてまい。	
\\	そんな無茶な手段は認めますまい!	そんなむちゃなしゅだんはみとめますまい!	
\\	その時までは決して彼に会うまいと心に決めていた。	そのときまではけっしてかれにあうまいとこころにきめていた。	
\\	あの人は、二度と嘘をつくまいと誓ったのです。	あのにんは、にどとうそをつくまいとちかったのです。	
\\	明日に行くのをやめよう。	あしたにいくのをやめよう。	
\\	肉を食べないようにしている。	にくをたべないようにしている。	
\\	あいつが大学に入ろうが入るまいが、俺とは関係ないよ。	あいつがだいがくにいろうがいるまいが、おれとはかんけいないよ。	
\\	時間があろうがあるまいが、間に合わせるしかない。	じかんがあろうがあるまいが、まにあわせるしかない。	
\\	最近のウィルスは強力で、プログラムを実行しようがしまいが、ページを見るだけで感染するらしい。	さいきんのウィルスはきょうりょくで、プログラムをじっこうしようがしまいが、ページをみるだけでかんせんするらしい。	
\\	今後50年、人間が直面するであろう問題に正面から向き合って、自ら解決をはかりつつ、そのノウハウが次の産業となるシナリオを考えたい。	こんご50ねん、にんげんがちょくめんするであろうもんだいにしょうめんからむきあって、みずからかいけつをはかりつつ、そのノウハウがつぎのさんぎょうとなるシナリオをかんがえたい。	
\\	「であろう」 
\\	もちろん、生徒数減少の現在、学科の新設は困難であろうが、職業科の統廃合や科内コースの改編などで時代に合わせた変革が求められているはずである。	もちろん、せいとすうげんしょうのげんざい、がっかのしんせつはこんなんであろうが、しょくぎょうかのとうはいごうやかないコースのかいへんなどでじだいにあわせたへんかくがもとめられているはずである。	
\\	「であろう」 
\\	どんな商品でもネットで販売するだけで売上が伸びるというものではなかろう。	どんなしょうひんでもネットではんばいするだけでうりあがのびるというものではなかろう。	
\\	「かろう」 
\\	「い」 
\\	運動を始めるのが早かろうが遅かろうが、健康にいいというのは変わりません。	うんどうをはじめるのがはやかろうがおそかろうが、けんこうにいいというのはかわりません。	
\\	「かろう」 
\\	「い」 
\\	休日であろうが、なかろうが、この仕事では関係ないみたい。	きゅうじつであろうが、なかろうが、このしごとではかんけいないみたい。	
\\	「かろう」 
\\	「い」 
\\	このドキュメントは間違えだらけで、全然役に立たない。	このドキュメントはまちがえだらけで、ぜんぜんやくにたたない。	
\\	「だらけ」 
\\	携帯を2年間使ってたら、傷だらけになった。	けいたいを2ねんかんつかってたら、きずだらけになった。	
\\	「だらけ」 
\\	この埃だれけのテレビをちゃんと拭いてくれない?	このほこりだれけのテレビをちゃんとぬぐいてくれない?	
\\	「だらけ」 
\\	彼は油まみれになりながら、車の修理に頑張りました。	かれはあぶらまみれになりながら、くるまのしゅうりにがんばりました。	
\\	「まみれ」 
\\	たった1キロを走っただけで、汗まみれになるのは情けない。	たった1キロをはしっただけで、あせまみれになるのはなさけない。	
\\	「まみれ」 
\\	白ずくめ団体は去年ニューズになっていた。	しろずくめだんたいはきょねんニューズになっていた。	
\\	「ずくめ」 
\\	女の子と共通の話題ができて、自分の体も健康になる。いいことずくめですよ。	おんなのこときょうつうのわだいができて、じぶんのからだもけんこうになる。いいことずくめですよ。	
\\	「ずくめ」 
\\	彼女は、教授の姿を見るが早いか、教室から逃げ出した。	かのじょは、きょうじゅのすがたをみるがはやいか、きょうしつからにげだした。	
\\	「が早いか」 
\\	「食べてみよう」と言うが早いか、口の中に放り込んだ。	「たべてみよう」というがはやいか、くちのなかにほうりこんだ。	
\\	「が早いか」 
\\	「食べてみよう」と言ったが早いか、口の中に放り込んだ。	「たべてみよう」といっったがはやいか、くちのなかにほうりこんだ。	
\\	「が早いか」 
\\	私の顔を見るや、何か言おうとした。	わたしのかおをみるや、なにかいおうとした。	
\\	「や/や否や」 
\\	搭乗のアナウンスが聞こえるや否や、みんながゲートの方へ走り出した。	とうじょうのアナウンスがきこえるやひや、みんながゲートのほうへはしりだした。	
\\	「や/や否や」 
\\	子供が掃除するそばから散らかすから、もうあきらめたくなった。	こどもがそうじするそばからちらかすから、もうあきらめたくなった。	
\\	「そばから」 
\\	教科書を読んだそばから忘れてしまうので勉強ができない。	きょうかしょをよんだそばからわすれてしまうのでべんきょうができない。	
\\	「そばから」 
\\	昼間だから絶対込んでいると思いきや、一人もいなかった。	ひるまだからぜったいこんでいるとおもいきや、ひとりもいなかった。	
\\	「思いきや」 
\\	このレストランは安いと思いきや、会計は5千円以上だった!	このレストランはやすいとおもいきや、かいけいは5せんえんいじょうだった!	
\\	「思いきや」 
\\	散歩がてら、タバコを買いに行きました。	さんぽがてら、タバコをかいにいきました。	
\\	「~がてら」 
\\	博物館を見がてらに、お土産を買うつもりです。	はくぶつかんをけんがてらに、おみやげをかうつもりです。	
\\	「~がてら」 
\\	事情を2時間かけて説明をしたあげく、納得してもらえなかった。	じじょうを2じかんかけてせつめいをしたあげく、なっとくしてもらえなかった。	
\\	「~あげく(挙句)」 
\\	先生と相談のあげく、退学することにした。	せんせいとそうだんのあげく、たいがくすることにした。	
\\	「~あげく(挙句)」 
\end{CJK}
\end{document}