\documentclass[8pt]{extreport} 
\usepackage{hyperref}
\usepackage{CJKutf8}
\begin{document}
\begin{CJK}{UTF8}{min}
\\	嗚呼	
\\	表示感叹或者震惊。(物事に深く感じたり驚いたりした気持ちを直接表す語。) 
\\	表示对别人的呼唤。(呼びかけに用いる語。) 
\\	表示同意、肯定。(同意したり肯定したりする応答の語。)	ああ		【感叹词】
\\	相	
\\	互相。(双方が互いに同じ関係に立つと捕らえられることをあらわす。) 
\\	表示积极性。(動作・状態の積極性を表す。)	あい		【名词】
\\	相変わらず	
\\	照旧,仍旧。	あいかわらず		【副词】
\\	愛想	
\\	亲切,和蔼。 
\\	招待,款待。 
\\	顾客付的钱。	あいそ		【名词】
\\	相対	
\\	面对面。 
\\	并非独立存在,而是依靠和其他事物的关系而存在。	あいたい		【名词】
\\	間柄	
\\	交际。 
\\	联系。	あいだがら		【名词】
\\	愛憎	
\\	爱憎。(愛することと憎むこと。)	あいにく		【名词】
\\	合間	
\\	缝隙,缝。 
\\	空儿,空闲时间,间歇,余暇。事物间断的短暂时间。	あいま		【名词】
\\	曖昧	
\\	含糊,可疑。	あいまい		【形容动词/ナ形容词】
\\	敢えて	
\\	敢于,硬要。 
\\	(后接否定式)未必,并不。	あえて		【副词】
\\	仰ぐ	
\\	仰,瞻。 
\\	尊为,推为。 
\\	仰赖,仰仗,依靠。 
\\	请,求。 
\\	请求批准。 
\\	饮,服。	あおぐ		【他动词・五段/一类】
\\	垢	
\\	污垢,油泥。(老化した皮膚に汗・脂・ほこりが交って出来る汚れ。) 
\\	水锈,水碱。(みずあか。)	あか		【名词】
\\	銅	
\\	铜。	あかがね		【名词】
\\	証	
\\	证据,证明。同証し。	あかし		【名词】
\\	赤字	
\\	入不敷出,亏空,赤字。 
\\	红字,校正的字。	あかじ		【名词】
\\	明かす	
\\	说出,揭露,揭破。 
\\	过夜。	あかす		【他动词・五段/一类】
\\	赤ちゃん	
\\	小宝宝,小宝贝,小娃娃,婴儿。	あかちゃん		【名词】
\\	明白	
\\	明白,明显。 
\\	同:明瞭	あからさま		【名・形容动词/ナ形容词】
\\	赤らむ	
\\	变红。	あからむ		【自动词・五段/一类】
\\	明るい	
\\	明亮。 
\\	明朗,快活,光明。 
\\	熟悉,精通。 
\\	颜色鲜艳的。 
\\	有希望的。 
\\	公正的,廉洁的。	あかるい		【形容词/イ形容词】
\\	上がり	
\\	上,往上。上升,指往上运动。上涨,涨价。进步,长进。(位置・地位・価格などが上がること。単独で用いることは少ない。) 
\\	收入,收获;卖项,销售额。利益。(収穫。利益。収入。売り上げ。) 
\\	完成,做的成绩。(物が出来上がること。できぐあい。仕上がり。) 
\\	成果。完成效果。(出来栄え。) 
\\	为茧移入蚕蔟,作茧。(カイコの上蔟。) 
\\	终了。结束。(物事が一段落すること。) 
\\	刚泡的茶。(寿司相关用语)指为了清除口中的味道最后上的茶。(〔鮨屋・料理屋などで〕「あがり花」の略。お茶。) 
\\	赢,满格。双六游戏中,棋子进入最后棋盘格,亦指其棋盘格。(双六(すごろく)で、駒が最後の区画に入ること。また、その区画。) 
\\	赢牌,和牌。玩扑克或麻将时,手里的牌凑齐而获胜。(トランプや麻雀で、手がそろって勝つこと。) 
\\	(接尾)…出身,当过…。表示以前曾经是什么职业、身份或状态。(前に、その職業・身分・状態だったことを表す。多く、好ましくないという気持ちが伴う。) 
\\	(接尾)刚…。表示持续的状态结束不久之意。(続いていた状態が終わって間がない意を表す。)	あがり		【接头/接尾】
\\	上がる	
\\	上;登。 
\\	(地位、程度、质量、价格、价值等)提高;上升。 
\\	发出(声音、响声)。 
\\	(从浴池、田地、船舶等中)出来。 
\\	(源自进入高一阶段之意)上学。 
\\	(事情)完成,结束。 
\\	产生;获得;课获得利益、效果、收入等。 
\\	(用钱)解决。 
\\	怯场;沉不住气。 
\\	(给神、佛)供上。 
\\	去;访问。 
\\	追溯,回溯。 
\\	(因皇宫坐落于北面)(在日本京都)向北行;上殿;进宫。	あがる		【自动词・五段/一类】
\\	商人	
\\	商人,做买卖的,做生意的人(あきびと的转化词)。	あきうど		【名词】
\\	空間	
\\	空间。(ものがなく、空いているところ。) 
\\	空隙,空的地方,空子。(上下、四方の広がり。)	あきま		【名词】
\\	諦め	
\\	死心,想开,达观,断念头,打消……的念头。	あきらめ		【名词】
\\	呆れる	
\\	(因意外事)吃惊,呆若木鸡。	あきれる		【自动词・一段/二类】
\\	悪	
\\	不好,坏。(悪い、不快である、害になるなどの意を表す) 
\\	过度,过分,过头,过火。(程度が過ぎている意を表す)	あく		【接头/接尾】
\\	灰	
\\	灰となる。/①烧成灰。 ②火葬。	あく		【惯用句】
\\	悪日	
\\	凶日;倒霉的日子	あくび		【名词】
\\	明くる	
\\	下,次,翌,第二。	あくる		【连体词】
\\	憧れ	
\\	憧憬;向往。(あこがれること。)	あこがれ		【名词】
\\	顎	
\\	下巴,下颚,钓鱼钩的倒须。	あご		【名词】
\\	麻	
\\	麻(大麻、苎麻、黄麻、亚麻等的总称)。大麻。 
\\	麻纤维。麻纱。麻布,夏布。	あさ		【名词】
\\	明後日	
\\	后天(あすの次の日。みょうごにち)。	あさって		【名词】
\\	朝寝坊	
\\	睡懒觉(的人)。	あさねぼう		【名・自动词・サ变/三类】
\\	浅ましい	
\\	卑鄙,下流。 
\\	可耻,可叹。 
\\	可怜,凄惨。	あさましい		【形容词/イ形容词】
\\	字	
\\	闾”大体上指与“字”相对应的“村庄”。并不是地号单位。 
\\	「闾」はほぼ「字」に相当する「むらざと」のこと。地番の単位ではない。	あざ		【名词】
\\	欺く	
\\	欺骗。 
\\	赛过,超过。	あざむく		【他动词・五段/一类】
\\	鮮やか	
\\	鲜明,色、形美而清晰。 
\\	巧妙,精湛,优美。	あざやか		【形容动词/ナ形容词】
\\	あざ笑う	
\\	嘲笑,看不起对方而嘲讽的笑。	あざわらう		【他动词・五段/一类】
\\	味わい	
\\	风味。味道。 
\\	趣味,有趣。 
\\	食物的味道。风味。 
\\	事物的乐趣。趣味。妙趣。风趣。	あじわい		【名词】
\\	東	
\\	特指关东地区。(特に関東地方をさしていった。) 
\\	东,吾妻。中世时从京都来看,指镰仓或镰仓幕府的词语。(中世に、京都からみて、鎌倉または鎌倉幕府をさしていった語。) 
\\	东,吾妻。江户时代从京都(关西)来看,指江户的词语。(江戸時代、上方からみて、江戸をさしていった語。) 
\\	同[吾妻]	あずま		【名词】
\\	焦る	
\\	焦躁,急躁,着急。	あせる		【自他・五段/一类】
\\	彼処	
\\	那;那里;那种情况;那种局面 
\\	彼処;彼所 
\\	あそこ		【代词】
\\	値	
\\	价格;东西的价值。	あたい		【名词】
\\	値する	
\\	值,价钱相当于……。 
\\	值得,有……价值。 
\\	同:価する。	あたいする		【自动词・サ变/三类】
\\	私	
\\	我。	あたし		【代词】
\\	当たり	
\\	打中,射中,命中。 
\\	中彩,中签,中奖。 
\\	称心如意,成功。 
\\	着落,头绪。 
\\	待人,对待。 
\\	(碰上对手的)气势。力量。 
\\	(围棋)叫吃。 
\\	(棒球)击球。 
\\	(钓鱼时)鱼上钩的感觉。	あたり		【名词】
\\	当たり前	
\\	当然,用不着说的。 
\\	普通,一般,平常。	あたりまえ		【形容动词/ナ形容词】
\\	他人	
\\	别人。(自分以外の人。) 
\\	(没有血亲关系的)外人。(血筋のつながらない人。) 
\\	局外者。(そのことに関係のない人。) 
\\	《相关惯用语》	あだびと		【名词】
\\	彼方此方	
\\	这边那边,到处,各处。指示各种各样的场所、方向。	あちこち		【代词】
\\	彼方	
\\	远方。(あちらの方。)	あちら		【名词】
\\	悪化	
\\	恶化;变坏。	あっか		【自动词・サ变/三类】
\\	呆気ない	
\\	太简单的;没意思的;没劲的;不尽兴的;不过瘾的。	あっけない		【形容词/イ形容词】
\\	悪口	
\\	谩骂,骂,毁谤。	あっこう		【名词】
\\	圧迫	
\\	压迫。用强力压住。 
\\	使用武力或权利等进行压制。	あっぱく		【名・他动词・サ变/三类】
\\	扱い	
\\	使用;操纵。 
\\	(当……)对待,看待。	あつかい		【名词】
\\	集まる	
\\	聚集。	あつまる		【自动词・五段/一类】
\\	誂える	
\\	定做,定购。 
\\	定,订做。	あつらえる		【他动词・一段/二类】
\\	圧力	
\\	压力。 
\\	制服力,压力。	あつりょく		【名词】
\\	当て	
\\	目的,目标。 
\\	期待,指望;依赖,依靠。 
\\	垫敷物,垫布。	あて		【名词】
\\	宛	
\\	碰上;期待;依赖。	あて		【名词】
\\	当て字	
\\	【豆知识】 
\\	别字”“白字”是指错别字。比如把“包子”写成“饱子”,“发脾气”写成“发皮气”。
\\	别字”“白字”は読み書きを誤っている字。“包子”(パオズ)を“饱子”と、“发脾气”(かんしゃくをおこす)を“发皮气”と書いたり、“绽”を“(定)”と読んだりすること。) 
\\	同:宛て字	あてじ		【名词】
\\	当てはまる	
\\	完全合适,适合,适应。	あてはまる		【自动词・五段/一类】
\\	当てはめる	
\\	适用,应用。	あてはめる		【他动词・一段/二类】
\\	宛てる	
\\	发,给;充当。	あてる		【他动词・一段/二类】
\\	跡継ぎ	
\\	后任,继承,接班人,后嗣。	あとつぎ		【名词】
\\	後回し	
\\	推迟,往后推,缓办。	あとまわし		【名词】
\\	貴女	
\\	尊贵女人。 
\\	您。书信等中对女性表示敬意的用语。	あなた		【名词】
\\	彼の	
\\	彼,那个,他的,对方的。(あの。)	あの		【接续词】
\\	溢れる	
\\	溢出,漾出;充满。 
\\	挤满。 
\\	充满。	あふれる		【自动词・一段/二类】
\\	油絵	
\\	油画。	あぶらえ		【名词】
\\	炙る	
\\	烤,晒。 
\\	烤干,烘。 
\\	烘火取暖。	あぶる		【他动词・五段/一类】
\\	甘える	
\\	撒娇,与人亲近熟识以后,任性地讲话。 
\\	半心半意的,兴趣不大的。	あまえる		【自动词・一段/二类】
\\	甘口	
\\	(酒等)甜的,带甜味的。 
\\	喜甜族,喜好吃甜东西的人。 
\\	甜言蜜语。 
\\	呆子,傻子。	あまくち		【名・形容动词/ナ形容词】
\\	雨具	
\\	雨具。为防止被雨淋湿而使用的衣物或用具。如雨衣、雨鞋、雨伞之类。(雨の日に、雨を防ぐため使う衣類や道具。レーン-コート・雨靴・雨傘の類。)	あまぐ		【名词】
\\	天	
\\	天,天空。 
\\	天国,天堂。 
\\	天道,天理。 
\\	天命,运命。 
\\	天,苍天,上帝。 
\\	(字画、货物等的)上部;(书的)天头。	あまつ		【名词】
\\	網	
\\	拖网。由袋网和连结袋网的两袖网构成的渔网。网口左右分开,由船拖带在海底行进。(底引き網。) 
\\	网,铁丝网;法网。	あみ		【名词】
\\	天地	
\\	天和地。 
\\	世界,宇宙。 
\\	上下面。	あめつち		【名词】
\\	操る	
\\	操,掌握。 
\\	〔暗地里〕操纵,控制。 
\\	〔船、机械等〕开动,驾驶,驾驭。耍,耍弄。善于操作。	あやつる		【他动词・五段/一类】
\\	危ぶむ	
\\	担心,认为危险。 
\\	怀疑,不相信。 
\\	感觉没把握,担心。	あやぶむ		【他动词・五段/一类】
\\	過ち	
\\	过错,过失,罪过。 
\\	错误,错儿,失败。	あやまち		【名词】
\\	誤る	
\\	误,搞错,做错,错误。做不恰当的判断、选择、评价、行动等。(不適切な判断・選択・評価・行動などをする。間違える。やりそこなう。) 
\\	错误,犯错误。做出不好的事情,做出越轨的行为。(よくないことをする。道にはずれた行為をする。) 
\\	同:謬る 
\\	误,搞错,做错,错误。做不恰当的判断、选择、评价、行动等。(不適切な判断・選択・評価・行動などをする。間違える。やりそこなう。) 
\\	错误,犯错误。做出不好的事情,做出越轨的行为。(よくないことをする。道にはずれた行為をする。) 
\\	同:謬る	あやまる		【自他・五段/一类】
\\	歩み	
\\	步行,走。行走。〔歩くこと。歩行。〕 
\\	脚步,步伐,步调。〔足なみ。歩く調子。歩調。〕 
\\	进行(情况),进展。发展变化,进程,演变。事物的进展情况,变迁。〔物事の進行、または進歩発達の過程。推移。沿革。歴史。〕 
\\	间距。螺钉的牙与牙之间或沟槽与沟槽之间的距离。(等間隔で並んでいる物のおのおのの中心間の距離。例えば、ねじの山と山との距離など。)	あゆみ		【名词】
\\	歩む	
\\	行,走。 
\\	前进,进展,经历过。	あゆむ		【自动词・五段/一类】
\\	予め	
\\	预先;先。	あらかじめ		【副词】
\\	荒らす	
\\	使荒芜;骚扰;抢劫。	あらす		【他动词・五段/一类】
\\	粗筋	
\\	概略,概要。	あらすじ		【名词】
\\	争い	
\\	争,争论,争吵,纠纷,不和。 
\\	竞争。	あらそい		【名词】
\\	改まる	
\\	改,变,更新。(新しくなる。) 
\\	改善,革新。(改善される。) 
\\	故作庄重,一本正经,郑重其事。(儀式ばる。)	あらたまる		【自动词・五段/一类】
\\	荒っぽい	
\\	粗暴的,粗野的,粗糙的。 
\\	粗糙,粗枝大叶。	あらっぽい		【形容词/イ形容词】
\\	有難う	
\\	向对方表示谢意,谢谢。(相手に感謝の気持ちを表わすことば)。	ありがとう		【感叹词】
\\	有様	
\\	样子,光景,情况,存在状态。事物的状态。	ありさま		【名词】
\\	有りのまま	
\\	据实的,如实的,实事求是的。	ありのまま		【名・形容动词/ナ形容词】
\\	或る	
\\	某;有。(事物・人・時・場所などを漠然とさしていう語。また,それらをはっきりさせずにいう時にも用いる。)	ある		【连体词】
\\	或いは	
\\	或者。有时,一方面。 
\\	或许,有可能。	あるいは		【副词】
\\	彼此	
\\	彼此。(いろいろな物や事柄をさす。いろいろと。あれやこれやと。)	あれこれ		【副词】
\\	合わす	
\\	合,合并。	あわす		【他动词・五段/一类】
\\	合わせ	
\\	把鱼勾住。 
\\	菜肴。 
\\	比赛。 
\\	调合。	あわせ		【名词】
\\	慌ただしい	
\\	慌张,匆忙,不稳。(忙しい。せわしない。物の動きや周囲の状況が激しく変化する。)	あわただしい		【形容词/イ形容词】
\\	慌てる	
\\	惊慌,着慌,慌张。 
\\	急急忙忙,急忙。	あわてる		【自动词・一段/二类】
\\	暗殺	
\\	暗杀,行刺。	あんさつ		【名词】
\\	暗算	
\\	心算,不用手或(计算)工具,而在头脑里计算,头脑计算的结果。	あんざん		【名・他动词・サ变/三类】
\\	暗示	
\\	暗示。知觉、观念、意图、行为等通过语言或其他象征,不诉诸于理性地被传达和接受的现象。亦指为此而作为刺激的事物。	あんじ		【名・他动词・サ变/三类】
\\	案じる	
\\	担心,挂念。 
\\	思考,想。	あんじる		【他动词・一段/二类】
\\	安静	
\\	安静。	あんせい		【名・形容动词/ナ形容词】
\\	案の定	
\\	果然,果如所料。	あんのじょう		【副词】
\\	余り	
\\	过分,过度。	あんまり		【形容动词/ナ形容词】
\\	良い	
\\	好,良好,优良,优秀,优异;善良;聪明;要好。(すぐれている。) 
\\	贵重,高贵,珍贵;高尚,高雅。(貴重な。) 
\\	美丽;漂亮。(美しい。) 
\\	晴朗,爽朗;舒适,舒服,舒畅。(快い。) 
\\	好,佳,吉,喜庆。(めでたい) 
\\	合适,正好,好;恰当,适当;恰好,凑巧。(適当だ。妥当だ。折よく。) 
\\	好。(注意を促す。) 
\\	对;行,可以;够了。(かまわない。正しい。大丈夫だ。十分だ。) 
\\	(对……)有效;有好处;适合于……。(効果がある。利点がある。適する。) 
\\	(感情)和睦,亲密。(親しい。)	いい		【形容词/イ形容词】
\\	否	
\\	不,不对。否定对方的话时用的词语。(問いに対して答えが否定的であることを表す語。いいえ。) 
\\	不,岂止『書』。否定自己先前说的话时使用。(自分が先に言った言葉を打ち消す時に使う語。) 
\\	哦,那么,是啊。(否定も肯定もしないで、話の合間に何となく発する語。) 
\\	(副)或者,或许。也许,不然的话。(あるいは。やれ。)	いいえ		【感叹词】
\\	いい加減	
\\	适可而止,不过分。 
\\	不彻底,不疼不痒。 
\\	敷衍,搪塞,支吾,含糊,马马虎虎。 
\\	靠不住,胡乱。	いいかげん		【形容动词/ナ形容词】
\\	言い訳	
\\	分辩,辩解,辩白。 
\\	道歉,赔不是。	いいわけ		【名・自动词・サ变/三类】
\\	家出	
\\	出奔,逃出家门。离家出走。 
\\	外出。 
\\	僧侣出家。	いえで		【自动词・サ变/三类】
\\	家主	
\\	房东,房主,户主。(やぬし。)	いえぬし		【名词】
\\	如何	
\\	如何。(どんなに。) 
\\	为什么。(どうして…か。) 
\\	如何。(「ーあらむ」の形で危惧もしくは非難の意を表す。) 
\\	怎么样。(どれほど…ことか。) 
\\	怎么样。(「どうですか」と呼びかけすすめる語。)	いかが		【副词】
\\	生かす	
\\	使其发挥作用;使其继续生存。	いかす		【他动词・五段/一类】
\\	雷	
\\	雷。 
\\	像雷一样(的东西)。	いかずち		【名词】
\\	如何に	
\\	怎么也,无论怎么也。 
\\	如何,怎么样。 
\\	为什么,何故。 
\\	多么,非常。	いかに		【副词】
\\	如何にも	
\\	【连语】 
\\	的的确确,完全。(相手の考えを強く肯定する意を表す。なるほど。確かに。) 
\\	实在,真(的)。(程度・状態のはなはだしいことを表す。どう考えても。全く。実に。) 
\\	果然,诚然,的确。(まさしく。さも。)	いかにも		【副词】
\\	怒り	
\\	怒り心頭に発する。/非常愤怒。怒上心头。	いかり		【惯用句】
\\	怒る	
\\	生气,发怒,发火。	いかる		【自动词・五段/一类】
\\	歪む	
\\	歪,歪扭。 
\\	心术不正,乖僻。	いがむ		【自动词・五段/一类】
\\	粋	
\\	通晓人情世故。 
\\	精粹,精华。 
\\	纯粹;潇洒,风流,俊俏。	いき		【名・形容动词/ナ形容词】
\\	域外	
\\	域外,区域以外,境外。(ある一定の範囲の外。区域の外。)	いきがい		【名词】
\\	意気込む	
\\	振奋,鼓起干劲,干劲十足,兴致勃勃,热火朝天,精神百倍。	いきごむ		【自动词・五段/一类】
\\	経緯	
\\	经线和纬线。 
\\	经度和纬度。南北和东西。 
\\	事情的经过,原委。	いきさつ		【名词】
\\	行き違い	
\\	走岔开,错过。 
\\	弄错,差错,产生龃龉,失和,不睦。	いきちがい		【名词】
\\	行き成り	
\\	突然;冷不防;马上就	いきなり		【副词】
\\	異議	
\\	异议,不同的意见。(他人と異なる議論。異論。)	いぎ		【名词】
\\	軍	
\\	军队,军队的编制单位。(数個以上の軍団または師団をもって編制する大きな兵力の単位。) 
\\	军,队。(軍隊。軍勢。)	いくさ		【名词】
\\	戦	
\\	战斗,战争。	いくさ		【名词】
\\	育成	
\\	培养,培育,培训,扶植,扶育。	いくせい		【名・他动词・サ变/三类】
\\	幾多	
\\	许多,无数。	いくた		【副词】
\\	活ける	
\\	插;栽;使活下去 
\\	生ける;活ける 
\\	いける		【他动词・一段/二类】
\\	異見	
\\	异议,不同见解。	いけん		【名词】
\\	意向	
\\	意向,打算,意图。	いこう		【名词】
\\	移行	
\\	过渡,转移,移交。	いこう		【自动词・サ变/三类】
\\	碑	
\\	石碑。(事績を後世に伝えるため、文字などを刻んで建てる石。)	いしぶみ		【名词】
\\	衣装	
\\	服装。 
\\	〈劇〉剧装,戏装。 
\\	同:衣裳	いしょう		【名词】
\\	意地	
\\	心术,用心。(気立て。気性。) 
\\	固执,倔强,意气用事『成』,矫情。(自分の考えを通そうと思う気持ち。) 
\\	志气,气魄,要强心,逞强心,坚强的意志。(強い意志。) 
\\	〔欲〕嘴馋,贪食;贪婪。(物を貪ろうとする気持ち。特に、食べ物に対する執着。)	いじ		【名词】
\\	苛める	
\\	欺负;虐待;捉弄;折磨。(わざと苦痛を与えて、快感を味わう。限度を超えて、ひどい扱いをする。)	いじめる		【他动词・一段/二类】
\\	移住	
\\	移住;移居。	いじゅう		【名词】
\\	弄る	
\\	随便改动。 
\\	玩弄。 
\\	摆弄。拨弄,抚弄,玩赏。	いじる		【他动词・五段/一类】
\\	何れ	
\\	哪个,哪一方面。(これとかそれとか、はっきり定めず、または分からないままに、物事をさすのに使う語。どれ。どちら。) 
\\	反正,左右,早晚,横竖,归根到底,总,总归。(何にしても。どのみち。) 
\\	不久,最近,改日,过几天。(いつとは言えないが、近い将来。そのうちに。)	いずれ		【惯用句】
\\	異性	
\\	异性。(男女・雌雄の性が異なること。特に、男性から女性を、女性から男性をさしていう。) 
\\	〈化〉(同分)异构(现象)。(分子式あるいはそれに対応する化学式は同じであるが,構成する原子の立体的な配列その他が異なるため,物理的および化学的物質が異なる化学種が二つまたはそれ以上存在するとき,これらはたがいに異性体
\\	であるといい,またこの現象を異性という。 
\\	いせい		【名词】
\\	遺跡	
\\	故址,遗迹,古迹。保留有过去人类活动痕迹以及遗留建筑或遗物的地点。(過去の人間の営為の跡が残されている場所。遺構・遺物のある場所。考古学では住居址・墳墓・貝塚・城郭など、土地に固定して動かすことができないものをさす。) 
\\	(死者的)家业。(死者家业)继承人。(故人の残した領地・地位など。また、その相続人。) 
\\	同:遺蹟	いせき		【名词】
\\	依然	
\\	依然,仍然,仍旧,照旧。	いぜん		【副词・形容动词/ナ形容词】
\\	依存	
\\	发音也可为[いそん] 
\\	依存。(他のものをたよりとして存在すること。)	いそん		【自动词・サ变/三类】
\\	委託	
\\	委托,托付。 
\\	〈法〉委托。	いたく		【名・他动词・サ变/三类】
\\	悪戯	
\\	淘气,恶作剧。 
\\	玩笑,闹着玩儿;消遣。 
\\	摆弄,玩弄。 
\\	胡搞,乱搞,猥亵行为。	いたずら		【自动词・サ变/三类】
\\	頂	
\\	顶,上部。(物の一番高いところ。) 
\\	山巅,山顶。树尖。(山の一番高いこと。)	いただき		【名词】
\\	至って	
\\	极,极其,极为,甚为。	いたって		【副词】
\\	痛む	
\\	疼,疼痛。 
\\	苦恼,悲痛,痛苦,伤心。 
\\	破损,损坏。	いたむ		【自动词・五段/一类】
\\	痛める	
\\	使(肉体)疼痛;损伤。 
\\	使(精神、心灵)痛苦。 
\\	弄坏,损坏。	いためる		【他动词・一段/二类】
\\	炒める	
\\	【比较】 
\\	炒”是用少量的油在较短时间内用强火炒。比如说“炒蛋鸡蛋”“炒面”等。而“煎”是用少量的油把食材来回烧烤。比如说“煎鸡蛋”“煎饼”。“爆”是在比“炒”火力更强的条件下在短时间内炒。比如说“宫爆肉丁”和“酱爆鸡丁”等。 
\\	炒”は少量の油で短時間に強火で炒める。たとえば“炒鸡蛋”(いり卵)、“炒面”(焼きそば)。“煎”は少量の油で材料をかき回さずに炒り焼く。たとえば“煎鸡蛋”(目玉焼)、“煎饼”(クレープ。パンケーキのように薄く焼いた主食)。“爆”は“炒”よりも強火で短時間に炒める。たとえば“宫爆肉丁”や“酱爆鸡丁”。	いためる		【他动词・一段/二类】
\\	労る	
\\	生病。 
\\	保养。补养,养生。 
\\	安慰,慰劳。犒劳。 
\\	怜恤,照拂,照顾。关怀,怜悯,悯惜,悯恤。满怀同情心地和善对待力量弱小者、有困难的人、病人等。	いたわる		【自他・五段/一类】
\\	市	
\\	市;城市,都市。	いち		【名词】
\\	一々	
\\	一一,一个一个,逐一,逐个。(ひとつひとつ) 
\\	全部,一个个,一件件。(ことごとく) 
\\	一一,详细。ことこまかに
\\	いちいち		【名词・副词】
\\	一概に	
\\	(后接否定语)一概,笼统地,无区别地。	いちがいに		【副词】
\\	一見	
\\	过路客,新客人。 
\\	新客,(妓女)首次接待的客人。	いちげん		【名词】
\\	一言	
\\	一句话;三言两语。	いちげん		【名词】
\\	一定	
\\	一定,适度。 
\\	一定,固定。 
\\	规定,统一。	いちじょう		【名・自他・サ变/三类】
\\	著しい	
\\	明显的,显著的。 
\\	非常,太。	いちじるしい		【形容词/イ形容词】
\\	一同	
\\	全体,大家。	いちどう		【名词】
\\	一人	
\\	一人。	いちにん		【名词】
\\	一部	
\\	一本,一册,一部,一份;一套。 
\\	一部分(人)。	いちぶ		【名词】
\\	一部分	
\\	一部分。	いちぶぶん		【名词】
\\	一別	
\\	一别;分别	いちべつ		【他动词・サ变/三类】
\\	一面	
\\	一面。 
\\	另一面。 
\\	全体,满,一片。 
\\	第一版(面),头版。 
\\	一张;一面。	いちめん		【名词】
\\	一目	
\\	一目,一看。 
\\	一只眼。 
\\	一个子儿。	いちもく		【自动词・サ变/三类】
\\	一様	
\\	平常,普通。 
\\	一样,同样。	いちよう		【名・形容动词/ナ形容词】
\\	一律	
\\	一律,一个样子。 
\\	一律,一概,没有差别。	いちりつ		【名・形容动词/ナ形容词】
\\	一連	
\\	一列;一排(ひとつらね。いちれん。) 
\\	同:一行	いちれん		【名词】
\\	一括	
\\	一包在内,总括起来,一揽子。	いっかつ		【名・他动词・サ变/三类】
\\	一気	
\\	一口气。(一息の漢語的表現。)	いっき		【副词】
\\	一挙に	
\\	一举。	いっきょに		【副词】
\\	一切	
\\	下接表示否定的词带有强调的语气。全然不,完全不。(下に打ち消しの語を伴って、打ち消しを強調する気持ちを表す。全然。全く。例外なく。) 
\\	一切,全部,所有,一点儿不剩。(全部。すべて。残らず。)	いっさい		【名词・副词】
\\	一心	
\\	同心,齐心,一条心,一个心眼儿。 
\\	专心,一心一意。 
\\	(佛教语)一心,唯一绝对之心,作为万象根源之心。	いっしん		【名词】
\\	一帯	
\\	某周围全部。 
\\	一连串。	いったい		【名词・副词】
\\	一敗	
\\	一败,一负,输一次。(勝負に1回敗れること。)	いっぱい		【名词】
\\	一変	
\\	一变,完全改变,突然改变。	いっぺん		【名・自他・サ变/三类】
\\	何時	
\\	几点钟。	いつ		【名词】
\\	何時か	
\\	不知不觉,不知什么时候。(時がたつのに気がつかないさま。) 
\\	曾经,以前。(過去の不定の時を表す。いつぞや。以前。) 
\\	早晚,迟早,(总)有一天,改日。(未来の不定の時を表す。そのうちに。)	いつか		【副词】
\\	何時でも	
\\	无论什么时候,随时,经常,总是。(どんな時でも。常に。)	いつでも		【副词】
\\	何時の間にか	
\\	不知不觉	いつのまにか		【副词】
\\	何時までも	
\\	不管到什么时候,永远,始终。(いつの時までも。とこしえに。永久に。) 
\\	到底,毕竟,始终。(どこまでも。あくまでも。)	いつまでも		【副词】
\\	何時も	
\\	无论何时,经常。(常に。どんな時でも。)	いつも		【副词】
\\	意図	
\\	意图,企图,打算。	いと		【名・他动词・サ变/三类】
\\	営む	
\\	营,办,从事。 
\\	经营。 
\\	营造,建造。 
\\	做(佛事等)。	いとなむ		【他动词・五段/一类】
\\	暇	
\\	闲,空闲。(仕事や義務に拘束されず、自由にできる時間があるさま。) 
\\	同:閑	いとま		【形容动词/ナ形容词】
\\	異動	
\\	调动。	いどう		【名・自他・サ变/三类】
\\	挑む	
\\	找碴。挑衅。对抗。挑战。(戦いを仕掛ける。) 
\\	竞争。争持。(競い争う。) 
\\	挑逗。调情。(恋慕をしかける。)	いどむ		【自他・五段/一类】
\\	稲光	
\\	日本地名,位于鸟取县西伯郡大山町。	いなびかり		【名词】
\\	古	
\\	旧;旧东西;旧衣物(古くなったもの。使い古したもの)。	いにしえ		【名词】
\\	祈り	
\\	祈祷,祷告。(祈ること。祈祷。) 
\\	同:祷り	いのり		【名词】
\\	鼾	
\\	鼾声。(睡眠中、呼吸に伴って鼻や口からうるさい音を出すこと。また、その音。気道、特に軟口蓋(なんこうがい)の振動によって出る。)	いびき		【名词】
\\	今更	
\\	现在才,现在开始。(今になって、また。) 
\\	现在重新。(あらためて。) 
\\	事到如今,事已至此,已到这般地步。(いまとなっては。)	いまさら		【副词】
\\	未だ	
\\	<后多接否定语>未,尚未。	いまだ		【副词】
\\	移民	
\\	移民,侨民。以从事劳动为目的而移居国外,也指其人。	いみん		【自动词・サ变/三类】
\\	卑しい	
\\	低贱,下贱,卑微。 
\\	卑鄙,粗俗,下流,下作,不雅。 
\\	贪婪。 
\\	吝啬,小气。 
\\	寒碜,衣衫褴褛,粗糙,简陋。	いやしい		【形容词/イ形容词】
\\	意欲	
\\	意志,热情,积极性。(積極的に何かをしようと思う気持ち。)	いよく		【名词】
\\	入口	
\\	入口。进入的地方,进口。(入るところ。はいりぐち。いりくち。) 
\\	开端,起头,头绪。事物的开始,亦指事物的最初阶段。(物事の始め。また、物事の最初の段階。)	いりくち		【名词】
\\	衣料	
\\	衣料。做衣服的布料。(衣服、また、その材料である布地などの総称。) 
\\	衣服。(衣服)	いりょう		【名词】
\\	威力	
\\	威力。威势,威慑。压倒对方的强大力量,使人感到吃惊的高强的性能、力量。(相手を圧倒する強い力。非常にすばらしい性能・力。)	いりょく		【名词】
\\	入る	
\\	进,入,进入;在内,归入,有,含有,包括在其范围内;装入,容纳,放入,归入,容纳入某容量之中;得到,到手,收入,入主,变为自己所有;加入,进入,成为组织的一员;硬加入,挤入;出现、产生裂纹等;为感官所感知;精力充沛;进,入,到达某时期或某阶段;饮(酒);备好茶,茶已准备好。	いる		【自动词・五段/一类】
\\	衣類	
\\	衣服,衣裳。	いるい		【名词】
\\	色々	
\\	各种各样的。 
\\	种种、方方面面。	いろいろ		【名・形容动词/ナ形容词】
\\	異論	
\\	异议,不同意见。	いろん		【名词】
\\	所謂	
\\	所谓。世人常说的。所说的。(世にいわれている。いうところの。)	いわゆる		【连体词】
\\	印	
\\	记号,标记。(他と紛れないように見分けるための心覚えとするもの。) 
\\	证据。(あることを証明すること。) 
\\	信号。(合図。) 
\\	心意。(心持ちを表す。)	いん		【名词】
\\	員	
\\	人员,人数。(特に定められた人や物の数。) 
\\	成员,一员。(一定の任をもつ人。団体などの構成メンバー。)	いん		【名词】
\\	印鑑	
\\	图章,戳儿,戳子。 
\\	印鉴。	いんかん		【名词】
\\	陰気	
\\	忧闷,阴郁,郁闷,阴沉,阴暗。 
\\	阴气。万物衰败,将要消亡之气。(万物が衰え消滅しようとする気。)	いんき		【名词】
\\	隠居	
\\	退休;放弃户主权。 
\\	隐居,闲居。 
\\	(家里的)老人。	いんきょ		【自动词・サ变/三类】
\\	上下	
\\	上下。 
\\	上级和下级。 
\\	上院和下院。	うえした		【名词】
\\	浮かぶ	
\\	浮;漂;飘;漂浮;飘浮。 
\\	浮;泛;浮现。 
\\	浮出;露出。 
\\	想起;想出;涌上心头;浮现于脑海。	うかぶ		【自动词・五段/一类】
\\	受かる	
\\	考中,考上,及格。	うかる		【自动词・五段/一类】
\\	受け入れ	
\\	接纳,收容。 
\\	承认,答应。	うけいれ		【名词】
\\	受け入れる	
\\	接受,接收,收纳,收进。 
\\	接受,听从他人意见。 
\\	接纳,迎接。	うけいれる		【他动词・一段/二类】
\\	受け継ぐ	
\\	继承,承继。	うけつぐ		【他动词・五段/一类】
\\	受け付ける	
\\	受理,接受,采纳,听取。 
\\	容纳,采纳。对外来的事物表示出相应的反应。 
\\	接受,听从。答应他人的委托、申诉。	うけつける		【他动词・一段/二类】
\\	受け止める	
\\	接住。 
\\	挡住攻击;架住。 
\\	理解;接受意见、批评等,并予以处理解决。	うけとめる		【他动词・一段/二类】
\\	受け取り	
\\	指接受别人给自己的东西。收。领。领取。(他から自分の方にくるものをうけとること。) 
\\	作为收到的凭证的单据。收据。收条。(受けとったしるしの書きつけ。) 
\\	同:受取り、受取、請取、請取り	うけとり		【名词】
\\	受身	
\\	被动。受到来自他方的攻击而呈守势,亦指其状态。(ほかから攻撃されて守勢になること。また、その状態。受け太刀。) 
\\	被动。消极的态度,等待来自外界的推动的态度。(消極的な態度。ほかからの働きかけを待つ態度。) 
\\	(语法)被动态,被动式。(文法で、他者からの動作・作用を受けるものを主語として述べるもの。口語では助動詞「れる」「られる」、文語では「る」「らる」(古くは「ゆ」「らゆ」)を付けて言い表す。「殺される」「ほめられる」の類。なお、「雨に降られる」のように、自動詞に受け身の助動詞を付けた言い方(迷惑の受け身といわれる)もある。受動態。) 
\\	受身。柔道运动中,当被对方摔出时,为了防止受伤而采取的倒的访法。(柔道で、相手に投げられたとき、けがをしないように倒れる方法。)	うけみ		【名词】
\\	動き	
\\	心跳得厉害;心脏悸动。(ふだんより強い、心臓の鼓動。)	うごき		【自动词・サ变/三类】
\\	潮	
\\	潮,潮水。(しお。) 
\\	海水。(海の水。) 
\\	潮流,潮。(潮流。) 
\\	清鲜汤。(潮汁のこと。)	うしお		【名词】
\\	氏	
\\	摄氏
\\	。(〔考案者であるスウェーデンのセルシウスの中国音訳「摂爾思」による。) 
\\	同:摂氏	うじ		【名词】
\\	渦	
\\	漩涡	うず		【名词】
\\	埋まる	
\\	(被)埋上,埋着。 
\\	填满。 
\\	填补,弥补。	うずまる		【自动词・五段/一类】
\\	嘘つき	
\\	说谎(的人)。(爱)撒谎(的人)。 
\\	同:嘘吐き	うそつき		【名词】
\\	打ち合わせ	
\\	商量,磋商;碰头。	うちあわせ		【自动词・サ变/三类】
\\	打ち合わせる	
\\	使……相碰,互击,对打。 
\\	商量,商洽,碰头。	うちあわせる		【他动词・一段/二类】
\\	打ち切る	
\\	停止,截止,结束,中止。 
\\	下完棋。	うちきる		【他动词・五段/一类】
\\	打ち消し	
\\	消除;否认;否定。	うちけし		【名词】
\\	打ち込む	
\\	打进,钉进。砸入,攻入。打到里面去,打入深处。(たたいて中に入れる。) 
\\	射入,击入,打进。将球、子弹等射进对方阵地。((多く「撃ち込む」と書く)球・弾丸などを相手の陣に入れる。) 
\\	劈,猛刺,反复劈。在剑道中指向对手发起攻击。(刀できりかかる。剣道で,相手に打ちかかる。) 
\\	迷恋,埋头,全神贯注。热衷于某件事。(精神を集中する。夢中になる。) 
\\	刺中,说中。准确地捅到人的弱点。(人の弱みを的確に突く。急所を突く。) 
\\	打入。在围棋中,在对方的阵中置入自己的棋子。(囲碁で,相手の陣の中に,自分の石を置く。) 
\\	浇灌,灌注。将混凝土灌入所定的地方。(コンクリートを所定の場所に流し込む。) 
\\	熟练击球。在棒球运动中,充分进行击球训练。(野球やゴルフで,球を打つ練習を十分にする。) 
\\	投进,扔进。挥霍光。将钱全部赌光。(財産を使い果たす。)	うちこむ		【他动词・五段/一类】
\\	団扇	
\\	为纳凉或增强火势而制的扇风用具。 
\\	日本古时大将用指挥扇。 
\\	相扑裁判用指挥扇。 
\\	日本家徽名。图案为团扇、军扇或羽扇。	うちわ		【名词】
\\	内訳	
\\	细目,细项,详细内容,将金钱、物品按用途等分类成项。	うちわけ		【名词】
\\	訴え	
\\	诉讼,控告,控诉。 
\\	申诉,诉说,呼吁。	うったえ		【名词】
\\	鬱陶しい	
\\	郁闷的,阴郁的,阴暗的,阴沉的。(重苦しく陰気である。心が晴れ晴れしない。) 
\\	腻烦的,麻烦的,烦忧的。(煩わしい。妨げになって煩い。)	うっとうしい		【形容词/イ形容词】
\\	写し	
\\	抄本,本,摹本,誊本。 
\\	抄本,抄件,誊本,副本,抄写下来备用的文书。 
\\	临摹画,仿制品,仿照原品制作,亦指其制品。	うつし		【名词】
\\	空ろ	
\\	空洞,空虚。 
\\	发呆,空虚的样子。 
\\	同:虚ろ	うつろ		【形容动词/ナ形容词】
\\	器	
\\	容器,器皿;器具。 
\\	才干,人才。	うつわ		【名词】
\\	雨天	
\\	雨天。(雨の降る天候。あまぞら。雨の降る日。あめふり。)	うてん		【名词】
\\	腕前	
\\	能力,本事,才干。手艺。	うでまえ		【名词】
\\	饂飩	
\\	切面,面条。(小麦粉を原料にして作っためん類。)	うどん		【名词】
\\	促す	
\\	促使,促进。 
\\	催促,促使。	うながす		【他动词・五段/一类】
\\	唸る	
\\	呻吟,哼哼。 
\\	(兽类)吼,啸,嗥。 
\\	发呜呜声,轰鸣。 
\\	赞叹,叫好。 
\\	卖力吟唱,卖力哼唱。	うなる		【自动词・五段/一类】
\\	自惚れ	
\\	骄傲,自满,自负,自大。(うぬぼれること。自負。) 
\\	同:己惚れ	うぬぼれ		【名词】
\\	甘い	
\\	甜。 
\\	淡。 
\\	甜蜜。 
\\	宽;姑息;好说话。 
\\	藐视,小看,看得简单。 
\\	不锐利,不快,钝的。	うまい		【形容词/イ形容词】
\\	生まれつき	
\\	天性,秉性,生就的性格。	うまれつき		【名词】
\\	海路	
\\	海路。	うみじ		【名词】
\\	産む	
\\	分娩,生孩子。(胎児や卵を母体から外に出す。出産する。分娩する。) 
\\	新生,创作出。(物事を新たに生じる。新しく作り出す。) 
\\	产生。(ある事態・判断を生じさせる。) 
\\	同:生む	うむ		【他动词・五段/一类】
\\	埋め込む	
\\	埋入,塞入。	うめこむ		【他动词・五段/一类】
\\	梅干	
\\	咸梅干,腌的梅子。	うめぼし		【名词】
\\	末	
\\	末;底。(年・月の終わりの方。)	うら		【接头/接尾】
\\	裏返し	
\\	翻里作面,表里相反。(裏を返して表とすること。また、その状態。)	うらがえし		【名词】
\\	売り出し	
\\	开始出售。 
\\	甩卖,减价出售,贱卖。 
\\	出名,初露头角。	うりだし		【名词】
\\	売り出す	
\\	出售;甩卖;初露头角。	うりだす		【他动词・五段/一类】
\\	潤う	
\\	润,湿。因含水分而发潮,因适度潮湿而有生气。 
\\	宽绰起来。 
\\	得利,受惠。	うるおう		【自动词・五段/一类】
\\	五月蝿い	
\\	讨厌的;烦人的;爱唠叨的;吵得慌的 
\\	煩い;五月蝿い 
\\	うるさい		【名词】
\\	売れ行き	
\\	销路,销售(情况)。(商品等の売れ行き具合。売れ方。)	うれゆき		【名词】
\\	浮気	
\\	爱情不专一,乱搞男女关系,外遇。 
\\	见异思迁,心思不专。	うわき		【名词】
\\	上手	
\\	好,高明,擅长,善于,拿手,能手。 
\\	善于奉承,会说话。	うわて		【形容动词/ナ形容词】
\\	上回る	
\\	超过,越出。	うわまわる		【自动词・五段/一类】
\\	植わる	
\\	栽着。	うわる		【自动词・五段/一类】
\\	運営	
\\	办理,经营,管理。	うんえい		【名・他动词・サ变/三类】
\\	運送	
\\	运送,运输,搬运。	うんそう		【名・他动词・サ变/三类】
\\	運賃	
\\	运费。	うんちん		【名词】
\\	云々	
\\	等等。云云。使引用文章、词句的后部含糊或省略时使用的词语。	うんぬん		【名词】
\\	運搬	
\\	搬运,运输。	うんぱん		【名・他动词・サ变/三类】
\\	運命	
\\	命运,运气。	うんめい		【名词】
\\	運輸	
\\	运输,运送,搬运。	うんゆ		【名词】
\\	運用	
\\	运用,活用。	うんよう		【名・他动词・サ变/三类】
\\	会	
\\	会;会议;集会;会见。	え		【名词】
\\	重	
\\	重,层;严重。	え		【名・形容动词/ナ形容词】
\\	映写	
\\	放映。	えいしゃ		【名・他动词・サ变/三类】
\\	英字	
\\	英国文字,英文。(英語を書き表す文字。)	えいじ		【名词】
\\	衛生	
\\	卫生。	えいせい		【名词】
\\	映像	
\\	映像,影像 
\\	形象,印象	えいぞう		【名词】
\\	英雄	
\\	英雄。才智出众、勇武有力且能成就大业的人。(知恵・才能が優れ、気力・武力にも秀で、大事業を成し遂げた人。) 
\\	《相关惯用语》	えいゆう		【名词】
\\	液	
\\	液,液体。(水のように流動する物質。液体。汁。)	えき		【名词】
\\	役	
\\	役务(公役)。 
\\	任务(役目),工作;职务(職務)。 
\\	角色。	えき		【名词】
\\	閲覧	
\\	阅览。	えつらん		【名・他动词・サ变/三类】
\\	獲物	
\\	猎(获)物
\\	捕获(物)。(他人から奪い取って手に入れたもの。) 
\\	战利品,缴获物,收获。(戦いや勝負事に勝ってとった物。)	えもの		【名词】
\\	襟	
\\	折りえり/翻领	えり		【名词】
\\	縁	
\\	缘,缘分;血缘;相知之情;关系;机缘。	えん		【名词】
\\	塩	
\\	塩を効く。/有效,咸淡正好。	えん		【惯用句】
\\	艶	
\\	光泽,光亮,润泽。 
\\	艳事,风流事。	えん		【名词】
\\	園	
\\	园,花园。(庭。庭園。また,花・野菜・果樹を栽培する区域。) 
\\	地方,场所。(何かの行われる場所。)	えん		【名词】
\\	円滑	
\\	圆滑,圆满,顺利,协调。	えんかつ		【名・形容动词/ナ形容词】
\\	縁側	
\\	后厦,套廊,廊子,走廊。 
\\	缘侧,一般指鲆鱼、多宝鱼或鲽鱼的背鳍附近的肉。	えんがわ		【名词】
\\	沿岸	
\\	沿岸
\\	沿海。	えんがん		【名词】
\\	婉曲	
\\	婉转,委婉。	えんきょく		【形容动词/ナ形容词】
\\	演習	
\\	军队演习。 
\\	专题研究讨论。 
\\	练习。	えんしゅう		【自动词・サ变/三类】
\\	演出	
\\	(会议等的)组织安排。 
\\	演出;导演。	えんしゅつ		【名・他动词・サ变/三类】
\\	演じる	
\\	表演,扮演。	えんじる		【他动词・一段/二类】
\\	演ずる	
\\	扮演。(劇・映画・舞踊などを行う。また、その中で或る役をつとめる。) 
\\	(把不好的东西)做给别人看。(人目につくようなことをする。多く、よくないことにいう。)	えんずる		【他动词・サ变/三类】
\\	沿線	
\\	沿线。(鉄道の線路や幹線道路に沿った所・地域。)	えんせん		【名词】
\\	縁談	
\\	亲事,婚事,提亲,说媒。	えんだん		【名词】
\\	遠方	
\\	远方,远处。	えんぽう		【名词】
\\	円満	
\\	圆满。美满。 
\\	完美无缺状。没有缺点。	えんまん		【形容动词/ナ形容词】
\\	尾	
\\	1、尾にひれをつける。/添枝加叶,渲染夸张。	お		【惯用句】
\\	甥	
\\	侄,侄儿,侄子;甥,外甥。(その人の兄弟や姉妹がもうけた男の子。)	おい		【名词】
\\	追い込む	
\\	赶进,撵进; 逼入,使陷入;(赛跑等)接近终点最后加劲;(工作等)到紧要关头作最后努力。 
\\	紧排,挤排,移前。 
\\	使……内攻。	おいこむ		【他动词・五段/一类】
\\	美味しい	
\\	味美的;好吃的。语气比「うまい」客气。	おいしい		【形容词/イ形容词】
\\	追い出す	
\\	赶出,逐出,轰出,撵出,驱逐出去。	おいだす		【他动词・五段/一类】
\\	於いて	
\\	以「…において」的形式使用。 
\\	在,于。(場所·時間を表す。) 
\\	在……方面,对于……。(場合·事柄を表す。)	おいて		【接续词】
\\	老いる	
\\	老,年老,上年纪。 
\\	衰老。 
\\	时候将尽,垂暮。	おいる		【自动词・一段/二类】
\\	負う	
\\	背,负;担负;蒙受;受照顾;相称。	おう		【他动词・五段/一类】
\\	応急	
\\	应急。(急場のまにあわせ。)	おうきゅう		【名词】
\\	応募	
\\	应募,报名参加。	おうぼ		【名・自动词・サ变/三类】
\\	大方	
\\	高明的人,有学识的人,大方。	おおかた		【名词】
\\	大柄	
\\	身量大,骨架大。 
\\	大花样,大花纹。	おおがら		【名・形容动词/ナ形容词】
\\	大げさ	
\\	夸大,夸张,铺张,小题大做。	おおげさ		【形容动词/ナ形容词】
\\	大事	
\\	重要,要紧,宝贵,保重,爱护。 
\\	小心,慎重。 
\\	大祸。	おおごと		【名词】
\\	大ざっぱ	
\\	粗心,粗略,草率。(こまかい事にこだわらないさま。粗雑なさま。おおまか。) 
\\	同:大雑把	おおざっぱ		【形容动词/ナ形容词】
\\	大筋	
\\	梗概,概略,主要经过,主要内容。(大きくつかんだ大体の筋。)	おおすじ		【名词】
\\	大空	
\\	天空。	おおぞら		【名词】
\\	大幅	
\\	大幅度,广泛,间距大。 
\\	宽幅。	おおはば		【名・形容动词/ナ形容词】
\\	大水	
\\	洪水,大水。	おおみず		【名词】
\\	公	
\\	公爵。 
\\	国家,公,公家。	おおやけ		【名词】
\\	可笑しい	
\\	可笑的,滑稽的,奇怪的,可疑的。	おかしい		【形容词/イ形容词】
\\	犯す	
\\	犯,干违背规则、法律、道德的事。 
\\	奸污,强奸。	おかす		【他动词・五段/一类】
\\	侵す	
\\	侵犯,侵入;侵权,侵害。	おかす		【他动词・五段/一类】
\\	臆病	
\\	胆怯,胆小,怯懦。	おくびょう		【名・形容动词/ナ形容词】
\\	遅らす	
\\	推迟,拖延,使后退。 
\\	抛下。 
\\	先死。	おくらす		【他动词・五段/一类】
\\	遅れ	
\\	晚。(遅れること。あとになること。)	おくれ		【名词】
\\	起こす	
\\	立起,扶起;唤醒;闹事,掀起;引起;翻土;剥下;把扑克牌等的背面翻成正面;(把速记或磁带录音)改成文字。	おこす		【他动词・五段/一类】
\\	行い	
\\	行为;行动;动作。(しわざ。ふるまい。動作。) 
\\	举止;品行。(品行。行状。身持ち。)	おこない		【名词】
\\	厳か	
\\	庄严,严肃,肃穆,庄重,郑重。威严。(いかめしく、近づきにくいさま。威厳があるさま。)	おごそか		【形容动词/ナ形容词】
\\	傲る	
\\	骄傲,傲慢。(地位・権力・財産・才能などを誇って、思い上がった振る舞いをする。)	おごる		【自动词・五段/一类】
\\	押さえる	
\\	压,摁,按。 
\\	堵住,捂。 
\\	抓住,捉住。 
\\	抓住(要点),认识,理解。 
\\	确保,掌握,控制。	おさえる		【他动词・一段/二类】
\\	お先に	
\\	那就先……了。(「さき」の尊敬語・丁寧語。)	おさきに		【副词】
\\	収まる	
\\	容纳,收纳;复原,回复。	おさまる		【自动词・五段/一类】
\\	納まる	
\\	纳入,收纳;复原;满意;缴纳;解决;理解;平息;被接收。	おさまる		【自动词・五段/一类】
\\	治まる	
\\	安定,平定;平息。	おさまる		【自动词・五段/一类】
\\	お産	
\\	生产,分娩,生孩子。(「産」の丁寧語。出産。)	おさん		【名词】
\\	教え	
\\	宗教的教义。说教,教法。 
\\	教导,教诲,教训,教育,指教。	おしえ		【名词】
\\	押し込む	
\\	【自他动・一类】 
\\	塞,塞进,塞入,硬往里装。(狭いところに、強引に入れる。) 
\\	闯进去抢劫(偷盗)。(他人の家などに強引に入る。)	おしこむ		【名词】
\\	惜しむ	
\\	珍惜;惋惜;吝惜。	おしむ		【他动词・五段/一类】
\\	お喋り	
\\	爱说话的人,健谈。聊天,闲谈。	おしゃべり		【名・自动词・サ变/三类】
\\	お洒落	
\\	好打扮的,爱漂亮的。	おしゃれ		【名・自动词・サ变/三类】
\\	押し寄せる	
\\	涌来,涌过来,涌上来,蜂拥而至,麇集而来。 
\\	推到一旁,挪到一边。〔押して近づける〕	おしよせる		【自他・一段/二类】
\\	お邪魔します	
\\	打扰了。	おじゃまします		【惯用句】
\\	雄	
\\	雄鸡。	おす		【名词】
\\	お世辞	
\\	恭维(话),奉承(话),献殷勤(的话);应酬话。	おせじ		【名词】
\\	襲う	
\\	袭击,侵袭,侵扰。 
\\	因袭,沿袭,继承,承袭。 
\\	突然赶往,冲到,闯到。 
\\	突遭受灾害。 
\\	心情突然遭扰乱。	おそう		【他动词・五段/一类】
\\	遅くとも	
\\	最晚,至迟。	おそくとも		【副词】
\\	恐らく	
\\	恐怕,或许,大概,一定。(多分。きっと。)	おそらく		【副词】
\\	恐れ	
\\	畏惧,害怕,恐惧。(恐れること。)	おそれ		【名词】
\\	恐れ入る	
\\	服输,折服。 
\\	实在不好意思。 
\\	吃惊。	おそれいる		【自动词・五段/一类】
\\	煽てる	
\\	煽动,怂恿。 
\\	捧,奉承,恭维,给人戴高帽子。	おだてる		【他动词・一段/二类】
\\	落ち込む	
\\	落进;塌陷;跌落;低落。	おちこむ		【自动词・五段/一类】
\\	落ち着き	
\\	沉着,镇静,安定,安详,稳重。(態度や言動が穏やかで安定していること。) 
\\	放得稳,稳定,稳当。	おちつき		【名词】
\\	落ち葉	
\\	落叶。 
\\	枯叶色,带红的黄褐色。 
\\	(贵族的)私生子。	おちば		【名词】
\\	落ちる	
\\	亡命;逃遁。 
\\	(柔道)气绝;断气,晕过去。 
\\	陷落。 
\\	没考中;落选,落后。 
\\	掉,落;脱落;剥落;褪落。 
\\	掉,落;脱落;剥落;褪落。 
\\	落下,降落,掉下来,坠落。	おちる		【自动词・一段/二类】
\\	乙	
\\	(天干的第二位)乙。	おつ		【名词】
\\	お使い	
\\	「使い」的敬语表达。 
\\	出去一趟。为了买东西、完成某事等而外出一会儿。 
\\	使者,代表。	おつかい		【名词】
\\	お手上げ	
\\	束手无策,毫无办法,没辙,只好放弃,只好认输。	おてあげ		【名词】
\\	弟	
\\	弟弟。 
\\	后辈。	おと		【名词】
\\	落とす	
\\	(从高处)扔下。使落下。使降落。 
\\	遗漏。漏掉。丢失。失落。 
\\	使落入(圈套)。使陷入(困境)。使掉进……。 
\\	攻陷。攻克。 
\\	(地位、程度、价值等)贬低。减低。降低。 
\\	使落第。使落选。使不合格。淘汰。 
\\	驱除。弄掉。去掉。 
\\	中(标)。抽中。 
\\	失去。丧失。丧。亡。	おとす		【他动词・五段/一类】
\\	訪れる	
\\	通信问候。 
\\	到来,来临。	おとずれる		【自动词・一段/二类】
\\	大人しい	
\\	温顺的,温和敦厚的;素净的,淡雅的,朴素的。	おとなしい		【形容词/イ形容词】
\\	少女	
\\	少女,小姑娘(小学生から中学くらいの年齡の女の子)。 
\\	同:乙女	おとめ		【名词】
\\	お供	
\\	陪同的人,随员。	おとも		【名词】
\\	衰える	
\\	势头消失,衰弱。 
\\	衰亡,衰败。	おとろえる		【自动词・一段/二类】
\\	脅かす	
\\	威胁,恫吓,威慑,威逼,用力量等手段吓唬。 
\\	威逼,胁迫,威胁其权力和地位等。	おどかす		【他动词・五段/一类】
\\	脅す	
\\	威胁,恐吓。	おどす		【他动词・五段/一类】
\\	驚き	
\\	惊恐,惊惧,惊愕,惊骇,害怕,震惊,惊异,惊讶,吃惊。	おどろき		【名词】
\\	同い年	
\\	同年,同岁。	おないどし		【名词】
\\	女子	
\\	女子,女人,女性,妇女。 
\\	女儿,女孩子。 
\\	反:男子	おなご		【名词】
\\	自ずから	
\\	自然地,自动地,自然而然地。	おのずから		【副词】
\\	お早う	
\\	早啊,您早,早安(朝の挨拶) 
\\	『比較』“早啊”はくだけた言い方で,“您早”がていねい.“早安”は旧時代の上品な表現.	おはよう		【感叹词】
\\	怯える	
\\	害怕,胆怯,恐惧。 
\\	做恶梦,梦魇,被恶梦魇住。	おびえる		【自动词・一段/二类】
\\	夥しい	
\\	很多,大量。 
\\	过甚。 
\\	夸张。	おびただしい		【形容词/イ形容词】
\\	帯びる	
\\	佩带;携带。 
\\	担任,担负,负重。 
\\	带有,含有。	おびる		【他动词・一段/二类】
\\	覚え	
\\	记事。 
\\	信任,宠信,器重。 
\\	自信,信心。 
\\	体验,经验。 
\\	记忆,记忆力,记性。	おぼえ		【名词】
\\	お巡りさん	
\\	警察,巡警。(親しみの意をこめて警察官を呼ぶ語。)	おまわりさん		【名词】
\\	お宮	
\\	神社。	おみや		【名词】
\\	面	
\\	附近;嘴脸,脸,面孔;表面,外表。	おも		【名词】
\\	思い付き	
\\	一时想起,随便一想,偶然的想法,凭一时高兴,主意,设想。	おもいつき		【名・形容动词/ナ形容词】
\\	面白い	
\\	愉快的。 
\\	被吸引的,精彩的,有趣的。 
\\	滑稽可笑的。 
\\	最好的。	おもしろい		【形容词/イ形容词】
\\	玩具	
\\	玩物,玩弄品。 
\\	玩具,玩意儿。	おもちゃ		【名词】
\\	重なる	
\\	重叠。 
\\	重合,(几件事情)赶在一起。	おもなる		【自动词・五段/一类】
\\	趣	
\\	事情,大致内容,情况,旨趣,要点。 
\\	情形,感觉。 
\\	情趣,风趣。 
\\	据说,听说。	おもむき		【名词】
\\	赴く	
\\	赴,奔赴,前往,去。 
\\	趋向,倾向,趋于,走向。	おもむく		【自动词・五段/一类】
\\	重役	
\\	要职,担任要职的人。 
\\	董事,与董事地位相称的职务。	おもやく		【名词】
\\	重んじる	
\\	重视,注重。 
\\	尊重,器重,敬重。	おもんじる		【他动词・一段/二类】
\\	重んずる	
\\	重视,尊重。(おもおもしいものとする。たっとぶ。尊重する。) 
\\	同:おもんじる	おもんずる		【他动词・サ变/三类】
\\	お休み	
\\	睡觉,就寝。 
\\	晚安;再见;您歇着吧。	おやすみ		【感叹词】
\\	凡そ	
\\	大概,大体上,大约。(大体のところ。約。) 
\\	凡,凡是,一般地。(話を切り出すときに用いる。そもそも、一体。) 
\\	完全,全然,根本。多用于否定。まったく。)	およそ		【副词】
\\	及び	
\\	(连接并列的体言或与体言同等资格的词语时使用)以及,还有,和,与。	および		【接续词】
\\	及ぶ	
\\	波及;达到;扩及;以至于做某事;必要,赶上,匹敌。	およぶ		【自动词・五段/一类】
\\	檻	
\\	笼,围栏,牢。	おり		【名词】
\\	折り返す	
\\	折回,叠回;翻回,卷回。 
\\	反复。 
\\	返回,折回去。	おりかえす		【他动词・五段/一类】
\\	織物	
\\	纺织品,织物,织品。	おりもの		【名词】
\\	織る	
\\	织,编织。 
\\	织布(等)。 
\\	组合(成一个物体)。	おる		【他动词・五段/一类】
\\	俺	
\\	我;俺;咱。(男が同輩・目下の者に対して使う一人称。)	おれ		【代词】
\\	愚か	
\\	愚蠢,愚笨,糊涂,傻。	おろか		【形容动词/ナ形容词】
\\	疎か	
\\	敷衍,马虎,草率,不认真。 
\\	过失,疏忽。	おろそか		【形容动词/ナ形容词】
\\	終わる	
\\	做完。	おわる		【接头/接尾】
\\	音色	
\\	音色,音品。	おんいろ		【名词】
\\	温和	
\\	温和,温暖。 
\\	温和;温柔,柔和。	おんわ		【形容动词/ナ形容词】
\\	仮	
\\	临时,暂时〔まにあわせ〕。 
\\	假,不是真的〔にせ〕。 
\\	假说,假定〔仮定〕。	か		【名词】
\\	科	
\\	专业,系。(学問、教育の場で系統別に分類したもの。) 
\\	科。(物事を区分した、その一つ。) 
\\	生物分类。(生物分類のリンネ式階層分類における基本的階級の1つ。) 
\\	专业,系。(学問、教育の場で系統別に分類したもの。) 
\\	科。(物事を区分した、その一つ。) 
\\	生物分类。(生物分類のリンネ式階層分類における基本的階級の1つ。)	か		【名词】
\\	個	
\\	个体,个人,自己自身。(ひとりの人。自分自身。) 
\\	个。计算物件的量词。(物の数を数える語。)	か		【名词】
\\	下位	
\\	低位。 
\\	次于(某人)的地位。 
\\	低的地位,下级。	かい		【名词】
\\	階	
\\	台阶。阶梯。(階段。)	かい		【名词】
\\	改悪	
\\	改恶,改坏了。	かいあく		【名・他动词・サ变/三类】
\\	改革	
\\	改革。	かいかく		【名・他动词・サ变/三类】
\\	貝殻	
\\	贝壳。被覆贝外侧的壳,由外套膜分泌的石灰质形成,保护贝自身。	かいがら		【名词】
\\	階級	
\\	级别。(社会や組織の中での身分・地位など段階。特に、軍隊での位。) 
\\	阶级;等级;阶层。(生産手段や生産から得る利益などに関して対立する関係にある社会的集団。一定の社会で、身分・職業・学歴・財産などを同じくする人々によって形成される集団。)	かいきゅう		【名词】
\\	海峡	
\\	海峡。	かいきょう		【名词】
\\	会見	
\\	会见。	かいけん		【自动词・サ变/三类】
\\	介護	
\\	看护,护理。(日常生活を助けること。)	かいご		【名・他动词・サ变/三类】
\\	開催	
\\	召开(会议),举办,举行。	かいさい		【名・他动词・サ变/三类】
\\	回収	
\\	回收;收回。	かいしゅう		【名・他动词・サ变/三类】
\\	改修	
\\	改建,重建,翻修。修理,修复。着手对不好处或损坏处进行修改复原。(悪い所やいたんだ所に手を入れて直すこと。)	かいしゅう		【名・他动词・サ变/三类】
\\	怪獣	
\\	怪兽。(正体の知れない不思議な動物。)	かいじゅう		【名词】
\\	解除	
\\	解除。 
\\	废除。	かいじょ		【名・他动词・サ变/三类】
\\	回送	
\\	调回,开回(空车等)。 
\\	转寄,转送。	かいそう		【名・他动词・サ变/三类】
\\	階層	
\\	阶层。社会地位相当的人组成的集团。依据职业、学历、财产、年龄、身份、人种等各种标准或综合标准构成。(社会的地位が大体等しい人々の集団。職業・学歴・財産・年齢・身分・人種などさまざまな基準やその重なったもので作られる。) 
\\	层。(建物の層の重なり。) 
\\	分层,分级。数据等构成上下层的重叠。(データなどの上下に層を成した重なり。)	かいそう		【名词】
\\	開拓	
\\	开垦,开荒。 
\\	开拓,开辟。	かいたく		【名・他动词・サ变/三类】
\\	会談	
\\	会谈,面谈;谈判。	かいだん		【自动词・サ变/三类】
\\	改定	
\\	修改,重新规定。(従来のきまりなどを改め定めること。)	かいてい		【名・他动词・サ变/三类】
\\	改訂	
\\	修订。	かいてい		【名・他动词・サ变/三类】
\\	街道	
\\	大道,大街,通衢。	かいどう		【名词】
\\	介入	
\\	介入,干预;插手,参与;染指;干涉。置身于事件、纷争等之中。	かいにゅう		【名・自动词・サ变/三类】
\\	海抜	
\\	拔海,海拔。	かいばつ		【名词】
\\	介抱	
\\	护理,服侍,照顾。	かいほう		【名・他动词・サ变/三类】
\\	解剖	
\\	分析,剖析。 
\\	解剖。	かいぼう		【名・他动词・サ变/三类】
\\	回覧	
\\	传阅。(書類や本などを順にまわして見ること。) 
\\	同:廻覧	かいらん		【名・他动词・サ变/三类】
\\	改良	
\\	改良。(物事の悪いところを改めて、前より良くすること。改善。)	かいりょう		【名・他动词・サ变/三类】
\\	回路	
\\	电路,回路,线路。(電源から出た電流が、再び電源に入るまでの道筋。電気回路。)	かいろ		【名词】
\\	顧みる	
\\	往回看,回头看。 
\\	回顾。 
\\	顾虑。	かえりみる		【他动词・一段/二类】
\\	省みる	
\\	反省 
\\	自省,自问。	かえりみる		【他动词・一段/二类】
\\	返る	
\\	归还;还原;返回。	かえる		【自动词・五段/一类】
\\	顔付き	
\\	相貌。 
\\	表情。	かおつき		【名词】
\\	掲げる	
\\	悬挂,高举。 
\\	刊登,登载。 
\\	张贴。 
\\	提出(主义、方针等)。 
\\	拨亮(灯心)。 
\\	往上提。	かかげる		【他动词・一段/二类】
\\	踵	
\\	脚后跟。(足の裏の後ろの部分。かかと。) 
\\	同:きびす	かかと		【名词】
\\	係り	
\\	担任者。	かかり		【接头/接尾】
\\	課外	
\\	课外。(学校から履修するようにと与えられた学科·課業のほかのもの。)	かがい		【名词】
\\	書き取り	
\\	抄写;记录。 
\\	听写;默写。	かきとり		【自动词・サ变/三类】
\\	書き取る	
\\	记录。	かきとる		【他动词・五段/一类】
\\	掻き回す	
\\	搅,搅拌,搅动。 
\\	搅乱,搅和。	かきまわす		【他动词・五段/一类】
\\	欠く	
\\	缺,缺乏,缺少。 
\\	弄坏。 
\\	欠,缺欠,怠慢。	かく		【他动词・五段/一类】
\\	角	
\\	角,犄角『口』。 
\\	角形物。	かく		【名词】
\\	核	
\\	果实。 
\\	核心。 
\\	地核。 
\\	细胞核。 
\\	原子核。 
\\	核武器。	かく		【名词】
\\	格	
\\	资格,等级。标准,水平。地位,身分。品格。(そのものの値打ちによってできた段階・位・身分など。) 
\\	(语法)格。	かく		【名词】
\\	佳句	
\\	佳句;好的俳句	かく		【名词】
\\	格差	
\\	差距,差别;等级差别; 质量差别;价格差别,差价。(等级などの差。)	かくさ		【名词】
\\	拡散	
\\	〈理〉漫射,〈化〉渗滤。 
\\	扩散。	かくさん		【名词】
\\	各種	
\\	各种各样,每一种,种种。形形色色的种类。(いろいろな種類。さまざま。種種。諸種。)	かくしゅ		【名词】
\\	隔週	
\\	每隔一周,隔周。(一週間おき。)	かくしゅう		【名词】
\\	確信	
\\	坚信,确信。〔かたく信じること。〕 
\\	有信心,有把握。〔たしかな自信。〕	かくしん		【自他・サ变/三类】
\\	革新	
\\	革新。	かくしん		【名・他动词・サ变/三类】
\\	確定	
\\	确定。(確かに決まること。)	かくてい		【名・他动词・サ变/三类】
\\	獲得	
\\	获得,取得,争取。	かくとく		【名・他动词・サ变/三类】
\\	確保	
\\	确保。	かくほ		【名・他动词・サ变/三类】
\\	確立	
\\	确立,确定。	かくりつ		【名・自他・サ变/三类】
\\	賭け	
\\	赌博。(勝負事などで金品を出し合い,勝者がその金品を取ること。賭け事。) 
\\	赌一把,碰运气。(運を天に任せて思い切ってやってみること。)	かけ		【名词】
\\	掛け	
\\	穿着,戴着;……折;座位坐...人;付出。	かけ		【接头/接尾】
\\	駆け足	
\\	策马急驰,使马快跑(疾驰)。 
\\	跑步似的,走马看(观)花地,急急忙忙地,草率地。 
\\	快跑,跑步。 
\\	同:駈け足	かけあし		【自动词・サ变/三类】
\\	家計	
\\	家计,家庭经济,家中收支情况。	かけい		【名词】
\\	駆けっこ	
\\	赛跑。(かけくらべ。)	かけっこ		【自动词・サ变/三类】
\\	駆ける	
\\	跑,快跑,奔跑。 
\\	跑,策马疾驰。	かける		【自动词・一段/二类】
\\	賭ける	
\\	赌,赌博。 
\\	亦作「懸ける」)冒着危险做,豁上,赌。	かける		【他动词・一段/二类】
\\	加工	
\\	加工。 
\\	加工,指对天然物品或他人的动产进行加工,改变其形状。	かこう		【名・他动词・サ变/三类】
\\	化合	
\\	化合。	かごう		【自动词・サ变/三类】
\\	嵩張る	
\\	增大,体积大。	かさばる		【自动词・五段/一类】
\\	嵩む	
\\	增大,体积大。 
\\	增多。	かさむ		【自动词・五段/一类】
\\	風車	
\\	风车。(風の力を受けて回る仕掛けの車。米つきや水くみなどに使う。) 
\\	风车儿。(軸に羽根が三、四枚付いていて、持って走ると回転する仕掛けのおもちゃ。)	かざぐるま		【名词】
\\	華奢	
\\	奢华,纤细,窈窕,苗条。 
\\	奢华。 
\\	削薄,纤弱,不结实。	かしゃ		【名・形容动词/ナ形容词】
\\	箇所	
\\	(特定的)地方,部分。(その物のある所。場所。)	かしょ		【名词】
\\	火傷	
\\	火伤,烫伤。	かしょう		【名词】
\\	箇条書き	
\\	分条写,分项写;一条一条地写,一项一项地写,列举。(いくつかの項目を読みやすくするために箇条書きを用いる。文中にいくつもの項目を並べていくと、他の文字や記号に埋もれてしまい、項目の確認がしづらくなるからである。)	かじょうがき		【名词】
\\	微か	
\\	微弱;略微;微暗,朦胧,模糊。 
\\	可怜,贫穷,微贱。	かすか		【形容动词/ナ形容词】
\\	霞む	
\\	有雾霭,起霞。 
\\	(亦作「翳む」)朦胧,看不清。 
\\	暗淡,不醒目,看不清,因其他事物的存在而变得不再醒目。	かすむ		【自动词・五段/一类】
\\	化する	
\\	化为,变成。化,使化为〔变成〕。(形や性質が別のものに変わる。また、変わらせる。変える。) 
\\	教育,感化。(影響を受けて変わる。同化する。また、導いて変わらせる。感化する。教化する。)	かする		【自他・サ变/三类】
\\	擦る	
\\	涂上;擦上(ぬりつける) 
\\	推诿;转嫁(罪や責任を他人に負わせる) 
\\	なすりつける	かする		【他动词・五段/一类】
\\	火星	
\\	英语中,红色行星的通称。 
\\	〈天〉火星太阳系第四行星。	かせい		【名词】
\\	化石	
\\	变成石头似的。 
\\	旧事物或老规矩原封不动地保存下来。 
\\	化石。	かせき		【名词】
\\	河川	
\\	河川。	かせん		【名词】
\\	化繊	
\\	化学纤维。(「化学繊維」の略。)	かせん		【名词】
\\	過疎	
\\	(人口)过稀,过少。	かそ		【名词】
\\	難い	
\\	难。不容易的,困难的。(むずかしい。) 
\\	同:むずかしい	かたい		【形容词/イ形容词】
\\	片思い	
\\	单恋,单相思。(男女の一方だけが相手を恋い慕うこと。)	かたおもい		【名词】
\\	敵	
\\	敌,敌人,仇敌,对头,敌对分子。〔かたき。〕 
\\	对手,敌手。〔競争の相手。〕 
\\	障碍
\\	大敌,有害。〔障害となるもの。〕 
\\	《相关惯用语》	かたき		【名词】
\\	気質	
\\	气质,风格,脾气,性情,性格。	かたぎ		【名词】
\\	片言	
\\	片言,简短的话。	かたこと		【名词】
\\	片付け	
\\	收拾,整理。	かたづけ		【名词】
\\	傾く	
\\	倾斜,偏,歪。 
\\	西斜。 
\\	有……倾向,倾向于……。 
\\	衰落,衰微。	かたぶく		【自动词・五段/一类】
\\	傾ける	
\\	使……倾斜,使……歪。 
\\	倾注。 
\\	倾,败家,使灭亡。	かたむける		【他动词・一段/二类】
\\	固める	
\\	使…凝固。 
\\	集中到一起。 
\\	使坚定。 
\\	加强防守。 
\\	身穿。	かためる		【他动词・一段/二类】
\\	偏る	
\\	偏于一,集中于一方;偏颇,不公平,不平衡。	かたよる		【自动词・五段/一类】
\\	傍ら	
\\	旁边。(はたに偏ったところ。わき。) 
\\	一边……一边……,一面……一面……。(しながら、また一方で。)	かたわら		【名词】
\\	課題	
\\	(提出的)题目。(題。) 
\\	课题,任务。(任務。)	かだい		【名词】
\\	花壇	
\\	花坛,花圃。	かだん		【名词】
\\	画期	
\\	划时代(時代を限ること)。	かっき		【名词】
\\	活発	
\\	活泼,活跃。	かっぱつ		【形容动词/ナ形容词】
\\	且つ	
\\	并且。既…又…。(その上に。それとともに。)	かつ		【接续词】
\\	割	
\\	分开。切开。割开。(割ること。割った物) 
\\	(得失的)比例。(損であること。あるいは得であること。また、損か得かの度合。) 
\\	分配。分摊。(割前。割当て。) 
\\	分组表。相扑比赛的对手名单。(大相撲の取組表。) 
\\	均摊。(物と物との比。) 
\\	稀释。冲淡。(歩合) 
\\	(以「…のわりに」的形式)与相应的程度相比…。同…不相适合。((「…割に」の形で)基準となるものの程度と比べてどうあるかをいう語。) 
\\	同:割り	かつ		【名词】
\\	嘗て	
\\	曾,曾经。以前,过去。(以前に、前に。昔。) 
\\	前所未有,(后接否定语)从未有过,未尝有过。((下に打ち消しの語を伴って)今まで一度も。ついぞ。) 
\\	同:曽て	かつて		【副词】
\\	日付	
\\	日期;年月日。	かづけ		【名词】
\\	門	
\\	门,门前,门外,门口(家の門)。 
\\	家,家族,家庭(家)。	かど		【名词】
\\	叶う	
\\	能实现,能如愿以偿。	かなう		【自动词・五段/一类】
\\	叶える	
\\	满足愿望。	かなえる		【他动词・一段/二类】
\\	金槌	
\\	钉锤,锤子。(釘などを打ちつける道具。 頭部が鉄製の鎚。) 
\\	不会游泳(的人)。(泳ぎのまったくできないこと。また,そういう人。)	かなづち		【名词】
\\	鉄棒	
\\	铁棍,铁棒,铁条。(鉄製の棒。特に、鉄尖棒,頭部にいくつかの鉄の輪をつけた、長いつえのような鉄の棒。夜回りや行列の先頭に立つ者などが地面に突いて鳴らして歩く。) 
\\	单杠。(体操器具の一。てつぼう。)	かなぼう		【名词】
\\	加入	
\\	加入,参加。	かにゅう		【自动词・サ变/三类】
\\	金庫	
\\	金库,保险柜(金銀・宝物などを入れておく庫。かねぐら)。	かねぐら		【名词】
\\	予言	
\\	预言,预告。	かねごと		【名・他动词・サ变/三类】
\\	兼ねて	
\\	早已,早先,老早。	かねて		【副词】
\\	下品	
\\	〈仏〉(极乐净土分为上中下三等的)最下级的净土。(極楽往生を九段階に分けた九品の下位の三段階。) 
\\	最低的阶级;下等。〔下等〕	かひん		【名词】
\\	花粉	
\\	花粉。(種子植物の雄性配偶体。雄しべの葯(やく)の中で減数分裂によって作られる半数性の単細胞。直接または風・虫・鳥などによって雌しべの柱頭に運ばれる。)	かふん		【名词】
\\	株式	
\\	股,股份。(総資本を等分した単位。) 
\\	股票。(株券。) 
\\	股权。(株主権。)	かぶしき		【名词】
\\	気触れる	
\\	起斑疹;着迷;热衷	かぶれる		【自动词・一段/二类】
\\	貨幣	
\\	货币,钱币。	かへい		【名词】
\\	構え	
\\	框。 
\\	准备。 
\\	姿势,架势,架式。 
\\	构造,格局,门面,外观。	かまえ		【名词】
\\	構える	
\\	建造,成家,开店。 
\\	对对方摆出某种态度。 
\\	手中持物做某种姿势。 
\\	假托,捏造。 
\\	准备好。 
\\	企图,策划。	かまえる		【自他・一段/二类】
\\	加味	
\\	加进,放进,掺加;采纳,采取。 
\\	调味,添加调味料,加佐料。	かみ		【名词】
\\	噛み切る	
\\	咬断,咬破。(物をかんで切る。食い切る。)	かみきる		【他动词・一段/二类】
\\	過密	
\\	过密,过于集中。	かみつ		【名・形容动词/ナ形容词】
\\	瓶	
\\	瓮;缸(口が大きく,胴が丸く深い土器・陶器。また,壺の大形のものもいう。容器,食物の保存・加工用など様々に用いられる)。 
\\	花瓶;酒瓶(瓶子 に同じ)。	かめ		【名词】
\\	かも知れない	
\\	接于终止形或体言下,表也许,说不定,也未可知,保不住。(断定はできないが、その可能性があることを表す。)	かもしれない		【惯用句】
\\	粥	
\\	稀饭。	かゆ		【名词】
\\	痒い	
\\	痒,发痒。	かゆい		【形容词/イ形容词】
\\	身体	
\\	同「体(からだ)」。	からだ		【名词】
\\	体付き	
\\	体格,体形,姿态,身材。	からだつき		【名词】
\\	絡む	
\\	卷,绕。 
\\	牵涉。 
\\	胡搅蛮缠,无理取闹,找碴儿。 
\\	缠住,纠缠。 
\\	扯上关系。 
\\	与某事相关。 
\\	和「からめる」的意思差不多。	からむ		【自动词・五段/一类】
\\	借り	
\\	借款,欠债,赊帐。(借りること。また、借りたもの。特に借金。) 
\\	应该报答的恩,应予报复的怨。必须还报对方的精神负担。(相手に報いなければならないような精神的負担。負い目。)	かり		【名词】
\\	加留多	
\\	扑克牌;骨牌;日本的纸牌 
\\	加留多;歌留多;骨牌 
\\	かるた		【名词】
\\	涸れる	
\\	干涸,枯竭。 
\\	(能力、感情、财力等)挖尽,用尽。	かれる		【自动词・一段/二类】
\\	過労	
\\	过劳,疲劳过度。	かろう		【自动词・サ变/三类】
\\	辛うじて	
\\	总算是…,好容易才…,勉勉强强…。	かろうじて		【副词】
\\	側	
\\	四周,周围。 
\\	壳。	かわ		【名词】
\\	可愛い	
\\	可爱,讨人喜欢。 
\\	宝贵的。 
\\	小巧玲珑。	かわいい		【形容词/イ形容词】
\\	可愛がる	
\\	爱,喜爱,疼爱。 
\\	教训,严加管教。	かわいがる		【他动词・五段/一类】
\\	可哀想	
\\	令人同情的,令人怜悯的,招人同情的。(気の毒なさま。同情を誘うさま。)	かわいそう		【形容动词/ナ形容词】
\\	交わす	
\\	交替,交换。 
\\	交错,交叉。 
\\	(接在动词的连用形后)相互……。	かわす		【他动词・五段/一类】
\\	代わる代わる	
\\	交替,轮流,轮换。	かわるがわる		【副词】
\\	乾	
\\	西北方,乾。(十二支で表した方位で、戌と亥の間。北西の方角。)	かん		【名词】
\\	管	
\\	库利基管
\\	考利基电子管
\\	热阴极电子射线管。
\\	線管の一。熱せられた陰極から出た電子を高圧で加速・集束して陽極に衝突させ、
\\	線を発生させるもの。) 
\\	管子,管道。(内部がからで筒状のもの。) 
\\	管乐器,吹奏乐器。(笛・筆など、くだ状の物を数えるのに用いる。)	かん		【名词】
\\	幹	
\\	干,秆,茎。(草木のみきや茎。)	かん		【名词】
\\	冠	
\\	冠冕,帽子。(かんむり。) 
\\	最优秀。盖世无双。(最も優れているさま。最高と認められるさま。)	かん		【名词】
\\	観	
\\	观,外观。映入眼内的印象、样子。观感,印象;景象,样子。(目に映った印象。物事の様子・状態。) 
\\	观。佛教中指在心中观察特定的想象对象和内心本性等,以通达佛教真理的方法。(〔仏〕 特定の想念や心の本性などを心の中で観察し、仏教の真理に達する方法。)	かん		【名词】
\\	簡易	
\\	简易,简单,简便。	かんい		【形容动词/ナ形容词】
\\	感慨	
\\	感慨。(心に深く感じて、しみじみとした気持ちになること。)	かんがい		【名词】
\\	寒気	
\\	寒气,寒冷,寒冷的气候。亦指寒冷的程度。	かんき		【名词】
\\	簡潔	
\\	简洁。	かんけつ		【形容动词/ナ形容词】
\\	還元	
\\	还原,返回本来面目,返回原样,恢复原状。 
\\	〈化〉还原。	かんげん		【名・自他・サ变/三类】
\\	刊行	
\\	出版发行,刊行。	かんこう		【名・他动词・サ变/三类】
\\	慣行	
\\	例行,习以为常的行动,经常实行。(ふだん、習慣として行うこと。) 
\\	惯例,常规,习惯做法,习俗。(古くからの習わしとして行われていること。〕	かんこう		【名・他动词・サ变/三类】
\\	勧告	
\\	劝告,向对方传达某种事项,并劝告其采取适当措施以符合该事项,日本在进行行政指导等时常用,但无法律约束力。	かんこく		【名・他动词・サ变/三类】
\\	看護	
\\	护理,看护。(傷病人などの手当てをしたり、世話をしたりすること。)	かんご		【名・他动词・サ变/三类】
\\	漢語	
\\	音读汉语词。 
\\	汉语
\\	汉语词
\\	。	かんご		【名词】
\\	換算	
\\	换算,折合。换成另一种单位计算某数值。(数値をほかの単位に換えた計算すること。)	かんさん		【名・他动词・サ变/三类】
\\	監視	
\\	监视;监视人。(不都合なことの怒らぬように見張ること。)	かんし		【名・他动词・サ变/三类】
\\	慣習	
\\	习惯;习俗;老规矩;常规。(ある社会で,長い間にみんなに認められるようになって,いつもそのようにする決まりとなっているならわし。世間のしきたり。なれること。習慣となること。 
\\	かんしゅう		【名词】
\\	観衆	
\\	观众。(スポーツなどの見物人たち。)	かんしゅう		【名词】
\\	干渉	
\\	干涉,干预。 
\\	<理>干扰,干涉。	かんしょう		【名・自动词・サ变/三类】
\\	感触	
\\	感触,触觉。手或身体接触其他东西时的感觉。(手や体がほかのものに触れたときに得られる感じ。肌触り。手ざわり。) 
\\	感触。由外界刺激而引起的感觉。(外界の刺激によって感じること。) 
\\	感触,觉察。从对方态度等隐约得到的感受。(相手の態度などからそれとなく感じ取れるもの。)	かんしょく		【自动词・サ变/三类】
\\	肝心	
\\	首要;重要;紧要;关键。 
\\	同:肝腎	かんじん		【名・形容动词/ナ形容词】
\\	歓声	
\\	欢声,欢呼声。	かんせい		【名词】
\\	感染	
\\	感染。(病原体が体中に侵入すること。また、他の影響を受けてその風に染まること。)	かんせん		【自动词・サ变/三类】
\\	幹線	
\\	干线。(鉄道・道路・電話などで、重要地域を結ぶ主要な線。)	かんせん		【名词】
\\	関税	
\\	关税。(貨物が国境を通過する際課せられる税。輸入税と輸出税がある。)	かんぜい		【名词】
\\	簡素	
\\	简单朴素,简朴;简化;没有修饰,质朴。	かんそ		【名・形容动词/ナ形容词】
\\	観点	
\\	观点,看法,见地,角度。作为判断之基础的立场。	かんてん		【名词】
\\	感度	
\\	敏感度,灵敏度,灵敏性,感应度,感光度。	かんど		【名词】
\\	幹部	
\\	干部。	かんぶ		【名词】
\\	勘弁	
\\	原谅,饶,饶恕,宽恕,容忍。	かんべん		【名・他动词・サ变/三类】
\\	完璧	
\\	(意为无瑕的宝玉)完美,完璧,完美无缺,完好无损,尽善尽美。	かんぺき		【名・形容动词/ナ形容词】
\\	感無量	
\\	无限感慨,感慨万千,感慨难言,深感,痛感。	かんむりょう		【形容动词/ナ形容词】
\\	勧誘	
\\	劝,劝说;劝诱。	かんゆう		【名・他动词・サ变/三类】
\\	関与	
\\	干预,参与。 
\\	同:干与	かんよ		【名・自动词・サ变/三类】
\\	慣用	
\\	惯用,习用。	かんよう		【名词】
\\	寛容	
\\	宽恕,宽容,容许,容忍。(心が広く、他人の言動などをよく受け入れること。)	かんよう		【名・他动词・サ变/三类】
\\	観覧	
\\	观看,参观。	かんらん		【名词】
\\	官僚	
\\	官僚,官吏。(国の行政面の仕事に従事する人たち。)	かんりょう		【名词】
\\	慣例	
\\	惯例,老规矩。(しきたり。ならわし。)	かんれい		【名词】
\\	還暦	
\\	花甲,满六十岁。(60年で再び生まれた年の干支にかえるところから、数え年61歳のこと。また、その祝い。華甲。)	かんれき		【名词】
\\	貫禄	
\\	尊严,威严,威信。	かんろく		【名词】
\\	緩和	
\\	缓和,和缓,放宽。	かんわ		【名・自他・サ变/三类】
\\	蓋	
\\	盖儿,盖子(はこなどの入れものの口や穴に、上からあてておおいふさぐもの)。	がい		【名词】
\\	街	
\\	商业区;居住区;市场,店铺。	がい		【名词】
\\	外貨	
\\	外国货币,外币,外汇。(外国の貨幣。) 
\\	进口货,外国货。(外国の品物や商品。)	がいか		【名词】
\\	外観	
\\	在表现人的外表时用“外表”一词。 人の見かけについては“外表”を用いる。	がいかん		【名词】
\\	外相	
\\	外交大臣,外相;外长,外交部长。(外務大臣のこと。)	がいしょう		【名词】
\\	害する	
\\	杀害,陷害。 
\\	危害。 
\\	妨碍。 
\\	伤害,损害,毁坏。	がいする		【他动词・サ变/三类】
\\	概説	
\\	概论,概述,概说。	がいせつ		【名・他动词・サ变/三类】
\\	該当	
\\	符合,相当。	がいとう		【自动词・サ变/三类】
\\	街頭	
\\	街头。	がいとう		【名词】
\\	概念	
\\	概念。关于事物的大致想法。逻辑学上指从个别事物中抽取具有共同本质的特征,归纳成一般的观念。	がいねん		【名词】
\\	外来	
\\	外来;舶来。(よそあるいは外国から来ること。) 
\\	门诊。(入院患者と違ってよそから来て診療を受ける人。)	がいらい		【名词】
\\	概略	
\\	概况,概要,梗概,概略。(物事のあらまし。大略。概要。)	がいりゃく		【名词】
\\	学芸	
\\	学业和技艺,文艺。	がくげい		【名词】
\\	学士	
\\	学士。 
\\	学生。	がくし		【名词】
\\	学説	
\\	李森科学说。 
\\	学说。	がくせつ		【名词】
\\	楽譜	
\\	乐谱,谱子。	がくふ		【名词】
\\	学歴	
\\	学历。(その人がどんな学校を出たかという経歴。)	がくれき		【名词】
\\	崖	
\\	崖,悬崖,绝壁,川崖,河崖。	がけ		【名词】
\\	合唱	
\\	合唱。	がっしょう		【名・他动词・サ变/三类】
\\	合致	
\\	一致;吻合。	がっち		【自动词・サ变/三类】
\\	月日	
\\	年月;日期。 
\\	月日。	がっぴ		【名词】
\\	合併	
\\	企业合并。 
\\	合并,归并。	がっぺい		【名・自他・サ变/三类】
\\	癌	
\\	癌、癌瘤、癌肿、癌症。 
\\	症结、要害、祸根。	がん		【名词】
\\	眼科	
\\	眼科。	がんか		【名词】
\\	眼球	
\\	眼球,眼珠。(脊椎動物の視覚をつかさどる、一対の球状の器官。眼窩内に収まり、強膜または角膜・脈絡膜・網膜の3層の膜に包まれ、内部に水晶体・硝子体などが入っている。外からの光が角膜および両凸レンズ状の水晶体を通って網膜上に像を結ぶ。めだま。)	がんきゅう		【名词】
\\	眼鏡	
\\	眼镜。	がんきょう		【名词】
\\	頑固	
\\	顽固,固执。 
\\	久治不愈的病,痼疾,顽疾。	がんこ		【名・形容动词/ナ形容词】
\\	願書	
\\	祈愿书。 
\\	申请书。	がんしょ		【名词】
\\	頑丈	
\\	坚固;结实;强健,健壮。	がんじょう		【形容动词/ナ形容词】
\\	元年	
\\	元年。	がんねん		【名词】
\\	贋物	
\\	赝品,假货。(にせもの。まやかしもの。)	がんぶつ		【名词】
\\	元来	
\\	原来,本来,最初,当初。	がんらい		【副词】
\\	生	
\\	生;生命。(いのち。せい。)	き		【名词】
\\	期	
\\	期,时期,时机,季节。(ある一定の期間。)	き		【名词】
\\	企画	
\\	规划,计划。	きかく		【名・他动词・サ变/三类】
\\	規格	
\\	规格,标准。(工業製品の品質・大きさ・形状などについて定められた標準) 
\\	规格,规范。(判断の基準となる社会的な標準)	きかく		【名词】
\\	着飾る	
\\	盛装,打扮。	きかざる		【他动词・五段/一类】
\\	器官	
\\	器官。	きかん		【名词】
\\	季刊	
\\	季刊。	きかん		【名词】
\\	危害	
\\	危害。	きがい		【名词】
\\	気兼ね	
\\	多心,拘泥,顾及,顾虑。	きがね		【名・自动词・サ变/三类】
\\	気軽	
\\	轻松愉快,舒畅,爽快,随随便便。	きがる		【形容动词/ナ形容词】
\\	危機	
\\	危机,险关。	きき		【名词】
\\	聞き取り	
\\	听懂,听取。听后理解,听后知道。 
\\	听力。听懂外语的能力。 
\\	同:聴き取り	ききとり		【名词】
\\	効き目	
\\	效验,效力	ききめ		【名词】
\\	帰京	
\\	返京,回首都。	ききょう		【名・自动词・サ变/三类】
\\	基金	
\\	基金。	ききん		【名词】
\\	棄権	
\\	弃权。	きけん		【名・他动词・サ变/三类】
\\	喜劇	
\\	笑剧,喜剧。 
\\	闹剧,滑稽剧。搞笑事儿。	きげき		【名词】
\\	起源	
\\	起源。事物发生的根源,起因,根源,起初。 
\\	同:おこり、はじまり	きげん		【名词】
\\	機構	
\\	结构。(機械の内部の構造。) 
\\	机构,组织。(機械的に構成されている仕組み。) 
\\	组织;机构。(機械の内部の構造。しくみ。メカニズム。) 
\\	构造;结构。(会社、団体などの組織。また、その仕組み。)	きこう		【名词】
\\	既婚	
\\	已婚。	きこん		【名词】
\\	記載	
\\	记载,写上;刊登。	きさい		【名・他动词・サ变/三类】
\\	気障	
\\	(在服装、言行方面)装模作样,矫揉造作,令人作呕。	きざ		【形容动词/ナ形容词】
\\	兆し	
\\	萌芽。 
\\	兆头,预兆。	きざし		【名词】
\\	兆	
\\	苗头,预兆,征兆。〔きざし。物事の起こる前ぶれ。〕 
\\	〈数〉万亿。兆。数的单位,1亿的1万倍。(数の単位。)	きざし		【名词】
\\	軋む	
\\	两物相擦发出声音,吱吱嘎嘎响。 
\\	着急。 
\\	摩擦,相互摩擦。	きしむ		【自动词・五段/一类】
\\	気象	
\\	气象。〔大気中の現象。〕 
\\	天性,秉性,脾气,性体。〔気性,心だて,気だて,気質。〕	きしょう		【名词】
\\	記述	
\\	描述。 
\\	记述。	きじゅつ		【名・他动词・サ变/三类】
\\	奇数	
\\	奇数,单数。	きすう		【名词】
\\	築く	
\\	砌;修筑;将土石垒起。	きずく		【他动词・五段/一类】
\\	傷付く	
\\	受伤;负伤;弄出瑕疵;遭受损害。	きずつく		【自动词・五段/一类】
\\	規制	
\\	规定(章则);限制,控制。	きせい		【名・他动词・サ变/三类】
\\	汽船	
\\	轮船。	きせん		【名词】
\\	寄贈	
\\	赠送,捐赠。	きそう		【名词】
\\	貴族	
\\	贵族。	きぞく		【名词】
\\	鍛える	
\\	磨练。 
\\	锻造,打制,锤炼。 
\\	锻炼。	きたえる		【他动词・一段/二类】
\\	来る	
\\	来,到来。(くる。) 
\\	引起,发生。(おこる。) 
\\	下(次的)。(近くやってくる。)	きたる		【自动词・五段/一类】
\\	気立て	
\\	性情;性格。(その人に備わっている気持ちの傾向。心だて。気質。性質。)	きだて		【名词】
\\	几帳面	
\\	规规矩矩的;一丝不苟的;严格的。	きちょうめん		【形容动词/ナ形容词】
\\	切っ掛け	
\\	原因,起首;开端;机会;一刹那;记号;梆子。(歌舞伎开闭幕或换道具时打的。)(物事を始めるための手がかりや機会。また、物事が始まる原因や動機。歌舞伎などで、演技や効果などの開始、次への進行を指示する種々の合図。)	きっかけ		【名词】
\\	喫茶	
\\	喝茶。(茶を飲むこと。きっちゃ。)	きっさ		【名词】
\\	規定	
\\	规定	きてい		【名词】
\\	起点	
\\	起点,出发点。(一連の動作・作用がそこから起こる点。)	きてん		【名词】
\\	軌道	
\\	〈鉄〉轨道,钢轨。(電車などの軌条車両を走らせるための構造物からなる道。) 
\\	〈機〉轨道。(物体が運動するときに描く一定の道筋。) 
\\	〈天〉轨道。(天体が運行する道筋。) 
\\	轨道。(物事の経過していく道筋。) 
\\	路线,方针。(方針。)	きどう		【名词】
\\	甲	
\\	甲壳。(カメ・カニなどの体をおおう堅い殻。甲羅。) 
\\	〔手脚的〕外侧面,手背,脚背。((手足的表面)手の甲。足の甲 。) 
\\	甲,十干的第一。(十干の第一。きのえ。) 
\\	甲,第一位。(等級・成績などをつけるときに用いて、一番上位を表す。) 
\\	甲,甲方。(二人以上の人、または二つ以上の物事がある時、その一つの名にかえて用いる。)	きのえ		【名词】
\\	気配	
\\	气氛,感觉。 
\\	景气,行情。	きはい		【名词】
\\	規範	
\\	规范,模范,标准。(行動や判断の基準・手本。)	きはん		【名词】
\\	気品	
\\	品格;气度;气派,文雅,斯文,温文尔雅。(表情や行ないにあらわれる上品な感じ。芸術作品などについてもいう。)	きひん		【名词】
\\	気風	
\\	风气,风尚;习气。(気性。特に、ある集団や地域内の人々が共通に持っている気質。) 
\\	特性;气质;风度,气派。(気質。性格。)	きふう		【名词】
\\	起伏	
\\	起卧。 
\\	荣枯,盛衰,浮沉,起落。 
\\	起伏,凸凹,高低。	きふく		【名词】
\\	規模	
\\	规模。 
\\	范围。	きぼ		【名词】
\\	気まぐれ	
\\	没准脾气,反复无常。 
\\	一时冲动,一时高兴。	きまぐれ		【名・形容动词/ナ形容词】
\\	生真面目	
\\	过于认真,一本正经,死心眼的人。	きまじめ		【名・形容动词/ナ形容词】
\\	期末	
\\	期末。(ある一定期間の終り。)	きまつ		【名词】
\\	決まり悪い	
\\	不好意思,难为情,怕人耻笑而心中不安。	きまりわるい		【形容词/イ形容词】
\\	決まる	
\\	决定,规定;决定胜负;得体;符合要求;一定是。	きまる		【自动词・五段/一类】
\\	記名	
\\	记名;签名。	きめい		【名・自动词・サ变/三类】
\\	脚色	
\\	添枝加叶,夸大其词。 
\\	改写(改编)成戏剧或电影。	きゃくしょく		【名・他动词・サ变/三类】
\\	脚本	
\\	脚本,剧本。	きゃくほん		【名词】
\\	客観	
\\	客观。(主観の認識・行為の対象となるもの。特定の認識作用や関心を越えた一般的もの。)	きゃっかん		【名词】
\\	規約	
\\	规章,章程,规约,协约。(関係者の間で、相談してきめた規則。組織・団体内で、その成員に適用するために定めた規約。)	きやく		【名词】
\\	救援	
\\	救援;支援。(他人の危険や困難をすくいたすけること。加勢。)	きゅうえん		【名・他动词・サ变/三类】
\\	究極	
\\	究竟,最终。	きゅうきょく		【名词】
\\	球根	
\\	〈植〉球根,鳞茎。	きゅうこん		【名词】
\\	救済	
\\	救济,救助。	きゅうさい		【名・他动词・サ变/三类】
\\	給食	
\\	供餐,供食,供给饮食,提供伙食。学校、公司等供给学生、职员等的膳食。	きゅうしょく		【名词】
\\	給仕	
\\	工勤人员。 
\\	勤杂,茶水服务,服务员,侍者。	きゅうじ		【自动词・サ变/三类】
\\	休戦	
\\	停战,休战。	きゅうせん		【名词】
\\	旧知	
\\	故知,老友。	きゅうち		【名词】
\\	宮殿	
\\	宫殿。	きゅうでん		【名词】
\\	窮乏	
\\	贫穷,贫困。	きゅうぼう		【名・自动词・サ变/三类】
\\	丘陵	
\\	丘陵。	きゅうりょう		【名词】
\\	共	
\\	们,表示多数。(複数化の接尾語。) 
\\	表示谦逊的说法。(一人称の語に付き、へりくだった気持ちを表す。)	きょう		【名词】
\\	供	
\\	随从,扈从,从者,伴侣。(人の後ろにつき従って行くこと。) 
\\	同:伴	きょう		【名词】
\\	驚異	
\\	惊异,惊奇;奇事,惊人的事,不可思议的事。	きょうい		【名词】
\\	教員	
\\	教员,教师。	きょういん		【名词】
\\	教科	
\\	教授科目。	きょうか		【名词】
\\	協会	
\\	协会。由为某种目的而集聚的会员协作组织并维持的团体。(ある目的のために集まった会員が協力して組織し、維持していく団体。)	きょうかい		【名词】
\\	共感	
\\	同感,共鸣。	きょうかん		【名・自动词・サ变/三类】
\\	共学	
\\	同校。	きょうがく		【名词】
\\	協議	
\\	协议,商议,协商,磋商。	きょうぎ		【名词】
\\	教訓	
\\	教训。	きょうくん		【名・他动词・サ变/三类】
\\	境遇	
\\	境遇,处境,环境,遭遇。(その人の置かれた環境や身辺の諸事情。身の上。境涯。)	きょうぐう		【名词】
\\	強行	
\\	强行,硬干。	きょうこう		【名・他动词・サ变/三类】
\\	強硬	
\\	强硬。	きょうこう		【形容动词/ナ形容词】
\\	凶作	
\\	歉收。(ひどい不作。) 
\\	凶作の年。歉收年。	きょうさく		【名词】
\\	共産	
\\	共产	きょうさん		【名词】
\\	教材	
\\	教材。(教育目的を達成するために、児童・生徒の学習に供する素材。カリキュラムまたは単元を構成する内容そのものをさすこともある。)	きょうざい		【名词】
\\	教習	
\\	教习,训练。	きょうしゅう		【名・他动词・サ变/三类】
\\	郷愁	
\\	乡愁,怀念故乡的忧伤的心情。(他郷にあって故郷を懐かしく思う気持ち。ノスタルジア。) 
\\	怀念,思念,想念。(過去のものや遠い昔などにひかれる気持ち。)	きょうしゅう		【名词】
\\	教職	
\\	教师的职务。(児童・生徒・学生を教育指導する職務。) 
\\	教导信徒的职务。(キリスト教で信徒を教導する職務。)	きょうしょく		【名词】
\\	享受	
\\	享有,归己所有,享受。	きょうじゅ		【名・他动词・サ变/三类】
\\	興じる	
\\	同:興ずる 
\\	高兴,兴高采烈,感觉有趣,感觉愉快。	きょうじる		【自动词・一段/二类】
\\	強制	
\\	强制,强迫。	きょうせい		【名・他动词・サ变/三类】
\\	共存	
\\	共存,共处。	きょうそん		【名・自动词・サ变/三类】
\\	姉妹	
\\	姊妹。 
\\	同一系统之物。	きょうだい		【名词】
\\	協調	
\\	协调,合作。	きょうちょう		【名・自动词・サ变/三类】
\\	協定	
\\	协定,商定,拟订协议。	きょうてい		【名・他动词・サ变/三类】
\\	脅迫	
\\	胁迫,威胁,威逼,恫吓,恐吓。	きょうはく		【名・他动词・サ变/三类】
\\	共鳴	
\\	共鸣,共振,谐振。 
\\	感情产生共鸣,同情,同感。	きょうめい		【名・自动词・サ变/三类】
\\	郷里	
\\	乡里,故乡,家乡,老家。(自分の生まれ育った所。故郷。ふるさと。むらざと。いなか。)	きょうり		【名词】
\\	強烈	
\\	强烈,力量、作用、刺激等强大而猛烈。	きょうれつ		【形容动词/ナ形容词】
\\	共和	
\\	共和。(独裁と違って、二人以上の人が共同協議の上 政務などを執り行うこと。)	きょうわ		【名词】
\\	曲	
\\	窝;弯。((川や道などの)折れ曲がって入りくんだ所。) 
\\	阴暗处。(奥まった隅の所。物陰の暗い所。) 
\\	黑眼圈;阴影。(濃い色や陰の部分。陰影。) 
\\	同:隈、阿	きょく		【名词】
\\	局限	
\\	局限,限定。	きょくげん		【名词】
\\	極端	
\\	极限;极端。	きょくたん		【名・形容动词/ナ形容词】
\\	居住	
\\	居住。(住むこと。) 
\\	住址,住处。(居どころ。)	きょじゅう		【自动词・サ变/三类】
\\	拒絶	
\\	拒绝。	きょぜつ		【名・他动词・サ变/三类】
\\	拒否	
\\	拒绝,否决。	きょひ		【名・他动词・サ变/三类】
\\	許容	
\\	容许,宽容。	きょよう		【名・他动词・サ变/三类】
\\	距離	
\\	距离。间隔。 
\\	(数)距离。	きょり		【名词】
\\	寄与	
\\	贡献,有助于……。 
\\	寄与,受某事物变化的影响。	きよ		【名・自动词・サ变/三类】
\\	清らか	
\\	清澈,清洁,洁白,纯洁。	きよらか		【形容动词/ナ形容词】
\\	気楽	
\\	舒畅,舒适,舒服,舒坦;安乐,安闲,安逸,轻松,轻快。 
\\	无挂虑,无顾虑,舒心,无忧无虑『成』,坦然。	きらく		【形容动词/ナ形容词】
\\	煌びやか	
\\	光辉灿烂,灿烂夺目,华丽。	きらびやか		【形容动词/ナ形容词】
\\	切り	
\\	限度,终结,界限。(かぎり。限界。限度。) 
\\	段落。(締めくくり。物事がそこで終わりになる切れ目。区切り。ひと区切り。) 
\\	表演结束的部分。(芸能で、終わりの部分。)	きり		【名词】
\\	桐	
\\	梧桐,泡桐。(ゴマノハグサ科の落葉高木。古くから各地で植栽される。葉は長い柄がある大きな広卵形で,軟腺毛を密生。初夏,枝頂に淡紫色の花を多数円錐状につける。材は軽く狂いが少ないので笥(たんす)・琴・下駄などにする。)	きり		【名词】
\\	切り替える	
\\	更换;改接。	きりかえる		【他动词・一段/二类】
\\	気流	
\\	气流。(温度や地形の変化によって大気中に起こる空気の流れ。)	きりゅう		【名词】
\\	奇麗	
\\	美丽,漂亮,好看。(美しい。) 
\\	洁净,干净。(よごれがない。) 
\\	漂亮。(上手で立派だ。) 
\\	完全,彻底,干干净净。(残りがない。) 
\\	同:綺麗	きれい		【形容动词/ナ形容词】
\\	切れ目	
\\	断开处,裂缝,缝隙。(切れてできたあと。切れた所。) 
\\	间断,中断。(とぎれたところ。) 
\\	段落。(くぎり。段落。) 
\\	断绝时,罄尽时。(絶える時。尽きる時。はて。おわり。)	きれめ		【名词】
\\	際	
\\	时候;时机,机会。 
\\	之间;边缘。 
\\	时候;时机,机会。 
\\	之间;边缘。	きわ		【名词】
\\	木綿	
\\	木棉,木棉树的异名;棉花。	きわた		【名词】
\\	極めて	
\\	极,极其,非常。	きわめて		【副词】
\\	近眼	
\\	近视,近视眼。(屈折異常のひとつで、眼球内に入ってきた平行光線が、調節力を働かせていない状態で、網膜上の正しい位置ではなく、もっと手前に焦点を結んでしまう状態。)	きんがん		【名词】
\\	緊急	
\\	紧急,急迫。	きんきゅう		【形容动词/ナ形容词】
\\	近々	
\\	最近,不久;近距离地,直接;屡次,常常,经常	きんきん		【副词】
\\	近郊	
\\	近郊,郊区。(都市周辺の地域。) 
\\	『参考』都市を取りまく郊外全体を“郊区”といい,“郊区”はさらに“近郊”“中郊”“远郊”,または“西郊”“东郊”などに分かれる.	きんこう		【名词】
\\	均衡	
\\	均衡,平衡,平均。	きんこう		【自动词・サ变/三类】
\\	近視	
\\	近视。	きんし		【名词】
\\	禁じる	
\\	禁止,禁;控制。	きんじる		【他动词・一段/二类】
\\	禁ずる	
\\	禁止,不准,忌,戒(除)。(してはいけないとさしとめる。禁止する。) 
\\	抑制,控制。(ある気持ちを抑える。)	きんずる		【他动词・サ变/三类】
\\	勤勉	
\\	勤勉,勤劳。	きんべん		【名・形容动词/ナ形容词】
\\	勤務	
\\	就职,工作,职务。	きんむ		【名・自动词・サ变/三类】
\\	禁物	
\\	切忌。 
\\	严禁,忌讳。	きんもつ		【名词】
\\	勤労	
\\	领取报酬劳动。 
\\	勤劳,劳动。	きんろう		【自动词・サ变/三类】
\\	議案	
\\	议案。	ぎあん		【名词】
\\	戯曲	
\\	剧本;戏剧。	ぎきょく		【名词】
\\	議決	
\\	议决,表决。	ぎけつ		【名・他动词・サ变/三类】
\\	議事堂	
\\	会议厅。 
\\	国会议事堂,国会大厦。	ぎじどう		【名词】
\\	犠牲	
\\	牺牲。(天地・宗廟を祭る時に供える生きた動物。いけにえ。また、供犠(くぎ)のために殺した動物、ごく稀には植物(穀物など)。) 
\\	牺牲。(自分の意志によらず戦争・天災・事故の巻きぞえなどで生命を失ったり傷ついたりすること。) 
\\	代价。(身命を捧げて他のために尽くすこと。ある目的を達成するために、それに伴う損失を顧みないこと。)	ぎせい		【名词】
\\	偽造	
\\	伪造,假造。	ぎぞう		【名・他动词・サ变/三类】
\\	議題	
\\	议题,讨论题目。	ぎだい		【名词】
\\	技能	
\\	技能;本领。(物事を行う腕前。技量。)	ぎのう		【名词】
\\	逆転	
\\	反转,倒转,逆转。 
\\	倒过来,反过来;倒退;恶化。 
\\	(飞机的)空中翻跟头。	ぎゃくてん		【名・自他・サ变/三类】
\\	行	
\\	去,出行。去,尤指徒步旅行等。(行くこと。特に,旅行などで歩きまわること。)) 
\\	条,行。(くだり・すじ。) 
\\	行。原指乐府的乐曲,故事的一种体裁,唐朝以后多为叙事诗。(もとが楽府の楽曲のこと。古詩の一体。唐代以降は,多く叙事詩。)	ぎょう		【名词】
\\	業者	
\\	同业者。 
\\	工商业者。	ぎょうしゃ		【名词】
\\	行政	
\\	与司法、立法并行的国家作用之一。(国家作用の一つ。) 
\\	根据法律、政令等而执行的政务。(内閣以下の国の機関または公共団体が法律・政令その他法規の範囲内で行う政務。)	ぎょうせい		【名词】
\\	業績	
\\	业绩,成就。	ぎょうせき		【名词】
\\	業務	
\\	业务,工作。(日常継続して行われる職業上の仕事。)	ぎょうむ		【名词】
\\	玉	
\\	玻璃球。(玩具の一種,ガラス製の球で遊戯用・観賞用に用いる。)	ぎょく		【名词】
\\	漁船	
\\	渔船。	ぎょせん		【名词】
\\	漁村	
\\	渔村。	ぎょそん		【名词】
\\	義理	
\\	情义,情面,情分,人情,正义,情理,礼节。 
\\	情理,道理,缘由。 
\\	亲戚关系。	ぎり		【名词】
\\	疑惑	
\\	疑惑,疑心,疑虑。(本当かどうか、不正があるのではないかなどと疑いをもつこと。)	ぎわく		【名词】
\\	吟味	
\\	吟诵诗歌仔细体会其含义,玩味。 
\\	斟酌,考虑,拣选,选择。 
\\	审问,审讯。 
\\	监督。	ぎんみ		【名・他动词・サ变/三类】
\\	区	
\\	地区,区域;区;自治市。	く		【名词】
\\	食い違う	
\\	不一致,有分歧,有所不同,发生龃龉,事情没有很好地协调一致。 
\\	交错,错位。	くいちがう		【自动词・五段/一类】
\\	空腹	
\\	空腹,空肚子,空心,饿。	くうふく		【名词】
\\	区画	
\\	区划,划分一定的场所。	くかく		【名・他动词・サ变/三类】
\\	区間	
\\	区间,区段,段。	くかん		【名词】
\\	茎	
\\	茎;秆;梗。(高等植物において、植物体を支え、根から吸収した水分や養分を師部・木部を通して各部に運ぶ器官。)	くき		【名词】
\\	区切り	
\\	段落。(詩文などの句の切れ目。) 
\\	阶段。(物事のきり。)	くぎり		【名词】
\\	潜る	
\\	潜水。 
\\	穿过,钻过。 
\\	渡过。 
\\	钻空子,乘人不备。	くぐる		【自动词・五段/一类】
\\	種々	
\\	种种,样样,各种各样。(「いろいろ·さまざま」の意の雅語の表現。)	くさぐさ		【名词】
\\	草臥れる	
\\	累,疲劳,疲乏。(くたくたに疲れる。疲労する。) 
\\	用旧,使用过久,穿旧。(長く使って、みすぼらしくなる。)	くたびれる		【自动词・一段/二类】
\\	下らない	
\\	无聊的,无用的。(つまらない。価値がない。取るに足りない。)	くだらない		【连体词】
\\	件	
\\	上述;曾经提过的事物,过去谈过。	くだん		【名词】
\\	口ずさむ	
\\	吟,诵,,哼。	くちずさむ		【他动词・五段/一类】
\\	嘴	
\\	鸟嘴。	くちばし		【名词】
\\	朽ちる	
\\	衰败,衰亡。 
\\	默默无闻而终,埋没一生。 
\\	腐朽,腐烂,腐坏。	くちる		【自动词・一段/二类】
\\	屈折	
\\	弯曲,曲折。 
\\	歪曲,不正常,不自然。 
\\	屈折,折射。	くっせつ		【名・自动词・サ变/三类】
\\	くっ付く	
\\	贴着,黏在一起。(ぴったりとつく。接着する。) 
\\	从属,归属。(つき従う。属する。) 
\\	结为夫妇。(男女が親しくなり、夫婦となる。)	くっつく		【自动词・五段/一类】
\\	くっ付ける	
\\	贴着,黏在一起。(ぴったりとつく。接着する。) 
\\	从属,归属。(つき従う。属する。) 
\\	结为夫妇。(男女が親しくなり、夫婦となる。)	くっつける		【自动词・五段/一类】
\\	覆す	
\\	彻底改变,推翻。 
\\	推翻,打倒。 
\\	打翻,弄翻,翻转。	くつがえす		【他动词・五段/一类】
\\	諄い	
\\	冗长无趣的,啰唆的,唠叨的,喋喋不休的。 
\\	过浓的。	くどい		【形容词/イ形容词】
\\	国境	
\\	国境,边境,边界。	くにざかい		【名词】
\\	首飾り	
\\	项链,首饰。(宝石・貴金属などをつないだ、首にかける装飾品。ネックレス。) 
\\	同:頚飾り	くびかざり		【名词】
\\	首輪	
\\	脖圈,脖子套。 
\\	项圈,项链。	くびわ		【名词】
\\	組み合わせ	
\\	(数学)组合。 
\\	配合,组成,编组。	くみあわせ		【名词】
\\	組み合わせる	
\\	编在一起;配合,编组。	くみあわせる		【他动词・一段/二类】
\\	組み込む	
\\	编入,排入。 
\\	入伙。	くみこむ		【他动词・五段/一类】
\\	蔵	
\\	收藏,所有。(所有していること。)	くら		【名词】
\\	苦しめる	
\\	使痛苦;使为难。	くるしめる		【他动词・一段/二类】
\\	包む	
\\	包上;笼罩;隐藏;束起。	くるむ		【他动词・五段/一类】
\\	呉れる	
\\	表示主语所表现的主体给说话者做某事	くれる		【他动词・一段/二类】
\\	玄人	
\\	内行,行家,里手,专家。 
\\	妓女,艺妓。	くろうと		【名词】
\\	黒字	
\\	黑色的字。(黒い色で書いた字。) 
\\	盈余,赚钱。(収入が支出よりも多いこと。利益が出ること。また,利益。)	くろじ		【名词】
\\	愚痴	
\\	牢骚,抱怨,埋怨。	ぐち		【名词】
\\	軍艦	
\\	军舰。	ぐんかん		【名词】
\\	群集	
\\	【自动·三类】 
\\	群集,聚集,人群,群众,群落。 
\\	群集,聚集。(人が多く群がり集まること。また、その集まった人々。ぐんじゅ。むれ。) 
\\	人群,群。(人々が大ぜい群がり集まること。また、その人々。)	ぐんしゅう		【名词】
\\	軍事	
\\	军事,有关军队,军备,战争的事项。	ぐんじ		【名词】
\\	軍備	
\\	军备,军事设备。 
\\	战争准备,备战。	ぐんび		【名词】
\\	刑	
\\	刑罚。(国家が犯罪者を懲らしめるために行う制裁。)	けい		【名词】
\\	系	
\\	系统,系列;方面,部门;系。	けい		【名词】
\\	経過	
\\	时间流逝,过去。 
\\	经过,过程。	けいか		【自动词・サ变/三类】
\\	軽快	
\\	动作轻快。 
\\	心情轻松愉快,舒畅,高兴,痛快。 
\\	病情好转,病愈。	けいかい		【名・形容动词/ナ形容词】
\\	警戒	
\\	小心谨慎,警戒,防范。	けいかい		【名・他动词・サ变/三类】
\\	計器	
\\	【比較】 
\\	仪器”と“仪表”: 
\\	仪表”は温度・気圧・電気量・血圧などを測定する文字盤・針のある計器。“仪器”は実験・計量・観測・作図などに用いる精密度の高い器具や装置。	けいき		【名词】
\\	契機	
\\	契机,转机,动机,起因(ものごとをかえたり、発展させたりする原因になるもの)。	けいき		【名词】
\\	敬具	
\\	敬启,谨上,谨启,敬上。	けいぐ		【名词】
\\	軽減	
\\	减轻。	けいげん		【名・自他・サ变/三类】
\\	掲載	
\\	刊登,登载。	けいさい		【名・他动词・サ变/三类】
\\	傾斜	
\\	倾斜,或倾斜状态。	けいしゃ		【名・自动词・サ变/三类】
\\	形成	
\\	形成,组成。	けいせい		【名・他动词・サ变/三类】
\\	形勢	
\\	形势,局势。(情勢。なりゆき。様子。また,勢力の優劣の状態。)	けいせい		【名词】
\\	軽率	
\\	轻率,草率,疏忽,马虎。	けいそつ		【形容动词/ナ形容词】
\\	携帯	
\\	携带。	けいたい		【名・他动词・サ变/三类】
\\	形態	
\\	形态;样子;形状(物のかたち。また,組織的に組み立てられたものの,外に表れているかたち。ありさま)。	けいたい		【名词】
\\	刑罰	
\\	刑罚。	けいばつ		【名词】
\\	経費	
\\	经费;开销;开支,费用。	けいひ		【名词】
\\	警部	
\\	警部。	けいぶ		【名词】
\\	軽蔑	
\\	轻蔑,轻侮,侮蔑,蔑视,看不起,轻视,藐视。	けいべつ		【名・他动词・サ变/三类】
\\	経歴	
\\	经历,履历,来历,经过。(今まで経験してきた仕事・身分・地位・学業などの事柄。履歴。)	けいれき		【名词】
\\	経路	
\\	路径;途径;路线。(通過する道すじ。また、たどるべき手順、筋道。方途・方法。)	けいろ		【名词】
\\	汚す	
\\	弄脏,污染。 
\\	玷辱,损伤,败坏。 
\\	奸污,污辱,凌辱。 
\\	忝列,忝居。	けがす		【他动词・五段/一类】
\\	汚らわしい	
\\	令人厌恶,卑鄙,不干净,肮脏。 
\\	(疫病等)不净,污秽。	けがらわしい		【形容词/イ形容词】
\\	汚れ	
\\	污垢;污渍;脏。(よごれること。よごれた痕。) 
\\	月经。(月経。)	けがれ		【名词】
\\	汚れる	
\\	污染。 
\\	失去神圣性。 
\\	失去贞操。 
\\	在服丧期、产期、经期等忌讳朝拜神佛。	けがれる		【自动词・一段/二类】
\\	獣	
\\	野兽,家畜。	けだもの		【名词】
\\	吝嗇	
\\	【形动】 
\\	吝啬。(むやみに金品を惜しむこと。また、そういう人や、そのさま。)	けち		【名词】
\\	結核	
\\	结核,结核病。(結核菌の感染によって起こる慢性の感染症。感染症予防法の2類感染症の一。) 
\\	〈矿〉结核,凝岩。(堆積物や堆積岩中に生じた塊で周囲より硬いもの。)	けっかく		【名词】
\\	血管	
\\	血管。	けっかん		【名词】
\\	決行	
\\	坚决进行。	けっこう		【名・他动词・サ变/三类】
\\	決算	
\\	决算,结算,清帐。	けっさん		【名・自他・サ变/三类】
\\	決勝	
\\	决赛,决胜负。(最終的に勝負を決めること。)	けっしょう		【名词】
\\	結晶	
\\	原子分子等组成的固体结晶。 
\\	(事物的)成果,结晶。	けっしょう		【自动词・サ变/三类】
\\	結成	
\\	结成,组成,组建,成立组织或团体。	けっせい		【名・他动词・サ变/三类】
\\	結束	
\\	捆束,捆扎。 
\\	团结。	けっそく		【名・他动词・サ变/三类】
\\	決意	
\\	决心,决意;下决心。	けつい		【名・自他・サ变/三类】
\\	決議	
\\	决议,决定,议决,会议上通过事项或意见,亦指已经决定的事项。	けつぎ		【名・他动词・サ变/三类】
\\	結合	
\\	结合。	けつごう		【名・自他・サ变/三类】
\\	決断	
\\	【自他サ】 
\\	决断,果断,当机立断。(思いきってはっきりきめること。)	けつだん		【名词】
\\	欠乏	
\\	【自动・三类】 
\\	缺乏,缺少,不足(必要な物が不足すること)。	けつぼう		【名词】
\\	蹴飛ばす	
\\	踢开,踢倒。 
\\	拒绝。 
\\	同:蹴っ飛ばす	けとばす		【他动词・五段/一类】
\\	貶す	
\\	贬低,毁谤。	けなす		【他动词・五段/一类】
\\	煙たい	
\\	烟气熏人,呛人,烟雾弥漫。 
\\	使人拘束,令人发怵。难以无拘无束地接近,感到局促不安。	けむたい		【形容词/イ形容词】
\\	煙る	
\\	冒烟;模糊不清。	けむる		【自动词・五段/一类】
\\	家来	
\\	家臣,臣下。(主君に忠誠を誓って仕える人。家臣。) 
\\	仆众。(家に召し使う者。従者。家人。)	けらい		【名词】
\\	圏	
\\	区域,范围。(限られた区域、範囲。まわりをかこったもの。)	けん		【名词】
\\	権	
\\	权力,权限。(権利。) 
\\	图谋,计策,谋划。(はかりごと。)	けん		【名词】
\\	権威	
\\	权势,威权,势力。 
\\	权威;威信。	けんい		【名词】
\\	兼業	
\\	兼任,副业。	けんぎょう		【名・他动词・サ变/三类】
\\	権限	
\\	权利。 
\\	权限。	けんげん		【名词】
\\	健在	
\\	健在。 
\\	健壮,不衰老。	けんざい		【名・形容动词/ナ形容词】
\\	懸賞	
\\	悬赏,赏品,招募。(正解を出した人、優秀な作品を出した人、物を探し出した人などに与えるという条件で出す、賞金や賞品。また、そのような賞金・賞品をかけること。)	けんしょう		【名词】
\\	検事	
\\	〈法〉检察官。(検察官の官名の一つ。)	けんじ		【名词】
\\	健全	
\\	健全,坚实,稳固。状态、见解等稳妥可靠,可以让人放心的状态。 
\\	健康,健全。身体和精神上没有缺陷,呈健康状态。	けんぜん		【形容动词/ナ形容词】
\\	見地	
\\	见地,观点,立场。	けんち		【名词】
\\	賢明	
\\	贤明,英明,高明。	けんめい		【名・形容动词/ナ形容词】
\\	倹約	
\\	节约,节省,俭省,俭约。	けんやく		【名・他动词・サ变/三类】
\\	兼用	
\\	一物两用。 
\\	为达到同一目的二者都可兼用。	けんよう		【名・他动词・サ变/三类】
\\	権力	
\\	权力。(他人をおさえつけ支配する力。支配者が被支配者に加える強制力。)	けんりょく		【名词】
\\	劇団	
\\	剧团,剧院。(演劇の上演などを目的として組織された団体。)	げきだん		【名词】
\\	激励	
\\	激励,鼓励,鼓舞,鞭策。	げきれい		【名・他动词・サ变/三类】
\\	月謝	
\\	月酬,作为受教的谢礼,每月支付的酬金。学费。	げっしゃ		【名词】
\\	月賦	
\\	按月分期付款。月供。(代金などの全額を一時に払わないで、月々に割り当てて分割払いにすること。月賦払い。また、その月々の支払い金。)	げっぷ		【名词】
\\	下痢	
\\	腹泻,泻肚,拉肚子。	げり		【自动词・サ变/三类】
\\	原形	
\\	原形,旧观。(もとのかたち。変化する前のかたち。) 
\\	同:元形	げんけい		【名词】
\\	現行	
\\	现行,正在实行。	げんこう		【名词】
\\	原作	
\\	底本。 
\\	原作,原著,原文。	げんさく		【名词】
\\	原子	
\\	原子。(物質を構成する基本的た粒子で化学元素としての特性を失わない最小の微粒子。)	げんし		【名词】
\\	元首	
\\	元首。	げんしゅ		【名词】
\\	原書	
\\	原文书,外文书。 
\\	原书,原著。	げんしょ		【名词】
\\	減少	
\\	减少。	げんしょう		【名・自他・サ变/三类】
\\	現場	
\\	现场 
\\	现场,工地。	げんじょう		【名词】
\\	元素	
\\	元素。	げんそ		【名词】
\\	原則	
\\	原则。	げんそく		【名词】
\\	現像	
\\	显影,冲洗。(写真で、露出されたフィルム・印画紙を薬品で処理して、写された像を目に見えるようにすること。また、その処理。)	げんぞう		【名・他动词・サ变/三类】
\\	現地	
\\	当地,现住地方。 
\\	现场。	げんち		【名词】
\\	限定	
\\	限定,限制。	げんてい		【名・他动词・サ变/三类】
\\	原点	
\\	座标的交点,原点。 
\\	出发点;根据;根源。 
\\	基准点,原点。	げんてん		【名词】
\\	原典	
\\	原著。	げんてん		【名词】
\\	減点	
\\	扣分,减少的分数。	げんてん		【名・他动词・サ变/三类】
\\	原爆	
\\	原子弹。	げんばく		【名词】
\\	原文	
\\	原文。	げんぶん		【名词】
\\	厳密	
\\	严密,周密,严格,细密周到地严格进行。	げんみつ		【形容动词/ナ形容词】
\\	原油	
\\	低硫磺原油。(低硫黄原油。 硫黄分が重量比で1パーセント以下の原油。
\\	原油。)	げんゆ		【名词】
\\	言論	
\\	言论。	げんろん		【名词】
\\	故	
\\	已故的,加在人名等之前,表示该人已死亡;以前的,加在官职前表示以前的官职。	こ		【接头/接尾】
\\	戸	
\\	户。(家や世帯の数を数える語。)	こ		【接头/接尾】
\\	児	
\\	人和动物生出来的崽。(人間や動物から,生まれ出るもの。特に,生まれ出て間もないもの。) 
\\	年幼的孩子。(まだ一人前になっていない人間。年少の男女。) 
\\	(親しみの気持ちで若い女性をいう語。芸子をさす場合もある。)	こ		【名词】
\\	恋する	
\\	恋爱。(恋をする。)	こいする		【自他・サ变/三类】
\\	溝	
\\	沟,水沟;下水道。	こう		【名词】
\\	校	
\\	学校。(学びや。学校。) 
\\	校对(校正。) 
\\	计算校对的量词。(助数詞。校正の回数を数えるのに用いる。)	こう		【接头/接尾】
\\	好意	
\\	好意,美意,善意(相手のことが気に入り、すきになる気持ち)。	こうい		【名词】
\\	行為	
\\	行为,行动,举动。(おこない。) 
\\	不法行為。不法行为。	こうい		【名词】
\\	交易	
\\	交易,贸易。 
\\	相互交换。	こうえき		【名・自动词・サ变/三类】
\\	公演	
\\	公演。	こうえん		【名・自他・サ变/三类】
\\	後悔	
\\	后悔,懊悔。	こうかい		【名・他动词・サ变/三类】
\\	公開	
\\	公开;(设施等)开放。(电影等)上映、播出。	こうかい		【名・他动词・サ变/三类】
\\	航海	
\\	航海,航行。	こうかい		【名・自动词・サ变/三类】
\\	工学	
\\	工学。	こうがく		【名词】
\\	皇居	
\\	皇宫。天皇平常居住之所,皇宫,宫城。	こうきょ		【名词】
\\	好況	
\\	繁荣,景气,兴盛。	こうきょう		【名词】
\\	鉱業	
\\	矿业。 
\\	同:撂業	こうぎょう		【名词】
\\	興業	
\\	振兴事业,新兴事业。(新たに事業をおこすこと。)	こうぎょう		【名词】
\\	高原	
\\	高原。	こうげん		【名词】
\\	考古学	
\\	考古学。(古い時代の生活文化を、出土した遺物や発掘した遺跡から確認·追究しようとする科学。狭義では、有史以前の古代人の生活環境や古生物の探求を指す。)	こうこがく		【名词】
\\	交互	
\\	互相,交替。	こうご		【名词・副词】
\\	工作	
\\	制作。 
\\	修理,工程。 
\\	手工。 
\\	活动,工作。	こうさく		【名・他动词・サ变/三类】
\\	耕作	
\\	耕种。	こうさく		【他动词・サ变/三类】
\\	鉱山	
\\	矿山。	こうざん		【名词】
\\	講習	
\\	讲习,学习。	こうしゅう		【名・他动词・サ变/三类】
\\	交渉	
\\	关系;联系。 
\\	谈判;交涉。	こうしょう		【名・他动词・サ变/三类】
\\	高尚	
\\	高尚,高深。	こうしょう		【名・形容动词/ナ形容词】
\\	行進	
\\	(列队)行进;游行。	こうしん		【名・自动词・サ变/三类】
\\	香辛料	
\\	香辣调味料;佐料。	こうしんりょう		【名词】
\\	口述	
\\	口述。	こうじゅつ		【名・他动词・サ变/三类】
\\	控除	
\\	扣除。	こうじょ		【名・他动词・サ变/三类】
\\	向上	
\\	向上,提高;进步。	こうじょう		【名词】
\\	降水	
\\	降水。	こうすい		【名词】
\\	洪水	
\\	洪水,涨大水。 
\\	洪流。	こうずい		【名词】
\\	公然	
\\	公然;公开。	こうぜん		【名・形容动词/ナ形容词】
\\	抗争	
\\	反抗,对抗。	こうそう		【名・自动词・サ变/三类】
\\	構想	
\\	设想。构思。	こうそう		【名・他动词・サ变/三类】
\\	拘束	
\\	拘束;束缚;约束;限制。	こうそく		【名・他动词・サ变/三类】
\\	後退	
\\	后退,倒退;向后方退却;后退到。	こうたい		【自动词・サ变/三类】
\\	光沢	
\\	光泽。	こうたく		【名词】
\\	公団	
\\	公营,公团。	こうだん		【名词】
\\	好調	
\\	顺利,情况良好。	こうちょう		【名・形容动词/ナ形容词】
\\	口頭	
\\	口头,不用文书而用话语形式表达。	こうとう		【名词】
\\	講読	
\\	讲解。	こうどく		【名・他动词・サ变/三类】
\\	購読	
\\	订阅,购阅。	こうどく		【他动词・サ变/三类】
\\	購入	
\\	购买,买进,购入;购置;采购。	こうにゅう		【名・他动词・サ变/三类】
\\	公認	
\\	公认,正式承认。	こうにん		【名・他动词・サ变/三类】
\\	光熱費	
\\	照明与瓦斯等燃料费。	こうねつひ		【名词】
\\	荒廃	
\\	荒芜,荒废。	こうはい		【自动词・サ变/三类】
\\	購買	
\\	买,购买。	こうばい		【名・他动词・サ变/三类】
\\	好評	
\\	好评,称赞。	こうひょう		【名・形容动词/ナ形容词】
\\	交付	
\\	交付,交给。	こうふ		【名・他动词・サ变/三类】
\\	降伏	
\\	投降,降服。(降り伏すること。敗戦を認め敵に服従すること。降参すること。) 
\\	同:降服	こうふく		【自动词・サ变/三类】
\\	興奮	
\\	兴奋,激动。 
\\	刺激。 
\\	心理学的兴奋。(精神活動がある限度を超えること。)	こうふん		【自动词・サ变/三类】
\\	公募	
\\	公开招募,公开募集,公开征集。	こうぼ		【名・他动词・サ变/三类】
\\	巧妙	
\\	巧妙,做事的方法巧妙。	こうみょう		【形容动词/ナ形容词】
\\	公用	
\\	公用。 
\\	公事,公务。	こうよう		【名词】
\\	小売	
\\	零售。	こうり		【名词】
\\	効率	
\\	效率。	こうりつ		【名词】
\\	公立	
\\	公立。	こうりつ		【名词】
\\	小柄	
\\	身材短小。 
\\	小花样,碎花纹。	こがら		【名・形容动词/ナ形容词】
\\	小切手	
\\	支票。	こぎって		【名词】
\\	国産	
\\	国产。(自国で生産・産出すること。また、その産物。その国の産物。)	こくさん		【名词】
\\	国土	
\\	乡土。 
\\	土地,大地。 
\\	国土,领土,国家的土地。	こくど		【名词】
\\	告白	
\\	告白,表白,说出心中的秘密。 
\\	基督教中表示公开自己的信仰。或者是向神坦白自己的罪行等。 
\\	广而告之,自白,告诉别人。	こくはく		【名・他动词・サ变/三类】
\\	国防	
\\	国防。(外敵の侵略に対して国家を防衛すること。)	こくぼう		【名词】
\\	国有	
\\	国有。	こくゆう		【名词】
\\	国連	
\\	联合国的略语。	こくれん		【名词】
\\	漕ぐ	
\\	划,摇;蹬。	こぐ		【他动词・五段/一类】
\\	焦げ茶	
\\	浓茶色,深棕色,古铜色。	こげちゃ		【名词】
\\	個々	
\\	每个。各个,个别,一个一个,各自。	ここ		【名词】
\\	箇箇	
\\	每个;各自 
\\	個個;箇箇 
\\	ここ		【名词】
\\	心地	
\\	感觉,心情。	ここち		【接头/接尾】
\\	心得	
\\	经验,知识,心得,体会。 
\\	规则。 
\\	代理。暂代。	こころえ		【名词】
\\	心掛け	
\\	留心,注意;用心,努力;为人,人品,作风。(良い·悪いの点から見た日常の心の持ち方。)	こころがけ		【名词】
\\	心掛ける	
\\	留心,注意,记在心里。	こころがける		【他动词・一段/二类】
\\	志	
\\	志。纪传体史书中记述各部类历史称呼的部分。(紀伝体の史書で、部門史の部分の称を記述した部分。) 
\\	志。纪传体史书中记述各部类历史称呼的部分。(紀伝体の史書で、部門史の部分の称を記述した部分。)	こころざし		【名词】
\\	志す	
\\	志,立志,志愿。兴致所向,向心中立下的目标前进。(気持ちがそちらに向く。心の中で目標を定める。心中に決めた目標に向かって進む。)	こころざす		【自他・五段/一类】
\\	心強い	
\\	胆壮,有把握,受鼓舞,放心。 
\\	刚毅,刚强。	こころづよい		【形容词/イ形容词】
\\	心細い	
\\	心中不安的,心中没底的,觉得无依靠的,没把握的。	こころぼそい		【形容词/イ形容词】
\\	試み	
\\	尝试,试一试,试试看。	こころみ		【名词】
\\	試みる	
\\	试验一下。	こころみる		【他动词・一段/二类】
\\	快い	
\\	高兴,愉快,爽快。	こころよい		【形容词/イ形容词】
\\	凝らす	
\\	使凝固,使凝结。 
\\	使集中。 
\\	一心钻研。	こごらす		【他动词・五段/一类】
\\	凝る	
\\	凝固,凝集。 
\\	酸痛。 
\\	热衷,狂热,专心致志,入迷。 
\\	讲究,下功夫,精致。	こごる		【自动词・五段/一类】
\\	拵える	
\\	制造,做。 
\\	筹款,凑钱。 
\\	化妆,打扮。 
\\	捏造,虚构,假装。	こしらえる		【他动词・一段/二类】
\\	孤児	
\\	孤儿 。(両親のいない子。みなしご。) 
\\	比喻意义,孤独的人。((比喩的に)仲間のないこと。孤立した存在。)	こじ		【名词】
\\	拗れる	
\\	复杂化,恶化,缠绵不愈。 
\\	别扭,执拗。	こじれる		【自动词・一段/二类】
\\	故人	
\\	故人,旧友。(古い友人。) 
\\	死者,亡人。(死んだ人。)	こじん		【名词】
\\	梢	
\\	(意为树的尽头)树梢	こずえ		【名词】
\\	個性	
\\	个性,具有与他人不同的特有气质。 
\\	个别性,个体特有的性质。	こせい		【名词】
\\	戸籍	
\\	户口,户籍。	こせき		【名词】
\\	小銭	
\\	零钱,零用钱;少量资金。(小額の金。細かい金。)	こぜに		【名词】
\\	固体	
\\	固体。物体的三种形态之一。具有一定形状的物质。	こたい		【名词】
\\	答え	
\\	回答,答复,答应;解答,答案。	こたえ		【名词】
\\	堪える	
\\	耐,抗,经受,承受;忍耐,忍受;坚持住。 
\\	值得,有做的价值。 
\\	胜任,能担当。	こたえる		【自动词・一段/二类】
\\	古代	
\\	古代。(古い時代。)	こだい		【名词】
\\	誇張	
\\	夸张,夸大,添枝加叶。	こちょう		【名・他动词・サ变/三类】
\\	滑稽	
\\	滑稽,可笑。 
\\	诙谐,戏谑。	こっけい		【名・形容动词/ナ形容词】
\\	国交	
\\	建交,邦交。	こっこう		【名词】
\\	骨董品	
\\	古董古玩,古董。 
\\	老古董。年纪大没用的人。	こっとうひん		【名词】
\\	骨	
\\	麻烦,辛苦的,困难的。	こつ		【形容动词/ナ形容词】
\\	固定	
\\	固定;不从一定的状态发生变化。 
\\	固定;坚持不从一个场所移动。	こてい		【名・自他・サ变/三类】
\\	事柄	
\\	事情,事体;事态。	ことがら		【名词】
\\	悉く	
\\	所有;一切(同のこらず) 
\\	悉く;尽く 
\\	ことごとく		【副词】
\\	言伝	
\\	传话,捎口信。(伝えたい言葉を他の人に取り次いでもらうこと。また、その言葉。伝言。ことづけ。) 
\\	传说,传闻。(間接に人から伝え聞くこと。また、伝え聞いた話。伝聞。)	ことづて		【名词】
\\	殊に	
\\	特别,格外,分外,尤其。(その物事の程度が、同種の他のものとひどくかけ離れていることを表わす。)	ことに		【副词】
\\	事によると	
\\	说不定,可能。(時と場合による。)	ことによると		【惯用句】
\\	孤独	
\\	孤儿,单身汉。	こどく		【名词】
\\	粉々	
\\	把东西弄得粉碎,碎得极细小的样子。粉碎的,稀巴烂。 
\\	同:粉微塵、木っ端微塵	こなごな		【形容动词/ナ形容词】
\\	此の	
\\	这;这个	この		【连体词】
\\	この間	
\\	这个期间。	このあいだ		【副词】
\\	この頃	
\\	近来,这些天来,最近时期;现在。(話し手の当面している現在。)	このごろ		【副词】
\\	好ましい	
\\	令人喜欢的,感觉良好的。 
\\	令人满意的。	このましい		【形容词/イ形容词】
\\	個別	
\\	个别(ひとりひとり、または、ひとつひとつを別別にあつかうこと)。	こべつ		【名词】
\\	零す	
\\	洒,撒,泼,漏,掉,落。(不注意から器を傾けたりして,中の液体・粉末・粒状の物を外に出してしまう。) 
\\	发牢骚,抱怨,鸣不平。(不平・愚痴などを言う。ぼやく。) 
\\	同:溢す	こぼす		【他动词・五段/一类】
\\	零れる	
\\	洒,洒落。 
\\	溢出,流出。 
\\	流露,显露,表露。漾溢出。 
\\	闪现,显现。瞬间出现。微微露出。 
\\	谢,凋。	こぼれる		【自动词・一段/二类】
\\	細やか	
\\	浓厚,细腻,深厚。 
\\	细致,入微。 
\\	意味深长。	こまやか		【形容动词/ナ形容词】
\\	混む	
\\	拥挤,混乱。(人や物がいっぱい集まる。混雑する。)	こむ		【自动词・五段/一类】
\\	込める	
\\	装填。 
\\	包括在内,计算在内。 
\\	集中精力。	こめる		【他动词・一段/二类】
\\	篭る	
\\	闭门不出。 
\\	充满。 
\\	包含,蕴含。 
\\	闭关。 
\\	闷在里面。 
\\	同:籠る、隠る、籠もる	こもる		【自动词・五段/一类】
\\	固有	
\\	固有,本来就有的。 
\\	特有,仅某事物具有,天生即有。	こゆう		【名・形容动词/ナ形容词】
\\	雇用	
\\	雇佣,雇用,就业,就业岗位。	こよう		【名・他动词・サ变/三类】
\\	暦	
\\	历法。 
\\	日历。	こよみ		【名词】
\\	孤立	
\\	孤立。脱离其他事物而单独存在。 
\\	孤立,无依靠,独自存在。毫无援助。	こりつ		【名・自动词・サ变/三类】
\\	懲りる	
\\	因为吃过苦头不敢再尝试,惩前毖后。	こりる		【自动词・一段/二类】
\\	此れ	
\\	这;此;这个人;此人;现在;此时	これ		【代词】
\\	魂	
\\	精神;精力;心魂。 
\\	魂,灵魂,魂魄。	こん		【名词】
\\	根気	
\\	耐性,耐心,毅力,精力。	こんき		【名词】
\\	根拠	
\\	根据,依据。	こんきょ		【名词】
\\	混血	
\\	混血。	こんけつ		【名・自动词・サ变/三类】
\\	昆虫	
\\	昆虫。昆虫纲节肢动物的总称。	こんちゅう		【名词】
\\	根底	
\\	根底,基础。	こんてい		【名词】
\\	混同	
\\	混同,混为一谈。将必须分清的事物作同样事物对待。 
\\	合并。相对立的两种法律上的地位归于同一人,例如债权人和债务人变为同一人等。成为物权、债权全都消灭的原因。	こんどう		【名・自他・サ变/三类】
\\	今日は	
\\	你好。	こんにちは		【惯用句】
\\	今晩は	
\\	晚上好,你好。(夜、人に会ったり、他家を訪問した際などにいうあいさつの語。) 
\\	注:“晚上好”はテレビ・ラジオ・教室・演壇などで大勢に呼びかけるときに用いる。個人同士のあいさつでは、夜でも“你好”を使うことが多い。	こんばんは		【感叹词】
\\	根本	
\\	根本,根源。	こんぽん		【名词】
\\	御	
\\	增添尊敬,礼貌之意;表示敬意。	ご		【接头/接尾】
\\	語彙	
\\	语汇。	ごい		【名词】
\\	濠	
\\	护城河;沟,渠(同みぞ) 
\\	堀;濠 
\\	ごう		【名词】
\\	業	
\\	事情,事业,工作;行为。	ごう		【名词】
\\	号	
\\	号,别名。(学者、画家などの本名のほかに用いる雅名。) 
\\	期,号。(番号、順序を表す語。) 
\\	号。(乗り物の愛称。)	ごう		【名词】
\\	合意	
\\	同意,双方意见一致。	ごうい		【自动词・サ变/三类】
\\	合議	
\\	协议,协商,集议。	ごうぎ		【名・自他・サ变/三类】
\\	強気	
\\	刚强;顽强;激烈;果断;了不起;漂亮。 
\\	同:豪儀、豪気	ごうぎ		【形容动词/ナ形容词】
\\	護衛	
\\	护卫,保卫,警卫。	ごえい		【名・他动词・サ变/三类】
\\	語句	
\\	词句,词;语句,语。(語と句。文章を組み立てている、ひとまとまりの言葉。)	ごく		【名词】
\\	極楽	
\\	极乐世界,天堂。 
\\	安乐无忧的处境,天堂。	ごくらく		【名词】
\\	ご苦労様	
\\	您辛苦了。	ごくろうさま		【惯用句】
\\	語源	
\\	语源;词源。(ある単語の、もとの形や意味。)	ごげん		【名词】
\\	誤差	
\\	〈数・測〉误差。 
\\	偏差。	ごさ		【名词】
\\	ご馳走	
\\	款待,宴请;好吃的食物,高档的酒菜、酒席。	ごちそう		【名词】
\\	毎	
\\	每。(名詞に付いて,それぞれの,そのたびごとの,の意を表す。) 
\\	每。(名詞に付いて,それぞれの,そのたびごとの,の意を表す。)	ごと		【接续词】
\\	碁盤	
\\	棋盘。(碁を打つのに用いる盤。方形で4脚、表面に縦横各19条の罫を引いて、361の目を作ったもの。)	ごばん		【名词】
\\	誤魔化す	
\\	欺骗,欺瞒,蒙混,愚弄〔だます〕。 
\\	蒙蔽,掩盖,搪塞,敷衍〔とりつくろう)。 
\\	弄虚作假,捣鬼;舞弊;侵吞〔いんちきをする)。	ごまかす		【他动词・五段/一类】
\\	佐	
\\	佐,日本地名。(茨城県つくば市にある地区。)	さ		【名词】
\\	差異	
\\	差异;差别。(違い。隔たり。) 
\\	同:差違	さい		【名词】
\\	歳	
\\	岁,年岁。(助数詞。年齢・年数を数えるのに用いる。) 
\\	⇒ねんれい,とし(年) 
\\	岁,年岁。(助数詞。年齢・年数を数えるのに用いる。) 
\\	⇒ねんれい,とし(年)	さい		【接头/接尾】
\\	再会	
\\	再会,再见,重新相逢。(長く別れ別れになっていた人どうしが、再びめぐりあうこと。)	さいかい		【自动词・サ变/三类】
\\	災害	
\\	灾害。(異常な自然現象や人為的原因によって、人間の社会生活や人命に受ける被害。)	さいがい		【名词】
\\	細菌	
\\	细菌。单细胞微生物中没有核膜的原核生物的一大类。呈球状、杆状、螺旋状等,不具有叶绿体、线粒体等。原则上以两等分分裂繁殖。虽然亦存在对动植物带有病原性的细菌,但在生态系统中,对自然界的物质循环起着广泛而重要的作用。(単細胞の微生物で、核膜のない原核生物の一群。球状・桿状・螺旋(らせん)状などを呈し、葉緑体・ミトコンドリアなどをもたない。原則として二個に分裂してふえる。動植物に対して病原性をもつものもあるが、広く生態系の中にあって物質循環に重要な役割を果たしている。分裂菌類。バクテリア。)	さいきん		【名词】
\\	細工	
\\	手工艺,工艺品。 
\\	想办法,耍花招。	さいく		【名・他动词・サ变/三类】
\\	採掘	
\\	开采,采矿。	さいくつ		【名・他动词・サ变/三类】
\\	採決	
\\	表决。	さいけつ		【名・他动词・サ变/三类】
\\	再建	
\\	重新建筑,重新建造。重建失去了的建筑物。 
\\	重新建设;重新建立。重建衰败的团体、组织、文明等。	さいけん		【名・他动词・サ变/三类】
\\	再現	
\\	再现,再次出现,重新出现。	さいげん		【名・自他・サ变/三类】
\\	採算	
\\	(收支的)核算。(計算上、収支のひきあうこと。原価・諸費・利潤を加えて販売代価を算定すること。)	さいさん		【名词】
\\	採集	
\\	采集,收集。	さいしゅう		【名・他动词・サ变/三类】
\\	再生	
\\	再生,重生,死而复生,再造,重新给予生命。 
\\	新生,改造,重新作人。 
\\	再生,更生,重制。 
\\	再生;重放,放录音,播放。	さいせい		【名・自他・サ变/三类】
\\	最善	
\\	最善,最好。(行うことができる範囲の中で最も良いこと。) 
\\	全力。(できる限り。ベスト。)	さいぜん		【名词】
\\	採択	
\\	选择,选定。 
\\	采纳;通过。	さいたく		【名・他动词・サ变/三类】
\\	再発	
\\	(疾病)复发,重新发作,又发生,再发生。(同じ病気や事故などがもう一度起こること。)	さいはつ		【自动词・サ变/三类】
\\	栽培	
\\	栽培,种植。(植え育てること。)	さいばい		【名・他动词・サ变/三类】
\\	細胞	
\\	基层组织;组成分子,成员。 
\\	细胞。	さいぼう		【名词】
\\	採用	
\\	录用,采用。	さいよう		【名・他动词・サ变/三类】
\\	遮る	
\\	遮蔽;遮断,阻挡。	さえぎる		【他动词・五段/一类】
\\	冴える	
\\	寒冷,寒凉,冷峭,清寒料峭。 
\\	清澈,鲜明。 
\\	清爽,清醒。 
\\	清晰,灵敏,精巧,纯熟。 
\\	挺棒。	さえる		【自动词・一段/二类】
\\	竿	
\\	竿,竹竿,竿子。竹子去掉枝叶后做成的细长杆。(枝葉を取り去って作った竹の細長い棒。) 
\\	钓竿。(釣り竿。) 
\\	船篙。支撑岸边或水底,使船前进的长竿。(舟をこぐ道具。岸辺や水底につっぱって舟を進ませるための長い棒。) 
\\	杆儿。从三味线音箱延伸出的上面张弦的长柄,亦指三味线。(三味線の胴から上の、糸を張る長い柄。また、三味線。) 
\\	标杆。秤杆,杆秤。(測量や建築に使う間竿(けんざお)。) 
\\	杆。计数挂在竿上旗子的量词。(竿に付けた旗を数えるのに用いる。) 
\\	抬,个。计数箱柜等的量词。(箪笥(たんす)・長持などを数えるのに用いる。) 
\\	根,条,块。计数羊羹及与其相似的细长形点心等的量词。(羊羹(ようかん)やそれに類した細長い菓子などを数えるのに用いる。) 
\\	杆。计数晒在竿子上的洗濯物的单位。(竿に干した洗濯物などを数えるのに用いる。)	さお		【接头/接尾】
\\	栄える	
\\	靓丽,亮丽;显得美丽而漂亮。	さかえる		【自动词・一段/二类】
\\	杯	
\\	杯。 
\\	酒杯。 
\\	碗,匙,杯,桶,船只。	さかずき		【接头/接尾】
\\	逆立ち	
\\	倒立,拿大顶,竖蜻蜒。 
\\	颠倒。 
\\	竭尽全力。	さかだち		【自动词・サ变/三类】
\\	盛る	
\\	旺,旺盛。 
\\	繁盛,兴隆。 
\\	发情交尾。	さかる		【自动词・五段/一类】
\\	差額	
\\	差额,两笔金额之差,扣除后的金额。	さがく		【名词】
\\	下がる	
\\	下降,降落。 
\\	悬垂,下垂。 
\\	降价,降低。 
\\	降温。 
\\	(功能、本领)退步,衰退。 
\\	推移,时代前进。 
\\	向后倒退,后退。 
\\	(从地位高的人的身旁)退出(下)。 
\\	放学,下班,自学校、机关、工作单位等处回家。 
\\	(政府机关等)发给,发下。 
\\	(日本京都指)往南去。	さがる		【自动词・五段/一类】
\\	一昨昨日	
\\	大前天。	さきおととい		【副词】
\\	先に	
\\	以前,以往。	さきに		【副词】
\\	詐欺	
\\	诈骗,欺诈,欺骗。 
\\	诈骗,欺诈行为,法律上指欺骗他人使其错误地交出财物而自己得到财产利益的行为。	さぎ		【名词】
\\	作	
\\	著作,作品。(作品。) 
\\	年收,收成。(作柄。)	さく		【名词】
\\	策	
\\	计策,策略,方案
\\	方略。(はかりごと。文字を記した竹札。くじ。占いの筮。官吏登用試験の問題。政教に関する謀を述べる漢文の一体。永字八法の一つ。) 
\\	策,谋略。计谋。(計略。ばかりごと。) 
\\	对策,对事物、事件的处理方法、手段。(物事や事件に対して行う処置・手段。)	さく		【名词】
\\	柵	
\\	栅栏。	さく		【名词】
\\	削減	
\\	削减;缩减,减去。	さくげん		【名・自他・サ变/三类】
\\	錯誤	
\\	不相符合,错谬。 
\\	错误。	さくご		【名词】
\\	作戦	
\\	作战,作战策略。(勝つための戦略。) 
\\	作战行动;军事行动;战役。(軍事上の対敵行動。)	さくせん		【名词】
\\	作物	
\\	作物,农作物,庄稼。(農産物。)	さくぶつ		【名词】
\\	叫び	
\\	叫,叫喊;喊叫声;呼声;欢呼声。(叫ぶこと。また、その声。)	さけび		【名词】
\\	裂ける	
\\	裂,裂开,破裂。	さける		【自动词・一段/二类】
\\	捧げる	
\\	献出,贡献。 
\\	献,供,供奉。 
\\	双手擎举,捧举。	ささげる		【他动词・一段/二类】
\\	差し掛かる	
\\	来到,临到,靠近,路过。 
\\	逼近,临近。 
\\	垂悬,笼罩在……上。	さしかかる		【自动词・五段/一类】
\\	指図	
\\	指示。 
\\	指定。	さしず		【名・他动词・サ变/三类】
\\	差し出す	
\\	伸出,探出;寄出;提出;派出。	さしだす		【他动词・五段/一类】
\\	差し支える	
\\	妨碍,障碍,有影响。不方便。	さしつかえる		【自动词・一段/二类】
\\	差し引く	
\\	扣除,减去。 
\\	抵补,相抵。 
\\	潮水升降,增减。	さしひく		【他动词・五段/一类】
\\	些事	
\\	琐事,细节。(取るに足らないつまらないこと。ささいなこと。) 
\\	同:さ事	さじ		【名词】
\\	授ける	
\\	教授,传授。 
\\	授予,赋予,赐给。	さずける		【他动词・一段/二类】
\\	嘸	
\\	(和推量呼应)想必…吧;想必一定…吧(きっと、さだめし、さぞや)	さぞ		【副词】
\\	定まる	
\\	定,决定,规定。确定,明确。 
\\	安定,平定,稳定,固定,定下来。	さだまる		【自动词・五段/一类】
\\	定める	
\\	决定;选定。 
\\	规定,制定。 
\\	平定;平静;放松。 
\\	平定,平息。	さだめる		【他动词・一段/二类】
\\	錯覚	
\\	错觉;错误的观念;误会,错认为。	さっかく		【自动词・サ变/三类】
\\	早急	
\\	紧急地,尽快,火急地,火速地,赶忙地。	さっきゅう		【名・形容动词/ナ形容词】
\\	察する	
\\	推察,推测,揣测,想象。观察,察知。判断,了解到。 
\\	谅察,体谅。推测他人的心情而给予同情。	さっする		【他动词・サ变/三类】
\\	冊	
\\	1冊の本。/一本书。	さつ		【接头/接尾】
\\	殺人	
\\	杀人。	さつじん		【名词】
\\	悟る	
\\	悟,醒悟,觉悟,理解,领悟,认识,认清。 
\\	发现,察觉,看破。 
\\	〈仏〉开悟,悟道,了悟。	さとる		【他动词・五段/一类】
\\	真実	
\\	真实,真话,实际,事实。 
\\	实在地,真的,真心。	さな		【名・副・形容动词/ナ形容词】
\\	裁く	
\\	裁判,审判,排解,从中调停;评理。	さばく		【他动词・五段/一类】
\\	様	
\\	【形动】 
\\	样子,方式,方法,法子(ありさま。様子。すがた)。 
\\	样〔かたち〕。 
\\	风格〔様式)。	さま		【名词】
\\	三味線	
\\	三味线,日本三弦
\\	。(撥弦(はつげん)楽器の一。) 
\\	同:さみせん	さみせん		【名词】
\\	侍	
\\	武士(武士)。 
\\	有骨气,行动果断的人物,了不起的人物(なかなかの人物)。	さむらい		【名词】
\\	然も	
\\	〔その上〕而,而且,并且. 
\\	〔それでもなお〕而,但,却.	さも		【接续词】
\\	作用	
\\	作用,起作用。	さよう		【自动词・サ变/三类】
\\	左様なら	
\\	告别;离开(同さよなら) 
\\	然様なら;左様なら 
\\	さようなら		【自动词・サ变/三类】
\\	爽やか	
\\	(天气)清爽,爽朗;(心情)爽快。 
\\	(口齿)爽利;清楚;(嗓音)嘹亮,鲜明。	さわやか		【形容动词/ナ形容词】
\\	障る	
\\	有坏影响,有害。 
\\	妨碍,障碍,阻碍。	さわる		【自动词・五段/一类】
\\	酸化	
\\	氧化。	さんか		【自动词・サ变/三类】
\\	山岳	
\\	山岳。(やま。特に、高く険しい山々。)	さんがく		【名词】
\\	産休	
\\	产假。	さんきゅう		【名词】
\\	桟橋	
\\	(为了上下高处而架设的带斜坡的)跳板。 
\\	码头。	さんきょう		【名词】
\\	参議院	
\\	参议院,参院。日本国宪法下,同众议院一起构成国会的一院。日本国会的上院。	さんぎいん		【名词】
\\	産後	
\\	产后。(出産したあと。)	さんご		【名词】
\\	産出	
\\	出产,生产。	さんしゅつ		【名・他动词・サ变/三类】
\\	参照	
\\	参阅,参照。	さんしょう		【名・他动词・サ变/三类】
\\	参上	
\\	拜访,造访,趋谒,趋候。	さんじょう		【自动词・サ变/三类】
\\	賛成	
\\	赞成,赞同,同意。	さんせい		【名・自动词・サ变/三类】
\\	賛美	
\\	赞美,歌颂。	さんび		【名・他动词・サ变/三类】
\\	産婦人科	
\\	妇产科。(産科と婦人科。)	さんふじんか		【名词】
\\	産物	
\\	物产,产品,结果。	さんぶつ		【名词】
\\	山腹	
\\	山腹、山腰。(山頂と山麓との間の部分。)	さんぷく		【名词】
\\	山脈	
\\	山脉。群山绵延相连,呈带状延伸的山地。	さんみゃく		【名词】
\\	財	
\\	钱财,财宝。(財産。富。) 
\\	财产,财富。(人間の生活にとって貴重な物。)	ざい		【名词】
\\	財源	
\\	财源,创造财富的本钱,费用的来源。	ざいげん		【名词】
\\	在庫	
\\	库存。 
\\	存货。	ざいこ		【自动词・サ变/三类】
\\	財政	
\\	财政。(国または地方公共団体が、その存立を維持し活動するために必要な財力を取得し、これを管理・処分する一切の作用。) 
\\	经济情况。(個人・家庭・団体などの経済状態。)	ざいせい		【名词】
\\	座談会	
\\	座谈会。	ざだんかい		【名词】
\\	雑貨	
\\	杂货。	ざっか		【名词】
\\	雑	
\\	粗糙,粗率,粗枝大叶,潦草。	ざつ		【形容动词/ナ形容词】
\\	雑談	
\\	闲谈,闲聊,聊天儿『口』,谈闲天;说闲话。(よもやま話。)	ざつだん		【自动词・サ变/三类】
\\	雑木	
\\	杂木,杂树。	ざつぼく		【名词】
\\	残金	
\\	余额,余下的钱。(手元に残っている金銭。残高。) 
\\	余下的欠款,未付款。(未払いの金。)	ざんきん		【名词】
\\	残酷	
\\	残酷,残忍,冷酷,凶狠。	ざんこく		【名・形容动词/ナ形容词】
\\	残高	
\\	余额,结余。	ざんだか		【名词】
\\	死	
\\	死,死亡。 
\\	死罪。 
\\	出局。	し		【名词】
\\	次	
\\	下次,下回;其次,第二;下一(个);下面;接着。 
\\	次,第二。其次,次等。 
\\	隔壁,邻室。 
\\	驿站。 
\\	接二连三(地);接连不断(地)。	し		【名词】
\\	仕上がり	
\\	做完,完成。 
\\	做成的情况。	しあがり		【名词】
\\	仕上げ	
\\	最后一道工序;完工阶段。 
\\	做完;完成;润饰,做完的结果。	しあげ		【名・他动词・サ变/三类】
\\	仕上げる	
\\	完成。	しあげる		【他动词・一段/二类】
\\	明々後日	
\\	大后天。	しあさって		【名词】
\\	飼育	
\\	养,饲养(家畜)。(家畜などを養い育てること。)	しいく		【名・他动词・サ变/三类】
\\	強いて	
\\	强迫,强,硬。	しいて		【副词】
\\	強いる	
\\	强迫,强使。	しいる		【他动词・一段/二类】
\\	仕入れる	
\\	购入;取得。	しいれる		【他动词・一段/二类】
\\	歯科	
\\	牙科。	しか		【名词】
\\	資格	
\\	资格;身分。 
\\	资格;资质。	しかく		【名词】
\\	視覚	
\\	视觉。	しかく		【名词】
\\	仕掛け	
\\	制作中,做到中途;推动,发动,对人做工作;装置,结构,机关;(钓鱼用的)钓丝上所栓鱼钩、坠子、浮子等组件。	しかけ		【名词】
\\	仕掛ける	
\\	开始做,着手。 
\\	做到中途。 
\\	主动地做。 
\\	挑衅,寻衅。 
\\	装置,装设,设置,布置。	しかける		【他动词・一段/二类】
\\	然し	
\\	然而,可是。(前の話の内容を否定したり その内容から予想される事と反対の事を述べたりすることを表わす。) 
\\	同:併し	しかし		【接续词】
\\	然しながら	
\\	然而,可是,但是。(そっくりそのまま。まるで。さながら。)	しかしながら		【接续词】
\\	而も	
\\	〔その上〕而,而且,并且. 
\\	〔それでもなお〕而,但,却.	しかも		【接续词】
\\	市街	
\\	繁华街区,大街。 
\\	大街小巷。	しがい		【名词】
\\	指揮	
\\	指挥,主持。发令调度人员。 
\\	指挥。通过身体动作及指挥棒统率音乐演奏。	しき		【他动词・サ变/三类】
\\	色彩	
\\	倾向,色彩,特色,事物的性质、倾向。 
\\	颜色,色彩,彩色。	しきさい		【名词】
\\	式場	
\\	举行仪式的场所,会场,礼堂。	しきじょう		【名词】
\\	為来り	
\\	惯例;常规 
\\	仕来り;為来り 
\\	しきたり		【名词】
\\	頻りに	
\\	频繁地;屡次,再三;不断地,不停地;一直地。(同じことが何度も引き続き起こるさま。ひっきりなし。) 
\\	热心;强烈。(程度・度合いが著しいさま。むやみ。やたら。)	しきりに		【副词】
\\	仕切る	
\\	(相扑)摆架势:相扑比赛时,力士在交手之前站在比赛场上,两手向下,采取进攻姿势。 
\\	掌管,主持。 
\\	结帐,清账。 
\\	隔开,间隔开,区分开。	しきる		【他动词・五段/一类】
\\	資金	
\\	资金,资本。	しきん		【名词】
\\	施行	
\\	实施,施行,实行。实际进行。 
\\	施行,实施。使得公布的法令生效。	しぎょう		【名・他动词・サ变/三类】
\\	仕組み	
\\	(戏剧、小说等的)结构,情节。 
\\	结构,构造。 
\\	(事物的)结构,构成,策划,安排。	しくみ		【名词】
\\	死刑	
\\	死刑。(生命を絶つ刑罰。生命刑。)	しけい		【名词】
\\	湿気る	
\\	潮湿,发潮。	しける		【自动词・五段/一类】
\\	思考	
\\	思考,考虑。	しこう		【名・他动词・サ变/三类】
\\	志向	
\\	志向,意向。	しこう		【名・他动词・サ变/三类】
\\	嗜好	
\\	嗜好。爱好,兴趣。	しこう		【名・他动词・サ变/三类】
\\	視察	
\\	视察,考察。	しさつ		【名・他动词・サ变/三类】
\\	資産	
\\	资产。 
\\	财产,资产。	しさん		【名词】
\\	刺繍	
\\	刺绣。	ししゅう		【名・他动词・サ变/三类】
\\	支持	
\\	支撑,维持。 
\\	支持,赞成;拥护。	しじ		【名・他动词・サ变/三类】
\\	指示	
\\	指示。 
\\	指示,指令。	しじ		【名・他动词・サ变/三类】
\\	雫	
\\	水滴(水などの液体がしたたり落ちること。また,その水など)。	しずく		【名词】
\\	沈める	
\\	把东西沉入水中;使降落;击倒;落魄。	しずめる		【他动词・一段/二类】
\\	施設	
\\	孤儿院
\\	养老院;母子宿舍;接生站。 
\\	设施,设备。	しせつ		【名・他动词・サ变/三类】
\\	子息	
\\	儿子,令郎。	しそく		【名词】
\\	慕う	
\\	爱慕,怀念,想念,思慕。 
\\	敬慕,敬仰,景仰。 
\\	追随,跟随;追缠。	したう		【他动词・五段/一类】
\\	従って	
\\	因此,从而。	したがって		【接续词】
\\	下心	
\\	心字底。 
\\	本心。内心。 
\\	预谋。企图。	したごころ		【名词】
\\	親しむ	
\\	亲近,亲密,接近。 
\\	爱好,喜好。 
\\	接近,不离。持续不断地,职业般地做某事。	したしむ		【自动词・五段/一类】
\\	下調べ	
\\	预先调查,事前参考。 
\\	预习。	したしらべ		【自动词・サ变/三类】
\\	下地	
\\	准备,基础,底子。 
\\	原来就具备的性质,素质。 
\\	(未经涂抹的)墙底。	したじ		【名词】
\\	認める	
\\	看见,看到。(目にとめる。) 
\\	认识,赏识,重视。(みどころがあると考える。高く評価する。) 
\\	承认。(承認する。) 
\\	断定,认为。(見て判断する。) 
\\	准许,同意。(見てよしとする。かまわないとして許す。受け入れる。許可する。)	したためる		【他动词・一段/二类】
\\	仕立てる	
\\	缝制,制作,尤指缝制衣服。 
\\	培养,造就;掌握技术等。 
\\	特别准备。 
\\	乔装。	したてる		【他动词・一段/二类】
\\	下取り	
\\	用旧物折价贴钱换取新物,用旧物贴换新物。	したどり		【名词】
\\	下火	
\\	火势渐微,火将熄灭。(火の燃え方が弱まること。) 
\\	〔流行,势力等)衰退,微弱,不时兴,不流行。(一時盛んだった物事が、盛りを過ぎて衰えてくること。) 
\\	底火。在茶道添炭调整火候时,预先添在茶炉里对的火种。(茶道の炭手前で、あらかじめ風炉や炉に入れておく種火。)	したび		【名词】
\\	失格	
\\	失掉资格。 
\\	不及格。	しっかく		【自动词・サ变/三类】
\\	確り	
\\	股票市场形势好。	しっかり		【形容动词/ナ形容词】
\\	質素	
\\	朴素,俭朴,简陋。	しっそ		【名・形容动词/ナ形容词】
\\	失調	
\\	失调,不调和,不平衡,失常。	しっちょう		【名・自动词・サ变/三类】
\\	嫉妬	
\\	嫉妒,忌妒;吃醋。	しっと		【名・他动词・サ变/三类】
\\	尻尾	
\\	尾巴。 
\\	尾状物。 
\\	末尾,末端。	しっぽ		【名词】
\\	室	
\\	房间。(部屋。) 
\\	身份高贵者的妻子。(貴人の妻。奥方。) 
\\	窖,洞。(いわや。) 
\\	刀鞘。(鞘のこと。)	しつ		【名词】
\\	質疑	
\\	质疑,提出疑问,提问。在国会会议上,议员对议题或演说报告提出质疑并要求大臣、委员等给予说明。	しつぎ		【名・自动词・サ变/三类】
\\	躾	
\\	教育,管教,训练,教养。对孩子等教授礼仪成规使之懂得礼节、礼貌,亦指所具有的礼节、礼貌修养。(子供などに礼儀作法を教えて身に付けさせること。また、身についた礼儀作法。) 
\\	粗缝,临时缝上,绷线。为使平缝更加准确美观而预先进行粗略缝合,或为使做好的衣服不走行而缝折痕处。(本縫うを正確に、きれいにするためにあらがじめざっと縫い合わせておくこと。)	しつけ		【名词】
\\	仕付ける	
\\	做惯;织上;种植;长大成人;打败。	しつける		【他动词・一段/二类】
\\	指摘	
\\	指出;指摘。	してき		【名・他动词・サ变/三类】
\\	視点	
\\	视线的集中点。(視線の注がれるところ。) 
\\	观点。(ものを見る立場。観点。) 
\\	绘画上的视点。(絵画の遠近法で、画面上において遠方へ向かう平行線が集結する一点。消点。焦点。)	してん		【名词】
\\	萎びる	
\\	枯萎,干瘪,蔫。	しなびる		【自动词・一段/二类】
\\	屎尿	
\\	屎尿,大小便。(大便と小便。糞尿。)	しにょう		【名词】
\\	凌ぐ	
\\	忍耐,忍受;维持,抵御;躲避,排除。 
\\	闯过;摆脱;应付,对付;冒着,顶住。 
\\	凌驾,超过。	しのぐ		【他动词・五段/一类】
\\	始発	
\\	〔场所的〕起点。(乗り物の運転区間で、最初にその場所から発車すること。 
\\	〔时间方面〕头班车。(列車・電車・バスなどが一日のうちで、最初に出発すること。また、その車。)	しはつ		【名词】
\\	芝	
\\	结缕草。(イネ科の多年草,シバ・コウライシバ・オニシバ・イトシバなどの総称。) 
\\	日本地名。(東京都港区の地名。旧区名。増上寺・東京タワーがある。)	しば		【名词】
\\	暫く	
\\	暂,片刻,且慢,一会儿。虽不太长但亦非很短的时间;很久,许久。时间要经过得稍长一点的;暂且,暂时,姑且。不管将来如何,指是眼前时刻。	しばらく		【副词】
\\	痺れる	
\\	麻木;发麻;激动。	しびれる		【自动词・一段/二类】
\\	渋い	
\\	涩味的;古朴的;阴沉的;吝啬的。	しぶい		【形容词/イ形容词】
\\	私物	
\\	个人私有物。	しぶつ		【名词】
\\	司法	
\\	司法。	しほう		【名词】
\\	脂肪	
\\	脂肪。	しぼう		【名词】
\\	志望	
\\	志愿,愿望。希望将来能实现某种志向的心愿。(将来自分が「そうなりたい」「こうなりたい」とのぞむこと。)	しぼう		【名・他动词・サ变/三类】
\\	萎む	
\\	枯萎,凋萎。(植物が生気をなくして縮まる。しおれる。しなびる。) 
\\	瘪,消缩。饱满的东西松弛软缩下来。(張り詰めていたものが、緩み縮む。) 
\\	同:凋む	しぼむ		【自动词・五段/一类】
\\	仕舞	
\\	(名词后)结束,中止。(やめること。終えること。閉じること。) 
\\	(动词未然形的“ず”后)本想……,本来打算……(…するつもりであったこと、…したいと思っていたことをせずに終わってしまう意を表す。)	しまい		【接头/接尾】
\\	仕舞う	
\\	完了,光了,尽了。(补助动词)表示该动作全部结束或该状态完成。往往表示某事的非志愿发生。(動詞の連用形に助詞「て(で)」を添えた形に付いて、その動作がすっかり終わる、その状態が完成することを表す。終わったことを強調したり、不本意である、困ったことになった、などの気持ちを添えたりすることもある。) 
\\	同:終う、了う	しまう		【他动词・五段/一类】
\\	始末	
\\	(事情的)始末,颠末,原委。 
\\	情形,情况;(坏的)结果,结局,地步,(落到)……样子。 
\\	处理,应付。 
\\	俭省,节约,撙节。	しまつ		【名・他动词・サ变/三类】
\\	使命	
\\	使命,任务。	しめい		【名词】
\\	締め切り	
\\	截止,届满。	しめきり		【名词】
\\	僕	
\\	男仆人;我,男子指自己的词,明治时期作为书生用语开始在口语中使用。原本是自谦表达,现在主要用于成年前的男性对平辈以下的自称。	しもべ		【代词】
\\	社交	
\\	社交,交际。	しゃこう		【名词】
\\	謝絶	
\\	谢绝;拒绝。	しゃぜつ		【名・他动词・サ变/三类】
\\	社宅	
\\	公司的职工宿舍,公司职工住宅。(その会社の社員が住むために建ててある家。)	しゃたく		【名词】
\\	吃逆	
\\	打嗝。呃逆,打嗝儿。	しゃっくり		【自动词・サ变/三类】
\\	喋る	
\\	说,讲。 
\\	说出,泄漏。 
\\	喋喋不休,多嘴多舌,饶舌,能言善辩,能说会道。	しゃべる		【自他・五段/一类】
\\	斜面	
\\	斜面,倾斜面,斜坡。	しゃめん		【名词】
\\	洒落る	
\\	漂亮打扮;说俏皮话;别致;狂妄。	しゃれる		【自动词・一段/二类】
\\	視野	
\\	视场。 
\\	眼光,眼界,见识,思路。 
\\	视野。	しや		【名词】
\\	種	
\\	种类(一定の基準によって分類・類別したもの。種類。たぐい)。 
\\	(生物)种(生物分類学上の基本単位)。	しゅ		【名词】
\\	衆	
\\	众多,众人。(人数が多いこと。もろもろの人。) 
\\	一伙人。伙伴。(ある集団を形づくる特定の人々。また、なかま。) 
\\	对一伙人的敬称,爱称。(複数の人を敬意または親愛の意を添えて言い表す。)	しゅう		【接头/接尾】
\\	周	
\\	周,圈子,周围。(まわり。) 
\\	〈史〉周朝
\\	〈数〉周。	しゅう		【名词】
\\	収益	
\\	收益。	しゅうえき		【名词】
\\	修学	
\\	修学,学习。(学問をおさめること。学んで知識を得ること。しゅがく。)	しゅうがく		【自动词・サ变/三类】
\\	周期	
\\	周期。	しゅうき		【名词】
\\	衆議院	
\\	众议院。	しゅうぎいん		【名词】
\\	就業	
\\	开始工作,上班干活儿。 
\\	就业,有一定职业,有工作。	しゅうぎょう		【自动词・サ变/三类】
\\	修行	
\\	修行,托钵,巡礼。 
\\	修(学),练(武),学习(技艺)。	しゅうぎょう		【自动词・サ变/三类】
\\	集計	
\\	合计,总计。	しゅうけい		【名・他动词・サ变/三类】
\\	襲撃	
\\	袭击。	しゅうげき		【名・他动词・サ变/三类】
\\	収支	
\\	收支。	しゅうし		【名词】
\\	終始	
\\	始末。	しゅうし		【名词】
\\	修士	
\\	硕士。	しゅうし		【名词】
\\	収集	
\\	收集,搜集。	しゅうしゅう		【名・他动词・サ变/三类】
\\	修飾	
\\	修饰,装饰,润饰。(つくろいかざること。) 
\\	〈語〉修饰。(文法用語。体言または用言に、その表す意味を限定するために他の語をつけ加えること。)	しゅうしょく		【名・他动词・サ变/三类】
\\	終日	
\\	一天到晚,终日。(一日じゅう。ひねもす。)	しゅうじつ		【名词・副词】
\\	執着	
\\	同 
\\	しゅうちゃく 
\\	贪恋,留恋,不肯舍弃,执著,固执。	しゅうじゃく		【名・自动词・サ变/三类】
\\	収容	
\\	收容,容纳。	しゅうよう		【名・他动词・サ变/三类】
\\	修了	
\\	学习完(一定课程)。	しゅうりょう		【名・他动词・サ变/三类】
\\	守衛	
\\	门岗,门卫。(人の出入りを監視し、建物の警備を職務とする人。)	しゅえい		【名词】
\\	主演	
\\	主演,主角。	しゅえん		【自动词・サ变/三类】
\\	主観	
\\	主观。	しゅかん		【名词】
\\	祝賀	
\\	庆贺,祝贺。	しゅくが		【名・他动词・サ变/三类】
\\	宿命	
\\	宿命,注定的命运。	しゅくめい		【名词】
\\	主権	
\\	主权。(他国の干渉によって侵されることの無い、国家の意思力(統治権)。)	しゅけん		【名词】
\\	手芸	
\\	手工艺。	しゅげい		【名词】
\\	主催	
\\	主办,举办。	しゅさい		【名・自他・サ变/三类】
\\	取材	
\\	取材,采访。	しゅざい		【名・自动词・サ变/三类】
\\	趣旨	
\\	趣旨,宗旨,主要内容。	しゅし		【名词】
\\	主食	
\\	主食(品)
\\	。(日常の食事の中心となる食品。日本では,米飯・めん類・パンなどの類。)	しゅしょく		【名词】
\\	主人公	
\\	主人公,主人翁。	しゅじんこう		【名词】
\\	主体	
\\	核心。 
\\	主体。	しゅたい		【名词】
\\	主題	
\\	主题。主要题目。(主要な題目。メーン-タイトル。) 
\\	主题,题目,课题。成为谈话、文章、研究等的核心问题。中心思想,主要内容。(談話・文章・研究などの、中心となる問題。テーマ。) 
\\	主题。艺术作品中,作者所要表达的基本思想。(芸術作品に、作者があらわそうとする基本的な思想。テーマ。) 
\\	主题。直接表现乐曲中心构思的旋律。也有具有多个主题的乐曲。中心旋律。(楽曲の中心となる楽想を端的に表現している音楽的素材で、楽曲全体の発展の基礎となるもの。長い旋律から短い動機的音型に至るまでその規模はさまざまで、一曲が複数の主題をもつこともある。)	しゅだい		【名词】
\\	出血	
\\	【自サ】 
\\	出血。(血が出る。) 
\\	牺牲血本,亏本。(犠牲を払うこと。)	しゅっけつ		【名词】
\\	出産	
\\	生孩子;生产;分娩。	しゅっさん		【自他・サ变/三类】
\\	出社	
\\	(到公司)上班。(会社へ出勤すること) 
\\	到公司上班	しゅっしゃ		【自动词・サ变/三类】
\\	出生	
\\	同しゅっしょう。	しゅっしょう		【名・自动词・サ变/三类】
\\	出世	
\\	成功,出息,发迹。(立派な地位・身分となること。)	しゅっせ		【自动词・サ变/三类】
\\	出費	
\\	开支,支出,开销。拿出费用,亦指其费用。破钞。	しゅっぴ		【自动词・サ变/三类】
\\	出品	
\\	展出作品,展出产品。	しゅっぴん		【名・自他・サ变/三类】
\\	出演	
\\	演出;出场,登台。	しゅつえん		【自动词・サ变/三类】
\\	出題	
\\	(考试、诗歌)出题。	しゅつだい		【自动词・サ变/三类】
\\	出動	
\\	出动。	しゅつどう		【自动词・サ变/三类】
\\	主導	
\\	主导,主动。成为中心而起引导作用。	しゅどう		【名・他动词・サ变/三类】
\\	主任	
\\	主任。(ある任務にあたる者の中で中心となる者。また,その役職。)	しゅにん		【名词】
\\	首脳	
\\	首脑,首领,领导人物。(その組織の中で、中心となって働く最高の責任者。)	しゅのう		【名词】
\\	守備	
\\	【他サ】 
\\	守备;防备;防守。(自分の領域を侵されないように、自分の陳を守ること。守り。)	しゅび		【名词】
\\	手法	
\\	手法,技巧(美術作品などをつくるときの、独特の方法)。 
\\	同:技法	しゅほう		【名词】
\\	私有	
\\	私有。	しゆう		【名・他动词・サ变/三类】
\\	諸	
\\	诸多,众多(多くの)。 
\\	一起,共同(いっしょに)。 
\\	两个,双方(二つの。両方の)。	しょ		【接头/接尾】
\\	象	
\\	象,大象。长鼻目象科哺乳类动物的总称,通过化石发现的大象种类虽然很多,但现在仅存大型非洲象和印度象两种。长长的鼻子是由鼻与上唇延长而成。上颌门齿长大,长成“象牙”,是现代陆生动物中最大的动物。佛教把白象视为神圣。 
\\	同:ざう	しょう		【名词】
\\	傷	
\\	伤,创伤;瑕疵;缺陷,毛病。	しょう		【名词】
\\	商	
\\	商,商业。〔品物の売り買い。あきない。〕 
\\	商人。〔商人。あきないを業とする人。〕 
\\	商,商数。(ある数を他の数で割って得た数値。)	しょう		【名词】
\\	消去	
\\	消失,消去,涂掉。 
\\	消去,消元法,通过代入或加减等方法从几个方程式中把未知元去掉。	しょうきょ		【名・自他・サ变/三类】
\\	衝撃	
\\	精神的打击,冲动。(精神のショック。) 
\\	冲击,冲撞。(物体の打撃。)	しょうげき		【名词】
\\	証言	
\\	证言;作证。	しょうげん		【名・他动词・サ变/三类】
\\	証拠	
\\	证据,证明。	しょうこ		【名词】
\\	消耗	
\\	消耗,消费,耗尽,耗费。(使ってなくすこと) 
\\	疲乏,劳累。(疲れ果てる) 
\\	消耗;耗尽	しょうこう		【名・自他・サ变/三类】
\\	照合	
\\	对照,对证,查对,核对。	しょうごう		【他动词・サ变/三类】
\\	詳細	
\\	详细,详情。	しょうさい		【名・形容动词/ナ形容词】
\\	昇進	
\\	升进;晋级。	しょうしん		【自动词・サ变/三类】
\\	少数	
\\	少数。数量较少。	しょうすう		【名词】
\\	称する	
\\	称,名字叫……。 
\\	假称,伪称,冒充。 
\\	称赞。	しょうする		【名・自他・サ变/三类】
\\	消息	
\\	消息,信息,信。
\\	状況や用件などを手紙などで知らせること。また、その手紙や連絡。音信。音沙汰。たより。) 
\\	情况。(人や物事の、その時々のありさま。動静。状況。事情。)	しょうそく		【名词】
\\	承諾	
\\	同意,答应,承诺,应诺;应允;允许。(人からのお願いをききいれること。)	しょうだく		【名・他动词・サ变/三类】
\\	象徴	
\\	象征。	しょうちょう		【名・他动词・サ变/三类】
\\	小児科	
\\	儿科,小儿科'。(小児の内科的な病気を専門に扱う医学の分野。)	しょうにか		【名词】
\\	証人	
\\	证人。(ある事実を証明する人。ある事を証明するために事実を述べる人。) 
\\	保人,保证人。(ある人の身元・人柄などを保証する人。保証人。)	しょうにん		【名词】
\\	照明	
\\	照明,照亮。 
\\	舞台灯光,照明。	しょうめい		【名・他动词・サ变/三类】
\\	勝利	
\\	胜,胜利。(戦いや争いなどで、相手に勝つこと。)	しょうり		【自动词・サ变/三类】
\\	奨励	
\\	奖励,鼓励。(高く評価してそれをするようにすすめること。)	しょうれい		【名・他动词・サ变/三类】
\\	職員	
\\	职员。	しょくいん		【名词】
\\	植民地	
\\	殖民地。	しょくみんち		【名词】
\\	職務	
\\	职务,任务。	しょくむ		【名词】
\\	諸君	
\\	诸位,各位。(多くの人々をさす語。主として男性が、同輩ないし、それ以下の人々に対し、軽い敬意の念をもって用いる。)	しょくん		【代词】
\\	所在	
\\	住处,所在地。(物や人が存在すること。存在する場所。ありか。)	しょざい		【名词】
\\	所々	
\\	这儿那儿,有些地方。(あちらこちら。ここかしこ。)	しょしょ		【名词】
\\	所持	
\\	所持,所有;携带。(持っていること。持ち歩くこと。) 
\\	〔法〕所持。人对物事实上的支配。(法律で、物を事実上支配していると認められる状態。)	しょじ		【名・他动词・サ变/三类】
\\	所属	
\\	所属;附属(人やものが団体や組織に属していること)。 
\\	参加,加入(入る)。 
\\	テニス部に所属する	しょぞく		【自动词・サ变/三类】
\\	処置	
\\	处理,按手续处理。 
\\	医疗处理。	しょち		【名・他动词・サ变/三类】
\\	所定	
\\	指定,规定。(定まっていること。定めてあること。)	しょてい		【名词】
\\	所得	
\\	所得,收入,收益,所得,所有(物)。	しょとく		【名词】
\\	初版	
\\	初版,第一版。(刊行された書物の最初の版。また、その本。第1版。)	しょはん		【名词】
\\	処罰	
\\	处罚,处分。	しょばつ		【自他・サ变/三类】
\\	書評	
\\	书评。(書物の内容を批評・紹介すること。また、その文章。)	しょひょう		【名词】
\\	処分	
\\	处分,处罚。 
\\	法律处置,适用于法律条款。 
\\	处理。	しょぶん		【名・他动词・サ变/三类】
\\	庶民	
\\	老百姓,群众,平民,庶民,百姓。(特別の地位、資格、権能などを持たない、一般の人たち。)	しょみん		【名词】
\\	庶務	
\\	庶务,总务;杂务。(特別の名目のない一般の事務。いろいろの雑多な事務。)	しょむ		【名词】
\\	私用	
\\	私事。 
\\	私用,个人使用。	しよう		【名・他动词・サ变/三类】
\\	仕様	
\\	做法,方法,手段。 
\\	「仕様書」的略语。	しよう		【名词】
\\	使用人	
\\	佣人,雇工。(雇われ、働く人。)	しようにん		【名词】
\\	調べ	
\\	调查,检查。(調べること。調査。検査。) 
\\	审查,审问。调查。讯问,查问。(尋問。取調べ。) 
\\	演奏,乐曲。亦指其旋律。(音楽を奏でること。演奏。また、そのメロディー。) 
\\	(诗歌、音乐等的)调子,音调。((音楽や詩歌の持つ)調子。)	しらべ		【名词】
\\	知り合い	
\\	相识,结识,认识,相识的人,熟人,朋友。	しりあい		【名词】
\\	退く	
\\	向后退,离开。(いる場所から引き下がる。) 
\\	退避。(逃げる。)	しりぞく		【自动词・五段/一类】
\\	退ける	
\\	搬开,移开,挪开。移挪走。 
\\	另当别论,置于一旁。 
\\	(辅助动词)解决难题。表示出色地完成了一件困难事;表示敢于做难做的事。	しりぞける		【他动词・一段/二类】
\\	記す	
\\	做记号,加上符号。 
\\	铭记。 
\\	预示前兆。	しるす		【他动词・五段/一类】
\\	指令	
\\	指令,指示,通知。	しれい		【名・他动词・サ变/三类】
\\	皺	
\\	皱纹。皱折,褶子。(皮膚や紙・布などの表面にできる細い筋目。)	しわ		【名词】
\\	進化	
\\	进化。生物在漫长的岁月中,逐渐演变产生更复杂、更多样的生物的历史变化过程。有种类的多样化和为适应环境而产生的形态、技能、行为等方面的变化。这种变化不一定是进步性的。 
\\	进化。事物逐渐向好的方面发展变化。	しんか		【自动词・サ变/三类】
\\	殿	
\\	老爷;大人。(うやまって呼んだ言い方。)	しんがり		【名词】
\\	審議	
\\	审议,详细,评议。	しんぎ		【名・他动词・サ变/三类】
\\	進行	
\\	前进。 
\\	进展,进行。 
\\	发展,恶化。	しんこう		【名・自他・サ变/三类】
\\	新興	
\\	新兴。新兴起。(既成のものに対して、新たに興ること。)	しんこう		【名词】
\\	振興	
\\	振兴。	しんこう		【名・自他・サ变/三类】
\\	申告	
\\	向上级陈述。 
\\	国民向行政官厅报告。	しんこく		【名・他动词・サ变/三类】
\\	新婚	
\\	新婚,刚刚结婚不久。	しんこん		【自动词・サ变/三类】
\\	審査	
\\	审查。	しんさ		【名・他动词・サ变/三类】
\\	紳士	
\\	绅士,君子,正人君子。 
\\	男人。	しんし		【名词】
\\	進出	
\\	进入,打入;挤进;参加;向…发展,进展。	しんしゅつ		【名・自动词・サ变/三类】
\\	信者	
\\	信徒,追随者,崇拜者。	しんじゃ		【名词】
\\	真珠	
\\	珍珠。	しんじゅ		【名词】
\\	心中	
\\	心中。 
\\	男女相爱的证明。	しんじゅう		【名词】
\\	心情	
\\	心情。	しんじょう		【名词】
\\	新人	
\\	新人,新手,新参加的人,新一代人。	しんじん		【名词】
\\	神聖	
\\	神圣。	しんせい		【名・形容动词/ナ形容词】
\\	親善	
\\	亲善,友好。	しんぜん		【名词】
\\	真相	
\\	真相。	しんそう		【名词】
\\	新築	
\\	新建的房屋。 
\\	新建,新盖。翻盖,重盖。	しんちく		【名・他动词・サ变/三类】
\\	進呈	
\\	赠送,奉送。	しんてい		【名・他动词・サ变/三类】
\\	進展	
\\	进展,发展,进步。	しんてん		【自动词・サ变/三类】
\\	神殿	
\\	神殿,祭神的大殿。	しんでん		【名词】
\\	進度	
\\	进度。进展的程度。(物事の進みぐあい。)	しんど		【名词】
\\	振動	
\\	摇动,振动,振荡。 
\\	〈理〉振荡,振动,摆动。	しんどう		【自动词・サ变/三类】
\\	新入生	
\\	(学校)新生。(新入学の生徒。)	しんにゅうせい		【名词】
\\	信任	
\\	信任。	しんにん		【名・他动词・サ变/三类】
\\	審判	
\\	审判,判决,裁判。 
\\	〈体〉裁判(员)。 
\\	上帝的审判
\\	。	しんばん		【名・他动词・サ变/三类】
\\	神秘	
\\	神秘,奥秘。	しんぴ		【形容动词/ナ形容词】
\\	辛抱	
\\	耐心工作,耐。 
\\	忍,忍耐,忍受。	しんぼう		【名・自动词・サ变/三类】
\\	真理	
\\	道理,真理。(ほんとうの道理。) 
\\	合理。(その物事に関して、例外無くあてはまり、それ以外には考えられないとされる知識·判断。)	しんり		【名词】
\\	侵略	
\\	侵略。(他国に侵入してその領土や財物を奪いとること。)	しんりゃく		【名・他动词・サ变/三类】
\\	診療	
\\	诊疗。	しんりょう		【他动词・サ变/三类】
\\	進路	
\\	进路,前进的道路、方向。	しんろ		【名词】
\\	自覚	
\\	自知,认识到。觉醒,觉悟,自觉。感觉。	じかく		【名・他动词・サ变/三类】
\\	地方	
\\	音乐伴奏;乐队。(舞踊で、伴奏の音楽を受け持つ人。また、その音楽。)	じかた		【名词】
\\	自我	
\\	自我,自己。〔自分。〕 
\\	个性,自我主张。〔自分の主張。〕 
\\	意识主体,自我。(哲学では意識や行為をつかさどる主体としての私。)	じが		【名词】
\\	磁気	
\\	磁气,磁力。	じき		【名词】
\\	磁器	
\\	瓷器。(焼き物の一。陶器より高温で焼成。素地(きじ)はガラス化し,透明または半透明の白色で硬く,吸水性がない。)	じき		【名词】
\\	事業	
\\	事业;功业,业迹。 
\\	企业,实业。	じぎょう		【名词】
\\	地形	
\\	地形。 
\\	基础工程。 
\\	打地基。	じぎょう		【名词】
\\	軸	
\\	车轴。(車の心棒) 
\\	挂轴(掛けじく);卷轴(巻物の軸);书画(書画);画轴(絵の)。 
\\	〈数・理〉座标轴;轴。(回転するときの中心。) 
\\	核心,中心。(活動の中心。) 
\\	茎;蒂。(植物のくき,へた。) 
\\	(笔等的)杆儿。(柄。)	じく		【接头/接尾】
\\	自己	
\\	自我。(おのれ。自分自身。(哲)(必ずしも人格に限らず)何らかの同一性、統一性を持った存在自身。)	じこ		【名词】
\\	事項	
\\	事项,项目。	じこう		【名词】
\\	時刻表	
\\	(铁路等的)时刻表。(列車・バス・航空機など、乗り物の発着する時刻を記した表。時間表。)	じこくひょう		【名词】
\\	地獄	
\\	地狱。 
\\	受苦的地方,地狱。 
\\	喷火口,喷(热)水口,泉眼。	じごく		【名词】
\\	時差	
\\	时差。 
\\	错开时间。 
\\	时差。	じさ		【名词】
\\	自在	
\\	自由自在,自如,随意。	じざい		【名・形容动词/ナ形容词】
\\	自主	
\\	自主。	じしゅ		【名词】
\\	自首	
\\	自首,投案。	じしゅ		【名・自动词・サ变/三类】
\\	辞職	
\\	辞职。	じしょく		【名・他动词・サ变/三类】
\\	自信	
\\	信心,把握,自信。(自分で自分の能力や価値などを信じること。自分の考え方や行動が正しいと信じて疑わないこと。)	じしん		【名词】
\\	事前	
\\	事前,未然。	じぜん		【名词】
\\	自尊心	
\\	自尊心。	じそんしん		【名词】
\\	持続	
\\	持续。(保ち続けること。)	じぞく		【名・自他・サ变/三类】
\\	字体	
\\	字体,字形。	じたい		【名词】
\\	辞退	
\\	【他动·三类】 
\\	辞退,谢绝(へりくだって断ること。遠慮して権利や地位などを放棄すること)。	じたい		【名词】
\\	実質	
\\	实质,本质。事物的内容,本质。	じっしつ		【名词】
\\	実践	
\\	根据某种理论或主义实际行动。 
\\	实行,实际进行。	じっせん		【名・他动词・サ变/三类】
\\	実態	
\\	实际状态;真实情况。	じったい		【名词】
\\	実費	
\\	实际费用,成本实际费用。	じっぴ		【名词】
\\	十分	
\\	十分钟。(一時間の六分の一。)	じっぷん		【名词】
\\	実	
\\	〔果実〕果实;[果物]水果 
\\	〔種子〕种子 
\\	〔汁の具〕汤里的青菜〔肉〕 
\\	〔内容〕内容 
\\	実を結ぶ 
\\	〔実がなる〕结果 
\\	〔よい結果を生む〕成功;实现	じつ		【名词】
\\	実業家	
\\	实业家。	じつぎょうか		【名词】
\\	実情	
\\	实际情况。	じつじょう		【名词】
\\	自転	
\\	自行转动,自转(地球等)。	じてん		【名词】
\\	自動詞	
\\	(语法)自动词	じどうし		【名词】
\\	地主	
\\	地主。	じぬし		【名词】
\\	耳鼻科	
\\	耳鼻科。	じびか		【名词】
\\	地元	
\\	当地。与某事直接相关之地。 
\\	本地。自己居住之地。	じもと		【名词】
\\	弱	
\\	【接尾】 
\\	弱。(よわいこと) 
\\	[接在数量词后]不足,近;弱。(端数を切り上げた数字を表す語)	じゃく		【名词】
\\	若干	
\\	多少。若干。几多。 
\\	そくばく(若干)	じゃっかん		【副词】
\\	砂利	
\\	(俗语)儿童,小孩儿,崽子,剧场中的小孩观众。 
\\	小石头,砂石,碎石子,砾石。	じゃり		【名词】
\\	じゃん拳	
\\	猜猜猜,猜拳,划拳。(片手で、石(ぐう)・紙(ぱあ)・はさみ(ちょき)のいずれかの形を同時に出し合って勝負を決めること。また、その遊び。石ははさみに、はさみは紙に、紙は石に勝つ。)	じゃんけん		【名词】
\\	住	
\\	居住,住所(すみか)	じゅう		【名词】
\\	従業員	
\\	工作人员,业务员,职工。	じゅうぎょういん		【名词】
\\	従事	
\\	做,从事。	じゅうじ		【自动词・サ变/三类】
\\	充実	
\\	充实。	じゅうじつ		【名词】
\\	十字路	
\\	十字路,歧路。	じゅうじろ		【名词】
\\	絨毯	
\\	地毯。	じゅうたん		【名词】
\\	柔軟	
\\	柔软。 
\\	灵活。	じゅうなん		【形容动词/ナ形容词】
\\	重複	
\\	【自动·三类】 
\\	重复。(同じ物事が幾度も重なること。かさなりあうこと。じゅうふく。) 
\\	同:ちょうふく	じゅうふく		【名词】
\\	重宝	
\\	贵重宝物。(その家や寺社に伝えられた、大切な宝物。)	じゅうほう		【名词】
\\	従来	
\\	过去,从前,以前。	じゅうらい		【名词・副词】
\\	塾	
\\	塾,学校,学堂。私塾,私人学堂;补习班,教授技艺和学问的地方。 
\\	大门两侧的房间。	じゅく		【名词】
\\	樹木	
\\	树木。	じゅもく		【名词】
\\	樹立	
\\	树立,建立,确立。	じゅりつ		【名・自他・サ变/三类】
\\	準急	
\\	〈铁〉准快车,普通旅客快车。(急行列車よりやや停車駅の数の多い列車。)	じゅんきゅう		【名词】
\\	準じる	
\\	按照,依照,以……为标准。 
\\	比照,准照,按同等资格处理。	じゅんじる		【自动词・一段/二类】
\\	準ずる	
\\	准,以……为标准,按照,依照。遵从某根据。 
\\	准,按……看待,适用于,比照。仿照其他办理。 
\\	同:准ずる	じゅんずる		【自动词・サ变/三类】
\\	助	
\\	帮助,帮忙;援助。(仕事をたすけること。また、その者。てつだい。支援。) 
\\	(曲艺团体中挑班演员的)助演(代演);帮腔的。(応援のために出演すること。また、その役者。代演。) 
\\	情妇。(不良仲間の隠語。「なごすけ」の略。女の子。)	じょ		【名词】
\\	情	
\\	情,感情,情素。(感情。) 
\\	同情。(情け。思いやりの心。) 
\\	情感,恩情,情意。 
\\	性欲,爱情。(異性を慕う気持ち。) 
\\	情况。(ありさま。様子。)	じょう		【名词】
\\	畳	
\\	叠放。(積み重ねること。) 
\\	榻榻米。(たたみ。) 
\\	张。(畳を数える語。)	じょう		【名词】
\\	嬢	
\\	姑娘。少女,小姐。 
\\	小姐。接在未婚女性的姓名下作敬称。 
\\	小姐。接在职业名称后表示从事该职业的女性。 
\\	同:娘	じょう		【接头/接尾】
\\	状	
\\	情形,情况,状况。 
\\	书,文书,信件。 
\\	状,形状。	じょう		【名词】
\\	上位	
\\	上位,上座,高级。	じょうい		【名词】
\\	上演	
\\	上演,演出。	じょうえん		【名・他动词・サ变/三类】
\\	城下	
\\	城下。城墙之下。城旁边。(城壁の下。城のそば。)	じょうか		【名词】
\\	乗客	
\\	乘客,旅客。	じょうかく		【名词】
\\	上空	
\\	高空,天空。(空の上の方。またある地点の上に広がる空。)	じょうくう		【名词】
\\	上司	
\\	上司,上级。	じょうし		【名词】
\\	情緒	
\\	情趣,风趣,情绪。	じょうしょ		【名词】
\\	上昇	
\\	上升;上涨。	じょうしょう		【自动词・サ变/三类】
\\	情勢	
\\	情势,形势。(変化して行く物事の成り行きやようす。) 
\\	同:成り行き	じょうせい		【名词】
\\	情熱	
\\	热情,激情。	じょうねつ		【名词】
\\	丈夫	
\\	男子汉,大丈夫。(成年男性,一人前の男。)	じょうふ		【名词】
\\	譲歩	
\\	让步。部分地或全部地放弃自己的主张,与对方的意见达成妥协。((人の道を譲る意から)自分の主張の一部または全部を曲げて、相手の意見と折り合いをつけること。)	じょうほ		【自动词・サ变/三类】
\\	条約	
\\	条约。	じょうやく		【名词】
\\	上陸	
\\	上陆,登陆,登岸。	じょうりく		【名・自动词・サ变/三类】
\\	蒸留	
\\	蒸馏。	じょうりゅう		【名・他动词・サ变/三类】
\\	除外	
\\	除外,免除,不在此限。	じょがい		【名・他动词・サ变/三类】
\\	助言	
\\	忠告,建议,从旁教导。	じょげん		【名・自动词・サ变/三类】
\\	徐行	
\\	【自动・三类】 
\\	徐行,慢行慢行,徐行。(ゆっくり進むこと。)	じょこう		【名词】
\\	女史	
\\	女士。用在有社会地位和声望的妇女名下的词语。(見識や教養が豊かで、社会的に活動している女性。また、そういう女性に対する敬称として名前の下につける語。) 
\\	日本古时从事文书工作的女官。(律令制の女官の一。後宮で、文書の事をつかさどった。) 
\\	古代中国从事记录皇后礼事,后宫事务的女官。(古代中国で、王后の礼事、後宮の記録などをつかさどった女官。)	じょし		【名词】
\\	助詞	
\\	助词,日语词类的一种,没有活用的附属词,可接在各种词后,表示词与词的关系,增加种种意义。	じょし		【名词】
\\	助動詞	
\\	〈语〉助动词。(活用語の一つ。おもに動詞に付属して叙述の意味を助け補う言葉。)	じょどうし		【名词】
\\	自立	
\\	自立,独立。办事没有他人的帮助或操纵,完全凭借自己的力量。(ほかの助けや支配なしに自分ひとりで物事を行うこと。ひとりたち。独立。)	じりつ		【自动词・サ变/三类】
\\	人	
\\	人,人们。 
\\	人种,民族。 
\\	某地的人,某地出生的人。 
\\	专业人士。 
\\	人,天、地、人三者顺序中最下者的称呼。	じん		【名词】
\\	人格	
\\	人格,人品。 
\\	个人。	じんかく		【名词】
\\	人材	
\\	人材。(才知ある人物。役に立つ人物。)	じんざい		【名词】
\\	迅速	
\\	迅速。	じんそく		【名・形容动词/ナ形容词】
\\	人体	
\\	人品,人格。 
\\	人体,人的身体。	じんたい		【名词】
\\	人民	
\\	人民。	じんみん		【名词】
\\	人目	
\\	世人的眼目,众目;旁人看见(世間の人びとが見ていること。人びとの目)。	じんもく		【名词】
\\	水気	
\\	潮气,湿气。 
\\	水蒸气。 
\\	水肿。	すいき		【名词】
\\	水源	
\\	水源。	すいげん		【名词】
\\	推進	
\\	推进,推动。	すいしん		【名・他动词・サ变/三类】
\\	水洗	
\\	水洗,水冲。(水で洗い流すこと。)	すいせん		【名・他动词・サ变/三类】
\\	吹奏	
\\	吹奏。	すいそう		【名・他动词・サ变/三类】
\\	推測	
\\	推测;猜测;估计;臆测。	すいそく		【名・他动词・サ变/三类】
\\	水田	
\\	水田。(稲を作る田。) 
\\	日本地名。(日本の地名。福岡県と徳島県にある。)	すいでん		【名词】
\\	推理	
\\	推理,推论,推断。	すいり		【名・他动词・サ变/三类】
\\	数詞	
\\	〈语〉数词。(数を表わすために用いられる語。)	すうし		【名词】
\\	崇拝	
\\	崇拜。	すうはい		【名・他动词・サ变/三类】
\\	据え付ける	
\\	安装;装配;固定。	すえつける		【他动词・一段/二类】
\\	据える	
\\	安设;放,安放,放置。 
\\	摆列。 
\\	使就……职位。 
\\	沉着(不动)。 
\\	灸治。 
\\	让(人)坐在……的上面(位置)。 
\\	盖,印。	すえる		【他动词・一段/二类】
\\	清々しい	
\\	神清气爽的,清爽的。	すがすがしい		【形容词/イ形容词】
\\	過ぎ	
\\	超过;开外。(「時間がそこまでたった」「年齢がそれより上である」という意味を表わす。) 
\\	过度,太;过分。(「必要以上である」「程度が必要以上でよくない」という意味を表わす。)	すぎ		【接头/接尾】
\\	救い	
\\	救,救援,搭救,拯救。(救うこと。助けること。) 
\\	挽救,补偿,劝慰。(人の気持ちをなぐさめ、明るくするもの。)	すくい		【名词】
\\	掬う	
\\	抄取;捞取;掬取,捧;舀;撇。(液体や粉末の中に手・さじなどを入れて,一部分を取り出す。液体の中や表面にいるものを網などで取り出す。) 
\\	下绊子;抄起(对方的腿等)。(下から上へ曲線をえがくようにして横にはらう。また,下から上へ急にもちあげる。 
\\	すくう		【他动词・五段/一类】
\\	少なくとも	
\\	少说也要,保守估计也要…(最小の程度や最低の段階に言及するさま。うちわにみても。少なくも)。 
\\	至少(ほかの物事はともかくとして。せめて)。	すくなくとも		【副词】
\\	直ぐ	
\\	马上,立刻,立即。(時間的に間を置かないさま。ただちに。) 
\\	非常,紧,(距离)极近。(距離が非常に近いさま。) 
\\	直,直爽。笔直。亦指坦率。(まっすぐで曲がっていないさま。心が素直なさま。道などが直線的なさま。まっすぐなさま。ありのまま。) 
\\	同:直	すぐ		【形容动词/ナ形容词】
\\	健やか	
\\	健壮,健康,健全。	すこやか		【形容动词/ナ形容词】
\\	濯ぐ	
\\	洗涮。 
\\	雪耻。 
\\	同:雪ぐ	すすぐ		【他动词・五段/一类】
\\	進み	
\\	进度,进展。(物事がはかどること。また、その度合い。) 
\\	前进,进步。(進歩·上達の度合。)	すすみ		【名词】
\\	勧め	
\\	规劝,劝诱,劝告,劝诫。(すすめること。)	すすめ		【名词】
\\	裾	
\\	衣服的下摆。 
\\	山麓。 
\\	下端,末端。 
\\	河的下游。	すそ		【名词】
\\	廃れる	
\\	衰微,衰落,(被)淘汰。 
\\	过时,不再流行。	すたれる		【自动词・一段/二类】
\\	酸っぱい	
\\	酸。 
\\	酸”除了味道酸的意思以外,还有伴随着轻微疼痛的乏力感(如“肩膀酸痛”),苦恼难过(如“心酸”),傲慢(如“酸蛋”),嫉妒(如“酸味”“酸意”)等的用法。	すっぱい		【形容词/イ形容词】
\\	素敵	
\\	极好,绝妙,非常好的,极漂亮地。	すてき		【形容动词/ナ形容词】
\\	即ち	
\\	即;也就是说,换言之(言い換えれば。とりもなおさず)。 
\\	即是,就是,正是(そのときは。つまり)。	すなわち		【接续词】
\\	素早い	
\\	快速,敏捷,麻利,利落;反应快,理解、判断等迅速。	すばやい		【形容词/イ形容词】
\\	済ます	
\\	弄完,搞完,办完;结束。 
\\	偿清,还清。 
\\	对付,将就。	すます		【他动词・五段/一类】
\\	澄ます	
\\	澄清。 
\\	平心静气,静下心来。 
\\	集中注意力。 
\\	使……明亮,清澈,晶莹。 
\\	专心。 
\\	完全,完成。 
\\	洗净。 
\\	治理,平定。 
\\	说清道理。	すます		【他动词・五段/一类】
\\	済みません	
\\	对不起;抱歉;劳驾;谢谢(寒暄的客气语)	すみません		【接续词】
\\	天皇	
\\	天皇,日皇。	すめらぎ		【名词】
\\	刷り	
\\	印刷(的效果)
\\	。(印刷すること。また,印刷の具合。) 
\\	向丝绸上印花。(草木の汁で布に文様を染めつけること。)	すり		【名词】
\\	剃る	
\\	剃,刮。(ひげや髪の毛などをそぎ切る。)	する		【他动词・五段/一类】
\\	擦れ違い	
\\	交错,不一致,差开。(すれちがうこと。)	すれちがい		【名词】
\\	すれ違う	
\\	会车,会让。交错,错过去。在几乎相互擦着的近处相向通过。 
\\	错过,走两岔。本应遇上而没遇上。 
\\	不一致,不统一,有分歧。议论等的论点谈不到一块。	すれちがう		【自动词・五段/一类】
\\	擦れる	
\\	摩擦,蹭。互相摩擦。(物と物とがすれ合う。)	すれる		【自动词・一段/二类】
\\	図々しい	
\\	厚颜无耻的。(人に迷惑をかけても平気でいる。厚かましい。)	ずうずうしい		【形容词/イ形容词】
\\	ずぶ濡れ	
\\	全身湿透。(雨などが衣服にしみとおって、からだ全体がぬれること。びしょぬれ。ぐしょぬれ。)	ずぶぬれ		【名词】
\\	制	
\\	【接尾】 
\\	(古)封建帝王的命令。(天子の命令。支配者の命令。) 
\\	制;制度。
\\	きまり。おきて。)	せい		【名词】
\\	製	
\\	制造,制品,产品。(作られたものであること。作。)	せい		【名词】
\\	生育	
\\	生长;繁殖;生(育和抚)养。(育てること。育って大きくなること。)	せいいく		【名・自他・サ变/三类】
\\	成果	
\\	成果,成绩,成就;结果。(なしとげた結果。できあがったよい結果) 
\\	『比較』“成果”“成就”“成绩”: 
\\	成果”は行動により生まれたよい結果。 
\\	成就”は創造的なことがらや事業について。 
\\	成绩”は仕事・学習・労働について言い、意味が軽い。	せいか		【名词】
\\	正解	
\\	正确的解释,正解。(正しい解釈。) 
\\	正确的解答。(正しい解答。)	せいかい		【名・他动词・サ变/三类】
\\	正規	
\\	正常,额定,正规,标准。 
\\	正规,正式规定。	せいき		【名词】
\\	正義	
\\	正义。 
\\	正确的意义。	せいぎ		【名词】
\\	生計	
\\	生计;生活。(生活をしていくための方法・手段。くらし。)	せいけい		【名词】
\\	政権	
\\	政权。	せいけん		【名词】
\\	精巧	
\\	精巧,玲珑;精密,连细微之处都做得十分精致巧妙。	せいこう		【形容动词/ナ形容词】
\\	制裁	
\\	制裁。对违反社会或团体的规则、习惯等的人所给予的惩罚和处罚。	せいさい		【名・他动词・サ变/三类】
\\	政策	
\\	政策,方针。 
\\	策略。	せいさく		【名词】
\\	清算	
\\	(金銭の)结算;清算;结帐。(帳簿をしめる) 
\\	清理财产。(財産処分) 
\\	〔悪いことの〕清算,清除;结束,了结。(とりのぞく;終わらせる) 
\\	清算;结帐;清理财产;结束	せいさん		【名・他动词・サ变/三类】
\\	星座	
\\	星座,将天体上的恒星群比作神话里的人物、动物、器物等形态的星空区别。现国际上共确定88个星座,系根据希腊的星座整理而成。	せいざ		【名词】
\\	生死	
\\	生死,死活。	せいし		【名词】
\\	静止	
\\	静止。	せいし		【名・自动词・サ变/三类】
\\	青春	
\\	青春。	せいしゅん		【名词】
\\	聖書	
\\	《圣经》,基督教的经典,由《旧约圣经》和《新约圣经》两部分组成。	せいしょ		【名词】
\\	誠実	
\\	诚实。	せいじつ		【名・形容动词/ナ形容词】
\\	成熟	
\\	水果等长熟。 
\\	人的身体、精神等充分成长发育。 
\\	到了做某事的适当时期。	せいじゅく		【名・自动词・サ变/三类】
\\	清純	
\\	纯洁,清秀。	せいじゅん		【名・形容动词/ナ形容词】
\\	正常	
\\	正常。	せいじょう		【名・形容动词/ナ形容词】
\\	制する	
\\	制止,压制,控制。 
\\	控制。 
\\	至于自己的实力下。	せいする		【他动词・サ变/三类】
\\	整然	
\\	有条不紊,井井有条,整齐。	せいぜん		【形容动词/ナ形容词】
\\	盛装	
\\	盛装,华丽的装束。	せいそう		【名・自动词・サ变/三类】
\\	盛大	
\\	盛大,隆重,宏大。	せいだい		【形容动词/ナ形容词】
\\	清濁	
\\	清和浊。 
\\	正和邪。 
\\	君子和小人。 
\\	清音和浊音。	せいだく		【名词】
\\	制定	
\\	制定。(法律・規則などを定めること。特に、立法機関が一定の手続きによって法令を定めること。)	せいてい		【名・他动词・サ变/三类】
\\	静的	
\\	静的,不动的。	せいてき		【形容动词/ナ形容词】
\\	製鉄	
\\	炼铁。	せいてつ		【名词】
\\	晴天	
\\	晴天;晴。(空が晴れわたっていること。また,晴れた空。青天。)	せいてん		【名词】
\\	正当	
\\	正当;合乎道理,合理;合法;公正。正确。	せいとう		【名・形容动词/ナ形容词】
\\	成年	
\\	成年。	せいねん		【名词】
\\	制服	
\\	制服,学校、公司等所属人员穿着的有规定式样的服装。	せいふく		【名词】
\\	征服	
\\	克服。 
\\	征服,战胜,攻占。	せいふく		【名・他动词・サ变/三类】
\\	製法	
\\	(物品的)制法,制造方法,做法。	せいほう		【名词】
\\	精密	
\\	精密,细致,精细,精确,精致。	せいみつ		【名词】
\\	声明	
\\	〈仏〉 
\\	声明。印度五名之一,系研究音韵、语法、训诂的学问。(インドの五明の一。音韻・文法を研究する学問。) 
\\	生命。在日本,指做法会时僧人赞唱佛的声乐。(日本で、法会の祭、僧によって唱えられる声楽。)	せいめい		【名词】
\\	姓名	
\\	姓名。	せいめい		【名词】
\\	制約	
\\	条件,规定。(制限や条件をつけて、自由に活動させないこと。) 
\\	限制,制约。(物事の成立に必要な条件や規定。)	せいやく		【名・他动词・サ变/三类】
\\	生理	
\\	生理(现象)。(生きていることに伴う体の働き。) 
\\	月经。(月経。メンス。)	せいり		【名词】
\\	勢力	
\\	势力就是威势,指政治、经济、军事等方面的力量。(他をおさえ、支配下におくいきおいと力。特に、国家や政党などの社会的な集団がもつ、他の集団をおさえる力。) 
\\	势能的旧称(エネルギーの旧称。) 
\\	同:せいりょく	せいりょく		【名词】
\\	整列	
\\	排队,排列。	せいれつ		【名・自他・サ变/三类】
\\	急かす	
\\	催,催促。促使赶快做某事。	せかす		【他动词・五段/一类】
\\	伜	
\\	对他人谦称自己儿子的用语(自分の息子のことをへりくだっていう語。) 
\\	小孩子或者年轻人较为粗俗的说词。(子供や年の若い者をぞんざいにいう語。) 
\\	阴茎的俗称。(俗に、陰茎のこと。)	せがれ		【名词】
\\	責務	
\\	责任和义务。 
\\	职责,作为责任应该做的工作。	せきむ		【名词】
\\	世辞	
\\	巴结,奉承。(相手を喜ばせようとして、実際以上にほめる言葉。)	せじ		【名词】
\\	世帯	
\\	家庭,共同居住和处理生计问题的团体,可以有亲戚以外的人加入,也可以是一个人生活。 
\\	自立门户独立生活。 
\\	生活所必须的房子和用具。	せたい		【名词】
\\	世代	
\\	世代;一代,辈。	せだい		【名词】
\\	切開	
\\	切开,解剖。	せっかい		【名・他动词・サ变/三类】
\\	接触	
\\	靠近触摸。 
\\	联系。	せっしょく		【自动词・サ变/三类】
\\	設置	
\\	设置;安装。 
\\	设置,设立。	せっち		【名・他动词・サ变/三类】
\\	折衷	
\\	折中,折衷。	せっちゅう		【名・他动词・サ变/三类】
\\	設定	
\\	设立,制定,确定。	せってい		【名・他动词・サ变/三类】
\\	説得	
\\	说服,劝导。	せっとく		【名・他动词・サ变/三类】
\\	節	
\\	节儿,段儿。	せつ		【名词】
\\	切実	
\\	切实,迫切,殷切。 
\\	切实,由于受到直接影响而再不能轻视的状态。 
\\	贴切,极其恰当。	せつじつ		【形容动词/ナ形容词】
\\	接続詞	
\\	接续词,连接词。	せつぞくし		【名词】
\\	切ない	
\\	难受的,苦闷的,(因悲伤而感到)憋得慌的,难过的。 
\\	(身体感觉)难受的,痛苦的。	せつない		【形容词/イ形容词】
\\	設立	
\\	设立,成立。	せつりつ		【名・他动词・サ变/三类】
\\	攻め	
\\	进攻,攻击。 
\\	战术,攻击的方法。	せめ		【名词】
\\	世論	
\\	舆论。(世間一般の人が唱える論。社会大衆に共通な意見。) 
\\	同:輿論	せろん		【名词】
\\	前	
\\	前面,前方。 
\\	前面,前头,前边,连续部分的开始处。 
\\	前,在时间上指较早的时候。 
\\	前面,正面,对面。 
\\	阴部。 
\\	前科。	せん		【名词】
\\	繊維	
\\	细长的物质。(一般に、細い糸状の物質。) 
\\	从动植物中提取的细丝状物质。(生物体を組織する構造のうち、細い糸状物。)	せんい		【名词】
\\	選挙	
\\	选举,推选。	せんきょ		【名・他动词・サ变/三类】
\\	宣教	
\\	传教,布道。	せんきょう		【自动词・サ变/三类】
\\	宣言	
\\	宣言,宣布,宣告。	せんげん		【名・他动词・サ变/三类】
\\	先行	
\\	优先实行,提前实行。 
\\	先行,走在前头,领先,占先。	せんこう		【自动词・サ变/三类】
\\	選考	
\\	选拔。	せんこう		【名・他动词・サ变/三类】
\\	戦災	
\\	兵灾,战争之灾,战祸。	せんさい		【名词】
\\	専修	
\\	专修;专攻。(特定の学問や技術のみを専門に学ぶこと。専攻。)	せんしゅう		【名・他动词・サ变/三类】
\\	戦術	
\\	战术,策略。	せんじゅつ		【名词】
\\	潜水	
\\	潜水。(水中にもぐること。)	せんすい		【自动词・サ变/三类】
\\	先先月	
\\	上上月、前两月。	せんせんげつ		【名词】
\\	先先週	
\\	大上周	せんせんしゅう		【名词】
\\	先代	
\\	以前的时代,上一个时代。 
\\	(演艺界人士等的)前代,上一代。 
\\	上代主人,上一辈。	せんだい		【名词】
\\	先だって	
\\	前几日,前些日子。	せんだって		【名词・副词】
\\	先着	
\\	先到,先到达,先来到。	せんちゃく		【名词】
\\	先天的	
\\	先天的。	せんてんてき		【形容动词/ナ形容词】
\\	戦闘	
\\	战斗。	せんとう		【自动词・サ变/三类】
\\	潜入	
\\	潜入,溜进。(ひそかに入り込むこと。もぐりこむこと。)	せんにゅう		【自动词・サ变/三类】
\\	船舶	
\\	船的总称,一般指大型船。 
\\	轮船,海船,日本商法上指除橹桨船以外,以商业行为为目的的航船。	せんぱく		【名词】
\\	専用	
\\	专用。只限特定的人使用。 
\\	专用,只用于某种特定目的。	せんよう		【名・他动词・サ变/三类】
\\	戦力	
\\	军事力量,作战能力,能够进行战争的综合性有组织的军事力量。 
\\	(战斗)能力,完成某事的必要人员。	せんりょく		【名词】
\\	税務署	
\\	税务局。	ぜいむしょ		【名词】
\\	是正	
\\	订正;更正;矫正;纠正;改正。(誤った点を直して正しくすること。)	ぜせい		【名・他动词・サ变/三类】
\\	絶版	
\\	绝版,一度出版发行的读物不再继续出版发行。	ぜっぱん		【名词】
\\	絶望	
\\	绝望,完全丧失希望。	ぜつぼう		【名・自动词・サ变/三类】
\\	是非とも	
\\	一定,无论如何,务必。(どうしても。必ず。)	ぜひとも		【副词】
\\	膳	
\\	碗。(出来上がった飲食物を供えるワン。) 
\\	双。(箸二本を一そろえとして算える語。) 
\\	食案,小饭桌。(食物をのせる台。) 
\\	(摆在食案上的)饭菜。(飯・菜を膳にのせて出す料理。)	ぜん		【名词】
\\	禅	
\\	禅;禅宗;打坐。(精神を統一し、無我の境地に入って物事の真の姿を求めること。)	ぜん		【名词】
\\	全快	
\\	痊愈,疾病彻底治好。	ぜんかい		【名・自动词・サ变/三类】
\\	全盛	
\\	全盛,鼎盛。	ぜんせい		【名词】
\\	前提	
\\	事物发生的必要条件。(ある物事をなす土台となるもの。) 
\\	逻辑学上指推理成为结论依据的已知或假定的判断或命题。(推理を行う場合、結論の基礎となる判断。)	ぜんてい		【名词】
\\	前途	
\\	去路,前方。 
\\	将来,前程。	ぜんと		【名词】
\\	全滅	
\\	全歼,全灭,完全失败。	ぜんめつ		【名・自他・サ变/三类】
\\	善良	
\\	善良,正直而心地纯洁。	ぜんりょう		【名・形容动词/ナ形容词】
\\	前例	
\\	前例,先例。 
\\	前述例子。	ぜんれい		【名词】
\\	僧	
\\	僧,僧侣,和尚。
\\	出家し,仏門にはいって修行する人。ある宗教に入信してその修行をしている人。 
\\	そう		【名词】
\\	沿う	
\\	沿,顺;按照。	そう		【自动词・五段/一类】
\\	添う	
\\	增添;紧跟;结成夫妻一起生活;符合。	そう		【自动词・五段/一类】
\\	総	
\\	总,全体。(すべてのもの。全体。) 
\\	(接头词)全部。(全部の、すべての)	そう		【名词】
\\	相応	
\\	适应,相称。(程よいつりあうこと。)	そうおう		【形容动词/ナ形容词】
\\	総会	
\\	总会,大会,全会。集合团体中的全部成员召开的会议,通常构成该团体的最高决议机关。	そうかい		【名词】
\\	創刊	
\\	创刊。(新聞・雑誌など、定期刊行物を新しく刊行すること。)	そうかん		【名・他动词・サ变/三类】
\\	送金	
\\	【自动·三类】 
\\	寄钱;(為替による)汇款。	そうきん		【名词】
\\	走行	
\\	(汽车等的)行驶;行车。	そうこう		【自动词・サ变/三类】
\\	総合	
\\	总合,综合。(個々別々の物を一つにあわせまとめること。) 
\\	合题,综合。(弁証法で、相互に矛盾する定立と反定立とを止揚すること。)	そうごう		【名・他动词・サ变/三类】
\\	捜査	
\\	搜查。(捜査機関が、公訴の提起・維持のため、犯人および犯罪事実に関する証拠を発見・収集すること。) 
\\	查找,查访。(さがしてとりしらべること。)	そうさ		【名・他动词・サ变/三类】
\\	捜索	
\\	搜索,搜寻。(行方不明の人や物をさがし求めること。) 
\\	搜查。(刑事訴訟法に基づき、裁判所・検察官・司法警察職員などが、証拠物件や犯人を発見するために、人の身体・物件・住居などを強制的に調べること。)	そうさく		【名・他动词・サ变/三类】
\\	装飾	
\\	装饰,点缀。	そうしょく		【名词】
\\	操縦	
\\	驾驶。(機械を自分の思うとおりに動かすこと。特に、飛行機を運転すること。) 
\\	操纵,驾御,控制,支配。(他人を自分の思いどおりに動かすこと。)	そうじゅう		【名・他动词・サ变/三类】
\\	創造	
\\	创造。(新たに造ること。)	そうぞう		【名・他动词・サ变/三类】
\\	壮大	
\\	雄壮,宏大。	そうだい		【形容动词/ナ形容词】
\\	騒動	
\\	骚动,扰乱,闹事;风潮;暴乱;纠纷。 
\\	争端。	そうどう		【自动词・五段/一类】
\\	遭難	
\\	【自サ】 
\\	遇难,遇险。(山・海に行ったり乗り物・航空機を利用したりしている間に事故にあうこと。)	そうなん		【名词】
\\	相場	
\\	市价,行市。 
\\	汇率,兑换率,汇兑行情。 
\\	投机买卖,投机倒把。 
\\	社会的评价。	そうば		【名词】
\\	装備	
\\	装备,配备。配备或随身携带必要的武器、器具、附属品等,亦指其武器、器具等。(ある目的に必要な武器・器具・付属品などを備え付けたり、身に着けたりすること。また、その武器・器具など。)	そうび		【名・他动词・サ变/三类】
\\	創立	
\\	创立,创建,创办。	そうりつ		【名・他动词・サ变/三类】
\\	添える	
\\	添,附加;伴随。	そえる		【他动词・一段/二类】
\\	即座に	
\\	当即,立即,当场。	そくざに		【副词】
\\	促進	
\\	促进。	そくしん		【名・他动词・サ变/三类】
\\	即する	
\\	就,适应,结合,符合。	そくする		【自动词・サ变/三类】
\\	束縛	
\\	束缚,限制。	そくばく		【名・他动词・サ变/三类】
\\	側面	
\\	(物体的)侧面,旁面。 
\\	侧,旁。 
\\	方面。 
\\	(数学)侧面,立体图形中除上面和底面以外的面。	そくめん		【名词】
\\	其処	
\\	那儿,那里,那边。(話し手·聞き手の居る場所からそれほど遠くないと意識される所。) 
\\	那一点,那儿。(今(まで)述べた、場面や事態·事柄。) 
\\	那时。(その局面。) 
\\	同:其所	そこ		【代词】
\\	其処で	
\\	于是;因此;所以(同それで);(转话题时)那么(同さて;ところで)	そこで		【接续词】
\\	損なう	
\\	损坏,破损。 
\\	伤害,损害。 
\\	没成功,失败,错误,……坏;……错。失掉时机,耽误。差一点,险些。 
\\	损伤,死伤。	そこなう		【他动词・五段/一类】
\\	其処ら	
\\	那一带,那里。(聞き手のいる場所のあたりをおおまかにさす。そのあたり。そのへん。) 
\\	那样,那种程度,大约,左右。(聞き手に関する事柄のだいたいの程度をさす。そのくらい。)	そこら		【代词】
\\	素材	
\\	原材料,坯料。 
\\	素材,艺术作品的材料。	そざい		【名词】
\\	阻止	
\\	阻止,阻塞。(おさえとどめること。妨げること。)	そし		【名・他动词・サ变/三类】
\\	然して	
\\	然后;于是 
\\	然して;而して 
\\	そして		【接续词】
\\	訴訟	
\\	诉讼。(人の非を訴えること。)	そしょう		【自动词・サ变/三类】
\\	育ち	
\\	发育,成长,长势。亦指成长情况。(育つこと。また、成長のぐあい。) 
\\	教育,教养,抚养,出身,长大成人。成长时的环境。(成長するときの環境。育ち方。育てられ方。)	そだち		【名词】
\\	措置	
\\	措施,处理,处理办法。	そち		【名・他动词・サ变/三类】
\\	其方	
\\	那;那边(同そっち);你;汝(同おまえ;なんじ)	そちら		【代词】
\\	素っ気ない	
\\	冷淡,无情,不客气。	そっけない		【形容词/イ形容词】
\\	率直	
\\	直率;坦率。	そっちょく		【名・形容动词/ナ形容词】
\\	外方	
\\	一边,旁边. 
\\	そっぽを向く 
\\	〔よそを見る〕扭向一边. 
\\	〔相手にしない〕不理睬.	そっぽ		【名词】
\\	備え付ける	
\\	设置,备置,装置。	そなえつける		【他动词・一段/二类】
\\	備わる	
\\	设有,具有,具备。	そなわる		【自动词・五段/一类】
\\	その上	
\\	兼之,而且,再加上。加之,又。(それに加えて。さらに。)	そのうえ		【接续词】
\\	その為	
\\	因此。为此。由于那种原因(理由)。(そういうわけで。だから。それゆえ)	そのため		【接续词】
\\	其の儘	
\\	就那样;就照原样;一模一样	そのまま		【副词】
\\	聳える	
\\	耸立,峙立。 
\\	长得很高。	そびえる		【自动词・一段/二类】
\\	素朴	
\\	朴素。 
\\	单纯。	そぼく		【名・形容动词/ナ形容词】
\\	染まる	
\\	染上;沾染。	そまる		【自动词・五段/一类】
\\	背く	
\\	违抗,不遵从。 
\\	背叛。 
\\	抛弃,背离,离开。 
\\	违反,违约。 
\\	背着,背向。	そむく		【自动词・五段/一类】
\\	染める	
\\	染颜色;脸发红;沾手。	そめる		【他动词・一段/二类】
\\	逸らす	
\\	得罪人。使他人不愉快。 
\\	岔开,扭转。使转向别的方向。 
\\	偏离,错过,避开,躲开(目标)。	そらす		【他动词・五段/一类】
\\	反り	
\\	弯曲,翘曲。 
\\	弯度。 
\\	脾气不合。	そり		【名词】
\\	其れ	
\\	喂,瞧。(呼びかけ。) 
\\	那个。(相手の近くにあるもの。) 
\\	当时成为话题的事务,人。(相手の関心が向いていると思われるもの・事・人・時。) 
\\	那件事。(そのこと。) 
\\	代替前面说过的。(前に言ったこと。)	それ		【代词】
\\	其れから	
\\	其次;还有(同そして;それに加えて);然后;从那时起(同それに続いて;その後);(催促对方谈话时)后来又怎样;讲下去	それから		【接续词】
\\	各々	
\\	各,各自,每个。 
\\	大家,诸位,各位。	それぞれ		【代词】
\\	其れでも	
\\	虽然那样;即使那样;尽管如此(同それにもかかわらず)	それでも		【接续词】
\\	其れ程	
\\	那么;那样程度	それほど		【副词】
\\	其れ故	
\\	[それだから]因为那个;[だから]所以. 
\\	而且	それゆえ		【接续词】
\\	揃い	
\\	都是,全是,各个一样。(すべて。)	そろい		【名词】
\\	損失	
\\	损失。	そんしつ		【名词】
\\	存続	
\\	继续存在,永存,长存。	そんぞく		【名・自他・サ变/三类】
\\	沿い	
\\	沿,顺。	ぞい		【名词】
\\	像	
\\	像。(物のかたち。すがた。) 
\\	影像。(物体から出た光が光学系によって、屈折または反射した後、再び集合して生じた形象。実際にスクリーン上に映し出し得る実像と、眼に見えるだけの虚像との2種がある。映像。)	ぞう		【名词】
\\	増強	
\\	【他サ】 
\\	(人员设备的)加强。(人員·設備などをふやし組織の機能を強めるために、その内部の構造を充実させること。)	ぞうきょう		【名词】
\\	蔵相	
\\	藏相、大藏大臣。财政部长。(大蔵大臣のこと。)	ぞうしょう		【名词】
\\	増進	
\\	(体力、能力)增进,增加。	ぞうしん		【名・自他・サ变/三类】
\\	他意	
\\	他意,其他的想法。(他の考え。心に隠した別の考え。) 
\\	二心,异心。(ふたごころ。あだしごころ。二心。他心。)	たい		【名词】
\\	対応	
\\	对等,对应,对立,相对。 
\\	应付,适应。 
\\	协调,调和。	たいおう		【自动词・サ变/三类】
\\	退化	
\\	【自动·三类】 
\\	退化。(生物の個体発生または系統発生の過程において、器官・組織などが縮小・衰退、あるいは消失すること。人間の虫垂・尾骨などはその例。) 
\\	退步,倒退。(進歩が止まって以前の状態に逆戻りすること。また、衰えたり規模が小さくなったりすること。)	たいか		【名词】
\\	体格	
\\	体格;身体。(身長・骨格などからみた身体の外観的状態。からだつき。)	たいかく		【名词】
\\	大概	
\\	大概,概略,梗概。 
\\	差不多,不过分。	たいがい		【名词】
\\	退学	
\\	退学。	たいがく		【名・自动词・サ变/三类】
\\	大金	
\\	大量的金钱。(多くの金銭。たいきん。)	たいきん		【名词】
\\	待遇	
\\	接待,对待;服务。(客などをもてなすこと。) 
\\	待遇;工资,报酬.。(ある地位、役職に準じた扱いをする。) 
\\	表示给予某种地位、条件。多接在其他名词之后,做接尾词使用。 
\\	和其他词结合,构成新的复合词。	たいぐう		【名・他动词・サ变/三类】
\\	対決	
\\	对决,争锋相对。	たいけつ		【名・自动词・サ变/三类】
\\	体験	
\\	体验,(亲身)经验。	たいけん		【名・他动词・サ变/三类】
\\	対抗	
\\	对抗,抗衡。(互いに張り合うこと。負けまいとして競いあうこと。)	たいこう		【自动词・サ变/三类】
\\	対して	
\\	对待。	たいして		【他动词・五段/一类】
\\	大衆	
\\	群众,占社会大多数的一般人。 
\\	群众,社会学指不特定的多数人的集合。	たいしゅう		【名词】
\\	対処	
\\	处理,应付,对付。(処理。)	たいしょ		【自动词・サ变/三类】
\\	退職	
\\	退职。	たいしょく		【自动词・サ变/三类】
\\	退治	
\\	惩办,扑灭,消灭,退治,制伏,消除祸害。	たいじ		【名・他动词・サ变/三类】
\\	態勢	
\\	态势,姿态,阵势。	たいせい		【名词】
\\	対談	
\\	【自サ】 
\\	对谈;会谈;对话。(向かい合って話し合うこと。また、ある事柄について二人で話し合うこと。)	たいだん		【名词】
\\	対等	
\\	对等,同等,平等。	たいとう		【名・形容动词/ナ形容词】
\\	滞納	
\\	滞纳,拖欠,逾期未缴。(納めるべきものを、定められた期限を過ぎても納めないでいること。)	たいのう		【名・他动词・サ变/三类】
\\	対比	
\\	对比,对照。	たいひ		【名・他动词・サ变/三类】
\\	大部	
\\	大部头,指书的页数多,或全集等书籍的册数多。 
\\	大部分。	たいぶ		【名词】
\\	待望	
\\	期望,等待,期待。(それが起こるのを待ち望むこと。)	たいぼう		【名・他动词・サ变/三类】
\\	怠慢	
\\	懈怠,玩忽(职务),怠慢。懒散而不做该做的事。	たいまん		【名・形容动词/ナ形容词】
\\	対面	
\\	会面,见面。相会,碰面,遭遇,重逢。 
\\	对面。互相面对面。	たいめん		【名・自动词・サ变/三类】
\\	対立	
\\	对立。相互矛盾和争执。	たいりつ		【自动词・サ变/三类】
\\	体力	
\\	体力。身体能持续工作的能力,亦指对疾病的抵抗力。(継続的に物事を行うことが出来る、体の能力。また病気に対する抵抗力。) 
\\	体力。身体的运动能力。(体の運動能力。)	たいりょく		【名词】
\\	対話	
\\	【自サ】 
\\	对谈;对话;谈话。(向かい合って話すこと。また、その話。)	たいわ		【名词】
\\	耐える	
\\	堪,耐,抗,经受,承受。 
\\	值得。 
\\	胜任,担当,堪。	たえる		【自动词・一段/二类】
\\	絶える	
\\	断绝,终了,停止,消失。	たえる		【自动词・一段/二类】
\\	高	
\\	姓氏之一。	たか		【名词】
\\	高まる	
\\	高涨,提高,增长,兴奋。	たかまる		【自动词・五段/一类】
\\	逞しい	
\\	健壮,强壮,魁梧。体格结实强壮。(体つきががっしりしていて強そうである。) 
\\	旺盛,蓬勃,茁壮。充满活力。(活力に満ち溢れている。) 
\\	顽强,刚毅,坚韧不拔。意志坚强,无论遇到什么事都不灰心失望。(意志が強くて、どんなことにもくじけない。)	たくましい		【形容词/イ形容词】
\\	巧み	
\\	巧妙,精巧,灵巧,精彩,出色,技艺精进,巧。	たくみ		【形容动词/ナ形容词】
\\	類	
\\	类,同类。 
\\	匹敌。	たぐい		【名词】
\\	丈	
\\	手杖;拐杖;依靠;靠山。 
\\	同:杖	たけ		【名词】
\\	足し算	
\\	加法。两个或两个以上的数相加的计算。(二つ以上の数を加えてその合計を出す計算。加え算。寄せ算。)	たしざん		【名词】
\\	多数決	
\\	多数表决,多数决定。	たすうけつ		【名词】
\\	助け	
\\	帮助;援助。〔援助する。〕 
\\	救助;救济。〔金銭的な救助。〕 
\\	救命;饶命。〔助命。救う。〕 
\\	必需品,补助。(必要な品。足し。)	たすけ		【名词】
\\	携わる	
\\	参与,参加,从事,有关系。	たずさわる		【自动词・五段/一类】
\\	漂う	
\\	漂,飘荡;洋溢;露出;徘徊。	ただよう		【自动词・五段/一类】
\\	立方	
\\	立方。	たちかた		【名词】
\\	忽ち	
\\	忽然,突然。 
\\	转瞬间,立刻,马上;不大工夫。	たちまち		【副词】
\\	立ち寄る	
\\	靠近,走近;顺便到,顺路到,中途落脚。	たちよる		【自动词・五段/一类】
\\	達者	
\\	高手,达人。	たっしゃ		【名词】
\\	達成	
\\	达成,告成。	たっせい		【名・他动词・サ变/三类】
\\	尊い	
\\	珍贵的,宝贵的,尊贵的,高贵的。(崇高で近寄りがたい。神聖である。また、高貴である。)	たっとい		【形容词/イ形容词】
\\	貴い	
\\	珍贵的;贵重的;宝贵的;高贵的;尊贵的。 
\\	同:尊い	たっとい		【形容词/イ形容词】
\\	尊ぶ	
\\	敬重,尊崇。	たっとぶ		【他动词・五段/一类】
\\	絶つ	
\\	剪断,割断,斩断。 
\\	截断(后路)。 
\\	切断,阻断(资金等)。 
\\	断绝,解除关系。 
\\	失去联络。 
\\	绝望。 
\\	戒断。	たつ		【他动词・五段/一类】
\\	盾	
\\	盾,盾牌,挡箭牌。战斗时用来防护身体,免遭敌人箭、刀、矛、枪攻击的防御用武具。 
\\	盾,后盾。用作保护自己的手段的东西。	たて		【名词】
\\	建前	
\\	〈建〉上梁;上梁仪式。 
\\	主义,方针,主张;原则。	たてまえ		【名词】
\\	奉る	
\\	奉,献上。 
\\	恭维,捧。奉承。	たてまつる		【他动词・五段/一类】
\\	例え	
\\	譬喻,比喻,打比方。亦指用于比方的事或语言。(たとえること。また、たとえられた事柄や話。) 
\\	例,例子。指相同种类的事情。(同じ種類の物事。例。ためし。) 
\\	同:譬え、喩え	たとえ		【名词】
\\	仮令	
\\	(下面常与""とも
\\	""ても""连用)纵使;即使;那怕(同かりに;よしんば;たとえ) 
\\	仮令;縦令 
\\	たとえ		【副词】
\\	他動詞	
\\	他动词,日语中词的一类,相当于英语中的及物动词。(日本語で、詞の一つ。)	たどうし		【名词】
\\	辿り着く	
\\	好不容易走到,挣扎着走到。(尋ね求めながら、やっと目的地に行き着く。)	たどりつく		【他动词・五段/一类】
\\	辿る	
\\	(摸索着)前进。 
\\	追踪,追溯。 
\\	探索。 
\\	走向。 
\\	陷入迷茫。	たどる		【他动词・五段/一类】
\\	掌	
\\	手掌。从手腕起手的内侧面。(手首から先の、握った時に内側になる面。たなごころ。)	たなごころ		【名词】
\\	楽しむ	
\\	乐,快乐,享受,欣赏。 
\\	以……为消遣,使快活。 
\\	期待,以愉快的心情盼望。	たのしむ		【他动词・五段/一类】
\\	頼み	
\\	请求,恳求。求,亦指拜托的内容。 
\\	信赖,依靠。	たのみ		【名词】
\\	度々	
\\	屡次,屡屡。再三,反复,多次。(回数を重ねるさま。しばしば。何度も。)	たびたび		【副词】
\\	他方	
\\	【副】 
\\	他方,另一方向,另一方面。(他の方面。他の方向。) 
\\	另一方面,从另一方面。(一方では。別の面から見ると。)	たほう		【名词】
\\	多忙	
\\	繁忙,忙碌。	たぼう		【名・形容动词/ナ形容词】
\\	給う	
\\	给;赐 
\\	賜う;給う 
\\	给;赐予。(同おあたえになる;くださる;たまう)	たまう		【他动词・五段/一类】
\\	偶に	
\\	有时,偶尔。	たまに		【副词】
\\	堪らない	
\\	难以形容,…不得了。 
\\	难堪,忍受不了,不得了。 
\\	受不了。	たまらない		【惯用句】
\\	溜まり	
\\	积存;积存处。(たまること。) 
\\	聚集的地方,集中处;休息室,休息处。(屯している所。) 
\\	大酱汁;用大豆酱渗出的汁做的酱油。(たまりじょうゆ。) 
\\	经受,忍受。(たえること。)	たまり		【名词】
\\	賜る	
\\	拜领,蒙受赏赐,蒙赐。 
\\	赐,赐予,赏赐。	たまわる		【他动词・五段/一类】
\\	例	
\\	曾经有过的事或做过的事,先例,例子(これまでにあったことや、やったこと)。 
\\	同:前例	ためし		【名词】
\\	躊躇う	
\\	踌躇,犹豫,游移,迟疑,踟蹰不前。	ためらう		【自他・五段/一类】
\\	保つ	
\\	保持,维持。 
\\	保,保住,支持。 
\\	保持不变,保存住。	たもつ		【自他・五段/一类】
\\	容易い	
\\	容易,不难。(苦心を必要としない。わけなくできる。むずかしくない。) 
\\	轻易,轻率。(かるがるしい。軽率である。)	たやすい		【形容词/イ形容词】
\\	弛み	
\\	松弛,缓和度。	たるみ		【名词】
\\	弛む	
\\	松弛,弛缓,松懈。懒惰,散漫。内心轻松。	たるむ		【自动词・五段/一类】
\\	誰	
\\	谁。	たれ		【代词】
\\	垂れる	
\\	下垂。(木の枝などがだらりと垂れ下がる) 
\\	同:枝垂れる	たれる		【自动词・一段/二类】
\\	反	
\\	布匹的长度单位 一反约宽34厘米,长10米。土地面积单位, 一反等于992平方米。(尺貫法の面積の単位である。土地の面積に使われる反と、布の大きさを表す反とがある。これとは別に6間の長さを表す反もある。)	たん		【名词】
\\	単一	
\\	单一。 
\\	单独。 
\\	简单,单一。	たんいつ		【名・形容动词/ナ形容词】
\\	短歌	
\\	短歌。	たんか		【名词】
\\	担架	
\\	担架。	たんか		【名词】
\\	探検	
\\	探险。 
\\	同:探険	たんけん		【名・他动词・サ变/三类】
\\	短縮	
\\	缩短。	たんしゅく		【名・他动词・サ变/三类】
\\	箪笥	
\\	衣橱,衣柜;多屉柜。(引出の付いた、衣服などを入れておくための家具。) 
\\	衣柜,衣橱。用于保存衣物和小道具等的家具的总称。(衣服・小道具などを整理・保管するのに用いる家具の総称。)	たんす		【名词】
\\	炭素	
\\	〈化〉碳。元素符号是
\\	。非金属元素。	たんそ		【名词】
\\	短大	
\\	短期大学。(短期大学の略。)	たんだい		【名词】
\\	単調	
\\	单调,平庸,无变化,无抑扬顿挫。	たんちょう		【名・形容动词/ナ形容词】
\\	単独	
\\	单独,独自,单身,单个。	たんどく		【名词】
\\	短波	
\\	短波。	たんぱ		【名词】
\\	蛋白質	
\\	〈生化〉蛋白质,朊。	たんぱくしつ		【名词】
\\	第	
\\	第。(数を表す語に付いて、ものの順序を表すのに用いる。)	だい		【接头/接尾】
\\	第一	
\\	首先。 
\\	最好(的),最佳,首要,头等,最重要(的)。 
\\	第一,首先。	だいいち		【名词・副词】
\\	大胆	
\\	大胆,胆子大,勇敢,无畏。	だいたん		【名・形容动词/ナ形容词】
\\	台無し	
\\	作践。糟蹋。不能使用。	だいなし		【名・形容动词/ナ形容词】
\\	大便	
\\	大便,粪。(人間が肛門から排泄する食物のかす。くそ。)	だいべん		【名词】
\\	代弁	
\\	代(替赔)偿。 
\\	代办事务,代理(职务);代言,代辩。	だいべん		【名・他动词・サ变/三类】
\\	台本	
\\	(意为台词的书)脚本,戏剧、电影等的脚本。	だいほん		【名词】
\\	代用	
\\	【他サ】 
\\	代用,代替,替用。(本来使うべきものの代わりにまにあわせとして臨時に使うこと。)	だいよう		【名词】
\\	打開	
\\	打开,开辟(途径),解决(问题)。	だかい		【名・他动词・サ变/三类】
\\	妥協	
\\	妥协,和解。	だきょう		【名・自动词・サ变/三类】
\\	妥結	
\\	妥协。	だけつ		【名・自动词・サ变/三类】
\\	打撃	
\\	打击。 
\\	击球。	だげき		【名词】
\\	駄作	
\\	拙劣的作品,无价值的作品。	ださく		【名词】
\\	脱出	
\\	逃出,逃脱,逃亡。	だっしゅつ		【名・自动词・サ变/三类】
\\	脱する	
\\	逃出,逃脱。脱离,离开。 
\\	脱落,漏掉。 
\\	摆脱。脱出。 
\\	脱离,脱。离开朋友或集体。 
\\	完稿,脱稿。 
\\	去掉,除掉。	だっする		【自他・サ变/三类】
\\	脱退	
\\	脱离,退出。	だったい		【自动词・サ变/三类】
\\	騙す	
\\	骗,欺骗。 
\\	哄。哄骗。 
\\	凑合。将就。勉强。	だます		【他动词・五段/一类】
\\	怠い	
\\	懒倦的;慵懒的	だるい		【形容词/イ形容词】
\\	壇	
\\	坛。(仲間うち意識によってささえられる、同じ学芸に従事する人たち。)	だん		【名词】
\\	団結	
\\	团结。同一心愿的许多人为了达到目的而结成集团。	だんけつ		【自动词・サ变/三类】
\\	断言	
\\	断言。	だんげん		【名・他动词・サ变/三类】
\\	断然	
\\	断然,绝对。程度差别明显。 
\\	断然。大胆地干。 
\\	断然,坚决,态度十分干脆。	だんぜん		【副词】
\\	旦那	
\\	〈仏〉施主,檀越。(施主·檀越·檀家。 
\\	主人,老爷。(あるじ。) 
\\	老爷(自分の);丈夫(他人の)(夫。) 
\\	先生。(商売での得意先。) 
\\	老爷。(尊敬する男性などに。) 
\\	檀那;旦那 
\\	公子。来源《大奥男女逆转》里阿信称呼水野为“旦那”。但其并非夫妇,而是青梅竹马长大的朋友。	だんな		【名词】
\\	断面	
\\	断面,截面,剖面。	だんめん		【名词】
\\	弾力	
\\	弹力,弹性。 
\\	弹性,灵活。指通融起作用。	だんりょく		【名词】
\\	治安	
\\	治安。	ちあん		【名词】
\\	近付く	
\\	挨近,靠近;临近;接近,交往;越来越相似。	ちかづく		【自动词・五段/一类】
\\	違える	
\\	违反,违背;使不一致,使有差别。	ちがえる		【他动词・一段/二类】
\\	契る	
\\	誓约,约定。 
\\	结为夫妇。 
\\	发生肉体关系。	ちぎる		【他动词・五段/一类】
\\	畜産	
\\	〈农〉畜产。	ちくさん		【名词】
\\	畜生	
\\	(源于梵语)畜生,动物,兽,畜类。鸟兽虫鱼的总称,前世作恶来生得到报应的形式之一。(〔仏〕〔梵 
\\	鳥獣虫魚の総称。前世の悪業の報いとして受ける生の形の一つ。愚かで、肉親をも傷つけ、苦しみが多い。人間に価しないものの意で、卑劣な人や不道徳な人をいう。ちきしょう。) 
\\	畜生,混帐东西,浑〔混〕蛋,他妈的。骂人时、后悔时、生气时说的话。(人をののしったり、悔しい時や腹が立った時などに発する語。)	ちくしょう		【名词】
\\	蓄積	
\\	积蓄,蓄积,积存,积累,积攒。(蓄え貯めること。)	ちくせき		【名・他动词・サ变/三类】
\\	知性	
\\	才智;智力;智能;理智。	ちせい		【名词】
\\	乳	
\\	乳,奶。(乳汁。ちち。) 
\\	乳房。(乳房。また、乳首。) 
\\	小环。旗、幕边缘上用以穿竿或绳等的小环,以及和服外褂、草鞋等上用以穿袋子的小环。(旗・幕・羽織・草鞋などの縁に、竿・ひもなどを通すためにつけた小さな輪。みみ。) 
\\	疙瘩。(釣り鐘の表面の多数並んだ小さな疣状の突起。)	ちち		【名词】
\\	父母	
\\	父母。(父と母。両親。)	ちちはは		【名词】
\\	縮まる	
\\	缩,抽,缩小,缩短,收缩,抽缩。〔収縮する。〕 
\\	恐惧,害怕。〔すくむ。縮んだ状態になる。〕	ちぢまる		【自动词・五段/一类】
\\	窒息	
\\	窒息。变得不能呼吸。熏,憋。	ちっそく		【名・自动词・サ变/三类】
\\	些とも	
\\	(下接否定语)一点(也不);毫(无)	ちっとも		【副词】
\\	秩序	
\\	秩序(正しい順序);条理(すじみち);次序。(順序。) 
\\	秩序。(社会の。)	ちつじょ		【名词】
\\	知的	
\\	有知识的,智慧的,智力的。 
\\	聪明的,理智的,理性的。	ちてき		【形容动词/ナ形容词】
\\	著	
\\	著,著作,著述,著书。(書物を書くこと。また、書物。著作。) 
\\	显著,明显,显出。(明らかであること。顕著。)	ちゃく		【名词】
\\	着	
\\	件,身;[そろいの]套。(衣服をかぞえる単位。) 
\\	着,名。(競技で到着の順序を数える語。) 
\\	着。(囲碁で、石を打つ回数を数えるのに用いる。) 
\\	到达;抵达到达,抵达。(いくつくこと。到着。)	ちゃく		【接头/接尾】
\\	着手	
\\	下手。 
\\	刑法上指实施犯罪的开始。	ちゃくしゅ		【名・自动词・サ变/三类】
\\	着色	
\\	着色,上颜色;涂假色。(物に色をつけること。)	ちゃくしょく		【自动词・サ变/三类】
\\	着目	
\\	着眼,注目。	ちゃくもく		【名・自动词・サ变/三类】
\\	着陸	
\\	着陆,降落。	ちゃくりく		【名・自动词・サ变/三类】
\\	着工	
\\	开工,动工,兴工。(土木・建築などの工事を始めること。)	ちゃっこう		【自动词・サ变/三类】
\\	茶の間	
\\	(家庭中邻近厨房的)餐室,起居间;茶室。	ちゃのま		【名词】
\\	茶の湯	
\\	茶道,品茗会。(客を茶室に招き入れ、茶をたててすすめること(作法)。茶道。)	ちゃのゆ		【名词】
\\	昼間	
\\	白天。	ちゅうかん		【名词】
\\	宙返り	
\\	翻筋斗。	ちゅうがえり		【名・他动词・サ变/三类】
\\	中継	
\\	转播。 
\\	中继。	ちゅうけい		【名・他动词・サ变/三类】
\\	忠告	
\\	忠告,劝告。(心から相手のことを思い、早く悪いところを直す。)	ちゅうこく		【名・自他・サ变/三类】
\\	中指	
\\	中指。(五本の指の真ん中の指。)	ちゅうし		【名词】
\\	中傷	
\\	中伤,污蔑。	ちゅうしょう		【名・他动词・サ变/三类】
\\	忠実	
\\	忠诚,忠实。 
\\	照录,照转,尽量正确地转达。	ちゅうじつ		【名・形容动词/ナ形容词】
\\	中枢	
\\	中枢;中心。 
\\	枢纽,关键。	ちゅうすう		【名词】
\\	抽選	
\\	抽签,拈阄儿,抓阄儿。(くじを引くこと。くじびき。)	ちゅうせん		【自动词・サ变/三类】
\\	中断	
\\	中断。因出现一定的事由而失去至此的效力。 
\\	中断,中辍。中途暂时切断,或暂时断绝。	ちゅうだん		【名・自他・サ变/三类】
\\	中腹	
\\	半山腰。	ちゅうっぱら		【名词】
\\	中毒	
\\	中毒,上瘾。由于物质的毒性而引起功能障碍,如食物中毒、药物中毒、细菌中毒等。	ちゅうどく		【名・自动词・サ变/三类】
\\	仲人	
\\	调解人,说和人,调停人。(相対立している両者の間に入って仲裁する人。仲裁人。) 
\\	媒人,介绍人。(なこうど。媒酌人。)	ちゅうにん		【名词】
\\	昼飯	
\\	午饭。(ひるめし。昼食。) 
\\	同:中飯	ちゅうはん		【名词】
\\	中立	
\\	中立。(いずれにもかたよらずに中正の立場をとること。)	ちゅうりつ		【自动词・サ变/三类】
\\	中和	
\\	性格中和,温和。 
\\	中和,冲淡。 
\\	〈化〉中和,中和反应。 
\\	〈电〉中和,抵消。	ちゅうわ		【名・他动词・サ变/三类】
\\	腸	
\\	心肠,心地。 
\\	动物的内脏。 
\\	肠,内脏。	ちょう		【名词】
\\	蝶	
\\	蝴蝶。	ちょう		【名词】
\\	超	
\\	(接在数词后)超过……。	ちょう		【名词】
\\	調印	
\\	签订,签字,盖章。	ちょういん		【名・自动词・サ变/三类】
\\	聴覚	
\\	听觉。	ちょうかく		【名词】
\\	長官	
\\	知事,解放前用语。 
\\	长官,机关首长。	ちょうかん		【名词】
\\	聴講	
\\	听讲,旁听。(講義を聴くこと。)	ちょうこう		【名・他动词・サ变/三类】
\\	徴収	
\\	征收。	ちょうしゅう		【名・他动词・サ变/三类】
\\	聴診器	
\\	听诊器。	ちょうしんき		【名词】
\\	挑戦	
\\	挑战。(戦いを挑むこと。)	ちょうせん		【自动词・サ变/三类】
\\	調停	
\\	调停,调解。	ちょうてい		【名・他动词・サ变/三类】
\\	長編	
\\	长篇小说。 
\\	长篇。	ちょうへん		【名词】
\\	丁目	
\\	【参考】 
\\	中国城市地区划分不像日本那样仔细,一半只要写“……市……街(または……路)……号”邮包就能寄到。用“……巷”“……胡同”替代“……街”的用法也是有的。 
\\	中国の都市の番地は日本のように細分化されておらず、普通は“……市……街(または……路)……号”だけで郵便物などは届く。“……街”のかわりに“……巷”や“……胡同”などの番地が用いられることもある。	ちょうめ		【接头/接尾】
\\	調理	
\\	烹调,烹饪,做(菜),煮菜。	ちょうり		【名・他动词・サ变/三类】
\\	調和	
\\	调和,颜色配合,声音和谐,关系协调。	ちょうわ		【名・自动词・サ变/三类】
\\	直面	
\\	面临,面对。	ちょくめん		【名・自动词・サ变/三类】
\\	著書	
\\	著作,著述,写作,著书。	ちょしょ		【名词】
\\	貯蓄	
\\	储蓄,积蓄。储存金钱等财务。(金銭をたくわえること。また、その金銭。)	ちょちく		【名・他动词・サ变/三类】
\\	直感	
\\	直觉,直感,直观,直接感觉。	ちょっかん		【名・他动词・サ变/三类】
\\	一寸	
\\	一寸。一尺的十分之一。约3.03
\\	。(一尺の一〇分の一。約3.03センチメートル。) 
\\	极短的距离〔时间〕。比喻很少。(短い距離・時間・寸法。わずかなことのたとえ。)	ちょっと		【名词】
\\	著名	
\\	著名,有名,出名。	ちょめい		【名・形容动词/ナ形容词】
\\	塵取り	
\\	簸箕、畚箕。(掃き集めた塵を取り集める道具。)	ちりとり		【名词】
\\	治療	
\\	治疗,医疗,医治,治。	ちりょう		【名・他动词・サ变/三类】
\\	賃金	
\\	工资,薪金,作为劳动报酬支付的金钱。	ちんぎん		【名词】
\\	沈殿	
\\	沉淀,因溶液中的化学反应而产生的不溶性物质或温度等变化后达到饱和状态的溶质,在液体中形成固体出现的现象,亦指其固体。 
\\	沉淀,不溶于液体的物质沉到液体底层。	ちんでん		【名・自动词・サ变/三类】
\\	沈没	
\\	醉得人事不省。 
\\	沉没,船等沉入水中。	ちんぼつ		【名・自动词・サ变/三类】
\\	沈黙	
\\	沉默,默不作声。	ちんもく		【名・自动词・サ变/三类】
\\	陳列	
\\	展览,陈列。(見せるために物品をならべておくこと。)	ちんれつ		【名・他动词・サ变/三类】
\\	追及	
\\	追赶;追上,赶上。(あとから追いかけること。) 
\\	追究;追问。(責任、欠点からどこまでも問いただすこと。)	ついきゅう		【名・他动词・サ变/三类】
\\	追跡	
\\	追踪,跟踪追赶,追缉。	ついせき		【名・他动词・サ变/三类】
\\	次いで	
\\	接着,随后。(その次に。ひき続いて。)	ついで		【接续词】
\\	追放	
\\	放逐,驱逐(出境);赶出,赶走,赶出去。(追い払うこと。追い出すこと。危険人物または不法入国者を国外に退去させること。) 
\\	驱除,肃清,洗刷,扫除。(不要または有害なものとして、その社会から追い払うこと。) 
\\	开除,革职,清洗。(一定の職業・地位から退かせること。)	ついほう		【名・他动词・サ变/三类】
\\	費やす	
\\	消耗,耗费;白费,浪费。	ついやす		【他动词・五段/一类】
\\	墜落	
\\	坠落,掉下;摔下。	ついらく		【自动词・サ变/三类】
\\	通	
\\	通。流通。(とおること。) 
\\	道路。大街。马路。(通路。人や車などが通るための、まちなかの道) 
\\	传音。声音传送。声音传递方式。(声や音などがよく伝わること。) 
\\	声誉。名声。评价。(受け入れられること。了解されること。広く知れわたっていること。評判。) 
\\	通顺。易懂。理解。(通じること。理解。のみこみ。) 
\\	同样。原样。照样。表示与前面所述状态相同。(それと同じ経過をたどる、また同じ状態にあること。それと同様なこと。) 
\\	(助数)计算组、种类或次数等的量词。(方式・様式の種類を数える語)	つう		【名词】
\\	痛感	
\\	痛感,深切地感觉到(认识到)。	つうかん		【名・他动词・サ变/三类】
\\	通常	
\\	通常,平常,普通,一般。	つうじょう		【名词・副词】
\\	痛切	
\\	深切;迫切。	つうせつ		【名词】
\\	杖	
\\	手杖,拐杖;棍子。走路时手里拄着作为步行辅助工具的细长木棍或竹棍。(歩く時、手に持って地面につき、歩行の助けとする細長い木や竹の棒。) 
\\	滑雪杖。(スキーのストック。) 
\\	依靠,靠山。可以依赖的事物。(頼りにするもの。)	つえ		【名词】
\\	遣い	
\\	使用,使用的人。(使うこと。使う人。) 
\\	派遣,派出去的人,使者。(用足しのために人をさしむけること。また、その人。使者。) 
\\	代表神、佛之类的动物,神佛的使者。(神仏の使者とされる動物。つかわしめ。) 
\\	同:使い	つかい		【名词】
\\	使い道	
\\	用途,用处。 
\\	用法,使法。	つかいみち		【名词】
\\	仕える	
\\	(对长辈)服侍,侍奉,伺候。 
\\	做官,当官,服务。	つかえる		【自动词・一段/二类】
\\	司る	
\\	主持,担任。 
\\	掌管,司掌。 
\\	管理,支配。	つかさどる		【他动词・五段/一类】
\\	束の間	
\\	一刹那,转眼之间,转瞬,瞬息间。	つかのま		【名词】
\\	付き	
\\	付きが回る。/时来运转。	つき		【惯用句】
\\	付き合う	
\\	陪伴,奉陪,应酬;交际,交往,来往。	つきあう		【自动词・五段/一类】
\\	月並み	
\\	每月,月月,按月。 
\\	平凡,平庸;陈腐。	つきなみ		【名词】
\\	尽きる	
\\	尽,罄尽;到头;只限于。	つきる		【自动词・一段/二类】
\\	継ぎ目	
\\	接缝。焊口。物体的相连接部分。 
\\	后嗣。继承人。	つぎめ		【名词】
\\	吐く	
\\	说出。 
\\	呼吸,出气。	つく		【他动词・五段/一类】
\\	尽くす	
\\	尽,尽力,力。 
\\	为……尽力,报效,贡献力量。 
\\	达到极点。	つくす		【他动词・五段/一类】
\\	作り	
\\	制造,构造,结构,样式。制作器物,亦指制成后的样子。 
\\	化妆,打扮。 
\\	体格。 
\\	生鱼片。 
\\	假装,装作。故意装出某种样子。 
\\	同:造り	つくり		【名词】
\\	造り	
\\	假装;装做。 
\\	农作物。 
\\	化妆;打扮。 
\\	身材;体格。 
\\	(建筑物)的构造,制造(的样式)。 
\\	同:作り	つくり		【名词】
\\	造る	
\\	铸造。 
\\	建筑,修(建)。 
\\	建造。 
\\	创造。	つくる		【他动词・五段/一类】
\\	繕う	
\\	修理,修缮。 
\\	整理,修饰。 
\\	敷衍,弥缝。	つくろう		【他动词・五段/一类】
\\	接ぐ	
\\	连接。接合。(結び合わせたり、足したりして、一続きのものとする。) 
\\	同:つぐ	つぐ		【他动词・五段/一类】
\\	継ぐ	
\\	连接,连上。 
\\	缝补。 
\\	添加,补给,续上。 
\\	继续;接,接上。 
\\	继承,承袭,嗣。	つぐ		【他动词・五段/一类】
\\	付け加える	
\\	增加,添加,附加,补充。	つけくわえる		【他动词・一段/二类】
\\	告げる	
\\	告诉,通知,传话。 
\\	报告,宣告。 
\\	告知,宣言,告。	つげる		【他动词・一段/二类】
\\	辻褄	
\\	条理,道理;首尾,前后。	つじつま		【名词】
\\	途中	
\\	途中,从某场所向另一场所行进之间,来去的半路中。 
\\	中半途,半道,事情尚未做完之前。	つちゅう		【名词】
\\	突っ張る	
\\	顶上,支上,支撑。 
\\	抽筋;突然剧痛。 
\\	(相扑)使劲猛推。 
\\	顶,猛烈反驳;坚持己见。 
\\	虚张声势。	つっぱる		【自他・五段/一类】
\\	筒	
\\	〔さいころを入れて振る筒〕筒。摇骰子的盒。亦指掷色子时摇筒的人。(双六(すごろく)や博打(ばくち)で、さいころを中に入れて振るつつ。また、それを振る人。) 
\\	局东,赌头,庄家。((「胴」とも書く)「筒元(どうもと)」「筒親(どうおや)」「筒取(どうとり)」の略。)	つつ		【名词】
\\	突く	
\\	连续不断的刺,戳,捅。(何回も突く。また、軽く突く。) 
\\	煽动,挑唆。(けしかける。煽動する。) 
\\	挑毛病。(欠点や落度を取り上げて咎める。) 
\\	用筷子夹。(箸で食べ物をつまみとって食べる。)	つつく		【他动词・五段/一类】
\\	慎む	
\\	谨慎,慎重,小心。 
\\	节制。 
\\	斋戒。 
\\	恭谨,有礼貌。	つつしむ		【自动词・五段/一类】
\\	伝言	
\\	传话,口信;带口信。	つてごと		【名・他动词・サ变/三类】
\\	勤まる	
\\	能担任,胜任。	つとまる		【自动词・五段/一类】
\\	勤め先	
\\	工作地点;工作岗位。(勤めているところ。勤務先。)	つとめさき		【名词】
\\	努めて	
\\	尽量,竭力,努力,特别注意。	つとめて		【副词】
\\	抓る	
\\	掐,拧。(つめや指先で皮膚を少しつまんでねじる。)	つねる		【他动词・五段/一类】
\\	募る	
\\	激化;越来越严重。	つのる		【自动词・五段/一类】
\\	唾	
\\	唾液,唾沫。	つば		【名词】
\\	呟く	
\\	嘟哝,嘟囔,发牢骚,唠叨,唧咕。	つぶやく		【自动词・五段/一类】
\\	瞑る	
\\	瞑目,闭眼。	つぶる		【他动词・五段/一类】
\\	壷	
\\	壶;罐;坛;罐;坛子。 
\\	坑洼,中庭。 
\\	瓷钵。 
\\	要点;要害;关键。 
\\	企图;估计;预料。 
\\	穴位,灸点。	つぼ		【名词】
\\	蕾	
\\	未成年(的人)。 
\\	花骨朵,花蕾,蓓蕾。	つぼみ		【名词】
\\	躓く	
\\	绊(倒),摔倒,跌交。(行走时)因脚尖被东西挡挂住,导致身体踉跄。 
\\	跌跤,栽跟头,失败,受挫,受到挫折。因事情中途遇到障碍而不能顺利进行下去,中途失败。	つまずく		【自动词・五段/一类】
\\	摘む	
\\	摘;剪齐。	つまむ		【他动词・五段/一类】
\\	詰らない	
\\	堆满,挤满;(沟通)无法进行;变短。	つまらない		【他动词・五段/一类】
\\	詰まり	
\\	归根到底。总之。(結局。要するに。) 
\\	就是说。用以加强语气,即。(強めのために使う。すなわち。)	つまり		【副词】
\\	積もり	
\\	打算,意图,企图。(前もってもっている考え。意図。心ぐみ。) 
\\	估计,预计。(予想して計算すること。見積もり。) 
\\	就当作,就算是。(実際はそうでないのに、そうなったと仮定した気持ち。) 
\\	宴会时最后一杯(瓶)酒。(酒宴で、その酌で終わりにすること。おつもり。)	つもり		【名词】
\\	露	
\\	短暂,无常。 
\\	泪。 
\\	露,露水。	つゆ		【名词】
\\	強まる	
\\	强烈起来;强硬起来。	つよまる		【自动词・五段/一类】
\\	強める	
\\	加强,增强。	つよめる		【他动词・一段/二类】
\\	連なる	
\\	成行,成排,成列,连接,连绵。 
\\	牵连,关联。 
\\	列席,参加。 
\\	末了归一。	つらなる		【自动词・五段/一类】
\\	貫く	
\\	贯通,穿透,穿过。 
\\	贯彻始终,达到目的,坚持到底。	つらぬく		【他动词・五段/一类】
\\	連ねる	
\\	连成一排,排列成行。 
\\	连接,连上,罗列。连通,相接,连成串。 
\\	(作为其中一员)加入团体或组织等。 
\\	会同,伴同。参加,联,列。	つらねる		【他动词・一段/二类】
\\	釣り	
\\	つり道具。/钓鱼具。	つり		【名词】
\\	吊り革	
\\	(电车等的)吊环;吊带。	つりかわ		【名词】
\\	吊るす	
\\	挂、吊、悬。用带子、绳索等悬挂东西。	つるす		【他动词・五段/一类】
\\	手当て	
\\	准备;预备(ある物事を予測して用意しておくこと。準備)。 
\\	(对伤病的)处置,救治
\\	病気やけがの処置を施すこと。また、その処置)。 
\\	报酬;工资(労働の報酬として支払われる金銭)。 
\\	(基本工资之外的报酬)补贴,津贴(基本の賃金のほかに諸費用として支払われる金銭)。	てあて		【名・他动词・サ变/三类】
\\	体	
\\	样子;打扮。(有り様。様子。) 
\\	情况,状态。(物事のようす。) 
\\	姿态。(見た目)	てい		【名词】
\\	定義	
\\	定义。	ていぎ		【名・他动词・サ变/三类】
\\	提携	
\\	协作,合作,协作。互相帮助,同心协力干事业等。	ていけい		【自动词・サ变/三类】
\\	体裁	
\\	样子,样式,门面,外表,外形。 
\\	体面,体统。 
\\	(应有的)形式,局面,体裁。 
\\	奉承话,溜须拍马。	ていさい		【名词】
\\	定食	
\\	套餐,份儿饭,经济饭,包饭。	ていしょく		【名词】
\\	提示	
\\	出示,提出。	ていじ		【名・他动词・サ变/三类】
\\	訂正	
\\	订正;修订。(誤りを正しく改めること。)	ていせい		【名・他动词・サ变/三类】
\\	停滞	
\\	停滞,停顿。 
\\	滞销;积压。	ていたい		【自动词・サ变/三类】
\\	邸宅	
\\	宅邸。住宅。	ていたく		【名词】
\\	定年	
\\	退休年龄。 
\\	同:停年	ていねん		【名词】
\\	堤防	
\\	圩堤,堤防,堤坝,堤,防止水浪侵入沿岸和河川泛滥的构筑物。	ていぼう		【名词】
\\	手遅れ	
\\	耽误,为时已晚,错过时机。	ておくれ		【名词】
\\	手数	
\\	费心,麻烦。 
\\	费事。	てかず		【名词】
\\	手掛かり	
\\	抓头,抓手,攀附之处。(手をかける所。とりつく所。) 
\\	(侦察犯罪等的)线索。用以思考或调查的头绪。(考えたり調べたりするためのよすがとなるもの。いとぐち。) 
\\	同:手懸かり	てがかり		【名词】
\\	手掛ける	
\\	亲自动手,亲手做。 
\\	亲自照料,亲自照管,亲自培养。	てがける		【他动词・一段/二类】
\\	手軽	
\\	简单,简便,轻易。	てがる		【名词】
\\	的	
\\	的,靶子。(弓や銃砲の発射練習をする時、目標として立ってておくもの。) 
\\	目标,目的。(めあて。目標。目的。また、世間の非難や興味の対象となるもの。) 
\\	〔要点〕要害,要点。 
\\	家徽的名称。(紋所の名。)	てき		【名词】
\\	適応	
\\	适应,顺应。(その状況によくかなうこと。) 
\\	生物的诸性状适应在生活环境下的生存和繁殖。(生物の形態・習性などの形質が、その環境で生活・繁殖するのに適合していること、あるいはそう判断できること。)	てきおう		【自动词・サ变/三类】
\\	適宜	
\\	酌情,酌量;随意。 
\\	适宜,适当,合适。	てきぎ		【形容动词/ナ形容词】
\\	適性	
\\	适于某人的性质(资质,才能),适应性。	てきせい		【名词】
\\	手際	
\\	做出的结果。 
\\	手腕,本领。 
\\	(处理事物的)手法,技巧;程序。	てぎわ		【名词】
\\	手順	
\\	(工作的)次序,步骤,程序。	てじゅん		【名词】
\\	手錠	
\\	手铐。	てじょう		【名词】
\\	手近	
\\	手边,眼前,近旁。 
\\	尽人皆知,常见,浅近。	てぢか		【名・形容动词/ナ形容词】
\\	鉄鋼	
\\	钢铁。(鉄を主成分とする,銑鉄(せんてつ)・鋼鉄などの鉄材の総称。)	てっこう		【名词】
\\	徹する	
\\	彻,贯彻,透彻,贯穿;铭刻(于心)。渗透到深处。 
\\	彻底,贯彻始终,始终如一。 
\\	彻(夜),通(宵)。经过某段时间。	てっする		【自他・サ变/三类】
\\	鉄片	
\\	铁片。(鉄の破片。鉄の小片。)	てっぺん		【名词】
\\	手配	
\\	准备,筹备,安排。 
\\	通缉,逮捕犯人时的指令和部署。	てはい		【名・自动词・サ变/三类】
\\	手筈	
\\	程序;步骤;事前的准备。(物事を行うのに、前もってする必要な準備。)	てはず		【名词】
\\	手引き	
\\	拽,用手拉。 
\\	引路,向导。 
\\	辅导(初学者),启蒙。	てびき		【他动词・サ变/三类】
\\	手本	
\\	字帖,画帖,范本。(絵・字のテキスト。) 
\\	模范,榜样,样板。(模範。)	てほん		【名词】
\\	手回し	
\\	准备,预先筹划;布置;安排。 
\\	用手摇动。〔機械を手で回すこと。また、その機械。〕	てまわし		【名词】
\\	手元	
\\	身边,手头,手里(手の下。手の届くあたり)。 
\\	生活,生计,手头(手元にある金。懐具合)。 
\\	手的动作(手つき。手の動き)。 
\\	把手(器具などの,手で持つ部分)。	てもと		【名词】
\\	照り返す	
\\	反射,反照,反光。把从别处获得的光和热反挡回去。(他から受けた光や熱をはね返す。)	てりかえす		【他动词・五段/一类】
\\	店	
\\	商店,店铺。 
\\	格子门状的房间。	てん		【名词】
\\	点火	
\\	点火,点燃。	てんか		【名・自动词・サ变/三类】
\\	転回	
\\	【自他动・三类】 
\\	回转,转变。(くるりと回って、向きを変えること。向きがぐるりと変わること。) 
\\	〈体〉回旋,旋转。(体操で、からだを回転させる運動。前方転回・後方転回・空中転回など。) 
\\	〈乐〉(和音中、上下音的)转换。(和音で、根音以外の音が最下声となるように、音の上下関係を置き換えること。)	てんかい		【名词】
\\	転換	
\\	【自他动·三类】 
\\	转换,转变。(別のものに変えること。向きを変えること。また、変わること)	てんかん		【名词】
\\	転居	
\\	迁居,搬家。(住居を変えること。引っ越し。やどがえ。転宅。)	てんきょ		【自动词・サ变/三类】
\\	転勤	
\\	转职,转调工作。	てんきん		【名・自动词・サ变/三类】
\\	点検	
\\	检点,检查。	てんけん		【名・他动词・サ变/三类】
\\	転校	
\\	转校,转学。	てんこう		【自动词・サ变/三类】
\\	天国	
\\	天堂,天国。(神や天使が住む、天上の理想郷。キリスト教・イスラム教では、信仰を貫いた者が死後に赴き永生を得る所とする。) 
\\	理想境界,乐园。(理想的な世界。悩みや危険のない楽しい所。)	てんごく		【名词】
\\	天才	
\\	天才。	てんさい		【名词】
\\	天災	
\\	天灾,自然灾害。	てんさい		【名词】
\\	展示	
\\	展示,展出,陈列。(作品などを並べて、多くの人にみせること。)	てんじ		【名・他动词・サ变/三类】
\\	天井	
\\	顶点,最高限度。 
\\	物体内部最高处。 
\\	顶棚,天花板。	てんじょう		【名词】
\\	転じる	
\\	转。改换(方向、状态),亦指改变。	てんじる		【自他・一段/二类】
\\	点線	
\\	虚线,点线。	てんせん		【名词】
\\	天体	
\\	天体。恒星、行星、星云、星团、星际物质、太阳、月亮等存在于宇宙间的物体的总称。(恒星・惑星・星雲・星団・星間物質・太陽・月など、宇宙に存在する物体の総称。)	てんたい		【名词】
\\	転転	
\\	转来转去;辗转;滚转貌;叽哩咕噜	てんてん		【自动词・サ变/三类】
\\	転任	
\\	调职,调动工作。	てんにん		【名・自动词・サ变/三类】
\\	展望	
\\	展望,瞭望,眺望。	てんぼう		【名・他动词・サ变/三类】
\\	転落	
\\	顛落】 
\\	滚下,掉下。(ころげ落ちること。) 
\\	落魄,潦倒。(上位から下位に一挙に落ちること。急激に落ちぶれること。) 
\\	堕落,沦落。(堕落すること。身をもちくずすこと。) 
\\	同:顛落	てんらく		【自动词・サ变/三类】
\\	出合う	
\\	〔ゆきあう〕遇见,遇上,遇到,碰见,碰上,碰到,相逢. 
\\	〔あいびきする〕约会;[こっそり]幽会. 
\\	〔川などが〕汇合.	であう		【自动词・五段/一类】
\\	出入り口	
\\	出入口。	でいりぐち		【名词】
\\	出来物	
\\	了不起的人物,出色的人物。(人格・才能のすぐれた人物。)	できもの		【名词】
\\	出切る	
\\	全部出去,全部出来。(全部出る。残らず出てしまう。出つくす 。)	できる		【自动词・五段/一类】
\\	出くわす	
\\	偶遇,碰见,遇上。偶然遇见。	でくわす		【自动词・五段/一类】
\\	出鱈目	
\\	荒唐;胡来,瞎胡闹;胡扯,瞎说,胡说八道『成』。	でたらめ		【名・形容动词/ナ形容词】
\\	出直し	
\\	从头开始做,重新开始。 
\\	回来再去。	でなおし		【名词】
\\	電源	
\\	电源。	でんげん		【名词】
\\	伝説	
\\	传说,口传。(うわさ。風説。)	でんせつ		【名词】
\\	電線	
\\	电线。作为输送电流的导体所使用的金属线。	でんせん		【名词】
\\	伝達	
\\	传达,转达。(命令・連絡事項などを伝えること。つぎつぎに伝え届けること。)	でんたつ		【名・他动词・サ变/三类】
\\	伝来	
\\	(从外国)传来,传入。(外国から伝わってくること。渡来。) 
\\	祖传,世传。(代々うけつぎつたえること。相伝。)	でんらい		【自动词・サ变/三类】
\\	問い合わせる	
\\	询问,打听,就不明确处进行询问、查询。	といあわせる		【他动词・一段/二类】
\\	問屋	
\\	批发商,批发店。 
\\	同:といや	といや		【名词】
\\	問う	
\\	问,询问;调查,追究;当做问题。	とう		【他动词・五段/一类】
\\	棟	
\\	座,栋,所,幢。	とう		【助数词】
\\	等	
\\	等。(等しい。) 
\\	等,等级;[賞]等奖,……名。(順位。階級。クラス。) 
\\	等,等等,诸如此类。(同種のものを並べあげて、その他にもまだあることを表わす語。…など。)	とう		【名词】
\\	陶器	
\\	陶器,陶瓷器。	とうき		【名词】
\\	等級	
\\	星等。表示天体亮度级别的数值,星等越大,亮度越暗,亮星是从零等向负星等排列,如太阳的视星等为负27等。 
\\	等级。按地位高低、质量优劣划分的级别。	とうきゅう		【名词】
\\	討議	
\\	【自他サ】 
\\	讨论。(ある題目について意見を戦わせること。)	とうぎ		【名词】
\\	登校	
\\	上学;到校。	とうこう		【自动词・サ变/三类】
\\	統合	
\\	统一,综合,统归为一。	とうごう		【名・他动词・サ变/三类】
\\	倒産	
\\	倒闭,破产。	とうさん		【名・自动词・サ变/三类】
\\	投資	
\\	投资。	とうし		【名・自动词・サ变/三类】
\\	統治	
\\	统治。	とうじ		【名・他动词・サ变/三类】
\\	統制	
\\	统一,统管,统制。将分散的事物集中起来统一控制管理。(ばらばらになっているものをひとつにまとめて治めること。) 
\\	统一管理。凭借国家权力来对言论、经济活动等加以限制。(国家権力で言論・経済活動などに制限を加えること。) 
\\	协调一致。有意识地统一协调身心动作。(心身の動きを意図的にひとつ纏め上げること。)	とうせい		【名・他动词・サ变/三类】
\\	当選	
\\	当选,中选。(選挙で選びだされること。) 
\\	中彩,中签。(くじに当たること。)	とうせん		【自动词・サ变/三类】
\\	逃走	
\\	逃走,逃跑。	とうそう		【自动词・サ变/三类】
\\	統率	
\\	统率。(多くの人をまとめて率いること。統御。)	とうそつ		【名・他动词・サ变/三类】
\\	到達	
\\	到达,达到。(ある状態・目的に行きつくこと。)	とうたつ		【自动词・サ变/三类】
\\	到底	
\\	无论如何也,怎么也。	とうてい		【副词】
\\	投入	
\\	投入,扔进去。	とうにゅう		【名词】
\\	当人	
\\	本人,当事人。	とうにん		【名词】
\\	逃亡	
\\	逃跑匿身。 
\\	日本实行律令制时,指班田制下的公民擅自离开本籍移住他乡,当时属犯罪行为,若逃亡者始终未被逮捕,数年后则由官府没收其口分田。	とうぼう		【名・自动词・サ变/三类】
\\	冬眠	
\\	冬眠。	とうみん		【自动词・サ变/三类】
\\	登録	
\\	登记,注册。	とうろく		【名・他动词・サ变/三类】
\\	討論	
\\	讨论。(意見を述べ合い議論すること。ディスカッション。)	とうろん		【名・自他・サ变/三类】
\\	遠ざかる	
\\	远离,离远,离去。 
\\	疏远。远离。	とおざかる		【自动词・五段/一类】
\\	遠回り	
\\	绕远,绕道。(遠い方の道をゆくこと。また、回り遠いこと。)	とおまわり		【名・自动词・サ变/三类】
\\	通りかかる	
\\	恰巧路过。(ちょうど、そこを通る。)	とおりかかる		【自动词・五段/一类】
\\	兎角	
\\	(多用假名书写)种种;这个那个;动不动;动辄;不知不觉之间 
\\	兎角;左右 
\\	とかく		【自动词・サ变/三类】
\\	咎める	
\\	责难,非难,责备不好的行为或罪行。 
\\	盘问。	とがめる		【他动词・一段/二类】
\\	時折	
\\	有时,偶尔。	ときおり		【副词】
\\	跡切れる	
\\	间断;中断;断绝 
\\	跡切れる;途切れる 
\\	とぎれる		【自动词・一段/二类】
\\	説く	
\\	宣传,提倡;劝说,说服。 
\\	说明。	とく		【他动词・五段/一类】
\\	特技	
\\	拿手的技术。	とくぎ		【名词】
\\	特産	
\\	特产。	とくさん		【名词】
\\	特集	
\\	专刊,专集。特辑,特别节目。杂志、报纸、广播、电视等为特定主题而编辑的一期、一栏或一个节目。 
\\	同:特輯	とくしゅう		【名・他动词・サ变/三类】
\\	得点	
\\	得分,得分数,在考试、体育比赛等中得分或所得的分数。	とくてん		【名・自动词・サ变/三类】
\\	特派	
\\	特别派遣。	とくは		【名・他动词・サ变/三类】
\\	特有	
\\	特有,特别具备的。	とくゆう		【名・形容动词/ナ形容词】
\\	研ぐ	
\\	擦亮。磨光。 
\\	〔用磨石〕磨快。 
\\	淘。在水中搓洗。 
\\	同:磨ぐ	とぐ		【他动词・五段/一类】
\\	刺	
\\	刺。植物体表面针状的硬突起物,多为枝的变形物,具有叶、茎、托叶的性质。(植物の体表にある針状の硬い突起物。多くは枝が変形したものであるが、葉・茎・托葉の性質を持つ。刺毛。) 
\\	刺。动物的消化器官及身体表面的尖状附属突起物,有的是由毛蜕变成的(如豪猪等),有的是由鳞变形而成的(如刺鲀),有的是表皮形成的(如海胆等)。(動物の消化器や体表にある先の鋭くとがった付属突起物。毛が変質したもの(ヤマアラシなど)、うろこが変形したもの(ハリセンボン)、表皮から作られたもの(ウニ)などがある。) 
\\	刺儿。竹子、木头、鱼的小骨等尖细物。(竹・木・魚の小骨などのとがった細片。) 
\\	伤人心的言辞或作法。(人の心を突き刺すような言葉や仕打ち。)	とげ		【名词】
\\	遂げる	
\\	完成(任务),达到(愿望);最终形成(某种结果)。	とげる		【他动词・一段/二类】
\\	床	
\\	地板。 
\\	演唱”净琉璃”所设的高台。	とこ		【名词】
\\	年頃	
\\	成年的男女,成人的年龄。 
\\	适龄,特指该结婚的年龄。 
\\	大约的年龄,大约岁数。	としごろ		【名词】
\\	年寄り	
\\	年寄りの冷や水。/老人不量力。	としより		【惯用句】
\\	戸締り	
\\	锁门,紧锁门户。	とじまり		【名・自动词・サ变/三类】
\\	途上	
\\	道上,路上,中途。 
\\	事情正在进行中。	とじょう		【名词】
\\	綴じる	
\\	订上。 
\\	缝在一起。	とじる		【他动词・一段/二类】
\\	途絶える	
\\	断绝,杜绝,中断。	とだえる		【自动词・一段/二类】
\\	特許	
\\	特别许可,专利,特许,专利权。	とっきょ		【名・他动词・サ变/三类】
\\	疾っくに	
\\	老早;很早以前。(ずっと以前に。とうに。)	とっくに		【副词】
\\	特権	
\\	特权。	とっけん		【名词】
\\	咄嗟	
\\	瞬间,立刻,刹那间,猛然。	とっさ		【名词】
\\	取っ手	
\\	把儿,把手,拉手,装在家具、机械类等上的柄。	とって		【名词】
\\	突破	
\\	突破,冲破。	とっぱ		【他动词・サ变/三类】
\\	突如	
\\	突如其来地,冷不防,突然。	とつじょ		【副词】
\\	整える	
\\	弄齐。整理。整顿。 
\\	备齐。备至。准备。 
\\	谈妥。办妥。使达成。 
\\	同:調える。	ととのえる		【他动词・一段/二类】
\\	届け	
\\	配送,送达。 
\\	申请,登记。 
\\	报告书,通知,告示。	とどけ		【名词】
\\	滞る	
\\	堵塞;积压。 
\\	拖延,迟延;耽搁,迟误。 
\\	拖欠。	とどこおる		【自动词・五段/一类】
\\	止まる	
\\	停止,停顿,停下。(もとと同じ所にあって動かない。) 
\\	停留,逗留。(滞在する。宿泊する。) 
\\	停止,止住。(終止する。) 
\\	限于,止于。(その範囲内に限られる。) 
\\	同:留まる	とどまる		【自动词・五段/一类】
\\	留める	
\\	固定;钉住;别;扣。(その位置から動いたり離れたりしないように、固定する。) 
\\	留下;扣留。(引きとめる。また、その場にとどめおく。) 
\\	留心,注目;记,留在(心上);介意。(心をよせる。留意する。目や耳などのはたらきを、ある点に集中させる。) 
\\	限于。(限る。) 
\\	同:止める、停める	とどめる		【他动词・一段/二类】
\\	止める	
\\	停下,止。使正在动的事物停止。(動いている物をとめる。抑止する。) 
\\	止住,阻止住。先停下,使先保持某状态。(とめておく。そのままにしておく。) 
\\	同:留める	とどめる		【他动词・五段/一类】
\\	唱える	
\\	念;诵。 
\\	声明,提出。提倡,倡导,主张。	となえる		【他动词・一段/二类】
\\	兎に角	
\\	(现多用假名)无论如何;总之;姑且。(いろいろ内部に秘められた事情は有るだろうが、忌憚の無い所は次の通りである、ということを表わす。)	とにかく		【副词】
\\	殿様	
\\	(对贵族、主君的敬称)老爷,大人;(有钱而不谙世故的)老爷。	とのさま		【名词】
\\	幕	
\\	幕,帐幕,帷幕,帷幔。(ものの隔てや目隠しとして張りめぐらしたり垂らしたりする、横に長く縫い合わせた布。) 
\\	〈劇〉幕。(芝居等で演技の一段落。通常、幕が上がってから降りるまで。) 
\\	场面,时候。(場面。場合。)	とばり		【名词】
\\	帳	
\\	帐本。(帳面。)	とばり		【接头/接尾】
\\	扉	
\\	门;门扇。(開き戸の戸。) 
\\	扉页;(雑志正文前的)第一頁(印有題目,前言等)。(書籍の見返しの次にあって、書名・著者名などを記したページ。また、雑誌の本文の前の第1ページ。タイトル‐ページ。)	とびら		【名词】
\\	徒歩	
\\	走,徒步,步行。	とほ		【自动词・サ变/三类】
\\	乏しい	
\\	缺乏,不足;贫穷,贫乏。	とぼしい		【形容词/イ形容词】
\\	富	
\\	资源,富源。 
\\	财富,财产,资产,钱财。	とみ		【名词】
\\	富む	
\\	富裕。 
\\	丰富。	とむ		【自动词・五段/一类】
\\	兎も角	
\\	(现多用假名)姑且不论;暂且不谈;无论如何;不管怎样	ともかく		【副词】
\\	共稼ぎ	
\\	【自动・三类】 
\\	夫妇都工作,双职工。(夫婦がどちらも勤めに出て生計を立てること。共働き。)	ともかせぎ		【名词】
\\	灯	
\\	灯。(火をつけて闇を照らすのに用いる物。)	ともしび		【名词】
\\	伴う	
\\	带,伴随。 
\\	随着,跟着,伴随。 
\\	相符,相称。 
\\	带,带领。	ともなう		【自他・五段/一类】
\\	共働き	
\\	(夫妇)共同赚钱。	ともばたらき		【名词】
\\	取り扱い	
\\	对待,待遇,处理,办理。	とりあつかい		【名词】
\\	鳥居	
\\	(神社入口处)牌坊。日本神社入口处所建的大门。用以表示神域。	とりい		【名词】
\\	取り替え	
\\	调换,更换。(とりかえること。交換。)	とりかえ		【名词】
\\	取り組む	
\\	着手,对付,解决,应对。 
\\	较量,与……为对手,同……比赛。 
\\	全力处理,埋头苦干,下大气力去做。 
\\	(相扑)扭住(对方)。	とりくむ		【自动词・五段/一类】
\\	取り締まり	
\\	管理(者),监管,监督,控制,取缔。 
\\	<写作「取締」>「取締役」的略语。	とりしまり		【名词】
\\	取り締まる	
\\	管束,约束,管理。 
\\	管制;取缔。	とりしまる		【他动词・五段/一类】
\\	取り調べる	
\\	调查,审讯。	とりしらべる		【他动词・一段/二类】
\\	取り立てる	
\\	索取,催缴,催收;征收。 
\\	提拔,提升。 
\\	偏袒,偏爱。 
\\	(特别)提出,提及。	とりたてる		【他动词・一段/二类】
\\	取り次ぐ	
\\	传达,转达,通报。 
\\	代办;经销,代销,经售;代购。	とりつぐ		【他动词・五段/一类】
\\	取り除く	
\\	除掉,去掉;拆掉,拆除(取り壊し);解除,消除。	とりのぞく		【他动词・五段/一类】
\\	取り引き	
\\	做买卖,做交易。 
\\	交易,贸易。	とりひき		【名・自他・サ变/三类】
\\	取り巻く	
\\	包围,围绕,围住。 
\\	捧场,逢迎,拍马屁,阿谀奉承。	とりまく		【他动词・五段/一类】
\\	取り混ぜる	
\\	搀混,搀和,搀在一起,混在一块儿。	とりまぜる		【他动词・一段/二类】
\\	取り戻す	
\\	收回。 
\\	恢复。	とりもどす		【他动词・五段/一类】
\\	取り寄せる	
\\	拉到手边。 
\\	索取,要来。 
\\	让寄来。	とりよせる		【他动词・一段/二类】
\\	取り分	
\\	应得的份额。(自分が取るべき分。分け前。取り前。)	とりわけ		【名词】
\\	副	
\\	副本,抄件,副件。〔控え。〕 
\\	副。附属物。附加物。〔そえ。〕 
\\	一般には俗字の“付”も併用される。副;副本;副件(同ひかえ;うつし)	とりわけ		【接头/接尾】
\\	蕩ける	
\\	溶化;融解。(溶けてやわらかになる、金属が熱せられて液状になる。) 
\\	心荡神驰;神魂飘荡。(心を奪われて、うっとりとした気持ちになる。心のしまりがなくなる。)	とろける		【自动词・一段/二类】
\\	胴	
\\	躯干,身腰。(体の中心をなす部分、特に腹部のあたり。) 
\\	腹部,腰部,中间部分。〔物の中腹あたり。〕 
\\	共鸣箱,弦乐器或打击乐器的共鸣部分。(太鼓·三味線などで、音が反響するように、中空にした部分。) 
\\	胸铠,护胸。一击。(剣道の防具。剣道で、決まり手の一。) 
\\	船腹。(和船の腹部。)	どう		【名词】
\\	同	
\\	同。相等。相同。(同じであること。) 
\\	一同。一起干。(共にすること。) 
\\	同一。该。(前に挙げたものを再び繰り返す場合に用いる語。) 
\\	伙伴。(仲間になること。)	どう		【名词】
\\	同意	
\\	同义。(同じ意味。) 
\\	意见相同。(同じ意見。) 
\\	同意,赞成。(賛成すること。)	どうい		【自动词・サ变/三类】
\\	動員	
\\	【他サ】 
\\	动员;调动;发动。(通常何らかの目的の為に物資・人員を集中することをいう。)	どういん		【名词】
\\	同感	
\\	【自动·三类】 
\\	同感。同意。赞同。同一见解。与他人感觉相同、意见一致。(同じように感ずること。同じ考え。)	どうかん		【名词】
\\	動機	
\\	动机。引起行动的直接原因或目的。	どうき		【名词】
\\	同級	
\\	同等级;等级相同;同班;同年级。(同じ等級(学級)。)	どうきゅう		【名词】
\\	同居	
\\	同居。 
\\	同住,住在一起。 
\\	同时存在。 
\\	同居,多指夫妻或者未婚男女同住,相似的情况也会使用“同住”一词。	どうきょ		【自动词・サ变/三类】
\\	動向	
\\	动向。	どうこう		【名词】
\\	同士	
\\	同伴,伙伴,同好,志趣相同者。(互いにある共通の関係にある人。同じ志の人。)	どうし		【名词】
\\	同志	
\\	同志。志同道合者。(同じ志の人。)	どうし		【名词】
\\	如何しても	
\\	如何。(どんなに。) 
\\	为什么。(どうして…か。) 
\\	如何。(「ーあらむ」の形で危惧もしくは非難の意を表す。) 
\\	怎么样。(どれほど…ことか。) 
\\	怎么样。(「どうですか」と呼びかけすすめる語。)	どうしても		【副词】
\\	同情	
\\	汉语中的“同情”也有赞成别人的行为的意思。 
\\	中国語の“同情”には他人の行動に賛成・共感する意もある。	どうじょう		【自动词・サ变/三类】
\\	道場	
\\	<日本地名>道场。 
\\	练功场,练武场。 
\\	道场,修行的地方。	どうじょう		【名词】
\\	何卒	
\\	请。(相手に対して強く願い望む気持ちを表す。どうか。なんとか。「どうぞ」の改まった言い方。) 
\\	设法,想办法。(どうにかして。なんとかして。)	どうぞ		【副词】
\\	同調	
\\	同一步调,赞同。 
\\	调音,调准音调。	どうちょう		【名・自他・サ变/三类】
\\	動的	
\\	具有动感的,生动的。	どうてき		【形容动词/ナ形容词】
\\	同等	
\\	同等的等级,程度。 
\\	同等的资格,能力。	どうとう		【名・形容动词/ナ形容词】
\\	堂々	
\\	坦荡。 
\\	威严庄重。	どうどう		【形容动词/ナ形容词】
\\	導入	
\\	导入,引进,引入,输入。 
\\	引用。	どうにゅう		【名・他动词・サ变/三类】
\\	同封	
\\	附在信内,和信一起。	どうふう		【名・他动词・サ变/三类】
\\	同盟	
\\	同盟,联盟,结盟。	どうめい		【名・自动词・サ变/三类】
\\	動揺	
\\	动摇,摇摆,来回晃动。〔揺れ動くこと。〕 
\\	动摇,不安。(心や気持ちがゆれ動くこと。平静を失うこと。)	どうよう		【自动词・サ变/三类】
\\	動力	
\\	动力,原动力。	どうりょく		【名词】
\\	読者	
\\	读者。	どくしゃ		【名词】
\\	独自	
\\	独自,个人。	どくじ		【形容动词/ナ形容词】
\\	独占	
\\	独占。(ひとりじめにすること。) 
\\	垄断,专营。(市場における売手または買手がただ一人の状態。)	どくせん		【名・他动词・サ变/三类】
\\	独創	
\\	【他サ】 
\\	独创。(他のまねでなく、独自の考えで物事を作り出すこと。)	どくそう		【名词】
\\	何処	
\\	何处;哪里。(どこ。)	どこ		【代词】
\\	何処か	
\\	(不肯定的)某处;什么地方 
\\	〔ある場所〕哪里,什么地方;有的地方,某点. 
\\	〔どことなく〕总觉得,好象.	どこか		【接续词】
\\	土産	
\\	特产,土产;(赠给人的)礼品,礼物。	どさん		【名词】
\\	土台	
\\	木基础梁,日本式木结构建筑中,为固定柱子而横置于基础上的木材或钢架。 
\\	台基,基础,建筑物最下面的部分。 
\\	(事物的)基础。	どだい		【名词】
\\	何方	
\\	哪边,哪面,哪儿。(不明または不特定の方向・場所をさす。) 
\\	哪个。(特に二つのものの中から何か一つを選ぶとき,限定しないままそのうちの一つを取り立てて指す。)	どちら		【代词】
\\	土手	
\\	堤,堤坝;河堤。	どて		【名词】
\\	何の	
\\	表示回避、搪塞的语气词。(軽く否定したり受け流したりするときに用いる語。)	どの		【感叹词】
\\	土俵	
\\	相扑场地(台)。 
\\	土袋子。	どひょう		【名词】
\\	土木	
\\	土木(工程)
\\	。(土木工学、また、土木工事の略。)	どぼく		【名词】
\\	何々	
\\	哪些,什么。	どれどれ		【代词】
\\	度忘れ	
\\	一时想不起来、突然忘记。(ふと忘れてとっさに思い出せないこと。)	どわすれ		【名・他动词・サ变/三类】
\\	鈍感	
\\	钝感,感觉迟钝。	どんかん		【名・形容动词/ナ形容词】
\\	内閣	
\\	内阁,政府。	ないかく		【名词】
\\	乃至	
\\	到,至。(…より…まで。) 
\\	或,或者。(あるいは。)	ないし		【接续词】
\\	内臓	
\\	〈解〉内脏。(動物の胸腔や腹腔にある器官の総称。消化呼吸系・泌尿生殖系・内分泌系の器官をいう。普通は腹腔内にある胃・腸・肝・腎・膵などをいう。漢方では五臓六腑という。)	ないぞう		【名词】
\\	内部	
\\	内部;里面。 
\\	内幕,内情。	ないぶ		【名词】
\\	内乱	
\\	内乱,叛乱。	ないらん		【名词】
\\	内陸	
\\	内陆,内地。	ないりく		【名词】
\\	苗	
\\	【参考】 
\\	中文中的“苗”除了指作物的苗之外,还有“蒜苗”这样的苗,“鱼苗”、“猪苗”这样刚生来的动物,“火苗”这样形状类似苗的东西,后继者、子孙等的意思。 
\\	中国語の“苗”は,作物の苗のほか,
\\	蒜苗”(にんにくの芽)のように芽 
\\	鱼苗”“猪苗”のように生まれたばかりの動物 
\\	火苗”(ほのお)のように形が苗に似ているもの 
\\	後継者・子孫などの意味がある。	なえ		【名词】
\\	尚	
\\	还,仍然,依然,犹,尚。(あいかわらず。) 
\\	更;还;再。(さらに。) 
\\	尚未。(まだ。) 
\\	犹,犹如。(あたかも。)	なお		【副词】
\\	尚更	
\\	更,更加,越发。(ある条件が加わることで,物事の程度が以前よりいっそう進むさま。)	なおさら		【副词】
\\	中程	
\\	时间,期间的中间,中途。 
\\	场所的正中间。 
\\	中间,中等程度。	なかほど		【名词】
\\	流し	
\\	流,冲,流放。 
\\	洗碗池,水池子。 
\\	澡塘内冲洗身体处,搓背。 
\\	串街(卖唱、揽客)。 
\\	(串街揽客的)出租汽车。	ながし		【名词】
\\	長々	
\\	非常长。长时间。长久。冗长。(時間の非常に長いさま。物が長く伸びているさま。) 
\\	同:長長	ながなが		【副词】
\\	殴る	
\\	狠狠地揍。 
\\	打倒。 
\\	殴打;打,揍。	なぐる		【他动词・五段/一类】
\\	嘆く	
\\	慨叹,叹惋。 
\\	叹息,叹气。悲叹,哀叹。	なげく		【自他・五段/一类】
\\	投げ出す	
\\	抛出,扔下;甩。 
\\	抛弃,放弃,丢弃。 
\\	豁出,拿出。	なげだす		【他动词・五段/一类】
\\	和やか	
\\	平静,安详;和谐,和睦。 
\\	温和,和蔼。	なごやか		【形容动词/ナ形容词】
\\	名残	
\\	惜别,依恋。 
\\	余音,余韵。	なごり		【名词】
\\	情け	
\\	仁慈,慈悲,同情。 
\\	爱情,恋情。 
\\	情趣,雅兴,风趣。	なさけ		【名词】
\\	情け深い	
\\	富于同情心的,仁慈的,善良的,有慈悲心的。	なさけぶかい		【形容词/イ形容词】
\\	為さる	
\\	なす”、“する”的敬语)为,做。(「する」「なす」の尊敬語。) 
\\	接在动词连用形后表示尊敬之意。(動詞の連用形に付いて、尊敬の意を表す。)	なさる		【他动词・五段/一类】
\\	詰る	
\\	堆满,挤满;(沟通)无法进行;变短。	なじる		【他动词・五段/一类】
\\	何故	
\\	何故;为何。(どういうわけで。)	なぜ		【副词】
\\	何故なら	
\\	因为;原因是(同なぜかというと)	なぜなら		【接续词】
\\	名高い	
\\	有名,著名,出名;成名的,扬名的;广为人知的。	なだかい		【形容词/イ形容词】
\\	雪崩	
\\	雪崩,崩落的雪。(山の斜面に積もった大量の雪が、急激にくずれ落ちる現象。)	なだれ		【名词】
\\	懐く	
\\	「だく」的文语说法,抱。(「だく」の文語的な言い方。) 
\\	怀有,感到。(ある考え、気持ちを心の中に持つ。) 
\\	同:抱く	なつく		【他动词・五段/一类】
\\	名付ける	
\\	命名,起名,叫作。	なづける		【他动词・一段/二类】
\\	何気ない	
\\	假装没事,坦然自若。 
\\	无意,无心。 
\\	无意。没有任何想法,没有特别深的意图。 
\\	没事儿,没关系。并未特别介意,若无其事。	なにげない		【形容词/イ形容词】
\\	何しろ	
\\	无论怎样,不管怎样,反正,总之;毕竟,到底。(とにかく。つまりは。) 
\\	因为,由于。(なにしろ…なので。なににせよ。)	なにしろ		【副词】
\\	何も	
\\	什么也……,全都。(なにもかも。それを含めてみんな。) 
\\	并(不),什么(也不)。(ぜんぜん。少しも。全く。一つも。) 
\\	并(不)。何必,(不)必……,(不)特别……((打ち消しを伴って)特別に。わざわざ。)	なにも		【接续词】
\\	何より	
\\	比什么(都好),再好不过,最好;首先。(他のどんな物事にもましてよいこと。最上であること。最もよいこと。) 
\\	最,胜过其他任何东西。(他の何にもまさって。この上なく。最も。「なによりも」の形でも用いる。)	なにより		【接续词】
\\	七日	
\\	七天,七日。(日の数の七つ。7日間。1週間。) 
\\	七号,每月第七天。特别还指正月初七,七夕节的七月七日等。(月の第7の日。特に、正月7日、七夕の7月7日などをいう。) 
\\	孩子生下来第七天。(人が生まれて七日目の日。お七夜。) 
\\	头七,人死后七日做的佛教法事。(人の死後、七日ごとに営む法事。また、その日。特に、最初の「初七日」と最後の「七七日」。) 
\\	同:なのか 
\\	七天,七日。 
\\	七号,每月第七天。特别还指正月初七,七夕节的七月七日等。 
\\	孩子生下来第七天。 
\\	头七,人死后七日做的佛教法事。 同:なのか	なぬか		【名词】
\\	名札	
\\	姓名牌,名牌,名签。	なふだ		【名词】
\\	生臭い	
\\	腥,膻,腥臊。 
\\	血腥。 
\\	不守清规,带俗气。 
\\	荤,被欲望等所左右。	なまぐさい		【形容词/イ形容词】
\\	生温い	
\\	微温,有点热气。 
\\	不够严格,马马虎虎,不够彻底。	なまぬるい		【形容词/イ形容词】
\\	生身	
\\	真人,肉身,肉体,活人。	なまみ		【名词】
\\	鉛	
\\	铅。	なまり		【名词】
\\	鈍る	
\\	变弱;动摇。 
\\	变钝,不快。 
\\	变迟钝。	なまる		【自动词・五段/一类】
\\	並み	
\\	并列,排列(並んでいること) 
\\	普通,一般,平常(普通であること)	なみ		【名词】
\\	滑らか	
\\	光滑,滑溜,平滑,滑润;(说话)流利,流畅。(文章等)通畅,顺溜。	なめらか		【形容动词/ナ形容词】
\\	嘗める	
\\	含,舔。(舌の先でなでる。ねぶる。) 
\\	尝味道。(味わう。翫味する。) 
\\	尝受,经历。((つらい事や苦しい事を)十分に経験する。) 
\\	轻视,小看。((「なめ(無礼)」の動詞化という)相手、または事を、頭から馬鹿にしてかかる。みくびる。) 
\\	烧光,吞没。((比喩的に炎の舌先がふれて)焼き尽くす。) 
\\	同:舐める	なめる		【他动词・一段/二类】
\\	悩ましい	
\\	难过的,难受的,痛苦的,苦恼的。 
\\	恼人的,令人神魂颠倒的,内心为色情所动的。	なやましい		【形容词/イ形容词】
\\	悩ます	
\\	使烦恼,困扰。	なやます		【他动词・五段/一类】
\\	悩み	
\\	烦恼,苦恼,痛苦。 
\\	病,病患。	なやみ		【名词】
\\	平均	
\\	平均。 
\\	平均值。 
\\	平衡,均衡。	ならし		【名・自他・サ变/三类】
\\	慣らす	
\\	使习惯,使惯于。(なれるようにする。なれ親しませる。なれさせる。) 
\\	驯服,养熟。((動物を)手なずける。なつける。飼いならす。) 
\\	练习,学习。(習う。練習する。)	ならす		【他动词・五段/一类】
\\	馴らす	
\\	驯养,调教。	ならす		【他动词・五段/一类】
\\	並びに	
\\	和,及,以及。连接两个事物,表示其处于并列的关系。(二つの事柄を結び付けて、並列の関係にあることを表す。および。また。)	ならびに		【接续词】
\\	成り立つ	
\\	成立,谈妥。 
\\	构成,组成,形成。 
\\	划得来。 
\\	能维持;站得住脚。	なりたつ		【自动词・五段/一类】
\\	成る丈	
\\	尽量;尽可能	なるたけ		【副词】
\\	慣れ	
\\	司空见惯。 
\\	习惯,熟习。	なれ		【名词】
\\	馴れ馴れしい	
\\	狎昵,熟不拘礼。过分亲昵,熟头熟脑,嬉皮笑脸。(あまりに遠慮がなさすぎる。)	なれなれしい		【形容词/イ形容词】
\\	難	
\\	难,困难;灾难,苦难;问难,责难。	なん		【名词】
\\	南	
\\	南方;南风。	なん		【名词】
\\	何だか	
\\	是什么。(なにやら。) 
\\	总觉得,总有点,不由得。(理由がなにかはわからないが。なぜか。何となく。なにやら。 
\\	なんだか		【副词】
\\	何て	
\\	多么,何等。(何とまあ。) 
\\	什么样的,什么的。なんという的省略说法。(「なんという」のごくくだけた言い方。)	なんて		【接续词】
\\	何と	
\\	(用于请求同意、试探反应等场合)那么,那。(相手に呼びかけ同意を求めたり、反応を探ったりするときに用いる語。どうだ。)	なんと		【感叹词】
\\	何となく	
\\	(不知为什么)总觉得,不由得,无意中,无意中,不为何。(はっきりした理由や目的もなく。わかもなく。)	なんとなく		【副词】
\\	何とも	
\\	真的,实在。 
\\	无关紧要『成』,没什么;怎么也,什么也。	なんとも		【副词】
\\	荷	
\\	东西,货物。运输的物品或货物,亦指行李。 
\\	负担,责任,累赘。成为责任、负担的事项。(責任・負担となる事柄。) 
\\	《相关惯用语》	に		【名词】
\\	似通う	
\\	相似,类似。	にかよう		【自动词・五段/一类】
\\	賑わう	
\\	热闹,拥挤,繁华。人来的很多而热闹。(人が大勢出てにぎやかになる。) 
\\	繁荣,兴旺,兴隆,兴盛,繁盛。(富み栄える。繁盛する。) 
\\	丰盛。变丰富。(豊かになる。)	にぎわう		【自动词・五段/一类】
\\	悪い	
\\	坏,不好;恶性;恶劣;有害,不利;不对,错误;不吉利,不吉祥;不佳,不舒畅;不适合,不方便;坏,腐败;不好意思。对不住。用于口语;劣质的。	にくい		【形容词/イ形容词】
\\	憎しみ	
\\	憎,恨。憎恶,憎恨。厌恶的心情。	にくしみ		【名词】
\\	肉親	
\\	骨肉亲,亲人。指血缘关系很近的人。	にくしん		【名词】
\\	肉体	
\\	肉体,血肉的身体。	にくたい		【名词】
\\	逃げ出す	
\\	逃出,逃走,逃掉,溜掉。 
\\	开始逃跑。	にげだす		【自动词・五段/一类】
\\	西日	
\\	西照的阳光,午后的阳光。 
\\	夕阳。	にしび		【名词】
\\	滲む	
\\	渗,沁,洇,润。 
\\	渐渐地渗出;浸出,流出。 
\\	反映出。 
\\	渗入,深深地刻在。	にじむ		【自动词・五段/一类】
\\	日夜	
\\	日夜,昼夜。(よるひる。) 
\\	经常不断地,总是。(いつも。)	にちや		【名词・副词】
\\	日当	
\\	日本地名。(日本の地名。)	にっとう		【名词】
\\	荷造り	
\\	捆行李,包装,打包。	にづくり		【名・自他・サ变/三类】
\\	担う	
\\	肩负,担负,承担。 
\\	担,挑。	になう		【他动词・五段/一类】
\\	二人	
\\	二人,两个人,一对;二人,两个人,一对。	ににん		【名词】
\\	入手	
\\	【他动·三类】 
\\	得到,取得,到手(手に入れること。自分のものとすること)。	にゅうしゅ		【名词】
\\	入賞	
\\	获奖,得奖,受奖。(競技会・展覧会などで,賞をもらうことのできる順位にはいること。)	にゅうしょう		【自动词・サ变/三类】
\\	入浴	
\\	洗澡,入浴,沐浴。(風呂に入ること。入湯。)	にゅうよく		【自他・サ变/三类】
\\	尿	
\\	「しと」的音变。小便的幼儿语。小便。(「しと」の音変化。小便をいう幼児語。おしっこ。)	にょう		【名词】
\\	俄か	
\\	突然,忽然,骤然,遽然。 
\\	马上,立刻。 
\\	临时,暂时,一时,暂短。 
\\	即兴滑稽小剧。	にわか		【形容动词/ナ形容词】
\\	認識	
\\	哲学上指对客观世界从感觉、知觉乃至记忆、思考的意识活动和作用。 
\\	认知,识别,了解,理解。	にんしき		【名・他动词・サ变/三类】
\\	妊娠	
\\	妊娠,怀孕,怀胎。	にんしん		【自动词・サ变/三类】
\\	人情	
\\	爱情。 
\\	人情。	にんじょう		【名词】
\\	任務	
\\	任务,职责。	にんむ		【名词】
\\	任命	
\\	任命,命其担任某官职或职务。	にんめい		【名・他动词・サ变/三类】
\\	抜かす	
\\	遗漏;漏掉;跳过。	ぬかす		【他动词・五段/一类】
\\	抜け出す	
\\	溜;摆脱,脱出。 
\\	开始脱落。	ぬけだす		【自动词・五段/一类】
\\	盗み	
\\	偷盗,盗窃。(盗むこと。)	ぬすみ		【名词】
\\	沼	
\\	池塘,沼泽,池沼。	ぬま		【名词】
\\	音	
\\	音に聞く/ 传闻;闻名。(うわさ、評判。)	ね		【惯用句】
\\	値打ち	
\\	声价,品格。 
\\	价值。 
\\	价格,价钱。 
\\	估价,评价,定价。	ねうち		【名词】
\\	寝かせる	
\\	使入睡。(眠るようにする。) 
\\	放平。(横さまに倒す。横にする。) 
\\	存放,存着。(品物や金銭などを活用せずに手元にとどめておく。) 
\\	使发酵。(発酵·熟成させるために、一定の温度でしばらくそのままにしておく。)	ねかせる		【他动词・一段/二类】
\\	捻子	
\\	螺丝;螺钉 
\\	捻子;螺子;螺旋 
\\	ねじ		【名词】
\\	ねじ回し	
\\	螺丝起子,螺丝刀。	ねじまわし		【名词】
\\	鼠	
\\	鼠,老鼠,耗子。(人家の付近などにすむ、敏捷な小動物。)	ねず		【名词】
\\	妬む	
\\	嫉妒。(他人の幸運や長所を見て自分にはそれが望み得ないことを不満に思い、相手に悪い事が起こればいいと思う。) 
\\	同:嫉む	ねたむ		【他动词・五段/一类】
\\	強請る	
\\	央求。强求。(たのむ。)	ねだる		【他动词・五段/一类】
\\	熱湯	
\\	热水,开水。煮得滚开的热水。(煮えたっている熱い湯。煮え湯。)	ねっとう		【名词】
\\	熱意	
\\	热忱,热情。	ねつい		【名词】
\\	熱量	
\\	热量。	ねつりょう		【名词】
\\	粘り	
\\	坚韧顽强,韧性,韧劲。 
\\	粘,粘性,粘度。	ねばり		【名词】
\\	粘る	
\\	发粘,粘糊;柔软易延伸,易附着于物上的状态。 
\\	坚持,有耐性,顽强到底。 
\\	同:がんばる	ねばる		【自动词・五段/一类】
\\	値引き	
\\	降〔减〕价。(値段を普通より安くして売ること。)	ねびき		【名・他动词・サ变/三类】
\\	根回し	
\\	为移栽或使果树高产修根,整根。 
\\	事前疏通,为使交涉或会议等顺利进行,事先采取措施,进行疏通。	ねまわし		【名・他动词・サ变/三类】
\\	眠たい	
\\	困,困倦,昏昏欲睡『成』。(一刻も早く、寝たい状態だ。)	ねむたい		【形容动词/ナ形容词】
\\	練る	
\\	炼。 
\\	推敲。 
\\	熬制。 
\\	锻炼。 
\\	掺和。 
\\	游行。 
\\	锻造。	ねる		【自他・五段/一类】
\\	念	
\\	思念,念头。 
\\	注意。 
\\	(佛教上指)极短时间。	ねん		【名词】
\\	年鑑	
\\	年鉴。	ねんかん		【名词】
\\	年号	
\\	年号。	ねんごう		【名词】
\\	燃焼	
\\	燃烧。(燃えること。)	ねんしょう		【自动词・サ变/三类】
\\	年長	
\\	年长,年纪大(的人)。	ねんちょう		【名・形容动词/ナ形容词】
\\	燃料	
\\	燃料。(燃焼させて熱源とする材料。)	ねんりょう		【名词】
\\	年輪	
\\	年轮。 
\\	技艺经验,历史的年轮。	ねんりん		【名词】
\\	脳	
\\	脑。(あたま。) 
\\	智力,脑力。(頭脳のはたらき。)	のう		【名词】
\\	農耕	
\\	耕作,种田。(田畑を耕すこと。)	のうこう		【名词】
\\	農場	
\\	【日本地名】农场。	のうじょう		【名词】
\\	農地	
\\	农业用土地,耕地。	のうち		【名词】
\\	納入	
\\	缴纳,交纳。(品物や金を納めること。)	のうにゅう		【名・他动词・サ变/三类】
\\	逃す	
\\	逸失,放过,错过。	のがす		【他动词・五段/一类】
\\	逃れる	
\\	逃跑,逃出,逃遁,逃脱。 
\\	逃避,规避,避免,摆脱,躲避。	のがれる		【自动词・一段/二类】
\\	軒並み	
\\	屋檐栉比,鳞次栉比。	のきなみ		【名词】
\\	鋸	
\\	锯。(木材などのほか、堅い素材をひき切る工具。)	のこぎり		【名词】
\\	望ましい	
\\	最好,最理想的。	のぞましい		【形容词/イ形容词】
\\	臨む	
\\	面临,面对,临。 
\\	临近,面临,遇到某种机会、场面。 
\\	亲临,莅临,出席;(隆重的集会等),也指参加。 
\\	对待,处理。	のぞむ		【自动词・五段/一类】
\\	乗っ取る	
\\	攻占,夺取,劫持。	のっとる		【他动词・五段/一类】
\\	長閑	
\\	晴朗,舒适。(天気がよくて穏やかなさま。) 
\\	悠闲,宁静。(静かでのんびりとして落ち着いているさま。)	のどか		【形容动词/ナ形容词】
\\	罵る	
\\	骂,丑诋。	ののしる		【他动词・五段/一类】
\\	延べ	
\\	展锻,延展,压延品。指压延,经压延的物品。 
\\	累计,合计,共计,总计。计算时同一事物即使被包含多次也只按一次计算,既不同事物的总和。 
\\	延,延期。	のべ		【名词】
\\	飲み込む	
\\	(囫囵)咽下,吞下。 
\\	理解,领会,熟悉。 
\\	咽,忍住,止住。使劲控制住话语或哈欠等。 
\\	吞没,容纳。把人或东西拉进漩涡或裂缝中,亦指体育场等同时接受更多的人。	のみこむ		【他动词・五段/一类】
\\	乗り換え	
\\	改乘,换乘。 
\\	为了在途中使用而事前准备好的交通工具。 
\\	交叉。生殖细胞减数分裂时,同源染色体的局部交换,出现遗传基因重组交换。亦见于体细胞分裂。 
\\	同:乗換え、乗り替え、乗替え、乗換	のりかえ		【名词】
\\	乗り込む	
\\	乘上,坐进。 
\\	进入,开进。 
\\	开进,到达,进入。	のりこむ		【自动词・五段/一类】
\\	刃	
\\	刀,剑,刃具。 
\\	刀刃,刀纹。	は		【名词】
\\	派	
\\	派。流派,派别。(流儀・宗旨などの分かれ。なかま。思想などのわかれ。)	は		【名词】
\\	把握	
\\	握在手中;掌握,充分理解。	はあく		【名・他动词・サ变/三类】
\\	肺	
\\	肺。 
\\	肺病,肺脏。	はい		【名词】
\\	廃棄	
\\	扔掉。 
\\	废弃,废除。	はいき		【名词】
\\	配給	
\\	配给,配售,定量供应。	はいきゅう		【名・他动词・サ变/三类】
\\	配偶者	
\\	配偶,夫或妻。	はいぐうしゃ		【名词】
\\	拝啓	
\\	(写在书信的开头)敬启。	はいけい		【名词】
\\	背景	
\\	背景;(舞台上的)布景;后盾;靠山	はいけい		【名词】
\\	背後	
\\	背后。〔物のうしろ。背中の方。後方。〕 
\\	背地,幕后,背后。〔裏側。物事の表面に現れていない陰の部分。〕	はいご		【名词】
\\	灰皿	
\\	烟灰碟,烟灰缸。	はいさら		【名词】
\\	廃止	
\\	废止,废除,作废。	はいし		【名・他动词・サ变/三类】
\\	拝借	
\\	借。	はいしゃく		【名・他动词・サ变/三类】
\\	排除	
\\	排除。	はいじょ		【名・他动词・サ变/三类】
\\	排水	
\\	排水。(地表や地中、あるいは施設内などから過剰または不用な水を排除すること。)	はいすい		【自动词・サ变/三类】
\\	敗戦	
\\	战败,战争、比赛等失败。	はいせん		【名・自动词・サ变/三类】
\\	配置	
\\	配置,安置,布置,部署。	はいち		【名词】
\\	配布	
\\	散发,广泛分发。	はいふ		【名・他动词・サ变/三类】
\\	配分	
\\	分配。(配り分けること。)	はいぶん		【名・他动词・サ变/三类】
\\	配慮	
\\	关怀,照顾,照料,关照。	はいりょ		【名・他动词・サ变/三类】
\\	配列	
\\	排列。	はいれつ		【名・他动词・サ变/三类】
\\	映える	
\\	照,映照。 
\\	好看,漂亮;显眼,夺目。 
\\	适合,协调。	はえる		【自动词・一段/二类】
\\	破壊	
\\	破坏。(打ち壊すこと。)	はかい		【名・自他・サ变/三类】
\\	墓地	
\\	墓地,坟地。	はかち		【名词】
\\	捗る	
\\	进展。	はかどる		【自动词・五段/一类】
\\	諮る	
\\	商量,磋商,协商,咨询。 
\\	同:図る,謀る	はかる		【他动词・五段/一类】
\\	図る	
\\	意料,料想,估计。 
\\	安排,考虑,照顾。 
\\	图谋,策划,谋求。	はかる		【他动词・五段/一类】
\\	剥がす	
\\	剥下,揭下。	はがす		【他动词・五段/一类】
\\	破棄	
\\	破坏。(破り捨てること。) 
\\	毁约。(取り決めなどを一方的に取り消すこと。) 
\\	撤销原判。(上訴裁判所が上訴に理由があるとして原判決を取り消すこと。)	はき		【名・他动词・サ变/三类】
\\	泊	
\\	宿,晚,夜。(宿をとること。宿泊の数をかぞえる語。)	はく		【接头/接尾】
\\	薄弱	
\\	软弱,孱弱『書』;薄弱,不坚定。 
\\	不足,不充分。	はくじゃく		【名词】
\\	白状	
\\	坦白,招认,供认,认罪。说出自己所犯的罪过或秘密。	はくじょう		【名・他动词・サ变/三类】
\\	剥ぐ	
\\	剥下,揭下。 
\\	扒下。 
\\	剥夺。	はぐ		【他动词・五段/一类】
\\	派遣	
\\	派遣,派出。	はけん		【名・他动词・サ变/三类】
\\	励ます	
\\	鼓励,激励,勉励;厉声,提高嗓门。	はげます		【他动词・五段/一类】
\\	励む	
\\	努力,刻苦,辛勤,奋勉,勤勉。	はげむ		【自动词・五段/一类】
\\	剥げる	
\\	剥落。 
\\	褪色。	はげる		【自动词・一段/二类】
\\	鋏	
\\	剪刀,剪子。〔二枚の刃ではさんで物を切る道具。裁ち鋏・花鋏など。〕 
\\	剪票钳。〔切符などに穴をあける道具。パンチ。〕 
\\	螯足,螃蟹夹子。〔(「螯」「鉗」と書く)カニ・サソリなどの脚の、可動指があって、餌(えさ)などをはさむことのできる部分。〕 
\\	剪子,在猜拳的时候。〔じゃんけんで、二本の指を伸ばした形。〕	はさみ		【名词】
\\	柱	
\\	〈建〉柱子,支柱。(荷重を支える材。) 
\\	顶梁柱,靠山。(たよりになる人。) 
\\	尊,位(仏像);具。(遺骨。)	はしら		【名词】
\\	橋渡し	
\\	桥梁,中人,搭桥,当介绍人。 
\\	架桥。	はしわたし		【名词】
\\	恥	
\\	耻;耻辱(不名誉な);羞耻(不体裁な);丢人。(はじをかく。)	はじ		【名词】
\\	弾く	
\\	弹。 
\\	排斥。 
\\	打算盘。 
\\	漏接球。棒球运动中,守门队员接球时触手而未接住。	はじく		【他动词・五段/一类】
\\	始め	
\\	初,始。初始。(はじめること。) 
\\	开头,初,始,起先。刚开始不久的阶段,时候,也用作副词。(はじめたばかりの段階・時。副詞的にも用いる。) 
\\	初,始起源,起因,亦指先例。(起源。起こり。また、先例。) 
\\	头一个,老大,前者,初,始。许多中的第一个,亦指前面的东西。(多くのもののうち、第一番目のもの、また、先の方のもの。) 
\\	以…为首。表示具有代表性的例子。(それが代表的な例であることを表す。) 
\\	同:初め	はじめ		【名词】
\\	始めまして	
\\	(初次见面时说的寒暄语)幸会,初次见面,请多关照。(初対面の人にいうあいさつの語。はじめてお目にかかります、の意。) 
\\	同:初めまして	はじめまして		【接续词】
\\	恥じらう	
\\	害羞,羞涩。(恥ずかしがる。はにかむ。)	はじらう		【他动词・五段/一类】
\\	恥じる	
\\	害羞,羞愧,惭愧。 
\\	败坏名誉,不相称。	はじる		【自他・一段/二类】
\\	蓮	
\\	〈植〉莲,荷。(池・沼・水田に栽培される多年草。)	はす		【名词】
\\	筈	
\\	应该,理应;会。(当然そうなること) 
\\	该,当。(予定) 
\\	道理。(理由,わけ) 
\\	箭尾。(矢はず) 
\\	弓两端系弦的地方。(弓はず) 
\\	手掌的虎口。(相撲で) 
\\	应当;理应(形式名词用法);(表理由,缘故)道理(形式名词用法);箭尾	はず		【名词】
\\	弾む	
\\	弹起,反弹。 
\\	起劲,上劲,来劲。 
\\	(呼吸)急促,胸口扑通扑通地跳。 
\\	(情绪)高涨。	はずむ		【自动词・五段/一类】
\\	破損	
\\	破损,损坏。	はそん		【名・自他・サ变/三类】
\\	機	
\\	機が熟す。/正是时候。	はた		【惯用句】
\\	果たして	
\\	果然。 
\\	果真;真的;到底;与表示疑问、假定的词一起使用。	はたして		【副词】
\\	果たす	
\\	(接动词连用形后表示)完全……;彻底……;……光。 
\\	完成;实现。	はたす		【他动词・五段/一类】
\\	裸足	
\\	赤脚,赤足,光着脚。 
\\	(由赤脚逃跑之意转指)根本敌不过,甘拜下风。	はだし		【名词】
\\	蜂蜜	
\\	蜂蜜。	はちみつ		【名词】
\\	発掘	
\\	发掘,挖掘,发现。	はっくつ		【名・他动词・サ变/三类】
\\	発生	
\\	发生。 
\\	孳生。	はっせい		【名・自动词・サ变/三类】
\\	発足	
\\	出发;动身。 
\\	(新成立的团体等)开始活动。	はっそく		【名・自动词・サ变/三类】
\\	初	
\\	初。	はつ		【名词】
\\	発	
\\	颗,发。	はつ		【助数词】
\\	発育	
\\	发育,成长。	はついく		【名・自动词・サ变/三类】
\\	発芽	
\\	发芽。	はつが		【名・自动词・サ变/三类】
\\	発言	
\\	发言。(言葉を発すること。口頭で意見を述べること。)	はつげん		【名・他动词・サ变/三类】
\\	発病	
\\	发病,得病。	はつびょう		【名・自动词・サ变/三类】
\\	初耳	
\\	初次听到,首次听到,前所未闻(而感觉稀奇)。	はつみみ		【名词】
\\	果て	
\\	[境界的]边,边际;尽头。 
\\	最后,末了,结局;下场。	はて		【名词】
\\	果てる	
\\	……之极。 
\\	死。 
\\	终,尽,完。	はてる		【自动词・一段/二类】
\\	話し合い	
\\	商量,商谈,协商。	はなしあい		【名词】
\\	甚だ	
\\	非常,极其。 
\\	全然,完全。	はなはだ		【副词】
\\	華々しい	
\\	华丽,绚丽,辉煌,轰轰烈烈。显赫,灿烂美丽而好看。 
\\	同:花々しい	はなばなしい		【形容词/イ形容词】
\\	花びら	
\\	花瓣。	はなびら		【名词】
\\	華やか	
\\	华丽,华美,华贵。 
\\	辉煌,显赫,活跃;引人注目『成』;显眼。	はなやか		【形容动词/ナ形容词】
\\	阻む	
\\	阻止,阻挡;挡。	はばむ		【他动词・五段/一类】
\\	浜	
\\	海滨,湖滨。 
\\	吃下来的棋子。提子。 
\\	港口。横滨。	はま		【名词】
\\	浜辺	
\\	海滨,湖滨。	はまべ		【名词】
\\	填まる	
\\	套上。(ぴったり入る。) 
\\	恰好合适。(しっくりと合う。) 
\\	陷入,掉进。(落ちいる。) 
\\	中(计)。(だまされる。)	はまる		【自动词・五段/一类】
\\	生やす	
\\	使(植物等)生长,蓄(发),留(须)。	はやす		【他动词・五段/一类】
\\	早める	
\\	提前,提早。把日期、时间等往前移。(期日・時刻などを早くする。くりあげる。《早》) 
\\	快速,加速。使增加速度,催促。(速さを増させる。急がせる。《速》) 
\\	同:速める	はやめる		【他动词・一段/二类】
\\	流行	
\\	流行;时髦;时兴(流行ること。その時々の人々に好まれ、人気のある物事)。	はやり		【名词】
\\	腹立ち	
\\	生气,愤怒。	はらだち		【名词】
\\	原っぱ	
\\	杂草丛生的旷野,空地。	はらっぱ		【名词】
\\	張り紙	
\\	便签,附笺。 
\\	招贴,广告,标语。 
\\	贴纸,糊纸。	はりがみ		【名词】
\\	遥か	
\\	遥远,远远,远为。	はるか		【形容动词/ナ形容词】
\\	破裂	
\\	破裂。(勢いよく裂けること。または、話し合いがまとまらないで,物別れになること。決裂。)	はれつ		【自动词・サ变/三类】
\\	腫れる	
\\	肿,肿胀。	はれる		【自动词・一段/二类】
\\	班	
\\	班,组。(仕事・行動を一緒にするために分けられたグループ。)	はん		【名词】
\\	判	
\\	(纸或书本的规格)开数[紙の大きさ]开数.	はん		【名词】
\\	繁栄	
\\	繁荣,昌盛,兴旺。(栄えること。繁盛。)	はんえい		【自动词・サ变/三类】
\\	反感	
\\	反感。	はんかん		【名词】
\\	版画	
\\	版画,木刻。	はんが		【名词】
\\	反響	
\\	反响,响应,回音,声波碰到物体发生反射而产生声音。 
\\	反响,反应。	はんきょう		【名・自动词・サ变/三类】
\\	判決	
\\	判决。	はんけつ		【名词】
\\	反撃	
\\	反击,还击。(軍を引き返して、追って来る敵をうつこと。攻撃を受けた者が逆に攻撃に転ずること。)	はんげき		【自动词・サ变/三类】
\\	反射	
\\	反射。	はんしゃ		【名・自他・サ变/三类】
\\	繁殖	
\\	【自动·三类】 
\\	繁殖,孳生,滋生(動物や植物が生まれてふえること。生殖により個体数がふえて再生産が行われること)。	はんしょく		【名词】
\\	繁盛	
\\	繁荣昌盛,兴旺,兴隆。	はんじょう		【名・自动词・サ变/三类】
\\	反する	
\\	违反。 
\\	相反。 
\\	造反。	はんする		【自动词・サ变/三类】
\\	判定	
\\	判定,判断。经判断顶下某一个。 
\\	判定,判断。在拳击、摔跤、柔道等比赛中,当过了规定时间仍不分胜负时,裁判员根据比赛情形裁定胜负,亦指裁判员的裁定。	はんてい		【名・他动词・サ变/三类】
\\	半端	
\\	零头,零星;不彻底;零数;无用的人。	はんぱ		【名・形容动词/ナ形容词】
\\	反発	
\\	弹回,回弹,排斥。撞回,顶回去。 
\\	顶撞,反抗。指顶嘴。 
\\	回升,反弹。有下跌趋势的行情暂时上涨。 
\\	同:反撥	はんぱつ		【名・自他・サ变/三类】
\\	反乱	
\\	叛乱、反叛。	はんらん		【名词】
\\	氾濫	
\\	泛滥。(河川の水が堤防からあふれ出ること。) 
\\	充斥,过多。(好ましくないものが広がりはびこること。)	はんらん		【自动词・サ变/三类】
\\	黴菌	
\\	细菌。	ばいきん		【名词】
\\	倍率	
\\	倍数。 
\\	倍率。被扩大或缩小的物体与原物的比例。物理上的倍率指使用望远镜等的光学系的角倍率。	ばいりつ		【名词】
\\	馬鹿らしい	
\\	愚蠢的;无聊的;不值得。(無意味でくだらなく見えるさま。) 
\\	同:馬鹿馬鹿しい	ばからしい		【形容词/イ形容词】
\\	漠然	
\\	含混,含糊;笼统。	ばくぜん		【形容动词/ナ形容词】
\\	爆弾	
\\	炸弹。 
\\	爆炸性的。	ばくだん		【名词】
\\	爆破	
\\	爆破,炸毁。	ばくは		【名・他动词・サ变/三类】
\\	暴露	
\\	曝晒,风吹日晒。(風雨にさらすこと。また、さらされること。) 
\\	暴露,泄露,败露,揭露。(むき出しにすること。特に、悪事・秘密などをあばいて明るみに出すこと。また、それらが明るみに出ること。)	ばくろ		【名・自他・サ变/三类】
\\	化ける	
\\	化,变。 
\\	化装,乔装,改装。 
\\	突变,化为。变成另一种全然不同的东西。	ばける		【自动词・一段/二类】
\\	罰	
\\	惩罚,处罚。	ばち		【名词】
\\	発条	
\\	发条;弹簧 
\\	发条;弹簧	ばね		【名词】
\\	万	
\\	数的单位,万;1000的10倍,亦指数量多。	ばん		【名词】
\\	万人	
\\	万人,众人。(すべての人。ばんにん。)	ばんじん		【名词】
\\	万能	
\\	万能,全能;效果,效验。	ばんのう		【名・形容动词/ナ形容词】
\\	番目	
\\	连接部分,接合处。(二つのものが組み合わさった部分。)	ばんめ		【名词】
\\	非	
\\	非,不。(そうではないという意味。)	ひ		【名词】
\\	費	
\\	费,费用(物事を行うのに必要な金銭)。	ひ		【名词】
\\	延いては	
\\	进而	ひいては		【副词】
\\	控室	
\\	休息室。(控えて待っているための部屋。)	ひかえしつ		【名词】
\\	控える	
\\	等候,等待。 
\\	在旁边,在近旁。	ひかえる		【自动词・一段/二类】
\\	悲観	
\\	悲观;失望。	ひかん		【名・自他・サ变/三类】
\\	匹	
\\	头,只,条,尾。计数兽、鸟、鱼、虫等的量词。(獣、鳥、魚、虫などを数える語。) 
\\	匹。以2反为织物单位,用于做一套成人和服加一件羽织外褂等时。(2反分がひと続きの織物の単位。大人の着物と羽織を対で作るときなどに用いる。)	ひき		【接头/接尾】
\\	引き上げる	
\\	吊起,曳起拉上来,提到高处。 
\\	打捞。 
\\	提高(物价)。 
\\	返回,收回,取回,撤回。	ひきあげる		【他动词・一段/二类】
\\	率いる	
\\	带领。 
\\	率领,统率。	ひきいる		【他动词・一段/二类】
\\	引き受ける	
\\	承担,承接,接受。 
\\	继承。 
\\	承当。 
\\	保证,担保。 
\\	签署(汇票),汇票的付款人作为票据金额的付款人签字。	ひきうける		【他动词・一段/二类】
\\	引き下げる	
\\	拉下;划价,降低,减低。 
\\	使后退,撤回。	ひきさげる		【他动词・一段/二类】
\\	引きずる	
\\	使物体的下侧贴着地面或地板移动,拖。 
\\	尽管本人不愿意却硬拉着去,拖,拉。 
\\	不立即把事做完,而是使其拖延。 
\\	无法忘怀,无法舍弃。 
\\	一般用“ひきずられる”的形式,表示影响他人行动、行为,给予影响。	ひきずる		【他动词・五段/一类】
\\	引き取る	
\\	退出。 
\\	取回。 
\\	断气。	ひきとる		【自他・五段/一类】
\\	引き分け	
\\	平局,和局,不分胜负。(試合や勝負事で、勝負がつかないまま終わられること。)	ひきわけ		【名词】
\\	否決	
\\	否决。(会議に出た議案を、認められないと議決すること。)	ひけつ		【名・他动词・サ变/三类】
\\	非行	
\\	不正当的行为,违背道德规范的行为,流氓行为。严重不良行为,失足。特指青少年的违法行为和违反社会规范的行为等。(よくない行い。特に青少年が法律で禁じられたことや社会規範に反した行いなどをすること。)	ひこう		【名词】
\\	日頃	
\\	【副】 
\\	平时,素日,平常。(ふだん。へいぜい)	ひごろ		【名词】
\\	久しい	
\\	好久,许久。过去很长时间。 
\\	久违。好久才见面。	ひさしい		【形容词/イ形容词】
\\	久し振り	
\\	【形动】 
\\	(隔了)好久
\\	许久。(前にそのことを経験してから、再び同じことになるまでに長い日数のあったこと。また、そのさま。)	ひさしぶり		【名词】
\\	悲惨	
\\	悲惨,凄惨。	ひさん		【名・形容动词/ナ形容词】
\\	秘書	
\\	秘书。 
\\	秘藏的书籍。	ひしょ		【名词】
\\	比重	
\\	比重。 
\\	(所占的)比例,对比。	ひじゅう		【名词】
\\	密か	
\\	秘密地,悄悄地。	ひそか		【形容动词/ナ形容词】
\\	浸す	
\\	浸,泡。把物体放进液体中。 
\\	浸,泡。使整个物体饱含液体。	ひたす		【他动词・五段/一类】
\\	一向	
\\	完全,全然,一向,总,一点儿也……。	ひたすら		【副词】
\\	左利き	
\\	左撇子。 
\\	喜欢喝酒,亦指好喝酒的人。	ひだりきき		【名词】
\\	引っ掻く	
\\	用力搔,抓,挠。(つめで掻く。)	ひっかく		【他动词・五段/一类】
\\	引っ掛ける	
\\	挂(上)。 
\\	披上。 
\\	欺骗。 
\\	少量的酒一口喝掉。稍微喝一些。 
\\	溅(水);吐(唾沫)。 
\\	借引子,借机会。	ひっかける		【他动词・一段/二类】
\\	必修	
\\	必修,必学。	ひっしゅう		【名词】
\\	未	
\\	十二支第八;未,未时。	ひつじ		【名词】
\\	必然	
\\	必然。	ひつぜん		【名词】
\\	一息	
\\	喘口气。 
\\	一口气。 
\\	一把劲,稍努把劲儿。	ひといき		【名词】
\\	単	
\\	(缝纫)单衣。(裏をつけないで仕立てた衣類の総称。特に、裏をつけない長着。)	ひとえ		【名词】
\\	人柄	
\\	人品,人格。 
\\	品质好,品格好的事物。	ひとがら		【名・形容动词/ナ形容词】
\\	一頃	
\\	前些日子,曾有一时。	ひところ		【名词・副词】
\\	人質	
\\	债务的担保人。 
\\	人质。	ひとじち		【名词】
\\	酷い	
\\	凄惨,惨。极悲惨,可怜之极。(あまりにも悲惨である。いあましい。) 
\\	酷,残酷,残暴,狠毒,残忍毒辣。无情且冷酷,无慈悲。(思いやりがなくひどい。無慈悲だ。) 
\\	同:惨い	ひどい		【形容词/イ形容词】
\\	日取り	
\\	日期,日子,日程。	ひどり		【名词】
\\	雛	
\\	雏;小鸡。(鳥の成長の最初の段階をいう。) 
\\	黄口小儿。(幼稚、未熟な者。)	ひな		【名词】
\\	日向	
\\	日本地名。(日本のある県。)	ひなた		【名词】
\\	非難	
\\	非难,责备,责难,谴责,指责,非议。(相手の欠点や過失を取り上げて責めること。) 
\\	同:批難	ひなん		【名・他动词・サ变/三类】
\\	避難	
\\	避难,逃难。(災難を避けること。災害を避けて、安全な場所へ立ちのくこと。)	ひなん		【自动词・サ变/三类】
\\	日の丸	
\\	太阳形。(太陽をかたどった赤色や金色の丸。古くから、扇や旗さし物などに用いられた。) 
\\	太阳旗,日本国旗。(白地に、太陽をかたどった赤い丸を描いた旗。日の丸の旗。)	ひのまる		【名词】
\\	火花	
\\	火花。 
\\	火星。	ひばな		【名词】
\\	日々	
\\	天天,每天。	ひび		【名词】
\\	悲鳴	
\\	恐惧惊讶时的惨叫,惊叫声。 
\\	哀鸣,悲鸣。	ひめい		【自动词・サ变/三类】
\\	百科事典	
\\	百科全书。	ひゃっかじてん		【名词】
\\	百科辞典	
\\	百科辞典	ひゃっかじてん		【名词】
\\	冷やかす	
\\	冷却,使……凉。 
\\	嘲弄,戏弄,嘲笑,开玩笑,愚弄。 
\\	只询价不买。	ひやかす		【他动词・五段/一类】
\\	日焼け	
\\	晒干。 
\\	晒黑。	ひやけ		【自动词・サ变/三类】
\\	票	
\\	票,选票。	ひょう		【名词】
\\	標語	
\\	标语。(主張・信条や行動の目標、指示内容などをわかりやすく簡潔に言い表した語句。スローガン。モットー。)	ひょうご		【名词】
\\	平たい	
\\	平易,简明,浅易,浅显。简单,易懂。 
\\	平,平坦,扁,扁平。开阔且没有高低凹凸。	ひらたい		【形容词/イ形容词】
\\	比率	
\\	比率,比例。	ひりつ		【名词】
\\	肥料	
\\	肥料。施与土壤、植物,使土壤肥沃并有助于植物生长的物质。(土壌をこやし、植物の生育に役立って増収をもたらす効果をもつ土壌・植物に施す物質。窒素・リン・カリウムは肥料の三要素と呼ばれ重要である。こやし。こえ。)	ひりょう		【名词】
\\	比例	
\\	举例比较。 
\\	比例,正比。 
\\	成比例,相称。	ひれい		【名・自动词・サ变/三类】
\\	疲労	
\\	疲劳,疲乏,疲倦,疲惫。	ひろう		【自动词・サ变/三类】
\\	広まる	
\\	扩大;传播,蔓延。	ひろまる		【自动词・五段/一类】
\\	貧困	
\\	贫困,贫穷;贫乏。	ひんこん		【名・形容动词/ナ形容词】
\\	品質	
\\	质量。	ひんしつ		【名词】
\\	品種	
\\	种类。 
\\	品种。	ひんしゅ		【名词】
\\	貧弱	
\\	贫乏,欠缺。 
\\	贫弱,逊色,不漂亮,寒碜。	ひんじゃく		【名・形容动词/ナ形容词】
\\	頻繁	
\\	频繁,屡次。	ひんぱん		【名・形容动词/ナ形容词】
\\	美	
\\	美,美丽。(美しいこと。) 
\\	出色。(りっぱなこと。)	び		【名词】
\\	微笑	
\\	微笑。	びしょう		【自动词・サ变/三类】
\\	美術	
\\	美术。	びじゅつ		【名词】
\\	吃驚	
\\	【自动·三类】 
\\	吃惊,吓一跳。(おどろくこと。びっくりすること。)	びっくり		【副词】
\\	描写	
\\	描写,描绘,描画,描述。	びょうしゃ		【名・他动词・サ变/三类】
\\	微量	
\\	微量,少量,极少的量。	びりょう		【名词】
\\	敏感	
\\	敏感,感觉敏锐,灵敏。	びんかん		【名・形容动词/ナ形容词】
\\	貧乏	
\\	穷,贫穷,贫困,贫苦。	びんぼう		【名・形容动词/ナ形容词・自动词・サ变/三类】
\\	歩	
\\	步行。 
\\	(距离单位)步。 
\\	《相关惯用语》	ふ		【名词】
\\	不意	
\\	冷不防,忽然,突然,抽冷子,意外,想不到,出其不意。	ふい		【名・形容动词/ナ形容词】
\\	封	
\\	封,封口,封上。(封筒・容器などが開いたり、不法に開けられたりしないように閉じること。また、その閉じた部分。) 
\\	封条。(封じ目につけるしるし。)	ふう		【名词】
\\	封鎖	
\\	封锁,冻结。(とじて出入りまたは出し入れさせないこと。敵国または重大な義務違反を行なった国の港などへの船舶の出入りを海軍力によって遮断すること。海上封鎖。)	ふうさ		【名・他动词・サ变/三类】
\\	風習	
\\	风俗习惯,习惯,风俗。(その土地の慣わし。習慣。)	ふうしゅう		【名词】
\\	風俗	
\\	风俗。(一定の社会集団に広く行われている生活上のさまざまなならわし。しきたり。風習。)	ふうぞく		【名词】
\\	風土	
\\	风土,水土。(土地の状態。住民の習慣や文化に影響を及ぼす、気候・地形・地質など。)	ふうど		【名词】
\\	不可欠	
\\	不可缺少,必需,必须。	ふかけつ		【名・形容动词/ナ形容词】
\\	深める	
\\	加深,加强。	ふかめる		【他动词・一段/二类】
\\	不吉	
\\	不吉利,不祥。	ふきつ		【名・形容动词/ナ形容词】
\\	不況	
\\	不景气,萧条。	ふきょう		【名词】
\\	布巾	
\\	抹布。	ふきん		【名词】
\\	福	
\\	福,福气。幸福,幸运。(さいわい。しあわせ。幸運。) 
\\	日本地名,福。位于福井县福井市。	ふく		【名词】
\\	復旧	
\\	恢复原状,修复,使重新回到先前的状态,被破坏或被搞乱的东西恢复原来状态。	ふくきゅう		【名・自他・サ变/三类】
\\	複合	
\\	复合,合成。	ふくごう		【名・自他・サ变/三类】
\\	福祉	
\\	福利,福祉。	ふくし		【名词】
\\	覆面	
\\	蒙面,蒙上脸。 
\\	不出面,不露名。(匿名。)	ふくめん		【名・自动词・サ变/三类】
\\	膨れる	
\\	胀,鼓出。 
\\	噘嘴,不高兴。	ふくれる		【自动词・一段/二类】
\\	老ける	
\\	上年纪,老。亦指看起来像老人。 
\\	老,老化。因陈旧而质量变差。	ふける		【自动词・五段/一类】
\\	布告	
\\	公布,宣布,宣告,公告,布告。	ふこく		【名・他动词・サ变/三类】
\\	富豪	
\\	富豪,大财主,富翁。	ふごう		【名词】
\\	負債	
\\	负债,欠债。	ふさい		【名词】
\\	相応しい	
\\	适合,适称,相称。	ふさわしい		【形容词/イ形容词】
\\	不在	
\\	不在,不在家。	ふざい		【名词】
\\	負傷	
\\	负伤,受伤;挂彩,挂花。(きずを負うこと。けが。)	ふしょう		【名・自他・サ变/三类】
\\	不審	
\\	可疑,疑问,不清楚。	ふしん		【名・形容动词/ナ形容词】
\\	不振	
\\	成绩不好,形势不佳,业绩不兴旺,萧条,萎缩。	ふしん		【名・形容动词/ナ形容词】
\\	不順	
\\	不顺,不调,异常。	ふじゅん		【名・形容动词/ナ形容词】
\\	付属	
\\	【自动·三类】 
\\	附属。(主となるものに付き従っていること)	ふぞく		【名词】
\\	負担	
\\	承担,负担(的责任,义务,工作)。(身に引き受けること。) 
\\	『比較』“承担”“负担”:“承担”は動詞。“责任,工作,业务”に対していう。“负担”は名詞,動詞。“责任,工作,费用”に対していう。 
\\	负;背(行李等)。(負う。)	ふたん		【名・他动词・サ变/三类】
\\	不調	
\\	破裂,失败,不成功,谈不拢。 
\\	不顺利,一时不振,萎靡。	ふちょう		【名・形容动词/ナ形容词】
\\	復活	
\\	恢复,复兴,复辟。 
\\	复活,苏生。	ふっかつ		【名・自他・サ变/三类】
\\	復興	
\\	复兴,重建。衰败的事物再度兴隆、昌盛。	ふっこう		【名词】
\\	沸騰	
\\	沸腾。	ふっとう		【自动词・サ变/三类】
\\	仏	
\\	佛(仏陀);佛像(仏像);释迦
\\	。(釈迦。) 
\\	死者,亡魂。(死者。)	ふつ		【名词】
\\	不図	
\\	偶然,一下,突然,忽然。(思いがけなく、突然起こるさま。不意に。たやすく。簡単に。動作の素早いさま。)	ふと		【副词】
\\	不当	
\\	不正当,不当,不合理,非法。	ふとう		【形容动词/ナ形容词】
\\	不動産	
\\	不动产。	ふどうさん		【名词】
\\	赴任	
\\	赴任,上任。	ふにん		【自动词・サ变/三类】
\\	不評	
\\	声誉不佳,名誉坏,评价低,不受欢迎。	ふひょう		【名词】
\\	不便	
\\	不便,不方便,不便利。	ふびん		【名词】
\\	不服	
\\	不服从。 
\\	异议,抗议。 
\\	不满意,不心服。	ふふく		【名・形容动词/ナ形容词】
\\	普遍	
\\	普遍。(広く行き渡ること。ある範囲のすべてのものに共通し、例外は考えられないこと。)	ふへん		【名词】
\\	踏まえる	
\\	作为根据(立足点)。 
\\	踏;踩。	ふまえる		【他动词・一段/二类】
\\	文	
\\	文章,作文,论文。(文章) 
\\	文学。(武に対して、学問や文芸など。) 
\\	句,句子。(言語の単位の一つ。言葉で表現する際の、完結した内容を表す最小の単位。文字で表す場合には、通常、分の切れ目に句点「。」を打つ。センテンス。)	ふみ		【名词】
\\	不明	
\\	不详,不明。 
\\	蠢,无能。	ふめい		【名・形容动词/ナ形容词】
\\	扶養	
\\	扶养。	ふよう		【他动词・サ变/三类】
\\	振り	
\\	振动,摆动。 
\\	(外表的)样子,打扮。 
\\	装做,假装。 
\\	(舞蹈、戏剧的)动作,姿势。 
\\	陌生。 
\\	突然;临时。 
\\	甩袖。女和服从腋下到袖下的开口部分。	ふり		【名词】
\\	振り出し	
\\	摇出,晃出,簸出,摇晃出,亦指摇晃出东西的容器。 
\\	出发点,亦指事物之初,开始,开端,最初。 
\\	开票,出票,发票,开出票据、支票等。	ふりだし		【名词】
\\	不良	
\\	不好,不良,坏,次。 
\\	品行不端,品质不好,小流氓。	ふりょう		【名・形容动词/ナ形容词】
\\	浮力	
\\	浮力。	ふりょく		【名词】
\\	付録	
\\	附带的东西,增刊。(書籍·新聞·雑誌に添えてある紙面または冊子その他のもの。おまけ。) 
\\	(另外的)记录。(付け加えて記録すること。また、その記録。) 
\\	同:附録	ふろく		【名・他动词・サ变/三类】
\\	分	
\\	份儿。 
\\	身分,地位。 
\\	本分。 
\\	程度,状态,情况。 
\\	身分。 
\\	份。	ふん		【名词】
\\	憤慨	
\\	【自他动·三类】 
\\	气愤。愤慨。(いきどおりなげくこと。ひどく腹をたてること。)	ふんがい		【名词】
\\	紛失	
\\	遗失,丢失,失落。	ふんしつ		【名・自他・サ变/三类】
\\	噴出	
\\	喷出;射出。	ふんしゅつ		【自动词・サ变/三类】
\\	紛争	
\\	纠纷,争端。	ふんそう		【自动词・サ变/三类】
\\	奮闘	
\\	奋战。(力を出してたたかうこと。) 
\\	奋斗。(力を出して事に当たること。)	ふんとう		【自动词・サ变/三类】
\\	粉末	
\\	粉末。	ふんまつ		【名词】
\\	部	
\\	局部。部分。(分けること。) 
\\	统治,统辖。(統べる。) 
\\	部。组织的部门之一。(官庁・会社などの業務組織の区分の一つ。) 
\\	部首。(部首。) 
\\	部。用于清点分开的东西。(数冊あわせて完結する書類をまとめて数える語。) 
\\	部。册。份。用于计量书、报等量词。(冊。)	ぶ		【名词】
\\	部下	
\\	部下,部属,属下。	ぶか		【名词】
\\	侮辱	
\\	侮辱,凌辱。	ぶじょく		【名・他动词・サ变/三类】
\\	武装	
\\	武装,军事装备。	ぶそう		【自动词・サ变/三类】
\\	斑	
\\	斑,斑点。斑驳,花斑。(まだら。ぶち。)	ぶち		【名词】
\\	物資	
\\	物资。	ぶっし		【名词】
\\	物体	
\\	物体。	ぶったい		【名词】
\\	物議	
\\	物议,众人的批评,群众的批评。	ぶつぎ		【名词】
\\	仏像	
\\	佛像,佛的雕塑像和画像等的总称,通常指雕塑像。	ぶつぞう		【名词】
\\	無難	
\\	无灾无难。 
\\	没有缺点,无可非议『成』,说得过去。	ぶなん		【名・形容动词/ナ形容词】
\\	部門	
\\	部门,部类,方面。	ぶもん		【名词】
\\	ぶら下げる	
\\	佩带,悬挂。 
\\	提。	ぶらさげる		【他动词・一段/二类】
\\	武力	
\\	武力,兵力。	ぶりょく		【名词】
\\	無礼	
\\	没有礼貌,不恭敬,失礼。	ぶれい		【名・形容动词/ナ形容词】
\\	文化財	
\\	文物,文化遗产,文化财富。(人間の精神的な働きがかわって生み出されたもので、文化的価値を有するもの。)	ぶんかざい		【名词】
\\	分業	
\\	分工(合作)。(手分けをして、仕事をすること。) 
\\	分工(序)。(ある製品を生産するために、その生産の全工程を分割し、労働者がそれぞれの工程を分担すること。)	ぶんぎょう		【名・他动词・サ变/三类】
\\	文語	
\\	书面语。 
\\	文言,古文语。	ぶんご		【名词】
\\	分散	
\\	分散,散开。(分け散らすこと。) 
\\	〈理〉色散,频散,弥散。(物理学で、同一媒質中の波の進行速度が、振動数によって変化する現象。光が波長によりスペクトルに分解することなど。) 
\\	〈数〉方差。(数学で、資料の散らばりぐあいを表す値。各値と平均値との差を2乗し、算術平均したもの。分散の正の平方根が標準偏差となる。)	ぶんさん		【自动词・サ变/三类】
\\	分子	
\\	化学元素的分子。 
\\	团体中的一分子,成员。 
\\	数学与分母相对的分子,分数。	ぶんし		【名词】
\\	文書	
\\	文书。(かきもの。)	ぶんしょ		【名词】
\\	分担	
\\	分担。(仕事などを分けて受け持つこと。)	ぶんたん		【名・他动词・サ变/三类】
\\	分配	
\\	配分,配给,分给。	ぶんぱい		【名词】
\\	分母	
\\	分母。	ぶんぼ		【名词】
\\	分離	
\\	分离,分开,,脱离,隔离。(わけはなすこと。)	ぶんり		【名・自他・サ变/三类】
\\	兵器	
\\	兵器,武器;军火。(航空機・火器・通信機材・車両など,戦闘用のすべての資材。狭義には,敵を殺傷・破壊する器材をいう。)	へいき		【名词】
\\	閉口	
\\	闭口(无言);为难;没办法;折服。	へいこう		【名・自动词・サ变/三类】
\\	平行	
\\	平行。 
\\	并行。	へいこう		【自动词・サ变/三类】
\\	閉鎖	
\\	封闭,关闭。	へいさ		【名・他动词・サ变/三类】
\\	兵士	
\\	战士,士兵。	へいし		【名词】
\\	平常	
\\	平常,照常。	へいじょう		【名词】
\\	平方	
\\	日本人的姓氏。 
\\	日本地名。	へいほう		【名词】
\\	並列	
\\	并列,并排。	へいれつ		【名・自他・サ变/三类】
\\	辟易	
\\	感到为难;感到束手无策(閉口すること。うんざりすること)。 
\\	辟易。为势所迫而退缩。(勢いに押されて、しりごみすること。)	へきえき		【自动词・サ变/三类】
\\	臍	
\\	脐,肚脐。(ほぞ。) 
\\	小坑。(ものの表面の中央部にある小さなくぼみやでっぱり。) 
\\	中心,中央,关键点。(物の中央、中心。また、重要な部分。ポイント。) 
\\	附:	へそ		【名词】
\\	隔たる	
\\	相隔,离,距。 
\\	不同,不一致,有差别。 
\\	疏远,发生隔阂,隔膜。	へだたる		【自动词・五段/一类】
\\	謙る	
\\	谦恭,谦逊。 
\\	同:謙る,遜る	へりくだる		【自动词・五段/一类】
\\	経る	
\\	(时间)经过。 
\\	(空间)经过,通过。 
\\	经过,经历,结束。	へる		【自动词・一段/二类】
\\	編	
\\	编;编辑,编纂,编著;篇,首,册,卷。	へん		【名词】
\\	偏	
\\	偏,偏袒,不公平。 
\\	(汉字的)偏旁。	へん		【名词】
\\	変革	
\\	变革,改革,变化。	へんかく		【名・自动词・サ变/三类】
\\	返還	
\\	归还。(もとの所へ返すこと。)	へんかん		【名・他动词・サ变/三类】
\\	偏見	
\\	偏见,僻见,偏执。	へんけん		【名词】
\\	返済	
\\	偿还,还债,还款。	へんさい		【名・他动词・サ变/三类】
\\	変遷	
\\	变迁。	へんせん		【名・自动词・サ变/三类】
\\	返答	
\\	回答,回话。	へんとう		【名词】
\\	変動	
\\	【自动·三类】 
\\	变动,波动,改变,变化(物事が変わり動くこと)。	へんどう		【名词】
\\	弁解	
\\	辩解,分辩,辩明。辩白。(言い訳をすること。言い訳。)	べんかい		【名・自他・サ变/三类】
\\	便宜	
\\	方便,便利。	べんぎ		【名・形容动词/ナ形容词】
\\	弁護	
\\	辩护,辩解。(その人の利益となることを主張して助けること。その人のために言い開きをして助けること。)	べんご		【名・他动词・サ变/三类】
\\	弁償	
\\	赔,赔偿。	べんしょう		【名・他动词・サ变/三类】
\\	弁論	
\\	【自动·三类】 
\\	辩论。(大勢の前で、意見を述べること。) 
\\	辩护,申辩,申述,陈述。(互いに論じ合うこと。また、その議論。)	べんろん		【名词】
\\	穂	
\\	物体的尖端。 
\\	麦穗。	ほ		【名词】
\\	保育	
\\	保育。(乳幼児を保護し、育てること。)	ほいく		【名・他动词・サ变/三类】
\\	法案	
\\	法案,法律草案。	ほうあん		【名词】
\\	崩壊	
\\	崩溃;蜕变,衰变。 
\\	衰变。	ほうかい		【名词】
\\	法学	
\\	法学。	ほうがく		【名词】
\\	放棄	
\\	放弃,丢弃,抛弃。 
\\	弃权。	ほうき		【名・他动词・サ变/三类】
\\	封建	
\\	封建。(皇帝・天子・王などが,直属の公領以外の土地を諸侯などに分け与え領有させること。また,その制度。西洋では,主君が封土を与え,臣下が軍役奉仕を誓うという双務契約にもとづく主従関係。)	ほうけん		【名词】
\\	豊作	
\\	丰收。	ほうさく		【名词】
\\	方策	
\\	方策,典籍。 
\\	方策,方略,对策,谋划策略。	ほうさく		【名词】
\\	奉仕	
\\	服务,不计报酬地效劳。 
\\	廉价销售商品。 
\\	侍奉(神佛或君主)。	ほうし		【名・自动词・サ变/三类】
\\	方式	
\\	方式。(形式。) 
\\	手续。(手続き。) 
\\	方法。(やり方。)	ほうしき		【名词】
\\	放射	
\\	从一点向四面八方射出。 
\\	物体向外辐射红外线、电磁波等,也指放射出的电磁波、粒子线的总称。	ほうしゃ		【名・他动词・サ变/三类】
\\	放射能	
\\	放射能。	ほうしゃのう		【名词】
\\	報酬	
\\	报酬。(労働や物の使用などに対するお礼の金銭や物品)	ほうしゅう		【名词】
\\	放出	
\\	放出,喷出,使喷出。 
\\	投放,发放。	ほうしゅつ		【名・自他・サ变/三类】
\\	報じる	
\\	(新闻、事件、时间等的)报道,报时。 
\\	报恩。 
\\	报仇。 
\\	同:報ずる	ほうじる		【自他・一段/二类】
\\	報ずる	
\\	报,报答。(むくいる。かえす。) 
\\	报告,报知,报导。(知らせる。告げ伝える。)	ほうずる		【自他・サ变/三类】
\\	放置	
\\	放置,搁置,置之不理。	ほうち		【名・他动词・サ变/三类】
\\	法廷	
\\	法庭。	ほうてい		【名词】
\\	報道	
\\	报道。	ほうどう		【名・他动词・サ变/三类】
\\	褒美	
\\	赞美,表扬,褒奖,也指奖赏、奖品。	ほうび		【名词】
\\	葬る	
\\	埋葬,葬埋。把尸体、遗骨等埋入墓穴或土中。 
\\	葬送,弃而不顾;遮掩,秘而不宣『成』。不让世人知道,亦指舍弃。	ほうむる		【他动词・五段/一类】
\\	放り込む	
\\	投入,扔进去。	ほうりこむ		【他动词・五段/一类】
\\	放り出す	
\\	抛出去,扔出去。 
\\	(中途)放弃,丢开。 
\\	开除,推出门外。 
\\	毫不吝惜地拿出。	ほうりだす		【他动词・五段/一类】
\\	飽和	
\\	蒸气、电流、磁力、溶质等在一定条件下达到最大限度的状态。 
\\	泛指事物达到最高限度。	ほうわ		【名・自动词・サ变/三类】
\\	保温	
\\	保温。保持温度,尤指保暖。	ほおん		【名・自动词・サ变/三类】
\\	捕獲	
\\	捕获。 
\\	(在海上)缴获,国际法上指战争时期缴获敌国或被认为有敌意的中立国的船只。	ほかく		【名・他动词・サ变/三类】
\\	保管	
\\	保管。(大切なものを壊したり無くしたりしないように保存すること。)	ほかん		【名・他动词・サ变/三类】
\\	補給	
\\	补给,补充,供给。	ほきゅう		【名・他动词・サ变/三类】
\\	補強	
\\	补充,加强,强化。	ほきょう		【名・他动词・サ变/三类】
\\	保険	
\\	保险。 
\\	保证。	ほけん		【名词】
\\	捕鯨	
\\	捕鲸。	ほげい		【名词】
\\	誇る	
\\	夸耀。 
\\	先进,杰出。	ほこる		【自他・五段/一类】
\\	綻びる	
\\	开绽,绽线。 
\\	(笑容等)张开。 
\\	(花朵等)开放。	ほころびる		【自动词・一段/二类】
\\	保護	
\\	【他动·三类】 
\\	保护(危険・破壊・困難などが及ばないように,かばい守ること)。	ほご		【名词】
\\	干し物	
\\	晾晒物;晒衣物。(日に干して乾かすこと。)	ほしもの		【名词】
\\	保守	
\\	保守。 
\\	保养。	ほしゅ		【名・他动词・サ变/三类】
\\	保障	
\\	保障。(障害のないように保つこと。)	ほしょう		【名・他动词・サ变/三类】
\\	補償	
\\	赔偿。 
\\	心理学上指心理机能的一部分过强时,与其相对的另一部分机能也相应增强,以维持心理整体的平衡。	ほしょう		【名・他动词・サ变/三类】
\\	補充	
\\	补充。	ほじゅう		【名・他动词・サ变/三类】
\\	補助	
\\	补助。(助ける。)	ほじょ		【名・他动词・サ变/三类】
\\	舗装	
\\	铺修,铺路。(道路面の耐久力を増すため、煉瓦・木塊・土石・コンクリート・アスファルトなどで路面を築造すること。)	ほそう		【名・他动词・サ变/三类】
\\	補足	
\\	补足,补充。	ほそく		【名・他动词・サ变/三类】
\\	発作	
\\	发作。	ほっさ		【自动词・サ变/三类】
\\	頬っぺた	
\\	颊,脸蛋。((ほお)の口頭語の表現。顔の両側の目の下から耳にかけての柔らかな部分。)	ほっぺた		【名词】
\\	辺り	
\\	附近;周围。	ほとり		【名词】
\\	殆ど	
\\	大体。(大方。) 
\\	几乎。(今少しで。)	ほとんど		【名词・副词】
\\	解く	
\\	解开(绳结等)。 
\\	解开疑问,解答。	ほどく		【他动词・五段/一类】
\\	施す	
\\	施舍,周济。 
\\	施,施行,为大众做事。 
\\	施加,(为期待某种效果而)加上。 
\\	遍布,遍及。	ほどこす		【他动词・五段/一类】
\\	保母	
\\	保育员,保姆。(保育所・養護施設(現在の児童養護施設)などの児童福祉施設で、児童の保育にあたる女子職員の俗称。正式名称は男女とも保育士。)	ほぼ		【名词】
\\	保養	
\\	消遣。 
\\	保养,休养,疗养。	ほよう		【名词】
\\	捕虜	
\\	俘虏。	ほりょ		【名词】
\\	滅びる	
\\	灭亡,灭绝。	ほろびる		【自动词・一段/二类】
\\	滅ぼす	
\\	使灭亡;毁灭。	ほろぼす		【他动词・五段/一类】
\\	本格	
\\	原则,正规,规范。(元来そうあるべき方式や手続きに従っていること。) 
\\	正式。(本来の格式を備えていること。本式。正式。)	ほんかく		【名词】
\\	本気	
\\	真实,认真,当真,真的。	ほんき		【名・形容动词/ナ形容词】
\\	本質	
\\	本质,骨子里。(物事の本来の性質や姿。)	ほんしつ		【名词】
\\	本体	
\\	真相,本来面目。 
\\	实体,本质。 
\\	主体。 
\\	主机,主要部分。 
\\	佛像本尊。	ほんたい		【名词】
\\	本音	
\\	真正的音色。 
\\	真话,真心话。	ほんね		【名词】
\\	本の	
\\	不过;仅仅;些许	ほんの		【连体词】
\\	本能	
\\	与生俱来的能力,本能。	ほんのう		【名词】
\\	本場	
\\	原产地,主要产地。 
\\	发源地,本地。正宗的。	ほんば		【名词】
\\	本文	
\\	本文,正文,原文。相对注释等而言的原来的文章。 
\\	本文,原文。书籍中,除插图、序言等之外的主要部分的文章。(書物のうちで、さし絵・序文などを除いた、主となっている文。ほんぶん。) 
\\	原文,引文,引语。古典等中可作依据的文句。(古典などにある、よりどころとすべき文句。)	ほんぶん		【名词】
\\	本名	
\\	本名,真名。	ほんみょう		【名词】
\\	防衛	
\\	防卫。	ぼうえい		【名・他动词・サ变/三类】
\\	防火	
\\	防火,预防火灾。	ぼうか		【名词】
\\	妨害	
\\	妨害,妨碍。	ぼうがい		【名・他动词・サ变/三类】
\\	紡績	
\\	纺纱;纺织。(短い繊維を平行に並べ、引き伸ばして撚りをかけ、一本の糸にすること。)	ぼうせき		【名词】
\\	呆然	
\\	发呆,发愣。	ぼうぜん		【形容动词/ナ形容词】
\\	膨脹	
\\	膨胀;增加;扩大;发展 
\\	膨脹;膨張 
\\	ぼうちょう		【自动词・サ变/三类】
\\	冒頭	
\\	谈话起首,文章开头。 
\\	引子,开场。	ぼうとう		【名词】
\\	暴風	
\\	暴风。	ぼうふう		【名词】
\\	暴力	
\\	暴力,武力。粗暴的力,蛮横的力。	ぼうりょく		【名词】
\\	募金	
\\	募捐(寄付金などをつのること)。	ぼきん		【名・自动词・サ变/三类】
\\	牧師	
\\	牧师。	ぼくし		【名词】
\\	母校	
\\	母校。	ぼこう		【名词】
\\	没収	
\\	没收,查抄,充公。(強制的に取り上げること。また、刑法上の付加刑の一。)	ぼっしゅう		【名・他动词・サ变/三类】
\\	坊ちゃん	
\\	令郎,您家的男孩儿,小弟弟,小朋友。 
\\	大少爷,公子哥儿;少爷作风的人。	ぼっちゃん		【名词】
\\	没落	
\\	没落,衰败,破产。	ぼつらく		【自动词・サ变/三类】
\\	枚	
\\	片,张,块,件,幅,扇,个。(薄く平たいもの。また、それを数える語。) 
\\	块,个。(大判金および銀貨を数える語。) 
\\	块。(田地の一区画。) 
\\	片,张,块,件,幅,扇,个。(薄く平たいもの。また、それを数える語。) 
\\	块,个。(大判金および銀貨を数える語。) 
\\	块。(田地の一区画。)	まい		【名词】
\\	埋蔵	
\\	埋藏,蕴藏。(主として地下にうずめかくすこと。うずもれていること。天然資源が地中にうまっていること。)	まいぞう		【名・他动词・サ变/三类】
\\	舞う	
\\	飘荡,飞舞。 
\\	舞蹈。	まう		【自动词・五段/一类】
\\	真上	
\\	正上方,头顶上。	まうえ		【名词】
\\	前売り	
\\	预售;预先售票。	まえうり		【名・他动词・サ变/三类】
\\	前置き	
\\	引子,开场白。(本題に入る前に述べること。また、その言葉。)	まえおき		【名词】
\\	前もって	
\\	预先,事先。(前から。あらかじめ。かねてから。)	まえもって		【副词】
\\	任す	
\\	委托,托付,听任,任凭,尽力,尽量。	まかす		【他动词・五段/一类】
\\	負かす	
\\	打败,击败,战胜;说服,驳倒。 
\\	还价,驳价,打价。	まかす		【他动词・五段/一类】
\\	賄う	
\\	供给(饭食)。 
\\	临时凑合,用有限的资金、人手来办事。	まかなう		【他动词・五段/一类】
\\	曲がる	
\\	弯,弯曲。 
\\	转弯。 
\\	歪曲,不合道理。 
\\	倾斜。 
\\	乖僻。	まがる		【自动词・五段/一类】
\\	巻	
\\	卷,书,书本,书册,卷帙。(書籍。) 
\\	手卷,卷子。(巻きもの。)	まき		【名词】
\\	紛らわしい	
\\	易弄错的,容易混淆的,不易分辨的。	まぎらわしい		【形容词/イ形容词】
\\	紛れる	
\\	混同,混淆,难以辨别。 
\\	掺混,混杂,混进。 
\\	忘怀,忘(忧)。	まぎれる		【自动词・一段/二类】
\\	膜	
\\	膜。(生物体内の諸器官をおおい,または境をなしている薄い層。) 
\\	皮,薄皮,薄膜。(物の表面をおおう薄い皮。)	まく		【名词】
\\	捲る	
\\	撩起,掀起。(おおっているものをはがす。) 
\\	翻开。(上に重なっているものをはがすように上げる。)	まくる		【他动词・五段/一类】
\\	真心	
\\	真心,诚心,诚意,丹心,精心。(真実の心。偽りや飾りのない心。)	まこころ		【名词】
\\	誠	
\\	真实。(事実のとおりであること。) 
\\	真诚。(偽り飾らない情。) 
\\	诚然。(じつに。)	まこと		【名词・副词】
\\	真に	
\\	真的,实在,诚然。	まことに		【副词】
\\	正しく	
\\	的确,确实,正是,没错儿。(ある事柄がまちがいなく成り立つさま。確かに。まさに。 
\\	まさしく		【副词】
\\	正に	
\\	方,恰,当今;方今,正当。 
\\	应当,应该。 
\\	快要,将,即将,将要。 
\\	真正;的确,确实;实在。	まさに		【副词】
\\	勝る	
\\	比……好,胜过,强过,凌驾。	まさる		【自动词・五段/一类】
\\	増し	
\\	增加,追加。	まし		【名词】
\\	真下	
\\	正下面,正下方。	ました		【名词】
\\	況して	
\\	何况,况且,更不能。(いわんや。そのうえ。なおさら。) 
\\	更,更加。
\\	いっそう。さらに。もっと。)	まして		【副词】
\\	交える	
\\	掺杂。 
\\	交叉。 
\\	交换。	まじえる		【他动词・一段/二类】
\\	交わる	
\\	混杂,搀杂。 
\\	交往,交际,来往,打交道『口』。 
\\	交叉。 
\\	性交,发生肉体关系;交尾。 
\\	(数学)相交。	まじわる		【自动词・五段/一类】
\\	麻酔	
\\	麻醉。	ますい		【名词】
\\	益々	
\\	越发,更加。(前よりも一層。いよいよ。)	ますます		【副词】
\\	不味い	
\\	不好吃;难吃。(味が悪いと感じられる状態だ。) 
\\	拙劣;笨拙;不高明,不好。(へただ。) 
\\	丑,难看。(美しくない。) 
\\	不妙,不合适,不恰当。(ぐあいが悪い。)	まずい		【形容词/イ形容词】
\\	股	
\\	股,大腿(足の、ひざから上の部分)。	また		【名词】
\\	跨がる	
\\	跨;乘;骑;横跨;跨越 
\\	跨がる;股がる 
\\	またがる		【自动词・五段/一类】
\\	跨ぐ	
\\	跨立,跨过,跨越。	またぐ		【他动词・五段/一类】
\\	瞬き	
\\	【自动·三类】 
\\	眨眼。(まぶたを閉じて,すぐあくこと。)	またたき		【名词】
\\	待ち合わせ	
\\	等候,约会,碰头。	まちあわせ		【名词】
\\	間違う	
\\	弄错,犯过失。 
\\	不正确,有误。	まちがう		【自他・五段/一类】
\\	待ち遠しい	
\\	急切等待的,盼望已久的,等待得令人焦急的。	まちどおしい		【形容词/イ形容词】
\\	待ち望む	
\\	盼望,期待,翘企,殷切希望。	まちのぞむ		【他动词・五段/一类】
\\	末期	
\\	临死,临终。(一生の終わりの時。)	まっき		【名词】
\\	真っ二つ	
\\	切割成两半。	まっぷたつ		【名词】
\\	纏まり	
\\	解决,谈妥;凑齐;概括;统一。(まとまること。まとまったぐあい。) 
\\	同:纏り	まとまり		【名词】
\\	纏め	
\\	总结,概括,归纳。〔意見などのまとめたもの。〕 
\\	汇集,汇总。〔資料などのまとめたもの。〕 
\\	解决,完结,有结果,达成协议。〔商談などのまとめたもの。〕 
\\	调停,调解,仲裁。〔いさかいなどのまとめたもの。〕	まとめ		【名词】
\\	免れる	
\\	免;避免	まぬかれる		【他动词・一段/二类】
\\	招き	
\\	招待,邀请,招聘。(招くこと。招待。招聘。) 
\\	招揽观众(顾客)的招牌(装饰物)。(「招き看板」に同じ。)	まねき		【名词】
\\	麻痺	
\\	麻痹,瘫痪。指功能迟钝或停止。 
\\	麻痹,麻木。	まひ		【自动词・サ变/三类】
\\	眩しい	
\\	晃眼,耀眼。 
\\	光辉耀眼,光彩夺目『成』,刺目。	まぶしい		【形容词/イ形容词】
\\	目蓋	
\\	眼睑;眼皮	まぶた		【名词】
\\	間々	
\\	有时,偶尔。	まま		【副词】
\\	間もなく	
\\	没过多少时间的样子。不久。不多时。不大工夫。不一会儿。一会儿。 
\\	同:間も無く	まもなく		【副词】
\\	眉	
\\	眉”以熟语的形式使用,而不单独使用。 
\\	眉”は熟語の形で使い、単独では用いない。 
\\	同:まゆずみ、まゆげ	まゆ		【名词】
\\	鞠	
\\	球.	まり		【名词】
\\	丸ごと	
\\	整个,完整。囫囵个儿。保持原来形状的全部,原样全部。(切り分けたり一部を除いたりしない、もとの形のまま全部。そっくり全部。まるのまま。)	まるごと		【副词】
\\	丸で	
\\	完全,简直,全然(同全く);宛如;仿佛	まるで		【副词】
\\	丸々	
\\	双圈儿。 
\\	连续两个符号。	まるまる		【名词】
\\	丸める	
\\	弄圆;团,揉成团;卷。 
\\	笼络,拉拢,操纵,耍弄。 
\\	剃头。 
\\	四舍五入。	まるめる		【他动词・一段/二类】
\\	満月	
\\	圆月,望月,满月。(月が地球から見て太陽と正反対の位置に来た時、月の全面がまんまるに輝いて見える現象。)	まんげつ		【名词】
\\	満場	
\\	全场,满堂。	まんじょう		【名词】
\\	真ん中	
\\	正中,中间,正当中。	まんなか		【名词】
\\	真ん前	
\\	正前方,正前面,(正)对面。(そのまっすぐ前。真正面。)	まんまえ		【名词】
\\	真ん丸い	
\\	圆滚的,圆圆的。	まんまるい		【形容词/イ形容词】
\\	見合い	
\\	相抵,平衡,相称。 
\\	相亲,相看。	みあい		【自动词・サ变/三类】
\\	見合わせる	
\\	互相看;互视。 
\\	比较着看;对照。 
\\	推迟;暂缓。	みあわせる		【他动词・一段/二类】
\\	見落とす	
\\	看漏,没看出来,忽略过去。	みおとす		【他动词・五段/一类】
\\	未開	
\\	文明未开化。 
\\	土地未开垦。 
\\	花朵没开。 
\\	未开拓。	みかい		【名词】
\\	味覚	
\\	味觉。	みかく		【名词】
\\	見掛ける	
\\	看到,看见。	みかける		【他动词・一段/二类】
\\	三日月	
\\	新月;月牙『口』,娥眉月。	みかずき		【名词】
\\	見方	
\\	看事物的立场。 
\\	见解,看法;想法。 
\\	看法。	みかた		【名词】
\\	見苦しい	
\\	难看的,寒碜的,丢脸的。	みぐるしい		【形容词/イ形容词】
\\	見込み	
\\	前景,估计,预计,未来的可能性。 
\\	巴望,盼望,期望。 
\\	碗或钵里面中央部分的总称。 
\\	外观。	みこみ		【名词】
\\	未婚	
\\	未婚,未婚者。	みこん		【名词】
\\	惨め	
\\	惨,凄惨,悲惨,惨痛。	みじめ		【形容动词/ナ形容词】
\\	未熟	
\\	未熟,生。 
\\	未成熟,不熟练。 
\\	发育不充分,发育不成熟。	みじゅく		【名・形容动词/ナ形容词】
\\	微塵	
\\	切碎,碎末。 
\\	一点,一丁点,极少量。 
\\	微小(物),极小(物)。 
\\	微尘。	みじん		【名词】
\\	見せびらかす	
\\	卖弄,夸示,显示,炫耀。	みせびらかす		【他动词・五段/一类】
\\	見せ物	
\\	杂耍。(珍しい物・曲芸・手品などを人に見せる興行。 
\\	叫人笑话,被人瞧热闹,洋相,当众出丑。(多くの人 におもしろがって見られること。また、そのもの。)	みせもの		【名词】
\\	満たす	
\\	满足。 
\\	充满;填满;装满。	みたす		【他动词・五段/一类】
\\	乱す	
\\	弄乱,扰乱。	みだす		【他动词・五段/一类】
\\	乱れる	
\\	乱,紊乱,杂乱。 
\\	(心)乱,困惑。 
\\	(心)乱,困惑。 
\\	骚动,动乱,不安定。 
\\	骚动,动乱,不安定。	みだれる		【自动词・一段/二类】
\\	未知	
\\	未知,还不知道的事。	みち		【名词】
\\	導く	
\\	引导,指导,领导。 
\\	引路,领道。 
\\	导致。	みちびく		【他动词・五段/一类】
\\	身近	
\\	切身;身边。	みぢか		【形容动词/ナ形容词】
\\	密集	
\\	密集,稠密。	みっしゅう		【自动词・サ变/三类】
\\	密接	
\\	密切。 
\\	密接,紧连。	みっせつ		【名・自动词・サ变/三类】
\\	見っともない	
\\	不像样的,不体面的,难看的。	みっともない		【形容词/イ形容词】
\\	蜜	
\\	蜜,蜂蜜。〔はちみつ〕 
\\	糖蜜。〔シロップ〕 
\\	甜水,甜汁。〔あまい液〕	みつ		【名词】
\\	密度	
\\	物质单位体积的质量。 
\\	密度。 
\\	文章,讲话等内容的充实程度。	みつど		【名词】
\\	見積り	
\\	报价,报价单,估价。(あらかじめ大体の計算をすること。)	みつもり		【名词】
\\	未定	
\\	未定,未决定。	みてい		【名词】
\\	見通し	
\\	眺望,远望。 
\\	看穿,看透。 
\\	看到底,看完。 
\\	预期,指望。	みとおし		【名词】
\\	源	
\\	水源,发源地。 
\\	水源,河源,河流的发源地。 
\\	起源。 
\\	起源,由来。	みなもと		【名词】
\\	身なり	
\\	服饰,装束,打扮。(衣服をつけた姿。よそおい。) 
\\	同:身形	みなり		【名词】
\\	峰	
\\	峰,山峰,山顶。〔山頂。〕 
\\	刀背。〔刃の背。〕 
\\	东西的凸起部分。〔高くなった所。〕	みね		【名词】
\\	見逃す	
\\	看漏,错过看的机会。 
\\	饶恕,宽恕。 
\\	放过,放跑,放走。	みのがす		【他动词・五段/一类】
\\	見晴らし	
\\	眺望,景致。开阔地环视到远处。	みはらし		【名词】
\\	身振り	
\\	姿态,动作。(意志や感情を表す(そうとする)体の動き。)	みぶり		【名词】
\\	脈	
\\	脉,血管。 
\\	脉搏。 
\\	脉。接续,衔接,连接。连贯而自成体系的东西。 
\\	未来的希望。对未来的期望。	みゃく		【名词】
\\	未練	
\\	依恋,恋恋不舍『成』。 
\\	不干脆,怯懦。	みれん		【名・形容动词/ナ形容词】
\\	見渡す	
\\	远望,具望,展望。	みわたす		【他动词・五段/一类】
\\	民宿	
\\	在民家投宿。(民家に宿泊すること。民泊。) 
\\	家庭旅店,接待旅客住宿的民家。(一般民家が許可を得て営む簡易な宿泊施設。)	みんしゅく		【名词】
\\	民族	
\\	民族。拥有共同的语言、文化,并拥有属于相同群体的意识的人组成的集团。	みんぞく		【名词】
\\	民俗	
\\	民俗,民间风俗。	みんぞく		【名词】
\\	六	
\\	六,六,六个。(むっつ。数の名。六番目)	む		【名词】
\\	無意味	
\\	无意义,没意思,没价值,无聊,无益。	むいみ		【名・形容动词/ナ形容词】
\\	向き	
\\	方向。(方向) 
\\	适合;合乎;对路。(適すること) 
\\	认真;郑重其事『成』(本気) 
\\	倾向;趋向。(傾向) 
\\	人;人们。(その方面の人)	むき		【名词】
\\	無口	
\\	不爱说话,沉默寡言。	むくち		【名・形容动词/ナ形容词】
\\	向け	
\\	向……。(……の方に。)	むけ		【接头/接尾】
\\	婿	
\\	婿,女婿,姑爷。〔娘の夫。〕 
\\	新郎。〔花婿。〕 
\\	同:壻	むこ		【名词】
\\	無効	
\\	无效,无效果;失效。	むこう		【名・形容动词/ナ形容词】
\\	無言	
\\	无言,不说话,沉默。	むごん		【名词】
\\	毟る	
\\	拔,薅,撕,揪。 
\\	撕下来,剔下来。 
\\	强夺,豪夺,威逼夺取。	むしる		【他动词・五段/一类】
\\	無邪気	
\\	天真无邪,单纯,天真,纯朴,老实,听话。 
\\	天真烂漫,幼稚,孩子气。	むじゃき		【名・形容动词/ナ形容词】
\\	結び	
\\	系,结,连结,打结。(結ぶこと。) 
\\	结束,末尾,终结。(締めくくること。終わり。) 
\\	饭团子。(にぎり飯。おむすび。) 
\\	〈語〉结尾,结语。(文法では、係の語に呼応して文を終わらせる語型。)	むすび		【名词】
\\	結び付き	
\\	联系,联合。(結びつくこと。二つのものの密接な関係。つながり。)	むすびつき		【名词】
\\	結び付く	
\\	有关系,有联系。 
\\	结成一体,有密切关系。	むすびつく		【自动词・五段/一类】
\\	結び付ける	
\\	结牢,结上,套结。系上,拴上,拴绑上。 
\\	结合,联结,联系,挂钩。	むすびつける		【他动词・一段/二类】
\\	無線	
\\	无线,不用电线。(電線のない。) 
\\	无线电。(無線電信、無線電話。)	むせん		【名词】
\\	無駄遣い	
\\	浪费,乱花钱,乱用。(必要もないことに金品を使うこと。浪費。)	むだづかい		【名词】
\\	無断	
\\	擅自,私自,事前未经允许,自做主张『成』,事先不打招呼。(ことわらない。)	むだん		【名词】
\\	無知	
\\	无知,无知识,愚笨,无智慧。 
\\	同:無智	むち		【名・形容动词/ナ形容词】
\\	無茶	
\\	毫无道理,岂有此理『成』。胡乱,胡来,乱来。 
\\	离谱,格外。(坏的)程度不寻常。	むちゃ		【名・形容动词/ナ形容词】
\\	無茶苦茶	
\\	毫无道理,岂有此理。指荒唐沒道理的事。 
\\	乱,混乱,乱七八糟,乱哄哄。 
\\	过分,格外,非常。	むちゃくちゃ		【名・形容动词/ナ形容词】
\\	空しい	
\\	空的,空洞的,没有内容的。 
\\	无事实的。 
\\	无意义的,徒劳的,变幻无常的。	むなしい		【形容词/イ形容词】
\\	無念	
\\	懊悔,遗憾。	むねん		【名・形容动词/ナ形容词】
\\	無能	
\\	无才能,无用。	むのう		【名・形容动词/ナ形容词】
\\	無闇に	
\\	蛮干,乱来,做事不考虑后果。 
\\	过度,过分,超过限度。	むやみに		【副词】
\\	無用	
\\	没有用处,无用,不起作用 。 
\\	没必要,无需。 
\\	没事,无事。 
\\	不许,不得,不准,禁止。	むよう		【名・形容动词/ナ形容词】
\\	群がる	
\\	聚,聚集。	むらがる		【自动词・五段/一类】
\\	無論	
\\	不用说,当然。	むろん		【副词】
\\	名産	
\\	名产。(その土地の名高い産物。)	めいさん		【名词】
\\	名称	
\\	名称。	めいしょう		【名词】
\\	命中	
\\	命中。	めいちゅう		【名・自动词・サ变/三类】
\\	名簿	
\\	名簿,名册。	めいぼ		【名词】
\\	明瞭	
\\	明了,明确,明显,清晰,清醒。清楚明白。	めいりょう		【名・形容动词/ナ形容词】
\\	明朗	
\\	明朗,开朗。 
\\	明朗。无不正当或隐瞒之事。	めいろう		【形容动词/ナ形容词】
\\	目方	
\\	重量,分量。	めかた		【名词】
\\	恵み	
\\	恩惠,恩泽,周济,施舍。	めぐみ		【名词】
\\	恵む	
\\	怜恤,照顾。 
\\	恩赐,施舍。	めぐむ		【他动词・五段/一类】
\\	目覚しい	
\\	惊人的,出奇的,特别突出的。出色得令人惊奇的,优秀得令人惊愕的。	めざましい		【形容词/イ形容词】
\\	目覚める	
\\	睡醒,觉醒。 
\\	自觉,醒悟,觉醒。 
\\	发生,发动。	めざめる		【自动词・一段/二类】
\\	召す	
\\	召见,召唤。贵人招呼人到自己身边来。 
\\	应召担任。应邀就任某官职。 
\\	吃,喝(的敬语)。 
\\	尊敬贵人或对方,描述其动作、状态等的词语。穿,乘,感冒,喜好等。 
\\	买(的敬语)。 
\\	入浴,洗澡,沐浴等的敬语。 
\\	〔由被召到神的身边转意〕
\\	死
\\	的敬语。	めす		【他动词・五段/一类】
\\	雌	
\\	雌,母。	めす		【名词】
\\	滅茶苦茶	
\\	不合理,没有道理。(すじ道のたたない。) 
\\	失度,胡闹,荒谬。(度はずれ。) 
\\	杂乱无章,乱七八糟。(乱雑なさま)	めちゃくちゃ		【形容动词/ナ形容词】
\\	目付き	
\\	眼神。	めつき		【名词】
\\	滅亡	
\\	灭亡。	めつぼう		【名・自动词・サ变/三类】
\\	目眩	
\\	头晕,眼花,眩晕。(目がくらみこと。目がくらんで倒れそうになること。) 
\\	同:眩暈	めまい		【自动词・サ变/三类】
\\	目盛	
\\	刻度;度数。(物差•秤•温度計•基盤などに長さ•量•度•区画を示すしるしをつけること。) 
\\	同:目盛り	めもり		【名词】
\\	面会	
\\	会见,见面,会面。	めんかい		【名・自动词・サ变/三类】
\\	免除	
\\	免除。 
\\	免除;民法上指凭债权人单方对债务人所作的意思表示而使债务消灭。	めんじょ		【名・他动词・サ变/三类】
\\	面する	
\\	面对,面向。 
\\	面临,直接面对。	めんする		【自动词・サ变/三类】
\\	面目	
\\	面目,脸面。(物事のありさま。) 
\\	名誉,威信,体面。(世間や周囲に対する体面・立場・名誉。)	めんぼく		【名词】
\\	設ける	
\\	预备,准备。 
\\	设立,制定,设置。	もうける		【他动词・一段/二类】
\\	申し入れる	
\\	提议,提出,要求。	もうしいれる		【他动词・一段/二类】
\\	申し込み	
\\	提议,提出要求(特定の契約を締結しようとする意思表示。相手方の承諾によって契約は成立する)。 
\\	申请;应征,报名(申し込むこと。また,その手続き)。 
\\	预约〔予約)。	もうしこみ		【名词】
\\	申し出る	
\\	提议,建议,提出,报名。	もうしでる		【他动词・一段/二类】
\\	申し分	
\\	分辨,辩解,辩解的理由(意见);不足,欠缺,缺点。	もうしぶん		【名词】
\\	盲点	
\\	盲点,暗点。 
\\	漏洞,空隙。	もうてん		【名词】
\\	猛烈	
\\	猛烈,凶猛,激烈。	もうれつ		【形容动词/ナ形容词】
\\	目録	
\\	目次。 
\\	目录。 
\\	清单,礼单。 
\\	证书。	もくろく		【名词】
\\	目論見	
\\	计划,策划,意图,企图。	もくろみ		【名词】
\\	模型	
\\	模型,仿照实物制作的物品。(実物にまねて作ったもの。)	もけい		【名词】
\\	模索	
\\	摸索。	もさく		【名・他动词・サ变/三类】
\\	若し	
\\	要,要是,如果,假如;假设,倘若。(定していない物事,事実に反する物事を仮定して次に述べる物事の条件とする意を表す。かりに。確実ではないが,十分にあり得る事態を想定する気持ちを表す。あるいはひょっとして。もしかすると。もしかしたら。)	もし		【副词】
\\	若しかしたら	
\\	或许;可能	もしかしたら		【副词】
\\	若しかすると	
\\	也许,或许,可能。(ひょっとすると。もしかしたら。)	もしかすると		【副词】
\\	若しくは	
\\	或,或者。	もしくは		【接续词】
\\	若しも	
\\	假使,万一。(「もし」を強めた語。)	もしも		【副词】
\\	持ち	
\\	持久性,耐久性。(品質等が変わらなず、長く使えること。) 
\\	负担。(負担すること。) 
\\	有很多。(程度が高いさま。) 
\\	适合……用。(所有すること。)	もち		【名词】
\\	持ち切り	
\\	热门话题。	もちきり		【名词】
\\	以て	
\\	主要以「…をもって」的形式,用于作格助动词使用时(格助詞的に用いられる場合。「をもって」の形で用いられることが多い): 
\\	以,用,拿。表示手段、方法、素材等,相当于「…で」「…によって」。(手段・方法・材料などを表す。…で。…でもって。…によって。) 
\\	因为,由于。表示原因、理由等,相当于「…により」。(原因・理由などを表す。…の理由で。…により。) 
\\	在,于。表示动作实施的时间,相当于「に」。(動作の行われる時を表す。に。) 
\\	以。表示动作、作用实施时的状态。(動作・作用の行われる際の状態を表す。) 
\\	仅起强调作用。加强语感。把、将。(単なる強めとして用いる。) 
\\	作接续助词使用时(接続助詞的に用いられる場合。): 
\\	并且,而且。接形容动词、断定助动词「だ」的连用形后,表示「…に加えて」等意。(形容動詞、断定の助動詞「だ」の連用形に付いて、下に続ける。「…の上に」「…に加えて」などの意を表す。かつ。) 
\\	接动词连用形下,表示「…しながら」之意。(動詞の連用形に付いて、下の動詞に続ける。「…しながら」の意を表す。)	もって		【接续词】
\\	専ら	
\\	专擅,独揽。 
\\	专(门);主要;净;专心致志『成』。	もっぱら		【副词】
\\	持て成す	
\\	对待,接待。(応対する。) 
\\	款待,招待;请吃饭。(歓待する。)	もてなす		【他动词・五段/一类】
\\	持てる	
\\	有人捧;能保持。	もてる		【自动词・一段/二类】
\\	基	
\\	〈化〉亚氨基(イミノ基)。	もとい		【名词】
\\	物置き	
\\	库房;堆房(同なや)	ものおき		【名词】
\\	物好き	
\\	好奇,好事(者)。有好奇心的人。	ものずき		【名・形容动词/ナ形容词】
\\	物足りない	
\\	不够满意的,感到欠缺的。	ものたりない		【形容词/イ形容词】
\\	最早	
\\	(事到如今)已经。(今となっては。もう。) 
\\	再快也…,已经…(早くも。すでに。)	もはや		【副词】
\\	模範	
\\	模范,榜样,标准,典型。	もはん		【名词】
\\	模倣	
\\	模仿;效仿。(自分で創り出すのではなく、すでにあるものをまねならうこと。他者と類似あるいは同一の行動をとること。幼児の学習過程、社会的流行、さらには高度の文化活動など、文化的・社会的に重要な意義をもつ。) 
\\	同:摸倣	もほう		【名・他动词・サ变/三类】
\\	揉める	
\\	发生争执,起纠纷。 
\\	忧虑不安;焦虑不安。	もめる		【自动词・一段/二类】
\\	腿	
\\	大腿,下至膝盖以上连接腰的部分。	もも		【名词】
\\	催す	
\\	举行,举办;主办。 
\\	感觉(要……)。 
\\	有征兆;萌起;感觉;预示。	もよおす		【自他・五段/一类】
\\	漏らす	
\\	漏,漏掉;露出;洒。遗尿;尿床,尿炕。 
\\	遗漏。 
\\	泄漏,走漏,透露。 
\\	流露,发出,发泄。	もらす		【他动词・五段/一类】
\\	漏る	
\\	漏,泄,漏出,透出。液体、光、空气等一点儿一点儿地向外泄出。 
\\	同:洩る	もる		【自动词・五段/一类】
\\	漏れる	
\\	漏,漏。 
\\	泄露,走漏。 
\\	遗漏,被排除。	もれる		【自动词・一段/二类】
\\	脆い	
\\	脆的,易碎的,易坏的。 
\\	脆弱的,不坚强的。	もろい		【形容词/イ形容词】
\\	問	
\\	问,问题,设问。 
\\	问,提问,质问。	もん		【名词】
\\	矢	
\\	箭。 
\\	楔子。	や		【名词】
\\	哉	
\\	由疑问或反问终助词「か」后接感叹或强调终助词「な」构成。表示语气轻微的自问或者疑问。(係助詞「か」の文末用法に詠嘆の終助詞「な」が付いてできたもの。中古以降の語。体言およびそれに準ずるもの,活用語の連体形に付く。文末にあって,詠嘆・感動の意を表す。…だなあ。…なあ。)	や		【接续词】
\\	野外	
\\	野外,郊外,原野。(野原・郊外。〕 
\\	户外,室外。(建物の外。屋外。〕	やがい		【名词】
\\	夜行	
\\	夜间行动。
\\	夜間に出歩くこと。また、夜、活動すること。やぎょう。) 
\\	夜间的火车,夜车。(「夜行列車」の略。)	やぎょう		【自动词・サ变/三类】
\\	役職	
\\	在政府机关或公司里具有掌管其组织的工作责任的职务。官职,职务。要职。(官庁や会社などで、その組織を運営する責任のある職務。重役・局長・部長・課長など)	やくしょく		【名词】
\\	役立つ	
\\	有用,有益,有帮助。	やくだつ		【自动词・五段/一类】
\\	役場	
\\	区(乡、村)公所,办事处。(町長・村長・地方公務員が事務を扱う所。また、その建物。)	やくば		【名词】
\\	夜具	
\\	寝具,卧具。	やぐ		【名词】
\\	優	
\\	优美。(やさしいこと。) 
\\	优待。充足。(てあついこと。) 
\\	优越。(上手なこと。) 
\\	优秀。(優れていること。) 
\\	优伶。(役者。) 
\\	悠闲自在。不果断。(のんびりしているさま。)	やさ		【名词】
\\	屋敷	
\\	房地,宅地,房屋和地产。 
\\	公馆,宅邸,住宅。	やしき		【名词】
\\	養う	
\\	养育;供养,扶养;喂养,饲养;疗养,修养;养;养成,培养;收养(子女)。	やしなう		【他动词・五段/一类】
\\	社	
\\	神社,神殿,庙,圣祠。(社殿。神社。)	やしろ		【名词】
\\	野心	
\\	野心,奢望,雄心。	やしん		【名词】
\\	易い	
\\	容易,简单。(容易だ。とかくそうなりがちだ。)	やすい		【形容词/イ形容词】
\\	安っぽい	
\\	不值钱的,看起来像便宜货的。 
\\	卑鄙的,让人瞧不起的。 
\\	轻率的,不足取的。	やすっぽい		【形容词/イ形容词】
\\	休める	
\\	使休息,使安心。 
\\	使停歇,暂停。	やすめる		【他动词・一段/二类】
\\	野生	
\\	野生(动植物)。动植物在山野里自然生长。(動植物が自然に山野で生育すること。) 
\\	野(人)。未经别人管教,完全自然地生长发育的人。(人が教育などによって整えられたりせず自然のままの状態で育つこと。) 
\\	野生。男子对自己的一中谦称。(一人称の人代名詞。男子が自分をへりくだっていう語。近世に多用される。)	やせい		【名词】
\\	矢鱈に	
\\	过分,非常;大量。 
\\	胡乱,随便,任意;不分好歹,没有差别。	やたらに		【副词】
\\	夜中	
\\	夜间。	やちゅう		【名词】
\\	奴	
\\	奴仆。(江戸時代の武家の奴僕。) 
\\	侠客。(任侠。) 
\\	崇尚侠义。(任侠を崇める。) 
\\	仆人。(家来。) 
\\	俘虏。(俘虜。) 
\\	「奴凧」的略。(「奴凧」の略。) 
\\	「奴豆腐」的略。(「奴豆腐」の略。)	やっこ		【名词】
\\	矢っ張り	
\\	仍然,依然,也,毕竟还是。(「やはり」の口語形。)	やっぱり		【副词】
\\	野党	
\\	在野党。	やとう		【名词】
\\	病	
\\	病。 
\\	毛病,坏习惯,恶癖。	やまい		【名词】
\\	闇	
\\	黑暗;黑夜。 
\\	(心中)无数,糊涂,辨别不清;不知所措『成』,不知如何是好。 
\\	黑暗。 
\\	黑市;黑市交易;黑货。	やみ		【名词】
\\	病む	
\\	得病,患病。 
\\	烦恼,痛心,忧伤,忧虑。	やむ		【自他・五段/一类】
\\	止むを得ない	
\\	不得已;无可奈何	やむをえない		【接续词】
\\	遣り通す	
\\	做完;完成 
\\	一度やると決めたことはやりとおす。「做到底」	やりとおす		【他动词・五段/一类】
\\	やり遂げる	
\\	完成,做完。	やりとげる		【他动词・一段/二类】
\\	遣る	
\\	派去,派遣,送去,打发去。(その場の勢い・なりゆきにまかせて他方へ行かせる。) 
\\	给。(身分が同等以下の者に与える。) 
\\	做,搞,干。(自ら物事を行う。する。)	やる		【他动词・五段/一类】
\\	和らげる	
\\	使缓和,使柔和。 
\\	使明白易懂。	やわらげる		【他动词・一段/二类】
\\	優位	
\\	优越地位,优势。	ゆうい		【名・形容动词/ナ形容词】
\\	憂鬱	
\\	忧郁,忧虑不安。	ゆううつ		【名・形容动词/ナ形容词】
\\	有益	
\\	有益,有意义,有好处。	ゆうえき		【名・形容动词/ナ形容词】
\\	優越	
\\	优越,优秀。	ゆうえつ		【自动词・サ变/三类】
\\	勇敢	
\\	勇敢。	ゆうかん		【名・形容动词/ナ形容词】
\\	有機	
\\	〈化〉有机(化学),有机(物)。((生命力を有する意)有機化学または有機化合物の略。)	ゆうき		【名词】
\\	夕暮れ	
\\	黄昏,傍晚。	ゆうぐれ		【名词】
\\	融資	
\\	融资,贷款。通融资金,贷款。	ゆうし		【名・自动词・サ变/三类】
\\	有する	
\\	有(持っている。所有する)。	ゆうする		【他动词・サ变/三类】
\\	融通	
\\	融通,顺利通过。 
\\	通融。 
\\	随机应变。	ゆうずう		【名・他动词・サ变/三类】
\\	優勢	
\\	优势。气势、形势比他人有利。	ゆうせい		【名・形容动词/ナ形容词】
\\	優先	
\\	优先。(他より先であること。)	ゆうせん		【自动词・サ变/三类】
\\	誘導	
\\	引导,诱导,导航。 
\\	电磁感应。 
\\	衍生。	ゆうどう		【名・他动词・サ变/三类】
\\	優美	
\\	优美,优雅而美丽。	ゆうび		【名・形容动词/ナ形容词】
\\	有望	
\\	(前途)有希望。	ゆうぼう		【形容动词/ナ形容词】
\\	遊牧	
\\	游牧。	ゆうぼく		【自动词・サ变/三类】
\\	夕焼け	
\\	晚霞,火烧云。	ゆうやけ		【名词】
\\	有力	
\\	有力,有势力,有权威;最有希望,可实现性胜过其他;有效力。	ゆうりょく		【名・形容动词/ナ形容词】
\\	幽霊	
\\	幽灵,死者的灵魂。 
\\	鬼魂,幽魂。 
\\	有名无实,虚设的。	ゆうれい		【名词】
\\	揺さぶる	
\\	摇动,摇晃。 
\\	使混乱,使…发生动摇,震惊,冲击。 
\\	摇撼(战术),棒球比赛中,投球手改变投球的路线或球种,迷惑击球手。	ゆさぶる		【他动词・五段/一类】
\\	茹でる	
\\	(用热水)煮,烫,焯。	ゆでる		【他动词・一段/二类】
\\	指差す	
\\	(用手指)指方向。	ゆびさす		【他动词・五段/一类】
\\	弓	
\\	弓。	ゆみ		【名词】
\\	揺らぐ	
\\	摇晃,晃动,摇曳。 
\\	摇摇欲坠,动摇。	ゆらぐ		【自动词・五段/一类】
\\	緩む	
\\	松,松弛。 
\\	松懈,松动。 
\\	缓和,放宽。 
\\	软化,变稀。 
\\	(行市)疲软。	ゆるむ		【自动词・五段/一类】
\\	緩める	
\\	放松。 
\\	松懈,懈弛,疏忽。 
\\	缓和,放慢。 
\\	放宽。	ゆるめる		【他动词・一段/二类】
\\	緩やか	
\\	缓慢,缓和。 
\\	宽松,宽大。 
\\	舒畅。	ゆるやか		【形容动词/ナ形容词】
\\	世	
\\	世间。 
\\	一生一世。 
\\	世,表示第几代。 
\\	时代。 
\\	地质年代的分期。 
\\	社会。 
\\	时代,时期。 
\\	(佛)世。	よ		【名词】
\\	好い	
\\	好的。
\\	「よい」に同じ。) 
\\	关系良好。(関係が良好である。特に、男女が相思相愛の仲である。) 
\\	过于。(十分過ぎる。その必要がない。)	よい		【形容词/イ形容词】
\\	要因	
\\	主要原因,主要因素。	よういん		【名词】
\\	溶液	
\\	溶液。	ようえき		【名词】
\\	用件	
\\	事,事情。	ようけん		【名词】
\\	養護	
\\	护养,护养;保健;特别加以保护地帮助成长。 
\\	养护,保育;根据儿童身心的成熟程度,对儿童加以保护,促进成长、发展。 
\\	养护,敬养;保护、照顾因身心健康状态及社会生活条件而状况不佳的人。	ようご		【名・他动词・サ变/三类】
\\	用紙	
\\	(特定用途的)纸张,专用纸。	ようし		【名词】
\\	様式	
\\	样式,方式。 
\\	一定的形式,格式。 
\\	式样,风格,格调。	ようしき		【名词】
\\	要する	
\\	必须,需要。(そのこと、あるいはそのものを必要とする。) 
\\	埋伏,伏击。(人を道に待ち受ける。待ち伏せする。) 
\\	要之,总而言之。(要約する。)	ようする		【他动词・サ变/三类】
\\	要請	
\\	请求,要求(必要なこととして,実現を願い求めること。乞い求めること)。	ようせい		【名・他动词・サ变/三类】
\\	養成	
\\	培养;造就;培训。(養育して成長させること。養い育てること。)	ようせい		【名・他动词・サ变/三类】
\\	様相	
\\	样子,情况,面貌。	ようそう		【名词】
\\	用品	
\\	用品,用具。(ある事に用いる品物。必要な品物。)	ようひん		【名词】
\\	洋風	
\\	西式,洋式。	ようふう		【名词】
\\	用法	
\\	用法。	ようほう		【名词】
\\	要望	
\\	要求,迫切期望。	ようぼう		【名・他动词・サ变/三类】
\\	余暇	
\\	余暇,业余时间。(余ったひまな時間。)	よか		【名词】
\\	予感	
\\	预感,预兆。(ある結果になることを前もってちょっとした事で感じること。)	よかん		【名・他动词・サ变/三类】
\\	余興	
\\	余兴。	よきょう		【名词】
\\	預金	
\\	存款。(金銭を銀行などの金融機関に預けること。また、その金銭。)	よきん		【名・他动词・サ变/三类】
\\	抑圧	
\\	压制,压迫;强行控制行动或自由等。 
\\	压抑;精神分析用语,无意识地抑制住不愉快的想法或感情,使之不上升为意识。 
\\	抑制;第二突变控制最初的突变产生的性状变化,使显现原先的性状。	よくあつ		【名・他动词・サ变/三类】
\\	浴室	
\\	浴室,在日本有和式及西式两种。	よくしつ		【名词】
\\	抑制	
\\	抑制,制止。(勢いや自由な活動などを、抑え止めること。(医)刺激によって興奮した神経細胞の活動が、他の神経細胞によって抑えられること。制止。)	よくせい		【名・他动词・サ变/三类】
\\	欲深い	
\\	贪婪,贪得无厌。	よくふかい		【形容词/イ形容词】
\\	欲望	
\\	欲望,欲求。想得到,希望得到满足的心情。(ほしいと思う心。不足を満たそうと強く求める気持ち。)	よくぼう		【名词】
\\	横綱	
\\	冠军;(冠军大力士的称号), 
\\	超群,出众,首屈一指(的人)。	よこづな		【名词】
\\	葦	
\\	芦苇。(イネ科の多年草。根茎は地中をはい、沼や川の岸に大群落をつくる。高さは2~3メートルになり、茎は堅く、円柱形で、細長い葉が互生する。) 
\\	同:芦	よし		【名词】
\\	予想	
\\	预料,预测,预计,预想。	よそう		【名・他动词・サ变/三类】
\\	余所見	
\\	【自サ】 
\\	往旁处看。(わき見。) 
\\	别人瞧着。(他人の見る目。) 
\\	假装没看见,熟视无睹;默许。(見て見ないふりをすること。)	よそみ		【名词】
\\	余地	
\\	空地。〔あいた土地。〕 
\\	余地,宽裕。剩余部分,尚有些富余。(残された部分。何かできるゆとり。)	よち		【名词】
\\	依って	
\\	基于,由于。 
\\	通过,利用,靠。 
\\	根据,基于,按照。 
\\	限于。 
\\	有关,在于,取决于。	よって		【自动词・五段/一类】
\\	余程	
\\	很,颇,相当,大量,在很大程度上。 
\\	很想……,差一点就……。	よっぽど		【副词】
\\	与党	
\\	执政的政党,执政党。 
\\	志同道合的伙伴。	よとう		【名词】
\\	呼び止める	
\\	招呼使站住,叫停。	よびとめる		【他动词・一段/二类】
\\	夜更かし	
\\	熬夜。	よふかし		【名・自动词・サ变/三类】
\\	夜更け	
\\	深夜。	よふけ		【名词】
\\	読み上げる	
\\	朗读、宣读。高声读。 
\\	读完,看完。	よみあげる		【他动词・五段/一类】
\\	寄り掛かる	
\\	依靠,靠。 
\\	依赖。	よりかかる		【自动词・五段/一类】
\\	宜しく	
\\	适当的。(ほどよく。適当に。) 
\\	应该,应当。(そうすることが必要であることを表す。) 
\\	(寒暄语)请问好,请致意,请关照。(「よろしくお願いします」「よろしくお伝えください」などの略。挨拶の語。) 
\\	附在体言后,表示十分得体之意。(いかにも
\\	のように。)	よろしく		【副词】
\\	弱まる	
\\	变弱,衰弱。(弱くなる)	よわまる		【自动词・五段/一类】
\\	弱める	
\\	使衰弱;削弱;减弱;减低。(弱くする。弱らせる。)	よわめる		【他动词・一段/二类】
\\	弱る	
\\	(因疾病、疲劳、年老)身体变弱。 
\\	为难。	よわる		【自动词・五段/一类】
\\	来場	
\\	到场,出席。	らいじょう		【自动词・サ变/三类】
\\	酪農	
\\	奶酪畜牧业。 
\\	乳畜业。主要饲养乳牛,生产牛乳及乳制品的产业。	らくのう		【名词】
\\	落下	
\\	下降,落下。	らっか		【名・自动词・サ变/三类】
\\	楽観	
\\	乐观。	らっかん		【名・他动词・サ变/三类】
\\	濫用	
\\	滥用;乱用。	らんよう		【他动词・サ变/三类】
\\	理屈	
\\	理论;道理;理;理由。 
\\	歪理;诡辩;借口。	りくつ		【名词】
\\	利子	
\\	利息,利钱。(金銭の貸借が行われた場合、その使用の対価として借り手が貸し手に支払う金銭。利息。)	りし		【名词】
\\	利潤	
\\	利润,红利。	りじゅん		【名词】
\\	理性	
\\	理性。不被情感左右。 
\\	理性。与感觉能力相对,之概念性思维的能力。 
\\	理性。康德的用语,广义上指先天能力的总称,狭义指区别与感性、悟性的、以理念统一悟性认识的能力。 
\\	理性。黑格尔的用语,区别于抽象概念的能力,即悟性的具体概念能力。 
\\	理性。与神的启示相对,指人的自然认知能力。 
\\	理性。支配宇宙和世界的原理。	りせい		【名词】
\\	利息	
\\	利息,利钱。	りそく		【名词】
\\	立体	
\\	立体。(いくつかの平面や曲面によって囲まれ,三次元の空間に広がりをもつ物体。また,その物体の占める空間を抽象化した,幾何学上の対象としての図形。)	りったい		【名词】
\\	立法	
\\	立法。	りっぽう		【名词】
\\	利点	
\\	优点,长处。	りてん		【名词】
\\	略語	
\\	略语,简语。缩写词。(もとの語形の一部分を省略して簡略にした語。「ロケーション」を「ロケ」、「短期大学」を「短大」、「西独逸」を「西独」などとする類。
\\	のように頭文字だけをとったものをもいう。)	りゃくご		【名词】
\\	略奪	
\\	掠夺,抢夺,抢劫,抢掠。 
\\	同:掠奪。	りゃくだつ		【名・他动词・サ变/三类】
\\	流	
\\	日本地名,位于福岛县东白川郡棚仓町。(地名、福島県東白川郡棚倉町にある。)	りゅう		【名词】
\\	流通	
\\	流通。	りゅうつう		【自动词・サ变/三类】
\\	了	
\\	明白,领悟。(わかること。さとること。) 
\\	终了,结束。(終わること。すむこと。)	りょう		【名词】
\\	料	
\\	费用,代价。价金,价款。(代金。代価。) 
\\	料,材料。派上用场的东西,供使用的材料。(用にあてるもの。使用に供する材料。)	りょう		【名词】
\\	領域	
\\	领域。 
\\	范围。	りょういき		【名词】
\\	了解	
\\	了解,理解,领会,明白。 
\\	谅解,体谅。	りょうかい		【名・他动词・サ变/三类】
\\	領海	
\\	领海。	りょうかい		【名词】
\\	両極	
\\	两极,阳极和阴极。 
\\	两极,南极和北极。 
\\	两极,两极端。	りょうきょく		【名词】
\\	良好	
\\	良好,优秀。	りょうこう		【名・形容动词/ナ形容词】
\\	良識	
\\	明智。	りょうしき		【名词】
\\	良質	
\\	质量良好,上等,优质。	りょうしつ		【形容动词/ナ形容词】
\\	了承	
\\	理解;同意,晓得;谅解;了察。(事情をくんで納得すること。承知すること。承諾。) 
\\	同:諒承、領承	りょうしょう		【名・他动词・サ变/三类】
\\	良心	
\\	良心。	りょうしん		【名词】
\\	領地	
\\	领土,领地。	りょうち		【名词】
\\	領土	
\\	领土。	りょうど		【名词】
\\	両立	
\\	两立;并存。	りょうりつ		【名・自动词・サ变/三类】
\\	旅客	
\\	旅客;乘客。(旅人の意の漢語的表現。)	りょかく		【名词】
\\	旅券	
\\	护照。(外国へ旅行する人の国籍・身分を証明する公文書。パスポート。)	りょけん		【名词】
\\	履歴	
\\	履历;经历。	りれき		【名词】
\\	理論	
\\	理论。原理,学说,见解。承认个别经验、事实中存在的规律性,并在此基础上统一形成的思想。亦指原理上系统化的假说。	りろん		【名词】
\\	輪	
\\	环。轮。(丸のこと。) 
\\	车。(車の輪。) 
\\	轮流。(順番に。) 
\\	外缘。(外部。) 
\\	轮。佛教指地水火风空五大元素。 
\\	朵。用于数花。(花の数える語。)	りん		【名词】
\\	林業	
\\	林业。(土地に林木を仕立てて育成し、これを経済的に利用することを目的とする生産業。)	りんぎょう		【名词】
\\	類似	
\\	类似,相似。	るいじ		【自动词・サ变/三类】
\\	類推	
\\	类推,类比推理。	るいすい		【名・他动词・サ变/三类】
\\	冷酷	
\\	冷酷无情,铁石心肠『成』。	れいこく		【名・形容动词/ナ形容词】
\\	冷蔵	
\\	【他动・三类】 
\\	冷藏,冷冻。(飲食物などの腐敗を防ぐため、セ氏零度に近い低温で貯蔵すること。)	れいぞう		【名词】
\\	冷淡	
\\	冷淡,冷漠,不热心,不关心。 
\\	不热情,不亲热,冷心肠。	れいたん		【名・形容动词/ナ形容词】
\\	恋愛	
\\	爱情,恋爱'。(特定の異性に特別の愛情を感じて恋い慕うこと。また、男女が互いにそのような感情をもつこと。)	れんあい		【自动词・サ变/三类】
\\	連休	
\\	连续的假日,连休。	れんきゅう		【名词】
\\	連日	
\\	连日,接连几天。	れんじつ		【名词】
\\	連中	
\\	伙伴,同伙;演艺团体的成员们。同れんじゅう。	れんじゅう		【名词】
\\	連帯	
\\	连带;互相结成一体,心情一致。 
\\	连带,两个以上的人共同负责。	れんたい		【名・自动词・サ变/三类】
\\	連邦	
\\	联邦。(二つ以上の、対内的には国家に準ずる行政機構を備えた自治体が 平等な関係で結合した国家。)	れんぽう		【名词】
\\	連盟	
\\	联盟;联合会。	れんめい		【名词】
\\	老衰	
\\	衰老。	ろうすい		【名・自动词・サ变/三类】
\\	朗読	
\\	朗读,朗诵。	ろうどく		【名・他动词・サ变/三类】
\\	浪費	
\\	浪费。乱用钱。	ろうひ		【名・他动词・サ变/三类】
\\	労力	
\\	费力,出力。 
\\	劳动力,劳力。	ろうりょく		【名词】
\\	碌に	
\\	很好地,正经地。	ろくに		【副词】
\\	露骨	
\\	露骨,直率,坦率,明显。十分显露地表示感情或真心。	ろこつ		【名・形容动词/ナ形容词】
\\	論議	
\\	议论,辩论,争论。(問答によって理非を明らかにすること。互いに意見を述べて論じ合うこと。) 
\\	能乐的一部分。(能の構成部分の一つ。シテと地謡または他の役とが問答形式で交互に謡う拍子に乗る部分。) 
\\	同:論義	ろんぎ		【名・他动词・サ变/三类】
\\	論理	
\\	逻辑,论理。(考えや議論などを進めていく筋道。ロジック。) 
\\	道理,规律。(事物の間にある法則的な連関。)	ろんり		【名词】
\\	枠	
\\	纺线,篗子,(缠线的)木制线框。 
\\	边,缘,框,轮廓。 
\\	框,框子。 
\\	限制,范围,框框。	わく		【名词】
\\	惑星	
\\	行星。 
\\	前途不可限量的人。	わくせい		【名词】
\\	技	
\\	技,技能;本领。(技芸。技術。) 
\\	招儿,招数。(柔剣道、相撲等で、相手に仕掛けて負かすための一定の型に基づく動作。)	わざ		【名词】
\\	態と	
\\	故意地(同こいに);特意地;有意	わざと		【副词】
\\	煩わしい	
\\	繁琐的,麻烦的,复杂的。 
\\	烦恼的,腻烦的,令人心烦的。	わずらわしい		【形容词/イ形容词】
\\	渡り鳥	
\\	到处奔走谋生的人。 
\\	候鸟。	わたりどり		【名词】
\\	詫び	
\\	赔不是,道歉,表示歉意。(謝罪。また、その言葉。詫び言。)	わび		【名词】
\\	和風	
\\	日式,日风。 
\\	微风。	わふう		【名词】
\\	和文	
\\	和文。用日语书写的文章。(日本語で書かれた文章。) 
\\	主要指用平假名书写的文章,见于平安时代的物语等。(和語。主に平仮名を用いて書かれた文。平安時代の物語などにみられる。)	わぶん		【名词】
\\	藁	
\\	稻草;麦杆。	わら		【名词】
\\	割合に	
\\	(表示与其基准相比不符)虽然……但是。(けれど。) 
\\	比较地,比预想地。(比較的。かなり。割に。)	わりあいに		【副词】
\\	割り当て	
\\	分配,分摊,分派,分担,分摊额。	わりあて		【名词】
\\	割り込む	
\\	挤进,加塞儿。 
\\	硬加入,插嘴。 
\\	跌破,股票下跌到某价格之下。	わりこむ		【他动词・五段/一类】
\\	割り算	
\\	〈数〉除法。(ある数を、もう一つの数で割ること。)	わりざん		【名词】
\\	悪者	
\\	坏人。(悪いことをする人。)	わるもの		【名词】
\\	ローマ字	
\\	罗马字母。(ラテン語を表記する文字。ギリシャ文字に由来する音素文字で,現在,各文字体系中最も普及し国際性をもつ。本来のアルファベットは二四文字であったが,現在ではこれに 
\\	と 
\\	を加えた二六文字が用いられる。ラテン文字。)	ローマじ		【名词】
\end{CJK}
\end{document}