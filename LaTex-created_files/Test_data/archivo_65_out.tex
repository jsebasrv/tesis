\documentclass[8pt]{extreport} 
\usepackage{hyperref}
\usepackage{CJKutf8}
\begin{document}
\begin{CJK}{UTF8}{min}
\\	大学	だいがく	
\\	大学に行ってもっと勉強したいです。	だいがくにいってもっとべんきょうしたいです	
\\	どこ	どこ	
\\	あの本をどこに置きましたか。	あのほんをどこにおきましたか	
\\	終わる	おわる	
\\	会議は4時に終わります。	かいぎはよじにおわります	
\\	学生	がくせい	
\\	彼は真面目な学生です。	かれはまじめながくせいです	
\\	知る	しる	
\\	誰もその話を知らない。	だれもそのはなしをしらない	
\\	払う	はらう	
\\	私が払いましょう。	わたしがはらいましょう	
\\	年	とし	
\\	新しい年が始まりました。	あたらしいとしがはじまりました	
\\	長い	ながい	
\\	彼女の髪はとても長い。	かのじょのかみはとてもながい	
\\	教える	おしえる	
\\	彼は数学を教えています。	かれはすうがくをおしえています	
\\	落ちる	おちる	
\\	猿も木から落ちる。	さるもきからおちる	
\\	古い	ふるい	
\\	私は古い車が好きです。	わたしはふるいくるまがすきです	
\\	なる	なる	
\\	彼は医者になりました。	かれはいしゃになりました	
\\	覚える	おぼえる	
\\	妹は平仮名を全部覚えました。	いもうとはひらがなをぜんぶおぼえました	
\\	死ぬ	しぬ	
\\	犬が病気で死にました。	いぬがびょうきでしにました	
\\	働く	はたらく	
\\	姉は銀行で働いています。	あねはぎんこうではたらいています	
\\	午前	ごぜん	
\\	午前9時のニュースです。	ごぜんくじのにゅーすです	
\\	それ	それ	
\\	それはとってもいい話だ。	それはとってもいいはなしだ	
\\	顔	かお	
\\	彼はタオルで顔を拭きました。	かれはたおるでかおをふきました	
\\	質問	しつもん	
\\	質問のある方はどうぞ。	しつもんのあるかたはどうぞ	
\\	見る	みる	
\\	私は絵を見るのが好きです。	わたしはえをみるのがすきです	
\\	少ない	すくない	
\\	今年は雨が少ないです。	ことしはあめがすくないです	
\\	名前	なまえ	
\\	あなたの名前を教えてください。	あなたのなまえをおしえてください	
\\	決める	きめる	
\\	帰国することに決めました。	きこくすることにきめました	
\\	これ	これ	
\\	これをください。	これをください	
\\	一つ	ひとつ	
\\	それを一つください。	それをひとつください	
\\	作る	つくる	
\\	今、朝ご飯を作っています。	いまあさごはんをつくっています	
\\	出来る	できる	
\\	彼女はイタリア語が出来ます。	かのじょはいたりあごができます	
\\	読む	よむ	
\\	彼女は雑誌を読んでいます。	かのじょはざっしをよんでいます	
\\	どちら	どちら	
\\	肉と魚とどちらが好きですか。	にくとさかなとどちらがすきですか	
\\	こう	こう	
\\	こう小さい字は読めない。	こうちいさいじはよめない	
\\	重い	おもい	
\\	この鞄は重いです。	このかばんはおもいです	
\\	誰	だれ	
\\	誰と会ってみたいですか。	だれとあってみたいですか	
\\	見せる	みせる	
\\	その写真を見せてください。	そのしゃしんをみせてください	
\\	好き	すき	
\\	私はワインが好きです。	わたしはわいんがすきです	
\\	皆	みんな	
\\	皆にお菓子を上げましょう。	みんなにおかしをあげましょう	
\\	建てる	たてる	
\\	私たちは来年、家を建てます。	わたしたちはらいねんいえをたてます	
\\	待つ	まつ	
\\	あなたが来るのを待っています。	あなたがくるのをまっています	
\\	体	からだ	
\\	私は体が丈夫だ。	わたしはからだがじょうぶだ	
\\	見つける	みつける	
\\	新しい仕事を見つけました。	あたらしいしごとをみつけました	
\\	どう	どう	
\\	あなたはどう思いますか。	あなたはどうおもいますか	
\\	少し	すこし	
\\	少し疲れました。	すこしつかれました	
\\	感じる	かんじる	
\\	膝に痛みを感じます。	ひざにいたみをかんじます	
\\	冬	ふゆ	
\\	カナダの冬はとても寒いです。	かなだのふゆはとてもさむいです	
\\	話す	はなす	
\\	母と電話で話しました。	ははとでんわではなしました	
\\	兄	あに	
\\	兄は水泳が得意です。	あにはすいえいがとくいです	
\\	送る	おくる	
\\	彼の家に荷物を送りました。	かれのいえににもつをおくりました	
\\	国	くに	
\\	私の国について少しお話しましょう。	わたしのくにについてすこしおはなししましょう	
\\	置く	おく	
\\	彼は鞄を椅子の上に置きました。	かれはかばんをいすのうえにおきました	
\\	姉	あね	
\\	姉は大学生です。	あねはだいがくせいです	
\\	聞く	きく	
\\	彼女はラジオを聞いています。	かのじょはらじおをきいています	
\\	かなり	かなり	
\\	彼はかなり英語が上手です。	かれはかなりえいごがじょうずです	
\\	乗る	のる	
\\	駅からはタクシーに乗ってください。	えきからはたくしーにのってください	
\\	思う	おもう	
\\	私もそう思います。	わたしもそうおもいます	
\\	出る	でる	
\\	今朝は早く家を出ました。	けさははやくいえをでました	
\\	勉強	べんきょう	
\\	私は日本語を勉強しています。	わたしはにほんごをべんきょうしています	
\\	行く	いく	
\\	日曜日は図書館に行きます。	にちようびはとしょかんにいきます	
\\	気持ち	きもち	
\\	彼の気持ちが分からない。	かれのきもちがわからない	
\\	朝	あさ	
\\	気持ちのいい朝です。	きもちのいいあさです	
\\	頭	あたま	
\\	今朝から頭が痛い。	けさからあたまがいたい	
\\	よく	よく	
\\	彼女はよく旅行に行きます。	かのじょはよくりょこうにいきます	
\\	帰る	かえる	
\\	家に帰ろう。	うちにかえろう	
\\	忘れる	わすれる	
\\	約束を忘れないでください。	やくそくをわすれないでください	
\\	仕事	しごと	
\\	3月は仕事が忙しい。	さんがつはしごとがいそがしい	
\\	多い	おおい	
\\	京都にはお寺が多い。	きょうとにはおてらがおおい	
\\	数	かず	
\\	グラスの数が足りません。	ぐらすのかずがたりません	
\\	車	くるま	
\\	弟が車を買った。	おとうとがくるまをかった	
\\	胸	むね	
\\	胸に少し痛みがあります。	むねにすこしいたみがあります	
\\	秋	あき	
\\	彼女は秋に結婚します。	かのじょはあきにけっこんします	
\\	いつも	いつも	
\\	彼女はいつも元気だ。	かのじょはいつもげんきだ	
\\	病気	びょうき	
\\	祖父が病気になった。	そふがびょうきになった	
\\	部屋	へや	
\\	私の部屋は2階にあります。	わたしのへやはにかいにあります	
\\	会う	あう	
\\	また会いましょう。	またあいましょう	
\\	狭い	せまい	
\\	私の部屋は狭いです。	わたしのへやはせまいです	
\\	変える	かえる	
\\	旅行の日程を変えました。	りょこうのにっていをかえました	
\\	言う	いう	
\\	上司が「一杯、飲もう。」と言った。	じょうしがいっぱいのもうといった	
\\	持つ	もつ	
\\	私は車を持っています。	わたしはくるまをもっています	
\\	春	はる	
\\	今年の春は暖かいね。	ことしのはるはあたたかいね	
\\	親	おや	
\\	親の愛は有り難い。	おやのあいはありがたい	
\\	上げる	あげる	
\\	この本、あなたに上げます。	このほんあなたにあげます	
\\	安い	やすい	
\\	この服はとても安かった。	このふくはとてもやすかった	
\\	一	いち	
\\	一から始めましょう。	いちからはじめましょう	
\\	未だ	まだ	
\\	宿題は未だ終わっていません。	しゅくだいはまだおわっていません	
\\	私	わたし	
\\	私が行きましょう。	わたしがいきましょう	
\\	時間	じかん	
\\	今は時間がありません。	いまはじかんがありません	
\\	彼女	かのじょ	
\\	彼女は
\\	です。	かのじょはおーえるです	
\\	もっと	もっと	
\\	もっと近くに来てください。	もっとちかくにきてください	
\\	足	あし	
\\	彼は足が長い。	かれはあしがながい	
\\	ここ	ここ	
\\	ここに本があります。	ここにほんがあります	
\\	難しい	むずかしい	
\\	この本は難しいですね。	このほんはむずかしいですね	
\\	軽い	かるい	
\\	この靴はとても軽い。	このくつはとてもかるい	
\\	学校	がっこう	
\\	学校は8時半に始まります。	がっこうははちじはんにはじまります	
\\	飛ぶ	とぶ	
\\	鳥が飛んでいます。	とりがとんでいます	
\\	そこ	そこ	
\\	そこに座ってください。	そこにすわってください	
\\	やる	やる	
\\	一緒に宿題をやろう。	いっしょにしゅくだいをやろう	
\\	夫	おっと	
\\	私の夫はサラリーマンです。	わたしのおっとはさらりーまんです	
\\	飛行機	ひこうき	
\\	息子は飛行機のおもちゃが好きです。	むすこはひこうきのおもちゃがすきです	
\\	別	べつ	
\\	別の本も見せてください。	べつのほんもみせてください	
\\	起きる	おきる	
\\	私は毎朝6時に起きます。	わたしはまいあさろくじにおきます	
\\	売る	うる	
\\	彼は家を売った。	かれはいえをうった	
\\	書く	かく	
\\	彼に手紙を書きました。	かれにてがみをかきました	
\\	戻る	もどる	
\\	今、会社に戻ります。	いまかいしゃにもどります	
\\	住む	すむ	
\\	彼は会社の近くに住んでいる。	かれはかいしゃのちかくにすんでいる	
\\	買う	かう	
\\	郵便局で切手を買いました。	ゆうびんきょくできってをかいました	
\\	取る	とる	
\\	テストでいい点を取った。	てすとでいいてんをとった	
\\	早い	はやい	
\\	まだ学校へ行くには早い時間です。	まだがっこうへいくにははやいじかんです	
\\	もう	もう	
\\	彼はもう帰りました。	かれはもうかえりました	
\\	言葉	ことば	
\\	この言葉の意味が分かりません。	このことばのいみがわかりません	
\\	どれ	どれ	
\\	この中でどれが好きですか。	このなかでどれがすきですか	
\\	貰う	もらう	
\\	彼女からプレゼントを貰らいました。	かのじょからぷれぜんとをもらいました	
\\	起こる	おこる	
\\	町で大事件が起こりました。	まちでだいじけんがおこりました	
\\	続く	つづく	
\\	工事は3月まで続きます。	こうじはさんがつまでつづきます	
\\	新しい	あたらしい	
\\	彼の車は新しい。	かれのくるまはあたらしい	
\\	そう	そう	
\\	私もそう思います。	わたしもそうおもいます	
\\	走る	はしる	
\\	彼は毎晩3キロ走っています。	かれはまいばんさんきろはしっています	
\\	始める	はじめる	
\\	テストを始めてください。	てすとをはじめてください	
\\	今	いま	
\\	彼は今、勉強しています。	かれはいまべんきょうしています	
\\	勝つ	かつ	
\\	今日はヤンキースが勝った。	きょうはやんきーすがかった	
\\	先生	せんせい	
\\	私は日本語の先生になりたいです。	わたしはにほんごのせんせいになりたいです	
\\	短い	みじかい	
\\	彼は足が短い。	かれはあしがみじかい	
\\	立つ	たつ	
\\	彼はステージに立った。	かれはすてーじにたった	
\\	今年	ことし	
\\	今年はイタリアに旅行したい。	ことしはいたりあにりょこうしたい	
\\	後	あと	
\\	仕事の後、映画を見た。	しごとのあとえいがをみた	
\\	使う	つかう	
\\	このパソコンを使ってください。	このぱそこんをつかってください	
\\	娘	むすめ	
\\	私の娘はアメリカにいます。	わたしのむすめはあめりかにいます	
\\	入れる	いれる	
\\	彼は鞄に手帳を入れた。	かれはかばんにてちょうをいれた	
\\	呼ぶ	よぶ	
\\	ウェイターを呼びましょう。	うぇいたーをよびましょう	
\\	考える	かんがえる	
\\	よく考えてください。	よくかんがえてください	
\\	友達	ともだち	
\\	私には友達がたくさんいます。	わたしにはともだちがたくさんいます	
\\	色	いろ	
\\	素敵な色のセーターですね。	すてきないろのせーたーですね	
\\	息子	むすこ	
\\	うちの息子は大学1年生です。	うちのむすこはだいがくいちねんせいです	
\\	同じ	おなじ	
\\	彼の日本語のレベルは私と同じ位だ。	かれのにほんごのれべるはわたしとおなじくらいだ	
\\	午後	ごご	
\\	明日の午後、お客様が来る。	あすのごごおきゃくさまがくる	
\\	円	えん	
\\	そこに大きな円を描いて。	そこにおおきなえんをかいて	
\\	いつ	いつ	
\\	彼女はいつ来ますか。	かのじょはいつきますか	
\\	次	つぎ	
\\	次はいつ会いましょうか。	つぎはいつあいましょうか	
\\	酒	さけ	
\\	彼女は酒に強い。	かのじょはさけにつよい	
\\	広い	ひろい	
\\	彼の家はとても広い。	かれのいえはとてもひろい	
\\	若い	わかい	
\\	彼はまだ若いです。	かれはまだわかいです	
\\	呉れる	くれる	
\\	友達が誕生日プレゼントを呉れた。	ともだちがたんじょうびぷれぜんとをくれた	
\\	子供	こども	
\\	電車で子供が騒いでいた。	でんしゃでこどもがさわいでいた	
\\	美しい	うつくしい	
\\	この絵は美しいです。	このえはうつくしいです	
\\	白い	しろい	
\\	彼は歯が白い。	かれははがしろい	
\\	分かる	わかる	
\\	質問の意味は分かりましたか。	しつもんのいみはわかりましたか	
\\	気	き	
\\	彼は意外に気が小さい。	かれはいがいにきがちいさい	
\\	とても	とても	
\\	この本はとても面白い。	このほんはとてもおもしろい	
\\	起こす	おこす	
\\	明日、6時に起こしてください。	あすろくじにおこしてください	
\\	する	する	
\\	友達と一緒に宿題をした。	ともだちといっしょにしゅくだいをした	
\\	本	ほん	
\\	本を1冊買いました。	ほんをいっさつかいました	
\\	食べる	たべる	
\\	昨日タイカレーを食べました。	きのうたいかれーをたべました	
\\	楽しむ	たのしむ	
\\	今日は一人の時間を楽しみたい。	きょうはひとりのじかんをたのしみたい	
\\	家	うち	
\\	家に遊びに来てください。	うちにあそびにきてください	
\\	結婚	けっこん	
\\	彼女は来月結婚します。	かのじょはらいげつけっこんします	
\\	生きる	いきる	
\\	皆一生懸命生きている。	みんないっしょうけんめいいきている	
\\	低い	ひくい	
\\	彼は背が低い。	かれはせがひくい	
\\	所	ところ	
\\	私は友達の所に泊まった。	わたしはともだちのところにとまった	
\\	直ぐ	すぐ	
\\	直ぐ行きます。	すぐいきます	
\\	歩く	あるく	
\\	駅まで歩きましょう。	えきまであるきましょう	
\\	近く	ちかく	
\\	駅の近くで食事をした。	えきのちかくでしょくじをした	
\\	近い	ちかい	
\\	駅はここから近いです。	えきはここからちかいです	
\\	高い	たかい	
\\	これがこの町で一番高いビルです。	これがこのまちでいちばんたかいびるです	
\\	日記	にっき	
\\	私は毎日、日記を付けています。	わたしはまいにちにっきをつけています	
\\	漢字	かんじ	
\\	漢字は中国から来ました。	かんじはちゅうごくからきました	
\\	耳	みみ	
\\	彼女は耳がよく聞こえません。	かのじょはみみがよくきこえません	
\\	腕	うで	
\\	彼の腕は太い。	かれのうではふとい	
\\	去年	きょねん	
\\	私は去年フランスへ行った。	わたしはきょねんふらんすへいった	
\\	一緒に	いっしょに	
\\	一緒に帰りましょう。	いっしょにかえりましょう	
\\	上げる	あげる	
\\	彼は荷物を網棚に上げた。	かれはにもつをあみだなにあげた	
\\	肩	かた	
\\	肩が凝りました。	かたがこりました	
\\	元気	げんき	
\\	お蔭様で元気です。	おかげさまでげんきです	
\\	週	しゅう	
\\	その次の週は空いてますか。	そのつぎのしゅうはあいてますか	
\\	飲む	のむ	
\\	友達とお酒を飲んでいます。	ともだちとおさけをのんでいます	
\\	遠い	とおい	
\\	家から学校までは遠いです。	いえからがっこうまではとおいです	
\\	電車	でんしゃ	
\\	私は電車で通学しています。	わたしはでんしゃでつうがくしています	
\\	自転車	じてんしゃ	
\\	毎日、駅まで自転車で行きます。	まいにちえきまでじてんしゃでいきます	
\\	笑う	わらう	
\\	赤ん坊が笑っています。	あかんぼうがわらっています	
\\	赤い	あかい	
\\	赤い薔薇を買いました。	あかいばらをかいました	
\\	遅い	おそい	
\\	前の車はとても遅い。	まえのくるまはとてもおそい	
\\	遊ぶ	あそぶ	
\\	子供たちが公園で遊んでいる。	こどもたちがこうえんであそんでいる	
\\	右	みぎ	
\\	右のポケットにハンカチが入っています。	みぎのぽけっとにはんかちがはいっています	
\\	一番	いちばん	
\\	彼はクラスで一番背が高い。	かれはくらすでいちばんせがたかい	
\\	首	くび	
\\	麒麟の首は長い。	きりんのくびはながい	
\\	弱い	よわい	
\\	その子は体が少し弱い。	そのこはからだがすこしよわい	
\\	歌う	うたう	
\\	私たちは大きな声で歌いました。	わたしたちはおおきなこえでうたいました	
\\	授業	じゅぎょう	
\\	今日は日本語の授業があります。	きょうはにほんごのじゅぎょうがあります	
\\	着る	きる	
\\	今日はスーツを着ています。	きょうはすーつをきています	
\\	図書館	としょかん	
\\	図書館で料理の本を借りた。	としょかんでりょうりのほんをかりた	
\\	緑	みどり	
\\	この町には緑がたくさんあります。	このまちにはみどりがたくさんあります	
\\	速い	はやい	
\\	彼は走るのが速い。	かれははしるのがはやい	
\\	今月	こんげつ	
\\	今月はとても忙しい。	こんげつはとてもいそがしい	
\\	寝る	ねる	
\\	もう寝よう。	もうねよう	
\\	今日	きょう	
\\	今日は仕事がありません。	きょうはしごとがありません	
\\	紙	かみ	
\\	紙と鉛筆はありますか。	かみとえんぴつはありますか	
\\	浴びる	あびる	
\\	私は朝、シャワーを浴びます。	わたしはあさしゃわーをあびます	
\\	消す	けす	
\\	昼間は電気を消してください。	ひるまはでんきをけしてください	
\\	負ける	まける	
\\	私たちのチームはその試合で負けた。	わたしたちのちーむはそのしあいでまけた	
\\	ドア	どあ	
\\	ドアを開けてください。	どあをあけてください	
\\	全部	ぜんぶ	
\\	それ、全部ください。	それぜんぶください	
\\	引く	ひく	
\\	このドアは引いてください。	このどあはひいてください	
\\	探す	さがす	
\\	彼は郵便局を探していました。	かれはゆうびんきょくをさがしていました	
\\	座る	すわる	
\\	私は窓側の席に座った。	わたしはまどがわのせきにすわった	
\\	彼	かれ	
\\	彼は私の上司です。	かれはわたしのじょうしです	
\\	晴れる	はれる	
\\	明日は晴れるといいですね。	あしたははれるといいですね	
\\	二日	ふつか	
\\	私は二日待った。	わたしはふつかまった	
\\	夏	なつ	
\\	私は夏が大好き。	わたしはなつがだいすき	
\\	口	くち	
\\	口を大きく開けてください。	くちをおおきくあけてください	
\\	掛ける	かける	
\\	夫の服をハンガーに掛けた。	おっとのふくをはんがーにかけた	
\\	線	せん	
\\	赤い線を2本引いて下さい。	あかいせんをにほんひいてください	
\\	強い	つよい	
\\	今日は風が強い。	きょうはかぜがつよい	
\\	曇る	くもる	
\\	明日は昼頃から曇るでしょう。	あすはひるごろからくもるでしょう	
\\	不味い	まずい	
\\	ここの料理は不味い。	ここのりょうりはまずい	
\\	神	かみ	
\\	彼は神を信じている。	かれはかみをしんじている	
\\	母	はは	
\\	昨日、母と話をしました。	きのうははとはなしをしました	
\\	要る	いる	
\\	予約は要りません。	よやくはいりません	
\\	父さん	とうさん	
\\	父さん、ありがとう。	とうさんありがとう	
\\	六日	むいか	
\\	六日前に日本に帰ってきました。	むいかまえににほんにかえってきました	
\\	昨日	きのう	
\\	昨日、友達に会った。	きのうともだちにあった	
\\	万	まん	
\\	この靴は1万円です。	このくつはいちまんえんです	
\\	易しい	やさしい	
\\	この問題はかなり易しいです。	このもんだいはかなりやさしいです	
\\	九	きゅう	
\\	野球は九人で1チームです。	やきゅうはきゅうにんでひとちーむです	
\\	風邪	かぜ	
\\	私は風邪を引きました。	わたしはかぜをひきました	
\\	小さい	ちいさい	
\\	小さい花が咲いています。	ちいさいはながさいています	
\\	大人	おとな	
\\	お酒は大人になってから。	おさけはおとなになってから	
\\	四	よん	
\\	ハワイは四回目です。	はわいはよんかいめです	
\\	千	せん	
\\	千円貸してください。	せんえんかしてください	
\\	今週	こんしゅう	
\\	今週は日本語のテストがあります。	こんしゅうはにほんごのてすとがあります	
\\	お手洗い	おてあらい	
\\	お手洗いはどこですか。	おてあらいはどこですか	
\\	左	ひだり	
\\	そこを左に曲がってください。	そこをひだりにまがってください	
\\	髪の毛	かみのけ	
\\	髪の毛が伸びたね。	かみのけがのびたね	
\\	見つかる	みつかる	
\\	眼鏡が見つかりません。	めがねがみつかりません	
\\	八つ	やっつ	
\\	あの家には時計が八つあります。	あのいえにはとけいがやっつあります	
\\	お腹	おなか	
\\	お腹が空きました。	おなかがすきました	
\\	七日	なのか	
\\	先月の七日に孫が生まれました。	せんげつのなのかにまごがうまれました	
\\	手	て	
\\	分かった人は手を上げてください。	わかったひとはてをあげてください	
\\	閉じる	とじる	
\\	教科書を閉じてください。	きょうかしょをとじてください	
\\	十	じゅう	
\\	その子は指で十数えました。	そのこはゆびでじゅうかぞえました	
\\	月曜日	げつようび	
\\	月曜日に会いましょう。	げつようびにあいましょう	
\\	男	おとこ	
\\	男の人が私たちに話しかけた。	おとこのひとがわたしたちにはなしかけた	
\\	答える	こたえる	
\\	私の質問に答えてください。	わたしのしつもんにこたえてください	
\\	大きい	おおきい	
\\	あの大きい建物は何ですか。	あのおおきいたてものはなんですか	
\\	ため	ため	
\\	これは医者のためのサイトです。	これはいしゃのためのさいとです	
\\	下	した	
\\	財布は机の下にあった。	さいふはつくえのしたにあった	
\\	始まる	はじまる	
\\	新しい仕事が始まりました。	あたらしいしごとがはじまりました	
\\	一日	ついたち	
\\	来月の一日は空いていますか。	らいげつのついたちはあいていますか	
\\	止める	やめる	
\\	話すのを止めてください。	はなすのをやめてください	
\\	有る	ある	
\\	私の机の上に書類がたくさん有ります。	わたしのつくえのうえにしょるいがたくさんあります	
\\	傍	そば	
\\	傍にいて下さい。	そばにいてください	
\\	ゲーム	げーむ	
\\	私たちはビデオゲームをした。	わたしたちはびでおげーむをした	
\\	上手	じょうず	
\\	妹は歌が上手です。	いもうとはうたがじょうずです	
\\	入る	はいる	
\\	寒いので中に入ってください。	さむいのでなかにはいってください	
\\	駄目	だめ	
\\	彼は駄目な男だ。	かれはだめなおとこだ	
\\	三	さん	
\\	彼女は三人の子供の母親だ。	かのじょはさんにんのこどものははおやだ	
\\	辺	へん	
\\	彼はこの辺に住んでいます。	かれはこのへんにすんでいます	
\\	お兄さん	おにいさん	
\\	あなたのお兄さんは何歳?	あなたのおにいさんはなんさい	
\\	他	ほか	
\\	他に方法がありません。	ほかにほうほうがありません	
\\	駅	えき	
\\	駅はどこですか。	えきはどこですか	
\\	テーブル	てーぶる	
\\	新しいテーブルを買いました。	あたらしいてーぶるをかいました	
\\	嫌い	きらい	
\\	私は煙草が嫌いです。	わたしはたばこがきらいです	
\\	あそこ	あそこ	
\\	あそこにバス停があります。	あそこにばすていがあります	
\\	出す	だす	
\\	彼は鞄から教科書を出した。	かれはかばんからきょうかしょをだした	
\\	太い	ふとい	
\\	彼女は足が太い。	かのじょはあしがふとい	
\\	先	さき	
\\	お先にどうぞ。	おさきにどうぞ	
\\	よう	よう	
\\	彼女は眠いようです。	かのじょはねむいようです	
\\	一杯	いっぱい	
\\	プールは人で一杯です。	ぷーるはひとでいっぱいです	
\\	終わり	おわり	
\\	夏休みももう終わりだ。	なつやすみももうおわりだ	
\\	二	に	
\\	その人には二回会った。	そのひとにはにかいあった	
\\	八日	ようか	
\\	八日からイギリスに行きます。	ようかからいぎりすにいきます	
\\	痛い	いたい	
\\	今日は頭が痛いです。	きょうはあたまがいたいです	
\\	寒い	さむい	
\\	この部屋は寒いです。	このへやはさむいです	
\\	人	ひと	
\\	彼は優しい人です。	かれはやさしいひとです	
\\	平仮名	ひらがな	
\\	私は平仮名を全部読めます。	わたしはひらがなをぜんぶよめます	
\\	閉める	しめる	
\\	ちゃんとドアを閉めてよ。	ちゃんとどあをしめてよ	
\\	近く	ちかく	
\\	私の家は駅の近くです。	わたしのいえはえきのちかくです	
\\	お姉さん	おねえさん	
\\	昨日、あなたのお姉さんに会ったよ。	きのうあなたのおねえさんにあったよ	
\\	金曜日	きんようび	
\\	金曜日の夜は友達と出掛けます。	きんようびのよるはともだちとでかけます	
\\	僕	ぼく	
\\	僕は学生です。	ぼくはがくせいです	
\\	違う	ちがう	
\\	答えが違います。	こたえがちがいます	
\\	四日	よっか	
\\	新学期は来月の四日からです。	しんがっきはらいげつのよっかからです	
\\	お願い	おねがい	
\\	お願いがあります。	おねがいがあります	
\\	ご飯	ごはん	
\\	私はパンよりご飯が好きだ。	わたしはぱんよりごはんがすきだ	
\\	閉まる	しまる	
\\	お店はもう閉まっていました。	おみせはもうしまっていました	
\\	方	ほう	
\\	彼は私の方を見ました。	かれはわたしのほうをみました	
\\	妹	いもうと	
\\	私の妹は小学生です。	わたしのいもうとはしょうがくせいです	
\\	曜日	ようび	
\\	曜日を間違えました。	ようびをまちがえました	
\\	乗せる	のせる	
\\	皿にケーキを乗せました。	さらにけーきをのせました	
\\	今晩	こんばん	
\\	今晩のパーティーは何時からですか。	こんばんのぱーてぃーはなんじからですか	
\\	悪い	わるい	
\\	煙草は体に悪い。	たばこはからだにわるい	
\\	火曜日	かようび	
\\	火曜日に会議があります。	かようびにかいぎがあります	
\\	大丈夫	だいじょうぶ	
\\	大丈夫ですか。	だいじょうぶですか	
\\	三つ	みっつ	
\\	コップを三つ買いました。	こっぷをみっつかいました	
\\	大学生	だいがくせい	
\\	姉は大学生です。	あねはだいがくせいです	
\\	五日	いつか	
\\	五月五日は祝日です	ごがついつかはしゅくじつです	
\\	甘い	あまい	
\\	この苺は甘い。	このいちごはあまい	
\\	高校生	こうこうせい	
\\	私の弟は高校生です。	わたしのおとうとはこうこうせいです	
\\	円	えん	
\\	カレーライスは700円です。	かれーらいすはななひゃくえんです	
\\	上	うえ	
\\	上を向いて。	うえをむいて	
\\	来年	らいねん	
\\	来年一緒に旅行しましょう。	らいねんいっしょにりょこうしましょう	
\\	片仮名	かたかな	
\\	お名前を片仮名で書いてください。	おなまえをかたかなでかいてください	
\\	汚い	きたない	
\\	彼の部屋はとても汚い。	かれのへやはとてもきたない	
\\	木曜日	もくようび	
\\	木曜日は仕事が休みです。	もくようびはしごとがやすみです	
\\	愛する	あいする	
\\	私は家族を愛しています。	わたしはかぞくをあいしています	
\\	生る	なる	
\\	柿の木に実がたくさん生っています。	かきのきにみがたくさんなっています	
\\	本当に	ほんとうに	
\\	あなたが本当に好きです。	あなたがほんとうにすきです	
\\	服	ふく	
\\	昨日、新しい服を買った。	きのうあたらしいふくをかった	
\\	百	ひゃく	
\\	私の祖母は百才です。	わたしのそぼはひゃくさいです	
\\	細い	ほそい	
\\	彼女は指が細いですね。	かのじょはゆびがほそいですね	
\\	空く	すく	
\\	レストランは空いていました。	れすとらんはすいていました	
\\	何	なに	
\\	夕食には何を食べたいですか。	ゆうしょくにはなにをたべたいですか	
\\	六つ	むっつ	
\\	息子は六つになりました。	むすこはむっつになりました	
\\	お母さん	おかあさん	
\\	お母さんによろしくお伝えください。	おかあさんによろしくおつたえください	
\\	道	みち	
\\	この道を真っ直ぐ行くと駅です。	このみちをまっすぐいくとえきです	
\\	悲しい	かなしい	
\\	その映画はとても悲しかった。	そのえいがはとてもかなしかった	
\\	締める	しめる	
\\	彼はシートベルトを締めた。	かれはしーとべるとをしめた	
\\	日	にち	
\\	私たちは先月11日に結婚しました。	わたしたちはせんげつじゅういちにちにけっこんしました	
\\	酸っぱい	すっぱい	
\\	この葡萄は酸っぱいです。	このぶどうはすっぱいです	
\\	三日	みっか	
\\	手紙が届くのに三日掛かりました。	てがみがとどくのにみっかかかりました	
\\	妻	つま	
\\	今日は妻の誕生日だ。	きょうはつまのたんじょうびだ	
\\	時	とき	
\\	時の経つのは早い。	ときのたつのははやい	
\\	嬉しい	うれしい	
\\	彼に会えて嬉しかった。	かれにあえてうれしかった	
\\	ご主人	ごしゅじん	
\\	ご主人はお元気ですか。	ごしゅじんはおげんきですか	
\\	冷たい	つめたい	
\\	冷たい飲み物をください。	つめたいのみものをください	
\\	綺麗	きれい	
\\	彼女はとても綺麗だ。	かのじょはとてもきれいだ	
\\	時計	とけい	
\\	時計を見たらちょうど3時だった。	とけいをみたらちょうどさんじだった	
\\	決まる	きまる	
\\	旅行の日程が決まりました。	りょこうのにっていがきまりました	
\\	良い	いい	
\\	彼女は良い友達です。	かのじょはいいともだちです	
\\	こっち	こっち	
\\	こっちに来て下さい。	こっちにきてください	
\\	曲がる	まがる	
\\	そこを左に曲がってください。	そこをひだりにまがってください	
\\	大好き	だいすき	
\\	私は犬が大好きだ。	わたしはいぬがだいすきだ	
\\	楽しい	たのしい	
\\	彼はとても楽しい人です。	かれはとてもたのしいひとです	
\\	欲しい	ほしい	
\\	僕は新しい靴が欲しいです。	ぼくはあたらしいくつがほしいです	
\\	土曜日	どようび	
\\	土曜日の夜はクラブに行きます。	どようびのよるはくらぶにいきます	
\\	開ける	あける	
\\	窓を開けてください。	まどをあけてください	
\\	辛い	からい	
\\	彼は辛いものが好きです。	かれはからいものがすきです	
\\	金	かね	
\\	これはかなり金がかかった。	これはかなりかねがかかった	
\\	女	おんな	
\\	店員は若い女の人でした。	てんいんはわかいおんなのひとでした	
\\	二十日	はつか	
\\	来月の二十日は弟の誕生日です。	らいげつのはつかはおとうとのたんじょうびです	
\\	飲み物	のみもの	
\\	何か飲み物が欲しいな。	なにかのみものがほしいな	
\\	喋る	しゃべる	
\\	彼女はよく喋るね。	かのじょはよくしゃべるね	
\\	四つ	よっつ	
\\	私は腕時計を四つ持っています。	わたしはうでどけいをよっつもっています	
\\	五	ご	
\\	五人で旅行に行きました。	ごにんでりょこうにいきました	
\\	月	つき	
\\	今夜は月がとても綺麗です。	こんやはつきがとてもきれいです	
\\	九つ	ここのつ	
\\	娘は明日九つになります。	むすめはあすここのつになります	
\\	来る	くる	
\\	彼は昼過ぎに来ます。	かれはひるすぎにきます	
\\	目	め	
\\	彼女は青い目をしています。	かのじょはあおいめをしています	
\\	暖かい	あたたかい	
\\	このコートはとても暖かい。	このこーとはとてもあたたかい	
\\	付ける	つける	
\\	電気を付けてください。	でんきをつけてください	
\\	昼	ひる	
\\	私は昼のドラマを毎日見ます。	わたしはひるのどらまをまいにちみます	
\\	水	みず	
\\	水を一杯ください。	みずをいっぱいください	
\\	店	みせ	
\\	私はこの店によく来ます。	わたしはこのみせによくきます	
\\	日曜日	にちようび	
\\	日曜日は海に行きました。	にちようびはうみにいきました	
\\	休む	やすむ	
\\	明日、会社を休みます。	あしたかいしゃをやすみます	
\\	見える	みえる	
\\	ここから富士山がよく見えます。	ここからふじさんがよくみえます	
\\	無い	ない	
\\	ここには何も無い。	ここにはなにもない	
\\	十日	とおか	
\\	十日後に帰ります。	とおかごにかえります	
\\	夜	よる	
\\	昨日の夜は家にいました。	きのうのよるはいえにいました	
\\	上る	のぼる	
\\	猫が屋根に上っている。	ねこがやねにのぼっている	
\\	直る	なおる	
\\	クーラーはまだ直りません。	くーらーはまだなおりません	
\\	九日	ここのか	
\\	九日に荷物が届きます。	ここのかににもつがとどきます	
\\	八	はち	
\\	林檎を八個ください。	りんごをはちこください	
\\	鼻	はな	
\\	鼻が痒いです。	はながかゆいです	
\\	二つ	ふたつ	
\\	ソフトクリームを二つください。	そふとくりーむをふたつください	
\\	お父さん	おとうさん	
\\	お父さんは会社員です。	おとうさんはかいしゃいんです	
\\	五つ	いつつ	
\\	桃を五つください。	ももをいつつください	
\\	どうして	どうして	
\\	どうして泣いているの。	どうしてないているの	
\\	動く	うごく	
\\	動かないで。	うごかないで	
\\	米	こめ	
\\	日本人はお米が大好きです。	にほんじんはおこめがだいすきです	
\\	弟	おとうと	
\\	弟は野球が好きです。	おとうとはやきゅうがすきです	
\\	時々	ときどき	
\\	彼は時々遅刻します。	かれはときどきちこくします	
\\	切る	きる	
\\	この紙を半分に切ってください。	このかみをはんぶんにきってください	
\\	六	ろく	
\\	彼には子供が六人います。	かれにはこどもがろくにんいます	
\\	黄色い	きいろい	
\\	彼女に黄色い薔薇を買いました。	かのじょにきいろいばらをかいました	
\\	開く	ひらく	
\\	32ページを開いてください。	さんじゅうにぺーじをひらいてください	
\\	父	ちち	
\\	私は父が大好きです。	わたしはちちがだいすきです	
\\	茶色	ちゃいろ	
\\	彼女は茶色の靴を履いています。	かのじょはちゃいろのくつをはいています	
\\	留学生	りゅうがくせい	
\\	彼は留学生です。	かれはりゅうがくせいです	
\\	あれ	あれ	
\\	あれは何ですか。	あれはなんですか	
\\	書き直す	かきなおす	
\\	この書類を書き直してください。	このしょるいをかきなおしてください	
\\	七つ	ななつ	
\\	この子は今年七つになります。	このこはことしななつになります	
\\	起きる	おきる	
\\	近所で盗難事件が起きました。	きんじょでとうなんじけんがおきました	
\\	熱い	あつい	
\\	このスープはとても熱い。	このすーぷはとてもあつい	
\\	食べ物	たべもの	
\\	日本の食べ物はとても美味しいです。	にっぽんのたべものはとてもおいしいです	
\\	一人	ひとり	
\\	そこには私一人しかいなかった。	そこにはわたしひとりしかいなかった	
\\	渇く	かわく	
\\	喉が渇きました。	のどがかわきました	
\\	前	まえ	
\\	その店の前で会いましょう。	そのみせのまえであいましょう	
\\	七	なな	
\\	バナナが七本あります。	ばなながななほんあります	
\\	泊まる	とまる	
\\	今日はこのホテルに泊まります。	きょうはこのほてるにとまります	
\\	後ろ	うしろ	
\\	後ろを向いて。	うしろをむいて	
\\	背	せ	
\\	彼女は背が高い。	かのじょはせがたかい	
\\	暑い	あつい	
\\	今日はとても暑い。	きょうはとてもあつい	
\\	多分	たぶん	
\\	彼女は多分家で寝ています。	かのじょはたぶんいえでねています	
\\	腰	こし	
\\	昨日から腰が痛い。	きのうからこしがいたい	
\\	触る	さわる	
\\	絵に触らないでください。	えにさわらないでください	
\\	自分	じぶん	
\\	宿題は自分でやりなさい。	しゅくだいはじぶんでやりなさい	
\\	四	し	
\\	四月に大学に入学しました。	しがつにだいがくににゅうがくしました	
\\	手伝う	てつだう	
\\	私が手伝いましょう。	わたしがてつだいましょう	
\\	晩	ばん	
\\	晩ご飯は食べましたか。	ばんごはんはたべましたか	
\\	優しい	やさしい	
\\	彼はとても優しい人です。	かれはとてもやさしいひとです	
\\	美味しい	おいしい	
\\	このケーキは美味しいね。	このけーきはおいしいね	
\\	天気	てんき	
\\	今日はいい天気ですね。	きょうはいいてんきですね	
\\	場合	ばあい	
\\	分からない場合は私に聞いてください。	わからないばあいはわたしにきいてください	
\\	水曜日	すいようび	
\\	水曜日はバイトがあります。	すいようびはばいとがあります	
\\	より	より	
\\	これから、より一層努力します。	これからよりいっそうどりょくします	
\\	首相	しゅしょう	
\\	今の首相はあまり力がない。	いまのしゅしょうはあまりちからがない	
\\	受ける	うける	
\\	彼は就職試験を受けた。	かれはしゅうしょくしけんをうけた	
\\	利用	りよう	
\\	私はよく図書館を利用します。	わたしはよくとしょかんをりようします	
\\	示す	しめす	
\\	彼は新製品に興味を示している。	かれはしんせいひんにきょうみをしめしている	
\\	行う	おこなう	
\\	その会社は来月、キャンペーンを行う。	そのかいしゃはらいげつきゃんぺーんをおこなう	
\\	政府	せいふ	
\\	そのデモについて政府は何もしなかった。	そのでもについてせいふはなにもしなかった	
\\	問題	もんだい	
\\	問題が一つあります。	もんだいがひとつあります	
\\	昨年	さくねん	
\\	昨年は地震が多い年でした。	さくねんはじしんがおおいとしでした	
\\	発表	はっぴょう	
\\	合格者が発表された。	ごうかくしゃがはっぴょうされた	
\\	治る	なおる	
\\	怪我はもう治りましたか。	けがはもうなおりましたか	
\\	開く	あく	
\\	電車のドアが開きました。	でんしゃのどあがあきました	
\\	販売	はんばい	
\\	前売券は窓口で販売しています。	まえうりけんはまどぐちではんばいしています	
\\	建設	けんせつ	
\\	新しいビルの建設が始まった。	あたらしいびるのけんせつがはじまった	
\\	事件	じけん	
\\	その事件の犯人はまだ捕まっていない。	そのじけんのはんにんはまだつかまっていない	
\\	持つ	もつ	
\\	この車はよく持っているね。	このくるまはよくもっているね	
\\	企業	きぎょう	
\\	彼女はアメリカの企業で働いています。	かのじょはあめりかのきぎょうではたらいています	
\\	研究	けんきゅう	
\\	彼は何年も地震の研究をしている。	かれはなんねんもじしんのけんきゅうをしている	
\\	現在	げんざい	
\\	現在の気温は30度です。	げんざいのきおんはさんじゅうどです	
\\	ドル	どる	
\\	この服は300ドルしました。	このふくはさんびゃくどるしました	
\\	必要	ひつよう	
\\	私にはたくさんのお金が必要だ。	わたしにはたくさんのおかねがひつようだ	
\\	機能	きのう	
\\	このソフトには色々な機能があります。	このそふとにはいろいろなきのうがあります	
\\	情報	じょうほう	
\\	学生たちはインターネットで色々な情報を集めた。	がくせいたちはいんたーねっとでいろいろなじょうほうをあつめた	
\\	開発	かいはつ	
\\	ダムの開発に住民は反対しています。	だむのかいはつにじゅうみんははんたいしています	
\\	一方	いっぽう	
\\	ここは一方通行です。	ここはいっぽうつうこうです	
\\	土地	とち	
\\	ここは父の土地です。	ここはちちのとちです	
\\	やる	やる	
\\	犬に餌をやった。	いぬにえさをやった	
\\	普及	ふきゅう	
\\	塵のリサイクルが普及している。	ごみのりさいくるがふきゅうしている	
\\	株	かぶ	
\\	最近株を始めました。	さいきんかぶをはじめました	
\\	もう	もう	
\\	コーヒーをもう一杯ください。	こーひーをもういっぱいください	
\\	実現	じつげん	
\\	夢を実現するには努力が必要です。	ゆめをじつげんするにはどりょくがひつようです	
\\	理由	りゆう	
\\	遅れた理由を教えてください。	おくれたりゆうをおしえてください	
\\	会談	かいだん	
\\	会談の内容が発表されました。	かいだんのないようがはっぴょうされました	
\\	存在	そんざい	
\\	宇宙人は存在すると思いますか。	うちゅうじんはそんざいするとおもいますか	
\\	採用	さいよう	
\\	その会社は女性を多く採用している。	そのかいしゃはじょせいをおおくさいようしている	
\\	関係	かんけい	
\\	彼はその事件に関係がない。	かれはそのじけんにかんけいがない	
\\	発生	はっせい	
\\	交差点で事故が発生した。	こうさてんでじこがはっせいした	
\\	集める	あつめる	
\\	弟は切手を集めています。	おとうとはきってをあつめています	
\\	会長	かいちょう	
\\	初めに会長が挨拶した。	はじめにかいちょうがあいさつした	
\\	対象	たいしょう	
\\	このアンケートは大学生が対象です。	このあんけーとはだいがくせいがたいしょうです	
\\	競争	きょうそう	
\\	2社は互いに競争している。	にしゃはたがいにきょうそうしている	
\\	状態	じょうたい	
\\	ここは道の状態がとても悪いです。	ここはみちのじょうたいがとてもわるいです	
\\	昭和	しょうわ	
\\	私の両親は昭和生まれです。	わたしのりょうしんはしょうわうまれです	
\\	生まれる	うまれる	
\\	姉夫婦に男の子が生まれました。	あねふうふにおとこのこがうまれました	
\\	全国	ぜんこく	
\\	次は全国のお天気です。	つぎはぜんこくのおてんきです	
\\	与える	あたえる	
\\	そのニュースは彼に大きなショックを与えた。	そのにゅーすはかれにおおきなしょっくをあたえた	
\\	命令	めいれい	
\\	彼女は命令に従わなかった。	かのじょはめいれいにしたがわなかった	
\\	政権	せいけん	
\\	政権が交代した。	せいけんがこうたいした	
\\	商品	しょうひん	
\\	この商品はよく売れている。	このしょうひんはよくうれている	
\\	可能	かのう	
\\	20キロのダイエットは可能だと思いますか。	にじゅっきろのだいえっとはかのうだとおもいますか	
\\	制度	せいど	
\\	来年から新しい制度が始まります。	らいねんからあたらしいせいどがはじまります	
\\	世界	せかい	
\\	私は世界旅行をしたい。	わたしはせかいりょこうをしたい	
\\	報告	ほうこく	
\\	昨日の会議について報告があります。	きのうのかいぎについてほうこくがあります	
\\	政治	せいじ	
\\	私は政治に関心がある。	わたしはせいじにかんしんがある	
\\	選ぶ	えらぶ	
\\	良い家を選ぶのは難しい。	よいいえをえらぶのはむずかしい	
\\	十分	じゅうぶん	
\\	お金はまだ十分あります。	おかねはまだじゅうぶんあります	
\\	社会	しゃかい	
\\	これは大きな社会問題になっている。	これはおおきなしゃかいもんだいになっている	
\\	用いる	もちいる	
\\	彼はその詩を用いて自分の気持ちを伝えた。	かれはそのしをもちいてじぶんのきもちをつたえた	
\\	家	いえ	
\\	ここが私の家です。	ここがわたしのいえです	
\\	最大	さいだい	
\\	これは世界最大の船です。	これはせかいさいだいのふねです	
\\	意見	いけん	
\\	あなたの意見が聞きたいです。	あなたのいけんがききたいです	
\\	対策	たいさく	
\\	一緒に対策を考えましょう。	いっしょにたいさくをかんがえましょう	
\\	社長	しゃちょう	
\\	伯母は小さな会社の社長です。	おばはちいさなかいしゃのしゃちょうです	
\\	今回	こんかい	
\\	まあ今回は許してあげよう。	まあこんかいはゆるしてあげよう	
\\	先ず	まず	
\\	帰ったら先ず手を洗いましょう。	かえったらまずてをあらいましょう	
\\	残る	のこる	
\\	料理がたくさん残りました。	りょうりがたくさんのこりました	
\\	国内	こくない	
\\	この携帯電話が使えるのは国内だけです。	このけいたいでんわがつかえるのはこくないだけです	
\\	構造	こうぞう	
\\	この建物の構造は複雑です。	このたてもののこうぞうはふくざつです	
\\	相手	あいて	
\\	試合の相手は誰ですか。	しあいのあいてはだれですか	
\\	病院	びょういん	
\\	病院はどこですか。	びょういんはどこですか	
\\	資金	しきん	
\\	私たちは今、結婚資金を貯めています。	わたしたちはいまけっこんしきんをためています	
\\	輸入	ゆにゅう	
\\	これはイタリアから輸入した服です。	これはいたりあからゆにゅうしたふくです	
\\	除く	のぞく	
\\	部長を除く全員が土曜日も働いた。	ぶちょうをのぞくぜんいんがどようびもはたらいた	
\\	頃	ころ	
\\	子供の頃、よくその公園で遊んだ。	こどものころよくそのこうえんであそんだ	
\\	投資	とうし	
\\	私は4つの会社に投資しています。	わたしはよっつのかいしゃにとうししています	
\\	代表	だいひょう	
\\	彼がクラスの代表だ。	かれがくらすのだいひょうだ	
\\	方針	ほうしん	
\\	今後の方針が決まった。	こんごのほうしんがきまった	
\\	表現	ひょうげん	
\\	彼は歌で自分の気持ちを表現した。	かれはうたでじぶんのきもちをひょうげんした	
\\	分野	ぶんや	
\\	音楽は彼の得意な分野です。	おんがくはかれのとくいなぶんやです	
\\	構成	こうせい	
\\	システムの構成を変えてみました。	しすてむのこうせいをかえてみました	
\\	以上	いじょう	
\\	飛行機が1時間以上遅れた。	ひこうきがいちじかんいじょうおくれた	
\\	備える	そなえる	
\\	災害に備えて大量の水を買い込んだ。	さいがいにそなえてたいりょうのみずをかいこんだ	
\\	目立つ	めだつ	
\\	彼女の大きな帽子はとても目立つ。	かのじょのおおきなぼうしはとてもめだつ	
\\	対する	たいする	
\\	その質問に対する答えが見つからなかった。	そのしつもんにたいするこたえがみつからなかった	
\\	管理	かんり	
\\	彼女が私のスケジュールを管理している。	かのじょがわたしのすけじゅーるをかんりしている	
\\	自由	じゆう	
\\	今日は自由な時間が多い。	きょうはじゆうなじかんがおおい	
\\	今後	こんご	
\\	今後ともよろしくお願いします。	こんごともよろしくおねがいします	
\\	事故	じこ	
\\	彼は事故で怪我をしました。	かれはじこでけがをしました	
\\	価格	かかく	
\\	ガソリンの価格がどんどん上がっている。	がそりんのかかくがどんどんあがっている	
\\	影響	えいきょう	
\\	私は彼から大きな影響を受けました。	わたしはかれからおおきなえいきょうをうけました	
\\	力	ちから	
\\	お相撲さんは、みんな力持ちだ。	おすもうさんはみんなちからもちだ	
\\	業者	ぎょうしゃ	
\\	引っ越しを業者に頼んだ。	ひっこしをぎょうしゃにたのんだ	
\\	解決	かいけつ	
\\	トラブルがやっと解決した。	とらぶるがやっとかいけつした	
\\	明らか	あきらか	
\\	明らかに彼が悪い。	あきらかにかれがわるい	
\\	内容	ないよう	
\\	この本の内容を説明してください。	このほんのないようをせつめいしてください	
\\	良く	よく	
\\	良く答えが分かりましたね。	よくこたえがわかりましたね	
\\	比べる	くらべる	
\\	今月と先月の売上を比べた。	こんげつとせんげつのうりあげをくらべた	
\\	事実	じじつ	
\\	それは全て事実ですか。	それはすべてじじつですか	
\\	上がる	あがる	
\\	私たちは2階に上がった。	わたしたちはにかいにあがった	
\\	ソフト	そふと	
\\	このソフトで日本語を勉強することができます。	このそふとでにほんごをべんきょうすることができます	
\\	処理	しょり	
\\	事務的な処理に1週間掛かります。	じむてきなしょりにいっしゅうかんかかります	
\\	方向	ほうこう	
\\	あの人たちは皆、同じ方向を見ている。	あのひとたちはみんなおなじほうこうをみている	
\\	生産	せいさん	
\\	ボルドーはワインの生産で有名だ。	ぼるどーはわいんのせいさんでゆうめいだ	
\\	部分	ぶぶん	
\\	この部分は問題ないです。	このぶぶんはもんだいないです	
\\	選挙	せんきょ	
\\	彼は選挙に出るつもりだ。	かれはせんきょにでるつもりだ	
\\	激しい	はげしい	
\\	激しい雨が降っています。	はげしいあめがふっています	
\\	新聞	しんぶん	
\\	今日の新聞、どこに置いた?	きょうのしんぶんどこにおいた	
\\	加える	くわえる	
\\	塩、胡椒を加えてください。	しおこしょうをくわえてください	
\\	増える	ふえる	
\\	この町は人口が増えた。	このまちはじんこうがふえた	
\\	提供	ていきょう	
\\	彼がパーティー会場を提供してくれました。	かれがぱーてぃーかいじょうをていきょうしてくれました	
\\	銀行	ぎんこう	
\\	銀行は3時まで開いています。	ぎんこうはさんじまであいています	
\\	変わる	かわる	
\\	信号が青に変わりました。	しんごうがあおにかわりました	
\\	段階	だんかい	
\\	この段階では、決断するのはまだ早い。	このだんかいではけつだんするのはまだはやい	
\\	最近	さいきん	
\\	それは最近話題の本ですね。	それはさいきんわだいのほんですね	
\\	中心	ちゅうしん	
\\	ここが建物の中心です。	ここがたてもののちゅうしんです	
\\	作業	さぎょう	
\\	作業するにはもっと広いスペースが必要だ。	さぎょうするにはもっとひろいすぺーすがひつようだ	
\\	対立	たいりつ	
\\	その2社は対立しています。	そのにしゃはたいりつしています	
\\	結果	けっか	
\\	試合の結果を早く知りたい。	しあいのけっかをはやくしりたい	
\\	計画	けいかく	
\\	彼は一人旅の計画を立てた。	かれはひとりたびのけいかくをたてた	
\\	設計	せっけい	
\\	兄は船の設計をしています。	あにはふねのせっけいをしています	
\\	立場	たちば	
\\	彼女は自分の立場を分かっていない。	かのじょはじぶんのたちばをわかっていない	
\\	都市	とし	
\\	東京は日本一大きな都市です。	とうきょうはにほんいちおおきなとしです	
\\	殆ど	ほとんど	
\\	お金が殆どありません。	おかねがほとんどありません	
\\	会社	かいしゃ	
\\	彼は小さな会社に勤めています。	かれはちいさなかいしゃにつとめています	
\\	規模	きぼ	
\\	この動物園は日本一の規模です。	このどうぶつえんはにっぽんいちのきぼです	
\\	原因	げんいん	
\\	この事故の原因は何ですか。	このじこのげんいんはなんですか	
\\	目的	もくてき	
\\	彼が来た目的が分かりません。	かれがきたもくてきがわかりません	
\\	予定	よてい	
\\	今日の予定を教えてください。	きょうのよていをおしえてください	
\\	期待	きたい	
\\	皆私たちに期待しています。	みんなわたしたちにきたいしています	
\\	大統領	だいとうりょう	
\\	フランスの大統領は誰ですか。	ふらんすのだいとうりょうはだれですか	
\\	党	とう	
\\	党の代表が質問に答えました。	とうのだいひょうがしつもんにこたえました	
\\	電話	でんわ	
\\	あとで電話します。	あとででんわします	
\\	あまり	あまり	
\\	このビールはあまり美味しくありません。	このびーるはあまりおいしくありません	
\\	工場	こうじょう	
\\	この会社の工場は海の近くにある。	このかいしゃのこうじょうはうみのちかくにある	
\\	語る	かたる	
\\	今夜は大いに語りましょう。	こんやはおおいにかたりましょう	
\\	唯	ただ	
\\	この温泉は唯です。	このおんせんはただです	
\\	文章	ぶんしょう	
\\	彼は文章がとてもうまい。	かれはぶんしょうがとてもうまい	
\\	後	のち	
\\	後に彼は総理大臣になりました。	のちにかれはそうりだいじんになりました	
\\	広がる	ひろがる	
\\	留学してから私の世界が広がった。	りゅうがくしてからわたしのせかいがひろがった	
\\	プログラム	ぷろぐらむ	
\\	受付でプログラムを受け取った。	うけつけでぷろぐらむをうけとった	
\\	これら	これら	
\\	今日はこれらの問題について話し合います。	きょうはこれらのもんだいについてはなしあいます	
\\	開始	かいし	
\\	運動会は9時開始です。	うんどうかいはくじかいしです	
\\	決定	けってい	
\\	会議で重要な決定がありました。	かいぎでじゅうようなけっていがありました	
\\	多く	おおく	
\\	毎年多くの人が海外へ旅行する。	まいとしおおくのひとがかいがいへりょこうする	
\\	地域	ちいき	
\\	この地域は雨が多い。	このちいきはあめがおおい	
\\	消える	きえる	
\\	突然、電気が消えた。	とつぜんでんきがきえた	
\\	システム	しすてむ	
\\	全てのシステムが停止した。	すべてのしすてむがていしした	
\\	計算	けいさん	
\\	その計算は間違っている。	そのけいさんはまちがっている	
\\	生活	せいかつ	
\\	日本での生活は楽しいです。	にほんでのせいかつはたのしいです	
\\	国会	こっかい	
\\	国会が再開した。	こっかいがさいかいした	
\\	略	ほぼ	
\\	仕事が略終わりました。	しごとがほぼおわりました	
\\	続ける	つづける	
\\	仕事を続けてください。	しごとをつづけてください	
\\	繰り返す	くりかえす	
\\	彼女は同じ間違いを繰り返した。	かのじょはおなじまちがいをくりかえした	
\\	まま	まま	
\\	電気が付いたままですよ。	でんきがついたままですよ	
\\	種類	しゅるい	
\\	薔薇には色々な種類があります。	ばらにはいろいろなしゅるいがあります	
\\	初めて	はじめて	
\\	東京に来るのは初めてです。	とうきょうにくるのははじめてです	
\\	一般	いっぱん	
\\	一般の方はこちらの席へどうぞ。	いっぱんのかたはこちらのせきへどうぞ	
\\	従来	じゅうらい	
\\	このプリンターは従来のものより速い。	このぷりんたーはじゅうらいのものよりはやい	
\\	コンピューター	こんぴゅーたー	
\\	新しいコンピューターを買った。	あたらしいこんぴゅーたーをかった	
\\	写真	しゃしん	
\\	写真は良い思い出になります。	しゃしんはいいおもいでになります	
\\	尚	なお	
\\	尚、雨の場合は中止です。	なおあめのばあいはちゅうしです	
\\	調査	ちょうさ	
\\	私たちがその問題を調査しています。	わたしたちがそのもんだいをちょうさしています	
\\	住宅	じゅうたく	
\\	ここは静かな住宅地だ。	ここはしずかなじゅうたくちだ	
\\	進める	すすめる	
\\	早く授業を進めましょう。	はやくじゅぎょうをすすめましょう	
\\	姿	すがた	
\\	遠くに彼女の姿が見えた。	とおくにかのじょのすがたがみえた	
\\	伴う	ともなう	
\\	その仕事は危険を伴う。	そのしごとはきけんをともなう	
\\	意味	いみ	
\\	それはどういう意味ですか。	それはどういういみですか	
\\	テレビ	てれび	
\\	私はテレビをあまり見ません。	わたしはてれびをあまりみません	
\\	一部	いちぶ	
\\	計画を一部変更しましょう。	けいかくをいちぶへんこうしましょう	
\\	強化	きょうか	
\\	国は国語教育を強化しています。	くにはこくごきょういくをきょうかしています	
\\	特徴	とくちょう	
\\	この曲線がこの車の特徴です。	このきょくせんがこのくるまのとくちょうです	
\\	改革	かいかく	
\\	彼は行政を改革したいと思っている。	かれはぎょうせいをかいかくしたいとおもっている	
\\	調べる	しらべる	
\\	この単語の意味を辞書で調べましょう。	このたんごのいみをじしょでしらべましょう	
\\	交渉	こうしょう	
\\	今、値段を交渉しています。	いまねだんをこうしょうしています	
\\	話	はなし	
\\	あなたの話は面白いね。	あなたのはなしはおもしろいね	
\\	重要	じゅうよう	
\\	これは重要な書類です。	これはじゅうようなしょるいです	
\\	いずれ	いずれ	
\\	いずれまたお会いしましょう。	いずれまたおあいしましょう	
\\	経済	けいざい	
\\	大学で経済を勉強しました。	だいがくでけいざいをべんきょうしました	
\\	女性	じょせい	
\\	そのパーティーに女性は何人来ますか。	そのぱーてぃーにじょせいはなんにんきますか	
\\	パソコン	ぱそこん	
\\	彼はパソコンを2台持っています。	かれはぱそこんをにだいもっています	
\\	経営	けいえい	
\\	我が社の経営は上手く行っています。	わがしゃのけいえいはうまくいっています	
\\	増加	ぞうか	
\\	島の人口は年々増加しています。	しまのじんこうはねんねんぞうかしています	
\\	外国	がいこく	
\\	母はまだ外国に行ったことがありません。	はははまだがいこくにいったことがありません	
\\	検討	けんとう	
\\	今日中にこの問題を検討してください。	きょうじゅうにこのもんだいをけんとうしてください	
\\	減る	へる	
\\	体重がかなり減りました。	たいじゅうがかなりへりました	
\\	工場	こうじょう	
\\	彼は食品工場で働いています。	かれはしょくひんこうじょうではたらいています	
\\	評価	ひょうか	
\\	最近、彼の評価が上がった。	さいきんかれのひょうかがあがった	
\\	関する	かんする	
\\	その問題に関する記事を読みました。	そのもんだいにかんするきじをよみました	
\\	技術	ぎじゅつ	
\\	彼は非常に高い技術を持っている。	かれはひじょうにたかいぎじゅつをもっている	
\\	変化	へんか	
\\	今年は変化の多い年でした。	ことしはへんかのおおいとしでした	
\\	全体	ぜんたい	
\\	全体の80パーセントが完成しました。	ぜんたいのはちじゅっぱーせんとがかんせいしました	
\\	声	こえ	
\\	彼は大きな声で話した。	かれはおおきなこえではなした	
\\	方法	ほうほう	
\\	いい方法を思い付きました。	いいほうほうをおもいつきました	
\\	家族	かぞく	
\\	うちは五人家族です。	うちはごにんかぞくです	
\\	実施	じっし	
\\	現在、スペシャルキャンペーンを実施中です。	げんざいすぺしゃるきゃんぺーんをじっしちゅうです	
\\	動き	うごき	
\\	先ず相手の動きを見ましょう。	まずあいてのうごきをみましょう	
\\	完成	かんせい	
\\	新しいホームページが完成した。	あたらしいほーむぺーじがかんせいした	
\\	参加	さんか	
\\	明日は市民マラソンに参加します。	あすはしみんまらそんにさんかします	
\\	強調	きょうちょう	
\\	彼は良いところだけを強調した。	かれはよいところだけをきょうちょうした	
\\	導入	どうにゅう	
\\	会社で新しいシステムを導入した。	かいしゃであたらしいしすてむをどうにゅうした	
\\	達する	たっする	
\\	気温は35度に達した。	きおんはさんじゅうごどにたっした	
\\	自動車	じどうしゃ	
\\	彼は自動車会社に就職した。	かれはじどうしゃがいしゃにしゅうしょくした	
\\	産業	さんぎょう	
\\	日本の自動車産業は世界的に有名だ。	にほんのじどうしゃさんぎょうはせかいてきにゆうめいだ	
\\	同様	どうよう	
\\	私たちは彼を家族同様に思っている。	わたしたちはかれをかぞくどうようにおもっている	
\\	訴える	うったえる	
\\	彼女は会社を訴えた。	かのじょはかいしゃをうったえた	
\\	拡大	かくだい	
\\	この図を拡大コピーしてください。	このずをかくだいこぴーしてください	
\\	伝える	つたえる	
\\	皆にこのことを伝えてください。	みんなにこのことをつたえてください	
\\	個人	こじん	
\\	これは私個人の意見です。	これはわたしこじんのいけんです	
\\	事情	じじょう	
\\	あなたの事情はよく分かりました。	あなたのじじょうはよくわかりました	
\\	直す	なおす	
\\	私がそれを直しました。	わたしがそれをなおしました	
\\	議員	ぎいん	
\\	彼は国会議員です。	かれはこっかいぎいんです	
\\	当然	とうぜん	
\\	彼女が怒るのも当然だ。	かのじょがおこるのもとうぜんだ	
\\	主張	しゅちょう	
\\	上司は私の主張を受け入れた。	じょうしはわたしのしゅちょうをうけいれた	
\\	迫る	せまる	
\\	締め切りが迫っています。	しめきりがせまっています	
\\	最後	さいご	
\\	今日が夏休み最後の日だ。	きょうがなつやすみさいごのひだ	
\\	記録	きろく	
\\	マラソンで世界記録が出た。	まらそんでせかいきろくがでた	
\\	運動	うんどう	
\\	彼はもっと運動した方がいい。	かれはもっとうんどうしたほうがいい	
\\	車	くるま	
\\	この道は車が多い。	このみちはくるまがおおい	
\\	見方	みかた	
\\	彼に対する見方が変わりました。	かれにたいするみかたがかわりました	
\\	通常	つうじょう	
\\	通常は夜8時まで営業しています。	つうじょうはよるはちじまでえいぎょうしています	
\\	向かう	むかう	
\\	今、会社に向かっています。	いまかいしゃにむかっています	
\\	反対	はんたい	
\\	私は反対です。	わたしははんたいです	
\\	世紀	せいき	
\\	新しい世紀の始まりです。	あたらしいせいきのはじまりです	
\\	全く	まったく	
\\	私には全く分かりません。	わたしにはまったくわかりません	
\\	施設	しせつ	
\\	そのホテルにはレジャー施設がたくさんある。	そのほてるにはれじゃーしせつがたくさんある	
\\	場所	ばしょ	
\\	会社の場所を教えてください。	かいしゃのばしょをおしえてください	
\\	大幅	おおはば	
\\	計画を大幅に変更した。	けいかくをおおはばにへんこうした	
\\	直接	ちょくせつ	
\\	彼に直接お願いしなさい。	かれにちょくせつおねがいしなさい	
\\	程度	ていど	
\\	この程度の怪我なら大丈夫です。	このていどのけがならだいじょうぶです	
\\	批判	ひはん	
\\	彼は同僚を批判した。	かれはどうりょうをひはんした	
\\	発言	はつげん	
\\	会議で全員が発言した。	かいぎでぜんいんがはつげんした	
\\	組織	そしき	
\\	彼はある組織のリーダーだ。	かれはあるそしきのりーだーだ	
\\	作品	さくひん	
\\	この絵はゴッホの作品だ。	このえはごっほのさくひんだ	
\\	課題	かだい	
\\	夏休みの課題は何ですか。	なつやすみのかだいはなんですか	
\\	能力	のうりょく	
\\	彼は能力のある社員です。	かれはのうりょくのあるしゃいんです	
\\	残す	のこす	
\\	彼女はメッセージを残しました。	かのじょはめっせーじをのこしました	
\\	要求	ようきゅう	
\\	彼は私の要求に答えた。	かれはわたしのようきゅうにこたえた	
\\	分	ぶん	
\\	このケーキはあなたの分です。	このけーきはあなたのぶんです	
\\	集まる	あつまる	
\\	駅前に人が集まっています。	えきまえにひとがあつまっています	
\\	材料	ざいりょう	
\\	サラダの材料を揃えました。	さらだのざいりょうをそろえました	
\\	記者	きしゃ	
\\	彼は新聞記者です。	かれはしんぶんきしゃです	
\\	政策	せいさく	
\\	新しい政策はあまり良いとは思えません。	あたらしいせいさくはあまりいいとはおもえません	
\\	文化	ぶんか	
\\	私はこの国の文化を勉強しています。	わたしはこのくにのぶんかをべんきょうしています	
\\	グループ	ぐるーぷ	
\\	店に学生のグループが来た。	みせにがくせいのぐるーぷがきた	
\\	実際	じっさい	
\\	彼は実際にはあまり背が高くない。	かれはじっさいにはあまりせがたかくない	
\\	占める	しめる	
\\	私に届くメールのうち、迷惑メールが7割を占めている。	わたしにとどくめーるのうちめいわくめーるがななわりをしめている	
\\	招く	まねく	
\\	両親を食事に招いた。	りょうしんをしょくじにまねいた	
\\	担当	たんとう	
\\	私はセールスを担当しています。	わたしはせーるすをたんとうしています	
\\	教育	きょういく	
\\	彼は海外で教育を受けました。	かれはかいがいできょういくをうけました	
\\	機械	きかい	
\\	新しい機械が壊れた。	あたらしいきかいがこわれた	
\\	最初	さいしょ	
\\	5ページの最初を見てください。	ごぺーじのさいしょをみてください	
\\	サービス	さーびす	
\\	この店はサービスがいい	このみせはさーびすがいい	
\\	年間	ねんかん	
\\	年間5万人がここを訪れます。	ねんかんごまんにんがここをおとずれます	
\\	成功	せいこう	
\\	遂に実験が成功した。	ついにじっけんがせいこうした	
\\	提案	ていあん	
\\	そのアイデアは彼の提案です。	そのあいであはかれのていあんです	
\\	言語	げんご	
\\	彼はアジアの言語を研究している。	かれはあじあのげんごをけんきゅうしている	
\\	市民	しみん	
\\	市民の安全は大切だ。	しみんのあんぜんはたいせつだ	
\\	若者	わかもの	
\\	最近の若者は本を読まない。	さいきんのわかものはほんをよまない	
\\	警察	けいさつ	
\\	警察を呼んでください。	けいさつをよんでください	
\\	装置	そうち	
\\	実験にはこの装置を使います。	じっけんにはこのそうちをつかいます	
\\	国際	こくさい	
\\	ここで国際会議が開かれます。	ここでこくさいかいぎがひらかれます	
\\	公開	こうかい	
\\	その映画は今日、公開されます。	そのえいがはきょうこうかいされます	
\\	会議	かいぎ	
\\	今日の午後、大事な会議があります。	きょうのごごだいじなかいぎがあります	
\\	年度	ねんど	
\\	売り上げは年度によって違います。	うりあげはねんどによってちがいます	
\\	過去	かこ	
\\	それは過去の話だ。	それはかこのはなしだ	
\\	具体的	ぐたいてき	
\\	具体的な例をいくつか見せてください。	ぐたいてきなれいをいくつかみせてください	
\\	基づく	もとづく	
\\	この話は真実に基づいています。	このはなしはしんじつにもとづいています	
\\	経験	けいけん	
\\	今日の試合はいい経験になりました。	きょうのしあいはいいけいけんになりました	
\\	再び	ふたたび	
\\	彼は再びここに戻ってきた。	かれはふたたびここにもどってきた	
\\	姿勢	しせい	
\\	あの子はいつも姿勢が悪い。	あのこはいつもしせいがわるい	
\\	輸出	ゆしゅつ	
\\	彼の会社は車を輸出しています。	かれのかいしゃはくるまをゆしゅつしています	
\\	注目	ちゅうもく	
\\	私たちはその会社に注目している。	わたしたちはそのかいしゃにちゅうもくしている	
\\	背景	はいけい	
\\	事件の背景に何があったのだろう。	じけんのはいけいになにがあったのだろう	
\\	判断	はんだん	
\\	彼の判断は正しい。	かれのはんだんはただしい	
\\	上昇	じょうしょう	
\\	地球の気温は上昇している。	ちきゅうのきおんはじょうしょうしている	
\\	簡単	かんたん	
\\	この料理はとても簡単です。	このりょうりはとてもかんたんです	
\\	ホテル	ほてる	
\\	今ホテルに着きました。	いまほてるにつきました	
\\	願う	ねがう	
\\	彼が元気になるよう願っています。	かれがげんきになるようねがっています	
\\	進む	すすむ	
\\	前に進んでください。	まえにすすんでください	
\\	取引	とりひき	
\\	私たちは中国の会社と取引しています。	わたしたちはちゅうごくのかいしゃととりひきしています	
\\	当たる	あたる	
\\	ボールが彼の頭に当たった。	ぼーるがかれのあたまにあたった	
\\	協力	きょうりょく	
\\	このプロジェクトには皆の協力が必要です。	このぷろじぇくとにはみんなのきょうりょくがひつようです	
\\	纏める	まとめる	
\\	彼女は荷物を纏めて出て行った。	かのじょはにもつをまとめてでていった	
\\	加工	かこう	
\\	この工場では魚を加工している。	このこうじょうではさかなをかこうしている	
\\	実行	じっこう	
\\	彼はその計画を実行した。	かれはそのけいかくをじっこうした	
\\	他	た	
\\	留学生はアジア人が多く、その他は3割です。	りゅうがくせいはあじあじんがおおくそのたはさんわりです	
\\	手術	しゅじゅつ	
\\	父は胸の手術をした。	ちちはむねのしゅじゅつをした	
\\	安全	あんぜん	
\\	安全が第一です。	あんぜんがだいいちです	
\\	事務所	じむしょ	
\\	後で事務所に来てください。	あとでじむしょにきてください	
\\	自然	しぜん	
\\	皆で自然を守りましょう。	みんなでしぜんをまもりましょう	
\\	発見	はっけん	
\\	新しい星が発見された。	あたらしいほしがはっけんされた	
\\	少年	しょうねん	
\\	少年たちがサッカーをしている。	しょうねんたちがさっかーをしている	
\\	患者	かんじゃ	
\\	患者は眠っています。	かんじゃはねむっています	
\\	契約	けいやく	
\\	その選手は新しいチームと契約した。	そのせんしゅはあたらしいちーむとけいやくした	
\\	許す	ゆるす	
\\	私は彼を許しました。	わたしはかれをゆるしました	
\\	危険	きけん	
\\	その地域は今、危険だ。	そのちいきはいまきけんだ	
\\	関連	かんれん	
\\	関連のセクションへ連絡した。	かんれんのせくしょんへれんらくした	
\\	矢張り	やはり	
\\	彼は矢張り遅刻しました。	かれはやはりちこくしました	
\\	半分	はんぶん	
\\	お菓子を友達に半分あげた。	おかしをともだちにはんぶんあげた	
\\	運ぶ	はこぶ	
\\	椅子を二階に運んでください。	いすをにかいにはこんでください	
\\	非常に	ひじょうに	
\\	これは非常に重要です。	これはひじょうにじゅうようです	
\\	出席	しゅっせき	
\\	午後は会議に出席します。	ごごはかいぎにしゅっせきします	
\\	結合	けつごう	
\\	ファイルを結合して1つにしました。	ふぁいるをけつごうしてひとつにしました	
\\	候補	こうほ	
\\	会長の候補は3人います。	かいちょうのこうほはさんにんいます	
\\	特に	とくに	
\\	特に質問はありません。	とくにしつもんはありません	
\\	家庭	かてい	
\\	彼は家庭を大切にしている。	かれはかていをたいせつにしている	
\\	話し合う	はなしあう	
\\	私たちはよく話し合いました。	わたしたちはよくはなしあいました	
\\	大型	おおがた	
\\	大型のテレビを買った。	おおがたのてれびをかった	
\\	海外	かいがい	
\\	彼は海外での生活が長いです。	かれはかいがいでのせいかつがながいです	
\\	そのまま	そのまま	
\\	そのままお待ちください。	そのままおまちください	
\\	文字	もじ	
\\	壁に文字が書いてあった。	かべにもじがかいてあった	
\\	諸国	しょこく	
\\	彼はヨーロッパ諸国を旅行した。	かれはよーろっぱしょこくをりょこうした	
\\	効果	こうか	
\\	この薬には胃を守る効果がある。	このくすりにはいをまもるこうかがある	
\\	攻撃	こうげき	
\\	2003年にアメリカはイラクを攻撃した。	にせんさんねんにあめりかはいらくをこうげきした	
\\	地球	ちきゅう	
\\	地球は丸い。	ちきゅうはまるい	
\\	確認	かくにん	
\\	もう一度、予約を確認した。	もういちどよやくをかくにんした	
\\	歴史	れきし	
\\	私は歴史に興味があります。	わたしはれきしにきょうみがあります	
\\	ビル	びる	
\\	私の会社はあのビルの8階です。	わたしのかいしゃはあのびるのはちかいです	
\\	海	うみ	
\\	海は広くて大きい。	うみはひろくておおきい	
\\	住民	じゅうみん	
\\	地域の住民が集まって話合いをした。	ちいきのじゅうみんがあつまってはなしあいをした	
\\	普通	ふつう	
\\	彼女は普通の女の子だ。	かのじょはふつうのおんなのこだ	
\\	当時	とうじ	
\\	彼女は当時、まだ3才だった。	かのじょはとうじまださんさいだった	
\\	はっきり	はっきり	
\\	今日は山がはっきり見える。	きょうはやまがはっきりみえる	
\\	通過	つうか	
\\	次の駅は通過します。	つぎのえきはつうかします	
\\	動物	どうぶつ	
\\	私は動物が大好きです。	わたしはどうぶつがだいすきです	
\\	映画	えいが	
\\	彼はよく映画を見ます。	かれはよくえいがをみます	
\\	木	き	
\\	台風で木が倒れた。	たいふうできがたおれた	
\\	努力	どりょく	
\\	もっと努力しよう。	もっとどりょくしよう	
\\	注意	ちゅうい	
\\	車に注意してください。	くるまにちゅういしてください	
\\	自宅	じたく	
\\	自宅に電話を下さい。	じたくにでんわをください	
\\	設ける	もうける	
\\	授業の終わりに復習の時間を設けた。	じゅぎょうのおわりにふくしゅうのじかんをもうけた	
\\	移る	うつる	
\\	彼女は新しい会社に移った。	かのじょはあたらしいかいしゃにうつった	
\\	選手	せんしゅ	
\\	彼はプロのサッカー選手だ。	かれはぷろのさっかーせんしゅだ	
\\	時代	じだい	
\\	今は便利さとスピードの時代だ。	いまはべんりさとすぴーどのじだいだ	
\\	異なる	ことなる	
\\	彼と私はいつも意見が異なる。	かれとわたしはいつもいけんがことなる	
\\	位置	いち	
\\	私の町は東京の北に位置します。	わたしのまちはとうきょうのきたにいちします	
\\	卒業	そつぎょう	
\\	私は去年、大学を卒業した。	わたしはきょねんだいがくをそつぎょうした	
\\	平和	へいわ	
\\	この国は平和です。	このくにはへいわです	
\\	驚く	おどろく	
\\	彼は血を見て驚いた。	かれはちをみておどろいた	
\\	各国	かっこく	
\\	各国の代表がニューヨークに集まった。	かっこくのだいひょうがにゅーよーくにあつまった	
\\	複雑	ふくざつ	
\\	このプログラムはとても複雑です。	このぷろぐらむはとてもふくざつです	
\\	大会	たいかい	
\\	夏には川辺で花火大会があります。	なつにはかわべではなびたいかいがあります	
\\	並ぶ	ならぶ	
\\	ここに並んでください。	ここにならんでください	
\\	周辺	しゅうへん	
\\	この周辺には大学が多い。	このしゅうへんにはだいがくがおおい	
\\	外国人	がいこくじん	
\\	日本に住む外国人が増えています。	にほんにすむがいこくじんがふえています	
\\	準備	じゅんび	
\\	明日の会議の準備をした。	あすのかいぎのじゅんびをした	
\\	責任	せきにん	
\\	彼は失敗の責任を取って、会社を辞めた。	かれはしっぱいのせきにんをとってかいしゃをやめた	
\\	深い	ふかい	
\\	あの池はとても深い。	あのいけはとてもふかい	
\\	空気	くうき	
\\	ここは空気が綺麗です。	ここはくうきがきれいです	
\\	基本	きほん	
\\	今、ジャズダンスの基本を習っています。	いまじゃずだんすのきほんをならっています	
\\	建物	たてもの	
\\	これは日本一古い建物です。	これはにほんいちふるいたてものです	
\\	貿易	ぼうえき	
\\	父は貿易の仕事をしています。	ちちはぼうえきのしごとをしています	
\\	子	こ	
\\	その子は日本語が分からない。	そのこはにほんごがわからない	
\\	人口	じんこう	
\\	その国の人口はどのくらいですか。	そのくにのじんこうはどのくらいですか	
\\	遅れる	おくれる	
\\	今朝彼女は学校に遅れました。	けさかのじょはがっこうにおくれました	
\\	経つ	たつ	
\\	あれから14年が経ちました。	あれからじゅうよねんがたちました	
\\	スポーツ	すぽーつ	
\\	あなたは何かスポーツをしていますか。	あなたはなにかすぽーつをしていますか	
\\	地震	じしん	
\\	日本は地震が多いです。	にほんはじしんがおおいです	
\\	条件	じょうけん	
\\	この条件では厳し過ぎます。	このじょうけんではきびしすぎます	
\\	道路	どうろ	
\\	この道路は3年前に出来ました。	このどうろはさんねんまえにできました	
\\	工事	こうじ	
\\	工事の音がうるさい。	こうじのおとがうるさい	
\\	英語	えいご	
\\	あなたは英語が話せますか。	あなたはえいごがはなせますか	
\\	合う	あう	
\\	この靴は私の足に合っている。	このくつはわたしのあしにあっている	
\\	状況	じょうきょう	
\\	この状況では出発は難しいです。	このじょうきょうではしゅっぱつはむずかしいです	
\\	雨	あめ	
\\	雨が降っています。	あめがふっています	
\\	限る	かぎる	
\\	このサービスは週末に限ります。	このさーびすはしゅうまつにかぎります	
\\	手紙	てがみ	
\\	友人から手紙を貰いました。	ゆうじんからてがみをもらいました	
\\	実験	じっけん	
\\	科学の授業で実験をした。	かがくのじゅぎょうでじっけんをした	
\\	音	おと	
\\	雨の音が聞こえる。	あめのおとがきこえる	
\\	高校	こうこう	
\\	妹は高校に通っています。	いもうとはこうこうにかよっています	
\\	教授	きょうじゅ	
\\	彼は化学の教授です。	かれはかがくのきょうじゅです	
\\	なぜ	なぜ	
\\	なぜ来なかったの。	なぜこなかったの	
\\	一定	いってい	
\\	申し込むには一定の資格が要る。	もうしこむにはいっていのしかくがいる	
\\	回る	まわる	
\\	月は地球のまわりを回っています。	つきはちきゅうのまわりをまわっています	
\\	分ける	わける	
\\	お菓子を皆で分けました。	おかしをみんなでわけました	
\\	訪れる	おとずれる	
\\	私は夏に京都を訪れました。	わたしはなつにきょうとをおとずれました	
\\	今度	こんど	
\\	今度はどこに行きたいですか。	こんどはどこにいきたいですか	
\\	傾向	けいこう	
\\	最近の若者は難しい本を読まない傾向がある。	さいきんのわかものはむずかしいほんをよまないけいこうがある	
\\	述べる	のべる	
\\	上司が意見を述べた。	じょうしがいけんをのべた	
\\	つまり	つまり	
\\	つまり、あなたは何も知らないのですね。	つまりあなたはなにもしらないのですね	
\\	開催	かいさい	
\\	京都で国際会議が開催された。	きょうとでこくさいかいぎがかいさいされた	
\\	毎日	まいにち	
\\	私たちは毎日散歩をします。	わたしたちはまいにちさんぽをします	
\\	製造	せいぞう	
\\	この工場ではエアコンを製造しています。	このこうじょうではえあこんをせいぞうしています	
\\	もし	もし	
\\	もし雨が降ったら、行きません。	もしあめがふったらいきません	
\\	もちろん	もちろん	
\\	もちろん一緒に行きます。	もちろんいっしょにいきます	
\\	彼ら	かれら	
\\	彼らはバスケットの選手です。	かれらはばすけっとのせんしゅです	
\\	期間	きかん	
\\	テスト期間は10日から15日までだ。	てすときかんはとおかからじゅうごにちまでだ	
\\	曾て	かつて	
\\	曾て私が学生だった頃のことです。	かつてわたしががくせいだったころのことです	
\\	説明	せつめい	
\\	この単語の意味を説明してください。	このたんごのいみをせつめいしてください	
\\	花	はな	
\\	綺麗な花が咲きました。	きれいなはながさきました	
\\	届く	とどく	
\\	昨日、父から手紙が届いた。	きのうちちからてがみがとどいた	
\\	帰国	きこく	
\\	彼は帰国しました。	かれはきこくしました	
\\	生徒	せいと	
\\	このクラスの生徒は30人です。	このくらすのせいとはさんじゅうにんです	
\\	時期	じき	
\\	今はあなたにとって大事な時期です。	いまはあなたにとってだいじなじきです	
\\	社員	しゃいん	
\\	彼は優秀な社員です。	かれはゆうしゅうなしゃいんです	
\\	雪	ゆき	
\\	クリスマスに雪が降りました。	くりすますにゆきがふりました	
\\	守る	まもる	
\\	彼は約束を守る人です。	かれはやくそくをまもるひとです	
\\	逮捕	たいほ	
\\	逃げていた犯人が逮捕されました。	にげていたはんにんがたいほされました	
\\	基準	きじゅん	
\\	判断の基準が示された。	はんだんのきじゅんがしめされた	
\\	活動	かつどう	
\\	彼は地域の活動に参加した。	かれはちいきのかつどうにさんかした	
\\	無理	むり	
\\	5時までに家に帰るのは無理です。	ごじまでにいえにかえるのはむりです	
\\	過ぎる	すぎる	
\\	時が過ぎるのは早い。	ときがすぎるのははやい	
\\	アパート	あぱーと	
\\	彼はアパートに住んでいます。	かれはあぱーとにすんでいます	
\\	ガス	がす	
\\	地震でガスが止まった。	じしんでがすがとまった	
\\	思い出す	おもいだす	
\\	大切な用事を思い出しました。	たいせつなようじをおもいだしました	
\\	戸	と	
\\	部屋の戸が開いています。	へやのとがあいています	
\\	味	あじ	
\\	この料理は味が薄い。	このりょうりはあじがうすい	
\\	迎える	むかえる	
\\	姉が空港まで迎えに来てくれます。	あねがくうこうまでむかえにきてくれます	
\\	ずっと	ずっと	
\\	父は休みの日はずっとテレビを見ている。	ちちはやすみのひはずっとてれびをみている	
\\	早く	はやく	
\\	成るべく早く来て下さい。	なるべくはやくきてください	
\\	レコード	れこーど	
\\	ジャズのレコードを掛けました。	じゃずのれこーどをかけました	
\\	並べる	ならべる	
\\	私は料理をテーブルに並べた。	わたしはりょうりをてーぶるにならべた	
\\	年寄り	としより	
\\	あの村にはお年寄りがたくさん住んでいます。	あのむらにはおとしよりがたくさんすんでいます	
\\	放送	ほうそう	
\\	その番組は来週放送されます。	そのばんぐみはらいしゅうほうそうされます	
\\	訪ねる	たずねる	
\\	彼はニューヨークの友達を訪ねた。	かれはにゅーよーくのともだちをたずねた	
\\	薄い	うすい	
\\	この電子辞書はとても薄い。	このでんしじしょはとてもうすい	
\\	科学	かがく	
\\	科学は常に進歩している。	かがくはつねにしんぽしている	
\\	犬	いぬ	
\\	この犬はとても賢い。	このいぬはとてもかしこい	
\\	正しい	ただしい	
\\	それは正しい答えです。	それはただしいこたえです	
\\	大きさ	おおきさ	
\\	この大きさの封筒が欲しいのですが。	このおおきさのふうとうがほしいのですが	
\\	料理	りょうり	
\\	母は料理が得意です。	はははりょうりがとくいです	
\\	戦争	せんそう	
\\	2003年にイラクで戦争があった。	にせんさんねんにいらくでせんそうがあった	
\\	通る	とおる	
\\	毎日、この道を通ります。	まいにちこのみちをとおります	
\\	例えば	たとえば	
\\	例えば、このソフトで日本語を勉強することが出来ます。	たとえばこのそふとでにほんごをべんきょうすることができます	
\\	明るい	あかるい	
\\	彼女は明るい性格です。	かのじょはあかるいせいかくです	
\\	増やす	ふやす	
\\	あの町は緑を増やしています。	あのまちはみどりをふやしています	
\\	ビール	びーる	
\\	夏はビールがとても美味しい。	なつはびーるがとてもおいしい	
\\	育てる	そだてる	
\\	彼女は三人の子を育てました。	かのじょはさんにんのこをそだてました	
\\	甘い	うまい	
\\	甘い寿司屋を見つけたよ。	うまいすしやをみつけたよ	
\\	夫婦	ふうふ	
\\	その夫婦はとても仲がいい。	そのふうふはとてもなかがいい	
\\	つもり(積もり)	つもり	
\\	明日から煙草を止めるつもりです。	あしたからたばこをやめるつもりです	
\\	島	しま	
\\	日本は島国です。	にっぽんはしまぐにです	
\\	通り	とおり	
\\	この通りは賑やかですね。	このとおりはにぎやかですね	
\\	パーティー	ぱーてぃー	
\\	明日、うちでパーティーを開きます。	あしたうちでぱーてぃーをひらきます	
\\	捨てる	すてる	
\\	塵を捨ててください。	ごみをすててください	
\\	分かれる	わかれる	
\\	グループの中で意見が分かれました。	ぐるーぷのなかでいけんがわかれました	
\\	売れる	うれる	
\\	今年の夏はクーラーがよく売れた。	ことしのなつはくーらーがよくうれた	
\\	クラス	くらす	
\\	この学校は1クラス30人です。	このがっこうはひとくらすさんじゅうにんです	
\\	北	きた	
\\	ロシアは日本の北にあります。	ろしあはにっぽんのきたにあります	
\\	大使館	たいしかん	
\\	彼は大使館に勤めています。	かれはたいしかんにつとめています	
\\	長男	ちょうなん	
\\	彼の長男は8才です。	かれのちょうなんははっさいです	
\\	コース	こーす	
\\	私は日本語コースを取っています。	わたしはにほんごこーすをとっています	
\\	入院	にゅういん	
\\	昨日、母が入院しました。	きのうははがにゅういんしました	
\\	大変	たいへん	
\\	大変なことが起こりました。	たいへんなことがおこりました	
\\	玄関	げんかん	
\\	玄関に花を飾りました。	げんかんにはなをかざりました	
\\	亡くなる	なくなる	
\\	一昨日、昔の友人が亡くなった。	おとといむかしのゆうじんがなくなった	
\\	流れる	ながれる	
\\	ラジオから美しい音楽が流れています。	らじおからうつくしいおんがくがながれています	
\\	聞こえる	きこえる	
\\	隣の部屋からテレビの音が聞こえる。	となりのへやからてれびのおとがきこえる	
\\	長さ	ながさ	
\\	このケーブルの長さは1メートルです。	このけーぶるのながさはいちめーとるです	
\\	信じる	しんじる	
\\	彼はキリストを信じている。	かれはきりすとをしんじている	
\\	頼む	たのむ	
\\	私はハンバーガーを頼みました。	わたしははんばーがーをたのみました	
\\	振る	ふる	
\\	犬が尻尾を振っている。	いぬがしっぽをふっている	
\\	動かす	うごかす	
\\	部屋の家具を動かしました。	へやのかぐをうごかしました	
\\	自然	しぜん	
\\	親が子供を守るのは自然なことだ。	おやがこどもをまもるのはしぜんなことだ	
\\	温度	おんど	
\\	今、部屋の温度は25度だ。	いまへやのおんどはにじゅうごどだ	
\\	涙	なみだ	
\\	母は涙を流していた。	はははなみだをながしていた	
\\	数学	すうがく	
\\	兄は数学の先生です。	あにはすうがくのせんせいです	
\\	約束	やくそく	
\\	約束は守ります。	やくそくはまもります	
\\	故障	こしょう	
\\	冷蔵庫が故障しました。	れいぞうこがこしょうしました	
\\	太陽	たいよう	
\\	太陽が雲に隠れた。	たいようがくもにかくれた	
\\	近所	きんじょ	
\\	近所にカナダ人が住んでいる。	きんじょにかなだじんがすんでいる	
\\	打つ	うつ	
\\	転んで膝を打ちました。	ころんでひざをうちました	
\\	教室	きょうしつ	
\\	私の教室は3階にあります。	わたしのきょうしつはさんがいにあります	
\\	困る	こまる	
\\	携帯を失くして困っています。	けーたいをなくしてこまっています	
\\	ページ	ぺーじ	
\\	教科書の36ページを開いてください。	きょうかしょのさんじゅうろくぺーじをひらいてください	
\\	地図	ちず	
\\	地図を見て来てください。	ちずをみてきてください	
\\	夏休み	なつやすみ	
\\	夏休みにお祖母ちゃんの家へ行きました。	なつやすみにおばあちゃんのうちへいきました	
\\	有名	ゆうめい	
\\	彼は有名な歌手です。	かれはゆうめいなかしゅです	
\\	受け取る	うけとる	
\\	彼からメールを受け取りました。	かれからめーるをうけとりました	
\\	倒れる	たおれる	
\\	強風で木が倒れた。	きょうふうできがたおれた	
\\	私達	わたしたち	
\\	私達は来月結婚します。	わたしたちはらいげつけっこんします	
\\	ニュース	にゅーす	
\\	夜のニュースを見ましたか。	よるのにゅーすをみましたか	
\\	教師	きょうし	
\\	彼は高校教師だ。	かれはこうこうきょうしだ	
\\	紹介	しょうかい	
\\	両親に彼女を紹介した。	りょうしんにかのじょをしょうかいした	
\\	入学	にゅうがく	
\\	妹は九月にアメリカの大学に入学します。	いもうとはくがつにあめりかのだいがくににゅうがくします	
\\	季節	きせつ	
\\	私の一番好きな季節は春です。	わたしのいちばんすきなきせつははるです	
\\	移す	うつす	
\\	机を窓の傍に移しました。	つくえをまどのそばにうつしました	
\\	どんどん	どんどん	
\\	ドアをどんどんと叩いた。	どあをどんどんとたたいた	
\\	テープ	てーぷ	
\\	彼女はその会話をテープに録音した。	かのじょはそのかいわをてーぷにろくおんした	
\\	横	よこ	
\\	横の長さは1メートルです。	よこのながさはいちめーとるです	
\\	殺す	ころす	
\\	私は生き物を殺すのが嫌いだ。	わたしはいきものをころすのがきらいだ	
\\	静か	しずか	
\\	今年の夏休みには静かな所へ行きたい。	ことしのなつやすみにはしずかなところへいきたい	
\\	中学生	ちゅうがくせい	
\\	息子は中学生です。	むすこはちゅうがくせいです	
\\	上手い	うまい	
\\	彼は野球が上手い。	かれはやきゅうがうまい	
\\	落とす	おとす	
\\	途中で財布を落としました。	とちゅうでさいふをおとしました	
\\	歌	うた	
\\	私はその歌を知らなかった。	わたしはそのうたをしらなかった	
\\	急ぐ	いそぐ	
\\	私たちは駅へ急ぎました。	わたしたちはえきへいそぎました	
\\	壁	かべ	
\\	壁に絵が掛かっている。	かべにえがかかっている	
\\	窓	まど	
\\	窓を開けてください。	まどをあけてください	
\\	相談	そうだん	
\\	相談したいことがあります。	そうだんしたいことがあります	
\\	席	せき	
\\	この席、空いてますか。	このせきあいてますか	
\\	サラリーマン	さらりーまん	
\\	父はサラリーマンです。	ちちはさらりーまんです	
\\	旅行	りょこう	
\\	彼女は旅行が好きです。	かのじょはりょこうがすきです	
\\	熱	ねつ	
\\	昨日の夜、熱が出ました。	きのうのよるねつがでました	
\\	高さ	たかさ	
\\	富士山の高さは3,776メートルです。	ふじさんのたかさはさんぜんななひゃく-ななじゅうろくめーとるです	
\\	留学	りゅうがく	
\\	1年間、アメリカに留学しました。	いちねんかんあめりかにりゅうがくしました	
\\	空港	くうこう	
\\	空港までリムジンバスで行った。	くうこうまでりむじんばすでいった	
\\	音楽	おんがく	
\\	私は音楽を聞くのが好きだ。	わたしはおんがくをきくのがすきだ	
\\	健康	けんこう	
\\	私は健康な生活を送っています。	わたしはけんこうなせいかつをおくっています	
\\	グラフ	ぐらふ	
\\	彼は売上をグラフにした。	かれはうりあげをぐらふにした	
\\	考え	かんがえ	
\\	それは良い考えです。	それはいいかんがえです	
\\	馬	うま	
\\	彼は牧場で馬に乗った。	かれはぼくじょうでうまにのった	
\\	バス	ばす	
\\	バスで行こう。	ばすでいこう	
\\	カメラ	かめら	
\\	カメラが壊れた。	かめらがこわれた	
\\	苦しい	くるしい	
\\	食べ過ぎてお腹が苦しい。	たべすぎておなかがくるしい	
\\	裏	うら	
\\	会社の裏に公園があります。	かいしゃのうらにこうえんがあります	
\\	特別	とくべつ	
\\	あなたは私にとって特別な人です。	あなたはわたしにとってとくべつなひとです	
\\	穴	あな	
\\	靴下に穴が開いている。	くつしたにあながあいている	
\\	食事	しょくじ	
\\	今日は上司と食事をする。	きょうはじょうしとしょくじをする	
\\	番組	ばんぐみ	
\\	私はこの番組が好きです。	わたしはこのばんぐみがすきです	
\\	珍しい	めずらしい	
\\	昨日珍しい果物を食べました。	きのうめずらしいくだものをたべました	
\\	必ず	かならず	
\\	必ずシートベルトを着けて下さい。	かならずしーとべるとをつけてください	
\\	工業	こうぎょう	
\\	そこは工業都市だ。	そこはこうぎょうとしだ	
\\	小説	しょうせつ	
\\	私は月に3冊くらい小説を読みます。	わたしはつきにさんさつくらいしょうせつをよみます	
\\	逃げる	にげる	
\\	鳥が窓から逃げたよ。	とりがまどからにげたよ	
\\	用意	ようい	
\\	食事の用意が出来ました。	しょくじのよういができました	
\\	首都	しゅと	
\\	東京は日本の首都です。	とうきょうはにっぽんのしゅとです	
\\	川	かわ	
\\	小さな川を渡りました。	ちいさなかわをわたりました	
\\	細かい	こまかい	
\\	彼女は細かいことにうるさい。	かのじょはこまかいことにうるさい	
\\	減らす	へらす	
\\	最近、食事を減らしています。	さいきんしょくじをへらしています	
\\	ビデオ	びでお	
\\	私はその番組をビデオに撮った。	わたしはそのばんぐみをびでおにとった	
\\	伸びる	のびる	
\\	髪が大分伸びたね。	かみがだいぶのびたね	
\\	中学	ちゅうがく	
\\	息子は中学に通っています。	むすこはちゅうがくにかよっています	
\\	海岸	かいがん	
\\	海岸を散歩しましょう。	かいがんをさんぽしましょう	
\\	成績	せいせき	
\\	成績が上がりました。	せいせきがあがりました	
\\	渡す	わたす	
\\	彼に手紙を渡しました。	かれにてがみをわたしました	
\\	隣	となり	
\\	隣の家には犬がいます。	となりのいえにはいぬがいます	
\\	代わる	かわる	
\\	上司に代わって会議に出た。	じょうしにかわってかいぎにでた	
\\	農業	のうぎょう	
\\	私は農業を勉強しています。	わたしはのうぎょうをべんきょうしています	
\\	ガラス	がらす	
\\	クリスタルガラスの花瓶を買いました。	くりすたるがらすのかびんをかいました	
\\	立てる	たてる	
\\	彼はケーキに蝋燭を立てた。	かれはけーきにろうそくをたてた	
\\	着く	つく	
\\	午後8時に大阪に着きます。	ごごはちじにおおさかにつきます	
\\	喜ぶ	よろこぶ	
\\	彼女はとても喜びました。	かのじょはとてもよろこびました	
\\	正確	せいかく	
\\	彼の計算は正確です。	かれのけいさんはせいかくです	
\\	夢	ゆめ	
\\	私の夢は先生になることです。	わたしのゆめはせんせいになることです	
\\	値段	ねだん	
\\	このベルトの値段は2500円でした。	このべるとのねだんはにせんごひゃくえんでした	
\\	やっと	やっと	
\\	やっと仕事が終わりました。	やっとしごとがおわりました	
\\	下げる	さげる	
\\	少し音量を下げてください。	すこしおんりょうをさげてください	
\\	小学校	しょうがっこう	
\\	家の近くに小学校があります。	いえのちかくにしょうがっこうがあります	
\\	借りる	かりる	
\\	彼にビデオを借りました。	かれにびでおをかりました	
\\	運転	うんてん	
\\	父は安全運転だ。	ちちはあんぜんうんてんだ	
\\	出発	しゅっぱつ	
\\	あと15分で出発です。	あとじゅうごふんでしゅっぱつです	
\\	雑誌	ざっし	
\\	この雑誌はよく売れています。	このざっしはよくうれています	
\\	野菜	やさい	
\\	私は毎日たくさん野菜を食べます。	わたしはまいにちたくさんやさいをたべます	
\\	マンション	まんしょん	
\\	彼はマンションに住んでいます。	かれはまんしょんにすんでいます	
\\	黒い	くろい	
\\	彼女は黒いドレスを着ています。	かのじょはくろいどれすをきています	
\\	合格	ごうかく	
\\	娘が入学試験に合格しました。	むすめがにゅうがくしけんにごうかくしました	
\\	秘密	ひみつ	
\\	これは秘密です。	これはひみつです	
\\	トラック	とらっく	
\\	トラックを運転できますか。	とらっくをうんてんできますか	
\\	飾る	かざる	
\\	テーブルの上に花を飾りました。	てーぶるのうえにはなをかざりました	
\\	夕方	ゆうがた	
\\	夕方、雨が降りました。	ゆうがたあめがふりました	
\\	風	かぜ	
\\	今日は風が強いです。	きょうはかぜがつよいです	
\\	就職	しゅうしょく	
\\	最近、若い人たちの就職が難しくなっています。	さいきんわかいひとたちのしゅうしょくがむずかしくなっています	
\\	通う	かよう	
\\	私はジムに通っています。	わたしはじむにかよっています	
\\	撮る	とる	
\\	写真をたくさん撮りました。	しゃしんをたくさんとりました	
\\	贈る	おくる	
\\	母に花を贈った。	ははにはなをおくった	
\\	何か	なにか	
\\	道に何か落ちています。	みちになにかおちています	
\\	公園	こうえん	
\\	公園に小さな池があります。	こうえんにちいさないけがあります	
\\	数字	すうじ	
\\	数字は苦手です。	すうじはにがてです	
\\	詳しい	くわしい	
\\	もっと詳しく説明してください。	もっとくわしくせつめいしてください	
\\	法律	ほうりつ	
\\	新しい法律が出来た。	あたらしいほうりつができた	
\\	暫く	しばらく	
\\	そこで暫く休んでいます。	そこでしばらくやすんでいます	
\\	毎年	まいとし	
\\	私は毎年、海外旅行に行きます。	わたしはまいとしかいがいりょこうにいきます	
\\	給料	きゅうりょう	
\\	来年から給料が上がります。	らいねんからきゅうりょうがあがります	
\\	規則	きそく	
\\	あの会社の規則は厳しいです。	あのかいしゃのきそくはきびしいです	
\\	テスト	てすと	
\\	明日は理科のテストです。	あしたはりかのてすとです	
\\	急に	きゅうに	
\\	急に用事を思い出した。	きゅうにようじをおもいだした	
\\	ゆっくり	ゆっくり	
\\	もっとゆっくり話してください。	もっとゆっくりはなしてください	
\\	降りる	おりる	
\\	次の駅で降ります。	つぎのえきでおります	
\\	忙しい	いそがしい	
\\	忙しいので手伝ってください。	いそがしいのでてつだってください	
\\	両方	りょうほう	
\\	チョコレートとケーキを両方ください。	ちょこれーととけーきをりょうほうください	
\\	タクシー	たくしー	
\\	タクシーを呼んでください。	たくしーをよんでください	
\\	電気	でんき	
\\	電気をつけてもらえる?	でんきをつけてもらえる	
\\	曲	きょく	
\\	私はこの曲が大好きです。	わたしはこのきょくがだいすきです	
\\	ラジオ	らじお	
\\	私は毎朝ラジオを聴きます。	わたしはまいあさらじおをききます	
\\	返す	かえす	
\\	図書館に本を返した。	としょかんにほんをかえした	
\\	教科書	きょうかしょ	
\\	日本語の教科書を忘れた。	にほんごのきょうかしょをわすれた	
\\	慣れる	なれる	
\\	新しい家にはもう慣れましたか。	あたらしいいえにはもうなれましたか	
\\	暗い	くらい	
\\	東の空が暗いです。	ひがしのそらがくらいです	
\\	来月	らいげつ	
\\	来月から大学生になります。	らいげつからだいがくせいになります	
\\	試験	しけん	
\\	彼は採用試験に合格した。	かれはさいようしけんにごうかくした	
\\	台風	たいふう	
\\	台風が近付いている。	たいふうがちかづいている	
\\	ポスト	ぽすと	
\\	手紙をポストに入れました。	てがみをぽすとにいれました	
\\	向こう	むこう	
\\	友達は向こうにいます。	ともだちはむこうにいます	
\\	兄弟	きょうだい	
\\	彼は3人兄弟です。	かれはさんにんきょうだいです	
\\	切れる	きれる	
\\	この鋏はよく切れますね。	このはさみはよくきれますね	
\\	中学校	ちゅうがっこう	
\\	息子の中学校は家から5分です。	むすこのちゅうがっこうはいえからごふんです	
\\	看護婦	かんごふ	
\\	母は看護婦です。	はははかんごふです	
\\	厚い	あつい	
\\	その辞書はとても厚い。	そのじしょはとてもあつい	
\\	ゴルフ	ごるふ	
\\	兄はゴルフを始めました。	あにはごるふをはじめました	
\\	野球	やきゅう	
\\	友達と野球をしました。	ともだちとやきゅうをしました	
\\	森	もり	
\\	私は森を歩くのが好きです。	わたしはもりをあるくのがすきです	
\\	確か	たしか	
\\	彼の昇進は確かだ。	かれのしょうしんはたしかだ	
\\	牛肉	ぎゅうにく	
\\	夕食に牛肉を買った。	ゆうしょくにぎゅうにくをかった	
\\	女の子	おんなのこ	
\\	あの女の子を知っていますか。	あのおんなのこをしっていますか	
\\	急	きゅう	
\\	急な坂道を上った。	きゅうなさかみちをのぼった	
\\	あちこち	あちこち	
\\	私たちは朝からあちこち散歩しました。	わたしたちはあさからあちこちさんぽしました	
\\	遠く	とおく	
\\	遠くに船が見えます。	とおくにふねがみえます	
\\	遊び	あそび	
\\	お正月には色々な遊びをします。	おしょうがつにはいろいろなあそびをします	
\\	逢う	あう	
\\	ついに素晴らしい女性に逢えた。	ついにすばらしいじょせいにあえた	
\\	コピー	こぴー	
\\	会議で書類のコピーを配った。	かいぎでしょるいのこぴーをくばった	
\\	新幹線	しんかんせん	
\\	新幹線で京都に行きました。	しんかんせんできょうとにいきました	
\\	信号	しんごう	
\\	信号が青になった。	しんごうがあおになった	
\\	奥	おく	
\\	鋏は机の奥にあった。	はさみはつくえのおくにあった	
\\	奥さん	おくさん	
\\	彼の奥さんは奇麗な方です。	かれのおくさんはきれいなかたです	
\\	面白い	おもしろい	
\\	この本は凄く面白かった。	このほんはすごくおもしろかった	
\\	未来	みらい	
\\	未来は誰にも分からない。	みらいはだれにもわからない	
\\	離婚	りこん	
\\	友人が離婚しました。	ゆうじんがりこんしました	
\\	教会	きょうかい	
\\	私たちは教会で結婚式をしました。	わたしたちはきょうかいでけっこんしきをしました	
\\	赤ちゃん	あかちゃん	
\\	ベッドで赤ちゃんが眠っています。	べっどであかちゃんがねむっています	
\\	レストラン	れすとらん	
\\	レストランでインド料理を食べました。	れすとらんでいんどりょうりをたべました	
\\	打つかる	ぶつかる	
\\	車が電柱に打つかった。	くるまがでんちゅうにぶつかった	
\\	天井	てんじょう	
\\	この部屋は天井が高いですね。	このへやはてんじょうがたかいですね	
\\	箱	はこ	
\\	この箱は重い。	このはこはおもい	
\\	招待	しょうたい	
\\	高校の時の先生を結婚式に招待した。	こうこうのときのせんせいをけっこんしきにしょうたいした	
\\	散歩	さんぽ	
\\	公園を散歩しようか。	こうえんをさんぽしようか	
\\	久し振り	ひさしぶり	
\\	明日、久し振りに友達に会います。	あしたひさしぶりにともだちにあいます	
\\	買い物	かいもの	
\\	母は買い物に出掛けています。	はははかいものにでかけています	
\\	親しい	したしい	
\\	週末、親しい友達を家に呼んだ。	しゅうまつしたしいともだちをいえによんだ	
\\	生む	うむ	
\\	うちの猫が子猫を生みました。	うちのねこがこねこをうみました	
\\	練習	れんしゅう	
\\	娘は今、バイオリンを練習しています。	むすめはいまばいおりんをれんしゅうしています	
\\	毎年	まいねん	
\\	毎年給料が上がる。	まいねんきゅうりょうがあがる	
\\	地下鉄	ちかてつ	
\\	私は地下鉄で通勤しています。	わたしはちかてつでつうきんしています	
\\	向く	むく	
\\	こっちを向いてください。	こっちをむいてください	
\\	往復	おうふく	
\\	往復切符をください。	おうふくきっぷをください	
\\	大勢	おおぜい	
\\	大勢で食事に出掛けました。	おおぜいでしょくじにでかけました	
\\	コート	こーと	
\\	寒かったのでコートを着た。	さむかったのでこーとをきた	
\\	踏む	ふむ	
\\	運転手がブレーキを踏んだ。	うんてんしゅがぶれーきをふんだ	
\\	見送る	みおくる	
\\	彼が外国に行くのを見送りました。	かれががいこくにいくのをみおくりました	
\\	小学生	しょうがくせい	
\\	うちの息子は来年、小学生になります。	うちのむすこはらいねんしょうがくせいになります	
\\	吹く	ふく	
\\	今日は北風が吹いている。	きょうはきたかぜがふいている	
\\	思い出	おもいで	
\\	旅行で楽しい思い出が出来ました。	りょこうでたのしいおもいでができました	
\\	課	か	
\\	今から課のミーティングがある。	いまからかのみーてぃんぐがある	
\\	階段	かいだん	
\\	私たちは駅の階段を駆け上がった。	わたしたちはえきのかいだんをかけあがった	
\\	デパート	でぱーと	
\\	私はデパートで靴を買った。	わたしはでぱーとでくつをかった	
\\	注文	ちゅうもん	
\\	レストランでピザを注文しました。	れすとらんでぴざをちゅうもんしました	
\\	配る	くばる	
\\	会議で書類を配った。	かいぎでしょるいをくばった	
\\	髪	かみ	
\\	昨日、髪を切りました。	きのうかみをきりました	
\\	仕方	しかた	
\\	ファイルのダウンロードの仕方が分かりません。	ふぁいるのだうんろーどのしかたがわかりません	
\\	台所	だいどころ	
\\	お母さんは台所にいます。	おかあさんはだいどころにいます	
\\	カード	かーど	
\\	支払いはカードでお願いします。	しはらいはかーどでおねがいします	
\\	ノート	のーと	
\\	ノートを開いてください。	のーとをひらいてください	
\\	数える	かぞえる	
\\	椅子の数を数えてください。	いすのかずをかぞえてください	
\\	食堂	しょくどう	
\\	大学の食堂は安い。	だいがくのしょくどうはやすい	
\\	ベッド	べっど	
\\	彼はベッドで寝ています。	かれはべっどでねています	
\\	回す	まわす	
\\	捻子は左に回すと外れます。	ねじはひだりにまわすとはずれます	
\\	石	いし	
\\	私は石に躓いた。	わたしはいしにつまづいた	
\\	建つ	たつ	
\\	ここに来年、家が建ちます。	ここにらいねんいえがたちます	
\\	得意	とくい	
\\	彼は歌が得意です。	かれはうたがとくいです	
\\	家賃	やちん	
\\	ここの家賃は12万円です。	ここのやちんはじゅうにまんえんです	
\\	アルバイト	あるばいと	
\\	兄はアルバイトをしています。	あにはあるばいとをしています	
\\	修理	しゅうり	
\\	車を修理に出した。	くるまをしゅうりにだした	
\\	予約	よやく	
\\	レストランを予約しました。	れすとらんをよやくしました	
\\	一昨年	おととし	
\\	一昨年初めて京都へ旅行しました。	おととしはじめてきょうとへりょこうしました	
\\	通勤	つうきん	
\\	毎朝、通勤に30分掛かります。	まいあさつうきんにさんじゅっぷんかかります	
\\	薬	くすり	
\\	この薬を必ず飲んでください。	このくすりをかならずのんでください	
\\	辞める	やめる	
\\	彼は会社を辞めます。	かれはかいしゃをやめます	
\\	燃える	もえる	
\\	山が燃えています。	やまがもえています	
\\	校長	こうちょう	
\\	あの人は高校の校長だ。	あのひとはこうこうのこうちょうだ	
\\	靴	くつ	
\\	靴が汚れた。	くつがよごれた	
\\	揺れる	ゆれる	
\\	風で木が揺れています。	かぜできがゆれています	
\\	フィルム	ふぃるむ	
\\	旅行のためにたくさんフィルムを買った。	りょこうのためにたくさんふぃるむをかった	
\\	絵	え	
\\	これは有名な画家の絵です。	これはゆうめいながかのえです	
\\	握る	にぎる	
\\	少女は母親の手を握った。	しょうじょはははおやのてをにぎった	
\\	休日	きゅうじつ	
\\	休日は家でよくテレビを見ます。	きゅうじつはいえでよくてれびをみます	
\\	番号	ばんごう	
\\	この番号に電話してください。	このばんごうにでんわしてください	
\\	運転手	うんてんしゅ	
\\	彼はタクシーの運転手です。	かれはたくしーのうんてんしゅです	
\\	医者	いしゃ	
\\	私は医者に相談した。	わたしはいしゃにそうだんした	
\\	育つ	そだつ	
\\	野菜がよく育っている。	やさいがよくそだっている	
\\	下がる	さがる	
\\	やっと熱が下がった。	やっとねつがさがった	
\\	強さ	つよさ	
\\	風の強さに驚きました。	かぜのつよさにおどろきました	
\\	巻く	まく	
\\	彼は頭にタオルを巻いていた。	かれはあたまにたおるをまいていた	
\\	重さ	おもさ	
\\	この荷物の重さを計ってください。	このにもつのおもさをはかってください	
\\	駐車場	ちゅうしゃじょう	
\\	車は駐車場に止めてください。	くるまはちゅうしゃじょうにとめてください	
\\	録音	ろくおん	
\\	そのラジオ番組はもう録音しました。	そのらじおばんぐみはもうろくおんしました	
\\	氷	こおり	
\\	グラスに氷を入れてください。	ぐらすにこおりをいれてください	
\\	愛	あい	
\\	彼女は愛を込めて手紙を書いた。	かのじょはあいをこめててがみをかいた	
\\	浅い	あさい	
\\	この川は浅いです。	このかわはあさいです	
\\	パン	ぱん	
\\	朝ご飯にはいつもパンを食べる。	あさごはんにはいつもぱんをたべる	
\\	匂い	におい	
\\	いい匂いがする。	いいにおいがする	
\\	商業	しょうぎょう	
\\	この町では商業が盛んだ。	このまちではしょうぎょうがさかんだ	
\\	オートバイ	おーとばい	
\\	彼はオートバイに乗っている。	かれはおーとばいにのっている	
\\	恐ろしい	おそろしい	
\\	昨夜恐ろしい夢を見た。	ゆうべおそろしいゆめをみた	
\\	退院	たいいん	
\\	母が今日退院します。	ははがきょうたいいんします	
\\	旅館	りょかん	
\\	京都では旅館に泊まりました。	きょうとではりょかんにとまりました	
\\	幸せ	しあわせ	
\\	良い友達がいて私は幸せだ。	いいともだちがいてわたしはしあわせだ	
\\	金持ち	かねもち	
\\	彼は金持ちです。	かれはかねもちです	
\\	エレベーター	えれべーたー	
\\	エレベーターで下に降りましょう。	えれべーたーでしたにおりましょう	
\\	コーヒー	こーひー	
\\	私は毎朝コーヒーを飲みます。	わたしはまいあさこーひーをのみます	
\\	疲れる	つかれる	
\\	今日は疲れました。	きょうはつかれました	
\\	赤	あか	
\\	信号が赤に変わりました。	しんごうがあかにかわりました	
\\	言い方	いいかた	
\\	そんな言い方をしてはいけません。	そんないいかたをしてはいけません	
\\	交通事故	こうつうじこ	
\\	彼は交通事故を起こした。	かれはこうつうじこをおこした	
\\	人形	にんぎょう	
\\	彼女は人形をたくさん持っています。	かのじょはにんぎょうをたくさんもっています	
\\	刺す	さす	
\\	虫に腕を刺されました。	むしにうでをさされました	
\\	吸う	すう	
\\	彼は大きく息を吸った。	かれはおおきくいきをすった	
\\	休み	やすみ	
\\	今日は休みです。	きょうはやすみです	
\\	祭り	まつり	
\\	彼女は祭りが大好きです。	かのじょはまつりがだいすきです	
\\	ピアノ	ぴあの	
\\	昔、ピアノを習っていました。	むかしぴあのをならっていました	
\\	付ける	つける	
\\	お負けを付けました。	おまけをつけました	
\\	安心	あんしん	
\\	それを聞いて安心しました。	それをきいてあんしんしました	
\\	預ける	あずける	
\\	鍵を彼に預けた。	かぎをかれにあずけた	
\\	気温	きおん	
\\	今日の気温は26度です。	きょうのきおんはにじゅうろくどです	
\\	趣味	しゅみ	
\\	私の趣味は映画とテニスです。	わたしのしゅみはえいがとてにすです	
\\	先月	せんげつ	
\\	先月、友達の結婚式があった。	せんげつともだちのけっこんしきがあった	
\\	盗む	ぬすむ	
\\	誰かが私の鞄を盗みました。	だれかがわたしのかばんをぬすみました	
\\	トイレ	といれ	
\\	トイレを掃除しましたか。	といれをそうじしましたか	
\\	凄い	すごい	
\\	凄い雨になった。	すごいあめになった	
\\	メモ	めも	
\\	メモを取ってください。	めもをとってください	
\\	男の子	おとこのこ	
\\	男の子たちがサッカーをしている。	おとこのこたちがさっかーをしている	
\\	貸す	かす	
\\	私は彼に本を貸しています。	わたしはかれにほんをかしています	
\\	郵便局	ゆうびんきょく	
\\	郵便局はどこですか。	ゆうびんきょくはどこですか	
\\	外人	がいじん	
\\	この町には外人が少ない。	このまちにはがいじんがすくない	
\\	銀	ぎん	
\\	彼は銀メダルを取った。	かれはぎんめだるをとった	
\\	グラム	ぐらむ	
\\	挽肉を200グラムください。	ひきにくをにひゃくぐらむください	
\\	便利	べんり	
\\	インターネットはとても便利です。	いんたーねっとはとてもべんりです	
\\	大学院	だいがくいん	
\\	彼は大学院に進みました。	かれはだいがくいんにすすみました	
\\	ワイン	わいん	
\\	ワインを少し飲みました。	わいんをすこしのみました	
\\	ホーム	ほーむ	
\\	もう直ぐこのホームに電車が来ます。	もうすぐこのほーむにでんしゃがきます	
\\	何度	なんど	
\\	あの店には何度も行きました。	あのみせにはなんどもいきました	
\\	キャンプ	きゃんぷ	
\\	友達とキャンプに行った。	ともだちときゃんぷにいった	
\\	左手	ひだりて	
\\	彼女は左手で字を書く。	かのじょはひだりてでじをかく	
\\	おかしい	おかしい	
\\	彼の様子がおかしい。	かれのようすがおかしい	
\\	ボート	ぼーと	
\\	池でボートに乗りました。	いけでぼーとにのりました	
\\	渡る	わたる	
\\	私たちは歩いて橋を渡った。	わたしたちはあるいてはしをわたった	
\\	ボール	ぼーる	
\\	彼はボールを投げた。	かれはぼーるをなげた	
\\	中々	なかなか	
\\	荷物が中々届きません。	にもつがなかなかとどきません	
\\	叩く	たたく	
\\	彼は子供のお尻を叩いた。	かれはこどものおしりをたたいた	
\\	売り場	うりば	
\\	靴売り場はどこですか。	くつうりばはどこですか	
\\	隠れる	かくれる	
\\	太陽が雲に隠れた。	たいようがくもにかくれた	
\\	ズボン	ずぼん	
\\	ズボンが汚れた。	ずぼんがよごれた	
\\	届ける	とどける	
\\	これを彼に届けてください。	これをかれにとどけてください	
\\	塗る	ぬる	
\\	壁にペンキを塗っています。	かべにぺんきをぬっています	
\\	品物	しなもの	
\\	その店は色々な品物を売っている。	そのみせはいろいろなしなものをうっている	
\\	砂糖	さとう	
\\	コーヒーに砂糖は入れますか。	こーひーにさとうはいれますか	
\\	内側	うちがわ	
\\	白線の内側に下がってください。	はくせんのうちがわにさがってください	
\\	遭う	あう	
\\	彼は交通事故に遭った。	かれはこうつうじこにあった	
\\	敷く	しく	
\\	生まれて初めて布団を敷いた。	うまれてはじめてふとんをしいた	
\\	是非	ぜひ	
\\	是非、うちに来てください。	ぜひうちにきてください	
\\	屋根	やね	
\\	屋根に烏が止まっています。	やねにからすがとまっています	
\\	泣く	なく	
\\	妹は直ぐに泣く。	いもうとはすぐになく	
\\	投げる	なげる	
\\	ボールをこっちに投げてください。	ぼーるをこっちになげてください	
\\	別々	べつべつ	
\\	別々に払いましょう。	べつべつにはらいましょう	
\\	テント	てんと	
\\	皆でテントを張りました。	みんなでてんとをはりました	
\\	おもちゃ	おもちゃ	
\\	赤ちゃんが自動車のおもちゃで遊んでいる。	あかちゃんがじどうしゃのおもちゃであそんでいる	
\\	付く	つく	
\\	靴に泥が付いています。	くつにどろがついています	
\\	進学	しんがく	
\\	彼女は来年、大学に進学する。	かのじょはらいねんだいがくにしんがくする	
\\	光る	ひかる	
\\	波がきらきら光っていました。	なみがきらきらひかっていました	
\\	毎週	まいしゅう	
\\	私は毎週母に電話をします。	わたしはまいしゅうははにでんわをします	
\\	卵	たまご	
\\	鶏は卵を産みます。	にわとりはたまごをうみます	
\\	挨拶	あいさつ	
\\	彼女は笑顔で挨拶した。	かのじょはえがおであいさつした	
\\	破る	やぶる	
\\	彼は約束を破った。	かれはやくそくをやぶった	
\\	葉	は	
\\	これは桜の葉です。	これはさくらのはです	
\\	立派	りっぱ	
\\	彼は立派な人です。	かれはりっぱなひとです	
\\	青	あお	
\\	私の好きな色は青です。	わたしのすきないろはあおです	
\\	牛	うし	
\\	牛が草を食べています。	うしがくさをたべています	
\\	正月	しょうがつ	
\\	お正月には大抵、家族が集まる。	おしょうがつにはたいていかぞくがあつまる	
\\	荷物	にもつ	
\\	彼は大きな荷物を持ってきた。	かれはおおきなにもつをもってきた	
\\	組み立てる	くみたてる	
\\	日曜日に本棚を組み立てます。	にちようびにほんだなをくみたてます	
\\	プール	ぷーる	
\\	私は夏休みにプールに行った。	わたしはなつやすみにぷーるにいった	
\\	右手	みぎて	
\\	私は右手で字を書きます。	わたしはみぎてでじをかきます	
\\	拾う	ひろう	
\\	道で財布を拾った。	みちでさいふをひろった	
\\	習う	ならう	
\\	私はダンスを習っています。	わたしはだんすをならっています	
\\	咲く	さく	
\\	桜の花が咲きました。	さくらのはながさきました	
\\	洗う	あらう	
\\	早く顔を洗いなさい。	はやくかおをあらいなさい	
\\	びっくりする(吃驚する)	びっくりする	
\\	大きな音にびっくりしました。	おおきなおとにびっくりしました	
\\	割る	わる	
\\	皿を落として割った。	さらをおとしてわった	
\\	美術館	びじゅつかん	
\\	昨日、美術館に行きました。	きのうびじゅつかんにいきました	
\\	カラー	からー	
\\	カラーコピーは一枚幾らですか。	からーこぴーはいちまいいくらですか	
\\	頑張る	がんばる	
\\	明日の試験、頑張ってね。	あしたのしけんがんばってね	
\\	見学	けんがく	
\\	今日、工場の見学に行きました。	きょうこうじょうのけんがくにいきました	
\\	青い	あおい	
\\	ここの海はとても青い。	ここのうみはとてもあおい	
\\	スキー	すきー	
\\	冬はよくスキーに行きます。	ふゆはよくすきーにいきます	
\\	親切	しんせつ	
\\	親切にして下さってどうも有り難う御座います。	しんせつにしてくださってどうもありがとうございます	
\\	不幸	ふこう	
\\	彼女の家族に不幸があった。	かのじょのかぞくにふこうがあった	
\\	蹴る	ける	
\\	ゴールキーパーがボールを蹴った。	ごーるきーぱーがぼーるをけった	
\\	糸	いと	
\\	母は糸を針に通した。	はははいとをはりにとおした	
\\	黒	くろ	
\\	黒のボールペンはありますか。	くろのぼーるぺんはありますか	
\\	断る	ことわる	
\\	私は彼のプロポーズを断った。	わたしはかれのぷろぽーずをことわった	
\\	池	いけ	
\\	池に鯉がいます。	いけにこいがいます	
\\	沈む	しずむ	
\\	ボートが川に沈んだ。	ぼーとがかわにしずんだ	
\\	迷惑	めいわく	
\\	人に迷惑を掛けてはいけません。	ひとにめいわくをかけてはいけません	
\\	尋ねる	たずねる	
\\	彼女は道を尋ねた。	かのじょはみちをたずねた	
\\	足りる	たりる	
\\	お金が足りなくて買えなかった。	おかねがたりなくてかえなかった	
\\	猫	ねこ	
\\	私は猫が大好きです。	わたしはねこがだいすきです	
\\	案内	あんない	
\\	私が中をご案内します。	わたしがなかをごあんないします	
\\	間に合う	まにあう	
\\	授業に間に合いました。	じゅぎょうにまにあいました	
\\	博物館	はくぶつかん	
\\	昨日、車の博物館に行った。	きのうくるまのはくぶつかんにいった	
\\	廊下	ろうか	
\\	廊下は走らないでください。	ろうかははしらないでください	
\\	習慣	しゅうかん	
\\	毎朝コーヒーを飲むのが習慣です。	まいあさこーひーをのむのがしゅうかんです	
\\	泳ぐ	およぐ	
\\	彼女はダイエットのために泳いでいる。	かのじょはだいえっとのためにおよいでいる	
\\	プラスチック	ぷらすちっく	
\\	このカップはプラスチックです。	このかっぷはぷらすちっくです	
\\	壊す	こわす	
\\	彼女が私の携帯を壊した。	かのじょがわたしのけーたいをこわした	
\\	通学	つうがく	
\\	毎朝、通学に1時間掛かる。	まいあさつうがくにいちじかんかかる	
\\	桜	さくら	
\\	桜は三月か四月に咲きます。	さくらはさんがつかしがつにさきます	
\\	帰り	かえり	
\\	仕事の帰りにビールを飲んだ。	しごとのかえりにびーるをのんだ	
\\	酷い	ひどい	
\\	妹と酷い喧嘩をした。	いもうととひどいけんかをした	
\\	週末	しゅうまつ	
\\	週末は家でゆっくりします。	しゅうまつはうちでゆっくりします	
\\	真っ赤	まっか	
\\	彼の顔は真っ赤でした。	かれのかおはまっかでした	
\\	濃い	こい	
\\	私は濃い味が好きだ。	わたしはこいあじがすきだ	
\\	怖い	こわい	
\\	私は犬が怖いです。	わたしはいぬがこわいです	
\\	作文	さくぶん	
\\	日本語で作文を書きました。	にほんごでさくぶんをかきました	
\\	休憩	きゅうけい	
\\	少し休憩しましょう。	すこしきゅうけいしましょう	
\\	財布	さいふ	
\\	会社に財布を忘れた。	かいしゃにさいふをわすれた	
\\	越える	こえる	
\\	私たちは高い山を越えました。	わたしたちはたかいやまをこえました	
\\	削る	けずる	
\\	彼は家族のために仕事の時間を削った。	かれはかぞくのためにしごとのじかんをけずった	
\\	胃	い	
\\	昨日から胃が痛い。	きのうからいがいたい	
\\	今まで	いままで	
\\	今までどこにいたのですか。	いままでどこにいたのですか	
\\	涼しい	すずしい	
\\	夕方は涼しくなりますよ。	ゆうがたはすずしくなりますよ	
\\	ペン	ぺん	
\\	ペンを貸してください。	ぺんをかしてください	
\\	鍵	かぎ	
\\	出掛ける時は鍵を掛けてください。	でかけるときはかぎをかけてください	
\\	そろそろ	そろそろ	
\\	そろそろ始めましょうか。	そろそろはじめましょうか	
\\	込む	こむ	
\\	電車が込んでいる。	でんしゃがこんでいる	
\\	引っ越す	ひっこす	
\\	来月、大阪に引っ越します。	らいげつおおさかにひっこします	
\\	小包	こづつみ	
\\	フランスの友達から小包が届いた。	ふらんすのともだちからこづつみがとどいた	
\\	踊る	おどる	
\\	彼女はクラブで踊るのが好きです。	かのじょはくらぶでおどるのがすきです	
\\	配達	はいたつ	
\\	彼は新聞配達をしている。	かれはしんぶんはいたつをしている	
\\	毎晩	まいばん	
\\	姉は毎晩日記を書いています。	あねはまいばんにっきをかいています	
\\	アルバム	あるばむ	
\\	私は彼のアルバムを見た。	わたしはかれのあるばむをみた	
\\	知らせる	しらせる	
\\	皆に会議の日にちを知らせた。	みんなにかいぎのひにちをしらせた	
\\	割れる	われる	
\\	コップが落ちて割れた。	こっぷがおちてわれた	
\\	帽子	ぼうし	
\\	暑いので帽子を被りましょう。	あついのでぼうしをかぶりましょう	
\\	封筒	ふうとう	
\\	その手紙を封筒に入れた。	そのてがみをふうとうにいれた	
\\	太る	ふとる	
\\	私の姉は直ぐ太ります。	わたしのあねはすぐふとります	
\\	歩道	ほどう	
\\	歩道を歩きましょう。	ほどうをあるきましょう	
\\	セーター	せーたー	
\\	このセーターはウールだ。	このせーたーはうーるだ	
\\	週刊誌	しゅうかんし	
\\	その週刊誌はゴシップばかりだ。	そのしゅうかんしはごしっぷばかりだ	
\\	切手	きって	
\\	ここに切手を貼ってください。	ここにきってをはってください	
\\	尊敬	そんけい	
\\	祖父は家族みんなに尊敬されています。	そふはかぞくみんなにそんけいされています	
\\	真ん中	まんなか	
\\	道の真ん中に人が立っている。	みちのまんなかにひとがたっている	
\\	空く	あく	
\\	後ろの席が空いています。	うしろのせきがあいています	
\\	ダンス	だんす	
\\	彼女はダンスが大好きです。	かのじょはだんすがだいすきです	
\\	きっと	きっと	
\\	明日はきっと雨が降ります。	あしたはきっとあめがふります	
\\	棚	たな	
\\	大きな棚はとても便利です。	おおきなたなはとてもべんりです	
\\	おかしい	おかしい	
\\	彼の話はおかしかった。	かれのはなしはおかしかった	
\\	折れる	おれる	
\\	強風で木の枝が折れた。	きょうふうできのえだがおれた	
\\	恋人	こいびと	
\\	彼は恋人を失った。	かれはこいびとをうしなった	
\\	駅員	えきいん	
\\	駅員に聞きましょう。	えきいんにききましょう	
\\	国籍	こくせき	
\\	私は日本国籍です。	わたしはにほんこくせきです	
\\	テキスト	てきすと	
\\	テキストを読んでください。	てきすとをよんでください	
\\	草	くさ	
\\	庭に草が生えている。	にわにくさがはえている	
\\	ビニール	びにーる	
\\	塵はそのビニール袋に入れてください。	ごみはそのびにーるぶくろにいれてください	
\\	水泳	すいえい	
\\	母は健康のために水泳をしている。	はははけんこうのためにすいえいをしている	
\\	気に入る	きにいる	
\\	新しい靴がとても気に入りました。	あたらしいくつがとてもきにいりました	
\\	読み	よみ	
\\	母に読み書きを習いました。	ははによみかきをならいました	
\\	宿題	しゅくだい	
\\	宿題、手伝ってくれない?	しゅくだいてつだってくれない	
\\	動物園	どうぶつえん	
\\	昨日子供たちと動物園に行きました。	きのうこどもたちとどうぶつえんにいきました	
\\	ナイフ	ないふ	
\\	ナイフで手を切った。	ないふでてをきった	
\\	ゼロ	ぜろ	
\\	今日の交通事故はゼロです。	きょうのこうつうじこはぜろです	
\\	全然	ぜんぜん	
\\	この本は全然面白くなかった。	このほんはぜんぜんおもしろくなかった	
\\	写る	うつる	
\\	このカメラはよく写りますよ。	このかめらはよくうつりますよ	
\\	メニュー	めにゅー	
\\	メニューをください。	めにゅーをください	
\\	朝刊	ちょうかん	
\\	今日の朝刊に面白い記事があった。	きょうのちょうかんにおもしろいきじがあった	
\\	茶	ちゃ	
\\	私たちは毎日お茶を飲みます。	わたしたちはまいにちおちゃをのみます	
\\	独身	どくしん	
\\	私の兄はまだ独身です。	わたしのあにはまだどくしんです	
\\	パスポート	ぱすぽーと	
\\	パスポートが見つからない。	ぱすぽーとがみつからない	
\\	緑色	みどりいろ	
\\	彼は緑色のジャケットを着ています。	かれはみどりいろのじゃけっとをきています	
\\	コート	こーと	
\\	新しいコートでテニスをしました。	あたらしいこーとでてにすをしました	
\\	お喋り	おしゃべり	
\\	妹はとてもお喋りです。	いもうとはとてもおしゃべりです	
\\	噛む	かむ	
\\	もっとよく噛みなさい。	もっとよくかみなさい	
\\	誕生日	たんじょうび	
\\	彼女の誕生日は7月16日です。	かのじょのたんじょうびはしちがつじゅうろくにちです	
\\	毛	け	
\\	猫の毛がセーターに付いた。	ねこのけがせーたーについた	
\\	注射	ちゅうしゃ	
\\	彼は注射があまり好きではありません。	かれはちゅうしゃがあまりすきではありません	
\\	冷える	ひえる	
\\	クーラーで体が冷えた。	くーらーでからだがひえた	
\\	次男	じなん	
\\	次男は今、海外にいます。	じなんはいまかいがいにいます	
\\	貯金	ちょきん	
\\	貯金は十分にあります。	ちょきんはじゅうぶんにあります	
\\	丁寧	ていねい	
\\	彼女はいつも丁寧に仕事をします。	かのじょはいつもていねいにしごとをします	
\\	少しも	すこしも	
\\	あなたは少しも悪くない。	あなたはすこしもわるくない	
\\	震える	ふるえる	
\\	彼は寒くて震えていました。	かれはさむくてふるえていました	
\\	柔道	じゅうどう	
\\	私は柔道を習っています。	わたしはじゅうどうをならっています	
\\	少々	しょうしょう	
\\	塩を少々入れてください。	しおをしょうしょういれてください	
\\	油	あぶら	
\\	水と油は混ざらない。	みずとあぶらはまざらない	
\\	オレンジ	おれんじ	
\\	私はオレンジが好きです。	わたしはおれんじがすきです	
\\	恥ずかしい	はずかしい	
\\	とても恥ずかしかった。	とてもはずかしかった	
\\	スカート	すかーと	
\\	彼女はあまりスカートは履かない。	かのじょはあまりすかーとははかない	
\\	外国語	がいこくご	
\\	外国語を習うのは難しい。	がいこくごをならうのはむずかしい	
\\	鉛筆	えんぴつ	
\\	鉛筆を貸して下さい。	えんぴつをかしてください	
\\	ドラマ	どらま	
\\	私はドラマを見るのが好きです。	わたしはどらまをみるのがすきです	
\\	素晴らしい	すばらしい	
\\	素晴らしい景色ですね。	すばらしいけしきですね	
\\	広さ	ひろさ	
\\	その家の広さはどれ位ですか。	そのいえのひろさはどれくらいですか	
\\	ゴム	ごむ	
\\	ゴムが伸びてしまった。	ごむがのびてしまった	
\\	林檎	りんご	
\\	林檎を一つください。	りんごをひとつください	
\\	舌	した	
\\	舌を出してください。	したをだしてください	
\\	着物	きもの	
\\	彼女は着物がよく似合います。	かのじょはきものがよくにあいます	
\\	速さ	はやさ	
\\	新幹線の速さはどれ位ですか。	しんかんせんのはやさはどれくらいですか	
\\	黒板	こくばん	
\\	答えを黒板に書いてください。	こたえをこくばんにかいてください	
\\	磨く	みがく	
\\	靴を磨いてください。	くつをみがいてください	
\\	幼稚園	ようちえん	
\\	娘は幼稚園に通っています。	むすめはようちえんにかよっています	
\\	お茶	おちゃ	
\\	お茶を入れましょうか。	おちゃをいれましょうか	
\\	林	はやし	
\\	私たちは林の中に入っていった。	わたしたちははやしのなかにはいっていった	
\\	我慢	がまん	
\\	彼の我が侭には我慢出来ません。	かれのわがままにはがまんできません	
\\	住所	じゅうしょ	
\\	この住所に行ってください。	このじゅうしょにいってください	
\\	洋服	ようふく	
\\	今日は洋服を買いに行きます。	きょうはようふくをかいにいきます	
\\	掃除	そうじ	
\\	週末は部屋の掃除をしました。	しゅうまつはへやのそうじをしました	
\\	積もる	つもる	
\\	雪が積もっています。	ゆきがつもっています	
\\	椅子	いす	
\\	そのお年寄りは椅子に座った。	そのおとしよりはいすにすわった	
\\	間違い	まちがい	
\\	この文には間違いがあります。	このぶんにはまちがいがあります	
\\	随分	ずいぶん	
\\	随分早く終わったね。	ずいぶんはやくおわったね	
\\	ピンク	ぴんく	
\\	娘がピンクのドレスを着ている。	むすめがぴんくのどれすをきている	
\\	果物	くだもの	
\\	デザートに果物を食べましょう。	でざーとにくだものをたべましょう	
\\	暑さ	あつさ	
\\	今年の夏は暑さが厳しい。	ことしのなつはあつさがきびしい	
\\	葡萄	ぶどう	
\\	私は葡萄が好きです。	わたしはぶどうがすきです	
\\	畳	たたみ	
\\	そのホテルには畳の部屋がある。	そのほてるにはたたみのへやがある	
\\	お祖父さん	おじいさん	
\\	私のお祖父さんは毎日散歩します。	わたしのおじいさんはまいにちさんぽします	
\\	弁当	べんとう	
\\	今日は弁当を持ってきました。	きょうはべんとうをもってきました	
\\	次女	じじょ	
\\	うちの次女は春から中学生です。	うちのじじょははるからちゅうがくせいです	
\\	曲げる	まげる	
\\	膝を曲げてください。	ひざをまげてください	
\\	貧乏	びんぼう	
\\	彼は昔は貧乏だった。	かれはむかしはびんぼうだった	
\\	郵便	ゆうびん	
\\	さっき郵便が届きました。	さっきゆうびんがとどきました	
\\	会話	かいわ	
\\	親子の会話は大切です。	おやこのかいわはたいせつです	
\\	スイッチ	すいっち	
\\	彼はカーラジオのスイッチを入れた。	かれはかーらじおのすいっちをいれた	
\\	映画館	えいがかん	
\\	彼と近くの映画館に行きました。	かれとちかくのえいがかんにいきました	
\\	指輪	ゆびわ	
\\	彼女に指輪をプレゼントしました。	かのじょにゆびわをぷれぜんとしました	
\\	見物	けんぶつ	
\\	皆で東京見物をした。	みんなでとうきょうけんぶつをした	
\\	下着	したぎ	
\\	私は下着を手で洗う。	わたしはしたぎをてであらう	
\\	ボタン	ぼたん	
\\	ボタンを押してください。	ぼたんをおしてください	
\\	正直	しょうじき	
\\	彼女はとても正直だ。	かのじょはとてもしょうじきだ	
\\	火事	かじ	
\\	火事です。119番に電話してください。	かじです。ひゃくじゅうきゅうばんにでんわしてください。	
\\	ギター	ぎたー	
\\	彼女はギターが得意です。	かのじょはぎたーがとくいです	
\\	寂しい	さびしい	
\\	これは寂しい曲ですね。	これはさびしいきょくですね	
\\	迷う	まよう	
\\	道に迷いました。	みちにまよいました	
\\	余る	あまる	
\\	夕食の料理がたくさん余った。	ゆうしょくのりょうりがたくさんあまった	
\\	柔らかい	やわらかい	
\\	布団がとても柔らかい。	ふとんがとてもやわらかい	
\\	出掛ける	でかける	
\\	主人はもう出掛けました。	しゅじんはもうでかけました	
\\	半年	はんとし	
\\	日本に来て半年になります。	にっぽんにきてはんとしになります	
\\	押さえる	おさえる	
\\	ドアを押さえてください。	どあをおさえてください	
\\	バイオリン	ばいおりん	
\\	彼女はバイオリンを習っています。	かのじょはばいおりんをならっています	
\\	冷蔵庫	れいぞうこ	
\\	飲み物は冷蔵庫にあります。	のみものはれいぞうこにあります	
\\	鏡	かがみ	
\\	彼女は鏡を見て髪を直した。	かのじょはかがみをみてかみをなおした	
\\	敬語	けいご	
\\	お客様には敬語を使いなさい。	おきゃくさまにはけいごをつかいなさい	
\\	下りる	おりる	
\\	そこの階段を下りてください。	そこのかいだんをおりてください	
\\	ポケット	ぽけっと	
\\	財布をポケットに仕舞った。	さいふをぽけっとにしまった	
\\	水道	すいどう	
\\	東京は水道の水が不味い。	とうきょうはすいどうのみずがまずい	
\\	飛行場	ひこうじょう	
\\	バスが飛行場に着きました。	ばすがひこうじょうにつきました	
\\	踊り	おどり	
\\	この踊りは易しいですよ。	このおどりはやさしいですよ	
\\	新年	しんねん	
\\	新年明けましておめでとうございます。	しんねんあけましておめでとうございます	
\\	プレゼント	ぷれぜんと	
\\	誕生日にプレゼントを貰いました。	たんじょうびにぷれぜんとをもらいました	
\\	この間	このあいだ	
\\	この間彼女に会った。	このあいだかのじょにあった	
\\	可愛い	かわいい	
\\	彼女の赤ちゃんは可愛いです。	かのじょのあかちゃんはかわいいです	
\\	苦手	にがて	
\\	私は料理が苦手です。	わたしはりょうりがにがてです	
\\	傘	かさ	
\\	電車に傘を忘れた。	でんしゃにかさをわすれた	
\\	見送り	みおくり	
\\	駅に友人の見送りに行きました。	えきにゆうじんのみおくりにいきました	
\\	日	ひ	
\\	夏は日が長い。	なつはひがながい	
\\	禁煙	きんえん	
\\	彼は今、禁煙しています。	かれはいまきんえんしています	
\\	幾つ	いくつ	
\\	娘さんは幾つになりましたか。	むすめさんはいくつになりましたか	
\\	きつい	きつい	
\\	このシャツは少しきついです。	このしゃつはすこしきついです	
\\	さっき	さっき	
\\	さっきの話を続けましょう。	さっきのはなしをつづけましょう	
\\	あなた	あなた	
\\	これはあなたの鞄ですか。	これはあなたのかばんですか	
\\	皆さん	みなさん	
\\	皆さんにお話があります。	みなさんにおはなしがあります	
\\	取れる	とれる	
\\	シャツのボタンが取れた。	しゃつのぼたんがとれた	
\\	固い	かたい	
\\	私の上司は頭が固い。	わたしのじょうしはあたまがかたい	
\\	神社	じんじゃ	
\\	京都には神社がたくさんある。	きょうとにはじんじゃがたくさんある	
\\	こちら	こちら	
\\	受付はこちらです。	うけつけはこちらです	
\\	受け付ける	うけつける	
\\	郵便物は5時まで受け付けています。	ゆうびんぶつはごじまでうけつけています	
\\	テニス	てにす	
\\	彼らはよくテニスをしています。	かれらはよくてにすをしています	
\\	電池	でんち	
\\	新しい電池を入れましょう。	あたらしいでんちをいれましょう	
\\	やっぱり	やっぱり	
\\	やっぱり旅が大好きだ。	やっぱりたびがだいすきだ	
\\	救急車	きゅうきゅうしゃ	
\\	誰か救急車を呼んでください。	だれかきゅうきゅうしゃをよんでください	
\\	クリスマス	くりすます	
\\	クリスマスにはケーキを食べます。	くりすますにはけーきをたべます	
\\	引っ越し	ひっこし	
\\	去年、引っ越ししました。	きょねんひっこししました	
\\	いつでも	いつでも	
\\	いつでもうちに来てください。	いつでもうちにきてください	
\\	鳴る	なる	
\\	今朝、5時に電話が鳴った。	けさごじにでんわがなった	
\\	お洒落	おしゃれ	
\\	あの子はお洒落だね。	あのこはおしゃれだね	
\\	申し込む	もうしこむ	
\\	彼女はそのセミナーに申し込んだ。	かのじょはそのせみなーにもうしこんだ	
\\	ビザ	びざ	
\\	学生ビザを持っています。	がくせいびざをもっています	
\\	ジェット機	じぇっとき	
\\	ジェット機が飛んでいる。	じぇっときがとんでいる	
\\	夕刊	ゆうかん	
\\	そのニュースは夕刊で見ました。	そのにゅーすはゆうかんでみました	
\\	毎朝	まいあさ	
\\	私は毎朝ジョギングをします。	わたしはまいあさじょぎんぐをします	
\\	グラス	ぐらす	
\\	これは綺麗なグラスですね。	これはきれいなぐらすですね	
\\	牛乳	ぎゅうにゅう	
\\	私は毎朝、牛乳を飲む。	わたしはまいあさぎゅうにゅうをのむ	
\\	店員	てんいん	
\\	あの店員はとても親切です。	あのてんいんはとてもしんせつです	
\\	真面目	まじめ	
\\	彼は真面目な人です。	かれはまじめなひとです	
\\	ネクタイ	ねくたい	
\\	父の日にネクタイをプレゼントした。	ちちのひにねくたいをぷれぜんとした	
\\	いつか	いつか	
\\	僕はいつかアフリカに行きたい。	ぼくはいつかあふりかにいきたい	
\\	植える	うえる	
\\	庭に薔薇を植えました。	にわにばらをうえました	
\\	喫茶店	きっさてん	
\\	喫茶店でコーヒーを飲んだ。	きっさてんでこーひーをのんだ	
\\	物	もの	
\\	彼女の家には物がたくさんあります。	かのじょのいえにはものがたくさんあります	
\\	カーテン	かーてん	
\\	朝、カーテンを開けた。	あさかーてんをあけた	
\\	欠席	けっせき	
\\	風邪のため今日は欠席します。	かぜのためきょうはけっせきします	
\\	厚さ	あつさ	
\\	私は板の厚さを測った。	わたしはいたのあつさをはかった	
\\	引っ張る	ひっぱる	
\\	娘が私の手を引っ張った。	むすめがわたしのてをひっぱった	
\\	焼ける	やける	
\\	肉が焼けました。	にくがやけました	
\\	先週	せんしゅう	
\\	先週は海に行った。	せんしゅうはうみにいった	
\\	アイスクリーム	いすくりーむ	
\\	弟はアイスクリームが大好きです。	おとうとはあいすくりーむがだいすきです	
\\	シャツ	しゃつ	
\\	このシャツはアイロンが必要だ。	このしゃつはあいろんがひつようだ	
\\	別れる	わかれる	
\\	駅で友だちと別れました。	えきでともだちとわかれました	
\\	滅多に	めったに	
\\	彼女は滅多に怒りません。	かのじょはめったにおこりません	
\\	握手	あくしゅ	
\\	彼らは握手をした。	かれらはあくしゅをした	
\\	切符	きっぷ	
\\	東京までの切符を買った。	とうきょうまでのきっぷをかった	
\\	間違える	まちがえる	
\\	電話番号を間違えました。	でんわばんごうをまちがえました	
\\	ジュース	じゅーす	
\\	このジュースは甘すぎる。	このじゅーすはあますぎる	
\\	バッグ	ばっぐ	
\\	私は黒いバッグを持っています。	わたしはくろいばっぐをもっています	
\\	バイク	ばいく	
\\	兄はバイクが大好きです。	あにはばいくがだいすきです	
\\	丸い	まるい	
\\	地球は丸い。	ちきゅうはまるい	
\\	冷房	れいぼう	
\\	冷房を入れてください。	れいぼうをいれてください	
\\	危ない	あぶない	
\\	その道は車が多くて危ない。	そのみちはくるまがおおくてあぶない	
\\	預かる	あずかる	
\\	荷物を預かってください。	にもつをあずかってください	
\\	理科	りか	
\\	私は理科が得意です。	わたしはりかがとくいです	
\\	腐る	くさる	
\\	林檎が腐っている。	りんごがくさっている	
\\	苦い	にがい	
\\	私は苦いコーヒーが好きです。	わたしはにがいこーひーがすきです	
\\	登る	のぼる	
\\	私たちは昨年、富士山に登りました。	わたしたちはさくねんふじさんにのぼりました	
\\	出口	でぐち	
\\	出口はあそこです。	でぐちはあそこです	
\\	ワイシャツ	わいしゃつ	
\\	彼は白いワイシャツを着ている。	かれはしろいわいしゃつをきている	
\\	スプーン	すぷーん	
\\	カレーライスはスプーンで食べます。	かれーらいすはすぷーんでたべます	
\\	ラーメン	らーめん	
\\	夕食にラーメンを食べました。	ゆうしょくにらーめんをたべました	
\\	偉い	えらい	
\\	彼は偉い学者です。	かれはえらいがくしゃです	
\\	履く	はく	
\\	彼女はブーツを履いています。	かのじょはぶーつをはいています	
\\	きらきら	きらきら	
\\	星がきらきら光っている。	ほしがきらきらひかっている	
\\	踏切	ふみきり	
\\	その踏切は長い。	そのふみきりはながい	
\\	ビデオテープ	びでおてーぷ	
\\	私はビデオテープを30本持っています。	わたしはびでおてーぷをさんじゅっぽんもっています	
\\	着せる	きせる	
\\	娘に可愛いドレスを着せた。	むすめにかわいいどれすをきせた	
\\	ブラウス	ぶらうす	
\\	彼女は白いブラウスを着ている。	かのじょはしろいぶらうすをきている	
\\	梅雨	つゆ	
\\	梅雨は6月頃です。	つゆはろくがつごろです	
\\	億	おく	
\\	世界人口は65億人だ。	せかいじんこうはろくじゅうごおくにんだ	
\\	一生懸命	いっしょうけんめい	
\\	彼は毎日一生懸命働いている。	かれはまいにちいっしょうけんめいはたらいている	
\\	美人	びじん	
\\	彼のお母さんは美人です。	かれのおかあさんはびじんです	
\\	エスカレーター	えすかれーたー	
\\	3階までエスカレーターで行きましょう。	さんかいまでえすかれーたーでいきましょう	
\\	ミルク	みるく	
\\	赤ちゃんにミルクを上げた。	あかちゃんにみるくをあげた	
\\	クーラー	くーらー	
\\	暑いのでクーラーをつけました。	あついのでくーらーをつけました	
\\	下り	くだり	
\\	もう直ぐ下りの電車が出発します。	もうすぐくだりのでんしゃがはっしゃします	
\\	早口	はやくち	
\\	彼女は早口だ。	かのじょははやくちだ	
\\	嘘	うそ	
\\	嘘をついてはいけません。	うそをついてはいけません	
\\	足す	たす	
\\	母は味噌汁に水を足した。	はははみそしるにみずをたした	
\\	ハンドバッグ	はんどばっぐ	
\\	ハンドバッグを忘れました。	はんどばっぐをわすれました	
\\	滑る	すべる	
\\	彼は雪の上を滑った。	かれはゆきのうえをすべった	
\\	腕時計	うでどけい	
\\	私の腕時計は遅れている。	わたしのうでどけいはおくれている	
\\	従兄弟	いとこ	
\\	従兄弟と私は同じ年です。	いとことわたしはおなじとしです	
\\	塵	ごみ	
\\	今日は塵の日だ。	きょうはごみのひだ	
\\	可哀相	かわいそう	
\\	その可哀相な子供たちは食べるものがない。	そのかわいそうなこどもたちはたべるものがない	
\\	ベル	べる	
\\	玄関のベルが鳴った。	げんかんのべるがなった	
\\	冷める	さめる	
\\	彼への気持ちが冷めた。	かれへのきもちがさめた	
\\	掲示板	けいじばん	
\\	掲示板のお知らせを見ましたか。	けいじばんのおしらせをみましたか	
\\	貸し出す	かしだす	
\\	その本は貸し出し中です。	そのほんはかしだしちゅうです	
\\	昼寝	ひるね	
\\	私の子供は毎日昼寝をします。	わたしのこどもはまいにちひるねをします	
\\	饂飩	うどん	
\\	私はお昼に饂飩を食べました。	わたしはおひるにうどんをたべました	
\\	洗濯機	せんたっき	
\\	新しい洗濯機を買いました。	あたらしいせんたっきをかいました	
\\	年賀状	ねんがじょう	
\\	昨日、年賀状を出しました。	きのうねんがじょうをだしました	
\\	ラケット	らけっと	
\\	テニスのラケットを買いました。	てにすのらけっとをかいました	
\\	冗談	じょうだん	
\\	冗談は止めてください。	じょうだんはやめてください	
\\	お嬢さん	おじょうさん	
\\	お嬢さんはお幾つですか。	おじょうさんはおいくつですか	
\\	交差点	こうさてん	
\\	次の交差点を左に曲がってください。	つぎのこうさてんをひだりにまがってください	
\\	ベルト	べると	
\\	ベルトがきつくなりました。	べるとがきつくなりました	
\\	靴下	くつした	
\\	この靴下は3足で1000円です。	このくつしたはさんそくでせんえんです	
\\	破れる	やぶれる	
\\	シャツが破れている。	しゃつがやぶれている	
\\	パトカー	ぱとかー	
\\	あそこにパトカーがいる。	あそこにぱとかーがいる	
\\	本屋	ほんや	
\\	駅前に本屋があります。	えきまえにほんやがあります	
\\	サッカー	さっかー	
\\	彼はサッカーの選手です。	かれはさっかーのせんしゅです	
\\	再来年	さらいねん	
\\	次のオリンピックは再来年です。	つぎのおりんぴっくはさらいねんです	
\\	毛布	もうふ	
\\	この毛布は暖かい。	このもうふはあたたかい	
\\	手帳	てちょう	
\\	新しい手帳を買いました。	あたらしいてちょうをかいました	
\\	チャンネル	ちゃんねる	
\\	テレビのチャンネルを変えてください。	てれびのちゃんねるをかえてください	
\\	裏返す	うらがえす	
\\	彼は紙を裏返しました。	かれはかみをうらがえしました	
\\	沸く	わく	
\\	お風呂が沸きました。	おふろがわきました	
\\	辞書	じしょ	
\\	彼女は辞書をよく使います。	かのじょはじしょをよくつかいます	
\\	失礼	しつれい	
\\	ではそろそろ失礼します。	ではそろそろしつれいします	
\\	ハンバーガー	はんばーがー	
\\	今日の昼ご飯はハンバーガーでした。	きょうのひるごはんははんばーがーでした	
\\	乾杯	かんぱい	
\\	お二人の未来に乾杯しましょう。	おふたりのみらいにかんぱいしましょう	
\\	止める	とめる	
\\	車を止めて。	くるまをとめて	
\\	上着	うわぎ	
\\	暑いので上着を脱ぎました。	あついのでうわぎをぬぎました	
\\	ソース	そーす	
\\	ソースはどれですか。	そーすはどれですか	
\\	改札口	かいさつぐち	
\\	改札口で会いましょう。	かいさつぐちであいましょう	
\\	スープ	すーぷ	
\\	母がコーンスープを作っている。	ははがこーんすーぷをつくっている	
\\	乗り物	のりもの	
\\	自転車は便利な乗り物です。	じてんしゃはべんりなのりものです	
\\	不便	ふべん	
\\	私の家は駅から遠くて不便です。	わたしのいえはえきからとおくてふべんです	
\\	シャワー	しゃわー	
\\	朝、急いでシャワーを浴びました。	あさいそいでしゃわーをあびました	
\\	伯父-叔父	おじ	
\\	叔父は銀行に勤めています。	おじはぎんこうにつとめています	
\\	手袋	てぶくろ	
\\	寒いので手袋をしました。	さむいのでてぶくろをしました	
\\	事務室	じむしつ	
\\	事務室でコピーを取って来ます。	じむしつでこぴーをとってきます	
\\	タイプライター	たいぷらいたー	
\\	母は古いタイプライターを持っています。	はははふるいたいぷらいたーをもっています	
\\	今朝	けさ	
\\	今朝テレビでそのニュースを知りました。	けさてれびでそのにゅーすをしりました	
\\	暖房	だんぼう	
\\	冬は暖房が必要です。	ふゆはだんぼうがひつようです	
\\	蜜柑	みかん	
\\	蜜柑を1つください。	みかんをひとつください	
\\	バナナ	ばなな	
\\	私は毎朝バナナを食べます。	わたしはまいあさばななをたべます	
\\	天気予報	てんきよほう	
\\	明日の天気予報は雨です。	あしたのてんきよほうはあめです	
\\	折る	おる	
\\	祖父は足の骨を折りました。	そふはあしのほねをおりました	
\\	テープレコーダー	てーぷれこーだー	
\\	テープレコーダーで自分の声を録音しました。	てーぷれこーだーでじぶんのこえをろくおんしました	
\\	鳴く	なく	
\\	どこかで猫が鳴いている。	どこかでねこがないている	
\\	うるさい(煩い)	うるさい	
\\	この通りは車の音がうるさい。	このとおりはくるまのおとがうるさい	
\\	紅茶	こうちゃ	
\\	温かい紅茶が飲みたい。	あたたかいこうちゃがのみたい	
\\	都合	つごう	
\\	今日は都合が悪くて行けません。	きょうはつごうがわるくていけません	
\\	ナイロン	ないろん	
\\	このジャケットはナイロンで出切ています。	このじゃけっとはないろんでできています	
\\	田んぼ	たんぼ	
\\	この辺は田んぼがたくさんあります。	このへんはたんぼがたくさんあります	
\\	ライター	らいたー	
\\	店にライターを忘れました。	みせにらいたーをわすれました	
\\	年下	としした	
\\	彼は奥さんより年下です。	かれはおくさんよりとししたです	
\\	自炊	じすい	
\\	私は毎日、自炊している。	わたしはまいにちじすいしている	
\\	片道	かたみち	
\\	東京まで片道切符を買った。	とうきょうまでかたみちきっぷをかった	
\\	来週	らいしゅう	
\\	続きは来週やりましょう。	つづきはらいしゅうやりましょう	
\\	謝る	あやまる	
\\	彼は直ぐに謝りました。	かれはすぐにあやまりました	
\\	お礼	おれい	
\\	彼女にお礼の手紙を書きました。	かのじょにおれいのてがみをかきました	
\\	遅刻	ちこく	
\\	遅刻しないでください。	ちこくしないでください	
\\	用事	ようじ	
\\	父は用事で出掛けています。	ちちはようじででかけています	
\\	脱ぐ	ぬぐ	
\\	靴を脱いでください。	くつをぬいでください	
\\	交通費	こうつうひ	
\\	会社までの交通費は一ヶ月8,000円です。	かいしゃまでのこうつうひはいっかげつはっせんえんです	
\\	乗り換える	のりかえる	
\\	次の駅で地下鉄に乗り換えます。	つぎのえきでちかてつにのりかえます	
\\	邪魔	じゃま	
\\	邪魔です、どいてください。	じゃまですどいてください	
\\	トマト	とまと	
\\	私はトマトが大好きです。	わたしはとまとがだいすきです	
\\	留守	るす	
\\	父は今、留守です。	ちちはいまるすです	
\\	停車	ていしゃ	
\\	この電車は東京まで停車致しません。	このでんしゃはとうきょうまでていしゃいたしません	
\\	寿司	すし	
\\	彼女は寿司を初めて食べました。	かのじょはすしをはじめてたべました	
\\	コップ	こっぷ	
\\	私は毎朝コップ一杯の水を飲む。	わたしはまいあさこっぷいっぱいのみずをのむ	
\\	インク	いんく	
\\	プリンターのインクを買った。	ぷりんたーのいんくをかった	
\\	チケット	ちけっと	
\\	この遊園地のチケットは3000円です。	このゆうえんちのちけっとはさんぜんえんです	
\\	パチンコ	ぱちんこ	
\\	彼は毎日パチンコをしています。	かれはまいにちぱちんこをしています	
\\	電話帳	でんわちょう	
\\	電話帳を見せてください。	でんわちょうをみせてください	
\\	蚊	か	
\\	蚊に足を刺された。	かにあしをさされた	
\\	真っ白	まっしろ	
\\	外は雪で真っ白だった。	そとはゆきでまっしろだった	
\\	ぶらぶら	ぶらぶら	
\\	朝、近所をぶらぶらした。	あさきんじょをぶらぶらした	
\\	レモン	れもん	
\\	紅茶にレモンを入れて飲んだ。	こうちゃにれもんをいれてのんだ	
\\	記念日	きねんび	
\\	今日は両親の結婚記念日です。	きょうはりょうしんのけっこんきねんびです	
\\	アナウンサー	あなうんさー	
\\	私はアナウンサーになりたい。	わたしはあなうんさーになりたい	
\\	役に立つ	やくにたつ	
\\	私は人々の役に立ちたいと思っています。	わたしはひとびとのやくにたちたいとおもっています	
\\	昼休み	ひるやすみ	
\\	昼休みに公園に行った。	ひるやすみにこうえんにいった	
\\	追い越す	おいこす	
\\	大きなトラックが私たちを追い越した。	おおきなとらっくがわたしたちをおいこした	
\\	スチュワーデス	すちゅわーです	
\\	彼女はスチュワーデスになった。	かのじょはすちゅわーですになった	
\\	おかず	おかず	
\\	晩ご飯のおかずは何?	ばんごはんのおかずはなに	
\\	いかが(如何)	いかが	
\\	お加減はいかがですか。	おかげんはいかがですか	
\\	遠慮	えんりょ	
\\	私は遠慮します。	わたしはえんりょします	
\\	皿	さら	
\\	空いたお皿をお下げします。	あいたおさらをおさげします	
\\	乾く	かわく	
\\	夏は洗濯物が直ぐ乾く。	なつはせんたくものがすぐかわく	
\\	訳	やく	
\\	その文の訳を読みました。	そのぶんのやくをよみました	
\\	サラダ	さらだ	
\\	サラダをたくさん食べました。	さらだをたくさんたべました	
\\	交番	こうばん	
\\	あそこの交番で道を聞きましょう。	あそこのこうばんでみちをききましょう	
\\	叔父さん、伯父さん	おじさん	
\\	昨日伯父さんに会いました。	きのうおじさんにあいました	
\\	三角	さんかく	
\\	紙を三角に切りました。	かみをさんかくにきりました	
\\	真っ直ぐ	まっすぐ	
\\	この道を真っ直ぐ行ってください。	このみちをまっすぐいってください	
\\	マッチ	まっち	
\\	マッチを持っていますか。	まっちをもっていますか	
\\	豆腐	とうふ	
\\	私は毎日豆腐を食べます。	わたしはまいにちとうふをたべます	
\\	仮名	かな	
\\	お名前に仮名を振ってください。	おなまえにかなをふってください	
\\	治す	なおす	
\\	早く風邪を治してください。	はやくかぜをなおしてください	
\\	西暦	せいれき	
\\	西暦1964年に東京でオリンピックがあった。	せいれきせんきゅうひゃくろくじゅうよねんにとうきょうでおりんぴっくがあった	
\\	急行	きゅうこう	
\\	ちょうど急行電車が来た。	ちょうどきゅうこうでんしゃがきた	
\\	ストーブ	すとーぶ	
\\	うちではまだストーブを使っています。	うちではまだすとーぶをつかっています	
\\	近付く	ちかづく	
\\	女性が近付いてきました。	じょせいがちかづいてきました	
\\	干す	ほす	
\\	母は洗濯物を干しています。	はははせんたくものをほしています	
\\	大分	だいぶ	
\\	大分ピアノが上手くなりました。	だいぶぴあのがうまくなりました	
\\	花見	はなみ	
\\	友達と花見をしました。	ともだちとはなみをしました	
\\	四季	しき	
\\	日本には四季がある。	にほんにはしきがある	
\\	布団	ふとん	
\\	母が布団を干している。	ははがふとんをほしている	
\\	箸	はし	
\\	箸を上手に使えるよ。	はしをじょうずにつかえるよ	
\\	ガソリンスタンド	がそりんすたんど	
\\	この近くにガソリンスタンドはありますか。	このちかくにがそりんすたんどはありますか	
\\	書き方	かきかた	
\\	彼はその漢字の書き方が分からない。	かれはそのかんじのかきかたがわからない	
\\	変	へん	
\\	変な音が聞こえます。	へんなおとがきこえます	
\\	押し入れ	おしいれ	
\\	布団を押し入れにしまいました。	ふとんをおしいれにしまいました	
\\	居る	いる	
\\	今日は一日中家に居ました。	きょうはいちにちじゅううちにいました	
\\	八百屋	やおや	
\\	八百屋で蜜柑を買いました。	やおやでみかんをかいました	
\\	ケーキ	けーき	
\\	誕生日にケーキを食べました。	たんじょうびにけーきをたべました	
\\	間違う	まちがう	
\\	あなたは間違っている。	あなたはまちがっている	
\\	下ろす	おろす	
\\	棚からその箱を下ろしてください。	たなからそのはこをおろしてください	
\\	フォーク	ふぉーく	
\\	フォークを貰えますか。	ふぉーくをもらえますか	
\\	ぐっすり	ぐっすり	
\\	赤ちゃんがぐっすり寝ている。	あかちゃんがぐっすりねている	
\\	炊く	たく	
\\	母は毎朝ご飯を炊く。	はははまいあさごはんをたく	
\\	詰まらない	つまらない	
\\	彼の話は詰まらないですね。	かれのはなしはつまらないですね	
\\	真っ黒	まっくろ	
\\	インクで手が真っ黒になった	いんくでてがまっくろになった	
\\	ドライブ	どらいぶ	
\\	今日は群馬までドライブしました。	きょうはぐんままでどらいぶしました	
\\	皆様	みなさま	
\\	皆様、こんにちは。	みなさまこんにちは	
\\	高等学校	こうとうがっこう	
\\	弟が高等学校を卒業しました。	おとうとがこうとうがっこうをそつぎょうしました	
\\	怒る	おこる	
\\	彼女が嘘をついたので、彼は怒った。	かのじょがうそをついたのでかれはおこった	
\\	引き出し	ひきだし	
\\	財布は引き出しの中にあります。	さいふはひきだしのなかにあります	
\\	真っ暗	まっくら	
\\	外は真っ暗です。	そとはまっくらです	
\\	バター	ばたー	
\\	パンにバターを塗って食べました。	ぱんにばたーをぬってたべました	
\\	扇風機	せんぷうき	
\\	暑いから扇風機をつけよう。	あついからせんぷうきをつけよう	
\\	どうぞ	どうぞ	
\\	こちらへどうぞ。	こちらへどうぞ	
\\	リボン	りぼん	
\\	プレゼントにリボンを付けた。	ぷれぜんとにりぼんをつけた	
\\	バレーボール	ばれーぼーる	
\\	妹はバレーボールが得意です。	いもうとはばれーぼーるがとくいです	
\\	カレンダー	かれんだー	
\\	カレンダーに予定を書いた。	かれんだーによていをかいた	
\\	消防署	しょうぼうしょ	
\\	この町には消防署が1つしかありません。	このまちにはしょうぼうしょがひとつしかありません	
\\	向こう側	むこうがわ	
\\	私の家は川の向こう側にあります。	わたしのいえはかわのむこうがわにあります	
\\	受付	うけつけ	
\\	受付は9時からです。	うけつけはくじからです	
\\	角	かど	
\\	次の角で左に曲がってください。	つぎのかどでひだりにまがってください	
\\	スーツ	すーつ	
\\	あのスーツはそんなに高くない。	あのすーつはそんなにたかくない	
\\	凍る	こおる	
\\	寒い朝は道路が凍ります。	さむいあさはどうろがこおります	
\\	悲しむ	かなしむ	
\\	父は友だちの死を悲しんでいます。	ちちはともだちのしをかなしんでいます	
\\	ハンカチ	はんかち	
\\	ハンカチで手を拭きました。	はんかちでてをふきました	
\\	化粧	けしょう	
\\	彼女は化粧が上手い。	かのじょはけしょうがうまい	
\\	タオル	たおる	
\\	私はタオルで顔を拭いた。	わたしはたおるでかおをふいた	
\\	釣る	つる	
\\	昨日大きな魚を釣りました。	きのうおおきなさかなをつりました	
\\	革	かわ	
\\	革のベルトを買いました。	かわのべるとをかいました	
\\	ステレオ	すてれお	
\\	彼はステレオで音楽を聞いた。	かれはすてれおでおんがくをきいた	
\\	売店	ばいてん	
\\	駅の売店で雑誌を買った。	えきのばいてんでざっしをかった	
\\	スーパーマーケット	すーぱーまーけっと	
\\	スーパーマーケットで人参を買った。	すーぱーまーけっとでにんじんをかった	
\\	転ぶ	ころぶ	
\\	お祖母ちゃんがお風呂場で転んだ。	おばあちゃんがおふろばでころんだ	
\\	代える	かえる	
\\	社長に代えて部長を出席させます。	しゃちょうにかえてぶちょうをしゅっせきさせます	
\\	写す	うつす	
\\	彼は友達の答えを写した。	かれはともだちのこたえをうつした	
\\	売り切れる	うりきれる	
\\	その本は直ぐ売り切れた。	そのほんはすぐうりきれた	
\\	家内	かない	
\\	家内は九州出身です。	かないはきゅうしゅうしゅっしんです	
\\	チーズ	ちーず	
\\	チーズを一切れ食べました。	ちーずをひときれたべました	
\\	消防車	しょうぼうしゃ	
\\	消防車が4台も来た。	しょうぼうしゃがよんだいもきた	
\\	髭	ひげ	
\\	父は髭を生やしています。	ちちはひげをはやしています	
\\	贈り物	おくりもの	
\\	素敵な贈り物をどうも有り難う。	すてきなおくりものをどうもありがとう	
\\	冷やす	ひやす	
\\	頭を冷やしなさい。	あたまをひやしなさい	
\\	発車	はっしゃ	
\\	バスが発車します。	ばすがはっしゃします	
\\	覚める	さめる	
\\	今朝は6時に目が覚めた。	けさはろくじにめがさめた	
\\	チョコレート	ちょこれーと	
\\	妹はチョコレートが大好きです。	いもうとはちょこれーとがだいすきです	
\\	泥棒	どろぼう	
\\	近所に泥棒が入った。	きんじょにどろぼうがはいった	
\\	襖	ふすま	
\\	襖を閉めてください。	ふすまをしめてください	
\\	叔母、伯母	おば	
\\	伯母は大阪に住んでいます。	おばはおおさかにすんでいます	
\\	片付ける	かたづける	
\\	早く部屋を片付けなさい。	はやくへやをかたづけなさい	
\\	本棚	ほんだな	
\\	これはとても大きな本棚ですね。	これはとてもおおきなほんだなですね	
\\	今夜	こんや	
\\	今夜は月がとても綺麗です。	こんやはつきがとてもきれいです	
\\	デート	でーと	
\\	デートで遊園地に行きました。	でーとでゆうえんちにいきました	
\\	そちら(其方)	そちら	
\\	夕方そちらに着きます。	ゆうがたそちらにつきます	
\\	四角	しかく	
\\	紙を四角に切ってください。	かみをしかくにきってください	
\\	桃	もも	
\\	私の一番好きな果物は桃です。	わたしのいちばんすきなくだものはももです	
\\	歯医者	はいしゃ	
\\	私は歯医者が嫌いです。	わたしははいしゃがきらいです	
\\	色々	いろいろ	
\\	彼は色々なことを知っている。	かれはいろいろなことをしっている	
\\	降ろす	おろす	
\\	彼は車から荷物を降ろした。	かれはくるまからにもつをおろした	
\\	歯磨き	はみがき	
\\	歯磨きはしましたか。	はみがきはしましたか	
\\	ボールペン	ぼーるぺん	
\\	ボールペンで名前を書いてください。	ぼーるぺんでなまえをかいてください	
\\	カップ	かっぷ	
\\	水を1カップ入れてください。	みずをいちかっぷいれてください	
\\	ぐらぐら	ぐらぐら	
\\	地震で家がぐらぐらと揺れた。	じしんでいえがぐらぐらとゆれた	
\\	停留所	ていりゅうじょ	
\\	バスの停留所で10分待ちました。	ばすのていりゅうじょでじゅっぷんまちました	
\\	早起き	はやおき	
\\	祖父は早起きです。	そふははやおきです	
\\	晴れ	はれ	
\\	明日の天気は晴れです。	あすのてんきははれです	
\\	東	ひがし	
\\	日は東から上ります。	ひはひがしからのぼります	
\\	黄色	きいろ	
\\	信号は黄色でした。	しんごうはきいろでした	
\\	鋏	はさみ	
\\	この鋏はよく切れる。	このはさみはよくきれる	
\\	突き当たる	つきあたる	
\\	突き当たったら右に曲がってください。	つきあたったらみぎにまがってください	
\\	二階	にかい	
\\	兄は二階にいます。	あにはにかいにいます	
\\	明日	あした	
\\	では、また明日。	ではまたあした	
\\	日陰	ひかげ	
\\	暑いので日陰で休みましょう。	あついのでひかげでやすみましょう	
\\	無くす	なくす	
\\	今日、鍵を無くしました。	きょうかぎをなくしました	
\\	文房具	ぶんぼうぐ	
\\	新しい文房具を買いました。	あたらしいぶんぼうぐをかいました	
\\	寝坊	ねぼう	
\\	今朝は寝坊しました。	けさはねぼうしました	
\\	美容院	びよういん	
\\	父は美容院で髪を切ります。	ちちはびよういんでかみをきります	
\\	読み方	よみかた	
\\	この漢字の読み方を教えてください。	このかんじのよみかたをおしえてください	
\\	忘れ物	わすれもの	
\\	学校に忘れ物をしました。	がっこうにわすれものをしました	
\\	どなた(何方)	どなた	
\\	失礼ですが、どなたですか。	しつれいですがどなたですか	
\\	真っ青	まっさお	
\\	空が真っ青です。	そらがまっさおです	
\\	ウール	うーる	
\\	このウールのセーターは暖かい。	このうーるのせーたーはあたたかい	
\\	ジャム	じゃむ	
\\	この苺でジャムを作りましょう。	このいちごでじゃむをつくりましょう	
\\	飴	あめ	
\\	缶に飴が入っています。	かんにあめがはいっています	
\\	柔らか	やわらか	
\\	彼の声は柔らかだ。	かれのこえはやわらかだ	
\\	掛け算	かけざん	
\\	弟は掛け算を習っている。	おとうとはかけざんをならっている	
\\	サンドイッチ	さんどいっち	
\\	今日のお昼はサンドイッチです。	きょうのおひるはさんどいっちです	
\\	冬休み	ふゆやすみ	
\\	冬休みにお祖父ちゃんの家に行きます。	ふゆやすみにおじいちゃんのうちにいきます	
\\	床屋	とこや	
\\	昨日床屋で髪を切った。	きのうとこやでかみをきった	
\\	トイレットペーパー	といれっとぺーぱー	
\\	トイレットペーパーを交換しました。	といれっとぺーぱーをこうかんしました	
\\	四角い	しかくい	
\\	こっちの四角いテーブルを買おうよ。	こっちのしかくいてーぶるをかおうよ	
\\	怠ける	なまける	
\\	怠けていないで、手伝って。	なまけていないでてつだって	
\\	鍋	なべ	
\\	鍋にスープが入っているよ。	なべにすーぷがはいっているよ	
\\	入場券	にゅうじょうけん	
\\	入場券は3000円です。	にゅうじょうけんはさんぜんえんです	
\\	誘う	さそう	
\\	彼女をデートに誘った。	かのじょをでーとにさそった	
\\	ピクニック	ぴくにっく	
\\	今日はピクニックに行きましょう。	きょうはぴくにっくにいきましょう	
\\	どっち(何方)	どっち	
\\	どっちの色が好きですか。	どっちのいろがすきですか	
\\	掛かる	かかる	
\\	壁に大きな時計が掛かっています。	かべにおおきなとけいがかかっています	
\\	酔っ払い	よっぱらい	
\\	彼はただの酔っ払いです。	かれはただのよっぱらいです	
\\	たくさん(沢山)	たくさん	
\\	昨日はたくさん泳ぎました。	きのうはたくさんおよぎました	
\\	ガールフレンド	がーるふれんど	
\\	昔は彼女のことをガールフレンドと言っていました。	むかしはかのじょのことをがーるふれんどといっていました	
\\	畑	はたけ	
\\	母は畑仕事が好きだ。	ははははたけしごとがすきだ	
\\	図書室	としょしつ	
\\	図書室で勉強した。	としょしつでべんきょうした	
\\	まあまあ	まあまあ	
\\	彼の成績はまあまあです。	かれのせいせきはまあまあです	
\\	灰皿	はいざら	
\\	灰皿をください。	はいざらをください	
\\	嫌	いや	
\\	私は待つのが嫌だ。	わたしはまつのがいやだ	
\\	アクセサリー	あくせさりー	
\\	このアクセサリーは素敵ね。	このあくせさりーはすてきね	
\\	大抵	たいてい	
\\	朝食は大抵7時頃に食べます。	ちょうしょくはたいていしちじごろにたべます	
\\	下る	くだる	
\\	小さな船が川を下っています。	ちいさなふねがかわをくだっています	
\\	蕎麦	そば	
\\	お昼に蕎麦を食べた。	おひるにそばをたべた	
\\	スケート	すけーと	
\\	湖でスケートをした。	みずうみですけーとをした	
\\	キャッシュカード	きゃっしゅかーど	
\\	キャッシュカードでお金を下ろした。	きゃっしゅかーどでおかねをおろした	
\\	バス停	ばすてい	
\\	次のバス停で降ります。	つぎのばすていでおります	
\\	塩	しお	
\\	もうちょっと塩を入れて。	もうちょっとしおをいれて	
\\	ボーイフレンド	ぼーいふれんど	
\\	私のボーイフレンドはハンサムな方でした。	わたしのぼーいふれんどははんさむなかたでした	
\\	パジャマ	ぱじゃま	
\\	このパジャマを着て。	このぱじゃまをきて	
\\	イヤリング	いやりんぐ	
\\	彼女は素敵なイヤリングをしている。	かのじょはすてきないやりんぐをしている	
\\	曇り	くもり	
\\	今日は一日曇りでした。	きょうはいちにちくもりでした	
\\	お宅	おたく	
\\	先生のお宅はどちらですか。	せんせいのおたくはどちらですか	
\\	訳す	やくす	
\\	この文を訳してください。	このぶんをやくしてください	
\\	ラッシュアワー	らっしゅあわー	
\\	私はラッシュアワーの電車が嫌いです。	わたしはらっしゅあわーのでんしゃがきらいです	
\\	覚ます	さます	
\\	子供が目を覚ました。	こどもがめをさました	
\\	横断歩道	おうだんほどう	
\\	あそこに横断歩道があります。	あそこにおうだんほどうがあります	
\\	乗り換え	のりかえ	
\\	次の駅で乗り換えです。	つぎのえきでのりかえです	
\\	痒い	かゆい	
\\	背中が痒いです。	せなかがかゆいです	
\\	マフラー	まふらー	
\\	彼女は首にマフラーを巻いていた。	かのじょはくびにまふらーをまいていた	
\\	自習	じしゅう	
\\	明日は自習の時間がある。	あすはじしゅうのじかんがある	
\\	突き当たり	つきあたり	
\\	この先は突き当たりです。	このさきはつきあたりです	
\\	馬鹿	ばか	
\\	私は馬鹿だった。	わたしはばかだった	
\\	番地	ばんち	
\\	その建物の番地は分かりますか。	そのたてもののばんちはわかりますか	
\\	引き算	ひきざん	
\\	娘は学校で引き算を習っている。	むすめはがっこうでひきざんをならっている	
\\	片付く	かたづく	
\\	仕事が大分片付いた。	しごとがだいぶかたづいた	
\\	夕飯	ゆうはん	
\\	夕飯は寿司でした。	ゆうはんはすしでした	
\\	汚す	よごす	
\\	彼は服を汚した。	かれはふくをよごした	
\\	味噌汁	みそしる	
\\	私は毎日味噌汁を飲みます。	わたしはまいにちみそしるをのみます	
\\	臭い	くさい	
\\	納豆は臭い。	なっとうはくさい	
\\	両替	りょうがえ	
\\	1万円を両替してください。	いちまんえんをりょうがえしてください	
\\	例	れい	
\\	一つ例を上げてください。	ひとつれいをあげてください	
\\	名字	みょうじ	
\\	あなたの名字は何ですか。	あなたのみょうじはなんですか	
\\	納豆	なっとう	
\\	私は納豆をよく食べます。	わたしはなっとうをよくたべます	
\\	酔っ払う	よっぱらう	
\\	昨夜は酔っ払いました。	ゆうべはよっぱらいました	
\\	お辞儀	おじぎ	
\\	皆、社長にお辞儀をした。	みんなしゃちょうにおじぎをした	
\\	西瓜	すいか	
\\	夏は西瓜が美味しい。	なつはすいかがおいしい	
\\	中	なか	
\\	父は家の中にいます。	ちちはいえのなかにいます	
\\	苺	いちご	
\\	この苺はとても甘い。	このいちごはとてもあまい	
\\	定期券	ていきけん	
\\	定期券は1万2千円でした。	ていきけんはいちまんにせんえんでした	
\\	ハム	はむ	
\\	ハムサンドをください。	はむさんどをください	
\\	畳む	たたむ	
\\	布団を畳んでください。	ふとんをたたんでください	
\\	航空便	こうくうびん	
\\	航空便で書類が届きました。	こうくうびんでしょるいがとどきました	
\\	乾かす	かわかす	
\\	濡れた服を乾かした。	ぬれたふくをかわかした	
\\	からから	からから	
\\	お喋りして喉がからからになりました。	おしゃべりしてのどがからからになりました	
\\	ローマ字	ろーまじ	
\\	ローマ字で名前を書いてください。	ろーまじでなまえをかいてください	
\\	宜しい	よろしい	
\\	レポートはこれで宜しいですか。	れぽーとはこれでよろしいですか	
\\	大嫌い	だいきらい	
\\	私はテストが大嫌い。	わたしはてすとがだいきらい	
\\	床	ゆか	
\\	床が濡れている。	ゆかがぬれている	
\\	免許証	めんきょしょう	
\\	免許証を見せてください。	めんきょしょうをみせてください	
\\	速達	そくたつ	
\\	これを速達で送りたいのですが。	これをそくたつでおくりたいのですが	
\\	魚	さかな	
\\	魚がたくさん泳いでいます。	さかながたくさんおよいでいます	
\\	カセットテープ	かせっとてーぷ	
\\	昔はカセットテープを使っていました。	むかしはかせっとてーぷをつかっていました 。	
\\	アイロン	あいろん	
\\	彼女はシャツにアイロンをかけた。	かのじょはしゃつにあいろんをかけた	
\\	消しゴム	けしごむ	
\\	消しゴムを貸して下さい。	けしごむをかしてください	
\\	お巡りさん	おまわりさん	
\\	あのお巡りさんに道を聞きましょう。	あのおまわりさんにみちをききましょう	
\\	持って行く	もっていく	
\\	水を持って行きましょう。	みずをもっていきましょう	
\\	トランプ	とらんぷ	
\\	友達とトランプをして遊びました。	ともだちととらんぷをしてあそびました	
\\	カレーライス	かれーらいす	
\\	子供はカレーライスが好きです。	こどもはかれーらいすがすきです	
\\	送り仮名	おくりがな	
\\	その送り仮名は間違っています。	そのおくりがなはまちがっています	
\\	欲しがる	ほしがる	
\\	子供がジュースを欲しがっています。	こどもがじゅーすをほしがっています	
\\	特急	とっきゅう	
\\	東京まで特急で3時間掛かります。	とうきょうまでとっきゅうでさんじかんかかります	
\\	入り口	いりぐち	
\\	入り口は向こうです。	いりぐちはむこうです	
\\	糊	のり	
\\	糊で2枚の紙を貼り合わせた。	のりでにまいのかみをはりあわせた	
\\	あちら(彼方)	あちら	
\\	あちらに行ってみよう。	あちらにいってみよう	
\\	二十歳	はたち	
\\	妹は来年、二十歳になります。	いもうとはらいねんはたちになります	
\\	休校	きゅうこう	
\\	学校は今週は休校です。	がっこうはこんしゅうはきゅうこうです	
\\	万年筆	まんねんひつ	
\\	父に万年筆をプレゼントしました。	ちちにまんねんひつをぷれぜんとしました	
\\	コーラ	こーら	
\\	私の兄はコーラが好きだ。	わたしのあにはこーらがすきだ	
\\	ハイキング	はいきんぐ	
\\	昨日は友人とハイキングに行きました。	きのうはゆうじんとはいきんぐにいきました	
\\	どきどき	どきどき	
\\	彼女に会うとどきどきします。	かのじょにあうとどきどきします	
\\	スリッパ	すりっぱ	
\\	スリッパを履いてください。	すりっぱをはいてください	
\\	売り切れ	うりきれ	
\\	チケットはもう売り切れだって。	ちけっとはもううりきれだって	
\\	例文	れいぶん	
\\	例文を3つ作ってください。	れいぶんをみっつつくってください	
\\	歯	は	
\\	私の歯は丈夫です。	わたしのははじょうぶです	
\\	ネックレス	ねっくれす	
\\	妻にネックレスをプレゼントしました。	つまにねっくれすをぷれぜんとしました	
\\	万歳	ばんざい	
\\	勝った、万歳!	かったばんざい	
\\	主人	しゅじん	
\\	主人は今、留守です。	しゅじんはいまるすです	
\\	バスケットボール	ばすけっとぼーる	
\\	姉はバスケットボールの選手です。	あねはばすけっとぼーるのせんしゅです	
\\	にこにこ	にこにこ	
\\	彼女はいつもにこにこしています。	かのじょはいつもにこにこしています	
\\	欠伸	あくび	
\\	父が欠伸をした。	ちちがあくびをした	
\\	葉書	はがき	
\\	家族に葉書を書いています。	かぞくにはがきをかいています	
\\	無くなる	なくなる	
\\	もうお金が無くなりました。	もうおかねがなくなりました	
\\	よろしく(宜しく)	よろしく	
\\	よろしくお願いします。	よろしくおねがいします	
\\	暇	ひま	
\\	明日は暇ですか。	あしたはひまですか	
\\	七	しち	
\\	そのグループのメンバーは全部で七人だ。	そのぐるーぷのめんばーはぜんぶでしちにんだ	
\\	歯ブラシ	はぶらし	
\\	新しい歯ブラシが必要だ。	あたらしいはぶらしがひつようだ	
\\	ふらふら	ふらふら	
\\	私は疲れてふらふらです。	わたしはつかれてふらふらです	
\\	目覚まし時計	めざましどけい	
\\	7時に目覚まし時計が鳴りました。	しちじにめざましどけいがなりました	
\\	外側	そとがわ	
\\	白線の外側を歩かないでください。	はくせんのそとがわをあるかないでください	
\\	豚	ぶた	
\\	豚は凄く鼻が良い。	ぶたはすごくはながいい	
\\	塩辛い	しおからい	
\\	海の水は塩辛い。	うみのみずはしおからい	
\\	零す	こぼす	
\\	ソファにワインを零した。	そふぁにわいんをこぼした	
\\	西	にし	
\\	太陽は西に沈みます。	たいようはにしにしずみます	
\\	洗面所	せんめんじょ	
\\	洗面所で顔を洗った。	せんめんじょでかおをあらった	
\\	二人	ふたり	
\\	今日は妻と二人で食事をします。	きょうはつまとふたりでしょくじをします	
\\	九	く	
\\	私は九月に行く予定です。	わたしはくがつにいくよていです	
\\	包む	つつむ	
\\	プレゼントを綺麗な紙で包みました。	ぷれぜんとをきれいなかみでつつみました	
\\	止む	やむ	
\\	雨が止みました。	あめがやみました	
\\	ノック	のっく	
\\	入る時はドアをノックしてください。	はいるときはどあをのっくしてください	
\\	横書き	よこがき	
\\	この本は横書きです。	このほんはよこがきです	
\\	チョーク	ちょーく	
\\	チョークで黒板に字を書きました。	ちょーくでこくばんにじをかきました	
\\	ぺらぺら	ぺらぺら	
\\	彼女はドイツ語がぺらぺらです。	かのじょはどいつごがぺらぺらです	
\\	エアメール	えあめーる	
\\	彼にエアメールを送りました。	かれにえあめーるをおくりました	
\\	蕎麦屋	そばや	
\\	昼は蕎麦屋に行きました。	ひるはそばやにいきました	
\\	幾ら	いくら	
\\	この靴は幾らですか。	このくつはいくらですか	
\\	レシート	れしーと	
\\	レシートを財布に入れました。	れしーとをさいふにいれました	
\\	褒める	ほめる	
\\	頑張ったので褒められました。	がんばったのでほめられました	
\\	セロテープ	せろてーぷ	
\\	セロテープはありますか。	せろてーぷはありますか	
\\	いつごろ(何時頃)	いつごろ	
\\	いつごろ夏休みを取りますか。	いつごろなつやすみをとりますか	
\\	本箱	ほんばこ	
\\	雑誌を本箱に入れました。	ざっしをほんばこにいれました	
\\	ピンポン	ぴんぽん	
\\	昨日の夕方、友達とピンポンをした。	きのうのゆうがたともだちとぴんぽんをした	
\\	寒さ	さむさ	
\\	今日は厳しい寒さになるでしょう。	きょうはきびしいさむさになるでしょう	
\\	レインコート	れいんこーと	
\\	雨なのでレインコートを着ました。	あめなのでれいんこーとをきました	
\\	祝日	しゅくじつ	
\\	5月3日は祝日だ。	ごがつみっかはしゅくじつだ	
\\	換える	かえる	
\\	車のタイヤを換えた。	くるまのたいやをかえた	
\\	そっち(其方)	そっち	
\\	そっちが私の部屋です。	そっちがわたしのへやです	
\\	持って来る	もってくる	
\\	その椅子を持って来てください。	そのいすをもってきてください	
\\	缶詰	かんづめ	
\\	缶詰のフルーツはとても甘い。	かんづめのふるーつはとてもあまい	
\\	絵葉書	えはがき	
\\	友達から絵葉書が届きました。	ともだちからえはがきがとどきました	
\\	不親切	ふしんせつ	
\\	その店員は不親切だった。	そのてんいんはふしんせつだった	
\\	昼間	ひるま	
\\	昼間は仕事で忙しいです。	ひるまはしごとでいそがしいです	
\\	鋤焼	すきやき	
\\	昨日の夜は鋤焼を食べた。	きのうのよるはすきやきをたべた	
\\	ウィスキー	うぃすきー	
\\	このウィスキーは強いよ。	このうぃすきーはつよいよ	
\\	ぎらぎら	ぎらぎら	
\\	太陽がぎらぎらしている。	たいようがぎらぎらしている	
\\	段々	だんだん	
\\	段々仕事が楽しくなってきました。	だんだんしごとがたのしくなってきました	
\\	刺身	さしみ	
\\	私は刺身は食べません。	わたしはさしみはたべません	
\\	怪我	けが	
\\	彼女は腕を怪我した。	かのじょはうでをけがした	
\\	ティッシュ	てぃっしゅ	
\\	ティッシュを取ってください。	てぃっしゅをとってください	
\\	昼前	ひるまえ	
\\	昼前に会議があった。	ひるまえにかいぎがあった	
\\	ぺこぺこ	ぺこぺこ	
\\	私はお腹がぺこぺこです。	わたしはおなかがぺこぺこです	
\\	書留	かきとめ	
\\	これを書留で送りたいのですが。	これをかきとめでおくりたいのですが	
\\	シャープペンシル	しゃーぷぺんしる	
\\	私のシャープペンシルがありません。	わたしのしゃーぷぺんしるがありません	
\\	掃く	はく	
\\	床を箒で掃きました。	ゆかをほうきではきました	
\\	予習	よしゅう	
\\	明日の予習をしましょう。	あしたのよしゅうをしましょう	
\\	がらがら	がらがら	
\\	映画館はがらがらでした。	えいがかんはがらがらでした	
\\	暖める	あたためる	
\\	今、車を暖めています。	いまくるまをあたためています	
\\	止まる	とまる	
\\	今朝、事故で電車が止まりました。	けさじこででんしゃがとまりました	
\\	明後日	あさって	
\\	明後日は休日です。	あさってはきゅうじつです	
\\	アイス	あいす	
\\	暑いのでアイスを食べました。	あついのであいすをたべました	
\\	ランチ	らんち	
\\	一緒にランチに行きませんか。	いっしょにらんちにいきませんか	
\\	字	じ	
\\	もっと大きく字を書いてください。	もっとおおきくじをかいてください	
\\	足し算	たしざん	
\\	娘は学校で足し算を習った。	むすめはがっこうでたしざんをならった	
\\	深さ	ふかさ	
\\	このプールの深さは2メートルです。	このぷーるのふかさはにめーとるです	
\\	気を付ける	きをつける	
\\	体に気を付けてください。	からだにきをつけてください	
\\	沸かす	わかす	
\\	お湯を沸かしてください。	おゆをわかしてください	
\\	一昨日	おととい	
\\	一昨日彼から電話がありました。	おとといかれからでんわがありました	
\\	下手	へた	
\\	私は歌が下手だ。	わたしはうたがへただ	
\\	紫	むらさき	
\\	彼女は紫のドレスを着ていた。	かのじょはむらさきのどれすをきていた	
\\	姉さん	ねえさん	
\\	姉さん、ごめんね。	ねえさんごめんね	
\\	濡れる	ぬれる	
\\	雨で濡れてしまった。	あめでぬれてしまった	
\\	庭	にわ	
\\	庭に白い猫がいます。	にわにしろいねこがいます	
\\	零	れい	
\\	今ちょうど零時です。	いまちょうどれいじです	
\\	表	おもて	
\\	表に人が来ています。	おもてにひとがきています	
\\	眼鏡	めがね	
\\	彼は眼鏡をかけています。	かれはめがねをかけています	
\\	ちょっと	ちょっと	
\\	ちょっとだけ待っていてください。	ちょっとだけまっていてください	
\\	助ける	たすける	
\\	彼女は病気の犬を助けた。	かのじょはびょうきのいぬをたすけた	
\\	点く	つく	
\\	部屋の電気が点いています。	へやのでんきがついています	
\\	木綿	もめん	
\\	彼女は木綿のシャツを着ています。	かのじょはもめんのしゃつをきています	
\\	貼る	はる	
\\	机にシールを貼った。	つくえにしーるをはった	
\\	焼く	やく	
\\	今、魚を焼いています。	いまさかなをやいています	
\\	人参	にんじん	
\\	私は人参が嫌いだ。	わたしはにんじんがきらいだ	
\\	風邪薬	かぜぐすり	
\\	この風邪薬を飲みなさい。	このかぜぐすりをのみなさい	
\\	曲がり角	まがりかど	
\\	ポストはそこの曲がり角にあります。	ぽすとはそこのまがりかどにあります	
\\	雲	くも	
\\	今日は雲が多い。	きょうはくもがおおい	
\\	茄子	なす	
\\	夕飯に茄子の天ぷらを食べました。	ゆうはんになすのてんぷらをたべました	
\\	音	おと	
\\	このピアノは音があまり良くない。	このぴあのはおとがあまりよくない	
\\	祖母	そぼ	
\\	祖母は京都で生まれました。	そぼはきょうとでうまれました	
\\	顎	あご	
\\	顎が痛い。	あごがいたい	
\\	半日	はんにち	
\\	今日は半日だけ仕事だ。	きょうははんにちだけしごとだ	
\\	ご馳走する	ごちそうする	
\\	今日の夕食は私がご馳走します。	きょうのゆうしょくはわたしがごちそうします	
\\	灰色	はいいろ	
\\	今日の空は灰色だ。	きょうのそらははいいろだ	
\\	間	あいだ	
\\	雲の間から月が出た。	くものあいだからつきがでた	
\\	朝食	ちょうしょく	
\\	朝食に納豆を食べました。	ちょうしょくになっとうをたべました	
\\	空	そら	
\\	空の写真を撮るのが好きです。	そらのしゃしんをとるのがすきです	
\\	天ぷら(天麩羅)	てんぷら	
\\	私は蝦の天ぷらが好きです。	わたしはえびのてんぷらがすきです	
\\	眠る	ねむる	
\\	昨日は8時間眠りました。	きのうははちじかんねむりました	
\\	掃除機	そうじき	
\\	掃除機が壊れた。	そうじきがこわれた	
\\	温い	ぬるい	
\\	風呂が温かった。	ふろがぬるかった	
\\	拭く	ふく	
\\	タオルで体を拭きました。	たおるでからだをふきました	
\\	暖まる	あたたまる	
\\	まだ部屋が暖まらない。	まだへやがあたたまらない	
\\	お釣り	おつり	
\\	母はお釣りを確かめた。	はははおつりをたしかめた	
\\	煙草屋	たばこや	
\\	私は煙草屋でライターを買った。	わたしはたばこやでらいたーをかった	
\\	薬缶	やかん	
\\	薬缶でお湯を沸かしました。	やかんでおゆをわかしました	
\\	肘	ひじ	
\\	彼は肘に怪我をした。	かれはひじにけがをした	
\\	醤油	しょうゆ	
\\	もう少し醤油を足してください。	もうすこししょうゆをたしてください	
\\	上り	のぼり	
\\	これは上り電車です。	これはのぼりでんしゃです	
\\	お祖母さん	おばあさん	
\\	彼女はお祖母さんと住んでいる。	かのじょはおばあさんとすんでいる	
\\	丈夫	じょうぶ	
\\	祖母は身体が丈夫だ。	そぼはからだがじょうぶだ	
\\	山	やま	
\\	山の空気は綺麗だ。	やまのくうきはきれいだ	
\\	答え	こたえ	
\\	彼はその問題の答えが分からない。	かれはそのもんだいのこたえがわからない	
\\	孫	まご	
\\	昨日、孫が生まれました。	きのうまごがうまれました	
\\	又	また	
\\	明日、又来ます。	あしたまたきます	
\\	長女	ちょうじょ	
\\	うちの長女は10歳です。	うちのちょうじょはじゅっさいです	
\\	伯母さん、叔母さん	おばさん	
\\	叔母さん一家が遊びに来ました。	おばさんいっかがあそびにきました	
\\	机	つくえ	
\\	新しい机を買ってもらいました。	あたらしいつくえをかってもらいました	
\\	半月	はんつき	
\\	半月前に日本に来ました。	はんつきまえににほんにきました	
\\	始めに	はじめに	
\\	始めにスープが出ます。	はじめにすーぷがでます	
\\	お医者さん	おいしゃさん	
\\	熱があるのでお医者さんに行った。	ねつがあるのでおいしゃさんにいった	
\\	そんなに	そんなに	
\\	一度にそんなにたくさんは出来ない。	いちどにそんなにたくさんはできない	
\\	煙草	たばこ	
\\	彼は煙草を吸いません。	かれはたばこをすいません	
\\	ざあざあ	ざあざあ	
\\	雨がざあざあ降っている。	あめがざあざあふっている	
\\	再来月	さらいげつ	
\\	再来月まで予約で一杯です。	さらいげつまでよやくでいっぱいです	
\\	緩い	ゆるい	
\\	このズボンは緩いです。	このずぼんはゆるいです	
\\	人気	にんき	
\\	このバンドはとても人気があるよ。	このばんどはとてもにんきがあるよ	
\\	姪	めい	
\\	私の姪は3才です。	わたしのめいはさんさいです	
\\	心配	しんぱい	
\\	明日のプレゼンテーションが心配だ。	あすのぷれぜんてーしょんがしんぱいだ	
\\	ようこそ	ようこそ	
\\	日本へようこそ。	にっぽんへようこそ	
\\	旗	はた	
\\	旗が風に揺れている。	はたがかぜにゆれている	
\\	昨夜	ゆうべ	
\\	昨夜、流れ星を見ました。	ゆうべながれぼしをみました	
\\	金	きん	
\\	彼女は金のネックレスをしています。	かのじょはきんのねっくれすをしています	
\\	白	しろ	
\\	白は雪の色です。	しろはゆきのいろです	
\\	丁度	ちょうど	
\\	値段は丁度3万円です。	ねだんはちょうどさんまんえんです	
\\	あんなに	あんなに	
\\	あんなにいい人はいません。	あんなにいいひとはいません	
\\	喧嘩	けんか	
\\	喧嘩は止めて。	けんかはやめて	
\\	膝	ひざ	
\\	スキーで膝を怪我しました。	すきーでひざをけがしました	
\\	鶏	にわとり	
\\	その鶏は毎朝鳴きます。	そのにわとりはまいあさなきます	
\\	寺	てら	
\\	あそこに古いお寺があります。	あそこにふるいおてらがあります	
\\	雷	かみなり	
\\	雷が鳴っています。	かみなりがなっています	
\\	書き取る	かきとる	
\\	話しのポイントを書き取った。	はなしのぽいんとをかきとった	
\\	アイスコーヒー	いすこーひー	
\\	アイスコーヒーをください。	あいすこーひーをください	
\\	お菓子	おかし	
\\	お母さんと一緒にお菓子を作りました。	おかあさんといっしょにおかしをつくりました	
\\	お土産	おみやげ	
\\	父はお土産にお菓子を買ってきた。	ちちはおみやげにおかしをかってきた	
\\	ソファー	そふぁー	
\\	このソファーは気持ちがいい。	このそふぁーはきもちがいい	
\\	右側	みぎがわ	
\\	彼女はいつも私の右側を歩きます。	かのじょはいつもわたしのみぎがわをあるきます	
\\	厳しい	きびしい	
\\	私の上司はとても厳しい。	わたしのじょうしはとてもきびしい	
\\	大人しい	おとなしい	
\\	私の彼女はとても大人しいです。	わたしのかのじょはとてもおとなしいです	
\\	可	か	
\\	このアルバイトは「学生可」ですね。	このあるばいとはがくせいかですね	
\\	毎月	まいつき	
\\	私は毎月貯金をしています。	わたしはまいつきちょきんをしています	
\\	湯	ゆ	
\\	お風呂のお湯が一杯です。	おふろのおゆがいっぱいです	
\\	空手	からて	
\\	彼は空手を習っています。	かれはからてをならっています	
\\	弾く	ひく	
\\	彼はギターを弾きます。	かれはぎたーをひきます	
\\	汚れる	よごれる	
\\	エプロンをしないと服が汚れます。	えぷろんをしないとふくがよごれます	
\\	眠い	ねむい	
\\	今日はとても眠いです。	きょうはとてもねむいです	
\\	港	みなと	
\\	港に船が着きました。	みなとにふねがつきました	
\\	平成	へいせい	
\\	彼女は平成3年生まれです。	かのじょはへいせいさんねんうまれです	
\\	似ている	にている	
\\	私は母に似ています。	わたしはははににています	
\\	半袖	はんそで	
\\	今日は、半袖のシャツを着よう。	きょうははんそでのしゃつをきよう	
\\	蒔く	まく	
\\	日本では、二月に豆を蒔く行事がある。	にほんではにがつにまめをまくぎょうじがある	
\\	大切	たいせつ	
\\	これは母が大切にしていた指輪です。	これはははがたいせつにしていたゆびわです	
\\	人々	ひとびと	
\\	あの村の人々はとても親切です。	あのむらのひとびとはとてもしんせつです	
\\	復習	ふくしゅう	
\\	昨日の復習をしましたか。	きのうのふくしゅうをしましたか	
\\	船便	ふなびん	
\\	アメリカの友人から船便が届いた。	あめりかのゆうじんからふなびんがとどいた	
\\	とんとん	とんとん	
\\	母の肩をとんとん叩いた。	ははのかたをとんとんたたいた	
\\	町	まち	
\\	ここは小さな町です。	ここはちいさなまちです	
\\	初め	はじめ	
\\	初めは上手く出来ませんでした。	はじめはうまくできませんでした	
\\	一人で	ひとりで	
\\	今日は一人で映画を見ます。	きょうはひとりでえいがをみます	
\\	灰	はい	
\\	火事で全て灰になった。	かじですべてはいになった	
\\	氏名	しめい	
\\	ここに住所と氏名を書いてください。	ここにじゅうしょとしめいをかいてください	
\\	橋	はし	
\\	あの橋は日本で一番長い。	あのはしはにほんでいちばんながい	
\\	茶碗	ちゃわん	
\\	お茶碗にご飯を装いました。	おちゃわんにごはんをよそいました	
\\	指	ゆび	
\\	彼は指が太い。	かれはゆびがふとい	
\\	風呂	ふろ	
\\	父は今お風呂に入っています。	ちちはいまおふろにはいっています	
\\	壊れる	こわれる	
\\	会社のパソコンが壊れた。	かいしゃのぱそこんがこわれた	
\\	夜中	よなか	
\\	夜中に電話がありました。	よなかにでんわがありました	
\\	割り算	わりざん	
\\	割り算は小学校で習います。	わりざんはしょうがっこうでならいます	
\\	細か	こまか	
\\	それは細かな問題です。	それはこまかなもんだいです	
\\	星	ほし	
\\	今夜は星がよく見えます。	こんやはほしがよくみえます	
\\	行き	ゆき	
\\	東京行きの列車に乗った。	とうきょうゆきのれっしゃにのった	
\\	こんなに	こんなに	
\\	こんなに面白い本は初めて読んだ。	こんなにおもしろいほんははじめてよんだ	
\\	洗濯	せんたく	
\\	一週間、洗濯をしていない。	いっしゅうかんせんたくをしていない	
\\	連れて行く	つれていく	
\\	私も連れて行ってください。	わたしもつれていってください	
\\	ウェートレス	うぇーとれす	
\\	ウェートレスを呼んだ。	うぇーとれすをよんだ	
\\	火傷	やけど	
\\	彼は手に火傷をしました。	かれはてにやけどをしました	
\\	昼食	ちゅうしょく	
\\	昼食にお寿司を食べました。	ちゅうしょくにおすしをたべました	
\\	返事	へんじ	
\\	手紙の返事を出しました。	てがみのへんじをだしました	
\\	夕食	ゆうしょく	
\\	夕食は7時です。	ゆうしょくはしちじです	
\\	袖	そで	
\\	このシャツは袖が短い。	このしゃつはそでがみじかい	
\\	流行る	はやる	
\\	去年はスニーカーが流行りました。	きょねんはすにーかーがはやりました	
\\	朝御飯	あさごはん	
\\	七時に朝御飯を食べました	しちじにあさごはんをたべました	
\\	再来週	さらいしゅう	
\\	再来週は忙しいです。	さらいしゅうはいそがしいです	
\\	良い	よい	
\\	彼は良い人です。	かれはよいひとです	
\\	大事	だいじ	
\\	お体をお大事に。	おからだをおだいじに	
\\	ご馳走	ごちそう	
\\	テーブルの上にご馳走が並んでいる。	てーぶるのうえにごちそうがならんでいる	
\\	南	みなみ	
\\	私の家は町の南にあります。	わたしのいえはまちのみなみにあります	
\\	袋	ふくろ	
\\	袋は要りません。	ふくろはいりません	
\\	形	かたち	
\\	その椅子は変わった形をしている。	そのいすはかわったかたちをしている	
\\	降る	ふる	
\\	明日は雨が降るでしょう。	あすはあめがふるでしょう	
\\	郵便屋さん	ゆうびんやさん	
\\	郵便屋さんはもう来ましたか。	ゆうびんやさんはもうきましたか	
\\	お祝い	おいわい	
\\	彼の就職のお祝いをしよう。	かれのしゅうしょくのおいわいをしよう	
\\	瓶	びん	
\\	瓶ビールを注文した。	びんびーるをちゅうもんした	
\\	本当	ほんとう	
\\	その話は本当ですか。	そのはなしはほんとうですか	
\\	石鹸	せっけん	
\\	石鹸で手を洗ってください。	せっけんでてをあらってください	
\\	連れて来る	つれてくる	
\\	息子が友達を連れて来ました。	むすこがともだちをつれてきました	
\\	長袖	ながそで	
\\	今日は寒いので長袖を着ました。	きょうはさむいのでながそでをきました	
\\	兄さん	にいさん	
\\	兄さん、おめでとう。	にいさんおめでとう	
\\	年上	としうえ	
\\	彼は私より年上です。	かれはわたしよりとしうえです	
\\	火	ひ	
\\	火を消して。	ひをけして	
\\	枕	まくら	
\\	私は低い枕が好きです。	わたしはひくいまくらがすきです	
\\	倒す	たおす	
\\	そのスキーヤーはポールを倒した。	そのすきーやーはぽーるをたおした	
\\	櫛	くし	
\\	彼は櫛で髪を解かした。	かれはくしでかみをとかした	
\\	外	そと	
\\	外は暑いよ。	そとはあついよ	
\\	便所	べんじょ	
\\	便所はそこです。	べんじょはそこです	
\\	缶	かん	
\\	コーヒーはその缶に入っています。	こーひーはそのかんにはいっています	
\\	船	ふね	
\\	私たちは船に乗った。	わたしたちはふねにのった	
\\	風呂屋	ふろや	
\\	昨日、友達とお風呂屋さんに行った。	きのうともだちとおふろやさんにいった	
\\	歌手	かしゅ	
\\	その歌手は歌が下手だ。	そのかしゅはうたがへただ	
\\	汚れ	よごれ	
\\	靴の汚れを落としました。	くつのよごれをおとしました	
\\	血	ち	
\\	血が出ていますよ。	ちがでていますよ	
\\	行き	いき	
\\	行きは新幹線で行った。	いきはしんかんせんでいった	
\\	剃る	そる	
\\	父は毎日髭を剃ります。	ちちはまいにちひげをそります	
\\	誰か	だれか	
\\	誰かに聞いてみてください。	だれかにきいてみてください	
\\	表	ひょう	
\\	この表を見てください。	このひょうをみてください	
\\	痩せる	やせる	
\\	私は少し痩せました。	わたしはすこしやせました	
\\	肉	にく	
\\	昨日の夜は肉をたくさん食べました。	きのうのよるはにくをたくさんたべました	
\\	上がる	あがる	
\\	彼は人前だと上がってしまう。	かれはひとまえだとあがってしまう	
\\	勤める	つとめる	
\\	私は銀行に勤めています。	わたしはぎんこうにつとめています	
\\	じゃが芋	じゃがいも	
\\	ポテトサラダを作るから、じゃが芋を買ってきて。	ぽてとさらだをつくるからじゃがいもをかってきて	
\\	台	だい	
\\	そこにちょうど良い台がある。	そこにちょうどよいだいがある	
\\	絶対に	ぜったいに	
\\	絶対に駄目です。	ぜったいにだめです	
\\	祖父	そふ	
\\	祖父は元気です。	そふはげんきです	
\\	不動産屋	ふどうさんや	
\\	私の父は不動産屋です。	わたしのちちはふどうさんやです	
\\	零れる	こぼれる	
\\	彼女の目から涙が零れた。	かのじょのめからなみだがこぼれた	
\\	柱	はしら	
\\	この家の柱は太い。	このいえのはしらはふとい	
\\	泊める	とめる	
\\	友達をうちに泊めてあげました。	ともだちをうちにとめてあげました	
\\	両親	りょうしん	
\\	私の両親は大阪に住んでいます。	わたしのりょうしんはおおさかにすんでいます	
\\	悪戯	いたずら	
\\	その子は悪戯が大好きだ。	そのこはいたずらがだいすきだ	
\\	方	かた	
\\	次の方、どうぞ。	つぎのかたどうぞ	
\\	叱る	しかる	
\\	父親が子供を叱っている。	ちちおやがこどもをしかっている	
\\	骨	ほね	
\\	彼は足の骨を折りました。	かれはあしのほねをおりました	
\\	あっち(彼方)	あっち	
\\	郵便局はあっちです。	ゆうびんきょくはあっちです	
\\	景色	けしき	
\\	ここは景色が綺麗ですね。	ここはけしきがきれいですね	
\\	左側	ひだりがわ	
\\	画面の左側を見てください。	がめんのひだりがわをみてください	
\\	ウェーター	うぇーたー	
\\	ウェーターが水を運んできました。	うぇーたーがみずをはこんできました	
\\	仕舞う	しまう	
\\	彼女は大切な書類を机に仕舞った。	かのじょはたいせつなしょるいをつくえにしまった	
\\	昼過ぎ	ひるすぎ	
\\	今日は昼過ぎまで寝ていた。	きょうはひるすぎまでねていた	
\\	鳥	とり	
\\	あそこに綺麗な鳥がいます。	あそこにきれいなとりがいます	
\\	騒ぐ	さわぐ	
\\	電車の中で騒がないでください。	でんしゃのなかでさわがないでください	
\\	残念	ざんねん	
\\	その試合は残念な結果になった。	そのしあいはざんねんなけっかになった	
\\	紐	ひも	
\\	靴の紐が切れた。	くつのひもがきれた	
\\	この頃	このごろ	
\\	この頃、よく肩が凝る。	このごろよくかたがこる	
\\	爪	つめ	
\\	爪が伸びています。	つめがのびています	
\\	豚肉	ぶたにく	
\\	豚肉は美味しいです。	ぶたにくはおいしいです	
\\	被る	かぶる	
\\	帽子を被って外出した。	ぼうしをかぶってがいしゅつした	
\\	四つ角	よつかど	
\\	あそこの四つ角を左に曲がってください。	あそこのよつかどをひだりにまがってください	
\\	鞄	かばん	
\\	その黒い鞄は僕のです。	そのくろいかばんはぼくのです	
\\	クリーニング	くりーにんぐ	
\\	スーツをクリーニングに出しました。	すーつをくりーにんぐにだしました	
\\	丸	まる	
\\	紙に大きな丸を書きました。	かみにおおきなまるをかきました	
\\	元旦	がんたん	
\\	元旦に彼から年賀状が来た。	がんたんにかれからねんがじょうがきた	
\\	車道	しゃどう	
\\	車道の工事が始まりました。	しゃどうのこうじがはじまりました	
\\	拍手	はくしゅ	
\\	大きな拍手が上がった。	おおきなはくしゅがあがった	
\\	人間	にんげん	
\\	人間の心は複雑です。	にんげんのこころはふくざつです	
\\	パーセント	ぱーせんと	
\\	この村の80パーセントは老人です。	このむらのはちじゅっぱーせんとはろうじんです	
\\	賑やか	にぎやか	
\\	浅草は賑やかな街です。	あさくさはにぎやかなまちです	
\\	着替える	きがえる	
\\	彼は綺麗な服に着替えた。	かれはきれいなふくにきがえた	
\\	何	なん	
\\	質問は何ですか。	しつもんはなんですか	
\\	客	きゃく	
\\	その店は若い客が多いです。	そのみせはわかいきゃくがおおいです	
\end{CJK}
\end{document}