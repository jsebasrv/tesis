\documentclass[8pt]{extreport} 
\usepackage{hyperref}
\usepackage{CJKutf8}
\begin{document}
\begin{CJK}{UTF8}{min}
\\	さんがにち
\\	あけましておめでとうございます。
\\	ぞうに
\\	おせちりょうり
\\	おとしだま
\\	ふくろ
\\	にねんまいり
\\	がんたん
\\	ごちそうさま。
\\	かんぱい
\\	おいしい
\\	いただきます。
\\	いただく
\\	かんぱい
\\	なっとう
\\	まあまあ
\\	はじめまして
\\	わたし
\\	よろしく
\\	おねがいします。
\\	やまぐち
\\	わたなべ
\\	はじめまして。
\\	はじめまして。
\\	わたしはやまぐちちぐさです。
\\	わたしはやまぐちちぐさです。
\\	よろしくおねがいします。
\\	よろしくおねがいします。
\\	はじめまして。
\\	はじめまして。
\\	ゴールデンウィークが おわってから、ぜんぜん やるきも でないんだよね。
\\	ゴールデンウィークが おわってから、ぜんぜん やるきも でないんだよね。
\\	あ、おれも
\\	あさ ぜんぜん おきられないし、いちにちじゅう だるいんだよな。
\\	あ、おれも
\\	あさ ぜんぜん おきられないし、いちにちじゅう だるいんだよな。
\\	あ、ばばせんせいだ。
\\	あ、ばばせんせいだ。
\\	こんにちは。
\\	こんにちは。
\\	こんにちは。どうした?ふたりとも、ねむそうなかおして。
\\	こんにちは。どうした?ふたりとも、ねむそうなかおして。
\\	いま、「さいきん なんだか だるくて・・・」というはなしを していたところだったんですよ。
\\	いま、「さいきん なんだか だるくて・・・」というはなしを していたところだったんですよ。
\\	あー、それは ごがつびょうじゃないですか。
\\	あー、それは ごがつびょうじゃないですか。
\\	ごがつびょう?
\\	ごがつびょう?
\\	そう。だいがく いちねんの しがつは、あたらしいことばかりで、 まいにち きんちょうするでしょう。
\\	そう。だいがく いちねんの しがつは、あたらしいことばかりで、 まいにち きんちょうするでしょう。
\\	で、そのつかれが、 5がつの れんきゅうのあとに、 いっきに でてきたんだよ。
\\	で、そのつかれが、 5がつの れんきゅうのあとに、 いっきに でてきたんだよ。
\\	どうしたら・・・いいんですか。
\\	どうしたら・・・いいんですか。
\\	あたらしい かんきょうに なれるのは、むずかしいんだから、 あせるひつようは ないよ。
\\	あたらしい かんきょうに なれるのは、むずかしいんだから、 あせるひつようは ないよ。
\\	ま、きらくにね。
\\	ま、きらくにね。
\\	はい
\\	ありがとうございます
\\	はい
\\	ありがとうございます
\\	あ、そうそう。 ぼく、 ごがつびょうについての ほんを かいているから、 ほんやで かってね。
\\	あ、そうそう。 ぼく、 ごがつびょうについての ほんを かいているから、 ほんやで かってね。
\\	ごうこん
\\	そうぞうする
\\	はずむ
\\	きょうつう
\\	わだいになる
\\	こうし
\\	りょうりきょうしつ
\\	だらけ
\\	つづく
\\	かんけい
\\	みりょく
\\	しょたいめん
\\	さんかしゃ
\\	わたしは、いま、
\\	りょうりきょうしつにきています。
\\	わたしは、いま、
\\	りょうりきょうしつにきています。
\\	きょうは、さいきんわだいになっている、「りょうりごうこん」についてレポートします。
\\	きょうは、さいきんわだいになっている、「りょうりごうこん」についてレポートします。
\\	では、
\\	りょうりきょうしつのこうしもりひかりせんせいにおはなしをきいてみましょう。
\\	では、
\\	りょうりきょうしつのこうしもりひかりせんせいにおはなしをきいてみましょう。
\\	せんせいよろしくおねがいします。
\\	せんせいよろしくおねがいします。
\\	よろしくおねがいします。
\\	よろしくおねがいします。
\\	せんせい、「りょうりごうこん」というのは・・・
\\	せんせい、「りょうりごうこん」というのは・・・
\\	りょうりきょうしつでのごうこん・・・
\\	りょうりきょうしつでのごうこん・・・
\\	つまり、しょたいめんのだんじょのグループがりょうりをいっしょにして、しりあうことができるパーティーです。
\\	つまり、しょたいめんのだんじょのグループがりょうりをいっしょにして、しりあうことができるパーティーです。
\\	りょうりごうこんのみりょくは、なんでしょうか。
\\	りょうりごうこんのみりょくは、なんでしょうか。
\\	まず、りょうりのべんきょうになります。
\\	まず、りょうりのべんきょうになります。
\\	そして、りょうりごうこんでしりあったカップルはながくつづくといわれています。
\\	そして、りょうりごうこんでしりあったカップルはながくつづくといわれています。
\\	そうなんですかー。どうしてかんけいがつづきやすいんですか。
\\	そうなんですかー。どうしてかんけいがつづきやすいんですか。
\\	いっしょにりょうりをつくるので、あいてのせいかくがわかりやすいんです。
\\	いっしょにりょうりをつくるので、あいてのせいかくがわかりやすいんです。
\\	きょうつうのわだいがあって、はなしもはずみますし。
\\	きょうつうのわだいがあって、はなしもはずみますし。
\\	なるほど。
\\	なるほど。
\\	さんかしゃのなかには、「けっこんごのせいかつがそうぞうしやすい」っていうひともいるんですよ。
\\	さんかしゃのなかには、「けっこんごのせいかつがそうぞうしやすい」っていうひともいるんですよ。
\\	りょうりがならえて、さいこうのパートナーもみつけられるんですね。
\\	りょうりがならえて、さいこうのパートナーもみつけられるんですね。
\\	まさに、いいことだらけ
\\	いいなー。わたしもさんかしたーい。
\\	まさに、いいことだらけ
\\	いいなー。わたしもさんかしたーい。
\\	もの
\\	けいひん
\\	こうどく
\\	たくはいびん
\\	がくひ
\\	ひっこし
\\	さしあげる
\\	しょうひんけん
\\	けいやく
\\	はさむ
\\	どかす
\\	たくはいびんでーす。
\\	たくはいびんでーす。
\\	はーい。
\\	はーい。
\\	すみませーん、わたくし、あかさかしんぶんのものなんですが・・・
\\	すみませーん、わたくし、あかさかしんぶんのものなんですが・・・
\\	え・・(こころのこえ
\\	たくはいびんじゃないじゃん
\\	どうしよう。うちしんぶんいらないし。)
\\	え・・(こころのこえ
\\	たくはいびんじゃないじゃん
\\	どうしよう。うちしんぶんいらないし。)
\\	しんぶんをこうどくしませんか。
\\	しんぶんをこうどくしませんか。
\\	あー、もうしわけないんですけど、もうまにあっていますので、だいじょうぶです。
\\	あー、もうしわけないんですけど、もうまにあっていますので、だいじょうぶです。
\\	(こころのこえ
\\	あー
\\	あー
\\	このひと、ドアにあしをはさんでる
\\	(こころのこえ
\\	あー
\\	あー
\\	このひと、ドアにあしをはさんでる
\\	いま、けいやくすると、けいひんがつきますよ。
\\	いま、けいやくすると、けいひんがつきますよ。
\\	せんざいごはこ、しょうひんけんさんぜんえんぶん、そして、ディズニーランドのパスポートをさしあげます。
\\	せんざいごはこ、しょうひんけんさんぜんえんぶん、そして、ディズニーランドのパスポートをさしあげます。
\\	わたし、らいねんのしがつにはひっこしするんですけど・・・
\\	わたし、らいねんのしがつにはひっこしするんですけど・・・
\\	では、ひっこすまで、しんぶんをこうどくしませんか?
\\	では、ひっこすまで、しんぶんをこうどくしませんか?
\\	すみません、けっこうです
\\	すみません、けっこうです
\\	ドアをしめますので、あしどかしてください
\\	ドアをしめますので、あしどかしてください
\\	あれ?もしかして・・・あかさかしょうがっこうのしゅっしん?
\\	あれ?もしかして・・・あかさかしょうがっこうのしゅっしん?
\\	はい?あ、けんちゃん
\\	はい?あ、けんちゃん
\\	ひさしぶり
\\	ひさしぶり
\\	けんちゃん。。。しんぶんのセールスしているんだ。
\\	けんちゃん。。。しんぶんのセールスしているんだ。
\\	あ・・・うん。がくひのためにね。
\\	あ・・・うん。がくひのためにね。
\\	わかった
\\	じゃ、けいやくする。
\\	わかった
\\	じゃ、けいやくする。
\\	ちしき
\\	しゃこうせい
\\	そんをする
\\	のべる
\\	けつろん
\\	そもそも
\\	りけい
\\	ぶんけい
\\	しっかりと
\\	とにかく
\\	だます
\\	くだらない
\\	さっか
\\	みなさんこんばんは。
\\	マキコです。
\\	みなさんこんばんは。
\\	マキコです。
\\	きょうのゲストは ベストセラーさっかの、なつめごろうさんです。よろしくおねがいします。
\\	きょうのゲストは ベストセラーさっかの、なつめごろうさんです。よろしくおねがいします。
\\	よろしく。
\\	よろしく。
\\	では、さいしょのしつもんをよみますね。
\\	では、さいしょのしつもんをよみますね。
\\	マキコさん、なつめせんせい、こんにちは。
\\	マキコさん、なつめせんせい、こんにちは。
\\	はい。こんにちは。
\\	はい。こんにちは。
\\	わたしは、こうこういちねんせいです。
\\	わたしは、こうこういちねんせいです。
\\	「すうがく」や「ぶつり」などのりけいのかもくはきらいではないのですが、「こくご」がいちばんすきです。
\\	「すうがく」や「ぶつり」などのりけいのかもくはきらいではないのですが、「こくご」がいちばんすきです。
\\	しょうらいは、しょうせつかになりたいとおもっているので、りけいのかもくはべんきょうしなくてもよいですか。
\\	しょうらいは、しょうせつかになりたいとおもっているので、りけいのかもくはべんきょうしなくてもよいですか。
\\	けつろんをのべます。「ぜったいにべんきょうしなさい
\\	けつろんをのべます。「ぜったいにべんきょうしなさい
\\	そもそも、ぶんけいとりけいにわけるかんがえかたがおかしいんだ。
\\	そもそも、ぶんけいとりけいにわけるかんがえかたがおかしいんだ。
\\	どうして、おかしいんですか。
\\	どうして、おかしいんですか。
\\	にんげん、ちしきをもっていて、そんをすることはないんだよ。
\\	にんげん、ちしきをもっていて、そんをすることはないんだよ。
\\	たとえば、わたしはしょうせつかだけど、
\\	しょうせつをかくときにはりけいのちしきがひつようになる。
\\	たとえば、わたしはしょうせつかだけど、
\\	しょうせつをかくときにはりけいのちしきがひつようになる。
\\	なるほどー。
\\	なるほどー。
\\	りけいのひとはしゃこうせいがないというイメージもあるみたいですけど。
\\	りけいのひとはしゃこうせいがないというイメージもあるみたいですけど。
\\	くだらない。どうして、そんなステレオタイプにだまされるんだ。
\\	くだらない。どうして、そんなステレオタイプにだまされるんだ。
\\	とにかく、しっかりといまは、なんでもべんきょうするように
\\	とにかく、しっかりといまは、なんでもべんきょうするように
\\	とくしゅうする
\\	たりる
\\	にほんかくち
\\	おうこく
\\	えきまえ
\\	いぶくろ
\\	てつ
\\	むり(な)
\\	しゅるい
\\	べつばら
\\	ねえ、ユウキ、このまえテレビで「デザートおうこく」をとくしゅうしているのをみたんだけど、いかない?
\\	ねえ、ユウキ、このまえテレビで「デザートおうこく」をとくしゅうしているのをみたんだけど、いかない?
\\	デザートおうこく?なにそれ?
\\	デザートおうこく?なにそれ?
\\	「フードテーマパーク」っていうのかな。きいたことない?
\\	「フードテーマパーク」っていうのかな。きいたことない?
\\	にほんかくちのおいしいデザートを、たべることができるんだよ。
\\	にほんかくちのおいしいデザートを、たべることができるんだよ。
\\	へー。ホントに、さおりはたべるのがすきだよな。ま、いいよ。しゅうまつ、いってみようか。
\\	へー。ホントに、さおりはたべるのがすきだよな。ま、いいよ。しゅうまつ、いってみようか。
\\	えっと、このチケットで、デザートをにじゅっしゅるいまでたべることができるのね。
\\	えっと、このチケットで、デザートをにじゅっしゅるいまでたべることができるのね。
\\	ひとりでにじゅっしゅるいもたべるのはむりじゃない?
\\	ひとりでにじゅっしゅるいもたべるのはむりじゃない?
\\	じゅっしゅるいようのチケットをかったほうがよくない?
\\	じゅっしゅるいようのチケットをかったほうがよくない?
\\	だいじょうぶだいじょうぶ。わたしはてつのいぶくろのもちぬしだから。
\\	だいじょうぶだいじょうぶ。わたしはてつのいぶくろのもちぬしだから。
\\	あ、プリンだ
\\	ゆうき、たべよう
\\	あ、プリンだ
\\	ゆうき、たべよう
\\	あーよくたべた。たのしかったねー。
\\	あーよくたべた。たのしかったねー。
\\	でも、20こじゃ、ぜんぜんたりなかったね。
\\	でも、20こじゃ、ぜんぜんたりなかったね。
\\	(ウップ)え?あ。。。そう?おれ、はらが(ゴロゴロゴロ)・・・。うっ。
\\	(ウップ)え?あ。。。そう?おれ、はらが(ゴロゴロゴロ)・・・。うっ。
\\	ね、ゆうき。えきまえにおいしいクレープやさんがあるの。クレープたベてかえろう
\\	ね、ゆうき。えきまえにおいしいクレープやさんがあるの。クレープたベてかえろう
\\	はらいっぱいだよ。これいじょうたべるのはむり。(ウップ)
\\	はらいっぱいだよ。これいじょうたべるのはむり。(ウップ)
\\	デザートはべつはら
\\	デザートはべつはら
\\	しょうがない
\\	かりる
\\	かしだしちゅう
\\	じたく
\\	さくひん
\\	げつがく
\\	とうろくする
\\	(はぁ・・・)ただいま。
\\	(はぁ・・・)ただいま。
\\	おかえり。どうしたの?げんきないけど。
\\	おかえり。どうしたの?げんきないけど。
\\	が かしだしちゅうだったのよ。かんこくドラマ。
\\	が かしだしちゅうだったのよ。かんこくドラマ。
\\	つぎのストーリーがきになってきになって、しょうがないのよ。
\\	つぎのストーリーがきになってきになって、しょうがないのよ。
\\	あのみせは、かりられている
\\	ばかりで、ほんとうにこまるわ。
\\	あのみせは、かりられている
\\	ばかりで、ほんとうにこまるわ。
\\	だったら、ネットで
\\	をかりればよいじゃない?
\\	だったら、ネットで
\\	をかりればよいじゃない?
\\	おみせにはおいてないさくひんもあるって、ともだちがいってたよ。
\\	おみせにはおいてないさくひんもあるって、ともだちがいってたよ。
\\	インターネットでかりられるの?
\\	インターネットでかりられるの?
\\	そうそう。じぶんがみたい
\\	をよやくすると、じたくにとどくんだって。
\\	そうそう。じぶんがみたい
\\	をよやくすると、じたくにとどくんだって。
\\	ほんとうに
\\	かえすときは?
\\	ほんとうに
\\	かえすときは?
\\	みおわった
\\	をポストにいれるだけ。
\\	みおわった
\\	をポストにいれるだけ。
\\	へーべんりね
\\	でもたかいんでしょ?
\\	へーべんりね
\\	でもたかいんでしょ?
\\	そんなにたかくないとおもうよ。ちょっと、しらべるね。
\\	そんなにたかくないとおもうよ。ちょっと、しらべるね。
\\	あ、げつがくにせんえんで、はちまいまでかりることができるって。
\\	あ、げつがくにせんえんで、はちまいまでかりることができるって。
\\	おかあさんがいつもかりているおみせよりやすいじゃない。
\\	おかあさんがいつもかりているおみせよりやすいじゃない。
\\	さいしょのいっかげつは、ただで、かりることができるってかいてあるよ。
\\	さいしょのいっかげつは、ただで、かりることができるってかいてあるよ。
\\	すごい。いますぐ、とうろくして。
\\	すごい。いますぐ、とうろくして。
\\	いいよ。
\\	いいよ。
\\	じゃ、わたしのアメリカのドラマもいっしょにかりてね。
\\	じゃ、わたしのアメリカのドラマもいっしょにかりてね。
\\	ダメよ。じぶんでかりなさい。
\\	ダメよ。じぶんでかりなさい。
\\	しゅっちょう
\\	うわさ
\\	いけん
\\	きほんりょうきん
\\	しくみ
\\	かいてんずし
\\	まわる
\\	かさんする
\\	きかい
\\	こんにちは、みなさん。
\\	こんにちは、みなさん。
\\	かいしゃでかいてんずしをたべられるといううわさをきいたのですが、ほんとうなのでしょうか。
\\	かいしゃでかいてんずしをたべられるといううわさをきいたのですが、ほんとうなのでしょうか。
\\	きょうは、ここのかいしゃにかいてんずしがきているときいたのですが。
\\	きょうは、ここのかいしゃにかいてんずしがきているときいたのですが。
\\	(コンコン)しつれいします。
\\	(コンコン)しつれいします。
\\	こんにちは。
\\	こんにちは。
\\	すごい
\\	オフィスのなかで、おすしがまわっています
\\	すごい
\\	オフィスのなかで、おすしがまわっています
\\	らっしゃい
\\	ようこそ、しゅっちょうかいてんずしへ
\\	らっしゃい
\\	ようこそ、しゅっちょうかいてんずしへ
\\	しゅっちょうかいてんずし?このきかいは、もってきたんですか。
\\	しゅっちょうかいてんずし?このきかいは、もってきたんですか。
\\	そうだよ。きほんりょうきんはにまん。
\\	そうだよ。きほんりょうきんはにまん。
\\	それにたべたふんのりょうきんをかさんするシステムだよ。
\\	それにたべたふんのりょうきんをかさんするシステムだよ。
\\	そして、なんと、おすしひとさらは、ふつうのかいてんずしとは、かわらないねだん
\\	そして、なんと、おすしひとさらは、ふつうのかいてんずしとは、かわらないねだん
\\	なるほどー。
\\	なるほどー。
\\	きかいをセットアップするだいきんと、いたまえさんにきてもらうだいきんが2まんえん。
\\	きかいをセットアップするだいきんと、いたまえさんにきてもらうだいきんが2まんえん。
\\	あとは、おすしをたべたりょうきんをはらえばいいんですね。
\\	あとは、おすしをたべたりょうきんをはらえばいいんですね。
\\	それは、おてごろですね
\\	それは、おてごろですね
\\	だろ?
\\	だろ?
\\	どうして、しゅっちょうかいてんずしをはじめたんですか?
\\	どうして、しゅっちょうかいてんずしをはじめたんですか?
\\	じたくやかいしゃでかいてんずしをたのしみたいといういけんがあってね。
\\	じたくやかいしゃでかいてんずしをたのしみたいといういけんがあってね。
\\	すばらしいですね
\\	すばらしいですね
\\	かいしゃやイベントだけでなく、ひとがたくさんあつまるホームパーティーなどでも、おすすめですね
\\	かいしゃやイベントだけでなく、ひとがたくさんあつまるホームパーティーなどでも、おすすめですね
\\	やせる
\\	じつは
\\	ほそい
\\	くだもの
\\	はだ
\\	ぺこぺこ
\\	やしょく
\\	ひさしぶりー。あれー?まゆちゃん、やせたー?
\\	ひさしぶりー。あれー?まゆちゃん、やせたー?
\\	あ、わかる?すこしほそくなったでしょ。
\\	あ、わかる?すこしほそくなったでしょ。
\\	じつは、ダイエットしているんだー。
\\	じつは、ダイエットしているんだー。
\\	そうなんだー
\\	なにしてるの?
\\	そうなんだー
\\	なにしてるの?
\\	フルーツダイエット。すごくかんたんだよ。
\\	フルーツダイエット。すごくかんたんだよ。
\\	まいあさ、くだものをたべるようにするだけ。
\\	まいあさ、くだものをたべるようにするだけ。
\\	へー
\\	いつごろからはじめたの?
\\	へー
\\	いつごろからはじめたの?
\\	はんとしまえ。
\\	はんとしまえ。
\\	はだもきれいになったんじゃない?
\\	はだもきれいになったんじゃない?
\\	いいなぁ・・・。わたしもがんばろうっと。
\\	いいなぁ・・・。わたしもがんばろうっと。
\\	いま、なにかしているの?
\\	いま、なにかしているの?
\\	うん。じつはウォーキングダイエットをはじめたの。
\\	うん。じつはウォーキングダイエットをはじめたの。
\\	へー。けんこうによさそう
\\	へー。けんこうによさそう
\\	まいあさ、こうえんをあるくようにしているの。
\\	まいあさ、こうえんをあるくようにしているの。
\\	それから、ゆうはんもはやいじかんにたべるようにしているんだ。
\\	それから、ゆうはんもはやいじかんにたべるようにしているんだ。
\\	へー。でも、まいあさあるくのは、たいへんでしょ。えらいねぇ。
\\	へー。でも、まいあさあるくのは、たいへんでしょ。えらいねぇ。
\\	あたしあるくのすきだから、だいじょうぶ。でも・・・
\\	あたしあるくのすきだから、だいじょうぶ。でも・・・
\\	でも?
\\	でも?
\\	ウォーキングはじめてからふとったんだよね。
\\	ウォーキングはじめてからふとったんだよね。
\\	うんどうすると、おなかがすごいすくから、ごはんをたくさんたべるようになって・・・。
\\	うんどうすると、おなかがすごいすくから、ごはんをたくさんたべるようになって・・・。
\\	ああ・・・。
\\	ああ・・・。
\\	それに、ゆうはんのじかんがはやいから、よるじゅうじにはおなかぺこぺこなの。
\\	それに、ゆうはんのじかんがはやいから、よるじゅうじにはおなかぺこぺこなの。
\\	だから、やしょくをたべるようになって・・・。さんキロくらいふとった。
\\	だから、やしょくをたべるようになって・・・。さんキロくらいふとった。
\\	でんしれんじ
\\	あがる
\\	えいようそ
\\	ぽたぽた
\\	へらす
\\	あげる
\\	こわれる
\\	こわす
\\	おとす
\\	おちる
\\	あぶら
\\	きのう
\\	よぶん
\\	はい、きょうごしょうかいするしんせいひんは、こちら
\\	はい、きょうごしょうかいするしんせいひんは、こちら
\\	シャーポのでんしれんじ「ヘルカロ」です
\\	シャーポのでんしれんじ「ヘルカロ」です
\\	たいへんにんきのしょうひんですね。
\\	たいへんにんきのしょうひんですね。
\\	そうです
\\	しおとあぶらのりょうをへらすことができるので、とてもけんこうにいいんです。
\\	そうです
\\	しおとあぶらのりょうをへらすことができるので、とてもけんこうにいいんです。
\\	なるほど。
\\	なるほど。
\\	たとえば、からあげをつくりましょう。
\\	たとえば、からあげをつくりましょう。
\\	はい。
\\	はい。
\\	このヘルカロはよぶんなあぶらをおとします。
\\	このヘルカロはよぶんなあぶらをおとします。
\\	みえますか?あぶらがおちていますよね。
\\	みえますか?あぶらがおちていますよね。
\\	うわー、ほんとうですね。ぽたぽたおちてる・・・。
\\	うわー、ほんとうですね。ぽたぽたおちてる・・・。
\\	このように、ヘルシーにからあげをあげることもできるんです。
\\	このように、ヘルシーにからあげをあげることもできるんです。
\\	はい、どうぞ。たべてください。
\\	はい、どうぞ。たべてください。
\\	(もぐもぐ)おいしい
\\	ヘルシーなのに、おいしくあがるんですね。
\\	(もぐもぐ)おいしい
\\	ヘルシーなのに、おいしくあがるんですね。
\\	そうなんですよー。しかも、この「ヘルカロ」はビタミンなどもこわしません。
\\	そうなんですよー。しかも、この「ヘルカロ」はビタミンなどもこわしません。
\\	えいようそがこわれないのはすばらしいですね。
\\	えいようそがこわれないのはすばらしいですね。
\\	さらに
\\	あたためたりょうりのカロリーは、じどうでひょうじされます。
\\	さらに
\\	あたためたりょうりのカロリーは、じどうでひょうじされます。
\\	それは、べんりなきのうですね
\\	さすが、「ヘルカロ」
\\	それは、べんりなきのうですね
\\	さすが、「ヘルカロ」
\\	で、きになるおねだんは?
\\	で、きになるおねだんは?
\\	こんかいもおどろきのかかく
\\	こんかいもおどろきのかかく
\\	なんと、さんまんきゅうせんはっぴゃくえん
\\	どうでしょう
\\	なんと、さんまんきゅうせんはっぴゃくえん
\\	どうでしょう
\\	なれる
\\	けしょうひん
\\	さいふ
\\	せつやく
\\	ながれ
\\	しゃいんしょくどう
\\	こっそり
\\	えいぎょうする
\\	びよう
\\	みかにあうのは、そつぎょういらいだよな。どう?しごとはなれた?
\\	みかにあうのは、そつぎょういらいだよな。どう?しごとはなれた?
\\	うん、まあまあかなー。さいきんやっとなれてきたかんじ。
\\	うん、まあまあかなー。さいきんやっとなれてきたかんじ。
\\	そうかー、おれもようやくしごとのながれがわかってきたよ。
\\	そうかー、おれもようやくしごとのながれがわかってきたよ。
\\	みかは、けしょうひんがいしゃではたらいているんだよね?
\\	みかは、けしょうひんがいしゃではたらいているんだよね?
\\	そう。うちのかいしゃ、じょせいがおおいから、しゃいんしょくどうのメニューのよこにはカロリーがかいてあるんだ。
\\	そう。うちのかいしゃ、じょせいがおおいから、しゃいんしょくどうのメニューのよこにはカロリーがかいてあるんだ。
\\	びようとけんこうのために、やさいをたくさんつかったメニューもたくさんあって、けっこうおいしいんだ。
\\	びようとけんこうのために、やさいをたくさんつかったメニューもたくさんあって、けっこうおいしいんだ。
\\	おんなのひとがおおいとそうなんだね。
\\	おんなのひとがおおいとそうなんだね。
\\	あと、けんこうをチェックするためのコーナーもあるんだよー。
\\	あと、けんこうをチェックするためのコーナーもあるんだよー。
\\	かつおのかいしゃもしゃいんしょくどうあるでしょ?
\\	かつおのかいしゃもしゃいんしょくどうあるでしょ?
\\	あるある。うちのしゃいんしょくどう、むりょうなんだ。
\\	あるある。うちのしゃいんしょくどう、むりょうなんだ。
\\	えーー
\\	タダなの?
\\	えーー
\\	タダなの?
\\	そう。すごいだろ?しゃいんのために、ってしゃちょうがきめたんだって。
\\	そう。すごいだろ?しゃいんのために、ってしゃちょうがきめたんだって。
\\	いいなー。おさいふにやさしいね。
\\	いいなー。おさいふにやさしいね。
\\	うん。でも、せつやくのためだけじゃなくて、はたらこうっていうきもちがつよくなるよ。
\\	うん。でも、せつやくのためだけじゃなくて、はたらこうっていうきもちがつよくなるよ。
\\	しかも、かいしゃにながくのこっているひとのために、よるもえいぎょうしてるんだ。
\\	しかも、かいしゃにながくのこっているひとのために、よるもえいぎょうしてるんだ。
\\	うらやましい
\\	こんど、こっそりたべにいきたい
\\	うらやましい
\\	こんど、こっそりたべにいきたい
\\	しきり
\\	こってり
\\	めん
\\	こうもく
\\	ちゅうもんする
\\	おーだーようし
\\	しゅうちゅうする
\\	のれん
\\	りょうわき
\\	しょっけん
\\	かため
\\	こんにちは、ラーメンに くわしい、 いわがみさんと いっしょに、さいきん わだいになっている ユニークなラーメンやさんに いってみようと おもいます。
\\	こんにちは、ラーメンに くわしい、 いわがみさんと いっしょに、さいきん わだいになっている ユニークなラーメンやさんに いってみようと おもいます。
\\	いわがみさん、よろしくおねがいします。
\\	いわがみさん、よろしくおねがいします。
\\	こんかい しょうかいしたい ラーメンてんは こちらの「いちばん
\\	(いちばんぼし)です。
\\	こんかい しょうかいしたい ラーメンてんは こちらの「いちばん
\\	(いちばんぼし)です。
\\	では、さっそく はいってみましょう。
\\	では、さっそく はいってみましょう。
\\	(ガラガラガラ)まず さいしょに、 このきかいに おかねをいれて、しょっけんを かいます。
\\	(ガラガラガラ)まず さいしょに、 このきかいに おかねをいれて、しょっけんを かいます。
\\	おや? カウンターの それぞれ せきの りょうわきに しきりと、まえに のれんが ありますねー。
\\	おや? カウンターの それぞれ せきの りょうわきに しきりと、まえに のれんが ありますねー。
\\	そうなんですよ。
\\	そうなんですよ。
\\	このしきりは、 おきゃくさんに ラーメンの あじに しゅうちゅうしてもらうために つくってあります。
\\	このしきりは、 おきゃくさんに ラーメンの あじに しゅうちゅうしてもらうために つくってあります。
\\	このカウンターせきに すわると、りょうわきも まえも みえませんからね。
\\	このカウンターせきに すわると、りょうわきも まえも みえませんからね。
\\	なるほどー。
\\	なるほどー。
\\	あと、このシステムがあるため、 じょせいのおきゃくさんも おおいんです。
\\	あと、このシステムがあるため、 じょせいのおきゃくさんも おおいんです。
\\	ああ、ひとりでは ラーメンやさんに はいりにくい・・・と かんじるじょせいは おおいですからね。
\\	ああ、ひとりでは ラーメンやさんに はいりにくい・・・と かんじるじょせいは おおいですからね。
\\	そのとおりです。
\\	そのとおりです。
\\	それから、このおみせの ラーメンのスープは トンコツあじなんですが・・・はい、オーダーようしと えんぴつ。
\\	それから、このおみせの ラーメンのスープは トンコツあじなんですが・・・はい、オーダーようしと えんぴつ。
\\	ん?このオーダーようしに じぶんでかいて、ちゅうもんするんですか?
\\	ん?このオーダーようしに じぶんでかいて、ちゅうもんするんですか?
\\	そうです。じぶんの たべたいあじになるように、 あじを えらぶんです。
\\	そうです。じぶんの たべたいあじになるように、 あじを えらぶんです。
\\	オーダーようしに かいてある こうもくに マルを してください。
\\	オーダーようしに かいてある こうもくに マルを してください。
\\	へー。じゃ、わたしは・・・・スープは...こってり。(マルをするおと)めんは かため・・・
\\	へー。じゃ、わたしは・・・・スープは...こってり。(マルをするおと)めんは かため・・・
\\	うかがう
\\	うらみ
\\	あやしい
\\	おそろしい
\\	かこ
\\	すいりしょうせつか
\\	ほうせきてん
\\	てんしゅ
\\	にぎる
\\	みなさん、こんばんは。
\\	マキコ です。
\\	みなさん、こんばんは。
\\	マキコ です。
\\	きょうは、すいりしょうせつか、あずまごろう せんせいに せんせいの しんさく「ダイングメッセージ」について おはなしを うかがいたいと おもいます。
\\	きょうは、すいりしょうせつか、あずまごろう せんせいに せんせいの しんさく「ダイングメッセージ」について おはなしを うかがいたいと おもいます。
\\	せんせい よろしくおねがいします。
\\	せんせい よろしくおねがいします。
\\	はい。このしょうせつは、「ダイングメッセージ」で はんにんが わかったところから、ものがたりが はじまります。
\\	はい。このしょうせつは、「ダイングメッセージ」で はんにんが わかったところから、ものがたりが はじまります。
\\	「あるほうせきてんの てんしゅが ころされた。とうぜん、かぎは しめてあった。
\\	「あるほうせきてんの てんしゅが ころされた。とうぜん、かぎは しめてあった。
\\	みせの ゆかには、ダイアモンドが なんこも おちていた。
\\	みせの ゆかには、ダイアモンドが なんこも おちていた。
\\	てんしゅは ダイアモンドを ひとつ にぎっていたが、おみせの ほうせきや げんきんは ひとつも ぬすまれていなかった。
\\	てんしゅは ダイアモンドを ひとつ にぎっていたが、おみせの ほうせきや げんきんは ひとつも ぬすまれていなかった。
\\	けいさつは てんしゅに うらみを もったものが はんにんであると かんがえ、のだ せつこ、なかむら ゆきこ、さいとう じゅんこの 3にんが あやしいと かんがえた。
\\	けいさつは てんしゅに うらみを もったものが はんにんであると かんがえ、のだ せつこ、なかむら ゆきこ、さいとう じゅんこの 3にんが あやしいと かんがえた。
\\	ひとりの けいじが、ようぎしゃの たんじょうびを しらべた。
\\	ひとりの けいじが、ようぎしゃの たんじょうびを しらべた。
\\	のだは 3がつうまれ、なかむらは 4がつうまれで、さいとうは 12がつうまれである。」
\\	のだは 3がつうまれ、なかむらは 4がつうまれで、さいとうは 12がつうまれである。」
\\	おおお、かんたんに はんにんが わかりましたね。わたしにも はんにんが だれか わかります。
\\	おおお、かんたんに はんにんが わかりましたね。わたしにも はんにんが だれか わかります。
\\	そして、そこから、はんにんの かなしいじんせい、ころされたてんしゅの おそろしいかこが わかっていく・・・という ものがたり です。
\\	そして、そこから、はんにんの かなしいじんせい、ころされたてんしゅの おそろしいかこが わかっていく・・・という ものがたり です。
\\	では、ここで コマーシャルが はいります。
\\	では、ここで コマーシャルが はいります。
\\	れきし
\\	ふぁいるする
\\	ぷらもでる
\\	ねんだいじゅん
\\	ほうもんする
\\	せんごくぶしょう
\\	ながめる
\\	しりょう
\\	ほんだな
\\	しろ
\\	みなさんは、「れきじょ」という ことばを きいたことは ありますか?
\\	みなさんは、「れきじょ」という ことばを きいたことは ありますか?
\\	「れきしが すきな おんなの ひと」を みじかくして、「れきじょ」ですが、 こんかいは、 そんな「れきじょ」さん、 やまだ はなさんの おうちを ほうもんしたいと おもいます。
\\	「れきしが すきな おんなの ひと」を みじかくして、「れきじょ」ですが、 こんかいは、 そんな「れきじょ」さん、 やまだ はなさんの おうちを ほうもんしたいと おもいます。
\\	はーい。(ガチャ)
\\	はーい。(ガチャ)
\\	こんにちは、よろしく おねがいします。
\\	こんにちは、よろしく おねがいします。
\\	こちらこそ。
\\	こちらこそ。
\\	おじゃまします。
\\	おじゃまします。
\\	おっと、ものすごい かずの プラモデルが かざって ありますね。
\\	おっと、ものすごい かずの プラモデルが かざって ありますね。
\\	はい。これは、せんごくぶしょうの プラモデルなんです。
\\	はい。これは、せんごくぶしょうの プラモデルなんです。
\\	おおお、ほんだなは、さすがに れきし かんけいの ほん ばかり ですね。
\\	おおお、ほんだなは、さすがに れきし かんけいの ほん ばかり ですね。
\\	これは ねんだいじゅんに ならべて あるんですね?
\\	これは ねんだいじゅんに ならべて あるんですね?
\\	はい。これが、わたしの れきしファイルです。
\\	はい。これが、わたしの れきしファイルです。
\\	あつめた しりょうを ファイルして おくんです。
\\	あつめた しりょうを ファイルして おくんです。
\\	あと、きゅうじつに おとずれた おしろの しゃしんも いれて おきますね。
\\	あと、きゅうじつに おとずれた おしろの しゃしんも いれて おきますね。
\\	これを ながめるのが すき なんですよー。
\\	これを ながめるのが すき なんですよー。
\\	すごい ですねー。ところで、 どうして れきしが すきに なったんですか?
\\	すごい ですねー。ところで、 どうして れきしが すきに なったんですか?
\\	うーん。まえに つきあっていた かれしが、 れきしの ゲームを しているのを みて、 おもしろそうだな・・・と おもったのが、 きっかけ です。
\\	うーん。まえに つきあっていた かれしが、 れきしの ゲームを しているのを みて、 おもしろそうだな・・・と おもったのが、 きっかけ です。
\\	なるほど。
\\	なるほど。
\\	きづいたら、 かれしよりも れきしの ほうが すきに なって いましたが・・・ははは・・・。
\\	きづいたら、 かれしよりも れきしの ほうが すきに なって いましたが・・・ははは・・・。
\\	よほう
\\	こんしゅういっぱい
\\	へんこうする
\\	はれおとこ
\\	のんびりする
\\	てるてる坊主
\\	おににかなぼう
\\	あめおんな
\\	こんにちは、ごごからあめのよほうですので、かさをわすれないでください。
\\	こんにちは、ごごからあめのよほうですので、かさをわすれないでください。
\\	また、こんしゅういっぱいは、あめがつづくようです。
\\	また、こんしゅういっぱいは、あめがつづくようです。
\\	えー、あめがふるのー?
\\	えー、あめがふるのー?
\\	なーんだ。こうえんでおべんとうたべたかったのにー。
\\	なーんだ。こうえんでおべんとうたべたかったのにー。
\\	そうね。じゃあ、プランをへんこうする?
\\	そうね。じゃあ、プランをへんこうする?
\\	いや、だいじょうぶ。こうえんにいけば、はれるから。
\\	いや、だいじょうぶ。こうえんにいけば、はれるから。
\\	ほんとう
\\	ほんとう
\\	そうだ。おとうさんは、はれおとこなんだよ。
\\	そうだ。おとうさんは、はれおとこなんだよ。
\\	はれおとこ?
\\	はれおとこ?
\\	おとうさんが、だいじなイベントで、そとにでかけるときは、あめがふらないんだよ。
\\	おとうさんが、だいじなイベントで、そとにでかけるときは、あめがふらないんだよ。
\\	そういうひとを「はれおとこ」「はれおんな」っていうんだ。
\\	そういうひとを「はれおとこ」「はれおんな」っていうんだ。
\\	すごーい
\\	すごーい
\\	そういっても・・・そとをみてよ。あめがふりそう。
\\	そういっても・・・そとをみてよ。あめがふりそう。
\\	そんなことをいっっているから、あめがふっちゃうんだよ。
\\	そんなことをいっっているから、あめがふっちゃうんだよ。
\\	よほうもいっているし。きょうは、いえでのんびりしない?
\\	よほうもいっているし。きょうは、いえでのんびりしない?
\\	えー、おそといきたい。
\\	えー、おそといきたい。
\\	きのう、てるてるぼうずつくったの。だから、あめふらないよ。
\\	きのう、てるてるぼうずつくったの。だから、あめふらないよ。
\\	はれおとことてるてるぼうずで、おににかなぼうだろう?
\\	はれおとことてるてるぼうずで、おににかなぼうだろう?
\\	うーん。
\\	うーん。
\\	どうして、そんなにいやなんだよ?
\\	どうして、そんなにいやなんだよ?
\\	じつは、わたし、あめおんななのよ。
\\	じつは、わたし、あめおんななのよ。
\\	そうか、あめおんなだったのか・・・。それは、こまったな。
\\	そうか、あめおんなだったのか・・・。それは、こまったな。
\\	もー、おそといこうよぉー。
\\	もー、おそといこうよぉー。
\\	ひだりきき
\\	むりやり
\\	のうかがくしゃ
\\	むり
\\	そうだんする
\\	ききて
\\	はったつ
\\	げいじゅつか
\\	たまる
\\	みなさん、こんばんは。
\\	マキコです。
\\	みなさん、こんばんは。
\\	マキコです。
\\	きょうのゲストは、のうかがくしゃのさかいきょうじゅです。せんせい、よろしくおねがいします。
\\	きょうのゲストは、のうかがくしゃのさかいきょうじゅです。せんせい、よろしくおねがいします。
\\	よろしくおねがいします。
\\	よろしくおねがいします。
\\	では、リスナーからのしつもんです。
\\	では、リスナーからのしつもんです。
\\	「こんにちは。わたしにはしょうがっこういちねんせいのむすこがいます。
\\	「こんにちは。わたしにはしょうがっこういちねんせいのむすこがいます。
\\	むすこは、ひだりてでじをかいたり、はしをもったりします。」
\\	むすこは、ひだりてでじをかいたり、はしをもったりします。」
\\	おお、ひだりききのむすこさんですね。いいじゃないですか。
\\	おお、ひだりききのむすこさんですね。いいじゃないですか。
\\	「さいきん、むすこが、がっこうでせんせいに「みぎてをつかいなさい」と、いわれたようです。
\\	「さいきん、むすこが、がっこうでせんせいに「みぎてをつかいなさい」と、いわれたようです。
\\	むすこはみぎてにかえたくないらしいのですが、どうしたらいいでしょうか。」
\\	むすこはみぎてにかえたくないらしいのですが、どうしたらいいでしょうか。」
\\	なるほど・・・。
\\	なるほど・・・。
\\	せんせい、これはどうなんでしょう?
\\	せんせい、これはどうなんでしょう?
\\	なおすひつようはありません。ききてをむりにかえるのは、のうのはったつによくないです。
\\	なおすひつようはありません。ききてをむりにかえるのは、のうのはったつによくないです。
\\	のうのはったつ・・・。
\\	のうのはったつ・・・。
\\	あ、そういえば、げいじゅつかはひだりききがおおいらしいですね。
\\	あ、そういえば、げいじゅつかはひだりききがおおいらしいですね。
\\	そうですね。ダビンチやミケランジェロもひだりききだったようです。
\\	そうですね。ダビンチやミケランジェロもひだりききだったようです。
\\	また、ききてをかえるのは、たいへんなストレスがたまるようです。
\\	また、ききてをかえるのは、たいへんなストレスがたまるようです。
\\	え、そうなんですか。
\\	え、そうなんですか。
\\	ですので、むすこさんがひだりききのままでいいように、がっこうとそうだんするのがいいとおもいます。
\\	ですので、むすこさんがひだりききのままでいいように、がっこうとそうだんするのがいいとおもいます。
\\	そうですね。それがいちばんよさそうですね。
\\	そうですね。それがいちばんよさそうですね。
\\	せんせい、きょうはありがとうございました。
\\	せんせい、きょうはありがとうございました。
\\	いえいえ。
\\	いえいえ。
\\	それでは。
\\	それでは。
\\	めんせつ
\\	てんちょう
\\	りれきしょ
\\	じきゅう
\\	はたらきはじめる
\\	こうじちゅう
\\	ぎりぎり
\\	うける
\\	うーん。もうちょっと・・・。
\\	うーん。もうちょっと・・・。
\\	ヤベ
\\	きょうはカフェのバイトのめんせつだった
\\	ヤベ
\\	きょうはカフェのバイトのめんせつだった
\\	なんだよー。このみち、こうじちゅう?ふざけんなよ・・・。
\\	なんだよー。このみち、こうじちゅう?ふざけんなよ・・・。
\\	あ
\\	ここをひだりにまがっていけばおみせにいけるんだった
\\	あ
\\	ここをひだりにまがっていけばおみせにいけるんだった
\\	めんせつはいちじからだったよな。いまごじゅっぷんだから、ギリギリまにあうな。
\\	めんせつはいちじからだったよな。いまごじゅっぷんだから、ギリギリまにあうな。
\\	(ドアをあけるおと・ギイィィ)あの、すみません。おかだともうします。めんせつをうけにきたのですが・・・。
\\	(ドアをあけるおと・ギイィィ)あの、すみません。おかだともうします。めんせつをうけにきたのですが・・・。
\\	あ、めんせつのかたですね。じゃ、こちらへどうぞ。
\\	あ、めんせつのかたですね。じゃ、こちらへどうぞ。
\\	はい。
\\	はい。
\\	どうぞ、すわってください。
\\	どうぞ、すわってください。
\\	わたしは、てんちょうのかたおかです。よろしく。
\\	わたしは、てんちょうのかたおかです。よろしく。
\\	よろしくおねがいします。
\\	よろしくおねがいします。
\\	ええっと・・・おかだけんじくんだったね。
\\	ええっと・・・おかだけんじくんだったね。
\\	あ、はい。そうです。
\\	あ、はい。そうです。
\\	このまえおくってくれた、りれきしょをみたんだけど、まえにも、レストランではたらいていたことがあるんだったね。
\\	このまえおくってくれた、りれきしょをみたんだけど、まえにも、レストランではたらいていたことがあるんだったね。
\\	はい。ウェイターをしてました。あと、のみものやデザートをつくったりもしていました。
\\	はい。ウェイターをしてました。あと、のみものやデザートをつくったりもしていました。
\\	ふーん。おかだくんはだいがくせいだったね。
\\	ふーん。おかだくんはだいがくせいだったね。
\\	そうです。
\\	そうです。
\\	よるおそくてもだいじょうぶ?
\\	よるおそくてもだいじょうぶ?
\\	よるは、じゅうにじまででしたよね。
\\	よるは、じゅうにじまででしたよね。
\\	みせは、じゅうにじまでだけど、スタッフはいちじまでだね。
\\	みせは、じゅうにじまでだけど、スタッフはいちじまでだね。
\\	なんで
\\	ふゆやすみ
\\	ピアノ
\\	おわる
\\	きんじょ
\\	ひく
\\	かよう
\\	こうこうせい
\\	こうこう
\\	おこす
\\	やばい
\\	(ピピピピー。)
\\	(ピピピピー。)
\\	うーん。え?はちじ?(バタバタバタバタ)
\\	うーん。え?はちじ?(バタバタバタバタ)
\\	もう
\\	おかあさん、なんで おこして くれなかった の?
\\	もう
\\	おかあさん、なんで おこして くれなかった の?
\\	なんかいも おこした わ よ。
\\	なんかいも おこした わ よ。
\\	いって きます
\\	いって きます
\\	わたしのなまえは のざわ なみ。たばたこうこうに かよっている こうこう にねんせい。
\\	わたしのなまえは のざわ なみ。たばたこうこうに かよっている こうこう にねんせい。
\\	おっす、ノザワナ。
\\	おっす、ノザワナ。
\\	やめてよ。「のざわ」か、「なみ」ってよんでっ て なんども いっているでしょう。
\\	やめてよ。「のざわ」か、「なみ」ってよんでっ て なんども いっているでしょう。
\\	いそげ、ちこくするぞ。
\\	いそげ、ちこくするぞ。
\\	かれは うちのきんじょにすんでいる なすみのる。おなじ たばたこうこう にねんせい。
\\	かれは うちのきんじょにすんでいる なすみのる。おなじ たばたこうこう にねんせい。
\\	くちも あたまも わるいけど、みのるの ひくピアノは すごい。
\\	くちも あたまも わるいけど、みのるの ひくピアノは すごい。
\\	みのるのおかあさんは、ピアノの せんせいで いえに ピアノが さんだい も ある。
\\	みのるのおかあさんは、ピアノの せんせいで いえに ピアノが さんだい も ある。
\\	ねぇ、みのる、ふゆやすみの まえに でた しゅくだい、おわった?
\\	ねぇ、みのる、ふゆやすみの まえに でた しゅくだい、おわった?
\\	へ?
\\	へ?
\\	ほんを じゅっさつよんで、レポートをひゃくまいかく しゅくだい。
\\	ほんを じゅっさつよんで、レポートをひゃくまいかく しゅくだい。
\\	ハワイ
\\	ようし
\\	もんだい
\\	おとしだま
\\	すわる
\\	にあう
\\	あける
\\	ブレスレット
\\	きにいる
\\	わかる
\\	ぜんぶ
\\	あけましておめでとう。
\\	あけましておめでとう。
\\	ことしも よろしくね、さくらこ。ハワイはどうだった。
\\	ことしも よろしくね、さくらこ。ハワイはどうだった。
\\	おおばさん、ハワイ いったの?
\\	おおばさん、ハワイ いったの?
\\	うん。ごはくなのかで いってきた。おばあちゃんが すんでいるんだ。
\\	うん。ごはくなのかで いってきた。おばあちゃんが すんでいるんだ。
\\	はい、これ おみやげ。きに いる かどうか わからないけど。
\\	はい、これ おみやげ。きに いる かどうか わからないけど。
\\	ありがとう
\\	あけても いい?
\\	ありがとう
\\	あけても いい?
\\	うん。
\\	うん。
\\	ブレスレット
\\	かわいい
\\	にほんも?いいの?
\\	ブレスレット
\\	かわいい
\\	にほんも?いいの?
\\	にあうかどうか・・・。
\\	にあうかどうか・・・。
\\	うるさい。
\\	うるさい。
\\	おおばさん、おれの おみやげは?
\\	おおばさん、おれの おみやげは?
\\	あるよ。あまいものが すきかどうか わからなかったけど、チョコ かったんだ。
\\	あるよ。あまいものが すきかどうか わからなかったけど、チョコ かったんだ。
\\	サンキュー。
\\	サンキュー。
\\	はい、すわって。あけまして おめでとう。
\\	はい、すわって。あけまして おめでとう。
\\	せんせいから おとしだまが あります。
\\	せんせいから おとしだまが あります。
\\	みんなが やすみちゅうに べんきょうしたかどう かしりたいので、いまから テストを します。
\\	みんなが やすみちゅうに べんきょうしたかどう かしりたいので、いまから テストを します。
\\	もんだいようしは ぜんぶで ごまい。テストじかんは さんじゅっぷん。
\\	もんだいようしは ぜんぶで ごまい。テストじかんは さんじゅっぷん。
\\	あそぶ
\\	たぶん
\\	あやまる
\\	わすれる
\\	はなし
\\	ひき
\\	ふえる
\\	かぞく
\\	あそびにいく
\\	なんだ
\\	ねぇねぇ、いまから、うちに あそびに こない?かぞくが いっぴき ふえたんだ。
\\	ねぇねぇ、いまから、うちに あそびに こない?かぞくが いっぴき ふえたんだ。
\\	かぞくが いっぴき?
\\	かぞくが いっぴき?
\\	そう。そのかぞくは これ。ジャーン
\\	この しゃしんが なにか わかる?
\\	そう。そのかぞくは これ。ジャーン
\\	この しゃしんが なにか わかる?
\\	ワンちゃんだ
\\	ねぇ、みのるも さくらこ の うちに あそびにいこうよ。
\\	ワンちゃんだ
\\	ねぇ、みのるも さくらこ の うちに あそびにいこうよ。
\\	う…ん。いきたいんだけど…。ちょっと…いま なんじか わかる?
\\	う…ん。いきたいんだけど…。ちょっと…いま なんじか わかる?
\\	よじはんだよ。
\\	よじはんだよ。
\\	やべ。ちょっと せんせいに はなしが あるんだ。あとで おおばさんのうちに いくよ。
\\	やべ。ちょっと せんせいに はなしが あるんだ。あとで おおばさんのうちに いくよ。
\\	うちが どこか わかる?がっこうの まえの しろいうち だよ。
\\	うちが どこか わかる?がっこうの まえの しろいうち だよ。
\\	ああ、あのいえね。わかった。
\\	ああ、あのいえね。わかった。
\\	まってるね。
\\	まってるね。
\\	あ、くりばやしくん。
\\	あ、くりばやしくん。
\\	のざわさん、みのる、どこに いったか しってる?
\\	のざわさん、みのる、どこに いったか しってる?
\\	せんせいと はなしが あるって いってたよ。どんな はなしか しらないけど。
\\	せんせいと はなしが あるって いってたよ。どんな はなしか しらないけど。
\\	ああ、しゅくだいわすれたから、あやまりに いったんだよ。たぶん。
\\	ああ、しゅくだいわすれたから、あやまりに いったんだよ。たぶん。
\\	なーんだ。
\\	なーんだ。
\\	すてる
\\	たいじゅう
\\	ゆるす
\\	えさ
\\	むせきにん
\\	つれていく
\\	じゅうい
\\	ひろう
\\	かわいそう
\\	えんりょ
\\	いいなぁ。やっぱり、いぬ、ほしいなぁ。
\\	いいなぁ。やっぱり、いぬ、ほしいなぁ。
\\	たいじゅうは1キロぐらい?
\\	たいじゅうは1キロぐらい?
\\	うん。いま、800グラムぐらいかな。
\\	うん。いま、800グラムぐらいかな。
\\	なんさい?
\\	なんさい?
\\	なんさいかわからないの。すていぬだったんだ。
\\	なんさいかわからないの。すていぬだったんだ。
\\	さむがっていたし、かわいそうだから、ひろってきたんだ。
\\	さむがっていたし、かわいそうだから、ひろってきたんだ。
\\	このいぬ・・・あし、どうしたの?いたそう。
\\	このいぬ・・・あし、どうしたの?いたそう。
\\	いたがっているんだよね。あした、じゅういさんにつれていくんだ。
\\	いたがっているんだよね。あした、じゅういさんにつれていくんだ。
\\	いぬをすてるむせきにんなひとがいるんだ…。ゆるせない。
\\	いぬをすてるむせきにんなひとがいるんだ…。ゆるせない。
\\	あ、そとにでたがってる。
\\	あ、そとにでたがってる。
\\	ちがうよ。えさをほしがっているんだよ。
\\	ちがうよ。えさをほしがっているんだよ。
\\	みなさん、いらっしゃい。
\\	みなさん、いらっしゃい。
\\	おじゃましています。
\\	おじゃましています。
\\	はい、こうちゃとケーキ。
\\	はい、こうちゃとケーキ。
\\	どうぞ、おかまいなく。(グー)あ・・・。
\\	どうぞ、おかまいなく。(グー)あ・・・。
\\	えんりょしないでいいのよ。わかいんだから。
\\	えんりょしないでいいのよ。わかいんだから。
\\	すみません。じゃ、えんりょなくいただきます。
\\	すみません。じゃ、えんりょなくいただきます。
\\	やおや
\\	しょうひぜい
\\	ちょうだい
\\	はんがく
\\	たまねぎ
\\	なす
\\	きゃべつ
\\	おすすめ
\\	おもい
\\	からだ
\\	おや
\\	アルバイト
\\	まいど
\\	らっしゃい。あ、せんせい。
\\	らっしゃい。あ、せんせい。
\\	あ、のざわさん?やおやさんで アルバイト?
\\	あ、のざわさん?やおやさんで アルバイト?
\\	あ、うちのおや、やおやを してるんです。
\\	あ、うちのおや、やおやを してるんです。
\\	たいへんだね。
\\	たいへんだね。
\\	はい。おもいもの はこばなきゃいけないから、からだ いたくなっちゃうし、おかねも もらえないし、ほんと、やに なりますよ。
\\	はい。おもいもの はこばなきゃいけないから、からだ いたくなっちゃうし、おかねも もらえないし、ほんと、やに なりますよ。
\\	で、せんせい、なに、かいます?
\\	で、せんせい、なに、かいます?
\\	おすすめ、ある?
\\	おすすめ、ある?
\\	きょうは…キャベツは にわりびき、なすとたまねぎは はんがくっすよ。
\\	きょうは…キャベツは にわりびき、なすとたまねぎは はんがくっすよ。
\\	じゃ、なすと たまねぎと キャベツ、ちょうだい。
\\	じゃ、なすと たまねぎと キャベツ、ちょうだい。
\\	はい。
\\	はい。
\\	キャベツ、よんぶんのいちしか いらないんだけど…。
\\	キャベツ、よんぶんのいちしか いらないんだけど…。
\\	あ、いいっすよ。きります。
\\	あ、いいっすよ。きります。
\\	えっと…。ひゃくにじゅうたすひゃくたすにじゅうごは、にひゃくよんじゅうご。で、しょうひぜいがごパーセントだから…。
\\	えっと…。ひゃくにじゅうたすひゃくたすにじゅうごは、にひゃくよんじゅうご。で、しょうひぜいがごパーセントだから…。
\\	にひゃくよんじゅうごかけるいってんぜろごは…にひゃくごじゅうななえんです
\\	にひゃくよんじゅうごかけるいってんぜろごは…にひゃくごじゅうななえんです
\\	はい。
\\	はい。
\\	まいどあり。
\\	まいどあり。
\\	こいぬ
\\	かう
\\	ひつよう
\\	ペットショップ
\\	ずっと
\\	かなしい
\\	かいぬし
\\	ふじゆう
\\	うまれつき
\\	ねんれい
\\	さがす
\\	どうじょう
\\	からて
\\	ほか
\\	あか
\\	おみまい
\\	じゅぎょう
\\	うわさ
\\	あかい
\\	おる
\\	こし
\\	ころぶ
\\	れんしゅう
\\	せんせい、くりばやしくんはやすみですか?
\\	せんせい、くりばやしくんはやすみですか?
\\	ああ、どうじょうにいくときに、ころんであしをおったそうだ。
\\	ああ、どうじょうにいくときに、ころんであしをおったそうだ。
\\	ええ?ころんでこしをうった?いたそう。
\\	ええ?ころんでこしをうった?いたそう。
\\	なあ、きいたか?うわさによると、くりばやし、どうじょうにいったときに、こしをおったみたいだぞ。
\\	なあ、きいたか?うわさによると、くりばやし、どうじょうにいったときに、こしをおったみたいだぞ。
\\	ええ?たいへん
\\	じゅぎょうがおわってからみんなでおみまいにいこうよ。
\\	ええ?たいへん
\\	じゅぎょうがおわってからみんなでおみまいにいこうよ。
\\	おみまいにいくまえに、おはなをかわない?
\\	おみまいにいくまえに、おはなをかわない?
\\	じゃ、このあかいはながいいよ。くりばやし、あかがすきだし。
\\	じゃ、このあかいはながいいよ。くりばやし、あかがすきだし。
\\	おみまいにいくとき、あかいはなをもっていっちゃいけないってきいたことがある。
\\	おみまいにいくとき、あかいはなをもっていっちゃいけないってきいたことがある。
\\	へー。じゃ、ほかのいろにしよう。
\\	へー。じゃ、ほかのいろにしよう。
\\	こんにちは。これ、おみまい。
\\	こんにちは。これ、おみまい。
\\	おみまい?
\\	おみまい?
\\	からてのれんしゅうちゅうにこしをおったって…。
\\	からてのれんしゅうちゅうにこしをおったって…。
\\	ぼくは、「どうじょうにいくときに、ころんであしをうった」っていったんだけど…。
\\	ぼくは、「どうじょうにいくときに、ころんであしをうった」っていったんだけど…。
\\	ペパロニピザ
\\	しんせき
\\	マカロニ
\\	てんちょう
\\	まぜる
\\	おゆ
\\	こむぎこ
\\	生地
\\	レシピぼん
\\	ちゅうもん
\\	そもそも
\\	20ふんくらいまえに、ペパロニピザを3まいちゅうもんしたのざわですけど...まだですか。
\\	20ふんくらいまえに、ペパロニピザを3まいちゅうもんしたのざわですけど...まだですか。
\\	もうしわけございません。いまつくっているところなので、もうすこしおまちください。
\\	もうしわけございません。いまつくっているところなので、もうすこしおまちください。
\\	てんちょう、ピザできましたか?
\\	てんちょう、ピザできましたか?
\\	いま、レシピほんをよんだところなんだ。
\\	いま、レシピほんをよんだところなんだ。
\\	いまからきじをつくるところだから、もうすこしじかんがかかるなぁ。
\\	いまからきじをつくるところだから、もうすこしじかんがかかるなぁ。
\\	さてと、こむぎこ300
\\	とおゆ200
\\	をまぜて・・・。
\\	さてと、こむぎこ300
\\	とおゆ200
\\	をまぜて・・・。
\\	ところで、てんちょう、ペパロふたってなんですか。
\\	ところで、てんちょう、ペパロふたってなんですか。
\\	マカロニのしんせきだろ。
\\	マカロニのしんせきだろ。
\\	さすがてんちょう。でも、はじめてピザのちゅうもんがきましたね。
\\	さすがてんちょう。でも、はじめてピザのちゅうもんがきましたね。
\\	そうだな、そもそも、うちはそばやだからな。
\\	そうだな、そもそも、うちはそばやだからな。
\\	あ、もしもし?あ、すみません。いま、おみせをでたところなんです。もうすこしおまちください。
\\	あ、もしもし?あ、すみません。いま、おみせをでたところなんです。もうすこしおまちください。
\\	てんちょう、いそいでください。
\\	てんちょう、いそいでください。
\\	むね
\\	つらい
\\	ゆめ
\\	さそう
\\	としょかん
\\	まいる
\\	もてる
\\	にゅうがく
\\	どきどきする
\\	であう
\\	わらう
\\	なすみのるさん。であったばかりで、まだ あなたのことを よく しらないけど、ちょうタイプです。
\\	なすみのるさん。であったばかりで、まだ あなたのことを よく しらないけど、ちょうタイプです。
\\	マジで?
\\	マジで?
\\	みのるくんを みるたび、むねが どきどきするの。だいすき。わたしと つきあって。
\\	みのるくんを みるたび、むねが どきどきするの。だいすき。わたしと つきあって。
\\	ええ?
\\	ええ?
\\	こうこうに にゅうがくしていらい、ずっと なすくんのことが すきでした。わたしと けっこんしてください。
\\	こうこうに にゅうがくしていらい、ずっと なすくんのことが すきでした。わたしと けっこんしてください。
\\	うそ?
\\	うそ?
\\	みのるのことが だいすき。
\\	みのるのことが だいすき。
\\	あなたのことを あいしています
\\	あなたのことを あいしています
\\	ムニャムニャ・・・・。もてるおとこは つらいなぁ。でへへへへ。まいったなぁー。
\\	ムニャムニャ・・・・。もてるおとこは つらいなぁ。でへへへへ。まいったなぁー。
\\	ねぇ、「としょかんで みんなで いっしょに べんきょうしよう」ってさそったひとは だれだった?
\\	ねぇ、「としょかんで みんなで いっしょに べんきょうしよう」ってさそったひとは だれだった?
\\	みのるだね。でも・・・きたとたん ねちゃったね。
\\	みのるだね。でも・・・きたとたん ねちゃったね。
\\	いいゆめ みているんでしょう。わらいながら ねてる。
\\	いいゆめ みているんでしょう。わらいながら ねてる。
\\	かんせい
\\	きゅうけい
\\	いみ
\\	せかい
\\	へいわ
\\	ふつう
\\	ねむる
\\	ほる
\\	うら
\\	すずめ
\\	ねこ
\\	しゅうごう
\\	ここはにっこうとうしょうぐうです。
\\	ここはにっこうとうしょうぐうです。
\\	これは1636ねんにとくがわいえみつによってかんせいされました。
\\	これは1636ねんにとくがわいえみつによってかんせいされました。
\\	では、あるきましょう。
\\	では、あるきましょう。
\\	これはねむりねこです。
\\	これはねむりねこです。
\\	「ひだりじんごろうによってつくられた」といわれています。
\\	「ひだりじんごろうによってつくられた」といわれています。
\\	みてください。ネコのうらには、スズメがほられています。
\\	みてください。ネコのうらには、スズメがほられています。
\\	ふつう、スズメはネコにたべられてしまいますよね。
\\	ふつう、スズメはネコにたべられてしまいますよね。
\\	でもすぐそこにスズメがいるのに、ネコはスズメをたべないでねむっていますね。
\\	でもすぐそこにスズメがいるのに、ネコはスズメをたべないでねむっていますね。
\\	これは「へいわなせかいがとくがわばくふによってつくられた」といういみだといわれています。
\\	これは「へいわなせかいがとくがわばくふによってつくられた」といういみだといわれています。
\\	では、いまからいちじかん、おひるきゅうけいをとりましょう。
\\	では、いまからいちじかん、おひるきゅうけいをとりましょう。
\\	しゅうごうじかんは1じはん。しゅうごうばしょはバス。いいですか?
\\	しゅうごうじかんは1じはん。しゅうごうばしょはバス。いいですか?
\\	とくに、なすくん、のざわさん、ちこくをしないでください。
\\	とくに、なすくん、のざわさん、ちこくをしないでください。
\\	はーい。
\\	はーい。
\\	まさゆめ
\\	まったく
\\	えんそく
\\	くち
\\	ける
\\	ビンテージ
\\	ジーンズ
\\	おきにいり
\\	かむ
\\	ほえる
\\	しあわせ
\\	あらう
\\	このまえ、たくさんのおんなのこたちにすかれるゆめをみたんだ。まさゆめかな。
\\	このまえ、たくさんのおんなのこたちにすかれるゆめをみたんだ。まさゆめかな。
\\	みのるはしあわせだな・・・。ぼく、こんしゅう、ほんとうについてなかったよ。
\\	みのるはしあわせだな・・・。ぼく、こんしゅう、ほんとうについてなかったよ。
\\	そう?
\\	そう?
\\	げつよう、いぬにほえられて、ころんであしをうったし・・・。
\\	げつよう、いぬにほえられて、ころんであしをうったし・・・。
\\	いぬにほえられたからころんだのか?でも、かまれなくてよかったじゃん。
\\	いぬにほえられたからころんだのか?でも、かまれなくてよかったじゃん。
\\	まあね。で、かように、おきにいりのジーンズを おやに あらわれたんだ。
\\	まあね。で、かように、おきにいりのジーンズを おやに あらわれたんだ。
\\	え?ジーンズ あらわないの?
\\	え?ジーンズ あらわないの?
\\	あらわないよ。あれ、ビンテージジーンズで たかかったんだよ。
\\	あらわないよ。あれ、ビンテージジーンズで たかかったんだよ。
\\	・・・っていうか、おやが せんたくしてくれるんだ。
\\	・・・っていうか、おやが せんたくしてくれるんだ。
\\	いいなぁ。うちのおや、せんたくしてくれないよ。
\\	いいなぁ。うちのおや、せんたくしてくれないよ。
\\	ふーん。それから、すいようびはどうじょうでせんぱいにかおをけられて、くちのなかをきっちゃったし。
\\	ふーん。それから、すいようびはどうじょうでせんぱいにかおをけられて、くちのなかをきっちゃったし。
\\	うわ・・・。
\\	うわ・・・。
\\	きのうのえんそくは、あめにふられたし、まったく、ついてないよ。
\\	きのうのえんそくは、あめにふられたし、まったく、ついてないよ。
\\	にわ
\\	こっか
\\	こんご
\\	かんそうぶん
\\	せいと
\\	きょうしつ
\\	たつ
\\	しりつ
\\	うたう
\\	こうりつ
\\	なすみのるのははです。せんせい、しつもんがあるのですが。
\\	なすみのるのははです。せんせい、しつもんがあるのですが。
\\	あ、はい。
\\	あ、はい。
\\	(モンスターペアレントか?)
\\	(モンスターペアレントか?)
\\	このがっこうではせいとにそうじさせているんですか。
\\	このがっこうではせいとにそうじさせているんですか。
\\	ええ、そうですよ。きょうしつやトイレ、にわをそうじさせます。
\\	ええ、そうですよ。きょうしつやトイレ、にわをそうじさせます。
\\	せいとたちにたくさんほんをよませるそうですね。
\\	せいとたちにたくさんほんをよませるそうですね。
\\	はい。そのあとで、かんそうぶんもかかせます。
\\	はい。そのあとで、かんそうぶんもかかせます。
\\	せいとにきみがよをうたわせるとききましたが、それはほんとうですか。
\\	せいとにきみがよをうたわせるとききましたが、それはほんとうですか。
\\	ええ。たたせてこっかをうたわせますよ。せいとにこっかをおぼえさせるためです。…
\\	ええ。たたせてこっかをうたわせますよ。せいとにこっかをおぼえさせるためです。…
\\	おかあさん、うちのがっこうはしりつですからね。もしいやだったら、こうりつにいかせてください。
\\	おかあさん、うちのがっこうはしりつですからね。もしいやだったら、こうりつにいかせてください。
\\	いえいえ。きにいりました。こんごもうちのむすこをよろしくおねがいします。
\\	いえいえ。きにいりました。こんごもうちのむすこをよろしくおねがいします。
\\	はあ…。
\\	はあ…。
\\	せんせい、これ、つまらないものですが、どうぞ。
\\	せんせい、これ、つまらないものですが、どうぞ。
\\	じゃ、しつれいします。
\\	じゃ、しつれいします。
\\	なんなんだ?
\\	なんなんだ?
\\	あにき
\\	やきゅう
\\	そる
\\	とこや
\\	おどろく
\\	しんぱいする
\\	はじめる
\\	まるがり
\\	ファン
\\	まゆげ
\\	どうしてないてるの?さくらこ。
\\	どうしてないてるの?さくらこ。
\\	だって…みのるくんが…。
\\	だって…みのるくんが…。
\\	こら、みのる。おんなのこをなかせるな。なにしたのよ。
\\	こら、みのる。おんなのこをなかせるな。なにしたのよ。
\\	なかせたつもりはないんだけど。
\\	なかせたつもりはないんだけど。
\\	ちがう、ちがう…。なみ、しんぱいさせてごめん。でも、みのるくんがわらわせるから…。
\\	ちがう、ちがう…。なみ、しんぱいさせてごめん。でも、みのるくんがわらわせるから…。
\\	わらわせたつもりもないんだけど。
\\	わらわせたつもりもないんだけど。
\\	なんだ、わらってたの?おどろかせないでよ。
\\	なんだ、わらってたの?おどろかせないでよ。
\\	み、そのあたまどうしたの?
\\	み、そのあたまどうしたの?
\\	きのう、あにきのとこやにかみをきりにいったんだ。
\\	きのう、あにきのとこやにかみをきりにいったんだ。
\\	ああ、おにいさん、とこやさんだよね。
\\	ああ、おにいさん、とこやさんだよね。
\\	さいしょはたのしくはなしていたんだけど、やきゅうのはなしをはじめて、いからせちゃったんだ。
\\	さいしょはたのしくはなしていたんだけど、やきゅうのはなしをはじめて、いからせちゃったんだ。
\\	ほら、おれはジャイアンツファンだけど、あにきはタイガースファンだから。
\\	ほら、おれはジャイアンツファンだけど、あにきはタイガースファンだから。
\\	ことし、ジャイアンツちょうしいいけど、タイガースはちょうしわるいからね。
\\	ことし、ジャイアンツちょうしいいけど、タイガースはちょうしわるいからね。
\\	ああ。で、まるがりにされて、まゆげもそられた。
\\	ああ。で、まるがりにされて、まゆげもそられた。
\\	もふく
\\	そうしき
\\	はく
\\	ミニスカート
\\	ファッションショー
\\	よむ
\\	こうし
\\	マナー
\\	せいふく
\\	いけん
\\	こんにちは。ちかもりわたるの「ちょこっとマナー」のじかんです。
\\	こんにちは。ちかもりわたるの「ちょこっとマナー」のじかんです。
\\	マナーこうしのもちだかねせんせいにいろいろおしえてもらいましょう。
\\	マナーこうしのもちだかねせんせいにいろいろおしえてもらいましょう。
\\	よろしくおねがいいたします。
\\	よろしくおねがいいたします。
\\	まず、あるこうこうせいからのしつもんを よませてください。
\\	まず、あるこうこうせいからのしつもんを よませてください。
\\	「しんせきのおそうしきにいきます。
\\	「しんせきのおそうしきにいきます。
\\	せいふくはつまらないので、くろいミニスカートをはこうとおもっています。
\\	せいふくはつまらないので、くろいミニスカートをはこうとおもっています。
\\	どうおもいますか。いけんをきかせてください。」
\\	どうおもいますか。いけんをきかせてください。」
\\	じゃ、しつもんさせてください。
\\	じゃ、しつもんさせてください。
\\	どうしてミニスカートをはきたいんですか。
\\	どうしてミニスカートをはきたいんですか。
\\	ちょっと、いわせてください。
\\	ちょっと、いわせてください。
\\	おそうしきはファッションショーじゃないんです。もふくをきなさい。
\\	おそうしきはファッションショーじゃないんです。もふくをきなさい。
\\	あ、せんせい、このひとはこうこうせいみたいです。
\\	あ、せんせい、このひとはこうこうせいみたいです。
\\	じゃ、せいふくがいちばんいいとおもいますよ。
\\	じゃ、せいふくがいちばんいいとおもいますよ。
\\	しぬ
\\	このたび
\\	こうでん
\\	せんじつ
\\	イグアナ
\\	しんせつ
\\	きょうじゅ
\\	りゅうがくせい
\\	にんげん
\\	ぜったいに
\\	ちかもりわたるのちょこっとマナーのじかんです。
\\	ちかもりわたるのちょこっとマナーのじかんです。
\\	きょうももちだかねせんせいがみなさんのしつもんにこたえてくれますよ。
\\	きょうももちだかねせんせいがみなさんのしつもんにこたえてくれますよ。
\\	ではさいしょのしつもんです。
\\	ではさいしょのしつもんです。
\\	「わたしはりゅうがくせいです。わたしのきょうじゅはとてもしんせつで、だいがくでアルバイトさせてくれたり、おいしいりょうりをたべさせてくれたりします。
\\	「わたしはりゅうがくせいです。わたしのきょうじゅはとてもしんせつで、だいがくでアルバイトさせてくれたり、おいしいりょうりをたべさせてくれたりします。
\\	せんじつ、せんせいがかっていたイグアナがしんだので、おそうしきにいくつもりです。
\\	せんじつ、せんせいがかっていたイグアナがしんだので、おそうしきにいくつもりです。
\\	こうでんはひつようですか。」
\\	こうでんはひつようですか。」
\\	うーん。にんげんのそうしきではないのでこうでんはひつようないとおもいますよ。
\\	うーん。にんげんのそうしきではないのでこうでんはひつようないとおもいますよ。
\\	きょうじゅにはなんといっったらいいですか?
\\	きょうじゅにはなんといっったらいいですか?
\\	「このたびは、ごしゅうしょうさまでした」とか「おくやみをもうしあげます」ですか。
\\	「このたびは、ごしゅうしょうさまでした」とか「おくやみをもうしあげます」ですか。
\\	そうですね。「このたびは、どうも…」だけでもいいです。
\\	そうですね。「このたびは、どうも…」だけでもいいです。
\\	「あたらしいペットをかったほうがいい」とか ぜったいに いっってはいけませんよ。
\\	「あたらしいペットをかったほうがいい」とか ぜったいに いっってはいけませんよ。
\\	けいさつ
\\	はらう
\\	つかまる
\\	はずかしい
\\	しんこんりょこう
\\	ひろうえん
\\	おいわい
\\	しゅっせき
\\	ほっかいどう
\\	ばっきん
\\	ももやくん、けっこん おめでとう。
\\	ももやくん、けっこん おめでとう。
\\	ありがとう。
\\	ありがとう。
\\	かんぱい
\\	かんぱい
\\	けっこんしきに しゅっせきできなくて、ごめん。
\\	けっこんしきに しゅっせきできなくて、ごめん。
\\	しごと やすめなかったんだ。
\\	しごと やすめなかったんだ。
\\	これ、おいわい。きにいるかどうかわからないけど。
\\	これ、おいわい。きにいるかどうかわからないけど。
\\	ありがとう。
\\	ありがとう。
\\	で、どうだった、けっこんしきは?
\\	で、どうだった、けっこんしきは?
\\	うーん。にんのまえで キスをさせられて、はずかしかった。
\\	うーん。にんのまえで キスをさせられて、はずかしかった。
\\	ははは。ひろうえんは どうだった。
\\	ははは。ひろうえんは どうだった。
\\	ああ。たくさん おさけを のませられたよ。
\\	ああ。たくさん おさけを のませられたよ。
\\	つぎのにち、いちにちじゅう、きもちわるくてたいへんだった。
\\	つぎのにち、いちにちじゅう、きもちわるくてたいへんだった。
\\	へぇー。しんこんりょこうへは もういったの?
\\	へぇー。しんこんりょこうへは もういったの?
\\	ああ、ほっかいどうに いったんだ。たのしかったよ。
\\	ああ、ほっかいどうに いったんだ。たのしかったよ。
\\	でも…。けいさつにつかまっちゃったんだ。
\\	でも…。けいさつにつかまっちゃったんだ。
\\	なんで。スピード?
\\	なんで。スピード?
\\	ああ、ばっきんはらわせられた。
\\	ああ、ばっきんはらわせられた。
\\	いくら?
\\	いくら?
\\	8まん。
\\	8まん。
\\	うわー。8まんえんもはらわせられたの?
\\	うわー。8まんえんもはらわせられたの?
\\	それはごしゅうしょうさま。
\\	それはごしゅうしょうさま。
\\	ぬすむ
\\	はなしかける
\\	えらぶ
\\	いちおく
\\	ダイヤ
\\	ゆびわ
\\	なみだ
\\	だます
\\	とつぜん
\\	ひっこす
\\	けむり
\\	にげる
\\	じょせい
\\	ニュースです。ごご2じはんころ、
\\	ジュエリーから1おくえんのダイヤがぬすまれました。
\\	ニュースです。ごご2じはんころ、
\\	ジュエリーから1おくえんのダイヤがぬすまれました。
\\	2じころ、65さいくらいのだんせいとわかいじょせいのおきゃくさまがはいってきました。
\\	2じころ、65さいくらいのだんせいとわかいじょせいのおきゃくさまがはいってきました。
\\	ゆびわをみにきたといっていました。
\\	ゆびわをみにきたといっていました。
\\	とつぜん、だんせいのカバンからしろいけむりがでてきたんです。
\\	とつぜん、だんせいのカバンからしろいけむりがでてきたんです。
\\	なみだがたくさんでてきて、なにもみえなくなりました。
\\	なみだがたくさんでてきて、なにもみえなくなりました。
\\	え?そのだんせいですか?はい。もちろん、ダイヤをもってにげていきました。
\\	え?そのだんせいですか?はい。もちろん、ダイヤをもってにげていきました。
\\	おとこのひとがはなしかけてきたんです。
\\	おとこのひとがはなしかけてきたんです。
\\	ジュエリーはどこかときかれました。
\\	ジュエリーはどこかときかれました。
\\	ひっこしてきたばかりで、みちがわからないっていっていました。
\\	ひっこしてきたばかりで、みちがわからないっていっていました。
\\	むすめのゆびわをみにいくんだけど、えらんでくれませんかっていわれたんです。
\\	むすめのゆびわをみにいくんだけど、えらんでくれませんかっていわれたんです。
\\	とてもやさしそうなひとだったので、つれてきてあげたんです…。
\\	とてもやさしそうなひとだったので、つれてきてあげたんです…。
\\	わたし、だまされたんですか。
\\	わたし、だまされたんですか。
\\	さいこう
\\	かえる
\\	ぼうし
\\	れんきゅう
\\	ちばけん
\\	きゅうに
\\	さいきん
\\	くらい
\\	ふる
\\	オタマジャクシ
\\	こんにちは。てんきよほうのおじかんです。
\\	こんにちは。てんきよほうのおじかんです。
\\	さいきん、あたたかくなってきましたね。
\\	さいきん、あたたかくなってきましたね。
\\	きょうかられんきゅうがはじまりますが、きょうのごごからあつくなってくるでしょう。
\\	きょうかられんきゅうがはじまりますが、きょうのごごからあつくなってくるでしょう。
\\	そとにいくとき、ぼうしをわすれないでくださいね。
\\	そとにいくとき、ぼうしをわすれないでくださいね。
\\	うわー。ネズミーランドさいこう。たのしいね。
\\	うわー。ネズミーランドさいこう。たのしいね。
\\	おなかすいてきた。
\\	おなかすいてきた。
\\	じゃ、なにかたべよう。
\\	じゃ、なにかたべよう。
\\	うー。きゅうにすずしくなってきた。
\\	うー。きゅうにすずしくなってきた。
\\	そらもくらくなってきた。あめがふりそう・・・。
\\	そらもくらくなってきた。あめがふりそう・・・。
\\	あ、ふってきた。
\\	あ、ふってきた。
\\	いたい
\\	いたい
\\	いてっ
\\	あれ?これ、おたまじゃくし?
\\	いてっ
\\	あれ?これ、おたまじゃくし?
\\	ええ?おたまじゃくしがふってきたの?
\\	ええ?おたまじゃくしがふってきたの?
\\	いま、はいったニュースです。ごぜん11じはんごろ、おたまじゃくしがふりました。ちばけんのとうきょうネズミーランドできょうごぜん11じはんごろおたまじゃくし30びき、かえる2ひきがふったそうです。
\\	いま、はいったニュースです。ごぜん11じはんごろ、おたまじゃくしがふりました。ちばけんのとうきょうネズミーランドできょうごぜん11じはんごろおたまじゃくし30びき、かえる2ひきがふったそうです。
\\	つぎのニュースです。
\\	つぎのニュースです。
\\	4がつ1にちに
\\	バーガーがはつばいした「わびさびバーガー」がにんきのようです。
\\	4がつ1にちに
\\	バーガーがはつばいした「わびさびバーガー」がにんきのようです。
\\	ハンバーガーから、にく、やさい、ピクルスなどのぐをすべてぬいたパンだけのたいへんシンプルなハンバーガーです。
\\	ハンバーガーから、にく、やさい、ピクルスなどのぐをすべてぬいたパンだけのたいへんシンプルなハンバーガーです。
\\	わびさびバーガーをはつめいしたチン・ゲンサイしゃちょうは
\\	わびさびバーガーをはつめいしたチン・ゲンサイしゃちょうは
\\	「いままで30ねんかんハンバーガーをつくってきた。
\\	「いままで30ねんかんハンバーガーをつくってきた。
\\	にくがきらいなので、ハンバーガーをおいしいとおもったことはない。
\\	にくがきらいなので、ハンバーガーをおいしいとおもったことはない。
\\	もともと、エイプリルフールのじょうだんとしてわびさびバーガーをはつばいした。
\\	もともと、エイプリルフールのじょうだんとしてわびさびバーガーをはつばいした。
\\	どうしてにんきがでてきたかわからないが、これからも、おもしろいハンバーガーをつくっていきたい。」
\\	どうしてにんきがでてきたかわからないが、これからも、おもしろいハンバーガーをつくっていきたい。」
\\	というコメントをはっぴょうしました。
\\	というコメントをはっぴょうしました。
\\	テリヤキ
\\	たつ
\\	にる
\\	ふっとう
\\	なべ
\\	とける
\\	たれ
\\	す
\\	しょうゆ
\\	ひっくりかえす
\\	にわとりのテリヤキをつくりましょう。
\\	にわとりのテリヤキをつくりましょう。
\\	ざいりょうは、とりにくいちまい、しょうゆごじゅう
\\	、すごじゅう
\\	、さとうさんじゅうグラム。
\\	ざいりょうは、とりにくいちまい、しょうゆごじゅう
\\	、すごじゅう
\\	、さとうさんじゅうグラム。
\\	みなさん、ざいりょうはありますか。
\\	みなさん、ざいりょうはありますか。
\\	はーい。
\\	はーい。
\\	では、つくりはじめましょう。
\\	では、つくりはじめましょう。
\\	しょうゆと、す、と さとうをまぜてください。
\\	しょうゆと、す、と さとうをまぜてください。
\\	ほら、のざわさん、なすくん、つくりはじめてください。
\\	ほら、のざわさん、なすくん、つくりはじめてください。
\\	さとうがとけるまで、まぜつづけてくださいね。
\\	さとうがとけるまで、まぜつづけてくださいね。
\\	まぜおわりましたか。
\\	まぜおわりましたか。
\\	つぎに、なべにたれととりにくをいれます。
\\	つぎに、なべにたれととりにくをいれます。
\\	それから、ひをつけてください。
\\	それから、ひをつけてください。
\\	せんせい、ふっとうしました。
\\	せんせい、ふっとうしました。
\\	じゃ、ひをよわくしてください。じゅっぷんかんにてください。
\\	じゃ、ひをよわくしてください。じゅっぷんかんにてください。
\\	せんせいじゅっぷんたちました。
\\	せんせいじゅっぷんたちました。
\\	じゃ、ひっくりかえして、またじゅっぷんにてください。
\\	じゃ、ひっくりかえして、またじゅっぷんにてください。
\\	つくりおわりましたか?
\\	つくりおわりましたか?
\\	せんせい、たべはじめてもいいですか。
\\	せんせい、たべはじめてもいいですか。
\\	はい、いいですよ。
\\	はい、いいですよ。
\\	うーん。おいしかった。
\\	うーん。おいしかった。
\\	みのる、もうたべおわったの?はやいなー。
\\	みのる、もうたべおわったの?はやいなー。
\\	こたえる
\\	しかし
\\	かんがえる
\\	すくない
\\	つまり
\\	やく
\\	きじ
\\	ぶんしょう
\\	いしきする
\\	げんじつ
\\	ここに「じょうようかんじ」についてのきじがあります。
\\	ここに「じょうようかんじ」についてのきじがあります。
\\	のざわさん、よんでくれますか。
\\	のざわさん、よんでくれますか。
\\	はい。「ぶんしょうをかくときに、じょうようかんじをいしきするかどうか、しつもんをした。
\\	はい。「ぶんしょうをかくときに、じょうようかんじをいしきするかどうか、しつもんをした。
\\	いしきするとこたえたひとはさんじゅうはちぱーせんと。いしきしないとこたえたひとはやくろくじゅっぱーせんとであった。」
\\	いしきするとこたえたひとはさんじゅうはちぱーせんと。いしきしないとこたえたひとはやくろくじゅっぱーせんとであった。」
\\	ありがとう。つまり、「このかんじはじょうようかんじかな」とかんがえながら、かんじをつかっているひとはすくないんだね。
\\	ありがとう。つまり、「このかんじはじょうようかんじかな」とかんがえながら、かんじをつかっているひとはすくないんだね。
\\	じゃ、つぎをおおばさん、おねがいします。
\\	じゃ、つぎをおおばさん、おねがいします。
\\	はい。
\\	はい。
\\	「ゆううつ(ゆう・うつ)のうつ、ごい(ご・い)のい、しょほうせん(しょ・ほう・せん)のせんは、2010ねんにじょうようかんじになった。
\\	「ゆううつ(ゆう・うつ)のうつ、ごい(ご・い)のい、しょほうせん(しょ・ほう・せん)のせんは、2010ねんにじょうようかんじになった。
\\	しかし、3つのかんじがすべてかけるとこたえたひとはよんぱーせんと、かけないとこたえたひとはごじゅうはちぱーせんとであった。
\\	鬱 
\\	憂鬱 
\\	彙 
\\	語彙 
\\	箋 
\\	処方箋 
\\	しかし、3つのかんじがすべてかけるとこたえたひとはよんぱーせんと、かけないとこたえたひとはごじゅうはちぱーせんとであった。
\\	鬱 
\\	憂鬱 
\\	彙 
\\	語彙 
\\	箋 
\\	処方箋 
\\	これがげんじつなのであろう。」
\\	これがげんじつなのであろう。」
\\	よかった。
\\	よかった。
\\	かけないひとはおれだけじゃないんだ
\\	かけないひとはおれだけじゃないんだ
\\	ばんごうふだ
\\	かしきり
\\	かた
\\	タトゥー
\\	うで
\\	あきたけん
\\	りょかん
\\	おんせん
\\	とる
\\	ろてんぶろ
\\	いらっしゃいませ。
\\	いらっしゃいませ。
\\	こちらのばんごうふだをおとりになって、おまちください。
\\	こちらのばんごうふだをおとりになって、おまちください。
\\	はい。
\\	はい。
\\	さんばんのばんごうふだをおもちのおきゃくさま、カウンターへどうぞ。
\\	さんばんのばんごうふだをおもちのおきゃくさま、カウンターへどうぞ。
\\	あ、はい。
\\	あ、はい。
\\	こんしゅうまつ、つまといっしょにおんせんにいこうとおもっているんですが、まだよやくできますか。
\\	こんしゅうまつ、つまといっしょにおんせんにいこうとおもっているんですが、まだよやくできますか。
\\	どちらのおんせんをおかんがえですか。
\\	どちらのおんせんをおかんがえですか。
\\	なまえはちょっとわからないんですが、このしゃしんのおんせんりょかんです。
\\	なまえはちょっとわからないんですが、このしゃしんのおんせんりょかんです。
\\	あ、これはあきたけんのりょかんですね。いまはなつなので、ゆきはございませんが…。
\\	あ、これはあきたけんのりょかんですね。いまはなつなので、ゆきはございませんが…。
\\	だいじょうぶです。
\\	だいじょうぶです。
\\	にめいさま、いっぱくふつかですね。おたばこはおすいになりますか。
\\	にめいさま、いっぱくふつかですね。おたばこはおすいになりますか。
\\	すいません。
\\	すいません。
\\	きゃくしつがかり
\\	ほんじつ
\\	ごろ
\\	しく
\\	ふとん
\\	ひじょうぐち
\\	かい
\\	あんない
\\	りよう
\\	へや
\\	かんない
\\	ももやさま、きゃくしつかかりのなしほんともうします。
\\	ももやさま、きゃくしつかかりのなしほんともうします。
\\	おへやまでごあんないいたします。
\\	おへやまでごあんないいたします。
\\	あ、はい。おねがいします。
\\	あ、はい。おねがいします。
\\	おにもつおもちいたします。
\\	おにもつおもちいたします。
\\	あ、すみません。
\\	あ、すみません。
\\	では、かんないのせつめいをいたします。
\\	では、かんないのせつめいをいたします。
\\	こちらはひじょうぐちでございます。
\\	こちらはひじょうぐちでございます。
\\	おふろはすべて1かいにございます。
\\	おふろはすべて1かいにございます。
\\	ごりようじかんはあさ4じから、よる12じまででございます。
\\	ごりようじかんはあさ4じから、よる12じまででございます。
\\	ももやようのおへやはこちらでございます。
\\	ももやようのおへやはこちらでございます。
\\	うわー、いいへやですね。
\\	うわー、いいへやですね。
\\	ゆかたとタオルはこちらにいっっております。
\\	ゆかたとタオルはこちらにいっっております。
\\	ではおちゃをおいれいたしますね。
\\	ではおちゃをおいれいたしますね。
\\	ほんじつ、ごゆうしょくは7じでいかがでしょうか。
\\	ほんじつ、ごゆうしょくは7じでいかがでしょうか。
\\	はい。
\\	はい。
\\	では、7じごろにゆうしょくをおへやにおはこびします。
\\	では、7じごろにゆうしょくをおへやにおはこびします。
\\	いまからおふろにいくので、そのまにふとんをしいてもらえますか。
\\	いまからおふろにいくので、そのまにふとんをしいてもらえますか。
\\	かしこまりました。では、しつれいいたします。
\\	かしこまりました。では、しつれいいたします。
\\	かいがい
\\	てがみ
\\	ふつう
\\	なるほど
\\	めうえ
\\	そだつ
\\	にがて
\\	けいご
\\	このまえ
\\	ていねい
\\	ちかもりわたるのちょこっとマナーのおじかんです。
\\	ちかもりわたるのちょこっとマナーのおじかんです。
\\	もちだかねせんせい、きょうもよろしくおねがいいたします。
\\	もちだかねせんせい、きょうもよろしくおねがいいたします。
\\	はい。
\\	はい。
\\	さいしょのしつもんです。
\\	さいしょのしつもんです。
\\	「こんにちは。おれはかいがいでそだったので、けいごがにがてだ。
\\	「こんにちは。おれはかいがいでそだったので、けいごがにがてだ。
\\	このまえせんぱいに『てつだってほしいですか』といっったら、むっとされた。
\\	このまえせんぱいに『てつだってほしいですか』といっったら、むっとされた。
\\	なにがわるかったんだ?」
\\	なにがわるかったんだ?」
\\	あら、かわいそう。
\\	あら、かわいそう。
\\	でもめうえのひとに「てほしいですか」はつかわないほうがいいですね。
\\	でもめうえのひとに「てほしいですか」はつかわないほうがいいですね。
\\	「てつだいましょうか」とか「てつだわせてください」。
\\	「てつだいましょうか」とか「てつだわせてください」。
\\	もっとかんたんに「てつだいます」ということもできますよ。
\\	もっとかんたんに「てつだいます」ということもできますよ。
\\	なるほど。
\\	なるほど。
\\	あの・・・せんせい、ふつうてがみで「おれ」はつかわないですよね。
\\	あの・・・せんせい、ふつうてがみで「おれ」はつかわないですよね。
\\	そうですね。
\\	そうですね。
\\	しらないひとにてがみをかくときは、「です」「ます」をつかってていねいにかきましょうね。
\\	しらないひとにてがみをかくときは、「です」「ます」をつかってていねいにかきましょうね。
\\	ひま
\\	ひさしぶり
\\	さいきん
\\	ちゃんと
\\	しめきり
\\	ろんぶん
\\	だす
\\	ただ
\\	ひさびさ
\\	らいしゅう
\\	こんしゅう
\\	じゅぎょう
\\	すずきさん、ひさしぶりですね。このじゅぎょうはきらいですか?
\\	すずきさん、ひさしぶりですね。このじゅぎょうはきらいですか?
\\	いいえ、やまだせんせいのじゅぎょうはだいすきです。ただ、さいきんいそがしいんです。
\\	いいえ、やまだせんせいのじゅぎょうはだいすきです。ただ、さいきんいそがしいんです。
\\	ろんぶんは?
\\	ろんぶんは?
\\	そうですね。
\\	そうですね。
\\	しめきりはらいしゅうですよ
\\	しめきりはらいしゅうですよ
\\	はい。ちゃんとだします。
\\	はい。ちゃんとだします。
\\	がんばってね。
\\	がんばってね。
\\	がんばります
\\	がんばります
\\	まね
\\	スタバ
\\	らくしょう
\\	まさか
\\	ずるすぎる
\\	ネット
\\	ばかり
\\	おとな
\\	マジで
\\	うまい
\\	きんえん
\\	そうかも
\\	ぎりぎりセーフ
\\	スノボ
\\	きぶん
\\	かしら
\\	モカラテ
\\	うん
\\	ところで
\\	ぼく
\\	さゆりきた
\\	さゆりちゃん、こっち。
\\	さゆりきた
\\	さゆりちゃん、こっち。
\\	ふたりはいつもスタバね。なんで?
\\	ふたりはいつもスタバね。なんで?
\\	きんえんだし、モカラテがうまい
\\	きんえんだし、モカラテがうまい
\\	なるほど、ところで、ひさしぶり、ひでおとメル
\\	なるほど、ところで、ひさしぶり、ひでおとメル
\\	ひさびさ
\\	ひさびさ
\\	ひさびさ
\\	ひさびさ
\\	まさか、ひでおのまねしてるの?
\\	まさか、ひでおのまねしてるの?
\\	なかなかいいにほんごのせんせいだし
\\	なかなかいいにほんごのせんせいだし
\\	ありがとうございますー。なかなかいいがくせいだな。
\\	ありがとうございますー。なかなかいいがくせいだな。
\\	ああ、せんせいのおかげで、ほんとにかんしゃしています・・・。
\\	ああ、せんせいのおかげで、ほんとにかんしゃしています・・・。
\\	あいかわらずね。ふたりはいつおとなになるのかしら?
\\	あいかわらずね。ふたりはいつおとなになるのかしら?
\\	ところで、さいきんなにしている?
\\	ところで、さいきんなにしている?
\\	ずっとろんぶん。いまだしたばかり。たいへんだったよ
\\	ずっとろんぶん。いまだしたばかり。たいへんだったよ
\\	ちゃんとだした?
\\	ちゃんとだした?
\\	うん、ぎりぎりセーフだった
\\	いまはものすごくいいきぶん
\\	うん、ぎりぎりセーフだった
\\	いまはものすごくいいきぶん
\\	よかった
\\	よかった
\\	よかったなー。
\\	よかったなー。
\\	ふたりは?ろんぶんどうした?
\\	ふたりは?ろんぶんどうした?
\\	ぼくはちゃんとネットでかったよ
\\	ぼくはちゃんとネットでかったよ
\\	あいかわらず、おもしろくない。メルとかわらない
\\	あいかわらず、おもしろくない。メルとかわらない
\\	そうかも。
\\	そうかも。
\\	もういっかいきくけど、ふたりはろんぶんをちゃんとだした?
\\	もういっかいきくけど、ふたりはろんぶんをちゃんとだした?
\\	らくしょうだった
\\	しゅうかんまえにだしたよ。
\\	らくしょうだった
\\	しゅうかんまえにだしたよ。
\\	マジで?
\\	マジで?
\\	うん。
\\	うん。
\\	にほんごはたいへんだったでしょう?
\\	にほんごはたいへんだったでしょう?
\\	にほんご??えいごでかいたよ。
\\	にほんご??えいごでかいたよ。
\\	ずるすぎる
\\	フェアじゃない。
\\	ずるすぎる
\\	フェアじゃない。
\\	まあね。
\\	まあね。
\\	ぼくもきょうだした。すごい、いいきぶんだ。こんしゅうスノボは?
\\	ぼくもきょうだした。すごい、いいきぶんだ。こんしゅうスノボは?
\\	いく、いく
\\	いく、いく
\\	わたしもいく。
\\	わたしもいく。
\\	じゃ、きまりだ
\\	じゃ、きまりだ
\\	ようこは?
\\	ようこは?
\\	わからない。
\\	わからない。
\\	ぼくもわからない。
\\	ぼくもわからない。
\\	かのじょはどこにいるのかしら?しめきりは3じ、いまは2じはん。だいじょうぶかな?
\\	かのじょはどこにいるのかしら?しめきりは3じ、いまは2じはん。だいじょうぶかな?
\\	ようこにでんわして。
\\	ようこにでんわして。
\\	[でんわをかける・・・]
\\	[でんわをかける・・・]
\\	まにあう
\\	なおす
\\	おまたせしました.
\\	けん
\\	きぼう
\\	すいせんじょう
\\	すいせん
\\	たいしたもん
\\	テーマ
\\	やっぱり
\\	しつれい
\\	がくちょう
\\	もしもしようこちゃん?
\\	もしもしようこちゃん?
\\	うん、えりちゃんひさしぶり。
\\	うん、えりちゃんひさしぶり。
\\	だいじょうぶ?さいきんなにしているの?みんなしんぱいしているよ。
\\	だいじょうぶ?さいきんなにしているの?みんなしんぱいしているよ。
\\	ずっとろんぶんだった。たいへんだったよ。ギリギリまにあった。
\\	ずっとろんぶんだった。たいへんだったよ。ギリギリまにあった。
\\	いまどこ?
\\	いまどこ?
\\	いま、だいがくでがくちょうまっているところ。
\\	いま、だいがくでがくちょうまっているところ。
\\	えっ、どうしたの?
\\	えっ、どうしたの?
\\	わからないけど、ろんぶんのけんでがくちょうがわたしをよんだ。
\\	わからないけど、ろんぶんのけんでがくちょうがわたしをよんだ。
\\	えー!?
\\	えー!?
\\	おまたせしました。やまぐちさん、どうぞ。
\\	おまたせしました。やまぐちさん、どうぞ。
\\	もう、いくね。あとでかけなおす。じゃあね。
\\	もう、いくね。あとでかけなおす。じゃあね。
\\	しつれいします。
\\	しつれいします。
\\	どうぞ。すわって。きょうよんだりゆうはきみのろんぶんのこと
\\	どうぞ。すわって。きょうよんだりゆうはきみのろんぶんのこと
\\	です。
\\	です。
\\	すみません。やっぱりそのホリエモンのテーマがわるかったですか?
\\	すみません。やっぱりそのホリエモンのテーマがわるかったですか?
\\	いやいやいや、すばらしいろんぶんだ。たいしたものだよ。
\\	いやいやいや、すばらしいろんぶんだ。たいしたものだよ。
\\	ほんとうですか?
\\	ほんとうですか?
\\	ほんとうだよ。それで
\\	にきみをすいせんしたいんだが、ど
\\	ほんとうだよ。それで
\\	にきみをすいせんしたいんだが、ど
\\	うかね?
\\	うかね?
\\	ですか
\\	ですか
\\	おまたせ
\\	しゅうしょくかつどう
\\	ふたりきり
\\	そつぎょうりょこう
\\	ピンチ・ヒッター
\\	たのしみ
\\	トリノオリンピック
\\	さんせい
\\	なにいってるの?
\\	ようこ、おまたせ
\\	ようこ、おまたせ
\\	おはよう、メル
\\	おはよう、メル
\\	さゆりとひでおは?
\\	さゆりとひでおは?
\\	さゆりはろんぶんをかきなおしてるところ。ひでおはまだしゅうかつちゅう。
\\	にきまったけど、いまのじょうたいではにゅうしゃできないかも。ふたりはそつぎょうりょこうにいけるかどうかまだわからない。
\\	さゆりはろんぶんをかきなおしてるところ。ひでおはまだしゅうかつちゅう。
\\	にきまったけど、いまのじょうたいではにゅうしゃできないかも。ふたりはそつぎょうりょこうにいけるかどうかまだわからない。
\\	ふたりきりか
\\	すてきだね
\\	ふたりきりか
\\	すてきだね
\\	なにいってるの?ピンチ・ヒッターがいるよ
\\	わたしがよんだの。あっ
\\	いらっしゃった
\\	なにいってるの?ピンチ・ヒッターがいるよ
\\	わたしがよんだの。あっ
\\	いらっしゃった
\\	がくちょう
\\	がくちょう
\\	どうも、どうも。ひしぶりだね、メルはげんき?
\\	どうも、どうも。ひしぶりだね、メルはげんき?
\\	おかげさまでげんきです。
\\	おかげさまでげんきです。
\\	このりょこうたのしみですね。どこがいい?
\\	このりょこうたのしみですね。どこがいい?
\\	トリノがいい
\\	まだオリンピックをやってます
\\	トリノがいい
\\	まだオリンピックをやってます
\\	でも、きっぷをかえるかどうかわからないし。がくちょう、ごきぼうはありますか?
\\	・ご 
\\	でも、きっぷをかえるかどうかわからないし。がくちょう、ごきぼうはありますか?
\\	・ご 
\\	そうですね。わたしのそつぎょうりょこうはおきなわでした。もういちど、いきたいなあ。
\\	そうですね。わたしのそつぎょうりょこうはおきなわでした。もういちど、いきたいなあ。
\\	おきなわもいいですね。わたしはさんせいです。
\\	おきなわもいいですね。わたしはさんせいです。
\\	わたしも
\\	です
\\	わたしも
\\	です
\\	じゃ、きまり
\\	じゃ、きまり
\\	いきましょう
\\	いきましょう
\\	とうちゃく
\\	ゴーヤチャンプルー
\\	あわもり
\\	がりがり
\\	きがえる
\\	シュノーケル
\\	なつかしい
\\	おきなわ
\\	やっと
\\	タコライス
\\	とうちゃく
\\	とうちゃく
\\	やっとうみについた
\\	きれいね
\\	おきなわはさいこう
\\	やっとうみについた
\\	きれいね
\\	おきなわはさいこう
\\	そうだね。なつかしいな。さんじゅうねんぶりだ。
\\	そうだね。なつかしいな。さんじゅうねんぶりだ。
\\	きもちいい
\\	うみがだいすき
\\	きもちいい
\\	うみがだいすき
\\	シュノーケルをやってみたいな。
\\	シュノーケルをやってみたいな。
\\	いいかんがえですね
\\	わたしもやってみたいです。
\\	いいかんがえですね
\\	わたしもやってみたいです。
\\	いまはよじはんで、まだやっているかどうか・・・
\\	いまはよじはんで、まだやっているかどうか・・・
\\	きいてみよう
\\	きいてみよう
\\	まだだいじょうぶ。はやくきがえよう。
\\	まだだいじょうぶ。はやくきがえよう。
\\	えーしんじられない
\\	がくちょうすごい
\\	かっこういいー
\\	えーしんじられない
\\	がくちょうすごい
\\	かっこういいー
\\	えぇぇぇ、すごいからだしてます
\\	えぇぇぇ、すごいからだしてます
\\	まあね。
\\	まあね。
\\	すごい
\\	メルにぜんぜんかっている
\\	すごいがくちょう
\\	すごい
\\	メルにぜんぜんかっている
\\	すごいがくちょう
\\	メル、がりがりだね。だいじょうぶ?こんや、あわもりをのみながら、ゴーヤチャン
\\	メル、がりがりだね。だいじょうぶ?こんや、あわもりをのみながら、ゴーヤチャン
\\	べんごし
\\	ゆうしゅう
\\	スコール
\\	かみなり
\\	いなずま
\\	おみやげ
\\	ハブアイス
\\	ストロー
\\	ひさびさ。げんき?
\\	ひさびさ。げんき?
\\	げんきじゃない。おきなわはたいへんだった
\\	げんきじゃない。おきなわはたいへんだった
\\	どうしたの?
\\	どうしたの?
\\	シュノーケルするところでスコールがきて、かみなりはごろごろ、あめはざあざあ。いのちがあぶなかった。ぎりぎりでにげた。
\\	シュノーケルするところでスコールがきて、かみなりはごろごろ、あめはざあざあ。いのちがあぶなかった。ぎりぎりでにげた。
\\	たいへんだったね。
\\	たいへんだったね。
\\	たいへんだったけど、メルとがくちょうのほうがばいぐらいたいへんだった。
\\	たいへんだったけど、メルとがくちょうのほうがばいぐらいたいへんだった。
\\	たいへん?なにやったの?
\\	たいへん?なにやったの?
\\	それはちょっといいにくい。ほんにんからはなしをきいて。
\\	それはちょっといいにくい。ほんにんからはなしをきいて。
\\	じゃ、ふたりはいまどこ?
\\	じゃ、ふたりはいまどこ?
\\	それもいいにくいな。
\\	それもいいにくいな。
\\	おしえて
\\	おしえて
\\	むり。おしえない。でも、ゆうしゅうなべんごしがひつよう。ところで、おみやげ
\\	むり。おしえない。でも、ゆうしゅうなべんごしがひつよう。ところで、おみやげ
\\	ありがとう。なんだろう?えぇぇぇ、これは…ハブアイス
\\	ありがとう。なんだろう?えぇぇぇ、これは…ハブアイス
\\	そうそう
\\	すごいでしょう
\\	はやくたべてみて。
\\	そうそう
\\	すごいでしょう
\\	はやくたべてみて。
\\	とけてる
\\	どろどろ
\\	とけてる
\\	どろどろ
\\	ひこうきのなかあつかったから。まあ、でもあじはかわってないよ。ストローでのんでみて。
\\	ひこうきのなかあつかったから。まあ、でもあじはかわってないよ。ストローでのんでみて。
\\	うん。わかった。おいしい
\\	ありがとう。
\\	うん。わかった。おいしい
\\	ありがとう。
\\	でしょう?えっ、ひでおは?かれのぶんもあるの。
\\	でしょう?えっ、ひでおは?かれのぶんもあるの。
\\	かれのしゅうかつはヤバい。しごとがみつかるかどうか…
\\	かれのしゅうかつはヤバい。しごとがみつかるかどうか…
\\	えぇぇぇ…でもじぶんのほうがもっとしんぱい。しごとまだきまってないんだ。ねぇ、たのみがあるんだけど。らいしゅう、だいじなめんせつがあるからてつだってほしい。
\\	えぇぇぇ…でもじぶんのほうがもっとしんぱい。しごとまだきまってないんだ。ねぇ、たのみがあるんだけど。らいしゅう、だいじなめんせつがあるからてつだってほしい。
\\	いいよ。
\\	いいよ。
\\	ありがとう。たすかった
\\	ありがとう。たすかった
\\	おもち
\\	そっと
\\	とうとう
\\	せめて
\\	せっせと
\\	ニコニコ
\\	おおみそか
\\	じそう
\\	むかしばなし
\\	むらはずれ
\\	はげしい
\\	つもる
\\	うしろすがた
\\	とんでもない
\\	ゆうしゅうな
\\	こまる
\\	げんざい
\\	ちまつり
\\	ふほうしんにゅう
\\	えさ
\\	サメ
\\	すいぞくかん
\\	べろべろ
\\	ちからになる
\\	じかんをさく
\\	みつもり
\\	はい。
\\	はい。
\\	もしもし。ようこ
\\	どうなってる?
\\	もしもし。ようこ
\\	どうなってる?
\\	いま、せんせいとあうところ。もうちょっとまってて。
\\	いま、せんせいとあうところ。もうちょっとまってて。
\\	これがぼくらのさいごのでんわ、たのむよ。
\\	これがぼくらのさいごのでんわ、たのむよ。
\\	わかったけど、ふたりのせいできょうのめんせついけな
\\	わかったけど、ふたりのせいできょうのめんせついけな
\\	いよ。まったく、とんでもないふたりだよ
\\	いよ。まったく、とんでもないふたりだよ
\\	ごめん。よろしく
\\	ごめん。よろしく
\\	どうぞおはいりください。
\\	どうぞおはいりください。
\\	しつれいします。
\\	しつれいします。
\\	どうぞ。すわってください。
\\	どうぞ。すわってください。
\\	じかんをさいていただいてありがとうございます。
\\	じかんをさいていただいてありがとうございます。
\\	いえいえ、がくちょうがむかしのゆうじんなのでちから
\\	いえいえ、がくちょうがむかしのゆうじんなのでちから
\\	になりますよ。
\\	になりますよ。
\\	ありがとうございます。となべせんせいがついています
\\	ありがとうございます。となべせんせいがついています
\\	のできっともんだいないとおもいます。
\\	のできっともんだいないとおもいます。
\\	ところで、テレビでみましたけど、もういちどあのよる
\\	ところで、テレビでみましたけど、もういちどあのよる
\\	かいとり
\\	かち
\\	うらやましい
\\	こどものころ
\\	(~と)よぶ
\\	したのなまえ
\\	フェラーリ
\\	ランボルギーニ
\\	こんなところで
\\	ぼろい
\\	あいしゃ
\\	しょぶん
\\	いらっしゃいませ。「ピーターくるまかいとり」へ
\\	いらっしゃいませ。「ピーターくるまかいとり」へ
\\	ようこそ。
\\	ようこそ。
\\	すみません。くるまをうりたいのですが。
\\	すみません。くるまをうりたいのですが。
\\	はい。かしこまりました。ごあいしゃはどちらでし
\\	はい。かしこまりました。ごあいしゃはどちらでし
\\	ょうか。
\\	ょうか。
\\	あのぼろいあかのミニです。
\\	あのぼろいあかのミニです。
\\	ずいぶんふるいものですね。しょうしょうおまちく
\\	ずいぶんふるいものですね。しょうしょうおまちく
\\	ださい。こちらですわってまっていてください。
\\	ださい。こちらですわってまっていてください。
\\	はい。はぁ~。べんごしりょうはらえないから、たか
\\	はい。はぁ~。べんごしりょうはらえないから、たか
\\	くうりたいなあ。
\\	くうりたいなあ。
\\	ようこちゃん
\\	どうしたの、こんなところで。
\\	ようこちゃん
\\	どうしたの、こんなところで。
\\	となべせんせい
\\	こんなところで。
\\	となべせんせい
\\	こんなところで。
\\	したのなまえはりょう。りょうってよんでよ。
\\	したのなまえはりょう。りょうってよんでよ。
\\	はい。
\\	はい。
\\	きょう、くるまをうりにきたんだよ。
\\	きょう、くるまをうりにきたんだよ。
\\	おかねがひつようなんですか。
\\	おかねがひつようなんですか。
\\	こんかいのべんごりょうであたらしいくるまをかうつ
\\	こんかいのべんごりょうであたらしいくるまをかうつ
\\	もりなんだよ。こどものころ、ずっとイタリアのランボルギー
\\	もりなんだよ。こどものころ、ずっとイタリアのランボルギー
\\	ニがほしくて、あるいは、フェラーリもわるくないなあ。まだ
\\	ニがほしくて、あるいは、フェラーリもわるくないなあ。まだ
\\	きめてないけどね。べんごしでよかったよ。
\\	きめてないけどね。べんごしでよかったよ。
\\	フレンチレストラン
\\	やくそく
\\	そうだん
\\	えんりょ
\\	かいけつ
\\	しんゆう
\\	どうしても
\\	こういうとき
\\	やばい
\\	むかえにいく
\\	れんらく
\\	もしもし。りょうですけど。
\\	もしもし。りょうですけど。
\\	はい。
\\	はい。
\\	どようびの6じにとうきょうでいちばんおいしい
\\	どようびの6じにとうきょうでいちばんおいしい
\\	フレンチレストランをよやくしているので、5じはんにぼ
\\	フレンチレストランをよやくしているので、5じはんにぼ
\\	くのあたらしいフェラーリでむかえにいきます。
\\	くのあたらしいフェラーリでむかえにいきます。
\\	はい、わかりました。
\\	はい、わかりました。
\\	では、どようびに。むふっふっふ・・・
\\	では、どようびに。むふっふっふ・・・
\\	どうしょう
\\	かなりやばい。どうしよう
\\	ほん
\\	どうしょう
\\	かなりやばい。どうしよう
\\	ほん
\\	とうにこまった。あっ、わかった
\\	こういうときにはさと
\\	とうにこまった。あっ、わかった
\\	こういうときにはさと
\\	こだ。こうこうのしんゆうで、こまったときえんりょなく
\\	こだ。こうこうのしんゆうで、こまったときえんりょなく
\\	そうだんしていいってやくそくした。きっとさとこがたす
\\	そうだんしていいってやくそくした。きっとさとこがたす
\\	けてくれる。しかし4ねんぶりで、だいじょうぶかな。
\\	けてくれる。しかし4ねんぶりで、だいじょうぶかな。
\\	さとこですけど。
\\	さとこですけど。
\\	もしもし、ひさしぶり・・・
\\	もしもし、ひさしぶり・・・
\\	ようこちゃん?
\\	ようこちゃん?
\\	すでに
\\	ロマネコンティ
\\	ソムリエさん
\\	プラス
\\	そういうわけで
\\	むかし
\\	かのうせいがある
\\	うしなう
\\	めんきょ
\\	けん
\\	さいねんしょうごうかくしゃ
\\	しほうしけん
\\	こくさいべんごし
\\	べんご しりょう
\\	よくきたね。どうぞ。
\\	よくきたね。どうぞ。
\\	じつは、きょう、わたしのゆうじんもいっしょなん
\\	じつは、きょう、わたしのゆうじんもいっしょなん
\\	です。
\\	です。
\\	かのじょもかわいい?
\\	かのじょもかわいい?
\\	とんでもないひとね。
\\	とんでもないひとね。
\\	いやぁ、かわいい、かわいい。はじめまして。わた
\\	いやぁ、かわいい、かわいい。はじめまして。わた
\\	しは・・・
\\	しは・・・
\\	いや、もうわかってます。ようこからとなべせんせ
\\	いや、もうわかってます。ようこからとなべせんせ
\\	いのはなしはもうすでにきいていますから。たいしたことあ
\\	いのはなしはもうすでにきいていますから。たいしたことあ
\\	りませんね。わたしはたかはしさとこです。こくさいべんご
\\	りませんね。わたしはたかはしさとこです。こくさいべんご
\\	しとして、げんざい、ニューヨークとチューリヒでしごとし
\\	しとして、げんざい、ニューヨークとチューリヒでしごとし
\\	ています。しほうしけんのさいねんしょうごうかくしゃで
\\	ています。しほうしけんのさいねんしょうごうかくしゃで
\\	す。せかいでいちばんゆうしゅうなべんごしでございます。
\\	す。せかいでいちばんゆうしゅうなべんごしでございます。
\\	さて、こんやのこのミーティングなんですが、もちろんしご
\\	さて、こんやのこのミーティングなんですが、もちろんしご
\\	とのはなしでしょうね。そうじゃないとあなた、めんきょを
\\	とのはなしでしょうね。そうじゃないとあなた、めんきょを
\\	うしなうかのうせいがありますよ。
\\	うしなうかのうせいがありますよ。
\\	うっかり
\\	ずうずうしい
\\	まったく
\\	なによりも
\\	あばよ
\\	なじむ
\\	ありえない
\\	ろくでなし
\\	なかなかむずかしいケースですね。どうやったらふたりをたすけられるか・・・
\\	なかなかむずかしいケースですね。どうやったらふたりをたすけられるか・・・
\\	そうですか?かんたんですよ。いったい、なんねんべんごしやってるんですか。まったく。
\\	そうですか?かんたんですよ。いったい、なんねんべんごしやってるんですか。まったく。
\\	そ、そうですか。さすが、さいねんしょうごうかくしゃ
\\	そ、そうですか。さすが、さいねんしょうごうかくしゃ
\\	あっ、そういえば、メルからてがみがきたんだった
\\	うっかりわすれるところだった。いま、よむね。
\\	あっ、そういえば、メルからてがみがきたんだった
\\	うっかりわすれるところだった。いま、よむね。
\\	ようこちゃん、げんきにしてますか。ゆうしゅうなべんごしにはちゃんとあえましたか。がくちょうのしんゆうならきっとしんせつなひとだろうとおもいます。
\\	ようこちゃん、げんきにしてますか。ゆうしゅうなべんごしにはちゃんとあえましたか。がくちょうのしんゆうならきっとしんせつなひとだろうとおもいます。
\\	そのとおりですよ
\\	そのとおりですよ
\\	なにいってるんですか、ずうずうしい。
\\	なにいってるんですか、ずうずうしい。
\\	まあ、まって。つづきがあるから。えっと、
\\	まあ、まって。つづきがあるから。えっと、
\\	こんなことをいうとようこちゃんがあきれるかもしれないけど、ぼくとがくちょうはここでのせいかつをたのしんでます。
\\	こんなことをいうとようこちゃんがあきれるかもしれないけど、ぼくとがくちょうはここでのせいかつをたのしんでます。
\\	なんだって
\\	なんだって
\\	どうして?
\\	どうして?
\\	ここはさいこう
\\	しょくじがただでたべほうだい。そして、なによりも、まいにちただでにほんごのレッスンしほうだい
\\	とうきょうではだれもにほんごではなしかけてくれなかったのに、ここではみんなぜんぶにほんご
\\	ここはさいこう
\\	しょくじがただでたべほうだい。そして、なによりも、まいにちただでにほんごのレッスンしほうだい
\\	とうきょうではだれもにほんごではなしかけてくれなかったのに、ここではみんなぜんぶにほんご
\\	そりゃ、そうだろう。
\\	そりゃ、そうだろう。
\\	ようこちゃんにはわるいけど、しばらくここでのせいかつをたのしみたくなってきました。
\\	ようこちゃんにはわるいけど、しばらくここでのせいかつをたのしみたくなってきました。
\\	あきれた
\\	あきれた
\\	がくちょうもいっぱいともだちができて、まいにちたのしそうです。ぼくらにもうしばらくじかんをください。またてがみかきます。あばよ
\\	メル。
\\	がくちょうもいっぱいともだちができて、まいにちたのしそうです。ぼくらにもうしばらくじかんをください。またてがみかきます。あばよ
\\	メル。
\\	いいなあ、メルくん、すっかりなじんでて。うらやましいなあ。
\\	いいなあ、メルくん、すっかりなじんでて。うらやましいなあ。
\\	こんなろくでなしとつきあってたの?こんなひとたちのためにわたしをよんだなんてありえない
\\	ようこにはがっかりよ。
\\	こんなろくでなしとつきあってたの?こんなひとたちのためにわたしをよんだなんてありえない
\\	ようこにはがっかりよ。
\\	なりたくうこう
\\	たいどがわるい
\\	たいどがいい
\\	たいど
\\	おきゃくさま
\\	かんけいある
\\	かんけいない
\\	かんけい
\\	おしゃれ(な)
\\	とにかく
\\	こえる
\\	たびだつ
\\	もういい、ようこ、わたしかえる
\\	こんなことでじかんをむだにしたくない。
\\	もういい、ようこ、わたしかえる
\\	こんなことでじかんをむだにしたくない。
\\	ちょっとまって、さとこ、もうすこしだけはなしをきいて
\\	おねがい
\\	おねがいだから
\\	ちょっとまって、さとこ、もうすこしだけはなしをきいて
\\	おねがい
\\	おねがいだから
\\	タクシー
\\	タクシー
\\	こんばんは、どこまでですか。
\\	こんばんは、どこまでですか。
\\	なりたくうこうへ。
\\	なりたくうこうへ。
\\	くうこうですとじゅうまんをこえますよ。
\\	くうこうですとじゅうまんをこえますよ。
\\	いくらでもかまわない。とにかくくうこうへいって。はやくここをでたいの。はやくじぶんのおしゃれなせいかつにもどりたい。やっぱりもどってきたのはしっぱいだった。はやくここをわすれたい。
\\	いくらでもかまわない。とにかくくうこうへいって。はやくここをでたいの。はやくじぶんのおしゃれなせいかつにもどりたい。やっぱりもどってきたのはしっぱいだった。はやくここをわすれたい。
\\	どうしましたか。けんかでもしましたか。
\\	どうしましたか。けんかでもしましたか。
\\	けんかしてきたばかりよ 。でもあなたにかんけいないでしょう。
\\	けんかしてきたばかりよ 。でもあなたにかんけいないでしょう。
\\	いや、かんけいあります。わたしのおきゃくさまですから。おきゃくさまのしあわせがわたしのしあわせです。おきゃくさまがかなしいとわたしもかなしいです。
\\	いや、かんけいあります。わたしのおきゃくさまですから。おきゃくさまのしあわせがわたしのしあわせです。おきゃくさまがかなしいとわたしもかなしいです。
\\	すみません。やっぱりわたしのたいどがわるかった。ごめんなさい。
\\	すみません。やっぱりわたしのたいどがわるかった。ごめんなさい。
\\	いや、もんだいございません。 でも、そんなきもちでたびたたないでください。わたしでよければ、きぶんてんかんのおてつだいをしますよ。
\\	いや、もんだいございません。 でも、そんなきもちでたびたたないでください。わたしでよければ、きぶんてんかんのおてつだいをしますよ。
\\	そうね。そういわれたらおなかがすいてきたわ。
\\	そうね。そういわれたらおなかがすいてきたわ。
\\	もしよかったら、いいラーメンやをしってますよ。とうきょうでいちばんきたないけど、いちばんやすくて、いちばんおいしいラーメンです。
\\	もしよかったら、いいラーメンやをしってますよ。とうきょうでいちばんきたないけど、いちばんやすくて、いちばんおいしいラーメンです。
\\	じゃあ、そこへいって。もしよかったら、うんてんしゅさんもいっしょにどう?
\\	じゃあ、そこへいって。もしよかったら、うんてんしゅさんもいっしょにどう?
\\	よろこんでおともします。
\\	よろこんでおともします。
\\	ごぶさた
\\	めじるし
\\	しるし
\\	おにあい
\\	いがい(と)
\\	せけん
\\	わっかない
\\	(じかんが)たつ
\\	でかせぎ
\\	ぎゃく
\\	いっしょうけんめい
\\	じもと
\\	ちょう
\\	きがあう
\\	おいしいというしるし
\\	ゴキブリ
\\	かせぐ
\\	いらっしゃい
\\	いらっしゃい
\\	いらっしゃいませ、なんめいさまですか。あらまあ、りゅうたくん、おひさしぶりです。げんきですか。
\\	いらっしゃいませ、なんめいさまですか。あらまあ、りゅうたくん、おひさしぶりです。げんきですか。
\\	おかげさまで。ごぶさたしていました。おげんきですか。
\\	おかげさまで。ごぶさたしていました。おげんきですか。
\\	げんきいっぱいですよ。おともだちですか。
\\	げんきいっぱいですよ。おともだちですか。
\\	そうです。
\\	そうです。
\\	はじめまして、よろしくおねがいします。
\\	はじめまして、よろしくおねがいします。
\\	こちらこそ、さぁー、こちらへどうぞ。
\\	こちらこそ、さぁー、こちらへどうぞ。
\\	うそじゃなかったんですね。ほんとうにきたないな。あ
\\	ゴキブリ
\\	うそじゃなかったんですね。ほんとうにきたないな。あ
\\	ゴキブリ
\\	まあまあ、だいじょうぶですよ。ゴキブリはおいしいというしるしですよ。きにしないで。
\\	まあまあ、だいじょうぶですよ。ゴキブリはおいしいというしるしですよ。きにしないで。
\\	なんになさいますか。
\\	なんになさいますか。
\\	みそラーメンください
\\	みそラーメンください
\\	みそラーメンおねがいします。
\\	みそラーメンおねがいします。
\\	おふたり、きがあいますね。おとうさん、みそにちょう。
\\	おふたり、きがあいますね。おとうさん、みそにちょう。
\\	あいおう
\\	あいおう
\\	すごいおばあさんですね。
\\	すごいおばあさんですね。
\\	でしょう
\\	ここにくるとじもとをおもいだします。
\\	でしょう
\\	ここにくるとじもとをおもいだします。
\\	ごしゅっしんはどちらですか。
\\	ごしゅっしんはどちらですか。
\\	ほっかいどうのいなかからきました。
\\	ほっかいどうのいなかからきました。
\\	え?わたしもほっかいどうです。でもいなかがだいきらいで、どうしてもほっかいどうをでたくて、いっしょうけんめいとうきょうのだいがくにはいりました。
\\	え?わたしもほっかいどうです。でもいなかがだいきらいで、どうしてもほっかいどうをでたくて、いっしょうけんめいとうきょうのだいがくにはいりました。
\\	わたしはぎゃくです。ここがきらいだけど、でかせぎのためにでてきました。たまにはかえったりしますか。
\\	わたしはぎゃくです。ここがきらいだけど、でかせぎのためにでてきました。たまにはかえったりしますか。
\\	ぜんぜんかえってないです。もうろくねんたちます。
\\	ぜんぜんかえってないです。もうろくねんたちます。
\\	ところで、ほっかいどうのどこですか。
\\	ところで、ほっかいどうのどこですか。
\\	わっかないです。
\\	わっかないです。
\\	え?わっかない?わたしもわっかないです。
\\	え?わっかない?わたしもわっかないです。
\\	ほんとうにわっかない?じゃ、いっしょですね。
\\	ほんとうにわっかない?じゃ、いっしょですね。
\\	はい
\\	はい
\\	ろっぽんぎヒルズ
\\	しょうたいけん
\\	みおくり
\\	ぶじ
\\	ゆくえふめい
\\	おちつく
\\	ようす
\\	かしら
\\	たからもの
\\	ロボット
\\	レジデンス
\\	しょうひんけん
\\	ときは2010ねん、とうきょうのろっぽんぎヒルズにこころのやさしいおじいさんとおばあさんがくらしていました。
\\	ときは2010ねん、とうきょうのろっぽんぎヒルズにこころのやさしいおじいさんとおばあさんがくらしていました。
\\	ろっぽんぎヒルズレジデンンス25かいでございます。
\\	ろっぽんぎヒルズレジデンンス25かいでございます。
\\	ドアひらくおと
\\	ドアひらくおと
\\	ただいま。
\\	ただいま。
\\	おかえり。
\\	おかえり。
\\	ねえ、ばあちゃん。こどもがいなくてさびしいとおもうことはないかい。ばあちゃんはずっとほしかったのに、わしがずっとロボットのしごとでいそがしくて・・・もうこんなとしになっちゃったなあ。
\\	ねえ、ばあちゃん。こどもがいなくてさびしいとおもうことはないかい。ばあちゃんはずっとほしかったのに、わしがずっとロボットのしごとでいそがしくて・・・もうこんなとしになっちゃったなあ。
\\	あなた、なにいっているの。あなたとあなたのてでつくったこのロボットたちがわたしのたからものですよ。わたしはいま、じゅうぶんしあわせですよ。ほらみて。ここにいる500のロボットたちはみんなかぞくですよ。
\\	あなた、なにいっているの。あなたとあなたのてでつくったこのロボットたちがわたしのたからものですよ。わたしはいま、じゅうぶんしあわせですよ。ほらみて。ここにいる500のロボットたちはみんなかぞくですよ。
\\	ありがとう。ばあさん。ほんとうにありがとう。ところで、あしたはおしょうがつだよ。すしでもかってこようか。
\\	ありがとう。ばあさん。ほんとうにありがとう。ところで、あしたはおしょうがつだよ。すしでもかってこようか。
\\	いいね。いきましょう。
\\	いいね。いきましょう。
\\	くるまのなか
\\	くるまのなか
\\	ひさしぶりのドライブですね。
\\	ひさしぶりのドライブですね。
\\	そうだね。
\\	そうだね。
\\	このレクサスにのるのははじめて。ルーフをオープンして。。。いいきもち
\\	このレクサスにのるのははじめて。ルーフをオープンして。。。いいきもち
\\	ほんとに。
\\	ほんとに。
\\	あっ
\\	あれみて
\\	こんなところで何やってるのかしら。
\\	あっ
\\	あれみて
\\	こんなところで何やってるのかしら。
\\	アイボだ
\\	あぶない
\\	アイボだ
\\	あぶない
\\	ひかれた
\\	かわいそうに。
\\	ひかれた
\\	かわいそうに。
\\	たすけないと...
\\	たすけないと...
\\	じいさん、きをつけて。
\\	じいさん、きをつけて。
\\	おじいさん、アイボをとってくる。
\\	おじいさん、アイボをとってくる。
\\	だいじょうぶ?
\\	だいじょうぶ?
\\	かなりひどいよ。はやくうちへつれてかえらないと。
\\	かなりひどいよ。はやくうちへつれてかえらないと。
\\	はやくかえりましょう。
\\	はやくかえりましょう。
\\	みっかご
\\	みっかご
\\	もうみっかだが、あまりようすがかわらないなあ。ぜんぜんうごかないし。もうだめかな。
\\	もうみっかだが、あまりようすがかわらないなあ。ぜんぜんうごかないし。もうだめかな。
\\	そんなにかんたんにあきらめないで。ほら。
\\	そんなにかんたんにあきらめないで。ほら。
\\	うごいた
\\	よかった。ほんとうによかった。じゃ、さんぽにいこう。
\\	うごいた
\\	よかった。ほんとうによかった。じゃ、さんぽにいこう。
\\	あなた、まだはやいですよ。ちゃんとなおるまでまちましょう。
\\	あなた、まだはやいですよ。ちゃんとなおるまでまちましょう。
\\	そうだな。さすが、ばあちゃん。
\\	そうだな。さすが、ばあちゃん。
\\	いっしゅうかんご
\\	いっしゅうかんご
\\	ただいま。
\\	ただいま。
\\	ふたりともおかえり。
\\	ふたりともおかえり。
\\	ばあさん、ついにかんぜんになおったよ
\\	ばあさん、ついにかんぜんになおったよ
\\	あなたがなおしたんでしょう。
\\	あなたがなおしたんでしょう。
\\	まあね。
\\	まあね。
\\	せっかくかぞくになったのに、さびしいけど・・・げんきになったから、そろそろ
\\	のみせへかえさなきゃ。
\\	せっかくかぞくになったのに、さびしいけど・・・げんきになったから、そろそろ
\\	のみせへかえさなきゃ。
\\	まだはやすぎるよ。
\\	まだはやすぎるよ。
\\	じいさん。
\\	じいさん。
\\	はい、わかりました。アイボ、そろそろほんとのおうちへかえろうか。
\\	はい、わかりました。アイボ、そろそろほんとのおうちへかえろうか。
\\	じゃ、いっしょにいきましょう。
\\	じゃ、いっしょにいきましょう。
\\	おばあさんとおじいさんが
\\	につく
\\	おばあさんとおじいさんが
\\	につく
\\	ここがうわさの
\\	ビルですか。
\\	ここがうわさの
\\	ビルですか。
\\	そうですね。アイボよ、おまえはここからにげてきたのか。ひしぶりのふるさとはどうだい?ともだちがまってるよ。ほら、みんなこっちをみてる。よっしゃ、ここからはひとりでいきなさい。ばあさん、かえりましょう。みてるとさびしい。
\\	そうですね。アイボよ、おまえはここからにげてきたのか。ひしぶりのふるさとはどうだい?ともだちがまってるよ。ほら、みんなこっちをみてる。よっしゃ、ここからはひとりでいきなさい。ばあさん、かえりましょう。みてるとさびしい。
\\	おばあさんとおじいさんがかえろうとすると、おばあさんがなにかにきづく。
\\	おばあさんとおじいさんがかえろうとすると、おばあさんがなにかにきづく。
\\	みて
\\	アイボがみんなこっちにきてる
\\	みて
\\	アイボがみんなこっちにきてる
\\	かいけい
\\	かんじょう
\\	かのうせい
\\	ごちそう
\\	いる・
\\	いっらしゃる
\\	ふたん
\\	てんかん
\\	おんがえし
\\	そろそろいきましょうか。
\\	そろそろいきましょうか。
\\	そうですね。いきましょう。
\\	そうですね。いきましょう。
\\	おばあさん、おかいけいおねがいします。
\\	おばあさん、おかいけいおねがいします。
\\	ええと、みそラーメン二つだから、700えんでございます。
\\	ええと、みそラーメン二つだから、700えんでございます。
\\	ええ、あんなにおいしかったのに、そんなにやすいんですか。しんじられないです。ごちそうさまでした。
\\	ええ、あんなにおいしかったのに、そんなにやすいんですか。しんじられないです。ごちそうさまでした。
\\	きにいってくれてよかった。
\\	きにいってくれてよかった。
\\	まいど、あのね、ちょっときいて、きいて、このラーメンはね、めんのこしがいのちで・・・
\\	まいど、あのね、ちょっときいて、きいて、このラーメンはね、めんのこしがいのちで・・・
\\	おじいさん、いいから
\\	また、いっらしゃってね。
\\	おじいさん、いいから
\\	また、いっらしゃってね。
\\	もちろん。またきます。
\\	もちろん。またきます。
\\	またかのじょもいっしょにね。
\\	またかのじょもいっしょにね。
\\	さ、どうでしょうね。
\\	さ、どうでしょうね。
\\	かのうせいは なくもないですね。ね、りゅうたさん。
\\	かのうせいは なくもないですね。ね、りゅうたさん。
\\	りゅうたくん、がんばってね。
\\	りゅうたくん、がんばってね。
\\	どうも。きぶんはどうですか。
\\	どうも。きぶんはどうですか。
\\	すっきりしました。ほんとうにありがとうございます。
\\	すっきりしました。ほんとうにありがとうございます。
\\	じゃ、くうこうでしたっけ?
\\	じゃ、くうこうでしたっけ?
\\	そうですね。
\\	そうですね。
\\	どこへいくんでしたっけ?
\\	どこへいくんでしたっけ?
\\	チューリッヒにかえるつもりでしたけど、いまからかいがいへいくきぶんではなくなりました。あのおばあさんとあって、あのラーメンをたべて、ひさしぶりにほっかいどうへかえりたくなりました。
\\	チューリッヒにかえるつもりでしたけど、いまからかいがいへいくきぶんではなくなりました。あのおばあさんとあって、あのラーメンをたべて、ひさしぶりにほっかいどうへかえりたくなりました。
\\	いいな〜。わたしもかえりたいな〜。でもでかせぎちゅうだし。
\\	いいな〜。わたしもかえりたいな〜。でもでかせぎちゅうだし。
\\	じゃ、このさいいっしょにいきませんか。
\\	じゃ、このさいいっしょにいきませんか。
\\	わ、わたしですか。しかし、しごとがありますし。おかねもなくて。
\\	わ、わたしですか。しかし、しごとがありますし。おかねもなくて。
\\	もんだいありません。わたしがぜんぶふたんしますよ。どうですか。
\\	もんだいありません。わたしがぜんぶふたんしますよ。どうですか。
\\	あなたのおかげで、わたしのきぶんがてんかんできて、なにかおんがえししたいんです。せっかくおなじじもとじゃないですか。
\\	あなたのおかげで、わたしのきぶんがてんかんできて、なにかおんがえししたいんです。せっかくおなじじもとじゃないですか。
\\	ぶちょう
\\	ビジネス
\\	まさか
\\	さっき
\\	たのむ
\\	じつは
\\	もどる
\\	やめる
\\	とくべつ
\\	もらう); 
\\	いただく
\\	ただいま。かえりました。
\\	ただいま。かえりました。
\\	おそいぞ。なにしてた?
\\	おそいぞ。なにしてた?
\\	すみませんぶちょう、とくべつなおきゃくさまがいまして。
\\	すみませんぶちょう、とくべつなおきゃくさまがいまして。
\\	なんだそりゃ
\\	とくべつなきゃく?きゃくにとくべつもなにもないだろ。これはビジネスだ
\\	なんだそりゃ
\\	とくべつなきゃく?きゃくにとくべつもなにもないだろ。これはビジネスだ
\\	すみませんでした。
\\	すみませんでした。
\\	にどとないようにきをつけろ。
\\	にどとないようにきをつけろ。
\\	はい、わかりました。
\\	はい、わかりました。
\\	はやくしごともどれ。
\\	はやくしごともどれ。
\\	じつは、ちょっとはなしがあります。
\\	じつは、ちょっとはなしがあります。
\\	はなしってなんだ?
\\	はなしってなんだ?
\\	ここでおせわになりはじめてもうさんねんです。
\\	ここでおせわになりはじめてもうさんねんです。
\\	やめるとかいうつもりか?むりだよ。きみはうちのいちばんいいドライバーだから。さっきはわるかった。すまん。たのむ、やめないでくれ
\\	やめるとかいうつもりか?むりだよ。きみはうちのいちばんいいドライバーだから。さっきはわるかった。すまん。たのむ、やめないでくれ
\\	まさか。ただ、ここにはいってから、まだいちどもやすんでいません。あしたからいっしゅうかんやすみをいただきたいんですが。
\\	まさか。ただ、ここにはいってから、まだいちどもやすんでいません。あしたからいっしゅうかんやすみをいただきたいんですが。
\\	ああ、そういうことか。いいけど。でもこんかいだけだぞ。で何をするつもりなんだ?
\\	ああ、そういうことか。いいけど。でもこんかいだけだぞ。で何をするつもりなんだ?
\\	ほっかいどうへかえってきます。
\\	ほっかいどうへかえってきます。
\\	よし、わかった。はい、いまからスタートな。
\\	よし、わかった。はい、いまからスタートな。
\\	いそぐ
\\	やきたて
\\	たんき
\\	せいかく
\\	おもいやりのある
\\	さいしゅうあんない
\\	まにあう
\\	おまたせしました。
\\	おまたせしました。
\\	あーまにあいました。もうなまえよばれたから、あまりじかんがないですよ。しゅっぱつまで10ぷんですから、いそがないと。
\\	あーまにあいました。もうなまえよばれたから、あまりじかんがないですよ。しゅっぱつまで10ぷんですから、いそがないと。
\\	ごめんなさい。おみやげです。とうきょうでいちばんうまいパンやさんによってきました。やきたてです。
\\	ごめんなさい。おみやげです。とうきょうでいちばんうまいパンやさんによってきました。やきたてです。
\\	パンや
\\	パンやにいってておくれたんですか
\\	パンや
\\	パンやにいってておくれたんですか
\\	そうです。さとこさんのごかぞくのおみやげをかってきました。
\\	そうです。さとこさんのごかぞくのおみやげをかってきました。
\\	わたしのかぞくですか。
\\	わたしのかぞくですか。
\\	もちろん。さとこさんがかうひまがないとおもって、わたしはかいにいきました。
\\	もちろん。さとこさんがかうひまがないとおもって、わたしはかいにいきました。
\\	すみませんでした。たんきなせいかくでごめんなさい。りゅうたさんみたいなおもいやりのあるひととあうのはひさしぶりです。
\\	すみませんでした。たんきなせいかくでごめんなさい。りゅうたさんみたいなおもいやりのあるひととあうのはひさしぶりです。
\\	いいえ、こちらこそ。とうきょうにきてから、さとこさんみたいなやさしいひとははじめてです。
\\	いいえ、こちらこそ。とうきょうにきてから、さとこさんみたいなやさしいひとははじめてです。
\\	さっぽろゆき321びんでごしゅっぱつのたかはしさん、さいしゅうあんないです。
\\	さっぽろゆき321びんでごしゅっぱつのたかはしさん、さいしゅうあんないです。
\\	わたしたちだ。
\\	わたしたちだ。
\\	まにあうかな?
\\	まにあうかな?
\\	はしろう。
\\	はしろう。
\\	とうき
\\	とうしゃ
\\	とうにん
\\	りりく
\\	ちゃくりく
\\	ぼうえき
\\	きかん
\\	せかいぼうえききかん
\\	せいふきかん
\\	とうてん
\\	びん
\\	きゅうに
\\	れいせいに
\\	きおく
\\	ゆがむ
\\	とうじ
\\	まほう
\\	なんとなく
\\	おちつく
\\	メモリー
\\	さっぽろゆき321びんはまもなくさっぽろくうこうへちゃくりくいたします。げんざいのさっぽろのてんこうははれ。きおんはにじゅうどとなっております。またのごりようをこころからおまちしております。
\\	さっぽろゆき321びんはまもなくさっぽろくうこうへちゃくりくいたします。げんざいのさっぽろのてんこうははれ。きおんはにじゅうどとなっております。またのごりようをこころからおまちしております。
\\	りゆうがわからないけど、きゅうにどきどきしてきました。
\\	りゆうがわからないけど、きゅうにどきどきしてきました。
\\	だいじょうぶですか。
\\	だいじょうぶですか。
\\	おかしいです。すごくおかしいです。
\\	おかしいです。すごくおかしいです。
\\	なにが?
\\	なにが?
\\	こんなきぶんひさしぶりです。わたしはふだんせかいぼうえききかんでれいせいにはなせるのに、なんでじもとのことをかんがえるときんちょうするんでしょう。
\\	こんなきぶんひさしぶりです。わたしはふだんせかいぼうえききかんでれいせいにはなせるのに、なんでじもとのことをかんがえるときんちょうするんでしょう。
\\	まあ、じんせいとはそういうものです。じかんがたつほど、きおくがゆがんでいく。
\\	まあ、じんせいとはそういうものです。じかんがたつほど、きおくがゆがんでいく。
\\	どういうこと?
\\	どういうこと?
\\	こころがきおくをすこしずつかえていく。たとえば、むかしのかのじょがそうです。とうじ、かのじょはすごくきれいでした。わかれてからぜんぜんあっていません。あたまのなかでじかんがとまっているから、かのじょはずっとうつくしいんです。そのうえ、こころのまほうでかのじょはもっときれいになります。しかし、じっさいにそのことあったら、あたまのなかのかのじょとぜんぜんちがう。まあ、いいたいことは、つまり、さとこさんのあたまのなかのわるいおもいでほど、げんじつはわるくないんです。もどったらわかりますよ。
\\	こころがきおくをすこしずつかえていく。たとえば、むかしのかのじょがそうです。とうじ、かのじょはすごくきれいでした。わかれてからぜんぜんあっていません。あたまのなかでじかんがとまっているから、かのじょはずっとうつくしいんです。そのうえ、こころのまほうでかのじょはもっときれいになります。しかし、じっさいにそのことあったら、あたまのなかのかのじょとぜんぜんちがう。まあ、いいたいことは、つまり、さとこさんのあたまのなかのわるいおもいでほど、げんじつはわるくないんです。もどったらわかりますよ。
\\	ありがとう、りゅうたさん。りゅうたさんといっしょにいるとなんとなくおちつける。
\\	ありがとう、りゅうたさん。りゅうたさんといっしょにいるとなんとなくおちつける。
\\	こんやくしゃ
\\	そうだん
\\	ぴったり
\\	タイプ
\\	ふくしゃちょう
\\	いきとうごう
\\	おもてさんどう
\\	(でんしゃのあんない)まもなくドアがしまります。ごちゅういください。
\\	(でんしゃのあんない)まもなくドアがしまります。ごちゅういください。
\\	ああ、ぎりぎりまにあった。
\\	ああ、ぎりぎりまにあった。
\\	よかった。まさか
\\	こんなところで
\\	よかった。まさか
\\	こんなところで
\\	だれ?
\\	だれ?
\\	だいがくのどうきゅうせい。ようこちゃん。いつもはなしてたじゃない。
\\	だいがくのどうきゅうせい。ようこちゃん。いつもはなしてたじゃない。
\\	すごい。あの、ようこちゃん?
\\	すごい。あの、ようこちゃん?
\\	げんきそうにみえないけど、だいじょうぶかな。ようこちゃん。
\\	げんきそうにみえないけど、だいじょうぶかな。ようこちゃん。
\\	さゆりちゃん
\\	こんなところで
\\	さゆりちゃん
\\	こんなところで
\\	ほんとう。ようこちゃん、げんき?
\\	ほんとう。ようこちゃん、げんき?
\\	あんまり。
\\	あんまり。
\\	なんかあった?
\\	なんかあった?
\\	まあ、さいきんいろいろ。しんゆうがいなくなったり、しごとがなくなったり。かなりたいへんだったんだ。
\\	まあ、さいきんいろいろ。しんゆうがいなくなったり、しごとがなくなったり。かなりたいへんだったんだ。
\\	そうだんすればよかったのに。ちからになれたかも。
\\	そうだんすればよかったのに。ちからになれたかも。
\\	まあ、でも、わたしのことはどうでもいいよ。さゆりちゃんは?
\\	まあ、でも、わたしのことはどうでもいいよ。さゆりちゃんは?
\\	かなりいいかんじ。わたしたち、いま、おもてさんどへむかってるとこ。きょうはこんやくしゃのかれといっしょにへやをさがしにいくの。
\\	かなりいいかんじ。わたしたち、いま、おもてさんどへむかってるとこ。きょうはこんやくしゃのかれといっしょにへやをさがしにいくの。
\\	こんやくしゃ
\\	こんやくしゃ
\\	はじめまして。たにヒロともうします。どうぞよろしくおねがいします。よくさゆりから、はなしをきいています。
\\	はじめまして。たにヒロともうします。どうぞよろしくおねがいします。よくさゆりから、はなしをきいています。
\\	どうも。はじめまして。すてき
\\	ぴったり
\\	しかもあたしのタイプ
\\	うらやましい。どこでしりあったの?
\\	どうも。はじめまして。すてき
\\	ぴったり
\\	しかもあたしのタイプ
\\	うらやましい。どこでしりあったの?
\\	しごとで。かれはふくしゃちょうだよ。
\\	しごとで。かれはふくしゃちょうだよ。
\\	いいな〜。ぶじにそつぎょうした?
\\	いいな〜。ぶじにそつぎょうした?
\\	うん、あぶなかったけどね。わたしそのひからずっとついてるみたい。じつは、このしごと、きまってたひとがいたんだけど、そのひとがこなかったから、わたしがはいって、ヒロとであって、あっというまにいきとうごう。
\\	うん、あぶなかったけどね。わたしそのひからずっとついてるみたい。じつは、このしごと、きまってたひとがいたんだけど、そのひとがこなかったから、わたしがはいって、ヒロとであって、あっというまにいきとうごう。
\\	かいしゃはほとんどおとこだったから、さゆりがはいってくれて、ほんとうによかったです。
\\	かいしゃはほとんどおとこだったから、さゆりがはいってくれて、ほんとうによかったです。
\\	どこのかいしゃ?
\\	どこのかいしゃ?
\\	じゃなくて、
\\	っていうかいしゃ。
\\	じゃなくて、
\\	っていうかいしゃ。
\\	え、
\\	え、
\\	しってる?ね、ようこちゃん、どうかした?なんかいってよ。
\\	しってる?ね、ようこちゃん、どうかした?なんかいってよ。
\\	かう
\\	どこからともなく
\\	わかれわかれ
\\	こまりもの
\\	さいかい
\\	あまのがわ
\\	みずかさ
\\	まちにまった
\\	ひとめみる
\\	あまりにも
\\	カササギ
\\	むかえる
\\	きし
\\	しゅくふくする
\\	としごろ
\\	ひさしぶりにさっぽろのくうきをすった。ほんとうにきれいですね。
\\	ひさしぶりにさっぽろのくうきをすった。ほんとうにきれいですね。
\\	そうですね。やっぱり、ほっかいどうがいいです。
\\	そうですね。やっぱり、ほっかいどうがいいです。
\\	でも、いま10じです。わっかないまでのでんしゃがなくなった。どうしよう?
\\	でも、いま10じです。わっかないまでのでんしゃがなくなった。どうしよう?
\\	よかった。じつは、いくまえに、みせたいところある。ぼくをしんようしている?
\\	よかった。じつは、いくまえに、みせたいところある。ぼくをしんようしている?
\\	まあ。
\\	まあ。
\\	じゃ、いきましょう。
\\	じゃ、いきましょう。
\\	はい、とうちゃく。おりましょう。
\\	はい、とうちゃく。おりましょう。
\\	ほしがきれい
\\	ほしがきれい
\\	きょうはなんのひでしょう?
\\	きょうはなんのひでしょう?
\\	ああ
\\	たなばただ
\\	ああ
\\	たなばただ
\\	そうだよ。
\\	そうだよ。
\\	あ
\\	あまのがわがみえる
\\	あのものがたり、なつかしいわ。おぼえてる?
\\	あ
\\	あまのがわがみえる
\\	あのものがたり、なつかしいわ。おぼえてる?
\\	むかしむかし、あまのがわのそばにはてんのかみさまがすんでいました。
\\	むかしむかし、あまのがわのそばにはてんのかみさまがすんでいました。
\\	すごい
\\	すごい
\\	てんのかみさまには、ひとりのむすめがいました。なまえを
\\	てんのかみさまには、ひとりのむすめがいました。なまえを
\\	おりひめといいました。
\\	おりひめといいました。
\\	おっ
\\	おっ
\\	おりひめははたをおって、かみさまたちのきものをつくるしごとをしていました。おりひめがやがてとしごろになり、てんのかみさまはむすめに、おむこさんをむかえてやろうとおもいました。いろいろさがしてみつけたのが、あまのがわのきしでてんのウシをかっている、ひこぼしというわかものです。
\\	おりひめははたをおって、かみさまたちのきものをつくるしごとをしていました。おりひめがやがてとしごろになり、てんのかみさまはむすめに、おむこさんをむかえてやろうとおもいました。いろいろさがしてみつけたのが、あまのがわのきしでてんのウシをかっている、ひこぼしというわかものです。
\\	やるね。ひこぼしは、とてもりっぱなわかものでした。おりひめも、かがやくばかりにうつくしいむすめです。ふたりはあいてをひとめみただけで、すきになりました。
\\	やるね。ひこぼしは、とてもりっぱなわかものでした。おりひめも、かがやくばかりにうつくしいむすめです。ふたりはあいてをひとめみただけで、すきになりました。
\\	ふたりはけっこんして、たのしいせいかつをおくるようになりました。でも、なかがよすぎるのもこまりもので、ふたりはしごとをわすれて、あそんでばかりいるようになったのです。
\\	ふたりはけっこんして、たのしいせいかつをおくるようになりました。でも、なかがよすぎるのもこまりもので、ふたりはしごとをわすれて、あそんでばかりいるようになったのです。
\\	「おりひめがはたおりをしないので、みんなのきものがふるくてボロボロです。はやくあたらしいきものをつくってください。」
\\	「おりひめがはたおりをしないので、みんなのきものがふるくてボロボロです。はやくあたらしいきものをつくってください。」
\\	「ひこぼしがせわをしないので、ウシたちがびょうきになってしまいます。」
\\	「ひこぼしがせわをしないので、ウシたちがびょうきになってしまいます。」
\\	てんのかみさまに、みんながもんくをいいにくるようになりました。かみさまは、すっかりおこってしまい、「ふたりはあまのがわの、ひがしとにしにわかれてくらすがよい
\\	と、いって、おりひめとひこぼしを、わかれわかれにしたのです。
\\	てんのかみさまに、みんながもんくをいいにくるようになりました。かみさまは、すっかりおこってしまい、「ふたりはあまのがわの、ひがしとにしにわかれてくらすがよい
\\	と、いって、おりひめとひこぼしを、わかれわかれにしたのです。
\\	でもてんのかみさまは、おりひめがあまりにもかなしそうにしているのをみて、こういいました。
\\	でもてんのかみさまは、おりひめがあまりにもかなしそうにしているのをみて、こういいました。
\\	「いちねんにいちどだけ、しちがつなのかのよるだけ、ひこぼしとあってもよろしい」
\\	「いちねんにいちどだけ、しちがつなのかのよるだけ、ひこぼしとあってもよろしい」
\\	それから、いちねんにいちどあえるひだけをたのしみにして、おりひめはまいにち、いっしょうけんめいはたをおりました。
\\	それから、いちねんにいちどあえるひだけをたのしみにして、おりひめはまいにち、いっしょうけんめいはたをおりました。
\\	あまのがわのむこうのひこぼしも、てんのウシをかうしごとにせいをだしました。そして、まちにまったしちがつなのかのよる、おりひめはあまのがわをわたって、ひこぼしのところへあいにいきます。
\\	あまのがわのむこうのひこぼしも、てんのウシをかうしごとにせいをだしました。そして、まちにまったしちがつなのかのよる、おりひめはあまのがわをわたって、ひこぼしのところへあいにいきます。
\\	でも、あめがふるとあまのがわのみずかさがふえるため、おりひめはかわをわたることができません。そんなときは、どこからともなくカササギというとりがとんできて、あまのがわにはしをかけてくれるのです。
\\	でも、あめがふるとあまのがわのみずかさがふえるため、おりひめはかわをわたることができません。そんなときは、どこからともなくカササギというとりがとんできて、あまのがわにはしをかけてくれるのです。
\\	このまま
\\	かまわない
\\	ゆっくり
\\	ぐずぐずする
\\	みぬく
\\	ベルト
\\	おくる
\\	なれる
\\	うんてんしゅう
\\	ほんとうにたのしかった。こんやどうする?かなりおそくなった。
\\	ほんとうにたのしかった。こんやどうする?かなりおそくなった。
\\	そうですね。じゃ、ねないでこのままわっかないへかえる?
\\	そうですね。じゃ、ねないでこのままわっかないへかえる?
\\	でも、くるまだとりゅうたはずっとうんてんしてリラックスできないから。でんしゃのほうがらくでなにもかんがえなくていい。
\\	でも、くるまだとりゅうたはずっとうんてんしてリラックスできないから。でんしゃのほうがらくでなにもかんがえなくていい。
\\	ぼくはぜんぜんかまわない。うんてんすることとかんがえることがすきだし。ぜんぜんもんだいない。なれてるよ。まいにちおきゃくさんをおそくまでおくっているから。ほんとうにへいき。
\\	ぼくはぜんぜんかまわない。うんてんすることとかんがえることがすきだし。ぜんぜんもんだいない。なれてるよ。まいにちおきゃくさんをおそくまでおくっているから。ほんとうにへいき。
\\	あたしはおきゃくさんじゃない。
\\	あたしはおきゃくさんじゃない。
\\	もちろん。
\\	もちろん。
\\	ゆっくりしたいの。さいきん、さいごにゆっくりしたのはいつ?
\\	ゆっくりしたいの。さいきん、さいごにゆっくりしたのはいつ?
\\	まあ、
\\	まあ、
\\	きまり。ゆっくりいきましょう。
\\	きまり。ゆっくりいきましょう。
\\	さとこ、いくらぐずぐずしてもいつかはかえらないといけないよ。
\\	さとこ、いくらぐずぐずしてもいつかはかえらないといけないよ。
\\	ね、りゅうた、ちょっときになってるんだけど、
\\	ね、りゅうた、ちょっときになってるんだけど、
\\	なに?
\\	なに?
\\	わたしはたくさんのゆうしゅうなひととであったのに、りゅうたみたいなひとははじめて。タクシーのうんてんしゅさんらしくない。
\\	わたしはたくさんのゆうしゅうなひととであったのに、りゅうたみたいなひとははじめて。タクシーのうんてんしゅさんらしくない。
\\	いままでなんにんのタクシーうんてんしゅとはなしたことがある?
\\	いままでなんにんのタクシーうんてんしゅとはなしたことがある?
\\	ほら、またふかいはなしをする。こんなにするどくてかわっているひと、なにかうらがあるみたい。ぜったいみぬいてみせる。
\\	ほら、またふかいはなしをする。こんなにするどくてかわっているひと、なにかうらがあるみたい。ぜったいみぬいてみせる。
\\	みぬいてみてください。そのあいだどうしよう?
\\	みぬいてみてください。そのあいだどうしよう?
\\	じゃ、かえろうか。
\\	じゃ、かえろうか。
\\	りょうかい
\\	ベルトしてください。いきましょう。
\\	りょうかい
\\	ベルトしてください。いきましょう。
\\	はまべ
\\	たまてばこ
\\	みあたる
\\	すら
\\	すれちがう
\\	ろうじん
\\	きり
\\	ほんの
\\	もくもくと
\\	いきもの
\\	としおいる
\\	かめ
\\	たたく
\\	ける
\\	りゅうぐうじょう
\\	りっぱ(な)
\\	おとひめ
\\	おおよろこび
\\	こいしい
\\	すっかり
\\	ごらん
\\	しんこきゅう
\\	なんて
\\	いってくる
\\	おてあらい
\\	せんこきゅう
\\	つらい
\\	じけん
\\	ね、ついたよ。わっかない
\\	ね、ついたよ。わっかない
\\	いまなんじ?
\\	いまなんじ?
\\	7じはん。あさごはんをたべよう。
\\	7じはん。あさごはんをたべよう。
\\	ね、あのみせまだあるね。なつかしいな~。そこでたべよう。
\\	ね、あのみせまだあるね。なつかしいな~。そこでたべよう。
\\	あのみせ?
\\	あのみせ?
\\	うん。
\\	うん。
\\	いいよ・・・とうちゃく。
\\	いいよ・・・とうちゃく。
\\	(ドアひらくおと)
\\	(ドアひらくおと)
\\	ちょっとおてあらいにいってくる。
\\	ちょっとおてあらいにいってくる。
\\	りゅうた
\\	りゅうたなの?
\\	りゅうた
\\	りゅうたなの?
\\	おひさしぶりです。
\\	おひさしぶりです。
\\	りゅうたがこのまちにもどってくるとおもわなかった。あのじけんのあと、きゅうにいなくなって・・・どうしたの?どこにいってたの?なにかんがえてるの?つらかったのはじぶんだけとおもっているの?いなくなっちゃうなんて
\\	りゅうたがこのまちにもどってくるとおもわなかった。あのじけんのあと、きゅうにいなくなって・・・どうしたの?どこにいってたの?なにかんがえてるの?つらかったのはじぶんだけとおもっているの?いなくなっちゃうなんて
\\	おちついておちついて
\\	わかった、わかったから。しんこきゅうしてごらん。
\\	おちついておちついて
\\	わかった、わかったから。しんこきゅうしてごらん。
\\	とにかく、あなたにあいたがっているひとがいっぱいいるよ。いこう
\\	とにかく、あなたにあいたがっているひとがいっぱいいるよ。いこう
\\	わかった。ちょっとまってね。ともだちにメモをかく。かのじょにこのメモを。よろしくおねがいします。
\\	わかった。ちょっとまってね。ともだちにメモをかく。かのじょにこのメモを。よろしくおねがいします。
\\	ごかい
\\	かこ
\\	とつぜん
\\	むきあう
\\	しょうじき
\\	てだすけをする
\\	〜ころ
\\	すみません。わたしのともだちをしりませんか。
\\	すみません。わたしのともだちをしりませんか。
\\	あ、りゅうたくん?
\\	あ、りゅうたくん?
\\	どうしてしっているんですか。
\\	どうしてしっているんですか。
\\	このまちでりゅうたくんをしらないひとはいないよ。
\\	このまちでりゅうたくんをしらないひとはいないよ。
\\	え。どういうこと?
\\	え。どういうこと?
\\	たぶんこのてがみをみれば、わかるんじゃないかな。
\\	たぶんこのてがみをみれば、わかるんじゃないかな。
\\	このてがみをよむころにはぼくはもういませんが、ごかいしないでください。じつは、さとこだけではなく、ぼくもこのまちにわすれられないかこがあります。こんなきゅうなことになるとおもっていなかったけど、とつぜんそれとむきあうときがきてしまいました。いまはせつめいできないけど、つぎにあうときにはかならずせつめいします。しょうじきいうとさとことこんなにしたしくなってここでわかれるのはつらいけど、でも、ぼくのしごとはおわりました。このたびのもくてきはさとこがかことむきあうてだすけをすることだったから。またあうひまでおげんきで。りゅうた
\\	このてがみをよむころにはぼくはもういませんが、ごかいしないでください。じつは、さとこだけではなく、ぼくもこのまちにわすれられないかこがあります。こんなきゅうなことになるとおもっていなかったけど、とつぜんそれとむきあうときがきてしまいました。いまはせつめいできないけど、つぎにあうときにはかならずせつめいします。しょうじきいうとさとことこんなにしたしくなってここでわかれるのはつらいけど、でも、ぼくのしごとはおわりました。このたびのもくてきはさとこがかことむきあうてだすけをすることだったから。またあうひまでおげんきで。りゅうた
\\	なんか、どこかでみたようなかおだね。
\\	なんか、どこかでみたようなかおだね。
\\	いや、そんなことないとおもいます。しつれいします。
\\	いや、そんなことないとおもいます。しつれいします。
\\	にわ
\\	とうぜん
\\	あわせる
\\	かおをあわせる
\\	ごぞんじ
\\	おかげさまで
\\	せきてい
\\	ほこり
\\	はい、ここでいいです。
\\	はい、ここでいいです。
\\	2600えんでございます。どうもありがとうございました。
\\	2600えんでございます。どうもありがとうございました。
\\	なつかしい〜。子どものころあのきにのぼったり、そのにわでバーベキューをやったり、でもまって。なんでこんなにいえがぼろぼろになっているの。むかしとずいぶんちがう。ただいま
\\	おかあさん、おとうさん、みなでかけてるの?
\\	なつかしい〜。子どものころあのきにのぼったり、そのにわでバーベキューをやったり、でもまって。なんでこんなにいえがぼろぼろになっているの。むかしとずいぶんちがう。ただいま
\\	おかあさん、おとうさん、みなでかけてるの?
\\	こんにちは。
\\	こんにちは。
\\	ああ、どうも。こんにちは。
\\	ああ、どうも。こんにちは。
\\	となりのすずきともうします。だれかをおさがしですか。
\\	となりのすずきともうします。だれかをおさがしですか。
\\	はい、ここにすんでいるたかはしですが。
\\	はい、ここにすんでいるたかはしですが。
\\	ああ、さとこちゃん。おかえりなさい。
\\	ああ、さとこちゃん。おかえりなさい。
\\	えっ、なんでわかるんですか。
\\	えっ、なんでわかるんですか。
\\	とうぜんですよ。あなたがここにいたころは、まいにちかおあわせていましたよ。ざんねんですが、いまはだれもこのいえにすんでいませんよ。
\\	とうぜんですよ。あなたがここにいたころは、まいにちかおあわせていましたよ。ざんねんですが、いまはだれもこのいえにすんでいませんよ。
\\	どういうこと?なにがあったの?おしえてください
\\	どういうこと?なにがあったの?おしえてください
\\	ながいはなしになります。おちゃをのみながら、ゆっくりはなしてあげます。わたしのいえにいきましょう。
\\	ながいはなしになります。おちゃをのみながら、ゆっくりはなしてあげます。わたしのいえにいきましょう。
\\	ただいま。
\\	ただいま。
\\	ごぶさたしています。
\\	ごぶさたしています。
\\	あら、さとこちゃん?おひさしぶりです。げんきにしてた?
\\	あら、さとこちゃん?おひさしぶりです。げんきにしてた?
\\	はいおかげさまで。
\\	はいおかげさまで。
\\	おとなのじょせいになったわね。こくさいべんごしのしごとはどう?
\\	おとなのじょせいになったわね。こくさいべんごしのしごとはどう?
\\	ええ、なんでそんなことをごぞんじなんですか。
\\	ええ、なんでそんなことをごぞんじなんですか。
\\	さとこちゃんはこのまちのほこりよ。さとこちゃんがやっていることはみんなしってる。
\\	さとこちゃんはこのまちのほこりよ。さとこちゃんがやっていることはみんなしってる。
\\	ほんとうですか。
\\	ほんとうですか。
\\	ほんとうよ。
\\	ほんとうよ。
\\	ありがとうございます。だけど、すずきさん、おねがいします。わたしのかぞくはどこへ?
\\	ありがとうございます。だけど、すずきさん、おねがいします。わたしのかぞくはどこへ?
\\	わたしはおちゃをもってきます。
\\	わたしはおちゃをもってきます。
\\	さとこちゃん
\\	さとこちゃん
\\	ええ、なんでないているんですか。
\\	ええ、なんでないているんですか。
\\	かなしいしらせがあるのよ。れんらくしようとしたんだけど。れんらくがとれなくて。
\\	かなしいしらせがあるのよ。れんらくしようとしたんだけど。れんらくがとれなくて。
\\	かぞくは?
\\	かぞくは?
\\	しょくどう
\\	リハビリ
\\	せめる
\\	けが
\\	しおくり
\\	すむ
\\	ねえさん、ひさしぶり。
\\	ねえさん、ひさしぶり。
\\	りゅうた
\\	いつかえってきたの?
\\	りゅうた
\\	いつかえってきたの?
\\	けさ。しょくどうにはいったら、さくらとあって、すぐここにきた。ちょうしはどう?
\\	けさ。しょくどうにはいったら、さくらとあって、すぐここにきた。ちょうしはどう?
\\	よくなってきた。
\\	よくなってきた。
\\	おかねはちゃんととどいてる?
\\	おかねはちゃんととどいてる?
\\	うん。ありがとう。
\\	うん。ありがとう。
\\	リハビリはうまくいっている?
\\	リハビリはうまくいっている?
\\	うん。がんばってる。りゅうた、なんであのじこのあと、すぐまちをでたの?りゅうたがひつようだったのに。
\\	うん。がんばってる。りゅうた、なんであのじこのあと、すぐまちをでたの?りゅうたがひつようだったのに。
\\	ごめん。ほんとうにごめん。でも、おかねがひつようだとおもったから。たくさんひつようってわかっていて。
\\	ごめん。ほんとうにごめん。でも、おかねがひつようだとおもったから。たくさんひつようってわかっていて。
\\	あのじこはりゅうたのせいじゃないよ。じぶんをせめるのをやめないと・・・
\\	あのじこはりゅうたのせいじゃないよ。じぶんをせめるのをやめないと・・・
\\	でも、ぼくがうんてんしていたら、きっとみんなはだいじょうぶだった。おとうさんとおかあさんもあのかぞくもしんでなかったし、ねえさんもけがをせずにすんだ。
\\	でも、ぼくがうんてんしていたら、きっとみんなはだいじょうぶだった。おとうさんとおかあさんもあのかぞくもしんでなかったし、ねえさんもけがをせずにすんだ。
\\	いえ、もうやめましょう。すんだことをいってもしかたがないわ。あなたはいっしょうけんめいわたしのためにしおくりしてくれているのに。ごめんね。
\\	いえ、もうやめましょう。すんだことをいってもしかたがないわ。あなたはいっしょうけんめいわたしのためにしおくりしてくれているのに。ごめんね。
\\	いいよ、ねえさんははやくよくなることだけをかんがえて。もうどこにもいかないから、あんしんして。
\\	いいよ、ねえさんははやくよくなることだけをかんがえて。もうどこにもいかないから、あんしんして。
\\	ぼうそうぞく
\\	ごうてい
\\	ろてんぶろ
\\	おいだす
\\	なかま
\\	のぞく
\\	いきのした
\\	ぶっころす
\\	とける
\\	しきっている
\\	がいする
\\	こしかける
\\	らんぼう
\\	きぶんをがいする
\\	おはか
\\	てんごく
\\	たいへん
\\	なぐさめる
\\	だきかかえる
\\	じこ
\\	いちにんまえ
\\	おやこうこう
\\	こや
\\	おせんこう
\\	おしょう
\\	おしりあい
\\	すみません、こどものときいらいおはかまいりしていないので、どうしたらいいのかわからなくて。おしえていただけますか?
\\	すみません、こどものときいらいおはかまいりしていないので、どうしたらいいのかわからなくて。おしえていただけますか?
\\	あ、そうですか。まず、おはかをきれいにおそうじしてあげてください。それからおはなとおせんこうをあげてあげたらいいでしょう。
\\	あ、そうですか。まず、おはかをきれいにおそうじしてあげてください。それからおはなとおせんこうをあげてあげたらいいでしょう。
\\	はい、わかりました。
\\	はい、わかりました。
\\	バケツやなにかは、あのこやにあるものをつかってください。
\\	バケツやなにかは、あのこやにあるものをつかってください。
\\	どうも、ありがとうございます。
\\	どうも、ありがとうございます。
\\	おとうさん、おかあさん、どうして
\\	おとうさん、おかあさん、どうして
\\	わたし、なにもおやこうこうできなくてごめんね。やっと、いちにんまえになって、これからだとおもってかえってきたのに
\\	おとうさんとおかあさんがじこにあったこともしらずに
\\	わたし、なにもおやこうこうできなくてごめんね。やっと、いちにんまえになって、これからだとおもってかえってきたのに
\\	おとうさんとおかあさんがじこにあったこともしらずに
\\	あー、さとこちゃんか。おぼえてないだろうな。あのころはまだこんなにちいさかったから。むかしよくおとうさんとおかあさんとおぼんのおはかまいりにきていたんだが。こんなにおおきくなって。
\\	あー、さとこちゃんか。おぼえてないだろうな。あのころはまだこんなにちいさかったから。むかしよくおとうさんとおかあさんとおぼんのおはかまいりにきていたんだが。こんなにおおきくなって。
\\	あのころのことはなんとなくおぼえています。わたしがおはかにくるのがこわくてないているのを、ちちがだきかかえてくれて、ははが「さとこがきたからおじいちゃんもおばあちゃんもよろこんでいるわよ。」ってなぐさめてくれて。
\\	あのころのことはなんとなくおぼえています。わたしがおはかにくるのがこわくてないているのを、ちちがだきかかえてくれて、ははが「さとこがきたからおじいちゃんもおばあちゃんもよろこんでいるわよ。」ってなぐさめてくれて。
\\	このたびはいろいろ、たいへんだったね。でもおとうさんとおかあさんはまいとしここへきてはさとこちゃんのことをじまんにはなしてくれていたよ。
\\	このたびはいろいろ、たいへんだったね。でもおとうさんとおかあさんはまいとしここへきてはさとこちゃんのことをじまんにはなしてくれていたよ。
\\	ほんとうですか?
\\	ほんとうですか?
\\	ああ、せかいいちのむすめだって。ほんとうにこんなにりっぱになって。てんごくのおとうさんもおかあさんもしんぱいいらないな。
\\	ああ、せかいいちのむすめだって。ほんとうにこんなにりっぱになって。てんごくのおとうさんもおかあさんもしんぱいいらないな。
\\	でも、わたしこれからひとりでどうしたらいいのか。
\\	でも、わたしこれからひとりでどうしたらいいのか。
\\	さとこちゃん、さとこちゃんのおとうさんとおかあさんは、てんごくへいってもいつもさとこちゃんといっしょだよ。つらいけどじしんをもって生きていきなさい。
\\	さとこちゃん、さとこちゃんのおとうさんとおかあさんは、てんごくへいってもいつもさとこちゃんといっしょだよ。つらいけどじしんをもって生きていきなさい。
\\	ありがとうございます。なんだかきぶんがらくになりました。わたしがんばります。てんごくにいるおとうさんとおかあさんのためにも。
\\	ありがとうございます。なんだかきぶんがらくになりました。わたしがんばります。てんごくにいるおとうさんとおかあさんのためにも。
\\	おや、あのかたもおしりあいですか。
\\	おや、あのかたもおしりあいですか。
\\	りゅうたさん
\\	りゅうたさん
\\	ざんこくな
\\	じじつ
\\	とちゅう
\\	きづく
\\	だまる
\\	はんのう
\\	なおる
\\	おわび
\\	なにいえばいいか
\\	なにいえばいいか
\\	なんてざんこくなせかい
\\	さいしょからしっていたの?
\\	なんてざんこくなせかい
\\	さいしょからしっていたの?
\\	いや、そんなことはない。さいしょはしらなかった。それはじじつだよ。
\\	いや、そんなことはない。さいしょはしらなかった。それはじじつだよ。
\\	でも、とちゅうできづいてたでしょう。でしょう
\\	でも、とちゅうできづいてたでしょう。でしょう
\\	それは
\\	それは
\\	なんでいわなかったの。なんでだまってたの?なんで
\\	なんでいわなかったの。なんでだまってたの?なんで
\\	おちついて、おちついて。
\\	おちついて、おちついて。
\\	こんなひろいせかいでなんであなたとであってしまったの??なんでだまっていたの。しっていたんでしょう?
\\	こんなひろいせかいでなんであなたとであってしまったの??なんでだまっていたの。しっていたんでしょう?
\\	ほんとうにごめんなさい。じかんがたてばたつほどぼくの、ぼくが
\\	つまり、いいづらくなって、だって、さとこはあんなにしあわせで
\\	しょうじきいうとわからない。
\\	ほんとうにごめんなさい。じかんがたてばたつほどぼくの、ぼくが
\\	つまり、いいづらくなって、だって、さとこはあんなにしあわせで
\\	しょうじきいうとわからない。
\\	もうききたくない。
\\	もうききたくない。
\\	さとこ、きみはぼくがなにいっても、なんてせつめいしても、ぜったいにきかなかった。
\\	さとこ、きみはぼくがなにいっても、なんてせつめいしても、ぜったいにきかなかった。
\\	そんなことはわからないでしょ。
\\	そんなことはわからないでしょ。
\\	こういうばあい、ぼくがなにいっても、さとこはおこっていた。それはとうぜんなはんのうだ。いまはたしかにつらいけど、しったばかりだから。じかんがたてば、きずはなおるんだ。
\\	こういうばあい、ぼくがなにいっても、さとこはおこっていた。それはとうぜんなはんのうだ。いまはたしかにつらいけど、しったばかりだから。じかんがたてば、きずはなおるんだ。
\\	ぼくはまいにちこのじこをふせぐことができたとおもいかえすんだ。そのじじつからはにげられない。ぼくはおわびとさようならをいいにきた。おげんきで。
\\	ぼくはまいにちこのじこをふせぐことができたとおもいかえすんだ。そのじじつからはにげられない。ぼくはおわびとさようならをいいにきた。おげんきで。
\\	くだらない
\\	ごとし
\\	や
\\	こういん
\\	えんがない
\\	えんがある
\\	こういんやのごとし
\\	いなくなってしまう
\\	いつのまにか
\\	えん
\\	でんわがなる
\\	でんわがなる
\\	はい、ようこです。
\\	はい、ようこです。
\\	さとこですけど。
\\	さとこですけど。
\\	え、さとこ、だいじょうぶ?どうしてでんわくれなかったの?
\\	え、さとこ、だいじょうぶ?どうしてでんわくれなかったの?
\\	いま、わっかないにいるの。
\\	いま、わっかないにいるの。
\\	え、もどったの?どうして?しごとは?
\\	え、もどったの?どうして?しごとは?
\\	もうもどらない。
\\	もうもどらない。
\\	え、どうして?
\\	え、どうして?
\\	ようこ、ごめんなさい。あんなくだらないことでおこったりして。ほんとうにごめんなさい。
\\	ようこ、ごめんなさい。あんなくだらないことでおこったりして。ほんとうにごめんなさい。
\\	さとこ、だいじょうぶ?
\\	さとこ、だいじょうぶ?
\\	いつのまにかみんないなくなってしまう。こういんやのごとしね。
\\	いつのまにかみんないなくなってしまう。こういんやのごとしね。
\\	いったいどうしたの?なんだかこわいわ。
\\	いったいどうしたの?なんだかこわいわ。
\\	ごめん、いろいろあったから。でも、わたしはだいじょうぶよ
\\	ただ、ようこのこえをききたかったの。いつさいごになるがわからないから。このにしゅうかんでいろんなことをまなんで、ようこのことがどれだけたいせつかわかった。そういうわけであやまりたかったの。ゆるしてくれるかな?
\\	ごめん、いろいろあったから。でも、わたしはだいじょうぶよ
\\	ただ、ようこのこえをききたかったの。いつさいごになるがわからないから。このにしゅうかんでいろんなことをまなんで、ようこのことがどれだけたいせつかわかった。そういうわけであやまりたかったの。ゆるしてくれるかな?
\\	もちろん。わたしはなんにもきにしてないよ。
\\	もちろん。わたしはなんにもきにしてないよ。
\\	わたしはこれからしばらくわっかないにのこるけど、ちょっとしょうかいしたいひとがいるの。
\\	わたしはこれからしばらくわっかないにのこるけど、ちょっとしょうかいしたいひとがいるの。
\\	だれ?
\\	だれ?
\\	かれのなまえはりゅうた。わたしにはえんがなかったけど、あなたにならピッタリだとおもうわ。
\\	かれのなまえはりゅうた。わたしにはえんがなかったけど、あなたにならピッタリだとおもうわ。
\\	え、どういうこと?
\\	え、どういうこと?
\\	もういかなくちゃ。
\\	もういかなくちゃ。
\\	あ、さとこ、まって
\\	あ、さとこ、まって
\\	もよう
\\	しょたいめん
\\	ぐうぜん
\\	かたまる
\\	にじ
\\	ふしぎ
\\	おまたせしてすみません。
\\	おまたせしてすみません。
\\	いいえ、わたしもいまきたところです。ほんとうにごめんなさい、しょたいめんなのにきゅうによびだしたりして。
\\	いいえ、わたしもいまきたところです。ほんとうにごめんなさい、しょたいめんなのにきゅうによびだしたりして。
\\	いいんですよ。さいきんしごとをくびになって、ひましてたところなんで。
\\	いいんですよ。さいきんしごとをくびになって、ひましてたところなんで。
\\	え、そうなんですか。わたしもいましごとがなくて。なんかぐうぜんですね。
\\	え、そうなんですか。わたしもいましごとがなくて。なんかぐうぜんですね。
\\	そうですね。
\\	そうですね。
\\	さいきんなにもかもがうまくいかなくて。
\\	さいきんなにもかもがうまくいかなくて。
\\	そういうこともありますよ。でも、あめふってじかたまるっていうじゃないですか。
\\	そういうこともありますよ。でも、あめふってじかたまるっていうじゃないですか。
\\	それもそうね。
\\	それもそうね。
\\	それにきれいなにじだってあめのあとにしかでませんからね。
\\	それにきれいなにじだってあめのあとにしかでませんからね。
\\	りゅうたさんてすごい、ロマンチックですね。
\\	りゅうたさんてすごい、ロマンチックですね。
\\	ははっ、はずかしいな。
\\	ははっ、はずかしいな。
\\	えがおもすてき。ねえ、きょうどうしましょうか?
\\	えがおもすてき。ねえ、きょうどうしましょうか?
\\	よよぎこうえんでピクニックはどうですか
\\	ざいりょうをかいにいって。
\\	よよぎこうえんでピクニックはどうですか
\\	ざいりょうをかいにいって。
\\	でも、てんきよほうによるときょうはごごからあめだそうです。
\\	でも、てんきよほうによるときょうはごごからあめだそうです。
\\	そうなんですか。
\\	そうなんですか。
\\	あの
\\	あの
\\	どうぞ。
\\	どうぞ。
\\	なんでだろう。しょたいめんなのにこんなに
\\	なんていうか
\\	なんでだろう。しょたいめんなのにこんなに
\\	なんていうか
\\	ええ
\\	ふしぎですよね。そうだ、じつはこのちかくにすごくおいしいラーメンやさんがあるんです。とうきょうでいちばんきたないけどいちばんおいしいラーメンやです。
\\	ええ
\\	ふしぎですよね。そうだ、じつはこのちかくにすごくおいしいラーメンやさんがあるんです。とうきょうでいちばんきたないけどいちばんおいしいラーメンやです。
\\	はい、ぜひ。
\\	はい、ぜひ。
\\	グラフィックデザイナー
\\	めずらしい
\\	フィリピン人
\\	めいし
\\	きぎょう
\\	おじょうさん
\\	だいファン
\\	ちょうせん
\\	ひとりぐらし
\\	さあ、もういかないとでんしゃにおくれちゃう。
\\	さあ、もういかないとでんしゃにおくれちゃう。
\\	まもなくさんばんせんよりじゅうにじよんじゅうななふんはつやまびこごじゅうにごうとうきょうゆきはっしゃいたします。
\\	まもなくさんばんせんよりじゅうにじよんじゅうななふんはつやまびこごじゅうにごうとうきょうゆきはっしゃいたします。
\\	とうきょうでひとりぐらしかー。わたしにできるかな?でも、ふくしまのいなかでおわるわけにはいかないもんね。やっとけっしんしたんだもん。がんばらないと。わたし、ファイト
\\	とうきょうでひとりぐらしかー。わたしにできるかな?でも、ふくしまのいなかでおわるわけにはいかないもんね。やっとけっしんしたんだもん。がんばらないと。わたし、ファイト
\\	えーと、ろくれつめのみぎからにばんめっと。お、ここか。しつれいしますよ。
\\	えーと、ろくれつめのみぎからにばんめっと。お、ここか。しつれいしますよ。
\\	はい、どうぞ。
\\	はい、どうぞ。
\\	いやー、やっぱりでんしゃのたびはいいねー。おや、こんなにおおきなカバンをもって、おじょうさんどちらまで?
\\	いやー、やっぱりでんしゃのたびはいいねー。おや、こんなにおおきなカバンをもって、おじょうさんどちらまで?
\\	とうきょうです。しごとをさがしに。
\\	とうきょうです。しごとをさがしに。
\\	どんなおしごとですか。
\\	どんなおしごとですか。
\\	グラフィックデザイナーです。
\\	グラフィックデザイナーです。
\\	それはいい。わかいうちになんでもちょうせんしないと。
\\	それはいい。わかいうちになんでもちょうせんしないと。
\\	わたしそんなにわかくないですよ。
\\	わたしそんなにわかくないですよ。
\\	はっはっは、わたしのとしになるとみんなこどもにみえてねー。
\\	はっはっは、わたしのとしになるとみんなこどもにみえてねー。
\\	わたし、むらかみアグネスともうします。
\\	わたし、むらかみアグネスともうします。
\\	おわかいのにしっかりしてるね。おくれました。よこたみつおです。
\\	おわかいのにしっかりしてるね。おくれました。よこたみつおです。
\\	はじめまして。
\\	はじめまして。
\\	はじめまして。アグネスかめずらしいなまえだね。
\\	はじめまして。アグネスかめずらしいなまえだね。
\\	ええ、ははがフィリピンじんで。
\\	ええ、ははがフィリピンじんで。
\\	どおりで。
\\	どおりで。
\\	でも、ただちちがアグネスチャンのだいファンっていうだけなんですけどね。ほんとうのところ。
\\	でも、ただちちがアグネスチャンのだいファンっていうだけなんですけどね。ほんとうのところ。
\\	それはいい、わたしもだいファンだ。まあ、これもなにかのえんだ。とうきょうでこまったことがあったらいつでもれんらくしなさい。これはわたしのめいしです。
\\	それはいい、わたしもだいファンだ。まあ、これもなにかのえんだ。とうきょうでこまったことがあったらいつでもれんらくしなさい。これはわたしのめいしです。
\\	どうもありがとうございます。
\\	どうもありがとうございます。
\\	さあ、これからあたらしいせいかつがはじまる。ちょっとしんぱいだけどがんばらないと。あっ、そういえばあのひとどこかおおきなきぎょうのしゃちょうさんだって。さいさきいいかも。
\\	さあ、これからあたらしいせいかつがはじまる。ちょっとしんぱいだけどがんばらないと。あっ、そういえばあのひとどこかおおきなきぎょうのしゃちょうさんだって。さいさきいいかも。
\\	かぎ
\\	タバコ
\\	フロント
\\	モーニングコール
\\	かしこまる
\\	ほんじつは
\\	ひがしにほんをごりよういただきましてまことにありがとうございます。やまびこごじゅうにごうまもなくとうきょうえきにとうちゃくいたします。
\\	ほんじつは
\\	ひがしにほんをごりよういただきましてまことにありがとうございます。やまびこごじゅうにごうまもなくとうきょうえきにとうちゃくいたします。
\\	おじょうさん、おきて
\\	とうきょうにつきましたよ。
\\	おじょうさん、おきて
\\	とうきょうにつきましたよ。
\\	あっ、どうも。
\\	あっ、どうも。
\\	まあ、これからいろいろあるとおもうがじぶんのゆめにむけてがんばってください。
\\	まあ、これからいろいろあるとおもうがじぶんのゆめにむけてがんばってください。
\\	はい、どうもありがとうございました。
\\	はい、どうもありがとうございました。
\\	それではわたしはこれで。
\\	それではわたしはこれで。
\\	はい、しつれいします。
\\	はい、しつれいします。
\\	そうそう、それからなにかあったらいつでもれんらくしてきなさい。
\\	そうそう、それからなにかあったらいつでもれんらくしてきなさい。
\\	はい、わかりました。
\\	はい、わかりました。
\\	それでは。
\\	それでは。
\\	さあ、まずはホテルにいかないと。
\\	さあ、まずはホテルにいかないと。
\\	こんにちは。
\\	こんにちは。
\\	こんにちは。よやくしたむらかみです。
\\	こんにちは。よやくしたむらかみです。
\\	はい、しょうしょうおまちください。はい、むらかみアグネスさまですね。こちらがおへやのかぎになります。おにかいのみぎがわです。
\\	はい、しょうしょうおまちください。はい、むらかみアグネスさまですね。こちらがおへやのかぎになります。おにかいのみぎがわです。
\\	はい、ありがとうございます。
\\	はい、ありがとうございます。
\\	あれ、このへやきんえんじゃないんだ。わたしタバコすわないのに。フロントにいってかえてもらおうっと。(でんわ)すみません、きんえんのおへやにかえていただきたいんですが。
\\	あれ、このへやきんえんじゃないんだ。わたしタバコすわないのに。フロントにいってかえてもらおうっと。(でんわ)すみません、きんえんのおへやにかえていただきたいんですが。
\\	はい、もうしわけございません。いますぐかえさせていただきます。
\\	はい、もうしわけございません。いますぐかえさせていただきます。
\\	はい、おねがいします。
\\	はい、おねがいします。
\\	こちらのおへやでよろしいでしょうか?
\\	こちらのおへやでよろしいでしょうか?
\\	はい、だいじょうぶです。ありがとうございます。それから、あしたのひちじにモーニングコールをおねがいします。
\\	はい、だいじょうぶです。ありがとうございます。それから、あしたのひちじにモーニングコールをおねがいします。
\\	はい、かしこまりました。
\\	はい、かしこまりました。
\\	さあ、おふろにはいってさっぱりしよう。あしたはへやさがしにいかないといけないからちゃんとおきないと。いいへやがみつかればいいけど。ファイト!
\\	さあ、おふろにはいってさっぱりしよう。あしたはへやさがしにいかないといけないからちゃんとおきないと。いいへやがみつかればいいけど。ファイト!
\\	ふどうさん
\\	ふどうさんや
\\	たたみ
\\	ワンルーム
\\	よさん
\\	じょう
\\	とほ
\\	ひあたり
\\	できれば
\\	いらっしゃい。
\\	いらっしゃい。
\\	こんにちは。へやをさがしにきたんですが。
\\	こんにちは。へやをさがしにきたんですが。
\\	いや、おきゃくさんはついてる。なんといってもここはせかいでいちばんしんせつなふどうさんやですから。わたしにまかせておけばかならずいいへやみつかるよ。
\\	いや、おきゃくさんはついてる。なんといってもここはせかいでいちばんしんせつなふどうさんやですから。わたしにまかせておけばかならずいいへやみつかるよ。
\\	よろしくおねがいします。
\\	よろしくおねがいします。
\\	どういうところをおさがしですか?
\\	どういうところをおさがしですか?
\\	ワンルームでキッチンとおふろとトイレがついているところをさがしているんですが。
\\	ワンルームでキッチンとおふろとトイレがついているところをさがしているんですが。
\\	そうですねー。あ、このちかくによじょうはんのアパートがあるけど。
\\	そうですねー。あ、このちかくによじょうはんのアパートがあるけど。
\\	もっとおおきなおへやはないですか。ろくじょうぐらいの。
\\	もっとおおきなおへやはないですか。ろくじょうぐらいの。
\\	えー、ごよさんは?つきどれくらい?
\\	えー、ごよさんは?つきどれくらい?
\\	まあ、ご、ろくまんぐらいで。できればえきからちかいほうがいいんですけど。
\\	まあ、ご、ろくまんぐらいで。できればえきからちかいほうがいいんですけど。
\\	えきのちかくはすこしたかくなるけど。ああ、ろくまんごせんえんでえきからとほごふんってところがあるよ。
\\	えきのちかくはすこしたかくなるけど。ああ、ろくまんごせんえんでえきからとほごふんってところがあるよ。
\\	ああ、きれいなところですね。ほかにはありますか?
\\	ああ、きれいなところですね。ほかにはありますか?
\\	あとはつきごまんえんのところがあるよ。あたらしくはないがアパートのにかいでひあたりはいいね。えきからはすこしあるくけど。
\\	あとはつきごまんえんのところがあるよ。あたらしくはないがアパートのにかいでひあたりはいいね。えきからはすこしあるくけど。
\\	どのくらいですか?
\\	どのくらいですか?
\\	まあ、じゅうごふんぐらいかな。
\\	まあ、じゅうごふんぐらいかな。
\\	わるくはないですね。
\\	わるくはないですね。
\\	ところでおじょうさんはどちらのかた?
\\	ところでおじょうさんはどちらのかた?
\\	ふくしまです。
\\	ふくしまです。
\\	へー、いいところだね。
\\	へー、いいところだね。
\\	なにもありませんが。しぜんはきれいです。
\\	なにもありませんが。しぜんはきれいです。
\\	このふたつみてみますか?
\\	このふたつみてみますか?
\\	はい。ぜひ。
\\	はい。ぜひ。
\\	きょうはこれからほかのところをみせないといけないから、あしたはあいてますか?
\\	きょうはこれからほかのところをみせないといけないから、あしたはあいてますか?
\\	はい、じゃあ、あしたおねがいします。
\\	はい、じゃあ、あしたおねがいします。
\\	さく
\\	たまらない
\\	じかんをさく
\\	ゴキブリ
\\	ていれ
\\	うんでいのさ
\\	せんろ
\\	かぎり
\\	わりに
\\	てつづき
\\	こんにちは。
\\	こんにちは。
\\	おいそがしいなかじかんをさいていただきましてありがとうございます。
\\	おいそがしいなかじかんをさいていただきましてありがとうございます。
\\	いやいや、きにしないで。これがしごとだから。それにきょうはそんなにいそがしくないんだよ。
\\	いやいや、きにしないで。これがしごとだから。それにきょうはそんなにいそがしくないんだよ。
\\	そうですか。
\\	そうですか。
\\	じゃあ、ひとつめのところにいってみましょうか。
\\	じゃあ、ひとつめのところにいってみましょうか。
\\	はい、おねがいします。
\\	はい、おねがいします。
\\	ここが、6まん5せんえんのアパートです。
\\	ここが、6まん5せんえんのアパートです。
\\	え、しゃしんとすこしちがいませんか?あっゴキブリ。ここってたかいわりにはきたないかぎり。ちょっとここは
\\	え、しゃしんとすこしちがいませんか?あっゴキブリ。ここってたかいわりにはきたないかぎり。ちょっとここは
\\	まあ、えきからちかいからとてもべんりですけどね。
\\	まあ、えきからちかいからとてもべんりですけどね。
\\	ちかいのはいいですけどせんろのとなりだとすこしうるさくないですか?
\\	ちかいのはいいですけどせんろのとなりだとすこしうるさくないですか?
\\	まあ、すぐになれるとおもうけど。
\\	まあ、すぐになれるとおもうけど。
\\	ちょっと、しゃしんでみたのとうんでいのさがあるな。
\\	ちょっと、しゃしんでみたのとうんでいのさがあるな。
\\	まあ、こんなかんじかな。つぎのところもいってみますか?
\\	まあ、こんなかんじかな。つぎのところもいってみますか?
\\	そうですね。
\\	そうですね。
\\	さあ、こちらです。ふるいたてものだけど、ていれれされてるからきれいだよ。
\\	さあ、こちらです。ふるいたてものだけど、ていれれされてるからきれいだよ。
\\	へー、とてもかんじのいいところですね。
\\	へー、とてもかんじのいいところですね。
\\	ひあたりもいいですよ。さあ、どうぞ。
\\	ひあたりもいいですよ。さあ、どうぞ。
\\	うわー、あたらしいたたみ、このにおいがたまらない。
\\	うわー、あたらしいたたみ、このにおいがたまらない。
\\	このまえかえてもらったばかりですからね。まどのおそとをみてごらんなさい。
\\	このまえかえてもらったばかりですからね。まどのおそとをみてごらんなさい。
\\	はい、すごい、ひもはいるしきれいですね。
\\	はい、すごい、ひもはいるしきれいですね。
\\	まあ、えきからもそうとおくはないから。
\\	まあ、えきからもそうとおくはないから。
\\	つき5まんえんですよね。
\\	つき5まんえんですよね。
\\	そうです。
\\	そうです。
\\	わたし、ここにきめます。
\\	わたし、ここにきめます。
\\	ありがとうございます。ではおみせにもどっててつづきさせていただきますね。
\\	ありがとうございます。ではおみせにもどっててつづきさせていただきますね。
\\	まちにまった
\\	かいだし
\\	さんぽがてら
\\	そろえる
\\	はいたつ
\\	いこう
\\	いぜん
\\	せっかく
\\	いらっしゃいませ。なにかおさがしですか?
\\	いらっしゃいませ。なにかおさがしですか?
\\	はい、じつはひとりぐらしをはじめるのでいろいろかいだしに。
\\	はい、じつはひとりぐらしをはじめるのでいろいろかいだしに。
\\	あ、そうですか。いまちょうどふゆのだいセールをやっていまして、ベッドやたななどおやすくなっていますので、そちらもぜひごらんになっていってください。
\\	あ、そうですか。いまちょうどふゆのだいセールをやっていまして、ベッドやたななどおやすくなっていますので、そちらもぜひごらんになっていってください。
\\	はい、ありがとうございます。へー、ちょうどよかった。いろいろそろえられそう。うわー、このテーブルかわいい。なんだかひつよういじょうにかっちゃうかも。  
\\	はい、ありがとうございます。へー、ちょうどよかった。いろいろそろえられそう。うわー、このテーブルかわいい。なんだかひつよういじょうにかっちゃうかも。  
\\	このテーブルといすはおかいどくですよ。
\\	このテーブルといすはおかいどくですよ。
\\	あのー、はいたつはできますか?
\\	あのー、はいたつはできますか?
\\	はい、やっております。
\\	はい、やっております。
\\	じゃあ、はこぶしんぱいはないわね。
\\	じゃあ、はこぶしんぱいはないわね。
\\	ただ、きょうのぶんはあしたいこうになってしまいますが。
\\	ただ、きょうのぶんはあしたいこうになってしまいますが。
\\	はい、だいじょうぶです。せっかくのひとりぐらしだしいろいろそろえちゃおうかな。これに、あっ、あれも。
\\	はい、だいじょうぶです。せっかくのひとりぐらしだしいろいろそろえちゃおうかな。これに、あっ、あれも。
\\	えー、ごうけいがさんまんよんせんろっぴゃくごじゅうえんになります。
\\	えー、ごうけいがさんまんよんせんろっぴゃくごじゅうえんになります。
\\	はい。
\\	はい。
\\	どうも、ありがとうございました。それではあすのごぜんちゅうにおうちのほうへおうかがいします。
\\	どうも、ありがとうございました。それではあすのごぜんちゅうにおうちのほうへおうかがいします。
\\	こんなにかっちゃった。でも、たのしみー。かえってどこになにをおくかかんがえなくちゃ。でも、ぜんぶひとりでできるかな。まあ、やってみないと。
\\	こんなにかっちゃった。でも、たのしみー。かえってどこになにをおくかかんがえなくちゃ。でも、ぜんぶひとりでできるかな。まあ、やってみないと。
\\	ひとざとはなれた
\\	おてだま
\\	いしけり
\\	たいそう
\\	しんあい
\\	なかよく
\\	ぜんぶ
\\	くみたてる
\\	はこ
\\	かぐ
\\	せいり
\\	いろいろ
\\	かんせい
\\	こんにちは、
\\	です。
\\	こんにちは、
\\	です。
\\	はーい、こちらにおねがいします。
\\	はーい、こちらにおねがいします。
\\	はい、こちらでよろしいですか?
\\	はい、こちらでよろしいですか?
\\	はい、おねがいします。
\\	はい、おねがいします。
\\	それではどうもありがとうございました。
\\	それではどうもありがとうございました。
\\	はい、どうも、って、あれ、でもぜんぶはこにはいったままだ。えー、じぶんでくみたてないといけないの?どうしよう
\\	でも、やってみないと。よし、がんばるぞ。
\\	はい、どうも、って、あれ、でもぜんぶはこにはいったままだ。えー、じぶんでくみたてないといけないの?どうしよう
\\	でも、やってみないと。よし、がんばるぞ。
\\	うわー、おもったよりたいへん。
\\	うわー、おもったよりたいへん。
\\	けいたいがなる。
\\	けいたいがなる。
\\	もしもし、え、もしかしてみつこ?
\\	もしもし、え、もしかしてみつこ?
\\	うん、アグネス?ひさしぶり。
\\	うん、アグネス?ひさしぶり。
\\	えー、なんねんぶりかな。げんきにしてた?
\\	えー、なんねんぶりかな。げんきにしてた?
\\	うん、あいかわらずよ。アグネスはどうしてるの?
\\	うん、あいかわらずよ。アグネスはどうしてるの?
\\	じつはねいまとうきょうにでてきてるんだ。
\\	じつはねいまとうきょうにでてきてるんだ。
\\	うそー、なんでもっとはやくいってくれないのよ。
\\	うそー、なんでもっとはやくいってくれないのよ。
\\	だってまだでてきたばっかりよ。こっちでしごとをさがそうとおもって。
\\	だってまだでてきたばっかりよ。こっちでしごとをさがそうとおもって。
\\	そうなんだ。で、いまどこにすんでるの?
\\	そうなんだ。で、いまどこにすんでるの?
\\	なかののちかくなんだけど。
\\	なかののちかくなんだけど。
\\	へー、きょうはなにしてるの?
\\	へー、きょうはなにしてるの?
\\	さっきアイキアからかぐがとどいたから、それをくみたてたり、いろいろせいりしようとおもって。
\\	さっきアイキアからかぐがとどいたから、それをくみたてたり、いろいろせいりしようとおもって。
\\	あしたはひま?
\\	あしたはひま?
\\	うん。
\\	うん。
\\	またせる
\\	やせる
\\	おしゃれ
\\	ふんいき
\\	ずうずうしい
\\	しょくば
\\	どうりょう
\\	ごうこん
\\	みちこ、ごめーん。まった?
\\	みちこ、ごめーん。まった?
\\	ううん、そんなにまたされてないよ。アグネス、ひさしぶり
\\	ねえ、ちょっとやせた?
\\	ううん、そんなにまたされてないよ。アグネス、ひさしぶり
\\	ねえ、ちょっとやせた?
\\	ときどきそういわれるけど、かわってないよ。みちこはきれいになったね
\\	ときどきそういわれるけど、かわってないよ。みちこはきれいになったね
\\	またあ。
\\	またあ。
\\	このおみせ、すごくおしゃれだね。
\\	このおみせ、すごくおしゃれだね。
\\	いいふんいきでしょ?
\\	いいふんいきでしょ?
\\	あ、いたいた。みちこー。
\\	あ、いたいた。みちこー。
\\	このせきでいいの?
\\	このせきでいいの?
\\	どうぞどうぞ。
\\	どうぞどうぞ。
\\	え?
\\	え?
\\	おお、このひとがみちこがいってたひとか。うわあ、ほんとうにきれいなひとだなあ。あ、すみません、となりにすわってもいいですか?
\\	おお、このひとがみちこがいってたひとか。うわあ、ほんとうにきれいなひとだなあ。あ、すみません、となりにすわってもいいですか?
\\	おい、おまえちょっとずうずうしいよ。
\\	おい、おまえちょっとずうずうしいよ。
\\	ちょっとちょっと、みちこ。どうなってるの?
\\	ちょっとちょっと、みちこ。どうなってるの?
\\	ああ、ゴメンね。アグネスをしょうかいして、ってたのまれたの。こちらがまつだひろしくん。わたしのしょくばのどうりょう。
\\	ああ、ゴメンね。アグネスをしょうかいして、ってたのまれたの。こちらがまつだひろしくん。わたしのしょくばのどうりょう。
\\	どうもー、こんにちはー。で、こっちがおれのちゅうがっこうじだいのともだちで、はしもとゆうすけっていいます。こうみえてもいしゃなんですよ。
\\	どうもー、こんにちはー。で、こっちがおれのちゅうがっこうじだいのともだちで、はしもとゆうすけっていいます。こうみえてもいしゃなんですよ。
\\	どうも、はじめまして。きょうはひろしにつれてこられちゃって…
\\	どうも、はじめまして。きょうはひろしにつれてこられちゃって…
\\	まあ、おいしゃさまなの
\\	かっこいい
\\	まあ、おいしゃさまなの
\\	かっこいい
\\	いやいや、たいしたことないですよ。
\\	いやいや、たいしたことないですよ。
\\	むらかみアグネスさんっていうんでしょ?みちこからはなしをきかされて、いちどあいたいっておもってたんだ。
\\	むらかみアグネスさんっていうんでしょ?みちこからはなしをきかされて、いちどあいたいっておもってたんだ。
\\	なにこれ、ごうコン?ひょっとしてわたし、りようされた?
\\	なにこれ、ごうコン?ひょっとしてわたし、りようされた?
\\	やまおく
\\	きょうそう
\\	にんきもの
\\	けらい
\\	すばらしい
\\	ぶし
\\	こうこうじだい
\\	さびしい
\\	れんらく
\\	きそくただしい
\\	かわる
\\	(でんわがなる)
\\	(でんわがなる)
\\	はい、もしもし。
\\	はい、もしもし。
\\	アグネス?
\\	アグネス?
\\	おかあさん
\\	あさからどうしたの?
\\	おかあさん
\\	あさからどうしたの?
\\	まあいやだ、このこったら。あさどころかもうおひるすぎよ。いままでねていたの?
\\	まあいやだ、このこったら。あさどころかもうおひるすぎよ。いままでねていたの?
\\	ええ?あ、ほんとうだ。うーん、きのうちょっとおそくまでともだちとのんでたから。
\\	ええ?あ、ほんとうだ。うーん、きのうちょっとおそくまでともだちとのんでたから。
\\	あら、もうそっちでおともだちができたの?
\\	あら、もうそっちでおともだちができたの?
\\	ううん、こうこうじだいのともだちのたなかみつこ。いまとうきょうにいて、れんらくくれたんだ。
\\	ううん、こうこうじだいのともだちのたなかみつこ。いまとうきょうにいて、れんらくくれたんだ。
\\	まあ、そうなの。どうしてるかとおもったけど、げんきみたいね。ちょっとおとうさんとかわるわ。
\\	まあ、そうなの。どうしてるかとおもったけど、げんきみたいね。ちょっとおとうさんとかわるわ。
\\	もしもし、アグネスか?げんきでやってるか?ひとりでさびしくないか?
\\	もしもし、アグネスか?げんきでやってるか?ひとりでさびしくないか?
\\	やだ、おとうさん。わたし、こどもじゃないんだから。さびしいどころか、まいにちいそがしくて、おとうさんとおかあさんにでんわするのもわすれちゃってたわ。
\\	やだ、おとうさん。わたし、こどもじゃないんだから。さびしいどころか、まいにちいそがしくて、おとうさんとおかあさんにでんわするのもわすれちゃってたわ。
\\	そうか、それはよかった…。アグネスはさびしくないんだな…。
\\	そうか、それはよかった…。アグネスはさびしくないんだな…。
\\	ん、どうしたの、おとうさん?
\\	ん、どうしたの、おとうさん?
\\	いやいやなんでもない。またかあさんにかわるよ。
\\	いやいやなんでもない。またかあさんにかわるよ。
\\	アグネス、あそぶのもいいけどちゃんときそくただしいせいかつをして、はやくしごとをみつけなさい。
\\	アグネス、あそぶのもいいけどちゃんときそくただしいせいかつをして、はやくしごとをみつけなさい。
\\	はあい。ねえおかあさん、おとうさんはどうしたの?ちょっとげんきがなかったけど。
\\	はあい。ねえおかあさん、おとうさんはどうしたの?ちょっとげんきがなかったけど。
\\	げんきがないどころか。おとうさんったらまいにち、こどものごろのアグネスのしゃしんをみながらないてるのよ。
\\	げんきがないどころか。おとうさんったらまいにち、こどものごろのアグネスのしゃしんをみながらないてるのよ。
\\	なにをいうんだかあさん
\\	アグネス、おとうさんはげんきだぞ。ないてなんかいないぞ
\\	なにをいうんだかあさん
\\	アグネス、おとうさんはげんきだぞ。ないてなんかいないぞ
\\	おとうさん、あんなつよがりいってるけど、アグネスがいなくてさびしいのよ。
\\	おとうさん、あんなつよがりいってるけど、アグネスがいなくてさびしいのよ。
\\	あーあ、なんだかしんぱいになってきちゃった。おとうさん、だいじょうぶかしら。
\\	あーあ、なんだかしんぱいになってきちゃった。おとうさん、だいじょうぶかしら。
\\	かみさま
\\	がまん
\\	おてあらい
\\	いっしょうけんめい
\\	ころがる
\\	ほうほう
\\	りれきしょ
\\	しょく
\\	けん
\\	しょうち
\\	たすかる
\\	めんせつ
\\	さっそく
\\	うかがう
\\	ぜひ
\\	(でんわがなる)
\\	(でんわがなる)
\\	はい、もしもし。
\\	はい、もしもし。
\\	むらかみアグネスさんですか?
\\	むらかみアグネスさんですか?
\\	はい、そうです。
\\	はい、そうです。
\\	デザインオフィスのすぎたともうします。せんじつりれきしょをおおくりいただいたけんでおでんわしました。
\\	デザインオフィスのすぎたともうします。せんじつりれきしょをおおくりいただいたけんでおでんわしました。
\\	あ、はい
\\	おでんわありがとうございます。
\\	あ、はい
\\	おでんわありがとうございます。
\\	グラフィックデザイナーのしょくをごきぼうということで、まちがいないですね。
\\	グラフィックデザイナーのしょくをごきぼうということで、まちがいないですね。
\\	はい。よろしくおねがいします。
\\	はい。よろしくおねがいします。
\\	さっそくですが、ちかいうちにこちらにきていただいて、ちょくせつおはなしをおききしたいのですが。
\\	さっそくですが、ちかいうちにこちらにきていただいて、ちょくせつおはなしをおききしたいのですが。
\\	わかりました
\\	きょうすぐにでもうかがいます。
\\	わかりました
\\	きょうすぐにでもうかがいます。
\\	いえ、きょうはちょっと…。すいようびのゆうがた4じではいかがでしょうか?
\\	いえ、きょうはちょっと…。すいようびのゆうがた4じではいかがでしょうか?
\\	あ、もうしわけありません、そのひはよていがはいっておりまして…。ごごのはやいうちならあいているんですが…。
\\	あ、もうしわけありません、そのひはよていがはいっておりまして…。ごごのはやいうちならあいているんですが…。
\\	それではもくようびのごぜんちゅうはいかがですか?
\\	それではもくようびのごぜんちゅうはいかがですか?
\\	はい、それならうかがえます。
\\	はい、それならうかがえます。
\\	では、もくようびの11じに。
\\	では、もくようびの11じに。
\\	しょうちいたしました。あ、すみません、かいしゃはあおやまでしたよね?
\\	しょうちいたしました。あ、すみません、かいしゃはあおやまでしたよね?
\\	そうです。おもてさんどうのえきから5ふんくらいです。ちずをファックスしましょうか?
\\	そうです。おもてさんどうのえきから5ふんくらいです。ちずをファックスしましょうか?
\\	ぜひおねがいします。まだとうきょうになれていないので、ちずがあるとたすかります。
\\	ぜひおねがいします。まだとうきょうになれていないので、ちずがあるとたすかります。
\\	わかりました。りれきしょにあるこちらのばんごうでいいんですね。では、のちほどファックスをおおくりします.
\\	わかりました。りれきしょにあるこちらのばんごうでいいんですね。では、のちほどファックスをおおくりします.
\\	ごしんせつにありがとうございます。どうぞよろしくおねがいいたします。
\\	ごしんせつにありがとうございます。どうぞよろしくおねがいいたします。
\\	(でんわをきる)
\\	(でんわをきる)
\\	うっかり
\\	とおりかかる
\\	おもいっきり
\\	めがける
\\	かすむ
\\	うっとり
\\	すがた
\\	ながめる
\\	ごきげん
\\	てっぺん
\\	ひがくれる
\\	こりる
\\	むかしむかし、はこね(はこね)のやまに、うっかりてんからおっこちてきた、あまのじゃくというものがすんでいました。あまのじゃくはたいへんなちからもちでしたが、ふしぎなことに、ちからがでるのひがくれてからのよるだけでした。あるはれたひ、あまのじゃくははこねのやまのてっぺんにたって、まわりをグルリとみわたしていました。
\\	むかしむかし、はこね(はこね)のやまに、うっかりてんからおっこちてきた、あまのじゃくというものがすんでいました。あまのじゃくはたいへんなちからもちでしたが、ふしぎなことに、ちからがでるのひがくれてからのよるだけでした。あるはれたひ、あまのじゃくははこねのやまのてっぺんにたって、まわりをグルリとみわたしていました。
\\	おう、きょうはとおくのやままでよくみえるのう。だが、わしのはこねやまがいちばんいいやまじゃ。
\\	おう、きょうはとおくのやままでよくみえるのう。だが、わしのはこねやまがいちばんいいやまじゃ。
\\	あまのじゃくはごきげんでしたが、ふとにしのほうをながめると、さっとかおいろがかわりました。
\\	あまのじゃくはごきげんでしたが、ふとにしのほうをながめると、さっとかおいろがかわりました。
\\	はこねやまのにしのくものあいだからは、にっぽんいちのふじさんが、そのうつくしいすがたをのぞかせていたのです。
\\	はこねやまのにしのくものあいだからは、にっぽんいちのふじさんが、そのうつくしいすがたをのぞかせていたのです。
\\	ううむ、ふじはやはりきれいなやまじゃのう。せたけもたかくて、ひとびとがあさにゆうにてをあわせるきもちもわかるわい。 
\\	ううむ、ふじはやはりきれいなやまじゃのう。せたけもたかくて、ひとびとがあさにゆうにてをあわせるきもちもわかるわい。 
\\	あまのじゃくは、ウットリとふじさんをながめていましたが、やがてくやしそうにいいました。
\\	あまのじゃくは、ウットリとふじさんをながめていましたが、やがてくやしそうにいいました。
\\	だめだ、だめだ
\\	ふじがいるおかげで、わしのはこねやまのうつくしさがかすんでしまう。にんげんどもははこねにしり(しり)をむけてふじばかりみておる。なんとかしなくては・・・
\\	だめだ、だめだ
\\	ふじがいるおかげで、わしのはこねやまのうつくしさがかすんでしまう。にんげんどもははこねにしり(しり)をむけてふじばかりみておる。なんとかしなくては・・・
\\	あまのじゃくはしばらくうでをくんでかんがえていましたが、やがていいことをおもいつきました。
\\	あまのじゃくはしばらくうでをくんでかんがえていましたが、やがていいことをおもいつきました。
\\	それはなんと、ふじさんのてっぺんのいわをうみへなげすててしまい、そのせたけをひくくしてやろうというのです。そのよる、ひとびとがねしずまってから、あまのじゃくはもっこ
\\	つちをはこぶどうぐ)をかついで、エッチラオッチラとふじさんにのぼりました。そして、てっぺんのいわをつかむともっこにいれて、ふじさんをくだり、かいがんからうみにめがけてなげこみました。あまのじゃくは、それからもまいばんふじさんにでかけては、てっぺんのいわをうみになげこみました。あまのじゃくがあんまりたくさんのいわをなげこんだので、うみにはいくつものしまができました。それが、「おおしま(おおしま)」「としま(としま)」「にいじま(にいじま)」「しきねじま(しきねじま)」「こうづしま(こうづしま)」「みやけじま(みやけじま)」「みくらじま(みくらじま)」の「いずしちしま(いずしちとう)」なのです。
\\	それはなんと、ふじさんのてっぺんのいわをうみへなげすててしまい、そのせたけをひくくしてやろうというのです。そのよる、ひとびとがねしずまってから、あまのじゃくはもっこ
\\	つちをはこぶどうぐ)をかついで、エッチラオッチラとふじさんにのぼりました。そして、てっぺんのいわをつかむともっこにいれて、ふじさんをくだり、かいがんからうみにめがけてなげこみました。あまのじゃくは、それからもまいばんふじさんにでかけては、てっぺんのいわをうみになげこみました。あまのじゃくがあんまりたくさんのいわをなげこんだので、うみにはいくつものしまができました。それが、「おおしま(おおしま)」「としま(としま)」「にいじま(にいじま)」「しきねじま(しきねじま)」「こうづしま(こうづしま)」「みやけじま(みやけじま)」「みくらじま(みくらじま)」の「いずしちしま(いずしちとう)」なのです。
\\	そして、なげそこなってちかくにおちたのが、「はつしま(はつしま)」になったということです。
\\	そして、なげそこなってちかくにおちたのが、「はつしま(はつしま)」になったということです。
\\	しかし、これだけのいわをとられても、ふじさんのせたけは、まだまだにっぽんいちです。
\\	しかし、これだけのいわをとられても、ふじさんのせたけは、まだまだにっぽんいちです。
\\	しやくしょ
\\	はいけん
\\	ねんきんてちょう
\\	ていじ
\\	てんしゅつしょうめいしょ
\\	ばんごうふだ
\\	こんざつ
\\	きにゅう
\\	しょるい
\\	まどぐち
\\	てんにゅうとどけ
\\	やくしょ
\\	いんかんしょうめい
\\	えーと、しやくしょはどこかしら。あ、ここね。
\\	えーと、しやくしょはどこかしら。あ、ここね。
\\	すみません、てんにゅうとどけのうけつけはどちらですか?
\\	すみません、てんにゅうとどけのうけつけはどちらですか?
\\	はい、3ばんのまどぐちになります。まず、あちらにあるしょるいにきにゅうしてからまどぐちでおもうしこみください。
\\	はい、3ばんのまどぐちになります。まず、あちらにあるしょるいにきにゅうしてからまどぐちでおもうしこみください。
\\	はい、わかりました。
\\	はい、わかりました。
\\	ただいまうけつけがしょうしょうこんざつしておりますので、あちらのばんごうふだをとってください。じゅんばんにおよびします。
\\	ただいまうけつけがしょうしょうこんざつしておりますので、あちらのばんごうふだをとってください。じゅんばんにおよびします。
\\	わかりました。ありがとうございます。
\\	わかりました。ありがとうございます。
\\	ばんごうふだ105ばんでおまちのかた、まどぐちまでどうぞ。
\\	ばんごうふだ105ばんでおまちのかた、まどぐちまでどうぞ。
\\	あ、わたしだわ
\\	あ、わたしだわ
\\	えーと、てんにゅうとどけをだしたいんですけど。
\\	えーと、てんにゅうとどけをだしたいんですけど。
\\	はい、てんしゅつしょうめいしょをいただけますか?
\\	はい、てんしゅつしょうめいしょをいただけますか?
\\	こちらです。
\\	こちらです。
\\	ほんじつ、うんてんめんきょしょうやパスポートなどのみぶんしょうめいしょはおもちですか?
\\	ほんじつ、うんてんめんきょしょうやパスポートなどのみぶんしょうめいしょはおもちですか?
\\	え…わたし、めんきょはもっていないんですけど…。
\\	え…わたし、めんきょはもっていないんですけど…。
\\	もうしわけございませんが、てんにゅうとどけのうけつけにあたっては、みぶんしょうめいしょをごていじいただくひつようがございまして。
\\	もうしわけございませんが、てんにゅうとどけのうけつけにあたっては、みぶんしょうめいしょをごていじいただくひつようがございまして。
\\	あのう、まえのしやくしょではねんきんてちょうをもっていったらだいじょうぶだったんですけど。だからきょうはねんきんてちょうをもってきました。
\\	あのう、まえのしやくしょではねんきんてちょうをもっていったらだいじょうぶだったんですけど。だからきょうはねんきんてちょうをもってきました。
\\	そうですか。ちょっとはいけんしてよろしいですか?しょうしょうおまちください。
\\	そうですか。ちょっとはいけんしてよろしいですか?しょうしょうおまちください。
\\	おまたせしました。こちらでだいじょうぶです。おひっこしにあたって、いんかんしょうめいのてつづきなどもなさいますか?
\\	おまたせしました。こちらでだいじょうぶです。おひっこしにあたって、いんかんしょうめいのてつづきなどもなさいますか?
\\	いいえ、それはこんかいひつようありません。
\\	いいえ、それはこんかいひつようありません。
\\	では、これでてつづきはしゅうりょうです。
\\	では、これでてつづきはしゅうりょうです。
\\	ありがとうございました。
\\	ありがとうございました。
\\	うみべ
\\	ちのけがひく
\\	こころあたり
\\	あやうい
\\	てごたえ
\\	おしあげる
\\	つよまる
\\	そらもよう
\\	はた
\\	すっかり
\\	むかしむかし、あるうみべに、おじいさんとおばあさんとむすことよめとがくらしていました。おじいさんとむすこはおきにでてさかなをとり、おばあさんとよめははたをおるまいにちでした。あるひのこと、おじいさんとむすこがおきへりょうにでていると、きゅうにそらもようがあやしくなってきました。
\\	むかしむかし、あるうみべに、おじいさんとおばあさんとむすことよめとがくらしていました。おじいさんとむすこはおきにでてさかなをとり、おばあさんとよめははたをおるまいにちでした。あるひのこと、おじいさんとむすこがおきへりょうにでていると、きゅうにそらもようがあやしくなってきました。
\\	こりゃ、おおあめになるぞ。
\\	こりゃ、おおあめになるぞ。
\\	おとっつぁん、あのしまへいこう。
\\	おとっつぁん、あのしまへいこう。
\\	ふたりはおおいそぎで、ちかくのしまへにげました。
\\	ふたりはおおいそぎで、ちかくのしまへにげました。
\\	だんだんあめかぜがつよまるなか、やっとふねをおかにおしあげて、ほらあなにこもり、おおあれにあれるうみをみながら、ふたりはジッとよるをあかしました。つぎのあさ、おおあめがやんだので、ふたりはふねをだしてさかなをとりはじめました。アミをうみにいれると、とてもおもいてごたえがあります。ふたりがなんとかアミをひきあげてみると、アミのなかにはにじゅうご、ろくさいのりっぱなきものをきたおとこがかかっていたのです。
\\	だんだんあめかぜがつよまるなか、やっとふねをおかにおしあげて、ほらあなにこもり、おおあれにあれるうみをみながら、ふたりはジッとよるをあかしました。つぎのあさ、おおあめがやんだので、ふたりはふねをだしてさかなをとりはじめました。アミをうみにいれると、とてもおもいてごたえがあります。ふたりがなんとかアミをひきあげてみると、アミのなかにはにじゅうご、ろくさいのりっぱなきものをきたおとこがかかっていたのです。
\\	おとっつぁん、こりゃあ。
\\	おとっつぁん、こりゃあ。
\\	うむ、ゆうべのおおあめにながされてきたおひとじゃろう。かわいそうなことだが、もうしんでいる。
\\	うむ、ゆうべのおおあめにながされてきたおひとじゃろう。かわいそうなことだが、もうしんでいる。
\\	ふたりはしまにあなをほると、そのおとこをていねいにうめてやりました。
\\	ふたりはしまにあなをほると、そのおとこをていねいにうめてやりました。
\\	きょうは、ひきあげよう。おばあさんにたのまれていたものをかってからかえろう。
\\	きょうは、ひきあげよう。おばあさんにたのまれていたものをかってからかえろう。
\\	ふたりはおおきなまちがあるみなとへ、ふねをこぎよせました。
\\	ふたりはおおきなまちがあるみなとへ、ふねをこぎよせました。
\\	おみそやおこめをかおうと、おこめやへいったら、そこのだんながこえをかけてきました。
\\	おみそやおこめをかおうと、おこめやへいったら、そこのだんながこえをかけてきました。
\\	もし、あなたたちは、さくやのおおあめのとき、どうしていましたか?
\\	もし、あなたたちは、さくやのおおあめのとき、どうしていましたか?
\\	はい、わしたちはあやういところでしまにのがれられました。
\\	はい、わしたちはあやういところでしまにのがれられました。
\\	そうでしたか、それはよろしゅうございました。ところでここへくるとちゅう、せんごくぶね(せんごくぶね・えどじだい、こめをせんごくほどつめるおおがたのわせん)をみかけませんでしたか?
\\	そうでしたか、それはよろしゅうございました。ところでここへくるとちゅう、せんごくぶね(せんごくぶね・えどじだい、こめをせんごくほどつめるおおがたのわせん)をみかけませんでしたか?
\\	いいや、みなかったですな。ですがきょう、わしらのアミにわかいおとこのしがいがかかって、しまにうめてきました。
\\	いいや、みなかったですな。ですがきょう、わしらのアミにわかいおとこのしがいがかかって、しまにうめてきました。
\\	しがいですと
\\	しがいですと
\\	なにか、こころあたりでもありなさるのか?
\\	なにか、こころあたりでもありなさるのか?
\\	じつは、むすこがおおさかにせんごくぶねでこめをつんででていったのですが、そこへあのおおあめ。しんぱいしているところです。
\\	じつは、むすこがおおさかにせんごくぶねでこめをつんででていったのですが、そこへあのおおあめ。しんぱいしているところです。
\\	そうじゃったか。
\\	そうじゃったか。
\\	ごめんどうをおかけしますが、わたしをそのしまへつれていってもらえますまいか?
\\	ごめんどうをおかけしますが、わたしをそのしまへつれていってもらえますまいか?
\\	ふたりはだんなをのせて、そのしまへもどりました。うめたしがいをほりかえしてみると、だんなのかおからちのけがびきました。
\\	ふたりはだんなをのせて、そのしまへもどりました。うめたしがいをほりかえしてみると、だんなのかおからちのけがびきました。
\\	むっ、むすこです。
\\	むっ、むすこです。
\\	ふたりはしがいをのせてふたたびみなとへひきかえし、りっぱなそうしきにもたちあいました。
\\	ふたりはしがいをのせてふたたびみなとへひきかえし、りっぱなそうしきにもたちあいました。
\\	あなたたちには、すっかりおせわになりました。わたしのこころからのおれいをみなとによういしました。どうかうけとってください。
\\	あなたたちには、すっかりおせわになりました。わたしのこころからのおれいをみなとによういしました。どうかうけとってください。
\\	いや、おれいなんぞいりません。
\\	いや、おれいなんぞいりません。
\\	じんじたんとう
\\	こせいてき
\\	ごじつ
\\	かんげい
\\	ねっしん
\\	じみ
\\	ほうそうし
\\	おんがくかんけい
\\	こじんてき
\\	おうぼ
\\	あいしょう
\\	ごめんください。ほんじつ4じからめんせつをしていただくよていになっている、むらかみアグネスともうします。
\\	ごめんください。ほんじつ4じからめんせつをしていただくよていになっている、むらかみアグネスともうします。
\\	はい、うかがっています。しょうしょうおまちください。
\\	はい、うかがっています。しょうしょうおまちください。
\\	うわあ、いよいよめんせつか。きんちょうするなあ。
\\	うわあ、いよいよめんせつか。きんちょうするなあ。
\\	それでは、こちらにどうぞ。
\\	それでは、こちらにどうぞ。
\\	(ドアをノックする)
\\	(ドアをノックする)
\\	どうぞおはいりください。
\\	どうぞおはいりください。
\\	しつれいします。
\\	しつれいします。
\\	はじめまして、じんじたんとうのすぎたともうします。
\\	はじめまして、じんじたんとうのすぎたともうします。
\\	せんじつはおでんわをありがとうございました。むらかみアグネスともうします。
\\	せんじつはおでんわをありがとうございました。むらかみアグネスともうします。
\\	どうぞおかけください。
\\	どうぞおかけください。
\\	では、しつれいします。
\\	では、しつれいします。
\\	まず、うちのかいしゃにおうぼされたきっかけは?
\\	まず、うちのかいしゃにおうぼされたきっかけは?
\\	こじんてきなりゆうなのですが、がくせいのころからおんがくがすきなんです。それで、じぶんのすきなアーティストの
\\	ジャケットのデザインを、こちらのデザインオフィス
\\	さんがたんとうされているのをしって、きょうみをもちました。
\\	こじんてきなりゆうなのですが、がくせいのころからおんがくがすきなんです。それで、じぶんのすきなアーティストの
\\	ジャケットのデザインを、こちらのデザインオフィス
\\	さんがたんとうされているのをしって、きょうみをもちました。
\\	ほう、わがしゃのしごとをごぞんじでしたか。では、やはりそういったおんがくかんけいのしごとをごきぼうですか?
\\	ほう、わがしゃのしごとをごぞんじでしたか。では、やはりそういったおんがくかんけいのしごとをごきぼうですか?
\\	ええ、やってみたいというきもちはあります。
\\	ええ、やってみたいというきもちはあります。
\\	ただ、うちのしごとはおんがくかんけいがとくにおおいというわけではないんです。たとえばほうそうしのデザインなど、じみなしごともおおいですよ。
\\	ただ、うちのしごとはおんがくかんけいがとくにおおいというわけではないんです。たとえばほうそうしのデザインなど、じみなしごともおおいですよ。
\\	どんなにじみなしごとでもかまいません。デザインというしごとをいっしょうけんめいべんきょうしたいとおもっています。
\\	どんなにじみなしごとでもかまいません。デザインというしごとをいっしょうけんめいべんきょうしたいとおもっています。
\\	なかなか、ねっしんですね。
\\	なかなか、ねっしんですね。
\\	おそれいります。けいけんがないので、きもちだけはまえむきに、とおもっています。
\\	おそれいります。けいけんがないので、きもちだけはまえむきに、とおもっています。
\\	いいですね、しごとにまえむきなひとはかんげいですよ。 それではごじつ、あらためてうちのしゃちょうとおあいになってください。ちゅうしょくでもごいっしょしながら、いろいろおはなしをうかがいたいので。
\\	いいですね、しごとにまえむきなひとはかんげいですよ。 それではごじつ、あらためてうちのしゃちょうとおあいになってください。ちゅうしょくでもごいっしょしながら、いろいろおはなしをうかがいたいので。
\\	はい、ありがとうございます
\\	はい、ありがとうございます
\\	じつは、うちのしゃちょうはかなりこせいてきなじんぶつなので…しゃちょうとのあいしょうがけっこうたいせつなのですよ。
\\	じつは、うちのしゃちょうはかなりこせいてきなじんぶつなので…しゃちょうとのあいしょうがけっこうたいせつなのですよ。
\\	ごいっしょする
\\	かくしあじ
\\	じしんさく
\\	かんじん
\\	みどころ
\\	げいじゅつか
\\	せつび
\\	インスピレーション
\\	ちょうどよかった
\\	しゃちょうしつ
\\	さいよう
\\	しつれいします。ほんじつ、しゃちょうにめんせつをしていただくよていのむらかみアグネスです。おひるをごいっしょするとうかがっているんですが。
\\	しつれいします。ほんじつ、しゃちょうにめんせつをしていただくよていのむらかみアグネスです。おひるをごいっしょするとうかがっているんですが。
\\	はい、おくのしゃちょうしつへどうぞ。
\\	はい、おくのしゃちょうしつへどうぞ。
\\	(ドアをノックする)
\\	(ドアをノックする)
\\	しつれいします。
\\	しつれいします。
\\	あ、きみ
\\	ちょうどよかった
\\	そこのしおをとってくれ
\\	あ、きみ
\\	ちょうどよかった
\\	そこのしおをとってくれ
\\	は?こ、これですか?
\\	は?こ、これですか?
\\	そうそう
\\	よし、これでばっちりだ
\\	ああ、きみがむらかみアグネスくん?
\\	そうそう
\\	よし、これでばっちりだ
\\	ああ、きみがむらかみアグネスくん?
\\	はい…あの…。
\\	はい…あの…。
\\	しゃちょうのふじもとだ。よろしく。まあ、すわってすわって。
\\	しゃちょうのふじもとだ。よろしく。まあ、すわってすわって。
\\	このテーブルのところでいいんでしょうか?
\\	このテーブルのところでいいんでしょうか?
\\	そうそう、そこ。じつはぼくはりょうりがしゅみなんだ
\\	りょうりをしているとデザインのインスピレーションもわくんだよ。きょうはきみにぼくのてりょうりをたべてもらおうとおもってね。
\\	そうそう、そこ。じつはぼくはりょうりがしゅみなんだ
\\	りょうりをしているとデザインのインスピレーションもわくんだよ。きょうはきみにぼくのてりょうりをたべてもらおうとおもってね。
\\	だから、しゃちょうしつにキッチンがあるんですか?
\\	だから、しゃちょうしつにキッチンがあるんですか?
\\	そうそう。なかなかいいせつびだろう?みなみフランスのレストランっぽいデザインにしてみたんだ。
\\	そうそう。なかなかいいせつびだろう?みなみフランスのレストランっぽいデザインにしてみたんだ。
\\	え、ええ。このデザインなんか、すてきですね。
\\	え、ええ。このデザインなんか、すてきですね。
\\	おお、これにちゅうもくするとは、なかなかげいじゅつかっぽいセンスがあるね
\\	デザイナーとしてみどころがあるぞ
\\	おお、これにちゅうもくするとは、なかなかげいじゅつかっぽいセンスがあるね
\\	デザイナーとしてみどころがあるぞ
\\	ほんとうですか?
\\	ほんとうですか?
\\	うん、でもかんじんなのはりょうりをあじわうセンスだな。まあ、これをたべてみてくれ。さいきんのじしんさくなんだ。
\\	うん、でもかんじんなのはりょうりをあじわうセンスだな。まあ、これをたべてみてくれ。さいきんのじしんさくなんだ。
\\	はい、いただきます
\\	はい、いただきます
\\	どうだい?ソースのかくしあじがわかるかな?
\\	どうだい?ソースのかくしあじがわかるかな?
\\	とてもおいしいです。えーと…このきいろっぽいの、なんでしょう…。ショウガかしら?
\\	とてもおいしいです。えーと…このきいろっぽいの、なんでしょう…。ショウガかしら?
\\	とつぜんの
\\	むりやり
\\	おりいって
\\	ほんかくてき
\\	みならい
\\	ついてる
\\	おやくにたつ
\\	(でんわがなる)
\\	(でんわがなる)
\\	はい、もしもし。
\\	はい、もしもし。
\\	こんにちは、むらかみアグネスさんですか?
\\	こんにちは、むらかみアグネスさんですか?
\\	そうですけど…。
\\	そうですけど…。
\\	とつぜんのでんわで、しつれいします。ぼく、はしもとゆうすけです。せんじつ、たなかみつこさんといっしょにおあいした、まつだのゆうじんです。
\\	とつぜんのでんわで、しつれいします。ぼく、はしもとゆうすけです。せんじつ、たなかみつこさんといっしょにおあいした、まつだのゆうじんです。
\\	え~と…ああ
\\	あのおいしゃさまの?
\\	え~と…ああ
\\	あのおいしゃさまの?
\\	そうです。まつだからむりやりでんわばんごうをききだしちゃいました。
\\	そうです。まつだからむりやりでんわばんごうをききだしちゃいました。
\\	そうなんですか。びっくりした~。
\\	そうなんですか。びっくりした~。
\\	おどろかせてごめんなさい。じつは、アグネスさんがデザイナーだってきいて、おりいっておねがいがあるんですけど…。
\\	おどろかせてごめんなさい。じつは、アグネスさんがデザイナーだってきいて、おりいっておねがいがあるんですけど…。
\\	いえ、わたし、ほんかくてきなデザイナーじゃなくて、まだみならいですよ
\\	いえ、わたし、ほんかくてきなデザイナーじゃなくて、まだみならいですよ
\\	いやいや、そういうセンスのあるひとのいけんをききたいんですよ。
\\	いやいや、そういうセンスのあるひとのいけんをききたいんですよ。
\\	わたしでおやくにたてるかしら。どんなことですか?
\\	わたしでおやくにたてるかしら。どんなことですか?
\\	うーん、でんわではちょっとせつめいしづらいなあ。もしさしつかえなければ、ちかいうちにいっしょにしょくじでもいかがですか?そのときにゆっくりごせつめいします。いそぐはなしじゃないんです。
\\	うーん、でんわではちょっとせつめいしづらいなあ。もしさしつかえなければ、ちかいうちにいっしょにしょくじでもいかがですか?そのときにゆっくりごせつめいします。いそぐはなしじゃないんです。
\\	ええ、わたしはかまいませんよ。
\\	ええ、わたしはかまいませんよ。
\\	そうですか、よかった。いつがいいですか?
\\	そうですか、よかった。いつがいいですか?
\\	らいしゅうからあたらしいしごとがはじまるので、さしつかえなければこんしゅうちゅうがいいんですが。
\\	らいしゅうからあたらしいしごとがはじまるので、さしつかえなければこんしゅうちゅうがいいんですが。
\\	わかりました。では、あさってのよるはいかがですか?
\\	わかりました。では、あさってのよるはいかがですか?
\\	ええ、だいじょうぶです。
\\	ええ、だいじょうぶです。
\\	では、じかんとばしょはまたれんらくします。ありがとう、たのしみにしてます
\\	では、じかんとばしょはまたれんらくします。ありがとう、たのしみにしてます
\\	(でんわきれる)
\\	(でんわきれる)
\\	にんきしょうひん
\\	カーディガン
\\	たまには
\\	てにとる
\\	ワンピース
\\	ウィンドー
\\	スリム
\\	イメージ
\\	いろちがい
\\	しちゃく
\\	おちついた
\\	いらっしゃいませ。なにかおさがしですか?
\\	いらっしゃいませ。なにかおさがしですか?
\\	ウィンドウにあったワンピースがすてきだな、とおもって。みせていただけますか?
\\	ウィンドウにあったワンピースがすてきだな、とおもって。みせていただけますか?
\\	こちらですね。どうぞ、おてにとってごらんください。これはこのはるのにんきしょうひんなんですよ
\\	はるらしいかんじで、おすすめです
\\	こちらですね。どうぞ、おてにとってごらんください。これはこのはるのにんきしょうひんなんですよ
\\	はるらしいかんじで、おすすめです
\\	でも、このピンクはわたしにはかわいらしすぎるかなあ。いろちがいはありますか?
\\	でも、このピンクはわたしにはかわいらしすぎるかなあ。いろちがいはありますか?
\\	もうしわけございません、こちらのしょうひんはこのいろだけなんです。でも、きっとおにあいだとおもいますよ。よろしかったら、ごしちゃくなさいますか?
\\	もうしわけございません、こちらのしょうひんはこのいろだけなんです。でも、きっとおにあいだとおもいますよ。よろしかったら、ごしちゃくなさいますか?
\\	そうですね…。
\\	そうですね…。
\\	どうぞ、こちらのしちゃくしつがあいてます。
\\	どうぞ、こちらのしちゃくしつがあいてます。
\\	(ノックのおと)
\\	(ノックのおと)
\\	おきゃくさま、いかがですか?
\\	おきゃくさま、いかがですか?
\\	うーん、あの…ちょっとちいさいみたいで…もうひとつうえのサイズはありますか?
\\	うーん、あの…ちょっとちいさいみたいで…もうひとつうえのサイズはありますか?
\\	はい、いまおもちします。
\\	はい、いまおもちします。
\\	(ノックのおと)
\\	(ノックのおと)
\\	おきゃくさま、いかがですか?
\\	おきゃくさま、いかがですか?
\\	ええ、だいじょうぶみたい。
\\	ええ、だいじょうぶみたい。
\\	わあ、とてもおにあいですよ
\\	おきゃくさまのイメージにぴったり
\\	わあ、とてもおにあいですよ
\\	おきゃくさまのイメージにぴったり
\\	ふとってみえませんか?
\\	ふとってみえませんか?
\\	ぜんぜんそんなことありませんよ
\\	おきゃくさまはあしがながいし、とてもスリムにみえます。
\\	ぜんぜんそんなことありませんよ
\\	おきゃくさまはあしがながいし、とてもスリムにみえます。
\\	ひかくてき
\\	えさ
\\	マウス
\\	よれよれ
\\	ぼさぼさ
\\	じっけん
\\	はくい
\\	じったい
\\	かたくるしい
\\	しんりょう
\\	はつしゅっきん
\\	こんばんは。ごめんなさい、おまたせしちゃいました?
\\	こんばんは。ごめんなさい、おまたせしちゃいました?
\\	あ、こんばんは
\\	こちらこそ、とつぜんおさそいしちゃってすみません。
\\	あ、こんばんは
\\	こちらこそ、とつぜんおさそいしちゃってすみません。
\\	はしもとさん、きょうはおいそがしかったんじゃないですか?
\\	はしもとさん、きょうはおいそがしかったんじゃないですか?
\\	いえいえ。ぼくもきょうはひかくてきひまだったんです。それより、はしもとさん、なんてかたくるしいので、ゆうすけでいいですよ。
\\	いえいえ。ぼくもきょうはひかくてきひまだったんです。それより、はしもとさん、なんてかたくるしいので、ゆうすけでいいですよ。
\\	じゃあ、ゆうすけさん。おしごとさきはびょういんなんですか?
\\	じゃあ、ゆうすけさん。おしごとさきはびょういんなんですか?
\\	ええ、そうです。このちかくにあるだいがくびょういんです。でも、ぼくはしんりょうじゃなくてけんきゅうのほうなんですよ。
\\	ええ、そうです。このちかくにあるだいがくびょういんです。でも、ぼくはしんりょうじゃなくてけんきゅうのほうなんですよ。
\\	わあ、けんきゅうなんて、かっこいい
\\	わあ、けんきゅうなんて、かっこいい
\\	いやー、じったいはひどいんです。ふだんはあたまもぼさぼさで、はくいもよれよれ。そのかっこうでよなかにじっけんようのマウスにえさをやってるところは、ちょっとじょせいにはみせられないね。
\\	いやー、じったいはひどいんです。ふだんはあたまもぼさぼさで、はくいもよれよれ。そのかっこうでよなかにじっけんようのマウスにえさをやってるところは、ちょっとじょせいにはみせられないね。
\\	あ、それはみてみたい。
\\	あ、それはみてみたい。
\\	やめてくれよー。それよりアグネスさんのほうがかっこいいよ、デザイナーなんて。
\\	やめてくれよー。それよりアグネスさんのほうがかっこいいよ、デザイナーなんて。
\\	ぜんぜん。それにまだちゃんとしごとをしているわけじゃないの。
\\	ぜんぜん。それにまだちゃんとしごとをしているわけじゃないの。
\\	そういえば、こっちでしごとをさがしてるんだっけ。
\\	そういえば、こっちでしごとをさがしてるんだっけ。
\\	そう
\\	このあいだめんせつにいって、さいようになったの。あしたがはつしゅっきんなんだ。
\\	そう
\\	このあいだめんせつにいって、さいようになったの。あしたがはつしゅっきんなんだ。
\\	えー、おめでとう
\\	じゃあ、おいわいにおいしいワインをあけようよ。きょうはぼくがおごるから。
\\	えー、おめでとう
\\	じゃあ、おいわいにおいしいワインをあけようよ。きょうはぼくがおごるから。
\\	え、いいの?あれ、そういえばなにかそうだんがあるってはなしじゃなかった?
\\	え、いいの?あれ、そういえばなにかそうだんがあるってはなしじゃなかった?
\\	いいのいいの、そのはなしはまたこんどでも。じゃあ、アグネスのしゅうしょくにカンパーイ
\\	いいのいいの、そのはなしはまたこんどでも。じゃあ、アグネスのしゅうしょくにカンパーイ
\\	ありがとう。カンパーイ
\\	ありがとう。カンパーイ
\\	むちゃな
\\	じんざい
\\	けいき
\\	しゅうしょくかつどう
\\	しようしょ
\\	ぷれぜん
\\	ていっぱい
\\	かかえる
\\	おいしいしごと
\\	ごういん
\\	きあいをいれる
\\	おはようございます、しゃちょう。
\\	おはようございます、しゃちょう。
\\	きのう、ゴリからでんわがあったらしいな。いまないようをかくにんしたんだが、まったく、むちゃなちゅうもんだよ。このいそがしいときに
\\	きみがでんわをうけたのか?
\\	きのう、ゴリからでんわがあったらしいな。いまないようをかくにんしたんだが、まったく、むちゃなちゅうもんだよ。このいそがしいときに
\\	きみがでんわをうけたのか?
\\	すみません、おことわりするべきでしたか?
\\	すみません、おことわりするべきでしたか?
\\	いや、ことわるのはむりだろう。あいつはむかしからごういんなんだ。それにいつものばいのきんがくをはらうといっている。おいしいしごとだよ。このしごとはうけるべきだな。
\\	いや、ことわるのはむりだろう。あいつはむかしからごういんなんだ。それにいつものばいのきんがくをはらうといっている。おいしいしごとだよ。このしごとはうけるべきだな。
\\	そうなんですか。
\\	そうなんですか。
\\	よし、このけんはきみがたんとうしなさい。はつしごとだ。
\\	よし、このけんはきみがたんとうしなさい。はつしごとだ。
\\	ええっ?わたしがですか?
\\	ええっ?わたしがですか?
\\	ほかのスタッフはみんな、いそぎのしごとをかかえていてていっぱいなんだよ。あとでゴリにはでんわをしておくから、こんしゅうのきんようびまでにプレゼンのじゅんびをしてくれ。できるか?
\\	ほかのスタッフはみんな、いそぎのしごとをかかえていてていっぱいなんだよ。あとでゴリにはでんわをしておくから、こんしゅうのきんようびまでにプレゼンのじゅんびをしてくれ。できるか?
\\	は、はい
\\	なんとかなるとおもいます。
\\	は、はい
\\	なんとかなるとおもいます。
\\	よし、いいぞ
\\	これがしようしょだ。くわしいことは、のむらやほかのスタッフにききなさい。
\\	よし、いいぞ
\\	これがしようしょだ。くわしいことは、のむらやほかのスタッフにききなさい。
\\	はい。ないようは、しゅうしょくかつどうのがくせいむけのパンフレットですか…。
\\	はい。ないようは、しゅうしょくかつどうのがくせいむけのパンフレットですか…。
\\	あいつのかいしゃはいまけいきがよくて、ことしはゆうしゅうなじんざいをたくさんさいようするきでいるんだ。きあいをいれろよ
\\	このしごとがとれたら、とくべつボーナスだ
\\	あいつのかいしゃはいまけいきがよくて、ことしはゆうしゅうなじんざいをたくさんさいようするきでいるんだ。きあいをいれろよ
\\	このしごとがとれたら、とくべつボーナスだ
\\	が、がんばります
\\	が、がんばります
\\	しぜんはかい
\\	しげん
\\	こかつ
\\	ゆうがい
\\	しがいせん
\\	しゃだん
\\	ひふがん
\\	はくないしょう
\\	みずかさ
\\	はいしゅつ
\\	たいきおせんぶっしつ
\\	どじょう
\\	しょくりょうきき
\\	さばくか
\\	どうとく
\\	ちきゅう
\\	しぜんかんきょう
\\	とりまく
\\	あっか
\\	はかい
\\	おせん
\\	オゾンそう
\\	いじょうきしょう
\\	さんせいう
\\	しんりんはかい
\\	すいしつおせん
\\	きょうのどうとくのじかんは、みんなでビデオをみますよ。
\\	きょうのどうとくのじかんは、みんなでビデオをみますよ。
\\	せんせい、どんなビデオをみるんですか。
\\	せんせい、どんなビデオをみるんですか。
\\	きょうは、ちきゅうのしぜんかんきょうが、どんどんわるくなっていることについてのビデオをみます。じゃあ、みんなテレビのまえにあつまって。
\\	きょうは、ちきゅうのしぜんかんきょうが、どんどんわるくなっていることについてのビデオをみます。じゃあ、みんなテレビのまえにあつまって。
\\	はーい。
\\	はーい。
\\	ねんねん、わたしたちをとりまくしぜんかんきょうはどんどんあっかしています。わたしたちにんげんは、しぜんをはかいし、おせんしつづけてきました。ちきゅうおんだんかやオゾンそうのはかい、いじょうきしょう、さんせいう、しんりんはかい、さばくか、しょくりょうきき、しげんのこかつ。じゅんちょうにぶんめいがはったつしてきたはずのちきゅうに、いまなにがおこっているのでしょうか。そして、わたしたちはいまなにをすべきなのでしょうか。しぜんはかいがすすむいま、どのようなえいきょうがでているのかみてみましょう。
\\	ねんねん、わたしたちをとりまくしぜんかんきょうはどんどんあっかしています。わたしたちにんげんは、しぜんをはかいし、おせんしつづけてきました。ちきゅうおんだんかやオゾンそうのはかい、いじょうきしょう、さんせいう、しんりんはかい、さばくか、しょくりょうきき、しげんのこかつ。じゅんちょうにぶんめいがはったつしてきたはずのちきゅうに、いまなにがおこっているのでしょうか。そして、わたしたちはいまなにをすべきなのでしょうか。しぜんはかいがすすむいま、どのようなえいきょうがでているのかみてみましょう。
\\	オゾンそうのはかい ー ゆうがいなしがいせんをしゃだんし、じんたいをまもるはたらきをするオゾンそう。 にんげんがつくりだしたフロンガスにより、オゾンそうにあながあいてじんたいへのえいきょうがしんぱいされています。 このオゾンそうがいっパーセント
\\	へるたびに、ひふがんはさんからろくパーセント(3〜6
\\	、はくないしょうはいっパーセント
\\	ぜんごのかんじゃがふえるといわれています。
\\	オゾンそうのはかい ー ゆうがいなしがいせんをしゃだんし、じんたいをまもるはたらきをするオゾンそう。 にんげんがつくりだしたフロンガスにより、オゾンそうにあながあいてじんたいへのえいきょうがしんぱいされています。 このオゾンそうがいっパーセント
\\	へるたびに、ひふがんはさんからろくパーセント(3〜6
\\	、はくないしょうはいっパーセント
\\	ぜんごのかんじゃがふえるといわれています。
\\	ちきゅうおんだんか ー フロンやガスによって、たいきちゅうの2さんかたんそのぞうかがげんいんとされています。 そのけっか、ここすうねんちきゅうのおんどがあがっているのです。あたたかくなったえいきょうで、なんきょくのこおりがとけてすこしずつみずかさがましているけいこうがみられます。 このまますすむと、すべてのたいりくがみずにしずんでしまうかのうせいもひていできません。
\\	ちきゅうおんだんか ー フロンやガスによって、たいきちゅうの2さんかたんそのぞうかがげんいんとされています。 そのけっか、ここすうねんちきゅうのおんどがあがっているのです。あたたかくなったえいきょうで、なんきょくのこおりがとけてすこしずつみずかさがましているけいこうがみられます。 このまますすむと、すべてのたいりくがみずにしずんでしまうかのうせいもひていできません。
\\	さんせいう ー じどうしゃなどからはいしゅつされたたいきおせんぶっしつが、つよいさんせいをともなってあめとしてふるげんしょうをいいます。 じんたいへのえいきょうだけでなく、ほかのせいぶつにただいなえいきょうをあたえます。 そして、どじょうやみずのせいしつをへんかさせしんりんそのものがかれてしまうきけんせいもあります。
\\	さんせいう ー じどうしゃなどからはいしゅつされたたいきおせんぶっしつが、つよいさんせいをともなってあめとしてふるげんしょうをいいます。 じんたいへのえいきょうだけでなく、ほかのせいぶつにただいなえいきょうをあたえます。 そして、どじょうやみずのせいしつをへんかさせしんりんそのものがかれてしまうきけんせいもあります。
\\	すいしつおせん ー わたしたちがいきていくうえでかかせないみず。うつくしいみずは、おもにせいかつはいすいがげんいんでうしなわれつつあります。これらがおせんされるということは、のみみずがおせんされるだけではなく、そこにすんでいたさかななどもおせんされるということにつながります。それらをせっしゅしてしまうと、たいないにちくせきしてけんこうしょうがいをひきおこすかのうせいもあります。
\\	すいしつおせん ー わたしたちがいきていくうえでかかせないみず。うつくしいみずは、おもにせいかつはいすいがげんいんでうしなわれつつあります。これらがおせんされるということは、のみみずがおせんされるだけではなく、そこにすんでいたさかななどもおせんされるということにつながります。それらをせっしゅしてしまうと、たいないにちくせきしてけんこうしょうがいをひきおこすかのうせいもあります。
\\	わたしたちのしらないまにかくじつにしぜんはかいはすすんでいます。しぜんはかいのむくいは、かくじつにいろいろなところでおこりつつあるのです。
\\	わたしたちのしらないまにかくじつにしぜんはかいはすすんでいます。しぜんはかいのむくいは、かくじつにいろいろなところでおこりつつあるのです。
\\	では、しぜんはかいをふせぐためにいまわたしたちは、なにをすべきなのでしょうか。まずみじかなところでできる、ゴミのぶんべつとしょり。 これをまもることでゆうがいガスをすくなからずふせぐことがかのうです。とうぜんポイすてはもってのほか。しかし、ひとりでじっこうしてもいみがありません。 ひとりひとりがそれをまもることでおおきなせいかをはっきし、しぜんはかいのよぼうにつながるのです。 いまは、それぞれがみじかでできることをじっこうしていくことがたいせつです。
\\	では、しぜんはかいをふせぐためにいまわたしたちは、なにをすべきなのでしょうか。まずみじかなところでできる、ゴミのぶんべつとしょり。 これをまもることでゆうがいガスをすくなからずふせぐことがかのうです。とうぜんポイすてはもってのほか。しかし、ひとりでじっこうしてもいみがありません。 ひとりひとりがそれをまもることでおおきなせいかをはっきし、しぜんはかいのよぼうにつながるのです。 いまは、それぞれがみじかでできることをじっこうしていくことがたいせつです。
\\	みんな、しぜんはかいがどれだけこわいものかわかったでしょう。しぜんをまもるために、みんなができることがあるっていっていたけど、それがなにかわかるひと。
\\	みんな、しぜんはかいがどれだけこわいものかわかったでしょう。しぜんをまもるために、みんなができることがあるっていっていたけど、それがなにかわかるひと。
\\	ハイハイハイ
\\	ハイハイハイ
\\	じゃあ、サトシ。
\\	じゃあ、サトシ。
\\	はい。ゴミをだすときは、ぶんべつしてだすことです。それと、ポイすてはぜったいしないことです。
\\	はい。ゴミをだすときは、ぶんべつしてだすことです。それと、ポイすてはぜったいしないことです。
\\	そのとおり。みんなもこれからはしぜんのことをすこしかんがえてこうどうすること。わかったわね。
\\	そのとおり。みんなもこれからはしぜんのことをすこしかんがえてこうどうすること。わかったわね。
\\	はーい。
\\	はーい。
\\	(じゅぎょうおわりのチャイムがなる)
\\	(じゅぎょうおわりのチャイムがなる)
\\	はい、きょうはここまで。
\\	はい、きょうはここまで。
\\	そうそう
\\	まかされる
\\	きたいされる
\\	いまさら
\\	しゅうでん
\\	はい、もしもし。
\\	はい、もしもし。
\\	アグネスさん?
\\	アグネスさん?
\\	あ、ゆうすけさん?こんばんは。
\\	あ、ゆうすけさん?こんばんは。
\\	おそくにごめんね。いま、だいじょうぶかな?
\\	おそくにごめんね。いま、だいじょうぶかな?
\\	えーと、いましごとちゅうなの。
\\	えーと、いましごとちゅうなの。
\\	えっ、こんなじかんまで?もうじゅうにじすぎてるよ
\\	どうしたの?
\\	えっ、こんなじかんまで?もうじゅうにじすぎてるよ
\\	どうしたの?
\\	うん、にゅうしゃそうそうおおきなしごとをまかされちゃって、ぜんぜんおわらないんだ。
\\	うん、にゅうしゃそうそうおおきなしごとをまかされちゃって、ぜんぜんおわらないんだ。
\\	そうか、たいへんだなあ。でも、すぐにだいじなしごとをまかされるなんて、きたいされてるね。
\\	そうか、たいへんだなあ。でも、すぐにだいじなしごとをまかされるなんて、きたいされてるね。
\\	うーん、どうかなあ…。あ、それより、なにかようじだった?
\\	うーん、どうかなあ…。あ、それより、なにかようじだった?
\\	いや、またしゅうまつにでもしょくじにどうかなあ、とおもって。
\\	いや、またしゅうまつにでもしょくじにどうかなあ、とおもって。
\\	そうねえ、このしごとのプレゼンがきんようびなの。だから、しゅうまつならいかれるとおもうわ。
\\	そうねえ、このしごとのプレゼンがきんようびなの。だから、しゅうまつならいかれるとおもうわ。
\\	そうか、よかった
\\	じゃあ、またでんわするよ。いそがしいところごめんね。がんばって。
\\	そうか、よかった
\\	じゃあ、またでんわするよ。いそがしいところごめんね。がんばって。
\\	うん、ありがとう。じゃあね。
\\	うん、ありがとう。じゃあね。
\\	(でんわきれる)
\\	(でんわきれる)
\\	(かいしゃのでんわがなる)
\\	(かいしゃのでんわがなる)
\\	はい、デザインオフィスジェイでございます。
\\	はい、デザインオフィスジェイでございます。
\\	あかさかのゴリだけど。
\\	あかさかのゴリだけど。
\\	あ、たんとうのむらかみでございます
\\	あ、たんとうのむらかみでございます
\\	おお、アンタか。おそくまでしごとさせて、わるいねえ。ちょっとかくにんしたいんだけどさ、おれ、きのうのでんわでひょうしはあおがいいっていったっけ?
\\	おお、アンタか。おそくまでしごとさせて、わるいねえ。ちょっとかくにんしたいんだけどさ、おれ、きのうのでんわでひょうしはあおがいいっていったっけ?
\\	ええ。
\\	ええ。
\\	やっぱりそうかー。あのさ、いまからあかにかえられる?
\\	やっぱりそうかー。あのさ、いまからあかにかえられる?
\\	えっ…と、しょうちしました。ごれんらくありがとうございました。
\\	えっ…と、しょうちしました。ごれんらくありがとうございました。
\\	したのこ
\\	じゅぎょうさんかん
\\	ちょうし
\\	いぶんか
\\	たまには
\\	あのさあ、きょう、したのこのじゅぎょうさんかんにはじめていったんだけど、すごくおもしろかったよ。
\\	あのさあ、きょう、したのこのじゅぎょうさんかんにはじめていったんだけど、すごくおもしろかったよ。
\\	へー。どんなだったの。
\\	へー。どんなだったの。
\\	みたのはさんすうのじゅぎょうだったんだけどさあ。
\\	みたのはさんすうのじゅぎょうだったんだけどさあ。
\\	はじめにボールがみっつ、ありました。あとでよっつ、ふえました。ぜんぶでいくつになったかな?。。わかるひと
\\	はじめにボールがみっつ、ありました。あとでよっつ、ふえました。ぜんぶでいくつになったかな?。。わかるひと
\\	ハイハイハイ
\\	ハイハイハイ
\\	じゃあ、レオくん。わかるかな。
\\	じゃあ、レオくん。わかるかな。
\\	はい。ななです。
\\	はい。ななです。
\\	そう
\\	ほんとう?ほかにだれか、ちがうとおもうひとは?
\\	そう
\\	ほんとう?ほかにだれか、ちがうとおもうひとは?
\\	ハイハイハイ
\\	ハイハイハイ
\\	じゃあ、マサミちゃん。
\\	じゃあ、マサミちゃん。
\\	せんせい、ななです。
\\	せんせい、ななです。
\\	ああ、そう。ななですか。ほかには?ほかのこたえは?
\\	ああ、そう。ななですか。ほかには?ほかのこたえは?
\\	ハイハイハイ
\\	ハイハイハイ
\\	ユウキくん、わかるかな。
\\	ユウキくん、わかるかな。
\\	はい。ななだとおもいます。
\\	はい。ななだとおもいます。
\\	うーん。ななかあ。ほかには?こたえわかるひとは?
\\	うーん。ななかあ。ほかには?こたえわかるひとは?
\\	ハイハイハイ
\\	ハイハイハイ
\\	ってまあ、ずっとこんなちょうしで、さらに にかいも、こどもたちのハイハイハイ
\\	がつづいてさあ。おどろいちゃったよ。
\\	ってまあ、ずっとこんなちょうしで、さらに にかいも、こどもたちのハイハイハイ
\\	がつづいてさあ。おどろいちゃったよ。
\\	なるほどね。しょうがくいちねんせいってそんなもんなのよね。
\\	なるほどね。しょうがくいちねんせいってそんなもんなのよね。
\\	まさに、いぶんかコミュニケーション。こどものせかいってすごいな。
\\	まさに、いぶんかコミュニケーション。こどものせかいってすごいな。
\\	やきゅうかんせん
\\	かって
\\	もってのほか
\\	たきゅうだん
\\	おうえんだんいん
\\	てすう
\\	しかたない
\\	きょうしゅく
\\	がわ
\\	こうほび
\\	むりょう
\\	しょうたい
\\	そしてもらえると たいへんありがたい です
\\	うちのこどもからきいたんですが、こんど、ヤクルトがしょうがくせいをむりょうでやきゅうかんせんにしょうたいしてくれるんですって。いっしょにいきませんか。
\\	うちのこどもからきいたんですが、こんど、ヤクルトがしょうがくせいをむりょうでやきゅうかんせんにしょうたいしてくれるんですって。いっしょにいきませんか。
\\	すごいですね。むりょうしょうたいなんて。。。ぜひ、ごいっしょさせてください。
\\	すごいですね。むりょうしょうたいなんて。。。ぜひ、ごいっしょさせてください。
\\	いくつかこうほびがあるんですが、どのしあいにしますか?やっぱり、きょじんせんかな。
\\	いくつかこうほびがあるんですが、どのしあいにしますか?やっぱり、きょじんせんかな。
\\	そうしましょう
\\	わたしはだいのきょじんファンなので、そうしてもらえると、たいへんありがたいです
\\	そうしましょう
\\	わたしはだいのきょじんファンなので、そうしてもらえると、たいへんありがたいです
\\	やまださんって、きょじんファンだったんですね。でも、ヤクルトがわのせきですよ。それでもいいですか?
\\	やまださんって、きょじんファンだったんですね。でも、ヤクルトがわのせきですよ。それでもいいですか?
\\	えー
\\	それは、きょじんファンとしてはたえられないです。わたしだけきょじんがわにすわってもいいですか?
\\	えー
\\	それは、きょじんファンとしてはたえられないです。わたしだけきょじんがわにすわってもいいですか?
\\	うーーん、たぶんそれはできないとおもいますよ。
\\	うーーん、たぶんそれはできないとおもいますよ。
\\	あのぉ、こんなことおねがいするのは、たいへんきょうしゅくなのですが、うちのこどもだけつれていってもらえませんか?
\\	あのぉ、こんなことおねがいするのは、たいへんきょうしゅくなのですが、うちのこどもだけつれていってもらえませんか?
\\	しかたないですね。じゃぁ、わたしがつれていきますよ。
\\	しかたないですね。じゃぁ、わたしがつれていきますよ。
\\	すみません。おてすうをおかけします。じつは、わたし、きょじんのおうえんだんいんなので、たきゅうだんのせきにすわるなんてことはもってのほかなんです。かってをいってもうしわけありませんが、どうぞよろしくおねがいします。
\\	すみません。おてすうをおかけします。じつは、わたし、きょじんのおうえんだんいんなので、たきゅうだんのせきにすわるなんてことはもってのほかなんです。かってをいってもうしわけありませんが、どうぞよろしくおねがいします。
\\	アンタ
\\	ロマン
\\	エロじじい
\\	やっぱり
\\	いちおう
\\	れっきとした
\\	おとめ
\\	しびれる
\\	いつよんでもドラゴン・ボールはさいこうだよな。
\\	いつよんでもドラゴン・ボールはさいこうだよな。
\\	アンタ、じゅぎょうちゅうにまんがよんでたでしょ。
\\	アンタ、じゅぎょうちゅうにまんがよんでたでしょ。
\\	おんなには、ドラゴン・ボールのロマンはわからないだろうなあ。「カ・メ・ハ・メ・はぁぁ~
\\	やっぱりコレだよ。
\\	おんなには、ドラゴン・ボールのロマンはわからないだろうなあ。「カ・メ・ハ・メ・はぁぁ~
\\	やっぱりコレだよ。
\\	なによ。ごくうなんてこどもじゃない。かめせんにんはただのエロじじいだし。やっぱり、ほくとのけんよ。
\\	なによ。ごくうなんてこどもじゃない。かめせんにんはただのエロじじいだし。やっぱり、ほくとのけんよ。
\\	ほくとのけん~
\\	ほくとのけん~
\\	「おまえはもう、しんでいる。」ケンシロウかっこうよすぎ
\\	「おまえはもう、しんでいる。」ケンシロウかっこうよすぎ
\\	おまえいちおう、おんなだろ。「どうぶつのおいしゃさん」とか、ふつうはそういうんだろ。
\\	おまえいちおう、おんなだろ。「どうぶつのおいしゃさん」とか、ふつうはそういうんだろ。
\\	なによ、そのいちおう、おんなって。れっきとしたおとめですう。「あたたたたたたたたたたたたたたたたたたたた・・・」ケンシロウのほくとひゃくれつけんってほんとうにさいこうだわ。しびれるう~。
\\	なによ、そのいちおう、おんなって。れっきとしたおとめですう。「あたたたたたたたたたたたたたたたたたたたた・・・」ケンシロウのほくとひゃくれつけんってほんとうにさいこうだわ。しびれるう~。
\\	ぜんかん
\\	マジ
\\	よじげん
\\	わくわく
\\	ひみつどうぐ
\\	だんぜん
\\	このあいだタツオのいえにあそびにいったら、タツオ、「ドラえもん」ぜんかんもっててさあ。
\\	このあいだタツオのいえにあそびにいったら、タツオ、「ドラえもん」ぜんかんもっててさあ。
\\	マジで。すごいね~。
\\	マジで。すごいね~。
\\	おとうさんにかってもらったんだって。
\\	おとうさんにかってもらったんだって。
\\	あの、ドラえもんがよじげんポケットからひみつどうぐをだすときの、あれ。「(ドラえもんになりきって)タケコプター」ってやつ。あれきくとすごくワクワクするよね。
\\	あの、ドラえもんがよじげんポケットからひみつどうぐをだすときの、あれ。「(ドラえもんになりきって)タケコプター」ってやつ。あれきくとすごくワクワクするよね。
\\	ひみつどうぐでしょ。すっごいワクワクする。わたしは「(ドラえもんになりきって)どこでもドア~」がいちばんわくわくするね。
\\	ひみつどうぐでしょ。すっごいワクワクする。わたしは「(ドラえもんになりきって)どこでもドア~」がいちばんわくわくするね。
\\	こいびとにするなら、シズカちゃんよりだんぜんドラミちゃんだね。
\\	こいびとにするなら、シズカちゃんよりだんぜんドラミちゃんだね。
\\	なんで。
\\	なんで。
\\	だって、よじげんポケットもってるじゃん。ドラミちゃん うたもうまいし。たのしいって。
\\	だって、よじげんポケットもってるじゃん。ドラミちゃん うたもうまいし。たのしいって。
\\	じょうりく
\\	かのうせい
\\	たいふういっか
\\	さいあく
\\	しちがつになってたいふうがおおいよな
\\	しちがつになってたいふうがおおいよな
\\	そうそう、しちがつからくがつだけでさんじゅっこくらいはにほんにじょうりくするもんね。
\\	そうそう、しちがつからくがつだけでさんじゅっこくらいはにほんにじょうりくするもんね。
\\	おおあめだし、きょうふうだし、、、けどたいふうがくればがっこうがやすみになるかのうせいもたかいけどね。どこにもいけないけど。。
\\	おおあめだし、きょうふうだし、、、けどたいふうがくればがっこうがやすみになるかのうせいもたかいけどね。どこにもいけないけど。。
\\	どうやら、に、さんにちしたら、たいふうよんごうがくるかもしれないらしいよ。
\\	どうやら、に、さんにちしたら、たいふうよんごうがくるかもしれないらしいよ。
\\	えーっ
\\	こんどのにちよう、つりにいこうとおもってたのに。。。ダメじゃん。。。
\\	えーっ
\\	こんどのにちよう、つりにいこうとおもってたのに。。。ダメじゃん。。。
\\	そりゃー、しょうがないわ。
\\	そりゃー、しょうがないわ。
\\	しかもたいふうがすぎたあとはたいふういっかですごいはれて、めっちゃきおんがあがって。いいことないわ。
\\	しかもたいふうがすぎたあとはたいふういっかですごいはれて、めっちゃきおんがあがって。いいことないわ。
\\	そうそう。さいあく。
\\	そうそう。さいあく。
\\	けさ
\\	ざんぎょう
\\	かどうか
\\	やりなおす
\\	しゅうせい
\\	ようぼう
\\	ないしょ
\\	いちだんらく
\\	おとす
\\	おはようございまーす。
\\	おはようございまーす。
\\	あ、のむらさんおはようございます。
\\	あ、のむらさんおはようございます。
\\	アグネスさん、けさもはやいね。きのうもおそくまでざんぎょうしていたでしょ?
\\	アグネスさん、けさもはやいね。きのうもおそくまでざんぎょうしていたでしょ?
\\	きのう、ゴリさんからでんわがあって。。。いろをかえなきゃならなくなったんです。だからまにあうかどうかしんぱいで。
\\	きのう、ゴリさんからでんわがあって。。。いろをかえなきゃならなくなったんです。だからまにあうかどうかしんぱいで。
\\	ええっ
\\	いまから?それはひどいねー。
\\	ええっ
\\	いまから?それはひどいねー。
\\	ちがういろでやりなおしてたら、ぜんたいをしゅうせいしないわけにはいかなくなってしまって。。。
\\	ちがういろでやりなおしてたら、ぜんたいをしゅうせいしないわけにはいかなくなってしまって。。。
\\	だいじょうぶ?プレゼンはあしたでしょう?
\\	だいじょうぶ?プレゼンはあしたでしょう?
\\	でも、クライアントのようぼうだからきかざるをえないですよねぇ。
\\	でも、クライアントのようぼうだからきかざるをえないですよねぇ。
\\	そうねえ。。。そういうこまったクライアント、おおいのよね。あ、いまの、しゃちょうにはないしょよ。
\\	そうねえ。。。そういうこまったクライアント、おおいのよね。あ、いまの、しゃちょうにはないしょよ。
\\	しゃちょうのおともだちですもんね。
\\	しゃちょうのおともだちですもんね。
\\	だから、いうことをきかないわけにいかないんだけどね。
\\	だから、いうことをきかないわけにいかないんだけどね。
\\	ああー、どうしよう。まにあうかしら。
\\	ああー、どうしよう。まにあうかしら。
\\	わたし、きょうのしごとがひとだんらくしたらてつだってあげるわ。このしごとはおとすわけにはいかないでしょう。
\\	わたし、きょうのしごとがひとだんらくしたらてつだってあげるわ。このしごとはおとすわけにはいかないでしょう。
\\	いいんですか?ごめんなさい。
\\	いいんですか?ごめんなさい。
\\	いいのよ。にゅうしゃしたばかりでこんなきついしごと、たいへんでしょう。あとでだれかほかにもてつだえるかどうかきいてみるわ。
\\	いいのよ。にゅうしゃしたばかりでこんなきついしごと、たいへんでしょう。あとでだれかほかにもてつだえるかどうかきいてみるわ。
\\	ありがとうございます
\\	ありがとうございます
\\	おたく
\\	ひるむ
\\	にほんとう
\\	ちんぴら
\\	ひっとう
\\	もひかん
\\	ぶつかる
\\	かげ
\\	いえじ
\\	かり
\\	かくご
\\	いてっ
\\	いてっ
\\	(いきなりかげからでてきて、ぶつかってきたのはあなたじゃないですか。なんなんですか)
\\	(いきなりかげからでてきて、ぶつかってきたのはあなたじゃないですか。なんなんですか)
\\	かたがいてーなー。どうしてくれんだよー。
\\	かたがいてーなー。どうしてくれんだよー。
\\	(いち、に、さん...さんにんぐみか。モヒカンをひっとうにぜんいんチンピラふう。これって・・・よにいうオタクがりってやつ
\\	(いち、に、さん...さんにんぐみか。モヒカンをひっとうにぜんいんチンピラふう。これって・・・よにいうオタクがりってやつ
\\	ゆるしてほしければ、ドラゴンボールのマンガぜんぶと
\\	をわたしな。
\\	ゆるしてほしければ、ドラゴンボールのマンガぜんぶと
\\	をわたしな。
\\	(あっ、やっぱりオタクがりってやつですか。。。)
\\	(あっ、やっぱりオタクがりってやつですか。。。)
\\	おい。(おとこ、もっていたにほんとうをだす。ひるむわかいおとこたち。)
\\	おい。(おとこ、もっていたにほんとうをだす。ひるむわかいおとこたち。)
\\	おっオタクが、そっ、そんなものだしていいのか。
\\	おっオタクが、そっ、そんなものだしていいのか。
\\	かくごしろよ。
\\	かくごしろよ。
\\	ほうめん
\\	グルメてんごく
\\	どくとく
\\	たべつくす
\\	きにゅうする
\\	こうくうけん
\\	よやく
\\	まちなみ
\\	うだるような
\\	いらっしゃいませ。
\\	いらっしゃいませ。
\\	らいげつにアジアほうめんにりょこうをかんがえているんですが、どこかオススメのばしょはありませんか。
\\	らいげつにアジアほうめんにりょこうをかんがえているんですが、どこかオススメのばしょはありませんか。
\\	なんにちかんぐらいをかんがえていらっしゃいますか。
\\	なんにちかんぐらいをかんがえていらっしゃいますか。
\\	いっしゅうかんくらいです。
\\	いっしゅうかんくらいです。
\\	ちょっとしらべてみますね。しょうしょうおまちください。にめいさまでよろしいですか。
\\	ちょっとしらべてみますね。しょうしょうおまちください。にめいさまでよろしいですか。
\\	はい。おねがいします。(さん、よんぷんご)
\\	はい。おねがいします。(さん、よんぷんご)
\\	おまたせしました。いま、いちばんのオススメはちゅうごくのしゃんはいですね。ななはくようかホテルつきで1めいさま68,000えんです。
\\	おまたせしました。いま、いちばんのオススメはちゅうごくのしゃんはいですね。ななはくようかホテルつきで1めいさま68,000えんです。
\\	しゃんはいですか。
\\	しゃんはいですか。
\\	しゃんはいはいまにんきですよ。ヨーロッパぶんかとちゅうごくぶんかがミックスされたまちなみはしゃんはいどくとくできれいですし、しゃんはいはアジアのグルメてんごくといわれてますから、たべものもすごくおいしいですよ。
\\	しゃんはいはいまにんきですよ。ヨーロッパぶんかとちゅうごくぶんかがミックスされたまちなみはしゃんはいどくとくできれいですし、しゃんはいはアジアのグルメてんごくといわれてますから、たべものもすごくおいしいですよ。
\\	どうする?(どうせきのユミにきく)
\\	どうする?(どうせきのユミにきく)
\\	まなつのしゃんはいはものすごくあついってきくけどだいじょうぶかしら。
\\	まなつのしゃんはいはものすごくあついってきくけどだいじょうぶかしら。
\\	このじき、アジアはどこでもうだるようなあつさだよ。
\\	このじき、アジアはどこでもうだるようなあつさだよ。
\\	そうよね。ちゅうかりょうりたべつくしっていうのもいいわねえ。
\\	そうよね。ちゅうかりょうりたべつくしっていうのもいいわねえ。
\\	タカオ
\\	じゃあ、きまり
\\	すいません、ではしゃんはいでおねがいします。
\\	タカオ
\\	じゃあ、きまり
\\	すいません、ではしゃんはいでおねがいします。
\\	ありがとうございます。ではさっそくよやくをいれたいとおもいますので、このようしにひつようじこうをごきにゅうください。
\\	ありがとうございます。ではさっそくよやくをいれたいとおもいますので、このようしにひつようじこうをごきにゅうください。
\\	わかりました。
\\	わかりました。
\\	はんせい
\\	にんしきする
\\	むだに
\\	へこむ
\\	はいき
\\	はいきしょぶん
\\	ぎりぎり
\\	しょうみきげん
\\	しんとうする
\\	はいきゅう
\\	ふくしだんたい
\\	ひえいり
\\	ひきとる
\\	むしょう
\\	かてい
\\	せいぞうはんばい
\\	きぎょう
\\	しゃかいは
\\	フードバンクっていうかつどう、しってる?このあいだ、たまたまざっしでよんだんだけど、それをよんだら、いままでのわたしのスーパーでのしょくひんのかいかたをはんせいさせられちゃったわ。
\\	フードバンクっていうかつどう、しってる?このあいだ、たまたまざっしでよんだんだけど、それをよんだら、いままでのわたしのスーパーでのしょくひんのかいかたをはんせいさせられちゃったわ。
\\	フードバンクってなんのこと?
\\	フードバンクってなんのこと?
\\	フードバンクって、きぎょうがしょくひんをせいぞうはんばいするかていで、しょうひんにならなくなってしまったものを、むしょうでひきとって、それをひえいりのふくしだんたいなどをとおして、たべものにこまっているひとびとへむしょうではいきゅうしようとするシステムなんですって。
\\	フードバンクって、きぎょうがしょくひんをせいぞうはんばいするかていで、しょうひんにならなくなってしまったものを、むしょうでひきとって、それをひえいりのふくしだんたいなどをとおして、たべものにこまっているひとびとへむしょうではいきゅうしようとするシステムなんですって。
\\	へぇー。なるほどね。
\\	へぇー。なるほどね。
\\	アメリカでは、ずいぶんしんとうしつつあるかつどうらしいよ。
\\	アメリカでは、ずいぶんしんとうしつつあるかつどうらしいよ。
\\	ふーん、だけど、なんで、それがスーパーでのしょくひんのかいかたとかんけいがあるわけ?
\\	ふーん、だけど、なんで、それがスーパーでのしょくひんのかいかたとかんけいがあるわけ?
\\	たとえば、ぎゅうにゅうをかうときは、しょうみきげんができるだけさきのぎゅうにゅうをえらんでかってたんだけど、みんながそういうかいかたをするから、しょうみきげんぎりぎりのぎゅうにゅうばかりがのこっちゃって、それがけっきょくはいきしょぶんになるわけよ。
\\	たとえば、ぎゅうにゅうをかうときは、しょうみきげんができるだけさきのぎゅうにゅうをえらんでかってたんだけど、みんながそういうかいかたをするから、しょうみきげんぎりぎりのぎゅうにゅうばかりがのこっちゃって、それがけっきょくはいきしょぶんになるわけよ。
\\	なるほどね。みんながなるべくしょうみきげんがちかいものをかっていけば、それだけはいきされるたべものがすくなくなるもんな。
\\	なるほどね。みんながなるべくしょうみきげんがちかいものをかっていけば、それだけはいきされるたべものがすくなくなるもんな。
\\	そういうことよ。それに、パッケージがちょっとでもへこんでたりするしょうひんはかわなかったりするでしょ。だから、なかみになんのもんだいもなくても、パッケージがよごれたりきずついたりしたしょうひんははいきしょぶんされているのがげんじょうらしいわ。
\\	そういうことよ。それに、パッケージがちょっとでもへこんでたりするしょうひんはかわなかったりするでしょ。だから、なかみになんのもんだいもなくても、パッケージがよごれたりきずついたりしたしょうひんははいきしょぶんされているのがげんじょうらしいわ。
\\	たしかに、わざわざパッケージがよごれているしょうひんはかわないよな。
\\	たしかに、わざわざパッケージがよごれているしょうひんはかわないよな。
\\	フードバンクのかつどうがしんとうすることはたいせつだけど、それとどうじに、わたしたちしょうひしゃもしょくひんがむだにはいきされていることをにんしきするひつようがあるわね。
\\	フードバンクのかつどうがしんとうすることはたいせつだけど、それとどうじに、わたしたちしょうひしゃもしょくひんがむだにはいきされていることをにんしきするひつようがあるわね。
\\	なんか、きょうのきみは、ずいぶんしゃかいはだね。
\\	なんか、きょうのきみは、ずいぶんしゃかいはだね。
\\	にどね
\\	とおまわり
\\	ろせん
\\	めどがたたない
\\	さいかい
\\	みあわせる
\\	ぜんせん
\\	はっせい
\\	じんしんじこ
\\	ちょうだのれつ
\\	(めざましのおとがなる)
\\	(めざましのおとがなる)
\\	うーん…あ、もうあさ?きのうもしゅうでんまでざんぎょうしてたから、ねむい…あとすこしだけねよう…。
\\	うーん…あ、もうあさ?きのうもしゅうでんまでざんぎょうしてたから、ねむい…あとすこしだけねよう…。
\\	しまった
\\	もうこんなじかん
\\	うっかりにどねしちゃったよー。きょうはあさからプレゼンだから、すこしはやくかいしゃにいってじゅんびしようとおもったのに、これじゃギリギリだわ。あさごはんをたべているばあいじゃないわね。
\\	しまった
\\	もうこんなじかん
\\	うっかりにどねしちゃったよー。きょうはあさからプレゼンだから、すこしはやくかいしゃにいってじゅんびしようとおもったのに、これじゃギリギリだわ。あさごはんをたべているばあいじゃないわね。
\\	(えきのおと)
\\	(えきのおと)
\\	おきゃくさまにおしらせします。さきほどじんしんじこがはっせいいたしましたかんけいで、ただいまぜんせんでうんてんをみあわせております。
\\	おきゃくさまにおしらせします。さきほどじんしんじこがはっせいいたしましたかんけいで、ただいまぜんせんでうんてんをみあわせております。
\\	ええーっ
\\	じんしんじこ
\\	いつうごくのかしら?
\\	ええーっ
\\	じんしんじこ
\\	いつうごくのかしら?
\\	あ、えきいんさん、すみません。うんてんさいかいまで、どれくらいかかりそうですか?
\\	あ、えきいんさん、すみません。うんてんさいかいまで、どれくらいかかりそうですか?
\\	もうしわけございません、ただいまさいかいのめどがたっておりません。
\\	もうしわけございません、ただいまさいかいのめどがたっておりません。
\\	そんなあ
\\	わたし、きょうだけはちこくできないのに…。
\\	そんなあ
\\	わたし、きょうだけはちこくできないのに…。
\\	バスでべつのろせんのえきにむかっていただけないでしょうか。
\\	バスでべつのろせんのえきにむかっていただけないでしょうか。
\\	そんなとおまわりしているばあいじゃないのよー。しかたがないわ、タクシーでいこう。
\\	そんなとおまわりしているばあいじゃないのよー。しかたがないわ、タクシーでいこう。
\\	うわあ、タクシーのりばもちょうだのれつ
\\	これじゃまにあわない。あ、そうだ、かいしゃにでんわしなきゃ。
\\	うわあ、タクシーのりばもちょうだのれつ
\\	これじゃまにあわない。あ、そうだ、かいしゃにでんわしなきゃ。
\\	はい、デザインオフィス
\\	でございます。
\\	はい、デザインオフィス
\\	でございます。
\\	もしもし、むらかみアグネスです。じつは、じんしんじこででんしゃがとまってしまって、かいしゃにおくれそうなんです。
\\	もしもし、むらかみアグネスです。じつは、じんしんじこででんしゃがとまってしまって、かいしゃにおくれそうなんです。
\\	じゅうたい
\\	せんとう
\\	ひるすぎ
\\	したみち
\\	うらみち
\\	かれし
\\	かのじょ
\\	まじ
\\	このさきじゅうたいちゅうだってさー。
\\	このさきじゅうたいちゅうだってさー。
\\	えー、どれぐらいじゅうたいしてるのー。
\\	えー、どれぐらいじゅうたいしてるのー。
\\	ラジオではだんごうざかをせんとうににじゅうご
\\	キロのじゅうたい。。。たぶん、かわぐちこにつくのはひるすぎになっちまうかも。
\\	ラジオではだんごうざかをせんとうににじゅうご
\\	キロのじゅうたい。。。たぶん、かわぐちこにつくのはひるすぎになっちまうかも。
\\	まーじーでー。したみちでいったらどんくらいかかる?
\\	まーじーでー。したみちでいったらどんくらいかかる?
\\	んー。。これくらいのじゅうたいだったら、うえでいっても、したみちでいくのとかわんねぇんじゃねーの。
\\	んー。。これくらいのじゅうたいだったら、うえでいっても、したみちでいくのとかわんねぇんじゃねーの。
\\	うらみちとかないの?
\\	うらみちとかないの?
\\	わからん。。。
\\	わからん。。。
\\	だからやすみのひにでかけるのはいやなんだよねー。
\\	だからやすみのひにでかけるのはいやなんだよねー。
\\	さかだち
\\	はずかしい
\\	ひとさま
\\	てしおにかける
\\	とぶ
\\	いちにんまえ
\\	しゅうとく
\\	わざ
\\	でかした
\\	ちゃくちする
\\	かんぺき(な)
\\	たろう(さる)、こうじ(トレーナー)
\\	たろう(さる)、こうじ(トレーナー)
\\	よし、たろう、そこでかべをジャンプしてさかだちでちゃくちするんだ
\\	よし、たろう、そこでかべをジャンプしてさかだちでちゃくちするんだ
\\	キーキーキー
\\	キーキーキー
\\	いいぞ、たろう
\\	でかした
\\	このわざをしゅうとくするのにいちねんもかかったなぁ。だが、これでおまえもいちにんまえのサルのいちいんだな。さかだちはできるし、こんなにたかいかべだってとべるんだ。
\\	いいぞ、たろう
\\	でかした
\\	このわざをしゅうとくするのにいちねんもかかったなぁ。だが、これでおまえもいちにんまえのサルのいちいんだな。さかだちはできるし、こんなにたかいかべだってとべるんだ。
\\	なんつったって、いちばんすごいのは、おまえはかいものにいけるし、りょうりもできる。へやのそうじだってできる。
\\	なんつったって、いちばんすごいのは、おまえはかいものにいけるし、りょうりもできる。へやのそうじだってできる。
\\	キーキーキー
\\	キーキーキー
\\	「うれしい
\\	っていってんのか?そうかそうか。おれもうれしいぞ、たろう
\\	このにねんかん、てしおにかけておまえをそだててきたんだ。ひとさまのまえにおまえをだしても、はずかしくない。あとは、おふろそうじとトイレそうじをおぼえれば、かんぺきだ
\\	「うれしい
\\	っていってんのか?そうかそうか。おれもうれしいぞ、たろう
\\	このにねんかん、てしおにかけておまえをそだててきたんだ。ひとさまのまえにおまえをだしても、はずかしくない。あとは、おふろそうじとトイレそうじをおぼえれば、かんぺきだ
\\	わく
\\	アイロンをかける
\\	えいぎょう
\\	いえで
\\	ちょうきょう
\\	とっつかまえる
\\	えてこう
\\	たろう(さる)、こうじ(トレーナー)
\\	たろう(さる)、こうじ(トレーナー)
\\	もうそろそろ、メシのじかんじゃねーの?おい、たろう、メシはまだか?
\\	はやくしろよーっ。
\\	もうそろそろ、メシのじかんじゃねーの?おい、たろう、メシはまだか?
\\	はやくしろよーっ。
\\	キー
\\	キー
\\	おっ、きょうはハンバーグかぁ。きょうはさかなのきぶんだったんだけど。。。まぁいいや…
\\	おっ、きょうはハンバーグかぁ。きょうはさかなのきぶんだったんだけど。。。まぁいいや…
\\	おい、たろう。おまえのけがちょっとはいってんだけど。。。
\\	おい、たろう。おまえのけがちょっとはいってんだけど。。。
\\	キー?
\\	キー?
\\	あっ、ふろはもうわいてんのか?メシくったらふろにいりたいんだから、さっさとよういしとけよ。あっ
\\	ワイシャツにアイロンはかけといたのか?あしたはだいじなえいぎょうがあるんだから、しっかりアイロンかけたシャツをきていかないと。
\\	あっ、ふろはもうわいてんのか?メシくったらふろにいりたいんだから、さっさとよういしとけよ。あっ
\\	ワイシャツにアイロンはかけといたのか?あしたはだいじなえいぎょうがあるんだから、しっかりアイロンかけたシャツをきていかないと。
\\	キーキーキー
\\	キーキーキー
\\	あー
\\	そうそうそう、あしたはかのじょくるんだから、よるはどっかそとでねろよなっ
\\	あー
\\	そうそうそう、あしたはかのじょくるんだから、よるはどっかそとでねろよなっ
\\	ええぇ???
\\	ええぇ???
\\	あっ、そうだそうだ、それから。。。
\\	あっ、そうだそうだ、それから。。。
\\	キーーーーーっっ
\\	キーーーーーっっ
\\	バタン(ドアがしまる)
\\	バタン(ドアがしまる)
\\	とうじょうてつづき
\\	しめきる
\\	はいけんする
\\	オカン
\\	とうき
\\	ちょうかりょうきん
\\	ととのう
\\	おあずかりする
\\	かかりいん、やまだ
\\	かかりいん、やまだ
\\	なりたはつ、バンクーバーいきのおきゃくさまのとうじょうてつづきをまもなくしめきらせていただきます
\\	なりたはつ、バンクーバーいきのおきゃくさまのとうじょうてつづきをまもなくしめきらせていただきます
\\	まったー
\\	まてまてまてまて
\\	ここにもうひとり、とうじょうしゃがいます
\\	まったー
\\	まてまてまてまて
\\	ここにもうひとり、とうじょうしゃがいます
\\	おきゃくさまおいそぎください、あとさんじゅっぷん(30ぷん)でバンクーバーいきがしゅっぱつしてしまいます。パスポートととうじょうけんをはいけんさせていただきます。
\\	おきゃくさまおいそぎください、あとさんじゅっぷん(30ぷん)でバンクーバーいきがしゅっぱつしてしまいます。パスポートととうじょうけんをはいけんさせていただきます。
\\	あいよっ
\\	あいよっ
\\	はい、おあずかりします。 。。。。おきゃくさま、このパスポート、どうやらおかあさまかどなたかのパスポートではないでしょうか?
\\	はい、おあずかりします。 。。。。おきゃくさま、このパスポート、どうやらおかあさまかどなたかのパスポートではないでしょうか?
\\	はっ
\\	やべー、しまった、オカンのパスポートもってきちゃったよ
\\	まってたしかにこのなかにいれたはずなんだ。。。ほらあった
\\	これでしょ、これ、はい
\\	はっ
\\	やべー、しまった、オカンのパスポートもってきちゃったよ
\\	まってたしかにこのなかにいれたはずなんだ。。。ほらあった
\\	これでしょ、これ、はい
\\	はい、たしかに。それではおにもつのほうなんですが、、、みっつ(3つ)でよろしいですか?
\\	はい、たしかに。それではおにもつのほうなんですが、、、みっつ(3つ)でよろしいですか?
\\	えーっとはい。
\\	えーっとはい。
\\	おそれいりますおきゃくさま。とうきでのおひとりさまのさいだいせきさいじゅうりょうはさんじゅっきろ
\\	となっております。おきゃくさまのおにもつはろくじゅっきろ
\\	。ちょうかりょうきんとして10まんえんいただくことになりますがよろしいですか?
\\	おそれいりますおきゃくさま。とうきでのおひとりさまのさいだいせきさいじゅうりょうはさんじゅっきろ
\\	となっております。おきゃくさまのおにもつはろくじゅっきろ
\\	。ちょうかりょうきんとして10まんえんいただくことになりますがよろしいですか?
\\	しょうがない。。。はいだいじょうぶですよ。
\\	しょうがない。。。はいだいじょうぶですよ。
\\	はい、ではおきゃくさま、しゅっぱつのじゅんびはすべてととのいました。よんじゅうご
\\	ばんゲートにおいそぎください
\\	はい、ではおきゃくさま、しゅっぱつのじゅんびはすべてととのいました。よんじゅうご
\\	ばんゲートにおいそぎください
\\	きゅうくつ
\\	たまらない
\\	ぜいかん
\\	ぬける
\\	しんどい
\\	ひからびる
\\	あきる
\\	いきなり
\\	ゲートよんじゅうご
\\	、ゲート45・・・どこだぁ?
\\	。。。あ、あった
\\	ゲートよんじゅうご
\\	、ゲート45・・・どこだぁ?
\\	。。。あ、あった
\\	やまださまですか?おいそぎください、とうきはていこくをすぎております。
\\	やまださまですか?おいそぎください、とうきはていこくをすぎております。
\\	(ひこうきしゅっぱつ)はぁ。なんとかまにあった。パッキングにはいろいろとてまどったからな。さて、ちょっとつかれたし、ひとねむりしよ。
\\	(ひこうきしゅっぱつ)はぁ。なんとかまにあった。パッキングにはいろいろとてまどったからな。さて、ちょっとつかれたし、ひとねむりしよ。
\\	(バンクーバーちゃく)ああああぁぁぁあ(あくび)、やっとバンクーバーについた。エコノミークラスはきゅうくつでたまらんなぁ。まぁ、なんとかぜいかんもすぐにぬけられたし、あんしんあんしん。
\\	(バンクーバーちゃく)ああああぁぁぁあ(あくび)、やっとバンクーバーについた。エコノミークラスはきゅうくつでたまらんなぁ。まぁ、なんとかぜいかんもすぐにぬけられたし、あんしんあんしん。
\\	あっ
\\	わすれてた、はやくトランクあけないとしんでしまう
\\	オカン、ついたよ
\\	バンクーバー
\\	あっ
\\	わすれてた、はやくトランクあけないとしんでしまう
\\	オカン、ついたよ
\\	バンクーバー
\\	(トランクから、おかあさんがでてくる。)イヤーしんどかった。。。かた、こしがいたい。。トランクのなかにじゅう
\\	じかんて、どんだけー
\\	(トランクから、おかあさんがでてくる。)イヤーしんどかった。。。かた、こしがいたい。。トランクのなかにじゅう
\\	じかんて、どんだけー
\\	トランクのなか、あついし、はらへるし、とちゅうでみずなしくなるし、ひからびるかとおもった。
\\	トランクのなか、あついし、はらへるし、とちゅうでみずなしくなるし、ひからびるかとおもった。
\\	けど、
\\	とにんてんどう
\\	をいっしょにいれといてよかった。あきずにすんだわ。それにしてもバンクーバーいちにちめ、いきなりあめかい。
\\	けど、
\\	とにんてんどう
\\	をいっしょにいれといてよかった。あきずにすんだわ。それにしてもバンクーバーいちにちめ、いきなりあめかい。
\\	けらい
\\	こよい
\\	ばくふ
\\	かいこく
\\	じょうしゅ
\\	わし
\\	ふんどし
\\	かしきる
\\	せいだいに
\\	との
\\	とのぉ
\\	おたんじょうび、おめでとうございます
\\	こんにちは11がつ23にち。マーキーさまの50さいのたんじょうびですぞ
\\	こうしてはおられません。せいだいにとののおたんじょうびをいわいましょう。
\\	とのぉ
\\	おたんじょうび、おめでとうございます
\\	こんにちは11がつ23にち。マーキーさまの50さいのたんじょうびですぞ
\\	こうしてはおられません。せいだいにとののおたんじょうびをいわいましょう。
\\	そうよのー、なにかクレージーなことをせねばの。
\\	そうよのー、なにかクレージーなことをせねばの。
\\	なにか、やりたいこと、ほしいものはありますか?
\\	なにか、やりたいこと、ほしいものはありますか?
\\	それでは、ひめじじょうをかしきってしまおう。
\\	それでは、ひめじじょうをかしきってしまおう。
\\	。。。。。えっ?ひめじじょうをですか?なにをいっってるんですか、との?
\\	。。。。。えっ?ひめじじょうをですか?なにをいっってるんですか、との?
\\	いや、だから、ひめじじょうでパーティーをせねば。ふんどしで。ひめじじょうじょうしゅ、さかいただてるにはワシからもうしておく。
\\	いや、だから、ひめじじょうでパーティーをせねば。ふんどしで。ひめじじょうじょうしゅ、さかいただてるにはワシからもうしておく。
\\	ひめじじょうで。ふんどしで。との、とのはさいきんいそがしくおつかれのようす。どうやらねつでもだされているのでしょう。すこしおやすみになられたほうがよいのではないかと。。。
\\	ひめじじょうで。ふんどしで。との、とのはさいきんいそがしくおつかれのようす。どうやらねつでもだされているのでしょう。すこしおやすみになられたほうがよいのではないかと。。。
\\	ワシはつかれてなどいないぞよ。ワシはほんきだ。あ、そうだ。アメリカからきているペリーくんもよんでしまおう。かれはアメリカじんだからパーティはなれておるはず。きっとたのしくなるぞー
\\	ワシはつかれてなどいないぞよ。ワシはほんきだ。あ、そうだ。アメリカからきているペリーくんもよんでしまおう。かれはアメリカじんだからパーティはなれておるはず。きっとたのしくなるぞー
\\	(パーティー 
\\	ひめじじょう)
\\	(パーティー 
\\	ひめじじょう)
\\	どうだ、けらい、たのしいだろ
\\	どうだ、けらい、たのしいだろ
\\	はい、だいへんたのしゅうございます
\\	はい、だいへんたのしゅうございます
\\	あ、さかいどの、こよいはおしろをかしていただきまことにありがたい。
\\	あ、さかいどの、こよいはおしろをかしていただきまことにありがたい。
\\	あ、ペリーくん
\\	きてたのかね
\\	どうだ、これがにっぽんのパーティーだ。すごいだろ?
\\	あ、ペリーくん
\\	きてたのかね
\\	どうだ、これがにっぽんのパーティーだ。すごいだろ?
\\	しんさつ
\\	こみあう
\\	い
\\	だんぞくてき
\\	はきけ
\\	げり
\\	しょうじょう
\\	しょくよく
\\	のど
\\	はい、こちらきたやまクリニックです。
\\	はい、こちらきたやまクリニックです。
\\	ほんじつうかがいたいのですが、しんさつじかんはなんじまでですか?
\\	ほんじつうかがいたいのですが、しんさつじかんはなんじまでですか?
\\	ほんじつはどようびですので、しんさつのうけつけはごご1ときまでとなっております。どようびはこみあいますので、おはやめにおいでください。
\\	ほんじつはどようびですので、しんさつのうけつけはごご1ときまでとなっております。どようびはこみあいますので、おはやめにおいでください。
\\	わかりました。ありがとうございます。
\\	わかりました。ありがとうございます。
\\	(まちあいしつ)
\\	(まちあいしつ)
\\	むらかみさん、むらかみアグネスさーん。
\\	むらかみさん、むらかみアグネスさーん。
\\	あ、はい。
\\	あ、はい。
\\	しんさつしつへどうぞ。
\\	しんさつしつへどうぞ。
\\	むらかみさん、ほんじつはどうされました?
\\	むらかみさん、ほんじつはどうされました?
\\	あのう、きのうのよるからいがいたくて…。
\\	あのう、きのうのよるからいがいたくて…。
\\	ふむ。どんないたみですか?しくしくいたいとか、しめづけられるようにいたいとか…。
\\	ふむ。どんないたみですか?しくしくいたいとか、しめづけられるようにいたいとか…。
\\	ええ、だんぞくてきにキリキリいたみます。
\\	ええ、だんぞくてきにキリキリいたみます。
\\	はきけやげりなど、ほかにしょうじょうはありますか?
\\	はきけやげりなど、ほかにしょうじょうはありますか?
\\	いえ、とくにありません。
\\	いえ、とくにありません。
\\	えーと、きのうなにかかわったものをたべたりしましたか。
\\	えーと、きのうなにかかわったものをたべたりしましたか。
\\	いいえ、きのうはほとんどなにもたべていないんです。
\\	いいえ、きのうはほとんどなにもたべていないんです。
\\	おや、いたくてたべられなかったんですか?
\\	おや、いたくてたべられなかったんですか?
\\	いえ、しごとでちょっとトラブルがあったもので、しょくよくがなくて。だいじなプレゼンのにちだったんですけど、ちこくしちゃったんです。しょくばのにんにめいわくをかけたうえに、そのしごとはけっきょくけいやくがとれなくて…ほとんどてつやでがんばったのに、むだになっちゃったとおもったら、もうなにものどをとうらなくて。
\\	いえ、しごとでちょっとトラブルがあったもので、しょくよくがなくて。だいじなプレゼンのにちだったんですけど、ちこくしちゃったんです。しょくばのにんにめいわくをかけたうえに、そのしごとはけっきょくけいやくがとれなくて…ほとんどてつやでがんばったのに、むだになっちゃったとおもったら、もうなにものどをとうらなくて。
\\	がっかい
\\	けんきゅうはっぴょう
\\	きんむ
\\	しんせんな
\\	かき
\\	ようだい
\\	かゆ
\\	(けいたいでんわがなる)
\\	(けいたいでんわがなる)
\\	はい、もしもし。
\\	はい、もしもし。
\\	もしもし?はしもとだけど、アグネスさん、きのうでんわくれた?
\\	もしもし?はしもとだけど、アグネスさん、きのうでんわくれた?
\\	うん、5かいくらいかけた。
\\	うん、5かいくらいかけた。
\\	ゴメンゴメン、このしゅうまつはがっかいでひろしまにきているんだ。
\\	ゴメンゴメン、このしゅうまつはがっかいでひろしまにきているんだ。
\\	ふーん。
\\	ふーん。
\\	きのうもいちにちけんきゅうはっぴょうがあって…それさえなければすぐれんらくできたんだけど、なにかだいじなようじだった?
\\	きのうもいちにちけんきゅうはっぴょうがあって…それさえなければすぐれんらくできたんだけど、なにかだいじなようじだった?
\\	もうようじはすんだからいいの。
\\	もうようじはすんだからいいの。
\\	えっ、なにそれ?
\\	えっ、なにそれ?
\\	ゆうすけさん、びょういんきんむだっていってたでしょ。だからおいしゃさんをしょうかいしてもらおうかとおもったの。でもれんらくがとれないから、ちかくのクリニックにいったんだ。
\\	ゆうすけさん、びょういんきんむだっていってたでしょ。だからおいしゃさんをしょうかいしてもらおうかとおもったの。でもれんらくがとれないから、ちかくのクリニックにいったんだ。
\\	いしゃ?どうしたの?
\\	いしゃ?どうしたの?
\\	いがいたいの。ストレスせいいえんだって。くすりをのんだら、だいぶよくなったけど、きょうはいちにちいえにいるわ。
\\	いがいたいの。ストレスせいいえんだって。くすりをのんだら、だいぶよくなったけど、きょうはいちにちいえにいるわ。
\\	いえん?そうか、タイミングわるいなあ。
\\	いえん?そうか、タイミングわるいなあ。
\\	なにが?
\\	なにが?
\\	いや、ひろしまからしんせんなかきをおくろうかとおもったんだけど、いによくないよね。おくるまえにれんらくできてよかったよ。
\\	いや、ひろしまからしんせんなかきをおくろうかとおもったんだけど、いによくないよね。おくるまえにれんらくできてよかったよ。
\\	かき
\\	かき
\\	あ、いや、アグネスさんのようだいもしんぱいだよ。だいじょうぶ?
\\	あ、いや、アグネスさんのようだいもしんぱいだよ。だいじょうぶ?
\\	もういいよ…またね。
\\	もういいよ…またね。
\\	(でんわきれる)
\\	(でんわきれる)
\\	あ、アグネスさん
\\	あ、アグネスさん
\\	ああーくやしい
\\	いさえいたくなければかきはだいこうぶつなのに
\\	しょうがない、おかゆでもたべるか…。
\\	ああーくやしい
\\	いさえいたくなければかきはだいこうぶつなのに
\\	しょうがない、おかゆでもたべるか…。
\\	かおいろ
\\	きがおもい
\\	ぎょかいるい
\\	すみびやき
\\	とりひきさき
\\	しんよう
\\	ちょくそう
\\	おはようございます。
\\	おはようございます。
\\	おはようございまーす。あ、アグネスさん、しゃちょうがおひるまえにしゃちょうしつへくるように、って。だいじょうぶ?かおいろわるいよ。
\\	おはようございまーす。あ、アグネスさん、しゃちょうがおひるまえにしゃちょうしつへくるように、って。だいじょうぶ?かおいろわるいよ。
\\	えー、なんだろう…まだいもいたいのに、きがおもいなあ。
\\	えー、なんだろう…まだいもいたいのに、きがおもいなあ。
\\	せんしゅう、たいへんだったもんね。でもしゃちょう、そんなにおこってなかったよ。
\\	せんしゅう、たいへんだったもんね。でもしゃちょう、そんなにおこってなかったよ。
\\	(ノックのおと)
\\	(ノックのおと)
\\	しつれいします。
\\	しつれいします。
\\	むらかみくんか。はいりなさい。
\\	むらかみくんか。はいりなさい。
\\	しゃちょう、せんしゅうのプレゼンはほんとうにもうしわけありませんでした…って、なになさってるんですか?
\\	しゃちょう、せんしゅうのプレゼンはほんとうにもうしわけありませんでした…って、なになさってるんですか?
\\	なにってきみ、みればわかるだろう。しんせんなぎょかいるいがてにはいったから、きょうはすみびやきにしているんだ。まあすわりなさい。きみならこのあじがわかるだろう。
\\	なにってきみ、みればわかるだろう。しんせんなぎょかいるいがてにはいったから、きょうはすみびやきにしているんだ。まあすわりなさい。きみならこのあじがわかるだろう。
\\	しゃちょう、じつはわたし、きょうはいが…
\\	しゃちょう、じつはわたし、きょうはいが…
\\	いやー、せんしゅうはゴリのけんでだいぶむりをさせたねえ。あいつはむかしからぼうじゃくぶじんなところがあってね。
\\	いやー、せんしゅうはゴリのけんでだいぶむりをさせたねえ。あいつはむかしからぼうじゃくぶじんなところがあってね。
\\	はあ…あ、いえ。
\\	はあ…あ、いえ。
\\	まあでも、このせかいもじゃくにくきょうしょくだからなあ。ビジネスにはごういんさもひつようなんだ。しかし、それでとりひきさきのしんようをうしなったらほんまつてんとうだろう、そうはおもわないか、きみ?
\\	まあでも、このせかいもじゃくにくきょうしょくだからなあ。ビジネスにはごういんさもひつようなんだ。しかし、それでとりひきさきのしんようをうしなったらほんまつてんとうだろう、そうはおもわないか、きみ?
\\	あ、はい、そうおもいます。
\\	あ、はい、そうおもいます。
\\	うんうん、そうなんだ。こんどあいつにはよくいっておかないと。あ、ちょうどかきがやけたぞ。ほら、たべなさい
\\	ひろしまからちょくそうだ
\\	うんうん、そうなんだ。こんどあいつにはよくいっておかないと。あ、ちょうどかきがやけたぞ。ほら、たべなさい
\\	ひろしまからちょくそうだ
\\	ポスター
\\	ためしに
\\	はいけい
\\	くれーむ
\\	ボランティアかつどう
\\	ねえねえ、アグネスさん、ちょっとそうだんしていい?
\\	ねえねえ、アグネスさん、ちょっとそうだんしていい?
\\	ええ、なんですか?
\\	ええ、なんですか?
\\	あのね、このポスターなんだけど、クライアントに、ぜんたいてきにくらいっていわれちゃったんだけど。どうおもう?
\\	あのね、このポスターなんだけど、クライアントに、ぜんたいてきにくらいっていわれちゃったんだけど。どうおもう?
\\	うーん…この、きのいろをもうすこしあかるいかんじにするとか。
\\	うーん…この、きのいろをもうすこしあかるいかんじにするとか。
\\	ぼくもそうおもって、ためしにやってみたんだけど、そうするとすごくうそっぽくなっちゃうの。ほら。
\\	ぼくもそうおもって、ためしにやってみたんだけど、そうするとすごくうそっぽくなっちゃうの。ほら。
\\	あー、ホントだ。なんか、へいめんてきですねえ。
\\	あー、ホントだ。なんか、へいめんてきですねえ。
\\	でしょう?でも、きほんてきなデザインはかえられないんだよね。
\\	でしょう?でも、きほんてきなデザインはかえられないんだよね。
\\	このあかるいいろのまま、かげをこくしたらこどもっぽいかんじがなくなりませんか?
\\	このあかるいいろのまま、かげをこくしたらこどもっぽいかんじがなくなりませんか?
\\	あ、そうか。でもそうすると、このはいけいとあわないかんじなんだよね。ここをもうすこしきいろっぽいいろにしてみようか。
\\	あ、そうか。でもそうすると、このはいけいとあわないかんじなんだよね。ここをもうすこしきいろっぽいいろにしてみようか。
\\	あ、いいかんじじゃないですか
\\	あ、いいかんじじゃないですか
\\	じゃあ、これでだしてみるか。あーあ、このかいしゃっていつもこんなかんじなんだよね。
\\	じゃあ、これでだしてみるか。あーあ、このかいしゃっていつもこんなかんじなんだよね。
\\	クレームがぐたいてきじゃないですよね。
\\	クレームがぐたいてきじゃないですよね。
\\	そうなんだよ
\\	このあいだも、ボランティアかつどうのポスターを、「もうすこしじょうねつてきなかんじでおねがいします」とかいわれてさ。
\\	そうなんだよ
\\	このあいだも、ボランティアかつどうのポスターを、「もうすこしじょうねつてきなかんじでおねがいします」とかいわれてさ。
\\	うわー、そういうの、いちばんこまりますよねー。
\\	うわー、そういうの、いちばんこまりますよねー。
\\	このところ
\\	るすがち
\\	いとこ
\\	こそだて
\\	いいとしして
\\	なげかわしい
\\	ていねん
\\	おちつく
\\	(でんわがなる)
\\	(でんわがなる)
\\	はい、むらかみです。
\\	はい、むらかみです。
\\	アグネス?
\\	アグネス?
\\	あ、おかあさん。ひさしぶりねー。
\\	あ、おかあさん。ひさしぶりねー。
\\	まったくこのこは、ちっともれんらくしてこないで。しんぱいしていたのよ。
\\	まったくこのこは、ちっともれんらくしてこないで。しんぱいしていたのよ。
\\	ごめんごめん、このところちょっとしごとがいそがしくて、るすがちだったから。
\\	ごめんごめん、このところちょっとしごとがいそがしくて、るすがちだったから。
\\	そうだったのね。まだいそがしいの?
\\	そうだったのね。まだいそがしいの?
\\	もうそれほどいそがしくないからだいじょうぶよ。
\\	もうそれほどいそがしくないからだいじょうぶよ。
\\	ところでアグネス、あなたまだしらないでしょ?あなたのいとこのゆみちゃん、こんどこどもがうまれるのよ。
\\	ところでアグネス、あなたまだしらないでしょ?あなたのいとこのゆみちゃん、こんどこどもがうまれるのよ。
\\	ええー
\\	ゆみちゃんまだわかいのに、もうママになっちゃうの?
\\	ええー
\\	ゆみちゃんまだわかいのに、もうママになっちゃうの?
\\	なにをいっているの。かのじょだってもう24さいよ。
\\	なにをいっているの。かのじょだってもう24さいよ。
\\	じゅうぶんわかいわよ。たしかけっこんしたときはまだがくせいだったのよね。
\\	じゅうぶんわかいわよ。たしかけっこんしたときはまだがくせいだったのよね。
\\	そうよ。おねえさんのゆきちゃんにはもうこどもがいるし、ゆみちゃんはこそだてにかんしてはしんぱいはいらないわね。それにくらべてうちは…。ちょっとアグネス、あなたまだけっこんしないの?もう30さいでしょう?
\\	そうよ。おねえさんのゆきちゃんにはもうこどもがいるし、ゆみちゃんはこそだてにかんしてはしんぱいはいらないわね。それにくらべてうちは…。ちょっとアグネス、あなたまだけっこんしないの?もう30さいでしょう?
\\	まだ29さいよ
\\	それに、いきなりけっこんってなによー。まだかれしだっていないのに。
\\	まだ29さいよ
\\	それに、いきなりけっこんってなによー。まだかれしだっていないのに。
\\	あーあ。いいとししてそれじゃあねえ…なげかわしい。さっきもおとうさんと、わたしたちももうわかくないし、そろそろまごのかおがみたいわねえ、なんてはなしてたところなのに。
\\	あーあ。いいとししてそれじゃあねえ…なげかわしい。さっきもおとうさんと、わたしたちももうわかくないし、そろそろまごのかおがみたいわねえ、なんてはなしてたところなのに。
\\	ちょっとおかあさん、きがはやすぎ。
\\	ちょっとおかあさん、きがはやすぎ。
\\	あなた、もうこちらにはかえらないでずっととうきょうにいるつもりなの?
\\	あなた、もうこちらにはかえらないでずっととうきょうにいるつもりなの?
\\	そんなのまだわからないわよ。
\\	そんなのまだわからないわよ。
\\	はやくおちついて、おとうさんとおかあさんをあんしんさせてちょうだい。おとうさんももうていねんがちかいのよ。
\\	はやくおちついて、おとうさんとおかあさんをあんしんさせてちょうだい。おとうさんももうていねんがちかいのよ。
\\	あー、はいはい。ようじはそれだけ?もうきるわよー。
\\	あー、はいはい。ようじはそれだけ?もうきるわよー。
\\	おとそ
\\	おせちりょうり
\\	えんぎもの
\\	マメに
\\	くろまめ
\\	ながいき
\\	めあて
\\	ぽちぶくろ
\\	うすい
\\	おしょうがつ
\\	おしょうがつ
\\	さぁ、みんなそろったかな。きょうからあたらしい1ねんがはじまるぞ。おとそでおいわいしよう。
\\	さぁ、みんなそろったかな。きょうからあたらしい1ねんがはじまるぞ。おとそでおいわいしよう。
\\	さぁ、どうぞ。おじいちゃんから。
\\	さぁ、どうぞ。おじいちゃんから。
\\	しんねんあけましておめでとうございます
\\	しんねんあけましておめでとうございます
\\	しんねんおめでとうございます
\\	しんねんおめでとうございます
\\	しんねんおめでとうございます
\\	しんねんおめでとうございます
\\	おとそ、ぼくものんでみたいなぁー
\\	おとそ、ぼくものんでみたいなぁー
\\	おっ
\\	だいちものんでみるか。ちょっとだけだぞ。
\\	おっ
\\	だいちものんでみるか。ちょっとだけだぞ。
\\	ごく。。うぁ
\\	ん
\\	ちょっとへんなあじだけど、あまいから、もっとのみたい
\\	ごく。。うぁ
\\	ん
\\	ちょっとへんなあじだけど、あまいから、もっとのみたい
\\	だめだめ、あまくたっておさけなんだから
\\	さぁさぁ、おせちりょうりたべましょう
\\	だめだめ、あまくたっておさけなんだから
\\	さぁさぁ、おせちりょうりたべましょう
\\	まぁ、こんなにいっぱいつくるの、たいへんだったでしょ。
\\	まぁ、こんなにいっぱいつくるの、たいへんだったでしょ。
\\	おせちりょうりはえんぎものだからね。1ねんかんマメにはたらきけんこうにくらせるようにくろまめたべてよ。
\\	おせちりょうりはえんぎものだからね。1ねんかんマメにはたらきけんこうにくらせるようにくろまめたべてよ。
\\	じゃぁ、おじいちゃんとおばあちゃんは、このえびたべてながいきしてね。
\\	じゃぁ、おじいちゃんとおばあちゃんは、このえびたべてながいきしてね。
\\	やさしいねぇー、このこは。。。ううう。。。
\\	やさしいねぇー、このこは。。。ううう。。。
\\	だって、おばあちゃんとおじいちゃんから、ずーっとおとしだまもらいたいんだもん
\\	だって、おばあちゃんとおじいちゃんから、ずーっとおとしだまもらいたいんだもん
\\	なんだ、めあてはおとしだまかい
\\	しょうがないねぇ。はい、おとしだま。はい、おにいちゃん。はい、だいち。
\\	なんだ、めあてはおとしだまかい
\\	しょうがないねぇ。はい、おとしだま。はい、おにいちゃん。はい、だいち。
\\	ありがとう
\\	ありがとう
\\	ねぇ、おにいちゃんのぽちぶくろさわらせて
\\	あれ、ぼくのよりうすいよ
\\	わーい、ぼくのほうがいっぱいはいってるんだ。わーい、わーい。
\\	ねぇ、おにいちゃんのぽちぶくろさわらせて
\\	あれ、ぼくのよりうすいよ
\\	わーい、ぼくのほうがいっぱいはいってるんだ。わーい、わーい。
\\	けいさつ
\\	ぼうそうぞく
\\	おう
\\	ふりきる
\\	はつひので
\\	まにあう
\\	ぶっとばす
\\	つかまる
\\	おとしだま
\\	そこのバイクいますぐとまりなさい
\\	そこのバイクいますぐとまりなさい
\\	もっとはやくはしってよ
\\	もっとはやくはしってよ
\\	わかってるよ、けど、ただ、ぼうそうぞくじゃないのになんでけいさつがおってくるんだよ〜。ぼうそうぞくはほかにいっぱいいるだろ、なんでおれなんだよー。しょうがない、こうなったら。
\\	わかってるよ、けど、ただ、ぼうそうぞくじゃないのになんでけいさつがおってくるんだよ〜。ぼうそうぞくはほかにいっぱいいるだろ、なんでおれなんだよー。しょうがない、こうなったら。
\\	は、はやいっ
\\	は、はやいっ
\\	ふぅ、なんとかふりきった。。。。
\\	ふぅ、なんとかふりきった。。。。
\\	あ、ひのでまでまにあわないじゃない。はやくして
\\	あ、ひのでまでまにあわないじゃない。はやくして
\\	よーし、かわぐちこまでぶっとばすぞ、しっかりつかまってろよ。
\\	よーし、かわぐちこまでぶっとばすぞ、しっかりつかまってろよ。
\\	ふじさんにいそぐ
\\	ふじさんにいそぐ
\\	よし、なんとかまにあった。
\\	よし、なんとかまにあった。
\\	うわぁ、はつひので、きれいねぇー。。。。。
\\	うわぁ、はつひので、きれいねぇー。。。。。
\\	(バイクがうごきだす。)
\\	(バイクがうごきだす。)
\\	あれ?おれのバイクがないぞ。。。
\\	あれ?おれのバイクがないぞ。。。
\\	、、、、あれ、、、、バイクの上、、、、
\\	、、、、あれ、、、、バイクの上、、、、
\\	ああああ
\\	サルがおれのバイクをうんてんしてる。。。。あれ、もしかしてあれは、、、、たろうじゃないか?
\\	アイツいつのまにバイクのうんてんができるようになったんだ。。。。しょうがない、そのバイクはたろうにおとしだまだ。あけましておめでとう、な、たろう
\\	ああああ
\\	サルがおれのバイクをうんてんしてる。。。。あれ、もしかしてあれは、、、、たろうじゃないか?
\\	アイツいつのまにバイクのうんてんができるようになったんだ。。。。しょうがない、そのバイクはたろうにおとしだまだ。あけましておめでとう、な、たろう
\\	おんしゃ
\\	こくさいがっかい
\\	おそれいる
\\	たんとう
\\	そうむか
\\	じんじか
\\	さしあげる
\\	あいにく
\\	おりかえし
\\	たいへんおまたせいたしました。やがみでございます。
\\	たいへんおまたせいたしました。やがみでございます。
\\	わたし、さっぽろどうのまつだいらともうします。
\\	わたし、さっぽろどうのまつだいらともうします。
\\	さっぽろどうのまつだいらさまですね。たんとうのものとかわりますのでしょうしょうおまちください。
\\	さっぽろどうのまつだいらさまですね。たんとうのものとかわりますのでしょうしょうおまちください。
\\	もうしわけございません。あいにくたんとうのものはがいしゅつしております。
\\	もうしわけございません。あいにくたんとうのものはがいしゅつしております。
\\	では、おてすうですがおりかえしおでんわをいただけますでしょうか。
\\	では、おてすうですがおりかえしおでんわをいただけますでしょうか。
\\	たしかにもうしつたえます。わたし、じんじかのやがみがうけたまわりました。
\\	たしかにもうしつたえます。わたし、じんじかのやがみがうけたまわりました。
\\	それでは、よろしくおねがいいたします。
\\	それでは、よろしくおねがいいたします。
\\	しつれいいたします。
\\	しつれいいたします。
\\	はけん
\\	きょうしゅく
\\	めいし
\\	しょうさい
\\	けんきゅうしゃ
\\	どうじつうやく
\\	しゅっせきしゃ
\\	すみません。わたし、さっぽろどうのまつだいらともうします。じんじぶのやまもとさまと3じにうちあわせのやくそくをしているのですが。
\\	すみません。わたし、さっぽろどうのまつだいらともうします。じんじぶのやまもとさまと3じにうちあわせのやくそくをしているのですが。
\\	さっぽろどうのまつだいらさまですね。じんじぶは3かいでございます。あちらのエレベーターをごりようください。
\\	さっぽろどうのまつだいらさまですね。じんじぶは3かいでございます。あちらのエレベーターをごりようください。
\\	たいへんおまたせいたしました。わたし、じんじぶ・はけんたんとうのやまもとじろうともうします。(めいしをさしだす)
\\	たいへんおまたせいたしました。わたし、じんじぶ・はけんたんとうのやまもとじろうともうします。(めいしをさしだす)
\\	これはきょうしゅくです。わたし、さっぽろどうのまつだいらひろしともうします。(めいしをさしだす)このたびはおいそがしいなか、おじかんをとっていただきありがとうございます。コンコン(かいぎしつをノックする)
\\	これはきょうしゅくです。わたし、さっぽろどうのまつだいらひろしともうします。(めいしをさしだす)このたびはおいそがしいなか、おじかんをとっていただきありがとうございます。コンコン(かいぎしつをノックする)
\\	ではさっそく、しょうさいについておしえていただけますでしょうか。
\\	ではさっそく、しょうさいについておしえていただけますでしょうか。
\\	はい。きたる2008ねん2がつ26、27にちにこくさいシンポジウムをかいさいいたします。
\\	はい。きたる2008ねん2がつ26、27にちにこくさいシンポジウムをかいさいいたします。
\\	シンガポール、フランス、イギリス、からけんきゅうしゃをまねき、にほんじんけんきゅうしゃをまじえセッションをおこうよていになっております。
\\	シンガポール、フランス、イギリス、からけんきゅうしゃをまねき、にほんじんけんきゅうしゃをまじえセッションをおこうよていになっております。
\\	つきましては、どうじつうやくがひつようとなり、おんしゃにどうじつうやくしゃのはけんをおねがいいたしたくぞんじます。
\\	つきましては、どうじつうやくがひつようとなり、おんしゃにどうじつうやくしゃのはけんをおねがいいたしたくぞんじます。
\\	なんかこくごのつうやくしゃがひつようなのでしょうか。
\\	なんかこくごのつうやくしゃがひつようなのでしょうか。
\\	きほんてきにはえいご、フランスごになります。
\\	きほんてきにはえいご、フランスごになります。
\\	しかい
\\	ぐたいてき
\\	ちょうしゅう
\\	きざい
\\	てはい
\\	こうそくじかん
\\	みつもり
\\	りょうきん
\\	はい。フランスから1めい、シンガポールから1めい、イギリスから2めいのよていです。それと、にほんじんけんきゅうしゃが3めいとしかいのかたが1めいです。
\\	はい。フランスから1めい、シンガポールから1めい、イギリスから2めいのよていです。それと、にほんじんけんきゅうしゃが3めいとしかいのかたが1めいです。
\\	ぜんぶで8めいですね。かいじょうぜんたいではどうでしょうか?
\\	ぜんぶで8めいですね。かいじょうぜんたいではどうでしょうか?
\\	100めいから200めいていどだとおもいます。
\\	100めいから200めいていどだとおもいます。
\\	かいじょうぜんたいのおおよそのさんかしゃすうは、よりぐたいてきにおしえていただけますか。ごよういするちょうしゅうようのイヤホンなどもかずがちがってきますので。
\\	かいじょうぜんたいのおおよそのさんかしゃすうは、よりぐたいてきにおしえていただけますか。ごよういするちょうしゅうようのイヤホンなどもかずがちがってきますので。
\\	はい、わかりました。
\\	はい、わかりました。
\\	どうじつうやくきざいのてはいはにんずうによってかわってきますのでよろしくおねがいいたします。
\\	どうじつうやくきざいのてはいはにんずうによってかわってきますのでよろしくおねがいいたします。
\\	かいぎのながさや、おもなないようなどもおおしえください。つうやくしゃもこうそくじかんによってりょうきんがちがいますので。
\\	かいぎのながさや、おもなないようなどもおおしえください。つうやくしゃもこうそくじかんによってりょうきんがちがいますので。
\\	はい。わかりました。しょうさいはすぐに
\\	いたします。それにめをとおしていただき、おみつもりのほうをおねがいできますでしょうか。
\\	はい。わかりました。しょうさいはすぐに
\\	いたします。それにめをとおしていただき、おみつもりのほうをおねがいできますでしょうか。
\\	わかりました。
\\	わかりました。
\\	みつもりしょ
\\	よさん
\\	もろもろ
\\	しょけいひ
\\	はいけん
\\	さいこう
\\	せってい
\\	じっせき
\\	まつだいらさん、
\\	きてるよ。
\\	まつだいらさん、
\\	きてるよ。
\\	ほーい。どれどれ。
\\	ほーい。どれどれ。
\\	うーん、まいったなあ。かんぜんによさんオーバーだわ。
\\	うーん、まいったなあ。かんぜんによさんオーバーだわ。
\\	どうじつうやくってこんなにたかいのね。
\\	どうじつうやくってこんなにたかいのね。
\\	しょけいひがすごいなあ。
\\	しょけいひがすごいなあ。
\\	もしもし。やまもとさまでいらっしゃいますか。
\\	もしもし。やまもとさまでいらっしゃいますか。
\\	はい。やまもとです。
\\	はい。やまもとです。
\\	さっぽろどうのまつだいらでございます。
\\	さっぽろどうのまつだいらでございます。
\\	さきほど、おくっていただいたみつもりしょのほうをはいけんさせていただきました。
\\	さきほど、おくっていただいたみつもりしょのほうをはいけんさせていただきました。
\\	はい。それで、いかがでしょうか。
\\	はい。それで、いかがでしょうか。
\\	はい。もろもろのしょけいひをふくめますと、じゃっかんよさんオーバーぎみですので、もういちどしょうさいのほうをさいこういたしまして、さいていしゅつさせていただきたいのですが。
\\	はい。もろもろのしょけいひをふくめますと、じゃっかんよさんオーバーぎみですので、もういちどしょうさいのほうをさいこういたしまして、さいていしゅつさせていただきたいのですが。
\\	わかりました。
\\	わかりました。
\\	あと、どうじつうやくのりょうきんなのですが、4じかんをこえますとりょうきんのほうはいちにちぶんとなるわけですよね。
\\	あと、どうじつうやくのりょうきんなのですが、4じかんをこえますとりょうきんのほうはいちにちぶんとなるわけですよね。
\\	はい。そうです。りょうきんにかんしましては、ややたかくせっていさせていただいておりますが、わがしゃのつうやくしゃはじっせきもじつりょくももうしぶんないとじしんをもっていわせていただきます。
\\	はい。そうです。りょうきんにかんしましては、ややたかくせっていさせていただいておりますが、わがしゃのつうやくしゃはじっせきもじつりょくももうしぶんないとじしんをもっていわせていただきます。
\\	はあ。わかりました。だいじなこくさいかいぎですので、どうぞよろしくおねがいいたします。
\\	はあ。わかりました。だいじなこくさいかいぎですので、どうぞよろしくおねがいいたします。
\\	(こりゃあ、せんぽうとよさんくみなおしかもな・・・とほほ)
\\	(こりゃあ、せんぽうとよさんくみなおしかもな・・・とほほ)
\\	けんこう
\\	きんえん
\\	ちょうせん
\\	しっぱい
\\	すう
\\	きょうはどうしましたか?
\\	きょうはどうしましたか?
\\	せんせい、わたし、けんこうのためにきんえんしたいんです。
\\	せんせい、わたし、けんこうのためにきんえんしたいんです。
\\	それなら、きんえんしたらいいんじゃないかね。
\\	それなら、きんえんしたらいいんじゃないかね。
\\	ええ、でも、14かいもきんえんにちょうせんしたんですがすべてしっぱいしました。タバコをすわないようにしようとおもえばおもうほど、タバコをすいたくなってくるんです。どうしたらいいでしょう?
\\	ええ、でも、14かいもきんえんにちょうせんしたんですがすべてしっぱいしました。タバコをすわないようにしようとおもえばおもうほど、タバコをすいたくなってくるんです。どうしたらいいでしょう?
\\	それじゃきんえんのガムを。。
\\	それじゃきんえんのガムを。。
\\	それはもうやりました
\\	きんえんガム、きんえんフィルター、すべてやりました。ですがだめなんです。。。
\\	それはもうやりました
\\	きんえんガム、きんえんフィルター、すべてやりました。ですがだめなんです。。。
\\	よしわかった。では、これをためしてみよう。ベッドによこになってごらん。
\\	よしわかった。では、これをためしてみよう。ベッドによこになってごらん。
\\	はい。。
\\	はい。。
\\	それではこのコインをじーっとみなさい。
\\	それではこのコインをじーっとみなさい。
\\	ほーら、だんだんねむたくなってくる・・だんだんきもちよくなってくる・・あなたはタバコをすいたくなくなってくる・・・
\\	ほーら、だんだんねむたくなってくる・・だんだんきもちよくなってくる・・あなたはタバコをすいたくなくなってくる・・・
\\	どんぞこ
\\	そううつ
\\	そめる
\\	ふくよう
\\	ふくさよう
\\	どうき
\\	しびれる
\\	どうされましたか。
\\	どうされましたか。
\\	こころがあんていしないんです。そわそわしているっていうか。
\\	こころがあんていしないんです。そわそわしているっていうか。
\\	どんなふうに。
\\	どんなふうに。
\\	ノリノリでイケイケのときもあれば、どんぞこにおちてなにもできないときもあるんです。
\\	ノリノリでイケイケのときもあれば、どんぞこにおちてなにもできないときもあるんです。
\\	はげしくやってくるかんじ?
\\	はげしくやってくるかんじ?
\\	はい。
\\	はい。
\\	そううつのけがあるなあ。
\\	そううつのけがあるなあ。
\\	まわりにもそういわれます。
\\	まわりにもそういわれます。
\\	では、くすりをだそう。まずひとつめはせいしんあんていざい。そしてふたつめはこれ。
\\	では、くすりをだそう。まずひとつめはせいしんあんていざい。そしてふたつめはこれ。
\\	なんですか。
\\	なんですか。
\\	うむ。これは、「かみをそめるようにせいかくをそめる」といわれているくらいつよいくすりだ。どんぞこでたちあがれないときだけふくようするように。
\\	うむ。これは、「かみをそめるようにせいかくをそめる」といわれているくらいつよいくすりだ。どんぞこでたちあがれないときだけふくようするように。
\\	2しゅうかんご
\\	2しゅうかんご
\\	どうですか。
\\	どうですか。
\\	せんせい
\\	あのつよいくすりまただしてください
\\	あれすごいっす
\\	あたまのなかがシュワーーッとしてスッキリするんです。ちからがみなぎってきます。
\\	せんせい
\\	あのつよいくすりまただしてください
\\	あれすごいっす
\\	あたまのなかがシュワーーッとしてスッキリするんです。ちからがみなぎってきます。
\\	つよいくすりだからのみすぎてはいけませんよ。ふくさようがでてくるかもしれないよ。
\\	つよいくすりだからのみすぎてはいけませんよ。ふくさようがでてくるかもしれないよ。
\\	ふくさよう
\\	ふくさよう
\\	うでがしびれてくるとか、どうきがはげしくなるとか、まあひとによっていろいろ。
\\	うでがしびれてくるとか、どうきがはげしくなるとか、まあひとによっていろいろ。
\\	せんせい
\\	うで、しびれてます。
\\	せんせい
\\	うで、しびれてます。
\\	ちょうしんき
\\	えいようしっちょう
\\	せいしんてきしょっく
\\	はく
\\	ちゅうしゃ
\\	すいじゃく
\\	てんてき
\\	きょうふしょう
\\	どうされましたか。
\\	どうされましたか。
\\	たいちょうがわるいんです。たべてもはいてしまうし。
\\	たいちょうがわるいんです。たべてもはいてしまうし。
\\	かおいろがまっさおですよ。ではちょうしんきをあてるので、うえをぬいでください。
\\	かおいろがまっさおですよ。ではちょうしんきをあてるので、うえをぬいでください。
\\	はい。
\\	はい。
\\	えいようしっちょうですね。どうされたんですか。
\\	えいようしっちょうですね。どうされたんですか。
\\	じつはかのじょにフラれてから、せいしんてきショックでほとんどなにもたべられなくなってしまって。むりにたべてもはいてしまうんです。
\\	じつはかのじょにフラれてから、せいしんてきショックでほとんどなにもたべられなくなってしまって。むりにたべてもはいてしまうんです。
\\	てんてきをうちましょう。まずはからだにえいようをあたえないといけません。
\\	てんてきをうちましょう。まずはからだにえいようをあたえないといけません。
\\	えーっ
\\	うでにうつんですか。
\\	えーっ
\\	うでにうつんですか。
\\	はい。からだがすいじゃくしていますから。おーい。てんてきのよういをおねがい。
\\	はい。からだがすいじゃくしていますから。おーい。てんてきのよういをおねがい。
\\	ちょ、ちょっとまってください。わたし、ちゅうしゃはダメなんです。
\\	ちょ、ちょっとまってください。わたし、ちゅうしゃはダメなんです。
\\	(かんじゃあばれる)
\\	(かんじゃあばれる)
\\	はい、おとなしくして。きみたち、ちょっとからだをおさえて。
\\	はい、おとなしくして。きみたち、ちょっとからだをおさえて。
\\	はい。はりがおれでもしたら、たいへんですよ。じっとしてください。
\\	はい。はりがおれでもしたら、たいへんですよ。じっとしてください。
\\	むしば
\\	ますい
\\	ぬく
\\	けいけん
\\	おくば
\\	いたむ
\\	レントゲン
\\	しかいし
\\	はいしゃ
\\	うったえる
\\	きょうは、どうしましたか。
\\	きょうは、どうしましたか。
\\	いっしゅうかんくらいまえからずっと、みぎのうえのはがいたむんですが…。
\\	いっしゅうかんくらいまえからずっと、みぎのうえのはがいたむんですが…。
\\	はい?すみませんね。みみがとおいんですよ。
\\	はい?すみませんね。みみがとおいんですよ。
\\	みぎの うえの はが いたいんです。
\\	みぎの うえの はが いたいんです。
\\	ああ。みぎうえのはね。みてみましょうか。はい、くちをおおきくあけて。はい、あーん。
\\	ああ。みぎうえのはね。みてみましょうか。はい、くちをおおきくあけて。はい、あーん。
\\	あー。
\\	あー。
\\	いたいのはこのはですか?(コンコン)
\\	いたいのはこのはですか?(コンコン)
\\	ちがう・・・とおもいます。
\\	ちがう・・・とおもいます。
\\	じゃ、このはかな? (コンコン)
\\	じゃ、このはかな? (コンコン)
\\	あのぉ。せんせい。レントゲンをとったりしないんですか。
\\	あのぉ。せんせい。レントゲンをとったりしないんですか。
\\	レントゲン?わたしをだれだとおもっているんですか。わたしは、80ねんしかいしをしているんです。80ねんですよ。きみがうまれるずっとまえからはいしゃをしているんです。むしばくらいわかります。
\\	レントゲン?わたしをだれだとおもっているんですか。わたしは、80ねんしかいしをしているんです。80ねんですよ。きみがうまれるずっとまえからはいしゃをしているんです。むしばくらいわかります。
\\	せんせい、けいけんがあろうがあるまいが、レントゲンはひつようだとおもいます。きちんととってください。
\\	せんせい、けいけんがあろうがあるまいが、レントゲンはひつようだとおもいます。きちんととってください。
\\	やっぱりむしばですね。ほら、このレントゲンをみてください。かなりふかいですよ。これはぬかざるをえないですね。はい、くちをあけて。いたかったらてをあげてください。
\\	やっぱりむしばですね。ほら、このレントゲンをみてください。かなりふかいですよ。これはぬかざるをえないですね。はい、くちをあけて。いたかったらてをあげてください。
\\	(ウィィイイイーン キーィイイイイン ガガガガガ)
\\	(ウィィイイイーン キーィイイイイン ガガガガガ)
\\	あががががあ。せ、せんせいいたい
\\	ま、ますいをしてください。ますいをおねがいします
\\	あががががあ。せ、せんせいいたい
\\	ま、ますいをしてください。ますいをおねがいします
\\	あと もうちょっとなので、がんばってください。
\\	あと もうちょっとなので、がんばってください。
\\	(ウィィイイイーン キーィイイイイン ガガガガガ)
\\	(ウィィイイイーン キーィイイイイン ガガガガガ)
\\	ぜっこうの
\\	はなみ
\\	しく
\\	はる
\\	しゅっぱん
\\	めじるし
\\	けんえんのなか
\\	ひより
\\	みてのとおり
\\	(おはなみたいけつ 
\\	きょうはこんでるなあ。すごいひとだ。
\\	(おはなみたいけつ 
\\	きょうはこんでるなあ。すごいひとだ。
\\	ぜっこうのはなみびよりですからね。
\\	ぜっこうのはなみびよりですからね。
\\	ばしょなどもうないのではないか。
\\	ばしょなどもうないのではないか。
\\	だいじょうぶです。ばしょをとっておきましたから。
\\	だいじょうぶです。ばしょをとっておきましたから。
\\	どうやって。
\\	どうやって。
\\	きのう、よなかにきてシートをしいておきました。「おおはらけ」ってかみをはっておきましたから。
\\	きのう、よなかにきてシートをしいておきました。「おおはらけ」ってかみをはっておきましたから。
\\	たしかあのへんなんですけど・・・あれっ、だれかいますね。
\\	たしかあのへんなんですけど・・・あれっ、だれかいますね。
\\	ああっ
\\	ああっ
\\	まさか・・・。
\\	まさか・・・。
\\	たじましゃちょうじゃないか。
\\	たじましゃちょうじゃないか。
\\	ああ、たじましゃちょうですねえ。(なんでみどりやましゅっぱんのたじましゃちょうがいるのよ。おおはらしゃちょうとけんえんのなかじゃん・・・)
\\	ああ、たじましゃちょうですねえ。(なんでみどりやましゅっぱんのたじましゃちょうがいるのよ。おおはらしゃちょうとけんえんのなかじゃん・・・)
\\	これはこれはたじまさん、こんなところでなにを・・・。
\\	これはこれはたじまさん、こんなところでなにを・・・。
\\	おおはらさんじゃないですか。みてのとおりおはなみですよ。おたくは?
\\	おおはらさんじゃないですか。みてのとおりおはなみですよ。おたくは?
\\	われわれもおはなみですよ。
\\	われわれもおはなみですよ。
\\	あのー、このあたりにうちのシートがしいてあったはずなんですけれど。
\\	あのー、このあたりにうちのシートがしいてあったはずなんですけれど。
\\	シート?おい、やまなか、シートここにしいてあったか?
\\	シート?おい、やまなか、シートここにしいてあったか?
\\	いえ。あさ6じにきたときはなにもなかったです。おたく、ばしょをまちがえてるんじゃないの。
\\	いえ。あさ6じにきたときはなにもなかったです。おたく、ばしょをまちがえてるんじゃないの。
\\	たしかにここです。このおおきいきがめじるしでしたから。あっ、あんなところに。
\\	たしかにここです。このおおきいきがめじるしでしたから。あっ、あんなところに。
\\	かんけいない
\\	うけてたつ
\\	おじけづく
\\	きそう
\\	よきょう
\\	ひしょ
\\	まね
\\	きたない
\\	こまる
\\	せんりょう
\\	しょうこ
\\	れんじゅう 
\\	れんちゅう
\\	(おはなみたいけつ2, くしゃくしゃになったシートとかみがゴミはこにすててあるのにきがつく。ひどい。ちょっとひどいじゃないですか。)
\\	(おはなみたいけつ2, くしゃくしゃになったシートとかみがゴミはこにすててあるのにきがつく。ひどい。ちょっとひどいじゃないですか。)
\\	われわれはかんけいないよ。われわれがやったなんてしょうこは?
\\	われわれはかんけいないよ。われわれがやったなんてしょうこは?
\\	ここをせんりょうしてるのが、あきらかなしょうこでしょ。
\\	ここをせんりょうしてるのが、あきらかなしょうこでしょ。
\\	あさきたときはなにもなかったからね。
\\	あさきたときはなにもなかったからね。
\\	ウソだ
\\	あなたたちがやったにきまっている
\\	ウソだ
\\	あなたたちがやったにきまっている
\\	きみィー、いいがかりはこまるよ。
\\	きみィー、いいがかりはこまるよ。
\\	みどりやましゅっぱんのれんちゅうはいつもきたないまねを
\\	みどりやましゅっぱんのれんちゅうはいつもきたないまねを
\\	なんだとー。
\\	なんだとー。
\\	まあまあまあまあ。おふたりともせっかくのきゅうじつなんですから。
\\	まあまあまあまあ。おふたりともせっかくのきゅうじつなんですから。
\\	ここはよきょうもかねて・・・ひとつ、おさけできめるのはどうでしょう?
\\	ここはよきょうもかねて・・・ひとつ、おさけできめるのはどうでしょう?
\\	おさけ?
\\	おさけ?
\\	おさけののみくらべをするんですよ。
\\	おさけののみくらべをするんですよ。
\\	のみくらべ?
\\	のみくらべ?
\\	あおやましゅっぱんさんとウチからだいひょうしゃをひとりだして、どちらがのみつづけられるかきそうんですよ。
\\	あおやましゅっぱんさんとウチからだいひょうしゃをひとりだして、どちらがのみつづけられるかきそうんですよ。
\\	でっ、かったほうがこのばしょをつかえるわけです。
\\	でっ、かったほうがこのばしょをつかえるわけです。
\\	なんですかそれ?
\\	なんですかそれ?
\\	あおやましゅっぱんさん、まさかおじけづいたんですか。
\\	あおやましゅっぱんさん、まさかおじけづいたんですか。
\\	なにを〜
\\	みどりやまのれんちゅうにはまけられん
\\	うけてたってやる。
\\	なにを〜
\\	みどりやまのれんちゅうにはまけられん
\\	うけてたってやる。
\\	あいて
\\	まるつぶれ
\\	メンツ
\\	てまえ
\\	よう
\\	ていど
\\	おおめだま
\\	いじ
\\	おそれられる
\\	そこなしの
\\	せいぎ
\\	あいてはそこなしのやまだとおそれられているおとこらしい。まけるおそれがあるぞ。
\\	あいてはそこなしのやまだとおそれられているおとこらしい。まけるおそれがあるぞ。
\\	どうするんだ。
\\	どうするんだ。
\\	わたしにいかせてください。
\\	わたしにいかせてください。
\\	だいじょうぶなのか。
\\	だいじょうぶなのか。
\\	わたしにもいじがあります。このばしょはわれわれのものです。
\\	わたしにもいじがあります。このばしょはわれわれのものです。
\\	(いっぱいめのテキーラ)
\\	(いっぱいめのテキーラ)
\\	いっぱいめ、にはいめ、さんばいめ
\\	いっぱいめ、にはいめ、さんばいめ
\\	(つぎつぎとテキーラをのむ)
\\	(つぎつぎとテキーラをのむ)
\\	うー、ま、まけられないぞぉ・・・どうしてもかつんだぁ。
\\	うー、ま、まけられないぞぉ・・・どうしてもかつんだぁ。
\\	ふふ。そのていどか。かてっこないさ。
\\	ふふ。そのていどか。かてっこないさ。
\\	まだまだまだだぁぁ。
\\	まだまだまだだぁぁ。
\\	やまださん、まさかよってるんじゃないでしょうね。
\\	やまださん、まさかよってるんじゃないでしょうね。
\\	だ、だいじょうぶですよ、ウィッ、ヒック。まけっこないですよ。
\\	だ、だいじょうぶですよ、ウィッ、ヒック。まけっこないですよ。
\\	あおぞらしゅっぱんにまけたら、しゃちょうからおおめだまですよ
\\	いいだしたてまえ、わたしのメンツもまるつぶれです。
\\	あおぞらしゅっぱんにまけたら、しゃちょうからおおめだまですよ
\\	いいだしたてまえ、わたしのメンツもまるつぶれです。
\\	じゅうきゅうはいめ
\\	にじゅう・・・ぱい
\\	じゅうきゅうはいめ
\\	にじゅう・・・ぱい
\\	ヒック・・・ヒック・・・か、かった・・・ウィッ、ヒック。ここはあおぞらしゅっぱんのも・・・
\\	ヒック・・・ヒック・・・か、かった・・・ウィッ、ヒック。ここはあおぞらしゅっぱんのも・・・
\\	(ドサッ)
\\	(ドサッ)
\\	やっぱりせいぎはかつということだな。さくらいよくやったぞ。
\\	やっぱりせいぎはかつということだな。さくらいよくやったぞ。
\\	しゅくはく
\\	いっぱくふつか
\\	あいにく
\\	だんたい
\\	まんしつ
\\	りょかん
\\	りよう
\\	ありがとうございます。りょかんわさびでございます。
\\	ありがとうございます。りょかんわさびでございます。
\\	あ、あのぉ… ろくがつはつかにしゅくはくをしたいのですが。
\\	あ、あのぉ… ろくがつはつかにしゅくはくをしたいのですが。
\\	ありがとうございます。なんめいさまでごしゅくはくでしょうか。
\\	ありがとうございます。なんめいさまでごしゅくはくでしょうか。
\\	おとなふたり、こどもふたりのよんめいです。あ、ぜんいんおなじへやでおねがいします。
\\	おとなふたり、こどもふたりのよんめいです。あ、ぜんいんおなじへやでおねがいします。
\\	よんめいさまごいっしつのごりようですね。
\\	よんめいさまごいっしつのごりようですね。
\\	いっぱくふつかのごりようでよろしいですか。
\\	いっぱくふつかのごりようでよろしいですか。
\\	あ、はい。そうです。
\\	あ、はい。そうです。
\\	しょうしょうおまちくださいませ。
\\	しょうしょうおまちくださいませ。
\\	(パソコンそうさするおと)
\\	(パソコンそうさするおと)
\\	おきゃくさま、もうしわけございません。あいにくですが、ろくがつはつかはだんたいのおきゃくさまのごよやくがいっておりまして、ぜんしつまんしつでございます。そのぜんじつろくがつじゅうくにちあるいは、つぎのにちろくがつにじゅういちにちでしたらおへやをごよういできるのですが・・・。
\\	おきゃくさま、もうしわけございません。あいにくですが、ろくがつはつかはだんたいのおきゃくさまのごよやくがいっておりまして、ぜんしつまんしつでございます。そのぜんじつろくがつじゅうくにちあるいは、つぎのにちろくがつにじゅういちにちでしたらおへやをごよういできるのですが・・・。
\\	あー。まえのひもつぎのひもちょっとダメだなぁ…。
\\	あー。まえのひもつぎのひもちょっとダメだなぁ…。
\\	せっかくおでんわいただいたのに、たいへんもうしわけございません。
\\	せっかくおでんわいただいたのに、たいへんもうしわけございません。
\\	そうですか。わかりました。ありがとうございました。
\\	そうですか。わかりました。ありがとうございました。
\\	(カチャ)
\\	(カチャ)
\\	どうだった?
\\	どうだった?
\\	ダメだって。よやくいっぱいだって。せっかくでんわしたのになぁ。
\\	ダメだって。よやくいっぱいだって。せっかくでんわしたのになぁ。
\\	よやく
\\	うかがう
\\	れんらくさき
\\	だいひょうしゃ
\\	つうじょう
\\	しめきる
\\	すでに
\\	もうしこむ
\\	ろてんぶろ
\\	くうしつ
\\	じゃっかん
\\	かかり
\\	たんとう
\\	しょうがくせい
\\	くのいちりょかん よやくたんとうがかり はっとりでございます。
\\	くのいちりょかん よやくたんとうがかり はっとりでございます。
\\	あ、もしもし、ろくがつはつかからいっぱくふつか、よんめいいっしつでしゅくはくをかんがえているのですが、まだあいていますか。
\\	あ、もしもし、ろくがつはつかからいっぱくふつか、よんめいいっしつでしゅくはくをかんがえているのですが、まだあいていますか。
\\	(カチャカチャカチャ)はい。まだじゃっかん くうしつがございます。
\\	(カチャカチャカチャ)はい。まだじゃっかん くうしつがございます。
\\	ホームページをみておでんわしているのですが、ろてんぶろつきいっぱくにしょくしゅくはくプランというのはまだもうしこむことができますか。
\\	ホームページをみておでんわしているのですが、ろてんぶろつきいっぱくにしょくしゅくはくプランというのはまだもうしこむことができますか。
\\	あ、もうしわけございません。そちらのプランはたいへんにんきがございまして、すでにしめきらせていただいております。つうじょうのおへやでしたら、ごよういできるのですが…。
\\	あ、もうしわけございません。そちらのプランはたいへんにんきがございまして、すでにしめきらせていただいております。つうじょうのおへやでしたら、ごよういできるのですが…。
\\	そうですか。じゃ、ふつうのへやのよやくをおねがいします。
\\	そうですか。じゃ、ふつうのへやのよやくをおねがいします。
\\	ありがとうございます。ではだいひょうしゃさまのおなまえとごれんらくさきをうかがってもよろしいでしょうか。
\\	ありがとうございます。ではだいひょうしゃさまのおなまえとごれんらくさきをうかがってもよろしいでしょうか。
\\	いが しんたろうです。でんわばんごうは03
\\	いが しんたろうです。でんわばんごうは03
\\	よんめいさまごいっしつのごりようですね。
\\	よんめいさまごいっしつのごりようですね。
\\	はい。おとなふたりこどもふたりです。こどもはふたりともしょうがくせいです。あ、それから、ゆうしょくなのですが、へやしょくにしてもらえますか。
\\	はい。おとなふたりこどもふたりです。こどもはふたりともしょうがくせいです。あ、それから、ゆうしょくなのですが、へやしょくにしてもらえますか。
\\	かしこまりました。
\\	かしこまりました。
\\	そうげい
\\	あれるぎー
\\	そうげいさーびす
\\	へやしょく
\\	いっさい
\\	じゅうぎょういん
\\	いちどう
\\	ナッツ
\\	あと、こどものひとりがナッツアレルギーなので、ナッツるいはいれないようにおねがいします。
\\	あと、こどものひとりがナッツアレルギーなので、ナッツるいはいれないようにおねがいします。
\\	はい。かしこまりました。
\\	はい。かしこまりました。
\\	でんしゃでいくつもりなのですが、しもだのえきまでむかえにきてもらうことはかのうですか。
\\	でんしゃでいくつもりなのですが、しもだのえきまでむかえにきてもらうことはかのうですか。
\\	はい、にじ、さんじ、よじ、ごじにそうげいサービスがございます。
\\	はい、にじ、さんじ、よじ、ごじにそうげいサービスがございます。
\\	つまり、それいがいのじかんならば、タクシーでいくしかないんですね。
\\	つまり、それいがいのじかんならば、タクシーでいくしかないんですね。
\\	さようでございます。しもだのえきからにじゅうごふんくらいかかるとおもいます。
\\	さようでございます。しもだのえきからにじゅうごふんくらいかかるとおもいます。
\\	ちなみに、チェックインはなんじからでしたっけ。
\\	ちなみに、チェックインはなんじからでしたっけ。
\\	チェックインはさんじから、チェックアウトはじゅういちじでございます。
\\	チェックインはさんじから、チェックアウトはじゅういちじでございます。
\\	わかりました。じゃぁ、にじのそうげいサービスをりようします。
\\	わかりました。じゃぁ、にじのそうげいサービスをりようします。
\\	かしこまりました。ではごよやくないようをかくにんさせていただきます。いが しんたろうさま、ろくがつはつかからいっぱくふつか、よんめいさまごいっしつのごりよう。でんわばんごうは03
\\	6969。ゆうしょくはおへやしょく、しもだのえきににじのそうげい、おこさまいちめいさまナッツアレルギーあり。いじょうでまちがいございませんか。
\\	かしこまりました。ではごよやくないようをかくにんさせていただきます。いが しんたろうさま、ろくがつはつかからいっぱくふつか、よんめいさまごいっしつのごりよう。でんわばんごうは03
\\	6969。ゆうしょくはおへやしょく、しもだのえきににじのそうげい、おこさまいちめいさまナッツアレルギーあり。いじょうでまちがいございませんか。
\\	だいじょうぶです。ちなみに、ペットはつれていってはだめなんですよね。
\\	だいじょうぶです。ちなみに、ペットはつれていってはだめなんですよね。
\\	そうですね。もうしわけないのですが、ペットはいっさいおことわりしております。
\\	そうですね。もうしわけないのですが、ペットはいっさいおことわりしております。
\\	わかりました。
\\	わかりました。
\\	ごよやくはわたし、はっとりがうけたまわりました。それではいがさまのごらいかんをじゅうぎょういんいちどうこころよりおまちもうしあげております。
\\	ごよやくはわたし、はっとりがうけたまわりました。それではいがさまのごらいかんをじゅうぎょういんいちどうこころよりおまちもうしあげております。
\\	よろしくおねがいします。
\\	よろしくおねがいします。
\\	にんじゃ
\\	あんない
\\	ほんじつ
\\	さしあげる
\\	さんさくする
\\	ばいてん
\\	だいよくじょう
\\	ひじょうぐち
\\	もうしつける
\\	おかみ
\\	なんなり
\\	いがさま。おまちしておりました。おかみのももちでございます。ほんじつはとおいところまで、あしをはこんでいただきまして、ありがとうございます。さぁ、どうぞ、おあがりください。おへやのじゅんびもできております。シノブ、いがさまをおへやまでごあんないしてさしあげて。いがさまのおへやをたんとうさせていただきますシノブでございます。なにかございましたら、なんなりとおもうしつけください。
\\	いがさま。おまちしておりました。おかみのももちでございます。ほんじつはとおいところまで、あしをはこんでいただきまして、ありがとうございます。さぁ、どうぞ、おあがりください。おへやのじゅんびもできております。シノブ、いがさまをおへやまでごあんないしてさしあげて。いがさまのおへやをたんとうさせていただきますシノブでございます。なにかございましたら、なんなりとおもうしつけください。
\\	あ、どうも。
\\	あ、どうも。
\\	(あるきながら)ひじょうぐちはあちらでございます。だいよくじょうのごりようはあさろくじからよるじゅうにじまでとなっております。きっさコーナー、ばいてんはよるのはちじまででございます。
\\	(あるきながら)ひじょうぐちはあちらでございます。だいよくじょうのごりようはあさろくじからよるじゅうにじまでとなっております。きっさコーナー、ばいてんはよるのはちじまででございます。
\\	せっかくだから、さんぽしたいんだけど、このへんに、おすすめのさんぽコースはある。
\\	せっかくだから、さんぽしたいんだけど、このへんに、おすすめのさんぽコースはある。
\\	とうりょかんのにわもさんさくするにはいいとおもいますが、せっかくですから、あしをのばしてこのさきのにんじゃでらにいくのもおもしろいとおもいます。…こちらが、いがさまのおへやでございます。
\\	とうりょかんのにわもさんさくするにはいいとおもいますが、せっかくですから、あしをのばしてこのさきのにんじゃでらにいくのもおもしろいとおもいます。…こちらが、いがさまのおへやでございます。
\\	(カチャ)ガラガラ
\\	(カチャ)ガラガラ
\\	うわーきれいで、おおきいへやだなぁ。うみがみえるー。
\\	うわーきれいで、おおきいへやだなぁ。うみがみえるー。
\\	(コポコポコポコポ)
\\	(コポコポコポコポ)
\\	ごゆうしょくはなんじからになさいますか。
\\	ごゆうしょくはなんじからになさいますか。
\\	うーん。さんぽして…、おふろはいって…、ゆっくりしてだから…、ろくじくらいかな。
\\	うーん。さんぽして…、おふろはいって…、ゆっくりしてだから…、ろくじくらいかな。
\\	かしこまりました。ではろくじにおしょくじをごよういさせていただきます。ではしつれいいたします。
\\	かしこまりました。ではろくじにおしょくじをごよういさせていただきます。ではしつれいいたします。
\\	(キキー)
\\	(キキー)
\\	おきゃくさま?なにかおっしゃいましたか?
\\	おきゃくさま?なにかおっしゃいましたか?
\\	え..いえ、べつに。キー…キーがないなぁ。。。なんてハハハ。
\\	え..いえ、べつに。キー…キーがないなぁ。。。なんてハハハ。
\\	けしょう
\\	サービスエリア
\\	しゃかんきょり
\\	ついとつじこ
\\	こうび
\\	ふきん
\\	えいきょう
\\	しゃせんきせい
\\	かたがわ
\\	おうてん
\\	くだり
\\	だんぞくてきに
\\	ジャンクション
\\	のぼり
\\	あいつぐ
\\	ぴーくをむかえる
\\	ゆーたーん らっしゅ
\\	じゅうたい
\\	はっせいする
\\	さっきからぜんぜんうごかないね。
\\	さっきからぜんぜんうごかないね。
\\	あぁ。ゴールデンウィークさいしゅうびだからな。
\\	あぁ。ゴールデンウィークさいしゅうびだからな。
\\	あぁあー。もっとはやくやどをでればよかったね。
\\	あぁあー。もっとはやくやどをでればよかったね。
\\	だれだよ。にじかんもけしょうをしていたのは。
\\	だれだよ。にじかんもけしょうをしていたのは。
\\	え?だれだったかなぁ。あ、じゅうたいじょうほうをきいてみよーっと。
\\	え?だれだったかなぁ。あ、じゅうたいじょうほうをきいてみよーっと。
\\	(ラジオをつける)
\\	(ラジオをつける)
\\	(ボリュームがだんだんあがる)
\\	(ボリュームがだんだんあがる)
\\	ゴールデンウィークさいしゅうびのきょうは、
\\	ターンラッシュがピークをむかえ、かくちでじゅうたいがあいついでいます。げんざいとうきょうへむかうちゅうおうじどうしゃどうのぼりはおおつきジャンクションをせんとうにだんぞくてきに44キロのじゅうたいとなっております。また、ちゅうおうどうくだりは、しょうごすぎにさがみこふきんではっせいしたトラックおうてんじこのため、かたがわいっしゃせんのしゃせんきせいがおこなわれております。げんばふきんをそうこうするばあいはじゅうぶんにちゅういしてください。このじこのえいきょうで、はちおうじインターふきんから8キロのじゅうたいがつづいています。じゅうたいのなかやじゅうたいのこうびでは、ついとつじこがはっせいしやすいので、(しゃかんきょりにじゅうぶんなちゅういがひつようです…。ボリュームをさげていく)
\\	ゴールデンウィークさいしゅうびのきょうは、
\\	ターンラッシュがピークをむかえ、かくちでじゅうたいがあいついでいます。げんざいとうきょうへむかうちゅうおうじどうしゃどうのぼりはおおつきジャンクションをせんとうにだんぞくてきに44キロのじゅうたいとなっております。また、ちゅうおうどうくだりは、しょうごすぎにさがみこふきんではっせいしたトラックおうてんじこのため、かたがわいっしゃせんのしゃせんきせいがおこなわれております。げんばふきんをそうこうするばあいはじゅうぶんにちゅういしてください。このじこのえいきょうで、はちおうじインターふきんから8キロのじゅうたいがつづいています。じゅうたいのなかやじゅうたいのこうびでは、ついとつじこがはっせいしやすいので、(しゃかんきょりにじゅうぶんなちゅういがひつようです…。ボリュームをさげていく)
\\	ちゅうおうどうのぼりじゅうたい44キロ? ありえない…。
\\	ちゅうおうどうのぼりじゅうたい44キロ? ありえない…。
\\	ねぇ、けんちゃん、あたし、トイレいきたくなっちゃった。
\\	ねぇ、けんちゃん、あたし、トイレいきたくなっちゃった。
\\	とりあえず、つぎのサービスエリアでおりて、めしでもたべるか。
\\	とりあえず、つぎのサービスエリアでおりて、めしでもたべるか。
\\	わだい
\\	ふくまれる
\\	のう
\\	しんぞう
\\	かろう
\\	じさつ
\\	とうろく
\\	とりひきさき
\\	うつびょう
\\	しいる
\\	しょうちょうする
\\	げんじょう
\\	はいけい
\\	ざんぎょう
\\	きんむ
\\	こきゅうき
\\	しっかん
\\	はつびょう
\\	いたる
\\	はっしょう
\\	みんなは、「かろうし」ということばをきいたことがありますか?
\\	みんなは、「かろうし」ということばをきいたことがありますか?
\\	しんぶんやテレビのニュースなどでもわだいになっていますね。きょうは、この「かろうし」についてみんなでかんがえようとおもいます。
\\	しんぶんやテレビのニュースなどでもわだいになっていますね。きょうは、この「かろうし」についてみんなでかんがえようとおもいます。
\\	じゃ、いまから30ぷんかん、グループにわかれて、かろうしのげんじょうやしゃかいてきなはいけいなどについて、きょうかしょやインターネットをしらべながらはなしあってください。
\\	じゃ、いまから30ぷんかん、グループにわかれて、かろうしのげんじょうやしゃかいてきなはいけいなどについて、きょうかしょやインターネットをしらべながらはなしあってください。
\\	う~ん、まずさ、かろうしのていぎって、なに?
\\	う~ん、まずさ、かろうしのていぎって、なに?
\\	きょういかしょにかいてあるよ。「かろうしとは、ちょうじかんのざんぎょうやきゅうじつなしのきんむをしいられたけっか、かろうやストレスがげんいんのひとつとなって、のうやしんぞう、こきゅうきなどのしっかんをはつびょうし、しぼうにいたることをいみします。また、かろうによりおおきなストレスをうけ、「うつびょう」をはっしょうし、じさつしてしまうかろうじさつもふくまれます。」だって。
\\	きょういかしょにかいてあるよ。「かろうしとは、ちょうじかんのざんぎょうやきゅうじつなしのきんむをしいられたけっか、かろうやストレスがげんいんのひとつとなって、のうやしんぞう、こきゅうきなどのしっかんをはつびょうし、しぼうにいたることをいみします。また、かろうによりおおきなストレスをうけ、「うつびょう」をはっしょうし、じさつしてしまうかろうじさつもふくまれます。」だって。
\\	なるほどね。そういえば、えいごでも、そのまま 
\\	でつうじるってきいたことがあるよ。
\\	なるほどね。そういえば、えいごでも、そのまま 
\\	でつうじるってきいたことがあるよ。
\\	ちょっとまって。。。(カチャカチャ)へぇ~。
\\	ちょっとまって。。。(カチャカチャ)へぇ~。
\\	「2002ねんにはじめてえいごのじしょにとうろくされ、これによって、かろうしがにほんじんのライフスタイルをしょうちょうすることばとしてせかいにしられるようになった。」だって。
\\	「2002ねんにはじめてえいごのじしょにとうろくされ、これによって、かろうしがにほんじんのライフスタイルをしょうちょうすることばとしてせかいにしられるようになった。」だって。
\\	ろうどうじかん
\\	ざんぎょうだい
\\	おやじ
\\	ねんかん
\\	かっこく
\\	おうべい
\\	くにべつ
\\	いちいん
\\	うみだす
\\	しんせいする
\\	のる
\\	ふろしきざんぎょう
\\	ほうこくする
\\	かんりしょく
\\	ぼやく
\\	みて、このくにべつのろうどうじかんのグラフ。1988ねんごろは、おうべいかっこくのねんかんろうどうじかんが1600じかんから1800じかんなのにたいして、にっぽんは2000じかんをこえている。でも、それいこう、にっぽんのろうどうじかんがみじかくなって、2006ねんにはアメリカのほうがにっぽんよりろうどうじかんがながくなってる。
\\	みて、このくにべつのろうどうじかんのグラフ。1988ねんごろは、おうべいかっこくのねんかんろうどうじかんが1600じかんから1800じかんなのにたいして、にっぽんは2000じかんをこえている。でも、それいこう、にっぽんのろうどうじかんがみじかくなって、2006ねんにはアメリカのほうがにっぽんよりろうどうじかんがながくなってる。
\\	でも、これ、ほんとうのすうじなのかなぁ。ぼくのおやじは、かんりしょくになったら、ざんぎょうだいがはらわれないってぼやいてたし、タカシのにいちゃんも、おかねがはらわれるわけじゃないのに、まいばんいえにかえってからもしごとしてるらしいし。。。ざんぎょうしていても、すうじとしてほうこくされてないケースがおおいんじゃないかな。
\\	でも、これ、ほんとうのすうじなのかなぁ。ぼくのおやじは、かんりしょくになったら、ざんぎょうだいがはらわれないってぼやいてたし、タカシのにいちゃんも、おかねがはらわれるわけじゃないのに、まいばんいえにかえってからもしごとしてるらしいし。。。ざんぎょうしていても、すうじとしてほうこくされてないケースがおおいんじゃないかな。
\\	ふろしきざんぎょうっていうことばもあるみたい。ここにのってる。
\\	ふろしきざんぎょうっていうことばもあるみたい。ここにのってる。
\\	おじさんがぼやいてたけど、かいしゃで、ノーざんぎょうデーとかきめて、ざんぎょうじかんをへらそうとしても、けっきょく、しごとがへるわけじゃないから、ざんぎょうはせざるをえないって。だけど、かいしゃはざんぎょうじかんをへらそうとしているから、ざんぎょうじかんをしんせいしづらいって。だから、けっきょくサービスざんぎょうになっちゃうって。。。
\\	おじさんがぼやいてたけど、かいしゃで、ノーざんぎょうデーとかきめて、ざんぎょうじかんをへらそうとしても、けっきょく、しごとがへるわけじゃないから、ざんぎょうはせざるをえないって。だけど、かいしゃはざんぎょうじかんをへらそうとしているから、ざんぎょうじかんをしんせいしづらいって。だから、けっきょくサービスざんぎょうになっちゃうって。。。
\\	なるほどね。サービスざんぎょうがあたりまえになっていることも、かろうしをうみだすいちいんじゃないかな。
\\	なるほどね。サービスざんぎょうがあたりまえになっていることも、かろうしをうみだすいちいんじゃないかな。
\\	ろうどうさいがい
\\	ほけんきん
\\	みとめる
\\	げきむ
\\	せいかつしゅうかん
\\	せきにん
\\	にんてい
\\	すいい
\\	そういえば、ときどき、ニュースで「かろうしのろうさいにんてい」のはなしをみみにするよね。
\\	そういえば、ときどき、ニュースで「かろうしのろうさいにんてい」のはなしをみみにするよね。
\\	えっ、ろうさいにんていってなに?
\\	えっ、ろうさいにんていってなに?
\\	ろうどうさいがいにんていのこと。しごとやつうきんのとちゅうでけがをしたり、びょうきになったり、しぼうしてしまったばあいは、ろうどうさいがいとしてほけんきんがはらわれるんだ。でも、かろうしのばあいは、ろうどうさいがいとみとめられることはすくないんだよ。
\\	ろうどうさいがいにんていのこと。しごとやつうきんのとちゅうでけがをしたり、びょうきになったり、しぼうしてしまったばあいは、ろうどうさいがいとしてほけんきんがはらわれるんだ。でも、かろうしのばあいは、ろうどうさいがいとみとめられることはすくないんだよ。
\\	なんで?
\\	なんで?
\\	まず、かろうしは、げきむのさいちゅうやちょくごではなくて、いっかげつごからすうかげつごにおこることがおおい。くわえて、せいかつしゅうかんがげんいんでのうやしんぞうのびょうきになることもあるから、なかなかげきむがげんいんでしにいたったということをしょうめいするのがむずかしいんだ。かいしゃがわもそれをさかてにとってせきにんをみとめようとしないんじゃないかな。
\\	まず、かろうしは、げきむのさいちゅうやちょくごではなくて、いっかげつごからすうかげつごにおこることがおおい。くわえて、せいかつしゅうかんがげんいんでのうやしんぞうのびょうきになることもあるから、なかなかげきむがげんいんでしにいたったということをしょうめいするのがむずかしいんだ。かいしゃがわもそれをさかてにとってせきにんをみとめようとしないんじゃないかな。
\\	ホントはらがたつね
\\	あっ、きょうかしょの25ページに、「かろうしのにんていすうのすいい」がのっているわ。
\\	ホントはらがたつね
\\	あっ、きょうかしょの25ページに、「かろうしのにんていすうのすいい」がのっているわ。
\\	かろうししんせいすうをみてみよう。1988ねんは676けんで、2005ねんは869けん。そんなにおおはばなぞうかはみられないな。でも、かろうしにんていすうは、1988ねんは29けんで2005ねんは330けん。やく10ばいにのびているよ。
\\	かろうししんせいすうをみてみよう。1988ねんは676けんで、2005ねんは869けん。そんなにおおはばなぞうかはみられないな。でも、かろうしにんていすうは、1988ねんは29けんで2005ねんは330けん。やく10ばいにのびているよ。
\\	とくに、2001ねんと2002ねんのかんでおおきくのびているけど、なにかあったのかな。
\\	とくに、2001ねんと2002ねんのかんでおおきくのびているけど、なにかあったのかな。
\\	きじゅん
\\	かいけつさく
\\	こんぽんてきな
\\	かんわする
\\	こうりょする
\\	はんていする
\\	かんれんせい
\\	へいきん
\\	かいせい
\\	さいばん
\\	2001ねんと2002ねんのあいだでにんていすうがおおきくのびたりゆうがわかったわよ。2001ねん12がつに、にんていきじゅんのかいせいがあって、「かこ6かげつかん、つきへいきん80じかんのざんぎょうがつづいていれば、かろうとはつびょうとのかんれんせいがつよい」とはんていすることになったらしいの。それまでのきじゅんでは、びょうきになる1しゅうかんまえまでのろうどうじかんしかこうりょされなかったんだって。かろうしにんていきじゅんがずいぶんかんわされたということらしいわ。
\\	2001ねんと2002ねんのあいだでにんていすうがおおきくのびたりゆうがわかったわよ。2001ねん12がつに、にんていきじゅんのかいせいがあって、「かこ6かげつかん、つきへいきん80じかんのざんぎょうがつづいていれば、かろうとはつびょうとのかんれんせいがつよい」とはんていすることになったらしいの。それまでのきじゅんでは、びょうきになる1しゅうかんまえまでのろうどうじかんしかこうりょされなかったんだって。かろうしにんていきじゅんがずいぶんかんわされたということらしいわ。
\\	なるほどねぇ。にんていされないより、されたほうがいいけど、かろうしをへらすためのこんぽんてきなかいけつさくにはならないよね。
\\	なるほどねぇ。にんていされないより、されたほうがいいけど、かろうしをへらすためのこんぽんてきなかいけつさくにはならないよね。
\\	ぼくも、いまは、おとなになったら、ぜったいサービスざんぎょうなんかしないぞっておもうよ。でも、じっさいにかいしゃではたらくようになったら、ノーとはいえないだろうな。まわりのひとがやっていれば、じぶんだけやらないわけにはいかないとか、じょうしににらまれたくないとか、そういうきもちになるかもしれない。
\\	ぼくも、いまは、おとなになったら、ぜったいサービスざんぎょうなんかしないぞっておもうよ。でも、じっさいにかいしゃではたらくようになったら、ノーとはいえないだろうな。まわりのひとがやっていれば、じぶんだけやらないわけにはいかないとか、じょうしににらまれたくないとか、そういうきもちになるかもしれない。
\\	そういうにほんじんのかんがえかたが、このもんだいをひきおこすげんいんになっているのかもしれないね。
\\	そういうにほんじんのかんがえかたが、このもんだいをひきおこすげんいんになっているのかもしれないね。
\\	ゆうきあるひとたちが、かいしゃをあいてにさいばんをおこして、かいしゃがわにサービスざんぎょうのはいしやざんぎょうだいのみはらいをみとめさせているけど、しゃかいをかえるためには、ひとりひとりがそういうこえをあげていくことがたいせつなのよね。
\\	ゆうきあるひとたちが、かいしゃをあいてにさいばんをおこして、かいしゃがわにサービスざんぎょうのはいしやざんぎょうだいのみはらいをみとめさせているけど、しゃかいをかえるためには、ひとりひとりがそういうこえをあげていくことがたいせつなのよね。
\\	そうだよ、かろうししてからではおそいもん
\\	そうだよ、かろうししてからではおそいもん
\\	しかいしゃ
\\	ルールいはん
\\	たしかに
\\	やむをえない
\\	せんたく
\\	いほう
\\	ほうりついはん
\\	くりかえし
\\	(テレビ)「あさまではなさナイト」のおじかんです。
\\	(テレビ)「あさまではなさナイト」のおじかんです。
\\	みなさんは、じてんしゃのまえとうしろにこどもをのせる3にんのりがルールいはんだということをごぞんじだったでしょうか。きょうはこのじてんしゃのさんにんのりについて、ゲストのみなさんとおはなしをしていきたいとおもいます。
\\	みなさんは、じてんしゃのまえとうしろにこどもをのせる3にんのりがルールいはんだということをごぞんじだったでしょうか。きょうはこのじてんしゃのさんにんのりについて、ゲストのみなさんとおはなしをしていきたいとおもいます。
\\	まず、ベストセラーしょうせつ「ダンディズム」のさっか、うすいジョーさんです。(はくしゅ)
\\	まず、ベストセラーしょうせつ「ダンディズム」のさっか、うすいジョーさんです。(はくしゅ)
\\	こんばんは。うすい ジョー(うすいじょう)です。
\\	こんばんは。うすい ジョー(うすいじょう)です。
\\	ぼくはママチャリ3にんのりにはだい、だい、だいはんたいですね。っていうのも、ぼくはママチャリにたいへんきけんなめにあわされたことがあるんです。あれは..
\\	ぼくはママチャリ3にんのりにはだい、だい、だいはんたいですね。っていうのも、ぼくはママチャリにたいへんきけんなめにあわされたことがあるんです。あれは..
\\	うすいさん、うすいさん、そのおはなしはのちほどゆっくりおねがいしますね~。
\\	うすいさん、うすいさん、そのおはなしはのちほどゆっくりおねがいしますね~。
\\	それから、「ママをたすけるかい」のかいちょう、いいのかな(いいの かな)さんです。(はくしゅ)
\\	それから、「ママをたすけるかい」のかいちょう、いいのかな(いいの かな)さんです。(はくしゅ)
\\	よろしくおねがいいたします。
\\	よろしくおねがいいたします。
\\	じてんしゃの3にんのりはたしかにきけんです。でも、2りいじょうのちいさなこどもをもつおやにとっては、3にんのりはやむをえないせんたくだとおもいますね。
\\	じてんしゃの3にんのりはたしかにきけんです。でも、2りいじょうのちいさなこどもをもつおやにとっては、3にんのりはやむをえないせんたくだとおもいますね。
\\	「やむをえない」ってなんですか
\\	「やむをえない」って。
\\	「やむをえない」ってなんですか
\\	「やむをえない」って。
\\	ママチャリの3にんのりはいほうなんですよ
\\	いほう
\\	つまり、ほうりついはんなんです。わかってますか。
\\	ママチャリの3にんのりはいほうなんですよ
\\	いほう
\\	つまり、ほうりついはんなんです。わかってますか。
\\	うすいさんがさきほどから、くりかえしつかっておられる、「ママチャリ」というのはじてんしゃのまえぶぶんにカゴのついたかいものようじてんしゃのことですね。
\\	うすいさんがさきほどから、くりかえしつかっておられる、「ママチャリ」というのはじてんしゃのまえぶぶんにカゴのついたかいものようじてんしゃのことですね。
\\	そうですね。「ちゃりんこ」というのはじてんしゃのことなので、ママのじてんしゃといういみなんです。
\\	そうですね。「ちゃりんこ」というのはじてんしゃのことなので、ママのじてんしゃといういみなんです。
\\	おかあさんたちにとって、とてもべんりなじてんしゃです。
\\	おかあさんたちにとって、とてもべんりなじてんしゃです。
\\	ほいくえん
\\	なげく
\\	ほろびる
\\	しょうしか
\\	げんそく
\\	とない
\\	さんせい
\\	きせい
\\	かくご
\\	ばっきん
\\	こそだて
\\	かんきょう
\\	まちのひとびとは、じてんしゃのぜんごにこどもをのせる3にんのりがいほうだということをしっているのでしょうか。
\\	まちのひとびとは、じてんしゃのぜんごにこどもをのせる3にんのりがいほうだということをしっているのでしょうか。
\\	え?だめなんですか?しりませんでした。でもじゃぁ、どうやってほいくえんにつれていったらいいんでしょうね。
\\	え?だめなんですか?しりませんでした。でもじゃぁ、どうやってほいくえんにつれていったらいいんでしょうね。
\\	へぇー。だめなんだ。でも、よくみかけますよね。
\\	へぇー。だめなんだ。でも、よくみかけますよね。
\\	はぁ?ガソリンのねあげよりもこまりますよ
\\	こそだてちゅうのははおやたちのことをまったくかんがえていないですよね。ばっきんかくごでのります
\\	はぁ?ガソリンのねあげよりもこまりますよ
\\	こそだてちゅうのははおやたちのことをまったくかんがえていないですよね。ばっきんかくごでのります
\\	じっさい、こどもをのせたママチャリのうんてんはあぶないっすよね。ふらふらして、ころびそうになるし。きせいさんせい
\\	じっさい、こどもをのせたママチャリのうんてんはあぶないっすよね。ふらふらして、ころびそうになるし。きせいさんせい
\\	いかがでしょうか。インタビューをしたにんのはんすうはじてんしゃの3にんのりがいほうだということをしらなかったそうです。
\\	いかがでしょうか。インタビューをしたにんのはんすうはじてんしゃの3にんのりがいほうだということをしらなかったそうです。
\\	とないはほいくえんへのくるまでのそうげいがげんそくてきにきんしのため、2にんのこどもをもつおやは、3にんのりせざるをえないんです。せいじかやせいふは「しょうしかだ、しょうしかだ。このままじゃ、くにがほろびる」となげくまえに、こそだてをしやすいかんきょうづくりをかんがえるべきだとおもいます。
\\	とないはほいくえんへのくるまでのそうげいがげんそくてきにきんしのため、2にんのこどもをもつおやは、3にんのりせざるをえないんです。せいじかやせいふは「しょうしかだ、しょうしかだ。このままじゃ、くにがほろびる」となげくまえに、こそだてをしやすいかんきょうづくりをかんがえるべきだとおもいます。
\\	でも、ママチャリの3にんのりはほんとうにきけんです。ぼくなんかこのまえママチャリにひかれそうになったんですよ。
\\	でも、ママチャリの3にんのりはほんとうにきけんです。ぼくなんかこのまえママチャリにひかれそうになったんですよ。
\\	ほう。じてんしゃに。
\\	ほう。じてんしゃに。
\\	ええ。あるひ、みちをあるいていたら、じてんしゃのうしろにこどもをのせたママチャリがぼくのほうにまっすぐぐやってきたんです。
\\	ええ。あるひ、みちをあるいていたら、じてんしゃのうしろにこどもをのせたママチャリがぼくのほうにまっすぐぐやってきたんです。
\\	うんどうしんけい
\\	ばつぐん
\\	おんぶ
\\	もうスピード
\\	せまる
\\	かんせいのほうそく
\\	ぶじ
\\	~さまさま
\\	もちろん、うんどうしんけいばつぐんのぼくはひらりとみをかわして、そのママチャリをさけました。すると、こんどはこどもをじてんしゃのまえとうしろにのせた、ママさんがすごいスピードでむかってきたんです。
\\	もちろん、うんどうしんけいばつぐんのぼくはひらりとみをかわして、そのママチャリをさけました。すると、こんどはこどもをじてんしゃのまえとうしろにのせた、ママさんがすごいスピードでむかってきたんです。
\\	じてんしゃのまえとうしろと?つまり3にんのりですね?
\\	じてんしゃのまえとうしろと?つまり3にんのりですね?
\\	そうです。そしてはんたいほうこうからは、こどもをじてんしゃのまえとうしろにのせ、さらにあかちゃんをおんぶした、ママさんがもうスピードでせまってきていました。
\\	そうです。そしてはんたいほうこうからは、こどもをじてんしゃのまえとうしろにのせ、さらにあかちゃんをおんぶした、ママさんがもうスピードでせまってきていました。
\\	ちょっとまってください。じてんしゃのまえとうしろとおんぶ?つまり4にんのりですか?
\\	ちょっとまってください。じてんしゃのまえとうしろとおんぶ?つまり4にんのりですか?
\\	はい。みぎからママチャリ、ひだりからもママチャリ。うしろはかべです。そして、まえからは…
\\	はい。みぎからママチャリ、ひだりからもママチャリ。うしろはかべです。そして、まえからは…
\\	まえからは?
\\	まえからは?
\\	まえからはなにがきたんですか?
\\	まえからはなにがきたんですか?
\\	トラックがもうスピードでバックしてきました。
\\	トラックがもうスピードでバックしてきました。
\\	ええ?トラックが?で?
\\	ええ?トラックが?で?
\\	さいわいうんてんしゅがぼくにきづいて、きゅうブレーキをふんだので、たすかったってわけです。
\\	さいわいうんてんしゅがぼくにきづいて、きゅうブレーキをふんだので、たすかったってわけです。
\\	おかあさんたちとこどもたちは?
\\	おかあさんたちとこどもたちは?
\\	ぜんいんぶじ。というのもそのトラックがはこんでいたのはマットレスだったんです。かんせいのほうそくで、つんでいたマットレスがおちて、クッションがわりになったってわけです。
\\	ぜんいんぶじ。というのもそのトラックがはこんでいたのはマットレスだったんです。かんせいのほうそくで、つんでいたマットレスがおちて、クッションがわりになったってわけです。
\\	かんせいのほうそくさまさまですね。
\\	かんせいのほうそくさまさまですね。
\\	ちゅうしゃする
\\	やちん
\\	けいざいてき
\\	けいしちょう
\\	みとめる
\\	ふくすう
\\	さきほど
\\	がいしゅつ
\\	ようじ
\\	でもねぇ、うすいさん。じゃぁ、おかあさんたちはどうしたらいいんでしょうね。あかちゃんをひとりいえにおいてがいしゅつしたほうがあんぜんですか。まだあるけないこどもがふたりいるおかあさんはどうやってほいくえんまでつれていったらいいんでしょうか。
\\	でもねぇ、うすいさん。じゃぁ、おかあさんたちはどうしたらいいんでしょうね。あかちゃんをひとりいえにおいてがいしゅつしたほうがあんぜんですか。まだあるけないこどもがふたりいるおかあさんはどうやってほいくえんまでつれていったらいいんでしょうか。
\\	くるまでいけばいいでしょ。くるまがないなら、えきのちかくにすめばいいんですよ。
\\	くるまでいけばいいでしょ。くるまがないなら、えきのちかくにすめばいいんですよ。
\\	さっきももうしあげましたが、とないではくるまのそうげいをきんしするほいくえんがおおいんですよ。ちゅうしゃスペースがありませんからね。それから、さきほど「えきのちかくにすめばいい」っておっしゃいましたね。 もちろん、えきのちかくにすめるにこしたことはありませんが、えきからちかいところはやちんがたかいんです。こどもがいるかぞくにとってはけいざいてきにきびしいとおもいませんか。
\\	さっきももうしあげましたが、とないではくるまのそうげいをきんしするほいくえんがおおいんですよ。ちゅうしゃスペースがありませんからね。それから、さきほど「えきのちかくにすめばいい」っておっしゃいましたね。 もちろん、えきのちかくにすめるにこしたことはありませんが、えきからちかいところはやちんがたかいんです。こどもがいるかぞくにとってはけいざいてきにきびしいとおもいませんか。
\\	くるまをかうけいざいりょくも、えきのちかくにすむけいざいりょくもないなら、こどもをつくらなければいいんですよ。
\\	くるまをかうけいざいりょくも、えきのちかくにすむけいざいりょくもないなら、こどもをつくらなければいいんですよ。
\\	はぁ?あなた、なんてこというんですか
\\	ひどいにもほどがあります
\\	じぶんのいっていることがわかっているんですか
\\	あたまおかしいんじゃないですか?
\\	はぁ?あなた、なんてこというんですか
\\	ひどいにもほどがあります
\\	じぶんのいっていることがわかっているんですか
\\	あたまおかしいんじゃないですか?
\\	(ポカポカ殴る音)
\\	(ポカポカ殴る音)
\\	けいしちょうは「じてんしゃのさんにんのりはきけんでみとめられない。しかしふくすうのようじをもつかていのたいへんさはりかいできる。さんにんのりでもあんぜんにのることができるじてんしゃがあれば、さんにんのりをみとめる」とコメントをだしています。
\\	けいしちょうは「じてんしゃのさんにんのりはきけんでみとめられない。しかしふくすうのようじをもつかていのたいへんさはりかいできる。さんにんのりでもあんぜんにのることができるじてんしゃがあれば、さんにんのりをみとめる」とコメントをだしています。
\\	おかあさんがたが「あんぜんな3にんのりじてんしゃ」にのるにこしたことはありませんが、その「あんぜんな3にんのりじてんしゃ」は6まんえんいじょうになるといわれているんです。
\\	おかあさんがたが「あんぜんな3にんのりじてんしゃ」にのるにこしたことはありませんが、その「あんぜんな3にんのりじてんしゃ」は6まんえんいじょうになるといわれているんです。
\\	6まんえんですか?ふつうのじてんしゃは1まんから2まんえんですから、3ばいいじょうのねだんになるわけですね。
\\	6まんえんですか?ふつうのじてんしゃは1まんから2まんえんですから、3ばいいじょうのねだんになるわけですね。
\\	だから、けいざいりょくがないなら…
\\	だから、けいざいりょくがないなら…
\\	(ポカポカ殴る音)
\\	(ポカポカ殴る音)
\\	いちにちもはやくやすくてあんぜんな3にんのりじてんしゃができるといいですね。では、このへんで。ごきげんよう。
\\	いちにちもはやくやすくてあんぜんな3にんのりじてんしゃができるといいですね。では、このへんで。ごきげんよう。
\\	おおう
\\	こうきあつ
\\	みこみ
\\	もよう
\\	きしょう よほうし
\\	おまけに
\\	こいつ
\\	ふこうへい
\\	クミ、そんなに、テレビをちかくでみたらめがわるくなるぞ。
\\	クミ、そんなに、テレビをちかくでみたらめがわるくなるぞ。
\\	うん…。
\\	うん…。
\\	では、ぜんこくのあしたのおてんきです。あすは、にほんふきんをおおうこうきあつのえいきょうで、はれるところがおおいみこみです。ただし、しこくからかんとうのたいへいようがわではあめがふりやすくなるでしょう。おおさかや、なごやはひるまえから、とうきょうはひるすぎからあめとなるので、あさ、あめがふっていなくても、わすれずにかさをおもちください。おきなわはくもりがちのてんきがつづくもようです。
\\	では、ぜんこくのあしたのおてんきです。あすは、にほんふきんをおおうこうきあつのえいきょうで、はれるところがおおいみこみです。ただし、しこくからかんとうのたいへいようがわではあめがふりやすくなるでしょう。おおさかや、なごやはひるまえから、とうきょうはひるすぎからあめとなるので、あさ、あめがふっていなくても、わすれずにかさをおもちください。おきなわはくもりがちのてんきがつづくもようです。
\\	えー。あしたはあめかぁ。
\\	えー。あしたはあめかぁ。
\\	ねぇ、けんちゃん、このひと、カッコいいね。
\\	ねぇ、けんちゃん、このひと、カッコいいね。
\\	あ、こいつね。おれ、だいがくじだい、おなじクラスだったんだ。あたまもいいし、せいかくもいい。おまけに、こいつのおやはせいじかでちょうかねもち。よのなか、ふこうへいだよなー。
\\	あ、こいつね。おれ、だいがくじだい、おなじクラスだったんだ。あたまもいいし、せいかくもいい。おまけに、こいつのおやはせいじかでちょうかねもち。よのなか、ふこうへいだよなー。
\\	けんちゃん。おねがい
\\	しょうかいして。
\\	けんちゃん。おねがい
\\	しょうかいして。
\\	え?
\\	え?
\\	じょ、じょうだんよ。じょうだん。わたしには、けんちゃんがいちばん。
\\	じょ、じょうだんよ。じょうだん。わたしには、けんちゃんがいちばん。
\\	うかれる
\\	ぶんめん
\\	スカイプ
\\	そうしん
\\	しゃこうじれい
\\	ゆうかちゃんから、ねんがじょうもらっちゃったよ。(はなうた)
\\	ゆうかちゃんから、ねんがじょうもらっちゃったよ。(はなうた)
\\	なに、うかれてるんだか。
\\	なに、うかれてるんだか。
\\	じゃ、おまえ、あんなかわいいこからねんがじょうもらえるものなら、もらってみろよ
\\	しかも、このぶんめん、みてくれよ。
\\	じゃ、おまえ、あんなかわいいこからねんがじょうもらえるものなら、もらってみろよ
\\	しかも、このぶんめん、みてくれよ。
\\	「ことしは、ぜひ、うえはらさんといっしょにゴルフさせていただきたいです
\\	だぜ~
\\	「ことしは、ぜひ、うえはらさんといっしょにゴルフさせていただきたいです
\\	だぜ~
\\	これで、さそわないのは、しつれいだよなっ
\\	これで、さそわないのは、しつれいだよなっ
\\	はぁ、おまえ、ねつでもあるんじゃないのか?
\\	はぁ、おまえ、ねつでもあるんじゃないのか?
\\	ほんとうに、あのゆうかちゃんをさそえるもんなら、さそってみろよ。
\\	ほんとうに、あのゆうかちゃんをさそえるもんなら、さそってみろよ。
\\	お~、みてろよ。さっそく、スカイプ、スカイプ。
\\	お~、みてろよ。さっそく、スカイプ、スカイプ。
\\	「ゆうかちゃん、ねんがじょうありがとう。ゴルフするなんて、しらなかったよ。さっそくだけど、こんどのにちようびは、どう?」 そうしん
\\	「ゆうかちゃん、ねんがじょうありがとう。ゴルフするなんて、しらなかったよ。さっそくだけど、こんどのにちようびは、どう?」 そうしん
\\	(ちゃくしんおん)
\\	(ちゃくしんおん)
\\	くちにあう
\\	くちうるさい
\\	くちだしする
\\	くちがうまい
\\	くちをきく
\\	くちをはさむ
\\	あとをつける
\\	くちごもる
\\	いいかげん〜して
\\	くちをつける
\\	ぎゃくこうか
\\	なぁ、どうしたんだよ。いいかげん、くちをきいてくれよ。のみものにもりょうりにもぜんぜんくちをつけないし。
\\	なぁ、どうしたんだよ。いいかげん、くちをきいてくれよ。のみものにもりょうりにもぜんぜんくちをつけないし。
\\	あ、「こんなやすいりょうりはわたしのくちにあいません。」なんていうのか?
\\	あ、「こんなやすいりょうりはわたしのくちにあいません。」なんていうのか?
\\	・・・せんしゅうの・・・せんしゅうのにちようびどこにいってたのよ….
\\	・・・せんしゅうの・・・せんしゅうのにちようびどこにいってたのよ….
\\	えっ
\\	えっ
\\	いきさきをきいたときに、くちごもっていたから、きになって、あなたのあとをつけたのよ。
\\	いきさきをきいたときに、くちごもっていたから、きになって、あなたのあとをつけたのよ。
\\	おまえ
\\	そんなことしたのか?
\\	しんじられない
\\	おまえ
\\	そんなことしたのか?
\\	しんじられない
\\	そりゃいけないよ、おくさん。
\\	そりゃいけないよ、おくさん。
\\	そういうことは、めをつぶっていたほうがいいとおもいますよ。
\\	そういうことは、めをつぶっていたほうがいいとおもいますよ。
\\	は?あなただれですか。かんけいないかたがくちをはさまないでください。
\\	は?あなただれですか。かんけいないかたがくちをはさまないでください。
\\	しんじられないのはこっちよ。もう、あなたとはくちもききたくないわ。
\\	しんじられないのはこっちよ。もう、あなたとはくちもききたくないわ。
\\	ごかいだよ。ごかい。おれがあいしているのは、きみだけなんだからさ。
\\	ごかいだよ。ごかい。おれがあいしているのは、きみだけなんだからさ。
\\	よっ、だんなさん。くちがうまいねぇ。
\\	よっ、だんなさん。くちがうまいねぇ。
\\	だから
\\	くちだししないでください
\\	っていったでしょ
\\	だから
\\	くちだししないでください
\\	っていったでしょ
\\	ひざかけ
\\	ずぼんした
\\	たいかんおんど
\\	すいしょうする
\\	せっていする
\\	さくげん
\\	はいしゅつりょう
\\	にさんかたんそ
\\	クールビズ
\\	ウォームビズ
\\	おくれをとる
\\	(オフィスにて)
\\	(オフィスにて)
\\	あっ、ひなこちゃん、かわいいひざかけしてるね
\\	あっ、ひなこちゃん、かわいいひざかけしてるね
\\	はい、ウォームビズですからね。
\\	はい、ウォームビズですからね。
\\	クールビズはしってるけど。ウォームビズってなんだっけ?
\\	クールビズはしってるけど。ウォームビズってなんだっけ?
\\	うえはらさん、おくれてますよ。
\\	うえはらさん、おくれてますよ。
\\	じゅういちがつからさんがつは、ウォームビズきかんで、にさんかたんそのはいしゅつりょうさくげんのために、オフィスのだんぼうをにじゅうさんどからにじゅうどにさげてせっていすることがすいしょうされているんです。
\\	じゅういちがつからさんがつは、ウォームビズきかんで、にさんかたんそのはいしゅつりょうさくげんのために、オフィスのだんぼうをにじゅうさんどからにじゅうどにさげてせっていすることがすいしょうされているんです。
\\	だから、さいきん、オフィスがさむいんだなぁ。
\\	だから、さいきん、オフィスがさむいんだなぁ。
\\	ぼくとしたことが、しらなかったなぁ~~。
\\	ぼくとしたことが、しらなかったなぁ~~。
\\	ひざかけをすると、たいかんおんどがにてんごどもあがるそうですよ。
\\	ひざかけをすると、たいかんおんどがにてんごどもあがるそうですよ。
\\	ゆうせんせき
\\	ゆずる
\\	どうどうと
\\	ふきん
\\	りょうて うち
\\	~なる いっぽう
\\	おち
\\	ぎゃくぎれ
\\	めつき
\\	ひかえる
\\	マナーモード
\\	ゆうき
\\	「ゆうせんせきふきんでは、けいたいでんわのでんげんはおきりください。
\\	「ゆうせんせきふきんでは、けいたいでんわのでんげんはおきりください。
\\	それいがいのばしょでは、マナーモードにせっていのうえ、つうわはおひかえください。」
\\	それいがいのばしょでは、マナーモードにせっていのうえ、つうわはおひかえください。」
\\	まったく、このひと、わかいくせにどうどうとゆうせんせきにすわって、そのうえ、けいたいメールまでしてるなんて、ゆるせないわ
\\	まったく、このひと、わかいくせにどうどうとゆうせんせきにすわって、そのうえ、けいたいメールまでしてるなんて、ゆるせないわ
\\	ひとこと、いってやろうかしら。
\\	ひとこと、いってやろうかしら。
\\	あっ、でも、めつきがわるいし、ぎゃくぎれされるのがおちかな。
\\	あっ、でも、めつきがわるいし、ぎゃくぎれされるのがおちかな。
\\	だめだめ、そんなことだから、にほんじんのマナーがわるくなるいっぽうなのよ。
\\	だめだめ、そんなことだから、にほんじんのマナーがわるくなるいっぽうなのよ。
\\	ここは、ゆうきをださなくちゃ。
\\	ここは、ゆうきをださなくちゃ。
\\	はなうた
\\	はなをあかす
\\	はなにかける
\\	はながきく
\\	はなにつく
\\	せいぜい
\\	どうき
\\	はながたかい
\\	ひょうしょうする
\\	ベストセールスマン
\\	はなをへしおる
\\	フフフン
\\	フフフン
\\	どうしたんだ?めずらしいな。はなうたなんかうたって。
\\	どうしたんだ?めずらしいな。はなうたなんかうたって。
\\	こんど、おれ、ベストセールスマンにひょうしょうされるんだ。
\\	こんど、おれ、ベストセールスマンにひょうしょうされるんだ。
\\	へぇえええ。す、すごいなぁ。おれも、どうきとして、はながたかいよ。おめでとう。
\\	へぇえええ。す、すごいなぁ。おれも、どうきとして、はながたかいよ。おめでとう。
\\	ま、うえはらも、せいぜいがんばれよ。おまえ、はながきくだろ。
\\	ま、うえはらも、せいぜいがんばれよ。おまえ、はながきくだろ。
\\	はながきく?
\\	はながきく?
\\	だって、かってくれそうなきゃくをみつけるのとくいじゃないか。
\\	だって、かってくれそうなきゃくをみつけるのとくいじゃないか。
\\	クンクンクンクン。かってください。ワン
\\	ワン
\\	ははは
\\	クンクンクンクン。かってください。ワン
\\	ワン
\\	ははは
\\	なんだ、そのいいかた。はなにつくな。
\\	なんだ、そのいいかた。はなにつくな。
\\	ベストセールスマンになったからって、そんなにはなにかけるなよ。
\\	ベストセールスマンになったからって、そんなにはなにかけるなよ。
\\	じにんする
\\	ふしょうじ
\\	せいか
\\	たてなおす
\\	しょりする
\\	すじ
\\	ミートミートのさいとうしゃちょう、じにんするらしいぞ
\\	ミートミートのさいとうしゃちょう、じにんするらしいぞ
\\	え?こんかいのふしょうじのせきにんをとってじにんするっていうこと?
\\	え?こんかいのふしょうじのせきにんをとってじにんするっていうこと?
\\	そうらしいな。
\\	そうらしいな。
\\	さいとうしゃちょうは、せきにんをもってこんごのたいさくをたてるべきよ。
\\	さいとうしゃちょうは、せきにんをもってこんごのたいさくをたてるべきよ。
\\	しゃちょうにはそういうせきにんがあるとおもうわ。それからやめてもおそくないんじゃない?
\\	しゃちょうにはそういうせきにんがあるとおもうわ。それからやめてもおそくないんじゃない?
\\	たしかに、そういうせきにんのとりかたもあるな。
\\	たしかに、そういうせきにんのとりかたもあるな。
\\	このまえのあまからせいかのしゃちょうは、すぐにじにんせずに、がんばってかいしゃをたてなおしたじゃない。
\\	このまえのあまからせいかのしゃちょうは、すぐにじにんせずに、がんばってかいしゃをたてなおしたじゃない。
\\	それが、しゃちょうとしてのせきにんあるたいどだとおもうわ。
\\	それが、しゃちょうとしてのせきにんあるたいどだとおもうわ。
\\	かたおもい
\\	ふしぎちゃん
\\	はぐれる
\\	そうなんする
\\	(でんわが)つうじる
\\	まちはずれ
\\	まぼろし
\\	たぬき
\\	まりこちゃん、かたおもいしているっていっていたよね? すきなひとってどんなひと?
\\	まりこちゃん、かたおもいしているっていっていたよね? すきなひとってどんなひと?
\\	えっ?う~ん、じつは、なまえ、しらないんだぁ。
\\	えっ?う~ん、じつは、なまえ、しらないんだぁ。
\\	いちどあったきりのひとだから。
\\	いちどあったきりのひとだから。
\\	(いちどあったきりか…。さすがふしぎちゃん。
\\	(いちどあったきりか…。さすがふしぎちゃん。
\\	ということは、まだ、おれにもかのうせいがあるってことか?)それって、どういうこと?
\\	ということは、まだ、おれにもかのうせいがあるってことか?)それって、どういうこと?
\\	ぜったい、わらわないでよ。。。
\\	ぜったい、わらわないでよ。。。
\\	ちかうよ。
\\	ちかうよ。
\\	じつはね、こうこうせいのとき、ともだちとやまへハイキングにでかけたんだけど、ともだちとはぐれちゃって、そうなんしそうになったことがあるの。
\\	じつはね、こうこうせいのとき、ともだちとやまへハイキングにでかけたんだけど、ともだちとはぐれちゃって、そうなんしそうになったことがあるの。
\\	やまおくだから、とうぜんでんわもつうじないでしょ。
\\	やまおくだから、とうぜんでんわもつうじないでしょ。
\\	そのうち、のみものはのみきってしまうし、たべものもたべきってしまって、ほんとうにこまりきっていたときに、とつぜん、かれが、どこからかあらわれたのよ。
\\	そのうち、のみものはのみきってしまうし、たべものもたべきってしまって、ほんとうにこまりきっていたときに、とつぜん、かれが、どこからかあらわれたのよ。
\\	それで?
\\	それで?
\\	かれは、むごんのままわたしのてをひいて、あるいていったの。
\\	かれは、むごんのままわたしのてをひいて、あるいていったの。
\\	ひとぎき
\\	おふろあがり
\\	えだまめ
\\	かならず
\\	たのみごと
\\	ばれる
\\	しんさく
\\	すてき
\\	たぬきねいり
\\	あぁ、いいゆだった。
\\	あぁ、いいゆだった。
\\	パパ、まずは、キーンとひえたビールでもどうぞ。
\\	パパ、まずは、キーンとひえたビールでもどうぞ。
\\	どうしたんだよ?いやにサービスがいいなぁ。
\\	どうしたんだよ?いやにサービスがいいなぁ。
\\	あら、ひとぎきのわるいことはいわないで
\\	いつも、サービスいいでしょ。
\\	あら、ひとぎきのわるいことはいわないで
\\	いつも、サービスいいでしょ。
\\	え~そうかぁ??まっ、そういうことにしておくよ。
\\	え~そうかぁ??まっ、そういうことにしておくよ。
\\	ゴクゴクゴク。あぁ~、ふろあがりはひえたビールにかぎる
\\	ゴクゴクゴク。あぁ~、ふろあがりはひえたビールにかぎる
\\	さっ、えだまめもどうぞ~。
\\	さっ、えだまめもどうぞ~。
\\	おっ
\\	サンキュウ~。やっぱり、ビールのつまみは、えだまめにかぎるよな~。
\\	おっ
\\	サンキュウ~。やっぱり、ビールのつまみは、えだまめにかぎるよな~。
\\	ところで、なんだい?はなしは?
\\	ところで、なんだい?はなしは?
\\	えっ
\\	なんでわかったの?
\\	えっ
\\	なんでわかったの?
\\	だって、きみがサービスよくしてくれるときは、かならずなにかたのみごとがあるじゃないか。
\\	だって、きみがサービスよくしてくれるときは、かならずなにかたのみごとがあるじゃないか。
\\	ばれちゃ、しょうがないわ。
\\	ばれちゃ、しょうがないわ。
\\	じつはねぇ、こんどのヴィトンのしんさくバック、とってもすてきなのよ~~。かって~。
\\	じつはねぇ、こんどのヴィトンのしんさくバック、とってもすてきなのよ~~。かって~。
\\	てまき
\\	みかく
\\	きゅうかく
\\	しかくしょうがい
\\	めかくし
\\	におい
\\	したざわり
\\	あてる
\\	しょっかく
\\	すし
\\	するどい
\\	きのう、いえで、てまきずしをたべたとき、ちょっとおもしろいゲームをやったよ。
\\	きのう、いえで、てまきずしをたべたとき、ちょっとおもしろいゲームをやったよ。
\\	えっ
\\	どんなゲームですか。
\\	えっ
\\	どんなゲームですか。
\\	ほかのひとがつくったてまきずしをめをつぶったままたべて、なかみのぐをあてるというゲームだ。
\\	ほかのひとがつくったてまきずしをめをつぶったままたべて、なかみのぐをあてるというゲームだ。
\\	へぇ~、おもしろそうですね。
\\	へぇ~、おもしろそうですね。
\\	だろう?めをつぶっているから、したざわりとか、におい、しょっかんがたいせつなんだ。
\\	だろう?めをつぶっているから、したざわりとか、におい、しょっかんがたいせつなんだ。
\\	けっこう、むずかしそうですね。
\\	けっこう、むずかしそうですね。
\\	めをつぶってたべるというてんでは、ブラインドレストランにつうじるものがありますね。
\\	めをつぶってたべるというてんでは、ブラインドレストランにつうじるものがありますね。
\\	なんだ、そのブラインドレストランっていうのは?
\\	なんだ、そのブラインドレストランっていうのは?
\\	めかくしをしてしょくじするレストランのことですよ。
\\	めかくしをしてしょくじするレストランのことですよ。
\\	しかくしょうがいをもつひとにたいするりかいをふかめることをもくてきにヨーロッパではじめられたものなんです。
\\	しかくしょうがいをもつひとにたいするりかいをふかめることをもくてきにヨーロッパではじめられたものなんです。
\\	ほ~。うえはら、いがいとものしりじゃないか。
\\	ほ~。うえはら、いがいとものしりじゃないか。
\\	あしがちにつかない
\\	いくら
\\	あしがはやい
\\	あしをのばす
\\	あしがでる
\\	ちょくげき
\\	ゆうきゅうきゅうか
\\	ねりなおす
\\	あしがとおのく
\\	あしたは、たかおくんとデートだ~。
\\	あしたは、たかおくんとデートだ~。
\\	わくわくするな~
\\	なんだか、あしがちにつかないかんじ
\\	さぁて、おべんとうは、なんにしようかな~。たかおくん、イクラすきだから、イクラのおにぎりにしようかな。
\\	わくわくするな~
\\	なんだか、あしがちにつかないかんじ
\\	さぁて、おべんとうは、なんにしようかな~。たかおくん、イクラすきだから、イクラのおにぎりにしようかな。
\\	あ、でも、いくらは、あしがはやいから、やめておいたほうがいいわね。。。
\\	あ、でも、いくらは、あしがはやいから、やめておいたほうがいいわね。。。
\\	あしたは、のりかちゃんとデートか。。。
\\	あしたは、のりかちゃんとデートか。。。
\\	てんぷ
\\	さいせい
\\	こくはく
\\	だうんろーど
\\	どうが
\\	たしか
\\	かくちょうし
\\	まめに
\\	さいきどう
\\	たちあがる
\\	ふぁいる
\\	ちょうし
\\	きょうゆう
\\	てすう
\\	うえはらさん、おはようございます。
\\	うえはらさん、おはようございます。
\\	あのぉ、てんぷファイルがあけられないんですが、みていただけますか。
\\	あのぉ、てんぷファイルがあけられないんですが、みていただけますか。
\\	じゃぁ、かのきょうゆうフォルダにいれておいてよ。
\\	じゃぁ、かのきょうゆうフォルダにいれておいてよ。
\\	いま、パソコンがたちあがったとおもったら、フリーズしちゃって、さいきどうしているところなんだ。
\\	いま、パソコンがたちあがったとおもったら、フリーズしちゃって、さいきどうしているところなんだ。
\\	さいきん、うえはらさんのコンピュータ、ちょうしわるいですね。
\\	さいきん、うえはらさんのコンピュータ、ちょうしわるいですね。
\\	うん、まめに、デフラグしないといけないな。
\\	うん、まめに、デフラグしないといけないな。
\\	ところで、そのファイルのかくちょうしはなんだった?
\\	ところで、そのファイルのかくちょうしはなんだった?
\\	たしか、
\\	だったとおもいます。
\\	たしか、
\\	だったとおもいます。
\\	なんだ、それなら、どうがファイルだから、
\\	でひらくはずだよ。
\\	なんだ、それなら、どうがファイルだから、
\\	でひらくはずだよ。
\\	それ、わたしの
\\	にはいってないとおもいます。じゃぁ、ダウンロードしなくちゃいけないんですね。
\\	それ、わたしの
\\	にはいってないとおもいます。じゃぁ、ダウンロードしなくちゃいけないんですね。
\\	あ、ちょっとまって。やっと、パソコンがたちあがった。どれどれ。。。
\\	あ、ちょっとまって。やっと、パソコンがたちあがった。どれどれ。。。
\\	くむ
\\	うらぎる
\\	なかをとりもつ
\\	しかた(が)ない
\\	みずくさい
\\	ないしょ
\\	ことわる
\\	つきあいがわるい
\\	どうりで
\\	わけ
\\	あれ?あそこにいるの、けんとあやかじゃない?
\\	あれ?あそこにいるの、けんとあやかじゃない?
\\	あ
\\	うでなんかくんじゃって、あのふたり、つきあってるっての?
\\	あ
\\	うでなんかくんじゃって、あのふたり、つきあってるっての?
\\	どうりで、あやか、さいきんつきあいがわるくなったわけだ。
\\	どうりで、あやか、さいきんつきあいがわるくなったわけだ。
\\	わたしたちがさそっても、りゆうをつけて、ことわっていたのには、こういうわけがあったのね。
\\	わたしたちがさそっても、りゆうをつけて、ことわっていたのには、こういうわけがあったのね。
\\	でも、ないしょにするなんて、みずくさいよねぇ。
\\	でも、ないしょにするなんて、みずくさいよねぇ。
\\	そうだね。でも、まぁ・・・しかたないか・・・。
\\	そうだね。でも、まぁ・・・しかたないか・・・。
\\	なんでよ?なにかわけがあるの?
\\	なんでよ?なにかわけがあるの?
\\	じつは、さゆりもけんのことがずーっとすきだったんだよね。
\\	じつは、さゆりもけんのことがずーっとすきだったんだよね。
\\	それで、さゆりは、あやかに、「けんとのなかをとりもって」ってたのんでいたのよ。
\\	それで、さゆりは、あやかに、「けんとのなかをとりもって」ってたのんでいたのよ。
\\	みかける
\\	いたばさみ
\\	ゆうじょう
\\	あいじょう
\\	ぎゃくに
\\	こくはくする
\\	あきらめる
\\	おもいつめる
\\	うちあける
\\	なかなか~ない
\\	びみょう
\\	あやか、けんとつきあってるの?
\\	あやか、けんとつきあってるの?
\\	きのう、けんといっしょにいるところみかけたんだけど。
\\	きのう、けんといっしょにいるところみかけたんだけど。
\\	う、うん。じつはね。。。いおういおうとおもってたんだけど、なかなかいいだせなくて。。
\\	う、うん。じつはね。。。いおういおうとおもってたんだけど、なかなかいいだせなくて。。
\\	どういうことなの?さゆりをうらぎったってわけ?
\\	どういうことなの?さゆりをうらぎったってわけ?
\\	うらぎるつもりはなかったの。
\\	うらぎるつもりはなかったの。
\\	じつは、わたしもずっとけんのことがすきだったのよ。でも、さゆりからけんのことをうちあけられたとき、ほんとうのことがいえなくて。。。
\\	じつは、わたしもずっとけんのことがすきだったのよ。でも、さゆりからけんのことをうちあけられたとき、ほんとうのことがいえなくて。。。
\\	さゆりも、おもいつめていたみたいだったから、けんのことはあきらめようっておもったわけ。
\\	さゆりも、おもいつめていたみたいだったから、けんのことはあきらめようっておもったわけ。
\\	それで、なんでつきあうことになったのよ?
\\	それで、なんでつきあうことになったのよ?
\\	けんに、さゆりのきもちをつたえにいったら、ぎゃくに、けんからこくはくされちゃって。
\\	けんに、さゆりのきもちをつたえにいったら、ぎゃくに、けんからこくはくされちゃって。
\\	もうじぶんのきもちにうそをつけなくて。。。
\\	もうじぶんのきもちにうそをつけなくて。。。
\\	そう。。あやかも、あいじょうとゆうじょうのいたばさみってくるしんでたってわけね。
\\	そう。。あやかも、あいじょうとゆうじょうのいたばさみってくるしんでたってわけね。
\\	それなら、さゆりにしょうじきにはなしたほうがいいよ。
\\	それなら、さゆりにしょうじきにはなしたほうがいいよ。
\\	こうれい
\\	かぞくりょこう
\\	コネクションルーム
\\	プレイルーム
\\	りょこうだいりてん
\\	しんせつする
\\	ちかば
\\	さんせだい
\\	いちどう
\\	しんせき
\\	ねんぱい
\\	いらっしゃいませ。 
\\	いらっしゃいませ。 
\\	あの~、こうれいのりょうしんとようじをつれて、いっぱくふつかのりょこうにいきたいんだけど。。。
\\	あの~、こうれいのりょうしんとようじをつれて、いっぱくふつかのりょこうにいきたいんだけど。。。
\\	ぜんいんでなんめいさまですか。
\\	ぜんいんでなんめいさまですか。
\\	えっと、りょうしんと、あねのかぞく4にんと、ぼくのかぞく4にんだから、ぜんぶで10にんだね。
\\	えっと、りょうしんと、あねのかぞく4にんと、ぼくのかぞく4にんだから、ぜんぶで10にんだね。
\\	ごしんせきごいちどうでいかれるんですね。
\\	ごしんせきごいちどうでいかれるんですね。
\\	うん、そう。だから、さんせだいのかぞくりょこうむきのプランがいいんだよね。
\\	うん、そう。だから、さんせだいのかぞくりょこうむきのプランがいいんだよね。
\\	そうですね。ちかばで、はこねはいかがですか。
\\	そうですね。ちかばで、はこねはいかがですか。
\\	やっぱり、はこねかね。じゃ、ホテルは、どこかおすすめある?
\\	やっぱり、はこねかね。じゃ、ホテルは、どこかおすすめある?
\\	しょうしょうおまちください。
\\	しょうしょうおまちください。
\\	(キーボードをたたくおと)ホテルあしのこのゆは、いかがでしょうか。
\\	(キーボードをたたくおと)ホテルあしのこのゆは、いかがでしょうか。
\\	ここは、さいきん、おこさまむけにプレイルームをしんせつしたばかりなんです。
\\	ここは、さいきん、おこさまむけにプレイルームをしんせつしたばかりなんです。
\\	それに、ごねんぱいのおきゃくさまむけのおしょくじもよういされています。
\\	それに、ごねんぱいのおきゃくさまむけのおしょくじもよういされています。
\\	みずもれ
\\	はいけんする
\\	げんぶつ
\\	さぎょういん
\\	ではらう
\\	もれる
\\	ねもと
\\	じゃぐち
\\	しゅうり
\\	みつもり
\\	もしもし、みずもれのしゅうりをおねがいしたいんですが。
\\	もしもし、みずもれのしゅうりをおねがいしたいんですが。
\\	はい、どういったしょうじょうですか。
\\	はい、どういったしょうじょうですか。
\\	だいどころのじゃぐちのねもとのあたりから、みずがピューピューもれるんです。
\\	だいどころのじゃぐちのねもとのあたりから、みずがピューピューもれるんです。
\\	さいしょのうちは、なんとかがまんしてつかってたんですけど、だんだんひどくなってきてしまって。。。
\\	さいしょのうちは、なんとかがまんしてつかってたんですけど、だんだんひどくなってきてしまって。。。
\\	いますぐにきてもらえませんか。
\\	いますぐにきてもらえませんか。
\\	いま、さぎょういんがではらっていまして。。。はやくても、あしたのおひるすぎになってしまいます。
\\	いま、さぎょういんがではらっていまして。。。はやくても、あしたのおひるすぎになってしまいます。
\\	あしたのごごですか。。。なんとか、あさのうちにきてもらえませんか。
\\	あしたのごごですか。。。なんとか、あさのうちにきてもらえませんか。
\\	うーん、わかりました。じゃ、なんとか、ちょうせいしてみます。
\\	うーん、わかりました。じゃ、なんとか、ちょうせいしてみます。
\\	おねがいします。それで、あしたすぐなおるんですよね。
\\	おねがいします。それで、あしたすぐなおるんですよね。
\\	はぁ、それは、じょうきょうによりますから。。。そのばでなおせないばあいもあります。
\\	はぁ、それは、じょうきょうによりますから。。。そのばでなおせないばあいもあります。
\\	じゃ、しゅうりだいは、だいたいいくらくらいかかりますか。
\\	じゃ、しゅうりだいは、だいたいいくらくらいかかりますか。
\\	ですから、げんぶつをはいけんしないと、おみつもりもできないのですが、、、
\\	ですから、げんぶつをはいけんしないと、おみつもりもできないのですが、、、
\\	とくしゅな
\\	しかも
\\	もうける
\\	いたる
\\	てんけんだい
\\	しゅっちょう
\\	ざっくり
\\	パッキン
\\	ぶひん
\\	そのて
\\	あ~、これは、とくしゅなすいせんですね。
\\	あ~、これは、とくしゅなすいせんですね。
\\	ぶひんこうかんではなおらないです。
\\	ぶひんこうかんではなおらないです。
\\	すいせんをぜんぶとりかえることになりますね。
\\	すいせんをぜんぶとりかえることになりますね。
\\	え~、パッキンのこうかんだけですむとおもってたんだけど?
\\	え~、パッキンのこうかんだけですむとおもってたんだけど?
\\	いえ、げんぶつをはいけんしてからでないと、わからないともうしあげたとおもいますが。。。
\\	いえ、げんぶつをはいけんしてからでないと、わからないともうしあげたとおもいますが。。。
\\	それで、ぜんぶとりかえたら、いくらくらいかかるの?
\\	それで、ぜんぶとりかえたら、いくらくらいかかるの?
\\	そうですね。ざっくりみつもって、7~8まんえんくらいになるとおもいます。
\\	そうですね。ざっくりみつもって、7~8まんえんくらいになるとおもいます。
\\	そんなにたかいの?
\\	そんなにたかいの?
\\	しゅじんとそうだんしてからでないと、きめられないわ。
\\	しゅじんとそうだんしてからでないと、きめられないわ。
\\	あとで、せいしきなみつもりしょおくってくださいね。
\\	あとで、せいしきなみつもりしょおくってくださいね。
\\	はい、わかりました。
\\	はい、わかりました。
\\	じゃ、きょうは、しゅっちょうてんけんだいとして3000えんいただきます。
\\	じゃ、きょうは、しゅっちょうてんけんだいとして3000えんいただきます。
\\	は?
\\	そんなのきいてないわよ。
\\	は?
\\	そんなのきいてないわよ。
\\	しゅうりにいたらなかったばあいは、てんけんだいをいただくことになっていますので。
\\	しゅうりにいたらなかったばあいは、てんけんだいをいただくことになっていますので。
\\	まったく
\\	しっかりしてるわね
\\	まったく
\\	しっかりしてるわね
\\	じつは、てんけんだいだけでもうけてるんじゃないの?
\\	じつは、てんけんだいだけでもうけてるんじゃないの?
\\	ただいま。
\\	ただいま。
\\	このみずもれ、すいせんをぜんぶとりかえなきゃなおらないって。
\\	このみずもれ、すいせんをぜんぶとりかえなきゃなおらないって。
\\	しかも、7~8まんもかかるって
\\	しかも、7~8まんもかかるって
\\	あー、やっぱり。
\\	あー、やっぱり。
\\	そのてのぎょうしゃは、すぐに、とりかえないとなおらないって、いうらしいぞ
\\	そのてのぎょうしゃは、すぐに、とりかえないとなおらないって、いうらしいぞ
\\	え、うそー
\\	え、うそー
\\	ふせいゆうし
\\	はらぐろい
\\	くくる
\\	いくさ
\\	やしなう
\\	たたかう
\\	てんきん
\\	かいがいふにん
\\	こくはつ
\\	はら
\\	おれ、はらをくくったから。
\\	おれ、はらをくくったから。
\\	えっ、なによ、いきなり?
\\	えっ、なによ、いきなり?
\\	かいしゃのふせいゆうしをこくはつすることにした。
\\	かいしゃのふせいゆうしをこくはつすることにした。
\\	えぇ
\\	なに?ふせいゆうし?こくはつ? 
\\	えぇ
\\	なに?ふせいゆうし?こくはつ? 
\\	もうちょっとくわしくせつめいしてよ。
\\	もうちょっとくわしくせつめいしてよ。
\\	このまえ、けいりのしりょうをみてたら、けいりたんとうのやくいんがふせいゆうしをしていることにきづいたんだよ。
\\	このまえ、けいりのしりょうをみてたら、けいりたんとうのやくいんがふせいゆうしをしていることにきづいたんだよ。
\\	それで、ぶちょうにそうだんしようとしたら、「はらをさってはなそう」っていわれたんだ。
\\	それで、ぶちょうにそうだんしようとしたら、「はらをさってはなそう」っていわれたんだ。
\\	ぶちょうもしってたらしい。
\\	ぶちょうもしってたらしい。
\\	だから、おれにもめをつぶっているようにせっとくしようとしたんだよ。
\\	だから、おれにもめをつぶっているようにせっとくしようとしたんだよ。
\\	そのうち、「このことは、ぼくのはらにしまっておくから。あ、たしか、くんはかいがいふにんをきぼうしていたねぇ・・・」なんていいだして・・・。
\\	そのうち、「このことは、ぼくのはらにしまっておくから。あ、たしか、くんはかいがいふにんをきぼうしていたねぇ・・・」なんていいだして・・・。
\\	きっと、ぼくをてんきんさせるはらなんだ。
\\	きっと、ぼくをてんきんさせるはらなんだ。
\\	はらぐろいわね。
\\	はらぐろいわね。
\\	しかも、こういうことは、はじめてじゃないんだ。
\\	しかも、こういうことは、はじめてじゃないんだ。
\\	だから、こんかいこそ、じじつをあきらかにしないと、おれのはらのむしがおさまらないんだよ。
\\	だから、こんかいこそ、じじつをあきらかにしないと、おれのはらのむしがおさまらないんだよ。
\\	わかった。わたしもはらをきめたわ。
\\	わかった。わたしもはらをきめたわ。
\\	あなたが、かいしゃとたたかうかん、わたしがあなたをやしなうからね
\\	あなたが、かいしゃとたたかうかん、わたしがあなたをやしなうからね
\\	よし、そうときまったら、「はらがへってはせんができぬ」っていうから、おいしいものでもたべにいこう
\\	よし、そうときまったら、「はらがへってはせんができぬ」っていうから、おいしいものでもたべにいこう
\\	俺、腹をくくったから。
\\	俺、腹をくくったから。
\\	きょうじゅ
\\	まねく
\\	ちょしょ
\\	うえる
\\	しょうげきてき
\\	じたい
\\	てんこうふじゅん
\\	ふさく
\\	ようふうか
\\	きょうは、とうとだいがくのくさやなぎきょうじゅをおまねきして、せんせいのさいきんのちょしょ「ちかいしょうらい、にほんがうえる
\\	について、おはなしをおうかがいしたいとおもいます。
\\	きょうは、とうとだいがくのくさやなぎきょうじゅをおまねきして、せんせいのさいきんのちょしょ「ちかいしょうらい、にほんがうえる
\\	について、おはなしをおうかがいしたいとおもいます。
\\	くさやなぎせんせい、よろしくおねがいします。
\\	くさやなぎせんせい、よろしくおねがいします。
\\	はい、こちらこそよろしくおねがいします。
\\	はい、こちらこそよろしくおねがいします。
\\	さっそくですが、このタイトル、かなりしょうげきてきですね。
\\	さっそくですが、このタイトル、かなりしょうげきてきですね。
\\	このゆたかなくにがうえるということが、じっさいにおこりうることなのでしょうか。
\\	このゆたかなくにがうえるということが、じっさいにおこりうることなのでしょうか。
\\	わたしは、おこりえるとかんがえています。
\\	わたしは、おこりえるとかんがえています。
\\	にほんのしょくりょうじきゅうりつは、1965ねんの73
\\	からさがりつづけ、いまは40
\\	しかないんです。
\\	にほんのしょくりょうじきゅうりつは、1965ねんの73
\\	からさがりつづけ、いまは40
\\	しかないんです。
\\	ですから、まんがいち、しょくりょうをゆにゅうできなくなるようなじたいになったら、にほんじんは、いまの4わりのしょくりょうでいきていかなければならないのです。
\\	ですから、まんがいち、しょくりょうをゆにゅうできなくなるようなじたいになったら、にほんじんは、いまの4わりのしょくりょうでいきていかなければならないのです。
\\	なるほど。。。そういうじたいがおきないともかぎらないですね。
\\	なるほど。。。そういうじたいがおきないともかぎらないですね。
\\	てんこうふじゅんで、ゆにゅうさきのくにでさくもつがふさくにならないともかぎらないですし。。。
\\	てんこうふじゅんで、ゆにゅうさきのくにでさくもつがふさくにならないともかぎらないですし。。。
\\	それに、いつまでもへいわがつづくとはかぎらないですからね。
\\	それに、いつまでもへいわがつづくとはかぎらないですからね。
\\	ところで、じきゅうりつのていかのおもなげんいんは、なんなんでしょう?
\\	ところで、じきゅうりつのていかのおもなげんいんは、なんなんでしょう?
\\	やはりにほんじんのしょくせいかつのようふうかです。
\\	やはりにほんじんのしょくせいかつのようふうかです。
\\	こめのじきゅうりつは100
\\	なのにくらべ、パンのざいりょうであるこむぎこのじきゅうりつは、10
\\	ていどしかありません。
\\	こめのじきゅうりつは100
\\	なのにくらべ、パンのざいりょうであるこむぎこのじきゅうりつは、10
\\	ていどしかありません。
\\	ですから、パンしょくがおおくなればなるほど、じきゅうりつはさがることになります。
\\	ですから、パンしょくがおおくなればなるほど、じきゅうりつはさがることになります。
\\	ひとで
\\	えいぎょう
\\	あんがい
\\	へいき
\\	そとまわり
\\	こしぬけ
\\	しかりとばす
\\	いけてる
\\	きさく
\\	あいさ、あなた、さいきん、しごとばっかりしてない?
\\	あいさ、あなた、さいきん、しごとばっかりしてない?
\\	しゅうしょくするとき、「しごとなんて、けっこんするまでのこしかけだ」っていっていたのに。
\\	しゅうしょくするとき、「しごとなんて、けっこんするまでのこしかけだ」っていっていたのに。
\\	にゅうしゃとうしょは、そのつもりだったんだけど、このかいしゃでこしをすえてがんばることにしたんだぁ。
\\	にゅうしゃとうしょは、そのつもりだったんだけど、このかいしゃでこしをすえてがんばることにしたんだぁ。
\\	なんで、また?なにかあったの?
\\	なんで、また?なにかあったの?
\\	さいしょは、えいぎょうアシスタントをしていたんだけど、さいきんのひとでぶそくでえいぎょうもさせられるようになったわけ。
\\	さいしょは、えいぎょうアシスタントをしていたんだけど、さいきんのひとでぶそくでえいぎょうもさせられるようになったわけ。
\\	で、わたし、けっこう、えいぎょうむきだってことにきづいたんだよ。
\\	で、わたし、けっこう、えいぎょうむきだってことにきづいたんだよ。
\\	それで、ほんごしをいれてはたらくことにしたってわけか。
\\	それで、ほんごしをいれてはたらくことにしたってわけか。
\\	えいぎょうさきでちょっとおこられると、こしがひけちゃうひとって、あんがいおおいのよね。でも、わたしは、ぜんぜんへいき。
\\	えいぎょうさきでちょっとおこられると、こしがひけちゃうひとって、あんがいおおいのよね。でも、わたしは、ぜんぜんへいき。
\\	それに、こしがかるいほうだから、そとまわりもくにならないし。
\\	それに、こしがかるいほうだから、そとまわりもくにならないし。
\\	かい
\\	しょうたいじょう
\\	ひきうける
\\	おめでたい
\\	てみじかに
\\	こぎつける
\\	せいだいな
\\	ひろうえん
\\	たまわる
\\	ゴールイン
\\	あいささんから、けっこんしきのしょうたいじょうがきてるよ。
\\	あいささんから、けっこんしきのしょうたいじょうがきてるよ。
\\	きたきた
\\	いよいよ、あいさもしゃちょうとゴールインか
\\	きたきた
\\	いよいよ、あいさもしゃちょうとゴールインか
\\	あいさもがんばったかいがあったよね。
\\	あいさもがんばったかいがあったよね。
\\	ちょっとみせて。おっ、さすがしゃちょうだけあって、いちりゅうホテルでやるんだな。。。
\\	ちょっとみせて。おっ、さすがしゃちょうだけあって、いちりゅうホテルでやるんだな。。。
\\	あれ、もえに、あいさつをたのみたいっていうかみがはいっているぞ。
\\	あれ、もえに、あいさつをたのみたいっていうかみがはいっているぞ。
\\	え
\\	ほんと? 「たけうちさまには、ゆうじんだいひょうとしてごあいさつをたまわりたく、おねがいもうしあげます」??
\\	え
\\	ほんと? 「たけうちさまには、ゆうじんだいひょうとしてごあいさつをたまわりたく、おねがいもうしあげます」??
\\	うっそー
\\	ちょっと、あいさにでんわしてきいてみる。
\\	うっそー
\\	ちょっと、あいさにでんわしてきいてみる。
\\	もしもし、あいさ?あたしだけど、ひろうえんであいさつなんて、きいてないけど?
\\	もしもし、あいさ?あたしだけど、ひろうえんであいさつなんて、きいてないけど?
\\	あ、そうそう、ぜひおねがいよ
\\	あ、そうそう、ぜひおねがいよ
\\	しょうたいきゃく300にんのせいだいなひろうえんだから、もえもあいさつのしかいがあるでしょ。
\\	しょうたいきゃく300にんのせいだいなひろうえんだから、もえもあいさつのしかいがあるでしょ。
\\	それに、もえのアドバイスのかいもあって、けっこんまでこぎつけたんだから
\\	よろしくね
\\	それに、もえのアドバイスのかいもあって、けっこんまでこぎつけたんだから
\\	よろしくね
\\	そう?
\\	そう?
\\	みじかくていいの。ほかにいっぱ~いあいさつをたのまなくちゃいけないひとがいるからさ。てみじかにたのむわ
\\	みじかくていいの。ほかにいっぱ~いあいさつをたのまなくちゃいけないひとがいるからさ。てみじかにたのむわ
\\	うーん。まぁ、おめでたいことだから、ことわっちゃいけないよね。・・・わかった。ひきうける。
\\	うーん。まぁ、おめでたいことだから、ことわっちゃいけないよね。・・・わかった。ひきうける。
\\	たまわる
\\	にじゅうねんらい
\\	しんぷ
\\	すえながい
\\	ひかれる
\\	とりくむ
\\	おめあて
\\	しごとにんげん
\\	あずかる
\\	しゅうしょく
\\	つづきまして、しんぷのごゆうじんをだいひょうして、たけうちさまにごあいさつをたまわりたいとぞんじます。
\\	つづきまして、しんぷのごゆうじんをだいひょうして、たけうちさまにごあいさつをたまわりたいとぞんじます。
\\	たけうちさま、よろしくおねがいいたします。
\\	たけうちさま、よろしくおねがいいたします。
\\	ただいま、ごしょうかいにあずかりましたたけうちともうします。
\\	ただいま、ごしょうかいにあずかりましたたけうちともうします。
\\	しんぷのゆうじんをだいひょうして、ひとことおいわいのあいさつをさせていただきます。
\\	しんぷのゆうじんをだいひょうして、ひとことおいわいのあいさつをさせていただきます。
\\	しんぷのあいささんとは、ちゅうがっこうからだいがくまでおなじがっこうにかよった20ねんらいのしんゆうです。
\\	しんぷのあいささんとは、ちゅうがっこうからだいがくまでおなじがっこうにかよった20ねんらいのしんゆうです。
\\	しゅうしょくしてからも、よくあって、あそんでおりました。
\\	しゅうしょくしてからも、よくあって、あそんでおりました。
\\	でも。。。たしか、しゅうしょくして、3ねんめくらいでしょうか。きゅうに、あいささんがしごとにんげんになってしまったのです。
\\	でも。。。たしか、しゅうしょくして、3ねんめくらいでしょうか。きゅうに、あいささんがしごとにんげんになってしまったのです。
\\	なにかあったのかとおもってきいてみますと、じつは、しごとというより、しゃちょうのたくやさんがおめあてだったわけです。
\\	なにかあったのかとおもってきいてみますと、じつは、しごとというより、しゃちょうのたくやさんがおめあてだったわけです。
\\	ちょ、ちょっと、そんなこといわないでよ
\\	(こころのなか)
\\	ちょ、ちょっと、そんなこといわないでよ
\\	(こころのなか)
\\	あいささんは、むかしからなにごとにもいっしょうけんめいでした。ですから、ねっしんにしごとにとりくむすがたに、たくやさんもひかれたのだとおもいます。
\\	あいささんは、むかしからなにごとにもいっしょうけんめいでした。ですから、ねっしんにしごとにとりくむすがたに、たくやさんもひかれたのだとおもいます。
\\	そうそう、わたしのいいところをいってよね。あ~、いろんなことがおもいだされる。なつかしい~。。
\\	そうそう、わたしのいいところをいってよね。あ~、いろんなことがおもいだされる。なつかしい~。。
\\	あれ、もえのスピーチ、よくきかないうちに、もうおわっちゃうわ。(こころのなか)
\\	あれ、もえのスピーチ、よくきかないうちに、もうおわっちゃうわ。(こころのなか)
\\	おふたりのすえながいおしあわせをおいのりして、わたしのおいわいのことばといたします。
\\	おふたりのすえながいおしあわせをおいのりして、わたしのおいわいのことばといたします。
\\	なさけない
\\	くびをきる
\\	にんじょう
\\	ぎり
\\	はくじょう
\\	おしうり
\\	じょう
\\	いなか
\\	でかせぎ
\\	つぶれる
\\	あれ、なに、このみず?
\\	あれ、なに、このみず?
\\	それね、ひるま、セールスのひとがきて、けいやくしちゃったの。
\\	それね、ひるま、セールスのひとがきて、けいやくしちゃったの。
\\	サーバーだいはサービスだっていうし、ペットボトルのみずをかうよりやすあがりだっていうからいいかなっておもって。。。
\\	サーバーだいはサービスだっていうし、ペットボトルのみずをかうよりやすあがりだっていうからいいかなっておもって。。。
\\	そんな、セールストークにひっかかって、なさけないなぁ。
\\	そんな、セールストークにひっかかって、なさけないなぁ。
\\	それに、そのひと、あおもりからでかせぎにきていて、こんげつあと10けん、けいやくをとらないと、おしょうがつにいなかにもかえれないっていうのよ。
\\	それに、そのひと、あおもりからでかせぎにきていて、こんげつあと10けん、けいやくをとらないと、おしょうがつにいなかにもかえれないっていうのよ。
\\	つい、じょうにながされちゃってね。
\\	つい、じょうにながされちゃってね。
\\	じょうにつけこむのは、おしうりのよくつかうてじゃないか。
\\	じょうにつけこむのは、おしうりのよくつかうてじゃないか。
\\	わたし、そういうはなしをきいちゃうと、どうしてもはくじょうなことできないのよ。
\\	わたし、そういうはなしをきいちゃうと、どうしてもはくじょうなことできないのよ。
\\	だって、わたしのちちは、とくにじょうにあついひとだったでしょ。
\\	だって、わたしのちちは、とくにじょうにあついひとだったでしょ。
\\	まぁ、たしかに、きみの、おとうさんは、ぎりとにんじょうをたいせつにするひとだったな。
\\	まぁ、たしかに、きみの、おとうさんは、ぎりとにんじょうをたいせつにするひとだったな。
\\	ふけいきになっても、じゅうぎょういんのくびをきらずにがんばっていたよね。
\\	ふけいきになっても、じゅうぎょういんのくびをきらずにがんばっていたよね。
\\	でも、けっきょく、かいしゃつぶれちゃったじゃないか。じょうにもろいのももんだいだよ。
\\	でも、けっきょく、かいしゃつぶれちゃったじゃないか。じょうにもろいのももんだいだよ。
\\	そうね。わたしも、ちちににて、じょうにもろいめんもあるわね。
\\	そうね。わたしも、ちちににて、じょうにもろいめんもあるわね。
\\	そういえば、あなたとつきあうまえ、あめのひもかぜのひも、まいにちわたしのいえにはなをとどけてくれたでしょ。
\\	そういえば、あなたとつきあうまえ、あめのひもかぜのひも、まいにちわたしのいえにはなをとどけてくれたでしょ。
\\	そのすがたをみて、つい、じょうにほだされてつきあうことにしたんだったわ。。。
\\	そのすがたをみて、つい、じょうにほだされてつきあうことにしたんだったわ。。。
\\	かねまわり
\\	ぼろもうけ
\\	もうかる
\\	ぎょうかい
\\	じょうほうしゅうしゅう
\\	ばかばかしい
\\	りし
\\	あずける
\\	ていきんり
\\	いたいめ
\\	ひびき、おまえ、ついはんとしまえにしんしゃをかったとおもったら、もう、またくるま、かいかえたのか?
\\	ひびき、おまえ、ついはんとしまえにしんしゃをかったとおもったら、もう、またくるま、かいかえたのか?
\\	なんで、そんなにかねまわりがいいんだよ。
\\	なんで、そんなにかねまわりがいいんだよ。
\\	まぁな。このていきんりだろ。ぎんこうにあずけても、りしがほとんどつかなくて、ばかばかしくてさ。
\\	まぁな。このていきんりだろ。ぎんこうにあずけても、りしがほとんどつかなくて、ばかばかしくてさ。
\\	だからさいきん、かぶをはじめたんだよ。
\\	だからさいきん、かぶをはじめたんだよ。
\\	かぶ?そんなの、どうせそんするのがおちだろ。
\\	かぶ?そんなの、どうせそんするのがおちだろ。
\\	そうともかぎらないぞ
\\	ちゃんときぎょうけんきゅうやじょうほうしゅうしゅうをすれば、なんとかもうかるもんだぜ。
\\	そうともかぎらないぞ
\\	ちゃんときぎょうけんきゅうやじょうほうしゅうしゅうをすれば、なんとかもうかるもんだぜ。
\\	へぇ、そうなのか。じゃ、おれもちょっとちょうせんしてみようかな。
\\	へぇ、そうなのか。じゃ、おれもちょっとちょうせんしてみようかな。
\\	しょしんしゃむきのかぶで、なにかおすすめはあるか?
\\	しょしんしゃむきのかぶで、なにかおすすめはあるか?
\\	そうだなぁ。かんきょうかんれんとかバイオテクノロジーかんけいとかのかぶがいいんじゃないか。
\\	そうだなぁ。かんきょうかんれんとかバイオテクノロジーかんけいとかのかぶがいいんじゃないか。
\\	これからせいちょうがみこめそうなぎょうかいだからな。
\\	これからせいちょうがみこめそうなぎょうかいだからな。
\\	なるほど。よし、おまえが、ぜったいそんしないってほしょうしてくれるなら、いちどかってみようかな。
\\	なるほど。よし、おまえが、ぜったいそんしないってほしょうしてくれるなら、いちどかってみようかな。
\\	なにいってるんだよ。そんなほしょう、だれができるか
\\	なにいってるんだよ。そんなほしょう、だれができるか
\\	かぶはじぶんのせきにんでかうもんだよ。
\\	かぶはじぶんのせきにんでかうもんだよ。
\\	じこせきにんだなんて、そんなつめたいこというなよ。
\\	じこせきにんだなんて、そんなつめたいこというなよ。
\\	ひびきは、ぼろもうけしているんだろう?
\\	ひびきは、ぼろもうけしているんだろう?
\\	おれだって、いろいろいたいめにあってるんだよ。
\\	おれだって、いろいろいたいめにあってるんだよ。
\\	ぜったいそんしないかぶなんてものがあるなら、おれのほうがしりたいよ
\\	ぜったいそんしないかぶなんてものがあるなら、おれのほうがしりたいよ
\\	なべぶぎょう
\\	ころあい
\\	てっそく
\\	ほんかくてき
\\	てがる
\\	もと
\\	とりがら
\\	にだす
\\	うでによりをかける
\\	めんどうくさい
\\	おじゃま~。お、こういちも、きてたのか。
\\	おじゃま~。お、こういちも、きてたのか。
\\	よぉ、しんぺい。おとこさんにんでぼうねんかいってのもさえないけどな。
\\	よぉ、しんぺい。おとこさんにんでぼうねんかいってのもさえないけどな。
\\	たしかにな(わらい)(クンクン)
\\	たしかにな(わらい)(クンクン)
\\	うまそうなにおいだな。きょうはなべか?
\\	うまそうなにおいだな。きょうはなべか?
\\	ぼうねんかいといえば、やっぱりなべでしょ
\\	おれ、なべにはちょっとうるさいよ。
\\	ぼうねんかいといえば、やっぱりなべでしょ
\\	おれ、なべにはちょっとうるさいよ。
\\	いわゆるなべぶぎょうってやつか。それは、たのしみだな。
\\	いわゆるなべぶぎょうってやつか。それは、たのしみだな。
\\	じゃーん、きょうのだしは、おれがうでによりをかけて、とりガラをにだしてとったスープだぞ
\\	じゃーん、きょうのだしは、おれがうでによりをかけて、とりガラをにだしてとったスープだぞ
\\	とりガラスープのもとはてがるだけど、ほんかくてきなあじをだすには、やっぱりとりガラからにだすにかぎるよ。
\\	とりガラスープのもとはてがるだけど、ほんかくてきなあじをだすには、やっぱりとりガラからにだすにかぎるよ。
\\	すげぇいいにおい
\\	さっそく、このにくとかやさい、いれていい?
\\	すげぇいいにおい
\\	さっそく、このにくとかやさい、いれていい?
\\	ちょ、ちょっとまて。いれるじゅんばんがあるんだ。
\\	ちょ、ちょっとまて。いれるじゅんばんがあるんだ。
\\	まずは、ひがとおりにくいやさいとにくるいからいれるのがてっそくなんだ。
\\	まずは、ひがとおりにくいやさいとにくるいからいれるのがてっそくなんだ。
\\	ころあいをみて、さいごにしろみざかなとか、かいるいをいれるんだ。ひがとおりすぎてかたくなっちゃうからな。
\\	ころあいをみて、さいごにしろみざかなとか、かいるいをいれるんだ。ひがとおりすぎてかたくなっちゃうからな。
\\	ほぉぉ。なるほど。
\\	ほぉぉ。なるほど。
\\	さ、そろそろ、このかきはたべごろだぞ。
\\	さ、そろそろ、このかきはたべごろだぞ。
\\	かきといえば、しろワインだよな。
\\	かきといえば、しろワインだよな。
\\	あっ
\\	しまった、おれとしたことが、しろワインをかいわすれた
\\	あっ
\\	しまった、おれとしたことが、しろワインをかいわすれた
\\	まぁ、いいよいいよ。
\\	まぁ、いいよいいよ。
\\	それより、このかき、さいこうだぜ
\\	めんどうくさいから、ここにあるかき、ぜんぶなべにいれちゃおうぜ。
\\	それより、このかき、さいこうだぜ
\\	めんどうくさいから、ここにあるかき、ぜんぶなべにいれちゃおうぜ。
\\	だめだめ
\\	つぎのかきをいれるのは、いまなべのなかにあるのをぜんぶたべきってからだよ
\\	だめだめ
\\	つぎのかきをいれるのは、いまなべのなかにあるのをぜんぶたべきってからだよ
\\	ほけん
\\	テロ
\\	むだ
\\	かにゅう
\\	おやしらず
\\	ひとりたび
\\	とうなん
\\	てんさい
\\	しょうがいほけん
\\	だいりてん
\\	もしもし?
\\	もしもし?
\\	ユウちゃん?ひさしぶり。だいがくは もうなつやすみに はいったんでしょ?いつ かえってくるの?おぼんには かえってくるんでしょ。
\\	ユウちゃん?ひさしぶり。だいがくは もうなつやすみに はいったんでしょ?いつ かえってくるの?おぼんには かえってくるんでしょ。
\\	それがさ、あしたから かいがいに ひとりたびに いこうとおもって…。
\\	それがさ、あしたから かいがいに ひとりたびに いこうとおもって…。
\\	あした?まったく、おやのこころ こしらず よね。ユウちゃんが かえってくるのを おかあさん たのしみに まっていたのに…。あ、ほけんには はいっておいてよ。
\\	あした?まったく、おやのこころ こしらず よね。ユウちゃんが かえってくるのを おかあさん たのしみに まっていたのに…。あ、ほけんには はいっておいてよ。
\\	なんだよ、それ。おれがしねばいいとかおもっているわけ?
\\	なんだよ、それ。おれがしねばいいとかおもっているわけ?
\\	ちがうわよ。とうなん とか、じこ とか、テロとか、てんさい とか あるかもしれないでしょ。くうこうに ほけんの だいりてんが あるから、そこで かいがいりょこうしょうがいほけんに はいってからいきなさい。
\\	ちがうわよ。とうなん とか、じこ とか、テロとか、てんさい とか あるかもしれないでしょ。くうこうに ほけんの だいりてんが あるから、そこで かいがいりょこうしょうがいほけんに はいってからいきなさい。
\\	えー。かねの むだだよ。
\\	えー。かねの むだだよ。
\\	そんなこといわずに かにゅうしなさい。ころばぬさきのつえっていうでしょ。いいわね。
\\	そんなこといわずに かにゅうしなさい。ころばぬさきのつえっていうでしょ。いいわね。
\\	はいはい。
\\	はいはい。
\\	「はい」は、いっかい。それから、みしらぬひとに こえをかけられても ついていったら ダメだからね。
\\	「はい」は、いっかい。それから、みしらぬひとに こえをかけられても ついていったら ダメだからね。
\\	はーい。
\\	はーい。
\\	それから…
\\	それから…
\\	かあさん、もういいかな。おやしらず が いたくて、これから はいしゃに いくんだけど…。
\\	かあさん、もういいかな。おやしらず が いたくて、これから はいしゃに いくんだけど…。
\\	そうなの。じゃ、はやく いってきなさい。じゃあ きるわよ、またね。
\\	そうなの。じゃ、はやく いってきなさい。じゃあ きるわよ、またね。
\\	ぱあ
\\	かいていかざん
\\	つなみ
\\	まさか
\\	わざわい
\\	はれる
\\	ふきん
\\	ふんか
\\	じしん
\\	ふく
\\	ただいま。
\\	ただいま。
\\	あれ、どうしたんだ、ゆうた?
\\	じまに いったんじゃないのか。
\\	あれ、どうしたんだ、ゆうた?
\\	じまに いったんじゃないのか。
\\	そのつもりだったんだけど…。
\\	そのつもりだったんだけど…。
\\	りょこうの ぜんじつに おやしらずを ぬいたらさ、ちは とまらないわ、かおは はれるわ、ねつは でるわ…。それで、しゅっぱつできなかったんだよ。
\\	りょこうの ぜんじつに おやしらずを ぬいたらさ、ちは とまらないわ、かおは はれるわ、ねつは でるわ…。それで、しゅっぱつできなかったんだよ。
\\	べつの ひの ひこうきに かえられただろうに。
\\	べつの ひの ひこうきに かえられただろうに。
\\	やすいこうくうけんだからさ、ほかのびんへのへんこうは できなくてさ、こうくうけんは ぱあだよ。
\\	やすいこうくうけんだからさ、ほかのびんへのへんこうは できなくてさ、こうくうけんは ぱあだよ。
\\	ま、ひこうきしか よやくしてなかったのは、ふこうちゅうの さいわいだったけど。
\\	ま、ひこうきしか よやくしてなかったのは、ふこうちゅうの さいわいだったけど。
\\	わかいんだから、また そのうち いけるさ。
\\	わかいんだから、また そのうち いけるさ。
\\	おとうさん、たいへん、
\\	じま ふきん の かいていかざんが ふんかして だいじしんだって。
\\	おとうさん、たいへん、
\\	じま ふきん の かいていかざんが ふんかして だいじしんだって。
\\	つなみも おきたみたいで、…あら?ゆうた、どうしたの?
\\	じま いかなかったの?
\\	つなみも おきたみたいで、…あら?ゆうた、どうしたの?
\\	じま いかなかったの?
\\	おやしらずをぬいたら ねつが でちゃってさ、いくの やめたんだ。
\\	おやしらずをぬいたら ねつが でちゃってさ、いくの やめたんだ。
\\	よかったー。さいおうが うまね。
\\	よかったー。さいおうが うまね。
\\	ぬいぐるみ
\\	はんばい
\\	はいご
\\	しゃさつ
\\	ぜんち
\\	はなれる
\\	くま
\\	おそう
\\	じゅうしょう
\\	おう
\\	たく
\\	ころす
\\	おかえりなさい。とうきょうしゅっちょうはどうだった。
\\	おかえりなさい。とうきょうしゅっちょうはどうだった。
\\	でかけたっきりれんらくないからしんぱいしてたのよ。
\\	でかけたっきりれんらくないからしんぱいしてたのよ。
\\	ごめん、いそがしくてさ。しゅっちょうはこれっきりにしてほしいよ。
\\	ごめん、いそがしくてさ。しゅっちょうはこれっきりにしてほしいよ。
\\	おかえり、パパ。
\\	おかえり、パパ。
\\	はい、おみやげ。さだこのすきなくまのプーさまのぬいぐるみだよ。
\\	はい、おみやげ。さだこのすきなくまのプーさまのぬいぐるみだよ。
\\	わーい。ありがとう。
\\	わーい。ありがとう。
\\	それ、とうきょうだけでしかかえないんだぞ。
\\	それ、とうきょうだけでしかかえないんだぞ。
\\	あら、ホント。とうきょうでのみはんばいってかいてある。なにかたべる?
\\	あら、ホント。とうきょうでのみはんばいってかいてある。なにかたべる?
\\	ビールだけもらうよ。
\\	ビールだけもらうよ。
\\	いま、よういするね。
\\	いま、よういするね。
\\	きょうのごご やまがたけんみなとしのじゅうたくがいにて、じょせいがくまにおそわれじゅうしょうをおいました。
\\	きょうのごご やまがたけんみなとしのじゅうたくがいにて、じょせいがくまにおそわれじゅうしょうをおいました。
\\	みなとし?すぐちかくじゃないか。
\\	みなとし?すぐちかくじゃないか。
\\	ほんじつごご、いちじはんごろ やまがたけんみなとしにすむもちだかねさんはちじゅうきゅうさいがじたくのにわでそうじをしていたところ はいごからくまにおそわれました。
\\	ほんじつごご、いちじはんごろ やまがたけんみなとしにすむもちだかねさんはちじゅうきゅうさいがじたくのにわでそうじをしていたところ はいごからくまにおそわれました。
\\	もちださんはかおのほねをおるなど、ぜんち2がつのけがをおいました。
\\	もちださんはかおのほねをおるなど、ぜんち2がつのけがをおいました。
\\	くまはそのご、もちださんたくからいちキロメートルはなれたこうえんであそんでいるところをしゃさつされました。
\\	くまはそのご、もちださんたくからいちキロメートルはなれたこうえんであそんでいるところをしゃさつされました。
\\	くま、ころされちゃったの?かわいそう。
\\	くま、ころされちゃったの?かわいそう。
\\	かくち
\\	おそれる
\\	さいせい
\\	ぶそく
\\	かいけつ
\\	じゅうたくち
\\	しょくりょう
\\	しゅつぼつ
\\	おんだんか
\\	かれる
\\	さんせいう
\\	ひがい
\\	ひとをおそったら、ころされちゃうのに、どうしてくまはひとをおそっちゃったんだろう。
\\	ひとをおそったら、ころされちゃうのに、どうしてくまはひとをおそっちゃったんだろう。
\\	くまがそんなことわかるはずないでしょ。おそわれたひとこそかわいそうよ。
\\	くまがそんなことわかるはずないでしょ。おそわれたひとこそかわいそうよ。
\\	くまはやまにしょくりょうがないからひとのすむところにたべものをさがしにくるんだろうね。ひとをおそれていればこそ、おそうらしいよ。
\\	くまはやまにしょくりょうがないからひとのすむところにたべものをさがしにくるんだろうね。ひとをおそれていればこそ、おそうらしいよ。
\\	かくちでくまのひがいがつづいています。なにがげんいんなのか、もりだいがくのくまがいきょうじゅにインタビューをしてみました。
\\	かくちでくまのひがいがつづいています。なにがげんいんなのか、もりだいがくのくまがいきょうじゅにインタビューをしてみました。
\\	くまがじゅうたくちにしゅつぼつするのはくまのえさがやまにほとんどないからなんですよ。おんだんかやさんせいうのせいできがかれてきているのもげんいんのひとつといえますね。
\\	くまがじゅうたくちにしゅつぼつするのはくまのえさがやまにほとんどないからなんですよ。おんだんかやさんせいうのせいできがかれてきているのもげんいんのひとつといえますね。
\\	つまりにんげんこそがすべてのげんいんをつくった…。
\\	つまりにんげんこそがすべてのげんいんをつくった…。
\\	そう。もりがさいせいしないと、ほんとうのいみでのもんだいはかいけつしないでしょうね。
\\	そう。もりがさいせいしないと、ほんとうのいみでのもんだいはかいけつしないでしょうね。
\\	くまもしょくりょうぶそくでたいへんなのね…。でも、こわいわ。あのこうえん、さだこのしょうがっこうからめとはなのさきでしょ。
\\	くまもしょくりょうぶそくでたいへんなのね…。でも、こわいわ。あのこうえん、さだこのしょうがっこうからめとはなのさきでしょ。
\\	わたしはだいじょうぶ。くまようスプレーももっているの。ほら。
\\	わたしはだいじょうぶ。くまようスプレーももっているの。ほら。
\\	それでこそおれのこだ。
\\	それでこそおれのこだ。
\\	じれい
\\	じれい
\\	かいがいえいぎょうぶ さとうげんきどの
\\	かいがいえいぎょうぶ さとうげんきどの
\\	ほんこんしてんてんきんをめいじる
\\	ほんこんしてんてんきんをめいじる
\\	みなさん、ごぶさたしてます。
\\	みなさん、ごぶさたしてます。
\\	とつぜんですが、かいしゃからじれいがでて、らいげつからほんこんにてんきんすることになりました。しゅっぱつまであといっかげつしかないのですが、みなさんさえよければ、のみましょう。
\\	とつぜんですが、かいしゃからじれいがでて、らいげつからほんこんにてんきんすることになりました。しゅっぱつまであといっかげつしかないのですが、みなさんさえよければ、のみましょう。
\\	えー。ちゅうがくじだいはアルファベットすらかけなかったげんきが、かいがいにてんきんすることになりました。
\\	えー。ちゅうがくじだいはアルファベットすらかけなかったげんきが、かいがいにてんきんすることになりました。
\\	げんきがげんきにいってかえってくることをいのってかんぱいしたいとおもいます。いってらっしゃーい。かんぱい
\\	げんきがげんきにいってかえってくることをいのってかんぱいしたいとおもいます。いってらっしゃーい。かんぱい
\\	かんぱい
\\	かんぱい
\\	おまえすごいな。ちゅうごくごまではなせるのか?
\\	おまえすごいな。ちゅうごくごまではなせるのか?
\\	いや、ぜんぜん。でも、しごとはえいごさえできればだいじょうぶなんだよ。
\\	いや、ぜんぜん。でも、しごとはえいごさえできればだいじょうぶなんだよ。
\\	いいなぁ。げんきはだいがくじだいにりゅうがくしたから、えいごができて。
\\	いいなぁ。げんきはだいがくじだいにりゅうがくしたから、えいごができて。
\\	おれにはかいがいちゅうざいいんになることはもちろんかいがいしゅっちょうすることすらゆめのゆめだよ。
\\	おれにはかいがいちゅうざいいんになることはもちろんかいがいしゅっちょうすることすらゆめのゆめだよ。
\\	でも、はるきのかいしゃってさいきんアメリカのかいしゃにばいしゅうされただろ?
\\	でも、はるきのかいしゃってさいきんアメリカのかいしゃにばいしゅうされただろ?
\\	ああ。しゃないぶんしょはすべてえいご。さいきんはかいぎまでえいごでするんだ。
\\	ああ。しゃないぶんしょはすべてえいご。さいきんはかいぎまでえいごでするんだ。
\\	けんとう
\\	ゆうじょう
\\	きざ
\\	ふる
\\	ふにん
\\	きかん
\\	あきる
\\	いろっぽい
\\	なやむ
\\	りくつ
\\	ほんこんへはひとりでいくの?
\\	ほんこんへはひとりでいくの?
\\	ああ。いま、かのじょいないからね。
\\	ああ。いま、かのじょいないからね。
\\	ふにんきかんはどのくらい?
\\	ふにんきかんはどのくらい?
\\	さあ。けんとうもつかないよ。
\\	さあ。けんとうもつかないよ。
\\	(だんしトイレ)
\\	(だんしトイレ)
\\	しほちゃんとは、うまくいっているのか?
\\	しほちゃんとは、うまくいっているのか?
\\	あんなにおんならしいことつきあっているなんてうらやましいよ。
\\	あんなにおんならしいことつきあっているなんてうらやましいよ。
\\	わかれたよ。おれ、あきっぽいからさ。
\\	わかれたよ。おれ、あきっぽいからさ。
\\	なんでだよ
\\	おれ...あいてがおまえだから、しほちゃんのことあきらめたんだぞ。
\\	なんでだよ
\\	おれ...あいてがおまえだから、しほちゃんのことあきらめたんだぞ。
\\	やっとほんとうのこといったな。おまえらしいな。
\\	やっとほんとうのこといったな。おまえらしいな。
\\	でもやすっぽいゆうじょうなんていらないんだよ。
\\	でもやすっぽいゆうじょうなんていらないんだよ。
\\	は?
\\	は?
\\	おれ、ふられたんだ。ちゅうがくのころからずっとすきだったひとがいて、そいつのことがわすれられないらしいぞ。
\\	おれ、ふられたんだ。ちゅうがくのころからずっとすきだったひとがいて、そいつのことがわすれられないらしいぞ。
\\	え?
\\	え?
\\	おまえいがいにいないだろう。
\\	おまえいがいにいないだろう。
\\	でも...。
\\	でも...。
\\	なにをなやんでいるんだよ。こんやがさいごのチャンスだろ。あたってくだけろよ。
\\	なにをなやんでいるんだよ。こんやがさいごのチャンスだろ。あたってくだけろよ。
\\	おお。
\\	おお。
\\	(じょしトイレ)
\\	(じょしトイレ)
\\	しほ、そのふく いろっぽい。
\\	しほ、そのふく いろっぽい。
\\	ありがとう。げんきくんにあえるからおしゃれしたんだけど、ざんねん。
\\	ありがとう。げんきくんにあえるからおしゃれしたんだけど、ざんねん。
\\	ちゅうがくのときはおとこらしくてすてきだったのに。
\\	ちゅうがくのときはおとこらしくてすてきだったのに。
\\	ふくつう
\\	はきけ
\\	しょくあたり
\\	リン
\\	どく
\\	うらないし
\\	げり
\\	はく
\\	さむけ
\\	めまい
\\	しょくちゅうどく
\\	しょうじょう
\\	おいしそう。いただきます。
\\	おいしそう。いただきます。
\\	あれ?いたた。
\\	あれ?いたた。
\\	しらゆきさん、しらゆきひめさん、どうぞおはいりください。
\\	しらゆきさん、しらゆきひめさん、どうぞおはいりください。
\\	どうしましたか。
\\	どうしましたか。
\\	おひるをたべてからおなかのちょうしがわるいんです。
\\	おひるをたべてからおなかのちょうしがわるいんです。
\\	げりとふくつうですか。ほかにしょうじょうはありますか。
\\	げりとふくつうですか。ほかにしょうじょうはありますか。
\\	たとえば、はきけがするとか。ねつがあるとか。
\\	たとえば、はきけがするとか。ねつがあるとか。
\\	ねつはわかりません。ここにくるまでになんかいもはいてしまいました。
\\	ねつはわかりません。ここにくるまでになんかいもはいてしまいました。
\\	いがまだむかむかします。さむけもしますし、めまいもします。
\\	いがまだむかむかします。さむけもしますし、めまいもします。
\\	てんけいてきなしょくちゅうどくのようですが、ちょっとしらべてみましょう。
\\	てんけいてきなしょくちゅうどくのようですが、ちょっとしらべてみましょう。
\\	やっぱりしょくあたりですか?なんだか、めがちかちかしてきたんですが...。
\\	やっぱりしょくあたりですか?なんだか、めがちかちかしてきたんですが...。
\\	どくせいのあるものをたべたとおもうんです。なにをたべましたか。
\\	どくせいのあるものをたべたとおもうんです。なにをたべましたか。
\\	このリンゴをひとくちだけたべたんです。
\\	このリンゴをひとくちだけたべたんです。
\\	これは、どこでてにいれたんですか。
\\	これは、どこでてにいれたんですか。
\\	うらないしふうのおばあさんにもらいました。
\\	うらないしふうのおばあさんにもらいました。
\\	せんもんてきなことは、しらべてみないとわかりませんが、リンけいのどくをたべてしまったのかもしれません。
\\	せんもんてきなことは、しらべてみないとわかりませんが、リンけいのどくをたべてしまったのかもしれません。
\\	とうあん
\\	すな
\\	かんどう
\\	かつやく
\\	だらけ
\\	きまつてすと
\\	あせ
\\	こうしえん
\\	どろ
\\	ばつ
\\	はぁ...。
\\	はぁ...。
\\	どうしたんですか?はらせんせい?
\\	どうしたんですか?はらせんせい?
\\	みてくださいよ。ばんどうくんのきまつテストのとうあん。
\\	みてくださいよ。ばんどうくんのきまつテストのとうあん。
\\	まちがいだらけなんですよ。べんきょうしたのかしら。
\\	まちがいだらけなんですよ。べんきょうしたのかしら。
\\	はは。こりゃひどいな。
\\	はは。こりゃひどいな。
\\	でも、うちのながしまのとうあんもひどいですよ。
\\	でも、うちのながしまのとうあんもひどいですよ。
\\	あら。バツだらけ。どんぐりのせいくらべね。
\\	あら。バツだらけ。どんぐりのせいくらべね。
\\	まあ、かれらはやきゅうぶだからしかたないですよ。
\\	まあ、かれらはやきゅうぶだからしかたないですよ。
\\	あさからばんまですなまみれ、あせまみれでやきゅうをれんしゅうしているんですから。
\\	あさからばんまですなまみれ、あせまみれでやきゅうをれんしゅうしているんですから。
\\	こうしえんがおわるまでは、べんきょうどころじゃないでしょうね。
\\	こうしえんがおわるまでは、べんきょうどころじゃないでしょうね。
\\	おおめにみましょうよ。
\\	おおめにみましょうよ。
\\	そうですね。
\\	そうですね。
\\	これでこうしえんにでてかつやくしてプロにはいれればいいことずくめですからね。
\\	これでこうしえんにでてかつやくしてプロにはいれればいいことずくめですからね。
\\	かれらはそんなりゆうでやきゅうをしているんじゃありませんよ。
\\	かれらはそんなりゆうでやきゅうをしているんじゃありませんよ。
\\	プロになれてもめがでるのはいちぶですし、プロせんしゅたちはてんさいだらけです。
\\	プロになれてもめがでるのはいちぶですし、プロせんしゅたちはてんさいだらけです。
\\	だいいち、ばんどうやながしまのレベルではプロになることもむりです。
\\	だいいち、ばんどうやながしまのレベルではプロになることもむりです。
\\	そうですか。
\\	そうですか。
\\	でも、さきのことをかんがえずにどろだらけになっていっしょうけんめいがんばっているかれらをみると、ぼくは、かんどうするんです。
\\	でも、さきのことをかんがえずにどろだらけになっていっしょうけんめいがんばっているかれらをみると、ぼくは、かんどうするんです。
\\	はぁ...。
\\	はぁ...。
\\	こうかてき
\\	ひょうろんか
\\	げんそうてき
\\	かんきょうおせん
\\	やけい
\\	きんみらい
\\	ふうけい
\\	とし
\\	きょうちょう
\\	じんこうてき
\\	みなさんは「こうじょうやけい」がブームになっていることをごぞんじですか?
\\	みなさんは「こうじょうやけい」がブームになっていることをごぞんじですか?
\\	こんかいは、やけいひょうろんかのみやざわけんさんにこうじょうをあんないしてもらいます。
\\	こんかいは、やけいひょうろんかのみやざわけんさんにこうじょうをあんないしてもらいます。
\\	よろしくおねがいします。
\\	よろしくおねがいします。
\\	さて、こうじょうやけいですが、どのようなてんがにんきなのでしょうか。
\\	さて、こうじょうやけいですが、どのようなてんがにんきなのでしょうか。
\\	ひゃくぶんはいっけんにしかず。こちらへどうぞ。
\\	ひゃくぶんはいっけんにしかず。こちらへどうぞ。
\\	うわ。きれい…。とてもじんこうてきで、げんそうてきなふうけいですね。
\\	うわ。きれい…。とてもじんこうてきで、げんそうてきなふうけいですね。
\\	そうでしょ。
\\	そうでしょ。
\\	きんみらいのとしにいるみたいです。
\\	きんみらいのとしにいるみたいです。
\\	にんきのりゆうはいろいろあるでしょうが、こうじょうやけいは
\\	のようなふうけいをみることができるというてんでにんきがあるのだとおもいます。
\\	にんきのりゆうはいろいろあるでしょうが、こうじょうやけいは
\\	のようなふうけいをみることができるというてんでにんきがあるのだとおもいます。
\\	なるほど。
\\	なるほど。
\\	こうじょうってマイナスめんばかりがきょうちょうされてきましたよね。かんきょうおせんとか。
\\	こうじょうってマイナスめんばかりがきょうちょうされてきましたよね。かんきょうおせんとか。
\\	はい。
\\	はい。
\\	いまのこうじょうはかんきょうめんをかんがえているからあんぜんですし、こうじょうはわれわれのせいかつにひつようなものです。
\\	いまのこうじょうはかんきょうめんをかんがえているからあんぜんですし、こうじょうはわれわれのせいかつにひつようなものです。
\\	こうじょうぜんたいのイメージをよくするというてんでも、こうじょうやけいけんがくはこうかてきなんですよ。
\\	こうじょうぜんたいのイメージをよくするというてんでも、こうじょうやけいけんがくはこうかてきなんですよ。
\\	なるほど。わたしもだれかすてきなひととこのやけいをみにきたいとおもいます。
\\	なるほど。わたしもだれかすてきなひととこのやけいをみにきたいとおもいます。
\\	すみませんねぇ。きょうはわたしといっしょで…。
\\	すみませんねぇ。きょうはわたしといっしょで…。
\\	あ、そういういみじゃないです。
\\	あ、そういういみじゃないです。
\\	しゅうねん
\\	しゅしょう
\\	じにん
\\	しつげん
\\	うわき
\\	なまけもの
\\	よびだす
\\	きんきゅう
\\	せんきょ
\\	ぜいきん
\\	しじりつ
\\	そうり
\\	おつかれ。
\\	おつかれ。
\\	たいへんでしたね。きゅう に よびだされて。
\\	たいへんでしたね。きゅう に よびだされて。
\\	ほんとうだよ。けっこん10しゅうねんだからレストランよやくしてしたづつみをうってたのに、いきなりかいしゃからよびだしだよ。
\\	ほんとうだよ。けっこん10しゅうねんだからレストランよやくしてしたづつみをうってたのに、いきなりかいしゃからよびだしだよ。
\\	おくさん、だいじょうぶでした?
\\	おくさん、だいじょうぶでした?
\\	したうちされたよ。で、なに?そうりじにん?
\\	したうちされたよ。で、なに?そうりじにん?
\\	こんなじかんにきんきゅうきしゃかいけんをするんだから、たぶんそうだとおもいますよ。
\\	こんなじかんにきんきゅうきしゃかいけんをするんだから、たぶんそうだとおもいますよ。
\\	ぜんかいのせんきょで、「ぜいきんはあげない」っていったのに、そのしたのねのかわかぬうちにしょうひぜいを10パーセントにするなんていうから、しじりつがさがったんだ。
\\	ぜんかいのせんきょで、「ぜいきんはあげない」っていったのに、そのしたのねのかわかぬうちにしょうひぜいを10パーセントにするなんていうから、しじりつがさがったんだ。
\\	それから、あのひと、ほんとうにはなすのへたですよね。
\\	それから、あのひと、ほんとうにはなすのへたですよね。
\\	したたらずなはなしかただし、しつげんもおおいし。
\\	したたらずなはなしかただし、しつげんもおおいし。
\\	「うわきはぶんかだ」でしたっけ?
\\	「うわきはぶんかだ」でしたっけ?
\\	そうそう。「ほっかいどうのひとはなまけもので、メロンもそだてられない」ともいったよな。
\\	そうそう。「ほっかいどうのひとはなまけもので、メロンもそだてられない」ともいったよな。
\\	あれは、まずかったな。
\\	あれは、まずかったな。
\\	いうこともころころかえるから、「にまいじた」なんてがいこくメディアにわるぐちかかれちゃいましたしね。
\\	いうこともころころかえるから、「にまいじた」なんてがいこくメディアにわるぐちかかれちゃいましたしね。
\\	ま、したさきさんずんなのはにっぽんのしゅしょうだけじゃないけどな。
\\	ま、したさきさんずんなのはにっぽんのしゅしょうだけじゃないけどな。
\\	たしかに。
\\	たしかに。
\\	あ、きた
\\	あ、きた
\\	だいいちいんしょう
\\	しんがく
\\	しゅほう
\\	かんしん
\\	てみやげ
\\	けんきゅうしつ
\\	しゅうしかてい
\\	ふかめる
\\	けんきゅう
\\	えんかくきょういく
\\	てんかだいがく
\\	てんかだいがく
\\	つちやひろしせんせい
\\	つちやひろしせんせい
\\	とつぜんのメールにてしつれいいたします。わたしは、げんざい、はねだだいがくで
\\	ラーニングについてけんきゅうしておりますが、さらにけんきゅうをふかめるべく、きだいがくのしゅうしかていへのしんがくをきぼうしております。
\\	とつぜんのメールにてしつれいいたします。わたしは、げんざい、はねだだいがくで
\\	ラーニングについてけんきゅうしておりますが、さらにけんきゅうをふかめるべく、きだいがくのしゅうしかていへのしんがくをきぼうしております。
\\	つちやせんせいのけんきゅうテーマである「こうかてきなえんかくきょういくのしゅほう」にかんしんをもっており、せんせいのごつごうのよろしいときに、おあいしてけんきゅうしつやだいがくいんでのけんきゅうせいかつについておはなしをうかがいたいとおもっております。
\\	つちやせんせいのけんきゅうテーマである「こうかてきなえんかくきょういくのしゅほう」にかんしんをもっており、せんせいのごつごうのよろしいときに、おあいしてけんきゅうしつやだいがくいんでのけんきゅうせいかつについておはなしをうかがいたいとおもっております。
\\	ごたぼうのところきょうしゅくではございますが、おじかんをいただくことがかのうかどうかごへんしんをいただければさいわいにぞんじます。
\\	ごたぼうのところきょうしゅくではございますが、おじかんをいただくことがかのうかどうかごへんしんをいただければさいわいにぞんじます。
\\	さとうまなぶ
\\	さとうまなぶ
\\	よし、かきおえた。でもねんにはねんをいれて、そうしんするまえにまちがいがないかだれかにチェックしてもらうべきだな。
\\	よし、かきおえた。でもねんにはねんをいれて、そうしんするまえにまちがいがないかだれかにチェックしてもらうべきだな。
\\	せんぱいにけんきゅうしつをみにいくべきだっていわれたけど、わからないことだらけだよ。
\\	せんぱいにけんきゅうしつをみにいくべきだっていわれたけど、わからないことだらけだよ。
\\	だいいちいんしょうはだいじだっていうから、けんきゅうしつへはスーツでいくべきかなぁ?
\\	だいいちいんしょうはだいじだっていうから、けんきゅうしつへはスーツでいくべきかなぁ?
\\	あと、てみやげはもっていくべきなのか?
\\	あと、てみやげはもっていくべきなのか?
\\	じゅけん
\\	にゅうし
\\	かんして
\\	がんしょ
\\	めいもん
\\	たいさく
\\	ほうもん
\\	かんじん
\\	いんせい
\\	しょるい
\\	よう。けんきゅうしつほうもんどうだった。
\\	よう。けんきゅうしつほうもんどうだった。
\\	せんぱいのアドバイスとおり、スーツでいって てみやげをもっていきました。
\\	せんぱいのアドバイスとおり、スーツでいって てみやげをもっていきました。
\\	かんじんなけんきゅうないようやにゅうしについてもきいてきたんだろう。
\\	かんじんなけんきゅうないようやにゅうしについてもきいてきたんだろう。
\\	はい。じぶんがやりたい
\\	ラーニングのシステムのけんきゅうができそうです。
\\	はい。じぶんがやりたい
\\	ラーニングのシステムのけんきゅうができそうです。
\\	にゅうしにかんしては、てんかだいいがいのだいがくからでもごうかくしにくいということはない…っていってました。
\\	にゅうしにかんしては、てんかだいいがいのだいがくからでもごうかくしにくいということはない…っていってました。
\\	へー。
\\	へー。
\\	けんきゅうしつのいんせいをなんにんかしょうかいしてくれて、いんせいからもいろいろきくことができました。
\\	けんきゅうしつのいんせいをなんにんかしょうかいしてくれて、いんせいからもいろいろきくことができました。
\\	そりゃよかった。
\\	そりゃよかった。
\\	せんせいにはちょくせつききづらいことでも、いんせいにはききやすいしな。
\\	せんせいにはちょくせつききづらいことでも、いんせいにはききやすいしな。
\\	はい。いんせいにてんかだいのだいがくいんにごうかくするのはかなりむずかしいから、てんかだいいがいのだいがくいんもじゅけんしておいたほうがいいっていわれました。
\\	はい。いんせいにてんかだいのだいがくいんにごうかくするのはかなりむずかしいから、てんかだいいがいのだいがくいんもじゅけんしておいたほうがいいっていわれました。
\\	ゆうしゅうながくせいばかりでえらびがたい…ってまいとしきょうじゅもなやむそうです。
\\	ゆうしゅうながくせいばかりでえらびがたい…ってまいとしきょうじゅもなやむそうです。
\\	そっか。
\\	そっか。
\\	なので、ていこくだいがくのだいがくいんにもがんしょをだそうとおもいます。
\\	なので、ていこくだいがくのだいがくいんにもがんしょをだそうとおもいます。
\\	てんかだいとていこくだいはこうおつつけがたいめいもんだな。
\\	てんかだいとていこくだいはこうおつつけがたいめいもんだな。
\\	しょるいのじゅんびとにゅうしたいさくもがんばれよ。
\\	しょるいのじゅんびとにゅうしたいさくもがんばれよ。
\\	あ、なにかそうだんあったら、れんらくしてこいよ。
\\	あ、なにかそうだんあったら、れんらくしてこいよ。
\\	ありがとうございます。
\\	ありがとうございます。
\\	よう。研究室訪問どうだった。
\\	よう。研究室訪問どうだった。
\\	しゅつだい
\\	じじょう
\\	ふび
\\	リストアップ
\\	そくたつ
\\	ひっきしけん
\\	せいせきしょうめいしょ
\\	けんきゅうけいかくしょ
\\	しぼう
\\	かきとめ
\\	けんきゅうけいかくしょ、しぼうりゆうしょ、せいせきしょうめいしょ…。でも、ふびがありえるかも。
\\	けんきゅうけいかくしょ、しぼうりゆうしょ、せいせきしょうめいしょ…。でも、ふびがありえるかも。
\\	ねんのため、もういちど、チェック…よし
\\	ねんのため、もういちど、チェック…よし
\\	つぎのかたどうぞ。
\\	つぎのかたどうぞ。
\\	これをかきとめでおくってください。あ…そくたつでおねがいします。
\\	これをかきとめでおくってください。あ…そくたつでおねがいします。
\\	あしたにはとどきますよね。
\\	あしたにはとどきますよね。
\\	ええ、あしたのごぜんちゅうにはとどくよていですが、こうつうなどのじじょうでおくれることもありえます。だいじょうぶですか。
\\	ええ、あしたのごぜんちゅうにはとどくよていですが、こうつうなどのじじょうでおくれることもありえます。だいじょうぶですか。
\\	はい。
\\	はい。
\\	ろっぴゃくろくじゅうえんです。
\\	ろっぴゃくろくじゅうえんです。
\\	けんじせんぱい、いま、じかん、だいじょうぶですか?
\\	けんじせんぱい、いま、じかん、だいじょうぶですか?
\\	おう。どうした?
\\	おう。どうした?
\\	せんぱいにみてもらったけんきゅうけいかくしょをおくったんですけど…。
\\	せんぱいにみてもらったけんきゅうけいかくしょをおくったんですけど…。
\\	あとはしけんとめんせつだな。
\\	あとはしけんとめんせつだな。
\\	はい…じつは、めんせつがぼくにがてで…。
\\	はい…じつは、めんせつがぼくにがてで…。
\\	じゃ、きかれそうなことをすべてリストアップすることだな。
\\	じゃ、きかれそうなことをすべてリストアップすることだな。
\\	リストアップ…。
\\	リストアップ…。
\\	ああ。めんせつではきんちょうして、あたまがまっしろになる…ってこともおこりうる。
\\	ああ。めんせつではきんちょうして、あたまがまっしろになる…ってこともおこりうる。
\\	こたえをいうれんしゅうもしておいたほうがいいぞ。
\\	こたえをいうれんしゅうもしておいたほうがいいぞ。
\\	なるほど。
\\	なるほど。
\\	ひっきしけんはだいじょうぶ?
\\	ひっきしけんはだいじょうぶ?
\\	しゅつだいされそうなものはいちおうぜんぶべんきょうしました。
\\	しゅつだいされそうなものはいちおうぜんぶべんきょうしました。
\\	さくせい
\\	れんけい
\\	システムこうちく
\\	じゅうじつ
\\	ぐたいてき
\\	ていじ
\\	こうりつてき
\\	かつよう
\\	かんけつ
\\	りろん
\\	はあく
\\	しつれいいたします。はねだだいがくよりきましたさとうまなぶです。
\\	しつれいいたします。はねだだいがくよりきましたさとうまなぶです。
\\	よろしくおねがいします。
\\	よろしくおねがいします。
\\	まず、ほんだいがくいんをしぼうするりゆうとけんきゅうテーマについてかんけつにのべてください。
\\	まず、ほんだいがくいんをしぼうするりゆうとけんきゅうテーマについてかんけつにのべてください。
\\	わたしのけんきゅうテーマは、
\\	ラーニングをかつようしたがいこくごがくしゅうにかんするものです。
\\	わたしのけんきゅうテーマは、
\\	ラーニングをかつようしたがいこくごがくしゅうにかんするものです。
\\	こうりつてきにがいこくごをマスターできるがくしゅうシステムをつくりたいとかんがえています。
\\	こうりつてきにがいこくごをマスターできるがくしゅうシステムをつくりたいとかんがえています。
\\	だいがくいんでは、システムこうちくのりろんをきそからべんきょうし、じっさいにシステムをつくるぎじゅつをみにつけたいため、きょういくのじゅうじつしたてんかだいがくいんをしぼうしています。
\\	だいがくいんでは、システムこうちくのりろんをきそからべんきょうし、じっさいにシステムをつくるぎじゅつをみにつけたいため、きょういくのじゅうじつしたてんかだいがくいんをしぼうしています。
\\	どんながくしゅうシステムをぐたいてきにイメージしていますか?
\\	どんながくしゅうシステムをぐたいてきにイメージしていますか?
\\	がくしゅうしゃのじゃくてんをはあくしながら、がくしゅうコンテンツをていじするかたちです。
\\	がくしゅうしゃのじゃくてんをはあくしながら、がくしゅうコンテンツをていじするかたちです。
\\	ようするに、フィードバックきのうがついたがくしゅうシステムということ?
\\	ようするに、フィードバックきのうがついたがくしゅうシステムということ?
\\	はい、そうです。いいかえれば、テストとれんけいしたがくしゅうシステムです。
\\	はい、そうです。いいかえれば、テストとれんけいしたがくしゅうシステムです。
\\	そぎょうけんきゅうについておしえてください。
\\	そぎょうけんきゅうについておしえてください。
\\	はい。がくしゅうしゃがたんごをこうりつてきにきおくできるようながくしゅうシステムをさくせいしました。
\\	はい。がくしゅうしゃがたんごをこうりつてきにきおくできるようながくしゅうシステムをさくせいしました。
\\	ということは、きほんてきなプログラミングのちしきはあるんですね。
\\	ということは、きほんてきなプログラミングのちしきはあるんですね。
\\	はい。
\\	はい。
\\	じゅけんばんごう
\\	ふんいき
\\	さいちゅう
\\	ろうにん
\\	いちらん
\\	なんとなく
\\	きょうそうりつ
\\	にらめっこ
\\	きょうじゅ
\\	ななころびやおき
\\	よう
\\	パソコンとにらめっこか?
\\	よう
\\	パソコンとにらめっこか?
\\	あ、せんぱい。いまてんかだいがくのごうかくしゃリストがじゅういちじからだいがくのホームページではっぴょうされるよていなんです。
\\	あ、せんぱい。いまてんかだいがくのごうかくしゃリストがじゅういちじからだいがくのホームページではっぴょうされるよていなんです。
\\	あといっぷんでじゅういちじ…。くびをながくしてまっているところか。ていこくだいもじゅけんしたんだろ?どうだったんだ?
\\	あといっぷんでじゅういちじ…。くびをながくしてまっているところか。ていこくだいもじゅけんしたんだろ?どうだったんだ?
\\	ふごうかくでした。ほかにもふたつだいがくいんをじゅけんしたんですが、それもダメでした。
\\	ふごうかくでした。ほかにもふたつだいがくいんをじゅけんしたんですが、それもダメでした。
\\	ことしはどこのだいがくいんもきょうそうりつがたかかったらしいからな。
\\	ことしはどこのだいがくいんもきょうそうりつがたかかったらしいからな。
\\	めんせつのふんいきでなんとなくごうかくできないのはわかっていたんですよね。きょうじゅたちはぼくのはなしのさいちゅうになんどもくびをひねってたんです。
\\	めんせつのふんいきでなんとなくごうかくできないのはわかっていたんですよね。きょうじゅたちはぼくのはなしのさいちゅうになんどもくびをひねってたんです。
\\	そうか…。
\\	そうか…。
\\	これで、てんかだいもおちたらどうしよう。ろうにんすることはおやがくびをたてにふらないだろうし…。
\\	これで、てんかだいもおちたらどうしよう。ろうにんすることはおやがくびをたてにふらないだろうし…。
\\	ななころびやおきっていうじゃないか、くよくよすんな。あ、もうじゅういちじだぞ。あ、でてるでてる。ごうかくしゃいちらん。
\\	ななころびやおきっていうじゃないか、くよくよすんな。あ、もうじゅういちじだぞ。あ、でてるでてる。ごうかくしゃいちらん。
\\	え?おれのじゅけんばんごうは…あった
\\	ありました、せんぱい。
\\	え?おれのじゅけんばんごうは…あった
\\	ありました、せんぱい。
\\	おめでとう。
\\	おめでとう。
\\	くびのかわいちまいでつながりました。
\\	くびのかわいちまいでつながりました。
\\	おれもくびをつっこんだかいがあったよ。
\\	おれもくびをつっこんだかいがあったよ。
\\	ねんしゅう
\\	りっこうほ
\\	だいじん
\\	ひはん
\\	ほうどう
\\	とうひょう
\\	せんきょ
\\	ちじ
\\	せいさく
\\	しせいかつ
\\	じゅうがつようかはちじせんきょのとうひょうびです。
\\	じゅうがつようかはちじせんきょのとうひょうびです。
\\	とうひょうじかんはあさしちじからよるはちじまでです。かならずとうひょうにいきましょう。
\\	とうひょうじかんはあさしちじからよるはちじまでです。かならずとうひょうにいきましょう。
\\	じゅうがつようかはちじせんきょのとうひょうびです。とうひょうじかんは…。
\\	じゅうがつようかはちじせんきょのとうひょうびです。とうひょうじかんは…。
\\	らいしゅうのにちようびはせんきょか。
\\	らいしゅうのにちようびはせんきょか。
\\	りっこうほしてるのはよにんだったっけ。
\\	りっこうほしてるのはよにんだったっけ。
\\	じみんとうのあべやまさんとみんしゅとうのすがやさん、きょうさんとうのちいさん。
\\	じみんとうのあべやまさんとみんしゅとうのすがやさん、きょうさんとうのちいさん。
\\	せいさくめんではあべやまさんがいいとおもうけど…おやのななひかりでぎいんになったおぼっちゃんだからな。
\\	せいさくめんではあべやまさんがいいとおもうけど…おやのななひかりでぎいんになったおぼっちゃんだからな。
\\	まえにだいじんをしていたときに「もうつかれた」っていってやめたんでしょ?
\\	まえにだいじんをしていたときに「もうつかれた」っていってやめたんでしょ?
\\	こどもじゃあるまいし…。こくみんをバカにしているよね。
\\	こどもじゃあるまいし…。こくみんをバカにしているよね。
\\	でも、なんでもかんでもマスコミにひはんされればいやになろうってもんだよ。
\\	でも、なんでもかんでもマスコミにひはんされればいやになろうってもんだよ。
\\	たしかに、どうでもいいしせいかつとかおくさんのこととかおもしろおかしくほうどうするのはもんだいだとおもうけど…。
\\	たしかに、どうでもいいしせいかつとかおくさんのこととかおもしろおかしくほうどうするのはもんだいだとおもうけど…。
\\	でもあべやまさん「サラリーマンのへいきんねんしゅうがどれくらいかしってるんですか?」というきしゃのしつもんに、「さあ、いっせんまんくらいですか?」ってこたえたんだって。
\\	でもあべやまさん「サラリーマンのへいきんねんしゅうがどれくらいかしってるんですか?」というきしゃのしつもんに、「さあ、いっせんまんくらいですか?」ってこたえたんだって。
\\	いっせんまん? はぁ…。かねもちのぼんぼんにはおれらのきもちはわかるまい。
\\	いっせんまん? はぁ…。かねもちのぼんぼんにはおれらのきもちはわかるまい。
\\	ひなんくんれん
\\	かくれる
\\	かんばん
\\	まどガラス
\\	さわぐ
\\	おおう
\\	ゆれる
\\	たいしょ
\\	かくほ
\\	あわてる
\\	「じしん・かみなり・かじ・おやじ」ってきいたことありますか。
\\	「じしん・かみなり・かじ・おやじ」ってきいたことありますか。
\\	これは、むかしのひとがこわいとおもっていたよっつのものですが、このなかのじしんとかじのたいしょのしかたをおぼえておきましょう。
\\	これは、むかしのひとがこわいとおもっていたよっつのものですが、このなかのじしんとかじのたいしょのしかたをおぼえておきましょう。
\\	じゃ、まずグラッとゆれたらどうしますか。
\\	じゃ、まずグラッとゆれたらどうしますか。
\\	つくえのしたにかくれます。
\\	つくえのしたにかくれます。
\\	そうですね。つくえがなかったらあたまをようふくやカバンでおおいます。
\\	そうですね。つくえがなかったらあたまをようふくやカバンでおおいます。
\\	ほかには?
\\	ほかには?
\\	まどがわれるかもしれないので、まどからはなれます。
\\	まどがわれるかもしれないので、まどからはなれます。
\\	ドアをあけます。
\\	ドアをあけます。
\\	そのとおり。にげみちをかくほするためですよね。
\\	そのとおり。にげみちをかくほするためですよね。
\\	そとにでます。
\\	そとにでます。
\\	うーん。じしんがおきているときにすぐにそとにとびだすのはかえってきけんですよ。
\\	うーん。じしんがおきているときにすぐにそとにとびだすのはかえってきけんですよ。
\\	でも、たてものがこわれるかもしれないし…。
\\	でも、たてものがこわれるかもしれないし…。
\\	ふるいたてものはそういうこともあります。
\\	ふるいたてものはそういうこともあります。
\\	でも、にほんのあたらしいたてものやがっこうならなかにいたほうがそとにでるよりむしろあんぜんなんです。
\\	でも、にほんのあたらしいたてものやがっこうならなかにいたほうがそとにでるよりむしろあんぜんなんです。
\\	おおきなまちではじしんでこわれたかんばんやまどガラスなどが、おちてきます。
\\	おおきなまちではじしんでこわれたかんばんやまどガラスなどが、おちてきます。
\\	だから、まだゆれているときにそとにでるとかえってあぶないんです。
\\	だから、まだゆれているときにそとにでるとかえってあぶないんです。
\\	ゆれがとまってから、あわてず、さわがずにひなんしてください。
\\	ゆれがとまってから、あわてず、さわがずにひなんしてください。
\\	それでは、ひなんくんれんをはじめましょう。
\\	それでは、ひなんくんれんをはじめましょう。
\\	せい
\\	せめる
\\	いとこ
\\	よめ
\\	こわい
\\	やめる
\\	すみ
\\	たいしょく
\\	てんしょく
\\	とりひきさき
\\	たいしょく?
\\	たいしょく?
\\	はい。いろいろかんがえて、それがいちばんいいかと…。
\\	はい。いろいろかんがえて、それがいちばんいいかと…。
\\	きみのようなゆうしゅうなひとがやめてしまうのはざんねんでならないよ。
\\	きみのようなゆうしゅうなひとがやめてしまうのはざんねんでならないよ。
\\	なにかあった?
\\	なにかあった?
\\	いえ。なにもありません。
\\	いえ。なにもありません。
\\	まいったよ。ウチのかのこがたいしょくしたいっていいだしてさ。
\\	まいったよ。ウチのかのこがたいしょくしたいっていいだしてさ。
\\	おおのあやのだろ?
\\	おおのあやのだろ?
\\	なんだよ。しりあいか?おまえもすみにおけないな。
\\	なんだよ。しりあいか?おまえもすみにおけないな。
\\	じつはおれのいとこのよめさんなんだよ。
\\	じつはおれのいとこのよめさんなんだよ。
\\	かのじょはごみのアシスタントだろ?
\\	かのじょはごみのアシスタントだろ?
\\	ごみはじぶんのミスをぜんぶかのじょのせいにしてせめるからはらがたってしかたなかったらしい。
\\	ごみはじぶんのミスをぜんぶかのじょのせいにしてせめるからはらがたってしかたなかったらしい。
\\	ごみか…。そういえば、あいつのアシスタントはすぐやめるな。
\\	ごみか…。そういえば、あいつのアシスタントはすぐやめるな。
\\	ごみはかどのあるいいかたばかりするから、かのじょはかいしゃにくるのがいやでならなかったそうだよ。
\\	ごみはかどのあるいいかたばかりするから、かのじょはかいしゃにくるのがいやでならなかったそうだよ。
\\	なんでおしえてくれなかったんだよ。
\\	なんでおしえてくれなかったんだよ。
\\	「だれにもいわないで」っていわれてたんだ。
\\	「だれにもいわないで」っていわれてたんだ。
\\	ま、でも、かのじょのてんしょくさきはウチのとりひきさきだ。
\\	ま、でも、かのじょのてんしょくさきはウチのとりひきさきだ。
\\	え?
\\	え?
\\	かのじょ、ウチのおきゃくさんのとこにてんしょくするんだ。
\\	かのじょ、ウチのおきゃくさんのとこにてんしょくするんだ。
\\	としん
\\	かいやく
\\	くむ
\\	よっぱらい
\\	こうがい
\\	じゅうたくローン
\\	てぜま
\\	こうしん
\\	こうにゅう
\\	といあわせ
\\	このへやのけいやくのこうしんもうすぐでしょ。どうする?
\\	このへやのけいやくのこうしんもうすぐでしょ。どうする?
\\	こうしんしていいんじゃないか。
\\	こうしんしていいんじゃないか。
\\	ひっこしてきたときはいなかだったけど、まわりにじゅうたくができるにしたがっておみせもふえて、べんりになってきたし。
\\	ひっこしてきたときはいなかだったけど、まわりにじゅうたくができるにしたがっておみせもふえて、べんりになってきたし。
\\	でも、こどもがおおきくなるにつれてへやがてぜまになってきたし、それに、さいきん、えきまえのガラがわるくなったとおもわない?
\\	でも、こどもがおおきくなるにつれてへやがてぜまになってきたし、それに、さいきん、えきまえのガラがわるくなったとおもわない?
\\	まあ、たしかにのみやがふえてくるにつれて、よっぱらいをよくみかけるようになってきたな。
\\	まあ、たしかにのみやがふえてくるにつれて、よっぱらいをよくみかけるようになってきたな。
\\	でしょ?やちんもったいないし、そろそろこうがいにいえをたてようよ。
\\	でしょ?やちんもったいないし、そろそろこうがいにいえをたてようよ。
\\	じゅうたくローンをくむならはやければはやいほどいいでしょ。
\\	じゅうたくローンをくむならはやければはやいほどいいでしょ。
\\	そうだな。でも、どこに?
\\	そうだな。でも、どこに?
\\	としんからはなれればはなれるほどとちはやすくなるけど、そのぶんつうきんが...。
\\	としんからはなれればはなれるほどとちはやすくなるけど、そのぶんつうきんが...。
\\	あ、ぐうぜんなんだけど、うちのおやがじっかのとなりのとちをこうにゅうしたらしいの。
\\	あ、ぐうぜんなんだけど、うちのおやがじっかのとなりのとちをこうにゅうしたらしいの。
\\	おでんわありがとうございます。
\\	ふどうさんです。
\\	おでんわありがとうございます。
\\	ふどうさんです。
\\	おんせいにしたがってすうじをごにゅうりょくください。
\\	おんせいにしたがってすうじをごにゅうりょくください。
\\	けいやくのこうしんのかたはいち、けいやくのかいやくのかたはにを、そのほかのおといあわせのかたはさんをにゅうりょくしてください。
\\	けいやくのこうしんのかたはいち、けいやくのかいやくのかたはにを、そのほかのおといあわせのかたはさんをにゅうりょくしてください。
\\	に。(ピッ)はあ...。おれもマスオさんになるのか。
\\	に。(ピッ)はあ...。おれもマスオさんになるのか。
\\	ほうしゃのう
\\	でんりょく
\\	もれ
\\	あんぜんせい
\\	ダム
\\	たかめる
\\	はつでん
\\	げんしりょく
\\	にさんかたんそ
\\	りそうてき
\\	「げんしりょくはつでんのあんぜんせいをたかめるとともに、クリーンエネルギーにもとりくんでいくひつようがある」か…。
\\	「げんしりょくはつでんのあんぜんせいをたかめるとともに、クリーンエネルギーにもとりくんでいくひつようがある」か…。
\\	ねぇ。しんぶんよみながらごはんたべるのやめて。
\\	ねぇ。しんぶんよみながらごはんたべるのやめて。
\\	はいはい。
\\	はいはい。
\\	ねぇ、パパ。クリーンエネルギーってなに?
\\	ねぇ、パパ。クリーンエネルギーってなに?
\\	ん?クリーンエネルギーっていうのは、かんたんにいえばかんきょうにやさしいエネルギーってことだ。
\\	ん?クリーンエネルギーっていうのは、かんたんにいえばかんきょうにやさしいエネルギーってことだ。
\\	たいようこうはつでんとかふうりょくはつでんとかだな。
\\	たいようこうはつでんとかふうりょくはつでんとかだな。
\\	ダムも?
\\	ダムも?
\\	ダム?ああ、すいりょくはつでんか。
\\	ダム?ああ、すいりょくはつでんか。
\\	そうだな、すいりょくもクリーンエネルギーだな。
\\	そうだな、すいりょくもクリーンエネルギーだな。
\\	ほうしゃのうもれのじこがあったのにげんぱつをやめることはできないの?
\\	ほうしゃのうもれのじこがあったのにげんぱつをやめることはできないの?
\\	げんしりょくはかりょくやすいりょくとならんで、にっぽんではメジャーなはつでんほうほうだから、いますぐやめるのはむりだろうなぁ。
\\	げんしりょくはかりょくやすいりょくとならんで、にっぽんではメジャーなはつでんほうほうだから、いますぐやめるのはむりだろうなぁ。
\\	わりあいはどのくらい?
\\	わりあいはどのくらい?
\\	たしか、かりょくはつでんがやくろくわりからななわり。げんしりょくがにわり。
\\	たしか、かりょくはつでんがやくろくわりからななわり。げんしりょくがにわり。
\\	すいりょくはいちわりじゃくじゃなかったかな。
\\	すいりょくはいちわりじゃくじゃなかったかな。
\\	え?じゃ、たいようこうとかふうりょくとかは?
\\	え?じゃ、たいようこうとかふうりょくとかは?
\\	ぜんぶあわせても、せいぜいいち、にパーセントだろう。
\\	ぜんぶあわせても、せいぜいいち、にパーセントだろう。
\\	えー。それだけ?
\\	えー。それだけ?
\\	じこさえおきなければ、げんしりょくはにさんかたんそもでないし、えられるでんりょくもおおきいから、りそうてきなんだ。じこさえおきなければね。
\\	じこさえおきなければ、げんしりょくはにさんかたんそもでないし、えられるでんりょくもおおきいから、りそうてきなんだ。じこさえおきなければね。
\\	うちゅう
\\	くみたてる
\\	うちゅうひこうし
\\	せっきん
\\	クレーター
\\	ちきゅう
\\	すいせい
\\	てんたいぼうえんきょう
\\	じんこうえいせい
\\	ふりょうひん
\\	ろくじゅうねんにいちど、ちきゅうにせっきんする
\\	すいせい(すいせい)ですが、げんざいもっともちきゅうにせっきんしており、こんやはぼうえんきょうをつかわなくてもはっきりみえるとおもわれます。
\\	ろくじゅうねんにいちど、ちきゅうにせっきんする
\\	すいせい(すいせい)ですが、げんざいもっともちきゅうにせっきんしており、こんやはぼうえんきょうをつかわなくてもはっきりみえるとおもわれます。
\\	ジャーン
\\	かっちゃった。
\\	ジャーン
\\	かっちゃった。
\\	え?
\\	え?
\\	ねだんもろくにみずにかったら、たかくてさ、なんとじゅうまんいじょうしたんだよ、このてんたいぼうえんきょう。
\\	ねだんもろくにみずにかったら、たかくてさ、なんとじゅうまんいじょうしたんだよ、このてんたいぼうえんきょう。
\\	うちゅうにろくにきょうみもないくせに、どうしてそんなにたかいのをかったの?
\\	うちゅうにろくにきょうみもないくせに、どうしてそんなにたかいのをかったの?
\\	インテリアにもなるとおもってさ。このフォルムいいだろ。
\\	インテリアにもなるとおもってさ。このフォルムいいだろ。
\\	あ、それから、これ、おまけでもらった。
\\	あ、それから、これ、おまけでもらった。
\\	なに?
\\	なに?
\\	うちゅうひこうしとじんこうえいせいのフィギュアとうちゅうしょく。
\\	うちゅうひこうしとじんこうえいせいのフィギュアとうちゅうしょく。
\\	うらやましいだろ。
\\	うらやましいだろ。
\\	もう
\\	ろくなものかわないんだから。
\\	もう
\\	ろくなものかわないんだから。
\\	だいいち、けっこんしきのためにちょきんするってきのうやくそくしたばかりじゃん。
\\	だいいち、けっこんしきのためにちょきんするってきのうやくそくしたばかりじゃん。
\\	カリカリしても、ろくなことがないぞ。
\\	カリカリしても、ろくなことがないぞ。
\\	さ、くみたてるぞ。
\\	さ、くみたてるぞ。
\\	よしできた。みてみよう。あれ?まっくらでなにもみえない…。
\\	よしできた。みてみよう。あれ?まっくらでなにもみえない…。
\\	せつめいしょもろくによまずにくみたてるからだよ。
\\	せつめいしょもろくによまずにくみたてるからだよ。
\\	おっかしいな。
\\	おっかしいな。
\\	このぼうえんきょうなら、つきのクレーターややまがみえるっていってたのに…。
\\	このぼうえんきょうなら、つきのクレーターややまがみえるっていってたのに…。
\\	ふりょうひんをかわされたんじゃない。いいカモにされたんだよ。
\\	ふりょうひんをかわされたんじゃない。いいカモにされたんだよ。
\\	たからくじ
\\	かべ
\\	うりば
\\	かせい
\\	さぎ
\\	しょうじ
\\	けんりしょ
\\	とうし
\\	ていきんり
\\	ならぶ
\\	「ふどうさんをかうなら、ていきんりのいまがかいどき」かぁ…。
\\	「ふどうさんをかうなら、ていきんりのいまがかいどき」かぁ…。
\\	でも、さきだつものがなきゃかえないよ。
\\	でも、さきだつものがなきゃかえないよ。
\\	ね、はるき。はるき?あれ?きゅうにみみがとおくなっちゃった?
\\	ね、はるき。はるき?あれ?きゅうにみみがとおくなっちゃった?
\\	みみがいたいなぁ。
\\	みみがいたいなぁ。
\\	てんたいぼうえんきょうなんてかってないできちんとちょきんしてよ。
\\	てんたいぼうえんきょうなんてかってないできちんとちょきんしてよ。
\\	はいはい。それは、みみにたこができるほどききました。
\\	はいはい。それは、みみにたこができるほどききました。
\\	おれだって、いろいろかんがえているんだ。とちにとうししたり…。
\\	おれだって、いろいろかんがえているんだ。とちにとうししたり…。
\\	うそ?とちかったの?それははつみみ。
\\	うそ?とちかったの?それははつみみ。
\\	ジャーン
\\	これかせいのとちのけんりしょ。
\\	ジャーン
\\	これかせいのとちのけんりしょ。
\\	は?こんなのさぎじゃん。かせいになんてすめっこないでしょ。
\\	は?こんなのさぎじゃん。かせいになんてすめっこないでしょ。
\\	そうだけど、ゆめがあるじゃん。
\\	そうだけど、ゆめがあるじゃん。
\\	ゆめねぇ…。
\\	ゆめねぇ…。
\\	じぶんだって、ゆめをかってるだろ。たからくじかったってこみみにはさんだぜ。
\\	じぶんだって、ゆめをかってるだろ。たからくじかったってこみみにはさんだぜ。
\\	どうしてしってんのよ。
\\	どうしてしってんのよ。
\\	しほがたからくじうりばでならんでいるのをげんきがみたっていってたんだよ。
\\	しほがたからくじうりばでならんでいるのをげんきがみたっていってたんだよ。
\\	あたったかどうか、しらべてみようぜ。
\\	あたったかどうか、しらべてみようぜ。
\\	あたるわけがないじゃない。
\\	あたるわけがないじゃない。
\\	やだ…あたってる?…さんおくあたった
\\	やだ…あたってる?…さんおくあたった
\\	シー。かべにみみありしょうじにめあり。
\\	シー。かべにみみありしょうじにめあり。
\\	だれがどこでききみみをたてているかわからないんだぞ。
\\	だれがどこでききみみをたてているかわからないんだぞ。
\\	このことはだれにもいうのはやめよう。
\\	このことはだれにもいうのはやめよう。
\\	こうかい
\\	かんど
\\	さまざま
\\	かぞえきれないほど
\\	ひかれる
\\	ひょうじゅんご
\\	はつおん
\\	ひょうげん
\\	しかた
\\	まるで~ない
\\	ふしぎ
\\	でんとうてき
\\	みなとまち
\\	げいにん
\\	ねた
\\	ごすろり
\\	とくしゅ
\\	ぶっかく
\\	ます
\\	しんせい
\\	びみょう
\\	よこちょう
\\	にくたいろうどうしゃ 
\\	ていねん
\\	ふらりと
\\	ねづよい
\\	かたすみ
\\	せっきゃくする
\\	きさく
\\	みりょうする
\\	さげる
\\	まるみえ
\\	のみやがい
\\	しゅうせん
\\	てんぽ
\\	みっしゅうする
\\	まもない
\\	かじ
\\	しょうしつする
\\	あけはなす
\\	もりあわせ
\\	にほんしゅ
\\	あさがた
\\	こばち
\\	あげもの
\\	にもの
\\	ひややっこ
\\	えだまめ
\\	かわきもの
\\	しおから
\\	のみほうだい
\\	はばひろい
\\	えいぎょうじかん
\\	チェーンてんかいする
\\	そろう
\\	めいがら
\\	ていばん
\\	からあげ
\\	つけもの
\\	たべほうだい
\\	てんないそうしょく
\\	ていきょうする
\\	ほうふ
\\	いたるところ
\\	きがるに
\\	ろじ
\\	みりょく
\\	かんかく
\\	くうかん
\\	きさく
\\	こくせき
\\	あふれだす
\\	かくれが
\\	かんけいしゃ
\\	かいてんする
\\	かんばん
\\	じょうれん
\\	ごうか
\\	ろここちょう
\\	つるす
\\	まっか
\\	そのなのとおり
\\	おもうぞんぶん
\\	外国人バー
\\	外国人バー
\\	店内の壁は、その名の通り真っ赤で、天井には何十個ものシャンデリアが吊るされています。
\\	店内の壁は、その名の通り真っ赤で、天井には何十個ものシャンデリアが吊るされています。
\\	ブドウの形をしたもの、ロココ調のクリスタルの豪華なものなど、見ているだけでも楽しいものばかりです。
\\	ブドウの形をしたもの、ロココ調のクリスタルの豪華なものなど、見ているだけでも楽しいものばかりです。
\\	その他、鹿の顔のオブジェなどユニークなオブジェも展示されています。
\\	その他、鹿の顔のオブジェなどユニークなオブジェも展示されています。
\\	常連の間では、
\\	の看板ともいえる日本人バーテンダーもっさんに会うために訪れる人も少なくありません。
\\	常連の間では、
\\	の看板ともいえる日本人バーテンダーもっさんに会うために訪れる人も少なくありません。
\\	やたい
\\	かいほうてき
\\	しんしんともに
\\	よそうがい
\\	おくない
\\	ちょうりだい
\\	だいしゃ
\\	もくぞう
\\	おーえる
\\	たちよる
\\	こうかした
\\	こうがい
\\	えんせん
\\	つどう
\\	うちとける
\\	どうせい
\\	とまどう
\\	ごうりてき
\\	ぎじれんあい
\\	しめいする
\\	ついかりょうきん
\\	いきつけ
\\	ともす
\\	わせいえいご
\\	ごうせいする
\\	かんばん
\\	ちいき
\\	しょうちょうする
\\	せいじ
\\	いきかう
\\	しこう
\\	うわべ
\\	ざんしん
\\	きょうれつ
\\	しじ
\\	あっとうてき
\\	まんさい
\\	ずらりと
\\	ちょくえいてん
\\	しょうてん
\\	はで
\\	よそおい
\\	ふんだんに
\\	になう
\\	やくめ
\\	あおる
\\	ふちどる
\\	かっこう
\\	よそおい
\\	つつむ
\\	はっしん
\\	へんか
\\	こぐ
\\	ととのえる
\\	うすげしょう
\\	きょうちょうする
\\	おもに
\\	きゅうそく
\\	とくちょう
\\	えいきょう
\\	かっこう
\\	とけこむ
\\	~にめがない
\\	しにせ
\\	きずく
\\	~といってもかごんではない
\\	ふけいき
\\	ひさしい
\\	じこひょうげん
\\	のぞく
\\	ながめる
\\	ざんしん
\\	はっそう
\\	びんかん
\\	ほこる
\\	せつりつする
\\	どぎもをぬく
\\	じょうしつ
\\	はだざわり
\\	ほどこす
\\	ざんしん
\\	げいのうぶんや
\\	ついきゅうする
\\	ひょうか
\\	しゅし
\\	ようし
\\	ひととき
\\	まきこむ
\\	はったつする
\\	どくじに
\\	こっけい
\\	けいこう
\\	たようか
\\	どくとく
\\	よろん
\\	おちゃのま
\\	かんばんばんぐみ
\\	かつどうする
\\	じゅしょうする
\\	ちめいど
\\	えいがかんとく
\\	かつやく
\\	ぜんだいみもん
\\	たさい
\\	とうじょう
\\	ぜつみょう
\\	ひょうひょうとした
\\	むひょうじょう
\\	すき
\\	ぜんてい
\\	ひなんする
\\	やくわりぶんたん
\\	まれ
\\	しじする
\\	ねっきょうてき
\\	しかいしゃ
\\	くんりんする
\\	ひにく
\\	きょうかんする
\\	はっきする
\\	くつじょく
\\	さいようしけん
\\	にらみつける
\\	いちやく
\\	れんぱつする
\\	ほぼ
\\	いつわる
\\	ひょうじゅんご
\\	そうとうする
\\	へいたん
\\	なだらか
\\	そくど
\\	タレント
\\	らいにちする
\\	さける
\\	みみにする
\\	ほうげん
\\	ひんぱんに
\\	じょうだんめいた
\\	いんしょう
\\	きょうつうご
\\	はぎれがいい
\\	よくよう
\\	いだく
\\	れいをあげる
\\	けいえんする
\\	かつやく
\\	ふくむ
\\	たく
\\	ものごと
\\	はんだんする
\\	かんじょう
\\	とくゆう
\\	はつげんする
\\	ちゅうけい
\\	こっかい
\\	いちじるしい
\\	めでたい
\\	せいじ
\\	むす
\\	ごび
\\	しょうりゃくする
\\	きしつ
\\	せっかち
\\	いしきする
\\	しんぼう
\\	むいしきに
\\	たつ
\\	しろくじちゅう
\\	しんせん
\\	じっか
\\	しみじみ
\\	みみにする
\\	しょうりゃくする
\\	ほんでん
\\	そびえたつ
\\	すむ
\\	しんせいな
\\	おがむ
\\	いろとりどり
\\	そうだいな
\\	まつる
\\	ちょうこく
\\	しょうちょうする
\\	あいにく
\\	せかいいさん
\\	そんざいかん
\\	しゅいろ
\\	くむ
\\	あふれだす
\\	わきみず
\\	すずむ
\\	ふきでる
\\	ぜっけい
\\	てっぺん
\\	あちこち
\\	しきち
\\	みわたす
\\	きよめ
\\	てんのう
\\	フラッと
\\	とかい
\\	さんさくする
\\	せいじゃくな
\\	にぎわう
\\	もより
\\	さんぱいきゃく
\\	こうれい
\\	こうたいごう
\\	あしをはこぶ
\\	やおよろず
\\	重要文化財
\\	にんしん
\\	こころをこめて
\\	そなえつけ
\\	えしゃくする
\\	こくほう
\\	くうかん
\\	しんせいな
\\	えんむすび
\\	くぐる
\\	ふりかえる
\\	ほごしゃ
\\	しゅごしん
\\	~たらず
\\	ゆったりと
\\	なんどとなく
\\	きゅうかくど
\\	いちれん
\\	しゅってん
\\	かざる
\\	じょうえいかい
\\	ばしょがら
\\	てんじ
\\	つるす
\\	かんきゃく
\\	ひみつけっしゃ
\\	はたらきかける
\\	ねもと
\\	こうきしん
\\	えいぞうさっか
\\	そうさくいよく
\\	げんだい
\\	せいさくする
\\	じゅんぱく
\\	シャレ
\\	こっけい
\\	さくひん
\\	しゅういつ
\\	せいきょう
\\	みごと
\\	しゅさい
\\	みなもと
\\	ごぞんじ
\\	はっしん
\\	おおはば
\\	しじ
\\	せんさい
\\	だいひょうかく
\\	いちやく
\\	ブーム
\\	すいとう
\\	しにせ
\\	つながり
\\	だいひょうきょく
\\	しゅだいか
\\	かれこれ
\\	しゅざい
\\	きさく
\\	みりょく
\\	あたりまえ
\\	まぼろし
\\	ほめたたえる
\\	いぜん
\\	だいたん
\\	はっしょう
\\	しぶい
\\	シミ
\\	うっすら
\\	ふうあい
\\	はてんこう
\\	せんしょく
\\	いかすみ
\\	ユニーク
\\	きぼ
\\	かいさい
\\	てんじ
\\	ごうどう
\\	かんしんする
\\	チャーター
\\	みうごき
\\	からむ
\\	ほこうしゃ
\\	はいかいする
\\	おびる
\\	はくりょく
\\	ていど
\\	じっしする
\\	つうじょう
\\	せいき
\\	あおる
\\	しゅうち
\\	れいかん
\\	どうようする
\\	ぶたい
\\	ごぞんじ
\\	ゆうりょく
\\	てんたい
\\	ぜいたく
\\	どうきょう
\\	くうかん
\\	またたくまに
\\	めずらしがる
\\	ちかごろ
\\	よほど
\\	ぼろ
\\	きょくたんに
\\	よ
\\	まとめる
\\	めいうつ
\\	けんちく
\\	ひじょうに
\\	ちょくせつ
\\	せんとう
\\	しょくにん
\\	てぎわ
\\	さすがに
\\	たくみ
\\	はけ
\\	しあげる
\\	しょようじかん
\\	みごと
\\	あふれる
\\	はっそう
\\	つらなる
\\	たて
\\	くくりつける
\\	きぐ
\\	しろもの
\\	せつごう
\\	てんじかい
\\	ギャラリー
\\	うんえい
\\	ちんどうぐ
\\	ぎゃくり
\\	しょうぞうが
\\	したう
\\	しっぴつ
\\	さかい
\\	きとくじょうたい
\\	いかいよう
\\	だっきゃくする
\\	わがはい
\\	げんどう
\\	ていきてきに
\\	きょとう
\\	はげむ
\\	りゅうちょう
\\	じんりょく
\\	しゃかいふっき
\\	やどす
\\	じせきのねん
\\	さいなまれる
\\	つづる
\\	あまやどり
\\	こうせき
\\	ふくどくじさつ
\\	すいこうする
\\	たんぺん
\\	けいさいする
\\	はぎとる
\\	たいまつ
\\	はしご
\\	とくひつすべき
\\	しょうがい
\\	いたい
\\	みすい
\\	しんじゅう
\\	だんことして
\\	よけいもの
\\	ひょうする
\\	ぼつご
\\	せいたん
\\	しゅうさい
\\	いっせんをかくす
\\	かたわら
\\	ちゃくそうをえる
\\	すじがき
\\	かれん
\\	いちざ
\\	こじ
\\	きょうぐう
\\	ついには
\\	とげる
\\	さいけん
\\	かんねんてき
\\	えんじょうする
\\	わきあがる
\\	がんぼう
\\	まぢかに
\\	どうじんし
\\	せんご
\\	そうりょ
\\	ちゅうしょうてき
\\	はいえん
\\	ひはん
\\	ぶんるい
\\	ほんそうする
\\	じゅうじ
\\	ひりょう
\\	かこくな
\\	ぼうとうぶ
\\	ぶんめい
\\	みがってな
\\	げんみつにいえば
\\	きんげんだい
\\	かんかする
\\	しぐさ
\\	おもかげ
\\	じょうじゅする
\\	しぼをつのらせる
\\	ようしたんれい
\\	こういをよせる
\\	ねづよい
\\	じゅしょうする
\\	きゃくほんか
\\	えんしゅつか
\\	かきたてる
\\	いきいきとしている
\\	いっそう
\\	いちやく
\\	かいしゃく
\\	しじ
\\	ひゆ
\\	たいきょく
\\	じょうしする
\\	こうほ
\\	じもくをあつめる
\\	あんじする
\\	ずいしょ
\\	げんそうてき
\\	きょうだん
\\	ぶたい
\\	れんそうする
\\	やなぎ
\\	すいてき
\\	せきひ
\\	ていぼう
\\	かすみ
\\	かたがき
\\	かとりせんこう
\\	しつげんちたい
\\	しんしょうふうけい
\\	かぐわしい
\\	ぐんせいする
\\	ふうぶつし
\\	おくゆかしい
\\	せいそ
\\	いやされる
\\	ゆうだいな
\\	そこはかとない
\\	しみじみとした
\\	きょうしゅう
\\	ずいひつ
\\	さくししゃ
\\	ここんとうざい
\\	したわしい
\\	さお
\\	しみる
\\	こがらし
\\	みをたてる
\\	こつこつ
\\	たつ
\\	みかぎる
\\	しょざい
\\	たんちょう
\\	きんせんにふれる
\\	ぶんご
\\	さんびか
\\	すみわたる
\\	じんと
\\	こぶな
\\	けいさいする
\\	ちゃくそうをえる
\\	いやく
\\	こじ
\\	なじみぶかい
\\	りっしんしゅっせ
\\	ゆらい
\\	はげむ
\\	こうきょうしせつ
\\	としのせ
\\	しょうぎょうしせつ
\\	みんよう
\\	あふれる
\\	せんりつ
\\	どうよう
\\	まなざし
\\	ぶつぎをかもす
\\	ぎおんご
\\	ぼこくご
\\	そぼくな
\\	すうし
\\	いとおしむ
\\	かいしゃくする
\\	くちづたえ
\\	こんれい
\\	ぞくする
\\	こうりつ
\\	みぶりてぶり
\\	かけごえ
\\	こぎだす
\\	たんちょうさ
\\	ごっかん
\\	きょうくん
\\	しゅってん
\\	しゅだい
\\	とうとぶ 
\\	たっとぶ
\\	きしょうてんけつ
\\	いろあせる
\\	せんりつ
\\	しんとうする
\\	ひつぜんてきに
\\	がっしょうきょく
\\	あいしゅう
\\	いくた
\\	かんしょう
\\	つづる
\\	ひかくてき
\\	ほうさく
\\	かざりつけ
\\	なわ
\\	おそなえ
\\	せっく
\\	いけい
\\	のきした
\\	かんそう
\\	ぼうとう
\\	ま
\\	めっする
\\	いる
\\	いわし
\\	ひいらぎ
\\	じゃき
\\	ぶっかく
\\	ふうしゅう
\\	たちばな
\\	こんき
\\	しょうぞく
\\	いましめる
\\	むびょうそくさい
\\	かねつする
\\	めいしん
\\	ぞうか
\\	よろい
\\	ふきながし
\\	ゆうもうかかん
\\	さいやく
\\	あん
\\	かぶと
\\	くずこ
\\	つつじょう
\\	りっしんしゅっせ
\\	カササギ
\\	みずかさ
\\	たんざく
\\	とたん
\\	てんてい
\\	へだてる
\\	かける
\\	つるす
\\	ふうしゅう
\\	れい
\\	たく
\\	どきょうする
\\	そうりょ
\\	れいこん
\\	ほうわ
\\	きせい
\\	たどる
\\	たんか
\\	いね
\\	かんしょう
\\	めいめい
\\	しだいに
\\	さかのぼる
\\	そんざい
\\	みなも
\\	めでる
\\	あとつぎ
\\	えきびょう
\\	きすう
\\	きげん
\\	ねんじゅうぎょうじ
\\	せつじつ
\\	きねん
\\	ならわし
\\	しそう
\\	きゅうれき
\\	きんぷん
\\	きんざいくしょくにん
\\	よくぼう
\\	しょゆう
\\	にくむ
\\	せいしん
\\	しょせつ
\\	ぶっきょう
\\	かね
\\	しんや
\\	さきがけ
\\	ほぞんしょく
\\	しょくたく
\\	いわずもがな
\\	いろどる
\\	すめし
\\	はっこう
\\	いたむ
\\	しょくする
\\	あまじょっぱい
\\	うまみ
\\	せいぶん
\\	レンコン
\\	めいじじだい
\\	つける
\\	おうぎがた
\\	こけい
\\	そくせき
\\	ゆせん
\\	レトルト
\\	いたって
\\	うめぼし
\\	しんかくか
\\	だんげん
\\	さずかる
\\	さしつかえ
\\	シャケ
\\	けいこう
\\	ぐざい
\\	ごかご
\\	こねる
\\	しょっかん
\\	ぶっしつ
\\	にてひなる
\\	わしょく
\\	りょうしゃ
\\	ぬめり
\\	こんぶ 
\\	しん
\\	しょうか
\\	ゆず
\\	しぼりじる
\\	キス
\\	ついで
\\	しょみん
\\	しきおりおり
\\	とく
\\	えきたい
\\	くし
\\	アジ
\\	ふうみ
\\	かわりだね
\\	かごん
\\	ゆげ
\\	かいそう
\\	ばんのう
\\	よだれ
\\	こうけい
\\	えんぶん
\\	せいひん
\\	けん
\\	ちぢれる
\\	しょうひ
\\	ひとくくり
\\	ねっとう
\\	いろん
\\	しんか
\\	となえる
\\	けいじょう
\\	でまわる
\\	ようき
\\	たんぱくしつ
\\	なおさら
\\	せっしゅ
\\	ちょう
\\	ねばりけ
\\	わがらし
\\	しょくもつせんい
\\	はっぽうスチロール
\\	ごろあわせ
\\	とう
\\	ととのえる
\\	のうみん
\\	ろうにゃくなんにょ
\\	ふきゅう
\\	てっぱん
\\	ひっとう
\\	このむ
\\	につまる
\\	のうさぎょう
\\	しぶみ
\\	ねる
\\	でんぷん
\\	しゅじゅ
\\	つぶ
\\	こくもつ
\\	なめらか
\\	でんらい
\\	つぶす
\\	いっぴん
\\	いしずえ
\\	せんしんこく
\\	かっきてき
\\	せしゅうせい
\\	けつえんしゅうだん
\\	とうようする
\\	ひきあいにだす
\\	てきかくに
\\	かごん
\\	いつわ
\\	こうせき
\\	うらぎり
\\	ひぼん
\\	どうぜん
\\	ひげき
\\	みかた
\\	どうじょうする
\\	けんしんてき
\\	えいゆう
\\	うなぎのぼり
\\	きずく
\\	きほん
\\	けんぽう
\\	しそう
\\	おぎなう
\\	しゅわん
\\	とく
\\	えいきょうをうける
\\	むよう
\\	さける
\\	いたわる
\\	ごうりてき
\\	ぼうとう
\\	はんこうする
\\	とっぴ
\\	れいこく
\\	せいこう
\\	げんどうりょく
\\	しゅっしん
\\	かたやぶり
\\	せいか
\\	ざつよう
\\	みぶん
\\	そういくふう
\\	しゅっせ
\\	ふところ
\\	がいしゅつする
\\	ふかのう
\\	かたき
\\	とういつする
\\	どうめい
\\	きょうかんをおぼえる
\\	あかし
\\	さしだす
\\	ぶけせいけん
\\	ようやく
\\	そうししゃ
\\	ひとじち
\\	ようしょうき
\\	しんらい
\\	はいけい
\\	ねらい
\\	かくほする
\\	ひっし
\\	きくみみをもつ
\\	せいさく
\\	かっこうの
\\	せまる
\\	きぐする
\\	たつ
\\	せつりつする
\\	やくしょく
\\	ちからをそそぐ
\\	はんぱつする
\\	しっそ
\\	おしい
\\	けついする
\\	じんぶつ
\\	あいしょう
\\	おいつめる
\\	ちしき
\\	どくがく
\\	しょうぞう
\\	せいど
\\	がくもん
\\	ゆたかな
\\	ほんやく
\\	すすめる
\\	にゅうしゅする
\\	がいねん
\\	はやくちことば
\\	いっせいに
\\	たいかんする
\\	せかいいさん
\\	みずうみ
\\	ゆうだいな
\\	あしをはこぶ
\\	たいへいよう
\\	かいがんせん
\\	かんこうめいしょ
\\	ゆうすう
\\	さいてん
\\	げんみつにいえば
\\	~に わたって
\\	かけごえ
\\	うてん
\\	はっしょう
\\	かいさいする
\\	いっせつに よると
\\	ふうしゅう
\\	どうき
\\	じょぶん
\\	そうたいせいりろん
\\	だいひょうさく
\\	わざ
\\	えがく
\\	こんとらすと
\\	こじん
\\	かいが
\\	わん
\\	すい
\\	とりあげる
\\	けんちく
\\	こうぞう
\\	たいこく
\\	めいしょ
\\	じんだい
\\	さいせんたん
\\	ひがい
\\	けねんする
\\	せっちする
\\	またがる
\\	ゆるやか
\\	あとをたたない
\\	ひょうこう
\\	あたえる
\\	ここんとうざい
\\	こころ を うごかす
\\	みりょうする
\\	ちゅうけいする
\\	たんのうする
\\	かんようく
\\	しんぴてき
\\	へいきん
\\	りょういき
\\	ながめる
\\	さかい
\\	ならぶ
\\	くべつ
\\	ひしょ
\\	しゅうり
\\	しょうする
\\	ゆねすこ
\\	とほうもない
\\	こくほう
\\	けっしょう
\\	げんぞん
\\	かちかん
\\	しょぞう
\\	わだいになる
\\	ちえ
\\	しんぼる
\\	きゅうしゅうする
\\	どくじに
\\	よびな
\\	とうろく
\\	しんどう
\\	ひょうか
\\	じゅれい
\\	かんがえだす
\\	どうくつ
\\	きりかえる
\\	せいび
\\	おすすめ
\\	さかのぼる
\\	こうひょう
\\	たっする
\\	たつ
\\	さんご
\\	にぎわう
\\	かんろく
\\	くわえる
\\	ちけい
\\	いち
\\	めんせき
\\	せいたいけい
\\	かたみち
\\	しめる
\\	めあて
\\	かんそく
\\	じゃっかん
\\	せんじゅつ
\\	このよを さる
\\	とう
\\	もちぬし
\\	きろく
\\	おもいうかぶ
\\	しゅえん
\\	もてはやす
\\	わずか
\\	きょうぎ
\\	こうえん
\\	あっとうてき
\\	はばひろい
\\	かいさん
\\	しじ
\\	こうしん
\\	ぶどう
\\	おこなう
\\	きばつ
\\	はんせん
\\	そうぎ
\\	どくとく
\\	がん
\\	きんきゅう
\\	きめぜりふ
\\	いやく
\\	永眠
\\	ちりょう
\\	さくし
\\	けんい
\\	はやくち
\\	てろっぷ
\\	あたえる
\\	ぎょうかい
\\	あたりまえ
\\	さっきょく
\\	ゆいいつむに
\\	もちいる
\\	くし
\\	ぜんしゃ
\\	くさわけ
\\	おおがかり
\\	こうしゃ
\\	おもいを よせる
\\	しゅし
\\	しじする
\\	ぐたいてき
\\	みりょうする
\\	れきだい
\\	みんかん
\\	きずつく
\\	しつれん
\\	とうとつ
\\	ろうろう
\\	ユーモア
\\	しんり
\\	はんめん
\\	もちぬし
\\	かばー
\\	しょとう
\\	かりすま
\\	あげく
\\	まっただなか
\\	いえる
\\	ふじょうり
\\	きつえん
\\	おおそうどう
\\	りゅうねん
\\	まえむき
\\	まるで
\\	でびゅー
\\	しょうけん
\\	へんせい
\\	じゅうだい
\\	ぜんしん
\\	かいきょ
\\	しんくたんく
\\	かぶか
\\	じゅうでん
\\	ばくはつ
\\	すいせい
\\	せつない
\\	はっしん
\\	きんきょう
\\	げいのうかい
\\	さきがけ
\\	こころまちにする
\\	おうぼ
\\	みとめる
\\	いんしょうてき
\\	しゅうねん
\\	だいべん
\\	いん
\\	ごい
\\	くちずさむ
\\	ちみつ
\\	はげます
\\	ひとがら
\\	ちょうこく
\\	こくほう
\\	にらむ
\\	ふんいき
\\	やくわり
\\	ふせぐ
\\	じいん
\\	はくりょく
\\	たくましい
\\	そんざいかん
\\	りんかく
\\	わし
\\	すみ
\\	しゅぎょう
\\	がんさく
\\	はいしゅつ
\\	すいぼくが
\\	くっきり
\\	がほう
\\	どくじ
\\	しっそ
\\	ゆうふく
\\	かんしん
\\	いきかた
\\	こうか
\\	もてなす
\\	とうじ
\\	くつろぐ
\\	しせい
\\	むきあう
\\	いんしょうは
\\	しゅよう
\\	ごしょ
\\	しょうぐん
\\	ひょうじょう
\\	こつぜんと
\\	しゅっぱんする
\\	いひょうをつく
\\	だいざい
\\	だいひょうてきな
\\	あざやかな
\\	こうず
\\	だいたんな
\\	ずがら
\\	がだん
\\	のうり
\\	よむ
\\	がんらい
\\	かんせい
\\	せんさいな
\\	かんしょうする
\\	そうさくする
\\	へいいな
\\	ぼうとう
\\	りゅうこうする
\\	ごく
\\	はっしょう
\\	はいかいし
\\	あそこ
\\	どこ
\\	マクドナルド
\\	エスカレーター
\\	はい、マクドナルド です。
\\	はい、マクドナルド です。
\\	マクドナルド は どこ に あります か。
\\	マクドナルド は どこ に あります か。
\\	マクドナルド は にかい に あります。
\\	マクドナルド は にかい に あります。
\\	にかい。。。 わかりました。
\\	にかい。。。 わかりました。
\\	すみません。エスカレーター は どこ に あります か?
\\	すみません。エスカレーター は どこ に あります か?
\\	エスカレーター? あそこ に あります。
\\	エスカレーター? あそこ に あります。
\\	かよう
\\	とおる
\\	かわりに
\\	はやく
\\	いじょう
\\	いか
\\	いがい
\\	いぜん
\\	いご
\\	えんりょ
\\	もうすぐ
\\	やっと
\\	やっぱり
\\	たしかに
\\	ちっとも
\\	べんり
\\	じゅうしょ
\\	みつける
\\	みつかる
\\	そだつ
\\	よやく
\\	やくそく
\\	こわれる
\\	こしょうする
\\	けんがく
\\	けんぶつ
\\	そだてる
\\	つる
\\	ふるさと
\\	がたい
\\	わすれる
\\	ゆめ
\\	かわ
\\	ふな
\\	やま
\\	おう
\\	うさぎ
\\	めぐる
\\	ねんね
\\	ぼうや
\\	おもり
\\	こえる
\\	さと
\\	みやげ
\\	でんでんだいこ
\\	しょうのふえ
\\	あたま
\\	くも
\\	しほう
\\	みおろす
\\	かみなり
\\	あおぞら
\\	そびえたつ
\\	かすみ
\\	すそ
\\	むすぶ
\\	ひらく
\\	うつ
\\	て
\\	また
\\	さくら
\\	やよい
\\	みわたす
\\	かすみ
\\	におい
\\	いざ
\\	やね
\\	まごい
\\	およぐ
\\	おとうさん
\\	こどもたち
\\	シャボンだま
\\	やね
\\	こわれる
\\	きえる
\\	うまれる
\\	かぜ
\\	まつむし
\\	すずむし
\\	よなが
\\	こおろぎ
\\	なく
\\	くつわむし
\\	あき
\\	なきだす
\\	なきとおす
\\	ちゃつぼ
\\	ぬける
\\	たわら
\\	いど
\\	おちゃわん
\\	かく
\\	ほそみち
\\	とおす
\\	ごよう
\\	おふだ
\\	おさめる
\\	まいる
\\	いびきをかく
\\	あける
\\	きょう
\\	きのう
\\	デート
\\	あした 
\\	あす
\\	しまる
\\	どあ
\\	ねむる
\\	おきる
\\	となり
\\	こまる
\\	うるさい
\\	よる
\\	ひる
\\	あさ
\\	ねこ
\\	いぬ
\\	うち
\\	ひき
\\	わたし
\\	ぼく
\\	さくら
\\	ざっし
\\	ふる
\\	あめ
\\	めくる
\\	ぺーじ
\\	ほん
\\	さく
\\	はなす
\\	いう
\\	ぺらぺらはなす
\\	ぺちゃくちゃはなす
\\	ともだち
\\	ひみつ
\\	ほかのひと
\\	ちゅう
\\	じゅぎょう
\\	がいこくご
\\	なつかしい
\\	サンリオ
\\	さく
\\	じゅぎょうちゅう
\\	きゃく
\\	かう
\\	いえ
\\	ちかく
\\	はなす
\\	うわ~。なつかしい。このカエル、けろけろけろっぴ?
\\	うわ~。なつかしい。このカエル、けろけろけろっぴ?
\\	そうそう。サンリオのキャラクター。
\\	そうそう。サンリオのキャラクター。
\\	いえのちかくのワンニャンランドでかったの。
\\	いえのちかくのワンニャンランドでかったの。
\\	へぇ~。おきゃくさんがいた?
\\	へぇ~。おきゃくさんがいた?
\\	うーん。パラパラ。
\\	うーん。パラパラ。
\\	そのうち
\\	クスクスわらう
\\	わらいだす
\\	げらげらわらう
\\	にこにこわらう
\\	かお
\\	しあわせ
\\	たくらむ
\\	なんで
\\	こと
\\	いい
\\	なにか
\\	にやにやわらう
\\	かんじ
\\	こども
\\	ころぶ
\\	なく
\\	まいにち
\\	おこる
\\	いう
\\	きもち
\\	しつこい
\\	こえ
\\	もんく
\\	しかる
\\	いう
\\	たいへん
\\	あやまる
\\	さいしょ
\\	わすれる
\\	たんじょうび
\\	せんげつ
\\	かぞく
\\	あかるい
\\	いっしょに
\\	べんきょうする
\\	うちのかあさん、べんきょうしろって、ガミガミうるさいんだよ。
\\	うちのかあさん、べんきょうしろって、ガミガミうるさいんだよ。
\\	たいへんだな。
\\	たいへんだな。
\\	うちのかあさんは、いっしょにテレビをみて、ゲラゲラわらってるよ。
\\	うちのかあさんは、いっしょにテレビをみて、ゲラゲラわらってるよ。
\\	いいなぁ、だいきのいえは。。。いつもニコニコあかるいかぞくだから。
\\	いいなぁ、だいきのいえは。。。いつもニコニコあかるいかぞくだから。
\\	でも、せんげつたいへんだったよ。
\\	でも、せんげつたいへんだったよ。
\\	とうさんがかあさんのたんじょうびをわすれちゃって。。。
\\	とうさんがかあさんのたんじょうびをわすれちゃって。。。
\\	コンピューター
\\	なんども
\\	ほんとうに
\\	となりの
\\	かべ
\\	ける
\\	あな
\\	つま
\\	おこる
\\	しゃちょう
\\	いわれる
\\	たからばこ
\\	みつける
\\	さいこすりらー
\\	れんぞく
\\	どきどきする
\\	わくわくする
\\	はらはらする
\\	めをさます
\\	おもいだす
\\	よごれる
\\	まっくろ
\\	かえる
\\	どうやって
\\	じぶん
\\	へや
\\	ぼく
\\	ほんとうに
\\	はは
\\	しょんぼりする
\\	がっくりくる
\\	いぬ
\\	にわ
\\	いちにちじゅう
\\	おもう
\\	しぬ
\\	ちち
\\	ぼんやりする
\\	しる
\\	ちょう
\\	かみ
\\	つかむ
\\	わたす
\\	みえる
\\	て
\\	まっしろい
\\	やめる
\\	つづける
\\	きこえる
\\	こえ
\\	かみ
\\	がっこう
\\	おこる
\\	ねぇねぇ、このはなし、しっている?
\\	ねぇねぇ、このはなし、しっている?
\\	あるひとががっこうのトイレにはいった。
\\	あるひとががっこうのトイレにはいった。
\\	そしたら、うえから「・・・かみをください・・・」ってこえがきこえた。
\\	そしたら、うえから「・・・かみをください・・・」ってこえがきこえた。
\\	やめて。わたし、こわいはなしきらいなの。ぞっとする。
\\	やめて。わたし、こわいはなしきらいなの。ぞっとする。
\\	つづけて
\\	つづけて
\\	わくわくする
\\	つづけて
\\	つづけて
\\	わくわくする
\\	そのひとは、「ちょっとまって」っていっって、トイレットペーパーをとって、うえをみたんだ。
\\	そのひとは、「ちょっとまって」っていっって、トイレットペーパーをとって、うえをみたんだ。
\\	そしたら、まっしろいてがみえた・・・。
\\	そしたら、まっしろいてがみえた・・・。
\\	ぎょっとしたけど、「はい」ってかみをわたしたんだ。
\\	ぎょっとしたけど、「はい」ってかみをわたしたんだ。
\\	しろいてがそのひとのあたまをつかんで...
\\	しろいてがそのひとのあたまをつかんで...
\\	ゆうべ
\\	すやすや
\\	ぐうぐう
\\	うとうと
\\	ぐっすり
\\	あかちゃん
\\	いびきをかく
\\	そふ
\\	ねむる
\\	ちち
\\	さら
\\	おとこ
\\	おく
\\	たべはじめる
\\	たべおわる
\\	いぬ
\\	ねこ
\\	いつまでも
\\	なめる
\\	さあ
\\	でていく
\\	ひゃくえんだま
\\	まわり
\\	しりあい
\\	こっち
\\	なんども
\\	にがて
\\	はじめて
\\	へや
\\	ひろう
\\	むすこ
\\	すぐに
\\	しずかに
\\	あぶない
\\	プラットホーム
\\	ころぶ
\\	よっぱらう
\\	さいきん
\\	あるく
\\	ろっかげつ
\\	わたしのむすこは、いっさいろっかげつだ。さいきん、よちよちあるきはじめたが、すぐにころぶ。
\\	わたしのむすこは、いっさいろっかげつだ。さいきん、よちよちあるきはじめたが、すぐにころぶ。
\\	よっぱらったひとが、プラットホームを、ふらふらとあるいている。とてもあぶない。
\\	よっぱらったひとが、プラットホームを、ふらふらとあるいている。とてもあぶない。
\\	わたしはとうきょうをふらふらあるくのがすきだ。
\\	わたしはとうきょうをふらふらあるくのがすきだ。
\\	どたばたあるかないでください。しずかにあるいてください。
\\	どたばたあるかないでください。しずかにあるいてください。
\\	おなか
\\	がいこくご
\\	ラベル
\\	きもち
\\	たいへん
\\	のど
\\	こんな
\\	ぜんぶ
\\	ウイスキーボンボン
\\	め
\\	かわく
\\	あー、おなかペコペコ。あ、チョコレートがある。
\\	あー、おなかペコペコ。あ、チョコレートがある。
\\	(キョロキョロ)(ムシャムシャムシャ)
\\	(キョロキョロ)(ムシャムシャムシャ)
\\	あ、パパもチョコたべる?
\\	あ、パパもチョコたべる?
\\	いや、いいよ。パパは、のどがカラカラだ。
\\	いや、いいよ。パパは、のどがカラカラだ。
\\	おお、ジュースがある。(ごくごくごく)
\\	おお、ジュースがある。(ごくごくごく)
\\	ただいま。どうしたの?めがとろんとしてる・・・。
\\	ただいま。どうしたの?めがとろんとしてる・・・。
\\	やだ、このチョコレートたべたの?
\\	やだ、このチョコレートたべたの?
\\	うん。ぜんぶぺろっとたべちゃった。
\\	うん。ぜんぶぺろっとたべちゃった。
\\	そしたら・・・ふらふらするの。(ヒック)
\\	そしたら・・・ふらふらするの。(ヒック)
\\	これ、ウイスキーボンボンよ。パパは?
\\	これ、ウイスキーボンボンよ。パパは?
\\	トイレによろよろあるいていったよ。(ヒック)
\\	トイレによろよろあるいていったよ。(ヒック)
\\	あなた、メイがたいへんなの。あなた?(ドンドンドン)
\\	あなた、メイがたいへんなの。あなた?(ドンドンドン)
\\	あけるわよ(カチャ)やだ、こんなところで、なんでぐうぐうねているの。
\\	あけるわよ(カチャ)やだ、こんなところで、なんでぐうぐうねているの。
\\	ん?さっき、のどがかわいて、ジュースをごくごくのんだんだ。
\\	ん?さっき、のどがかわいて、ジュースをごくごくのんだんだ。
\\	そのあと、どんどんきもちがわるくなって・・・。(ヒック)
\\	そのあと、どんどんきもちがわるくなって・・・。(ヒック)
\\	あのラベルよまなかったの?(ジロッ)
\\	あのラベルよまなかったの?(ジロッ)
\\	・・・チラッとみたけど、(ヒック)がいこくごで・・・わからなかったから・・・。(ヒック)
\\	・・・チラッとみたけど、(ヒック)がいこくごで・・・わからなかったから・・・。(ヒック)
\\	かぜ
\\	ストレス
\\	みみ
\\	なる
\\	い
\\	はら
\\	じつは
\\	のど
\\	すう
\\	やめる
\\	アレルギー
\\	はな
\\	ぶちょう
\\	(ハクション
\\	ハクション
\\	ハクション
\\	(ハクション
\\	ハクション
\\	ハクション
\\	かぜか?
\\	かぜか?
\\	いえ、めが しょぼしょぼして、はなも ずるずるでるので、アレルギーだと おもいます。(ゴホゴホ)
\\	いえ、めが しょぼしょぼして、はなも ずるずるでるので、アレルギーだと おもいます。(ゴホゴホ)
\\	ちょっと、タバコを やめたほうが、いいんじゃないか。
\\	ちょっと、タバコを やめたほうが、いいんじゃないか。
\\	そうかも しれないですね。
\\	そうかも しれないですね。
\\	タバコを すうと、ゲホゲホするし、のどもイ ガイガするんですよ。
\\	タバコを すうと、ゲホゲホするし、のどもイ ガイガするんですよ。
\\	じつは、せんしゅうくらいから、はらも ごろごろするんですよね。
\\	じつは、せんしゅうくらいから、はらも ごろごろするんですよね。
\\	おいおい、だいじょうぶかよ。
\\	おいおい、だいじょうぶかよ。
\\	ときどき きりきりと いたいんですよね。
\\	ときどき きりきりと いたいんですよね。
\\	ものを たべると、いが むかむかするし、ときどき、みみも キーンとするし。
\\	ものを たべると、いが むかむかするし、ときどき、みみも キーンとするし。
\\	いしゃに いけよ。
\\	いしゃに いけよ。
\\	まいにち、へとへとに なるまで、はたらいているので、ストレスですかね、ぶちょう。
\\	まいにち、へとへとに なるまで、はたらいているので、ストレスですかね、ぶちょう。
\\	くもる
\\	いや
\\	そと
\\	いそぐ
\\	ひかる
\\	なる
\\	かみなり
\\	ふる
\\	はれる
\\	くらい
\\	いやなてんきですね、せんぱい。
\\	いやなてんきですね、せんぱい。
\\	どんよりくもっているし、かみなりはごろごろなってるし、そとはくらいし。
\\	どんよりくもっているし、かみなりはごろごろなってるし、そとはくらいし。
\\	じめじめしているから、そのうち、ザーッとふりはじめるんじゃないか。
\\	じめじめしているから、そのうち、ザーッとふりはじめるんじゃないか。
\\	さっきまで、からっとはれてたのに・・・
\\	さっきまで、からっとはれてたのに・・・
\\	あ、いま、ぴかっとひかった。
\\	あ、いま、ぴかっとひかった。
\\	きゃ
\\	あー、いまのおおきかったですね。
\\	きゃ
\\	あー、いまのおおきかったですね。
\\	あ、ぽつぽつふってきた。
\\	あ、ぽつぽつふってきた。
\\	うわー。ざあざあぶりだ。いそげ
\\	えきまで、はしるぞ。
\\	うわー。ざあざあぶりだ。いそげ
\\	えきまで、はしるぞ。
\\	ちょっと、まって、せんぱい。ここ、つるつるしてて。あ、(ドスン)イタタタタ、・・・
\\	ちょっと、まって、せんぱい。ここ、つるつるしてて。あ、(ドスン)イタタタタ、・・・
\\	あーん。あたらしいスーツがびしょびしょ・・・。
\\	あーん。あたらしいスーツがびしょびしょ・・・。
\\	このまえ
\\	あつあつ
\\	じょうず
\\	さいこう
\\	おかし
\\	なる
\\	たつ
\\	なか
\\	そと
\\	かたい
\\	あおいちゃん、このまえは、ケーキありがとう。
\\	あおいちゃん、このまえは、ケーキありがとう。
\\	どうだった?
\\	どうだった?
\\	なんか、そとはカッチカチでなかはパッサパサだったよ。
\\	なんか、そとはカッチカチでなかはパッサパサだったよ。
\\	え?ほんとう?つくったときは、パサパサにみえなかったけど...。
\\	え?ほんとう?つくったときは、パサパサにみえなかったけど...。
\\	じかんがたって、かたくなったのかな...。
\\	じかんがたって、かたくなったのかな...。
\\	うちのおかあさん、おかしつくるのじょうずなんだ。
\\	うちのおかあさん、おかしつくるのじょうずなんだ。
\\	かあさんのケーキは、そとは、ふんわり、なかはしっとり。クッキーはさくさくでおいしいんだ。
\\	かあさんのケーキは、そとは、ふんわり、なかはしっとり。クッキーはさくさくでおいしいんだ。
\\	へー。
\\	へー。
\\	すてき
\\	かお
\\	けい
\\	けっこうです。
\\	きんにく
\\	おれ
\\	やける
\\	みせる
\\	かみ
\\	はだ
\\	マジ
\\	あおいって、ビジュアルけい すきなの?
\\	あおいって、ビジュアルけい すきなの?
\\	うん。だいすき。
\\	うん。だいすき。
\\	みんな、きれいで、すらっとしてて。
\\	みんな、きれいで、すらっとしてて。
\\	みて、このしゃしん、すてきじゃない?
\\	みて、このしゃしん、すてきじゃない?
\\	ええ?これ、おとこ?ガリガリ じゃない。
\\	ええ?これ、おとこ?ガリガリ じゃない。
\\	なに この、つるつるの はだ、サラサラの かみ。おんな の こ みたい。・・・
\\	なに この、つるつるの はだ、サラサラの かみ。おんな の こ みたい。・・・
\\	わたしは、もっと がっちりしたひと が すきだなぁ。
\\	わたしは、もっと がっちりしたひと が すきだなぁ。
\\	なになに、みせてみせて。・・・
\\	なになに、みせてみせて。・・・
\\	うーん、かお は かわいい けど いまいち。おれ は もうすこし、ぽっちゃりした こ の ほう が いいな。
\\	うーん、かお は かわいい けど いまいち。おれ は もうすこし、ぽっちゃりした こ の ほう が いいな。
\\	・・・これ、おとこ の ひと だよ。
\\	・・・これ、おとこ の ひと だよ。
\\	えええええ?マジ?おとこ?
\\	えええええ?マジ?おとこ?
\\	でも、おとこ は、こんがり やけた はだ に、むきむき が いい でしょ。
\\	でも、おとこ は、こんがり やけた はだ に、むきむき が いい でしょ。
\\	みる?おれ の きんにく?
\\	みる?おれ の きんにく?
\\	けっこう です。
\\	けっこう です。
\\	きつい
\\	ぜいたく
\\	はなしかた
\\	となり
\\	へや
\\	あいつ
\\	だらしない
\\	けっこう
\\	けとばす
\\	はっきり
\\	ゆうかちゃんって いいよね。ちゃきちゃき、さばさばしていて。
\\	ゆうかちゃんって いいよね。ちゃきちゃき、さばさばしていて。
\\	ゆうか?やめとけ やめとけ。
\\	ゆうか?やめとけ やめとけ。
\\	きついことを ビシッというし、ちゃらちゃらしていると、けとばされるぞ。
\\	きついことを ビシッというし、ちゃらちゃらしていると、けとばされるぞ。
\\	そうかな。
\\	そうかな。
\\	みんなは、ゆうかは てきぱきしてるっていうけど、おれは、せかせかしているだけだ とおもうね。
\\	みんなは、ゆうかは てきぱきしてるっていうけど、おれは、せかせかしているだけだ とおもうね。
\\	おっちょこちょいだしな。しかも、けっこうだらしないんだぜ。あいつのへや ぐちゃぐちゃなんだ。
\\	おっちょこちょいだしな。しかも、けっこうだらしないんだぜ。あいつのへや ぐちゃぐちゃなんだ。
\\	へー。よく しっているね。
\\	へー。よく しっているね。
\\	いえが となりだからな。
\\	いえが となりだからな。
\\	それより、あおいちゃんのほうが、いいだろう。ほのぼのした はなしかただし、のほほんとして、かわいいし。
\\	それより、あおいちゃんのほうが、いいだろう。ほのぼのした はなしかただし、のほほんとして、かわいいし。
\\	ぱっと しないんだよね。
\\	ぱっと しないんだよね。
\\	たしかに、おっとりしてて、かわいいけど、はっきりしないし、ぐずぐずしていて すきじゃないんだ。
\\	たしかに、おっとりしてて、かわいいけど、はっきりしないし、ぐずぐずしていて すきじゃないんだ。
\\	ぜいたくだなぁ。
\\	ぜいたくだなぁ。
\\	こんにちは。
\\	わたし
\\	どうぞ
\\	つま
\\	よろしくおねがいします
\\	ぼく
\\	なまえ
\\	ぐうぜん
\\	しゅじん
\\	こんにちは。わたしは、さとう けいこです。
\\	こんにちは。わたしは、さとう けいこです。
\\	どうぞ よろしく おねがいします。
\\	どうぞ よろしく おねがいします。
\\	けいこさん?(わらい) つまの なまえも けいこです。
\\	けいこさん?(わらい) つまの なまえも けいこです。
\\	あ、ぼくは たなかあきらです。
\\	あ、ぼくは たなかあきらです。
\\	それは、 ぐうぜん
\\	わたしの しゅじん もあきらです。
\\	それは、 ぐうぜん
\\	わたしの しゅじん もあきらです。
\\	えいご
\\	つうやく
\\	しゅじん
\\	ちゅうごくご
\\	むかし
\\	べんきょうする
\\	わすれる
\\	でも
\\	わたしのしゅじんは、つうやくをしています。
\\	わたしのしゅじんは、つうやくをしています。
\\	へー。えいご?
\\	へー。えいご?
\\	いえ。しゅじんは ちゅうごくごが できます。 
\\	いえ。しゅじんは ちゅうごくごが できます。 
\\	けいこさんも ちゅうごくごが わかる?
\\	けいこさんも ちゅうごくごが わかる?
\\	わたし?わたしも むかし、ちゅうごくごを べんきょうしました。
\\	わたし?わたしも むかし、ちゅうごくごを べんきょうしました。
\\	でも・・・わすれました。(わらい)
\\	でも・・・わすれました。(わらい)
\\	みなさん
\\	こうちょうせんせい
\\	せんせい
\\	あたらしい
\\	えいご
\\	にねんせい
\\	おしえる
\\	こちら
\\	しょうかい
\\	みなさん、おはようございます。
\\	みなさん、おはようございます。
\\	おはようございます。
\\	おはようございます。
\\	あたらしいせんせいを  しょうかいします。
\\	あたらしいせんせいを  しょうかいします。
\\	こちらは、あたらしいえいごの せんせいの さとう けいこせんせい。
\\	こちらは、あたらしいえいごの せんせいの さとう けいこせんせい。
\\	さとうせんせいは にねんせいの えいごを おしえます。
\\	さとうせんせいは にねんせいの えいごを おしえます。
\\	よろしく おねがいします。
\\	よろしく おねがいします。
\\	おこさん
\\	こ
\\	さい
\\	おとこのこ
\\	うちのこ
\\	こんしゅう
\\	バーベキュー
\\	じかん
\\	ぜひ
\\	けいこさんは、おこさんが いますか。
\\	けいこさんは、おこさんが いますか。
\\	はい。よんさいの おとこのこが います。
\\	はい。よんさいの おとこのこが います。
\\	ほんとうですか?うちのこも よんさい。
\\	ほんとうですか?うちのこも よんさい。
\\	あ、こんしゅうの にちようび、じかんがありますか。
\\	あ、こんしゅうの にちようび、じかんがありますか。
\\	バーベキューを しませんか。
\\	バーベキューを しませんか。
\\	はい
\\	ぜひ。
\\	はい
\\	ぜひ。
\\	でる
\\	りょこう
\\	かかる
\\	なんぷん
\\	ここ
\\	くうこう
\\	いく
\\	ばんせん
\\	ホーム(プラットホーム)
\\	つぎ
\\	でんしゃ
\\	ひこうき
\\	すみません。これは くうこうへ いきますか。
\\	すみません。これは くうこうへ いきますか。
\\	いえ、いきません。2ばんせんのホームに いってください。
\\	いえ、いきません。2ばんせんのホームに いってください。
\\	つぎのでんしゃは 2ばんせんから でます。
\\	つぎのでんしゃは 2ばんせんから でます。
\\	ここから、くうこうまで なんぷん かかりますか。
\\	ここから、くうこうまで なんぷん かかりますか。
\\	さんじゅっぷんくらいです。・・・りょこう?
\\	さんじゅっぷんくらいです。・・・りょこう?
\\	いえ、ひこうきを みに いきます。
\\	いえ、ひこうきを みに いきます。
\\	でんわ
\\	ご
\\	すぐ
\\	つく
\\	いい
\\	いま
\\	なんじ
\\	しんさつ
\\	しか
\\	うけつけ
\\	はい、おでんわ ありがとうございます。なかやましか です。
\\	はい、おでんわ ありがとうございます。なかやましか です。
\\	すみません。しんさつは、なんじから なんじまで ですか。
\\	すみません。しんさつは、なんじから なんじまで ですか。
\\	くじから ろくじまで です。
\\	くじから ろくじまで です。
\\	いまから、いいですか。
\\	いまから、いいですか。
\\	あ、はい。なんじに つきますか。
\\	あ、はい。なんじに つきますか。
\\	すぐです。ごふんごに つきます。
\\	すぐです。ごふんごに つきます。
\\	こうえん
\\	バーベキュー
\\	ちかく
\\	はし
\\	いりぐち
\\	わたる
\\	みえる
\\	いけ
\\	どこ
\\	ばしょ
\\	たなかさん?さとうです。ばしょが わかりません。
\\	たなかさん?さとうです。ばしょが わかりません。
\\	いま、どこに いますか。
\\	いま、どこに いますか。
\\	こうえんの いりぐちに います。
\\	こうえんの いりぐちに います。
\\	いけが みえますか。
\\	いけが みえますか。
\\	はい、みえます。
\\	はい、みえます。
\\	いけの はしを わたってください。
\\	いけの はしを わたってください。
\\	はしの ちかくで バーベキューを しています。
\\	はしの ちかくで バーベキューを しています。
\\	はし?ん?ちょっと わかりません。
\\	はし?ん?ちょっと わかりません。
\\	わかりました。じゃ、いりぐちで まっていてください。いま いきます
\\	わかりました。じゃ、いりぐちで まっていてください。いま いきます
\\	ソーセージ
\\	うちのこ
\\	おみず
\\	おちゃ
\\	のむ
\\	ジュース
\\	ワイン
\\	やさい
\\	ビール
\\	かう
\\	つま
\\	ソーセージと やさいを かってきました。
\\	ソーセージと やさいを かってきました。
\\	ありがとう ございます。
\\	ありがとう ございます。
\\	ビールや ワインや ジュースが あります。なにを のみますか。
\\	ビールや ワインや ジュースが あります。なにを のみますか。
\\	じゃ、おちゃ ありますか?
\\	じゃ、おちゃ ありますか?
\\	ありますよ。はい どうぞ。じゅんくん は?
\\	ありますよ。はい どうぞ。じゅんくん は?
\\	おみず。
\\	おみず。
\\	おみず?
\\	おみず?
\\	すみません。うちのこ、ジュースも おちゃも すきじゃなくて・・・。
\\	すみません。うちのこ、ジュースも おちゃも すきじゃなくて・・・。
\\	すぐ
\\	すごい
\\	つくる
\\	だれ
\\	じょうず
\\	あげる
\\	おばちゃん
\\	はいしゃ
\\	だんな
\\	くる
\\	さとうさん、だんなさんは?
\\	さとうさん、だんなさんは?
\\	あ、しゅじんは、はいしゃに いきました。すぐ きます。
\\	あ、しゅじんは、はいしゃに いきました。すぐ きます。
\\	はい、おばちゃん。これ、あげる。
\\	はい、おばちゃん。これ、あげる。
\\	これは なに?
\\	これは なに?
\\	つる
\\	つる
\\	うわー。じょうず。だれが つくったの?
\\	うわー。じょうず。だれが つくったの?
\\	ゆなが つくったの。
\\	ゆなが つくったの。
\\	すごいねぇ。
\\	すごいねぇ。
\\	さあ
\\	おしっこ
\\	ビール
\\	どうぞ
\\	つめたい
\\	もの
\\	ひつよう
\\	こっち
\\	おくれる
\\	すぐそこ
\\	あ、きた きた。パパー
\\	こっち こっち
\\	あ、きた きた。パパー
\\	こっち こっち
\\	すみません。おくれました。はじめまして。さとうです。
\\	すみません。おくれました。はじめまして。さとうです。
\\	はじめまして、たなかです。さあ、どうぞ。
\\	はじめまして、たなかです。さあ、どうぞ。
\\	なにを のみますか。
\\	なにを のみますか。
\\	さとう(おっと)えっと・・・つめたいものが のみたいです。
\\	さとう(おっと)えっと・・・つめたいものが のみたいです。
\\	ビール、ありますか。
\\	ビール、ありますか。
\\	パパー。ぼく、おしっこが したい。
\\	パパー。ぼく、おしっこが したい。
\\	さとう(おっと)ええ?ママは?
\\	さとう(おっと)ええ?ママは?
\\	パパが いい。
\\	パパが いい。
\\	そうか。わかった。
\\	そうか。わかった。
\\	あ、すぐそこに トイレが あります。
\\	あ、すぐそこに トイレが あります。
\\	たぶん、おかねが ひつようです。
\\	たぶん、おかねが ひつようです。
\\	フットサル
\\	とくい
\\	やきゅう
\\	くる
\\	こんど
\\	たのしい
\\	しあい
\\	たいへん
\\	れんしゅう
\\	サッカー
\\	へー。たなかさんは フットサルを するんですか。
\\	へー。たなかさんは フットサルを するんですか。
\\	ええ。れんしゅうは たいへんですが、しあいは たのしいです。
\\	ええ。れんしゅうは たいへんですが、しあいは たのしいです。
\\	こんど、れんしゅうに きませんか。
\\	こんど、れんしゅうに きませんか。
\\	うーん。やきゅうは とくいですが、サッカーは ちょっと・・・。
\\	うーん。やきゅうは とくいですが、サッカーは ちょっと・・・。
\\	おや
\\	こども
\\	でんわ
\\	ともだち
\\	じゅく
\\	うち 
\\	いえ
\\	けいさつ
\\	せんせい
\\	はい、たなかです。
\\	はい、たなかです。
\\	せんせい?うちの こどもが いないんです。
\\	せんせい?うちの こどもが いないんです。
\\	は?
\\	は?
\\	いえにも じゅくにも いません。
\\	いえにも じゅくにも いません。
\\	ともだちの うちへも、でんわ しましたが、いないんです。
\\	ともだちの うちへも、でんわ しましたが、いないんです。
\\	けいさつへは でんわを しましたか。
\\	けいさつへは でんわを しましたか。
\\	ええ?けいさつ?
\\	ええ?けいさつ?
\\	せいと
\\	べんきょう
\\	いい
\\	かえる
\\	ごめんなさい。
\\	わるい
\\	ぶじ
\\	おかあさん
\\	しんぱい
\\	ひこうき
\\	そうか。じゃ、くうこうで ひこうきを みていたんだね。
\\	そうか。じゃ、くうこうで ひこうきを みていたんだね。
\\	しんぱいしたんだぜ。
\\	しんぱいしたんだぜ。
\\	はい。すみません。
\\	はい。すみません。
\\	おかあさん、しんぱいしたわ。
\\	おかあさん、しんぱいしたわ。
\\	もう、べんきょうしたくなかったんだよ。
\\	もう、べんきょうしたくなかったんだよ。
\\	・・・おかあさん、わるかったわ。
\\	・・・おかあさん、わるかったわ。
\\	いつも「べんきょう、べんきょう」いって。
\\	いつも「べんきょう、べんきょう」いって。
\\	かあさんは わるくないよ。・・・ごめんなさい。
\\	かあさんは わるくないよ。・・・ごめんなさい。
\\	ま、ぶじで よかったよ。さ、かえるぞ。
\\	ま、ぶじで よかったよ。さ、かえるぞ。
\\	はい。
\\	はい。
\\	ぶじ
\\	はんせいする
\\	かわいそう
\\	だけ
\\	まいにち
\\	あそぶ
\\	ともだち
\\	よむ
\\	マンガ
\\	テレビ
\\	べんきょう
\\	ストレス
\\	おかあさん
\\	ぶじだったの?よかったじゃん。
\\	ぶじだったの?よかったじゃん。
\\	そうだなぁ。
\\	そうだなぁ。
\\	べんきょうのストレスかな。
\\	べんきょうのストレスかな。
\\	そうだろうなぁ。
\\	そうだろうなぁ。
\\	「べんきょうしろ」「テレビみるな」「マンガよむな」「ともだちとあそぶな」。。。
\\	「べんきょうしろ」「テレビみるな」「マンガよむな」「ともだちとあそぶな」。。。
\\	まいにち まいにち べんきょうだけ。。。ま、かわいそうだよな。
\\	まいにち まいにち べんきょうだけ。。。ま、かわいそうだよな。
\\	おかあさんも はんせいしたかな?
\\	おかあさんも はんせいしたかな?
\\	どうだろうね。
\\	どうだろうね。
\\	ほうちょう
\\	いたい
\\	ゆび
\\	きる
\\	てつだう
\\	かぼちゃ
\\	つくる
\\	スープ
\\	イタッ
\\	イタッ
\\	どうした?
\\	どうした?
\\	ほうちょうで ゆび きっちゃった。
\\	ほうちょうで ゆび きっちゃった。
\\	てつだうよ。
\\	てつだうよ。
\\	じゃ、ほうちょうで、そのかぼちゃ きって。
\\	じゃ、ほうちょうで、そのかぼちゃ きって。
\\	なにを つくるの?
\\	なにを つくるの?
\\	かぼちゃで スープを つくるの。
\\	かぼちゃで スープを つくるの。
\\	クイズ
\\	おおきい
\\	カナダ
\\	チーター
\\	はやぶさ
\\	なか
\\	とり
\\	どうぶつ
\\	はやい
\\	ばつ
\\	まる
\\	あまぞんがわ
\\	ながい
\\	かわ
\\	いちばん
\\	せかい
\\	ざんねん
\\	せかいで いちばん ながいかわは アマゾンがわです。まるか、ばつか?
\\	せかいで いちばん ながいかわは アマゾンがわです。まるか、ばつか?
\\	ばつ ナイルがわ
\\	ばつ ナイルがわ
\\	いちばん はやいどうぶつは チーター。
\\	いちばん はやいどうぶつは チーター。
\\	では、とりの なかで、なにが いちばんはやいですか。
\\	では、とりの なかで、なにが いちばんはやいですか。
\\	ハヤブサが いちばんはやいです。
\\	ハヤブサが いちばんはやいです。
\\	カナダとちゅうごくと、どちらが おおきいですか。
\\	カナダとちゅうごくと、どちらが おおきいですか。
\\	えっと・・・えっと・・・・ちゅうごくのほうが おおきいです。
\\	えっと・・・えっと・・・・ちゅうごくのほうが おおきいです。
\\	ざんねん
\\	ちゅうごくより カナダのほうが おおきいです。
\\	ざんねん
\\	ちゅうごくより カナダのほうが おおきいです。
\\	ろうじん
\\	かさ
\\	ピンク
\\	おりる
\\	でんしゃ
\\	のる
\\	きのう
\\	みぎ
\\	よる
\\	まがる
\\	かど
\\	わすれもの
\\	えき
\\	わすれる
\\	すみません。わすれものセンターは どこですか?
\\	すみません。わすれものセンターは どこですか?
\\	あのかどを まがってください。みぎに あります。
\\	あのかどを まがってください。みぎに あります。
\\	あの、きのうの よる 10じ ごろ、うえので でんしゃに のって、あかさかで でんしゃを おりました。
\\	あの、きのうの よる 10じ ごろ、うえので でんしゃに のって、あかさかで でんしゃを おりました。
\\	ピンクの かさを わすれました。
\\	ピンクの かさを わすれました。
\\	これですか?
\\	これですか?
\\	あ、そうです。
\\	あ、そうです。
\\	クイズ
\\	すわる
\\	はじめる
\\	てすと
\\	みんな
\\	べんきょうする
\\	じゅく
\\	おもしろい
\\	みる
\\	おれ
\\	きのうの クイズみた?おもしろかったね。
\\	きのうの クイズみた?おもしろかったね。
\\	じゅくで、みられなかった。
\\	じゅくで、みられなかった。
\\	おれも、べんきょうしていたから みられなかった。
\\	おれも、べんきょうしていたから みられなかった。
\\	ええ?みんな、べんきょうしていたの?
\\	ええ?みんな、べんきょうしていたの?
\\	はーい。テストを はじめますので、すわってください。
\\	はーい。テストを はじめますので、すわってください。
\\	てすと
\\	たいへん
\\	おなか
\\	あかい
\\	かお
\\	なんじかん
\\	いつも
\\	あたま
\\	かんたん
\\	もんだい
\\	たおれる
\\	テスト どうだった?
\\	テスト どうだった?
\\	うーん。きょうの テストは もんだいが かんたんだった。
\\	うーん。きょうの テストは もんだいが かんたんだった。
\\	ひろみは、あたまが いいなぁ。
\\	ひろみは、あたまが いいなぁ。
\\	ひろみは、いつも なんじかん べんきょうするの?
\\	ひろみは、いつも なんじかん べんきょうするの?
\\	うーん。あれ?かおが あかいよ。どうしたの?
\\	うーん。あれ?かおが あかいよ。どうしたの?
\\	え?そう?。。。なんだか、おなかが いたいんだよね。
\\	え?そう?。。。なんだか、おなかが いたいんだよね。
\\	さとうせんせい
\\	たいへん
\\	かおるが たおれた。
\\	さとうせんせい
\\	たいへん
\\	かおるが たおれた。
\\	あなた
\\	おとす
\\	ぎん
\\	かさ
\\	きん
\\	ちがう
\\	あげる
\\	しょうじき
\\	ゆめ
\\	あなたがおとしたかさは この、ぎんのかさ ですか。それとも、このきんのかさ ですか?
\\	あなたがおとしたかさは この、ぎんのかさ ですか。それとも、このきんのかさ ですか?
\\	ちがいます。
\\	ちがいます。
\\	わたしのおとしたかさは、ピンクのかさです。
\\	わたしのおとしたかさは、ピンクのかさです。
\\	あなたは、しょうじきですね。
\\	あなたは、しょうじきですね。
\\	では、あなたに、ぎんのかさと  きんのかさを  あげます。
\\	では、あなたに、ぎんのかさと  きんのかさを  あげます。
\\	はっ・・・ゆめ?
\\	はっ・・・ゆめ?
\\	まご
\\	ゆめ
\\	いい/よい
\\	ぼく
\\	つくる
\\	おりがみ
\\	きんいろ
\\	きんこんしき
\\	おじいさん
\\	おばあさん
\\	ぎんいろ
\\	おじいさん、いいゆめをみましたよ。
\\	おじいさん、いいゆめをみましたよ。
\\	ほう。そうですか。
\\	ほう。そうですか。
\\	おじいちゃーん、おばあちゃーん。こんにちは。
\\	おじいちゃーん、おばあちゃーん。こんにちは。
\\	きんこんしき、おめでとう。
\\	きんこんしき、おめでとう。
\\	おやおや、ありがとう。
\\	おやおや、ありがとう。
\\	これをあげる。きんのかさと、ぎんのかさ。
\\	これをあげる。きんのかさと、ぎんのかさ。
\\	おじいちゃんにはきんいろのかさ。おばあちゃんにはぎんいろのかさ。
\\	おじいちゃんにはきんいろのかさ。おばあちゃんにはぎんいろのかさ。
\\	だれがつくったの?
\\	だれがつくったの?
\\	ぼくが、おりがみでつくったんだ。
\\	ぼくが、おりがみでつくったんだ。
\\	びっくりする
\\	りょうしん
\\	いっしょに
\\	おわる
\\	しごと
\\	ひとり
\\	びょうき
\\	モデル
\\	なる
\\	きれい
\\	ごはん
\\	だめ
\\	かいがい
\\	びっくりしたわよ。だめよ。ごはんを たべなさい。
\\	びっくりしたわよ。だめよ。ごはんを たべなさい。
\\	でも、きれいに なりたいんです。モデルに なりたいんです。
\\	でも、きれいに なりたいんです。モデルに なりたいんです。
\\	なにを いっているの。びょうきに なるわよ。
\\	なにを いっているの。びょうきに なるわよ。
\\	いま、ひとりで すんでいるの?
\\	いま、ひとりで すんでいるの?
\\	いえ、あにと ふたりで すんでいます。
\\	いえ、あにと ふたりで すんでいます。
\\	りょうしんは しごとで かいがいに すんでいます。
\\	りょうしんは しごとで かいがいに すんでいます。
\\	わかった。しごと、ごじで おわるから、いっしょに うちで ごはんをたべましょう。
\\	わかった。しごと、ごじで おわるから、いっしょに うちで ごはんをたべましょう。
\\	え?せんせいの うちへ いくんですか?
\\	え?せんせいの うちへ いくんですか?
\\	おにいさんと ふたりで きて。
\\	おにいさんと ふたりで きて。
\\	あに
\\	あとで
\\	てつだう
\\	あおい
\\	とる
\\	おさら
\\	ほそい
\\	うんどうする
\\	もうす
\\	せんせい、あには あとで くると いっていました。
\\	せんせい、あには あとで くると いっていました。
\\	あら そう。
\\	あら そう。
\\	てつだいましょうか。
\\	てつだいましょうか。
\\	じゃ、その、あおいの とって。
\\	じゃ、その、あおいの とって。
\\	このあおいおさら ですか?
\\	このあおいおさら ですか?
\\	そう。ありがとう。
\\	そう。ありがとう。
\\	ほそくなりたいのは わかるけど、たべないのは だめよ。
\\	ほそくなりたいのは わかるけど、たべないのは だめよ。
\\	はい。これからは うんどうしよう とおもいます。
\\	はい。これからは うんどうしよう とおもいます。
\\	はじめまして。かおるの あにの すずきひろしと もうします。
\\	はじめまして。かおるの あにの すずきひろしと もうします。
\\	だいがく
\\	そつぎょう
\\	きめる
\\	だいがくいん
\\	なやむ
\\	なんども
\\	がくひ
\\	たかい
\\	かかる
\\	ひろしさんは、だいがくそつぎょうご、どうするの?
\\	ひろしさんは、だいがくそつぎょうご、どうするの?
\\	まだ なにを するか きめていません。
\\	まだ なにを するか きめていません。
\\	だいがくいんに いこうか どうか、なやんでいます。
\\	だいがくいんに いこうか どうか、なやんでいます。
\\	だいがくいんか・・・。
\\	だいがくいんか・・・。
\\	わたしも なんども だいがくいんにいこう とおもった。
\\	わたしも なんども だいがくいんにいこう とおもった。
\\	でも、がくひが たかいから…。
\\	でも、がくひが たかいから…。
\\	そうですね。いちねんに にひゃくまんえんも かかります。
\\	そうですね。いちねんに にひゃくまんえんも かかります。
\\	こうちょうせんせい
\\	おもう
\\	たぶん
\\	まえ
\\	えき
\\	ばしょ
\\	かんげいかい
\\	なれる
\\	いっしゅうかん
\\	がっこう
\\	このがっこうに きてから いっしゅうかんですね。
\\	このがっこうに きてから いっしゅうかんですね。
\\	どうですか。なれましたか。
\\	どうですか。なれましたか。
\\	はい。
\\	はい。
\\	きょう、しちじから あなたの かんげいかいを しますから、きてくださいね。
\\	きょう、しちじから あなたの かんげいかいを しますから、きてくださいね。
\\	ありがとうございます。
\\	ありがとうございます。
\\	なんじくらいまで ですか。
\\	なんじくらいまで ですか。
\\	たぶん、しちじから くじまでだとおもいますよ。
\\	たぶん、しちじから くじまでだとおもいますよ。
\\	ばしょは、えきのまえのバーイノベーティブです。
\\	ばしょは、えきのまえのバーイノベーティブです。
\\	しゅしゃせんたく
\\	しんきいってん
\\	きどあいらく
\\	ききいっぱつ
\\	たいきばんせい
\\	たんとうちょくにゅう
\\	たにんぎょうぎ
\\	だいたんふてき
\\	ぜったいぜつめい
\\	ぜんだいみもん
\\	ぜんとたなん
\\	たりきほんがん
\\	がっかり
\\	ふりかえる
\\	よこぎる
\\	なにものか
\\	さけぶ
\\	ぐらんど
\\	こうしゃ
\\	ひっぱる
\\	なにもいわず
\\	ごういんに
\\	なかば
\\	つかむ
\\	てくび
\\	そのとき
\\	くび
\\	よっていく
\\	あらわれる
\\	とろうとする
\\	あけようとする
\\	あご
\\	おでこ
\\	かげ
\\	とっさに
\\	かしげる
\\	くるししょうに
\\	のぞきこんでみる
\\	うんてんせきがわのドア
\\	どうやら
\\	たたく
\\	しめる
\\	といれのこしつ
\\	がっこう
\\	おりる
\\	こむ
\\	けっこう
\\	どうろ
\\	やってくる
\\	しんや
\\	きたいできそうだ
\\	びょういん
\\	かわった
\\	たすう
\\	もくげきしゃ
\\	しんれい
\\	たてもの
\\	さがしはじめる
\\	いがいに
\\	しょくいんしつ
\\	りかしつ
\\	きょうしつ
\\	どうろ
\\	おかしなこと
\\	あんぜんうんてん
\\	あしあと
\\	けいかん
\\	けんもん
\\	たたく
\\	ひたすら
\\	しゅんかん
\\	みつめる
\\	しらが
\\	にだい
\\	つぶれる
\\	こわれている
\\	つれていく
\\	まよう
\\	ばば
\\	かえす
\\	つきあう
\\	みがわり
\\	てら
\\	じんじゃ
\\	しまつ する
\\	あずける
\\	だきかかえる
\\	あばれる
\\	だんぼーる
\\	きをうしなう
\\	おはらい
\\	じゅうしょく
\\	いいかげんな
\\	なやむ
\\	さける
\\	よっぱらう
\\	よる
\\	とまる
\\	にらむ
\\	たんす
\\	にほんにんぎょう
\\	むく
\\	きみわるい
\\	なくなる
\\	かむ
\\	あと
\\	たましい
\\	クレープ
\\	なま
\\	ぼく
\\	どれどれ
\\	する
\\	きめる
\\	たべる
\\	におい
\\	クレープや
\\	なまクリーム
\\	おっ、クレープやだ。
\\	おっ、クレープやだ。
\\	すごくいいにおいだわ。ね、クレープたべない?
\\	すごくいいにおいだわ。ね、クレープたべない?
\\	いいねー。たべよう。たべよう。
\\	いいねー。たべよう。たべよう。
\\	なににしようかなあ。どれどれ。
\\	なににしようかなあ。どれどれ。
\\	わたしは、バナナチョコなまクリームにきめたっ
\\	わたしは、バナナチョコなまクリームにきめたっ
\\	ブロマイドしゃしん
\\	だらけ
\\	すごい
\\	アイドル
\\	ゆうめいじん
\\	こういう
\\	こっち
\\	このおみせ、しゃしんだらけだね。
\\	このおみせ、しゃしんだらけだね。
\\	すごいわね。だれのしゃしんかしら。
\\	すごいわね。だれのしゃしんかしら。
\\	うーん。にっぽんのアイドルとかゆうめいじんじゃない。
\\	うーん。にっぽんのアイドルとかゆうめいじんじゃない。
\\	どこもいっしょね。
\\	どこもいっしょね。
\\	しってる?こういうしゃしんをこっちでは、ブロマイドっていう。
\\	しってる?こういうしゃしんをこっちでは、ブロマイドっていう。
\\	へー。
\\	へー。
\\	きっぷ
\\	まつ
\\	すいか
\\	きのう
\\	はじめて
\\	よむ
\\	たしか
\\	かう
\\	きっぷかうから、ちょっとまってて。
\\	きっぷかうから、ちょっとまってて。
\\	えーっ、スイカじゃないのかよ。
\\	えーっ、スイカじゃないのかよ。
\\	スイカ?はあ?
\\	スイカ?はあ?
\\	あーそうか。きのう、はじめてにほんにきたんだっけ。
\\	あーそうか。きのう、はじめてにほんにきたんだっけ。
\\	しんじゅくまでいくら?よめないんだけど。
\\	しんじゅくまでいくら?よめないんだけど。
\\	たしか、130えんだったかなあ。150えんかも。
\\	たしか、130えんだったかなあ。150えんかも。
\\	ふむ
\\	がむ
\\	ちょっと
\\	ティッシュ
\\	はやく
\\	しんごう
\\	あお
\\	わたる
\\	あっ、なんかふんじゃったみたい。
\\	あっ、なんかふんじゃったみたい。
\\	どうしたの?
\\	どうしたの?
\\	あー、ガムふんじゃったよ。
\\	あー、ガムふんじゃったよ。
\\	ちょっとまって。はい、ティッシュ。
\\	ちょっとまって。はい、ティッシュ。
\\	ありがとう。
\\	ありがとう。
\\	はやくはやく。しんごうがあおのうちにわたりましょっ。
\\	はやくはやく。しんごうがあおのうちにわたりましょっ。
\\	タクシー
\\	はやい
\\	たかい
\\	やすい
\\	おそい
\\	でんしゃ
\\	まんいん
\\	いや
\\	はやい
\\	えびすまでどうやっていこうか。
\\	えびすまでどうやっていこうか。
\\	タクシーでいこうか。
\\	タクシーでいこうか。
\\	んん、、、いや、はやいけどたかいでしょ。バスはやすいけど、ちょっとおそいし。
\\	んん、、、いや、はやいけどたかいでしょ。バスはやすいけど、ちょっとおそいし。
\\	じゃあ、やっぱりでんしゃでいこうか。
\\	じゃあ、やっぱりでんしゃでいこうか。
\\	そうだね、やっぱりはやくてやすいしね。まんいんでんしゃのるのいやだけど。。。
\\	そうだね、やっぱりはやくてやすいしね。まんいんでんしゃのるのいやだけど。。。
\\	まちびと
\\	よかん
\\	いやな
\\	しんぱい
\\	つなぐ
\\	けいたい
\\	おくれる
\\	けいか
\\	おそい
\\	きもの
\\	こず
\\	じっか
\\	まちびとこず
\\	まちびとこず
\\	おそいわね。なにしているのかしら。
\\	おそいわね。なにしているのかしら。
\\	(30ぷんけいか)
\\	(30ぷんけいか)
\\	もういちじはんだわ。なにかあったのかしら。おくれるひとじゃないのに。
\\	もういちじはんだわ。なにかあったのかしら。おくれるひとじゃないのに。
\\	おなか
\\	けってい
\\	おおい
\\	りょう
\\	そこ
\\	まえに
\\	ランチ
\\	ちゅうか
\\	おひるごはん
\\	そろそろ
\\	へる
\\	マーボーどうふ
\\	おなかへったね。
\\	おなかへったね。
\\	そろそろ、おひるごはんにしようか。
\\	そろそろ、おひるごはんにしようか。
\\	そこのちゅうかにしない。ランチやってるし。
\\	そこのちゅうかにしない。ランチやってるし。
\\	いいよ。まえにそこでたべたことあるけど、りょうもおおいし、おいしかったよ。
\\	いいよ。まえにそこでたべたことあるけど、りょうもおおいし、おいしかったよ。
\\	じゃあ、けってい。わたしはまーぼうどうふにしよーっと。
\\	じゃあ、けってい。わたしはまーぼうどうふにしよーっと。
\\	やっと
\\	つく
\\	かんばん
\\	ほうめん
\\	でぐち
\\	へいさ
\\	ちゅう
\\	だいじょうぶ
\\	おっちょこちょい
\\	やっとえきについたね。
\\	やっとえきについたね。
\\	めいじじんぐうはどっちのでぐちだろう。
\\	めいじじんぐうはどっちのでぐちだろう。
\\	あそこに、かんばんがあるよ。
\\	あそこに、かんばんがあるよ。
\\	えっと、めいじじんぐうは・・・よよぎほうめんでぐちだって。
\\	えっと、めいじじんぐうは・・・よよぎほうめんでぐちだって。
\\	あれっ、でもいちばんでぐちはへいさちゅうだって。どうしよう・・・。
\\	あれっ、でもいちばんでぐちはへいさちゅうだって。どうしよう・・・。
\\	だいじょうぶだよ。めいじじんぐうはにばんでぐちってかいてあるじゃん。
\\	だいじょうぶだよ。めいじじんぐうはにばんでぐちってかいてあるじゃん。
\\	おれって、いつもおっちょこちょいだよなあ。
\\	おれって、いつもおっちょこちょいだよなあ。
\\	まっ、それがマモルのよいところでしょう。
\\	まっ、それがマモルのよいところでしょう。
\\	しゃしん
\\	こんど
\\	もういっかい
\\	ブス
\\	チーズ
\\	さんきゅー
\\	デジカメ
\\	カメラマン
\\	とる
\\	かんじ
\\	ねえ。せっかくだから、しゃしんとろうか。
\\	ねえ。せっかくだから、しゃしんとろうか。
\\	オーケー。わたしがとってあげる。なんたって、おとうさんカメラマンだからね、デジカメかして。
\\	オーケー。わたしがとってあげる。なんたって、おとうさんカメラマンだからね、デジカメかして。
\\	サンキュー。
\\	サンキュー。
\\	はい、チーズ。とれたよ。どう?
\\	はい、チーズ。とれたよ。どう?
\\	えーっなにこれー。わたしこんなにブスじゃないわよ
\\	えーっなにこれー。わたしこんなにブスじゃないわよ
\\	じゃあ、もういっかい。はい、チーズ。こんどはどう?
\\	じゃあ、もういっかい。はい、チーズ。こんどはどう?
\\	いいかんじじゃない。わたしらしさがすごくでてるわ。さすが、ヒカルね。
\\	いいかんじじゃない。わたしらしさがすごくでてるわ。さすが、ヒカルね。
\\	サンキュー。
\\	サンキュー。
\\	すごい
\\	さかだる
\\	めいがら
\\	ぜんこく
\\	くらもと
\\	やく
\\	しらべる
\\	いや
\\	かぞえる
\\	うわーすごい。なにあれー。
\\	うわーすごい。なにあれー。
\\	なんだろうね。いこう。
\\	なんだろうね。いこう。
\\	うわー、さかだるだあ。すごいかずだね。かぞえよう。
\\	うわー、さかだるだあ。すごいかずだね。かぞえよう。
\\	いち、に、さん、よん、、、、、、ひゃくはちじゅっこもあるよ。スッゲー。
\\	いち、に、さん、よん、、、、、、ひゃくはちじゅっこもあるよ。スッゲー。
\\	なんでここに、こんなにあるんだろう。
\\	なんでここに、こんなにあるんだろう。
\\	なんでだろうね。
\\	なんでだろうね。
\\	でもにほんしゅって、めいがらいくつあるんだろうね。
\\	でもにほんしゅって、めいがらいくつあるんだろうね。
\\	ぜんこくに、やくにせんのくらもと(さけをつくるばしょ)があるらしいよ。だから、めいがらはかぞえられないくらいあるんじゃないかな。じゃあ、マモルしらべてみてよ。
\\	ぜんこくに、やくにせんのくらもと(さけをつくるばしょ)があるらしいよ。だから、めいがらはかぞえられないくらいあるんじゃないかな。じゃあ、マモルしらべてみてよ。
\\	えーっ。いやだよ。
\\	えーっ。いやだよ。
\\	ほんじつ
\\	もとめる
\\	きかい
\\	さいだい
\\	シューズ
\\	にんき
\\	ぜひ
\\	しな
\\	さいしゅうび
\\	セール
\\	たちよる
\\	ほんじつはセールのさいしゅうびでーす。
\\	ほんじつはセールのさいしゅうびでーす。
\\	いいしながたくさんでていますよー。
\\	いいしながたくさんでていますよー。
\\	ぜひ、おたちよりくださいませー。
\\	ぜひ、おたちよりくださいませー。
\\	ほんじつはセールのさいしゅうびでございます。
\\	ほんじつはセールのさいしゅうびでございます。
\\	ほんじつ、にんきのシューズがさいだいごじゅっパーセントオフでございます。
\\	ほんじつ、にんきのシューズがさいだいごじゅっパーセントオフでございます。
\\	このきかいにぜひおもとめくださいませー。
\\	このきかいにぜひおもとめくださいませー。
\\	おいしい
\\	ちょっと
\\	だめ
\\	すこし
\\	ほんとう
\\	くいしんぼう
\\	たべる
\\	おねがい
\\	たべさせる
\\	このクレープおいしいーっ
\\	このクレープおいしいーっ
\\	ちょっと、たべさせてー。
\\	ちょっと、たべさせてー。
\\	えーっ。ダメーっ。
\\	えーっ。ダメーっ。
\\	わたしのも、すこしたべていいから。おねがい
\\	わたしのも、すこしたべていいから。おねがい
\\	じゃあ、ちょっとだけだからね。
\\	じゃあ、ちょっとだけだからね。
\\	わーい。
\\	わーい。
\\	もうっ。アイは、ほんとうにくいしんぼうなんだからーっ
\\	もうっ。アイは、ほんとうにくいしんぼうなんだからーっ
\\	パスタ
\\	たべほうだい
\\	やすい
\\	ろくじゅっぷんかん
\\	おなか
\\	へる
\\	さっき
\\	ちゅうか
\\	ねえ、パスタたべほうだいだって。
\\	ねえ、パスタたべほうだいだって。
\\	いくら?
\\	いくら?
\\	きゅうひゃくはちじゅうえんだって。やすいね。
\\	きゅうひゃくはちじゅうえんだって。やすいね。
\\	じかんは?
\\	じかんは?
\\	えーっと、ろくじゅっぷんかん。たべよっか?
\\	えーっと、ろくじゅっぷんかん。たべよっか?
\\	おなかへってるの?さっき、ちゅうかたべたばっかりじゃん。
\\	おなかへってるの?さっき、ちゅうかたべたばっかりじゃん。
\\	パスタ
\\	だいすき
\\	すごい
\\	ならぶ
\\	くみ
\\	いや
\\	やめる
\\	うん。パスタはだいすきだし、たべたーい。
\\	うん。パスタはだいすきだし、たべたーい。
\\	うわっ、すごくならんでるよ。
\\	うわっ、すごくならんでるよ。
\\	えーっ。どれくらい?
\\	えーっ。どれくらい?
\\	ごくみ、ろっくみくらいはまってるよ。
\\	ごくみ、ろっくみくらいはまってるよ。
\\	まつのはいやね。やっぱり、やめようか。
\\	まつのはいやね。やっぱり、やめようか。
\\	やめましょう。クレープでもたべにいきましょうよ。
\\	やめましょう。クレープでもたべにいきましょうよ。
\\	すみません
\\	ください
\\	なる
\\	あと
\\	パン
\\	ぎゅうにゅう
\\	うけとる
\\	おかえし
\\	どうも
\\	すみません。このパンとあのパンをください。
\\	すみません。このパンとあのパンをください。
\\	はい、にひゃくじゅうえんになります。
\\	はい、にひゃくじゅうえんになります。
\\	あ、あと、ぎゅうにゅうもおねがいします。
\\	あ、あと、ぎゅうにゅうもおねがいします。
\\	ではさんびゃくにじゅうえんになります。(ごひゃくえんうけとる)ひゃくはちじゅうえんのおかえしです。
\\	ではさんびゃくにじゅうえんになります。(ごひゃくえんうけとる)ひゃくはちじゅうえんのおかえしです。
\\	はい、どうも。
\\	はい、どうも。
\\	ありがとうございました。
\\	ありがとうございました。
\\	すみません
\\	しんじゅく
\\	まっすぐ
\\	あるいて
\\	でんしゃ
\\	いく
\\	わかりました
\\	くらい
\\	だいたい
\\	すみません。しんじゅくはどこですか?
\\	すみません。しんじゅくはどこですか?
\\	しんじゅくですか。えっと、このみちをまっすぐいくとしんじゅくにいきますよ。
\\	しんじゅくですか。えっと、このみちをまっすぐいくとしんじゅくにいきますよ。
\\	あるいてどのくらいですか。
\\	あるいてどのくらいですか。
\\	そうですね、だいたい30ぷんくらいです。でんしゃのほうがはやいですよ。
\\	そうですね、だいたい30ぷんくらいです。でんしゃのほうがはやいですよ。
\\	わかりました。ありがとうございます。
\\	わかりました。ありがとうございます。
\\	おとこ
\\	とどく
\\	めんきょしょう
\\	ながい
\\	くろい
\\	おとす
\\	さいふ
\\	どうしました
\\	けいかん
\\	もしかして
\\	すみません
\\	すみません
\\	どうしましたか。
\\	どうしましたか。
\\	さいふをおとしてしまって。。。とどいてないですか。
\\	さいふをおとしてしまって。。。とどいてないですか。
\\	どんなさいふですか。
\\	どんなさいふですか。
\\	くろくて、ながくて、、、いちまんごせんえんぐらいはいってたかな、、、そうだ、なかにめんきょしょうがはいっています
\\	くろくて、ながくて、、、いちまんごせんえんぐらいはいってたかな、、、そうだ、なかにめんきょしょうがはいっています
\\	もしかして、これですか。
\\	もしかして、これですか。
\\	そうです
\\	よかった、ありがとうございます
\\	そうです
\\	よかった、ありがとうございます
\\	すみません
\\	やくざいし
\\	きゃく
\\	くすり
\\	ひどい
\\	はなみず
\\	あと
\\	だるい
\\	あたまがいたい
\\	ねつっぽい
\\	しょうじょう
\\	かぜ
\\	どうしました
\\	やっきょく
\\	すみません。
\\	すみません。
\\	どうしました?
\\	どうしました?
\\	かぜっぽいんですけど。。。
\\	かぜっぽいんですけど。。。
\\	どんなしょうじょうですか。
\\	どんなしょうじょうですか。
\\	ねつっぽくて、あたまがいたくて、だるくて。あと、はなみずがひどいです。
\\	ねつっぽくて、あたまがいたくて、だるくて。あと、はなみずがひどいです。
\\	そうですねぇ。。。このくすりとこのくすりをごはんのあとにのめば、すぐなおりますよ
\\	そうですねぇ。。。このくすりとこのくすりをごはんのあとにのめば、すぐなおりますよ
\\	おそい
\\	そんなに
\\	ナンバープレート
\\	きいろい
\\	しかくい
\\	しろい
\\	たしか
\\	でんわにでない
\\	でんわ
\\	おおきい
\\	あいつ、おそいなー。
\\	あいつ、おそいなー。
\\	でんわしてみた?
\\	でんわしてみた?
\\	してみたけど、でないんだよ。たしか、くるまでくるんだよね。
\\	してみたけど、でないんだよ。たしか、くるまでくるんだよね。
\\	そうそう。くるまは、しろくて、しかくくて、きいろいナンバープレートの。そんなにおおきくないよ。
\\	そうそう。くるまは、しろくて、しかくくて、きいろいナンバープレートの。そんなにおおきくないよ。
\\	あ、あれじゃない?
\\	あ、あれじゃない?
\\	たけしたどおり
\\	ひとのかず
\\	ここ
\\	ゆうめい(な)
\\	ほら
\\	あそこ
\\	なるほど
\\	うわー、すごいひとのかずだなあ。
\\	うわー、すごいひとのかずだなあ。
\\	すごいね。なんでだろうね。
\\	すごいね。なんでだろうね。
\\	わかった
\\	わかった
\\	えっ、なに?
\\	えっ、なに?
\\	ここが、ゆうめいなたけしたどおりだよ。ほらっ、あそこにかいてあるよ。
\\	ここが、ゆうめいなたけしたどおりだよ。ほらっ、あそこにかいてあるよ。
\\	なるほどー。ここなのね。
\\	なるほどー。ここなのね。
\\	だれ
\\	ものしり
\\	おどる
\\	まい
\\	おまもり
\\	たいおう
\\	さんぱい
\\	しごと
\\	はたらく
\\	じんじゃ
\\	くわしく
\\	しらない
\\	みこ
\\	あのひとってなにをするひとだろう。
\\	あのひとってなにをするひとだろう。
\\	えっ。しらないの。みこさんだよ。くわしくはしらないけど、じんじゃではたらいているひとだよ。
\\	えっ。しらないの。みこさんだよ。くわしくはしらないけど、じんじゃではたらいているひとだよ。
\\	どんなしごとしてるんだろうね。
\\	どんなしごとしてるんだろうね。
\\	さんぱいしゃへのたいおうとか、おまもりをあげるとか、まいをおどるとか・・・いろいろじゃないの。
\\	さんぱいしゃへのたいおうとか、おまもりをあげるとか、まいをおどるとか・・・いろいろじゃないの。
\\	さすが、マサル。ものしりだなあ。
\\	さすが、マサル。ものしりだなあ。
\\	まあな。
\\	まあな。
\\	きょう
\\	おちゃ
\\	つかれる
\\	そのまえに
\\	ひさしぶり
\\	じゃあ
\\	せんしゅう
\\	また
\\	カラオケ
\\	さんせい
\\	きょうなにしようか。。。カラオケいかない?
\\	きょうなにしようか。。。カラオケいかない?
\\	また?せんしゅうもいったじゃん。
\\	また?せんしゅうもいったじゃん。
\\	じゃあ、なにしたいんだよ。
\\	じゃあ、なにしたいんだよ。
\\	じゃあ、ひさしぶりにボーリングいこうよ。
\\	じゃあ、ひさしぶりにボーリングいこうよ。
\\	おお、いいね
\\	けどそのまえにつかれたからおちゃしようよ。
\\	おお、いいね
\\	けどそのまえにつかれたからおちゃしようよ。
\\	さんせい
\\	そうしよ。
\\	さんせい
\\	そうしよ。
\\	だめ
\\	エスカレーター
\\	みぎ
\\	ひだりがわ
\\	かんさい
\\	たつ
\\	ひと
\\	くる
\\	うるさい
\\	エスカレーター
\\	エスカレーター
\\	ちょっと、まさと、ダメだよ。
\\	ちょっと、まさと、ダメだよ。
\\	ん?なにがや。
\\	ん?なにがや。
\\	とうきょうでは、エスカレーターのひだりがわにたつの。
\\	とうきょうでは、エスカレーターのひだりがわにたつの。
\\	なにをいうとんねん。かんさいじんはみぎがわにたつんや。
\\	なにをいうとんねん。かんさいじんはみぎがわにたつんや。
\\	ほらっ、ひとがきた。
\\	ほらっ、ひとがきた。
\\	メンズカットサロン
\\	くばる
\\	こまる
\\	ちょうど
\\	かぜをひく
\\	うけとる
\\	わすれる
\\	メンズカットサロン「サブロー」です。よろしくおねがいします。
\\	メンズカットサロン「サブロー」です。よろしくおねがいします。
\\	ティッシュ、ください。
\\	ティッシュ、ください。
\\	(おとこのひとにくばりたいんだけどなぁ。まぁ、いいや。)はい、どうぞ。
\\	(おとこのひとにくばりたいんだけどなぁ。まぁ、いいや。)はい、どうぞ。
\\	ティッシュわすれちゃって、こまっていたの。ちょうどよかったにゃー
\\	ティッシュわすれちゃって、こまっていたの。ちょうどよかったにゃー
\\	あ、はぁ。。(このきせつは、みんなかぜひいているから ティッシュをうけとるのか・・・。)
\\	あ、はぁ。。(このきせつは、みんなかぜひいているから ティッシュをうけとるのか・・・。)
\\	ちょうずや
\\	おまいり
\\	ちゃんと
\\	おきよめ
\\	いれもの
\\	ひしゃく
\\	もちかえる
\\	すすぐ
\\	おかあさん、どうしたらいいのー。
\\	おかあさん、どうしたらいいのー。
\\	おててとおくちをきれいにするのよ。
\\	おててとおくちをきれいにするのよ。
\\	なんでー。
\\	なんでー。
\\	おまいりするまえは、ちゃんとおきよめしないといけないの。
\\	おまいりするまえは、ちゃんとおきよめしないといけないの。
\\	このいれものはー?
\\	このいれものはー?
\\	ひしゃくっていうのよ。まずみぎてにひしゃくをもって、ひだりてをきよめて、それからひだりてにもちかえて、みぎてをきよめるのよ。それからおくちをすすぐの。
\\	ひしゃくっていうのよ。まずみぎてにひしゃくをもって、ひだりてをきよめて、それからひだりてにもちかえて、みぎてをきよめるのよ。それからおくちをすすぐの。
\\	はーい。
\\	はーい。
\\	くずれおちる
\\	つつむ
\\	けはい
\\	まっくら
\\	びょうしつ
\\	そくし
\\	したじき
\\	ちょくげき
\\	さしかかる
\\	こうじげんば
\\	とりもどす
\\	いしき
\\	たれさがる
\\	ひふ
\\	ちかづく
\\	ゆれる
\\	きゅうきゅうしゃ
\\	みまわす
\\	かんりょう
\\	おおけが
\\	じばら
\\	えたいのしれない
\\	げんしょう
\\	びびる
\\	むだがね
\\	なっとくする
\\	たじろぐ
\\	えたいのしれない
\\	かつぎあげる
\\	じょうしゃきょひ
\\	まちあいしつ
\\	せいしんじょうたい
\\	はんそうする
\\	にらむ
\\	おふだ
\\	ぎょうしゃ
\\	ひっこす
\\	つごう
\\	せんこう
\\	におい
\\	くさい
\\	さむけ
\\	ずめん
\\	たんとうしゃ
\\	がんきゅう
\\	すりぬける
\\	れんらくをとる
\\	いちぶしじゅう
\\	じんもんする
\\	はくせん
\\	とつぜん
\\	ひきにく
\\	がんきゅう
\\	ひめい
\\	ぎょうそう
\\	ふりむく
\\	さけぶ
\\	むらがる
\\	いけづくり
\\	たまらない
\\	さめる
\\	あう
\\	うすぐらい
\\	うちがわ
\\	こわい
\\	しずか
\\	しょくば
\\	てっさく
\\	ひめい
\\	けはい
\\	くらやみ
\\	いしゅう
\\	しかく
\\	けいこうとう
\\	ひじょうかいだん
\\	じんしんじこ
\\	かんりがいしゃ
\\	つうち
\\	しょりする
\\	はっちゅう
\\	てつごうし
\end{CJK}
\end{document}