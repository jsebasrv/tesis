\documentclass[8pt]{extreport} 
\usepackage{hyperref}
\usepackage{CJKutf8}
\begin{document}
\begin{CJK}{UTF8}{min}
\\	どうしてたっているんでしょうか
\\	ミルズさんは明日きっとくるでしょ。
\\	せっけんのにおいがします
\\	夜中に外でひそひそ話す声がします。
\\	へんなにおいがします
\\	甘い花のかおりがします
\\	食事会のレストランをよやくしております
\\	ボーナスが出るんですって。
\\	明日は寒いでしょう。
\\	新聞によると、インドネシアでおおきいつなみがあったそうです
\\	なつかしい味がします。
\\	まるで夢を見ているようです
\\	ピアノのおとがします。
\\	あそこにいるのはミルズさんでしょうか
\\	薬のようににがいです
\\	これはだれかの忘れ物のようです
\\	明日会議は2時からだろうです
\\	シカさんの話では。。。
\\	あさってから中国しゅっちょうでしょう。
\\	ミルズさんは少し遅れるそうです。
\\	ミルズさんはスーパーマンのようです
\\	この車、いくらぐらいだろう
\\	雷がおちたようなおとがしました
\\	ミルズさんは休みに国にかえるだろうと思います。
\\	魚がくさったようなにおいがします
\\	レモンの味がします
\\	ミルズさんは明日来ます。
\\	あかちゃんはおなかがすいているようです
\\	ここすうねん、おちています
\\	24時間お金がおろせるようになりました。
\\	おしゃれな服やドレスを着た犬をよくみかけます。
\\	サイン会は作家の急病のために、キャンスルになりました。
\\	小学校の授業でパソコンを使うようになりました。
\\	漢字を勉強したら、道や駅のサインがわかるようになりました。
\\	1週間でひらがなが読めるようになりました。
\\	寝ている人が多いです。
\\	いつでも暖かいべんとが食べるようになりました。
\\	簡単ににもつが送れようになりました。
\\	これからも伸びていくでしょう。
\\	これからは人口がへっていくでしょう。
\\	2000年から少しずつ伸びてきました。
\\	大人も楽しめるゲエムソフトがはやっています。ゲ
\\	マンガを読んでいるサラリマンをよく見かけます。
\\	左側通行なのに、左ハンドルの車がけっこうおおいです。
\\	太って、早く走れなくなりました。
\\	人口がふえてきました。
\\	いろいろな自販機があります。
\\	けしょうをしている若い女の人をよく見かけます
\\	ペットをかっているひとは全体の約3割りです。
\\	動物が好きでも嫌いでもない人は3
\\	です。
\\	動物が好きな人は64
\\	です。
\\	2004年に「ワンワン」をぬきました。
\\	60年だいにビートルズがはやっていました。
\\	子どもたちがよるおそくかいものにいくようになりました。
\\	雪がたくさん降ったせいで、電車がとまって、りょこうにいけませんでした。
\\	24時間買い物ができるようになりました。
\\	今年は雪がたくさん降ったせいで、ながくスキーがたのしめました。
\\	24時間ふりこみができるようになりました。
\\	2003年までのびていました。
\\	夜遅い時間によっぱらったサラリマンをみかけます。
\\	~ことについて、どう思いますか。
\\	仕事のことでそうだんがあるんですが。
\\	~ことについて、ご意見をお願いします。
\\	今名前お書いています。
\\	友達にあって、食事をします。
\\	ひどいことだと思います。
\\	朝ごはんをさべずに、会社に行きます。
\\	何してるの。
\\	そうあったんですか。
\\	名前が聞いてあります。
\\	わからないことは、何でも聞いてください。
\\	何もお話しすることはありません。
\\	難しいもんだいだと思います。
\\	仕事が合って、パーチーにいけませんでした。
\\	わからないのは、どのもんだいですか。
\\	新しいジャケットを着て、会社に行きました。
\\	いいことじゃないでしょうか。
\\	うかがいたいことがあるのですが。
\\	私のしゅみは、しゃしんをとることです。
\\	大学ではどんあことをけんきゅうしているんですか。
\\	お願いしたいことがあるんですが。
\\	~ことを知っていますか。
\\	大切なことだと思います。
\\	~ことにかんしんがあります。
\\	どちらとも言いえません。
\\	サルが温泉に入りことをしっていますか。
\\	お願いしたいことがあるんですが。
\\	きのう、面白いことを聞きました。
\\	サルが温泉に入ってるのを見ました。
\\	ききたいことがあるんですが。
\\	子どもが朝ご飯を食べなくて、困っています。
\\	安全な野菜を作ることは、大切なことだと思います。
\\	ノーコメントです。
\\	どんなことでしょうか。な
\\	日本語が上手になるように、毎日勉強しています。
\\	きのうの午後、そうじをして、せんたくをしました。
\\	今日はしなければないことがたくさんある。
\\	銀行からお金を借りて家を買いました。
\\	よくねむれるように、寝るまえに少しお酒をのみました。
\\	明日、私が空港に着いたとき、荷物はまだ着いていないでしょう。
\\	通訳になれるように、毎日英語をべんきょうしています。
\\	今朝私が空港に着いたとき、荷物はもうついていました。
\\	時間がなかったので、駅まで歩きながら話ました。
\\	もう聞きましたか。
\\	ごぞんじだと思いますが。。
\\	荷物はもう空港に着いています。
\\	しんぎられません。
\\	もう知っていると思いますが。。。
\\	ええ、安くしますよ。
\\	へえ、初めて聞きました。
\\	それは知りませんでした。
\\	ごぞんじかもしれませんが。。。
\\	それは面白いですうね。
\\	この仕事を始めてから、ずっとこのパソコンを使っています。
\\	荷物はまだ空港に着いていません。
\\	きのうの午後、このパソコンを使っていました。
\\	そうなんですか。
\\	本当にそうなんですか。
\\	ごぞんじのように。。。
\\	ステーキを食べるなら、
\\	フーズのステーキソースを使って見てください。
\\	そうじをしたり、せんたくをしたりしていました。
\\	来年イギリスにりゅうがくすることにしました。
\\	ビルの前に車を止めてもいいです。
\\	休みにパリに行って、びじゅつかんをたずねたりしたいです。
\\	明日、雨でもゴルフに行きます。
\\	明日、ひまでも、パーティーには行きません。
\\	毎日働かなくてもいいです。
\\	毎日働かなければなりません。
\\	泳いだり、ヨガをしたりすることができます。
\\	会社をやめることにしましたが、まあ部長に話ていません。
\\	会社をやめることになりました。
\\	ビルの前にくるまをとめてはいけません。
\\	説明を読んでも、よくわかりません。
\\	遠くてもあるいていきます。
\\	会社をやめることにしました。
\\	部屋のかたずけてをてつだってくれて、ありがちおう。
\\	中村さんは鈴木さんに仕事をかわってもらいました。
\\	わざわざ駅までむらえに来てくれて、ありがとう。
\\	パソコンをしゅりしてもらえますか。
\\	友達は私に店の名前をおしえました。
\\	ええ、そのつもりです。
\\	お皿がよごれているので、とりにかえてもらえますか。
\\	サイズを
\\	サイズにとりかえてもらえませんか。
\\	友達がスキーをおしえてくれました。
\\	私は弟のしゅくだいを手つってやりました。
\\	荷物を今日の5じにとどけてもらえませんか。
\\	鈴木さんにレポートのないようをせつめいしてあげてください。
\\	悪いけどちょっと手つだってくれない。
\\	本当はやるつもりだったんですが、なかなか時間がなくて。。。
\\	しんさつの予約の時間をかえてもらいたいんですが。
\\	今日中に荷物をとりにきてもらえませんか。
\\	友達が母を病院につれていってくれました。
\\	もうタバコをすわないつもりです。
\\	ホテルの予約をしておいてくれて、たすかったよ。
\\	今日はランチにさぞってくれて、ありがとう。
\\	うちに帰ったらすぐやるつもりだったんだけどつい忘れちゃって。。。
\\	しけんのけっかを知らせてくれて、ありがとう。
\\	~さんい知らせるつもりだったんですが。。。
\\	話すつもりがったんですが、なかなか言いにくくて。。。つ
\\	今日中にこの仕事をすませるつもりだったんですが。。。
\\	スポーツクラブに行くつもりだったのに、ざんぎょうでいけなくなりました。
\\	にもつをもってくれて、ありがとう。
\\	早く来るつもりだったんだけど、電車に乗り遅れちゃって。。。
\\	自分でやるつもりだったんですが。。。
\\	昨日は仕事をかわってもらって、たすかりました。
\\	私のためにパーチーを開いてくれて、本当にありがとう。
\\	早く起きるつもりだったんですが。。。
\\	今電車に乗っているので、おりたらこちらから電話します。
\\	手に入りにくいチケットをとってもって、すみませんでした。
\\	友達は私に店の名前をおしえました。
\\	私は田中さんにフランス語を教えてくれました。
\\	もっと早く準備をするつもりだったんですが、なかなか時間がなくて。。。
\\	私の好き料理を作ってくれて、ありがとう。
\\	無理を言って予約をかえてもらって、すみません。
\\	あかんぼうに泣かれて、ぜんぜんねられなかった。
\\	私はむすこにパソコンをこわされました。
\\	雨にふられました。
\\	ちゅうしゃいはんでつかまった。
\\	どりょうに仕事のじゃまをされた。
\\	子どもがねているとき、買い物にいってきました。
\\	犬にかまれた。
\\	京都でこくさい会議がひらかれました。
\\	知らない人に写真を取られました。
\\	鈴木さんは男の人におされました。
\\	友達にいじめられた。
\\	日本語が上手だといわれた。
\\	母に大事なものをすてられた。
\\	妻にしなれて、困っている。
\\	こうつうじこにあった。
\\	どろぼうにはいれた。
\\	子どもにカメラをこわされた。
\\	友達にだまされた。
\\	さぎにあった。
\\	だれかに車をぶつけられた。
\\	すりにさいふをすられた。
\\	ゆうかいされた。
\\	じょうしにしかれた。
\\	このお寺は1000年前に建てられました。
\\	どろぼうだと思われた。
\\	みんなにわらわれた。
\\	ねむろうとしましたがねむれませんでした。
\\	ハムレットはシェイクスピアによって書かれました。
\\	ハッカーにデータを盗まれた。
\\	悪口を言われた。
\\	知らない男になぐられた。
\\	雨にふられて、びしょぬれになった。
\\	僕にてつだわせてください。
\\	うちの前に車をとめないでほしいんですが。。。
\\	私は子どもにもっと勉強してほしいです。
\\	お金をたくさん使わないでほしい。
\\	なやみを何でも話してほし。
\\	お酒をあまり飲まないでほしい。
\\	家をきれいにそうじしてほしい。
\\	中野さんは無農薬でりんごが作れると思います。
\\	子度もがゲームソフトをほしがっていたのでたんじょう日にかってやりました。
\\	むすこはサッカーせんしゅうになりたがっています。
\\	車がほしいです。
\\	子どものとき、私は母にピアノを習わせられました。
\\	せいとがさわいで、先生をこまらせました。
\\	子どものとき、よく病気になって両親をしんぱいされました。
\\	私は子どもを外国に行かせました。
\\	母はおとうとに部屋をそうじさせました。
\\	私にこの仕事をやらせて下さい。
\\	私は子どもに外国に行かせました。
\\	お
\\	ます
\\	します。
\\	お
\\	ます
\\	になります。
\\	とおしゃいます。
\\	でございます。
\\	でいらしゃいます。
\\	でございます。
\\	ごめいわくをおかけして、もうしわけありません。
\\	お待たせして、もうしわけありません。
\\	遅くなって、もうしわけありません。
\\	さそっていただいたのに、うかがえなくてもうしわけありません。
\\	家内にきれいな花をありがとうございます。
\\	おいしいワインをありがとうございます。
\\	黒田じょうむでいらしゃいます。
\\	フーズのかとうでございます。
\\	けんとうしております。
\\	おいそがしいところ。
\\	伝言をおねがいしてもよろしいでしょうか。
\\	部長、今日は何時に帰られますか。
\\	部長は明日アメリカにいかれます。
\\	7時よ。おきなさい!
\\	部長に明日までにレポーを書けと言われました。
\\	もっと勉強しろ。
\\	電話してくださいと伝えてください。
\\	中村さん、ミルズさんにレポートを出すように伝えてください
\\	鈴木さん、ミルズさんに会議は3じからだとつたえてください。
\\	中村さんがけっこんするんだって。
\\	ミルズさんはカナダ人だって。
\\	かれは何時に来るって。
\\	かれはどこにいるって。
\\	明日、おひまですか。
\\	ミルズさんに後から行くと伝えてください。
\\	部長にお電話をいただきたいとお伝えください。
\\	6時に帰るって、お母さんに伝えて。
\\	悪いけど今日は行けなくなったってヒロにいっといて。
\\	ミルズさんに私に電話するように伝えてください。
\\	鈴木さん、ミルズさんにレポートを書くようにいってください。
\\	連絡するようにいわれています。
\\	会議室にご案内するように言われています。
\\	妻に飲みすぎるなといわれた。
\\	夫に長電話をするなといわれた。
\\	何て書いてあるんですか。
\\	ここに車を止めるなって書いてあります。
\\	危険なので、この中に入るなって書いてあります。
\\	車に気をつけて。
\\	あやれ。
\\	ばかにするな。
\end{CJK}
\end{document}