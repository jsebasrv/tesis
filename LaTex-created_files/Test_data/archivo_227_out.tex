\documentclass[8pt]{extreport} 
\usepackage{hyperref}
\usepackage{CJKutf8}
\begin{document}
\begin{CJK}{UTF8}{min}
\\	彼らには、けがに対する補償を受ける権利がある	
\\	インフルエンザの流行で、現在までに200人もの人が亡くなっている。	
\\	それらの錠剤の効果は強烈だが持続性はない。	
\\	彼は爪を噛む癖がある。あれには本当にぞっとするよ。	
\\	うちの息子を抱こうとしてかがんだときに、おばあちゃんは腰を痛めた。	
\\	肩が凝るなら、薬草治療を試してごらんよ。	
\\	医学の飛躍的発展は人類全体に多大な恩恵をもたらしてきた。	
\\	クローン技術は人間にはもちろん、牛にも応用してはいけない。絶対にしてはならないことだ。	
\\	クジラは哺乳類に分類される。	
\\	進化論は私の想像力の範囲を超えている。	
\\	その生物学者は自分の歴史的発見に誇りを持っており、そのことを臆することなく口に出す。	
\\	倫理的な理由から、彼らはいわゆる遺伝子医療に反対している。	
\\	その病気の初期症状は高熱と喉の痛みです。	
\\	脳の構造は複雑だ。	
\\	病気のために年次会議を出席した代表者もいた。	
\\	私の娘婿は徐々に胃癌を克服しつつあり、今は明るく元気だ。	
\\	「これって偽物だろう?」 「何言ってるの。本物の骨董品よ。」 「まさか!」	
\\	「ほかにご注文は?」 「それだけです。」 「こちらでお召し上がりですか?」 「持ち帰ります。」	
\\	「この敷物はおいくらですか?」 「税込みで100ドルです。」 「じゃあ、これにします。」	
\\	そのパンツはあなたには似合わないと思う。こっちをはいてみて。今すごく流行っているのよ。	
\\	彼は「おつりはいいです」と言って、急いでタクシーを降りた。	
\\	ショッピングセンターをぶらぶらしていたら、偶然ケンに会った。	
\\	現金が足りなかったので、先週口座に入れた100ドルを引き出した。	
\\	彼女は帰宅途中にハンドバッグを奪われた。	
\\	「それは変だわ。どうしてそんなことが起きたのか教えて。」 「後で説明するよ。」	
\\	私が約束を破ったので、罰として、両親に門限を6時にされた。	
\\	諺にある通り、「嘘も方便(目的が手段を正当化する。)」	
\\	誰も頼ってくれる人がいないとわかったのだから、自分に力でやっていくしかない。	
\\	議長は私の話をさえぎって、「話し合いはここまでにして決を採りましょう。」と言った。	
\\	我々は会議で率直な意見を交わしたが、この件に関してはまだ全体の意見がまとまらない。	
\\	「あなたは彼と同じ意見?それとも違うの?」 「彼のほうを支持するよ。」	
\\	厳密に言えば、彼の見解は私のものといくらか異なる。	
\\	もう一度説明していただけませんか?お話の意図がわかりませんでした。	
\\	僕の案を支持してくれる人は最初はだれもいなかったけれど、最後にはみんなを説得して賛同してもらうことだできた。	
\\	「君の不屈の努力、勇気、そして知恵には感謝するよ。」 「お世辞でもうれしいわ!」	
\\	どうするかを決めるとき、僕は論理ではなく本能を当てにする。	
\\	彼は独創的かつ懸命な解決策を考えだし、直ちにそれを実行に移した。	
\\	客観的に見れば、前者は後者よりも劣っている。	
\\	あなたの考えは理論上すばらしそうだけれど、実際には機能しないと思います。	
\\	私たちは彼が相手に勝つことを期待したが彼はその期待に応えられなかった。	
\\	皮肉なことだが、彼らの最善の努力にもかからわず、任務は完全な失敗に終わった。	
\\	実地経験の不足は彼女の天賦の才能で補える。	
\\	彼女にはどんな障害をも克服できる素晴らしい能力がある。	
\\	あなたの強固な支援がなかったら、その取引は成立しなかったでしょう。感謝しています。	
\\	機内への可燃物の持ち込みは禁止されています。	
\\	ちょっとした事故が原因で、私たちのウィーン行きの便に遅れが出た。	
\\	飛行機が爆発して海に墜落し、乗っていた全員が死亡した。	
\\	彼らは休むことなく、その墜落事故の原因を調査した。	
\\	不幸にも、その大惨事で生き残った乗客はほとんどいなかった。	
\\	航空評論家が統計を詳細に分析した。	
\\	全従業員を代表いたしまして、犠牲者の方々に対してお悔やみ申し上げたいと存じます。	
\\	ある意味、その惨事の責任は彼にある。	
\\	車両は欠陥がないか3か月に1度検査される。	
\\	うちの光熱費がそんな額になっているのを知ってびっくりした。	
\\	「水道の栓を閉めなさい!」とお母さんがかんかんになって叫んだ。	
\\	食べ物が腐らないように蓋をしっかり閉めなさい。	
\\	「ハサミ貸してもらえる?」 「もちろん。はいどうぞ。」	
\\	針と糸の代わりに透明な接着剤を使っても結構です。	
\\	この織物を染めたら縮んでしまった。	
\\	この革のジャケットは処分するよ。両肘の部分がすり減って穴があいているから。	
\\	その混ぜ合わせたものに小麦粉を加えて、とろみがつくまでかき混ぜましょう	
\\	「残り物をどうしよう?」 「とりあえず冷蔵庫に入れておいて。後で電子レンジで温め直すよ。」	
\\	新しいパソコンに全額つぎ込むこともできたが、考え直して、まさかの時のために多少は残しておくことにした。	
\\	携帯電話はちょっとした必需品になったことだし、それなしではやっていけない。	
\\	イスにゆったりと座ってくつろいだ途端、妻に雑用を頼まれた。	
\\	彼女は赤ちゃんを寝かせて自分も隣に横になった。	
\\	ここにあるものを片付けなさい。邪魔よ。	
\\	子供たちが恐竜ごっこをしていたら、お母さんに叱られた。	
\\	彼が棚にぶつかり、とても高価な陶磁器が落ちて粉々に割れてしまった。	
\\	「うちの子たちをちょっと見ててもらえる?ケンを迎えに行かないといけないの。」 「いいよ。」	
\\	隣の家の人が庭をめちゃくちゃにされたと、私たちにがみがみ説教した。	
\\	ドリーはあらゆる面でお姉さんにそっくりだ。二人を見分けることはできないよ。	
\\	概して、双子は似たところが多い。	
\\	皿洗いとか洗濯とか、そういうことはもううんざり!	
\\	時間が空いた時の一番の楽しみは浜辺の散歩です。気持ちいいですよ。	
\\	「私、もうくたくた。日陰で休憩しましょう。」 「そうだね。」	
\\	「ボブ、この自販機、故障してるわ。」 「何だって!のどが渇いて死にそうだよ!」	
\\	彼は空腹と疲労で気を失ったが、しばらくして意識を取り戻した。	
\\	たった一つの家具を除けば、講堂はがらんどうだった。	
\\	家賃が安ければ、どんなマンションでも構いません。こだわりはありませんから。	
\\	彼の60歳のいとこが湖畔の不動産を相続した。	
\\	雑談はこの辺にして、本題に入りましょう。	
\\	ルーシーは、契約書にサインさせようとする強引な販売員に腹を立てた。	
\\	私たちの株式投資への意欲をそぐような要因がかなりある。	
\\	日本は様々な原料を輸入し、工業製品を輸出している。	
\\	経済発展の過程で、日本はその伝統的価値観を捨ててきた。	
\\	世界はかつてない危機に直面している。	
\\	要するに、その規制の目的は国内産業の保護だ。	
\\	世界での競争力を維持するためには、日本は大幅な規制緩和に真剣に着手しなければならない。	
\end{CJK}
\end{document}