\documentclass[8pt]{extreport} 
\usepackage{hyperref}
\usepackage{CJKutf8}
\begin{document}
\begin{CJK}{UTF8}{min}
\\	側[かわ, がわ]	
\\	〔方面〕 
\\	〔相対する二つの一方〕 
\\	〔取り囲むもの〕 
\\	(覆い) 
\\	(時計の) 
\\	わたしの 右側[みぎがわ]。	
\\	側	かわ, がわ	側	~がわ	
\\	お祖父さん・お爺さん[おじいさん]	
\\	【お祖父さん】 〔祖父〕 
\\	《口》 
\\	《口》 
\\	〔呼びかけ〕 
\\	【お爺さん】 〔男の老人〕 
\\	昔[むかし]々あるところにおじいさんと お婆[おばあ]さんが 住[す]んでおりました。	
\\	お祖父さん・お爺さん	おじいさん	おじいさん	おじいさん	
\\	暇[ひま]	
\\	〔自由に使える時間〕 
\\	〔合間〕 
\\	~な 〔することのない〕 
\\	〔手の空いた〕 
\\	〔要する時間〕 
\\	〔(商売の)閑散〕 ~な 
\\	〔休暇〕 
\\	(官吏などの) 
\\	〔解雇〕 
\\	お 暇[ひま]なら。	
\\	《文》 
\\	暇	ひま	暇	ひま	
\\	回[かい]	
\\	〔回数〕 
\\	〔競技の〕 
\\	【野球】 
\\	【クリケット】 
\\	【スヌーカーなど】 
\\	【音楽】 
\\	【解】 
\\	回[かい]で 成功[せいこう]する。	
\\	《口》 
\\	回	かい	回	〜かい	
\\	階[かい]	
\\	〔建物の層〕 
\\	【地質】 〔地質時代のそれぞれの期 
\\	に堆積した地層〕 
\\	【数】 〔微分方程式の〕 
\\	〔テンソルの〕 
\\	〔行列・群の〕 
\\	階建[かいだ]てのビル。	
\\	階	かい	階	〜かい	
\\	か月[かげつ]	
\\	か月	かげつ	か月	〜かげつ	
\\	個[こ]	
\\	つ〕 
\\	人〕 
\\	ナシ3 個[こ]。	
\\	個	こ	個	〜こ	
\\	語[ご]	〔単語〕 
\\	〈総称〉 
\\	〔専門語〕 
\\	〔語法〕 
\\	〔言語〕 
\\	[⇒ことば]	一語一語[いちごいちご] 訳[やく]す。	
\\	語	ご	語	〜ご	
\\	-歳[-さい]	
\\	彼[かれ]は30 歳[さい]にはまだなるまい。	
\\	-歳	-さい	歳	〜さい	
\\	-冊[-さつ]	(巻) 
\\	(部) 
\\	それはあと4 冊[さつ]で 完結[かんけつ]する。	
\\	-冊	-さつ	冊	〜さつ	
\\	-さん[-さん]	
\\	[⇒-さま]	山田[やまだ]さん。	
\\	-さん	-さん	さん	〜さん	
\\	時[じ]	〔時間〕 
\\	〔時刻〕 
\\	時間[じかん] 制[せい]で〕21 時[じ]。	
\\	と読む. 
\\	時間制では常に 
\\	を用いる, たとえば零時 
\\	は 
\\	時 
\\	分 
\\	は 
\\	と読む.	時	じ	時	〜じ	
\\	時間[じかん]	
\\	〔ある時刻と時刻の間〕 
\\	分〕 
\\	〔時刻; (決められた)特定の時間〕 
\\	〔学校の授業時間〕 
\\	〔物理学的に見た時間〕 
\\	時間[じかん]がまだ 余[あま]っている。	
\\	時間	じかん	時間	〜じかん	
\\	週間[しゅうかん]	
\\	[⇒しゅう7]	
\\	週間[しゅうかん]にわたって。	
\\	週間	しゅうかん	週間	〜しゅうかん	
\\	-人[-じん]	
\\	〔ひと〕 
\\	〔一定の分野で働く人〕 
\\	沖縄[おきなわ] 人[じん]。	
\\	-人	-じん	人	〜じん	
\\	-過ぎ[-すぎ]	
\\	〔時〕 
\\	〔度合い〕 
\\	過[す]ぎまで 生[い]きる。	
\\	-過ぎ	-すぎ	過ぎ	〜すぎ	
\\	-ずつ[-ずつ]	
\\	半額[はんがく]ずつ 納入[のうにゅう]する。	
\\	-ずつ	-ずつ	ずつ	〜ずつ	
\\	-台[-だい]	
\\	〔数値・数量のだいたいの範囲〕 
\\	[⇒-だい2] 
\\	〔機械や車両の数量〕 ⇒だいすう2.	スピードスケートの 男子[だんし]500メートルで35 秒[びょう] 台[だい]を 出[だ]した。	
\\	-台	-だい	台	〜だい	
\\	-達[-たち]	
\\	そういう 人[ひと]たち。	
\\	-達	-たち	達	〜たち	
\\	-中[-ちゅう]	
\\	〔…の中〕 
\\	〔範囲内〕 
\\	〔なかでも〕 
\\	〔…の間(に)〕 
\\	〔進行中〕 
\\	人[にん] 中[ちゅう]の 一[いち] 人[にん]に 選[えら]ばれる。	
\\	-中	-ちゅう	中	〜ちゅう	
\\	度[ど]	
\\	〔角度・経緯度〕 
\\	〔温度〕 
\\	〔(レンズの) 湾曲度〕 
\\	〔音程〕 
\\	〔アルコール分〕 
\\	➡英米ではアルコール分を 
\\	の割合としてあらわすこともある. 
\\	が米国では 
\\	英国では 
\\	〔回数〕 
\\	[⇒いちど 
\\	にど] 
\\	〔程度〕 
\\	〔限度〕 
\\	〔心の平静〕 
\\	度[ど]の 角[かく]。	
\\	度	ど	度	〜ど	
\\	等[など]	
\\	(略: 
\\	ロンドンなどでは。	
\\	等	など	など	〜など	
\\	日[にち]	
\\	わずか23 日[にち]で。	
\\	日	にち	日	〜にち	
\\	年[ねん]	
\\	年〕 
\\	〔年季〕 
\\	〔年号による年次〕 
\\	〔経過した年数〕 
\\	〔学年度〕 
\\	〔学年〕 
\\	(小・中・高) 
\\	(中・高) 
\\	➡米国の 
\\	は小中高を通しての数え方である. したがって高校 
\\	年生は 
\\	となる.	年[とし] 一[いち] 年[ねん]と。	
\\	年	ねん	年	〜ねん	
\\	-杯[-はい]	〔コップやスプーンに入れたものの量〕 
\\	〔タコや船などの数え方〕 
\\	茶[ちゃ]を2 杯[はい]。	
\\	-杯	-はい	はい	〜はい	
\\	半[はん]	
\\	〔半分〕 
\\	〔なかば〕 
\\	[⇒はん-2] 
\\	〔奇数〕 
\\	時間[じかん] 半[はん]。	
\\	半	はん	半	〜はん	
\\	-番[-ばん]	
\\	〔識別番号〕 
\\	〔物の順番や大きさなどを表す番号〕 
\\	〔順位〕 
\\	〔勝負などの回数〕 
\\	〔曲・歌詞の数〕	右[みぎ]から3 番目[ばんめ]の 男[おとこ]。	
\\	-番	-ばん	番	〜ばん	
\\	-匹[-ひき]	
\\	〔(小)動物など〕 
\\	〔布地 
\\	反〕 
\\	水槽[すいそう]の 中[なか]に20 匹[ひき]ほどの 熱帯魚[ねったいぎょ]が 泳[およ]いでいた。	
\\	-匹	-ひき	匹	〜ひき	
\\	分[ふん]	
\\	〔時・度の〕 
\\	〔衡量〕 
\\	分[ふん]。	
\\	分	ふん	分	〜ふん	
\\	-枚[-まい]	
\\	〔ページ〕 
\\	皿[さら]5 枚[まい]。	
\\	-枚	-まい	枚	〜まい	
\\	前[まえ]	
\\	〔前部〕 
\\	〔前方の位置〕 
\\	〔以前〕 
\\	〔ある事態より早い時〕 
\\	〔前科〕 
\\	*《口》 
\\	《口》 
\\	〔貴婦人の名に添える敬称〕 
\\	前[まえ]から 友人[ゆうじん]が 走[はし]ってきた。	
\\	前	まえ	前	〜まえ	
\\	家・屋[や]	
\\	〔住家〕 
\\	〔商店〕 
\\	〔飲食店〕 
\\	〔商人〕 
\\	中華料理[ちゅうかりょうり] 屋[や]。	
\\	家・屋	や	屋	〜や	
\\	-本[-ほん]	
\\	白墨[はくぼく]5 本[ほん]。	
\\	-本	-ほん	本	~ほん	
\\	ああ[ああ]	
\\	〔感動したときなどの発声〕 
\\	《文》 
\\	〔呼びかけ〕 
\\	ああ 思い出[おもいだ]した。	
\\	ああ	ああ	ああ	ああ	
\\	会う・逢う・遭う[あう]	
\\	〔約束して対面する・面談する〕 
\\	〔偶然出会う〕 
\\	《口》 
\\	《文》 
\\	*《口》 
\\	〔好ましくない事態に遭遇する〕 
\\	会[あ]ったことのない 人[ひと]。	
\\	会う・逢う・遭う	あう	会う	あう	
\\	青い[あおい]	
\\	〔青色の〕 
\\	〔緑の〕 
\\	〔青白い〕 
\\	〔未熟な〕 
\\	〔経験不足な〕 
\\	[⇒あおにさい]	ホウレン草[ほうれんそう]とか 青[あお]いものも 食[た]べなさい。	
\\	青い	あおい	青い	あおい	
\\	赤い[あかい]	
\\	〔色〕 
\\	(深紅) 
\\	(緋) 
\\	〔共産主義の〕 
\\	西[にし]の 空[そら]は 夕日[ゆうひ]で 赤[あか]かった。	
\\	赤い	あかい	赤い	あかい	
\\	明るい[あかるい]	
\\	〔暗いに対して〕 
\\	〔明朗だ〕 
\\	〔公明な〕 
\\	〔物事に通じている〕 
\\	《文》 
\\	《文》 
\\	この 電球[でんきゅう]は 明[あか]るい。	
\\	明るい	あかるい	明るい	あかるい	
\\	秋[あき]	
\\	秋[あき]も 深[ふか]まりました。	
\\	秋	あき	秋	あき	
\\	開く・明く・空く[あく]	
\\	〔閉じていたものがひらく〕 
\\	〔始まる〕 
\\	〔営業などが続く〕 
\\	〔空間ができる〕 
\\	〔からになる・欠員が生じる・用済みになる〕 〈物が主語〉 
\\	〈物・人が主語〉 
\\	現在[げんざい]80
\\	の 票[ひょう]が 開[ひら]いています。	
\\	開く・明く・空く	あく	開く	あく	
\\	開ける・明ける・空ける[あける]	
\\	〔とじていたものをひらく〕 
\\	〔解きあける〕 
\\	〔空間を作る・あきを設ける〕 
\\	〔中身をだして空にする〕 
\\	〔未使用の状態にする〕 (場所を) 
\\	(時間を) 
\\	(地位を) 
\\	私[わたし]はわくわくしながら 包[つつ]みを 開[あ]けた。	
\\	開ける・明ける・空ける	あける	開ける	あける	
\\	上げる・挙げる[あげる]	
\\	〔上方へ移動させる〕 
\\	〔程度を高くする〕 
\\	〔示す〕 
\\	〔自分の気持ちや考えなどを表す手段として声を出す〕 
\\	〔儀式などを取り行う〕 
\\	〔よい結果を生み出す〕 
\\	〔終える・仕上げる・済ます〕 
\\	〔能力など発揮する〕 
\\	[⇒あげて] 
\\	〔潮が満ちてくる〕 
\\	〔与える・やる・渡す〕 
\\	〔神仏に供える〕 
\\	〔食べた物を戻す〕 ⇒もどす, はく5. 
\\	〔他のためにする動作のていねい表現〕	
\\	重量挙[じゅうりょうあ]げで〕 彼[かれ]は 合計[ごうけい]340キロ 挙[あ]げた。	
\\	上げる・挙げる	あげる	あげる	あげる	
\\	朝[あさ]	
\\	〔夜明け前後しばらくの間〕 
\\	《詩》 
\\	〔朝食〕 
\\	約束[やくそく]の 朝[あさ]が 来[き]た。	
\\	朝	あさ	朝	あさ	
\\	朝ご飯[あさごはん]	
\\	ちゃんと 朝ご飯[あさごはん]を 食[た]べて 行[い]きなさい。	
\\	朝ご飯	あさごはん	朝ご飯	あさごはん	
\\	明後日[あさって]	
\\	明日[あした]でもあさってでもあなたの 都合[つごう]のいい 日[ひ]にいらっしゃい。	
\\	明後日	あさって	あさって	あさって	
\\	足・脚[あし]	
\\	〔足首から先〕 
\\	足[あし]が 疲[つか]れたときは 足[あし]を 高[たか]くして 休[やす]むといい。	
\\	足・脚	あし	足	あし	
\\	明日[あした]	
\\	〔今日の次の日〕 
\\	[=あす] 
\\	〔将来〕 
\\	[=あす]	あしたが 日曜[にちよう]だったらいいのにな。	
\\	明日	あした	明日	あした	
\\	あそこ[あそこ]	
\\	〔向こうに見えるあの場所〕 
\\	〔既知のあの場所・組織・集団など〕 
\\	〔あのような程度・あれほど〕	あそこに{で}。	
\\	あそこ	あそこ	あそこ	あそこ	
\\	遊ぶ[あそぶ]	
\\	〔愉快なことをして楽しく過ごす〕 
\\	〔なんの束縛もなく自由に過ごす〕 
\\	〔仕事など有意義なことを何もしないで過ごす〕 
\\	(定職につかないで) 
\\	(職場などでまともに仕事をせずに) 
\\	〔遊興する〕 
\\	〔遊学する〕 
\\	〔有効に利用されずにある〕 
\\	[⇒あそばせる 
\\	〔もてあそぶ〕	明日[あした] 来[き]て 僕[ぼく]らと 一[いち] 日[にち] 遊[あそ]ばないか。	
\\	遊ぶ	あそぶ	遊ぶ	あそぶ	
\\	暖かい・温かい[あたたかい]	
\\	〔気温や物の温度が適度に高い〕 
\\	〔心のこもった・思いやりのある・友好的な〕 
\\	《文》 
\\	〔色が暖色系である〕 
\\	〔経済状態がゆたかだ〕	暖[あたた]かい 地方[ちほう]ではもう 桜[さくら]が 咲[さ]いている。	
\\	暖かい・温かい	あたたかい	温かい	あたたかい	
\\	頭[あたま]	
\\	〔頭部〕 
\\	〔脳天〕 
\\	【解】 
\\	〔頭髪〕 
\\	〔髪型〕 
\\	〔頭の働き・頭脳〕 
\\	〔知力〕 
\\	〔考慮〕 
\\	〔考え〕 
\\	〔見方〕 
\\	〔首領〕 
\\	〔最初〕 
\\	〔上方〕 
\\	〔先端〕 
\\	〔人数・メンバー〕 ⇒あたまかず.	彼[かれ]は 頭[あたま]のうしろに 手[て]を 組[く]んでソファーに 寝転[ねころ]びあくびをした。	
\\	頭	あたま	頭	あたま	
\\	新しい[あたらしい]	
\\	〔新奇な〕 
\\	〔新鮮な〕 
\\	〔最近の〕 
\\	〔現代的〕 
\\	新[あたら]しい 魚[さかな]。	
\\	新しい	あたらしい	新しい	あたらしい	
\\	あちら[あちら]	
\\	〔あれ〕 
\\	〔向こうのほう〕 
\\	《文》 
\\	[⇒あっち] 
\\	〔外国〕 
\\	〔あの人〕 
\\	〈複数〉 
\\	〔相手・先方〕 
\\	あちらが 新製品[しんせいひん]でございます。	
\\	あちら	あちら	あちら	あちら	
\\	厚い[あつい]	
\\	〔物の表裏・上端下端のへだたりが大きい〕 
\\	〔(比喩的に)幅が広い〕 
\\	〔心入れの程度が大きい〕 
\\	〔手厚い〕 
\\	《文》 
\\	〔囲碁で〕 
\\	オリンピック 出場[しゅつじょう]までには 厚[あつ]い 壁[かべ]がある。	
\\	厚い	あつい	厚い	あつい	
\\	暑い[あつい]	
\\	今[いま]は 暑い盛[あついさか]りだ。	
\\	暑い	あつい	暑い	あつい	
\\	後[あと]	
\\	〔後方〕 
\\	[⇒うしろ] 
\\	〔これ[それ]以後〕 
\\	〔次〕 
\\	〔今後〕 
\\	〔最後〕 
\\	〔結果〕 
\\	〔結末〕 
\\	〔残り〕 
\\	〔なごり〕 
\\	〔子孫〕 
\\	《文》 
\\	〔後継者〕 
\\	[⇒あと2 
\\	〔後妻〕 
\\	〔追加〕 
\\	〔(接続詞的に)それから〕 
\\	半年後[はんとしご] 妻[つま]も 病[や]んで 夫[おっと]の 後[のち]を 追[お]った。	
\\	後	あと	後	あと	
\\	貴方[あなた]	
\\	〔呼びかけ〕 
\\	(妻が夫に) 
\\	(若夫婦など) 
\\	あなたたち
\\	方[かた]}。	
\\	貴方	あなた	貴方	あなた	
\\	兄[あに]	
\\	➡年長であることを特に表現しない最後の言い方が英語としては普通.	実[み]の 兄[あに]。	
\\	兄	あに	兄	あに	
\\	姉[あね]	
\\	➡年長であることを特に表現しない最後の言い方が英語としては普通.	実[み]の 姉[あね]。	
\\	姉	あね	姉	あね	
\\	あの[あの]	
\\	あの人[あのひと]。	
\\	あの	あの	あの	あの	
\\	アパート[アパート]	〔建物〕 
\\	〔名前の一部として〕 
\\	〔部屋〕 
\\	アパートの 管理人[かんりにん]。	
\\	アパート	アパート	アパート	アパート	
\\	浴びる[あびる]	
\\	〔かぶる〕 
\\	〔浸る〕 
\\	〔光などを〕 
\\	〔非難などを〕 
\\	返り血[かえりち]を 浴[あ]びて。	
\\	浴びる	あびる	浴びる	あびる	
\\	危ない[あぶない]	
\\	〔危険な〕 
\\	《口》 
\\	〔警告として〕 
\\	〔生命・容態などが〕 
\\	〔負けそう〕 
\\	〔疑わしい〕 
\\	〔心もとない〕 
\\	〔当てにならない〕 
\\	〔しっかりしない〕 
\\	[⇒あやしい 
\\	〔間一髪の〕 
\\	〔(人物が)正常ではない; 通例アブナイと表記〕 
\\	その 仕事[しごと]は 少[すこ]しも 危[あぶ]ないことはない。	
\\	危ない	あぶない	危ない	あぶない	
\\	甘い[あまい]	
\\	〔甘味の〕 
\\	〔塩加減が〕 
\\	[《文》 
\\	〔うっとりと快い〕 
\\	〔おだてるような・誠意のない〕 
\\	《文》 
\\	《文》 
\\	〔言いくるめるような〕 
\\	〔魅力的な〕 
\\	〔甘やかす〕 
\\	〔厳しくない〕 
\\	〔浅薄な〕 
\\	〔のん気な〕 
\\	〔与(くみ)しやすい〕 
\\	〔(野球などで)打ちやすい〕 
\\	〔鋭くない〕 
\\	〔(栓などが)ゆるい〕 
\\	【相場】 
\\	サトウキビは 甘[あま]い。	
\\	甘い	あまい	甘い	あまい	
\\	余り[あまり]	
\\	〔残余〕 
\\	〔余分〕 
\\	〔残り物〕 
\\	〔残高〕 
\\	〔余地〕 
\\	〔普通の程度を越えた状態〕 
\\	〔副詞〕 
\\	〔否定〕 
\\	余[あま]りの。	
\\	〔過剰の〕 
\\	〔余分の〕 
\\	【数】 
\\	余り	あまり	あまり	あまり	
\\	雨[あめ]	
\\	回の降雨〕 
\\	(通り雨) 
\\	雨続[あめつづ]き。	
\\	雨	あめ	雨	あめ	
\\	洗う[あらう]	
\\	〔水などで汚れを落とす〕 
\\	《文》 
\\	〔水などが物にかかる〕 
\\	〔調べ上げる〕 
\\	洗[あら]ってない 皿[さら]。	
\\	洗う	あらう	洗う	あらう	
\\	有る・在る[ある]	
\\	〔存在する〕 
\\	〔…に存する〕 
\\	〔…による〕 
\\	〔位置する〕 
\\	(山・建物などが) 
\\	(都市・国などが) 
\\	(道・川が) 
\\	〔持っている〕 
\\	《文》 
\\	(よいものを) 
\\	(悪いものを) 
\\	〔授かっている〕 
\\	〔付属してある〕 
\\	〔設備がある〕 
\\	〔含まれている〕 
\\	〔数量がある〕 
\\	〔売っている・手に入る〕 
\\	〔経験がある〕 
\\	〔起こる〕 
\\	《文》 
\\	〔開かれる・行われる〕 
\\	(会議などが) 
\\	〔…と書かれている〕 
\\	〔…であるので〕 
\\	〔動作・作用の結果が続いている; 準備されている〕	その 修正案[しゅうせいあん]には 改善[かいぜん]の 余地[よち]がある。	
\\	有る・在る	ある	有る	ある	
\\	有る・在る[ある]	
\\	〔存在する〕 
\\	〔…に存する〕 
\\	〔…による〕 
\\	〔位置する〕 
\\	(山・建物などが) 
\\	(都市・国などが) 
\\	(道・川が) 
\\	〔持っている〕 
\\	《文》 
\\	(よいものを) 
\\	(悪いものを) 
\\	〔授かっている〕 
\\	〔付属してある〕 
\\	〔設備がある〕 
\\	〔含まれている〕 
\\	〔数量がある〕 
\\	〔売っている・手に入る〕 
\\	〔経験がある〕 
\\	〔起こる〕 
\\	《文》 
\\	〔開かれる・行われる〕 
\\	(会議などが) 
\\	〔…と書かれている〕 
\\	〔…であるので〕 
\\	〔動作・作用の結果が続いている; 準備されている〕	あるいはそんなことがあったかもしれない。	
\\	有る・在る	ある	ある	ある	
\\	歩く[あるく]	
\\	《口》 
\\	【野球】 
\\	ペンギンは 歩[ある]くのはへただ。	
\\	歩く	あるく	歩く	あるく	
\\	あれ[あれ]	
\\	〔離れた所にあるもの・人を指して〕 
\\	〔妻・子供など親しい人・目下の者を指して〕 
\\	〔聞き手も知っている時・人・事・物・場所を指して〕 
\\	〔言いにくいことを漠然と指して〕	あれもこれもと 欲張[よくば]って 買い込[かいこ]んでしまった。	
\\	あれ	あれ	あれ	あれ	
\\	良い[いい]	
\\	〔悪・誤・劣・醜・凶に対して善・正・優・美・吉〕 (上等の) 
\\	(優秀な) 
\\	(良好な) 
\\	(見事な) 
\\	(正しい) 
\\	(上手な) 
\\	(美しい) 
\\	(社会的に上位の) 
\\	(裕福な) 
\\	(値が高い) 
\\	(親しい) 
\\	(吉の) 
\\	(幸運な) 
\\	[⇒よい3 
\\	〔適当〕 (適当な) 
\\	(効果的な) 
\\	(好適な) 
\\	(快適な) 
\\	(好都合な) 
\\	(有益な) 
\\	(望ましい) 
\\	(好ましい) 
\\	[⇒よい3 
\\	〔(「…でいい」などの形で) 許容・満足〕 [⇒よい3 
\\	〔(「…していい」などの形で) 賛成・承認・許可〕 
\\	〔(「もういい」などの形で) 十分・不要の意を示して〕 
\\	〔(「…だといい・すればいい・した方がいい」などの形で) 願望・勧誘〕 [⇒よい3 
\\	〔(「…していいな」などの形で) 賛嘆・羨望〕 
\\	〔(「…よりいい・…の方がいい」などの形で) 比較的まし〕 [⇒よい3 
\\	〔反語〕 
\\	〔念押し〕 ⇒いいか, いいかい, いいこと 
\\	いいね2. 
\\	〔(動詞について) …しやすい〕 
\\	[⇒よい3 
\\	非常[ひじょう]にいい。	
\\	良い	いい	良い	いい 
\\	よい	
\\	いいえ[いいえ]	〔答が否定のとき〕 
\\	〔答が肯定のとき〕 
\\	「これは 失礼[しつれい]しました」「いいえどういたしまして」。	
\\	いいえ	いいえ	いいえ	いいえ	
\\	言う[いう]	
\\	〔口をきく〕 
\\	〔口に出す〕 
\\	〔言葉で表す〕 
\\	〔告げる〕 
\\	〔主張する〕 
\\	〔断言する〕 
\\	〔説明する〕 
\\	〔評する〕 
\\	〔うわさする〕 
\\	〔報道する〕 
\\	〔称する〕 
\\	〔…と呼ぶ〕 
\\	〔動物などが声を出す〕 
\\	〔物が音をたてる〕 
\\	この 子[こ]はまだ 物[もの]が 言[い]えない。	
\\	言う	いう	言う	いう	
\\	家[いえ]	
\\	〔家屋〕 
\\	〔住宅〕 
\\	〔自宅〕 
\\	[⇒うち2] 
\\	〔家庭〕 
\\	〔所帯〕 
\\	〔家名〕 
\\	〔家督〕 
\\	〔家業〕 
\\	〔家柄〕 
\\	〔家系〕 
\\	〔由緒ある家柄〕 
\\	〔親族団体〕 
\\	〔家制度での一団〕 
\\	東京[とうきょう]の 家[いえ]をたたんで 郷里[きょうり]に 戻[もど]った。	
\\	家	いえ	家	いえ	
\\	行く[いく]	
\\	〔こちらから向こうへ移動する〕 
\\	《文》 
\\	《文》 
\\	〔去る〕 
\\	〔こちらから向こうへ届く〕 
\\	〔出かける・訪問する・出席する〕 
\\	(相手のいるところへ) 
\\	〔学校などに入る・通う〕 
\\	〔嫁になる〕 
\\	〔養子になる〕 
\\	〔時が過ぎる〕 
\\	〔数量が達する〕 
\\	〔物事が進む〕 
\\	〔気持ちが進む〕 
\\	〔物事をする〕 
\\	〔セックスでオルガスムスに達する〕 
\\	〔現実から遊離している〕 
\\	〔(「…ていく」の形で)物事の継続・進行を表す〕 
\\	〔(「…にいく」の形で) …しようとする・物事を志す〕	お願[おねが]い 行[い]かないで。	
\\	[《文》 
\\	行く	いく	行く	いく	
\\	幾つ[いくつ]	
\\	〔疑問〕 
\\	〔不定の数〕 
\\	〔何歳〕 
\\	リンゴはもういくつも 残[のこ]っていない。	
\\	幾つ	いくつ	いくつ	いくつ	
\\	幾ら[いくら]	
\\	〔不定の値段・数量〕 
\\	〔値段・数量についての疑問〕 
\\	〔(否定文で)わずか・少し〕 
\\	〔(「~でも」「~も」の形で肯定文で)たくさん・限りなく〕 
\\	〔たとえどんなに〕 
\\	いくらでも。	〔数・量・金額について〕 
\\	幾ら	いくら	いくら	いくら	
\\	池[いけ]	
\\	(小さな) 
\\	(庭園・公園などの) 
\\	(川・湖水などに通じる) 
\\	〔貯水池〕 
\\	池[いけ]の 魚[さかな]を 獲[え]ってはいけません。	
\\	〔掲示で〕 
\\	池	いけ	池	いけ	
\\	医者[いしゃ]	
\\	〔開業医〕 
\\	《口》 
\\	医者[いしゃ]に 診[み]てもらう{かかる}。	
\\	医者	いしゃ	医者	いしゃ	
\\	椅子[いす]	
\\	〔腰掛け〕 
\\	(背もたれのない) 
\\	(ピアノの) 
\\	(オーケストラ奏者の) 
\\	(ベンチ) 
\\	〔地位〕 
\\	[=ちい2]	椅子[いす]を 並[なら]べる。	
\\	椅子	いす	椅子	いす	
\\	忙しい[いそがしい]	
\\	君[きみ]はいつも 忙[いそが]しそうだ。	
\\	忙しい	いそがしい	忙しい	いそがしい	
\\	痛い[いたい]	
\\	〔痛む〕 
\\	[=いたむ2 
\\	〔つらい〕 
\\	〔困る〕 
\\	腹[はら]が 痛[いた]いと 言[い]う。	
\\	痛い	いたい	痛い	いたい	
\\	一[いち]	
\\	つ〕 
\\	〔全体の中の 
\\	つ〕 
\\	〔ローマ数字〕 
\\	【数】 
\\	〔最初〕 
\\	〔最も〕 
\\	〔すぐれた・重要な〕 
\\	〔或る〕 
\\	〔単なる・つまらない〕 
\\	〔さいころやトランプの〕 
\\	体育[たいいく]の 先生[せんせい]は
\\	一[いち] 二[に] 一[いち] 二[に]」と 掛け声[かけごえ]をかけた。	
\\	一	いち	一	いち	
\\	一月[いちがつ]	
\\	(略: 
\\	その 祭[まつ]りは 毎年[まいとし] 1月[いちがつ]の 第[だい]2 週[しゅう]に 行[おこな]われる。	
\\	一月	いちがつ	一月	いちがつ	
\\	一日[いちにち]	
\\	〔日数〕 
\\	〔終日〕 
\\	〔ある日〕 
\\	〔ついたち〕 
\\	一[いち] 日[にち]3 回[かい] 毎食[まいしょく] 後[ご]2 錠[じょう]ずつお 飲[の]みください。	
\\	一日	いちにち	一日	いちにち	
\\	一番[いちばん]	
\\	〔第一番〕 
\\	〔(勝負の)1 回〕 
\\	〔(謡曲などの)1 曲〕 
\\	〔もっとも〕 
\\	〔試しに〕 ⇒ためし (試しに).	一番[いちばん]に 現場[げんば]に 駆[か]けつける。	
\\	一番	いちばん	一番	いちばん	
\\	何時[いつ]	
\\	[⇒いつなんどき]	彼女[かのじょ]が 入院[にゅういん]したって?それいつの 話[はなし]?	
\\	何時	いつ	いつ	いつ	
\\	五日[いつか]	〔日付〕 (書き言葉で) 
\\	(話し言葉で) 
\\	〔日数〕 
\\	彼[かれ]は 五[ご] 日[にち]の 午後[ごご]に 到着[とうちゃく]する。	
\\	五日	いつか	五日	いつか	
\\	一緒[いっしょ]	
\\	〔ひとまとめ〕 
\\	〔同一視〕 
\\	〔無差別〕 
\\	[⇒いっしょくた] 
\\	〔同一〕 ⇒おなじ 
\\	〔共に〕 
\\	〔同じ時〕 ⇒どうじ3 
\\	一緒[いっしょ]になる。	〔ひとまとまりになる〕 
\\	〔結婚する〕 
\\	一緒	いっしょ	一緒	いっしょ	
\\	五つ[いつつ]	〔数〕 
\\	〔年齢〕 
\\	あの 亜弓[あゆみ]って 子[こ]、まだ13 歳[さい]だけど5つのときから 子役[こやく]として 映画[えいが]や 舞台[ぶたい]に 出演[しゅつえん]。	
\\	五つ	いつつ	五つ	いつつ	
\\	何時も[いつも]	
\\	〔常に〕 
\\	(平常) 
\\	(常習的に) 
\\	(必ず) 
\\	(絶えず) 
\\	(いつでも) 
\\	〈否定の構文で〉 
\\	[⇒いつでも] 
\\	〔普段〕	いつも 言[い]ってるでしょう 食[た]べる 前[まえ]には 手[て]を 洗[あら]いなさいって。	
\\	何時も	いつも	いつも	いつも	
\\	今[いま]	
\\	〔現在〕 
\\	(この瞬間) 
\\	(目下) 
\\	[⇒いまごろ 
\\	〔現代〕 
\\	〔当節〕 
\\	〔いましがた〕 
\\	〔さっき〕 
\\	〔直ちに〕 
\\	〔じきに〕 
\\	〔もうすぐ〕 
\\	〔さらに〕 
\\	〔もう〕 
\\	〔別の〕 
\\	[⇒いまひとつ] 
\\	〔この場で〕 
\\	今[いま]の 今[いま]まで。	
\\	今	いま	今	いま	
\\	意味[いみ]	
\\	〔表す内容〕 
\\	(単語や文の) 
\\	(記号の) 
\\	(態度や行為の) 
\\	(事実の) 
\\	〔主旨〕 
\\	《文》 
\\	(条文などの) 《文》 
\\	〔含み〕 
\\	~する (単語や文が) 
\\	(記号が) 
\\	(態度・行為や事実が) 
\\	〔含意する〕 
\\	〔価値〕 
\\	《文》 
\\	〔重要性〕 
\\	彼[かれ]の 言[い]っている 意味[いみ]がさっぱりわからない。	
\\	意味	いみ	意味	いみ	
\\	妹[いもうと]	
\\	➡年少であることを特に表現せず単に 
\\	と言うのが英米では普通.	いちばん 下[か]の 妹[いもうと]。	
\\	妹	いもうと	妹	いもうと	
\\	嫌[いや]	
\\	〔欲しない〕 
\\	〔気が進まない〕 
\\	〔許容できない〕 
\\	〔気に入らない〕 
\\	〔好かない〕 
\\	[⇒いやでも, いやに] 
\\	~な 〔不快な〕 
\\	〔不審な〕 
\\	〔変な〕 
\\	〔(困惑して)迷惑な〕 
\\	〔心外な〕	勉強[べんきょう]がいやで 大学[だいがく]を 中退[ちゅうたい]するのではない。	
\\	嫌	いや	いや	いや	
\\	入口[いりぐち]	
\\	入口	いりぐち	入口	いりぐち	
\\	要る[いる]	〔必要である〕 〈人が主語〉 
\\	〈物が主語〉 
\\	要[い]らない。	〔不必要な〕 
\\	〔不用になった〕 
\\	[⇒いらぬ].	要る	いる	いる	いる	
\\	入れる[いれる]	
\\	〔物を外から中へ移す〕 
\\	〔詰める〕 
\\	〔注入する〕 
\\	〔つっこむ〕 
\\	【電算】 
\\	〔中に移したままにする〕 
\\	〔収納する・しまう〕 
\\	〔収録する〕 
\\	〔はめこむ〕 
\\	〔挿入する〕 
\\	〔送り込む〕 
\\	〔加える〕 
\\	〔添加する〕 
\\	〔書き込む〕 
\\	〔導入する〕 
\\	〔仕入れる〕 
\\	〔納入する〕 
\\	〔入金する〕 
\\	〔ある場所に人や物を導き入れる; 通す〕 
\\	〔仲間に加える〕 
\\	〔所属させる〕 
\\	〔雇用する〕 
\\	〔学校や病院などに入らせる〕 
\\	〔収容する〕 
\\	〔間や途中に差し挟む〕 
\\	〔意思を伝える〕 
\\	〔修正を加える〕 
\\	〔気持ちや力を加える〕 
\\	〔含める〕 
\\	〔聞き入れる〕 
\\	〔受け入れる〕 
\\	〔応じる〕 
\\	〔従う〕 
\\	〔点数をつける〕 
\\	〔得点する〕 
\\	〔投票する〕 
\\	〔お茶などを淹れる〕 
\\	〔機械や道具を作動させる〕 
\\	ポケットに 手[て]を 入[い]れて 歩[ある]くな。	
\\	入れる	いれる	入れる	いれる	
\\	色[いろ]	
\\	〔色彩〕 
\\	〔色合い〕 
\\	〔肌の色〕 
\\	〔顔つき・表情〕 
\\	〔そぶり・態度〕 
\\	〔おもむき・様子〕 
\\	〔気配・きざし〕 
\\	〔風情〕 
\\	〔特色・カラー〕 
\\	〔調子〕 
\\	[⇒ねいろ, こわいろ] 
\\	〔情事〕 
\\	〔情欲〕 
\\	〔情人〕 
\\	〔密通の相手〕 
\\	〔容色〕 
\\	〔種類〕 
\\	【物】 〔クォークの〕 
\\	色[いろ]が 変[か]わったり 嫌[いや]な 匂[にお]いがしたりするようになった 食品[しょくひん]は 食[た]べないほうがいい。	
\\	色	いろ	色	いろ	
\\	色々[いろいろ]	~な 〔種々な〕 
\\	いろいろな 点[てん]。	
\\	色々	いろいろ	色々	いろいろ	
\\	上[うえ]	
\\	〔上部〕 
\\	〔表面〕 
\\	〔頂上; ものの上部〕 
\\	〔上の階〕 
\\	〔比較〕 
\\	〔身分・地位など〕 
\\	〔…に関しては〕 
\\	[=-じょう1] 
\\	〔さらにまた〕 
\\	〔…の後〕 
\\	〔…の結果〕 
\\	〔…した時〕 
\\	〔主人・貴人〕 
\\	上[うえ]に 述[の]べた 理由[りゆう]によって。	
\\	上	うえ	上	うえ	
\\	後ろ[うしろ]	
\\	[=あと1 
\\	後[うし]ろの。	
\\	後ろ	うしろ	後ろ	うしろ	
\\	薄い[うすい]	
\\	〔厚みが少ない〕 
\\	〔濃度が低い〕 (液体・味などが) 
\\	(印字・印画が) 
\\	〔淡い〕 
\\	〔少ない〕 
\\	くちびるが 薄[うす]い。	
\\	薄い	うすい	薄い	うすい	
\\	歌[うた]	
\\	〔歌謡〕 
\\	〔歌うこと〕 
\\	〔詩歌〕 
\\	〈総称〉 
\\	〔和歌〕 
\\	〔歌詞〕 
\\	歌[うた]を 歌[うた]う。	
\\	歌	うた	歌	うた	
\\	歌う[うたう]	
\\	〔節をつけて発声する〕 
\\	〔詩歌を作る〕 
\\	〔詩歌を朗読する〕 
\\	小声[こごえ]で 歌[うた]う。	
\\	歌う	うたう	歌う	うたう	
\\	内[うち]	
\\	〔ある空間の範囲内〕 
\\	[⇒うちがわ, なか1] 
\\	〔心の奥〕 
\\	〔本心〕 
\\	〔内実〕 
\\	〔自分の所属団体〕 
\\	(わが社) 
\\	[⇒うち2] 
\\	〔ある数量の範囲内〕 [⇒なか1] 
\\	〔ある部類の範囲内〕 
\\	〔ある時間・状況の範囲内〕 
\\	〔代筆の妻の署名〕	観客[かんきゃく]が 去[さ]った 後[のち]も 熱気[ねっき]が 劇場[げきじょう]の 内[うち]にこもっているようだった。	
\\	内	うち	内	うち	
\\	生まれる・産まれる[うまれる]	
\\	〔生じる〕 
\\	お前[おまえ]はこの 病院[びょういん]で 生[う]まれたんだよ。	
\\	生まれる・産まれる	うまれる	生まれる	うまれる	
\\	海[うみ]	
\\	〔海洋〕 
\\	〔海域〕 
\\	〔一面の広がり〕 
\\	【天】 〔月面の〕 
\\	〔硯(すずり)の〕 
\\	海[うみ]に 向[む]かって 海[うみ]のほうへ。	
\\	海	うみ	海	うみ	
\\	売る[うる]	
\\	〔販売する〕 
\\	〔(所持品など)売り払う〕 
\\	〔広める〕 
\\	〔裏切る・そむく〕 
\\	[=うらぎる] 
\\	〔しかける〕	損[そん]してまでも 売[う]ることはない。	
\\	売る	うる	売る	うる	
\\	上着[うわぎ]	
\\	〔婦人・子供服のゆるい〕 
\\	〔詰襟の軍服の〕 
\\	上着[うわぎ]を 着[き]て。	
\\	上着	うわぎ	上着	うわぎ	
\\	絵[え]	〔絵画〕 
\\	(彩色画) 
\\	(写生画・下画) 
\\	(挿絵) 
\\	(カット・挿絵) 
\\	(版画) 
\\	〔画法〕 
\\	娘[むすめ]の 姿[すがた]を 絵[え]に 写[うつ]す。	
\\	絵	え	絵	え	
\\	映画[えいが]	
\\	《口》 
\\	〈総称〉 
\\	巻[かん]ものの 映画[えいが]。	
\\	映画	えいが	映画	えいが	
\\	映画館[えいがかん]	
\\	銀行[ぎんこう]の 窓口[まどぐち]の 人[ひと]は 映画館[えいがかん]で 見[み]た 人[ひと]だと 分[わ]かった。	
\\	映画館	えいがかん	映画館	えいがかん	
\\	英語[えいご]	
\\	〔単語〕 
\\	私[わたし]は 英語[えいご]が 少[すこ]しもわからない。	
\\	英語	えいご	英語	えいご	
\\	ええ[ええ]	〔肯定〕 
\\	《口》 
\\	〔言いよどみ・ためらい〕 
\\	[⇒ええっ, ええと]	ええ 間違[まちが]いなく 行[い]きます。	
\\	ええ	ええ	ええ	ええ	
\\	駅[えき]	
\\	〔鉄道の〕 
\\	〔宿場〕 
\\	次[つぎ]の 次[つぎ]の 駅[えき]。	
\\	駅	えき	駅	えき	
\\	エレベータ[エレベータ]	
\\	エレベータ	エレベータ	エレベータ	エレベータ	
\\	円[えん]	
\\	〔輪・円形〕 
\\	(小型の) 
\\	〔日本の通貨単位〕 
\\	(記号 
\\	円[えん]の 面積[めんせき]はπ
\\	2で 表[あらわ]される。	
\\	は 
\\	と読む.	円	えん	円	えん	
\\	鉛筆[えんぴつ]	
\\	➡鉛筆の芯は, 柔らかいほうから 
\\	軟[やわ]らかい
\\	硬[かた]い} 鉛筆[えんぴつ]。	
\\	鉛筆	えんぴつ	鉛筆	えんぴつ	
\\	御[お]	
\\	すみません、お 名前[なまえ]が 聞き取[ききと]れませんでした。	
\\	御	お	御	お	
\\	美味しい[おいしい]	
\\	おいしい 食事[しょくじ]と 十分[じゅうぶん]な 睡眠[すいみん]が 風邪[かぜ]をなくしてくれる。	
\\	美味しい	おいしい	美味しい	おいしい	
\\	大きい, 大きな[おおきい, おおきな]	
\\	〔形が〕 
\\	〔容積が〕 
\\	〔面積が〕 
\\	〔数量が〕 
\\	〔音量が〕 
\\	〔程度・規模が〕 
\\	(ひどい) 
\\	〔成長した〕 
\\	〔年長の〕 
\\	〔重要[重大]な意味がある〕 
\\	〔誇大だ〕 
\\	〔尊大だ〕	昨日[きのう]の 火事[かじ]では 焼失[しょうしつ] 面積[めんせき]が 大[おお]きかった 割[わり]に 死傷者[ししょうしゃ]は 少[すく]なかった。	
\\	大きい, 大きな	おおきい, おおきな	大きい	おおきい	
\\	大勢[おおぜい]	
\\	〔群集〕 
\\	〔客などの〕 
\\	大勢[たいせい]で。	
\\	〔軍勢などが〕 
\\	大勢	おおぜい	おおぜい	おおぜい	
\\	お母さん[おかあさん]	
\\	お母[おかあ]さんアヒル。	
\\	お母さん	おかあさん	お母さん	おかあさん	
\\	お菓子[おかし]	
\\	彼[かれ]は 一[いち] 箱[はこ]の お菓子[おかし]を 友達[ともだち] 全員[ぜんいん]と 分け合[わけあ]った。	
\\	お菓子	おかし	お菓子	おかし	
\\	お金[おかね]	=かね1 
\\	子供[こども]が 遊[あそ]びに 使[つか]う)おもちゃの お金[おかね]。	
\\	お金	おかね	お金	おかね	
\\	起きる[おきる]	
\\	〔起床する〕 
\\	《文》 
\\	《口》 
\\	〔病床を離れる〕 
\\	〔目覚める〕 
\\	《文》 
\\	〔眠らずにいる〕 
\\	〔体を起こす〕 
\\	(倒れた人が) 
\\	(傾いていた船などが) 
\\	(倒れ伏していた植物が) 
\\	〔発生する〕 (事故・症状などが) 
\\	(気持ちが) 
\\	《文》 
\\	病気[びょうき]の) 母[はは]はいつになったら 起[お]きられますか」
\\	間[ま]もなくですよ」。	
\\	起きる	おきる	起きる	おきる	
\\	置く[おく]	
\\	〔物を据える〕 
\\	《口》 
\\	《口》 
\\	(寝かせて) 
\\	〔人をある立場に位置させる〕 
\\	〔人をある状況に据える〕 
\\	〔自分の籍を所属させる〕 
\\	〔心に定める〕 
\\	〔設置する〕 (設備を) 
\\	(機関を) 
\\	〔配置する〕 (人を) 
\\	(軍隊を) 
\\	〔雇う〕 
\\	〔残してそこを離れる〕 
\\	〔同居させる〕 
\\	〔保管する〕 
\\	〔放置する; 放任する; そのままの状態を続けさせる〕 
\\	〔抵当にする〕 
\\	〔商品として仕入れておく〕 
\\	[⇒あつかう 
\\	〔隔てる〕 (空間を) 
\\	(時間を) 
\\	[⇒-おき] 
\\	〔(箔(はく)や蒔絵(まきえ)を)ほどこす〕 
\\	〔(霜や露が)生じる〕 
\\	〔あらかじめやる〕 
\\	〔当座の処置をする〕 
\\	【碁】	金庫[きんこ]はいつも 寝室[しんしつ]に 置[お]いてある。	
\\	置く	おく	置く	おく	
\\	奥さん[おくさん]	
\\	[⇒おくさま]	前[まえ]の 奥[おく]さん。	
\\	〔亡くなった〕 
\\	〔離婚した〕 
\\	奥さん	おくさん	奥さん	おくさん	
\\	送る[おくる]	
\\	〔物や情報などを届ける〕 
\\	〔発送する〕 
\\	(商品を) 
\\	(船・車で) 
\\	(送信する) 
\\	(送金する) 
\\	〔人を派遣する・送り込む〕 
\\	〔人を送り出す・見送る〕 
\\	〔死者を見送る・死別する〕 
\\	〔去る[行く]人にある所まで付き添って行く〕 
\\	〔月日を過ごす〕 
\\	〔位置を次へ移す・順に受け渡す〕 
\\	〔送り仮名を付ける〕 
\\	彼[かれ]は 子供[こども]たちが 成人[せいじん]するまで 養育費[よういくひ]を 滞[とどこお]りなく 送[おく]ることを 約束[やくそく]した。	
\\	送る	おくる	送る	おくる	
\\	お酒[おさけ]	
\\	俺[おれ]、 甘党[あまとう]だけど、お 酒[さけ]も 飲[の]むよ。	
\\	お酒	おさけ	お酒	おさけ	
\\	お皿[おさら]	
\\	私[わたし]は お皿[おさら]を 洗[あら]ってからテレビを 見[み]た。	
\\	お皿	おさら	お皿	おさら	
\\	伯父さん・叔父さん・小父さん[おじさん]	
\\	〔伯父・叔父〕 
\\	《小児語》 
\\	〔年長の男性〕 
\\	和夫[かずお]おじさん。	
\\	伯父さん・叔父さん・小父さん	おじさん	伯父さん	おじさん	
\\	押す[おす]	
\\	〔前方などに力を加える・力を加えて動かす〕 
\\	(頭や角で) 
\\	(ひじなどでちょっと押す) 
\\	〔指などで上から押しつける・機械のボタンなどを操作する〕 
\\	〔圧迫する〕 
\\	(強く) 
\\	〔ミカンなどを汁を搾るために〕 
\\	〔空気などを圧縮する〕 
\\	〔印を押す〕 
\\	《文》 
\\	〔箔(はく)を押す〕 
\\	〔優勢である〕 
\\	〔相手に対し強引に働きかける・無理やり押し通す〕 
\\	〔無理をする・我慢する〕 
\\	〔確かめる〕	私[わたし]は 後[うし]ろから 押[お]されて 前[まえ]につんのめった。	
\\	押す	おす	押す	おす	
\\	遅い[おそい]	
\\	〔時刻・時期が〕 
\\	(遅刻して) 
\\	(遅まきの) 
\\	〔速度・テンポ・反応が〕 
\\	例年[れいねん]よりも 桜[さくら]の 開花[かいか]が 少[すこ]し 遅[おそ]い。	
\\	遅い	おそい	遅い	おそい	
\\	お茶[おちゃ]	
\\	〔飲みもの〕 
\\	(緑茶) 
\\	〔休憩〕 
\\	[⇒ちゃ 
\\	〔茶の湯〕 
\\	お茶[おちゃ]のいれ 方[かた]がうまい。	
\\	お茶	おちゃ	お茶	おちゃ	
\\	お手洗い[おてあらい]	〔便所〕 
\\	《口》 
\\	(公共の建物の) 
\\	お手洗[おてあら]いに 行[い]く。	
\\	お手洗い	おてあらい	お手洗い	おてあらい	
\\	お父さん[おとうさん]	
\\	《口》 
\\	子[こ]が 呼[よ]びかけて〕 お父[おとう]さん。	
\\	お父さん	おとうさん	お父さん	おとうさん	
\\	弟[おとうと]	
\\	一番上[いちばんうえ]の 弟[おとうと]。	
\\	弟	おとうと	弟	おとうと	
\\	男[おとこ]	
\\	〔男子〕 
\\	《口》 
\\	《口》 
\\	〔一人前の〕 
\\	〔男子の面目〕 
\\	〔男としての評価〕 
\\	〔情夫〕 
\\	おれはあいつみたいな 男[おとこ]が 大嫌[だいきら]いなんだ。	
\\	[《口》 
\\	男	おとこ	男	おとこ	
\\	男の子[おとこのこ]	
\\	《文・古》 
\\	〔男の赤ん坊〕 
\\	男の子[おとこのこ]らしいいたずら。	
\\	男の子	おとこのこ	男の子	おとこのこ	
\\	一昨日[おととい]	
\\	彼女[かのじょ]はおとといここに 来[き]たばかりだよ。	
\\	一昨日	おととい	一昨日	おととい	
\\	一昨年[おととし]	
\\	おととしの 夏[なつ]。	
\\	一昨年	おととし	一昨年	おととし	
\\	大人[おとな]	
\\	〔成人〕 
\\	〔幼児語で, おとなしいさま〕	大人[おとな]になりかけ(てい)る。	
\\	大人	おとな	大人	おとな	
\\	お腹[おなか]	
\\	おなかが 痛[いた]いんです。 胃腸薬[いちょうやく]をください。	
\\	お腹	おなか	お腹	おなか	
\\	同じ[おなじ]	
\\	〔ひとつのものである〕 
\\	(寸分たがわぬ) 《文》 
\\	〔(複数のものが)互いに等しい〕 
\\	(特に数値が) 
\\	(特に型などが) 
\\	〔(複数の考え方・条件などが)共通している〕 
\\	〔酷似している〕 
\\	〔差がない〕 
\\	〔どうせ・どっちみち〕	その2つの 事故[じこ]は 同[おな]じ 原因[げんいん]で 起[お]きた。	
\\	同じ	おなじ	同じ	おなじ	
\\	お兄さん[おにいさん]	〔兄〕 
\\	(呼びかけて) 
\\	(➡英語では家族間での呼びかけには名前を言うのが普通); 〔他人になれなれしく呼びかけて〕 
\\	(若い男性に対して) 
\\	(飲食店などの男性従業員に) 
\\	「ちょっとそこの お兄[おにい]さん」
\\	私[わたし]のことですか」。	
\\	お兄さん	おにいさん	お兄さん	おにいさん	
\\	お姉さん[おねえさん]	〔姉〕 
\\	(呼びかけて) 
\\	(➡英語では家族間での呼びかけには名前を言うのが普通); 〔年上の女性になれなれしく呼びかけて〕 
\\	(若い女性に対して) 
\\	(ウエートレス・仲居などに) 
\\	私[わたし]のほうが 彼女[かのじょ]より お姉[おねえ]さんなの。	
\\	お姉さん	おねえさん	お姉さん	おねえさん	
\\	お祖母さん・お婆さん[おばあさん]	
\\	【お祖母さん】 〔祖母〕 
\\	《口・小児語》 
\\	《小児語》 
\\	〔呼びかけ〕 
\\	【お婆さん】 〔老女〕 
\\	私[わたし]もとうとうおばあさんになりました。	
\\	お祖母さん・お婆さん	おばあさん	おばあさん	おばあさん	
\\	伯母さん・叔母さん・小母さん[おばさん]	
\\	【伯母さん・叔母さん】 〔伯母・叔母〕 
\\	《小児語》 
\\	【小母さん】 〔年長の女性〕 
\\	まさよおばさん。	
\\	伯母さん・叔母さん・小母さん	おばさん	伯母さん	おばさん	
\\	お風呂[おふろ]	
\\	私[わたし]が お風呂[おふろ]に 入[はい]ったとたんに 電話[でんわ]が 鳴[な]った。	
\\	お風呂	おふろ	お風呂	おふろ	
\\	お弁当[おべんとう]	
\\	=べんとう. 
\\	〔口元についた飯粒など〕	ほっぺに お弁当[おべんとう]がついてるよ。	
\\	お弁当	おべんとう	お弁当	おべんとう	
\\	覚える[おぼえる]	
\\	〔記憶する〕 
\\	〔暗記する〕 
\\	〔習得する〕 
\\	〔自覚する〕 
\\	〔思われる〕 
\\	覚[おぼ]えていろ!	〔痛い目にあわせてやるから〕 
\\	覚える	おぼえる	覚える	おぼえる	
\\	お水[おみず]	
\\	お水	おみず	お水	おみず	
\\	重い[おもい]	
\\	〔目方が多い・比重が大きい〕 
\\	〔動きが鈍い・容易に動かない〕 
\\	〔気分などが重苦しい〕 
\\	〔安定感や重量感がある〕 
\\	〔重大・重要な〕 
\\	〔深刻な〕 
\\	〔程度がはなはだしい〕 (罰などが) 
\\	(病気が) 
\\	【電算】 〔システムが鈍重な〕 
\\	相撲[すもう]は 体重[たいじゅう]の 重[おも]い 人[ひと]も 軽[かる]い 人[ひと]も 一緒[いっしょ]に 戦[たたか]うから 面白[おもしろ]い。	
\\	重い	おもい	重い	おもい	
\\	面白い[おもしろい]	
\\	〔愉快な・楽しい〕 
\\	〔滑稽(こっけい)な〕 
\\	〔奇妙な〕 
\\	〔興趣がある・心が引かれる〕 
\\	〔好ましい・望ましい〕 (通例は「面白くない」の形で) 
\\	〔(「面白くない」の形で) 不愉快な・不満な〕 
\\	まあ 面白[おもしろ]い。	
\\	面白い	おもしろい	面白い	おもしろい	
\\	泳ぐ[およぐ]	
\\	〔水中を〕 
\\	〔水面上を〕 
\\	〔人込みなどを〕 
\\	〔世間を〕 
\\	〔体をふらつかせる〕 
\\	〔衣服・靴などが大きすぎて〕 
\\	ここは 一年中[いちねんじゅう] 泳[およ]ぐのによい 所[ところ]だ。	
\\	泳ぐ	およぐ	泳ぐ	およぐ	
\\	下りる・降りる[おりる]	
\\	【下りる】 〔高い所から下方へ移動する・移動して下に着く〕 
\\	【下りる】 〔飛んでいるものが下方へ移動する・着地や着陸する〕 (鳥が) 
\\	(滑空して) 
\\	(着陸する) 
\\	【下りる】 〔物が操作されて上から下へ動く〕 
\\	【降りる】 〔霜などが地上にできる〕 
\\	【下りる】 〔支給・発給される・申し渡される〕 
\\	【降りる】 〔乗り物から出る〕 
\\	《文》 
\\	【下りる】 〔体外に出る〕 
\\	〔流産する〕 
\\	【下りる】 〔負担がなくなる〕 
\\	【降りる】 〔地位・役・勝負などから退く〕 
\\	山[やま]を 下[お]りる。	
\\	下りる・降りる	おりる	降りる	おりる	
\\	終わる[おわる]	
\\	〔物事が終了する・おしまいになる〕 
\\	〔完了する〕 
\\	〔(「…に[で]~」の形で) …の結果となる, …の状態で終わりを迎える〕 
\\	〔物事を終了する・終える〕 
\\	〔(動詞の連用形に付いて) 動作などを終了する, …てしまう〕 
\\	私[わたし]たちの 付き合[つきあ]いは 半年[はんとし] 前[まえ]に 終[お]わりました。	
\\	終わる	おわる	終わる	おわる	
\\	音楽[おんがく]	
\\	音楽[おんがく]の 三[さん] 要素[ようそ]。	
\\	音楽	おんがく	音楽	おんがく	
\\	女[おんな]	
\\	〔ヒトの性のうち女性・女子〕 
\\	〔成熟した女性, 女性としての特質を備えた女性〕 
\\	〔愛人・情婦〕 
\\	女[おんな]の 先生[せんせい]
\\	医者[いしゃ]}。	
\\	女	おんな	女	おんな	
\\	女の子[おんなのこ]	
\\	〔少女〕 
\\	〔娘〕 
\\	〔女の赤ちゃん〕 
\\	〔若い女性〕 
\\	女の子[おんなのこ] 向[む]きの 雑誌[ざっし]。	
\\	女の子	おんなのこ	女の子	おんなのこ	
\\	外国[がいこく]	
\\	拳銃[けんじゅう]が 外国[がいこく]から 持ち込[もちこ]まれた。	
\\	外国	がいこく	外国	がいこく	
\\	外国人[がいこくじん]	
\\	(外国からのお客さま) 
\\	〔主に法律で〕 
\\	〔日本人以外の人〕 
\\	外国人[がいこくじん]を 受け入[うけい]れる。	
\\	外国人	がいこくじん	外国人	がいこくじん	
\\	会社[かいしゃ]	
\\	(略: 
\\	(略: 
\\	〔職場〕 
\\	〔仕事〕 
\\	会社[かいしゃ]に 入[はい]る。	
\\	会社	かいしゃ	会社	かいしゃ	
\\	階段[かいだん]	
\\	【鉱山】 〔露天掘などの〕 
\\	〔段階・等級〕 
\\	階段[かいだん]の 下[した]から3 段[だん] 目[め]。	
\\	階段	かいだん	階段	かいだん	
\\	買物[かいもの]	
\\	買物	かいもの	買物	かいもの	
\\	買う[かう]	
\\	〔買い求める〕 
\\	〔招く〕 
\\	〔引き受ける〕 
\\	〔価値を認める〕 
\\	〔金を払って性的交渉をする〕 
\\	乾電池[かんでんち]は 買[か]ってあるし 水[すい]もある。	〔災害用備品として〕 
\\	買う	かう	買う	かう	
\\	返す・反す[かえす]	
\\	【反す・返す】 〔ひっくり返す〕 
\\	〔裏返す〕 
\\	〔逆にする〕 
\\	【返す】 〔元の場所・状態へ戻す〕 
\\	【返す】 〔元の所有者に戻す〕 
\\	〔返却する〕 
\\	〔送り返す〕 
\\	【返す】 〔同じことを相手に対してする〕 
\\	〔返礼する〕 
\\	〔報復する〕 
\\	〔補助動詞として〕 
\\	裏[うら]を 返[かえ]して 見[み]ると 名前[なまえ]が 書[か]いてあった。	
\\	返す・反す	かえす	返す	かえす	
\\	帰る[かえる]	
\\	〔元の場所に戻る〕 
\\	〔以前の状態に戻る〕 
\\	〔去る〕 
\\	家[いえ]に 帰[かえ]る。	
\\	帰る	かえる	帰る	かえる	
\\	顔[かお]	
\\	〔顔面〕 
\\	〔顔の色〕 
\\	〔目鼻だち〕 
\\	〔顔つき〕 
\\	(表情) 
\\	〔代表・役割・出現・出席・参加〕 
\\	〔面目〕 
\\	〔知名度〕 
\\	〔信用〕 
\\	〔影響力〕 
\\	かわ いい顔[いいかお]の 少女[しょうじょ]。	
\\	顔	かお	顔	かお	
\\	掛かる・懸かる[かかる]	
\\	〔固定されて垂れる〕 
\\	〔とめて固定される〕 
\\	[⇒ひっかかる 
\\	〔上に置かれる〕 
\\	〔表面に注がれる〕 (液体が) 
\\	(砂などが) 
\\	〔表面が覆われる〕 
\\	〔まわりに巻きつく〕 
\\	〔道具が作動する・道具の機能が働く・機能がセットされる・契約により掛け金が払われる〕 
\\	〔陥る〕 
\\	〔釣針や網につかまる〕 
\\	〔時間・労力などが費やされる〕 
\\	〔金が〕 
\\	〔着手する〕 
\\	〔取り組む〕 
\\	〔不利益を被る〕 
\\	〔課せられる〕 
\\	〔攻撃をしかける〕 
\\	〔(弱者が)挑む〕 
\\	〔働きかけられる〕 
\\	〔作用が及ぶ〕 
\\	〔ある時期・箇所に至る〕 
\\	〔託される・委ねられる・そのこと次第で物事が決まる〕 
\\	〔受診する〕 
\\	〔(心・目などに)留まる〕 
\\	〔扱われる・処理される〕 
\\	〔上演・上映される〕 
\\	〔兼ねる・関連をもつ〕	見[み]るとすでに 刀[かたな]の 柄[え](つか)に 手[て]がかかっていた。	
\\	掛かる・懸かる	かかる	かかる	かかる	
\\	鍵[かぎ]	
\\	〔錠の穴に入れて開閉する器具〕 
\\	〔錠〕 
\\	〔解決・成功などへの手がかり〕 
\\	一[いち] 束[たば]の 鍵[かぎ]。	
\\	鍵	かぎ	鍵	かぎ	
\\	書く・描く[かく]	
\\	【書く】 
\\	〔書き記す〕 
\\	〔記述する〕 
\\	【描く】 (線で) 
\\	(彩色して) 
\\	〔眉を〕 
\\	遺言書[いごんしょ]には 次[つぎ]のように 書[か]いてある。	
\\	書く・描く	かく	書く	かく	
\\	学生[がくせい]	
\\	〔一校全体の〕 
\\	学生[がくせい]に 人気[にんき]の 教授[きょうじゅ]。	
\\	学生	がくせい	学生	がくせい	
\\	掛ける・懸ける[かける]	
\\	〔固定させて垂らす〕 
\\	〔留めて固定させる〕 
\\	〔上に置く〕 
\\	〔表面に注ぐ〕 (液体を) 
\\	(塩などを) 
\\	〔表面を覆う〕 
\\	〔まわりに巻きつける〕 
\\	〔道具を機能させる・道具で処理する・機能をセットする・契約して掛け金を払う〕 
\\	〔仕組んで陥れる〕 
\\	〔費やす〕 
\\	(金を) 
\\	(時間を) 
\\	(労力を) 
\\	〔不利益・負担を被らせる〕 
\\	〔課する〕 
\\	〔働きかける〕 
\\	〔相手に作用を及ぼす〕 
\\	〔託す・委ねる〕 
\\	〔処理・対応する・心に留める〕 
\\	〔上演・上映する〕 
\\	〔関連づける・かけもちする〕 
\\	〔掛け算をする〕 
\\	〔交尾させる〕 =かけあわせる 
\\	やかんをガスにかける。	
\\	掛ける・懸ける	かける	かける	かける	
\\	傘[かさ]	〔雨傘〕 
\\	〔日傘〕 
\\	傘[かさ]がなくてずぶぬれになってしまった。	
\\	傘	かさ	傘	かさ	
\\	貸す[かす]	
\\	〔無償で〕 (金品を) 
\\	(労力を) 
\\	〔担保を取って〕 
\\	〔賃貸しする〕 
\\	(家屋や部屋を) 
\\	【法】 (土地を) 
\\	〔掛け売りする〕 
\\	顔[かお]を 貸[か]す。	
\\	貸す	かす	貸す	かす	
\\	風[かぜ]	
\\	〔空気の流れ〕 
\\	(そよ風) 
\\	(すき間風) 
\\	(扇風機などの) 
\\	〔世間の風潮〕 
\\	〔社会を変動させるもの〕 
\\	〔(「…風」の形で)その雰囲気・そぶり〕 
\\	三[さん] 日[にち] 三[さん] 晩[ばん] 風[ふう]が 吹き荒[ふきあ]れた。	
\\	風	かぜ	風	かぜ	
\\	風邪[かぜ]	
\\	お 風邪[かぜ]を 召[め]しませんように。	
\\	風邪	かぜ	風邪	かぜ	
\\	家族[かぞく]	〔縁・婚姻などによって結ばれ, 共同生活を行っている人々〕 
\\	〔本人以外の家族構成員〕 
\\	家族[かぞく] 同伴[どうはん]で。	
\\	家族	かぞく	家族	かぞく	
\\	方・片[かた]	
\\	【方】 〔方角〕 
\\	【方】 〔敬称で人〕 
\\	【片】 〔片方〕 
\\	東[ひがし]の 方[ほう]に。	
\\	方・片	かた	方	かた	
\\	片仮名[かたかな]	
\\	片仮名[かたかな]で 外来語[がいらいご]を 表記[ひょうき]する。	
\\	片仮名	かたかな	片仮名	かたかな	
\\	学校[がっこう]	
\\	〔専門・大学程度の〕 
\\	〔中学以上高等学校程度の〕 
\\	〔神学などの〕 
\\	〔総称的に〕 
\\	学校[がっこう]のない 日[ひ]は 釣[つ]りに 行[い]きます。	
\\	学校	がっこう	学校	がっこう	
\\	角[かど]	
\\	〔稜角〕 
\\	(岩石などの) 
\\	【建】 (繰り形などの) 
\\	〔すみ〕 
\\	〔曲がり角〕 
\\	〔人柄などのとげとげしい所〕 
\\	角[かく]の 丸[まる]い 小型[こがた]の 名刺[めいし]。	
\\	角	かど	角	かど	
\\	家内[かない]	
\\	〔家の内・家族〕 
\\	《口》 
\\	〔自分の妻〕 
\\	息子[むすこ]さんのご 結婚[けっこん]おめでとうございました. 家内中[かないじゅう]で 喜[よろこ]んでおります。	
\\	家内	かない	家内	かない	
\\	鞄[かばん]	〔袋状の〕 
\\	(旅行用) 
\\	(小型の手提げ) 
\\	(折り鞄) 
\\	(ランドセル) 
\\	(学生鞄) 
\\	〔箱状の〕 
\\	(旅行用の大型) 
\\	(小型) 
\\	(書類用) 
\\	(紙ばさみ式の) 
\\	鞄[かばん]を 肩[かた]に 掛[か]ける。	
\\	鞄	かばん	鞄	かばん	
\\	花瓶[かびん]	
\\	花瓶[かびん]にさす。	
\\	花瓶	かびん	花瓶	かびん	
\\	冠る[かぶる]	
\\	冠る	かぶる	冠る	かぶる	
\\	紙[かみ]	
\\	〔薄いもの〕	紙[かみ]を 破[やぶ]る。	
\\	紙	かみ	紙	かみ	
\\	カメラ[カメラ]	
\\	カメラのフラッシュを 浴[あ]びる。	
\\	カメラ	カメラ	カメラ	かめら	
\\	火曜日[かようび]	
\\	(略: 
\\	火曜日[かようび]までに 返事[へんじ]をいただけるよう お願[おねが]いします。	
\\	火曜日	かようび	火曜日	かようび	
\\	辛い[からい]	
\\	〔ぴりっと感じる〕 
\\	〔塩辛い〕 
\\	[=しおからい] 
\\	〔酒が〕 
\\	〔評価などが厳しい〕 
\\	《口》 
\\	スープが 真っ赤[まっか]でものすごく 辛[つら]そう。	
\\	辛い	からい	辛い	からい	
\\	体[からだ]	
\\	〔頭から足まで全体〕 (身体) 
\\	(五体) 
\\	〔胴体〕 
\\	ウイルスが 体[からだ]の 中[なか]に 入[はい]るのを 防[ふせ]ぐ。	
\\	体	からだ	体	からだ	
\\	借りる[かりる]	
\\	〔金品を一時使わせてもらう〕 
\\	〔賃借りする〕 
\\	〔貸し切り契約で乗り物を〕 
\\	〔他人の助力・協力などを得る〕 
\\	〔臨時に別の用途に使う〕 
\\	兄[あに]は 僕[ぼく]のグローブを 借[か]りたまま 返[かえ]してくれない。	
\\	借りる	かりる	借りる	かりる	
\\	軽い[かるい]	
\\	〔目方が少ない〕 
\\	〔軽快だ〕 
\\	〔軽薄だ〕 
\\	〔重量感がない〕 
\\	〔重大でない・軽微な〕 
\\	〔容易な〕 
\\	〔気軽な〕 
\\	〔ランクの低い〕 
\\	〔本格的でない・あっさりした〕 
\\	羽根[はね]
\\	綿毛[わたげ]}のように 軽[かる]い。	
\\	軽い	かるい	軽い	かるい	
\\	カレンダー[カレンダー]	〔暦〕 
\\	園芸[えんげい]カレンダー。	
\\	カレンダー	カレンダー	カレンダー	カレンダー	
\\	川・河[かわ]	
\\	〔流れ〕 
\\	〔小川〕 
\\	〔大河〕 
\\	この 川[かわ]は 日本海[にほんかい]へ 流[なが]れていく。	
\\	川・河	かわ	川	かわ	
\\	可愛い[かわいい]	
\\	〔愛情の対象として〕 
\\	〔魅力的な〕 
\\	〔可憐な〕 
\\	〔憎めない〕 
\\	〔好ましい〕 
\\	〔小さい〕 
\\	〔ペットに〕 お前[おまえ]はほんとうにかわいいねえ。	
\\	可愛い	かわいい	可愛い	かわいい	
\\	漢字[かんじ]	
\\	漢字[かんじ]で 書[か]く。	
\\	漢字	かんじ	漢字	かんじ	
\\	木・樹[き]	
\\	【木・樹】 〔樹木〕 
\\	(灌木) 
\\	木[き]を 切り出[きりだ]す。	
\\	木・樹	き	木	き	
\\	黄色い[きいろい]	
\\	まだ 黄色[きいろ]いサクランボ。	
\\	黄色い	きいろい	黄色い	きいろい	
\\	消える[きえる]	
\\	〔光を失う〕 
\\	(火が) 
\\	〔電源が切れる〕 
\\	〔溶けてなくなる〕 
\\	〔乾いてなくなる〕 
\\	〔見当たらなくなる〕 
\\	〔姿を隠す〕 
\\	(身を引く) 
\\	(死ぬ) 
\\	〔抹消される〕 
\\	〔抹殺される〕 
\\	〔感知されなくなる〕 (形が) 
\\	(音が) 
\\	(味やにおいが) 
\\	(痛みなどが) 
\\	〔検知されなくなる〕 
\\	〔感情が治まる〕 
\\	〔可能性がなくなる〕 
\\	〔特徴がなくなる〕 
\\	〔帳消しになる〕 
\\	窓[まど]の 明[あ]かりが 消[き]えた。	
\\	消える	きえる	消える	きえる	
\\	聞く[きく]	
\\	〔音声を耳に入れる〕 
\\	(関心をもって) 
\\	[⇒きかせる2, きかない1] 
\\	〔聞き知る〕 
\\	〔受け入れる・聞き入れる〕 (忠告を) 
\\	(要求を) 
\\	(願望を) 
\\	(訴えを) 
\\	〔悩みなどを受けとめる〕 
\\	〔質問する〕 
\\	〔照会する〕 
\\	〔香を〕 ⇒こう14.	彼[かれ]は 私[わたし]の 聞[き]いているところでそう 言[い]った。	
\\	聞く	きく	聞く	きく	
\\	北[きた]	
\\	(略: 
\\	~の 
\\	北[きた]
\\	寄[よ]り)に。	
\\	北	きた	北	きた	
\\	ギター[ギター]	
\\	ギターを 弾[ひ]く。	
\\	ギター	ギター	ギター	ギター	
\\	汚い[きたない]	
\\	〔不潔な〕 
\\	〔乱雑な〕 
\\	〔見苦しい〕 
\\	〔聞き苦しい〕 
\\	〔尾籠(びろう)な〕 
\\	〔卑怯な・卑劣な〕 
\\	《口》 
\\	〔けちな・欲深い〕 
\\	汚[きたな]い 部屋[へや]。	
\\	汚い	きたない	汚い	きたない	
\\	喫茶店[きっさてん]	
\\	きっさき【切っ先】 ローマ
\\	この 喫茶店[きっさてん]でよくおしゃべりしたわね。	
\\	喫茶店	きっさてん	喫茶店	きっさてん	
\\	切手[きって]	〔郵券〕 
\\	(切手・印紙など) 
\\	《口》 
\\	〔商品切手〕 =商品券 (⇒しょうひん2).	切手[きって]で1500 円[えん] 払[はら]う
\\	送[おく]る}。	
\\	切手	きって	切手	きって	
\\	切符[きっぷ]	
\\	〔乗車券〕 
\\	〔入場券〕 
\\	〔交通違反の〕 
\\	〔出場できる資格・権利〕 
\\	ドバイまでの 飛行機[ひこうき]の 通し切符[とおしきっぷ]を 買[か]う。	
\\	切符	きっぷ	切符	きっぷ	
\\	昨日[きのう]	
\\	昨日[きのう]の 木曜日[もくようび]には。	
\\	昨日	きのう	昨日	きのう	
\\	九[きゅう]	⇒く1.	彼[かれ]は 午後[ごご] 九[きゅう] 時[じ]にここへ 来[く]る 事[こと]になっている。	
\\	九	きゅう	九	きゅう	
\\	牛肉[ぎゅうにく]	
\\	牛肉[ぎゅうにく]の 薄切[うすぎ]り。	
\\	牛肉	ぎゅうにく	牛肉	ぎゅうにく	
\\	牛乳[ぎゅうにゅう]	
\\	牛乳[ぎゅうにゅう]をしぼる。	
\\	牛乳	ぎゅうにゅう	牛乳	ぎゅうにゅう	
\\	今日[きょう]	
\\	今日中[きょうじゅう]に。	
\\	今日	きょう	今日	きょう	
\\	教室[きょうしつ]	
\\	〔教場〕 
\\	〔階段教室〕 
\\	〔講習会〕 
\\	番[ばん] 教室[きょうしつ]。	
\\	教室	きょうしつ	教室	きょうしつ	
\\	兄弟[姉妹][きょうだい]	
\\	〔親を同じくする子供(たち)〕 
\\	(男性) 
\\	(女性) 
\\	〔師を同じくする人〕 
\\	〔特に親しい関係の人〕 
\\	二[に] 人[にん]は 兄弟[きょうだい]
\\	姉妹[しまい]}みたいに 似[に]ている。	
\\	兄弟[姉妹]	きょうだい	兄弟	きょうだい	
\\	去年[きょねん]	
\\	去年[きょねん]の3 月[つき]。	
\\	月以降に, 単に 
\\	というと今年の 
\\	月を意味することになる.
\\	去年	きょねん	去年	きょねん	
\\	嫌い[きらい]	
\\	〔嫌うこと〕 
\\	[⇒-ぎらい] ~な 
\\	〔気味〕 
\\	〔傾向〕 
\\	〔区別〕 
\\	嫌[きら]いな 物[もの]。	
\\	嫌い	きらい	嫌い	きらい	
\\	切る[きる]	
\\	〔分断する〕 
\\	つに) 
\\	(のこぎりで) 
\\	(はさみで) 
\\	(食卓で鳥肉などを) 
\\	【囲碁】 
\\	〔寸断する〕 
\\	(ぶつぶつと) 
\\	(刻む) 
\\	(薄く) 
\\	〔一部を取り去る〕 
\\	(切符の) 
\\	〔取り除く〕 
\\	〔傷をつける〕 
\\	(浅く) 
\\	(深く) 
\\	〔溝をつける〕 
\\	〔終了する〕 
\\	〔電話を〕 
\\	(話し中に) 
\\	〔電気器具の稼働を止める〕 
\\	〔絶縁する〕 
\\	〔解雇する〕 
\\	〔期限を限定する〕 
\\	〔数量を限定する〕 
\\	〔下回る〕 
\\	〔開封する〕 
\\	〔壁や床を抜く〕 
\\	〔水気を落とす〕 
\\	(振り落とす) 
\\	〔通り抜ける〕 
\\	〔伝票などを発行する〕 
\\	〔方向を変える〕 
\\	(自動車のハンドルで) 
\\	(テニスなどのボールで) 
\\	〔批判する〕 
\\	〔トランプを〕 
\\	分する) 
\\	分してパラパラと〕 
\\	〔切り札を〕 
\\	(相手より上の札で) 
\\	彼女[かのじょ]は 腐敗[ふはい]した 政界[せいかい]を 舌鋒[ぜっぽう] 鋭[するど]く 切[き]った。	
\\	切る	きる	切る	きる	
\\	着る[きる]	
\\	〔体につける〕 
\\	(引っかける) 
\\	(急いで) 
\\	〔体につけている〕 
\\	〔罪などを身に引き受ける〕	子供[こども]が 大[おお]きくなってこの 服[ふく]も 着[き]られなくなった。	
\\	着る	きる	着る	きる	
\\	きれい[きれい]	~な 
\\	〔美しい〕 
\\	(絵のような) 
\\	〔美貌の〕 
\\	きれいな 女性[じょせい]。	
\\	きれい	きれい	きれい	きれい	
\\	キロ[キロ]	
\\	〔キロメートル〕 
\\	(略: 
\\	〔キログラム〕 
\\	(略: 
\\	〔キロリットル〕 
\\	(略: 
\\	〔キロワット〕 
\\	(略: 
\\	時速[じそく]100キロ。	
\\	キロ	キロ	キロ	キロ	
\\	銀行[ぎんこう]	〔金融の〕 
\\	銀行[ぎんこう]に 金[きん]を 預[あづ]ける
\\	預金[よきん]する}。	
\\	銀行	ぎんこう	銀行	ぎんこう	
\\	金曜日[きんようび]	
\\	(略: 
\\	日[にち]の 金曜日[きんようび]。	
\\	金曜日	きんようび	金曜日	きんようび	
\\	九月[くがつ]	
\\	(略: 
\\	九月[くがつ]10 日[にち]。	
\\	九月	くがつ	九月	くがつ	
\\	薬[くすり]	
\\	〔薬剤〕 
\\	〔粉薬〕 
\\	〔丸薬〕 
\\	〔水薬〕 
\\	〔飲み薬〕 
\\	〔塗り薬〕 
\\	〔せんじ薬〕 
\\	〔強壮剤〕 
\\	〔薬物〕 
\\	〔化学薬品〕 
\\	〔農薬〕 
\\	〔防虫剤〕 
\\	〔消毒液〕 
\\	〔上薬〕 
\\	〔ためになるもの〕 
\\	〔少量の例え〕 
\\	〔麻薬〕 =やく6	薬[くすり]の 副作用[ふくさよう]。	
\\	薬	くすり	薬	くすり	
\\	下さい[ください]	
\\	〔いただきたい〕 
\\	〔その動作をする人に頼むときの丁寧な言い方〕	どうか 折り返[おりかえ]しご 返事[へんじ]をください。	
\\	下さい	ください	下さい	ください	
\\	果物[くだもの]	〈総称〉 
\\	〔個々〕 
\\	果物[くだもの]に 似[に]た。	
\\	果物	くだもの	果物	くだもの	
\\	口[くち]	
\\	〔摂食・発声の器官〕 (人間・動物の) 
\\	(動物・怪物などの) 
\\	【解】 
\\	(唇) 
\\	〔言葉〕 
\\	〔味覚〕 
\\	〔養う人数〕 
\\	[⇒くちかず 
\\	〔物を出し入れする場所〕 
\\	(急須の) 
\\	(ホースの) 
\\	(たるの) 
\\	(穴) 
\\	(細長い) 
\\	(すきま) 
\\	【船舶】 
\\	〔栓(せん)〕 
\\	〔人や物の出入りする場所〕 
\\	(門) 
\\	(入口) 
\\	(出口) 
\\	〔ある地点に通じるところの始まり〕 (登山口) 
\\	(川の) 
\\	(港の) 
\\	〔就職先〕 
\\	*《口》 
\\	〔欠員〕 
\\	〔嫁ぎ先〕 
\\	〔招き〕 
\\	〔需要〕 
\\	〔種類〕 
\\	(品目) 
\\	(商品の) 
\\	〔割り前〕 
\\	〔始まり〕 
\\	〔うわさ〕 
\\	〔腫れものなどの〕 
\\	口[くち]がかかる。	
\\	〔芸者などが〕 
\\	口	くち	口	くち	
\\	靴[くつ]	〔短靴〕 
\\	〔深靴〕 
\\	〔長靴〕 
\\	〈総称〉 
\\	足[あし]に 合[あ]っていない 靴[くつ]。	
\\	靴	くつ	靴	くつ	
\\	靴下[くつした]	〔短い〕 
\\	(くるぶしがやっと隠れる程度の長さの) 
\\	〔長い〕 
\\	(主に商用語) 
\\	〈総称〉 
\\	靴[くつ]を 脱[ぬ]いで) 靴下[くつした]だけになって。	
\\	靴下	くつした	靴下	くつした	
\\	国[くに]	
\\	〔国土〕 
\\	〔領土〕 
\\	〔国家〕 
\\	(海岸線のある) 《文》 
\\	[⇒くにぐに] 
\\	〔昔の行政区画の単位〕 
\\	〔地域〕 
\\	〔母国〕 
\\	〔故郷〕 
\\	国[くに]のために 命[いのち]を 捨[す]てる。	
\\	国	くに	国	くに	
\\	曇り[くもり]	
\\	〔曇天〕 
\\	〔ぼんやりしてよく見えない状態〕 
\\	(銀などの) 
\\	(鏡などの) 
\\	【化】 
\\	〔陰影〕 
\\	〔気持ちがすっきりしないこと〕 
\\	〔疑い〕 
\\	天気予報[てんきよほう]で〕 明日[あした]の 天気[てんき]は 晴[は]れ 時々[ときどき] 曇[くも]りでしょう。	
\\	曇り	くもり	曇り	くもり	
\\	暗い[くらい]	
\\	〔光がたりない〕 (場所が) 
\\	(電灯などが) 
\\	〔くすんだ・黒っぽい〕 
\\	〔うとい〕 
\\	〔希望が持てない〕 
\\	〔気分が晴れ晴れしない〕 
\\	〔陰気な〕 
\\	〔不幸な〕 
\\	暗[くら]い 部屋[へや]。	
\\	暗い	くらい	暗い	くらい	
\\	-ぐらい[-ぐらい]	=-くらい.			-ぐらい	-ぐらい	ぐらい	ぐらい	
\\	クラス[クラス]	
\\	〔学級〕 
\\	〔階級・等級〕 
\\	[⇒きゅう8] 
\\	【電算】 〔変数の〕 
\\	君[きみ]はクラスで 何[なん] 番[ばん]ですか。	
\\	クラス	クラス	クラス	クラス	
\\	グラム[グラム]	〔質量の単位〕 
\\	グラム1000 円[えん]の 牛肉[ぎゅうにく]。	
\\	➡食品業界では
\\	あたり…」のことを「グラム…」ということがあるが, 俗用であり, 英語の 
\\	にはこの用法はない.	グラム	グラム	グラム	グラム	
\\	来る[くる]	
\\	〔こちらに近づく〕 
\\	〔来訪する〕 
\\	(ひょっこり) 《口》 
\\	(顔を出す) 
\\	〔到着する〕 
\\	(列車が) 
\\	(手紙などが) 
\\	〔時間・季節がめぐる〕 
\\	〔期限になる〕 
\\	〔伝来する〕 
\\	〔由来する〕 
\\	〔起因する〕 
\\	〔影響が現れる〕 
\\	〔態度を取る〕 
\\	〔そのことを強調する〕 
\\	〔心に感じる〕 
\\	《口》 
\\	〔…して戻る〕 
\\	〔その状態になる〕 
\\	〔引き続き今におよぶ〕 
\\	〔ある現象が生じる・実現する〕	こっちへ 来[こ]いよ。	
\\	来る	くる	来る	くる	
\\	車[くるま]	
\\	〔車輪〕 
\\	〔ピアノなど, 重い家具の下につけるもの〕 
\\	〔自動車〕 
\\	〔タクシー〕 
\\	[⇒じどうしゃ] 
\\	〔人力車〕 
\\	大[おお]きな 車[くるま]のついた 荷車[にぐるま]。	
\\	車	くるま	車	くるま	
\\	黒い[くろい]	
\\	〔色が黒である〕 
\\	(浅黒い) 
\\	〔日に焼けた色をしている〕 
\\	〔汚れている〕 
\\	〔不正を感じさせる〕 
\\	〔不吉な感じのする〕 
\\	黒[くろ]い 傘[かさ]。	
\\	黒い	くろい	黒い	くろい	
\\	今朝[けさ]	
\\	今朝[けさ]から。	
\\	今朝	けさ	今朝	けさ	
\\	消す[けす]	
\\	〔燃えなくする〕 
\\	〔スイッチを切る〕 
\\	〔抹消・削除する〕 
\\	〔隠滅する〕 
\\	〔殺す〕 
\\	*《口》 
\\	*《俗》 
\\	〔感知できなくする〕 (形を) 
\\	(音を) 
\\	(味を) 
\\	(においを) 
\\	(痛みなどを) 
\\	〔検知できなくする〕 
\\	電気器具[でんききぐ]やガス 器具[きぐ]を 指[さ]して〕それを 消[け]してくれ。	
\\	消す	けす	消す	けす	
\\	結構[けっこう]	
\\	〔構造〕 
\\	〔けた構え〕 
\\	〔構成〕 
\\	〔筋・趣向〕 
\\	[⇒こうぞう, こうせい19] 
\\	~な 〔よい〕 
\\	〔申し分ない〕 
\\	〔すばらしい〕 
\\	〔十分な〕 
\\	〔満足できる〕 
\\	〔十分である〕 
\\	〔まあ十分に〕 
\\	〔ほどほどに〕 
\\	〔案外と〕 
\\	大寺院[だいじいん]の 荘厳[そうごん]な 結構[けっこう]。	
\\	結構	けっこう	けっこう	けっこう	
\\	結婚[けっこん]	
\\	~する 
\\	結婚[けっこん]している 人[ひと]。	
\\	結婚	けっこん	結婚	けっこん	
\\	月曜日[げつようび]	
\\	(略: 
\\	私[わたし]はメアリーにいくらか お金[おかね]を 借[か]りているので、 今度[こんど]の 月曜日[げつようび]までに お金[おかね]を 返[かえ]さなければならない。	
\\	月曜日	げつようび	月曜日	げつようび	
\\	玄関[げんかん]	〔入口〕 
\\	〔玄関の間〕 
\\	これからちょっと ご挨拶[ごあいさつ]にうかがいますが 玄関[げんかん]で 失礼[しつれい]します。	
\\	玄関	げんかん	玄関	げんかん	
\\	元気[げんき]	〔健康〕 
\\	〔活力〕 
\\	《口》 
\\	〔気力〕 
\\	《口》 
\\	~な 
\\	元気[げんき]をつける。	
\\	[=げんきづける].	元気	げんき	元気	げんき	
\\	五[ご]	
\\	〔さいの目・トランプ札の〕 
\\	さいころの 五[ご]。	
\\	五	ご	五	ご	
\\	コート[コート]	〔外套〕 
\\	(女性用) 
\\	暖[あたた]かくて)コートが 要[い]らない。	
\\	コート	コート	コート	コート	
\\	公園[こうえん]	
\\	日比谷公園[ひびやこうえん]。	
\\	公園	こうえん	公園	こうえん	
\\	交番[こうばん]	
\\	〔警察の〕 
\\	〔交替〕 
\\	交番[こうばん]に 届[とど]ける。	
\\	〔拾得物を〕 
\\	交番	こうばん	交番	こうばん	
\\	声[こえ]	
\\	〔ヒトの〕 
\\	(語気) 
\\	(有声音) 
\\	〔動物の〕 
\\	〔鳥や虫の〕 
\\	〔鐘や風の〕 
\\	〔依頼・勧誘〕 
\\	〔意思表示〕 
\\	〔要望〕 
\\	〔風評〕 
\\	〔到来〕	鳥[とり]の 声[こえ]。	
\\	声	こえ	声	こえ	
\\	五月[ごがつ]	
\\	五月の節句[ごがつのせっく]。	
\\	五月	ごがつ	五月	ごがつ	
\\	此処[ここ]	
\\	〔この場所〕 
\\	〔このこと〕 
\\	〔この事態〕 
\\	〔この時点〕 
\\	[⇒ここいちばん, ここぞ] 
\\	〔期間〕	
\\	本[ほん]を 開[ひら]いて〕ここの 意味[いみ]がよくわからない。	
\\	此処	ここ	ここ	ここ	
\\	午後[ごご]	
\\	〔正午から日没まで〕 
\\	土曜[どよう]の 午後[ごご]に。	
\\	午後	ごご	午後	ごご	
\\	九日[ここのか]	〔日付〕 (書き言葉で) 
\\	(話し言葉で) 
\\	〔日数〕 
\\	私[わたし]がロンドンについたのは 2月[にがつ]9 日[にち]の 朝[あさ]であった。	
\\	九日	ここのか	九日	ここのか	
\\	九つ[ここのつ]	
\\	〔数〕 
\\	〔年齢〕 
\\	[=く1] 
\\	〔昔の時刻〕 
\\	九[ここの]つの 子供[こども]。	
\\	九つ	ここのつ	九つ	ここのつ	
\\	ご主人[ごしゅじん]	
\\	あなたの ご主人[ごしゅじん]は 食事[しょくじ]にうるさいのかな?	
\\	ご主人	ごしゅじん	ご主人	ごしゅじん	
\\	午前[ごぜん]	
\\	〔日の出から正午まで〕 
\\	午前[ごぜん]の 授業[じゅぎょう]。	
\\	午前	ごぜん	午前	ごぜん	
\\	答える[こたえる]	
\\	〔返事・返答をする〕 
\\	〔解答する〕 
\\	私[わたし]はどう 答[こた]えてよいかわからなかった。	
\\	答える	こたえる	答える	こたえる	
\\	こちら[こちら]	
\\	〔この場所〕 
\\	〔手前側〕 
\\	〔この方向〕 
\\	〔これ・この人〕 
\\	〔貴方〕 
\\	〔当方〕 
\\	こちらへ お越[おこ]しの 節[ふし]はどうぞお 立ち寄[たちよ]りください。	
\\	こちら	こちら	こちら	こちら	
\\	コップ[コップ]	
\\	〔平底で大型の〕 
\\	酒[さけ]をコップで 飲[の]む。	
\\	コップ	コップ	コップ	コップ	
\\	今年[ことし]	
\\	その 作業[さぎょう]は 今年[ことし] 中[ちゅう] 続[つづ]きます。	
\\	今年	ことし	今年	ことし	
\\	言葉[ことば]	
\\	〔言語〕 
\\	〔ある言語〕 
\\	《文》 
\\	〔方言〕 
\\	〔単語〕 
\\	〔語句〕 
\\	〔用語〕 
\\	〔発言〕 
\\	〔助言や忠告〕 
\\	〔表現〕 
\\	〔箴言〕 
\\	〔ことわざ〕 
\\	〔口調〕 
\\	〔儀式などでの挨拶〕 
\\	〔幼児の言語獲得〕 
\\	《文》 
\\	言葉の壁[ことばのかべ]。	
\\	言葉	ことば	言葉	ことば	
\\	子供[こども]	
\\	〔子女〕 
\\	〔息子〕 
\\	〔娘〕 
\\	〈総称〉 
\\	《文》 
\\	【主として法】 
\\	〔胎児〕 
\\	[=こ1 
\\	〔幼少者〕 
\\	《口》 
\\	(男児) 
\\	(女児) 
\\	〔乳児〕 
\\	〔幼児〕 
\\	[=こ1 
\\	⇒がき 
\\	子供[こども]を 持[も]つ。	
\\	子供	こども	子供	こども	
\\	此の[この]	
\\	〔人や物〕 
\\	〔場所や状況や内容〕 
\\	〔時期〕 (去る) 
\\	(来たる) 
\\	この 机[つくえ]。	
\\	此の	この	この	この	
\\	御飯[ごはん]	〔炊いた米〕 
\\	〔食事〕 
\\	[=めし2]	ご飯[ごはん]を 蒸[む]らす。	
\\	御飯	ごはん	御飯	ごはん	
\\	困る[こまる]	
\\	〔不便を感じる〕 
\\	《口》 
\\	〔対処に窮する〕 
\\	*《口》 
\\	[⇒こまった] 
\\	〔迷惑する〕 
\\	〔苦痛である〕 
\\	[⇒こまった] 
\\	〔不都合である〕 
\\	〔貧窮する〕 
\\	突然[とつぜん]の 停電[ていでん]で 困[こま]った。	
\\	困る	こまる	困る	こまる	
\\	これ[これ]	[⇒これという, これくらい, これっぱかり, これでこそ, これでも, これなら, これほど] 
\\	〔このもの・この人〕 
\\	〔このこと〕 
\\	〔この状態・程度〕 
\\	〔現時点〕 
\\	〔今回〕 
\\	〔この場所〕 
\\	[⇒ここ2 
\\	これが 次女[じじょ]です。	
\\	これ	これ	これ	これ	
\\	-頃[-ごろ]	〔その前後の時分〕 
\\	あしたの 今[いま]ごろ。	⇒いまごろ1.	-頃	-ごろ	ごろ	ごろ	
\\	今月[こんげつ]	
\\	【商】 
\\	(略: 
\\	年[ねん] 前[まえ]の 今月[こんげつ]。	
\\	今月	こんげつ	今月	こんげつ	
\\	今週[こんしゅう]	
\\	今週中[こんしゅうじゅう]に。	
\\	今週	こんしゅう	今週	こんしゅう	
\\	こんな[こんな]	
\\	人生[じんせい]なんてこんなものだ。	
\\	こんな	こんな	こんな	こんな	
\\	今晩[こんばん]	
\\	今晩[こんばん]はもうやすみましたから 明日[あした]また お電話[おでんわ]ください。	
\\	今晩	こんばん	今晩	こんばん	
\\	さあ[さあ]	
\\	〔促して〕 
\\	[⇒さあさあ] 
\\	〔自分を励まして〕 
\\	〔きっかけの時点で〕 
\\	〔困難に直面して〕 
\\	〔確信がない時〕 
\\	《口》 
\\	〔肯定しにくい時〕 
\\	さあまずは 一[いち] 杯[はい]いこう。	
\\	さあ	さあ	さあ	さあ	
\\	魚[さかな]	
\\	〔釣りのための〕 
\\	〔魚肉〕 
\\	〔魚料理〕 
\\	釣[つ]りで〕 魚[さかな]がかかった。	
\\	魚	さかな	魚	さかな	
\\	先[さき]	
\\	〔先端〕 
\\	〔末端〕 
\\	〔先頭〕 
\\	〔早い順番〕 ~の 
\\	〔最初〕 
\\	〔前方〕 
\\	(進行方向) 
\\	〔さらに遠方〕 
\\	〔将来〕 
\\	〔前途〕 
\\	〔直後〕 
\\	[⇒さきざき 
\\	〔続き〕 
\\	〔結末〕 
\\	〔めざす場所〕 
\\	〔先方〕 
\\	[⇒さきさま] 
\\	〔以前〕 ~の 〔この前の〕 
\\	〔元の〕 
\\	先[さき]の 細[ほそ]い。	
\\	先	さき	先	さき	
\\	咲く[さく]	
\\	[⇒さかせる]	桜[さくら]の 咲[さ]くころ。	
\\	咲く	さく	咲く	さく	
\\	作文[さくぶん]	
\\	〔国語教育などで文章を作ること〕 
\\	〔作った文章〕 
\\	〔実質のない文章〕 
\\	~する 〔もっともらしく文章にまとめる〕 
\\	私[わたし]は 作文[さくぶん]が 得意[とくい]だ。	
\\	作文	さくぶん	作文	さくぶん	
\\	差す・射す[さす]	
\\	〔光が〕 
\\	〔影が〕 
\\	〔色が〕 
\\	〔注ぎ入れる〕 
\\	(注ぎ足す) 
\\	〔着色する〕 
\\	〔満潮になる〕 
\\	〔流れ込む〕 
\\	〔きざす〕 
\\	〔立てる〕 
\\	(穴に) 
\\	〔帯に差し込む〕 
\\	〔脇に入れる〕 
\\	【相撲】 
\\	【日舞】 
\\	彼女[かのじょ]の 頬[ほお]に 赤[あか]みがさしてきた。	
\\	差す・射す	さす	さす	さす	
\\	雑誌[ざっし]	
\\	〔専門的な〕 
\\	〔定期刊行の〕 
\\	雑誌[ざっし]のバックナンバー。	
\\	雑誌	ざっし	雑誌	ざっし	
\\	砂糖[さとう]	
\\	[=スクロース, サッカロース, しょとう4]	砂糖[さとう]を 入[い]れる。	
\\	砂糖	さとう	砂糖	さとう	
\\	寒い[さむい]	
\\	〔気温が低く感じられる〕 
\\	〔恐ろしさでぞっとする〕 
\\	〔心が満たされない〕 
\\	〔所持金が少ない〕 
\\	〔貧弱な〕 
\\	[=おさむい 
\\	寒[さむ]くて 震[ふる]える。	
\\	寒い	さむい	寒い	さむい	
\\	再来年[さらいねん]	
\\	来年[らいねん]か 再来年[さらいねん]。	
\\	再来年	さらいねん	再来年	さらいねん	
\\	三[さん]	
\\	〔ローマ数字〕 
\\	第[だい] 三[さん]。	
\\	三	さん	三	さん	
\\	三月[さんがつ]	
\\	(略: 
\\	この 製品[せいひん]は、 2月[にがつ]15 日[にち]までに ご注文[ごちゅうもん]いただければ、 3月[さんがつ]15 日[にち]までにお 届[とど]けすることができます。	
\\	三月	さんがつ	三月	さんがつ	
\\	散歩[さんぽ]	
\\	《口》 (健康のための) 
\\	~する 
\\	(のんびりと) 
\\	朝[あさ]の 散歩[さんぽ]。	
\\	散歩	さんぽ	散歩	さんぽ	
\\	四[し]	
\\	〔ローマ数字〕 
\\	第[だい] 四[よん]。	
\\	四	し	四	し	
\\	塩[しお]	
\\	〔調味料の〕 
\\	【化】 
\\	[=塩化ナトリウム (⇒えんか8)] 
\\	〔塩加減〕 ⇒しおかげん.	塩[しお]の 結晶[けっしょう]。	
\\	塩	しお	塩	しお	
\\	しかし[しかし]	
\\	〔逆接; だが〕 
\\	[⇒しかしながら] 
\\	〔話題転換; それはそれとして〕 
\\	〔感動; それにしても〕	雨[あめ]だ.しかし 行[い]かねばならない。	
\\	しかし	しかし	しかし	しかし	
\\	四月[しがつ]	
\\	(略: 
\\	次[つぎ]の 四月[しがつ]までに、あなたは10 年間[ねんかん] 英語[えいご]を 勉強[べんきょう]したことになる。	
\\	四月	しがつ	四月	しがつ	
\\	仕事[しごと]	
\\	〔しなければならないこと・働くこと〕 
\\	(割り当ての) 
\\	〔職業〕 
\\	〔雇用〕 
\\	〔義務〕 
\\	【物】 〔物体に力を加えて動かすこと〕 
\\	日雇[ひやと]い 仕事[しごと]。	
\\	仕事	しごと	仕事	しごと	
\\	辞書[じしょ]	
\\	〔用語解〕 
\\	〔専門語・術語の〕 
\\	〔宝典〕 
\\	ポケット 判[ばん]の 辞書[じしょ]。	
\\	辞書	じしょ	辞書	じしょ	
\\	静か[しずか]	~な 〔音の立たない〕 
\\	〔穏やかな〕 
\\	〔激しくない〕 
\\	〔しとやかな〕 
\\	〔ゆっくりした〕 
\\	〔動かない〕 
\\	〔ひっそりした〕 
\\	静[しず]かな 音楽[おんがく]。	
\\	静か	しずか	静か	しずか	
\\	下[した]	
\\	〔下部〕 
\\	〔下の方〕 
\\	〔見えない内側〕 
\\	〔地位・階級・年齢・能力・数量などが他より低いこと〕 
\\	〔その状態で〕	
\\	積[つ]みあげた 雑誌[ざっし]の 山[やま]で〕 先月[せんげつ] 号[ごう]は 今月[こんげつ] 号[ごう]の 下[した]になっているはずだ。	
\\	下	した	下	した	
\\	七月[しちがつ]	
\\	(略: 
\\	さくらんぼは 六月[ろくがつ]か 七月[しちがつ]に 熟[じゅく]す。	
\\	七月	しちがつ	七月	しちがつ	
\\	質問[しつもん]	
\\	〔国会での〕 
\\	~する 
\\	〔国会で〕 
\\	たたみかけるように 質問[しつもん]する。	
\\	質問	しつもん	質問	しつもん	
\\	自転車[じてんしゃ]	
\\	《口》 
\\	に対して〕 
\\	自転車[じてんしゃ]に 乗[の]る。	〔またがる〕 
\\	[《文》 
\\	〔乗って走る〕 
\\	自転車	じてんしゃ	自転車	じてんしゃ	
\\	自動車[じどうしゃ]	
\\	《口》 
\\	[⇒くるま 
\\	自動車[じどうしゃ]の 安全性[あんぜんせい]。	
\\	自動車	じどうしゃ	自動車	じどうしゃ	
\\	死ぬ[しぬ]	
\\	〔命がやむ〕 
\\	《口》 
\\	〔息を引き取る〕 
\\	《文》 
\\	〔命を失う〕 
\\	《文》 
\\	〔命を捨てる〕 
\\	〔一生を終わる〕 
\\	〔他界する〕 
\\	《文》 
\\	〔婉曲的に〕 《口》 
\\	《文》 
\\	〔勢いがなくなる〕 
\\	《口》 
\\	〔本来の力が発揮されない〕 
\\	【野球】 〔アウトになる〕 
\\	【囲碁】 〔相手に囲まれて石を取られる〕 
\\	人間[にんげん]はいつかは 死[し]ぬのだ。	
\\	死ぬ	しぬ	死ぬ	しぬ	
\\	字引[じびき]	
\\	字引	じびき	字引	じびき	
\\	自分[じぶん]	
\\	〔その人自身〕 
\\	〔私〕 
\\	自分[じぶん]より 弱い者[よわいもの]には 優[やさ]しくしなさい。	
\\	自分	じぶん	自分	じぶん	
\\	閉まる[しまる]	
\\	〔開き戸が〕 
\\	(音を立てて) 
\\	〔引き戸が〕 
\\	〔かぎが〕 
\\	〔業務が終わる; 休業する〕 
\\	雨戸[あまど]が 閉[し]まっている。	
\\	閉まる	しまる	閉まる	しまる	
\\	締める[しめる]	
\\	〔緩みをなくす〕 (ひねって) 
\\	(引っ張って) 
\\	〔装着する〕 
\\	〔緊張させる〕 
\\	〔合計する〕 
\\	《口》 
\\	[⇒しめて] 
\\	〔物事の決着を祝って手打ちをする〕 
\\	[⇒てじめ] 
\\	〔切り詰める〕 
\\	【料理】 
\\	シートベルトをお 締[し]めください。	
\\	締める	しめる	締める	しめる	
\\	閉める[しめる]	
\\	〔戸を〕 
\\	(引き戸を) 
\\	(開き戸を) 
\\	〔かぎを〕 
\\	〔一日の業務を終える〕 
\\	〔休業する・廃業する〕 
\\	カーテンを 閉[し]める。	
\\	閉める	しめる	閉める	しめる	
\\	じゃ(あ)[じゃ(あ)]	〔'では
\\	のなまり〕 
\\	[⇒-では]	「それは 困[こま]ります」「じゃあどうしろ って言[ってい]うのかね」。	
\\	じゃ(あ)	じゃ(あ)	じゃ	じゃ	
\\	写真[しゃしん]	
\\	容疑者[ようぎしゃ]の 写真[しゃしん]。	
\\	写真	しゃしん	写真	しゃしん	
\\	シャツ[シャツ]	
\\	〔下着の〕 
\\	(袖なしの) 
\\	絹[きぬ]のシャツ。	
\\	シャツ	シャツ	シャツ	シャツ	
\\	十[じゅう]	
\\	【商】 
\\	〔ローマ数字〕 
\\	今年[ことし]の 十[じゅう] 大[だい]ニュース。	
\\	十	じゅう	十	じゅう	
\\	十一月[じゅういちがつ]	
\\	(略: 
\\	日本[にっぽん]の 11月[じゅういちがつ]は 全国的[ぜんこくてき]に よい天気[よいてんき]です。	
\\	十一月	じゅういちがつ	十一月	じゅういちがつ	
\\	十月[じゅうがつ]	
\\	(略: 
\\	年[ねん] 10月[じゅうがつ]28 日[にち]までに、この 請求書[せいきゅうしょ] に対[にたい]するお 支払[しはら]いを お願[おねが]いします。	
\\	十月	じゅうがつ	十月	じゅうがつ	
\\	十二月[じゅうにがつ]	
\\	(略: 
\\	この 提案[ていあん]は、27の 企業[きぎょう]から40 人[にん]が 参加[さんか]した 12月[じゅうにがつ]17 日[にち]のDCAの 電話会議[でんわかいぎ]の 結果[けっか]によるものです。	
\\	十二月	じゅうにがつ	十二月	じゅうにがつ	
\\	授業[じゅぎょう]	〔教え授けること〕 
\\	〔授業の内容〕 
\\	~する 
\\	先生[せんせい]の 授業[じゅぎょう]を 受[う]ける。	
\\	授業	じゅぎょう	授業	じゅぎょう	
\\	宿題[しゅくだい]	
\\	〔学校の〕 
\\	〔未解決問題〕 
\\	宿題[しゅくだい]をみてやる。	
\\	宿題	しゅくだい	宿題	しゅくだい	
\\	上手[じょうず]	
\\	〔名人〕 
\\	《口》 
\\	~な 〔巧みな〕 
\\	〔熟練した〕 
\\	〔お世辞〕 =おじょうず 
\\	料理[りょうり]
\\	絵[え]}の 上手[じょうず]な 人[ひと]。	
\\	上手	じょうず	上手	じょうず	
\\	丈夫[じょうぶ]	~な 
\\	〔壮健な〕 
\\	〔老人が〕 
\\	祖父[そふ]は 丈夫[じょうぶ]で 長生[ながい]きした。	
\\	丈夫	じょうぶ	丈夫	じょうぶ	
\\	醤油[しょうゆ]	
\\	同社[どうしゃ]は 醤油[しょうゆ] その他[そのた]の 食品[しょくひん]を 生産[せいさん]する。	
\\	醤油	しょうゆ	醤油	しょうゆ	
\\	食堂[しょくどう]	
\\	(軍隊・工場などの) 
\\	(駅や列車内の) 
\\	(修道院・大学などの) 
\\	〔店〕 
\\	食堂[しょくどう]は5 時[じ]にあく。	
\\	食堂	しょくどう	食堂	しょくどう	
\\	知る[しる]	
\\	〔認識する・知識や情報などを持つ〕 
\\	〔理解する〕 
\\	〔認める・悟る〕 
\\	〔知り合いである〕 
\\	〔知り合いになる〕 
\\	〔気づく〕 
\\	〔発見する〕 
\\	〔推測する〕 
\\	〔感じる〕 
\\	〔覚えている〕 
\\	〔なじみのある〕 
\\	〔経験して覚える〕 
\\	〔関わる〕 
\\	新聞[しんぶん]で 君[きみ]の 受賞[じゅしょう]を 知[し]った。	
\\	知る	しる	知る	しる	
\\	白い[しろい]	
\\	〔色の〕 
\\	〔皮膚の〕 
\\	〔頭髪の〕 
\\	〔空白の〕 
\\	〔汚れていない〕 
\\	〔罪がない〕 
\\	色[いろ]が 白[しろ]い。	〔肌の〕 
\\	白い	しろい	白い	しろい	
\\	新聞[しんぶん]	
\\	〈集合的に〉 
\\	新聞[しんぶん]で 見[み]る。	
\\	新聞	しんぶん	新聞	しんぶん	
\\	水曜日[すいようび]	
\\	(略: 
\\	今度[こんど]の 水曜日[すいようび]ならいいですよ。	
\\	水曜日	すいようび	水曜日	すいようび	
\\	吸う[すう]	
\\	〔鼻や口から体内に入れる〕 
\\	〔すする〕 
\\	〔吸収する〕 
\\	〔キスをする〕 
\\	その 火事[かじ]で 煙[けむり]を 吸[す]って3 人[にん]が 死亡[しぼう]した。	
\\	吸う	すう	吸う	すう	
\\	スカート[スカート]	
\\	ゆったりしたスカート。	
\\	スカート	スカート	スカート	スカート	
\\	好き[すき]	
\\	〔心引かれること〕 
\\	~な 
\\	〔物好き〕 
\\	〔思うまま, 勝手気まま〕 
\\	あの人[あのひと]は 甘[あま]いものが 好[す]きなんです。	
\\	好き	すき	好き	すき	
\\	直ぐに[すぐに]	
\\	トムは 映画界[えいがかい]にデビューするとすぐに 10代[じゅうだい]の 若者[わかもの]の 間[ま]で 人気[にんき]が 出[で]た。	
\\	直ぐに	すぐに	すぐに	すぐに	
\\	少し[すこし]	
\\	〔量〕 
\\	〔数〕 
\\	〔度合い〕 
\\	〔時間〕 
\\	〔距離〕 
\\	ほんの少[ほんのすこ]し。	
\\	少し	すこし	少し	すこし	
\\	涼しい[すずしい]	
\\	ひと 風呂[ふろ] 浴[あ]びると 涼[すず]しくなるよ。	
\\	涼しい	すずしい	涼しい	すずしい	
\\	ストーブ[ストーブ]	
\\	〔薪・石炭の〕 
\\	〔炉〕 
\\	ストーブの 燃料[ねんりょう]。	
\\	ストーブ	ストーブ	ストーブ	ストーブ	
\\	スプーン[スプーン]	
\\	〔さじ〕 
\\	【ゴルフ】 
\\	番ウッド〕 
\\	➡英語ではウッドの番号で呼ぶのが普通 (⇒ウッド1).	スプーンに 山盛[やまも]り 一[いち] 杯[はい]の 砂糖[さとう]。	
\\	スプーン	スプーン	スプーン	スプーン	
\\	スポーツ[スポーツ]	
\\	〔その 
\\	つ〕 
\\	見[み]る
\\	自分[じぶん]でやる}スポーツ。	
\\	スポーツ	スポーツ	スポーツ	スポーツ	
\\	ズボン[ズボン]	
\\	〔主として半ズボン〕 
\\	ズボンの 前[ぜん]ボタン。	
\\	ズボン	ズボン	ズボン	ズボン	
\\	住む・棲む[すむ]	
\\	〔居住する〕 
\\	[《雅》 
\\	その 当時[とうじ] 横浜[よこはま]の 外国人[がいこくじん]はたいてい 山の手[やまのて]に 住[す]んでいた。	
\\	住む・棲む	すむ	住む	すむ	
\\	スリッパ[スリッパ]	
\\	〔日本風のつっかけ式の〕 
\\	このスリッパに 履[は]きかえてください。	
\\	スリッパ	スリッパ	スリッパ	スリッパ	
\\	する[する]	
\\	〔行う〕 
\\	〔試みる〕 
\\	〔して遊ぶ〕 
\\	〔実行する〕 
\\	〔行動する〕 
\\	[⇒やる 
\\	-とする 
\\	させる1] 
\\	(「…をする」の形で) 〔ある仕事や役をする〕 
\\	〔…を勤める〕 
\\	〔従事する〕 
\\	(「…にする」の形で) 〔ある役・状態にならせる〕 
\\	〔…に変える〕 
\\	(「…にする」の形で) 〔ある用にあてる〕 
\\	(「…にする」の形で) 〔…だと思う〕 
\\	(「…とする」の形で) 〔みなす・考える〕 
\\	[⇒-とする 
\\	(「…に[と]する」の形で) 〔決定する; 選択する〕 
\\	(「(仮に)…とする」の形で) 〔仮定する〕 
\\	[⇒-とする 
\\	(「…をする」の形で) 〔装身具を身につける〕 
\\	(「…(を)する」の形で) 〔ある形・状態である〕 
\\	(「…がする」の形で) 〔ある状態や動きが感じられる〕 
\\	〔時が経つ〕 
\\	〔それだけの値段である〕 
\\	コンピューターばかり 相手[あいて]にしていると 複雑[ふくざつ]な 人間関係[にんげんかんけい]を 回避[かいひ]するようになる。	
\\	する	する	する	する	
\\	座る[すわる]	〔腰かける〕 
\\	〔正座する〕 
\\	〔座・地位を占める〕 
\\	どしんとすわる。	
\\	座る	すわる	座る	すわる	
\\	セーター[セーター]	
\\	カシミヤのセーター。	
\\	セーター	セーター	セーター	セーター	
\\	背[せ]	
\\	〔背中〕 
\\	[⇒せなか 
\\	〔うしろ〕 
\\	〔身長〕 
\\	〔山の尾根〕 
\\	背[せ]に 腹[はら]はかえられない。	
\\	【諺】 
\\	【諺】.	背	せ	背	せい	
\\	生徒[せいと]	
\\	〔低年齢の〕 
\\	(男子) 
\\	(女子) 
\\	受持[うけも]ちの 生徒[せいと]。	
\\	生徒	せいと	生徒	せいと	
\\	石鹸[せっけん]	
\\	石鹸[せっけん]1 個[こ]。	
\\	石鹸	せっけん	石鹸	せっけん	
\\	背広[せびろ]	〔上着〕 
\\	〔上下〕 
\\	三つ揃[みつぞろ]いの 背広[せびろ]。	
\\	背広	せびろ	背広	せびろ	
\\	狭い[せまい]	〔幅のない〕 
\\	〔面積の小さい〕 
\\	〔範囲の小さい〕 
\\	〔窮屈な〕 
\\	狭[せま]い 活動範囲[かつどうはんい]。	
\\	狭い	せまい	狭い	せまい	
\\	ゼロ[ゼロ]	
\\	ゼロから 出発[しゅっぱつ]する。	
\\	ゼロ	ゼロ	ゼロ	ゼロ	
\\	千[せん]	
\\	〔ローマ数字〕 
\\	それらは 何千[なんぜん]という 数[かず]に 達[たっ]する。	
\\	千	せん	千	せん	
\\	先月[せんげつ]	
\\	〔商用文で〕 
\\	(略: 
\\	先月[せんげつ]5 日[にち]に。	
\\	先月	せんげつ	先月	せんげつ	
\\	先週[せんしゅう]	
\\	先週の今日[せんしゅうのきょう]。	
\\	先週	せんしゅう	先週	せんしゅう	
\\	先生[せんせい]	
\\	〔教師〕 
\\	〔技術指導者〕 
\\	〔医者〕 
\\	〔作家, 議員, 弁護士など〕	新任[しんにん]の 先生[せんせい]。	
\\	先生	せんせい	先生	せんせい	
\\	洗濯[せんたく]	
\\	~する 
\\	洗濯[せんたく]がきく。	
\\	洗濯	せんたく	洗濯	せんたく	
\\	全部[ぜんぶ]	
\\	〔名詞的用法で〕 
\\	《口》 
\\	〔副詞用法で〕 
\\	課程[かてい]の 全部[ぜんぶ]または 一部[いちぶ]。	
\\	全部	ぜんぶ	全部	ぜんぶ	
\\	そう[そう]	
\\	〔そのように〕 
\\	〔そのとおり〕 
\\	〔問いに答えて〕 
\\	《肯定文が後に続くとき》; 
\\	《否定文が後に続くとき》. 
\\	〔思いついたり考えたりする時の言葉〕 ⇒そうそう11 
\\	〔そんなに〕 
\\	〔(否定を伴って)たいして〕 
\\	あの人[あのひと]お 酒[さけ] 強[つよ]いのよね」「そうそう」。	
\\	《口》 
\\	そう	そう	そう	そう	
\\	掃除[そうじ]	
\\	〔はき掃除〕 
\\	〔ふき掃除〕 (机のほこりなどの) 
\\	(こすり洗い) 
\\	〔街路の〕 
\\	~する 
\\	室内[しつない]を 掃除[そうじ]する。	
\\	掃除	そうじ	掃除	そうじ	
\\	そうして[そうして]	〔そして〕 
\\	〔そうやって〕 
\\	[⇒そして]	彼[かれ]はだまって 頭[あたま]を 下[さ]げた.そうして 静[しず]かに 出[で]て 行[い]った。	
\\	そうして	そうして	そうして	そうして	
\\	そこ[そこ]	
\\	〔その場所〕 
\\	〔その時・その場面〕 
\\	〔それ・そのこと〕 
\\	おいそこにいるのはだれだ。	
\\	そこ	そこ	そこ	そこ	
\\	そちら[そちら]	
\\	〔相手のいるところ〕 
\\	[⇒そっち 
\\	〔相手側の方向〕 
\\	〔相手の近く(にある物)〕 
\\	〔あなた〕 
\\	〔あなたの側〕 
\\	〔その方〕 
\\	今[いま]からそちらへ 伺[うかが]います。	
\\	そちら	そちら	そちら	そちら	
\\	外[そと]	
\\	〔外側〕 
\\	〔屋外〕 
\\	〔外部・よそ〕 
\\	〔外面〕 
\\	家の外[いえのそと]で。	
\\	外	そと	外	そと	
\\	その[その]	
\\	〔前述の〕 
\\	〔強めて〕 
\\	〔それの〕 
\\	列挙[れっきょ]して〕その 一[いち]。	
\\	その	その	その	その	
\\	側[そば]	
\\	〔かたわら〕 
\\	〔付近〕 
\\	そばに 寄[よ]るな。	
\\	側	そば	そば	そば	
\\	空[そら]	
\\	〔天〕 
\\	《雅》 
\\	《詩・雅》 
\\	〔上空〕 
\\	〔空中〕 
\\	〔空模様〕 
\\	〔遠く離れた場所〕 
\\	〔参照するものがないこと〕 
\\	〔心が向いていないようす〕 
\\	〔偽り〕	空[そら]を 眺[なが]める。	
\\	空	そら	空	そら	
\\	それ[それ]	
\\	〔相手に近いもの〕 
\\	〔その時〕 
\\	〔そのこと〕	ああそれそれ。	
\\	それ	それ	それ	それ	
\\	それから[それから]	
\\	〔そのあと〕 
\\	〔それ以来〕 
\\	〔その次に〕 
\\	〔そして〕 
\\	神戸[こうべ]まで 列車[れっしゃ]で 行[い]きそれからは 船[ふね]に 乗[の]った。	
\\	それから	それから	それから	それから	
\\	それでは[それでは]	〔そうなら〕 
\\	〔そういう状態では〕 
\\	〔話の切り出しや締めくくりに〕 
\\	それでは 彼女[かのじょ]がかわいそうですよ。	
\\	それでは	それでは	それでは	それでは	
\\	大学[だいがく]	〔総合〕 
\\	〔単科または分科〕 
\\	大学[だいがく]を 出[で]て 銀行[ぎんこう]に 勤[つと]めている。	
\\	大学	だいがく	大学	だいがく	
\\	大使館[たいしかん]	
\\	大使館[たいしかん] 付[つ]き 陸軍[りくぐん] 武官[ぶかん]。	
\\	大使館	たいしかん	大使館	たいしかん	
\\	大丈夫[だいじょうぶ]	
\\	~な 〔安心な〕 
\\	〔安全な〕 
\\	〔確実な〕 
\\	〔確実に; 確かに〕 
\\	(念を押して) 
\\	心配[しんぱい]しなくていいもう 大丈夫[だいじょうぶ]。	
\\	大丈夫	だいじょうぶ	大丈夫	だいじょうぶ	
\\	大好き[だいすき]	⇒すき3.	…が 大好[だいす]きである。	
\\	《口》 
\\	大好き	だいすき	大好き	だいすき	
\\	大切[たいせつ]	~な 〔重要な〕 
\\	〔貴重な〕 
\\	[⇒だいじ3]	大切[たいせつ]に 育[そだ]ててきたウサギ。	
\\	大切	たいせつ	大切	たいせつ	
\\	大抵[たいてい]	
\\	~の 〔大部分の〕 
\\	〔ほとんどの〕 
\\	~の[な] 〔一通りの〕 
\\	〔通常の〕 
\\	〔(副詞的に) 大部分〕 
\\	〔通常〕 
\\	〔ほとんどいつも〕 
\\	〔恐らく〕 
\\	たいていの 日本人[にっぽんじん]。	
\\	大抵	たいてい	たいてい	たいてい	
\\	台所[だいどころ]	
\\	〔経済事情〕 
\\	台所[だいどころ]の 流[なが]し。	
\\	台所	だいどころ	台所	だいどころ	
\\	大変[たいへん]	
\\	~な 〔深刻な〕 
\\	〔非常な〕 
\\	〔尋常でない〕 
\\	~な 〔ひどい〕 
\\	〔困難な〕 
\\	〔苦しい〕 
\\	〔(強意語として副詞的に)非常に〕 
\\	〔一大事〕 
\\	大変[たいへん] 大変[たいへん]! お鍋[おなべ]がふきこぼれてる!	
\\	大変	たいへん	大変	たいへん	
\\	高い[たかい]	
\\	〔背丈などが〕 
\\	〔位置・場所などが〕 
\\	〔金額・価格などが〕 
\\	〔数値が大きい〕 
\\	〔程度が〕 
\\	〔優れている〕 
\\	〔高貴な〕 
\\	〔地位・等級などが〕 
\\	〔音が〕 
\\	〔声がよく通る〕 
\\	〔知名度が〕 
\\	土地[とち]は 母屋[もや]に 向[む]かって 高[たか]くなっている。	
\\	高い	たかい	高い	たかい	
\\	沢山[たくさん]	
\\	~の[な] 〔多量の〕 
\\	《口》 
\\	〔多数の〕 
\\	《口》 
\\	〔大量に〕 
\\	〔数多く〕 
\\	〔十分〕 
\\	あまりたくさんの 時間[じかん]は 残[のこ]っていない。	
\\	沢山	たくさん	沢山	たくさん	
\\	タクシー[タクシー]	
\\	タクシーに 乗[の]る。	
\\	タクシー	タクシー	タクシー	タクシー	
\\	出す[だす]	
\\	〔外へ移す〕 
\\	〔取り出す〕 
\\	〔解放する〕 
\\	〔放出する; 発する〕 
\\	〔表面化させる〕 
\\	(見せる) 
\\	〔表明する〕 
\\	〔食物などを供する〕 
\\	〔突き出す〕 
\\	〔露出させる〕 
\\	〔旗や掲示を掲げる〕 
\\	〔修理などのために預ける〕 
\\	〔行かせる〕 
\\	〔卒業させる〕 
\\	〔参加させる〕 
\\	〔出演させる〕 
\\	〔登場させる〕 
\\	〔発送する〕 
\\	(郵便で) 
\\	〔提出する〕 
\\	〔話題にする〕 
\\	〔発売する〕 
\\	〔刊行する; 発行する〕 
\\	〔公表する〕 
\\	〔展示する; 出品する〕 
\\	〔出店する; 開店する〕 
\\	〔ものを与える〕 
\\	〔(食事などを)提供する〕 
\\	〔金を与える〕 
\\	〔運行する〕 
\\	〔発生させる〕 
\\	〔輩出する〕 
\\	〔結果をもたらす〕 
\\	〔数値・記録を示す〕 
\\	〔程度を増す〕 
\\	〔高さや長さをもたせる〕 
\\	〔応対させる〕 
\\	大声[おおごえ]を 出[だ]す。	
\\	出す	だす	出す	だす	
\\	立つ[たつ]	
\\	〔足で体を支える〕 
\\	〔立ち上がる・起きあがる〕 
\\	〔立場にいる〕 
\\	〔立場をとる〕 
\\	〔任務につく〕 
\\	【立つ・発つ】 〔出発する〕 
\\	《文》 
\\	〔その場を離れる〕 
\\	〔直立する〕 
\\	〔(矢・とげなどが) 突き刺さる〕 
\\	〔(犬の耳などが) ぴんと起きあがる〕 
\\	〔突き出ている〕 
\\	〔陰茎・乳首が〕 
\\	[《文》 
\\	〔発生する〕 
\\	〔始まる〕 
\\	【立つ・起つ】 〔行動を起こす〕 
\\	[⇒たちあがる 
\\	〔立候補する〕 
\\	〔あらわである・明示される〕 
\\	〔成り立つ・確かなものになる〕 
\\	〔通用する〕 
\\	〔独立する〕 
\\	〔割り算で〕 
\\	〔技能がある〕 
\\	〔戸・ふすまなどが閉まっている〕 
\\	15を5で 割[われ]ると3が 立[た]つ。	
\\	立つ	たつ	立つ	たつ	
\\	建物[たてもの]	
\\	(大きな) 
\\	〔総称〕 
\\	建物[たてもの]の 基礎[きそ]。	
\\	建物	たてもの	建物	たてもの	
\\	楽しい[たのしい]	
\\	明日[あした]は 楽[たの]しい 遠足[えんそく]だ。	
\\	楽しい	たのしい	楽しい	たのしい	
\\	頼む[たのむ]	
\\	〔依頼する・願う〕 
\\	〔懇願する〕 
\\	〔委託する〕 
\\	〔雇う〕 
\\	(弁護士などを) 
\\	(医者を) 
\\	〔注文する〕 
\\	〔あてにする・頼る〕 
\\	[《口》 
\\	〔…に訴える〕 
\\	手紙[てがみ]はすべて 新[あたら]しい 住所[じゅうしょ]へ 転送[てんそう]するように 郵便局[ゆうびんきょく]に 頼[たの]んでおいた。	
\\	頼む	たのむ	頼む	たのむ	
\\	煙草[たばこ]	
\\	《口》 
\\	〔たばこ 
\\	本〕 
\\	《口》 
\\	〔葉巻き〕 
\\	【植】 〔ナス科の 
\\	年草〕 
\\	たばこの 葉[は]。	
\\	煙草	たばこ	たばこ	たばこ	
\\	多分[たぶん]	〔おそらく〕 
\\	《口》 
\\	〔十中八九まで〕 
\\	は「たぶん」より可能性が低く「もしかしたら…, …かもしれない」に近い.	たぶんそうでしょう。	
\\	多分	たぶん	多分	たぶん	
\\	食べ物[たべもの]	
\\	《口》 
\\	《俗》 
\\	〔ホテル・下宿屋などの〕 
\\	〔料理〕 
\\	〔食糧〕 
\\	〔食用になる物〕 
\\	[=しょくもつ]	私[わたし]は 和食[わしょく] 党[とう]なので 海外[かいがい]ではいつも 食べ物[たべもの]に 困[こま]る。	
\\	食べ物	たべもの	食べ物	たべもの	
\\	食べる[たべる]	
\\	〔食する〕 
\\	《口》 
\\	〔動物の食性について〕 
\\	〔生活する〕 
\\	[⇒くう2 
\\	くわせる 
\\	それはバターをつけて 熱[あつ]いうちに 食[た]べるんだ。	
\\	食べる	たべる	食べる	たべる	
\\	卵[たまご]	
\\	〔魚・エビ・カニ・カエルなどの〕 
\\	(塊) 
\\	(粒) 
\\	(はらご) 
\\	〔貝, 主にカキの〕 
\\	〔ロブスター・蚕などの〕 
\\	〔シラミなどの〕 
\\	【玉子】 〔寿司の〕 
\\	〔一人前ではない者〕	アヒル{ウズラ}の 卵[たまご]。	
\\	卵	たまご	卵	たまご	
\\	誰[だれ]	
\\	〔知らない人を問う場合〕 
\\	〔特定できない人を指す場合〕 
\\	〔すべての人を指す場合〕	あらっだれかと 思[おも]ったら 雅子[まさこ]じゃない。	
\\	誰	だれ	誰	だれ	
\\	誕生日[たんじょうび]	
\\	歳[さい]の 誕生日[たんじょうび]を 迎[むか]える。	
\\	➡これで「満 
\\	歳」の意.	誕生日	たんじょうび	誕生日	たんじょうび	
\\	段々[だんだん]	〔徐々に〕 
\\	〔程度が増して〕 
\\	だんだん 夜[よる]が 明[あ]けてきた。	
\\	段々	だんだん	だんだん	だんだん	
\\	小さい[ちいさい]	
\\	〔形・サイズが〕 
\\	(矮小(わいしょう)な) 
\\	〔数量が〕 
\\	〔音が〕 
\\	〔声が〕 
\\	〔程度・規模が〕 
\\	〔些細な〕 
\\	〔度量が〕 
\\	〔年齢が〕 
\\	小[ちい]さい 家[いえ]。	
\\	小さい	ちいさい	小さい	ちいさい	
\\	近い[ちかい]	
\\	〔時間的に〕 
\\	〔距離的に〕 
\\	〔関係が〕 
\\	〔数量がほぼ同じ〕 
\\	〔性質・内容・程度などの隔たりが少ない〕 
\\	今日[きょう]はなぜかトイレが 近[ちか]い。	
\\	近い	ちかい	近い	ちかい	
\\	違う[ちがう]	
\\	〔差がある〕 (両者の間に) 
\\	〔異なる〕 
\\	〔別の〕 
\\	〔食い違う・一致しない〕 
\\	〔間違っている〕 
\\	〔正常でなくなる〕 
\\	大[おお]きさ
\\	形[がた]}が 違[ちが]う。	
\\	違う	ちがう	違う	ちがう	
\\	地下鉄[ちかてつ]	
\\	〔ロンドンの〕 
\\	〔パリの〕 
\\	地下鉄[ちかてつ]の 駅[えき]。	
\\	地下鉄	ちかてつ	地下鉄	ちかてつ	
\\	地図[ちず]	
\\	〔地図書〕 
\\	〔海図〕 
\\	〔都市などの〕 
\\	企業[きぎょう]の 統合[とうごう]や 合併[がっぺい]は 業界[ぎょうかい]の 地図[ちず]を 塗り替[ぬりか]える。	
\\	地図	ちず	地図	ちず	
\\	父[ちち]	
\\	〔男親〕 
\\	《口》 
\\	〔家畜などの〕 
\\	[⇒ちちおや] 
\\	〔創始者〕 
\\	父[ちち]も 父[ちち]なら 子[こ]も 子[こ]だ。	
\\	【諺】.	父	ちち	父	ちち	
\\	茶色[ちゃいろ]	
\\	〔薄い茶色〕 
\\	〔くすんだ茶色〕 
\\	茶色[ちゃいろ]の。	
\\	茶色	ちゃいろ	茶色	ちゃいろ	
\\	茶碗[ちゃわん]	〔食事用〕 
\\	〔湯飲み〕 
\\	飯[めし]を) 茶碗[ちゃわん]に 盛[も]る。	
\\	茶碗	ちゃわん	茶碗	ちゃわん	
\\	丁度[ちょうど]	
\\	〔きっちり・ぴったり〕 
\\	〔おりよく〕 
\\	〔まるで〕 
\\	〔あたかも〕 
\\	円[えん]ちょうどにしてくれない?	〔値段など〕 
\\	丁度	ちょうど	ちょうど	ちょうど	
\\	ちょっと[ちょっと]	
\\	〔時間が短いようす〕 
\\	〔量や程度がわずかなようす〕 
\\	〔軽い気持ちで行うようす〕 
\\	〔容易には〕 
\\	〔かなり〕 
\\	ほんのちょっとのところで 電車[でんしゃ]に 乗り遅[のりおく]れた。	
\\	ちょっと	ちょっと	ちょっと	ちょっと	
\\	使う[つかう]	
\\	〔道具を役立てる〕 
\\	〔操作する〕 
\\	〔手段として利用する〕 
\\	(文字を) 
\\	(言葉を) 
\\	〔方法として採用する〕 
\\	〔材料として活用する〕 
\\	〔体や頭を働かせる〕 
\\	《文》 
\\	〔金や時間を費やす〕 
\\	〔消費する〕 
\\	〔雇う〕 
\\	(女官・廷吏などを) 
\\	〔雇っている〕 
\\	〔特殊技能などを用いる〕 
\\	〔働かせる〕 
\\	(仕事をあてる) 
\\	…を 使[つか]って。	
\\	使う	つかう	使う	つかう	
\\	疲れる[つかれる]	
\\	〔心身が〕 
\\	〔品質などが〕 
\\	[⇒くたびれる 
\\	長時間[ちょうじかん]パソコンを 使[つか]うと 目[め]が 疲[つか]れる。	
\\	疲れる	つかれる	疲れる	つかれる	
\\	次[つぎ]	
\\	〔後続〕 
\\	〔次席の位〕 
\\	〔宿場〕 
\\	次[つぎ]に 近畿地方[きんきちほう]の 天気予報[てんきよほう]を お伝[おつた]えします。	
\\	次	つぎ	次	つぎ	
\\	着く[つく]	
\\	〔目的地に至る〕 
\\	〔配達される〕 
\\	〔届いて触れる〕 
\\	〔位置を占める〕.	ロンドンには 何[なん] 時[じ]に 着[つ]きますか。	
\\	《口》 
\\	着く	つく	着く	つく	
\\	机[つくえ]	
\\	片袖[かたそで]の 机[つくえ]。	
\\	机	つくえ	机	つくえ	
\\	作る・造る[つくる]	
\\	〔組み立てる〕 
\\	〔縫製する〕 
\\	〔あつらえる〕 
\\	〔調理・料理する〕 
\\	〔製造する〕 
\\	〔建造・建設する〕 
\\	〔運用のために生みだす〕 
\\	〔子をもうける〕 
\\	〔形成する〕 
\\	〔組織する〕 
\\	〔設立・建国する〕 
\\	(築き上げる) 
\\	〔創作する〕 
\\	〔製作する〕 
\\	〔模造する〕 
\\	〔でっちあげる〕 
\\	〔発生させる〕 
\\	(不本意ながら) 
\\	〔関係を生じさせる〕 
\\	〔作成する〕 
\\	〔編集する〕 
\\	〔育成する〕 
\\	〔養成する〕 
\\	〔化粧する〕 
\\	〔扮装する〕 
\\	〔わざとよそおう〕 
\\	〔整える〕 
\\	〔都合する〕 
\\	〔確保する〕 
\\	〔耕作・栽培する〕 
\\	〔醸造する〕 
\\	〔蒸留する〕 
\\	〔合成する〕 
\\	〔鋳造する〕 
\\	〔創造する〕 
\\	スーツを1 着[ちゃく]オーダーして 作[つく]った。	
\\	作る・造る	つくる	作る	つくる	
\\	点ける[つける]	〔点火する〕 
\\	〔稼働させる〕 
\\	導火線[どうかせん]に 火[ひ]をつける。	
\\	点ける	つける	点ける	つける	
\\	勤める・務める[つとめる]	
\\	【勤める】 〔雇われて働く〕 
\\	【勤める】 〔勤行する〕 
\\	〔仏事を営む〕 
\\	【務める】 〔役目をする〕 
\\	パートタイムで 勤[つと]める。	
\\	勤める・務める	つとめる	勤める	つとめる	
\\	詰らない[つまらない]	
\\	その 女優[じょゆう]はつまらぬ 芝居[しばい]を 活気[かっき]づけた。	
\\	詰らない	つまらない	詰らない	つまらない	
\\	冷たい[つめたい]	
\\	〔冷淡な〕 
\\	冬[ふゆ]の 張[は]りつめた) 冷[つめ]たい 空気[くうき]。	
\\	冷たい	つめたい	冷たい	つめたい	
\\	強い[つよい]	
\\	〔能力が高い〕 
\\	〔優位な〕 
\\	〔得手である〕 
\\	〔耐久性がある〕 
\\	〔頑健である〕 
\\	[⇒つよくする] 
\\	〔程度が激しい〕 
\\	〔程度が高い〕 
\\	〔作用が大きい〕 
\\	〔頼もしい〕 
\\	〔勇ましい〕 
\\	〔断固とした〕 
\\	〔力のこもった〕 
\\	〔明確な〕 ⇒つよく, つよくする.	強[つよ]いチーム。	
\\	強い	つよい	強い	つよい	
\\	手[て]	
\\	〔上肢〕 
\\	(腕) 
\\	(猫などの) 
\\	寒[さむ]くて 手[て]の 感[かん]じがまるでない。	
\\	手	て	手	て	
\\	テープ[テープ]	
\\	〔薄くて細長い帯状のもの〕 
\\	〔粘着テープ〕 
\\	〔歓迎用にビルの窓などから投げる〕 
\\	【陸上競技】 〔ゴールの〕 
\\	〔開通式などで使うもの〕 
\\	〔出港する船や舞台に投げるもの〕 
\\	〔録音用の〕 
\\	〔録画用の〕 
\\	強風[きょうふう]に 備[そな]えて 窓[まど]ガラスにテープを 貼[は]る。	
\\	テープ	テープ	テープ	テープ	
\\	テーブル[テーブル]	
\\	〔卓〕 
\\	〔食卓〕 
\\	【電算】 〔表〕 
\\	食事[しょくじ]をするためにテーブルにつく。	
\\	テーブル	テーブル	テーブル	テーブル	
\\	テープ・レコーダー[テープ・レコーダー]	
\\	テープ・レコーダーで 録音[ろくおん]する。	
\\	テープ・レコーダー	テープ・レコーダー	テープレコーダー	テープレコーダー	
\\	出掛ける[でかける]	
\\	〔外出する〕 
\\	〔出発する〕 
\\	《口》 
\\	〔出ようとする〕 
\\	たまにはこちらへも お出[おで]かけください。	
\\	出掛ける	でかける	出かける	でかける	
\\	手紙[てがみ]	
\\	〔挨拶や用事を書いて送る文書〕 
\\	《文》 
\\	〔短信〕 
\\	〈集合的に〉 
\\	〔封書〕 
\\	〔葉書などでなく〕 《口》 
\\	《文》 
\\	金[きん]を 送[おく]ってくれと 手紙[てがみ]で 頼[たの]む。	
\\	手紙	てがみ	手紙	てがみ	
\\	出来る[できる]	
\\	〔新たに発生する〕 
\\	〔形成される〕 
\\	〔新たに作られる〕 
\\	〔組織される〕 
\\	〔設けられる〕 
\\	〔物が完成・完了する〕 
\\	〔用意が〕 
\\	〔生産される〕 
\\	〔実る〕 
\\	〔生育する〕 
\\	〔…を材料として作られる; …であるように作られる〕 
\\	〔(「できている」の形で)本来…するような性質をもっている〕 
\\	〔可能である・能力がある〕 〈人が主語〉 
\\	〈事が主語〉 
\\	〔(…しても)差し支えない〕 
\\	〔成績や能力がすぐれている〕 
\\	〔人格円満だ〕 
\\	〔(男女が)深い関係になる〕 
\\	【相場】 〔売買が成立する〕 
\\	雨[あめ]で 道[みち]の 所々[ところどころ]に 水[みず]たまりができた。	
\\	出来る	できる	出来る	できる	
\\	出口[でぐち]	
\\	〔入口に対する〕 
\\	〔排出口〕 
\\	出口[でぐち]はこちらです。	
\\	出口	でぐち	出口	でぐち	
\\	テスト[テスト]	
\\	〔学力・技能などの程度を知るための試験〕 
\\	(小テスト) 
\\	[⇒しけん3] ~する 
\\	〔性能の確認などのための試み〕 
\\	~する 
\\	テストを 受[う]ける。	
\\	テスト	テスト	テスト	テスト	
\\	では[では]	
\\	〔前言の状況を受けて〕 
\\	〔話に区切りをつけたり, 挨拶の初めに〕 
\\	入学[にゅうがく] 祝[いわ]いちょうだい」「では 合格[ごうかく]したんだね」。	
\\	では	では	では	では	
\\	デパート(メントストア)[デパート(メントストア)]	〔百貨店〕 
\\	デパートの 試食[ししょく]コーナーへ 行[い]くと、いろいろな 物[もの]がただで 食[た]べられるよ。	
\\	デパート(メントストア)	デパート(メントストア)	デパート	デパート	
\\	でも[でも]	
\\	〔だが〕 
\\	〔弁解の前置き〕 
\\	私[わたし]は 何[なに]の 力[ちから]もないでも 私[わたし]に 話[はな]してほしかった。	
\\	でも	でも	でも	でも	
\\	テレビ[テレビ]	
\\	(略: 
\\	〔受像機〕 
\\	*《口》 
\\	《口》 
\\	テレビ 出現[しゅつげん] 以前[いぜん]の。	
\\	テレビ	テレビ	テレビ	テレビ	
\\	天気[てんき]	
\\	〔空模様〕 
\\	〔晴天〕 
\\	こんなに 天気[てんき]の 悪[わる]い 晩[ばん]に。	
\\	天気	てんき	天気	てんき	
\\	電気[でんき]	
\\	〔電気・電力〕 
\\	〔電灯・電気照明〕 
\\	僻地[へきち]などに) 電気[でんき]を 引[ひ]く。	
\\	電気	でんき	電気	でんき	
\\	電車[でんしゃ]	
\\	電車[でんしゃ]は 下田[しもだ]までです。	
\\	電車	でんしゃ	電車	でんしゃ	
\\	電話[でんわ]	
\\	〔電話機〕 
\\	《口》 
\\	《口》 
\\	=でんわかいせん. 
\\	〔電話機で話すこと; その通話〕 
\\	~する 
\\	この 村[むら]には 電話[でんわ]が 通[つう]じていない。	
\\	電話	でんわ	電話	でんわ	
\\	戸[と]	
\\	〔引き戸〕 
\\	〔開き戸〕 
\\	(両開きの) 
\\	〔窓の〕 
\\	戸[と]が 開[ひら]かなくなってしまった。	
\\	戸	と	戸	と	
\\	ドア[ドア]	
\\	ドアは 強[つよ]く 押[お]すと 開[ひら]いた。	
\\	ドア	ドア	ドア	ドア	
\\	トイレ[トイレ]	〔化粧室・便所〕 
\\	〔船の〕 
\\	トイレで 新聞[しんぶん]を 読[よ]む。	
\\	トイレ	トイレ	トイレ	トイレ	
\\	どう[どう]	[⇒どういう, どうか, どうした, どうして, どうでも, どうにか, どうにも(こうにも), どうのこうの, どうみても, どうやら, 
\\	〔状態や意見をたずねる〕 
\\	〔勧める・誘う・提案する〕 
\\	〔疑問を表す〕 
\\	〔安否などを気遣う〕 
\\	〔どのようであるかを特定しないで言う〕	その 小説[しょうせつ]はどうでしたか。	
\\	どう	どう	どう	どう	
\\	どうして[どうして]	
\\	〔どのようにして〕 
\\	〔なぜ〕 
\\	〔それどころか〕 
\\	《口》 
\\	〔(反語で) どうしても…ない〕 
\\	これはどうしてするのですか。	
\\	どうして	どうして	どうして	どうして	
\\	どうぞ[どうぞ]	
\\	〔是非〕 
\\	どうぞ。	〔物を渡して〕 
\\	【通信】 
\\	どうぞ	どうぞ	どうぞ	どうぞ	
\\	動物[どうぶつ]	
\\	〔獣〕 
\\	(一地方分布の) 
\\	〈総称的に〉 
\\	動物[どうぶつ]に 似[に]た。	
\\	動物	どうぶつ	動物	どうぶつ	
\\	どうも[どうも]	
\\	〔どうしても〕 
\\	〔どうやら〕 
\\	〔見たところでは〕 
\\	〔なんとなく〕 
\\	〔強調〕 
\\	〔あいさつ〕	何度[なんど]もやってるんだがどうもうまくいかない。	
\\	どうも	どうも	どうも	どうも	
\\	遠い[とおい]	
\\	〔時間的に〕 
\\	〔距離的に〕 
\\	〔関係が〕 
\\	〔その段階まで至っていない〕 
\\	〔感覚が〕	遠い昔[とおいむかし]にさかのぼって。	
\\	遠い	とおい	遠い	とおい	
\\	十日[とおか]	〔日付〕 
\\	〔日数〕 
\\	三月[さんがつ] 十[じゅう] 日[にち]。	
\\	十日	とおか	十日	とおか	
\\	時々[ときどき]	
\\	〔時として〕 
\\	〔その時々〕 
\\	わけもなく 憂鬱[ゆううつ]になることが 時々[ときどき]ある。	
\\	時々	ときどき	時々	ときどき	
\\	時計[とけい]	
\\	〔掛け時計〕 
\\	〔置き時計〕 
\\	〔腕時計〕 
\\	〔懐中時計〕 
\\	〔航海用などの精密な〕 
\\	時計[とけい]が30 分[ふん] 進[すす]んでいた。	
\\	時計	とけい	時計	とけい	
\\	何処[どこ]	
\\	〔どの辺〕 
\\	どこへ 行[い]くのですか。	
\\	何処	どこ	どこ	どこ	
\\	所[ところ]	
\\	〔場所〕 
\\	(狭い) 
\\	(現場) 
\\	(所在地) 
\\	〔地方〕 
\\	〔余地〕 
\\	〔ある距離の地点〕 
\\	〔そば〕 
\\	〔住所〕 
\\	〔家庭〕 
\\	〔点〕 
\\	〔特徴〕 
\\	〔部分〕 
\\	〔一節〕 
\\	〔事態・状況〕 
\\	〔限り・範囲〕 
\\	〔時・時点〕 
\\	〔場面〕 
\\	〔場所柄〕 
\\	〔ある条件のもとでの結果を表す〕 
\\	〔…したら〕 
\\	いたる 所[ところ]に。	
\\	所	ところ	所	ところ	
\\	図書館[としょかん]	
\\	図書館[としょかん]を 利用[りよう]する。	
\\	図書館	としょかん	図書館	としょかん	
\\	どちら[どちら]	
\\	〔どこ〕 
\\	〔どの方向〕 
\\	〔(二者のうち) どのほう〕 
\\	〔どなた〕 
\\	どちらかで。	
\\	どちら	どちら	どちら	どちら	
\\	とても[とても]	
\\	〔どうしても〕 
\\	《俗》 
\\	〔まったく〕 
\\	〔たいへん〕 
\\	*《俗》 
\\	*《俗》 
\\	とてもかなわぬことだ。	
\\	とても	とても	とても	とても	
\\	どなた[どなた]	〔だれ〕 
\\	人[ひと]から 話[はな]しかけられて〕どなたでしたかね。	
\\	どなた	どなた	どなた	どなた	
\\	隣り[となり]	
\\	隣り	となり	隣り	となり	
\\	どの[どの]	
\\	最近[さいきん]はどの 映画[えいが]もつまらない。	
\\	どの	どの	どの	どの	
\\	飛ぶ[とぶ]	
\\	〔飛行する〕 (鳥が) 
\\	(チョウなどが) 
\\	(飛行機が) 
\\	〔空中を移動する〕 (ボールなどが) 
\\	〔急行する〕 
\\	〔逃亡する〕 
\\	〔飛沫になる〕 
\\	〔発せられる〕 
\\	〔途中が抜ける〕 
\\	〔切断される〕 
\\	〔外れて落ちる〕 
\\	〔薄れる〕 
\\	(色が) 
\\	(味が) 
\\	〔なくなる〕 
\\	(瞬時に) 
\\	〔攻撃がかかる〕 
\\	打球[だきゅう]が)よく 飛[と]んだなあ。	
\\	飛ぶ	とぶ	飛ぶ	とぶ	
\\	止まる・停まる[とまる]	
\\	〔停車する〕 
\\	〔停船する〕 
\\	〔稼働が停止する〕 
\\	〔途絶える〕 
\\	〔働かなくなる〕 
\\	〔止む〕 
\\	〔静まる〕 
\\	止[と]まれ。	⇒とまれ1.	止まる・停まる	とまる	止まる	とまる	
\\	友達[ともだち]	
\\	《口》 
\\	[=とも1 
\\	友[とも]だちが 多[おお]い
\\	少[すく]ない}。	
\\	友達	ともだち	友達	ともだち	
\\	土曜日[どようび]	
\\	(略: 
\\	私[わたし]たちは 土曜日[どようび]の 晩[ばん]にパーティーを 開[ひら]くつもりです。	
\\	土曜日	どようび	土曜日	どようび	
\\	鳥[とり]	
\\	〔鳥類〕 
\\	〔食用の〕 
\\	〈集合的に〉 
\\	〔鶏〕 (雌) 
\\	(雄) 
\\	〔鳥肉〕 
\\	飛[と]んでいる 鳥[とり]の 群[む]れ。	
\\	鳥	とり	鳥	とり	
\\	鶏肉[とりにく]	
\\	母[はは]はその 鶏肉[けいにく]を 私[わたし]たちに 切[き]って 分[わ]けてくれた。	
\\	鶏肉	とりにく	鶏肉	とりにく	
\\	取る[とる]	
\\	〔手に持つ〕 
\\	〔選ぶ〕 
\\	〔手に入れる〕 
\\	(取得する) 
\\	(獲得する) 
\\	〔確保する〕 
\\	〔奪う〕 
\\	〔なくす・取り除く〕 
\\	(しみなどを) 
\\	〔外す〕 
\\	〔記録する〕 
\\	(書いて) 
\\	〔必要とする〕 
\\	〔徴収する〕 
\\	〔解釈する〕 
\\	〔出前させる〕 
\\	〔注文する〕 
\\	〔定期購読する〕 
\\	〔定期購入する〕 
\\	〔迎え入れる〕 
\\	【数】	昼食[ちゅうしょく]に カツ丼[かつどん]を 取[と]る。	
\\	取る	とる	取る	とる	
\\	撮る・録る[とる]	
\\	テレビドラマを 撮[と]る。	
\\	撮る・録る	とる	撮る	とる	
\\	どれ[どれ]	
\\	どれがいいですか。	
\\	どれ	どれ	どれ	どれ	
\\	どんな[どんな]	
\\	〔どのような〕 
\\	〔(「どんな…でも」の形で) あらゆる状態・状況のもとで〕 
\\	〔状況を危ぶんでいる場合〕 
\\	〔相手のコメントなどを求める場合〕	宇宙[うちゅう]ステーションとはどんなものですか。	
\\	どんな	どんな	どんな	どんな	
\\	ナイフ[ナイフ]	
\\	〔折りたたみ式の〕 
\\	(大型の) ⇒ジャックナイフ; (小型の) 
\\	(ごく小型の) 
\\	〔食事用の〕 
\\	(魚肉用の) 
\\	(肉切り用の) 
\\	ナイフで 人[ひと]を 脅[おど]す。	
\\	ナイフ	ナイフ	ナイフ	ナイフ	
\\	長い・永い[ながい]	
\\	〔長引いた〕 
\\	長い間[ながいあいだ]の 争[あらそ]い。	
\\	長い・永い	ながい	長い	ながい	
\\	鳴く・啼く[なく]	
\\	オウムなどが 鳴[な]く。	
\\	鳴く・啼く	なく	鳴く	なく	
\\	夏[なつ]	
\\	夏[なつ]の 初[はじ]め
\\	終[お]わり}に。	
\\	夏	なつ	夏	なつ	
\\	夏休み[なつやすみ]	
\\	私[わたし]の 学校[がっこう]は7 月[つき]の21 日[にち]から 夏休[なつやす]みになる。	
\\	夏休み	なつやすみ	夏休み	なつやすみ	
\\	七つ[ななつ]	
\\	~の 〔数字〕 
\\	〔年齢〕 
\\	〔午前 
\\	時・午後 
\\	時〕 
\\	七[なな]つ 一[いち] 組[くみ]。	
\\	七つ	ななつ	七つ	ななつ	
\\	何[なに]	
\\	[⇒なん1] 
\\	〔既知のものをあいまいに指す〕 
\\	《口》 
\\	彼[かれ]は 何[なに]を 言[い]われてもじっと 辛抱[しんぼう]した。	
\\	何	なに	何	なに	
\\	七日[なのか]	〔日数〕 
\\	〔日付〕 
\\	6月[ろくがつ]7 日[にち]は 何時[いつ]がつごうがいいですか。	
\\	七日	なのか	七日	なのか	
\\	名前[なまえ]	
\\	〔姓に対する名〕 
\\	〔洗礼名〕 
\\	[⇒な1 
\\	名前[なまえ]だけは 知[し]っているが 大[たい]して 付き合[つきあ]いはない。	
\\	名前	なまえ	名前	なまえ	
\\	習う[ならう]	〔学習する〕 
\\	〔練習する〕 
\\	ピアノを 習[なら]う。	
\\	習う	ならう	習う	ならう	
\\	並ぶ[ならぶ]	
\\	〔整列する〕 
\\	(順番どおりに) 
\\	〔行列する〕 
\\	〔陳列される〕 
\\	〔隣り合う〕 
\\	〔追いつく〕 
\\	〔横一線になる〕 
\\	(競技で) 
\\	〔匹敵する〕 
\\	並[なら]ばせる。	〔整列〕 
\\	〔行列〕 
\\	並ぶ	ならぶ	並ぶ	ならぶ	
\\	並べる[ならべる]	
\\	〔順番どおりに〕 
\\	〔列に〕 
\\	〔平面に〕 
\\	〔陳列する〕 
\\	〔隣り合わせる〕 
\\	〔比較のために〕 
\\	〔列挙する〕 
\\	この 植木鉢[うえきばち]をベランダに 並[なら]べなさい。	
\\	並べる	ならべる	並べる	ならべる	
\\	為る[なる]	
\\	〔自然現象で〕 
\\	〔年齢が〕 
\\	〔(成り行きで)変わる〕 
\\	〔結果として〕 
\\	〔決まる〕 
\\	〔取り決め・規則などで〕 
\\	〔順番で〕 
\\	〔ある身分や関係に〕 
\\	〔ある時点や時期に〕 
\\	〔ある数量に〕 
\\	〔ある時間・年月が過ぎる〕 
\\	〔ある形状である〕 
\\	〔変換できる〕 
\\	〔役割を果たす〕 
\\	〔演じる〕 
\\	〔変身する〕 
\\	この 人形[にんぎょう]は 私[わたし]の お守[おまも]りになっている。	
\\	為る	なる	なる	なる	
\\	二[に]	
\\	〔ふたつ〕 
\\	〔ローマ数字〕 
\\	〔第二〕 
\\	二[に]の 太刀[たち]をあびせる。	
\\	二	に	二	に	
\\	二月[にがつ]	
\\	(略: 
\\	社[しゃ]のガードナー 氏[し]は、 2月[にがつ]20 日[にち]から 27日[にじゅうしちにち]までマリオットホテルに 滞在中[たいざいちゅう]で、ぜひあなたに 会[あ]いたいとのことです。	
\\	二月	にがつ	二月	にがつ	
\\	賑やか[にぎやか]	
\\	にぎやかなところが 好[す]きだ。	
\\	賑やか	にぎやか	賑やか	にぎやか	
\\	西[にし]	
\\	〔方角〕 
\\	(略: 
\\	【相撲】 
\\	西[にし]に。	〔西部に〕 
\\	〔西方に〕 
\\	〔境を接して西側に〕 
\\	西	にし	西	にし	
\\	日曜(日)[にちよう(び)]	
\\	(略: 
\\	【キリスト教】 
\\	日曜[にちよう]
\\	日[び])らしい。	
\\	日曜(日)	にちよう(び)	日曜日	にちようび	
\\	荷物[にもつ]	〔背負い荷・車馬の荷〕 
\\	〔手荷物〕 
\\	〔貨物〕 
\\	〔船荷〕 
\\	〔手回り品〕 
\\	〔郵送・宅配などの〕 
\\	[⇒に3, てにもつ, おにもつ]	荷物[にもつ]を 背負[せお]って。	
\\	荷物	にもつ	荷物	にもつ	
\\	ニュース[ニュース]	
\\	それはニュースだ。	
\\	ニュース	ニュース	ニュース	ニュース	
\\	庭[にわ]	
\\	〔家の敷地内の地面〕 
\\	〔あることをするための場〕 
\\	庭[にわ]でバーベキューをする。	
\\	庭	にわ	庭	にわ	
\\	脱ぐ[ぬぐ]	
\\	〔手早く〕 
\\	〔引っぱって〕 
\\	〔放り投げるように〕 
\\	〔裸になる〕 
\\	まだセーターを 脱[ぬ]ぐには 寒[さむ]い 季節[きせつ]だ。	
\\	脱ぐ	ぬぐ	脱ぐ	ぬぐ	
\\	ネクタイ[ネクタイ]	
\\	ネクタイを 解[と]く
\\	直[なお]す}。	
\\	ネクタイ	ネクタイ	ネクタイ	ネクタイ	
\\	寝る[ねる]	
\\	〔体を横たえる〕 
\\	〔眠る〕 
\\	〔仮眠する〕 
\\	〔熟睡する〕 
\\	《口》 
\\	[⇒ねむる] 
\\	〔寝床に入る〕 
\\	〔病床に就く〕 
\\	〔同衾(どうきん)する〕 
\\	〔横倒しになる〕 
\\	〔活用されずにいる〕 
\\	〔麹(こうじ)がほどよく熟成する〕 
\\	ベッドに 寝[ね]る。	
\\	寝る	ねる	寝る	ねる	
\\	-の[-の]	
\\	〔所有の関係〕 
\\	〔所属の関係〕 
\\	〔所在する場所・行為の場所; …にある, …に居る〕 
\\	〔時間の関係〕 
\\	〔…という形状・性質をしている〕 
\\	〔…で有名な〕 
\\	〔その材料でできている〕 
\\	〔…という内容である〕 
\\	〔数量・分量・順序〕 
\\	〔…に対する, …のための〕 
\\	〔…に関する〕 
\\	〔作者を表す場合; …による, …の作った〕 
\\	〔…からの〕 
\\	〔人と人の関係; …にあたる, …としての〕 
\\	〔動作・状態の主体を表わす場合〕 
\\	〔行為の対象を表わす場合〕 
\\	〔同格を表す場合〕 
\\	〔もの・こと〕 
\\	〔並べ立てるさま〕 
\\	〔文末に付ける場合〕	人[ひと]の 手[て]を 取[と]る。	
\\	-の	-の	の	の				
\\	ノート[ノート]	
\\	〔帳面〕 
\\	〔筆記したもの〕 
\\	~する 〔筆記する〕 
\\	〔記録としての覚え書き〕 
\\	〔注〕 
\\	【音楽】 〔音符・音〕 
\\	ノート3 冊[さつ]。	
\\	ノート	ノート	ノート	ノート	
\\	上る・登る[のぼる]	
\\	〔高い方へ進む〕 
\\	(川を) 
\\	(がけなどを) 
\\	段上に身を移す〕 
\\	〔上京する〕 
\\	〔京都へ行く〕 
\\	〔昇殿する〕 
\\	〔数量がそこまで達する〕 
\\	〔卓上に置かれる〕 
\\	〔採り上げられる〕 
\\	興奮[こうふん]してすっかり 頭[あたま]に 血[ち]が 上[のぼ]っていた。	
\\	上る・登る	のぼる	登る	のぼる	
\\	飲物[のみもの]	
\\	飲物	のみもの	飲物	のみもの	
\\	飲む・呑む[のむ]	
\\	〔(酒を除いて)水などを〕 
\\	(薬を常用する) 
\\	〔飲み下す〕 
\\	〔酒を飲む〕 
\\	《文》 
\\	[⇒のめる1 
\\	〔喫煙する〕 
\\	〔大きなものが小さなものを中に取り込む〕 
\\	[⇒のまれる 
\\	〔表に出さずこらえる〕 
\\	〔(提議などを)受諾する〕 
\\	〔軽視する〕 
\\	[⇒のまれる 
\\	〔隠し持つ〕 
\\	飲[の]めばやせる 薬[くすり]。	
\\	飲む・呑む	のむ	飲む	のむ	
\\	乗る[のる]	
\\	〔乗り込む〕 
\\	(船・航空機に) 
\\	《文》 
\\	〔上に身を置く〕 
\\	〔基準を超える〕 
\\	(上乗せされる) 
\\	〔従う〕 
\\	(運ばれる) 
\\	〔調子づく〕 
\\	〔反応する〕 
\\	〔相手になる〕 
\\	〔参加する〕 
\\	(だまされて) 
\\	〔なじむ〕 
\\	〔化粧品や脂などのつき具合〕 
\\	名古屋[なごや]から 高速道路[こうそくどうろ]に 乗[の]る。	
\\	乗る	のる	乗る	のる	
\\	歯[は]	
\\	〔口の中の〕 
\\	《口》 
\\	《俗》 
\\	〔器具・機械の〕 
\\	〔下駄の〕 
\\	歯[は]が 欠[か]ける。	
\\	歯	は	歯	は	
\\	灰皿[はいざら]	
\\	〔スタンドつきの〕 
\\	灰皿[はいざら]でたばこの 火[ひ]を もみ消[もみけ]す。	
\\	灰皿	はいざら	灰皿	はいざら	
\\	入る[はいる]	
\\	〔外から中へ移動する〕 
\\	〔中に収められる〕 
\\	〔収容する〕 
\\	〔仲間に加わる〕 
\\	〔入会する〕 
\\	〔会社などに〕 
\\	〔入学する〕 
\\	〔施設などに〕 
\\	〔含む〕 
\\	〔含まれる〕 
\\	〔加えられる〕 
\\	〔割り込む〕 
\\	〔導入される〕 
\\	〔設備される〕 
\\	〔自分のものになる〕 
\\	〔収入がある〕 
\\	〔入荷する〕 
\\	〔知覚される〕 
\\	〔気持ちや力がこもる〕 
\\	〔得点する〕 
\\	〔お茶などの用意ができる〕 
\\	〔機械や道具が作動する〕 
\\	〔ある時期になる〕 
\\	〔始まる〕 
\\	〔ある状態・段階になる〕 
\\	列車[れっしゃ]がホームに 入[はい]ってきた。	
\\	入る	はいる	入る	はいる	
\\	葉書[はがき]	(官製) 
\\	(私製) 
\\	(官製・私製) 
\\	葉書[はがき]で。	
\\	葉書	はがき	葉書	はがき	
\\	履く[はく]	
\\	〔履いている〕 
\\	履[は]いてみる。	
\\	履く	はく	履く	はく	
\\	箱[はこ]	
\\	〔木・紙・金属の〕 
\\	(小箱) 
\\	(大箱) 
\\	(畢 (ひつ)) 
\\	(貴重品を入れる) 
\\	(たばこなどの) 
\\	〔三味線〕 
\\	〔客車の車両〕 
\\	〔エレベーターのケージ〕 
\\	〔(行政が作る)建築物〕 ⇒はこもの.	箱[はこ]のふた。	
\\	箱	はこ	箱	はこ	
\\	橋[はし]	
\\	橋[はし]をかける
\\	渡[わた]す}。	
\\	〔仲立ちをする〕 
\\	橋	はし	橋	はし	
\\	箸[はし]	
\\	彼[かれ]は 箸[はし]にも 棒[ぼう]にもかからない 人間[にんげん]だ。	
\\	箸	はし	箸	はし	
\\	始まる[はじまる]	
\\	〔その時期になる〕 
\\	〔生じる〕 
\\	〔起源がある〕 
\\	演奏[えんそう]が 始[はじ]まる。	
\\	始まる	はじまる	始まる	はじまる	
\\	始め・初め[はじめ]	
\\	【始め】 〔始めること〕 
\\	【初め】 〔一続きのものの最初〕 
\\	〔初期〕 
\\	【始め・初め】 〔物事を始めたばかりのころ〕 
\\	〔最初〕 
\\	【始め】 〔起源〕 
\\	【初め】 〔順序の最初〕 
\\	始[はじ]めの 言葉[ことば]。	〔開会の辞〕 
\\	始め・初め	はじめ	始め	はじめ	
\\	初めて[はじめて]	
\\	〔最初・第一〕 
\\	〔生まれて初めて, 第一回目に〕 
\\	〔その結果としてやっと〕 
\\	これが 彼[かれ]からの 初[はじ]めての 手紙[てがみ]もう20 年[ねん]も 昔[むかし]のものよ。	
\\	初めて	はじめて	初めて	はじめて	
\\	走る[はしる]	
\\	〔人や動物が疾走する〕 
\\	〔乗り物が進む〕 
\\	〔船が〕 
\\	(蒸気で) 
\\	(するすると) 
\\	(早く) 
\\	[⇒はしらせる 
\\	【電算】 〔プログラムが動作する〕 
\\	〔急行する〕 
\\	〔ものごとを実行する; どんどん進める; 先走る〕 ⇒はしりすぎる 
\\	〔好ましくない方向に傾く〕 
\\	〔出奔する. 逃亡する〕 
\\	〔敗走する〕 
\\	〔一瞬現れる〕 
\\	〔なめらかに動く〕 
\\	〔道などが伸びる〕 
\\	野球[やきゅう]で) 走[はし]りながら 球[たま]をさばく。	
\\	走る	はしる	走る	はしる	
\\	バス[バス]	
\\	〔輸送機関の一つ〕 
\\	【電算】 〔母線〕 
\\	「あっバスが 来[き]たよ」「やっと 来[き]たか」。	
\\	バス	バス	バス	バス	
\\	バター[バター]	
\\	たっぷりバターを 付[つ]ける。	
\\	バター	バター	バター	バター	
\\	二十(歳)[はたち]	
\\	二十歳[はたち] 代[だい]の 人[ひと]。	
\\	二十(歳)	はたち	二十歳	はたち	
\\	働く[はたらく]	
\\	〔労働する〕 
\\	〔労役する〕 
\\	〔奮闘する〕 
\\	〔勤務する〕 
\\	〔(悪事などを)行う〕 
\\	〔(器官などが)動く〕 
\\	〔(精神などが)活動する〕 
\\	〔(薬などが)作用する〕 
\\	こつこつと 働[はたら]く。	
\\	働く	はたらく	働く	はたらく	
\\	八[はち]	
\\	〔ローマ数字〕 
\\	八[はち] 倍[ばい](の)。	
\\	八	はち	八	はち	
\\	八月[はちがつ]	
\\	(略: 
\\	私[わたし]たちは 8月[はちがつ]にキャンプに 行[い]った。	
\\	八月	はちがつ	八月	はちがつ	
\\	二十日[はつか]	
\\	〔日数〕 
\\	〔第 
\\	日〕 
\\	卒業式[そつぎょうしき]は 三月[さんがつ] 二[に] 十[じゅう] 日[にち]に 行[おこな]われます。	
\\	二十日	はつか	二十日	はつか	
\\	花・華[はな]	
\\	【花】 〔草木の〕 (一般に, また特に観賞花) 
\\	〔果樹の〕 
\\	花[はな]をつける。	〔木や草が〕 
\\	花・華	はな	花	はな	
\\	鼻[はな]	〔人・動物の〕 
\\	《戯言》 
\\	〔豚などの〕 
\\	〔犬・馬などの口鼻〕 
\\	〔象の〕 
\\	犬[いぬ]の 狆[ちん](ちん)は 鼻[はな]がぺしゃんこだ。	
\\	鼻	はな	鼻	はな	
\\	話[はなし]	
\\	〔おしゃべり・会話〕 
\\	〔聴衆に向かってする話〕 (演説) 
\\	(講話) 
\\	(言説) 
\\	〔話題〕 
\\	〔用のある話; その内容〕 (相談) 
\\	(談判) 
\\	〔文学作品〕 (物語) 
\\	(寓話) 
\\	(挿話) 
\\	〔事実談〕 
\\	(おもしろくした) 
\\	〔作り話〕 
\\	〔うわさ〕 
\\	〔道理〕 
\\	〔事情〕 
\\	〔こと〕 
\\	しかけていた 話[はなし]を 途中[とちゅう]でやめる。	
\\	話	はなし	話	はなし	
\\	話す[はなす]	
\\	〔会話する〕 
\\	〔語る・告げる〕 
\\	*《口》 
\\	〔述べる〕 
\\	(詳しく) 
\\	〔外国語をしゃべる〕 
\\	彼[かれ]は 話[はな]してみると 結構[けっこう] 面白[おもしろ]いやつだ。	
\\	話す	はなす	話す	はなす	
\\	母[はは]	
\\	〔女親〕 
\\	《小児語》 
\\	《口》 
\\	〔獣類の〕 
\\	〔物事のもと〕 
\\	育[そだ]ての 母[はは]。	
\\	母	はは	母	はは	
\\	速い・早い[はやい]	
\\	【速い】 〔動く速度が大きい〕 
\\	【音楽】 
\\	【速い・早い】 〔時間がかからないさま〕 
\\	〔敏活な〕 
\\	【早い】 〔時間が朝だ〕 
\\	〔標準より前だ〕 
\\	【早い】 〔時期尚早だ〕 
\\	【早い】 〔手っ取り早い〕 
\\	速[はや]い 馬[うま]。	
\\	速い・早い	はやい	早い	はやい	
\\	速い・早い[はやい]	
\\	【速い】 〔動く速度が大きい〕 
\\	【音楽】 
\\	【速い・早い】 〔時間がかからないさま〕 
\\	〔敏活な〕 
\\	【早い】 〔時間が朝だ〕 
\\	〔標準より前だ〕 
\\	【早い】 〔時期尚早だ〕 
\\	【早い】 〔手っ取り早い〕 
\\	速[はや]い 馬[うま]。	
\\	速い・早い	はやい	速い	はやい	
\\	春[はる]	
\\	〔季節〕 
\\	〔新年〕 
\\	〔人生の盛んな時期〕 
\\	〔思春期〕 
\\	〔春情〕 
\\	彼女[かのじょ]は2003 年[ねん] 春[はる]3 月[つき]の 生[う]まれです。	
\\	春	はる	春	はる	
\\	張る[はる]	
\\	〔伸ばし広げる〕 
\\	〔布を〕 
\\	〔引き張る・渡す〕 
\\	〔強く引っ張る〕 
\\	〔表面に張りつける〕 
\\	(内張りする) 
\\	【張る・貼る】 〔糊などで貼りつける〕 (紙などを) 
\\	(膏薬などを) 
\\	〔障子などを〕 
\\	〔掲示する〕 
\\	〔氷などが表面を覆う〕 
\\	〔満たす〕 
\\	〔ふくれる〕 
\\	〔満ちる〕 
\\	〔緊張する〕 
\\	〔筋肉がこる〕 
\\	〔対抗する〕 
\\	〔突き出す〕 
\\	〔催す・開く〕 
\\	〔構える・経営する〕 
\\	〔地位にある〕 
\\	〔外見をよく見せる〕 
\\	〔押し通す〕 
\\	〔見張りをする〕 
\\	〔平手で打つ〕 
\\	【相撲】 
\\	【将棋】 〔持ち駒をある場所に置く〕 
\\	〔賭ける〕 
\\	〔値がかさむ〕 
\\	【数】	テントを 張[は]る。	
\\	張る	はる	張る	はる	
\\	晴れる[はれる]	
\\	〔消えて見通しがよくなる〕 
\\	(降り止んで) 
\\	〔すっきりする〕 ⇒はれて. 
\\	〔疑いが消える〕 
\\	晴[は]れて 温[あたた]かい 日[ひ]。	
\\	晴れる	はれる	晴れる	はれる	
\\	晩[ばん]	
\\	[⇒よる1]	明日[あした]の 晩[ばん]。	
\\	晩	ばん	晩	ばん	
\\	パン[パン]	
\\	〔命の糧(かて)〕 
\\	【教会】 〔聖餐(せいさん)式用〕 
\\	一切[ひとき]れのパン。	
\\	パン	パン	パン	パン	
\\	ハンカチ, ハンカチーフ[ハンカチ, ハンカチーフ]	
\\	《口》 
\\	胸[むね]ポケットにのぞかせるハンカチ。	
\\	ハンカチ, ハンカチーフ	ハンカチ, ハンカチーフ	ハンカチ	ハンカチ	
\\	番号[ばんごう]	
\\	番号[ばんごう]を 呼[よ]ぶ。	〔窓口などで〕 
\\	番号	ばんごう	番号	ばんごう	
\\	晩ご飯[ばんごはん]	
\\	晩ご飯	ばんごはん	晩ご飯	ばんごはん	
\\	半分[はんぶん]	
\\	【法】 (財産などの) 
\\	彼[かれ]と 半分[はんぶん]ずつ 出資[しゅっし]して 事業[じぎょう]をやった。	
\\	半分	はんぶん	半分	はんぶん	
\\	東[ひがし]	
\\	(略: 
\\	【相撲】 
\\	犬[いぬ]が 西[にし] 向[む]きゃ 尾[お]は 東[ひがし]。	⇒いぬ1.	東	ひがし	東	ひがし	
\\	引く[ひく]	
\\	〔引っ張る〕 
\\	〔手前に寄せる〕 
\\	〔誘導する〕 
\\	(人の手を) 
\\	(牛馬を) 
\\	〔引きずる〕 
\\	〔連結させて進む〕 
\\	〔線状に描く〕 
\\	〔音声を長く伸ばす〕 
\\	〔敷設する〕 
\\	〔塗り広げる〕 
\\	(油を) 
\\	(ワックスを) 
\\	〔引用する〕 
\\	〔選び出す〕 
\\	〔調べる〕 
\\	〔引き算する〕 
\\	〔差を求める〕 
\\	〔減額する〕 
\\	〔割引する〕 
\\	〔関心を向かせる〕 
\\	〔誘い込む〕 
\\	〔受け継ぐ〕 
\\	【日舞】 
\\	いすを 後[うし]ろに 引[ひ]く。	
\\	引く	ひく	引く	ひく	
\\	弾く[ひく]	
\\	その 少女[しょうじょ]は 大変[たいへん] 上手[じょうず]にギターを 弾[ひ]く。	
\\	弾く	ひく	弾く	ひく	
\\	低い[ひくい]	
\\	〔位置が下の方にある〕 
\\	〔垂直方向に短い〕 
\\	〔数値が小さい〕 
\\	〔程度が下である〕 
\\	〔劣っている〕 
\\	〔卑しい〕 
\\	〔音階が下のほうである〕 
\\	〔音量が小さい〕 
\\	腰[こし]の 低[ひく]い。	
\\	[⇒こし1 
\\	低い	ひくい	低い	ひくい	
\\	飛行機[ひこうき]	
\\	〔初期の〕 
\\	飛行機[ひこうき] 上[じょう]のながめ。	
\\	飛行機	ひこうき	飛行機	ひこうき	
\\	左[ひだり]	
\\	〔方向・位置〕 
\\	[⇒ひだりがわ, ひだりて 
\\	〔左の手〕 
\\	〔酒好き〕 
\\	〔思想的な左翼〕 
\\	〈集合的に〉 
\\	〔左派〕 
\\	[⇒ひだりより1 
\\	交番[こうばん]はその 出口[でぐち]を 出[で]てすぐ 左[ひだり]にあります。	
\\	左	ひだり	左	ひだり	
\\	一つ[ひとつ]	
\\	個〕 
\\	〔差がない〕 
\\	〔それだけ・次第〕 
\\	〔せめてこれだけ〕 
\\	〔最低 
\\	つさえも〕 
\\	〔一例〕 
\\	〔一種〕 
\\	〔一方〕 
\\	歳〕 
\\	〔ちょっと〕	石鹸[せっけん] 一[ひと]つ。	
\\	一つ	ひとつ	一つ	ひとつ	
\\	一人[ひとり]	〔人数〕 
\\	[⇒ひとり2]	涙[なみだ]を 流[なが]さなかった 者[もの]はひとりとしていなかった。	
\\	一人	ひとり	一人	ひとり	
\\	百[ひゃく]	
\\	〔ローマ数字〕 
\\	彼[かれ]の 弟子[でし]は 何[なん] 百[ひゃく] 人[にん]もいる。	
\\	百	ひゃく	百	ひゃく	
\\	病院[びょういん]	
\\	〔診療所〕 
\\	〔施設内の〕 
\\	急病人[きゅうびょうにん]など〕すぐ 病院[びょういん]に 連[つ]れていけ。	
\\	病院	びょういん	病院	びょういん	
\\	病気[びょうき]	
\\	〔疾病〕 
\\	《文》 
\\	〔軽症の〕 
\\	〔局部の〕 
\\	〔医療用語〕 
\\	〔悪い癖〕 
\\	〔改善しないこと〕 
\\	病気[びょうき]の 治[なお]りが 速[はや]い。	〔病気が〕 
\\	《文》 
\\	〔人が〕 
\\	病気	びょうき	病気	びょうき	
\\	平仮名[ひらがな]	
\\	[⇒かな]			平仮名	ひらがな	平仮名	ひらがな	
\\	昼[ひる]	
\\	〔日の出から日の入りまで〕 
\\	[⇒ひるま] 
\\	〔正午〕 
\\	《文》 
\\	〔昼休み〕 
\\	〔昼食〕 ⇒ちゅうしょく.	
\\	月[つき]は 昼[ひる]の 時間[じかん]が 長[なが]い。	
\\	昼	ひる	昼	ひる	
\\	昼ご飯[ひるごはん]	
\\	今[いま]は 外出[そとだ]しないで。もうすぐ 昼ご飯[ひるごはん]にするとこだから。	
\\	昼ご飯	ひるごはん	昼ご飯	ひるごはん	
\\	広い[ひろい]	
\\	〔面積が〕 
\\	(空間が) 
\\	〔幅・間隔が〕 
\\	〔広範囲の〕 
\\	[⇒ひろく2, ひろし] 
\\	〔寛大な〕 
\\	あの人[あのひと]は 広[ひろ]い 世界[せかい]に 友人[ゆうじん]も 親類[しんるい]もない。	
\\	広い	ひろい	広い	ひろい	
\\	プール[プール]	
\\	〔水泳の〕 
\\	〔合同資金〕 
\\	~する 
\\	プールで 泳[およ]ぐ。	
\\	プール	プール	プール	プール	
\\	フィルム[フィルム]	
\\	〔感光材料〕 
\\	〔映画〕 ⇒えいが1. 
\\	〔薄膜〕 
\\	不燃性[ふねんせい]フィルム。	
\\	フィルム	フィルム	フィルム	フィルム	
\\	封筒[ふうとう]	
\\	自分[じぶん]の 住所氏名[じゅうしょしめい]を 書[か]き 返信用[へんしんよう]の 切手[きって]を 貼[は]った 封筒[ふうとう]。	
\\	封筒	ふうとう	封筒	ふうとう	
\\	フォーク[フォーク]	
\\	切り盛[きりも]り 用[よう] 大型[おおがた]フォーク。	
\\	フォーク	フォーク	フォーク	フォーク	
\\	吹く[ふく]	
\\	〔風が〕 
\\	(静かに) 
\\	(一面に) 
\\	〔口で〕 
\\	(息を) 
\\	〔楽器などを吹き鳴らす〕 
\\	〔(金属を)精錬する〕 
\\	〔鋳造する〕 
\\	〔表面に出す, 吹き出す〕 
\\	〔自慢する〕 
\\	〔失笑する〕 ⇒ふきだす1 
\\	そよ風[そよかぜ]が 吹[ふ]く 気持[きも]ちのいい 日[ひ]。	
\\	吹く	ふく	吹く	ふく	
\\	服[ふく]	〔衣服〕 
\\	(主に女性・子供の) 
\\	そろいの) 
\\	〔服装〕 
\\	(特殊な時候・目的の) 
\\	[⇒ようふく, きもの]	服[ふく]を 着[き]ている。	
\\	服	ふく	服	ふく	
\\	二つ[ふたつ]	
\\	〔数の〕 
\\	〔二番め〕 
\\	〔二歳〕 
\\	二[ふた]つにする
\\	分[わ]ける}。	
\\	二つ	ふたつ	二つ	ふたつ	
\\	豚肉[ぶたにく]	
\\	豚肉[ぶたにく]は 私[わたし]には 合[あ]わない。	
\\	豚肉	ぶたにく	豚肉	ぶたにく	
\\	二人[ふたり]	
\\	彼[かれ]らは 二人[ふたり]とも 出席[しゅっせき]している。	
\\	二人	ふたり	二人	ふたり	
\\	二日[ふつか]	〔日付〕 (書き言葉で) 
\\	(話し言葉で) 
\\	〔日数〕 
\\	二[に] 日[にち]めごとに。	
\\	二日	ふつか	二日	ふつか	
\\	太い[ふとい]	
\\	〔棒状のものの周りや線の幅などが大きい〕 
\\	〔肥大の〕 
\\	〔ズボンの〕 
\\	〔活字などの〕 
\\	〔声が低い〕 
\\	〔大胆だ・図太い〕 
\\	太[ふと]い 糸[いと]。	
\\	太い	ふとい	太い	ふとい	
\\	降る[ふる]	
\\	[⇒あめ1, ゆき3, あられ, みぞれ]	
\\	雪[ゆき]が〕 降[ふ]るだけ 降[ふ]ったからもう 降[ふ]らないだろう。	
\\	降る	ふる	降る	ふる	
\\	古い[ふるい]	
\\	〔昔に属するさま; 出来てから時間がたっているさま〕 
\\	(古代の) 
\\	〔古くなった〕 
\\	(古風な) 
\\	(昔からの) 
\\	〔古臭い・時代遅れだ〕 
\\	(言いふるした) 
\\	*《俗》 
\\	〔すたれた〕 
\\	*《俗》 
\\	〔かび臭い〕 
\\	古[ふる]い 教[おし]え。	
\\	古い	ふるい	古い	ふるい	
\\	ページ[ページ]	
\\	【電算】 〔メモリーの区画〕 
\\	〔ホームページ〕 
\\	ページを 追[お]って 読[よ]む。	
\\	ページ	ページ	ページ	ページ	
\\	下手[へた]	
\\	〔技術の拙い人〕 
\\	~な 〔技術の拙い〕 
\\	〔たどたどしい〕 
\\	〔不器用な〕 
\\	~な 〔生半可な〕 
\\	〔いい加減な〕 
\\	へたな 鉄砲[てっぽう]も 数[すう] 撃[う]ちゃ 当[あ]たる。	
\\	下手	へた	下手	へた	
\\	ベッド[ベッド]	
\\	ベッドの 頭部[とうぶ]。	
\\	ベッド	ベッド	ベッド	ベッド	
\\	部屋[へや]	
\\	〔一般の〕 
\\	〔アパートの〕 
\\	〔相撲の〕 
\\	客[きゃく]を 部屋[へや]に 通[とお]す。	
\\	部屋	へや	部屋	へや	
\\	辺[へん]	
\\	〔漠然とした場所〕 
\\	〔方面〕 
\\	〔付近〕 
\\	〔漠然とした事情・程度〕 
\\	〔多角形の各線分〕 
\\	【数】 〔等式・不等式の両側の項〕 
\\	[⇒うへん1, さへん, りょうへん] 
\\	【電】 〔ブリッジの〕 
\\	この辺[このへん]に 私[わたし]の 眼鏡[めがね] 置[お]いてなかった?	
\\	辺	へん	辺	へん	
\\	ペン[ペン]	
\\	【電算】 〔入力用の〕 
\\	ペンを 貸[か]してくれ。	
\\	ペン	ペン	ペン	ぺん	
\\	勉強[べんきょう]	
\\	〔勉学〕 
\\	~する 
\\	〔子供が〕 
\\	〔安売り・値引き〕 ~する 〔安く売る〕 
\\	〔値引きする〕 
\\	演技[えんぎ]
\\	料理[りょうり]}の 勉強[べんきょう]をする。	
\\	勉強	べんきょう	勉強	べんきょう	
\\	便利[べんり]	〔利便性〕 
\\	~な 
\\	便利[べんり]な 人[ひと]。	
\\	便利	べんり	便利	べんり	
\\	ボール・ペン[ボール・ペン]	
\\	このボール・ペン 書[か]けないよ。	
\\	ボール・ペン	ボール・ペン	ボールペン	ボールペン	
\\	帽子[ぼうし]	〔縁のある〕 
\\	〔縁なしひさし付きの〕 
\\	〈集合的に〉 
\\	黄色[きいろ]い 帽子[ぼうし]の 園児[えんじ]たち。	
\\	帽子	ぼうし	帽子	ぼうし	
\\	外・他[ほか]	
\\	〔他所〕 
\\	〔外部〕 
\\	〔それ以外・別〕 
\\	〔ある範囲のそと〕 
\\	ほかを 捜[さが]しましたか。	
\\	外・他	ほか	他	ほか	
\\	ポケット[ポケット]	
\\	《俗》 
\\	ズボンの 後[うし]ろポケット。	
\\	ポケット	ポケット	ポケット	ポケット	
\\	欲しい[ほしい]	
\\	もう少[もうすこ]し 早[はや]く 来[き]て 欲[ほ]しかった。	
\\	欲しい	ほしい	欲しい	ほしい	
\\	細い[ほそい]	
\\	〔狭い〕 
\\	〔小さい〕 
\\	〔やせている〕 
\\	細[ほそ]い 道[みち]
\\	川[がわ]}。	
\\	細い	ほそい	細い	ほそい	
\\	鈕・釦[ボタン]	
\\	〔衣類などの〕 
\\	〔機械などの〕 
\\	三[みっ]つボタンのスーツ。	
\\	鈕・釦	ボタン	ボタン	ボタン	
\\	ホテル[ホテル]	
\\	ホテルを 経営[けいえい]する。	
\\	ホテル	ホテル	ホテル	ホテル	
\\	本棚[ほんだな]	
\\	本棚[ほんだな]に 本[ほん]を 並[なら]べる。	
\\	本棚	ほんだな	本棚	ほんだな	
\\	本当に[ほんとうに]	〔まったく, 実に〕 
\\	その 事故[じこ]は 本当[ほんとう]に 昨年[さくねん] 起[お]こったのですか。	
\\	本当に	ほんとうに	本当に	ほんとうに	
\\	毎朝[まいあさ]	
\\	隣[となり]の 男性[だんせい]は 毎朝[まいあさ]ジョギングをしている と言[とい]った。	
\\	毎朝	まいあさ	毎朝	まいあさ	
\\	毎週[まいしゅう]	
\\	週につき〕 
\\	毎週[まいしゅう]2 回[かい]。	
\\	毎週	まいしゅう	毎週	まいしゅう	
\\	毎月[まいつき]	=まいげつ.	その 仕事[しごと]は 毎月[まいつき]50 万[まん] 円[えん]になる。	
\\	毎月	まいつき	毎月	まいつき/まいげつ	
\\	毎年[まいとし]	=まいねん.	毎年[まいとし] 友達[ともだち]から 便[たよ]りがある。	
\\	毎年	まいとし	毎年	まいとし/まいねん	
\\	毎日[まいにち]	
\\	日あたり〕 
\\	〔来る日も来る日も〕 
\\	〔明け暮れ〕 
\\	毎日[まいにち]の 雑務[ざつむ]。	
\\	毎日	まいにち	毎日	まいにち	
\\	毎晩[まいばん]	
\\	〔服薬の指示〕 
\\	(略: 
\\	毎晩[まいばん] 酒[さけ]を 飲[の]む。	
\\	毎晩	まいばん	毎晩	まいばん	
\\	曲がる[まがる]	⇒まがった. 
\\	〔形が変わる〕 (弓状に) 
\\	(くの字に) 
\\	【解】 〔関節が動く〕 
\\	〔方向が変わる〕 
\\	〔道を折れる〕 
\\	〔カーブに沿う〕 
\\	〔ずれる〕 
\\	〔ゆがむ〕 
\\	〔傾く〕 
\\	〔ひねくれる〕 
\\	ひざが 痛[いた]くて 曲[ま]がらない。	
\\	曲がる	まがる	曲がる	まがる	
\\	不味い[まずい]	
\\	食事[しょくじ]がまずいと 気[き]がめいるよね。	
\\	不味い	まずい	不味い	まずい	
\\	又[また]	
\\	〔再び〕 
\\	〔繰り返して〕 
\\	〔続けて〕 
\\	〔さらに〕 
\\	〔同じく〕 
\\	〔同時に〕 
\\	〔その上〕 
\\	〔一方では〕 
\\	〔驚きを表わして〕	老境[ろうきょう]にさしかかると1 人[にん]また1 人[にん]と 親[した]しい 人[ひと]が 亡[な]くなっていく。	
\\	又	また	また	また	
\\	まだ[まだ]	
\\	〔まだ…(ない)〕 
\\	〔今でもなお〕 
\\	〔今のところ〕 
\\	〔もっと〕 
\\	〔どちらかと言えば〕 
\\	その 事実[じじつ]はまだ 証明[しょうめい]されていない。	
\\	まだ	まだ	まだ	まだ	
\\	町・街[まち]	〔都会〕 
\\	〔街路〕 
\\	〔行政単位〕 
\\	〔近所〕 
\\	町[まち]に
\\	買い物[かいもの]に) 出[で]る。	
\\	町・街	まち	町	まち	
\\	待つ・俟つ[まつ]	
\\	〔時を過ごす〕 
\\	(期待して) 
\\	(雌伏して) 
\\	(覚悟して) 
\\	〔望みを託す〕 
\\	〔ゆだねる〕 
\\	伏[ふ]せ! 待[ま]て!よし!	〔犬に向かって〕 
\\	待つ・俟つ	まつ	待つ	まつ	
\\	マッチ[マッチ]	
\\	マッチのラベル 収集家[しゅうしゅうか]。	
\\	マッチ	マッチ	マッチ	マッチ	
\\	窓[まど]	
\\	〔上下する〕 
\\	〔観音開きの〕 
\\	〔船室の〕 
\\	【光】 
\\	【動・解】 〔骨の中などの窓状の孔〕 
\\	〔外界との連絡路〕 
\\	【電算】 =ウインドー 
\\	この 部屋[へや]は 窓[まど]が 大[おお]きくて 気持[きも]ちいい。	
\\	窓	まど	窓	まど	
\\	丸い・円い[まるい]	〔円形の〕 
\\	〔環状の〕 
\\	〔球形の〕 
\\	〔円筒形の〕 
\\	背中[せなか]が 丸[まる]い。	
\\	丸い・円い	まるい	丸い	まるい	
\\	万[まん]	〔千の 
\\	倍〕 
\\	〔多数〕 
\\	何[なん] 万[まん]という 人[ひと]。	
\\	万	まん	万	まん	
\\	万年筆[まんねんひつ]	
\\	万年筆[まんねんひつ]にインクを 入[い]れる。	
\\	万年筆	まんねんひつ	万年筆	まんねんひつ	
\\	磨く[みがく]	
\\	〔つやを出す〕 
\\	〔研磨する〕 
\\	(ラップ盤で) 
\\	〔きれいにする〕 
\\	〔洗練の度合いを高める〕 
\\	彼女[かのじょ]は 都会[とかい]の 水[みず]に 磨[みが]かれてすっかり あか抜[あかぬ]けた。	
\\	磨く	みがく	磨く	みがく	
\\	右[みぎ]	
\\	〔方向・位置〕 
\\	〔前段に書いたこと〕 
\\	〔上位〕 ⇒右に出る. 
\\	〔思想的な右翼〕 
\\	〔右派〕 
\\	[⇒みぎより1 
\\	右前[みぎまえ]に 着物[きもの]を 着[き]る。	
\\	右	みぎ	右	みぎ	
\\	短い[みじかい]	〔端から端まで・開始から終了までが〕 
\\	[⇒みじかく]	髪[かみ]の 短[みじか]い 女性[じょせい]。	
\\	短い	みじかい	短い	みじかい	
\\	店[みせ]	〔商店〕 
\\	〔売り場〕 
\\	〔露店の類〕 
\\	〔飲食店〕 
\\	〔金融機関の店頭〕 
\\	〔商社〕 
\\	こんな 所[ところ]に 店[みせ]を 出[だ]しちゃいかん. 交通[こうつう]の 邪魔[じゃま]になる。	
\\	店	みせ	店	みせ	
\\	見せる[みせる]	
\\	〔提示する〕 
\\	(差し出す) 
\\	〔相手の前に出る〕 
\\	〔目に触れさせる〕 
\\	(露出する) 
\\	〔可視化する〕 
\\	〔表に現す〕 
\\	(隠せなくて) 
\\	〔披露する・見せつける〕 
\\	〔装う〕 
\\	〔診察や鑑定をしてもらう〕 
\\	〔見る価値がある〕 
\\	カタログを 見[み]せて 説明[せつめい]する。	
\\	見せる	みせる	見せる	みせる	
\\	道・途[みち]	
\\	〔道路〕 
\\	(往来) 
\\	〔街道〕 
\\	〔通路〕 
\\	(門から玄関までの) 
\\	〔小道〕 
\\	(山間の) 
\\	〔路地〕 
\\	〔経路〕 
\\	(目的地までの) 
\\	〔行程〕 
\\	(距離) 
\\	〔所要時間〕 
\\	〔手段や方法〕 
\\	〔過程〕 
\\	〔履歴〕 
\\	〔前途〕 
\\	〔進路〕 
\\	〔分野〕 (職業) 
\\	(技術) 
\\	〔特殊な領域〕 
\\	〔倫理的な・宗教上のきまり〕 
\\	[⇒みちならぬ]	これはかのゲーテが 歩[ある]いた 道[みち]です。	
\\	道・途	みち	道	みち	
\\	三日[みっか]	
\\	日間〕 
\\	〔第 
\\	日〕 
\\	三[さん] 日[にち]めごとに。	
\\	三日	みっか	三日	みっか	
\\	三つ[みっつ]	〔数〕 
\\	〔年齢〕 
\\	アメリカ合衆国[あめりかがっしゅうこく]はアジアの3つの 国[くに]の 紛争[ふんそう]に 巻き込[まきこ]まれたようだ。	
\\	三つ	みっつ	三つ	みっつ	
\\	皆さま, 皆さん[みなさま, みなさん]	〔全員〕 
\\	〔呼びかけ〕 
\\	お 宅[たく]のみなさまによろしくお 伝[つた]えください。	
\\	皆さま, 皆さん	みなさま, みなさん	皆さん	みなさん	
\\	南[みなみ]	
\\	(略: 
\\	その 町[まち]から 南[みなみ]に 約[やく]5キロの 地点[ちてん]に。	
\\	南	みなみ	南	みなみ	
\\	耳[みみ]	
\\	〔聴覚器官〕 
\\	(バッタなどの) 
\\	〔聴覚〕 
\\	〔聴力〕 
\\	〔音感〕 
\\	〔話や音を聞くこと〕 ⇒こみみ. 
\\	〔耳介〕 
\\	[⇒みみたぶ, みみばね] 
\\	〔耳のような突起〕 
\\	〔へり〕 
\\	(織物の) 
\\	【機】 
\\	【建】 
\\	〔のれんのさお通し〕 
\\	〔疑問符〕 
\\	耳[みみ]の。	【解】 
\\	耳	みみ	耳	みみ	
\\	見る[みる]	
\\	〔形や色を目に入れる〕 
\\	《口》 
\\	〔視線を向ける〕 
\\	[⇒みるからに, みるめ1] 
\\	〔目撃する〕 
\\	〔会う〕 
\\	〔凝視する〕 
\\	〔監視する〕 
\\	〔鑑賞する〕 
\\	(見物する) 
\\	〔目を通す〕 
\\	〔添削する〕 
\\	〔精査する〕 
\\	〔確認する〕 
\\	(調べて) 
\\	〔静観する〕 
\\	〔観察する〕 
\\	〔注意を向ける〕 
\\	〔世話をする〕 
\\	〔介護する〕 
\\	〔身に受ける〕 
\\	〔結果が出る〕 
\\	〔評価する〕 
\\	〔判断する〕 
\\	〔見積もる〕 
\\	〔試みる〕 ⇒-みる.	見[み]たくもない 彼[かれ]の 踊[おど]りを 見[み]せられた。	
\\	見る	みる	見る	みる	
\\	皆[みな, みんな]	
\\	〔全員〕 
\\	〔全部; すべてのもの〕 
\\	皆[みな]で 決[き]めたことだ。	
\\	皆	みな, みんな	皆	みんな	
\\	六日[むいか]	〔日付〕 (書き言葉で) 
\\	(話し言葉で) 
\\	〔日数〕 
\\	六[ろく] 日[にち]のあやめ。	⇒あやめ2.	六日	むいか	六日	むいか	
\\	向こう[むこう]	
\\	〔反対側〕 
\\	〔正面・前方〕 
\\	〔離れたところ〕 
\\	〔…をへだてた先〕 
\\	〔行く先〕 
\\	〔今から先〕 
\\	〔相手方〕 
\\	向こう向[むこうむ]きになる。	
\\	向こう	むこう	向こう	むこう	
\\	難しい[むずかしい]	
\\	〔困難な〕 
\\	《口》 
\\	〔難解な〕 
\\	〔厄介な〕 
\\	〔微妙な〕 
\\	〔(顔つきの)不機嫌な〕 
\\	〔しかつめらしい〕 
\\	〔気難しい〕 
\\	〔好みがうるさい〕 
\\	難[むずか]しがる。	〔難しいと思う〕 
\\	難しい	むずかしい	難しい	むずかしい	
\\	六つ[むっつ]	〔数字〕 
\\	〔年齢〕 
\\	彼女[かのじょ]はそのケーキを6つに 分[わ]けた。	
\\	六つ	むっつ	六つ	むっつ	
\\	目[め]	
\\	〔視覚器官〕 
\\	《口》 
\\	《俗》 
\\	彼[かれ]は 目[め]をうるませた。	
\\	目	め	目	め	
\\	メートル[メートル]	
\\	〔長さの単位〕 
\\	〔計量器〕 =メーター. 
\\	〔酔い〕	酔[よ]ってメートルを 上[あ]げる。	
\\	《口》 
\\	〔気焔(きえん)を上げる〕 
\\	メートル	メートル	メートル	メートル	
\\	眼鏡[めがね]	
\\	〔目にかける器具〕 
\\	《口》 
\\	〔鑑識・鑑定〕 
\\	めがねをかけて。	
\\	眼鏡	めがね	めがね	めがね	
\\	もう[もう]	
\\	〔今では〕 
\\	〔もはや〕 
\\	〔間もなく〕 
\\	〔すでに〕 
\\	〔疑問文で〕 
\\	〔否定文で〕 
\\	〔今ごろは〕 
\\	〔さらに〕 
\\	〔別の〕 
\\	〔感動・あきれた気持ち・いらだちなどを表して〕	もうこれ 以上[いじょう]やることがないと 思[おも]ったら 気[き]が 楽[らく]になった。	
\\	もう	もう	もう	もう	
\\	木曜日[もくようび]	
\\	(略: 
\\	木曜日[もくようび]よりむしろ 金曜日[きんようび]においでいただきたい。	
\\	木曜日	もくようび	木曜日	もくようび	
\\	もしもし[もしもし]	
\\	もしもし 何[なに]かお 落[お]としになりましたよ。	
\\	もしもし	もしもし	もしもし	もしもし	
\\	勿論[もちろん]	
\\	もちろん、あなたを 手伝[てつだ]ってあげますよ。	
\\	勿論	もちろん	勿論	もちろん	
\\	持つ[もつ]	
\\	〔手などで重さを支え持つ〕 
\\	*《口》 
\\	〔手で物をつかむ〕 
\\	〔携帯・所持する〕 
\\	〔所有する〕 
\\	〔家族・友・家庭などを有する〕 
\\	〔属性を有する〕 
\\	(授かっている) 
\\	〔仕事として担当する〕 
\\	〔負担する〕 
\\	〔もうける〕 
\\	〔心に抱く〕 
\\	〔保つ〕 
\\	〔長持ちする〕 
\\	〔死なずにいる〕 
\\	[⇒もたせる2 
\\	物[もの]を 持[も]ってやる。	
\\	持つ	もつ	持つ	もつ	
\\	もっと[もっと]	〔数量〕 
\\	〔時〕 
\\	〔距離〕 
\\	〔程度〕 
\\	もっと 安[やす]いものがほしい。	
\\	もっと	もっと	もっと	もっと	
\\	物[もの]	
\\	〔(一般に)有形の物〕 
\\	《口》 
\\	〔物質〕 
\\	〔品物〕 
\\	〔物資〕 
\\	〔材料〕 
\\	〔資源〕 
\\	〔所有物〕 
\\	【法】 
\\	〔品物〕 
\\	〔その品質〕 
\\	〔漠然としたもの〕 
\\	〔抽象的なもの; 形式名詞的なもの〕 
\\	〔道理〕 
\\	〔(「…のもの」の形で) 所有〕 
\\	〔(名詞などについて) その部類に入るもの〕 
\\	相続税[そうぞくぜい]を 物[もの]で 納[おさ]める。	
\\	物	もの	物	もの	
\\	門[もん]	
\\	〔家の〕 
\\	【解】 〔内臓・血管などの〕 
\\	〔ある師を頂点とする芸術や学問の派閥〕 
\\	〔生物分類の〕 【動】 
\\	【植】 
\\	➡なお, 門以下の分類階級は, 綱 
\\	目 
\\	科 
\\	属 
\\	種 
\\	亜種 
\\	変種 
\\	品種 
\\	の順になる.	門[もん]が 閉[し]まっている。	
\\	門	もん	門	もん	
\\	問題[もんだい]	
\\	〔答えを求める問題〕 
\\	〔研究などの課題〕 
\\	〔取り上げるべき問題・課題〕 
\\	〔困った事柄・難点〕 
\\	〔不都合〕 
\\	〔疑問〕 
\\	〔物議・世間のうわさ〕 
\\	〔面倒な事件・ごたごた〕 
\\	〔あることに左右される事柄〕 
\\	教師[きょうし] 手作[てづく]りの 問題[もんだい]。	
\\	問題	もんだい	問題	もんだい	
\\	八百屋[やおや]	〔店〕 
\\	〔人〕 
\\	その 八百屋[やおや]は 客[きゃく]にとても 親切[しんせつ]だ。	
\\	八百屋	やおや	八百屋	やおや	
\\	野菜[やさい]	
\\	〔市場向けの〕 
\\	〔菜園の〕 
\\	野菜[やさい]を 作[つく]る。	
\\	野菜	やさい	野菜	やさい	
\\	優しい[やさしい]	〔柔和な〕 
\\	〔優雅な〕 
\\	〔愛情ある〕 
\\	〔温和な〕 
\\	〔親切な〕 
\\	〔温順な〕 
\\	〔人当たりのよい〕 
\\	気立[きだ]ての 優[やさ]しい。	
\\	優しい	やさしい	優しい	やさしい	
\\	安い・廉い[やすい]	〔金額が少ない〕 
\\	〔価値が低い〕 
\\	安[やす]かろう 悪[わる]かろう。	
\\	安い・廉い	やすい	安い	やすい	
\\	休み[やすみ]	
\\	〔休息〕 
\\	〔休憩時間〕 
\\	〔中入り〕 
\\	〔休止〕 
\\	〔休日・休暇〕 
\\	〔休業〕 
\\	(長期の) 
\\	〔欠席・欠勤〕 
\\	コンピューターを 昼夜[ちゅうや] 休[やす]みなく 稼働[かどう]させている。	
\\	休み	やすみ	休み	やすみ	
\\	休む[やすむ]	
\\	〔休憩する〕 
\\	〔ひと息つく〕 
\\	〔欠勤する〕 
\\	〔欠席する〕 
\\	〔休業する〕 
\\	〔寝る〕 (就寝する) 
\\	《口》 
\\	(眠る) 
\\	時間[じかん] 休[やす]んだらよくなるでしょう。	
\\	休む	やすむ	休む	やすむ	
\\	八つ[やっつ]	〔数字〕 
\\	〔年齢〕 
\\	八[やっ]つの 部分[ぶぶん]に 分[わ]けられた。	
\\	八つ	やっつ	八つ	やっつ	
\\	山[やま]	
\\	〔山岳〕 
\\	(峰) 
\\	(丘陵) 
\\	(円丘) 
\\	(高地) 
\\	〔積み重なったもの〕 
\\	〔大量・多数〕 
\\	〔突出したところ〕 
\\	〔極点〕 
\\	(病気の) 
\\	〔危機〕 
\\	〔難所〕 
\\	〔見通し〕 ⇒やまば. 
\\	〔当て推量〕 
\\	(投機的行為) 
\\	(場当たり) 
\\	〔山林〕 
\\	(財産としての山林) 
\\	〔鉱山・炭坑〕 
\\	〔犯罪事件〕 
\\	低気圧[ていきあつ]の 影響[えいきょう]で 海[うみ] 山[やま]とも 終日[しゅうじつ] 荒[あ]れるでしょう。	
\\	山	やま	山	やま	
\\	やる[やる]	
\\	〔与える〕 
\\	〔贈り物を〕 
\\	〔送る〕 
\\	〔行かせる・つかわす〕 
\\	〔移す〕 
\\	〔置く〕 
\\	〔向ける〕 
\\	〔する・行う〕 
\\	(スポーツを) 
\\	(ゲームを) 
\\	(役を) 
\\	〔試みる〕 
\\	〔実行する〕 
\\	〔取りかかる〕 
\\	〔経営する〕 
\\	〔職業として…する〕 
\\	〔開催する〕 
\\	〔興行する〕 
\\	〔上映する〕 
\\	〔放映する〕 
\\	〔飲食する〕 
\\	(たばこを) 
\\	〔危害を加える〕 
\\	〔殺す〕 
\\	《口》 
\\	《俗》 
\\	〔犯す〕 
\\	〔性交する〕 
\\	《俗》 
\\	《卑》 
\\	[⇒やっつける] 
\\	〔親切で…する〕 
\\	〔ことさら…してみせる〕 
\\	〔嫌がらせに…する〕 
\\	〔生活する・過ごす〕 ⇒やっていく, やってくる. 
\\	〔心を慰める〕	この 本[ほん]を 君[きみ]にやろう。	
\\	やる	やる	やる	やる	
\\	八日[ようか]	
\\	日間〕 
\\	〔第 
\\	日〕 
\\	月[つき]8 日[にち]。	
\\	八日	ようか	八日	ようか	
\\	洋服[ようふく]	〔衣服〕 
\\	〔和服に対して〕 
\\	洋服[ようふく]を 着[き]た 犬[いぬ]。	
\\	洋服	ようふく	洋服	ようふく	
\\	よく[よく]	
\\	〔十分に〕 
\\	(かなりよく) 
\\	〔注意して〕 
\\	〔詳しく〕 
\\	〔非常に・たいそう〕 
\\	〔うまく・立派に〕 
\\	〔正しく〕 
\\	〔巧みに〕 
\\	〔よいように〕 
\\	〔多く〕 
\\	〔普通〕 
\\	〔しばしば〕 
\\	〔常に〕 
\\	〔能力が十分あるさま〕 
\\	〔うれしい気持ち〕 
\\	〔称賛や驚きの気持ち・よくぞ〕 
\\	〔非難する気持ち・よくもまあ〕	よく 聞[き]いてみるとそのとき 彼[かれ]は 一[いち] 人[にん]ではなかったようだ。	
\\	よく	よく	よく	よく	
\\	横[よこ]	
\\	〔縦や垂直に対して水平〕 
\\	〔横幅〕 
\\	(船の) 
\\	〔前後に対して左右〕 
\\	〔かたわら〕 
\\	〔側面〕 
\\	〔階級の同じ者〕	横[よこ]に 長[なが]い。	
\\	横	よこ	横	よこ	
\\	四日[よっか]	
\\	日間〕 
\\	〔第 
\\	日〕 
\\	月[つき] 四[よん] 日[にち]。	
\\	四日	よっか	四日	よっか	
\\	四つ[よっつ]	〔数〕 
\\	〔年齢〕 
\\	[=よつ 
\\	四[よっ]つに 切[き]る。	
\\	四つ	よっつ	四つ	よっつ	
\\	呼ぶ[よぶ]	
\\	〔声をかける〕 
\\	(おおいと) 
\\	(後ろから) 
\\	(神などを) 
\\	〔呼び寄せる〕 
\\	(呼び込む) 
\\	〔芸人を〕 
\\	〔招待する〕 
\\	[⇒よばれる 
\\	〔引き起こす〕 
\\	〔引きつける〕 
\\	〔名づける〕 
\\	(学術上で) 
\\	〔悪名で〕 
\\	[⇒よばれる 
\\	順番[じゅんばん]に お呼[およ]びしていますので 今[いま]しばらく お待[おま]ちください。	
\\	呼ぶ	よぶ	呼ぶ	よぶ	
\\	読む[よむ]	
\\	〔本などを〕 
\\	〔誦する〕 
\\	[⇒よめる] 
\\	〔意味を理解する〕 
\\	〔(人の心などを)読み取る〕 
\\	〔予測する〕 
\\	〔数を数える〕 
\\	先生[せんせい]の 後[のち]について 読[よ]む。	
\\	読む	よむ	読む	よむ	
\\	夜[よる]	〔日没ごろから日の出ごろまでの間〕 
\\	〔日暮れ〕 
\\	〔日没から就寝時まで; 晩〕 
\\	[=よ4]	夜[よる]と 昼[ひる]を 取り違[とりちが]える。	
\\	夜	よる	夜	よる	
\\	来月[らいげつ]	
\\	〔公用文・商用文などで〕 
\\	(略: 
\\	来月[らいげつ] 一[いち] 日[にち]に。	
\\	来月	らいげつ	来月	らいげつ	
\\	来週[らいしゅう]	
\\	来週[らいしゅう]の 月曜日[げつようび]に。	
\\	来週	らいしゅう	来週	らいしゅう	
\\	来年[らいねん]	
\\	この 町[まち]に 来[き]てから 来年[らいねん]で10 年[ねん]だ。	
\\	来年	らいねん	来年	らいねん	
\\	ラジオ[ラジオ]	〔放送〕 
\\	〔受信機〕 
\\	ラジオの 波長[はちょう]を 合[あ]わせる。	
\\	ラジオ	ラジオ	ラジオ	ラジオ	
\\	立派[りっぱ]	
\\	~な 〔堂々たる〕 
\\	〔優れた〕 
\\	〔尊敬すべき〕 
\\	~な 〔要件を満たした〕 
\\	〔十分な〕 
\\	立派[りっぱ]な 風采[ふうさい]。	
\\	立派	りっぱ	立派	りっぱ	
\\	留学生[りゅうがくせい]	〔一般の〕 
\\	〔官費の〕 
\\	在日[ざいにち] 中国[ちゅうごく] 留学生[りゅうがくせい]。	
\\	留学生	りゅうがくせい	留学生	りゅうがくせい	
\\	両親[りょうしん]	
\\	両親[りょうしん]のわからない 子[こ]。	
\\	両親	りょうしん	両親	りょうしん	
\\	料理[りょうり]	
\\	〔調理〕 
\\	〔調理した食物〕 
\\	~する 〔火を使って〕 
\\	〔火を使わないで〕 
\\	〔処理〕 
\\	~する 
\\	料理[りょうり]を 出前[でまえ]する。	
\\	料理	りょうり	料理	りょうり	
\\	旅行[りょこう]	〔旅行すること〕 
\\	回の〕 
\\	〔数回の〕 
\\	〔短期の〕 
\\	〔周遊〕 
\\	〔遊覧〕 
\\	〔船の〕 
\\	~する 
\\	〔船旅〕 
\\	旅行[りょこう]の 日取[ひど]りを 決[き]める。	
\\	旅行	りょこう	旅行	りょこう	
\\	零[れい]	
\\	電話[でんわ]で〕104 番[ばん]。	
\\	零	れい	れい	れい	
\\	冷蔵庫[れいぞうこ]	
\\	《口》 
\\	〔冷凍用〕 
\\	冷蔵庫[れいぞうこ]に 魚肉[ぎょにく]を 入[い]れておく。	
\\	冷蔵庫	れいぞうこ	冷蔵庫	れいぞうこ	
\\	レコード[レコード]	
\\	〔記録〕 
\\	〔音盤〕 
\\	【電算】 〔ひとそろいの情報〕 
\\	レコードを 破[やぶ]る。	
\\	レコード	レコード	レコード	レコード	
\\	レストラン[レストラン]	
\\	《口》 
\\	ここは 私[わたし]が 知[し]っている 中[なか]で 一番[いちばん]よいレストランです。	
\\	レストラン	レストラン	レストラン	レストラン	
\\	練習[れんしゅう]	
\\	〔(テニスなどの)試合前の〕 
\\	〔リハーサル〕 
\\	~する 
\\	歌[うた]
\\	野球[やきゅう]}の 練習[れんしゅう]をする。	
\\	練習	れんしゅう	練習	れんしゅう	
\\	六[ろく]	
\\	〔ローマ数字〕 
\\	第[だい] 六[ろく]。	
\\	六	ろく	六	ろく	
\\	六月[ろくがつ]	
\\	(略: 
\\	6月[ろくがつ]3 日[にち]にシングルの 部屋[へや]を 予約[よやく]したいのですが。	
\\	六月	ろくがつ	六月	ろくがつ	
\\	ワイシャツ[ワイシャツ]	
\\	〔胸の固い礼装用〕 
\\	〔カラー・カフスのついたもの〕 
\\	ワイシャツのすそを 出[だ]して。	
\\	ワイシャツ	ワイシャツ	ワイシャツ	ワイシャツ	
\\	若い[わかい]	
\\	〔若年の・青年期の〕 
\\	〔年下で〕 
\\	〔活力にあふれている〕 
\\	〔衰えていない〕 
\\	[⇒わかわかしい] 
\\	〔生え出たばかり〕 
\\	〔ういういしい〕 
\\	〔歴史の浅い〕 
\\	〔未熟な〕 
\\	〔番号が小さい〕 
\\	若き日[わかきひ]の 思い出[おもいで]。	
\\	若い	わかい	若い	わかい	
\\	分かる[わかる]	
\\	〔理解できる〕 
\\	[⇒わからない, わかりにくい, わかりやすい] 
\\	〔使いこなせる〕 
\\	〔判明する〕 
\\	〔弁別できる〕 
\\	〔意味が通じる〕 
\\	〔気持ちが伝わる〕 
\\	〔自覚する〕 
\\	〔納得する〕 
\\	〔腑に落ちる〕 
\\	〔十分な情報がある; 予測できる〕 
\\	〔道理を知る〕 
\\	〔価値を知る〕 
\\	アンケートでは「わからない」という 回答[かいとう]が 多[おお]かった。	
\\	分かる	わかる	分かる	わかる	
\\	忘れる[わすれる]	⇒わすれがち, わすれさられる, わすれさる. 
\\	〔思い出せなくなる〕 (忘却する) 
\\	(時がたつにつれ) 〈事が主語〉 
\\	〔なすべきことをしない〕 
\\	〔物を置いたままにする〕 (置き忘れる) 
\\	(持って来るのを) 
\\	〔思い出さないようにする〕 
\\	〔顧みない〕 
\\	〔気づかずにいる〕 
\\	〔注意がそれる〕 
\\	忘[わす]れもしないそれは 去年[きょねん]の 夏[なつ]のことだ。	
\\	忘れる	わすれる	忘れる	わすれる	
\\	私[わたし]	〔自分〕 
\\	[⇒わたしたち]	私[わたし]ならこうします。	
\\	私	わたし	私	わたし	
\\	渡す[わたす]	
\\	〔手から手に〕 
\\	〔差し出す〕 
\\	〔管理を委ねる〕 
\\	〔支給する〕 
\\	〔架け渡す〕 
\\	〔対岸に移動させる〕 
\\	(船で) 
\\	名刺[めいし]を 渡[わた]す。	
\\	渡す	わたす	渡す	わたす	
\\	渡る[わたる]	
\\	〔向こう側へ移動する〕 (横切って) 
\\	〔海を越える〕 
\\	(外国へ) 
\\	(日本へ) 
\\	〔ある方向へ移動する〕 
\\	〔通り過ぎる〕 
\\	〔世渡りする〕 
\\	〔他人の所有になる〕 ((人手に)移る) 
\\	【法】 
\\	〔全員に配られる〕 (渡される) 
\\	川[かわ]を 歩[ある]いて
\\	泳[およ]いで} 渡[わた]る。	
\\	渡る	わたる	渡る	わたる	
\\	悪い[わるい]	
\\	〔感心しない〕 
\\	(道徳的に) 
\\	(法律的に) 
\\	〔性質が悪である〕 
\\	〔治療の困難な〕 
\\	〔能力が低い〕 
\\	〔品質が劣る〕 
\\	〔水準や理想から遠い〕 
\\	〔適当でない〕 
\\	〔望ましくない〕 
\\	〔害がある〕 
\\	〔通常でない〕 
\\	〔故障がある〕 
\\	〔申し訳ない〕 
\\	いたずらばかりして お前[おまえ]は 悪[わる]い 子[こ]だ。	
\\	悪い	わるい	悪い	わるい	
\end{CJK}
\end{document}