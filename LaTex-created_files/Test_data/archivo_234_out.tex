\documentclass[8pt]{extreport} 
\usepackage{hyperref}
\usepackage{CJKutf8}
\begin{document}
\begin{CJK}{UTF8}{min}
\\	そんなことするくらいなら...	
\\	るくらいなら 
\\	るぐらいなら とても嫌だと言う気持ち 
\\	方がマシだ  方がいい  なさい
\\	君ぐらいのものだ	=君しかいない (〜のは)
\\	くらい のものだ      
\\	ぐらい なものだ
\\	できるものとして	=できると考えて、
\\	る/
\\	い/
\\	である・な/
\\	である ものとして
\\	10月に行うものとする  硬い	=行う、と決める 
\\	る ものとする 契約書などに多く出てくる表現
\\	双方の
\\	ソウホウ 
\\	物に成る	
\\	慌てることのないように	=慌てないように 
\\	る ことのないよう(に)
\\	日頃(から)	
\\	状況を知ることなしに  硬い	=知らないで、知ることなく、
\\	る ことなしに(は)
\\	彼のことだから	1。=彼はだから、きっと 
\\	のことだから 推量が続く 2。
\\	る/
\\	い/
\\	な・である/
\\	である  
\\	ことだから 
\\	ことだし 話
\\	ーこの話は聞かなかったことにしてください ー大学を卒業したことになっているが	1。=聞かなかったのと同じように扱ってください、
\\	2。大学を卒業したことになっているが…=卒業したと言われている 
\\	た 
\\	かった  (という)ことにする 
\\	(だ)   (という)ことになる 
\\	(だ)
\\	ケチ	
\\	問題だとされる	=問題だと言われている、
\\	とされる
\\	生長	セイチョウ 
\\	促進	そくしん 
\\	学生時代のことが思い出される 硬い	=学生時代のことを自然と思い出す (〜が)
\\	られる <自発の受け身系> 意思と関係なく、ひとりでにその状態になると言う意味。 例:〜と考えられる   懐かしく思われる   完成が待たれる   息子の将来が案じられる   行ってしまったことが悔やまれる
\\	感心させられる	=感心している 
\\	させられる 
\\	自発の使役受け身系
\\	私に言わせれば	=私の意見では 
\\	に  言わせれば 
\\	から 言わせれば 
\\	=人
\\	移住	いじゅう 
\\	案じる	
\\	を
\\	待つ	
\\	甘え	
\\	忙しいとみえて	=忙しいようで, 
\\	とみえて       とみえる
\\	大きい地震が来るとみられる	=来ると考えられている 
\\	とみられる
\\	難しいとみると	=難しいと思うと・とわかると 
\\	とみると
\\	歩いて行くとすれば	=もし歩いて行ったら 
\\	とすれば       とすると =〜としたら
\\	危ないところを	=危ない状況だった。その時 
\\	な ところを 
\\	の
\\	2、3時間というところだ	=だいたい2、3時間ぐらいである状況だ, 
\\	/[文]  というところだ      といったところだ 
\\	は数量が入ることが多い。
\\	何回読んだところで...ない (ー)	=たとえ何回読んでも、
\\	たところで…ない 例:いくら
\\	たところで   どんなに
\\	たところで 悪い結果・状態が続くことが多い。
\\	忘れようとしたところで	=忘れようとしても、
\\	としたところで       としたって (話)
\\	走る家	
\\	電話するなりメールするなり	=電話でもおメールでもどの方法でも、
\\	とか
\\	とか、なんとかして 
\\	したりして 
\\	なり 
\\	なり 
\\	る なり
\\	ルなり)する なんなりと(=なんでも) 
\\	る なりなんなり
\\	できないならできるなりに	=できないならできる範囲で、
\\	それなりの
\\	=それにふさわしい
\\	私なりに=私が考える範囲で 
\\	るなら)
\\	る 
\\	いなら)
\\	い  なりに 
\\	なら)
\\	なりの
\\	なら)
\\	会うなり	=会うと同時に、あった途端に, 
\\	るなり
\\	素人	しろうと 
\\	申し付ける	
\\	雑用	
\\	先生であれ、学生であれ   硬い	=先生でも学生でも誰でも、
\\	であれ 誰であれ 
\\	どこであれ 
\\	なんであれ 
\\	雨が降ろうが雪が降ろうが	=雨が降っても雪が降っても, 
\\	うと
\\	うと] 
\\	うが
\\	うが] 
\\	よう 
\\	かろう    が 
\\	だろう  と 
\\	であろう 〜だろうとなかろうと
\\	どんなに高かろうが   硬い	=どんなに高くても 「疑問詞」 
\\	よう       
\\	かろう   が 
\\	だろう  と(も) 
\\	であろう
\\	過労	かろう 
\\	故意	こい 
\\	嬉しいというか、残念というか	=嬉しいと言えるし、また残念とも言える。とにかく 
\\	というか
\\	というか] 
\\	い  というか 
\\	無茶	
\\	無知	
\\	降ろうと降るまいと  硬い	=降っても降らなくても、どの場合でも 
\\	ようと
\\	るまいと 
\\	ようが
\\	るまいが 
\\	るまい・
\\	2まい 
\\	るまい くるー
\\	こまい、くるまい するー
\\	しまい、すまい、するまい わたしいようがいまいが関係なく、彼らはけんかを始めた。
\\	行こうか行くまいか  硬い	=行くか、行かないか、どちらにしようか。 
\\	ようか
\\	るまいか 
\\	るまい・
\\	2まい 
\\	るまい 来るー
\\	こまい・くるまい するー
\\	しまい・すまい・するまい
\\	来るにせよこないにせよ  硬い	=来る場合もこない場合も 
\\	にせよ
\\	にせよ)] 
\\	にしろ
\\	にしろ)] 
\\	に/
\\	に せよ        しろ 何にせよ・にしろ=どういう場合でも、
\\	誰にせよ・にしろ いつにせよ・にしろ どこにせよ・にしろ
\\	故 故に	ゆえ
\\	ゆえに
\\	空間	
\\	大喧嘩	オオケンカ
\\	童貞	ドウテイ 
\\	雑誌だの
\\	だの (ー)	=雑誌や
\\	やいろいろなもの, 
\\	とか
\\	とか、ほかに色々... 
\\	る・た/
\\	い/
\\	だった/
\\	だった  だの 不満を言う時よく使う。
\\	デザインといい、色といい	=デザインも色も 例:運動といい勉強といい、僕は何をやってもダメだ。
\\	子供が子供なら、親も親だ (ー)	=子供も親も両方よくない 
\\	1が 
\\	1も  
\\	1なら、
\\	2も
\\	2だ
\\	子供と言わず、大人と言わず	=子供も大人も誰でも・・アや
\\	だけでなく。。。 例:私は、牛肉と言わず、豚肉と言わず、肉は食べません。
\\	申し分がない	
\\	ぱっと	
\\	こんなに雪が降っては	=こんなに雪が降っていたら 
\\	ては 
\\	くては 
\\	では
\\	肝心の
\\	カンジン
\\	鏡を見てはため息をついている。	=鏡を見るたびに 
\\	1ては
\\	2の流れがなんども繰り返されることを表す
\\	食べては寝て、食べては寝て	=食べてそのあと・すぐ寝る。それを繰り返す 
\\	1ては
\\	2て、
\\	1ては
\\	2て) 
\\	1ては
\\	1ては
\\	ニキビ	
\\	辿りつく	タドリ 
\\	行きつ戻りつした	=行ったり戻ったり 
\\	2つ
\\	2つ 
\\	1と
\\	2は反対の意味
\\	抜く	
\\	いいとも悪いとも言えない	=いい・悪いと決めることはできない、いいか悪いか 
\\	とも 
\\	とも ー
\\	うんともすんとも言わない 
\\	あったらあったで なかったらなかったで	=あるのはいいが、その場合は 
\\	あったらいいが、ない場合は 
\\	たら
\\	たで 
\\	かったら
\\	かったで 
\\	かったら
\\	いで 
\\	なら
\\	で 
\\	なら
\\	で 否定形も使う
\\	喜んでいいのやら悲しんでいいのやら	=喜ぶべきなのか悲しむべきなのかどっちとも言えないが 例:+最近仕事の依頼が多く、趣味の時間が全く取れない。 
\\	る/
\\	い/
\\	な/
\\	な のやら 
\\	と
\\	は反対の意味 いいのやら悪いのやら あるのやらないのやら 嘘なのやら本当なのやら
\\	草むしり	
\\	便箋	ビンセン 
\\	いつまで続くのやら。	=いつまで続くのかわからない・いつまで続くだろう [疑問詞] 
\\	る/
\\	い/
\\	な のやら            ものやら            ことやら 何をしたらいいのやら どうすればいいのやら いつ帰れることやら いつになることやら
\\	見当	ケントウ 
\\	見当をつける
\\	いつになったら	
\end{CJK}
\end{document}