\documentclass[8pt]{extreport} 
\usepackage{hyperref}
\usepackage{CJKutf8}
\begin{document}
\begin{CJK}{UTF8}{min}
\\	上がる	
\\	あがる			
\\	今日は仕事が早く上がったんだ。	
\\	上がる 
\\	きょう は しごと が はやく あがった ん だ			
\\	私は
\\	業界で働いています。	
\\	業界 
\\	わたし は 
\\	ぎょうかい で はたらいて います			
\\	車は整備に出しています。	
\\	整備 
\\	くるま は せいび に だして います			
\\	業界	
\\	ぎょうかい			
\\	円高の影響で海外製品が安く買えますよ。	
\\	円高 
\\	えんだか の えいきょう で かいがい せいひん が やすく かえます よ			
\\	テレビも取材に来たほど有名なレストランです。	
\\	取材 
\\	てれび も しゅざい に きた ほど ゆうめい な れすとらん です			
\\	証券を売って資金にしようと思うの。	
\\	証券 
\\	しょうけん を うって しきん に しよう と おもう の			
\\	整備	
\\	せいび			
\\	当社は10年前に設立されたのよ。	
\\	設立 
\\	とうしゃ は じゅうねんまえ に せつりつ された の よ			
\\	彼は大手のメーカーに勤めています。	
\\	大手 
\\	かれ は おおて の めーかー に つとめて います			
\\	両社が合併に合意しました。	
\\	合意 
\\	りょうしゃ が がっぺい に ごうい しました			
\\	円高	
\\	えんだか			
\\	彼は一家のお頭だったの。	
\\	頭 
\\	かれ は いっか の おかしら だった の			
\\	あの会社の幹部は皆とても優秀だね。	
\\	幹部 
\\	あの かいしゃ の かんぶ は みんな とても ゆうしゅう だ ね			
\\	彼は新首相に就任しましたね。	
\\	就任 
\\	かれ は しんしゅしょう に しゅうにん しました ね			
\\	取材	
\\	しゅざい			
\\	お店に防犯カメラが設置されたわね。	
\\	設置 
\\	お みせ に ぼうはん かめら が せっち された わ ね			
\\	議会で区の来年の予定が話し合われたんだ。	
\\	議会 
\\	ぎかい で く の らいねん の よてい が はなしあわれた ん だ 。			
\\	庭の木、大きく成長したわね。	
\\	成長 
\\	にわ の き、 おおきく せいちょう した わ ね			
\\	証券	
\\	しょうけん			
\\	彼は意識を失いました。	
\\	意識 
\\	かれ は いしき を うしないました			
\\	新たな計画が進んでいます。	
\\	新た 
\\	あらた な けいかく が すすんで います			
\\	設立	
\\	せつりつ			
\\	大手チェーン店が進出している。	
\\	大手 
\\	進出 
\\	おおて ちぇーんてん が しんしゅつ して いる			
\\	この古い寺院は今でも完全な形を保っているんだ。	
\\	完全 
\\	この ふるい じいん は いま で も かんぜん な かたち を たもって いる ん だ			
\\	委員会はその問題について協議しましたわ。	
\\	協議 
\\	委員会 
\\	いいんかい は その もんだい に ついて きょうぎ しました わ			
\\	大手	
\\	おおて			
\\	需要が多過ぎて生産が追い付きません。	
\\	需要 
\\	じゅよう が おおすぎ て せいさん が おいつきません			
\\	事態は深刻です。	
\\	事態 
\\	深刻 
\\	じたい は しんこく です			
\\	入学式の後、教科書を購入してください。	
\\	購入 
\\	式 
\\	にゅうがくしき の あと きょうかしょ を こうにゅう して ください			
\\	合意	
\\	ごうい			
\\	ご指摘いただきありがとうございます。	
\\	指摘 
\\	ごしてき いただき ありがとう ございます			
\\	彼は政治に関心が強いね。	
\\	関心 
\\	かれ は せいじ に かんしん が つよい ね			
\\	外相は来週訪米の予定です。	
\\	外相 
\\	がいしょう は らいしゅう ほうべい の よてい です			
\\	頭	
\\	かしら			
\\	私たちはレンタカーで移動しました。	
\\	移動 
\\	わたしたち は れんたかー で いどう しました			
\\	コンピューターをネットワークに接続しました。	
\\	接続 
\\	こんぴゅーたー を ねっとわーく に せつぞく しました			
\\	当社の製品は3年間の保障つきです。	
\\	製品 
\\	とうしゃ の せいひん は さんねんかん の ほしょうつき です			
\\	幹部	
\\	かんぶ			
\\	日本は石油のほとんどを輸入しています。	
\\	石油 
\\	にっぽん は せきゆ の ほとんど を ゆにゅう して います			
\\	彼は金融関係の会社で働いています。	
\\	金融 
\\	かれ は きんゆう かんけい の かいしゃ で はたらいて います			
\\	就任	
\\	しゅうにん			
\\	彼女は昨日から私のことを避けているようなんだ。	
\\	避ける 
\\	かのじょ は きのう から わたし の こと を さけて いる よう なん だ			
\\	自分自身を疑ってはいけません。	
\\	自身 
\\	じぶん じしん を うたがって は いけません			
\\	ここは職員専用の出入り口です。	
\\	職員 
\\	ここ は しょくいん せんよう の でいりぐち です			
\\	設置	
\\	せっち			
\\	彼の支援がなかったらどうなっていたか。	
\\	支援 
\\	かれ の しえん が なかったら どう なって いた か			
\\	福祉予算は削減されないことに決まったのよ。	
\\	削減 
\\	ふくしよさん は さくげん されない こと に きまった の よ			
\\	あの会社との契約が成立しました。	
\\	成立 
\\	あの かいしゃ と の けいやく が せいりつ しました			
\\	議会	
\\	ぎかい			
\\	彼女はインターネットを活用しているの。	
\\	活用 
\\	かのじょ は いんたーねっと を かつよう して いる の			
\\	各地で大雨が降っています。	
\\	各地 
\\	かくち で おおあめ が ふって います			
\\	このプリンターで写真を出力できます。	
\\	出力 
\\	この ぷりんたー で しゃしん を しゅつりょく できます			
\\	成長	
\\	せいちょう			
\\	彼は地震で親を失いました。	
\\	失う 
\\	かれ は じしん で おや を うしないました			
\\	明日は市内を観光する予定です。	
\\	市内 
\\	あした は しない を かんこう する よてい です 。			
\\	写真を額に入れて飾ったんだ。	
\\	額 
\\	しゃしん を がく に いれて かざった ん だ			
\\	意識	
\\	いしき			
\\	全員で技術の向上に努めています。	
\\	向上 
\\	ぜんいん で ぎじゅつ の こうじょう に つとめて います			
\\	私は彼の要望に応じました。	
\\	応じる 
\\	わたし は かれ の ようぼう に おうじました			
\\	最近、銀行の金利が少し上がった。	
\\	金利 
\\	さいきん ぎんこう の きんり が すこし あがった			
\\	新た	
\\	あらた			
\\	問題を解決できないのが現状です。	
\\	現状 
\\	もんだい を かいけつ できない の が げんじょう です			
\\	その問題は審議中です。	
\\	審議 
\\	その もんだい は しんぎちゅう です			
\\	進出	
\\	しんしゅつ			
\\	現実は予想より厳しかった。	
\\	現実 
\\	げんじつ は よそう より きびしかった			
\\	彼の行動は理解できない。	
\\	行動 
\\	かれ の こうどう は りかい できない			
\\	完全	
\\	かんぜん			
\\	高速バスで東京に行ったんだ。	
\\	高速 
\\	こうそくばす で とうきょう に いった ん だ			
\\	それは果物の一種です。	
\\	一種 
\\	それ は くだもの の いっしゅ です			
\\	協議	
\\	きょうぎ			
\\	彼は暴力事件を起こしたんだよ。	
\\	起こす 
\\	かれ は ぼうりょく じけん を おこした ん だ よ			
\\	都心と地方では収入に大きな差があるね。	
\\	差 
\\	収入 
\\	としん と ちほう で は しゅうにゅう に おおき な さ が ある ね			
\\	大きくなったら政治家になりたいです。	
\\	政治家 
\\	おおきく なったら せいじか に なりたい です			
\\	需要	
\\	じゅよう			
\\	その会社は欧米に進出しているよね。	
\\	欧米 
\\	その かいしゃ は おうべい に しんしゅつ して いる よ ね			
\\	会計士に資金運用について相談したよ。	
\\	運用 
\\	かいけいし に しきん うんよう に ついて そうだん した よ			
\\	彼は営業を担当しています。	
\\	営業 
\\	かれ は えいぎょう を たんとう しています 。			
\\	事態	
\\	じたい			
\\	景気が回復してきたね。	
\\	景気 
\\	けいき が かいふく して きた ね			
\\	社長の辞任が決まりました。	
\\	辞任 
\\	しゃちょう の じにん が きまりました			
\\	学会は予定通りに進行しています。	
\\	進行 
\\	がっかい は よてい どおり に しんこう して います			
\\	購入	
\\	こうにゅう			
\\	今年の応募作品は水準が高かったね。	
\\	水準 
\\	ことし の おうぼ さくひん は すいじゅん が たかかった ね			
\\	首相は国家のリーダーだ。	
\\	国家 
\\	しゅしょう は こっか の りーだー だ			
\\	もっと案を出し合いましょう。	
\\	案 
\\	もっと あん を だしあいましょう			
\\	指摘	
\\	してき			
\\	医師に入院を勧められたの。	
\\	医師 
\\	いし に にゅういん を すすめられた の			
\\	家族で話し合う機会を持ちました。	
\\	機会 
\\	かぞく で はなしあう きかい を もちました			
\\	関心	
\\	かんしん			
\\	体がすっかり回復した。	
\\	回復 
\\	すっかり 
\\	からだ が すっかり かいふく した			
\\	彼女は何とか健康を維持しているわね。	
\\	維持 
\\	かのじょ は なんとか けんこう を いじ して いる わ ね			
\\	この研究所には最新の設備が揃っています。	
\\	設備 
\\	この けんきゅうじょ に は さいしん の せつび が そろって います			
\\	外相	
\\	がいしょう			
\\	彼は結局何を言いたかったのだろう。	
\\	結局 
\\	かれ は けっきょく なに を いいたかった の だろう			
\\	当時のことはしっかり記憶しているよ。	
\\	記憶 
\\	とうじ の こと は しっかり きおく して いる よ			
\\	現地の天気は雨です。	
\\	現地 
\\	げんち の てんき は あめ です			
\\	移動	
\\	いどう			
\\	2社が共同で新製品を開発したんだ。	
\\	共同 
\\	にしゃ が きょうどう で しんせいひん を かいはつ した ん だ			
\\	その事故では2人死亡したの。	
\\	死亡 
\\	その じこ で は ふたり しぼう した の			
\\	私は彼の意見を受け入れました。	
\\	受け入れる 
\\	わたし は かれ の いけん を うけいれました			
\\	接続	
\\	せつぞく			
\\	両者の間に摩擦が生じています。	
\\	生じる 
\\	りょうしゃ の あいだ に まさつ が しょうじて います			
\\	今度のパソコンは性能がすごく良い。	
\\	性能 
\\	こんど の ぱそこん は せいのう が すごく いい			
\\	私たちのチームに彼が加わった。	
\\	加わる 
\\	わたしたち の ちーむ に かれ が くわわった			
\\	製品	
\\	せいひん			
\\	あなたにまで迷惑が及んでごめんなさい。	
\\	及ぶ 
\\	あなた に まで めいわく が およんで ごめんなさい			
\\	彼は株式の売買で多額の利益を得たのさ。	
\\	株式 
\\	かれ は かぶしき の ばいばい で たがく の りえき を えた の さ			
\\	今は骨が形成される大切な時期です。	
\\	形成 
\\	いま は ほね が けいせい される たいせつ な じき です			
\\	石油	
\\	せきゆ			
\\	会場は人で一杯になりました。	
\\	会場 
\\	かいじょう は ひと で いっぱい に なりました			
\\	議長が開会の挨拶をしました。	
\\	議長 
\\	ぎちょう が かいかい の あいさつ を しました			
\\	彼は精神の安定が必要よ。	
\\	安定 
\\	かれ は せいしん の あんてい が ひつよう よ			
\\	金融	
\\	きんゆう			
\\	チーム全員に名札が配られました。	
\\	全員 
\\	ちーむ ぜんいん に なふだ が くばられました			
\\	父は議論好きです。	
\\	議論 
\\	ちち は ぎろんずき です			
\\	インターネットは情報の概念を変えたよね。	
\\	概念 
\\	いんたーねっと は じょうほう の がいねん を かえた よ ね			
\\	避ける	
\\	さける			
\\	位が上がれば責任も増えるものだ。	
\\	位 
\\	くらい が あがれ ば せきにん も ふえる もの だ			
\\	図書館で資料を借りて来ました。	
\\	資料 
\\	としょかん で しりょう を かりて きました			
\\	自身	
\\	じしん			
\\	ここは昔、軍事基地でした。	
\\	基地 
\\	ここ は むかし ぐんじ きち でした			
\\	以前彼はこの町に住んでいたんだ。	
\\	以前 
\\	いぜん かれ は この まち に すんで いた ん だ			
\\	小型のスーツケースを買いました。	
\\	小型 
\\	こがた の すーつけーす を かいました			
\\	職員	
\\	しょくいん			
\\	警察の調べで女性の身元がわかりました。	
\\	調べ 
\\	けいさつ の しらべ で じょせい の みもと が わかりました			
\\	日本は消費大国といわれています。	
\\	消費 
\\	にっぽん は しょうひ たいこく と いわれて います			
\\	その事業は国が運営しています。	
\\	運営 
\\	その じぎょう は くに が うんえい して います			
\\	支援	
\\	しえん			
\\	僕に隠していることは無いですか。	
\\	隠す 
\\	ぼく に かくして いる こと は ない です か			
\\	あの企業では学歴が重視されるよ。	
\\	重視 
\\	あの きぎょう で は がくれき が じゅうし される よ			
\\	一度の失敗であきらめてはいけないよ。	
\\	失敗 
\\	いちど の しっぱい で あきらめては いけない よ			
\\	削減	
\\	さくげん			
\\	私は世界情勢を知るために毎日ニュースを見るわ。	
\\	情勢 
\\	わたし は せかい じょうせい を しる ため に まいにち にゅーす を みる わ			
\\	彼は人物を描くのがうまいな。	
\\	描く 
\\	人物 
\\	かれ は じんぶつ を えがく の が うまい な			
\\	近く交通法が改正されます。	
\\	改正 
\\	ちかく こうつう ほう が かいせい されます			
\\	成立	
\\	せいりつ			
\\	彼は会社の重要な人物です。	
\\	人物 
\\	かれ は かいしゃ の じゅうよう な じんぶつ です			
\\	明日、委員会が開かれます。	
\\	委員会 
\\	あした いいんかい が ひらかれます			
\\	村では最近、不思議な現象が起きています。	
\\	現象 
\\	むら で は さいきん ふしぎ な げんしょう が おきて います			
\\	活用	
\\	かつよう			
\\	私たちは共通の趣味を持っています。	
\\	共通 
\\	わたしたち は きょうつう の しゅみ を もって います			
\\	そちらに着くのは6時前後です。	
\\	前後 
\\	そちら に つく の は ろくじ ぜんご です			
\\	新しい代表チームが結成されました。	
\\	結成 
\\	あたらしい だいひょう ちーむ が けっせい されました			
\\	各地	
\\	かくち			
\\	3人分の代金は計6000円です。	
\\	計 
\\	さんにんぶん の だいきん は けい ろくせんえん です			
\\	彼は大きな荷物を抱えているわ。	
\\	抱える 
\\	かれ は おおき な にもつ を かかえて いる わ			
\\	上司の指示に従った。	
\\	従う 
\\	じょうし の しじ に したがった			
\\	出力	
\\	しゅつりょく			
\\	戦後の日本は混乱していました。	
\\	戦後 
\\	せんご の にほん は こんらん して いました			
\\	彼は目的のためには手段を選ばなかったわね。	
\\	手段 
\\	かれ は もくてき の ため に は しゅだん を えらばなかった わ ね			
\\	失う	
\\	うしなう			
\\	代金には規定の手数料が含まれます。	
\\	規定 
\\	だいきん に は きてい の てすうりょう が ふくまれます			
\\	相手会社から強い圧力が掛かった。	
\\	圧力 
\\	あいてがいしゃ から つよい あつりょく が かかった			
\\	会社は従業員の数を増やす予定だ。	
\\	従業員 
\\	かいしゃ は じゅうぎょういん の かず を ふやす よてい だ			
\\	市内	
\\	しない			
\\	彼は新しい小説の構想を練っているの。	
\\	構想 
\\	かれ は あたらしい しょうせつ の こうそう を ねって いる の			
\\	日本の古い料理法についての記述を読んだんだ。	
\\	記述 
\\	にほん の ふるい りょうりほう に ついて の きじゅつ を よんだ ん だ			
\\	母は苦労して私たちを育てたの。	
\\	苦労 
\\	はは は くろう して わたしたち を そだてた の			
\\	額	
\\	がく			
\\	日時はご都合に合わせます。	
\\	合わせる 
\\	にちじ は ごつごう に あわせます			
\\	勉強に集中しなさい。	
\\	集中 
\\	べんきょう に しゅうちゅう しなさい			
\\	これは極めて重要な問題です。	
\\	極めて 
\\	これ は きわめて じゅうよう な もんだい です			
\\	向上	
\\	こうじょう			
\\	彼は営業で実績を上げたんだ。	
\\	実績 
\\	かれ は えいぎょう で じっせき を あげた ん だ			
\\	この犬は穏やかな性質だよ。	
\\	性質 
\\	この いぬ は おだやか な せいしつ だ よ			
\\	父は以前、国鉄に勤めていました。	
\\	国鉄 
\\	ちち は いぜん こくてつ に つとめて いました			
\\	応じる	
\\	おうじる			
\\	将来はパイロットになりたいです。	
\\	将来 
\\	しょうらい は パイロット に なりたい です 。			
\\	環境問題に関する記事を読んだの。	
\\	記事 
\\	かんきょう もんだい に かんする きじ を よんだ の			
\\	交差点付近は駐車禁止です。	
\\	禁止 
\\	こうさてん ふきん は ちゅうしゃ きんし です			
\\	金利	
\\	きんり			
\\	彼女は国際的に有名な歌手です。	
\\	国際的 
\\	かのじょ は こくさいてき に ゆうめい な かしゅ です			
\\	ここのパネルで機械全体を制御できます。	
\\	制御 
\\	ここ の ぱねる で きかい ぜんたい を せいぎょ できます			
\\	現状	
\\	げんじょう			
\\	彼らは家を建築中です。	
\\	建築 
\\	かれら は いえ を けんちくちゅう です			
\\	この会社の売り上げは昨年の2倍ね。	
\\	売り上げ 
\\	この かいしゃ の うりあげ は さくねん の に ばい ね			
\\	審議	
\\	しんぎ			
\\	図書館には各種の雑誌が揃っています。	
\\	各種 
\\	としょかん に は かくしゅ の ざっし が そろって います			
\\	彼は自分の人生を振り返ったわ。	
\\	人生 
\\	かれ は じぶん の じんせい を ふりかえった わ			
\\	現実	
\\	げんじつ			
\\	町の青年たちはボランティア活動をしています。	
\\	青年 
\\	まち の せいねんたち は ぼらんてぃあ かつどう を して います			
\\	政府は外交に力を入れているの。	
\\	外交 
\\	せいふ は がいこう に ちから を いれて いる の			
\\	これまでで最高の結果が出たよ。	
\\	最高 
\\	これまで で さいこう の けっか が でた よ			
\\	行動	
\\	こうどう			
\\	数学が平均点以下だった。	
\\	以下 
\\	すうがく が へいきんてん いか だった			
\\	アルミニウムは金属の一種です。	
\\	金属 
\\	あるみにうむ は きんぞく の いっしゅ です			
\\	日本語能力試験を受験したんだ。	
\\	受験 
\\	にほんご のうりょく しけん を じゅけん した ん だ			
\\	高速	
\\	こうそく			
\\	プログラムの過程に問題があるの。	
\\	過程 
\\	ぷろぐらむ の かてい に もんだい が ある の			
\\	あれは実は私の勘違いでした。	
\\	実は 
\\	あれ は じつは わたし の かんちがい でした			
\\	牛肉の輸入が厳しく規制されているわね。	
\\	規制 
\\	ぎゅうにく の ゆにゅう が きびしく きせい されて いる わ ね			
\\	一種	
\\	いっしゅ			
\\	ご注文の品は明日出荷致します。	
\\	出荷 
\\	ごちゅうもん の しな は あす しゅっか いたします			
\\	彼は本社で働くことを希望しています。	
\\	希望 
\\	かれ は ほんしゃ で はたらく こと を きぼう して います			
\\	正式な招待状を受け取りました。	
\\	正式 
\\	せいしき な しょうたいじょう を うけとりました			
\\	起こす	
\\	おこす			
\\	税に関する新しい法律が定められたぞ。	
\\	定める 
\\	ぜい に かんする あたらしい ほうりつ が さだめられた ぞ			
\\	コンピュータの電子回路が故障した。	
\\	回路 
\\	こんぴゅーた の でんし かいろ が こしょう した			
\\	彼の会社は電力を供給しています。	
\\	供給 
\\	かれ の かいしゃ は でんりょく を きょうきゅう して います			
\\	差	
\\	さ			
\\	この国は人口が減少しているわね。	
\\	減少 
\\	この くに は じんこう が げんしょう して いる わ ね			
\\	指定された席にお座りください。	
\\	指定 
\\	してい された せき に お すわり ください			
\\	政治家	
\\	せいじか			
\\	彼は人事を担当しているんだ。	
\\	人事 
\\	かれ は じんじ を たんとう して いる ん だ			
\\	僕と姉の性格は正反対です。	
\\	性格 
\\	ぼく と あね の せいかく は せいはんたい です			
\\	現代の科学の進歩には驚くよ。	
\\	現代 
\\	げんだい の かがく の しんぽ に は おどろく よ			
\\	欧米	
\\	おうべい			
\\	原則としてキャンセルできません。	
\\	原則 
\\	げんそく と して きゃんせる できません			
\\	彼は食生活を改善しました。	
\\	改善 
\\	かれ は しょくせいかつ を かいぜん しました			
\\	運用	
\\	うんよう			
\\	皆の意見が一致しました。	
\\	一致 
\\	みんな の いけん が いっち しました			
\\	棒をテープで固定しなさい。	
\\	固定 
\\	ぼう を てーぷ で こてい しなさい			
\\	その会社の経営の実態を調査中だ。	
\\	実態 
\\	その かいしゃ の けいえい の じったい を ちょうさちゅう だ			
\\	営業	
\\	えいぎょう			
\\	去る者は追わず。	
\\	去る 
\\	追う 
\\	さる もの は おわず			
\\	7時5分に駅で集合しましょう。	
\\	集合 
\\	しち じ ご ふん に えき で しゅうごう しましょう			
\\	彼は校則に違反しました。	
\\	違反 
\\	かれ は こうそく に いはん しました			
\\	景気	
\\	けいき			
\\	彼は軍事に関わる仕事を続けてきたんだ。	
\\	軍事 
\\	かれ は ぐんじ に かかわる しごと を つづけて きた ん だ			
\\	必死の思いで彼に頼んだよ。	
\\	思い 
\\	ひっし の おもい で かれ に たのんだ よ			
\\	彼女はクラブの会員です。	
\\	会員 
\\	かのじょ は くらぶ の かいいん です			
\\	辞任	
\\	じにん			
\\	なんとか親を説得してみるよ。	
\\	説得 
\\	なんとか おや を せっとく して みる よ			
\\	やっと宿題を終えた。	
\\	終える 
\\	やっと しゅくだい を おえた			
\\	核戦争は絶対に防ぐべきよ。	
\\	核 
\\	かくせんそう は ぜったい に ふせぐ べき よ			
\\	進行	
\\	しんこう			
\\	このことを決して忘れないでください。	
\\	決して 
\\	この こと を けっして わすれない で ください			
\\	厳しい練習が良い成果に結び付きました。	
\\	成果 
\\	きびしい れんしゅう が よい せいか に むすびつきました			
\\	学校で避難訓練がありました。	
\\	訓練 
\\	がっこう で ひなん くんれん が ありました			
\\	水準	
\\	すいじゅん			
\\	日本人の平均収入はどの位ですか。	
\\	位 
\\	収入 
\\	にほんじん の へいきん しゅうにゅう は どの くらい です か			
\\	私は彼と同じ世代です。	
\\	世代 
\\	わたし は かれ と おなじ せだい です			
\\	ようやく結論に至ったようね	
\\	至る 
\\	ようやく けつろん に いたった よう ね			
\\	国家	
\\	こっか			
\\	警官は怪しい男の後を追ったよ。	
\\	追う 
\\	けいかん は あやしい おとこ の あと を おった よ			
\\	あの国の経済は急速に発展しているのね。	
\\	急速 
\\	あの くに の けいざい は きゅうそく に はってん して いる の ね			
\\	案	
\\	あん			
\\	彼らは長い距離を歩き続けたんだ。	
\\	距離 
\\	かれら は ながい きょり を あるきつづけた ん だ			
\\	書類は形式を守って作ってください。	
\\	形式 
\\	しょるい は けいしき を まもって つくって ください			
\\	健康が一番大切だ。	
\\	健康 
\\	けんこう が いちばん たいせつ だ 。			
\\	医師	
\\	いし			
\\	彼は生徒の指導が上手ね。	
\\	指導 
\\	かれ は せいと の しどう が じょうず ね			
\\	彼は盗みの疑いをかけられたの。	
\\	疑い 
\\	かれ は ぬすみ の うたがい を かけられた の			
\\	優れた人材の確保は重要です。	
\\	確保 
\\	優れる 
\\	すぐれた じんざい の かくほ は じゅうよう です			
\\	機会	
\\	きかい			
\\	いろいろ工夫して仕事をやりとげたさ。	
\\	工夫 
\\	いろいろ くふう して しごと を やりとげた さ			
\\	試合の相手とユニフォームを交換したよ。	
\\	交換 
\\	しあい の あいて と ゆにふぉーむ を こうかん した よ			
\\	医療関係の仕事をしています。	
\\	医療 
\\	いりょう かんけい の しごと を して います			
\\	大蔵省	
\\	おおくらしょう			
\\	その会社はリサイクルを推進してるね。	
\\	推進 
\\	その かいしゃ は りさいくる を すいしん して いる ね			
\\	この頃は、ローマが勢力を伸ばしていました。	
\\	勢力 
\\	この ころ は ろーま が せいりょく を のばして いました			
\\	回復	
\\	かいふく			
\\	会議で航空の安全について話し合ったの。	
\\	航空 
\\	かいぎ で こうくう の あんぜん に ついて はなしあった の			
\\	2国間で条約が結ばれました。	
\\	条約 
\\	にこくかん で じょうやく が むすばれました			
\\	年金問題は行政の最大の課題のひとつよ。	
\\	行政 
\\	ねんきん もんだい は ぎょうせい の さいだい の かだい の ひとつ よ			
\\	維持	
\\	いじ			
\\	限りある資源を大切にしよう。	
\\	限り 
\\	かぎり ある しげん を たいせつ に しよう			
\\	この法律は消費者を守るためのものです。	
\\	消費 
\\	消費者 
\\	この ほうりつ は しょうひしゃ を まもる ため の もの です			
\\	石油はいろいろな製品の原料になります。	
\\	原料 
\\	せきゆ は いろいろ な せいひん の げんりょう に なります			
\\	設備	
\\	せつび			
\\	国の財政はとても苦しい状態だな。	
\\	財政 
\\	くに の ざいせい は とても くるしい じょうたい だ な			
\\	死を恐れるのは自然なことです。	
\\	死 
\\	し を おそれる の は しぜん な こと です			
\\	このレシピはいろいろ応用できます。	
\\	応用 
\\	この れしぴ は いろいろ おうよう できます			
\\	結局	
\\	けっきょく			
\\	ビザについて外務省に問い合わせた。	
\\	外務省 
\\	びざ に ついて がいむしょう に といあわせた			
\\	沖にボートが浮かんでいます。	
\\	浮かぶ 
\\	おき に ぼーと が うかんで います			
\\	今ドイツ語の基礎を学んでいます。	
\\	基礎 
\\	いま どいつご の きそ を まなんで います			
\\	記憶	
\\	きおく			
\\	地球環境の危機が叫ばれているのよ。	
\\	危機 
\\	ちきゅう かんきょう の きき が さけばれて いる の よ			
\\	彼は非常に優れた選手です。	
\\	優れる 
\\	かれ は ひじょうに すぐれた せんしゅ です			
\\	彼の意向を聞いてみましょう。	
\\	意向 
\\	かれ の いこう を きいて みましょう			
\\	現地	
\\	げんち			
\\	雨が一層激しくなったね。	
\\	一層 
\\	あめ が いっそう はげしく なった ね			
\\	叔父には心臓病の恐れがあります。	
\\	恐れ 
\\	おじ に は しんぞうびょう の おそれ が あります			
\\	共同	
\\	きょうどう			
\\	彼はその会社と雇用契約を結んだの。	
\\	雇用 
\\	かれ は その かいしゃ と こよう けいやく を むすんだ の			
\\	あの子は感じの良い子です。	
\\	感じ 
\\	あの こ は かんじ の いい こ です			
\\	死亡	
\\	しぼう			
\\	彼は地方で演説したのよ。	
\\	演説 
\\	かれ は ちほう で えんぜつ した の よ			
\\	彼に全てのことを伝えた。	
\\	全て 
\\	かれ に すべて の こと を つたえた			
\\	この仕事には少なくとも2週間必要でしょう。	
\\	少なくとも 
\\	この しごと に は すくなくとも にしゅうかん ひつよう でしょう			
\\	受け入れる	
\\	うけいれる			
\\	東京行きの最終電車は何時ですか。	
\\	最終 
\\	とうきょうゆき の さいしゅう でんしゃ は なんじ です か			
\\	彼らは教会で式を挙げました。	
\\	式 
\\	かれら は きょうかい で しき を あげました			
\\	ドラマの後半は来週放送されるんだ。	
\\	後半 
\\	どらま の こうはん は らいしゅう ほうそう される ん だ			
\\	生じる	
\\	しょうじる			
\\	その都市は商業が盛んよ。	
\\	盛ん 
\\	その とし は しょうぎょう が さかん よ			
\\	来月、隣の州に引っ越します。	
\\	州 
\\	らいげつ となり の しゅう に ひっこします			
\\	性能	
\\	せいのう			
\\	北欧では社会保障が確立されています。	
\\	確立 
\\	ほくおう で は しゃかい ほしょう が かくりつ されて います			
\\	去年は航空機の事故が多かった。	
\\	航空機 
\\	きょねん は こうくうき の じこ が おおかった			
\\	加わる	
\\	くわわる			
\\	焦点を絞って話しましょう。	
\\	焦点 
\\	しょうてん を しぼって はなしましょう			
\\	鼻水は風邪の初期症状のひとつです。	
\\	初期 
\\	はなみず は かぜ の しょき しょうじょう の ひとつ です			
\\	彼女は出席を拒否した。	
\\	拒否 
\\	かのじょ は しゅっせき を きょひ した			
\\	及ぶ	
\\	およぶ			
\\	まずは障害を取り除いてから計画を進めよう。	
\\	障害 
\\	まず は しょうがい を とりのぞいて から けいかく を すすめよう			
\\	産業はインドの主要産業よね。	
\\	主要 
\\	あいてぃー さんぎょう は いんど の しゅよう さんぎょう よ ね			
\\	これは深刻な問題です。	
\\	深刻 
\\	これ は しんこく な もんだい で す			
\\	株式	
\\	かぶしき			
\\	作業の効率を上げるにはどうしたらいいですか。	
\\	効率 
\\	さぎょう の こうりつ を あげる に は どう したら いい です か			
\\	彼の活躍で優勝したよ。	
\\	活躍 
\\	かれ の かつやく で ゆうしょう した よ			
\\	これからは自己の判断で行動してください。	
\\	自己 
\\	これ から は じこ の はんだん で こうどう して ください			
\\	形成	
\\	けいせい			
\\	その選手の獲得賞金は2億円だって。	
\\	獲得 
\\	その せんしゅ の かくとく しょうきん は におくえん だって			
\\	その監督はアカデミー賞を受賞したよな。	
\\	監督 
\\	その かんとく は あかでみーしょう を じゅしょう した よ な			
\\	その問題に対する策を皆で考えたの。	
\\	策 
\\	その もんだい に たいする さく を みんな で かんがえた の			
\\	会場	
\\	かいじょう			
\\	彼は怒りを抑えていたの。	
\\	抑える 
\\	かれ は いかり を おさえて いた の			
\\	親切は必ずしも喜ばれるわけではない。	
\\	必ずしも 
\\	しんせつ は かならずしも よろこばれる わけ で は ない			
\\	あの二人はそっくりで区別できない。	
\\	区別 
\\	そっくり 
\\	あの ふたり は そっくり で くべつ できない			
\\	議長	
\\	ぎちょう			
\\	将来は海の近くで暮らしたいな。	
\\	将来 
\\	暮らす 
\\	しょうらい は うみ の ちかく で くらしたい な			
\\	夕方以降にお電話を下さい。	
\\	以降 
\\	ゆうがた いこう に お でんわ を ください			
\\	大臣の資産が公表されたね。	
\\	資産 
\\	だいじん の しさん が こうひょう された ね			
\\	安定	
\\	あんてい			
\\	彼は緊張していたな。	
\\	緊張 
\\	かれ は きんちょう して いた な			
\\	その広告を新聞で見ました。	
\\	広告 
\\	その こうこく を しんぶん で みました			
\\	彼女は食事を制限しています。	
\\	制限 
\\	かのじょ は しょくじ を せいげん して います			
\\	全員	
\\	ぜんいん			
\\	自宅から職場まで1時間かかります。	
\\	職場 
\\	じたく から しょくば まで いちじかん かかります			
\\	最近ヨガをやる人が急増しています。	
\\	急増 
\\	さいきん よが を やる ひと が きゅうぞう して います			
\\	彼の今月の業績は素晴らしいです。	
\\	業績 
\\	かれ の こんげつ の ぎょうせき は すばらしい です			
\\	議論	
\\	ぎろん			
\\	このケースには
\\	が50枚入ります。	
\\	ケース 
\\	この けーす に は 
\\	が 
\\	まい はいります			
\\	彼には清潔なイメージがあるね。	
\\	イメージ 
\\	かれ に は せいけつ な いめーじ が ある ね			
\\	自分の失敗を他人のせいにするな。	
\\	せい 
\\	じぶん の しっぱい を たにん の せい に する な			
\\	概念	
\\	がいねん			
\\	彼女の様子がいつもと違う。	
\\	いつも 
\\	かのじょ の ようす が いつも と ちがう			
\\	大学に通うのが上京のきっかけでした。	
\\	きっかけ 
\\	だいがく に かよう の が じょうきょう の きっかけ でした			
\\	買い物をすっかり忘れていた。	
\\	すっかり 
\\	かいもの を すっかり わすれて いた			
\\	位	
\\	くらい			
\\	このシリーズは50話もあるそうよ。	
\\	シリーズ 
\\	この しりーず は ごじゅうわ も ある そう よ			
\\	その国ではエイズが大きな社会問題ね。	
\\	エイズ 
\\	その くに で は えいず が おおき な しゃかい もんだい ね			
\\	車のエンジンが調子悪い。	
\\	エンジン 
\\	くるま の えんじん が ちょうし わるい			
\\	慎重	
\\	しんちょう			
\\	もう一度慎重に見直しましょう。	
\\	慎重 
\\	もういちど しんちょう に みなおしましょう			
\\	この荷物は丁寧に扱って下さい。	
\\	扱う 
\\	この にもつ は ていねい に あつかって ください			
\\	一般に老人は早起きです。	
\\	一般に 
\\	いっぱんに ろうじん は はやおき です			
\\	扱う	
\\	あつかう			
\\	今日は野外で学習した。	
\\	学習 
\\	きょう は やがい で がくしゅう した			
\\	国連の本部はニューヨークにあります。	
\\	国連 
\\	こくれん の ほんぶ は にゅーよーく に あります			
\\	カウンターで料金を支払った。	
\\	支払う 
\\	かうんたー で りょうきん を しはらった			
\\	一般に	
\\	いっぱんに			
\\	彼の行為はみんなの誤解を招いたわ。	
\\	行為 
\\	かれ の こうい は みんな の ごかい を まねいた わ			
\\	困難にあってもあきらめてはいけないよ。	
\\	困難 
\\	こんなん に あって も あきらめて は いけない よ			
\\	容疑者のアリバイが証明されました。	
\\	証明 
\\	ようぎしゃ の ありばい が しょうめい されました			
\\	学習	
\\	がくしゅう			
\\	温かい歓迎を受けました。	
\\	歓迎 
\\	あたたかい かんげい を うけました			
\\	私は部下を信頼しています。	
\\	信頼 
\\	わたし は ぶか を しんらい して います			
\\	京都で3日間観光しました。	
\\	観光 
\\	きょうと で みっかかん かんこう しました			
\\	国連	
\\	こくれん			
\\	レシートで買い物の金額を確かめたよ。	
\\	金額 
\\	れしーと で かいもの の きんがく を たしかめた よ			
\\	このドアは自動よ。	
\\	自動 
\\	この どあ は じどう よ			
\\	これから高齢化が急激に進みます。	
\\	急激 
\\	これから こうれいか が きゅうげき に すすみます			
\\	支払う	
\\	しはらう			
\\	今日は組合の集まりがあるわ。	
\\	組合 
\\	きょう は くみあい の あつまり が ある わ			
\\	時計が12時を指してる。	
\\	指す 
\\	とけい が じゅうにじ を さして る			
\\	冷めないうちにどうぞお上がり下さい。	
\\	上がる 
\\	さめない うち に どうぞ おあがり ください			
\\	行為	
\\	こうい			
\\	これは一種独特の味がするね。	
\\	一種 
\\	これ は いっしゅ どくとく の あじ が する ね			
\\	大声で話すと周囲の人に迷惑ですよ。	
\\	周囲 
\\	おおごえ で はなす と しゅうい の ひと に めいわく です よ			
\\	撮影は3ヶ月かけて行われました。	
\\	撮影 
\\	さつえい は さんかげつ かけて おこなわれました			
\\	困難	
\\	こんなん			
\\	彼は支店に転勤したよ。	
\\	支店 
\\	かれ は してん に てんきん した よ			
\\	彼は委員に選ばれました。	
\\	委員 
\\	かれ は いいん に えらばれました			
\\	今日の日本の若者は欧米化している。	
\\	今日 
\\	こんにち の にほん の わかもの は おうべいか して いる			
\\	証明	
\\	しょうめい			
\\	進み具合は全て順調です。	
\\	順調 
\\	すすみ ぐあい は すべて じゅんちょう です			
\\	彼の会社は移転したよ。	
\\	移転 
\\	かれ の かいしゃ は いてん した よ			
\\	先生が星の観測に連れて行ってくれました。	
\\	観測 
\\	せんせい が ほし の かんそく に つれて いって くれました			
\\	歓迎	
\\	かんげい			
\\	彼女は心理学に興味があるよ。	
\\	興味 
\\	かのじょ は しんりがく に きょうみ が ある よ			
\\	明日の試合は何時からですか。	
\\	試合 
\\	あした の しあい は なんじ から です か			
\\	手の傷が悪化した。	
\\	悪化 
\\	て の きず が あっか した			
\\	信頼	
\\	しんらい			
\\	あの人の顔には自信があふれているね。	
\\	自信 
\\	あの ひと の かお に は じしん が あふれて いる ね			
\\	選挙では3つの政党が争っています。	
\\	政党 
\\	せんきょ で は みっつ の せいとう が あらそって います			
\\	彼は作家であり学者です。	
\\	学者 
\\	かれ は さっか で あり がくしゃ です			
\\	観光	
\\	かんこう			
\\	これは政府の承認を受けた資格です。	
\\	承認 
\\	これ は せいふ の しょうにん を うけた しかく です			
\\	夏休みを高原で過ごしました。	
\\	過ごす 
\\	なつやすみ を こうげん で すごしました			
\\	彼は課長に昇進しました。	
\\	課長 
\\	かれ は かちょう に しょうしん しました			
\\	金額	
\\	きんがく			
\\	地震の後、町は大混乱だったよ。	
\\	混乱 
\\	じしん の あと まち は だいこんらん だった よ			
\\	これからパスポートの申請に行きます。	
\\	申請 
\\	これから ぱすぽーと の しんせい に いきます			
\\	もう我慢の限界です。	
\\	限界 
\\	もう がまん の げんかい です			
\\	自動	
\\	じどう			
\\	教授はカナダ出身です。	
\\	出身 
\\	きょうじゅ は かなだ しゅっしん です			
\\	この仕事は来月の下旬には終わります。	
\\	下旬 
\\	この しごと は らいげつ の げじゅん に は おわります			
\\	新しい市長が選ばれました。	
\\	市長 
\\	あたらしい しちょう が えらばれました			
\\	急激	
\\	きゅうげき			
\\	あの巨大な建物は博物館です。	
\\	巨大 
\\	あの きょだい な たてもの は はくぶつかん です			
\\	高度な技術を持つ人材を集めました。	
\\	高度 
\\	こうど な ぎじゅつ を もつ じんざい を あつめました			
\\	姉は何に対しても積極的です。	
\\	積極的 
\\	あね は なに に たいして も せっきょくてき です			
\\	組合	
\\	くみあい			
\\	地図にはいろいろな記号が使われているのね。	
\\	記号 
\\	ちず に は いろいろ な きごう が つかわれて いる の ね			
\\	今日の話し合いでは結論が出なかった。	
\\	結論 
\\	きょう の はなしあい で は けつろん が でなかった			
\\	主婦の主な仕事は家事よ。	
\\	主婦 
\\	しゅふ の おも な しごと は かじ よ			
\\	指す	
\\	さす			
\\	私はもう決心を固めたの。	
\\	固める 
\\	わたし は もう けっしん を かためた の			
\\	緊急事態が発生した。	
\\	緊急 
\\	きんきゅう じたい が はっせい した			
\\	宇宙の謎は限りなく大きいの。	
\\	宇宙 
\\	うちゅう の なぞ は かぎり なく おおきい の			
\\	上がる	
\\	あがる			
\\	この機械の仕組みは複雑だ。	
\\	仕組み 
\\	この きかい の しくみ は ふくざつ だ			
\\	天皇は日本国の象徴です。	
\\	象徴 
\\	てんのう は にほんこく の しょうちょう です			
\\	それ以来彼女に会ってない。	
\\	以来 
\\	それ いらい かのじょ に あってない			
\\	一種	
\\	いっしゅ			
\\	独身者パーティーを企画した。	
\\	企画 
\\	どくしんしゃ ぱーてぃー を きかく した			
\\	もっと他の町との交流を深めましょう。	
\\	交流 
\\	もっと ほか の まち と の こうりゅう を ふかめましょう			
\\	あなたの事情を考慮して予定をたてました。	
\\	考慮 
\\	あなた の じじょう を こうりょ して よてい を たてました			
\\	周囲	
\\	しゅうい			
\\	裁判の様子はテレビで中継されたわよ。	
\\	裁判 
\\	さいばん の ようす は てれび で ちゅうけい された わ よ			
\\	彼から良い印象を受けたわ。	
\\	印象 
\\	かれ から いい いんしょう を うけた わ			
\\	山で登山客が熊に襲われたよ。	
\\	襲う 
\\	やま で とざん きゃく が くま に おそわれた よ			
\\	撮影	
\\	さつえい			
\\	感情とは複雑なものです。	
\\	感情 
\\	かんじょう と は ふくざつ な もの です			
\\	私は彼の意見に賛成だ。	
\\	賛成 
\\	わたし は かれ の いけん に さんせい だ			
\\	冷えて指の感覚がない。	
\\	感覚 
\\	ひえて ゆび の かんかく が ない			
\\	支店	
\\	してん			
\\	道で警官に呼び止められた。	
\\	警官 
\\	みち で けいかん に よびとめられた			
\\	これからは経費を節約しましょう。	
\\	経費 
\\	これ から は けいひ を せつやく しましょう			
\\	彼女は有名な作家です。	
\\	作家 
\\	かのじょ は ゆうめい な さっか です			
\\	委員	
\\	いいん			
\\	人類は2本の足で歩きます。	
\\	人類 
\\	じんるい は にほん の あし で あるきます			
\\	これは税金の無駄遣いだね。	
\\	税金 
\\	これ は ぜいきん の むだづかい だ ね			
\\	この映画は香港で製作されました。	
\\	製作 
\\	この えいが は ほんこん で せいさく されました			
\\	今日	
\\	こんにち			
\\	彼らは次の試合のために作戦を立てたのよ。	
\\	作戦 
\\	かれら は つぎ の しあい の ため に さくせん を たてた の よ			
\\	この本は15年前に出版された。	
\\	出版 
\\	この ほん は じゅうごねんまえ に しゅっぱん された			
\\	順調	
\\	じゅんちょう			
\\	この大学も国際化してきたな。	
\\	国際化 
\\	この だいがく も こくさいか して きた な			
\\	兄は商社に勤めています。	
\\	商社 
\\	あに は しょうしゃ に つとめて います			
\\	私たちには知る権利があります。	
\\	権利 
\\	わたしたち に は しる けんり が あります			
\\	移転	
\\	いてん			
\\	ここに去年の所得をご記入ください。	
\\	所得 
\\	ここ に きょねん の しょとく を ご きにゅう ください			
\\	彼はボールに回転を掛けたんだ。	
\\	回転 
\\	かれ は ぼーる に かいてん を かけた ん だ			
\\	久しぶりに家族全員で食卓を囲みました。	
\\	囲む 
\\	ひさしぶり に かぞく ぜんいん で しょくたく を かこみました			
\\	観測	
\\	かんそく			
\\	ものごとは根気よく継続することが大切です。	
\\	継続 
\\	もの ごと は こんき よく けいぞく する こと が たいせつ です			
\\	彼女は今、精神が不安定だ。	
\\	精神 
\\	かのじょ は いま せいしん が ふあんてい だ			
\\	興味	
\\	きょうみ			
\\	国会の会期が延長されたわね。	
\\	延長 
\\	こっかい の かいき が えんちょう された わ ね			
\\	これは外部には秘密です。	
\\	外部 
\\	これ は がいぶ に は ひみつ です			
\\	これを縮小して印刷してください。	
\\	縮小 
\\	これ を しゅくしょう して いんさつ して ください			
\\	試合	
\\	しあい			
\\	科学技術は目覚しく進歩しているの。	
\\	進歩 
\\	かがく ぎじゅつ は めざましく しんぽ して いる の			
\\	小屋に石炭の山があります。	
\\	石炭 
\\	こや に せきたん の やま が あります			
\\	少年よ大志を抱け。	
\\	抱く 
\\	しょうねん よ たいし を いだけ			
\\	悪化	
\\	あっか			
\\	あの一家は仲がいい。	
\\	一家 
\\	あの いっか は なか が いい			
\\	疲れたでしょう、一旦休みましょう。	
\\	一旦 
\\	つかれた でしょう いったん やすみましょう			
\\	彼は仕事に意欲を燃やしています。	
\\	意欲 
\\	かれ は しごと に いよく を もやして います			
\\	自信	
\\	じしん			
\\	彼の言葉が本当かは疑問です。	
\\	疑問 
\\	かれ の ことば が ほんとう か は ぎもん です			
\\	とても価値のある話を聞いたよ。	
\\	価値 
\\	とても かち の ある はなし を きいた よ			
\\	試合はすぐに再開されたの。	
\\	再開 
\\	しあい は すぐ に さいかい された の			
\\	政党	
\\	せいとう			
\\	彼女の詩は世界中で有名になったんだ。	
\\	詩 
\\	かのじょ の し は せかいじゅう で ゆうめい に なった ん だ			
\\	その事故で30人の死者が出たの。	
\\	死者 
\\	その じこ で さんじゅうにん の ししゃ が でた の			
\\	私は彼の説が正しいと思う。	
\\	説 
\\	わたし は かれ の せつ が ただしい と おもう			
\\	学者	
\\	がくしゃ			
\\	この表は年齢区分ごとの人口を表しています。	
\\	区分 
\\	この ひょう は ねんれい くぶん ごと の じんこう を あらわして います			
\\	食品は日曜日にまとめて買います。	
\\	食品 
\\	しょくひん は にちようび に まとめて かいます			
\\	彼女は久しぶりに故郷に帰りました。	
\\	故郷 
\\	かのじょ は ひさしぶり に こきょう に かえりました			
\\	承認	
\\	しょうにん			
\\	彼はこの国で大きな権力を持っているわ。	
\\	権力 
\\	かれ は この くに で おおき な けんりょく を もって いる わ			
\\	私は郊外に家を買いました。	
\\	郊外 
\\	わたし は こうがい に いえ を かいました			
\\	その男は彼女のバッグを奪ったぞ。	
\\	奪う 
\\	その おとこ は かのじょ の ばっぐ を うばった ぞ			
\\	過ごす	
\\	すごす			
\\	日本では絶えず米の品種を改良しているの。	
\\	改良 
\\	にほん で は たえず こめ の ひんしゅ を かいりょう して いる の			
\\	正しい答は後者です。	
\\	後者 
\\	ただしい こたえ は こうしゃ です			
\\	今後の方針が公式に発表されました。	
\\	公式 
\\	こんご の ほうしん が こうしき に はっぴょう されました			
\\	課長	
\\	かちょう			
\\	この本の作者は誰ですか。	
\\	作者 
\\	この ほん の さくしゃ は だれ です か			
\\	駅は逆方向です。	
\\	逆 
\\	えき は ぎゃくほうこう です			
\\	彼女は毎日の暮らしを楽しんでいますね。	
\\	暮らし 
\\	かのじょ は まいにち の くらし を たのしんで います ね			
\\	混乱	
\\	こんらん			
\\	わが社は今年度、黒字となりました。	
\\	黒字 
\\	わがしゃ は こんねんど くろじ と なりました			
\\	祖母は心臓が悪いんだ。	
\\	心臓 
\\	そぼ は しんぞう が わるい ん だ			
\\	ここにタイヤの跡があるわ。	
\\	跡 
\\	ここ に たいや の あと が ある わ			
\\	申請	
\\	しんせい			
\\	アンケートに回答しました。	
\\	回答 
\\	あんけーと に かいとう しました			
\\	休日は植物の世話をして過ごします。	
\\	植物 
\\	きゅうじつ は しょくぶつ の せわ を して すごします			
\\	試合の代表メンバーが確定したの。	
\\	確定 
\\	しあい の だいひょう めんばー が かくてい した の			
\\	限界	
\\	げんかい			
\\	荷物はここに重ねてください。	
\\	重ねる 
\\	にもつ は ここ に かさねて ください			
\\	ここは芸術を愛する国です。	
\\	芸術 
\\	ここ は げいじゅつ を あいする くに です			
\\	その会社は店舗を拡張していますね。	
\\	拡張 
\\	その かいしゃ は てんぽ を かくちょう して います ね			
\\	出身	
\\	しゅっしん			
\\	あなたの見解を聞かせてください。	
\\	見解 
\\	あなた の けんかい を きかせて ください			
\\	そのクイズ番組は脳を刺激するね。	
\\	刺激 
\\	その くいずばんぐみ は のう を しげき する ね			
\\	彼は会長に推薦されたよ。	
\\	推薦 
\\	かれ は かいちょう に すいせん された よ			
\\	下旬	
\\	げじゅん			
\\	今日は最高にいい気分だよ。	
\\	気分 
\\	きょう は さいこう に いい きぶん だ よ			
\\	公園で死体が見つかったの。	
\\	死体 
\\	こうえん で したい が みつかった の			
\\	そこの角で車同士が衝突したのよ。	
\\	衝突 
\\	そこ の かど で くるま どうし が しょうとつ した の よ			
\\	市長	
\\	しちょう			
\\	彼の成長は著しいです。	
\\	著しい 
\\	かれ の せいちょう は いちじるしい です			
\\	感動する映画でした。	
\\	感動 
\\	かんどう する えいが でした			
\\	新しい型のカメラを買いました。	
\\	型 
\\	あたらしい かた の かめら を かいました			
\\	巨大	
\\	きょだい			
\\	卒業の記念にみんなで旅行したよ。	
\\	記念 
\\	そつぎょう の きねん に みんな で りょこう した よ			
\\	手術後の経過は順調です。	
\\	経過 
\\	しゅじゅつ ご の けいか は じゅんちょう です			
\\	余りは次回使いましょう。	
\\	余り 
\\	あまり は じかい つかいましょう			
\\	高度	
\\	こうど			
\\	彼に協力を依頼しました。	
\\	依頼 
\\	かれ に きょうりょく を いらい しました			
\\	古い服を整理しました。	
\\	整理 
\\	ふるい ふく を せいり しました			
\\	彼らの海外公演が発表されたよ。	
\\	公演 
\\	かれら の かいがいこうえん が はっぴょう された よ			
\\	積極的	
\\	せっきょくてき			
\\	彼らは一斉に拍手したね。	
\\	一斉に 
\\	かれら は いっせいに はくしゅ した ね			
\\	彼は国民の支持を得たのよ。	
\\	支持 
\\	かれ は こくみん の しじ を えた の よ			
\\	その事件は次第に忘れられていったのさ。	
\\	次第に 
\\	その じけん は しだいに わすれられて いった の さ			
\\	記号	
\\	きごう			
\\	この商品は主に女性に人気がありまして。	
\\	主に 
\\	この しょうひん は おもに じょせい に にんき が ありまし て			
\\	彼は貴重な体験をしたわね。	
\\	貴重 
\\	かれ は きちょう な たいけん を した わ ね			
\\	例を幾つか挙げてみましょう。	
\\	挙げる 
\\	れい を いくつ か あげて みましょう			
\\	結論	
\\	けつろん			
\\	その川の水は汚染されています。	
\\	汚染 
\\	その かわ の みず は おせん されて います			
\\	見学者に機械の動く原理を説明しました。	
\\	原理 
\\	けんがくしゃ に きかい の うごく げんり を せつめい しました			
\\	ここは交通の便がよいですね。	
\\	交通 
\\	ここ は こうつう の べん が よい です ね			
\\	主婦	
\\	しゅふ			
\\	の値を求めなさい。	
\\	値 
\\	えっくす の あたい を もとめなさい			
\\	彼は悪い習慣を改めようとしているわね。	
\\	改める 
\\	かれ は わるい しゅうかん を あらためよう と して いる わ ね			
\\	合計金額を計算してください。	
\\	金額 
\\	合計 
\\	ごうけい きんがく を けいさん して ください			
\\	固める	
\\	かためる			
\\	宗教を持たない人もたくさんいるわ。	
\\	宗教 
\\	しゅうきょう を もたない ひと も たくさん いる わ			
\\	高校卒業後、系列の大学に進みました。	
\\	系列 
\\	こうこう そつぎょう ご けいれつ の だいがく に すすみました			
\\	テロ集団が警察に捕まりました。	
\\	集団 
\\	てろ しゅうだん が けいさつ に つかまりました			
\\	緊急	
\\	きんきゅう			
\\	彼は1年2組の生徒です。	
\\	組 
\\	かれ は いちねん にくみ の せいと です			
\\	地震のあとは津波に警戒してください。	
\\	警戒 
\\	じしん の あと は つなみ に けいかい して ください			
\\	検討の結果、案を修正したよ。	
\\	修正 
\\	けんとう の けっか あん を しゅうせい した よ			
\\	宇宙	
\\	うちゅう			
\\	私の一日の勤務時間は8時間です。	
\\	勤務 
\\	わたし の いちにち の きんむじかん は はちじかん です			
\\	このプロジェクトで私は彼と組んでいるんだ。	
\\	組む 
\\	この ぷろじぇくと で わたし は かれ と くんで いる ん だ			
\\	解答に誤りが3つ有りますよ。	
\\	誤り 
\\	かいとう に あやまり が みっつ あります よ			
\\	仕組み	
\\	しくみ			
\\	この辺一帯はリンゴ畑です。	
\\	一帯 
\\	この へん いったい は りんごばたけ です			
\\	彼女は最近元気がないですね。	
\\	元気 
\\	かのじょ は さいきん げんき が ない です ね			
\\	今月はエアコンの売り上げが好調です。	
\\	好調 
\\	こんげつ は えあこん の うりあげ が こうちょう です			
\\	象徴	
\\	しょうちょう			
\\	この製品を使用する前に、説明書をお読みください。	
\\	使用 
\\	この せいひん を しよう する まえ に せつめいしょ を およみ ください			
\\	私は前者の方が優れていると思います。	
\\	前者 
\\	わたし は ぜんしゃ の ほう が すぐれて いる と おもいます			
\\	一体何事ですか。	
\\	一体 
\\	いったい なにごと です か			
\\	以来	
\\	いらい			
\\	首相は内閣の改造を行いました。	
\\	改造 
\\	しゅしょう は ないかく の かいぞう を おこないました			
\\	そのことは既にみんな知っています。	
\\	既に 
\\	その こと は すでに みんな しって います			
\\	大いに学び、大いに遊びなさい。	
\\	大いに 
\\	おおいに まなび おおいに あそびなさい			
\\	企画	
\\	きかく			
\\	今日は学校で行事がありました。	
\\	行事 
\\	きょう は がっこう で ぎょうじ が ありました			
\\	彼は実験を試みたのよ。	
\\	試みる 
\\	かれ は じっけん を こころみた の よ			
\\	このワインは国産です。	
\\	国産 
\\	この わいん は こくさん です			
\\	交流	
\\	こうりゅう			
\\	私の話を落ち着いて聞いてください。	
\\	落ち着く 
\\	わたし の はなし を おちついて きいて ください			
\\	彼の発言に対してたくさんの抗議があったよ。	
\\	抗議 
\\	かれ の はつげん に たいして たくさん の こうぎ が あった よ			
\\	この薬は神経に作用します。	
\\	作用 
\\	神経 
\\	この くすり は しんけい に さよう します			
\\	考慮	
\\	こうりょ			
\\	私は一切その問題とは関係がありません。	
\\	一切 
\\	わたし は いっさい その もんだい と は かんけい が ありません			
\\	貧富の格差が大きくなっているな。	
\\	格差 
\\	ひんぷ の かくさ が おおきく なって いる な			
\\	日本語を教える資格を取りたいです。	
\\	資格 
\\	にほんご を おしえる しかく を とりたい です			
\\	裁判	
\\	さいばん			
\\	これは強力な接着剤ね。	
\\	強力 
\\	これ は きょうりょく な せっちゃくざい ね			
\\	証拠を見つけるまで逮捕はできません。	
\\	証拠 
\\	しょうこ を みつける まで たいほ は できません			
\\	先生に許可をもらって早退しました。	
\\	許可 
\\	せんせい に きょか を もらって そうたい しました			
\\	印象	
\\	いんしょう			
\\	コンサートは夜7時5分に終了しました。	
\\	終了 
\\	コンサート 
\\	こんさーと は よる しち じ ご ふん に しゅうりょう しました			
\\	代金は現金でお願いします。	
\\	現金 
\\	だいきん は げんきん で おねがい します			
\\	地球の資源を守りましょう。	
\\	資源 
\\	ちきゅう の しげん を まもりましょう			
\\	襲う	
\\	おそう			
\\	彼女が将来、大統領になることは確実です。	
\\	確実 
\\	かのじょ が しょうらい だいとうりょう に なる こと は かくじつ です			
\\	冬は空気が乾燥しますね。	
\\	乾燥 
\\	ふゆ は くうき が かんそう します ね			
\\	全ての人に人権がある。	
\\	人権 
\\	すべて の ひと に じんけん が ある			
\\	感情	
\\	かんじょう			
\\	金属は高温で溶かします。	
\\	高温 
\\	きんぞく は こうおん で とかします			
\\	彼は真剣に話を聞いていました。	
\\	真剣 
\\	かれ は しんけん に はなし を きいて いました			
\\	彼の決意は堅いな。	
\\	決意 
\\	かれ の けつい は かたい な			
\\	賛成	
\\	さんせい			
\\	その芸術祭は市が主催しています。	
\\	主催 
\\	その げいじゅつさい は し が しゅさい して います			
\\	あなたの意図はよく分かりました。	
\\	意図 
\\	あなた の いと は よく わかりました			
\\	留守番電話のメッセージを再生したの。	
\\	再生 
\\	るすばんでんわ の めっせーじ を さいせい した の			
\\	感覚	
\\	かんかく			
\\	そんなの常識だよ。	
\\	常識 
\\	そんな の じょうしき だ よ			
\\	彼は神経が細かいですね。	
\\	神経 
\\	かれ は しんけい が こまかい です ね			
\\	それは歴史上の大きな革命です。	
\\	革命 
\\	それ は れきしじょう の おおき な かくめい です			
\\	警官	
\\	けいかん			
\\	この線を軸にして図形を回転してください。	
\\	軸 
\\	この せん を じく に して ずけい を かいてん して ください			
\\	先生のお陰で大学に合格できました。	
\\	お陰 
\\	せんせい の おかげ で だいがく に ごうかく できました			
\\	お支払いはカードもお使いいただけます。	
\\	支払い 
\\	おしはらい は かーど も お つかい いただけます			
\\	経費	
\\	けいひ			
\\	彼は受験の失敗を恐れています。	
\\	恐れる 
\\	かれ は じゅけん の しっぱい を おそれて います			
\\	もうプリンは固まったかな。	
\\	固まる 
\\	もう ぷりん は かたまった か な			
\\	ごま油の代わりにオリーブ油を使いましょう。	
\\	代わり 
\\	ごまあぶら の かわり に おりーぶゆ を つかいましょう			
\\	作家	
\\	さっか			
\\	腹一杯食った。	
\\	食う 
\\	はらいっぱい くった			
\\	詳細はお気軽にお問い合わせ下さい。	
\\	詳細 
\\	しょうさい は おきがる に おといあわせ ください			
\\	彼女は京都の生まれです。	
\\	生まれ 
\\	かのじょ は きょうと の うまれ です			
\\	人類	
\\	じんるい			
\\	彼は少年の頃から学問が好きでした。	
\\	学問 
\\	かれ は しょうねん の ころ から がくもん が すき でした			
\\	雨の日は子供を室内で遊ばせます。	
\\	室内 
\\	あめ の ひ は こども を しつない で あそばせます			
\\	これは今までで最低の記録だ。	
\\	最低 
\\	これ は いま まで で さいてい の きろく だ			
\\	税金	
\\	ぜいきん			
\\	犯人は警察に捕まる前に自殺しました。	
\\	自殺 
\\	はんにん は けいさつ に つかまる まえ に じさつ しました			
\\	彼の講演は評判がいい。	
\\	講演 
\\	かれ の こうえん は ひょうばん が いい			
\\	これはとても実用的なサイトだね。	
\\	実用 
\\	これ は とても じつようてき な さいと だ ね			
\\	製作	
\\	せいさく			
\\	新しい車が売り出された。	
\\	売り出す 
\\	あたらしい くるま が うりだされた			
\\	自国の文化を大切にしましょう。	
\\	自国 
\\	じこく の ぶんか を たいせつ に しましょう			
\\	いろいろな花を組み合わせ花束を作りました。	
\\	組み合わせる 
\\	いろいろ な はな を くみあわせ はなたば を つくりました			
\\	作戦	
\\	さくせん			
\\	子供の自主性に重点を置いています。	
\\	重点 
\\	こども の じしゅせい に じゅうてん を おいて います			
\\	世界中で森林が失われています。	
\\	森林 
\\	せかいじゅう で しんりん が うしなわれて います			
\\	池に鴨の親子がいます。	
\\	親子 
\\	いけ に かも の おやこ が います			
\\	出版	
\\	しゅっぱん			
\\	世間の反応は冷たかったよ。	
\\	世間 
\\	せけん の はんのう は つめたかった よ			
\\	一瞬自分の耳を疑ったよ。	
\\	一瞬 
\\	いっしゅん じぶん の みみ を うたがった よ			
\\	公共の乗り物は誰でも利用できます。	
\\	公共 
\\	こうきょう の のりもの は だれ で も りよう できます			
\\	国際化	
\\	こくさいか			
\\	そんな事をしてはいけません。	
\\	事 
\\	そんな こと を して は いけません			
\\	商社	
\\	しょうしゃ			
\\	人には思想の自由がある。	
\\	思想 
\\	ひと に は しそう の じゆう が ある			
\\	旅行の間、犬の世話をしてください。	
\\	世話 
\\	りょこう の あいだ いぬ の せわ を して ください			
\\	権利	
\\	けんり			
\\	その地域では争いが絶えません。	
\\	争い 
\\	その ちいき で は あらそい が たえません			
\\	求人に多数の応募があった。	
\\	応募 
\\	きゅうじん に たすう の おうぼ が あった			
\\	キャンペーンに新しいイベントを組み込みました。	
\\	組み込む 
\\	きゃんぺーん に あたらしい いべんと を くみこみました			
\\	所得	
\\	しょとく			
\\	彼は軍隊に入ったよ。	
\\	軍隊 
\\	かれ は ぐんたい に はいった よ			
\\	布をもっと固く絞りなさい。	
\\	絞る 
\\	ぬの を もっと かたく しぼりなさい			
\\	鉄道の運賃が値上げされたね。	
\\	運賃 
\\	てつどう の うんちん が ねあげ された ね			
\\	回転	
\\	かいてん			
\\	洗濯に液体の洗剤を使っています。	
\\	液体 
\\	せんたく に えきたい の せんざい を つかって います			
\\	正しい順序で操作してください。	
\\	順序 
\\	ただしい じゅんじょ で そうさ して ください			
\\	今月のわが家の支出は15万円です。	
\\	支出 
\\	こんげつ の わがや の ししゅつ は じゅうごまんえん です			
\\	囲む	
\\	かこむ			
\\	一生のお願いがあります。	
\\	一生 
\\	いっしょう の おねがい が あります			
\\	彼に集中するよう注意を促しといたよ。	
\\	促す 
\\	かれ に しゅうちゅう する よう ちゅうい を うながし とい た よ			
\\	沖に小島が見えます。	
\\	沖 
\\	おき に こじま が みえます			
\\	継続	
\\	けいぞく			
\\	彼は知識の吸収が早いですね。	
\\	吸収 
\\	かれ は ちしき の きゅうしゅう が はやい です ね			
\\	東京には23の区がある。	
\\	区 
\\	とうきょう に は 
\\	の く が ある 。			
\\	私たちは酸素無しでは生きていけない。	
\\	酸素 
\\	わたしたち は さんそ なし で は いきて いけない			
\\	憲法	
\\	けんぽう			
\\	彼はビジネスクラスに乗ったの。	
\\	クラス 
\\	かれ は びじねすくらす に のった の			
\\	カーブを曲がるときはスピードを落としましょう。	
\\	スピード 
\\	かーぶ を まがる とき は すぴーど を おとしましょう			
\\	彼に会えば、彼がいかに良い人か分かります。	
\\	いかに 
\\	かれ に あえば かれ が いかに いい ひと か わかります			
\\	精神	
\\	せいしん			
\\	マラソンは雨の中でスタートしたの。	
\\	スタート 
\\	まらそん は あめ の なか で すたーと した の			
\\	このメニューはサラダと飲み物がセットになっています。	
\\	セット 
\\	この めにゅー は さらだ と のみもの が せっと に なって います			
\\	写真をアップで撮ろう。	
\\	アップ 
\\	しゃしん を あっぷ で とろう			
\\	延長	
\\	えんちょう			
\\	若者たちはエネルギーにあふれていますね。	
\\	エネルギー 
\\	わかものたち は えねるぎー に あふれて います ね			
\\	そんなことをしたら、かえってよくないよ。	
\\	かえって 
\\	そんなことをしたら、かえってよくないよ。			
\\	コンクリートにひびが入っていますね。	
\\	コンクリート 
\\	こんくりーと に ひび が はいって います ね			
\\	外部	
\\	がいぶ			
\\	明日はいよいよ出発の日です。	
\\	いよいよ 
\\	あす は いよいよ しゅっぱつ の ひ です			
\\	オリンピックは4年に1度開催されます。	
\\	オリンピック 
\\	おりんぴっく は よねん に いちど かいさい されます			
\\	ここでいったんストップしてください。	
\\	ストップ 
\\	ここでいったんストップしてください。			
\\	縮小	
\\	しゅくしょう			
\\	問題についてきちんと話し合ったよ。	
\\	きちんと 
\\	もんだい に ついて きちんと はなしあった よ			
\\	夏休みにはクラブの合宿に参加します。	
\\	クラブ 
\\	なつやすみ に は くらぶ の がっしゅく に さんか します			
\\	あの生徒はじっと先生の話を聞いていたね。	
\\	じっと 
\\	あの せいと は じっと せんせい の はなし を きいて いた ね			
\\	支配	
\\	しはい			
\\	その権力者による支配は50年以上続いたんです。	
\\	支配 
\\	その けんりょくしゃ に よる しはい は ごじゅうねん いじょう つづいた ん です			
\\	大きく息を吸ってください。	
\\	息 
\\	おおきく いき を すって ください			
\\	これ以上税が上がると生活できないね。	
\\	税 
\\	これ いじょう ぜい が あがる と せいかつ できない ね			
\\	息	
\\	いき			
\\	最近はテレビを見る回数が減りました。	
\\	回数 
\\	さいきん は てれび を みる かいすう が へりました			
\\	雪の結晶にはいろいろな形があります。	
\\	結晶 
\\	ゆき の けっしょう に は いろいろ な かたち が あります			
\\	欠点のない人間はいません。	
\\	欠点 
\\	けってん の ない にんげん は いません			
\\	税	
\\	ぜい			
\\	私たちは万歳をして喜びを表しました。	
\\	表す 
\\	わたしたち は ばんざい を して よろこび を あらわしました			
\\	彼は医学を学んでいます。	
\\	医学 
\\	かれ は いがく を まなんで います			
\\	そのチームには勢いがありますね。	
\\	勢い 
\\	その ちーむ に は いきおい が あります ね			
\\	回数	
\\	かいすう			
\\	その国には物資の援助が必要です。	
\\	援助 
\\	その くに に は ぶっし の えんじょ が ひつよう です			
\\	大雨で崖が崩れたね。	
\\	崩れる 
\\	おおあめ で がけ が くずれた ね			
\\	飛行機の座席はゆったりしていたよ。	
\\	座席 
\\	ひこうき の ざせき は ゆったり して いた よ			
\\	結晶	
\\	けっしょう			
\\	もっと照明を明るくしてください。	
\\	照明 
\\	もっと しょうめい を あかるく して ください			
\\	化粧品は化学工場で作られます。	
\\	化学 
\\	けしょうひん は かがく こうじょう で つくられます			
\\	彼女はショックで口もきけなかったわ。	
\\	ショック 
\\	かのじょ は しょっく で くち も きけなかった わ			
\\	欠点	
\\	けってん			
\\	屋上にアンテナが立っていますね。	
\\	アンテナ 
\\	おくじょう に あんてな が たって います ね			
\\	彼はあえて危険を冒したの。	
\\	あえて 
\\	かれ は あえて きけん を おかした			
\\	本にカバーを掛けました。	
\\	カバー 
\\	ほん に かばー を かけました			
\\	表す	
\\	あらわす			
\\	彼女は有能なジャーナリストだ。	
\\	ジャーナリスト 
\\	かのじょ は ゆうのう な じゃーなりすと だ			
\\	彼女はモデルのようにスタイルがいいね。	
\\	スタイル 
\\	かのじょ は もでる の よう に すたいる が いい ね			
\\	彼はインタビューに、はきはきと答えてたよ。	
\\	インタビュー 
\\	かれ は いんたびゅー に はきはき と こたえて た よ			
\\	医学	
\\	いがく			
\\	彼女たちはそれぞれ個性が強いですね。	
\\	個性 
\\	かのじょたち は それぞれ こせい が つよい です ね			
\\	地震は自然災害のひとつです。	
\\	災害 
\\	じしん は しぜん さいがい の ひとつ です			
\\	彼にも一応知らせておいたほうがいいだろう。	
\\	一応 
\\	かれ に も いちおう しらせて おいた ほう が いいだろう			
\\	勢い	
\\	いきおい			
\\	テレビの画面が明るすぎる。	
\\	画面 
\\	てれび の がめん が あかる すぎる			
\\	暮れは用事が多くて忙しいです。	
\\	暮れ 
\\	くれ は ようじ が おおくて いそがしい です			
\\	援助	
\\	えんじょ			
\\	私たちは公害を減らすよう努力しています。	
\\	公害 
\\	わたしたち は こうがい を へらす よう どりょく して います			
\\	この本を参考にして下さい。	
\\	参考 
\\	この ほん を さんこう に して ください			
\\	このデザインは数々の賞を受けています。	
\\	賞 
\\	この でざいん は かずかず の しょう を うけて います			
\\	崩れる	
\\	くずれる			
\\	労働は国民の義務です。	
\\	義務 
\\	ろうどう は こくみん の ぎむ です			
\\	突然の悲しみが一家を襲いました。	
\\	悲しみ 
\\	とつぜん の かなしみ が いっか を おそいました			
\\	生命は海から始まったと言われている。	
\\	生命 
\\	せいめい は うみ から はじまった と いわれて いる			
\\	座席	
\\	ざせき			
\\	彼ったら怒り爆発だったよ。	
\\	怒り 
\\	かれ ったら いかり ばくはつ だった よ			
\\	今年の夏は異常な暑さですね。	
\\	異常 
\\	ことし の なつ は いじょう な あつさ です ね			
\\	命より大切なものは無いよ。	
\\	命 
\\	いのち より たいせつ な もの は ない よ			
\\	照明	
\\	しょうめい			
\\	大勢が応援に駆けつけてくれたよ。	
\\	応援 
\\	おおぜい が おうえん に かけつけて くれた よ			
\\	お気に入りのカップが欠けてしまいました。	
\\	欠ける 
\\	おきにいり の かっぷ が かけて しまいました			
\\	化学	
\\	かがく			
\\	高速道路は混んでたよ。	
\\	高速道路 
\\	こうそく どうろ は こん でた よ 。			
\\	友達のお父さんが突然失業したの。	
\\	失業 
\\	ともだち の おとうさん が とつぜん しつぎょう した の			
\\	今日は8チームが出場しました。	
\\	出場 
\\	きょう は はち ちーむ が しゅつじょう しました			
\\	ショック	
\\	仮にそれが事実だとしてももう遅いよ。	
\\	仮に 
\\	かりに それ が じじつ だ と して も もう おそい よ			
\\	近代の技術の発展はものすごいです。	
\\	近代 
\\	きんだい の ぎじゅつ の はってん は ものすごい です			
\\	専門家を講師に招いた。	
\\	講師 
\\	せんもんか を こうし に まねいた			
\\	アンテナ	
\\	視点を変えて見てみましょう。	
\\	視点 
\\	してん を かえて みて みましょう			
\\	医者に症状を説明したよ。	
\\	症状 
\\	いしゃ に しょうじょう を せつめい した よ			
\\	一列に並んで前進して下さい。	
\\	前進 
\\	いちれつ に ならんで ぜんしん して ください			
\\	あえて	
\\	すみません、以後気を付けます。	
\\	以後 
\\	すみません いご き を つけます			
\\	店の前に長い行列ができていますよ。	
\\	行列 
\\	みせ の まえ に ながい ぎょうれつ が できて います よ			
\\	小学校で子供たちの劇を見ました。	
\\	劇 
\\	しょうがっこう で こどもたち の げき を みました			
\\	カバー	
\\	この詩を解釈してみましょう。	
\\	解釈 
\\	この し を かいしゃく して みましょう			
\\	一体何の騒ぎですか。	
\\	騒ぎ 
\\	いったい なん の さわぎ です か			
\\	彼は学校の劇で主役に選ばれたよ。	
\\	主役 
\\	かれ は がっこう の げき で しゅやく に えらばれた よ			
\\	ジャーナリスト	
\\	赤ちゃんの笑顔が可愛いい。	
\\	笑顔 
\\	あかちゃん の えがお が かわいい			
\\	窓に男性の影が映っています。	
\\	影 
\\	映る 
\\	まど に だんせい の かげ が うつって います			
\\	スタイル	
\\	彼女はラジオ局で働いています。	
\\	局 
\\	かのじょ は らじおきょく で はたらいて います			
\\	それは極端な意見だね。	
\\	極端 
\\	それ は きょくたん な いけん だ ね			
\\	車に駐車違反の警告を貼られました。	
\\	警告 
\\	くるま に ちゅうしゃ いはん の けいこく を はられました			
\\	インタビュー	
\\	午後8時に劇場の前で会いましょう。	
\\	劇場 
\\	ごご はちじ に げきじょう の まえ で あいましょう			
\\	原子力の安全な利用について考えましょう。	
\\	原子力 
\\	げんしりょく の あんぜん な りよう に ついて かんがえましょう			
\\	その殺人事件は白昼に起こったんだ。	
\\	殺人 
\\	その さつじん じけん は はくちゅう に おこった ん だ			
\\	個性	
\\	こせい			
\\	図を描いて説明しましょう。	
\\	図 
\\	ず を かいて せつめい しましょう			
\\	彼はよく刑事の役を演じるね。	
\\	演じる 
\\	かれ は よく けいじ の やく を えんじる ね			
\\	新聞の解説欄を読みました。	
\\	解説 
\\	しんぶん の かいせつ らん を よみました			
\\	災害	
\\	さいがい			
\\	高速道路で事故が重なった。	
\\	高速道路 
\\	重なる 
\\	こうそく どうろ で じこ が かさなった			
\\	私たちは高級ホテルに泊まったの。	
\\	高級 
\\	わたしたち は こうきゅう ほてる に とまった の			
\\	彼は差別をなくす運動をしています。	
\\	差別 
\\	かれ は さべつ を なくす うんどう を して います			
\\	一応	
\\	いちおう			
\\	大学卒業は一昨年です。	
\\	一昨年 
\\	だいがく そつぎょう は いっさくねん です			
\\	彼は高校の教員です。	
\\	教員 
\\	かれ は こうこう の きょういん です			
\\	彼らは裁判所の前で知らせを待っています。	
\\	裁判所 
\\	かれら は さいばんしょ の まえ で しらせ を まって います			
\\	画面	
\\	がめん			
\\	学校では音楽部に所属していました。	
\\	所属 
\\	がっこう で は おんがくぶ に しょぞく して いました			
\\	彼は額に汗をかいていたの。	
\\	汗 
\\	かれ は ひたい に あせ を かいていた の			
\\	彼は数々の名作を生み出した。	
\\	生み出す 
\\	かれ は かずかず の めいさく を うみだした			
\\	暮れ	
\\	くれ			
\\	雑音がひどくて声が聞こえません。	
\\	雑音 
\\	ざつおん が ひどくて こえ が きこえません			
\\	辺りを見回したの。	
\\	辺り 
\\	あたり を みまわした の			
\\	彼はお金をだいぶ稼いだらしいよ。	
\\	稼ぐ 
\\	かれ は おかね を だいぶ かせいだ らしい よ			
\\	公害	
\\	こうがい			
\\	私は今日、目の検査を受けます。	
\\	検査 
\\	わたし は きょう め の けんさ を うけます			
\\	この地方ではみかんの栽培が盛んです。	
\\	栽培 
\\	この ちほう で は みかん の さいばい が さかん です			
\\	ボールが速過ぎて受け止められなかったの。	
\\	受け止める 
\\	ぼーる が はや すぎて うけとめられ なかった の			
\\	参考	
\\	さんこう			
\\	俺の頼みを聞いてくれ。	
\\	俺 
\\	おれ の たのみ を きいて くれ			
\\	彼はとても有名な科学者だ。	
\\	科学者 
\\	かれ は とても ゆうめい な かがくしゃ だ			
\\	賞	
\\	しょう			
\\	火災の原因は放火だそうです。	
\\	火災 
\\	かさい の げんいん は ほうか だ そう です			
\\	何を根拠にそんな事を言うのですか。	
\\	根拠 
\\	なに を こんきょ に そんな こと を いう の です か			
\\	義務	
\\	ぎむ			
\\	海水から塩を作ります。	
\\	海水 
\\	かいすい から しお を つくります			
\\	今日は体の具合が悪いです。	
\\	具合 
\\	きょう は からだ の ぐあい が わるい です			
\\	勝負はまだ始まったばかりよ。	
\\	勝負 
\\	しょうぶ は まだ はじまった ばかり よ			
\\	悲しみ	
\\	かなしみ			
\\	意外にも彼は独身です。	
\\	意外 
\\	いがい に も かれ は どくしん です			
\\	ここは危険区域よ。	
\\	区域 
\\	ここ は きけん くいき よ			
\\	幸い、電車に嵐の影響はなかった。	
\\	幸い 
\\	さいわい でんしゃ に あらし の えいきょう は なかった			
\\	生命	
\\	せいめい			
\\	彼は機械の操作を誤った。	
\\	誤る 
\\	かれ は きかい の そうさ を あやまった			
\\	ペットを飼ったことはありますか。	
\\	飼う 
\\	ぺっと を かった こと は あります か			
\\	彼は学会で論文を発表したよ。	
\\	学会 
\\	かれ は がっかい で ろんぶん を はっぴょう した よ			
\\	怒り	
\\	いかり			
\\	事務の経験が3年あります。	
\\	事務 
\\	じむ の けいけん が さんねん あります			
\\	これは人工の湖です。	
\\	人工 
\\	これ は じんこう の みずうみ です			
\\	台風が接近していますね。	
\\	接近 
\\	たいふう が せっきん して います ね			
\\	異常	
\\	いじょう			
\\	経済的な効果は計り知れません。	
\\	経済的 
\\	けいざいてき な こうか は はかり しれません			
\\	年賀状を印刷したの。	
\\	印刷 
\\	ねんがじょう を いんさつ した の			
\\	電車が10分遅れで到着したんだ。	
\\	遅れ 
\\	でんしゃ が じゅっ ぷん おくれ で とうちゃく した ん だ			
\\	命	
\\	いのち			
\\	彼女の瞳は喜びで輝いていますね。	
\\	輝く 
\\	かのじょ の ひとみ は よろこび で かがやいて います ね			
\\	彼女は背の高い男性を好みますね。	
\\	好む 
\\	かのじょ は せ の たかい だんせい を このみます ね			
\\	あいうえお順に並べて下さい。	
\\	順 
\\	あいうえお じゅん に ならべて ください			
\\	応援	
\\	おうえん			
\\	その選手は今シーズンも好調だね。	
\\	シーズン 
\\	その せんしゅ は こん シーズン も こうちょう だ ね 。			
\\	重大な発表があります。	
\\	重大 
\\	じゅうだい な はっぴょう が あります			
\\	この荷物はかなりの重量ですね。	
\\	重量 
\\	この にもつ は かなり の じゅうりょう です ね			
\\	欠ける	
\\	かける			
\\	自分らしい生き方をしなさい。	
\\	生き方 
\\	じぶん らしい いきかた を しなさい			
\\	王の墓を見学したよ。	
\\	王 
\\	おう の はか を けんがく した よ			
\\	彼はあの通りに店を構えているの。	
\\	構える 
\\	かれ は あの とおり に みせ を かまえて いる の			
\\	高速道路	
\\	こうそくどうろ			
\\	足の傷が痛みます。	
\\	傷 
\\	あし の きず が いたみます			
\\	市は昨年から歩きタバコを禁じているの。	
\\	禁じる 
\\	し は さくねん から あるきたばこ を きんじて いる の			
\\	心臓は全身に血液を送り出している。	
\\	血液 
\\	しんぞう は ぜんしん に けつえき を おくりだして いる			
\\	失業	
\\	しつぎょう			
\\	父の手術が無事に済みました。	
\\	済む 
\\	ちち の しゅじゅつ が ぶじ に すみました			
\\	そんなことは絶対できないよ。	
\\	絶対 
\\	そんな こと は ぜったい できない よ			
\\	彼がいいアイデアを出したね。	
\\	アイデア 
\\	かれ が いい あいであ を だした ね			
\\	出場	
\\	しゅつじょう			
\\	一緒に食事しようか。	
\\	一緒 
\\	いっしょ に しょくじ しよう か			
\\	外貨を両替しました。	
\\	外貨 
\\	がいか を りょうがえ しました			
\\	この花はいい香りがしますね。	
\\	香り 
\\	この はな は いい かおり が します ね			
\\	仮に	
\\	かりに			
\\	これは世界で最小のコンピューターです。	
\\	最小 
\\	これ は せかい で さいしょう の こんぴゅーたー です			
\\	夏は水分を十分に補いましょう。	
\\	補う 
\\	なつ は すいぶん を じゅうぶん に おぎないましょう			
\\	この角度からは画面が見えにくいですね。	
\\	角度 
\\	この かくど からは がめん が みえ にくい です ね			
\\	近代	
\\	きんだい			
\\	昨日火山が爆発したの。	
\\	火山 
\\	きのう かざん が ばくはつ した の			
\\	私の父は公務員です。	
\\	公務員 
\\	わたし の ちち は こうむいん です			
\\	この事業には大きな意義があります。	
\\	意義 
\\	この じぎょう に は おおき な いぎ が あります			
\\	講師	
\\	こうし			
\\	人質が解放されてよかった。	
\\	解放 
\\	ひとじち が かいほう されて よかった			
\\	彼は新しい仕事に人生を賭けている。	
\\	賭ける 
\\	かれ は あたらしい しごと に じんせい を かけて いる			
\\	視点	
\\	してん			
\\	そのマラソン大会にはたくさんの国々から選手が集まったよ。	
\\	国々 
\\	その まらそん たいかい に は たくさん の くにぐに から せんしゅ が あつまった よ			
\\	ここは子育てに最適な環境です。	
\\	最適 
\\	ここ は こそだて に さいてき な かんきょう です			
\\	二人の実力は互角です。	
\\	実力 
\\	ふたり の じつりょく は ごかく です			
\\	症状	
\\	しょうじょう			
\\	彼は80年の生涯を閉じました。	
\\	生涯 
\\	かれ は はちじゅうねん の しょうがい を とじました			
\\	医師は異常なしと診断したんだ。	
\\	診断 
\\	いし は いじょう なし と しんだん した ん だ			
\\	彼はいかにもスポーツマンらしいですね。	
\\	いかにも 
\\	かれ は いかにも すぽーつまん らしい です ね			
\\	前進	
\\	ぜんしん			
\\	彼は一見サラリーマン風ですね。	
\\	一見 
\\	かれ は いっけん さらりーまんふう です ね			
\\	彼は柔軟に対応をした。	
\\	柔軟 
\\	かれ は じゅうなん に たいおう を した			
\\	活発な意見が交されたの。	
\\	活発 
\\	かっぱつ な いけん が かわされた の			
\\	以後	
\\	いご			
\\	このソフトで画像を合成できます。	
\\	合成 
\\	この そふと で がぞう を ごうせい できます			
\\	彼は株で財産の半分を失ったの。	
\\	財産 
\\	かれ は かぶ で ざいさん の はんぶん を うしなった の			
\\	ここは児童の通学路です。	
\\	児童 
\\	ここ は じどう の つうがくろ です			
\\	行列	
\\	ぎょうれつ			
\\	運動した翌日は全身の筋肉が痛い。	
\\	全身 
\\	うんどう した よくじつ は ぜんしん の きんにく が いたい			
\\	彼はくじ引きで一等賞を当てたよ。	
\\	当てる 
\\	かれ は くじびき で いっとう しょう を あてた よ			
\\	後ろからいきなり肩をたたかれた。	
\\	いきなり 
\\	うしろ から いきなり かた を たたかれた			
\\	劇	
\\	げき			
\\	水面に月が映っているね。	
\\	映る 
\\	水面 
\\	みなも に つき が うつって いる ね			
\\	改めてあなたのご意見を聞かせて下さい。	
\\	改めて 
\\	あらためて あなた の ごいけん を きかせ て ください			
\\	彼はやっと問題点に気付きました。	
\\	気付く 
\\	かれ は やっと もんだいてん に きづきました			
\\	解釈	
\\	かいしゃく			
\\	私立の大学は学費が高い。	
\\	私立 
\\	しりつ の だいがく は がくひ が たかい			
\\	最近、事件の真相が明らかになったよ。	
\\	真相 
\\	さいきん じけん の しんそう が あきらか に なった よ			
\\	背中がかゆい。	
\\	背中 
\\	せなか が かゆい			
\\	騒ぎ	
\\	さわぎ			
\\	彼は一流の選手です。	
\\	一流 
\\	かれ は いちりゅう の せんしゅ です			
\\	皆が人質の無事を祈っているわよ。	
\\	祈る 
\\	みんな が ひとじち の ぶじ を いのって いる わ よ			
\\	勝手なことばかり言わないでくれ。	
\\	勝手 
\\	かって な こと ばかり いわないで くれ			
\\	主役	
\\	しゅやく			
\\	彼女は体調を崩しています。	
\\	崩す 
\\	かのじょ は たいちょう を くずして います			
\\	メールを経由して広がるウィルスもあります。	
\\	経由 
\\	めーる を けいゆ して ひろがる うぃるす も あります			
\\	では、私たちは先に出発します。	
\\	先に 
\\	では わたしたち は さきに しゅっぱつ します			
\\	笑顔	
\\	えがお			
\\	政治は国民主体であるべきです。	
\\	主体 
\\	せいじ は こくみん しゅたい で ある べき です			
\\	この小説の主題は反戦です。	
\\	主題 
\\	この しょうせつ の しゅだい は はんせん です			
\\	この通りには商店が多いね。	
\\	商店 
\\	この とおり に は しょうてん が おおい ね			
\\	影	
\\	かげ			
\\	いつまでもあなたを忘れません。	
\\	いつまでも 
\\	いつまでも あなた を わすれません			
\\	彼は拳銃で撃たれたわ。	
\\	撃つ 
\\	かれ は けんじゅう で うたれた わ			
\\	この絵はスペインの画家が描きました。	
\\	画家 
\\	この え は すぺいん の がか が かきました			
\\	局	
\\	きょく			
\\	父はラッシュアワーを嫌っています。	
\\	嫌う 
\\	ちち は らっしゅあわー を きらって います			
\\	彼女はアレルギーに苦しんでいます。	
\\	苦しむ 
\\	かのじょ は あれるぎー に くるしんで います			
\\	極端	
\\	きょくたん			
\\	犯人は国外に逃げたようです。	
\\	国外 
\\	はんにん は こくがい に にげた よう です			
\\	彼は若いのにしっかりしてるね。	
\\	しっかり 
\\	かれ は わかい の に しっかり してる ね			
\\	母の手料理をゆっくり味わいました。	
\\	味わう 
\\	はは の てりょうり を ゆっくり あじわいました			
\\	警告	
\\	けいこく			
\\	その古い建物は官庁です。	
\\	官庁 
\\	その ふるい たてもの は かんちょう です			
\\	期限までに申し込みました。	
\\	期限 
\\	きげん まで に もうしこみました			
\\	水が沸騰して気体になったんだ。	
\\	気体 
\\	みず が ふっとう して きたい に なった ん だ			
\\	劇場	
\\	げきじょう			
\\	この句はどんな意味でしょうか。	
\\	句 
\\	この く は どんな いみ でしょう か			
\\	この店では新鮮な野菜が買えますよ。	
\\	新鮮 
\\	この みせ で は しんせん な やさい が かえます よ			
\\	前回の続きから始めます。	
\\	前回 
\\	ぜんかい の つづき から はじめます			
\\	原子力	
\\	げんしりょく			
\\	今日は沿岸の波が荒いでしょう。	
\\	沿岸 
\\	きょう は えんがん の なみ が あらい でしょう			
\\	覚悟はできています。	
\\	覚悟 
\\	かくご は できて います			
\\	殺人	
\\	さつじん			
\\	彼女は幸福な日々を過ごしているわ。	
\\	幸福 
\\	かのじょ は こうふく な ひび を すごして いる わ			
\\	部長は京都に出張中です。	
\\	出張 
\\	ぶちょう は きょうと に しゅっちょうちゅう です			
\\	家族に感謝しています。	
\\	感謝 
\\	かぞく に かんしゃ して います			
\\	図	
\\	ず			
\\	私は毎年夏に田舎に帰ります。	
\\	田舎 
\\	わたし は まいとし なつ に いなか に かえります			
\\	彼の講義はとても分かりやすいですね。	
\\	講義 
\\	かれ の こうぎ は とても わかり やすい です ね			
\\	ゆっくり呼吸してください。	
\\	呼吸 
\\	ゆっくり こきゅう して ください			
\\	演じる	
\\	えんじる			
\\	その選手は自主トレーニングを始めたの。	
\\	自主 
\\	その せんしゅ は じしゅとれーにんぐ を はじめた の			
\\	地面に何か絵が描いてあるぞ。	
\\	地面 
\\	じめん に なにか え が かいて ある ぞ			
\\	7月上旬に夏祭りがありますよ。	
\\	上旬 
\\	しちがつ じょうじゅん に なつまつり が あります よ			
\\	解説	
\\	かいせつ			
\\	お二人はどこで知り合ったのですか。	
\\	知り合う 
\\	おふたり は どこ で しりあった の です か			
\\	彼女は秘書と事務の担当を兼ねています。	
\\	兼ねる 
\\	かのじょ は ひしょ と じむ の たんとう を かねて います			
\\	夜の11時に帰宅しました。	
\\	帰宅 
\\	よる の じゅういちじ に きたく しました			
\\	重なる	
\\	かさなる			
\\	物事には限度があります。	
\\	限度 
\\	ものごと に は げんど が あります			
\\	彼女は高価な宝石を持っているわ。	
\\	高価 
\\	かのじょ は こうか な ほうせき を もって いる わ			
\\	高級	
\\	こうきゅう			
\\	彼女は通りがかりの人に救われたよ。	
\\	救う 
\\	かのじょ は とおりがかり の ひと に すくわれた よ			
\\	祖父は戦場に行ったことがあるそうです。	
\\	戦場 
\\	そふ は せんじょう に いった こと が ある そう です			
\\	差別	
\\	さべつ			
\\	彼は相変わらず忙しいですね。	
\\	相変わらず 
\\	かれ は あいかわらず いそがしい です ね			
\\	街で偶然友人に会いました。	
\\	偶然 
\\	まち で ぐうぜん ゆうじん に あいました			
\\	子供が歩き始めたとき、子供の成長を実感した。	
\\	実感 
\\	こども が あるきはじめた とき こども の せいちょう を じっかん した			
\\	一昨年	
\\	いっさくねん			
\\	その家の正面には大きなバルコニーがあるの。	
\\	正面 
\\	その いえ の しょうめん に は おおき な ばるこにー が ある の			
\\	彼は人民のための政治を行った。	
\\	人民 
\\	かれ は じんみん の ため の せいじ を おこなった			
\\	私は体力では誰にも劣りません。	
\\	劣る 
\\	わたし は たいりょく で は だれ に も おとりません			
\\	教員	
\\	きょういん			
\\	あの岩まで泳ごう。	
\\	岩 
\\	あの いわ まで およごう			
\\	古代の歴史について勉強しました。	
\\	古代 
\\	こだい の れきし に ついて べんきょう しました			
\\	コストがこんなに高くては採算が取れません。	
\\	採算 
\\	こすと が こんなに たかくて は さいさん が とれません			
\\	裁判所	
\\	さいばんしょ			
\\	車内に忘れ物があったよ。	
\\	車内 
\\	しゃない に わすれもの が あった よ			
\\	彼女は演劇を学んでいます。	
\\	演劇 
\\	かのじょ は えんげき を まなんで います			
\\	嬉しくて思わず涙が出ました。	
\\	思わず 
\\	うれしくて おもわず なみだ が でました			
\\	所属	
\\	しょぞく			
\\	新聞の活字が読みやすくなったね。	
\\	活字 
\\	しんぶん の かつじ が よみ やすく なった ね			
\\	古い校舎の修理が必要です。	
\\	校舎 
\\	ふるい こうしゃ の しゅうり が ひつよう です			
\\	お米の収穫が始まったね。	
\\	収穫 
\\	おこめ の しゅうかく が はじまった ね			
\\	汗	
\\	あせ			
\\	戦前の生活は今と全く違いました。	
\\	戦前 
\\	せんぜん の せいかつ は いま と まったく ちがいました			
\\	彼女はアンケートに答えたよ。	
\\	アンケート 
\\	かのじょ は あんけーと に こたえた よ			
\\	駅前に交番が有ります。	
\\	駅前 
\\	えきまえ に こうばん が あります			
\\	生み出す	
\\	うみだす			
\\	彼らは労働を強制されたんだ。	
\\	強制 
\\	かれら は ろうどう を きょうせい された ん だ			
\\	3社合同で新作の発表会を開きました。	
\\	合同 
\\	さんしゃ ごうどう で しんさく の はっぴょうかい を ひらきました			
\\	雑音	
\\	ざつおん			
\\	彼は優れた詩人です。	
\\	詩人 
\\	かれ は すぐれた しじん です			
\\	1週間後に出国します。	
\\	出国 
\\	いっしゅうかん ご に しゅっこく します			
\\	その計画に反対の人はほんの少数だったよ。	
\\	少数 
\\	その けいかく に はんたい の ひと は ほんの しょうすう だった よ			
\\	辺り	
\\	あたり			
\\	あなたの職業を教えてください。	
\\	職業 
\\	あなた の しょくぎょう を おしえて ください			
\\	彼は幼い時に親類に預けられたんだ。	
\\	親類 
\\	かれ は おさない とき に しんるい に あずけられた ん だ			
\\	雨で集まりが悪いですね。	
\\	集まり 
\\	あめ で あつまり が わるい です ね			
\\	稼ぐ	
\\	かせぐ			
\\	父は長い板を買って来たんだ。	
\\	板 
\\	ちち は ながい いた を かって きた ん だ			
\\	運動会は雨で延期されました。	
\\	延期 
\\	うんどうかい は あめ で えんき されました			
\\	自動車税を納めたよ。	
\\	税 
\\	納める 
\\	じどうしゃぜい を おさめた よ			
\\	検査	
\\	けんさ			
\\	彼は芸術的な才能にあふれているね。	
\\	才能 
\\	かれ は げいじゅつてき な さいのう に あふれて いる ね			
\\	湖の水面に小さく波が立っているな。	
\\	水面 
\\	みずうみ の すいめん に ちいさく なみ が たって いる な			
\\	スケジュールの調整は君に任せるわ。	
\\	スケジュール 
\\	すけじゅーる の ちょうせい は きみ に まかせる わ			
\\	栽培	
\\	さいばい			
\\	私の学校には制服がありません。	
\\	制服 
\\	わたし の がっこう に は せいふく が ありません			
\\	私が好きな楽器はギターです。	
\\	楽器 
\\	わたし が すき な がっき は ぎたー です			
\\	砂漠ではほとんど雨が降らないのよ。	
\\	砂漠 
\\	さばく で は ほとんど あめ が ふらない の よ			
\\	受け止める	
\\	うけとめる			
\\	資料は助手に預けておいてください。	
\\	助手 
\\	しりょう は じょしゅ に あずけて おいて ください			
\\	このカメラなら水中の写真が撮れますね。	
\\	水中 
\\	この かめら なら すいちゅう の しゃしん が とれます ね			
\\	彼女はとても素直で可愛いですね。	
\\	素直 
\\	かのじょ は とても すなお で かわいい です ね			
\\	俺	
\\	おれ			
\\	ご感想をお聞かせ下さい。	
\\	感想 
\\	ごかんそう を おきかせ ください			
\\	わが国の国土は70
\\	が森林です。	
\\	国土 
\\	わがくに の こくど は ななじゅっぱーせんと が しんりん です			
\\	私は車は持たない主義です。	
\\	主義 
\\	わたし は くるま は もたない しゅぎ です			
\\	科学者	
\\	かがくしゃ			
\\	プールで男の子が救助されました。	
\\	救助 
\\	ぷーる で おとこ の こ が きゅうじょ されました			
\\	彼は高齢を理由に社長を辞めたよ。	
\\	高齢 
\\	かれ は こうれい を りゆう に しゃちょう を やめた よ			
\\	彼らは互いに再会を喜んだの。	
\\	再会 
\\	かれら は たがいに さいかい を よろこんだ の			
\\	火災	
\\	かさい			
\\	亀の寿命は長いんだ。	
\\	寿命 
\\	かめ の じゅみょう は ながい ん だ			
\\	このビルは新聞社です。	
\\	新聞社 
\\	この びる は しんぶんしゃ です			
\\	彼は意志の強い人です。	
\\	意志 
\\	かれ は いし の つよい ひと です			
\\	根拠	
\\	こんきょ			
\\	なぜあなたは私を疑うのですか。	
\\	疑う 
\\	なぜ あなた は わたし を うたがう の です か			
\\	車にガソリンを入れました。	
\\	ガソリン 
\\	くるま に がそりん を いれました			
\\	これまで話し合った事柄をまとめてください。	
\\	事柄 
\\	これ まで はなしあった ことがら を まとめて ください			
\\	解散	
\\	かいさん			
\\	乗客は全員生存しています。	
\\	生存 
\\	じょうきゃく は ぜんいん せいぞん して います			
\\	祖父は戦死しました。	
\\	戦死 
\\	そふ は せんし しました			
\\	市長がテープをカットした。	
\\	カット 
\\	しちょう が てーぷ を かっと した			
\\	海水	
\\	かいすい			
\\	私の気持ちを誤解しているんじゃない!	
\\	誤解 
\\	わたし の きもち を ごかい して いる ん じゃ ない			
\\	索引はたいてい本の後ろについていますよ。	
\\	索引 
\\	さくいん は たいてい ほん の うしろ に ついて います よ			
\\	具合	
\\	ぐあい			
\\	お腹がいっぱいなのでお代わりは結構です。	
\\	結構 
\\	おなか が いっぱい な の で おかわり は けっこう です			
\\	彼は今日入ったばかりの新人です。	
\\	新人 
\\	かれ は きょう はいった ばかり の しんじん です			
\\	市場で新鮮な魚を買ってきました。	
\\	新鮮 
\\	市場 
\\	いちば で しんせん な さかな を かって きました			
\\	勝負	
\\	しょうぶ			
\\	容器から液がこぼれていますよ。	
\\	液 
\\	ようき から えき が こぼれて います よ			
\\	日本の気候は温暖です。	
\\	気候 
\\	にほん の きこう は おんだん です			
\\	彼は感情を込めてその歌を歌ったの。	
\\	込める 
\\	かれ は かんじょう を こめて その うた を うたった の			
\\	意外	
\\	いがい			
\\	その地方はお茶の産地です。	
\\	産地 
\\	その ちほう は おちゃ の さんち です			
\\	昼休みは正午からです。	
\\	正午 
\\	ひるやすみ は しょうご から です			
\\	仕事は全て完了しました。	
\\	完了 
\\	しごと は すべて かんりょう しました			
\\	競技	
\\	きょうぎ			
\\	彼は個人競技のスポーツが好きです。	
\\	競技 
\\	かれ は こじん きょうぎ の すぽーつ が すき です			
\\	私は私鉄の職員です。	
\\	私鉄 
\\	わたし は してつ の しょくいん です			
\\	その住宅地は便利な場所にあるね。	
\\	住宅地 
\\	その じゅうたくち は べんり な ばしょ に ある ね			
\\	私鉄	
\\	してつ			
\\	患者の呼吸は正常です。	
\\	正常 
\\	かんじゃ の こきゅう は せいじょう です			
\\	お互いを信頼することが大事です。	
\\	お互い 
\\	おたがい を しんらい する こと が だいじ です			
\\	この道は、ゆるい曲線を描いているよ。	
\\	曲線 
\\	この みち は ゆるい きょくせん を えがいて いる よ			
\\	住宅地	
\\	じゅうたくち			
\\	この国には様々な人種がいます。	
\\	人種 
\\	この くに に は さまざま な じんしゅ が います			
\\	窓の外に水田が広がっていたよ。	
\\	水田 
\\	まど の そと に すいでん が ひろがって いた よ			
\\	彼女は曖昧な返事をしたね。	
\\	曖昧 
\\	かのじょ は あいまい な へんじ を した ね			
\\	正常	
\\	せいじょう			
\\	運命には逆らえないよ。	
\\	運命 
\\	うんめい に は さからえない よ			
\\	店に苦情の電話をかけました。	
\\	苦情 
\\	みせ に くじょう の でんわ を かけました			
\\	お互い	
\\	おたがい			
\\	言論の自由は国民の権利です。	
\\	言論 
\\	げんろん の じゆう は こくみん の けんり です			
\\	背広をクリーニングに出しました。	
\\	背広 
\\	せびろ を くりーにんぐ に だしました			
\\	雲の間から太陽が現れました。	
\\	現れる 
\\	くも の あいだ から たいよう が あらわれました			
\\	曲線	
\\	きょくせん			
\\	夏祭りで花火を打ち上げます。	
\\	打ち上げる 
\\	なつまつり で はなび を うちあげます			
\\	博士はその道の権威です。	
\\	権威 
\\	はかせ は そのみち の けんい です			
\\	今日は3科目のテストを受けたよ。	
\\	科目 
\\	きょう は さんかもく の てすと を うけた よ			
\\	人種	
\\	じんしゅ			
\\	彼は求人広告で仕事を見つけたんだ。	
\\	求人 
\\	かれ は きゅうじん こうこく で しごと を みつけた ん だ			
\\	あなたの経歴をメールで送ってください。	
\\	経歴 
\\	あなた の けいれき を めーる で おくって ください			
\\	彼女は女の子を出産したんだ。	
\\	出産 
\\	かのじょ は おんな の こ を しゅっさん した ん だ			
\\	水田	
\\	すいでん			
\\	彼らは垂直のがけを登り始めたの。	
\\	垂直 
\\	かれら は すいちょく の がけ を のぼりはじめた の			
\\	彼はその時スターだったよ。	
\\	スター 
\\	かれ は その とき スター だった よ 。			
\\	彼は野球部の一員です。	
\\	一員 
\\	かれ は やきゅうぶ の いちいん です			
\\	曖昧	
\\	あいまい			
\\	外は一面の雪景色だったね。	
\\	一面 
\\	そと は いちめん の ゆきげしき だった ね			
\\	玉ねぎを細かく刻んでください。	
\\	刻む 
\\	たまねぎ を こまかく きざんで ください			
\\	電気系統を図面で確かめました。	
\\	系統 
\\	でんき けいとう を ずめん で たしかめました			
\\	運命	
\\	うんめい			
\\	他人に知れるとまずいことになるわね。	
\\	知れる 
\\	たにん に しれる と まずい こと に なる わ ね			
\\	顧客心理を理解することは重要です。	
\\	心理 
\\	こきゃく しんり を りかい する こと は じゅうよう です			
\\	父は情熱を持って仕事に打ち込んでいます。	
\\	情熱 
\\	ちち は じょうねつ を もって しごと に うちこんで います			
\\	苦情	
\\	くじょう			
\\	山は鮮やかな緑でした。	
\\	鮮やか 
\\	やま は あざやか な みどり でした			
\\	将来はプロのカメラマンになりたいです。	
\\	カメラマン 
\\	しょうらい は ぷろ の かめらまん に なりたい です			
\\	言論	
\\	げんろん			
\\	彼は書道3級です。	
\\	級 
\\	かれ は しょどう さんきゅう です			
\\	洪水でたくさんの家が流されたの。	
\\	洪水 
\\	こうずい で たくさん の いえ が ながされた の			
\\	何とか最悪の事態を避けることができました。	
\\	最悪 
\\	なんとか さいあく の じたい を さける こと が できました			
\\	背広	
\\	せびろ			
\\	彼は更に質問を続けたの。	
\\	更に 
\\	かれ は さらに しつもん を つづけた の			
\\	デジカメ市場は急速に拡大している。	
\\	市場 
\\	でじかめ しじょう は きゅうそく に かくだい して いる			
\\	現れる	
\\	あらわれる			
\\	今日、彼女は地味な服装をしていますね。	
\\	地味 
\\	きょう かのじょ は じみ な ふくそう を して います ね			
\\	卒業後の進路を迷っています。	
\\	進路 
\\	そつぎょう ご の しんろ を まよって います			
\\	彼はアルコールに強い体質です。	
\\	アルコール 
\\	かれ は あるこーる に つよい たいしつ です			
\\	打ち上げる	
\\	うちあげる			
\\	運動会で一位になったよ。	
\\	運動会 
\\	うんどうかい で いちい に なった よ			
\\	衛星が打ち上げられましたね。	
\\	打ち上げる 
\\	衛星 
\\	えいせい が うちあげられました ね			
\\	彼女のピアノの演奏は素晴らしいね。	
\\	演奏 
\\	かのじょ の ぴあの の えんそう は すばらしい ね			
\\	権威	
\\	けんい			
\\	蝶々が空中を舞っています。	
\\	空中 
\\	ちょうちょう が くうちゅう を まって います			
\\	彼は二度と刑務所から出ることができないの。	
\\	刑務所 
\\	かれ は にどと けいむしょ から でる こと が できない の			
\\	科目	
\\	かもく			
\\	今度こそタバコを止める決心をしました。	
\\	決心 
\\	こんど こそ たばこ を やめる けっしん を しました			
\\	個々の問題を解決しましょう。	
\\	個々 
\\	ここ の もんだい を かいけつ しましょう			
\\	求人	
\\	きゅうじん			
\\	ハイキングは自然に親しむ良い機会ですよ。	
\\	親しむ 
\\	はいきんぐ は しぜん に したしむ いい きかい です よ			
\\	ご親切は決して忘れません。	
\\	親切 
\\	ごしんせつ は けっして わすれません			
\\	豆腐は栄養のある食べ物です。	
\\	栄養 
\\	とうふ は えいよう の ある たべもの です			
\\	経歴	
\\	けいれき			
\\	私たちは大声で歌を歌ったの。	
\\	大声 
\\	わたしたち は おおごえ で うた を うたった の			
\\	今度の休暇にフィリピンに行きます。	
\\	休暇 
\\	こんど の きゅうか に ふぃりぴん に いきます			
\\	私は彼の成功を確信しています。	
\\	確信 
\\	わたし は かれ の せいこう を かくしん して います			
\\	出産	
\\	しゅっさん			
\\	この薬はあまりよく効かないよ。	
\\	効く 
\\	この くすり は あまり よく きかない よ			
\\	今日の1時間目は国語です。	
\\	国語 
\\	きょう の いちじかんめ は こくご です			
\\	氷は固体です。	
\\	固体 
\\	こおり は こたい です			
\\	垂直	
\\	すいちょく			
\\	姉と私は服の好みが似ています。	
\\	好み 
\\	あね と わたし は ふく の このみ が にて います			
\\	この棚は水平になっていませんね。	
\\	水平 
\\	この たな は すいへい に なって いません ね			
\\	スター	
\\	私は西洋の建築に興味があります。	
\\	西洋 
\\	わたし は せいよう の けんちく に きょうみ が あります			
\\	家族で彼の合格を祝ったよ。	
\\	祝う 
\\	かぞく で かれ の ごうかく を いわった よ			
\\	刑事が現場を調べているわ。	
\\	刑事 
\\	けいじ が げんば を しらべて いる わ			
\\	一員	
\\	いちいん			
\\	彼女はよく自分の成績を自慢するよね。	
\\	自慢 
\\	かのじょ は よく じぶん の せいせき を じまん する よ ね			
\\	彼女は出版社で働いています。	
\\	出版社 
\\	かのじょ は しゅっぱんしゃ で はたらいて います			
\\	消極的な人は成功しないよ。	
\\	消極的 
\\	しょうきょくてき な ひと は せいこう しない よ			
\\	一面	
\\	いちめん			
\\	この国はイギリスの植民地でした。	
\\	植民地 
\\	この くに は いぎりす の しょくみんち でした			
\\	彼女はずっと女優になるのが夢でした。	
\\	女優 
\\	かのじょ は ずっと じょゆう に なる の が ゆめ でした			
\\	刻む	
\\	きざむ			
\\	身長はどれくらいありますか。	
\\	身長 
\\	しんちょう は どれ くらい あります か			
\\	電気や水を節約しましょう。	
\\	節約 
\\	でんき や みず を せつやく しましょう			
\\	系統	
\\	けいとう			
\\	背中に痛みがあります。	
\\	痛み 
\\	せなか に いたみ が あります			
\\	運動会は9時に開会します。	
\\	開会 
\\	うんどうかい は くじ に かいかい します			
\\	知れる	
\\	しれる			
\\	お勘定をして下さい。	
\\	勘定 
\\	おかんじょう を して ください			
\\	彼らは漁業を営んでいるんだ。	
\\	漁業 
\\	かれら は ぎょぎょう を いとなん でいる ん だ 。			
\\	これは日本固有の鳥です。	
\\	固有 
\\	これ は にほん こゆう の とり です			
\\	心理	
\\	しんり			
\\	彼はカバンから札の束を取り出したんだ。	
\\	札 
\\	かれ は カバン から さつ の たば を とりだした ん だ 。			
\\	家計の収支が合わないの。	
\\	収支 
\\	家計 
\\	かけい の しゅうし が あわない の			
\\	車の振動で棚の荷物が落ちた。	
\\	振動 
\\	くるま の しんどう で たな の にもつ が おちた 。			
\\	情熱	
\\	じょうねつ			
\\	自分の信念に従いたいと思います。	
\\	信念 
\\	じぶん の しんねん に したがいたい と おもいます			
\\	せめてこれだけは約束してください。	
\\	せめて 
\\	せめて これ だけ は やくそく して ください 。			
\\	彼女は驚きを隠せませんでしたよ。	
\\	驚き 
\\	かのじょ は おどろき を かくせませんでした よ			
\\	鮮やか	
\\	あざやか			
\\	彼は私より一学年上です。	
\\	学年 
\\	かれ は わたし より ひと がくねん うえ です			
\\	彼女は家計を任されているの。	
\\	家計 
\\	かのじょ は かけい を まかされて いる の			
\\	ガス管が爆発しました。	
\\	管 
\\	がすかん が ばくはつ しました			
\\	明日	
\\	あす			
\\	テレビで明日の気象情報を確認したよ。	
\\	気象 
\\	てれび で あす の きしょう じょうほう を かくにん した よ			
\\	今日の議題は面白そうです。	
\\	議題 
\\	きょう の ぎだい は おもしろ そう です			
\\	彼らは東京近郊に住んでいる。	
\\	近郊 
\\	かれら は とうきょう きんこう に すんで いる			
\\	カメラマン	
\\	彼女は結構めがねが似合うね。	
\\	結構 
\\	かのじょ は けっこう めがね が にあう ね			
\\	賞金をもらったら公平に分けましょう。	
\\	公平 
\\	しょうきん を もらったら こうへい に わけましょう			
\\	今日はデパートが混雑していたよ。	
\\	混雑 
\\	きょう は でぱーと が こんざつ して いた よ			
\\	級	
\\	きゅう			
\\	信仰は心の支えです。	
\\	信仰 
\\	しんこう は こころ の ささえ です			
\\	ここは隣の市との境界です。	
\\	境界 
\\	ここ は となり の し と の きょうかい です			
\\	洪水	
\\	こうずい			
\\	小麦は色々な食べ物に使われている。	
\\	小麦 
\\	こむぎ は いろいろ な たべもの に つかわれて いる			
\\	火が四方に広がったんだよ。	
\\	四方 
\\	ひ が しほう に ひろがった ん だ よ			
\\	上司に相談してみます。	
\\	上司 
\\	じょうし に そうだん して みます			
\\	最悪	
\\	さいあく			
\\	食卓に花を飾りましょう。	
\\	食卓 
\\	しょくたく に はな を かざりましょう			
\\	会議の初日に市長がスピーチをしたよ。	
\\	初日 
\\	かいぎ の しょにち に しちょう が すぴーち を した よ			
\\	アリは土の中に巣を作ります。	
\\	巣 
\\	あり は つち の なか に す を つくります			
\\	更に	
\\	さらに			
\\	それは彼の推測にすぎない。	
\\	推測 
\\	それ は かれ の すいそく に すぎない			
\\	父は居間でテレビを見ている。	
\\	居間 
\\	ちち は いま で てれび を みて いる			
\\	心の内をお話し下さい。	
\\	内 
\\	こころ の うち を お はなし ください			
\\	市場	
\\	しじょう			
\\	彼女の訴えは認められたわ。	
\\	訴え 
\\	かのじょ の うったえ は みとめられた わ			
\\	大勢の方々にご出席いただきました。	
\\	方々 
\\	おおぜい の かたがた に ごしゅっせき いただきました			
\\	地味	
\\	じみ			
\\	仮定の話だけでは結論は出ません。	
\\	仮定 
\\	かてい の はなし だけ で は けつろん は でません			
\\	彼女の優勝に感激し興奮しました。	
\\	興奮 
\\	かのじょ の ゆうしょう に かんげき し こうふん しました			
\\	では早速書類をお送りします。	
\\	早速 
\\	では さっそく しょるい を おおくり します			
\\	進路	
\\	しんろ			
\\	彼の弱点はスタミナが足りないところです。	
\\	弱点 
\\	かれ の じゃくてん は すたみな が たりない ところ です			
\\	彼の言うことは信用できない。	
\\	信用 
\\	かれ の いう こと は しんよう できない			
\\	少年は網でその蝶を捕まえた。	
\\	網 
\\	しょうねん は あみ で その ちょう を つかまえた			
\\	アルコール	
\\	母は毎日てきぱきと家事をしているよ。	
\\	家事 
\\	はは は まいにち てきぱき と かじ を して いる よ			
\\	彼は大学で工学を勉強しました。	
\\	工学 
\\	かれ は だいがく で こうがく を べんきょう しました			
\\	彼は論文の仕上げに入ったの。	
\\	仕上げ 
\\	かれ は ろんぶん の しあげ に はいった の			
\\	運動会	
\\	うんどうかい			
\\	駅前に新しい書店ができました。	
\\	書店 
\\	えきまえ に あたらしい しょてん が できました			
\\	もうすぐ夜が明けるね。	
\\	明ける 
\\	もうすぐ よ が あける ね			
\\	私は木造の家屋が好きです。	
\\	家屋 
\\	わたし は もくぞう の かおく が すき です			
\\	衛星	
\\	えいせい			
\\	彼の態度は誠意を欠いています。	
\\	欠く 
\\	かれ の たいど は せいい を かいて います			
\\	入学式で学長の挨拶がありました。	
\\	学長 
\\	にゅうがくしき で がくちょう の あいさつ が ありました			
\\	演奏	
\\	えんそう			
\\	花の成長を観察して日記を書きなさい。	
\\	観察 
\\	はな の せいちょう を かんさつ して にっき を かきなさい			
\\	年をとると血管が硬くなります。	
\\	血管 
\\	とし を とる と けっかん が かたく なります			
\\	彼はやっと報告書を仕上げたよ。	
\\	仕上げる 
\\	かれ は やっと ほうこくしょ を しあげた よ			
\\	空中	
\\	くうちゅう			
\\	市役所で書類をもらって来たの。	
\\	市役所 
\\	しやくしょ で しょるい を もらって きた の			
\\	上流には滝があります。	
\\	上流 
\\	じょうりゅう に は たき が あります			
\\	ストレスがいろいろな病気の元になっているの。	
\\	ストレス 
\\	ストレス が いろいろ な びょうき の もと に なっている の 。			
\\	刑務所	
\\	けいむしょ			
\\	その昔航海士達は未踏の地を求め海洋に乗り出しました。	
\\	海洋 
\\	その むかし こうかいしたち は みとう の ち を もとめ かいよう に のりだしました			
\\	コードが短くてコンセントに届きません。	
\\	コード 
\\	こーど が みじかく て こんせんと に とどきません			
\\	警察がその男を捜しているの。	
\\	捜す 
\\	けいさつ が その おとこ を さがして いる の			
\\	決心	
\\	けっしん			
\\	先日の件はどうなりましたか。	
\\	先日 
\\	せんじつ の けん は どう なりました か			
\\	明日は恐らく晴れるでしょう。	
\\	恐らく 
\\	あした は おそらく はれる でしょう			
\\	その職は大卒の学歴が必要だ。	
\\	学歴 
\\	その しょく は だいそつ の がくれき が ひつよう だ			
\\	個々	
\\	ここ			
\\	そのことは承知しております。	
\\	承知 
\\	その こと は しょうち して おります			
\\	書類を10枚コピーしました。	
\\	書類 
\\	しょるい を じゅうまい こぴー しました			
\\	親しむ	
\\	したしむ			
\\	雪の重みで枝が折れそうだ。	
\\	重み 
\\	ゆき の おもみ で えだ が おれ そう だ			
\\	デパートは10時に開店しますよ。	
\\	開店 
\\	でぱーと は じゅうじ に かいてん します よ			
\\	親切	
\\	しんせつ			
\\	この病院は24時間看護です。	
\\	看護 
\\	この びょういん は にじゅうよじかん かんご です			
\\	儀式には作法があります。	
\\	儀式 
\\	ぎしき に は さほう が あります			
\\	霧の深い夜のことでした。	
\\	霧 
\\	きり の ふかい よる の こと でした			
\\	栄養	
\\	えいよう			
\\	日本の国旗は描くのが簡単です。	
\\	国旗 
\\	にほん の こっき は かく の が かんたん です			
\\	私は古典を読むのが好きです。	
\\	古典 
\\	わたし は こてん を よむ の が すき です			
\\	学生たちは自主的に勉強会を開いたわよ。	
\\	自主的 
\\	がくせいたち は じしゅてき に べんきょうかい を ひらいた わ よ			
\\	大声	
\\	おおごえ			
\\	そちらの品は半額になっています。	
\\	品 
\\	そちら の しな は はんがく に なって います			
\\	泡立てたクリームに少量のブランデーを加えます。	
\\	少量 
\\	あわだてた くりーむ に しょうりょう の ぶらんでー を くわえます			
\\	あなたの成績が下がったのは当たり前です。	
\\	当たり前 
\\	あなた の せいせき が さがった の は あたりまえ です			
\\	休暇	
\\	きゅうか			
\\	台風で山も海も荒れています。	
\\	荒れる 
\\	たいふう で やま も うみ も あれて います			
\\	会計を済ませて店を出たんだ。	
\\	会計 
\\	かいけい を すませて みせ を でた ん だ			
\\	彼の我慢強さには感心しました。	
\\	感心 
\\	かれ の がまんづよさ に は かんしん しました			
\\	確信	
\\	かくしん			
\\	店の看板を塗り替えました。	
\\	看板 
\\	みせ の かんばん を ぬりかえました			
\\	その瞬間、猫が跳び出しました。	
\\	瞬間 
\\	その しゅんかん ねこ が とびだしました			
\\	これを順番通りに並べ替えて下さい。	
\\	順番 
\\	これ を じゅんばん どおり に ならべ かえて ください			
\\	効く	
\\	きく			
\\	彼女は奨学金で大学に行きました。	
\\	奨学金 
\\	かのじょ は しょうがくきん で だいがく に いきました			
\\	この曲を聞くと青春の頃を思い出します。	
\\	青春 
\\	この きょく を きく と せいしゅん の ころ を おもいだします			
\\	部屋の明かりを点けましょう。	
\\	明かり 
\\	へや の あかり を つけましょう			
\\	国語	
\\	こくご			
\\	その町は織物業で有名です。	
\\	織物 
\\	その まち は おりものぎょう で ゆうめい です			
\\	彼は私と再会の約束を交わした。	
\\	交わす 
\\	かれ は わたし と さいかい の やくそく を かわした			
\\	このあたりは高層ビルが増えましたね。	
\\	高層 
\\	この あたり は こうそうびる が ふえました ね			
\\	固体	
\\	こたい			
\\	彼は大学の後輩です。	
\\	後輩 
\\	かれ は だいがく の こうはい です			
\\	あの国とは国交がない。	
\\	国交 
\\	あの くに と は こっこう が ない			
\\	彼は頻りに時計を気にしていた。	
\\	頻りに 
\\	かれ は しきりに とけい を き に して いた			
\\	好み	
\\	このみ			
\\	新幹線の最高時速は300キロです。	
\\	時速 
\\	しんかんせん の さいこう じそく は さんびゃくきろ です			
\\	水は水素と酸素でできています。	
\\	水素 
\\	みず は すいそ と さんそ で できて います			
\\	体育館に行って整列しなさい。	
\\	整列 
\\	たいいくかん に いって せいれつ しなさい			
\\	水平	
\\	すいへい			
\\	皆の顔と名前を一度には覚えられません。	
\\	一度に 
\\	みんな の かお と なまえ を いちどに は おぼえられません			
\\	ここでは喫煙できません。	
\\	喫煙 
\\	ここ で は きつえん できません 。			
\\	西洋	
\\	せいよう			
\\	子供の健全な心を育てましょう。	
\\	健全 
\\	こども の けんぜん な こころ を そだてましょう			
\\	彼はモデルと交際していたんだ。	
\\	交際 
\\	かれ は もでる と こうさい して いた ん だ			
\\	彼女は合理的な考え方をする人です。	
\\	合理的 
\\	かのじょ は ごうりてき な かんがえかた を する ひと です			
\\	祝う	
\\	いわう			
\\	誕生日に小鳥を買ってもらいました。	
\\	小鳥 
\\	たんじょうび に ことり を かって もらいました			
\\	彼女は結婚生活に失望していたの。	
\\	失望 
\\	かのじょ は けっこん せいかつ に しつぼう して いた の			
\\	街で知り合いを見かけたよ。	
\\	知り合い 
\\	まち で しりあい を みかけた よ			
\\	刑事	
\\	けいじ			
\\	娘が今年成人します。	
\\	成人 
\\	むすめ が ことし せいじん します			
\\	結婚のお祝いに食器を頂きました。	
\\	祝い 
\\	けっこん の おいわい に しょっき を いただきました			
\\	お風呂に花を浮かべて入ったの。	
\\	浮かべる 
\\	おふろ に はな を うかべて はいった の			
\\	自慢	
\\	じまん			
\\	何事も最初が肝心だ。	
\\	肝心 
\\	なにごと も さいしょ が かんじん だ			
\\	苦心して絵を描き上げました。	
\\	苦心 
\\	くしん して え を かきあげました			
\\	彼は形式的な質問をしただけだった。	
\\	形式的 
\\	かれ は けいしきてき な しつもん を した だけ だった			
\\	出版社	
\\	しゅっぱんしゃ			
\\	入場券をお持ちですか。	
\\	券 
\\	にゅうじょうけん を お もち です か			
\\	上司から厳重に注意されました。	
\\	厳重 
\\	じょうし から げんじゅう に ちゅうい されました			
\\	このフィルムを現像してください。	
\\	現像 
\\	この ふぃるむ を げんぞう して ください			
\\	消極的	
\\	しょうきょくてき			
\\	この道をまっすぐ進むと国道に出ます。	
\\	国道 
\\	この みち を まっすぐ すすむ と こくどう に でます			
\\	夕方の市街は車が渋滞するよ。	
\\	市街 
\\	ゆうがた の しがい は くるま が じゅうたい する よ			
\\	火事は無事消火されました。	
\\	消火 
\\	かじ は ぶじ しょうか されました			
\\	植民地	
\\	しょくみんち			
\\	少々のことは我慢します。	
\\	少々 
\\	しょうしょう の こと は がまん します			
\\	衣類の整理をしたよ。	
\\	衣類 
\\	いるい の せいり を した よ			
\\	女優	
\\	じょゆう			
\\	この先にキリスト教の教会があります。	
\\	キリスト教 
\\	この さき に きりすときょう の きょうかい が あります			
\\	駅前に公立の図書館があるよ。	
\\	公立 
\\	えきまえ に こうりつ の としょかん が ある よ			
\\	海外で語学の勉強をします。	
\\	語学 
\\	かいがい で ごがく の べんきょう を します			
\\	身長	
\\	しんちょう			
\\	毎朝7時5分に出勤します。	
\\	出勤 
\\	まいあさ しち じ ご ふん に しゅっきん します			
\\	彼は去年、家を新築しました。	
\\	新築 
\\	かれ は きょねん いえ を しんちく しました			
\\	私の住まいは東京にあります。	
\\	住まい 
\\	わたし の すまい は とうきょう に あります			
\\	節約	
\\	せつやく			
\\	私たちはみな彼の死を惜しんだの。	
\\	惜しむ 
\\	わたしたち は みな かれ の し を おしんだ の			
\\	彼の得意な学科は数学です。	
\\	学科 
\\	かれ の とくい な がっか は すうがく です			
\\	新学期が始まったね。	
\\	学期 
\\	しんがっき が はじまった ね			
\\	痛み	
\\	いたみ			
\\	彼は最近筋肉を鍛えているよ。	
\\	筋肉 
\\	かれ は さいきん きんにく を きたえて いる よ			
\\	父の表情が険しくなったの。	
\\	険しい 
\\	ちち の ひょうじょう が けわしく なった の			
\\	新しいコーチの指導は厳しかった。	
\\	コーチ 
\\	あたらしい こーち の しどう は きびしかった			
\\	開会	
\\	かいかい			
\\	その国は次第に資本主義になったわね。	
\\	資本主義 
\\	その くに は しだいに しほんしゅぎ に なった わ ね			
\\	空き地で工事が始まった。	
\\	空き地 
\\	あきち で こうじ が はじまった			
\\	彼は大きな過ちを犯している。	
\\	犯す 
\\	かれ は おおき な あやまち を おかして いる			
\\	勘定	
\\	かんじょう			
\\	屋上から富士山が見えました。	
\\	屋上 
\\	おくじょう から ふじさん が みえました			
\\	結婚すると新しい戸籍が作られます。	
\\	戸籍 
\\	けっこん する と あたらしい こせき が つくられます			
\\	漁業	
\\	ぎょぎょう			
\\	彼は握手をしようと手を差し出したの。	
\\	差し出す 
\\	かれ は あくしゅ を しようと て を さしだした の			
\\	彼はステージに立ったよ。	
\\	ステージ 
\\	かれ は すてーじ に たった よ			
\\	固有	
\\	こゆう			
\\	彼のアドバイスはいつも有り難いわね。	
\\	アドバイス 
\\	かれ の あどばいす は いつも ありがたい わ ね			
\\	このお鍋はアルミニウム製です。	
\\	アルミニウム 
\\	この おなべ は あるみにうむ せい です			
\\	彼はオーケストラを指揮しているの。	
\\	オーケストラ 
\\	かれ は おーけすとら を しき して いる の			
\\	札	
\\	さつ			
\\	今そこで隣のおじさんに会ったよ。	
\\	おじさん 
\\	いま そこで となり の おじさん に あった よ			
\\	今夜はオペラ鑑賞に行きます。	
\\	鑑賞 
\\	こんや は おぺら かんしょう に いきます			
\\	収支	
\\	しゅうし			
\\	高い山は気圧が低いね。	
\\	気圧 
\\	たかい やま は きあつ が ひくい ね			
\\	この時計はすぐ狂うの。	
\\	狂う 
\\	この とけい は すぐ くるう の			
\\	誰も彼の苦しみを理解していなかったな。	
\\	苦しみ 
\\	だれ も かれ の くるしみ を りかい して いなかった な			
\\	振動	
\\	しんどう			
\\	後期の授業が始まりました。	
\\	後期 
\\	こうき の じゅぎょう が はじまりました			
\\	開会式で選手たちが行進しました。	
\\	行進 
\\	かいかいしき で せんしゅたち が こうしん しました			
\\	来年のコンクールに出場するつもりです。	
\\	コンクール 
\\	らいねん の こんくーる に しゅつじょう する つもり です			
\\	信念	
\\	しんねん			
\\	出血がひどいので、医者に行ったほうがいい。	
\\	出血 
\\	しゅっけつ が ひどい の で いしゃ に いった ほう が いい			
\\	彼は消化不良を起こしたんだ。	
\\	消化 
\\	かれ は しょうか ふりょう を おこした ん だ			
\\	せめて	
\\	道路を横断するときは注意して。	
\\	横断 
\\	どうろ を おうだん する とき は ちゅうい して			
\\	友人の強引な誘いを断れませんでした。	
\\	強引 
\\	ゆうじん の ごういん な さそい を ことわれません でした			
\\	驚き	
\\	おどろき			
\\	陸上部は校庭で練習しています。	
\\	校庭 
\\	りくじょうぶ は こうてい で れんしゅう して います			
\\	彼は孤独な人生を送っていたんだ。	
\\	孤独 
\\	かれ は こどく な じんせい を おくって いた ん だ			
\\	彼がそのプロジェクトの指揮を取ったの。	
\\	指揮 
\\	かれ が その ぷろじぇくと の しき を とった の			
\\	学年	
\\	がくねん			
\\	ただ今の時刻は6時35分です。	
\\	時刻 
\\	ただいま の じこく は ろくじ さんじゅうごふん です			
\\	子供の純粋な心を傷付けてはいけません。	
\\	純粋 
\\	こども の じゅんすい な こころ を きずつけて は いけません			
\\	家計	
\\	かけい			
\\	助詞を変えると文の意味が変わります。	
\\	助詞 
\\	じょし を かえる と ぶん の いみ が かわります			
\\	子供は好奇心でいっぱいだね。	
\\	好奇心 
\\	こども は こうきしん で いっぱい だ ね			
\\	管	
\\	かん			
\\	今日は料理の実習があった。	
\\	実習 
\\	きょう は りょうり の じっしゅう が あった			
\\	この国道はよく渋滞します。	
\\	渋滞 
\\	この こくどう は よく じゅうたい します			
\\	その選手は2種目でメダルを取ったわ。	
\\	種目 
\\	その せんしゅ は にしゅもく で めだる を とった わ			
\\	気象	
\\	きしょう			
\\	彼の商売は儲かっているな。	
\\	商売 
\\	かれ の しょうばい は もうかって いる な			
\\	彼は私の親友です。	
\\	親友 
\\	かれ は わたし の しんゆう です			
\\	大学では物理を専攻していました。	
\\	専攻 
\\	だいがく で は ぶつり を せんこう して いました			
\\	議題	
\\	ぎだい			
\\	彼はアマチュア音楽家です。	
\\	アマチュア 
\\	かれ は あまちゅあ おんがくか です			
\\	彼女は語学力を仕事に生かしているね。	
\\	生かす 
\\	かのじょ は ごがくりょく を しごと に いかして いる ね			
\\	片手運転は危ないよ。	
\\	片手 
\\	かたて うんてん は あぶないよ			
\\	近郊	
\\	きんこう			
\\	彼女は性格がきついよね。	
\\	きつい 
\\	かのじょ は せいかく が きつい よ ね			
\\	バーゲンコーナーで
\\	を買いました。	
\\	コーナー 
\\	ばーげんこーなー で でぃーぶいでぃー を かいました			
\\	問題の根本を見直しましょう。	
\\	根本 
\\	もんだい の こんぽん を みなおしましょう			
\\	結構	
\\	けっこう			
\\	今日、合格の知らせをもらいました。	
\\	知らせ 
\\	きょう ごうかく の しらせ を もらいました			
\\	彼は素早くあたりを見回したの。	
\\	素早い 
\\	かれ は すばやく あたり を みまわした の			
\\	公平	
\\	こうへい			
\\	山の空気は澄んでいますね。	
\\	澄む 
\\	やま の くうき は すんで います ね			
\\	前期の売上はとても良かったわ。	
\\	前期 
\\	ぜんき の うりあげ は とても よかった わ			
\\	混雑	
\\	こんざつ			
\\	最近は悪質な事件が多いですね。	
\\	悪質 
\\	さいきん は あくしつ な じけん が おおい です ね			
\\	彼は金遣いが荒いな。	
\\	荒い 
\\	かれ は かねづかい が あらい な			
\\	1年生を受け持っています。	
\\	受け持つ 
\\	いちねんせい を うけもって います			
\\	信仰	
\\	しんこう			
\\	彼は研究に打ち込んでいます。	
\\	打ち込む 
\\	かれ は けんきゅう に うちこんで います			
\\	工場の煙突から煙が上がっている。	
\\	煙突 
\\	こうじょう の えんとつ から けむり が あがって いる			
\\	午後は外出の予定です。	
\\	外出 
\\	ごご は がいしゅつ の よてい です			
\\	境界	
\\	きょうかい			
\\	歌舞伎の芝居を見に行きました。	
\\	歌舞伎 
\\	かぶき の しばい を み に いきました 。			
\\	感激して泣いてしまいました。	
\\	感激 
\\	かんげき して ないて しまいました			
\\	漢語はもともと外来語です。	
\\	漢語 
\\	かんご は もともと がいらいご です			
\\	小麦	
\\	こむぎ			
\\	私たちは客席に座ったんだ。	
\\	客席 
\\	わたしたち は きゃくせき に すわった ん だ			
\\	会社の始業時間にぎりぎりで間に合ったよ。	
\\	ぎりぎり 
\\	かいしゃ の しぎょうじかん に ぎりぎり で まにあった よ			
\\	彼の発明品はみな実用的だよ。	
\\	実用的 
\\	かれ の はつめいひん は みな じつようてき だ よ			
\\	四方	
\\	しほう			
\\	床に洗剤をこぼしてしまいました。	
\\	洗剤 
\\	ゆか に せんざい を こぼして しまいました			
\\	イスラム教	
\\	イスラムきょう			
\\	これはイスラム教の寺院です。	
\\	イスラム教 
\\	これ は いすらむきょう の じいん です			
\\	論文にその本を引用したの。	
\\	引用 
\\	ろんぶん に その ほん を いんよう した の			
\\	引用	
\\	いんよう			
\\	うちの犬は気が荒い。	
\\	荒い 
\\	うち の いぬ は き が あらい			
\\	ウサギとカメの話を知っていますか。	
\\	ウサギ 
\\	うさぎ と かめ の はなし を しって います か			
\\	私は両親から多くを教わりました。	
\\	教わる 
\\	わたし は りょうしん から おおく を おそわりました			
\\	荒い	
\\	あらい			
\\	そのモダンな外観の建物が大使館です。	
\\	外観 
\\	その もだん な がいかん の たてもの が たいしかん です			
\\	電車は5分間隔で来ますよ。	
\\	間隔 
\\	でんしゃ は ごふん かんかく で きます よ			
\\	船がやっと岸に着いたよ。	
\\	岸 
\\	ふね が やっと きし に ついた よ			
\\	ウサギ	
\\	野球部はグラウンドで練習しています。	
\\	グラウンド 
\\	やきゅうぶ は ぐらうんど で れんしゅう して います			
\\	親類の家に4年間下宿しました。	
\\	下宿 
\\	しんるい の いえ に よねんかん げしゅく しました			
\\	彼は後方の座席に着いたの。	
\\	後方 
\\	かれ は こうほう の ざせき に ついた の			
\\	教わる	
\\	おそわる			
\\	暑さは峠を越したね。	
\\	越す 
\\	あつさ は とうげ を こした ね			
\\	「この会議は退屈だ」と同僚が私に囁いたの。	
\\	囁く 
\\	この かいぎ は たいくつ だ 
\\	と どうりょう が わたし に ささやいた の 。			
\\	古新聞をひもで縛ったよ。	
\\	縛る 
\\	ふるしんぶん を ひも で しばった よ 。			
\\	外観	
\\	がいかん			
\\	彼女は会計主任です。	
\\	主任 
\\	かのじょ は かいけい しゅにん です			
\\	雲一つない青空ですね。	
\\	青空 
\\	くも ひとつ ない あおぞら です ね			
\\	その頃彼はまだ赤ん坊だったよね。	
\\	赤ん坊 
\\	その ころ かれ は まだ あかんぼう だった よ ね			
\\	間隔	
\\	かんかく			
\\	彼は仲間を裏切ったの。	
\\	裏切る 
\\	かれ は なかま を うらぎった の			
\\	彼女は英文の手紙を書きました。	
\\	英文 
\\	かのじょ は えいぶん の てがみ を かきました			
\\	下流に小さな滝があるよ。	
\\	下流 
\\	かりゅう に ちいさ な たき が ある よ			
\\	岸	
\\	きし			
\\	彼はレントゲン技師です。	
\\	技師 
\\	かれ は れんとげん ぎし です			
\\	彼女はとても教養のある人ですね。	
\\	教養 
\\	かのじょ は とても きょうよう の ある ひと です ね			
\\	金銭のトラブルには関わりたくありません。	
\\	金銭 
\\	きんせん の とらぶる に は かかわりたく ありません			
\\	グラウンド	
\\	彼女は毛皮のコートを着ていたの。	
\\	毛皮 
\\	かのじょ は けがわ の こーと を きて いた の			
\\	そのカメラには最先端の技術が使われています。	
\\	最先端 
\\	その かめら に は さいせんたん の ぎじゅつ が つかわれて います			
\\	下宿	
\\	げしゅく			
\\	彼女の言っていることがさっぱり分からないの。	
\\	さっぱり 
\\	かのじょ の いって いる こと が さっぱり わからない の			
\\	彼は織田信長の子孫だよ。	
\\	子孫 
\\	かれ は おだのぶなが の しそん だ よ			
\\	丸いボタンを押すとシャッターが閉まります。	
\\	シャッター 
\\	まるい ぼたん を おすと しゃったー が しまります			
\\	後方	
\\	こうほう			
\\	私は彼の人格を尊敬しています。	
\\	人格 
\\	わたし は かれ の じんかく を そんけい して います			
\\	ほこりを吸い込んじゃった。	
\\	吸い込む 
\\	ほこり を すいこんじゃった			
\\	越す	
\\	こす			
\\	勧められて欲しくもないものを買っちゃったよ。	
\\	勧める 
\\	すすめられて ほしく も ない もの を かっちゃった よ			
\\	このスピーカーは音がいいですね。	
\\	スピーカー 
\\	この すぴーかー は おと が いい です ね			
\\	子供はたくさんの愛情が必要です。	
\\	愛情 
\\	大雪 
\\	こども は たくさん の あいじょう が ひつよう です			
\\	囁く	
\\	ささやく			
\\	雪の上にうさぎの足跡があった。	
\\	足跡 
\\	ゆき の うえ に うさぎ の あしあと が あった			
\\	台風で大雨になったの。	
\\	大雨 
\\	たいふう で おおあめ に なった の			
\\	この川の河口は太平洋に注いでいます。	
\\	河口 
\\	この かわ の かこう は たいへいよう に そそいで います			
\\	縛る	
\\	しばる			
\\	彼は元来、真面目な人です。	
\\	元来 
\\	かれ は がんらい まじめ な ひと です			
\\	いつでも気軽に遊びに来て下さい。	
\\	気軽 
\\	いつでも きがる に あそび に きて ください			
\\	フロアはパーティションで区切られています。	
\\	区切る 
\\	ふろあ は ぱーてぃしょん で くぎられて います			
\\	主任	
\\	しゅにん			
\\	色の組み合わせで印象が変わりますよ。	
\\	組み合わせ 
\\	いろ の くみあわせ で いんしょう が かわります よ			
\\	芝生がよく手入れされていますね。	
\\	芝生 
\\	しばふ が よく ていれ されて います ね			
\\	青空	
\\	あおぞら			
\\	申し込みの締め切りはいつですか。	
\\	締め切り 
\\	もうしこみ の しめきり は いつ です か			
\\	彼は出世するタイプだな。	
\\	出世 
\\	かれ は しゅっせ する たいぷ だ な			
\\	以前、セールスマンをしていたことがあります。	
\\	セールスマン 
\\	いぜん せーるすまん を して いた こと が あります			
\\	赤ん坊	
\\	あかんぼう			
\\	車掌が発車の合図をしたよ。	
\\	合図 
\\	しゃしょう が はっしゃ の あいず を した よ			
\\	父の教えは「自分に厳しく」です。	
\\	教え 
\\	ちち の おしえ は じぶん に きびしく です			
\\	彼女たちはおしゃべりに夢中です。	
\\	おしゃべり 
\\	かのじょたち は おしゃべり に むちゅう です			
\\	裏切る	
\\	うらぎる			
\\	会社帰りに仲間とカラオケに行きました。	
\\	カラオケ 
\\	かいしゃがえり に なかま と からおけ に いきました			
\\	これはスポーツ施設用の器具です。	
\\	器具 
\\	これ は すぽーつ しせつよう の きぐ です			
\\	彼女はこっそりダイエットを始めたの。	
\\	こっそり 
\\	かのじょ は こっそり だいえっと を はじめた の			
\\	英文	
\\	えいぶん			
\\	私は毎日ジョギングをしています。	
\\	ジョギング 
\\	わたし は まいにち じょぎんぐ を して います			
\\	服を買うのにあれこれ迷いました。	
\\	あれこれ 
\\	ふく を かう の に あれこれ まよいました			
\\	彼女は自分の姿を鏡に映したの。	
\\	映す 
\\	かのじょ は じぶん の すがた を かがみ に うつした の			
\\	下流	
\\	かりゅう			
\\	家から駅までおよそ1キロあります。	
\\	およそ 
\\	いえ から えき まで およそ いちきろ あります			
\\	高原には気持ちのいい風が吹いていたよ。	
\\	高原 
\\	こうげん に は きもち の いい かぜ が ふいていた よ 。			
\\	技師	
\\	ぎし			
\\	おじいちゃんにお小遣いをもらったよ。	
\\	小遣い 
\\	おじいちゃん に おこづかい を もらった よ			
\\	どのサークルに入るか迷っちゃった。	
\\	サークル 
\\	どの さーくる に はいる か まよっちゃった			
\\	この春に彼は社会人になったよ。	
\\	社会人 
\\	この はる に かれ は しゃかいじん に なった よ			
\\	教養	
\\	きょうよう			
\\	彼の趣味は切手収集です。	
\\	収集 
\\	かれ の しゅみ は きって しゅうしゅう です			
\\	毛皮を税関で没収されたんだ。	
\\	毛皮 
\\	税関 
\\	けがわ を ぜいかん で ぼっしゅう された ん だ			
\\	レストランは清潔が第一です。	
\\	清潔 
\\	れすとらん は せいけつ が だいいち です			
\\	金銭	
\\	きんせん			
\\	その海峡に橋が掛けられました。	
\\	海峡 
\\	その かいきょう に はし が かけられました			
\\	栄養が偏らないように食事に気を付けています。	
\\	偏る 
\\	えいよう が かたよらない よう に しょくじ に き を つけて います			
\\	その本は貸し出ししていません。	
\\	貸し出し 
\\	その ほん は かしだし して いません			
\\	毛皮	
\\	けがわ			
\\	汽車で街まで行った。	
\\	汽車 
\\	きしゃ で まち まで いった			
\\	キリストの母親の名前はマリアだ。	
\\	キリスト 
\\	きりすと の ははおや の なまえ は まりあ だ			
\\	寒くて唇が青くなってしまった。	
\\	唇 
\\	さむく て くちびる が あおく なって しまった			
\\	最先端	
\\	さいせんたん			
\\	月給は毎月25日に支給されます。	
\\	月給 
\\	げっきゅう は まいつき にじゅうごにち に しきゅう されます			
\\	この曲は誰が作曲したのですか。	
\\	作曲 
\\	この きょく は だれ が さっきょく した の です か			
\\	さっぱり	
\\	昨日は遅くまで残業しました。	
\\	残業 
\\	きのう は おそく まで ざんぎょう しました			
\\	そうあっさり言わないで。	
\\	あっさり 
\\	そう あっさり いわない で			
\\	子孫	
\\	しそん			
\\	午後に打ち合わせをしましょう。	
\\	打ち合わせ 
\\	ごご に うちあわせ を しましょう			
\\	年と共に体力が衰えています。	
\\	衰える 
\\	とし と とも に たいりょく が おとろえて います			
\\	彼から思いがけないことを聞いた。	
\\	思いがけない 
\\	かれ から おもいがけない こと を きいた			
\\	シャッター	
\\	この鐘の音はとてもきれいですね。	
\\	鐘 
\\	この かね の ね は とても きれい です ね			
\\	私は人前で話すことが苦痛です。	
\\	苦痛 
\\	わたし は ひとまえ で はなす こと が くつう です			
\\	人格	
\\	じんかく			
\\	日が暮れる前に帰りましょう。	
\\	暮れる 
\\	ひ が くれる まえ に かえりましょう			
\\	昔は結核でたくさんの人が亡くなった。	
\\	結核 
\\	むかし は けっかく で たくさん の ひと が なくなった			
\\	ここはかつてゴールドラッシュで栄えた町だよ。	
\\	栄える 
\\	ここ は かつて ごーるど らっしゅ で さかえた まち だ よ			
\\	吸い込む	
\\	すいこむ			
\\	連絡があり次第出発します。	
\\	次第 
\\	れんらく が あり しだい しゅっぱつ します			
\\	昼ごはんは簡単に済ませましょう。	
\\	済ませる 
\\	ひるごはん は かんたん に すませましょう			
\\	彼は朝から晩までせっせと働いたの。	
\\	せっせと 
\\	かれ は あさ から ばん まで せっせと はたらいた の			
\\	勧める	
\\	すすめる			
\\	今日は慌ただしい一日でした。	
\\	慌ただしい 
\\	きょう は あわただしい いちにち でした			
\\	外はもう薄暗くなりましたよ。	
\\	薄暗い 
\\	そと は もう うすぐらく なりました よ			
\\	カタログを見て注文したんだ。	
\\	カタログ 
\\	かたろぐ を みて ちゅうもん した ん だ			
\\	スピーカー	
\\	彼は自分の状況を客観的に見てみたのね。	
\\	客観的 
\\	かれ は じぶん の じょうきょう を きゃっかんてき に みて みた の ね			
\\	彼は気力にあふれていますね。	
\\	気力 
\\	かれ は きりょく に あふれて います ね			
\\	この郡は来年、市になりますよ。	
\\	郡 
\\	この ぐん は らいねん し に なります よ			
\\	愛情	
\\	あいじょう			
\\	あの子は下町育ちだ。	
\\	下町 
\\	あの こ は したまち そだち だ			
\\	昨日の夜、ジャズを聴きに行きました。	
\\	ジャズ 
\\	きのう の よる じゃず を きき に いきました			
\\	私は心身共に疲れていました。	
\\	心身 
\\	わたし は しんしん ともに つかれて いました			
\\	足跡	
\\	あしあと			
\\	氷は水に浮きます。	
\\	浮く 
\\	こおり は みず に うきます			
\\	運動場でサッカーをしよう。	
\\	運動場 
\\	うんどうじょう で さっかー を しよう			
\\	家具の配置を変えました。	
\\	家具 
\\	かぐ の はいち を かえました 。			
\\	大雨	
\\	おおあめ			
\\	日本映画で刀を見た。	
\\	刀 
\\	にほん えいが で かたな を みた			
\\	彼は工学部の教授です。	
\\	工学部 
\\	かれ は こうがくぶ の きょうじゅ です			
\\	ここにサインしてください。	
\\	サイン 
\\	ここにサインしてください。			
\\	河口	
\\	かこう			
\\	このトラックの車輪は頑丈そうですね。	
\\	車輪 
\\	この とらっく の しゃりん は がんじょう そう です ね			
\\	食器を全部新しくしました。	
\\	食器 
\\	しょっき を ぜんぶ あたらしく しました			
\\	娘は市立の学校に通っています。	
\\	市立 
\\	むすめ は しりつ の がっこう に かよって います			
\\	元来	
\\	がんらい			
\\	その絵は一見に値しますよ。	
\\	値する 
\\	その え は いっけん に あたい します よ			
\\	彼は偉大な事業をなしとげました。	
\\	偉大 
\\	かれ は いだい な じぎょう を なしとげ ました			
\\	彼は試合の途中で棄権したぞ。	
\\	棄権 
\\	かれ は しあい の とちゅう で きけん した ぞ			
\\	気軽	
\\	きがる			
\\	友人が骨折して外科に入院しました。	
\\	外科 
\\	ゆうじん が こっせつ して げか に にゅういん しました			
\\	僕たちは頑張って決勝まで進んだよ。	
\\	決勝 
\\	ぼくたち は がんばって けっしょう まで すすんだ よ			
\\	彼は公衆電話を探したの。	
\\	公衆 
\\	かれ は こうしゅう でんわ を さがした の			
\\	区切る	
\\	くぎる			
\\	彼は質問の答えをごまかしたね。	
\\	ごまかす 
\\	かれ は しつもん の こたえ を ごまかした ね			
\\	要らないファイルは削除して下さい。	
\\	削除 
\\	いらない ふぁいる は さくじょ して ください			
\\	組み合わせ	
\\	くみあわせ			
\\	彼女はインテリで、しかも美人ですね。	
\\	インテリ 
\\	かのじょ は いんてり で しかも びじん です ね			
\\	彼は家から追い出された。	
\\	追い出す 
\\	かれ は いえ から おいだされた			
\\	丘の上にホテルが建ちましたね。	
\\	丘 
\\	おか の うえ に ほてる が たちました ね			
\\	芝生	
\\	しばふ			
\\	彼は騙されたと思い込んでいるようです。	
\\	思い込む 
\\	かれ は だまされた と おもいこんで いる よう です			
\\	浜辺できれいな貝を拾いました。	
\\	貝 
\\	はまべ で きれい な かい を ひろいました			
\\	彼は空想にふけっているの。	
\\	空想 
\\	かれ は くうそう に ふけって いる の			
\\	締め切り	
\\	しめきり			
\\	あそこの3区画は売り出し中です。	
\\	区画 
\\	あそこ の さんくかく は うりだしちゅう です			
\\	この坂を上るのはすごくきついね。	
\\	坂 
\\	この さか を のぼる の は すごく きつい ね			
\\	応募受付は締め切りました。	
\\	締め切る 
\\	おうぼ うけつけ は しめきりました			
\\	出世	
\\	しゅっせ			
\\	彼女はその映画に出演しているよ。	
\\	出演 
\\	かのじょ は その えいが に しゅつえん して いる よ			
\\	彼は積極的に攻めたが勝てなかったな。	
\\	攻める 
\\	かれ は せっきょくてき に せめた が かてなかった な			
\\	彼はいちいち私に指図する。	
\\	いちいち 
\\	かれ は いちいち わたし に さしず する			
\\	セールスマン	
\\	僕が知っているのは一部分に過ぎない。	
\\	一部分 
\\	ぼく が しって いる の は いちぶぶん に すぎない			
\\	その日、国民は平和への祈りを捧げるの。	
\\	祈り 
\\	その ひ こくみん は へいわ へ の いのり を ささげる の			
\\	合図	
\\	あいず			
\\	彼女は長年の恨みを晴らした。	
\\	恨み 
\\	かのじょ は ながねん の うらみ を はらした			
\\	彼は国の英雄ですね。	
\\	英雄 
\\	かれ は くに の えいゆう です ね			
\\	その子は絵本が大好きです。	
\\	絵本 
\\	その こ は えほん が だいすき です			
\\	教え	
\\	おしえ			
\\	彼女の思いやりが嬉しかった。	
\\	思いやり 
\\	かのじょ の おもいやり が うれしかった			
\\	彼は顔色がよくありませんね。	
\\	顔色 
\\	かれ は かおいろ が よく ありません ね 。			
\\	今日大学に願書を送りました。	
\\	願書 
\\	きょう だいがく に がんしょ を おくりました			
\\	おしゃべり	
\\	幸運にもチケットを手に入れました。	
\\	幸運 
\\	こううん に も ちけっと を て に いれました			
\\	先生は試験の採点が終わったようね。	
\\	採点 
\\	せんせい は しけん の さいてん が おわった よう ね			
\\	明日はお座敷での宴会になります。	
\\	座敷 
\\	あした は おざしき で の えんかい に なります			
\\	カラオケ	
\\	彼は司会が上手ですね。	
\\	司会 
\\	かれ は しかい が じょうず です ね			
\\	この新聞の社説は面白いね。	
\\	社説 
\\	この しんぶん の しゃせつ は おもしろい ね			
\\	今日、病院で診察してもらったの。	
\\	診察 
\\	きょう びょういん で しんさつ して もらった の			
\\	器具	
\\	きぐ			
\\	従業員は明日からストライキです。	
\\	ストライキ 
\\	じゅうぎょういん は あす から すとらいき です			
\\	妹はマフラーを編みました。	
\\	編む 
\\	いもうと は まふらー を あみました			
\\	これはあんまり好きじゃないな。	
\\	あんまり 
\\	これ は あんまり すき じゃ ない な			
\\	こっそり	
\\	彼の会社は衣服を扱っているよ。	
\\	衣服 
\\	かれ の かいしゃ は いふく を あつかって いる よ			
\\	外来語は一般にカタカナで書かれます。	
\\	外来語 
\\	がいらいご は いっぱんに かたかな で かかれます			
\\	彼はアルバイトをして学費を稼いだんだ。	
\\	学費 
\\	かれ は アルバイト を して がくひ を かせいだ ん だ 。			
\\	ジョギング	
\\	こちらにお名前をご記入ください。	
\\	記入 
\\	こちら に おなまえ を ご きにゅう ください			
\\	彼女は強烈な個性の持ち主ですよ。	
\\	強烈 
\\	かのじょ は きょうれつ な こせい の もちぬし です よ			
\\	レモンの爽やかな香りがした。	
\\	爽やか 
\\	れもん の さわやか な かおり が した			
\\	あれこれ	
\\	実例を使って説明してください。	
\\	実例 
\\	じつれい を つかって せつめい して ください			
\\	猫が障子を破った。	
\\	障子 
\\	ねこ が しょうじ を やぶった			
\\	彼女は育児で忙しいよ。	
\\	育児 
\\	かのじょ は いくじ で いそがしい よ			
\\	映す	
\\	うつす			
\\	親友に悩みを打ち明けたの。	
\\	打ち明ける 
\\	しんゆう に なやみ を うちあけた の			
\\	彼の才能は羨ましい程だね。	
\\	羨ましい 
\\	かれ の さいのう は うらやましい ほど だ ね			
\\	彼は運のいい男です。	
\\	運 
\\	かれ は うん の いい おとこ です			
\\	およそ	
\\	弟は落ち着きが足りません。	
\\	落ち着き 
\\	おとうと は おちつき が たりません			
\\	会費が少し高いね。	
\\	会費 
\\	かいひ が すこし たかい ね			
\\	高原	
\\	こうげん			
\\	天気予報は当たる確率が高くなりましたね。	
\\	確率 
\\	てんき よほう は あたる かくりつ が たかく なりました ね			
\\	私たちは校歌を合唱したの。	
\\	合唱 
\\	わたしたち は こうか を がっしょう した の			
\\	小遣い	
\\	こづかい			
\\	来週、帰省します。	
\\	帰省 
\\	らいしゅう きせい します			
\\	明日は臨時に休業します。	
\\	休業 
\\	あした は りんじ に きゅうぎょう します			
\\	「大きい」は形容詞です。	
\\	形容詞 
\\	おおきい は けいようし です			
\\	サークル	
\\	彼は彼女に好意をもっています。	
\\	好意 
\\	かれ は かのじょ に こうい を もって います			
\\	この山は紅葉がとても美しい。	
\\	紅葉 
\\	この やま は こうよう が とても うつくしい			
\\	社会人	
\\	しゃかいじん			
\\	消防士は勇敢でなければならない。	
\\	消防 
\\	しょうぼうし は ゆうかん で なけれ ば ならない			
\\	今日は食欲がありません。	
\\	食欲 
\\	きょう は しょくよく が ありません			
\\	収集	
\\	しゅうしゅう			
\\	身なりのいい紳士が話しかけてきたの。	
\\	紳士 
\\	みなり の いい しんし が はなしかけて きた の			
\\	真実は一つです。	
\\	真実 
\\	しんじつ は ひとつ です			
\\	娘の存在が私の救いでした。	
\\	救い 
\\	むすめ の そんざい が わたし の すくい でした			
\\	税関	
\\	ぜいかん			
\\	朝早く砂浜を散歩しました。	
\\	砂浜 
\\	あさ はやく すなはま を さんぽ しました			
\\	彼は地位を悪用しています。	
\\	悪用 
\\	かれ は ちい を あくよう して います			
\\	清潔	
\\	せいけつ			
\\	彼の趣味は囲碁です。	
\\	囲碁 
\\	かれ の しゅみ は いご です			
\\	あまりうるさく言わないで下さい。	
\\	うるさい 
\\	あまり うるさく いわない で ください			
\\	あなたを驚かす話があります。	
\\	驚かす 
\\	あなた を おどろかす はなし が あります			
\\	海峡	
\\	かいきょう			
\\	親指を怪我しました。	
\\	親指 
\\	おやゆび を けがしました			
\\	このカーブは気をつけて。	
\\	カーブ 
\\	この かーぶ は き を つけて			
\\	ガイドつきの旅行を申し込んだの。	
\\	ガイド 
\\	がいど つき の りょこう を もうしこんだ の			
\\	偏る	
\\	かたよる			
\\	田んぼの中でかえるが合唱している。	
\\	かえる 
\\	たんぼ の なか で かえる が がっしょう している 。			
\\	ごみは各自で持ち帰ってください。	
\\	各自 
\\	ごみ は かくじ で もちかえって ください			
\\	彼は冷静で賢い男だね。	
\\	賢い 
\\	かれ は れいせい で かしこい おとこ だ ね			
\\	貸し出し	
\\	かしだし			
\\	子供に昔話を聞かせてあげたの。	
\\	聞かす 
\\	こども に むかしばなし を きかせてあげた の 。			
\\	彼女は気が利いている。	
\\	利く 
\\	かのじょ は き が きいて いる			
\\	代金を期日までにお支払いください。	
\\	期日 
\\	だいきん を きじつ まで に おしはらい ください			
\\	汽車	
\\	きしゃ			
\\	その国は共産主義の国だよ。	
\\	共産主義 
\\	その くに は きょうさんしゅぎ の くに だ よ			
\\	強い日差しが選手たちを苦しめたの。	
\\	苦しめる 
\\	つよい ひざし が せんしゅたち を くるしめた の			
\\	彼は県立の高校に通っているよ。	
\\	県立 
\\	かれ は けんりつ の こうこう に かよって いる よ			
\\	キリスト	
\\	彼女はスキーで足を骨折したの。	
\\	骨折 
\\	かのじょ は すきー で あし を こっせつ した の			
\\	猫は砂の上で転がったんだ。	
\\	転がる 
\\	ねこ は すな の うえ で ころがった ん だ			
\\	唇	
\\	くちびる			
\\	姉は大学に在学しています。	
\\	在学 
\\	あね は だいがく に ざいがく して います			
\\	日本の夏は湿度が高いです。	
\\	湿度 
\\	にほん の なつ は しつど が たかい です			
\\	何かの説明をする時は実物を使うとわかりやすいの。	
\\	実物 
\\	なにか の せつめい を する とき は じつぶつ を つかう と わかり やすい の			
\\	月給	
\\	げっきゅう			
\\	彼は上級のコースに上がったよ。	
\\	上級 
\\	かれ は じょうきゅう の こーす に あがった よ			
\\	地図内の小さい建物は省略してあります。	
\\	省略 
\\	ちずない の ちいさい たてもの は しょうりゃく して あります			
\\	寝室の壁紙を張り替えました。	
\\	寝室 
\\	しんしつ の かべがみ を はりかえました			
\\	作曲	
\\	さっきょく			
\\	水害でたくさんの人が家を失ったの。	
\\	水害 
\\	すいがい で たくさん の ひと が いえ を うしなった の			
\\	その子はおもちゃに飽きたようだね。	
\\	飽きる 
\\	その こ は おもちゃ に あきた よう だ ね			
\\	虫歯がずきずき痛みます。	
\\	痛む 
\\	むしば が ずきずき いたみます			
\\	残業	
\\	ざんぎょう			
\\	今ごろそんなこと言わないで。	
\\	今ごろ 
\\	いまごろ そんな こと いわない で			
\\	衛生には十分気を付けましょう。	
\\	衛生 
\\	えいせい に は じゅうぶん き を つけましょう			
\\	彼はお嬢さんと一緒でした。	
\\	お嬢さん 
\\	かれ は おじょうさん と いっしょ でした			
\\	あっさり	
\\	それは間接的な原因の一つです。	
\\	間接 
\\	それ は かんせつてき な げんいん の ひとつ です			
\\	先生が手紙を下さいました。	
\\	下さる 
\\	せんせい が てがみ を くださいました			
\\	彼女は恋をしてきれいになったね。	
\\	恋 
\\	かのじょ は こい を して きれい に なった ね			
\\	打ち合わせ	
\\	うちあわせ			
\\	彼には誰もが親しみを感じます。	
\\	親しみ 
\\	かれ に は だれ も が したしみ を かんじます			
\\	ポスターの位置が少しずれているよ。	
\\	ずれる 
\\	ぽすたー の いち が すこし ずれて いる よ			
\\	衰える	
\\	おとろえる			
\\	赤道に沿って旅をしました。	
\\	赤道 
\\	せきどう に そって たび を しました			
\\	趣味で油絵を描いています。	
\\	油絵 
\\	しゅみ で あぶらえ を かいています			
\\	生き物を大切にしましょう。	
\\	生き物 
\\	いきもの を たいせつ に しましょう			
\\	思いがけない	
\\	おもいがけない			
\\	明日は遠足で動物園に行きます。	
\\	遠足 
\\	あす は えんそく で どうぶつえん に いきます			
\\	彼の話はいつもオーバーです。	
\\	オーバー 
\\	かれ の はなし は いつも おーばー です			
\\	10年振りの大雪です。	
\\	形容詞 
\\	大雪 
\\	じゅうねん ぶり の おおゆき です			
\\	鐘	
\\	かね			
\\	彼らは穏やかな暮らしをしているの。	
\\	穏やか 
\\	かれら は おだやか な くらし を して い の			
\\	仮の申し込みをしました。	
\\	仮 
\\	かり の もうしこみ を しました			
\\	花瓶の花が枯れました。	
\\	枯れる 
\\	かびん の はな が かれました			
\\	苦痛	
\\	くつう			
\\	あの白い建物が県庁です。	
\\	県庁 
\\	あの しろい たてもの が けんちょう です			
\\	そのチームは守備が甘いね。	
\\	守備 
\\	その ちーむ は しゅび が あまい ね			
\\	初夏の高原は気持ちがいいですね。	
\\	初夏 
\\	しょか の こうげん は きもち が いい です ね			
\\	暮れる	
\\	くれる			
\\	この世に正義はないのだろうか。	
\\	正義 
\\	この よ に せいぎ は ない の だろう か			
\\	靴の底にガムがくっ付いてしまった。	
\\	くっ付く 
\\	くつ の そこ に がむ が くっついて しまった			
\\	朝日が昇りましたよ。	
\\	朝日 
\\	あさひ が のぼりました よ			
\\	結核	
\\	けっかく			
\\	その会社は衣料を扱っているんだ。	
\\	衣料 
\\	その かいしゃ は いりょう を あつかって いる ん だ			
\\	部屋にエアコンを取り付けたんだ。	
\\	エアコン 
\\	へや に えあこん を とりつけた ん だ			
\\	栄える	
\\	さかえる			
\\	リレーで彼は2人追い抜いたね。	
\\	追い抜く 
\\	りれー で かれ は ふたり おいぬいた ね			
\\	大通りでパレードが始まるよ。	
\\	大通り 
\\	おおどおり で ぱれーど が はじまる よ			
\\	元日には初詣でに行きます。	
\\	元日 
\\	がんじつ に は はつもうで に いきます			
\\	次第	
\\	しだい			
\\	菊の花を買ってきました。	
\\	菊 
\\	きく の はな を かって きました			
\\	彼らは気の毒な生活をしている。	
\\	気の毒 
\\	かれら は きのどく な せいかつ を して いる			
\\	彼はかなり器用な人です。	
\\	器用 
\\	かれ は かなり きよう な ひと です			
\\	済ませる	
\\	すませる			
\\	爪をかむ癖は直した方がいい。	
\\	癖 
\\	つめ を かむ くせ は なおした ほう が いい			
\\	氷を細かく砕いてください。	
\\	砕く 
\\	こおり を こまかく くだいて ください			
\\	親に逆らうのは良くない事です。	
\\	逆らう 
\\	おや に さからう の は よくない こと です			
\\	せっせと	
\\	お腹の脂肪を取りたい。	
\\	脂肪 
\\	おなか の しぼう を とりたい			
\\	彼はしょっちゅう遅刻しているな。	
\\	しょっちゅう 
\\	かれ は しょっちゅう ちこく して いる な			
\\	姉はスタンドの下で本を読んでいたんだ。	
\\	スタンド 
\\	あね は すたんど の した で ほん を よんで いた ん だ			
\\	慌ただしい	
\\	あわただしい			
\\	平和は人類の永遠のテーマです。	
\\	永遠 
\\	へいわ は じんるい の えいえん の てーま です			
\\	この部屋は飾りが多すぎるな。	
\\	飾り 
\\	この へや は かざり が おおすぎる な			
\\	古い蛍光灯を取り替えてください。	
\\	蛍光灯 
\\	ふるい けいこうとう を とりかえて ください			
\\	薄暗い	
\\	うすぐらい			
\\	古いお寿司を食べて下痢をしてしまったんだ。	
\\	下痢 
\\	ふるい おすし を たべて げり を して しまった ん だ			
\\	私は医者を志しています。	
\\	志す 
\\	わたし は いしゃ を こころざして います			
\\	カタログ	
\\	山で猿の親子を見ました。	
\\	猿 
\\	やま で さる の おやこ を みました			
\\	彼は人生の真理を求めて旅に出た。	
\\	真理 
\\	かれ は じんせい の しんり を もとめて たび に でた			
\\	客観的	
\\	きゃっかんてき			
\\	成績が落ちてがっかりした。	
\\	がっかり 
\\	せいせき が おちて がっかり した			
\\	一年生の課程を修了したの。	
\\	課程 
\\	いちねんせい の かてい を しゅうりょう した の			
\\	言葉の起源に大変興味があります。	
\\	起源 
\\	ことば の きげん に たいへん きょうみ が あります			
\\	気力	
\\	きりょく			
\\	機長の放送があったの。	
\\	機長 
\\	きちょう の ほうそう が あった の			
\\	あの寺の屋根は金色ですね。	
\\	金色 
\\	あの てら の やね は きんいろ です ね			
\\	郡	
\\	ぐん			
\\	彼女は車椅子に乗っています。	
\\	車椅子 
\\	かのじょ は くるまいす に のっています 。			
\\	月末までに申込書を送ってください。	
\\	月末 
\\	げつまつ まで に もうしこみしょ を おくって ください			
\\	この仕事には何日必要か見当もつかないね。	
\\	見当 
\\	この しごと に は なんにち ひつよう か けんとう も つかない ね			
\\	下町	
\\	したまち			
\\	僕は彼のしたことは肯定できないな。	
\\	肯定 
\\	ぼく は かれ の した こと を こうてい できない な			
\\	その薬は効力が強いよ。	
\\	効力 
\\	その くすり は こうりょく が つよい よ			
\\	ジャズ	
\\	先生にそんな言葉遣いをしてはいけません。	
\\	言葉遣い 
\\	せんせい に そんな ことばづかい を して は いけません			
\\	このふたはスライドします。	
\\	スライド 
\\	このふたはスライドします。			
\\	彼はとても誠実な人です。	
\\	誠実 
\\	かれ は とても せいじつ な ひと です			
\\	心身	
\\	しんしん			
\\	受験生は正門から入って下さい。	
\\	正門 
\\	じゅけんせい は せいもん から はいって ください			
\\	神道では先祖を大切にします。	
\\	先祖 
\\	しんとう で は せんぞ を たいせつ に します			
\\	案外	
\\	あんがい			
\\	彼は案外いい人かも知れない。	
\\	案外 
\\	かれ は あんがい いい ひと かも しれない			
\\	あなたのおっしゃる通りです。	
\\	おっしゃる 
\\	あなた の おっしゃる とおり です			
\\	彼女はクラシック音楽が好きです。	
\\	クラシック 
\\	かのじょ は くらしっく おんがく が すき です			
\\	おっしゃる	
\\	その道は今、下水の工事をしているよ。	
\\	下水 
\\	その みち は いま げすい の こうじ を して いる よ			
\\	作文を書くのに原稿用紙を使ったの。	
\\	原稿用紙 
\\	さくぶん を かく の に げんこうようし を つかった の			
\\	クラシック	
\\	彼は国費で留学しています。	
\\	国費 
\\	かれ は こくひ で りゅうがく して います			
\\	テレビは彼のいちばんの娯楽です。	
\\	娯楽 
\\	てれび は かれ の いちばん の ごらく です			
\\	下水	
\\	げすい			
\\	船から材木が降ろされていますね。	
\\	材木 
\\	ふね から ざいもく が おろされて います ね			
\\	こぼれた牛乳をさっと拭き取ったの。	
\\	さっと 
\\	こぼれた ぎゅうにゅう を さっと ふきとった の			
\\	原稿用紙	
\\	げんこうようし			
\\	ここは市民の運動場です。	
\\	運動場 
\\	ここ は しみん の うんどうじょう です			
\\	次回の会議は2週間後に行います。	
\\	次回 
\\	じかい の かいぎ は に しゅうかん ご に おこないます			
\\	国費	
\\	こくひ			
\\	咳が止まらないので病院に行ってきたの。	
\\	咳 
\\	せき が とまらない ので びょういん に いってきた の 。			
\\	廊下から大きな足音が聞こえた。	
\\	足音 
\\	ろうか から おおき な あしおと が きこえた			
\\	娯楽	
\\	ごらく			
\\	年号が改まりました。	
\\	改まる 
\\	ねんごう が あらたまりました			
\\	赤に黄色を合わすと何色になりますか。	
\\	合わす 
\\	あか に きいろ を あわすと なにいろ に なります か			
\\	お客様にお菓子を頂いた。	
\\	頂く 
\\	おきゃくさま に おかし を いただいた			
\\	材木	
\\	ざいもく			
\\	尾の長い鳥が飛んでいますね。	
\\	尾 
\\	お の ながい とり が とんで います ね			
\\	彼の話は大げさだ。	
\\	大げさ 
\\	かれ の はなし は おおげさ だ			
\\	夏季講習に申し込みした?	
\\	夏季 
\\	かき こうしゅう に もうしこみ した 
\\	さっと	
\\	彼は経済学部の学生です。	
\\	学部 
\\	かれ は けいざい がくぶ の がくせい です			
\\	子供たちは広場に駆けて行ったわよ。	
\\	駆ける 
\\	こどもたち は ひろば に かけて いった わ よ			
\\	運動場	
\\	うんどうじょう			
\\	なんて頑固な子だ。	
\\	頑固 
\\	なんて がんこ な こ だ			
\\	このシャツは絹でできています。	
\\	絹 
\\	この しゃつ は きぬ で できて います			
\\	チューリップは茎が長いです。	
\\	茎 
\\	ちゅーりっぷ は くき が ながい です			
\\	次回	
\\	じかい			
\\	彼らは軽快なステップで踊り出したの。	
\\	軽快 
\\	かれら は けいかい な すてっぷ で おどりだした の			
\\	地震の後、街は地獄のようだったよ。	
\\	地獄 
\\	じしん の あと まち は じごく の よう だった よ			
\\	この柿は渋いね。	
\\	渋い 
\\	この かき は しぶい ね			
\\	咳	
\\	せき			
\\	短い時間にずいぶん上達しましたね。	
\\	上達 
\\	みじかい じかん に ずいぶん じょうたつ しました ね			
\\	足の傷を消毒したわ。	
\\	消毒 
\\	あし の きず を しょうどく した わ			
\\	このセットで勝敗が決まります。	
\\	勝敗 
\\	この せっと で しょうはい が きまります			
\\	足音	
\\	あしおと			
\\	祖父の家に親戚が集まったんだ。	
\\	親戚 
\\	そふ の いえ に しんせき が あつまった ん だ			
\\	今日は頭痛がします。	
\\	頭痛 
\\	きょう は ずつう が します			
\\	彼の息子が家出したそうよ。	
\\	家出 
\\	かれ の むすこ が いえで した そう よ			
\\	改まる	
\\	あらたまる			
\\	ここの大家は近くに住んでいますよ。	
\\	大家 
\\	ここ の おおや は ちかく に すんで います よ			
\\	歌合戦は赤組の勝ちでしたね。	
\\	勝ち 
\\	うたがっせん は あか ぐみ の かち でした ね			
\\	予約をキャンセルしたよ。	
\\	キャンセル 
\\	よやく を きゃんせる した よ			
\\	合わす	
\\	あわす			
\\	香水のいい香りがした。	
\\	香水 
\\	こうすい の いい かおり が した 。			
\\	メイドがシーツを交換してくれたね。	
\\	シーツ 
\\	めいど が しーつ を こうかん して くれた ね			
\\	頂く	
\\	いただく			
\\	車を車庫に入れておきました。	
\\	車庫 
\\	くるま を しゃこ に いれて おきました			
\\	予算案が衆議院を通過したな。	
\\	衆議院 
\\	よさんあん が しゅうぎいん を つうか した な			
\\	受話器を取ってもらえますか。	
\\	受話器 
\\	じゅわき を とって もらえます か			
\\	尾	
\\	お			
\\	今年は去年よりも順位が上がりました。	
\\	順位 
\\	ことし は きょねん より も じゅんい が あがりました			
\\	服に墨がついちゃった。	
\\	墨 
\\	ふく に すみ が ついちゃった			
\\	彼女の親切に甘えました。	
\\	甘える 
\\	かのじょ の しんせつ に あまえました			
\\	大げさ	
\\	おおげさ			
\\	彼女は生け花の先生です。	
\\	生け花 
\\	かのじょ は いけばな の せんせい です			
\\	彼女は会場整理の係だったの。	
\\	係 
\\	かのじょ は かいじょう せいり の かかり だった の			
\\	彼女はインフルエンザにかかっていますね。	
\\	かかる 
\\	かのじょ は いんふるえんざ に かかって います ね			
\\	夏季	
\\	かき			
\\	学力を付けてその大学に進みたい。	
\\	学力 
\\	がくりょく を つけて その だいがく に すすみたい			
\\	博物館で昔の貨幣を見ました。	
\\	貨幣 
\\	はくぶつかん で むかし の かへい を みました			
\\	神様にお願いしました。	
\\	神様 
\\	かみさま に おねがい しました			
\\	学部	
\\	がくぶ			
\\	スタジアムは観衆で満員だったよ。	
\\	観衆 
\\	すたじあむ は かんしゅう で まんいん だった よ			
\\	北風が冷たいです。	
\\	北風 
\\	きたかぜ が つめたい です			
\\	彼女は今、休養中です。	
\\	休養 
\\	かのじょ は いま きゅうようちゅう です			
\\	駆ける	
\\	かける			
\\	見物人が大勢集まっているね。	
\\	見物人 
\\	けんぶつにん が おおぜい あつまって いる ね			
\\	彼は腕のいいコックです。	
\\	コック 
\\	かれ は うで の いい こっく です			
\\	その先生はいつも授業の前に雑談をするの。	
\\	雑談 
\\	その せんせい は いつも じゅぎょう の まえ に ざつだん を する の			
\\	頑固	
\\	がんこ			
\\	何だか寒気がします。	
\\	寒気 
\\	なんだか さむけ が します			
\\	うちでは夫も炊事をします。	
\\	炊事 
\\	うち で は おっと も すいじ を します			
\\	1日7時間は睡眠を取るようにしています。	
\\	睡眠 
\\	いちにち しちじかん は すいみん を とる よう に して います			
\\	絹	
\\	きぬ			
\\	明け方に雨が降り始めましたね。	
\\	明け方 
\\	あけがた に あめ が ふりはじめました ね			
\\	彼は悪酔いして暴れたんだ。	
\\	暴れる 
\\	かれ は わるよい して あばれた ん だ			
\\	茎	
\\	くき			
\\	学校での苛めはなかなか減らない。	
\\	苛め 
\\	がっこう で の いじめ は なかなか へらない 。			
\\	お芋の料理はお好きですか。	
\\	芋 
\\	お いも の りょうり は お すき です か			
\\	被災地では飲料水が不足しているの。	
\\	飲料水 
\\	ひさいち で は いんりょうすい が ふそく して いる の			
\\	軽快	
\\	けいかい			
\\	えびの天ぷらは私の好物です。	
\\	えび 
\\	えび の てんぷら は わたし の こうぶつ です			
\\	理科の時間に石の硬さを比べたよ。	
\\	硬さ 
\\	りか の じかん に いし の かたさ を くらべた よ			
\\	彼は球技が大好きです。	
\\	球 
\\	かれ は きゅうぎ が だいすき です			
\\	地獄	
\\	じごく			
\\	そのグリーンのスカーフ、素敵ですね。	
\\	グリーン 
\\	その ぐりーん の すかーふ すてき です ね			
\\	彼女は快い眠りについています。	
\\	快い 
\\	かのじょ は こころよい ねむり に ついて います			
\\	このホテルでは最上のサービスが受けられます。	
\\	最上 
\\	この ほてる で は さいじょう の さーびす が うけられます			
\\	渋い	
\\	しぶい			
\\	彼はいつもジーパンをはいているね。	
\\	ジーパン 
\\	かれ は いつも じーぱん を はいて いる ね			
\\	支度ができたら出かけましょう。	
\\	支度 
\\	したく が できたら でかけましょう			
\\	彼はジャーナリズムを専攻している。	
\\	ジャーナリズム 
\\	かれ は じゃーなりずむ を せんこう して いる			
\\	上達	
\\	じょうたつ			
\\	男の子が3人もいるので食費がかさみます。	
\\	食費 
\\	おとこ の こ が さんにん も いる の で しょくひ が かさみます			
\\	妹夫婦は新婚です。	
\\	新婚 
\\	いもうとふうふ は しんこん です			
\\	彼のスピーチは素晴らしかった。	
\\	スピーチ 
\\	かれ の スピーチ は すばらしかった 。			
\\	消毒	
\\	しょうどく			
\\	出席者を性別で分けてください。	
\\	性別 
\\	しゅっせきしゃ を せいべつ で わけて ください			
\\	ああうるさい人は苦手です。	
\\	ああ 
\\	ああ うるさい ひと は にがて です			
\\	勝敗	
\\	しょうはい			
\\	アメリカ英語とイギリス英語ではアクセントが違うことがあるね。	
\\	アクセント 
\\	あめりか えいご と いぎりす えいご で は あくせんと が ちがう こと が ある ね			
\\	同じ過ちを繰り返さないことだ。	
\\	過ち 
\\	おなじ あやまち を くりかえさない こと だ			
\\	親戚	
\\	しんせき			
\\	ここに受け取りの印鑑をお願いします。	
\\	印鑑 
\\	受け取り 
\\	ここ に うけとり の いんかん を おねがい します			
\\	それは最新の学説ね。	
\\	学説 
\\	それ は さいしん の がくせつ ね			
\\	この道は片側通行です。	
\\	片側 
\\	この みち は かたがわ つうこう です			
\\	頭痛	
\\	ずつう			
\\	彼は大きな荷物を担いで来たの。	
\\	担ぐ 
\\	かれ は おおき な にもつ を かついで きた の			
\\	彼は犬の柄の
\\	シャツを着ているよ。	
\\	柄 
\\	かれ は いぬ の がら の 
\\	しゃつ を きて いる よ			
\\	彼女は一晩中彼を看病したの。	
\\	看病 
\\	かのじょ は ひとばんじゅう かれ を かんびょう した の			
\\	家出	
\\	いえで			
\\	将来は気楽な生活がしたいです。	
\\	気楽 
\\	しょうらい は きらく な せいかつ が したい です			
\\	父は長い勤労生活を送ったんです。	
\\	勤労 
\\	ちち は ながい きんろう せいかつ を おくった ん です			
\\	区役所に書類を届けてください。	
\\	区役所 
\\	くやくしょ に しょるい を とどけて ください			
\\	大家	
\\	おおや			
\\	彼女はテレビのコマーシャルに出ているわ。	
\\	コマーシャル 
\\	かのじょ は てれび の こまーしゃる に でて いる わ			
\\	靴の底に釘が刺さってしまったんだ。	
\\	刺さる 
\\	くつ の そこ に くぎ が ささって しまった ん だ			
\\	祖父は市外の病院に通っているの。	
\\	市外 
\\	そふ は しがい の びょういん に かよって いる の			
\\	勝ち	
\\	かち			
\\	しつこい迷惑メールに困っている。	
\\	しつこい 
\\	しつこい めいわく めーる に こまって いる			
\\	今朝は庭の草に霜が降りていたの。	
\\	霜 
\\	けさ は にわ の くさ に しも が おりて いた の			
\\	お年寄りが道端にしゃがんでいますね。	
\\	しゃがむ 
\\	おとしより が みちばた に しゃがんで います ね			
\\	キャンセル	
\\	アジアの主食は米です。	
\\	主食 
\\	あじあ の しゅしょく は こめ です			
\\	書道をすると姿勢も良くなります。	
\\	書道 
\\	しょどう を する と しせい も よく なります			
\\	その男の行動は怪しかったわ。	
\\	怪しい 
\\	その おとこ の こうどう は あやしかった わ			
\\	香水	
\\	こうすい			
\\	そんなに慌ててどこに行くの。	
\\	慌てる 
\\	そんなに あわてて どこ に いく の			
\\	一同顔を見合わせました。	
\\	一同 
\\	いちどう かお を みあわせました			
\\	シーツ	
\\	校舎から歌声が聞こえて来たよ。	
\\	歌声 
\\	こうしゃ から うたごえ が きこえて きた よ			
\\	落ち葉の季節になりましたね。	
\\	落ち葉 
\\	おちば の きせつ に なりました ね			
\\	この帯は長過ぎます。	
\\	帯 
\\	この おび は なが すぎます			
\\	車庫	
\\	しゃこ			
\\	思い切って彼に相談します。	
\\	思い切って 
\\	おもいきって かれ に そうだん します			
\\	女の勘を甘く見てはいけません。	
\\	勘 
\\	おんな の かん を あまく みて は いけません			
\\	大会のスローガンを掲示したよ。	
\\	掲示 
\\	たいかい の すろーがん を けいじ した よ			
\\	衆議院	
\\	しゅうぎいん			
\\	彼女は毛糸のセーターを編みました。	
\\	毛糸 
\\	かのじょ は けいと の せーたー を あみました			
\\	このコンテストに優勝すると車がもらえるんだ。	
\\	コンテスト 
\\	この こんてすと に ゆうしょう する と くるま が もらえる ん だ			
\\	二人は婚約しています。	
\\	婚約 
\\	ふたり は こんやく して います			
\\	受話器	
\\	じゅわき			
\\	彼は最強チームの一員です。	
\\	最強 
\\	かれ は さいきょう ちーむ の いちいん です			
\\	合鍵の作製には2日ほどかかります。	
\\	作製 
\\	あいかぎ の さくせい に は ふつか ほど かかります			
\\	順位	
\\	じゅんい			
\\	知らない言葉を辞典で調べました。	
\\	辞典 
\\	しらない ことば を じてん で しらべました			
\\	墨	
\\	すみ			
\\	車掌さんが車内を回って来たよ。	
\\	車掌 
\\	しゃしょう さん が しゃない を まわって きた よ			
\\	お邪魔して済みません。	
\\	邪魔 
\\	おじゃま して すみません			
\\	これは日本の人名ですか。	
\\	人名 
\\	これ は にほん の じんめい です か			
\\	甘える	
\\	あまえる			
\\	ここの水族館にはイルカがいます。	
\\	水族館 
\\	ここ の すいぞくかん に は いるか が います			
\\	筋道を立てて考えなさい。	
\\	筋道 
\\	すじみち を たてて かんがえなさい			
\\	素敵なプレゼントをありがとう。	
\\	素敵 
\\	すてき な ぷれぜんと を ありがとう			
\\	生け花	
\\	いけばな			
\\	宿題を済ませてから遊びなさい。	
\\	済ます 
\\	しゅくだい を すませて から あそびなさい 。			
\\	生年月日を教えて。	
\\	生年月日 
\\	せいねんがっぴ を おしえて			
\\	彼は易しい言葉に言い換えたんだ。	
\\	言い換える 
\\	かれ は やさしい ことば に いいかえた ん だ			
\\	係	
\\	かかり			
\\	いい加減なことを言ってはいけません。	
\\	いい加減 
\\	いいかげん な こと を いって は いけません			
\\	疑わしい場所は全部調べよう。	
\\	疑わしい 
\\	うたがわしい ばしょ は ぜんぶ しらべよう			
\\	かかる	
\\	この器は上等ね。	
\\	器 
\\	この うつわ は じょうとう ね			
\\	その国の王様はとても賢い。	
\\	王様 
\\	その くに の おうさま は とても かしこい			
\\	学力	
\\	がくりょく			
\\	弟は新しい自転車に大喜びです。	
\\	大喜び 
\\	おとうと は あたらしい じてんしゃ に おおよろこび です			
\\	彼女は郊外の学園に通っているね。	
\\	学園 
\\	かのじょ は こうがい の がくえん に かよって いる ね			
\\	彼は過労のために倒れたの。	
\\	過労 
\\	かれ は かろう の ため に たおれた の			
\\	貨幣	
\\	かへい			
\\	錠を頑丈なものに替えました。	
\\	頑丈 
\\	じょう を がんじょう な もの に かえました			
\\	決まりを守ることは大切です。	
\\	決まり 
\\	きまり を まもる こと は たいせつ です			
\\	神様	
\\	かみさま			
\\	キャベツの千切りを添えたよ。	
\\	キャベツ 
\\	きゃべつ の せんぎり を そえた よ			
\\	草花を大切にしましょう。	
\\	草花 
\\	くさばな を たいせつ に しましょう			
\\	男女交互に並んでください。	
\\	交互 
\\	だんじょ こうご に ならんで ください			
\\	観衆	
\\	かんしゅう			
\\	全員、講堂に集まってください。	
\\	講堂 
\\	ぜんいん こうどう に あつまって ください			
\\	うどんは小麦粉から作られます。	
\\	小麦粉 
\\	うどん は こむぎこ から つくられます			
\\	日本の制服は紺が多いね。	
\\	紺 
\\	にほん の せいふく は こん が おおい ね			
\\	北風	
\\	きたかぜ			
\\	彼は歯科医師です。	
\\	歯科 
\\	かれ は しか いし です			
\\	日本とフランスの時差は8時間です。	
\\	時差 
\\	にほん と ふらんす の じさ は はちじかん です			
\\	牛乳は品切れだったよ。	
\\	品切れ 
\\	ぎゅうにゅう は しなぎれ だった よ			
\\	休養	
\\	きゅうよう			
\\	先月、そのコースを修了しました。	
\\	修了 
\\	せんげつ その こーす を しゅうりょう しました			
\\	パーティーのビンゴの賞品は何がいいだろう。	
\\	賞品 
\\	ぱーてぃー の びんご の しょうひん は なに が いいだろう			
\\	間違いに印を付けておきました。	
\\	印 
\\	まちがい に しるし を つけて おきました			
\\	見物人	
\\	けんぶつにん			
\\	これからも両国の親善を深めましょう。	
\\	親善 
\\	これから も りょうこく の しんぜん を ふかめましょう			
\\	彼女は若者のアイドルです。	
\\	アイドル 
\\	かのじょ は わかもの の あいどる です			
\\	コック	
\\	彼の予想は大当たりでした。	
\\	当たり 
\\	かれ の よそう は おおあたり でした			
\\	惜しい、もう少しで優勝だった。	
\\	惜しい 
\\	おしい もうすこし で ゆうしょう だった			
\\	雑談	
\\	ざつだん			
\\	この子は覚えが早いね。	
\\	覚え 
\\	この こ は おぼえ が はやい ね			
\\	彼は時々妹をからかいます。	
\\	からかう 
\\	かれ は ときどき いもうと を からかいます			
\\	寒気	
\\	さむけ			
\\	着替えを旅行カバンに詰めたよ。	
\\	着替え 
\\	きがえ を りょこう かばん に つめた よ			
\\	全部で6時間の行程です。	
\\	行程 
\\	ぜんぶ で ろくじかん の こうてい です			
\\	炊事	
\\	すいじ			
\\	小雨なので傘はいりません。	
\\	小雨 
\\	こさめ な の で かさ は いりません			
\\	彼の名前をご存じですか。	
\\	ご存じ 
\\	かれ の なまえ を ごぞんじ です か			
\\	睡眠	
\\	すいみん			
\\	彼女の心の叫びに誰も気付かなかったよ。	
\\	叫び 
\\	かのじょ の こころ の さけび に だれ も きづかなかった よ			
\\	その市の人口は減り続けているの。	
\\	市 
\\	その し の じんこう は へりつづけて いる の			
\\	彼は私費で留学したんだ。	
\\	私費 
\\	かれ は しひ で りゅうがく した ん だ			
\\	明け方	
\\	あけがた			
\\	この週刊誌を毎週買っているの。	
\\	週刊 
\\	この しゅうかんし を まいしゅう かって いる の			
\\	電車の終点で降りました。	
\\	終点 
\\	でんしゃ の しゅうてん で おりました			
\\	今朝の地震は震度3でしたよ。	
\\	震度 
\\	けさ の じしん は しんど さん でした よ			
\\	暴れる	
\\	あばれる			
\\	ここから道の幅が狭まっています。	
\\	狭まる 
\\	ここ から みち の はば が せばまって います			
\\	私たちの会社は今月は赤字よ。	
\\	赤字 
\\	わたしたち の かいしゃ は こんげつ は あかじ よ			
\\	何をそんなに焦っているのですか。	
\\	焦る 
\\	なに を そんなに あせって いる の です か			
\\	苛め	
\\	いじめ			
\\	宛先不明で手紙が戻ってきたの。	
\\	宛先 
\\	あてさき ふめい で てがみ が もどって きた の			
\\	案の定、彼は遅刻したな。	
\\	案の定 
\\	あんのじょう かれ は ちこく した な			
\\	第一志望の大学に受かりました。	
\\	受かる 
\\	だいいち しぼう の だいがく に うかりました			
\\	芋	
\\	いも			
\\	受け取りに判子をお願いします。	
\\	受け取り 
\\	うけとり に はんこ を おねがい します			
\\	彼は車を後ろ向きに駐車したの。	
\\	後ろ向き 
\\	かれ は くるま を うしろむき に ちゅうしゃ した の			
\\	大急ぎでその仕事を仕上げたよ。	
\\	大急ぎ 
\\	おおいそぎ で その しごと を しあげた よ			
\\	飲料水	
\\	いんりょうすい			
\\	弟はようやくベッドから起き上がった。	
\\	起き上がる 
\\	おとうと は ようやく ベッド から おきあがった 。			
\\	学校は生徒たちを午前中に帰したね。	
\\	帰す 
\\	がっこう は せいとたち を ごぜんちゅう に かえした ね			
\\	えび	
\\	彼には貸しがあるの。	
\\	貸し 
\\	かれ に は かし が ある の			
\\	金もうけが彼の趣味だ。	
\\	金もうけ 
\\	かねもうけ が かれ の しゅみ だ			
\\	手首の関節をひねっちゃった。	
\\	関節 
\\	てくび の かんせつ を ひねっちゃった			
\\	硬さ	
\\	かたさ			
\\	後悔しても、しょうがない。	
\\	後悔 
\\	こうかい して も しょうがない			
\\	人間は高等な生物と言われている。	
\\	高等 
\\	にんげん は こうとう な せいぶつ と いわれて いる			
\\	ロックは殊に若者に人気だ。	
\\	殊に 
\\	ろっく は ことに わかもの に にんき だ			
\\	球	
\\	きゅう			
\\	そこの酒屋さんでビールを買って来て。	
\\	酒屋 
\\	そこ の さかやさん で びーる を かって きて			
\\	彼は参議院議員です。	
\\	参議院 
\\	かれ は さんぎいん ぎいん です			
\\	姉は大学で自然科学を学んでいます。	
\\	自然科学 
\\	あね は だいがく で しぜんかがく を まなんで います			
\\	グリーン	
\\	彼は使用人を首にしたよ。	
\\	使用人 
\\	かれ は しようにん を くび に した よ			
\\	母に真珠のネックレスをもらいました。	
\\	真珠 
\\	はは に しんじゅ の ねっくれす を もらいました			
\\	彼は5人兄弟の末っ子です。	
\\	末っ子 
\\	かれ は ごにん きょうだい の すえっこ です			
\\	快い	
\\	こころよい			
\\	よく眠ったら気分がすっきりした。	
\\	すっきり 
\\	よく ねむったら きぶん が すっきり した			
\\	10問中9問正解しました。	
\\	正解 
\\	じゅうもんちゅう きゅうもん せいかい しました			
\\	もっと範囲を狭めて探しましょう	
\\	狭める 
\\	もっと はんい を せばめて さがしましょう			
\\	最上	
\\	さいじょう			
\\	彼女は夕食に天ぷらを揚げました。	
\\	揚げる 
\\	かのじょ は ゆうしょく に てんぷら を あげました			
\\	あなたは言い訳が多すぎます。	
\\	言い訳 
\\	あなた は いいわけ が おおすぎます			
\\	息子はもう異性を意識している。	
\\	異性 
\\	むすこ は もう いせい を いしき して いる			
\\	ジーパン	
\\	彼はタバコの煙を嫌がるの。	
\\	嫌がる 
\\	かれ は たばこ の けむり を いやがる の			
\\	彼を恨んではいけません。	
\\	恨む 
\\	かれ を うらんで は いけません			
\\	支度	
\\	したく			
\\	彼は永久に帰らぬ人となったのよ。	
\\	永久 
\\	かれ は えいきゅう に かえらぬ ひと と なった の よ			
\\	母は今、うちに居りません。	
\\	居る 
\\	はは は いま うち に おりません			
\\	夏休みには海水浴に行きます。	
\\	海水浴 
\\	なつやすみ に は かいすいよく に いきます			
\\	ジャーナリズム	
\\	「手が空く」は慣用句です。	
\\	慣用句 
\\	て が すく は かんようく です			
\\	この薬の効き目は素晴らしいの。	
\\	効き目 
\\	この くすり の ききめ は すばらしい の			
\\	食費	
\\	しょくひ			
\\	彼女は大変機嫌がいいね。	
\\	機嫌 
\\	かのじょ は たいへん きげん が いい ね			
\\	彼の背広に糸屑がついているわ。	
\\	屑 
\\	かれ の せびろ に いとくず が ついて いる わ			
\\	暦の上では今日から冬ですね。	
\\	暦 
\\	こよみ の うえ で は きょう から ふゆ です ね			
\\	新婚	
\\	しんこん			
\\	雨が降ってきたので傘を差しました。	
\\	差す 
\\	あめ が ふって きた の で かさ を さしました			
\\	今年も残暑が厳しかった。	
\\	残暑 
\\	ことし も ざんしょ が きびしかった			
\\	スピーチ	
\\	乗車券は無くさないように。	
\\	乗車券 
\\	じょうしゃけん は なくさない よう に			
\\	祖父は白髪が少ないほうです。	
\\	白髪 
\\	そふ は しらが が すくない ほう です			
\\	降りる駅で料金を精算して下さい。	
\\	精算 
\\	おりる えき で りょうきん を せいさん して ください			
\\	性別	
\\	せいべつ			
\\	途中であいにく雨が降り出したの。	
\\	あいにく 
\\	とちゅう で あいにく あめ が ふりだした の			
\\	彼女はお年寄りのために席を空けたんだ。	
\\	空ける 
\\	かのじょ は おとしより の ため に せき を あけた ん だ			
\\	この肉は脂が多い。	
\\	脂 
\\	この にく は あぶら が おおい			
\\	ああ	
\\	試験の前に英文を暗記したんだ。	
\\	暗記 
\\	しけん の まえ に えいぶん を あんき した ん だ			
\\	森の中にきれいな泉があるの。	
\\	泉 
\\	もり の なか に きれい な いずみ が ある の			
\\	アクセント	
\\	植木に水をやりました。	
\\	植木 
\\	うえき に みず を やりました			
\\	お話しを伺いたいのですが。	
\\	伺う 
\\	おはなし を うかがいたい の です が			
\\	梅の花が咲きました。	
\\	梅 
\\	うめ の はな が さきました			
\\	過ち	
\\	あやまち			
\\	裏口へお回りください。	
\\	裏口 
\\	うらぐち へ おまわり ください			
\\	彼は駅長です。	
\\	駅長 
\\	かれ は えきちょう です			
\\	甥と姪にお年玉をあげたの。	
\\	お年玉 
\\	おい と めい に おとしだま を あげた の			
\\	印鑑	
\\	いんかん			
\\	おばさん、こんにちは。	
\\	おばさん 
\\	おばさん、こんにちは。			
\\	お酒の飲み過ぎは健康に害があります。	
\\	害 
\\	お さけ の のみすぎ は けんこう に がい が あります 。			
\\	この池には亀がいますね。	
\\	亀 
\\	この いけ に は かめ が います ね			
\\	学説	
\\	がくせつ			
\\	彼は義理を大切にしています。	
\\	義理 
\\	かれ は ぎり を たいせつ に して います			
\\	漢字には音読みと訓読みがあります。	
\\	訓読み 
\\	かんじ に は おんよみ と くんよみ が あります			
\\	彼女は彼を軽蔑していたの。	
\\	軽蔑 
\\	かのじょ は かれ を けいべつ して いた の			
\\	片側	
\\	かたがわ			
\\	あなたが一緒にいてくれると心強い。	
\\	心強い 
\\	あなた が いっしょ に いて くれる と こころづよい			
\\	秋には木の葉が赤くなります。	
\\	木の葉 
\\	あき に は このは が あかく なります			
\\	担ぐ	
\\	かつぐ			
\\	何だか表が騒がしい。	
\\	騒がしい 
\\	なんだか おもて が さわがしい			
\\	あせらずに時機を待つべきです。	
\\	時機 
\\	あせらず に じき を まつ べき です			
\\	柄	
\\	がら			
\\	猿が高い木にジャンプした。	
\\	ジャンプ 
\\	さる が たかい き に じゃんぷ した			
\\	このバスの運賃は乗車するときに払います。	
\\	乗車 
\\	この ばす の うんちん は じょうしゃ する とき に はらいます			
\\	看病	
\\	かんびょう			
\\	昨日、生理が始まったの。	
\\	生理 
\\	きのう せいり が はじまった の			
\\	彼女は青白い顔をしているね。	
\\	青白い 
\\	かのじょ は あおじろい かお を して いる ね			
\\	気楽	
\\	きらく			
\\	彼は悪を憎んでいます。	
\\	悪 
\\	かれ は あく を にくんで います			
\\	あちらこちらで紅葉がきれいですね。	
\\	あちらこちら 
\\	あちらこちら で こうよう が きれい です ね			
\\	彼女は途中で後戻りしました。	
\\	後戻り 
\\	かのじょ は とちゅう で あともどり しました			
\\	勤労	
\\	きんろう			
\\	土地によって衣食住は変化する。	
\\	衣食住 
\\	とち によって いしょくじゅう は へんか する			
\\	彼は柔道で腰を傷めたんだ。	
\\	傷める 
\\	かれ は じゅうどう で こし を いためた ん だ			
\\	区役所	
\\	くやくしょ			
\\	いつのまにか夜が明けていたね。	
\\	いつのまにか 
\\	いつのまにか よ が あけて いた ね			
\\	今にも雨が降りそうですね。	
\\	今にも 
\\	いまにも あめ が ふりそう です ね			
\\	コマーシャル	
\\	彼女は生まれつき丈夫です。	
\\	生まれつき 
\\	かのじょ は うまれつき じょうぶ です			
\\	他人の権利を侵してはなりません。	
\\	侵す 
\\	たにん の けんり を おかしては なりません			
\\	刺さる	
\\	ささる			
\\	彼に食事をおごってもらった。	
\\	おごる 
\\	かれ に しょくじ を おごって もらった			
\\	彼女の主な仕事は接客だよ。	
\\	主 
\\	かのじょ の おも な しごと は せっきゃく だ よ			
\\	カラオケで思い切り歌ったの。	
\\	思い切り 
\\	からおけ で おもいきり うたった の			
\\	市外	
\\	しがい			
\\	今日のおやつはプリンだった。	
\\	おやつ 
\\	きょう の おやつ は ぷりん だった			
\\	今日は快晴ですね。	
\\	快晴 
\\	きょう は かいせい です ね			
\\	しつこい	
\\	庭に柿の実がなりました。	
\\	柿 
\\	にわ に かき の み が なりました			
\\	引っ越して環境が変わりました。	
\\	環境 
\\	ひっこして かんきょう が かわりました			
\\	英語の聞き取り試験を受けたよ。	
\\	聞き取り 
\\	えいご の ききとり しけん を うけた よ			
\\	霜	
\\	しも			
\\	貴重品は自分で持っていてください。	
\\	貴重品 
\\	きちょうひん は じぶん で もって いて ください			
\\	この服は窮屈になったな。	
\\	窮屈 
\\	この ふく は きゅうくつ に なった な			
\\	しゃがむ	
\\	休息をとることは大切です。	
\\	休息 
\\	きゅうそく を とる こと は たいせつ です			
\\	安全運転を心掛けてください。	
\\	心掛ける 
\\	あんぜん うんてん を こころがけて ください			
\\	主食	
\\	しゅしょく			
\\	夜道の一人歩きは心細いね。	
\\	心細い 
\\	よみち の ひとりあるき は こころぼそい ね			
\\	ごろごろ	
\\	雷がごろごろ鳴っている。	
\\	ごろごろ 
\\	かみなり が ごろごろ なって いる			
\\	この三角形の面積を出しなさい。	
\\	三角形 
\\	この さんかくけい の めんせき を だしなさい			
\\	このドレスはとても上品なデザインですね。	
\\	上品 
\\	この どれす は とても じょうひん な でざいん です ね			
\\	これは初級の教科書です。	
\\	初級 
\\	これ は しょきゅう の きょうかしょ です 。			
\\	今日から新学期が始まります。	
\\	新学期 
\\	きょう から しんがっき が はじまります			
\\	シューズ	
\\	食べ物の好き嫌いは特にありません。	
\\	好き嫌い 
\\	たべもの の すききらい は とくに ありません			
\\	湖の水は透き通っていたよ。	
\\	透き通る 
\\	みずうみ の みず は すきとおって いた よ			
\\	上品	
\\	じょうひん			
\\	正方形の紙を用意しましょう。	
\\	正方形 
\\	せいほうけい の かみ を ようい しましょう			
\\	西洋人は正座が苦手よ。	
\\	西洋人 
\\	せいようじん は せいざ が にがて よ			
\\	初級	
\\	しょきゅう			
\\	彼は足首を痛めています。	
\\	足首 
\\	かれ は あしくび を いためて います			
\\	スープを少し薄めましょうか。	
\\	薄める 
\\	すーぷ を すこし うすめましょう か			
\\	新学期	
\\	しんがっき			
\\	この傘は柄が丈夫だな。	
\\	柄 
\\	この かさ は え が じょうぶ だ な			
\\	シャツを裏返しに着ているよ。	
\\	裏返し 
\\	しゃつ を うらがえし に きて いる よ			
\\	何かお手伝いしましょうか。	
\\	お手伝い 
\\	なに か おてつだい しましょう か			
\\	好き嫌い	
\\	すききらい			
\\	初めてのキスは海岸でだったの。	
\\	キス 
\\	はじめて の きす は かいがん で だった の			
\\	旧ソビエトは今はロシアと呼ばれている。	
\\	旧 
\\	きゅうそびえと は いま は ろしあ と よばれて いる			
\\	石が粉々に砕けましたね。	
\\	砕ける 
\\	いし が こなごな に くだけました ね			
\\	透き通る	
\\	すきとおる			
\\	試合に負けてとても悔しい。	
\\	悔しい 
\\	しあい に まけて とても くやしい			
\\	猫は私の足元をくるくると回ったの。	
\\	くるくる 
\\	ねこ は わたし の あしもと を くるくる と まわった の			
\\	彼女はクモを怖がります。	
\\	怖がる 
\\	かのじょ は くも を こわがります			
\\	正方形	
\\	せいほうけい			
\\	彼女に本を返すように催促したの。	
\\	催促 
\\	かのじょ に ほん を かえす よう に さいそく した の			
\\	また仕事をサボっていますね。	
\\	サボる 
\\	また しごと を さぼって います ね			
\\	西洋人	
\\	せいようじん			
\\	台風が去って風が静まったね。	
\\	静まる 
\\	たいふう が さって かぜ が しずまった ね			
\\	新しい事務員が入りました。	
\\	事務員 
\\	あたらしい じむいん が はいりました			
\\	上等なワインを飲んだの。	
\\	上等 
\\	じょうとう な わいん を のんだ の			
\\	足首	
\\	あしくび			
\\	彼女のいちばんの楽しみはショッピングです。	
\\	ショッピング 
\\	かのじょ の いちばん の たのしみ は しょっぴんぐ です			
\\	彼女の家は神道です。	
\\	神道 
\\	かのじょ の いえ は しんとう です			
\\	今はほとんどのトイレが水洗ですよ。	
\\	水洗 
\\	いま は ほとんど の といれ が すいせん です よ			
\\	薄める	
\\	うすめる			
\\	彼はいつもスマートに行動するね。	
\\	スマート 
\\	かれ は いつも すまーと に こうどう する ね			
\\	玄関に靴がずらりと並んでいたな。	
\\	ずらり 
\\	げんかん に くつ が ずらり と ならんで いた な			
\\	今夜は天の川が見えますね。	
\\	天の川 
\\	こんや は あまのがわ が みえます ね			
\\	柄	
\\	え			
\\	彼は金に卑しいね。	
\\	卑しい 
\\	かれ は かね に いやしい ね			
\\	お客様がいらっしゃいました。	
\\	いらっしゃる 
\\	おきゃくさま が いらっしゃいました			
\\	ウナギの蒲焼きは美味しいね。	
\\	ウナギ 
\\	うなぎ の かばやき は おいしい ね			
\\	裏返し	
\\	うらがえし			
\\	仏像に手を合わせて拝みました。	
\\	拝む 
\\	ぶつぞう に て を あわせて おがみました			
\\	歩道に新しくガードが付きましたね。	
\\	ガード 
\\	ほどう に あたらしく がーど が つきました ね			
\\	たまには外食しましょう。	
\\	外食 
\\	たま に は がいしょく しましょう			
\\	お手伝い	
\\	おてつだい			
\\	大雨で崖が崩れたんだ。	
\\	崖 
\\	おおあめ で がけ が くずれた ん だ			
\\	かぼちゃのスープは美味しいです。	
\\	かぼちゃ 
\\	かぼちゃの すーぷ は おいしい です			
\\	キス	
\\	僕の財布は空っぽだよ。	
\\	空っぽ 
\\	ぼく の さいふ は からっぽ だ よ			
\\	彼女はいつも聞き手にまわるね。	
\\	聞き手 
\\	かのじょ は いつも ききて に まわる ね			
\\	彼のマジックはまるで奇跡です。	
\\	奇跡 
\\	かれ の まじっく は まるで きせき です			
\\	旧	
\\	きゅう			
\\	この本を君にあげます。	
\\	君 
\\	この ほん を きみ に あげます			
\\	彼は軽い近視です。	
\\	近視 
\\	かれ は かるい きんし です			
\\	砕ける	
\\	くだける			
\\	クイズ番組を見るのが好きです。	
\\	クイズ 
\\	くいず ばんぐみ を みる の が すき です			
\\	私は洗い物をしたあと、手にクリームをぬります。	
\\	クリーム 
\\	わたし は あらいもの を した あと て に くりーむ を ぬります			
\\	犬が自分のしっぽを追いかけてぐるぐる回っている。	
\\	ぐるぐる 
\\	いぬ が じぶん の しっぽ を おいかけて ぐるぐる まわって いる			
\\	悔しい	
\\	くやしい			
\\	漢字の読み方には、音と訓の2通りがあるの。	
\\	訓 
\\	かんじ の よみかた に は おん と くん の ふたとおり が ある の			
\\	この雑誌は月刊ですか。	
\\	月刊 
\\	この ざっし は げっかん です か			
\\	この暑さは高気圧のせいです。	
\\	高気圧 
\\	この あつさ は こうきあつ の せい です			
\\	くるくる	
\\	こちらを差し上げます。	
\\	差し上げる 
\\	こちら を さしあげます			
\\	それはさぞがっかりしたことでしょう。	
\\	さぞ 
\\	それはさぞがっかりしたことでしょう。			
\\	彼はサングラスが似合うね。	
\\	サングラス 
\\	かれ は さんぐらす が にあう ね			
\\	怖がる	
\\	こわがる			
\\	その部屋は始終、人が出入りしているね。	
\\	始終 
\\	その へや は しじゅう ひと が でいり して いる ね			
\\	消火器を交換しました。	
\\	消火器 
\\	しょうかき を こうかん しました			
\\	催促	
\\	さいそく			
\\	子供を小児科に連れて行くところです。	
\\	小児科 
\\	こども を しょうにか に つれて いく ところ です			
\\	大学で人文科学を専攻しました。	
\\	人文科学 
\\	だいがく で じんぶんかがく を せんこう しました			
\\	おしゃべりは後回しにしましょう。	
\\	後回し 
\\	おしゃべり は あとまわし に しましょう			
\\	サボる	
\\	彼は威張ってなんかいません。	
\\	威張る 
\\	かれ は いばって なんか いません			
\\	大麦はビールの原料になります。	
\\	大麦 
\\	おおむぎ は びーる の げんりょう に なります			
\\	静まる	
\\	しずまる			
\\	脅かさないでよ。	
\\	脅かす 
\\	おどかさない で よ			
\\	お前の言うことは信じられない。	
\\	お前 
\\	おまえ の いう こと は しんじられ ない			
\\	事務員	
\\	じむいん			
\\	この漢字の音読みは何ですか。	
\\	音読み 
\\	この かんじ の おんよみ は なん です か			
\\	インターネットで試験の解答を確認したんだ。	
\\	解答 
\\	いんたーねっと で しけん の かいとう を かくにん した ん だ			
\\	上等	
\\	じょうとう			
\\	花瓶にお花を生けました。	
\\	花瓶 
\\	かびん に おはな を いけました			
\\	彼女は娘に可愛らしい服を作りましたね。	
\\	可愛らしい 
\\	かのじょ は むすめ に かわいらしい ふく を つくりました ね			
\\	彼女はきっちり3時に来たわ。	
\\	きっちり 
\\	かのじょ は きっちり さんじ に きた わ			
\\	ショッピング	
\\	1年休学することにしました。	
\\	休学 
\\	いちねん きゅうがく する こと に しました			
\\	きゅうりに味噌をつけて食べたの。	
\\	きゅうり 
\\	きゅうり に みそ を つけて たべた の			
\\	インドでは英語は共通語だよ。	
\\	共通語 
\\	いんど で は えいご は きょうつうご だ よ			
\\	神道	
\\	しんとう			
\\	この庭は草木が枯れているね。	
\\	草木 
\\	この にわ は くさき が かれて いる ね			
\\	外は木枯らしが吹いているよ。	
\\	木枯らし 
\\	そと は こがらし が ふいて いる よ			
\\	私はベンチに腰掛けたの。	
\\	腰掛ける 
\\	わたし は べんち に こしかけた の			
\\	水洗	
\\	すいせん			
\\	旅先で思いがけない災難にあいました。	
\\	災難 
\\	たびさき で おもいがけない さいなん に あいました			
\\	絵を逆さにしてみてごらん。	
\\	逆さ 
\\	え を さかさ に して みて ごらん			
\\	スマート	
\\	列車の窓から雄大な山脈が見えたんだよ。	
\\	山脈 
\\	れっしゃ の まど から ゆうだい な さんみゃく が みえた ん だ よ			
\\	始発の電車に乗りました。	
\\	始発 
\\	しはつ の でんしゃ に のりました			
\\	ずらり	
\\	レバーを右に回すと締まります。	
\\	締まる 
\\	ればー を みぎ に まわす と しまります			
\\	うちの会社は週休2日です。	
\\	週休 
\\	うち の かいしゃ は しゅうきゅう ふつか です			
\\	天の川	
\\	あまのがわ			
\\	言いたいことを言ったら胸がすっとした。	
\\	すっと 
\\	いいたい こと を いったら むね が すっと した			
\\	その小学生は難しい本をすらすら読んだの。	
\\	すらすら 
\\	その しょうがくせい は むずかしい ほん を すらすら よんだ の			
\\	警察はそのグループを怪しんでいます。	
\\	怪しむ 
\\	けいさつ は その ぐるーぷ を あやしんで います			
\\	卑しい	
\\	いやしい			
\\	庭に穴を掘ってそれを埋めました。	
\\	埋める 
\\	にわ に あな を ほって それ を うめました			
\\	両親を敬うことは大切です。	
\\	敬う 
\\	りょうしん を うやまう こと は たいせつ です			
\\	その男はロビーをしばらくうろうろしていたよ。	
\\	うろうろ 
\\	その おとこ は ろびー を しばらく うろうろ して いた よ			
\\	いらっしゃる	
\\	彼はお世辞を言うのが上手い。	
\\	お世辞 
\\	かれ は おせじ を いう の が うまい			
\\	家族でお寺にお参りに行きました。	
\\	お参り 
\\	かぞく で おてら に おまいり に いきました			
\\	八百屋さんでたくさんおまけしてもらったよ。	
\\	おまけ 
\\	やおや さん で たくさん おまけ して もらった よ			
\\	ウナギ	
\\	この鞄は重たいです。	
\\	重たい 
\\	この かばん は おもたい です			
\\	改札で3時に会おう。	
\\	改札 
\\	かいさつ で さんじ に あおう			
\\	拝む	
\\	おがむ			
\\	ここで2つの海流が出合っている。	
\\	海流 
\\	ここ で ふたつ の かいりゅう が であって いる			
\\	背中をお母さんに掻いてもらったの。	
\\	掻く 
\\	せなか を おかあさん に かいて もらった の			
\\	ガード	
\\	眼科で視力検査をしました。	
\\	眼科 
\\	がんか で しりょく けんさ を しました			
\\	待ち合わせは2時だと勘違いしていました。	
\\	勘違い 
\\	まちあわせ は にじ だ と かんちがい して いました			
\\	外食	
\\	がいしょく			
\\	昨夜はテレビで喜劇を見たよ。	
\\	喜劇 
\\	さくや は てれび で きげき を みた よ			
\\	この建物は建築の基準に達していないよ。	
\\	基準 
\\	この たてもの は けんちく の きじゅん に たっしていない よ 。			
\\	この子は本当に行儀の良い子です。	
\\	行儀 
\\	この こ は ほんとう に ぎょうぎ の いい こ です			
\\	崖	
\\	がけ			
\\	この包丁は切れが良いな。	
\\	切れ 
\\	この ほうちょう は きれ が いい な			
\\	クッキーと紅茶をいただきました。	
\\	クッキー 
\\	くっきー と こうちゃ を いただきました			
\\	プラグをコンセントに挿し込んだよ。	
\\	コンセント 
\\	ぷらぐ を こんせんと に さしこんだ よ			
\\	かぼちゃ	
\\	この参考書はとても役に立つよ。	
\\	参考書 
\\	この さんこうしょ は とても やく に たつ よ			
\\	終電にやっと間に合った。	
\\	終電 
\\	しゅうでん に やっと まにあった			
\\	この先は徐行して下さい。	
\\	徐行 
\\	この さき は じょこう して ください			
\\	空っぽ	
\\	からっぽ			
\\	朝は雀の声で目が覚めます。	
\\	雀 
\\	あさ は すずめ の こえ で め が さめます			
\\	あなたのやり方はずるい。	
\\	ずるい 
\\	あなた の やりかた は ずるい			
\\	聞き手	
\\	ききて			
\\	彼はせっかちで困ります。	
\\	せっかち 
\\	かれ は せっかち で こまります			
\\	秋風が気持ちいいね。	
\\	秋風 
\\	あきかぜ が きもち いい ね			
\\	奇跡	
\\	きせき			
\\	彼の助けは本当に有り難いな。	
\\	有り難い 
\\	かれ の たすけ は ほんとう に ありがたい な			
\\	彼は胃腸が強くありません。	
\\	胃腸 
\\	かれ は いちょう が つよく ありません			
\\	彼はいつも受け身の姿勢で、自分からは何もしないんだ。	
\\	受け身 
\\	かれ は いつも うけみ の しせい で じぶん からは なに も しない ん だ			
\\	君	
\\	きみ			
\\	父はうちわで扇いでいるの。	
\\	うちわ 
\\	ちち は うちわ で あおいで いる の			
\\	大事なメールをうっかり消してしまった。	
\\	うっかり 
\\	だいじ な めーる を うっかり けして しまった			
\\	バスの回数券を買いました。	
\\	回数券 
\\	ばす の かいすうけん を かいました			
\\	近視	
\\	きんし			
\\	火力を弱めないと肉がこげるよ。	
\\	火力 
\\	かりょく を よわめない と にく が こげる よ			
\\	知り合いが急病で倒れた。	
\\	急病 
\\	しりあい が きゅうびょう で たおれた			
\\	クイズ	
\\	くだらないおしゃべりはやめなさい。	
\\	くだらない 
\\	くだらないおしゃべりはやめなさい。			
\\	赤い口紅を買いました。	
\\	口紅 
\\	あかい くちべに を かいました			
\\	クリーム	
\\	軽率な発言はしないよう気をつけなさい。	
\\	軽率 
\\	けいそつ な はつげん は しない よう き を つけなさい			
\\	そんなに謙遜しなくてもいい。	
\\	謙遜 
\\	そんな に けんそん しなくて も いい			
\\	秋は行楽にいい季節ですね。	
\\	行楽 
\\	あき は こうらく に いい きせつ です ね			
\\	ぐるぐる	
\\	この仕事は時給1000円です。	
\\	時給 
\\	この しごと は じきゅう せんえん です			
\\	十字路で車とバイクが衝突したぞ。	
\\	十字路 
\\	じゅうじろ で くるま と ばいく が しょうとつ した ぞ			
\\	私用で出かけなければなりません。	
\\	私用 
\\	しよう で でかけなければ なりません			
\\	訓	
\\	くん			
\\	英語を初歩から勉強しています。	
\\	初歩 
\\	えいご を しょほ から べんきょう して います			
\\	お酢を入れすぎて酸っぱい。	
\\	酢 
\\	お す を いれすぎて すっぱい			
\\	月刊	
\\	げっかん			
\\	この島は発電を水力に頼っているんだ。	
\\	水力 
\\	この しま は はつでん を すいりょく に たよって いる ん だ			
\\	何てずうずうしい人なんだろう。	
\\	ずうずうしい 
\\	なんて ずうずうしい ひと なん だろう			
\\	高気圧	
\\	こうきあつ			
\\	子供がすやすや眠っているね。	
\\	すやすや 
\\	こども が すやすや ねむって いる ね			
\\	この食堂はセルフサービスです。	
\\	セルフサービス 
\\	この しょくどう は せるふさーびす です			
\\	差し上げる	
\\	さしあげる			
\\	この子はアルファベットを全部言えます。	
\\	アルファベット 
\\	この こ は あるふぁべっと を ぜんぶ いえます			
\\	彼女は負けずに言い返したの。	
\\	言い返す 
\\	かのじょ は まけず に いいかえした の			
\\	父のいびきは大きいんだ。	
\\	いびき 
\\	ちち の いびき は おおきい ん だ			
\\	さぞ	
\\	彼はいやらしい男ね。	
\\	いやらしい 
\\	かれ は いやらしい おとこ ね			
\\	イントネーションの違いに気を付けて。	
\\	イントネーション 
\\	いんとねーしょん の ちがい に き を つけて			
\\	サングラス	
\\	来週の予定を打ち合わせましょう。	
\\	打ち合わせる 
\\	らいしゅう の よてい を うちあわせましょう			
\\	試合は雨天中止です。	
\\	雨天 
\\	しあい は うてん ちゅうし です			
\\	始終	
\\	しじゅう			
\\	社長の奥様はきれいな方です。	
\\	奥様 
\\	しゃちょう の おくさま は きれい な かた です			
\\	駅まで駆け足で行ったよ。	
\\	駆け足 
\\	えき まで かけあし で いった よ			
\\	消火器	
\\	しょうかき			
\\	この椅子はがたがたしているね。	
\\	がたがた 
\\	この いす は がたがた して いる ね			
\\	この鉢植えは観賞用です。	
\\	観賞 
\\	この はちうえ は かんしょうよう です			
\\	小児科	
\\	しょうにか			
\\	カンニングをした人は0点です。	
\\	カンニング 
\\	かんにんぐ を した ひと は れいてん です			
\\	3は奇数です。	
\\	奇数 
\\	さん は きすう です			
\\	彼は禁酒しています。	
\\	禁酒 
\\	かれ は きんしゅ して います			
\\	人文科学	
\\	じんぶんかがく			
\\	偶数は2で割り切れます。	
\\	偶数 
\\	ぐうすう は に で わりきれます			
\\	事故現場には入れません。	
\\	現場 
\\	じこ げんば に は はいれません			
\\	後回し	
\\	あとまわし			
\\	新しい国立劇場が完成しました。	
\\	国立 
\\	あたらしい こくりつ げきじょう が かんせい しました			
\\	シチューが焦げたよー。	
\\	焦げる 
\\	しちゅー が こげた よー			
\\	この座布団は座り心地がいいね。	
\\	座布団 
\\	この ざぶとん は すわり ごこち が いい ね			
\\	威張る	
\\	いばる			
\\	彼は最近、失恋したらしいの。	
\\	失恋 
\\	かれ は さいきん しつれん した らしい の			
\\	雨の日が続いて家の中が湿っぽいな。	
\\	湿っぽい 
\\	あめ の ひ が つづいて いえ の なか が しめっぽい な			
\\	主観だけで物事を見てはいけない。	
\\	主観 
\\	主観的 
\\	しゅかん だけ で ものごと を みて は いけない			
\\	大麦	
\\	おおむぎ			
\\	それは主観的な意見だ。	
\\	主観的 
\\	それ は しゅかんてき な いけん だ			
\\	小数は切り捨てて計算して下さい。	
\\	小数 
\\	しょうすう は きりすてて けいさん して ください			
\\	脅かす	
\\	おどかす			
\\	水平線に太陽が沈んでいった。	
\\	水平線 
\\	すいへいせん に たいよう が しずんで いった			
\\	彼はあぐらをかいてしゃべっていますね。	
\\	あぐら 
\\	かれ は あぐら を かいて しゃべって います ね			
\\	紫の朝顔が咲きました。	
\\	朝顔 
\\	むらさき の あさがお が さきました			
\\	お前	
\\	おまえ			
\\	彼は時々意地悪な質問をする。	
\\	意地悪 
\\	かれ は ときどき いじわる な しつもん を する 。			
\\	コンビニは至る所にあります。	
\\	至る所 
\\	こんびに は いたるところ に あります			
\\	あなたも今に分かるでしょう。	
\\	今に 
\\	あなた も いまに わかる でしょう			
\\	音読み	
\\	おんよみ			
\\	その子はいやいや学校へ行ったの。	
\\	いやいや 
\\	その こ は いやいや がっこう へ いった の			
\\	最近ウエストが太くなったな。	
\\	ウエスト 
\\	さいきん うえすと が ふとく なった な			
\\	解答	
\\	かいとう			
\\	彼は運送会社に勤めているの。	
\\	運送 
\\	かれ は うんそう がいしゃ に つとめて いる の			
\\	襟の大きいコートを買いました。	
\\	襟 
\\	えり の おおきい こーと を かいました			
\\	咳が少し治まりました。	
\\	治まる 
\\	せき が すこし おさまりました			
\\	花瓶	
\\	かびん			
\\	僕たちはオルガンに合わせて賛美歌を歌ったんだ。	
\\	オルガン 
\\	ぼくたち は おるがん に あわせて さんびか を うたった んだ			
\\	日本は温帯にあります。	
\\	温帯 
\\	にっぽん は おんたい に あります			
\\	この地方は温暖で暮らしやすいな。	
\\	温暖 
\\	この ちほう は おんだん で くらし やすい な			
\\	可愛らしい	
\\	かわいらしい			
\\	書き言葉と話し言葉はだいぶ違うことがあります。	
\\	書き言葉 
\\	かきことば と はなしことば は だいぶ ちがう こと が あります			
\\	陰で少し休憩しましょう。	
\\	陰 
\\	かげ で すこし きゅうけい しましょう			
\\	祖父は古い仮名遣いで書くことがあるの。	
\\	仮名遣い 
\\	そふ は ふるい かなづかい で かく こと が ある の			
\\	きっちり	
\\	部長は僕の仕事に辛い評価を出したんだ。	
\\	辛い 
\\	ぶちょう は ぼく の しごと に からい ひょうか を だした ん だ			
\\	彼はからからと笑いました。	
\\	からから 
\\	かれ は からから と わらいました			
\\	お店のガラス戸が割られたんだ。	
\\	ガラス戸 
\\	おみせ の がらすど が わられた ん だ			
\\	休学	
\\	きゅうがく			
\\	体育館で器械を使って運動したの。	
\\	器械 
\\	たいいくかん で きかい を つかって うんどう した の			
\\	彼の体型じゃ既製のサイズに合わないよ。	
\\	既製 
\\	かれ の たいけい じゃ きせい の さいず に あわない よ			
\\	きゅうり	
\\	彼女は驚いて気絶してしまったの。	
\\	気絶 
\\	かのじょ は おどろいて きぜつ して しまった の			
\\	彼女はキッチンで夕食を作っています。	
\\	きっちん 
\\	かのじょ は きっちん で ゆうしょく を つくって います			
\\	お客さんを客間にお通ししたわよ。	
\\	客間 
\\	おきゃくさん を きゃくま に おとおし した わ よ			
\\	共通語	
\\	きょうつうご			
\\	彼は勤勉な人です。	
\\	勤勉 
\\	かれ は きんべん な ひと です			
\\	この服は上下セットで買いました。	
\\	上下 
\\	この ふく は じょうげ せっと で かいました			
\\	草木	
\\	くさき			
\\	犬を鎖でつなぎました。	
\\	鎖 
\\	いぬ を くさり で つなぎました			
\\	紙をくしゃくしゃに丸めました。	
\\	くしゃくしゃ 
\\	かみ を くしゃくしゃ に まるめました			
\\	木枯らし	
\\	こがらし			
\\	きのう、弟と口げんかしました。	
\\	口げんか 
\\	きのう おとうと と くちげんか しました			
\\	景色が良かったので途中下車したよ。	
\\	下車 
\\	けしき が よかった の で とちゅう げしゃ した よ			
\\	彼は金持ちだけど、けちね。	
\\	けち 
\\	かれ は かねもち だ けど けち ね			
\\	腰掛ける	
\\	こしかける			
\\	ここには現住所を書いてください。	
\\	現住所 
\\	ここ に は げんじゅうしょ を かいて ください			
\\	恋する気持ちを歌にしました。	
\\	恋する 
\\	こいする きもち を うた に しました			
\\	災難	
\\	さいなん			
\\	まず、肉をパン粉の上で転がします。	
\\	転がす 
\\	まず にく を ぱんこ の うえ で ころがします			
\\	デパートの歳末大売出しが始まったよ。	
\\	歳末 
\\	でぱーと の さいまつ おおうりだし が はじまった よ			
\\	釘に引っかけて袖が裂けてしまったな。	
\\	裂ける 
\\	くぎ に ひっかけて そで が さけて しまった な			
\\	逆さ	
\\	さかさ			
\\	砂ぼこりで机がざらざらしているね。	
\\	ざらざら 
\\	すなぼこり で つくえ が ざらざら して いる ね			
\\	彼女はソファーに体を沈めたんだ。	
\\	沈める 
\\	かのじょ は そふぁ に からだ を しずめた ん だ			
\\	山脈	
\\	さんみゃく			
\\	今、論文の下書きをしているところです。	
\\	下書き 
\\	いま ろんぶん の したがき を して いる ところ です			
\\	うちは3人姉妹です。	
\\	姉妹 
\\	うち は さんにん しまい です			
\\	この本の第6章が特に好きです。	
\\	章 
\\	この ほん の だいろくしょう が とくに すき です			
\\	始発	
\\	しはつ			
\\	ちょっと小便しに行って来る。	
\\	小便 
\\	ちょっと しょうべん し に いって くる			
\\	都会の空はスモッグで灰色ね。	
\\	スモッグ 
\\	とかい の そら は すもっぐ で はいいろ ね			
\\	すりに財布をとられた。	
\\	すり 
\\	すり に さいふ を とられた			
\\	締まる	
\\	しまる			
\\	奴は裏門から出てきたぜ。	
\\	裏門 
\\	やつ は うらもん から でて きた ぜ			
\\	このレストランは衛生的で安心です。	
\\	衛生的 
\\	この れすとらん は えいせいてき で あんしん です			
\\	週休	
\\	しゅうきゅう			
\\	パトカーがバイクを追いかけてるぞ。	
\\	追い掛ける 
\\	ぱとかー が ばいく を おいかけて る ぞ			
\\	駅でようやく彼に追い付きました。	
\\	追い付く 
\\	えき で ようやく かれ に おいつきました			
\\	すっと	
\\	彼は男らしさを心掛けているね。	
\\	男らしい 
\\	かれ は おとこらしさ を こころがけて いる ね			
\\	昼食におにぎりを食べました。	
\\	おにぎり 
\\	ちゅうしょく に おにぎり を たべました			
\\	すらすら	
\\	お寺からお坊さんが出て来ましたよ。	
\\	お坊さん 
\\	おてら から おぼうさん が でて きました よ			
\\	このご恩は一生忘れません。	
\\	恩 
\\	この ごおん は いっしょう わすれません			
\\	怪しむ	
\\	あやしむ			
\\	彼女は女らしい。	
\\	女らしい 
\\	かのじょ は おんならしい			
\\	彼は恐い顔付きで話したの。	
\\	顔付き 
\\	かれ は こわい かおつき で はなした の			
\\	子供に構い過ぎてはいけない。	
\\	構う 
\\	こども に かまいすぎて は いけない			
\\	埋める	
\\	うめる			
\\	かみそりの刃で指を切りました。	
\\	かみそり 
\\	かみそり の は で ゆび を きりました			
\\	生徒たちががやがやと騒いでいますね。	
\\	がやがや 
\\	せいとたち が がやがや と さわいで います ね			
\\	この借りは必ず返します。	
\\	借り 
\\	この かり は かならず かえします			
\\	敬う	
\\	うやまう			
\\	壁に寒暖計が掛かっていました。	
\\	寒暖計 
\\	かべ に かんだんけい が かかって いました			
\\	漢和辞典で漢字の意味を調べたの。	
\\	漢和 
\\	かんわ じてん で かんじ の いみ を しらべた の			
\\	彼は急用で帰りました。	
\\	急用 
\\	かれ は きゅうよう で かえりました			
\\	うろうろ	
\\	会場で子供がきょろきょろしているわよ。	
\\	きょろきょろ 
\\	かいじょう で こども が きょろきょろ して いる わ よ			
\\	風邪でくしゃみが止まりません。	
\\	くしゃみ 
\\	かぜ で くしゃみ が とまりません			
\\	ぐずぐずしていると電車に乗り遅れますよ。	
\\	ぐずぐず 
\\	ぐずぐず して いる と でんしゃ に のりおくれます よ			
\\	お世辞	
\\	おせじ			
\\	犬に顔をなめられてくすぐったいよ。	
\\	くすぐったい 
\\	いぬ に かお を なめられて くすぐったい よ 。			
\\	たくさん歩いてくたびれました。	
\\	くたびれる 
\\	たくさん あるいて くたびれました			
\\	彼は獣のような目をしていたな。	
\\	獣 
\\	かれ は けもの の よう な め を して いた な			
\\	お参り	
\\	おまいり			
\\	腰掛けはありませんか。	
\\	腰掛け 
\\	こしかけ は ありません か			
\\	小指を切ってしまいました。	
\\	小指 
\\	こゆび を きって しまいました			
\\	お茶がとても熱かったので冷ましてから飲んだの。	
\\	冷ます 
\\	おちゃ が とても あつかった の で さまして から のんだ の			
\\	おまけ	
\\	おまけ			
\\	彼は三流大学を出たが、今は社長だよ。	
\\	三流 
\\	かれ は さんりゅう だいがく を でた が いま は しゃちょう だ よ			
\\	社会科学の分野に関心があります。	
\\	社会科学 
\\	しゃかいかがく の ぶんや に かんしん が あります			
\\	重たい	
\\	おもたい			
\\	それは少女向けの雑誌です。	
\\	少女 
\\	それ は しょうじょむけ の ざっし です			
\\	朝食に食パンを2枚食べました。	
\\	食パン 
\\	ちょうしょく に しょくぱん を にまい たべました			
\\	植物園には珍しい花がたくさんありますね。	
\\	植物園 
\\	しょくぶつえん に は めずらしい はな が たくさん あります ね			
\\	改札	
\\	かいさつ			
\\	成年になると独立した戸籍を作れます。	
\\	成年 
\\	せいねん に なる と どくりつ した こせき を つくれます			
\\	後書きをよく読んで下さい。	
\\	後書き 
\\	あとがき を よく よんで ください			
\\	海流	
\\	かいりゅう			
\\	子供を甘やかしてはいけないの。	
\\	甘やかす 
\\	こども を あまやかしては いけない の			
\\	母は台所で洗い物をしています。	
\\	洗い物 
\\	はは は だいどころ で あらいもの を して います			
\\	掻く	
\\	かく			
\\	幾ら呼んでも彼は返事をしなかったわ。	
\\	幾ら 
\\	いくら よんで も かれ は へんじ を しなかった わ			
\\	握り寿司の上を注文したよ。	
\\	上 
\\	にぎりずし の じょう を ちゅうもん した よ			
\\	眼科	
\\	がんか			
\\	一昨日、お電話を差し上げました。	
\\	差し上げる 
\\	一昨日 
\\	いっさくじつ おでんわ を さしあげました			
\\	今更後悔しても、もう遅いよ。	
\\	今更 
\\	いまさら こうかい して も もう おそい よ			
\\	勘違い	
\\	かんちがい			
\\	彼女は朝からいらいらしている。	
\\	いらいら 
\\	かのじょ は あさ から いらいら して いる			
\\	子供たちは冬でも薄着で通します。	
\\	薄着 
\\	こどもたち は ふゆ で も うすぎ で とおします			
\\	うちでは、朝食には必ず梅干しが出ます。	
\\	梅干 
\\	うち で は ちょうしょく に は かならず うめぼし が でます			
\\	喜劇	
\\	きげき			
\\	エチケットを守ることは大切です。	
\\	エチケット 
\\	えちけっと を まもる こと は たいせつ です			
\\	私の甥は3才です。	
\\	甥 
\\	わたし の おい は さんさい です			
\\	基準	
\\	きじゅん			
\\	彼は名案を思い付いたの。	
\\	思い付く 
\\	かれ は めいあん を おもいついた の			
\\	犬がお皿の匂いをクンクン嗅いでいるね。	
\\	嗅ぐ 
\\	いぬ が おさら の におい を くんくん かいで いる ね			
\\	行儀	
\\	ぎょうぎ			
\\	彼は仕事に命を懸けているの。	
\\	懸ける 
\\	かれ は しごと に いのち を かけている の 。			
\\	括弧の部分は省略できます。	
\\	括弧 
\\	かっこ の ぶぶん は しょうりゃく できます			
\\	かっとなって余計なことを言ってしまったな。	
\\	かっと 
\\	かっと なって よけい な こと を いって しまった な			
\\	切れ	
\\	きれ			
\\	紙屑は屑入れに入れなさい。	
\\	紙屑 
\\	かみくず は くず いれ に いれなさい 。			
\\	がらがら	
\\	誰かがガラガラとうがいをしているね。	
\\	がらがら 
\\	だれか が がらがら と うがい を して いる ね			
\\	彼女はかんかんに怒っています。	
\\	かんかん 
\\	かのじょ は かんかん に おこって います			
\\	かんかん	
\\	彼女は近眼です。	
\\	近眼 
\\	かのじょ は きんがん です			
\\	彼女は薬指に指輪をはめています。	
\\	薬指 
\\	かのじょ は くすりゆび に ゆびわ を はめて います			
\\	近眼	
\\	きんがん			
\\	彼女の下品なふるまいには我慢できない。	
\\	下品 
\\	かのじょ の げひん な ふるまい に は がまん できない			
\\	これまでの私とは違うんです。	
\\	これまで 
\\	これまで の わたし と は ちがうん です			
\\	薬指	
\\	くすりゆび			
\\	夕食の最中に電話がかかってきたの。	
\\	最中 
\\	ゆうしょく の さいちゅう に でんわ が かかって きた の			
\\	会社から「至急」との連絡があったんだ。	
\\	至急 
\\	かいしゃ から しきゅう と の れんらく が あった ん だ			
\\	まず第一に、しっかり下調べをしなさい。	
\\	下調べ 
\\	まず だいいち に しっかり したしらべ を しなさい			
\\	下品	
\\	げひん			
\\	梅雨時はじめじめする。	
\\	じめじめ 
\\	つゆどき は じめじめ する			
\\	先生に暑中見舞いを出しました。	
\\	暑中見舞い 
\\	せんせい に しょちゅうみまい を だしました			
\\	これまで	
\\	人をじろじろ見るものではありません。	
\\	じろじろ 
\\	ひと を じろじろ みる もの で は ありません			
\\	あなたの姓名を教えてください。	
\\	姓名 
\\	あなた の せいめい を おしえて ください			
\\	最中	
\\	さいちゅう			
\\	ありの群れが砂糖にたかっている。	
\\	あり 
\\	あり の むれ が さとう に たかっている 。			
\\	学割だとだいぶ安いな。	
\\	学割 
\\	がくわり だ と だいぶ やすい な			
\\	彼の話に私はあいづちを打ったの。	
\\	あいづち 
\\	かれ の はなし に わたし は あいづち を うった の			
\\	至急	
\\	しきゅう			
\\	暗いので足下に気を付けてください。	
\\	足下 
\\	くらい の で あしもと に き を つけて ください			
\\	普段から厚着しないようにしています。	
\\	厚着 
\\	ふだん から あつぎ しない よう に して います			
\\	手紙に宛名を書き込んだよ。	
\\	宛名 
\\	てがみ に あてな を かきこんだ よ			
\\	下調べ	
\\	したしらべ			
\\	雨降りで月が見えなかったね。	
\\	雨降り 
\\	あめふり で つき が みえなかった ね			
\\	彼はようやく姿を現しましたね。	
\\	現す 
\\	かれ は ようやく すがた を あらわしました ね			
\\	じめじめ	
\\	それは彼女の期待の現れです。	
\\	現れ 
\\	それ は かのじょ の きたい の あらわれ です			
\\	私からご連絡致します。	
\\	致す 
\\	わたし から ごれんらく いたします			
\\	暑中見舞い	
\\	しょちゅうみまい			
\\	次に、野菜を炒めてください。	
\\	炒める 
\\	つぎ に やさい を いためて ください			
\\	物価は一段と上昇しそうだね。	
\\	一段と 
\\	ぶっか は いちだんと じょうしょう しそう だ ね			
\\	じろじろ	
\\	私たちは庭園を一周しました。	
\\	一周 
\\	わたしたち は ていえん を いっしゅう しました			
\\	荷物が多いので大きな入れ物が必要です。	
\\	入れ物 
\\	にもつ が おおい の で おおき な いれもの が ひつよう です			
\\	彼はうわさを打ち消したわよ。	
\\	打ち消し 
\\	かれ は うわさ を うちけした わ よ			
\\	姓名	
\\	せいめい			
\\	英和辞書をよく使います。	
\\	英和 
\\	えいわ じしょ を よく つかいます			
\\	私たちは出発を1日遅らしたの。	
\\	遅らす 
\\	わたしたち は しゅっぱつ を いちにち おくらした の			
\\	海上で衝突事故が発生した。	
\\	海上 
\\	かいじょう で しょうとつ じこ が はっせい した			
\\	あり	
\\	僕たちは毎朝漢字の書き取りをします。	
\\	書き取り 
\\	ぼくたち は まいあさ かんじ の かきとり を します			
\\	母は鍋のシチューを掻き回しているよ。	
\\	掻き回す 
\\	はは は なべ の シチュー を かきまわしている よ 。			
\\	学割	
\\	がくわり			
\\	池がかちかちに凍っています。	
\\	かちかち 
\\	いけ が かちかち に こおって います			
\\	皆が代わる代わる彼らを祝福したよ。	
\\	代わる代わる 
\\	みな が かわるがわる かれら を しゅくふく した よ			
\\	あいづち	
\\	今日のフランス語の講座は休講です。	
\\	休講 
\\	きょう の ふらんすご の こうざ は きゅうこう です			
\\	今日はたくさん歩いてくたくたです。	
\\	くたくた 
\\	きょう は たくさん あるいて くたくた です			
\\	足下	
\\	あしもと			
\\	小さなバケツで水を汲んだの。	
\\	汲む 
\\	ちいさ な ばけつ で みず を くんだ の			
\\	先生に月謝を渡しましたか。	
\\	月謝 
\\	せんせい に げっしゃ を わたしました か			
\\	論文の項目を分かりやすく整理したわ。	
\\	項目 
\\	ろんぶん の こうもく を わかりやすく せいり した わ			
\\	厚着	
\\	あつぎ			
\\	彼らは交代でボートを漕いだんだ。	
\\	漕ぐ 
\\	かれら は こうたい で ぼーと を こいだ ん だ			
\\	こしょうを入れ過ぎてスープが辛い。	
\\	こしょう 
\\	こしょう を いれすぎて すーぷ が からい			
\\	宛名	
\\	あてな			
\\	あの山のすぐ近くが国境です。	
\\	国境 
\\	あの やま の すぐ ちかく が こっきょう です			
\\	それじゃあ上と下が逆さまだよ。	
\\	逆さま 
\\	それじゃあ うえ と した が さかさま だ よ			
\\	雨降り	
\\	あめふり			
\\	今日は久しぶりに日が射してるね。	
\\	射す 
\\	きょう は ひさしぶり に ひ が さして る ね			
\\	会長は田中氏に決定。	
\\	氏 
\\	かいちょう は たなかし に けってい			
\\	現す	
\\	あらわす			
\\	量より質の方が大事です。	
\\	質 
\\	りょう より しつ の ほう が だいじ です			
\\	新宿駅まで準急で行ったのよ。	
\\	準急 
\\	しんじゅくえき まで じゅんきゅう で いった の よ			
\\	彼女の肌はすべすべしているね。	
\\	すべすべ 
\\	かのじょ の はだ は すべすべ して いる ね			
\\	現れ	
\\	あらわれ			
\\	結婚して姓が変わりました。	
\\	姓 
\\	けっこん して せい が かわりました			
\\	せっかく来たんだからゆっくりして行きなさい。	
\\	せっかく 
\\	せっかく きたん だから ゆっくり して いきなさい			
\\	致す	
\\	いたす			
\\	部屋の空きはありますか。	
\\	空き 
\\	へや の あき は あります か			
\\	この商品は小売店でも買えます。	
\\	小売店 
\\	この しょうひん は こうりてん で も かえます			
\\	このブラウスは後ろ開きです。	
\\	開き 
\\	この ぶらうす は うしろあき です			
\\	炒める	
\\	いためる			
\\	彼は留学の夢を諦めていないよ。	
\\	諦める 
\\	かれ は りゅうがく の ゆめ を あきらめて いない よ			
\\	彼の頑固さにはあきれました。	
\\	あきれる 
\\	かれ の がんこさ に は あきれました			
\\	彼は悪女に騙されたんだ。	
\\	悪女 
\\	かれ は あくじょ に だまされた			
\\	一段と	
\\	いちだんと			
\\	海外に住むのは私の憧れです。	
\\	憧れ 
\\	憧れる 
\\	かいがい に すむ の は わたし の あこがれ です			
\\	彼はパイロットの職に憧れているんだ。	
\\	憧れ 
\\	憧れる 
\\	かれ は ぱいろっと の しょく に あこがれて いる ん だ			
\\	一周	
\\	いっしゅう			
\\	最近は暖かです。	
\\	暖か 
\\	さいきん は あたたか です			
\\	彼のあだなは「りき」です。	
\\	あだな 
\\	かれ の あだな は 
\\	りき 
\\	です 。			
\\	入れ物	
\\	いれもの			
\\	食事の後片付けを手伝ったの。	
\\	後片付け 
\\	しょくじ の あとかたづけ を てつだった の			
\\	バスタブからお湯があふれました。	
\\	あふれる 
\\	ばすたぶ から おゆ が あふれました			
\\	雨具の用意を忘れないように。	
\\	雨具 
\\	あまぐ の ようい を わすれない よう に			
\\	打ち消し	
\\	うちけし			
\\	嵐で庭の木が折れたよ。	
\\	嵐 
\\	あらし で にわ の き が おれた よ 。			
\\	彼は強盗に家の中を荒されたの。	
\\	荒す 
\\	かれ は ごうとう に いえ の なか を あらされた の			
\\	英和	
\\	えいわ			
\\	その映画はどんな粗筋ですか。	
\\	粗筋 
\\	その えいが は どんな あらすじ です か			
\\	その2国は資源をめぐって争っています。	
\\	争う 
\\	その にこく は しげん を めぐって あらそって います			
\\	この本では自然の大切さがよく著されているよ。	
\\	著す 
\\	この ほん で は しぜん の たいせつさ が よく あらわされて いる よ			
\\	遅らす	
\\	おくらす			
\\	有りのままを話して下さい。	
\\	有りのまま 
\\	ありのまま を はなして ください			
\\	この気持ちをうまく言い表わせません。	
\\	言い表わす 
\\	この きもち を うまく いいあらわせません			
\\	海上	
\\	かいじょう			
\\	突然何を言い出すのかと思った。	
\\	言い出す 
\\	とつぜん なに を いいだす の か と おもった			
\\	私はイカの刺し身が大好きです。	
\\	いか 
\\	わたし は いか の さしみ が だいすき です			
\\	書き取り	
\\	かきとり			
\\	最近彼のところに行き来してないの。	
\\	行き来 
\\	さいきん かれ の ところ に いきき して ない の			
\\	ここから先は行き止まりです。	
\\	行き止まり 
\\	ここ から さき は いきどまり です			
\\	掻き回す	
\\	かきまわす			
\\	動物を苛めてはいけません。	
\\	苛める 
\\	どうぶつ を いじめて は いけません			
\\	お店は一時休業になったんだ。	
\\	一時 
\\	おみせ は いちじ きゅうぎょう に なった ん だ			
\\	彼はいつのまにいなくなったの。	
\\	いつのまに 
\\	かれ は いつのまに いなく なった の			
\\	かちかち	
\\	春は一遍に花が咲く。	
\\	一遍に 
\\	はる は いっぺんに はな が さく			
\\	日本は稲の品種が多いよ。	
\\	稲 
\\	にっぽん は いね の ひんしゅ が おおい よ			
\\	代わる代わる	
\\	かわるがわる			
\\	彼はソファーで居眠りをしているよ。	
\\	居眠り 
\\	かれ は そふぁー で いねむり を して いる よ			
\\	市場にはインフレーションの影響が出ているね。	
\\	インフレーション 
\\	しじょう に は いんふれーしょん の えいきょう が でて いる ね			
\\	休講	
\\	きゅうこう			
\\	今度のウィークエンドは映画を見ます。	
\\	ウィークエンド 
\\	こんど の うぃーくえんど は えいが を みます			
\\	彼は上下おそろいの服を着ていますね。	
\\	上下 
\\	かれ は うえした おそろい の ふく を きています ね 。			
\\	冬はまめにうがいをします。	
\\	うがい 
\\	ふゆ は まめ に うがい を します			
\\	くたくた	
\\	もうすぐ夏休みなのでうきうきしています。	
\\	うきうき 
\\	もうすぐ なつやすみ な の で うきうき して います			
\\	パレードと観衆が道を埋めていたよ。	
\\	埋める 
\\	ぱれーど と かんしゅう が みち を うずめて いた よ			
\\	汲む	
\\	くむ			
\\	嘘つきは泥棒の始まりよ。	
\\	嘘つき 
\\	うそつき は どろぼう の はじまり よ 。			
\\	彼はうつむいて何かを考えているね。	
\\	うつむく 
\\	かれ は うつむいて なにか を かんがえて いる ね			
\\	彼女はうなずいて同意を示したの。	
\\	うなずく 
\\	かのじょ は うなずいて どうい を しめした の			
\\	月謝	
\\	げっしゃ			
\\	シャツを裏表に着ていますよ。	
\\	裏表 
\\	しゃつ を うらおもて に きて います よ			
\\	その噂は本当ですか。	
\\	噂 
\\	その うわさ は ほんとう です か			
\\	項目	
\\	こうもく			
\\	毎日金魚に餌をやります。	
\\	餌 
\\	まいにち きんぎょ に えさ を やります			
\\	木の枝を折ってはいけません。	
\\	枝 
\\	き の えだ を おって は いけません			
\\	漕ぐ	
\\	こぐ			
\\	彼は大金を得ましたよ。	
\\	得る 
\\	かれ は たいきん を えました よ			
\\	雲が空を覆っているね。	
\\	覆う 
\\	くも が そら を おおって いる ね			
\\	こしょう	
\\	その年、この地域では大水がありました。	
\\	大水 
\\	その とし この ちいき で は おおみず が ありました			
\\	日本では、大晦日にそばを食べます。	
\\	大晦日 
\\	にっぽん で は おおみそか に そば を たべます			
\\	私たちは舟を降りて、陸に上がったんだ。	
\\	陸 
\\	わたしたち は ふね を おりて おか に あがった ん だ			
\\	国境	
\\	こっきょう			
\\	小川がさらさら流れています。	
\\	小川 
\\	おがわ が さらさら ながれて います			
\\	弟は小さい頃は臆病だったんだ。	
\\	臆病 
\\	おとうと は ちいさい ころ は おくびょう だった ん だ			
\\	逆さま	
\\	さかさま			
\\	彼女には幼い息子がいます。	
\\	幼い 
\\	かのじょ に は おさない むすこ が います			
\\	話はこれでお仕舞いです。	
\\	お仕舞い 
\\	はなし は これ で おしまい です			
\\	射す	
\\	さす			
\\	うちの猫は雄です。	
\\	雄 
\\	うち の ねこ は おす です			
\\	デパートでお歳暮を送りました。	
\\	お歳暮 
\\	でぱーと で おせいぼ を おくりました			
\\	氏	
\\	し			
\\	彼は人をおだてるのが上手です。	
\\	おだてる 
\\	かれ は ひと を おだてる の が じょうず です			
\\	財布の落とし物を拾いました。	
\\	落とし物 
\\	さいふ の おとしもの を ひろいました			
\\	あなたもだいぶ大人になったね。	
\\	大人 
\\	あなた も だいぶ おとな に なった ね			
\\	質	
\\	しつ			
\\	彼は溺れている子を助けたんだ。	
\\	溺れる 
\\	かれ は おぼれて いる こ を たすけた ん だ			
\\	お目に掛かれて嬉しいです。	
\\	お目に掛かる 
\\	おめにかかれて うれしい です			
\\	誤解があったことをお詫びします。	
\\	お詫び 
\\	ごかい が あった こと を おわび します			
\\	準急	
\\	じゅんきゅう			
\\	ここの温泉はよく効くそうです。	
\\	温泉 
\\	ここ の おんせん は よく きく そう です			
\\	会は午後9時に終わりました。	
\\	会 
\\	かい は ごご くじ に おわりました			
\\	すべすべ	
\\	その国は市場の開放を求められているの。	
\\	開放 
\\	その くに は しじょう の かいほう を もとめられて いる の			
\\	もう一度原点に返って考えましょう。	
\\	返る 
\\	もういちど げんてん に かえって かんがえましょう			
\\	姓	
\\	せい			
\\	靴のかかとで彼の足を踏んでしまったの。	
\\	かかと 
\\	くつ の かかと で かれ の あし を ふん でしまった の 。			
\\	医師は人の命に関わる大切な職業だよ。	
\\	関わる 
\\	いし は ひと の いのち に かかわる たいせつ な しょくぎょう だ よ			
\\	せっかく	
\\	玄関でがさがさと音がしましたよ。	
\\	がさがさ 
\\	げんかん で がさがさ と おと が しました よ			
\\	弟は美味しそうにりんごをかじっていたよ。	
\\	かじる 
\\	おとうと は おいし そう に りんご を かじって いた よ			
\\	階下から微かな音が聞こえた。	
\\	微か 
\\	かいか から かすか な おと が きこえた			
\\	空き	
\\	あき			
\\	片方の目がかゆいです。	
\\	片方 
\\	かたほう の め が かゆい です			
\\	道に土の塊ができてたよ。	
\\	塊 
\\	みち に つち の かたまり が できてた よ			
\\	お日様が西に傾きましたね。	
\\	傾く 
\\	おひさま が にし に かたむきました ね			
\\	小売店	
\\	こうりてん			
\\	彼女は首を少し傾けて笑うの。	
\\	傾ける 
\\	かのじょ は くび を すこし かたむけて わらう の			
\\	格好のいい青年に会ったよ。	
\\	格好 
\\	かっこう の いい せいねん に あった よ			
\\	ここに生年月日を記入してください。	
\\	月日 
\\	ここ に せいねんがっぴ を きにゅう して ください			
\\	開き	
\\	あき			
\\	彼は水をがぶがぶ飲んだんだ。	
\\	がぶがぶ 
\\	かれ は みず を がぶがぶ のんだ ん だ			
\\	本にカバーを被せたの。	
\\	被せる 
\\	ほん に カバー を かぶせた の 。			
\\	諦める	
\\	あきらめる			
\\	お上は何を考えてるんだろうね。	
\\	上 
\\	おかみ は なに を かんがえて る ん だろう ね			
\\	犬が手に噛み付きました。	
\\	噛み付く 
\\	いぬ が て に かみつきました 。			
\\	このトランクは貨物で送ろう。	
\\	貨物 
\\	この とらんく は かもつ で おくろう			
\\	あきれる	
\\	この瓶はもう空ね。	
\\	空 
\\	この びん は もう から ね			
\\	木の上でからすが鳴いています。	
\\	からす 
\\	き の うえ で からす が ないて います			
\\	悪女	
\\	あくじょ			
\\	りんごの皮をむきましたよ。	
\\	皮 
\\	りんご の かわ を むきました よ			
\\	彼女は猫を可愛がっています。	
\\	可愛がる 
\\	かのじょ は ねこ を かわいがって います			
\\	憧れ	
\\	あこがれ			
\\	その間に彼は居なくなっていました。	
\\	間 
\\	その かん に かれ は いなく なって いました			
\\	祖父はがんで亡くなりました。	
\\	がん 
\\	そふ は がん で なくなりました			
\\	観客は興奮していたよ。	
\\	観客 
\\	かんきゃく は こうふん して いた よ			
\\	憧れる	
\\	あこがれる			
\\	台風で交通機関がストップしている。	
\\	機関 
\\	たいふう で こうつう きかん が すとっぷ して いる			
\\	暖か	
\\	あたたか			
\\	私は彼女の言葉に傷付きました。	
\\	傷付く 
\\	わたし は かのじょ の ことば に きずつきました			
\\	あなたを傷付けるつもりはありませんでした。	
\\	傷付ける 
\\	あなた を きずつける つもり は ありませんでした			
\\	多くの人が災害の犠牲になったの。	
\\	犠牲 
\\	おおく の ひと が さいがい の ぎせい に なった の			
\\	あだな	
\\	この箱にはぎっしり物が詰まっています。	
\\	ぎっしり 
\\	この はこ に は ぎっしり もの が つまって います			
\\	電車がぎゅうぎゅうに込んでいますね。	
\\	ぎゅうぎゅう 
\\	でんしゃ が ぎゅうぎゅう に こんで います ね			
\\	後片付け	
\\	あとかたづけ			
\\	5行以内で答えを書いてください。	
\\	行 
\\	ごぎょう いない で こたえ を かいて ください			
\\	ここのぎょうざは美味しいよ。	
\\	ぎょうざ 
\\	ここ の ぎょうざ は おいしい よ			
\\	余ったきれで人形を作りましょう。	
\\	きれ 
\\	あまった きれ で にんぎょう を つくりましょう			
\\	あふれる	
\\	ホテルの金庫に貴重品を入れたよ。	
\\	金庫 
\\	ほてる の きんこ に きちょうひん を いれた よ			
\\	狭い空間に物がたくさん置いてあるね。	
\\	空間 
\\	せまい くうかん に もの が たくさん おいて ある ね			
\\	釘を踏んで怪我をしました。	
\\	釘 
\\	くぎ を ふんで けが を しました			
\\	雨具	
\\	あまぐ			
\\	弟に足の裏をくすぐられた。	
\\	くすぐる 
\\	おとうと に あし の うら を くすぐられた			
\\	彼は壁に耳をくっ付けて隣の話を聞いていたの。	
\\	くっ付ける 
\\	かれ は かべ に みみ を くっつけて となり の はなし を きいて いた の			
\\	怪我人は軍の病院に運ばれたよ。	
\\	軍 
\\	けがにん は ぐん の びょういん に はこばれた よ			
\\	嵐	
\\	あらし			
\\	数百人の群集が集まっているの。	
\\	群集 
\\	すうひゃくにん の ぐんしゅう が あつまって いる の			
\\	彼はマイクで群衆に話しかけたんだ。	
\\	群衆 
\\	かれ は まいく で ぐんしゅう に はなしかけた ん だ			
\\	彼女の成績は上の下です。	
\\	下 
\\	かのじょ の せいせき は じょう の げ です			
\\	荒す	
\\	あらす			
\\	相撲の稽古はとても厳しいよ。	
\\	稽古 
\\	すもう の けいこ は とても きびしい よ			
\\	煙突から煙がまっすぐ上っていたの。	
\\	煙 
\\	えんとつ から けむり が まっすぐ あがって いた の			
\\	粗筋	
\\	あらすじ			
\\	彼女はいつまでもげらげら笑っていたな。	
\\	げらげら 
\\	かのじょ は いつ まで も げらげら わらって いた な			
\\	県の代表は2名です。	
\\	県 
\\	けん の だいひょう は にめい です			
\\	電話で航空券を予約しました。	
\\	航空券 
\\	でんわ で こうくうけん を よやく しました			
\\	争う	
\\	あらそう			
\\	家まで交替で荷物を持ちました。	
\\	交替 
\\	いえ まで こうたい で にもつ を もちました			
\\	強盗がカメラに写っていました。	
\\	強盗 
\\	ごうとう が かめら に うつって いました			
\\	ゴールデンウィークに海外旅行をします。	
\\	ゴールデンウィーク 
\\	ごーるでんうぃーく に かいがい りょこう を します			
\\	著す	
\\	あらわす			
\\	国民の安全が最も大切です。	
\\	国民 
\\	こくみん の あんぜん が もっとも たいせつ です			
\\	アジアの国名をいくつ知っていますか。	
\\	国名 
\\	あじあ の こくめい を いくつ しって います か			
\\	彼は素直な心を持っている。	
\\	心 
\\	かれ は すなお な こころ を もって いる			
\\	有りのまま	
\\	ありのまま			
\\	乞食が公園のベンチで寝ている。	
\\	乞食 
\\	こじき が こうえん の べんち で ねて いる			
\\	名前は五十音順に並んでいます。	
\\	五十音 
\\	なまえ は ごじゅうおんじゅん に ならんで います			
\\	冷えた手を擦って温めた。	
\\	擦る 
\\	ひえた て を こすって あたためた			
\\	言い表わす	
\\	いいあらわす			
\\	寒いのでこたつを出しました。	
\\	こたつ 
\\	さむい の で こたつ を だしました			
\\	彼女への伝言を言付けたの。	
\\	言付ける 
\\	かのじょ へ の でんごん を ことづけた の			
\\	日本のことわざをいくつくらい知っていますか。	
\\	ことわざ 
\\	にほん の ことわざ を いくつ くらい しって います か			
\\	言い出す	
\\	いいだす			
\\	その白い粉は赤ちゃんのミルクです。	
\\	粉 
\\	その しろい こな は あかちゃん の みるく です			
\\	長いこと御無沙汰いたしました。	
\\	御無沙汰 
\\	ながい こと ごぶさた いたしました			
\\	遅れてご免。	
\\	ご免 
\\	おくれて ごめん			
\\	いか	
\\	これからはもっと気を付けます。	
\\	これから 
\\	これから は もっと き を つけます			
\\	この際はっきり言っておきます。	
\\	際 
\\	この さい はっきり いって おきます			
\\	行き来	
\\	いきき			
\\	うちの会社は祭日は休みです。	
\\	祭日 
\\	うち の かいしゃ は さいじつ は やすみ です			
\\	2つの市の境に川が流れているの。	
\\	境 
\\	ふたつ の し の さかい に かわ が ながれて いる の			
\\	その件でしたら、先程上司から許可を得ております。	
\\	先程 
\\	その けん でしたら さきほど じょうし から きょか を えて おります			
\\	行き止まり	
\\	いきどまり			
\\	昨日は雨でしたね。	
\\	昨日 
\\	さくじつ は あめ でした ね			
\\	昨晩の雪がまだ庭に残っている。	
\\	昨晩 
\\	さくばん の ゆき が まだ にわ に のこって いる			
\\	苛める	
\\	いじめる			
\\	米はアジアでは大切な作物だ。	
\\	作物 
\\	こめ は あじあ で は たいせつ な さくもつ だ			
\\	昨夜の台風で電車が遅れているね。	
\\	昨夜 
\\	さくや の たいふう で でんしゃ が おくれて いる ね			
\\	彼女は助けを求めて大声で叫んだの。	
\\	叫ぶ 
\\	かのじょ は たすけ を もとめて おおごえ で さけんだ の			
\\	一時	
\\	いちじ			
\\	父親には一家を支える責任がある。	
\\	支える 
\\	ちちおや に は いっか を ささえる せきにん が ある			
\\	自転車に油をさしたの。	
\\	さす 
\\	じてんしゃ に あぶら を さした の 。			
\\	包丁が錆びてしまった。	
\\	錆びる 
\\	ほうちょう が さびて しまった			
\\	いつのまに	
\\	その都市には様々な人種が集まっているわ。	
\\	様々 
\\	その とし に は さまざま な じんしゅ が あつまって いる わ			
\\	左右を見てから横断歩道を渡りなさい。	
\\	左右 
\\	さゆう を みてから おうだん ほどう を わたりなさい			
\\	会議は明々後日に延期された。	
\\	明々後日 
\\	かいぎ は しあさって に えんき された			
\\	一遍に	
\\	いっぺんに			
\\	私は通勤に
\\	を使うんだ。	
\\	わたし は つうきん に じぇいあーる を つかう ん だ			
\\	明日の授業は時間割り通りです。	
\\	時間割り 
\\	あした の じゅぎょう は じかんわり どおり です			
\\	マンションの敷金を払いました。	
\\	敷金 
\\	まんしょん の しききん を はらいました			
\\	稲	
\\	いね			
\\	彼は事業に失敗したの。	
\\	事業 
\\	かれ は じぎょう に しっぱい した の			
\\	来月大阪に支社を開設します。	
\\	支社 
\\	らいげつ おおさか に ししゃ を かいせつ します			
\\	この部屋は暗くて湿気が多いね。	
\\	湿気 
\\	この へや は くらくて しっけ が おおい ね			
\\	居眠り	
\\	いねむり			
\\	昨日のコンサートは実に素晴らしかったよ。	
\\	実に 
\\	さくじつ の こんさーと は じつに すばらしかった よ			
\\	この漢字を字引で引いてみて。	
\\	字引 
\\	この かんじ を じびき で ひいて みて			
\\	足がしびれた。	
\\	しびれる 
\\	あし が しびれた			
\\	インフレーション	
\\	洗濯物がまだ湿っています。	
\\	湿る 
\\	せんたくもの が まだ しめって います			
\\	彼は友達に借金をしたんだ。	
\\	借金 
\\	かれ は ともだち に しゃっきん を した ん だ			
\\	ウィークエンド	
\\	彼は顔をじゃぶじゃぶ洗ったの。	
\\	じゃぶじゃぶ 
\\	かれ は かお を じゃぶじゃぶ あらった の			
\\	電話がじゃんじゃん掛かってきたな。	
\\	じゃんじゃん 
\\	でんわ が じゃんじゃん かかって きた な			
\\	上下	
\\	うえした			
\\	秋分の日は毎年9月23日頃です。	
\\	秋分 
\\	しゅうぶん の ひ は まいとし くがつ にじゅうさんにちごろ です			
\\	明日、重役会議が開かれます。	
\\	重役 
\\	あす じゅうやくかいぎ が ひらかれます			
\\	春夏秋冬の移り変わりを見るのが大好きです。	
\\	春夏秋冬 
\\	しゅんかしゅうとう の うつりかわり を みる の が だいすき です			
\\	うがい	
\\	春分の日は毎年3月20日頃です。	
\\	春分 
\\	しゅんぶん の ひ は まいとし さんがつ はつかごろ です			
\\	この箱の小をください。	
\\	小 
\\	この はこ の しょう を ください			
\\	うきうき	
\\	乗客の一人の具合が悪くなったの。	
\\	乗客 
\\	じょうきゃく の いち にん の ぐあい が わるく なった の 。			
\\	商人は数字に強いね。	
\\	商人 
\\	しょうにん は すうじ に つよい ね			
\\	私はコーヒー豆を小量で買うようにしています。	
\\	小量 
\\	わたし は コーヒー まめ を しょうりょう で かうよう に しています 。			
\\	埋める	
\\	うずめる			
\\	人間にとって水は食物より大切なの。	
\\	食物 
\\	にんげん に とって みず は しょくもつ より たいせつ なの			
\\	食料はこの箱に入っています。	
\\	食料 
\\	しょくりょう は この はこ に はいって います			
\\	このクラスの女子は18人です。	
\\	女子 
\\	この くらす の じょし は じゅうはちにん です			
\\	嘘つき	
\\	うそつき			
\\	彼は言葉遣いが少し女性的だね。	
\\	女性的 
\\	かれ は ことばづかい が すこし じょせいてき だ ね			
\\	彼は書物に囲まれて生活しているの。	
\\	書物 
\\	かれ は しょもつ に かこまれて せいかつ して いる の			
\\	知らず知らずのうちに疲れがたまっていたよ。	
\\	知らず知らず 
\\	しらずしらず の うち に つかれ が たまって いた よ			
\\	うつむく	
\\	今回の旅行ではヨーロッパの城を見て回ります。	
\\	城 
\\	こんかい の りょこう で は よーろっぱ の しろ を みて まわります			
\\	スカートにしわがよっているよ。	
\\	しわ 
\\	スカートにしわがよっているよ。			
\\	うなずく	
\\	師走に入ると忙しくなります。	
\\	師走 
\\	しわす に はいる と いそがしく なります			
\\	長い話合いの末、やっと同意に至った。	
\\	末 
\\	ながい はなしあい の すえ やっと どうい に いたった			
\\	裏表	
\\	うらおもて			
\\	転んで筋を痛めてしまいました。	
\\	筋 
\\	ころんで すじ を いためて しまいました			
\\	靴に砂が入ってしまった。	
\\	砂 
\\	くつ に すな が はいって しまった			
\\	噂	
\\	うわさ			
\\	部屋の隅にいすが積んであったの。	
\\	隅 
\\	へや の すみ に いす が つんで あった の			
\\	相撲取りはたいてい体がとても大きいね。	
\\	相撲 
\\	すもう とり は たいてい からだ が とても おおきい ね 。			
\\	あいつはずるがしこい顔をしているね。	
\\	ずるがしこい 
\\	あいつ は ずるがしこい かお を して いる ね			
\\	餌	
\\	えさ			
\\	鋭い刃物で指を怪我した。	
\\	鋭い 
\\	するどい はもの で ゆび を けが した			
\\	私がこの世に生を受けて80年が過ぎたわ。	
\\	生 
\\	わたし が このよ に せい を うけて はちじゅうねん が すぎた わ			
\\	書類は正と副の2通あります。	
\\	正 
\\	しょるい は せい と ふく の につう あります			
\\	枝	
\\	えだ			
\\	この会社では性による差別はありません。	
\\	性 
\\	この かいしゃ で は せい に よる さべつ は ありません			
\\	彼はきちんと正座して待っていたね。	
\\	正座 
\\	かれ は きちんと せいざ して まって いた ね			
\\	これは人の生死にかかわる問題です。	
\\	生死 
\\	これ は ひと の せいし に かかわる もんだい で す			
\\	得る	
\\	える			
\\	この手紙を清書してください。	
\\	清書 
\\	この てがみ を せいしょ して ください			
\\	どんなに頑張っても、せいぜい3位くらいにしかなれないだろうな。	
\\	せいぜい 
\\	どんな に がんばって も せいぜい さんい くらい に しか なれない だろう な			
\\	彼女は一生ぜいたくに暮らしたんだ。	
\\	ぜいたく 
\\	かのじょ は いっしょう ぜいたく に くらした ん だ			
\\	覆う	
\\	おおう			
\\	これは正当な要求です。	
\\	正当 
\\	これ は せいとう な ようきゅう です			
\\	海の底には不思議な生物がたくさんいるよ。	
\\	生物 
\\	うみ の そこ に は ふしぎ な せいぶつ が たくさん いる よ			
\\	子供が動物に接するのは良いことです。	
\\	接する 
\\	こども が どうぶつ に せっする の は よい こと です			
\\	大水	
\\	おおみず			
\\	世論はあなたの味方です。	
\\	世論 
\\	せろん は あなた の みかた です			
\\	お中元にビールが届きました。	
\\	お中元 
\\	おちゅうげん に びーる が とどきました			
\\	大晦日	
\\	おおみそか			
\\	子供たちが生き生きと遊んでいるね。	
\\	生き生きと 
\\	こどもたち が いきいきと あそんで いる ね			
\\	彼は一向に興味を示さないの。	
\\	一向に 
\\	かれ は いっこうに きょうみ を しめさない の			
\\	今学期は成績ががくんと落ちてしまった。	
\\	がくんと 
\\	こんがっき は せいせき が がくんと おちて しまった			
\\	陸	
\\	おか			
\\	ダイヤモンドがきらりと光った。	
\\	きらりと 
\\	だいやもんど が きらりと ひかった			
\\	薬をごくんと飲み込んだんだ。	
\\	ごくんと 
\\	くすり を ごくんと のみこんだ ん だ			
\\	小川	
\\	おがわ			
\\	硬くならないで、自然に話して下さい。	
\\	自然に 
\\	かたく ならない で しぜんに はなして ください			
\\	子供たちは順々にお菓子を受け取ったよ。	
\\	順々に 
\\	こどもたち は じゅんじゅんに おかし を うけとった よ			
\\	久しぶりに全速力で走りました。	
\\	全速力 
\\	ひさしぶり に ぜんそくりょく で はしりました			
\\	データ	
\\	去年のデータを見せてください。	
\\	データ 
\\	きょねん の データ を みせて ください 。			
\\	彼はいつも素早い対応をするね。	
\\	対応 
\\	かれ は いつも すばやい たいおう を する ね			
\\	私はただ彼女と話したかっただけです。	
\\	ただ 
\\	わたし は ただ かのじょ と はなしたかった だけ です			
\\	対応	
\\	たいおう			
\\	あの歌手は3部門で賞を取ったの。	
\\	部門 
\\	あの かしゅ は 
\\	ぶもん で しょう を とった の			
\\	野党が与党を厳しく非難していましたね。	
\\	野党 
\\	やとう が よとう を きびしく ひなん して いました ね			
\\	自民党から新しい党が分離したね。	
\\	分離 
\\	じみんとう から あたらしい とう が ぶんり した ね			
\\	ただ	
\\	ただ			
\\	話の展開についていけない。	
\\	展開 
\\	はなし の てんかい に ついていけない			
\\	三つの町が統合されて新しい市が誕生したのよ。	
\\	統合 
\\	みっつ の まち が とうごう されて あたらしい し が たんじょう した の よ			
\\	彼女の仕事はデータの入力です。	
\\	データ 
\\	入力 
\\	かのじょ の しごと は でーた の にゅうりょく です			
\\	部門	
\\	ぶもん			
\\	当初の計画ではもっと早く終わるはずでした。	
\\	当初 
\\	とうしょ の けいかく で は もっと はやく おわる はず でした			
\\	その事件については当局が調査しています。	
\\	当局 
\\	その じけん に ついて は とうきょく が ちょうさ して います			
\\	野党	
\\	やとう			
\\	これからは経営の合理化を図りたいと思います。	
\\	図る 
\\	これから は けいえい の ごうりか を はかりたい と おもいます			
\\	秋は虫の音が心地良いです。	
\\	音 
\\	あき は むし の ね が ここちいい です			
\\	ひらがなをカタカナに変換しました。	
\\	変換 
\\	ひらがな を かたかな に へんかん しました			
\\	分離	
\\	ぶんり			
\\	東京の地価は上がり続けているんだ。	
\\	地価 
\\	とうきょう の ちか は あがりつづけて いる ん だ			
\\	船が長い航海に乗り出したの。	
\\	乗り出す 
\\	ふね が ながい こうかい に のりだした の			
\\	政治の体制を変える必要があります。	
\\	体制 
\\	せいじ の たいせい を かえる ひつよう が あります			
\\	展開	
\\	てんかい			
\\	私の予想が当たった。	
\\	予想 
\\	わたし の よそう が あたった			
\\	車の費用、予算オーバーだ。	
\\	予算 
\\	費用 
\\	くるま の ひよう、 よさん おーばー だ			
\\	彼女は2才の子の母親です。	
\\	母親 
\\	かのじょ は 
\\	さい の こ の ははおや です			
\\	統合	
\\	とうごう			
\\	新しい車が発売された。	
\\	発売 
\\	あたらしい くるま が はつばい された			
\\	大雨で川の水かさが増しているな。	
\\	増す 
\\	おおあめ で かわ の みずかさ が まして いる な			
\\	その土地は民間企業に売却されたんだ。	
\\	民間 
\\	その とち は みんかん きぎょう に ばいきゃく された ん だ			
\\	入力	
\\	にゅうりょく			
\\	彼女はスペイン語に取り組んでいます。	
\\	取り組む 
\\	かのじょ は すぺいんご に とりくんで います			
\\	この携帯電話は操作が簡単です。	
\\	操作 
\\	この けいたい でんわ は そうさ が かんたん です			
\\	当初	
\\	とうしょ			
\\	自転車の乗り過ぎで股が痛いです。	
\\	股 
\\	じてんしゃ の のりすぎ で また が いたい です			
\\	昔のことを思い出したの。	
\\	昔 
\\	むかし の こと を おもいだした の			
\\	当局	
\\	とうきょく			
\\	彼の意見はもっともです。	
\\	もっとも 
\\	かれ の いけん は もっとも です			
\\	今日は本社で会議があります。	
\\	本社 
\\	きょう は ほんしゃ で かいぎ が あります			
\\	あの人は日本銀行の総裁です。	
\\	総裁 
\\	あの ひと は にっぽんぎんこう の そうさい です			
\\	図る	
\\	はかる			
\\	彼女の登場で会場は盛り上がったね。	
\\	登場 
\\	かのじょ の とうじょう で かいじょう は もりあがった ね			
\\	交通費は個人負担です。	
\\	負担 
\\	こうつうひ は こじん ふたん です			
\\	明日、病院に検査に行くので少し不安です。	
\\	不安 
\\	あす びょういん に けんさ に いく の で すこし ふあん です			
\\	音	
\\	ね			
\\	彼の意見は無視されたよ。	
\\	無視 
\\	かれ の いけん は むし された よ			
\\	社長自らがその会社と交渉したよ。	
\\	自ら 
\\	しゃちょう みずから が その かいしゃ と こうしょう した よ			
\\	変換	
\\	へんかん			
\\	彼は靴のメーカーで働いています。	
\\	メーカー 
\\	かれ は くつ の めーかー で はたらいて います			
\\	この5種類から選択できます。	
\\	選択 
\\	この 
\\	しゅるい から せんたく できます			
\\	仕事の見通しがたたない。	
\\	見通し 
\\	しごと の みとおし が たたない			
\\	地価	
\\	ちか			
\\	先月の利益は200万円でした。	
\\	利益 
\\	せんげつ の りえき は 
\\	まんえん でした			
\\	理解してくれてありがとう。	
\\	理解 
\\	りかい して くれて ありがとう			
\\	乗り出す	
\\	のりだす			
\\	被災者を救うための特別な措置が取られました。	
\\	措置 
\\	ひさいしゃ を すくう ため の とくべつ な そち が とられました			
\\	その計画に対する反対の声が高まっているわ。	
\\	高まる 
\\	その けいかく に たいする はんたい の こえ が たかまって いる わ			
\\	最近、教育レベルが低下しています。	
\\	低下 
\\	さいきん きょういく れべる が ていか して います			
\\	体制	
\\	たいせい			
\\	この場合、保険が適用されますか。	
\\	適用 
\\	この ばあい ほけん が てきよう されます か			
\\	女の子は話し方まで母親に似るわね。	
\\	似る 
\\	おんな の こ は はなしかた まで ははおや に にる わ ね			
\\	今の経済の動きを分析しています。	
\\	分析 
\\	いま の けいざい の うごき を ぶんせき して います			
\\	予想	
\\	よそう			
\\	彼は明日メキシコへ発ちます。	
\\	発つ 
\\	かれ は あした めきしこ へ たちます			
\\	私は旅行から多くの知識を得た。	
\\	知識 
\\	わたし は りょこう から おおく の ちしき を えた			
\\	知っている範囲で教えてください。	
\\	範囲 
\\	しって いる はんい で おしえて ください			
\\	予算	
\\	よさん			
\\	額に汗をかいたよ。	
\\	額 
\\	ひたい に あせ を かいた よ			
\\	夜中もテレビで台風の報道をしていた。	
\\	報道 
\\	よなか も てれび で たいふう の ほうどう を して いた			
\\	労働者がストをしたそうだよ。	
\\	労働者 
\\	ろうどうしゃ が すと を した そう だ よ			
\\	母親	
\\	ははおや			
\\	あの会社は目覚しく発展しているね。	
\\	発展 
\\	あの かいしゃ は めざましく はってん して いる ね			
\\	テレビで日米野球をやっていますよ。	
\\	日米 
\\	てれび で にちべい やきゅう を やって います よ			
\\	子供が老人の手を引いて歩いていたの。	
\\	老人 
\\	こども が ろうじん の て を ひいて あるいて いた の			
\\	発売	
\\	はつばい			
\\	自分のレベルに合った授業を選んでください。	
\\	レベル 
\\	じぶん の れべる に あった じゅぎょう を えらんで ください			
\\	教授はフランス文学の専門家。	
\\	専門家 
\\	専門 
\\	文学 
\\	きょうじゅ は ふらんす ぶんがく の せんもんか			
\\	川の上流は流れが速いよ。	
\\	流れ 
\\	かわ の じょうりゅう は ながれ が はやい よ			
\\	増す	
\\	ます			
\\	新幹線の速度はどれくらいですか。	
\\	速度 
\\	しんかんせん の そくど は どれ くらい です か			
\\	私は父を目標にしています。	
\\	目標 
\\	わたし は ちち を もくひょう に して います			
\\	彼女の様子を見てきます。	
\\	様子 
\\	かのじょ の ようす を みて きます			
\\	民間	
\\	みんかん			
\\	彼は建築家というよりむしろ芸術家ね。	
\\	むしろ 
\\	かれ は けんちくか と いう より むしろ げいじゅつか ね			
\\	その仕事に多数の応募があったよ。	
\\	多数 
\\	その しごと に たすう の おうぼ が あった よ			
\\	取り組む	
\\	とりくむ			
\\	警察は犯人を特定したらしいよ。	
\\	特定 
\\	けいさつ は はんにん を とくてい した らしい よ			
\\	その事件の直後、捜査本部が設置された。	
\\	本部 
\\	その じけん の ちょくご そうさ ほんぶ が せっち された			
\\	彼は高校時代からの友人です。	
\\	友人 
\\	かれ は こうこう じだい から の ゆうじん です			
\\	操作	
\\	そうさ			
\\	状況はますます悪くなったね。	
\\	ますます 
\\	じょうきょう は ますます わるく なった ね			
\\	その雑誌は年に4回発行されているんだ。	
\\	発行 
\\	その ざっし は ねん に 
\\	かい はっこう されて いる ん だ			
\\	旅行中は身の安全が第一です。	
\\	身 
\\	りょこうちゅう は み の あんぜん が だいいち です			
\\	股	
\\	また			
\\	このビルは大地震に耐えられるでしょうか。	
\\	耐える 
\\	この びる は おおじしん に たえられる でしょう か			
\\	半導体は様々な製品に使われているよ。	
\\	半導体 
\\	はんどうたい は さまざま な せいひん に つかわれて いる よ			
\\	昔	
\\	むかし			
\\	新しい内閣が誕生したね。	
\\	内閣 
\\	あたらしい ないかく が たんじょう した ね			
\\	その会社はいろいろな国に投資を分散しているね。	
\\	分散 
\\	その かいしゃ は いろいろ な くに に とうし を ぶんさん して いる ね			
\\	私は哲学を学んでいます。	
\\	学ぶ 
\\	わたし は てつがく を まなんで います			
\\	もっとも	
\\	この辺りの家賃の相場はいくらですか。	
\\	相場 
\\	この あたり の やちん の そうば は いくら です か			
\\	彼は素敵な男性です。	
\\	男性 
\\	かれ は すてき な だんせい です			
\\	本社	
\\	ほんしゃ			
\\	あいつが犯人です。	
\\	犯人 
\\	あいつ が はんにん です			
\\	ニュースでは標準語が使われるの。	
\\	標準 
\\	にゅーす で は ひょうじゅんご が つかわれる の			
\\	彼はしっかりと責任を果たしました。	
\\	果たす 
\\	かれ は しっかり と せきにん を はたしました			
\\	総裁	
\\	そうさい			
\\	彼は名の通った会社に就職しました。	
\\	名 
\\	かれ は な の とおった かいしゃ に しゅうしょく しました			
\\	平均で一日に8時間ぐらい働いています。	
\\	平均 
\\	へいきん で いちにち に 
\\	じかん ぐらい はたらいて います			
\\	途中で30分ほど休みましょう。	
\\	途中 
\\	とちゅう で 
\\	ぷん ほど やすみましょう			
\\	登場	
\\	とうじょう			
\\	彼はその問題を容易に解決したわ。	
\\	容易 
\\	かれ は その もんだい を ようい に かいけつ した わ			
\\	先日会社から独立しました。	
\\	独立 
\\	せんじつ かいしゃ から どくりつ しました			
\\	雨はだんだん強まります。	
\\	強まる 
\\	あめ は だんだん つよまります			
\\	負担	
\\	ふたん			
\\	課題は7月5日までに提出してください。	
\\	提出 
\\	かだい は 
\\	がつ 
\\	か まで に ていしゅつ して ください			
\\	彼はイギリスに派遣されました。	
\\	派遣 
\\	かれ は いぎりす に はけん されました			
\\	彼女はこの小説の主人公のモデルです。	
\\	モデル 
\\	かのじょ は この しょうせつ の しゅじんこう の もでる です			
\\	不安	
\\	ふあん			
\\	その件は終わったと認識しています。	
\\	認識 
\\	その けん は おわった と にんしき して います			
\\	サッカーは団体競技です。	
\\	団体 
\\	さっかー は だんたい きょうぎ です			
\\	彼の父親は先生です。	
\\	父親 
\\	かれ の ちちおや は せんせい です 。			
\\	無視	
\\	むし			
\\	ここが話題のレストランです。	
\\	話題 
\\	ここ が わだい の れすとらん です			
\\	イタリア旅行の費用は30万円です。	
\\	費用 
\\	いたりあ りょこう の ひよう は 
\\	まんえん です			
\\	彼はこの国に平和をもたらしたのよ。	
\\	もたらす 
\\	かれ は この くに に へいわ を もたらした の よ			
\\	自ら	
\\	みずから			
\\	私が司会を務めます。	
\\	務める 
\\	わたし が しかい を つとめます			
\\	彼は先生に反発していたよ。	
\\	反発 
\\	かれ は せんせい に はんぱつ して いた よ			
\\	メーカー	
\\	彼はとても表情が豊かですね。	
\\	表情 
\\	豊か 
\\	かれ は とても ひょうじょう が ゆたか です ね			
\\	私は彼のやり方には不満です。	
\\	不満 
\\	わたし は かれ の やりかた に は ふまん です			
\\	複数のレポートをひとつにまとめています。	
\\	複数 
\\	ふくすう の れぽーと を ひとつ に まとめて います			
\\	選択	
\\	せんたく			
\\	その国では英語は通じますか。	
\\	通じる 
\\	その くに で は えいご は つうじます か			
\\	まだ料金は払っていないけど。	
\\	料金 
\\	まだ りょうきん は はらって いない けど			
\\	車の部品を取り替えたんだ。	
\\	部品 
\\	くるま の ぶひん を とりかえた ん だ			
\\	見通し	
\\	みとおし			
\\	針に糸を通してください。	
\\	通す 
\\	はり に いと を とおして ください			
\\	メートルは長さの単位です。	
\\	単位 
\\	めーとる は ながさ の たんい です			
\\	その国の政府は各国に支援を要請したのよ。	
\\	要請 
\\	その くに の せいふ は かっこく に しえん を ようせい した の よ			
\\	利益	
\\	りえき			
\\	前年に比べて利益が上がりました。	
\\	前年 
\\	ぜんねん に くらべて りえき が あがりました			
\\	あの男の態度にみんなあきれてたよ。	
\\	態度 
\\	あの おとこ の たいど に みんな あきれて た よ			
\\	法律は私の専門です。	
\\	専門 
\\	ほうりつ は わたし の せんもん です			
\\	理解	
\\	りかい			
\\	応募者の年齢は問いません。	
\\	問う 
\\	おうぼしゃ の ねんれい は といません			
\\	入国手続きが終わりました。	
\\	手続き 
\\	にゅうこく てつづき が おわりました			
\\	人はそれぞれ考え方が違います。	
\\	それぞれ 
\\	ひと は それぞれ かんがえかた が ちがいます			
\\	措置	
\\	そち			
\\	彼の家は台風の被害にあったの。	
\\	被害 
\\	かれ の いえ は たいふう の ひがい に あった の			
\\	来週、妻の実家を訪問します。	
\\	訪問 
\\	らいしゅう つま の じっか を ほうもん します			
\\	彼女の人生は苦労の連続でした。	
\\	連続 
\\	かのじょ の じんせい は くろう の れんぞく でした			
\\	高まる	
\\	たかまる			
\\	1ポンドは454グラムに相当します。	
\\	相当 
\\	いちぽんど は 
\\	ぐらむ に そうとう します			
\\	彼はトップの成績で合格しました。	
\\	トップ 
\\	かれ は とっぷ の せいせき で ごうかく しました			
\\	私の免許は来年まで有効です。	
\\	有効 
\\	わたし の めんきょ は らいねん まで ゆうこう です			
\\	低下	
\\	ていか			
\\	彼は数学の博士だそうです。	
\\	博士 
\\	かれ は すうがく の はくし だ そう です			
\\	この商品はトラックで輸送します。	
\\	輸送 
\\	この しょうひん は とらっく で ゆそう します			
\\	最近、日本は健康ブームです。	
\\	ブーム 
\\	さいきん にほん は けんこう ぶーむ です			
\\	適用	
\\	てきよう			
\\	月の表面にはクレーターがたくさんあるね。	
\\	表面 
\\	つき の ひょうめん に は くれーたー が たくさん ある ね			
\\	怪我を防ぐためによくストレッチをしてください。	
\\	防ぐ 
\\	けが を ふせぐ ため に よく すとれっち を して ください			
\\	この製品は有害な化学物質を含んでいるわよ。	
\\	物質 
\\	この せいひん は ゆうがい な かがくぶっしつ を ふくんで いる わ よ			
\\	似る	
\\	にる			
\\	ようやくゴールが見えてきました。	
\\	ようやく 
\\	ようやく ごーる が みえて きました			
\\	昨日株価が暴落しました。	
\\	暴落 
\\	きのう かぶか が ぼうらく しました			
\\	彼はチームの一員です。	
\\	チーム 
\\	かれ は ちーむ の いちいん です			
\\	分析	
\\	ぶんせき			
\\	今日、珍しい体験をしました。	
\\	体験 
\\	きょう めずらしい たいけん を しました			
\\	通信技術の発達は目覚ましいな。	
\\	発達 
\\	通信 
\\	つうしん ぎじゅつ の はったつ は めざましい な			
\\	新型車に研究の成果が反映されています。	
\\	反映 
\\	しんがたしゃ に けんきゅう の せいか が はんえい されて います			
\\	発つ	
\\	たつ			
\\	チームの中では相互の信頼が大切です。	
\\	相互 
\\	ちーむ の なか で は そうご の しんらい が たいせつ です			
\\	その公園は町のほぼ中央に位置するんだ。	
\\	中央 
\\	その こうえん は まち の ほぼ ちゅうおう に いち する ん だ			
\\	知識	
\\	ちしき			
\\	今、スケジュールの調整をしています。	
\\	調整 
\\	いま すけじゅーる の ちょうせい を して います			
\\	電車の中では新聞を広げないで。	
\\	広げる 
\\	でんしゃ の なか で は しんぶん を ひろげない で			
\\	同じタイプで色の違うものはありますか。	
\\	タイプ 
\\	おなじ たいぷ で いろ の ちがう もの は あります か			
\\	範囲	
\\	はんい			
\\	判決が下された。	
\\	判決 
\\	はんけつ が くだされた			
\\	火を強めてください。	
\\	強める 
\\	ひ を つよめて ください			
\\	彼の職場は都内にあります。	
\\	都内 
\\	かれ の しょくば は とない に あります			
\\	額	
\\	ひたい			
\\	このエアコンはあまり電力を使いません。	
\\	電力 
\\	この えあこん は あまり でんりょく を つかいません			
\\	彼は絵を本格的に勉強しているんだ。	
\\	本格的 
\\	かれ は え を ほんかくてき に べんきょう して いる ん だ			
\\	報道	
\\	ほうどう			
\\	試験の日程が発表されました。	
\\	日程 
\\	しけん の にってい が はっぴょう されました			
\\	彼はビジネスで大成功を収めたな。	
\\	ビジネス 
\\	かれ は ビジネス で だい せいこう を おさめた な 。			
\\	彼、友達の意見を否定した。	
\\	否定 
\\	かれ ともだち の いけん を ひてい した			
\\	労働者	
\\	ろうどうしゃ			
\\	母はお釣りを確かめたの。	
\\	確かめる 
\\	はは は おつり を たしかめた の			
\\	私は朝早く投票を済ませました。	
\\	投票 
\\	わたし は あさ はやく とうひょう を すませました			
\\	自然の破壊が進んでいるのよ。	
\\	破壊 
\\	しぜん の はかい が すすんで いる の よ			
\\	発展	
\\	はってん			
\\	最近、社内ネットワークの調子が悪い。	
\\	ネットワーク 
\\	さいきん しゃない ねっとわーく の ちょうし が わるい			
\\	彼は大学で音楽の理論を勉強したんだよ。	
\\	理論 
\\	かれ は だいがく で おんがく の りろん を べんきょう した ん だ よ			
\\	日米	
\\	にちべい			
\\	今回は長期の滞在です。	
\\	長期 
\\	こんかい は ちょうき の たいざい です			
\\	プロジェクトメンバーと話し合いをしたの。	
\\	話し合い 
\\	メンバー 
\\	ぷろじぇくと めんばー と はなしあい を した の			
\\	老人	
\\	ろうじん			
\\	彼の声はよく響きます。	
\\	響く 
\\	かれ の こえ は よく ひびきます			
\\	これらの本を分類してください。	
\\	分類 
\\	これら の ほん を ぶんるい して ください			
\\	彼女はプレッシャーから解放されて本来の自分に戻ったな。	
\\	本来 
\\	かのじょ は ぷれっしゃー から かいほう されて ほんらい の じぶん に もどった な			
\\	レベル	
\\	この服、私にはやや小さいみたい。	
\\	やや 
\\	この ふく わたし に は やや ちいさい みたい			
\\	文字の大きさは統一してください。	
\\	統一 
\\	もじ の おおきさ は とういつ して ください			
\\	車に荷物を積んでください。	
\\	積む 
\\	くるま に にもつ を つんで ください			
\\	専門家	
\\	せんもんか			
\\	実家を4年間離れていました。	
\\	離れる 
\\	じっか を 
\\	ねんかん はなれて いました			
\\	彼は突然走り出したのよ。	
\\	突然 
\\	かれ は とつぜん はしりだした の よ			
\\	今、経済の動向は読みにくいわ。	
\\	動向 
\\	いま けいざい の どうこう は よみ にくい わ			
\\	流れ	
\\	ながれ			
\\	部長に仕事の相談をしたんだ。	
\\	部長 
\\	ぶちょう に しごと の そうだん を した ん だ			
\\	あの大学には派閥がたくさんあります。	
\\	派閥 
\\	あの だいがく に は はばつ が たくさん あります			
\\	仕事が終わったら連絡します。	
\\	連絡 
\\	しごと が おわったら れんらく します			
\\	速度	
\\	そくど			
\\	姉はよく旅をします。	
\\	旅 
\\	あね は よく たび を します			
\\	先月、娘が誕生しました。	
\\	誕生 
\\	せんげつ むすめ が たんじょう しました			
\\	二人の走者は同時にゴールしたよ。	
\\	同時 
\\	ふたり の そうしゃ は どうじ に ごーる した よ			
\\	目標	
\\	もくひょう			
\\	あの村は伝統を保守している。	
\\	保守 
\\	伝統 
\\	あの むら は でんとう を ほしゅ して いる			
\\	彼は常に姿勢がいい。	
\\	常に 
\\	かれ は つねに しせい が いい			
\\	日本語の単語をいくつ知っていますか。	
\\	単語 
\\	にほんご の たんご を いくつ しって います か			
\\	様子	
\\	ようす			
\\	選手は最高の状態で大会に臨んだわ。	
\\	臨む 
\\	せんしゅ は さいこう の じょうたい で たいかい に のぞんだ わ			
\\	来月の総会は東京で行われます。	
\\	総会 
\\	らいげつ の そうかい は とうきょう で おこなわれます			
\\	これは女性専用の車両です。	
\\	専用 
\\	これ は じょせい せんよう の しゃりょう です			
\\	むしろ	
\\	この電車は地上を走ります。	
\\	地上 
\\	この でんしゃ は ちじょう を はしります			
\\	通信教育はとても便利です。	
\\	通信 
\\	つうしん きょういく は とても べんり です			
\\	彼女はデモに参加したよ。	
\\	デモ 
\\	かのじょ は でも に さんか した よ			
\\	多数	
\\	たすう			
\\	彼は新聞記者を経て作家になりました。	
\\	経る 
\\	かれ は しんぶん きしゃ を へて さっか に なりました			
\\	その物語は京都が舞台だ。	
\\	舞台 
\\	その ものがたり は きょうと が ぶたい だ			
\\	みんなで役割を分担しましょう。	
\\	役割 
\\	みんな で やくわり を ぶんたん しましょう			
\\	特定	
\\	とくてい			
\\	その制度は廃止されました。	
\\	廃止 
\\	その せいど は はいし されました			
\\	オーダーを追加しました。	
\\	追加 
\\	おーだー を ついか しました			
\\	今年は比較的景気がいい。	
\\	比較的 
\\	ことし は ひかくてき けいき が いい			
\\	本部	
\\	ほんぶ			
\\	私は民族の歴史に興味があります。	
\\	民族 
\\	わたし は みんぞく の れきし に きょうみ が あります			
\\	ワープロで報告書を打ったよ。	
\\	ワープロ 
\\	わーぷろ で ほうこくしょ を うった よ			
\\	その爆撃で兵士が3人負傷しました。	
\\	兵士 
\\	その ばくげき で へいし が さんにん ふしょう しました			
\\	友人	
\\	ゆうじん			
\\	彼女は料理の腕が素人の領域を超えているわ。	
\\	領域 
\\	かのじょ は りょうり の うで が しろうと の りょういき を こえて いる わ			
\\	彼女は良い反応を示したわ。	
\\	反応 
\\	かのじょ は よい はんのう を しめした わ			
\\	グループのメンバーは10人です。	
\\	メンバー 
\\	グループ の メンバー は 
\\	にん です 。			
\\	ますます	
\\	自分を高めるのは大切なことです。	
\\	高める 
\\	じぶん を たかめる の は たいせつ な こと です			
\\	中世の農民は貧しかった。	
\\	農民 
\\	貧しい 
\\	ちゅうせい の のうみん は まずしかった			
\\	この問題はハードウェアの故障が原因です。	
\\	ハードウェア 
\\	この もんだい は はーどうぇあ の こしょう が げんいん です			
\\	発行	
\\	はっこう			
\\	以前の収入は今の倍はあった。	
\\	倍 
\\	いぜん の しゅうにゅう は いま の ばい は あった			
\\	これから身長と体重を測定します。	
\\	測定 
\\	これから しんちょう と たいじゅう を そくてい します			
\\	彼はよくマスコミにも登場するね。	
\\	マスコミ 
\\	かれ は よく ますこみ に も とうじょう する ね			
\\	身	
\\	み			
\\	彼は上司に怒りの目を向けたんだよ。	
\\	向ける 
\\	かれ は じょうし に いかり の め を むけた ん だ よ			
\\	日本は鉄道がとても発達しています。	
\\	鉄道 
\\	にっぽん は てつどう が とても はったつ して います			
\\	このメーカーの製品は高品質だわね。	
\\	メーカー 
\\	品質 
\\	この めーかー の せいひん は こうひんしつ だ わ ね			
\\	耐える	
\\	たえる			
\\	私は日本の伝統を大切にします。	
\\	伝統 
\\	わたし は にっぽん の でんとう を たいせつ に します			
\\	彼女は髪にリボンを結んだね。	
\\	結ぶ 
\\	かのじょ は かみ に りぼん を むすんだ ね			
\\	これは機械の内部の問題です。	
\\	内部 
\\	これ は きかい の ないぶ の もんだい です			
\\	半導体	
\\	はんどうたい			
\\	新聞の読者の投書欄は好きです。	
\\	読者 
\\	しんぶん の どくしゃ の とうしょらん は すき です			
\\	去年の売り上げと比較しましょう。	
\\	比較 
\\	きょねん の うりあげ と ひかく しましょう			
\\	内閣	
\\	ないかく			
\\	列車がホームに入ってきた。	
\\	列車 
\\	れっしゃ が ほーむ に はいって きた			
\\	ダイナマイトが爆発したんだ。	
\\	爆発 
\\	だいなまいと が ばくはつ した ん だ			
\\	私たちはがんの要因を研究しています。	
\\	要因 
\\	わたしたち は がん の よういん を けんきゅう して います			
\\	分散	
\\	ぶんさん			
\\	その殺人事件の捜査は2年間続きました。	
\\	捜査 
\\	その さつじん じけん の そうさ は 
\\	ねんかん つづきました			
\\	親子の対話は大切だよ。	
\\	対話 
\\	おやこ の たいわ は たいせつ だ よ			
\\	彼は今月の売上目標を達成した。	
\\	達成 
\\	かれ は こんげつ の うりあげ もくひょう を たっせい した			
\\	学ぶ	
\\	まなぶ			
\\	今年はひどい水不足よ。	
\\	不足 
\\	ことし は ひどい みずぶそく よ			
\\	その商品の宣伝をよく見かけます。	
\\	宣伝 
\\	その しょうひん の せんでん を よく みかけます			
\\	生産コスト増大のため、値上げします。	
\\	増大 
\\	せいさん こすと ぞうだい の ため ねあげ します			
\\	相場	
\\	そうば			
\\	電子辞書はとても便利です。	
\\	電子 
\\	でんし じしょ は とても べんり です			
\\	彼女の年齢は27です。	
\\	年齢 
\\	かのじょ の ねんれい は 
\\	です			
\\	私は料理人を目指しています。	
\\	目指す 
\\	わたし は りょうりにん を めざして います			
\\	男性	
\\	だんせい			
\\	彼は最後まで戦ったよ。	
\\	戦う 
\\	かれ は さいご まで たたかった よ			
\\	彼は農家に生まれました。	
\\	農家 
\\	かれ は のうか に うまれました			
\\	それは本人に聞いてください。	
\\	本人 
\\	それ は ほんにん に きいて ください			
\\	犯人	
\\	はんにん			
\\	最近甘いものを控えています。	
\\	控える 
\\	さいきん あまい もの を ひかえて います			
\\	データを30分おきに保存してください。	
\\	保存 
\\	でーた を 
\\	ぷん おき に ほぞん して ください			
\\	やがて雪も止むでしょう。	
\\	やがて 
\\	やがて ゆき も やむ でしょう			
\\	標準	
\\	ひょうじゅん			
\\	彼は旅行の直前に熱が出たの。	
\\	直前 
\\	かれ は りょこう の ちょくぜん に ねつ が でた の			
\\	彼は心が豊かな人です。	
\\	豊か 
\\	かれ は こころ が ゆたか な ひと です			
\\	果たす	
\\	はたす			
\\	とても雰囲気のいいお店ですね。	
\\	雰囲気 
\\	とても ふんいき の いい おみせ です ね			
\\	川に沿って歩いたんだ。	
\\	沿う 
\\	かわ に そって あるいた ん だ			
\\	彼は有能な知事ね。	
\\	知事 
\\	かれ は ゆうのう な ちじ ね			
\\	名	
\\	な			
\\	スミス夫人がいらっしゃいました。	
\\	夫人 
\\	すみすふじん が いらっしゃいました			
\\	筆者の趣旨を考えてください。	
\\	筆者 
\\	ひっしゃ の しゅし を かんがえて ください			
\\	平均	
\\	へいきん			
\\	この国の物価はここ10年でだいぶ上がったね。	
\\	物価 
\\	この くに の ぶっか は ここ 
\\	ねん で だいぶ あがった ね			
\\	走者が次々にゴールしました。	
\\	次々に 
\\	そうしゃ が つぎつぎに ごーる しました			
\\	途中	
\\	とちゅう			
\\	両者の意見を聞きましょう。	
\\	両者 
\\	りょうしゃ の いけん を ききましょう			
\\	彼女は若さを保とうと必死だ。	
\\	保つ 
\\	かのじょ は わかさ を たもとう と ひっし だ			
\\	犯人は警察に抵抗したの。	
\\	抵抗 
\\	はんにん は けいさつ に ていこう した の			
\\	容易	
\\	ようい			
\\	彼らは武器を取り、立ち上がった。	
\\	武器 
\\	かれら は ぶき を とり たちあがった			
\\	今、歯を治療しています。	
\\	治療 
\\	いま は を ちりょう して います			
\\	彼は道路に飛び出したの。	
\\	飛び出す 
\\	かれ は どうろ に とびだした の			
\\	独立	
\\	どくりつ			
\\	その手術の成功率は90
\\	だそうです。	
\\	率 
\\	その しゅじゅつ の せいこうりつ は 
\\	だ そう です			
\\	人に頼らないで、自分でやってごらん。	
\\	頼る 
\\	ひと に たよらない で じぶん で やって ごらん			
\\	車から燃料がもれていますよ。	
\\	燃料 
\\	くるま から ねんりょう が もれて います よ			
\\	強まる	
\\	つよまる			
\\	朝は家族全員が揃って食事します。	
\\	揃う 
\\	あさ は かぞく ぜんいん が そろって しょくじ します			
\\	遥か向こうに目的地が見えてきた。	
\\	遥か 
\\	はるか むこう に もくてきち が みえて きた			
\\	提出	
\\	ていしゅつ			
\\	彼はわずかな貯金で暮らしている。	
\\	わずか 
\\	かれ は わずか な ちょきん で くらして いる			
\\	彼は選挙に当選しました。	
\\	当選 
\\	かれ は せんきょ に とうせん しました			
\\	ここがいちばん面白い場面です。	
\\	場面 
\\	ここ が いちばん おもしろい ばめん です			
\\	派遣	
\\	はけん			
\\	仕事のやり方を教えてください。	
\\	やり方 
\\	しごと の やりかた を おしえて ください			
\\	心配しないで、単に眠いだけです。	
\\	単に 
\\	しんぱい しない で たんに ねむい だけ です			
\\	予定が変更になりました。	
\\	変更 
\\	よてい が へんこう に なりました			
\\	モデル	
\\	車を塀に寄せたよ。	
\\	寄せる 
\\	くるま を へい に よせた よ			
\\	彼女の行動パターンは興味深いな。	
\\	パターン 
\\	かのじょ の こうどう ぱたーん は きょうみぶかい な			
\\	その問題についてクラスで討論しました。	
\\	討論 
\\	その もんだい に ついて くらす で とうろん しました			
\\	認識	
\\	にんしき			
\\	ストレッチで筋肉を伸ばしましょう。	
\\	伸ばす 
\\	すとれっち で きんにく を のばしましょう			
\\	彼はついに弁護士の資格を取ったよ。	
\\	ついに 
\\	かれ は ついに べんごし の しかく を とった よ			
\\	それはとても不思議な話だね。	
\\	不思議 
\\	それ は とても ふしぎ な はなし だ ね			
\\	団体	
\\	だんたい			
\\	彼女には明確な目標があるね。	
\\	目標 
\\	明確 
\\	かのじょ に は めいかく な もくひょう が ある ね			
\\	音楽は私の日常の一部です。	
\\	日常 
\\	おんがく は わたし の にちじょう の いちぶ です			
\\	仏の教えを勉強しました。	
\\	仏 
\\	ほとけ の おしえ を べんきょう しました			
\\	父親	
\\	ちちおや			
\\	私たちのチームが優勝しました。	
\\	優勝 
\\	わたしたち の ちーむ が ゆうしょう しました			
\\	枠の中に答えを書いてください。	
\\	枠 
\\	わく の なか に こたえ を かいて ください			
\\	漢字は中国から伝わった。	
\\	伝わる 
\\	かんじ は ちゅうごく から つたわった			
\\	話題	
\\	わだい			
\\	チャンピオンがタイトルを防衛した。	
\\	防衛 
\\	ちゃんぴおん が たいとる を ぼうえい した			
\\	資格があると就職に有利です。	
\\	有利 
\\	しかく が ある と しゅうしょく に ゆうり です			
\\	二つの薬品を1対3の割合で混ぜたんだ。	
\\	割合 
\\	対 
\\	ふたつ の やくひん を 
\\	たい 
\\	の わりあい で まぜた ん だ			
\\	費用	
\\	ひよう			
\\	学校で勉強したことが役立った。	
\\	役立つ 
\\	がっこう で べんきょう した こと が やくだった			
\\	彼と私はいとこ同士です。	
\\	同士 
\\	かれ と わたし は いとこ どうし です			
\\	もたらす	
\\	広場に子供が沢山集まっていたよ。	
\\	広場 
\\	ひろば に こども が たくさん あつまって いた よ			
\\	彼はまるで子供のようにはしゃいだよ。	
\\	まるで 
\\	かれ は まるで こども の よう に はしゃいだ よ			
\\	務める	
\\	つとめる			
\\	その選手は怪我を乗り越えて復活したわね。	
\\	復活 
\\	その せんしゅ は けが を のりこえて ふっかつ した わ ね			
\\	彼女は異文化に対する理解を深めたね。	
\\	理解 
\\	深める 
\\	かのじょ は いぶんか に たいする りかい を ふかめた ね			
\\	来月中旬にフランスへ行きます。	
\\	中旬 
\\	らいげつ ちゅうじゅん に フランス へ いきます 。			
\\	反発	
\\	はんぱつ			
\\	警察は犯罪の防止に努めています。	
\\	防止 
\\	犯罪 
\\	けいさつ は はんざい の ぼうし に つとめて います			
\\	夜9時に東京に到着しました。	
\\	到着 
\\	よる 
\\	じ に とうきょう に とうちゃく しました			
\\	表情	
\\	ひょうじょう			
\\	業界は女性の比率が低い。	
\\	比率 
\\	あいてぃーぎょうかい は じょせい の ひりつ が ひくい			
\\	その二つの国に摩擦が生じたんだ。	
\\	摩擦 
\\	その ふたつ の くに に まさつ が しょうじた ん だ			
\\	彼は山道をどんどん進んで行ったの。	
\\	どんどん 
\\	かれ は やまみち を どんどん すすんで いった の			
\\	不満	
\\	ふまん			
\\	事件現場付近で怪しい人物を見た。	
\\	付近 
\\	じけん げんば ふきん で あやしい じんぶつ を みた			
\\	この土地の面積はどれ位ですか。	
\\	面積 
\\	この とち の めんせき は どれ くらい です か			
\\	祖母がうちの猫を抱いているわよ。	
\\	抱く 
\\	そぼ が うち の ねこ を だいて いる わ よ			
\\	複数	
\\	ふくすう			
\\	彼がそれほど悩んでいたとは知らなかった。	
\\	それほど 
\\	かれ が それほど なやんで いた と は しらなかった			
\\	みんなで環境を保護しましょう。	
\\	保護 
\\	みんな で かんきょう を ほご しましょう			
\\	彼は特殊な能力を持っています。	
\\	特殊 
\\	かれ は とくしゅ な のうりょく を もって います			
\\	通じる	
\\	つうじる			
\\	何か質問はありますか。	
\\	何か 
\\	なにか しつもん は あります か			
\\	彼は動作が機敏です。	
\\	動作 
\\	かれ は どうさ が きびん です			
\\	その学校の教育は特色がありますね。	
\\	特色 
\\	その がっこう の きょういく は とくしょく が あります ね			
\\	料金	
\\	りょうきん			
\\	機内に荷物を持ち込んだの。	
\\	持ち込む 
\\	きない に にもつ を もちこんだ の			
\\	数学は論理に基づく学問です。	
\\	論理 
\\	すうがく は ろんり に もとづく がくもん です			
\\	今日は流通の仕組みを勉強しましょう。	
\\	流通 
\\	きょう は りゅうつう の しくみ を べんきょう しましょう			
\\	部品	
\\	ぶひん			
\\	彼は会社で微妙な立場にあります。	
\\	微妙 
\\	かれ は かいしゃ で びみょう な たちば に あります			
\\	無理はしないでください。	
\\	無理 
\\	むり は しない で ください			
\\	通す	
\\	とおす			
\\	彼女は水玉模様のスカートをはいているね。	
\\	模様 
\\	かのじょ は みずたま もよう の すかーと を はいて いる ね			
\\	この道具の用途を説明します。	
\\	用途 
\\	道具 
\\	この どうぐ の ようと を せつめい します			
\\	この都市は工業地帯です。	
\\	地帯 
\\	この とし は こうぎょう ちたい です			
\\	単位	
\\	たんい			
\\	今日は授業を1時間に短縮します。	
\\	短縮 
\\	きょう は じゅぎょう を 
\\	じかん に たんしゅく します			
\\	雨で運動会が中止になったの。	
\\	中止 
\\	あめ で うんどうかい が ちゅうし に なった の			
\\	危ないので子供からライターを取り上げました。	
\\	取り上げる 
\\	あぶない の で こども から らいたー を とりあげました			
\\	要請	
\\	ようせい			
\\	私と彼は同じ年代です。	
\\	年代 
\\	わたし と かれ は おなじ ねんだい です			
\\	私は今月一杯で退職します。	
\\	退職 
\\	わたし は こんげつ いっぱい で たいしょく します			
\\	不況の影響で仕事が少ないね。	
\\	不況 
\\	ふきょう の えいきょう で しごと が すくない ね			
\\	前年	
\\	ぜんねん			
\\	日本では北海道に滞在しました。	
\\	滞在 
\\	にっぽん で は ほっかいどう に たいざい しました			
\\	彼は都心に住んでいます。	
\\	都心 
\\	かれ は としん に すんで います			
\\	態度	
\\	たいど			
\\	父は弁護士です。	
\\	弁護士 
\\	ちち は べんごし です			
\\	彼は私との結婚を望んでいます。	
\\	望む 
\\	かれ は わたし と の けっこん を のぞんで います			
\\	彼は結果に満足したようです。	
\\	満足 
\\	かれ は けっか に まんぞく した よう です			
\\	専門	
\\	せんもん			
\\	もはや彼の助けは必要じゃないの。	
\\	もはや 
\\	もはや かれ の たすけ は ひつよう じゃ ない の			
\\	作業を始める態勢は整っています。	
\\	態勢 
\\	さぎょう を はじめる たいせい は ととのって います			
\\	問う	
\\	とう			
\\	親が君を一度連れて来いって言うんだ。	
\\	連れる 
\\	おや が きみ を いちど つれて こい って いう ん だ			
\\	皆で暖かく見守りましょう。	
\\	見守る 
\\	みんな で あたたかく みまもりましょう			
\\	世の中にはいろいろな人がいます。	
\\	世の中 
\\	よのなか に は いろいろ な ひと が います			
\\	手続き	
\\	てつづき			
\\	相手チームのリードで前半が終わったよ。	
\\	前半 
\\	あいて ちーむ の りーど で ぜんはん が おわった よ			
\\	彼女の明るさに魅力を感じました。	
\\	魅力 
\\	かのじょ の あかるさ に みりょく を かんじました			
\\	直ちに会社に戻ってください。	
\\	直ちに 
\\	ただちに かいしゃ に もどって ください			
\\	それぞれ	
\\	この服のデザインは素敵ですね。	
\\	デザイン 
\\	この ふく の デザイン は すてき です ね 。			
\\	心理学の授業に登録しましたか。	
\\	登録 
\\	しんりがく の じゅぎょう に とうろく しました か			
\\	彼の説明で納得できました。	
\\	納得 
\\	かれ の せつめい で なっとく できました			
\\	被害	
\\	ひがい			
\\	彼女は4年間熱心に勉強したわ。	
\\	熱心 
\\	かのじょ は 
\\	ねんかん ねっしん に べんきょう した わ			
\\	彼は大のサッカーファンです。	
\\	ファン 
\\	かれ は だい の さっかーふぁん です			
\\	お申し込みなさるのでしたら、こちらにお並びください。	
\\	なさる 
\\	おもうしこみ なさる の でしたら こちら に お ならび ください			
\\	訪問	
\\	ほうもん			
\\	野菜が値上がりしていますね。	
\\	値上がり 
\\	やさい が ねあがり して います ね			
\\	東西に大きな道路が通っています。	
\\	東西 
\\	とうざい に おおき な どうろ が とおって います			
\\	レポートを1日で仕上げるのは不可能です。	
\\	不可能 
\\	れぽーと を 
\\	にち で しあげる の は ふかのう です			
\\	連続	
\\	れんぞく			
\\	島で火山が噴火したよ。	
\\	噴火 
\\	しま で かざん が ふんか した よ			
\\	彼は成功するに違いないわ。	
\\	違いない 
\\	かれ は せいこう する に ちがいない わ			
\\	彼は新しいことに挑戦している。	
\\	挑戦 
\\	かれ は あたらしい こと に ちょうせん して いる			
\\	相当	
\\	そうとう			
\\	彼は幅広い知識を持っています。	
\\	幅広い 
\\	かれ は はばひろい ちしき を もって います			
\\	全員の意見を総合してみましょう。	
\\	総合 
\\	ぜんいん の いけん を そうごう して みましょう			
\\	彼は物理学の博士です。	
\\	博士 
\\	博士 
\\	かれ は ぶつりがく の はかせ です			
\\	トップ	
\\	謎の物体が空を飛んでいます。	
\\	物体 
\\	なぞ の ぶったい が そら を とんで います			
\\	彼女は会社で高い地位に就いているよ。	
\\	地位 
\\	かのじょ は かいしゃ で たかい ちい に ついて いる よ			
\\	有効	
\\	ゆうこう			
\\	私の会社で社員を募集しています。	
\\	募集 
\\	わたし の かいしゃ で しゃいん を ぼしゅう して います			
\\	土を掘って木を植えました。	
\\	土 
\\	つち を ほって き を うえました			
\\	彼は十分反省しています。	
\\	反省 
\\	かれ は じゅうぶん はんせい して います			
\\	博士	
\\	はくし			
\\	今日は英語の文法を勉強します。	
\\	文法 
\\	きょう は えいご の ぶんぽう を べんきょう します			
\\	バス代が値上げされました。	
\\	値上げ 
\\	ばすだい が ねあげ されました			
\\	彼は昔、ドイツの大使でした。	
\\	大使 
\\	かれ は むかし どいつ の たいし でした			
\\	輸送	
\\	ゆそう			
\\	彼女は都の職員です。	
\\	都 
\\	かのじょ は と の しょくいん です			
\\	彼らは1日7時間労働している。	
\\	労働 
\\	かれら は 
\\	にち 
\\	じかん ろうどう して いる			
\\	コップの底が濡れていますよ。	
\\	底 
\\	こっぷ の そこ が ぬれて います よ			
\\	ブーム	
\\	彼女は涙を流したんだ。	
\\	流す 
\\	かのじょ は なみだ を ながした ん だ			
\\	夜は10時までに寝るのが望ましいの。	
\\	望ましい 
\\	よる は 
\\	じ まで に ねる の が のぞましい の			
\\	表面	
\\	ひょうめん			
\\	国民は総理大臣を非難しているね。	
\\	非難 
\\	こくみん は そうり だいじん を ひなん して いる ね			
\\	みんなの意見がまとまりました。	
\\	まとまる 
\\	みんな の いけん が まとまりました			
\\	彼はグループから抜けました。	
\\	抜ける 
\\	かれ は ぐるーぷ から ぬけました			
\\	防ぐ	
\\	ふせぐ			
\\	姉は福祉の勉強をしています。	
\\	福祉 
\\	あね は ふくし の べんきょう を して います			
\\	彼は貧しい家庭で育ったんだ。	
\\	貧しい 
\\	かれ は まずしい かてい で そだった ん だ			
\\	あの都市はオリンピックの有力な候補地です。	
\\	有力 
\\	あの とし は おりんぴっく の ゆうりょく な こうほち です			
\\	物質	
\\	ぶっしつ			
\\	最後まで共にがんばりましょう。	
\\	共に 
\\	さいご まで ともに がんばりましょう			
\\	私は農村で育ちました。	
\\	農村 
\\	わたし は のうそん で そだちました			
\\	試験で不正が見つかったよ。	
\\	不正 
\\	しけん で ふせい が みつかった よ			
\\	ようやく	
\\	この製品の保証期間は5年間となっております。	
\\	保証 
\\	この せいひん の ほしょう きかん は 
\\	ねん かん と なって おります			
\\	このゲームのルールは簡単です。	
\\	ルール 
\\	この げーむ の るーる は かんたん です			
\\	彼は長年の友人です。	
\\	長年 
\\	かれ は ながねん の ゆうじん です			
\\	暴落	
\\	ぼうらく			
\\	彼のことを見直しました。	
\\	見直す 
\\	かれ の こと を みなおしました			
\\	彼は父親の店を継いだの。	
\\	継ぐ 
\\	かれ は ちちおや の みせ を ついだ の			
\\	「さくら」という言葉を使って俳句を書いたよ。	
\\	俳句 
\\	「さくら」 という ことば を つかって はいく を かいた よ			
\\	チーム	
\\	事故現場の上空をヘリコプターが飛んでいます。	
\\	ヘリコプター 
\\	じこ げんば の じょうくう を へりこぷたー が とんで います			
\\	人々は喜びに沸いた。	
\\	喜び 
\\	ひとびと は よろこび に わいた			
\\	非常事態です。	
\\	非常 
\\	ひじょう じたい です			
\\	体験	
\\	たいけん			
\\	あの国は強力な兵器を持っている。	
\\	兵器 
\\	あの くに は きょうりょく な へいき を もって いる			
\\	彼はトップに次ぐ好成績でした。	
\\	次ぐ 
\\	かれ は とっぷ に つぐ こうせいせき でした			
\\	私は山の風景が好きです。	
\\	風景 
\\	わたし は やま の ふうけい が すき です			
\\	発達	
\\	はったつ			
\\	日本人は先輩、後輩の関係をとても大切に考えます。	
\\	先輩 
\\	にっぽんじん は せんぱい こうはい の かんけい を とても たいせつ に かんがえます			
\\	彼女は文学に興味を持っているのよ。	
\\	文学 
\\	かのじょ は ぶんがく に きょうみ を もっている の よ 。			
\\	良い成績が取れるように努めます。	
\\	努める 
\\	いい せいせき が とれる よう に つとめます			
\\	反映	
\\	はんえい			
\\	ともかく一度考え直しましょう。	
\\	ともかく 
\\	ともかく いちど かんがえなおしましょう			
\\	翌日、彼に会いに行きました。	
\\	翌日 
\\	よくじつ かれ に あい に いきました			
\\	かばんの中身を見せてください。	
\\	中身 
\\	かばん の なかみ を みせて ください			
\\	相互	
\\	そうご			
\\	疲れて肩が張っています。	
\\	張る 
\\	つかれて かた が はって います			
\\	評判の良いレストランに行きました。	
\\	評判 
\\	ひょうばん の いい れすとらん に いきました			
\\	中央	
\\	ちゅうおう			
\\	会議中、ついあくびをしてしまいました。	
\\	つい 
\\	かいぎちゅう つい あくび を して しまいました			
\\	新しい仕事を引き受けたよ。	
\\	引き受ける 
\\	あたらしい しごと を ひきうけた よ			
\\	調整	
\\	ちょうせい			
\\	危ないので避難してください。	
\\	避難 
\\	あぶない の で ひなん して ください			
\\	お振込みは3番の窓口です。	
\\	窓口 
\\	おふりこみ は 
\\	ばん の まどぐち です			
\\	私の趣味は読書です。	
\\	読書 
\\	わたし の しゅみ は どくしょ です			
\\	広げる	
\\	ひろげる			
\\	敗戦の原因は何だろう。	
\\	敗戦 
\\	はいせん の げんいん は なん だろう			
\\	3月に大学を卒業の見込みです。	
\\	見込み 
\\	がつ に だいがく を そつぎょう の みこみ です			
\\	彼女は臨時の店員です。	
\\	臨時 
\\	かのじょ は りんじ の てんいん です			
\\	タイプ	
\\	交渉が成立するという前提で話を進めます。	
\\	前提 
\\	こうしょう が せいりつ する と いう ぜんてい で はなし を すすめます			
\\	このズボンは上着と対になっています。	
\\	対 
\\	この ずぼん は うわぎ と つい に なって います			
\\	判決	
\\	はんけつ			
\\	この人とその人は、同一人物ですか。	
\\	同一 
\\	この ひと と その ひと は どういつ じんぶつ です か			
\\	トンネルを抜けると海が見えたよ。	
\\	トンネル 
\\	とんねる を ぬける と うみ が みえた よ			
\\	部下が素晴らしい働きをしたな。	
\\	働き 
\\	ぶか が すばらしい はたらき を した な			
\\	強める	
\\	つよめる			
\\	彼は受験のことで悩んでいます。	
\\	悩む 
\\	かれ は じゅけん の こと で なやんで います			
\\	彼は大きな悩みを抱えていました。	
\\	悩み 
\\	かれ は おおき な なやみ を かかえて いました			
\\	都内	
\\	とない			
\\	彼は応募の条件を満たしていない。	
\\	満たす 
\\	かれ は おうぼ の じょうけん を みたして いない			
\\	先頭の人はプラカードを持ってください。	
\\	先頭 
\\	せんとう の ひと は ぷらかーど を もって ください			
\\	この荷物を倉庫に運びましょう。	
\\	倉庫 
\\	この にもつ を そうこ に はこびましょう			
\\	電力	
\\	でんりょく			
\\	彼は都会での生活を楽しんでいるの。	
\\	都会 
\\	かれ は とかい で の せいかつ を たのしんで いる の			
\\	彼はスピーチの前に服装を整えた。	
\\	整える 
\\	かれ は すぴーち の まえ に ふくそう を ととのえた			
\\	本格的	
\\	ほんかくてき			
\\	彼は30代の半ばです。	
\\	半ば 
\\	かれ は 
\\	だい の なかば です			
\\	栄養バランスの良い食事をしよう。	
\\	バランス 
\\	えいよう ばらんす の いい しょくじ を しよう			
\\	日程	
\\	にってい			
\\	出発まで時間の余裕があります。	
\\	余裕 
\\	しゅっぱつ まで じかん の よゆう が あります			
\\	この円は直径6センチです。	
\\	直径 
\\	この えん は ちょっけい 
\\	せんち です			
\\	明日は高校の入試です。	
\\	入試 
\\	あした は こうこう の にゅうし です			
\\	ビジネス	
\\	この場でお礼を言わせてください。	
\\	場 
\\	この ば で おれい を いわせて ください			
\\	弟は美術を専攻しています。	
\\	美術 
\\	おとうと は びじゅつ を せんこう して います			
\\	ユーラシアは世界で最も大きい大陸です。	
\\	大陸 
\\	ゆーらしあ は せかい で もっとも おおきい たいりく です			
\\	否定	
\\	ひてい			
\\	どうしてもその訳を知りたい。	
\\	どうしても 
\\	どうしても その わけ を しりたい			
\\	私は早朝のジョギングを日課にしています。	
\\	早朝 
\\	わたし は そうちょう の じょぎんぐ を にっか に して います			
\\	長い戦いが終わった。	
\\	戦い 
\\	ながい たたかい が おわった			
\\	確かめる	
\\	たしかめる			
\\	彼は買収されたらしいわ。	
\\	買収 
\\	かれ は ばいしゅう された らしい わ			
\\	どうかお許しください。	
\\	どうか 
\\	どうか おゆるし ください			
\\	あなたは保険に加入していますか。	
\\	保険 
\\	あなた は ほけん に かにゅう して います か			
\\	投票	
\\	とうひょう			
\\	みんなで製品の名称を考えました。	
\\	名称 
\\	みんな で せいひん の めいしょう を かんがえました			
\\	彼はいつも黙って仕事をします。	
\\	黙る 
\\	かれ は いつも だまって しごと を します			
\\	お客様の要望を聞かせてください。	
\\	要望 
\\	おきゃくさま の ようぼう を きかせて ください			
\\	破壊	
\\	はかい			
\\	彼はスケートのプロです。	
\\	プロ 
\\	かれ は すけーと の ぷろ です			
\\	話を戻しましょう。	
\\	戻す 
\\	はなし を もどしましょう			
\\	彼が全面的に協力してくれるそうです。	
\\	全面的 
\\	かれ が ぜんめんてき に きょうりょく して くれる そう です			
\\	ネットワーク	
\\	その国は経済大国よ。	
\\	大国 
\\	その くに は けいざい たいこく よ			
\\	反論がある人はどうぞ。	
\\	反論 
\\	はんろん が ある ひと は どうぞ			
\\	理論	
\\	りろん			
\\	昨日の夜、社長ご夫妻と食事をしました。	
\\	夫妻 
\\	きのう の よる しゃちょうごふさい と しょくじ を しました			
\\	被災地に物資を送ったよ。	
\\	物資 
\\	ひさいち に ぶっし を おくった よ			
\\	今日は歴史的な日です。	
\\	歴史的 
\\	きょう は れきしてき な ひ です			
\\	長期	
\\	ちょうき			
\\	これは昨年の売り上げの統計です。	
\\	統計 
\\	これ は さくねん の うりあげ の とうけい です			
\\	彼は昨年入社したの。	
\\	入社 
\\	かれ は さくねん にゅうしゃ した の			
\\	これは素晴らしいチャンスだ。	
\\	チャンス 
\\	これ は すばらしい ちゃんす だ			
\\	話し合い	
\\	はなしあい			
\\	彼は適切な言葉で説明してくれました。	
\\	適切 
\\	かれ は てきせつ な ことば で せつめい して くれました			
\\	南北に山が広がっています。	
\\	南北 
\\	なんぼく に やま が ひろがって います			
\\	響く	
\\	ひびく			
\\	会社が二つの派閥に分裂しているんだ。	
\\	派閥 
\\	分裂 
\\	かいしゃ が ふたつ の はばつ に ぶんれつ して いる ん だ			
\\	旅行は父の老後の楽しみです。	
\\	楽しみ 
\\	りょこう は ちち の ろうご の たのしみ です			
\\	先生は私たちを導いてくれます。	
\\	導く 
\\	せんせい は わたしたち を みちびいて くれます			
\\	分類	
\\	ぶんるい			
\\	仕事の残りは家でします。	
\\	残り 
\\	しごと の のこり は いえ で します			
\\	彼は偉大な発明王です。	
\\	発明 
\\	かれ は いだい な はつめいおう です			
\\	このビルは完成までに2年を要した。	
\\	要する 
\\	この びる は かんせい まで に 
\\	ねん を ようした			
\\	本来	
\\	ほんらい			
\\	日中はずっと海で泳いでいました。	
\\	日中 
\\	にっちゅう は ずっと うみ で およいで いました			
\\	日中貿易は急激に伸びているわね。	
\\	日中 
\\	にっちゅう ぼうえき は きゅうげき に のびて いる わ ね			
\\	今日はどうも体の調子が悪い。	
\\	どうも 
\\	調子 
\\	きょう は どうも からだ の ちょうし が わるい			
\\	やや	
\\	先生の論文が雑誌に載りましたよ。	
\\	載る 
\\	論文 
\\	せんせい の ろんぶん が ざっし に のりました よ			
\\	最近犯罪が減っています。	
\\	犯罪 
\\	さいきん はんざい が へって います			
\\	統一	
\\	とういつ			
\\	霧が晴れて視界が開けたね。	
\\	開ける 
\\	きり が はれて しかい が ひらけた ね			
\\	彼はとても優秀な生徒です。	
\\	優秀 
\\	かれ は とても ゆうしゅう な せいと です			
\\	これが建物の側面の写真です。	
\\	側面 
\\	これ が たてもの の そくめん の しゃしん です			
\\	積む	
\\	つむ			
\\	会社が倒産しました。	
\\	倒産 
\\	かいしゃ が とうさん しました			
\\	最近、仕事がかなりハードです。	
\\	ハード 
\\	さいきん しごと が かなり はーど です			
\\	離れる	
\\	はなれる			
\\	彼は来年定年を迎える。	
\\	定年 
\\	かれ は らいねん ていねん を むかえる			
\\	私は留学生として日本に入国しました。	
\\	入国 
\\	わたし は りゅうがくせい として にっぽん に にゅうこく しました			
\\	外の温度はマイナス3度です。	
\\	マイナス 
\\	そと の おんど は まいなす 
\\	ど です			
\\	突然	
\\	とつぜん			
\\	とにかく現場へ行ってみましょう。	
\\	とにかく 
\\	とにかく げんば へ いって みましょう			
\\	婦人服売り場は5階でございます。	
\\	婦人 
\\	ふじんふく うりば は 
\\	かい で ございます			
\\	容器のふたはきちんと閉めましょう。	
\\	容器 
\\	ようき の ふた は きちんと しめましょう			
\\	動向	
\\	どうこう			
\\	若者は体力がありますね。	
\\	体力 
\\	わかもの は たいりょく が あります ね			
\\	面白い出来事がありました。	
\\	出来事 
\\	おもしろい できごと が ありました			
\\	部長	
\\	ぶちょう			
\\	この薬はまだ認可されていません。	
\\	認可 
\\	この くすり は まだ にんか されて いません			
\\	祖父の墓は近くにあります。	
\\	墓 
\\	そふ の はか は ちかく に あります			
\\	論文を仕上げないと卒業できませんよ。	
\\	論文 
\\	ろんぶん を しあげない と そつぎょう できません よ			
\\	派閥	
\\	はばつ			
\\	彼は新しいアイデアをすぐ取り入れます。	
\\	取り入れる 
\\	かれ は あたらしい あいであ を すぐ とりいれます			
\\	私があなたの安全を保障します。	
\\	保障 
\\	わたし が あなた の あんぜん を ほしょう します			
\\	お飲み物は無料でございます。	
\\	無料 
\\	お のみもの は むりょう で ございます			
\\	連絡	
\\	れんらく			
\\	展覧会は連日賑わいました。	
\\	連日 
\\	てんらんかい は れんじつ にぎわいました			
\\	能は元々大衆の文化でした。	
\\	大衆 
\\	のう は もともと たいしゅう の ぶんか でした			
\\	この問題を解けますか。	
\\	解く 
\\	この もんだい を とけます か			
\\	旅	
\\	たび			
\\	このパイプは詰まっていますよ。	
\\	パイプ 
\\	この ぱいぷ は つまって います よ			
\\	夜間は裏口から入ってください。	
\\	夜間 
\\	やかん は うらぐち から はいって ください			
\\	そんなことは想像できないよ。	
\\	想像 
\\	そんな こと は そうぞう できない よ			
\\	誕生	
\\	たんじょう			
\\	空がたちまち曇ってきたね。	
\\	たちまち 
\\	そら が たちまち くもって きた ね			
\\	湖に鳥の群れがいたよ。	
\\	群れ 
\\	みずうみ に とり の むれ が いた よ			
\\	当店ではお酒を取り扱っておりません。	
\\	取り扱う 
\\	とうてん で は お さけ を とりあつかって おりません			
\\	同時	
\\	どうじ			
\\	この島は観光で成り立っています。	
\\	成り立つ 
\\	この しま は かんこう で なりたって います			
\\	年末のセールはいつも込んでいるね。	
\\	年末 
\\	ねんまつ の せーる は いつも こんで いる ね			
\\	あの国は経済的に繁栄しているわね。	
\\	繁栄 
\\	あの くに は けいざいてき に はんえい して いる わ ね			
\\	保守	
\\	ほしゅ			
\\	彼はメガネを外しました。	
\\	外す 
\\	かれ は めがね を はずしました			
\\	その事故で多くの人が負傷したの。	
\\	負傷 
\\	その じこ で おおく の ひと が ふしょう した の			
\\	常に	
\\	つねに			
\\	全力で走れ。	
\\	全力 
\\	ぜんりょく で はしれ			
\\	この2社が市場を独占しています。	
\\	独占 
\\	この 
\\	しゃ が しじょう を どくせん して います			
\\	取り敢えずお知らせしておきます。	
\\	取り敢えず 
\\	とりあえず おしらせ して おきます			
\\	単語	
\\	たんご			
\\	私にとって動物は身近な存在です。	
\\	身近 
\\	わたし に とって どうぶつ は みぢか な そんざい です			
\\	例外は認めません。	
\\	例外 
\\	れいがい は みとめません			
\\	道具は全部揃っていますか。	
\\	道具 
\\	どうぐ は ぜんぶ そろって います か			
\\	臨む	
\\	のぞむ			
\\	彼は植物全般に詳しいね。	
\\	全般 
\\	かれ は しょくぶつ ぜんぱん に くわしい ね			
\\	勤務の体系を見直すことにしました。	
\\	体系 
\\	きんむ の たいけい を みなおす こと に しました			
\\	祖母は大正生まれです。	
\\	大正 
\\	そぼ は たいしょう うまれ です			
\\	総会	
\\	そうかい			
\\	彼はどんな環境にも適応できます。	
\\	適応 
\\	かれ は どんな かんきょう に も てきおう できます			
\\	私は雑誌の編集の仕事をしています。	
\\	編集 
\\	わたし は ざっし の へんしゅう の しごと を して います			
\\	ここで代金をお支払いください。	
\\	代金 
\\	ここ で だいきん を お しはらい ください			
\\	専用	
\\	せんよう			
\\	彼は単純な人です。	
\\	単純 
\\	かれ は たんじゅん な ひと です			
\\	沈んだ船を引き上げたんだ。	
\\	引き上げる 
\\	しずんだ ふね を ひきあげた ん だ			
\\	私の秘書はとても優秀です。	
\\	優秀 
\\	秘書 
\\	わたし の ひしょ は とても ゆうしゅう です			
\\	地上	
\\	ちじょう			
\\	学生たちは授業についていくのに必死です。	
\\	必死 
\\	がくせいたち は じゅぎょう に ついて いく の に ひっし です			
\\	そのパーティーはカジュアルな服装で大丈夫です。	
\\	服装 
\\	その ぱーてぃー は かじゅある な ふくそう で だいじょうぶ です			
\\	ついに山頂に到達しました。	
\\	到達 
\\	ついに さんちょう に とうたつ しました			
\\	通信	
\\	つうしん			
\\	パーティーの準備が整いました。	
\\	整う 
\\	ぱーてぃー の じゅんび が ととのいました			
\\	その町に爆弾が落とされたの。	
\\	爆弾 
\\	その まち に ばくだん が おとされた の			
\\	デモ	
\\	彼の言っていることは矛盾しています。	
\\	矛盾 
\\	かれ の いって いる こと は むじゅん して います			
\\	緑を取り戻す必要があります。	
\\	取り戻す 
\\	みどり を とりもどす ひつよう が あります			
\\	ワインのコルクを抜きました。	
\\	抜く 
\\	わいん の こるく を ぬきました			
\\	経る	
\\	へる			
\\	私は株の知識がないに等しいです。	
\\	等しい 
\\	わたし は かぶ の ちしき が ない に ひとしい です			
\\	人手が足りなくて忙しい。	
\\	人手 
\\	ひとで が たりなくて いそがしい			
\\	舞台	
\\	ぶたい			
\\	この経験はあなたにとってプラスになるでしょう。	
\\	プラス 
\\	この けいけん は あなた に とって ぷらす に なる でしょう			
\\	彼は彼女をリードしながら踊ったね。	
\\	リード 
\\	かれ は かのじょ を りーど しながら おどった ね			
\\	このソフトには多少問題がある。	
\\	多少 
\\	この そふと に は たしょう もんだい が ある			
\\	役割	
\\	やくわり			
\\	体の調子がとても良いです。	
\\	調子 
\\	からだ の ちょうし が とても いい です			
\\	猫が窓から外を眺めているよ。	
\\	眺める 
\\	ねこ が まど から そと を ながめて いる よ			
\\	棒の先端を持って下さい。	
\\	先端 
\\	ぼう の せんたん を もって ください			
\\	廃止	
\\	はいし			
\\	たとえ、嵐になっても絶対に行く。	
\\	たとえ 
\\	たとえ あらし に なって も ぜったい に いく			
\\	彼女は典型的なイタリア人です。	
\\	典型的 
\\	かのじょ は てんけいてき な いたりあじん です			
\\	追加	
\\	ついか			
\\	なるべく早く仕事を終わらせてください。	
\\	なるべく 
\\	なるべく はやく しごと を おわらせて ください			
\\	機械を分解してみたの。	
\\	分解 
\\	きかい を ぶんかい して みた の			
\\	役所で住民票をもらってきた。	
\\	役所 
\\	やくしょ で じゅうみんひょう を もらって きた			
\\	比較的	
\\	ひかくてき			
\\	彼は非暴力を説きました。	
\\	説く 
\\	かれ は ひぼうりょく を ときました			
\\	最近、体調が少し不安定です。	
\\	不安定 
\\	さいきん たいちょう が すこし ふあんてい です			
\\	叔父は役人として30年働きました。	
\\	役人 
\\	おじ は やくにん と して 
\\	ねん はたらきました			
\\	民族	
\\	みんぞく			
\\	降りてください、定員オーバーです。	
\\	定員 
\\	おりて ください ていいん おーばー です			
\\	腹が減って動けない。	
\\	腹 
\\	はら が へって うごけない			
\\	友達と百貨店で買い物をしました。	
\\	百貨店 
\\	ともだち と ひゃっかてん で かいもの を しました			
\\	ワープロ	
\\	ここに穴を掘りましょう。	
\\	掘る 
\\	ここ に あな を ほりましょう			
\\	立ち上がる	
\\	たちあがる			
\\	彼は急に立ち上がったの。	
\\	立ち上がる 
\\	かれ は きゅう に たちあがった の			
\\	飛行機の墜落事故があったんだ。	
\\	墜落 
\\	ひこうき の ついらく じこ が あった ん だ			
\\	このドアの高さを測ってください。	
\\	測る 
\\	この どあ の たかさ を はかって ください			
\\	墜落	
\\	ついらく			
\\	虫歯は予防できます。	
\\	予防 
\\	むしば は よぼう できます			
\\	姉が嫁に行ったの。	
\\	嫁 
\\	あね が よめ に いった の			
\\	入場券は当日でも買えますよ。	
\\	当日 
\\	にゅうじょうけん は とうじつ で も かえます よ			
\\	測る	
\\	はかる			
\\	私に何か用ですか。	
\\	用 
\\	わたし に なにか よう です か			
\\	この用紙に名前を書いてください。	
\\	用紙 
\\	この ようし に なまえ を かいて ください			
\\	講演のテーマは何ですか。	
\\	テーマ 
\\	こうえん の てーま は なん です か			
\\	予防	
\\	よぼう			
\\	彼女はまだ仕事の経験が乏しい。	
\\	乏しい 
\\	かのじょ は まだ しごと の けいけん が とぼしい。			
\\	先生が私の能力を引き出してくれました。	
\\	引き出す 
\\	せんせい が わたし の のうりょく を ひきだしてくれました			
\\	これは本物のダイヤモンドです。	
\\	本物 
\\	これ は ほんもの の だいやもんど です			
\\	嫁	
\\	よめ			
\\	今日、スポーツジムの申し込みをしました。	
\\	申し込み 
\\	きょう すぽーつじむ の もうしこみ を しました			
\\	彼は強敵を見事に破ったね。	
\\	破る 
\\	かれ は きょうてき を みごと に やぶった ね			
\\	私はまだ両親と同居しています。	
\\	同居 
\\	わたし は まだ りょうしん と どうきょ して います			
\\	当日	
\\	とうじつ			
\\	彼はこの仕事に適しています。	
\\	適する 
\\	かれ は この しごと に てきしています			
\\	あの駅で学生がたくさん乗り込んだね。	
\\	乗り込む 
\\	あの えき で がくせい が たくさん のりこんだ ね			
\\	部屋を暗くしてムードを出してみたよ。	
\\	ムード 
\\	へや を くらく して むーど を だして みた よ			
\\	用	
\\	よう			
\\	運転手が無線で話していたの。	
\\	無線 
\\	うんてんしゅ が むせん で はなして いた の			
\\	飛行機が無事着陸したわね。	
\\	着陸 
\\	ひこうき が ぶじ ちゃくりく した わ ね			
\\	用紙	
\\	ようし			
\\	定期演奏会は年に4回あります。	
\\	定期 
\\	ていきえんそうかい は ねん に 
\\	かい あります			
\\	その質問の適当な答えが見つかりません。	
\\	適当 
\\	その しつもん の てきとう な こたえ が みつかりません			
\\	兵隊が銃をかまえていたよ。	
\\	兵隊 
\\	へいたい が じゅう を かまえて いた よ			
\\	テーマ	
\\	彼は会社の寮に住んでいます。	
\\	寮 
\\	かれ は かいしゃ の りょう に すんで います			
\\	冷静に話し合いましょう。	
\\	冷静 
\\	れいせい に はなしあいましょう			
\\	その車は中古で買ったんだ。	
\\	中古 
\\	その くるま は ちゅうこ で かった ん だ			
\\	乏しい	
\\	とぼしい			
\\	彼女は中世の音楽が好きです。	
\\	中世 
\\	かのじょ は ちゅうせい の おんがく が すき です			
\\	停電のため仕事を中断しました。	
\\	中断 
\\	ていでん の ため しごと を ちゅうだん しました			
\\	そこをどいてください。	
\\	どく 
\\	そこをどいてください。			
\\	引き出す	
\\	ひきだす			
\\	彼女は流行に敏感だね。	
\\	敏感 
\\	かのじょ は りゅうこう に びんかん だ ね			
\\	私の家は川に面しています。	
\\	面する 
\\	わたし の いえ は かわ に めんして います			
\\	本物	
\\	ほんもの			
\\	そのふたつの町の中間には川が流れているの。	
\\	中間 
\\	その ふたつ の まち の ちゅうかん に は かわ が ながれて いる の			
\\	彼は国外に追放されました。	
\\	追放 
\\	かれ は こくがい に ついほう されました			
\\	この会社は急成長を遂げました。	
\\	遂げる 
\\	この かいしゃ は きゅうせいちょう を とげました			
\\	申し込み	
\\	もうしこみ			
\\	彼はイタリアのマーケットを開拓しました。	
\\	マーケット 
\\	かれ は いたりあ の まーけっと を かいたく しました			
\\	彼は高い理想を持っているの。	
\\	理想 
\\	かれ は たかい りそう を もって いる の			
\\	他方の視点からも見てみましょう。	
\\	他方 
\\	たほう の してん から も みて みましょう			
\\	破る	
\\	やぶる			
\\	私の弟は団地に住んでいます。	
\\	団地 
\\	わたし の おとうと は だんち に すんで います			
\\	果たして彼は現れるだろうか。	
\\	果たして 
\\	はたして かれ は あらわれるだろう か			
\\	同居	
\\	どうきょ			
\\	私たち夫婦は家事を分担しています。	
\\	分担 
\\	わたしたち ふうふ は かじ を ぶんたん して います			
\\	会社で面倒なことが起こったの。	
\\	面倒 
\\	かいしゃ で めんどう な こと が おこった の			
\\	適する	
\\	てきする			
\\	かばんに荷物を詰めました。	
\\	詰める 
\\	かばん に にもつ を つめました			
\\	彼女はファッションの専門家だ。	
\\	ファッション 
\\	かのじょ は ふぁっしょん の せんもんか だ			
\\	乗り込む	
\\	のりこむ			
\\	桜のつぼみが膨らんだのね。	
\\	膨らむ 
\\	さくら の つぼみ が ふくらんだ の ね			
\\	彼は不自由な暮らしに慣れたようだね。	
\\	不自由 
\\	かれ は ふじゆう な くらし に なれた よう だ ね			
\\	彼の翻訳は分かりやすいですよ。	
\\	翻訳 
\\	かれ の ほんやく は わかりやすい です よ			
\\	ムード	
\\	当日は天気に恵まれました。	
\\	当日 
\\	恵まれる 
\\	とうじつ は てんき に めぐまれました			
\\	新しい社員を雇いました。	
\\	雇う 
\\	あたらしい しゃいん を やといました			
\\	夏休みには家族で登山をします。	
\\	登山 
\\	なつやすみ に は かぞく で とざん を します			
\\	無線	
\\	むせん			
\\	努力と結果が比例していないの。	
\\	比例 
\\	どりょく と けっか が ひれい して いない の			
\\	その程度の努力では不十分です。	
\\	不十分 
\\	その ていど の どりょく で は ふじゅうぶん です			
\\	赤信号でブレーキを踏んだの。	
\\	ブレーキ 
\\	あかしんごう で ぶれーき を ふんだ の			
\\	着陸	
\\	ちゃくりく			
\\	ミスは誰にでもあります。	
\\	ミス 
\\	みす は だれ に で も あります			
\\	ヨットでクルージングを楽しみましたわ。	
\\	ヨット 
\\	よっと で くるーじんぐ を たのしみました わ			
\\	何とかお願いします。	
\\	何とか 
\\	なんとか おねがい します			
\\	定期	
\\	ていき			
\\	二人の愛はますます深まっているね。	
\\	深まる 
\\	ふたり の あい は ますます ふかまっている ね			
\\	彼は見事なジャンプを見せたね。	
\\	見事 
\\	かれ は みごと な じゃんぷ を みせた ね			
\\	彼は論理的な人です。	
\\	論理的 
\\	かれ は ろんりてき な ひと です			
\\	適当	
\\	てきとう			
\\	彼は俳優です。	
\\	俳優 
\\	かれ は はいゆう です			
\\	ここでは太陽エネルギーを使って発電しています。	
\\	発電 
\\	ここ で は たいよう えねるぎー を つかって はつでん して います			
\\	その病気は原因不明と言われているんだよ。	
\\	不明 
\\	その びょうき は げんいん ふめい と いわれている ん だ よ			
\\	兵隊	
\\	へいたい			
\\	京都から中継で放送しています。	
\\	中継 
\\	きょうと から ちゅうけい で ほうそう して います			
\\	不要になったパソコンを処分したんだ。	
\\	不要 
\\	ふよう に なった ぱそこん を しょぶん した ん だ			
\\	学生時代を懐かしく振り返ったんだ。	
\\	振り返る 
\\	がくせい じだい を なつかしく ふりかえった ん だ			
\\	寮	
\\	りょう			
\\	多くの文明は川の近くで始まった。	
\\	文明 
\\	おおく の ぶんめい は かわ の ちかく で はじまった			
\\	花瓶の向きを変えたの。	
\\	向き 
\\	かびん の むき を かえた の			
\\	冷静	
\\	れいせい			
\\	祖父は明治の生まれです。	
\\	明治 
\\	そふ は めいじ の うまれ です			
\\	その国の社会秩序は乱れているわ。	
\\	秩序 
\\	その くに の しゃかい ちつじょ は みだれて いる わ			
\\	この道は通行できません。	
\\	通行 
\\	この みち は つうこう できません			
\\	中古	
\\	ちゅうこ			
\\	彼は優秀な部下を持っているわね。	
\\	部下 
\\	かれ は ゆうしゅう な ぶか を もって いる わ ね			
\\	雪道で派手に転んでしまったの。	
\\	派手 
\\	ゆきみち で はで に ころんで しまった の			
\\	中世	
\\	ちゅうせい			
\\	冬は皮膚が乾燥しますね。	
\\	皮膚 
\\	ふゆ は ひふ が かんそう します ね			
\\	私は毎朝体重を測っています。	
\\	体重 
\\	わたし は まいあさ たいじゅう を はかって います			
\\	中断	
\\	ちゅうだん			
\\	明日から2週間、短期のアルバイトをします。	
\\	短期 
\\	あした から 
\\	しゅうかん たんき の あるばいと を します			
\\	もうすぐ目標の地点に到達します。	
\\	地点 
\\	もうすぐ もくひょう の ちてん に とうたつ します			
\\	これはとても手間のかかる料理です。	
\\	手間 
\\	これ は とても てま の かかる りょうり です			
\\	どく	
\\	彼女は絵がなかなか上手です。	
\\	なかなか 
\\	かのじょ は え が なかなか じょうず です			
\\	よく見たら、時計の針が12時を指してるよ。	
\\	針 
\\	よく みたら、 とけい の はり が 
\\	じ を さして る よ			
\\	彼女には日ごろからお世話になっています。	
\\	日ごろ 
\\	かのじょ に は ひごろ から おせわ に なって います			
\\	敏感	
\\	びんかん			
\\	みんな彼の温かい人柄が大好きだよ。	
\\	人柄 
\\	みんな かれ の あたたかい ひとがら が だいすき だ よ			
\\	これが参加者の名簿です。	
\\	名簿 
\\	これ が さんかしゃ の めいぼ です			
\\	私のチームは1回戦で敗れたよ。	
\\	敗れる 
\\	わたし の ちーむ は 
\\	かいせん で やぶれた よ			
\\	面する	
\\	めんする			
\\	道の両側にたくさんお店がありますよ。	
\\	両側 
\\	みち の りょうがわ に たくさん おみせ が あります よ			
\\	俺はロックを聞くのが好きだ。	
\\	ロック 
\\	おれ は ろっく を きく の が すき だ			
\\	中間	
\\	ちゅうかん			
\\	ガソリンが値下がりしました。	
\\	値下がり 
\\	がそりん が ねさがり しました			
\\	私たちの旅は始めはよかったんだ。	
\\	始め 
\\	わたしたち の たび は はじめ は よかった ん だ			
\\	追放	
\\	ついほう			
\\	私が一番好きな宝石はダイヤモンドなの。	
\\	宝石 
\\	わたし が いちばん すき な ほうせき は だいやもんど なの			
\\	沖縄方面にお出かけの方は台風にご注意ください。	
\\	方面 
\\	おきなわ ほうめん に おでかけ の かた は たいふう に ごちゅうい ください			
\\	わざわざ来てくれてありがとう。	
\\	わざわざ 
\\	わざわざ きて くれて ありがとう			
\\	遂げる	
\\	とげる			
\\	ここはいやな臭いがする。	
\\	臭い 
\\	ここ は いや な におい が する 。			
\\	急に出張を命じられました。	
\\	命じる 
\\	きゅう に しゅっちょう を めいじられました			
\\	彼は県庁で働いています。	
\\	庁 
\\	かれ は けんちょう で はたらいて います			
\\	マーケット	
\\	この本の著者はイギリス人です。	
\\	著者 
\\	この ほん の ちょしゃ は いぎりすじん です			
\\	彼女は肌がとても白いですね。	
\\	肌 
\\	かのじょ は はだ が とても しろい です ね			
\\	その噂はすぐに広まったよ。	
\\	広まる 
\\	その うわさ は すぐ に ひろまった よ			
\\	理想	
\\	りそう			
\\	お客さんが続々とやって来ましたよ。	
\\	続々 
\\	おきゃくさん が ぞくぞく と やってきました よ			
\\	今年から新しい仕事に就きます。	
\\	就く 
\\	ことし から あたらしい しごと に つきます			
\\	台風で停電したよ。	
\\	停電 
\\	たいふう で ていでん した よ			
\\	他方	
\\	たほう			
\\	彼はパイロットです。	
\\	パイロット 
\\	かれ は ぱいろっと です			
\\	その仕事の報酬として20万円もらいました。	
\\	報酬 
\\	その しごと の ほうしゅう と して 
\\	まんえん もらいました			
\\	団地	
\\	だんち			
\\	彼女はずいぶん大胆なことを言うね。	
\\	大胆 
\\	かのじょ は ずいぶん だいたん な こと を いう ね			
\\	彼はすごく知的な人です。	
\\	知的 
\\	かれ は すごく ちてき な ひと です			
\\	彼の家は人の出入りが多いね。	
\\	出入り 
\\	かれ の いえ は ひと の でいり が おおい ね			
\\	果たして	
\\	はたして			
\\	明日はどうやら雨らしいよ。	
\\	どうやら 
\\	あした は どうやら あめ らしい よ			
\\	ふと昔の事を思い出したんだ。	
\\	ふと 
\\	ふと むかし の こと を おもいだした ん だ			
\\	彼はベテランの運転手です。	
\\	ベテラン 
\\	かれ は べてらん の うんてんしゅ です			
\\	分担	
\\	ぶんたん			
\\	今年はボーナスがたくさん出ました。	
\\	ボーナス 
\\	ことし は ぼーなす が たくさん でました			
\\	月が満ちてきましたね。	
\\	満ちる 
\\	つき が みちて きました ね			
\\	この自転車の持ち主は誰ですか。	
\\	持ち主 
\\	この じてんしゃ の もちぬし は だれ です か			
\\	面倒	
\\	めんどう			
\\	子猫が線路に迷い込んだぞ。	
\\	線路 
\\	こねこ が せんろ に まよいこんだ ぞ			
\\	あの親子はそっくりだね。	
\\	そっくり 
\\	あの おやこ は そっくり だ ね			
\\	彼の人生は変化に富んでいるな。	
\\	富む 
\\	かれ の じんせい は へんか に とんで いる な			
\\	詰める	
\\	つめる			
\\	テニスコートの半面を使って試合をしたの。	
\\	半面 
\\	てにすこーと の はんめん を つかって しあい を した の			
\\	この地区は放射能に汚染されたんだ。	
\\	放射能 
\\	この ちく は ほうしゃのう に おせん された ん だ			
\\	暴力はいけません。	
\\	暴力 
\\	ぼうりょく は いけません			
\\	ファッション	
\\	我々の決意は固いです。	
\\	我々 
\\	われわれ の けつい は かたい です			
\\	この車の動力は電気です。	
\\	動力 
\\	この くるま の どうりょく は でんき です			
\\	お坊さんが鐘を鳴らしていますね。	
\\	鳴らす 
\\	おぼうさん が かね を ならして います ね			
\\	膨らむ	
\\	ふくらむ			
\\	南極でペンギンを見ました。	
\\	南極 
\\	なんきょく で ぺんぎん を みました			
\\	彼は物理を専攻している。	
\\	物理 
\\	かれ は ぶつり を せんこう して いる			
\\	今日は絶えず電話が鳴りました。	
\\	絶えず 
\\	きょう は たえず でんわ が なりました			
\\	不自由	
\\	ふじゆう			
\\	彼の服のセンスは独特よね。	
\\	独特 
\\	かれ の ふく の せんす は どくとく よ ね			
\\	彼は皮肉ばかり言います。	
\\	皮肉 
\\	かれ は ひにく ばかり いいます			
\\	ポケットに財布を突っ込んじゃった。	
\\	突っ込む 
\\	ぽけっと に さいふ を つっこん じゃった			
\\	翻訳	
\\	ほんやく			
\\	世界の平和が私たちの願いです。	
\\	願い 
\\	せかい の へいわ が わたし たち の ねがい です			
\\	日本人がノーベル賞を取ったよ。	
\\	ノーベル賞 
\\	にほんじん が のーべるしょう を とった よ			
\\	あの先生は生徒をみな平等に扱います。	
\\	平等 
\\	あの せんせい は せいと を みな びょうどう に あつかいます			
\\	恵まれる	
\\	めぐまれる			
\\	眼鏡のレンズを替えました。	
\\	レンズ 
\\	めがね の れんず を かえました			
\\	部屋の中を徹底的に探しました。	
\\	徹底的 
\\	へや の なか を てっていてき に さがしました			
\\	雇う	
\\	やとう			
\\	叔父は牧場を持っているんだ。	
\\	牧場 
\\	おじ は ぼくじょう を もって いる ん だ			
\\	昨日友人を見舞ったの。	
\\	見舞う 
\\	きのう ゆうじん を みまった の			
\\	静かな音楽で眠気を催したよ。	
\\	催す 
\\	しずか な おんがく で ねむけ を もよおした よ			
\\	登山	
\\	とざん			
\\	この絵は赤と黒の対比が美しいですね。	
\\	対比 
\\	この え は あか と くろ の たいひ が うつくしい です ね			
\\	最近疲れがたまっています。	
\\	疲れ 
\\	さいきん つかれ が たまって います			
\\	比例	
\\	ひれい			
\\	昨日、絵の展覧会に行ってきました。	
\\	展覧会 
\\	きのう え の てんらんかい に いって きました			
\\	この単語を発音してください。	
\\	発音 
\\	この たんご を はつおん して ください			
\\	その映画はよい批評を得ているんだ。	
\\	批評 
\\	その えいが は よい ひひょう を えて いる ん だ			
\\	不十分	
\\	ふじゅうぶん			
\\	パチンコで5000円損しました。	
\\	損 
\\	ぱちんこ で 
\\	えん そん しました			
\\	応募作品の大小は問いません。	
\\	大小 
\\	おうぼ さくひん の だいしょう は といません			
\\	彼女はついに捕まりました。	
\\	捕まる 
\\	かのじょ は ついに つかまりました			
\\	ブレーキ	
\\	嘆いていても何も変わりません。	
\\	嘆く 
\\	なげいていて も なに も かわりません			
\\	やっと運転免許を手に入れたよ。	
\\	免許 
\\	やっと うんてん めんきょ を て に いれた よ			
\\	応援団が試合を盛り上げたんだ。	
\\	盛り上げる 
\\	おうえんだん が しあい を もりあげた ん だ			
\\	ミス	
\\	彼はタイムを1秒縮めたの。	
\\	縮める 
\\	かれ は たいむ を 
\\	びょう ちぢめた の			
\\	このソフトには色々なツールが付属しています。	
\\	付属 
\\	この そふと に は いろいろ な つーる が ふぞく して います			
\\	妹は漫画が好きです。	
\\	漫画 
\\	いもうと は まんが が すき です			
\\	ヨット	
\\	彼は釣りの名人です。	
\\	名人 
\\	釣り 
\\	かれ は つり の めいじん です			
\\	前方に山が見えますね。	
\\	前方 
\\	ぜんぽう に やま が みえます ね			
\\	父が息子を励ましたの。	
\\	励ます 
\\	ちち が むすこ を はげました の			
\\	何とか	
\\	なんとか			
\\	うちの子はその本に夢中です。	
\\	夢中 
\\	うち の こ は その ほん に むちゅう です			
\\	新しい仕事の面接に行ってきました。	
\\	面接 
\\	あたらしい しごと の めんせつ に いって きました			
\\	月が庭を照らしている。	
\\	照らす 
\\	つき が にわ を てらして いる			
\\	深まる	
\\	ふかまる			
\\	寺のとなりに墓地があります。	
\\	墓地 
\\	てら の となり に ぼち が あります			
\\	明日、会議を開きます。	
\\	明日 
\\	みょうにち かいぎ を ひらきます			
\\	見事	
\\	みごと			
\\	町は台風で大きな損害を受けたんだ。	
\\	損害 
\\	まち は たいふう で おおきな そんがい を うけた ん だ			
\\	これは、中年の男性によく見られる症状です。	
\\	中年 
\\	これ は ちゅうねん の だんせい に よく みられる しょうじょう です			
\\	論理的	
\\	ろんりてき			
\\	私は通訳です。	
\\	通訳 
\\	わたし は つうやく です			
\\	「食べる」は動詞です。	
\\	動詞 
\\	たべる は どうし です			
\\	俳優	
\\	はいゆう			
\\	暑さでアイスクリームが溶けてしまった。	
\\	溶ける 
\\	あつさ で あいすくりーむ が とけて しまった			
\\	班のメンバーは5人です。	
\\	班 
\\	はん の めんばー は 
\\	にん です			
\\	彼は今、不利な立場にいます。	
\\	不利 
\\	かれ は いま ふり な たちば に います			
\\	発電	
\\	はつでん			
\\	最近のプロ野球は面白くなってきたね。	
\\	プロ野球 
\\	さいきん の ぷろやきゅう は おもしろく なって きた ね			
\\	私は別に気になりません。	
\\	別に 
\\	わたし は べつに き に なりません			
\\	彼は仕事の本質をよく理解しているわね。	
\\	本質 
\\	かれ は しごと の ほんしつ を よく りかい して いる わ ね			
\\	不明	
\\	ふめい			
\\	この商品は免税です。	
\\	免税 
\\	この しょうひん は めんぜい です			
\\	資格を有する人のみ応募できます。	
\\	有する 
\\	しかく を ゆうする ひと のみ おうぼ できます			
\\	私には仕事が第一です。	
\\	第一 
\\	わたし に は しごと が だいいち です			
\\	中継	
\\	ちゅうけい			
\\	このメガネをかけると平面が立体に見えます。	
\\	平面 
\\	この めがね を かける と へいめん が りったい に みえます			
\\	彼はとても早いペースで走っているね。	
\\	ペース 
\\	かれ は とても はやい ぺーす で はしって いる ね			
\\	子供が紙飛行機を飛ばしています。	
\\	飛ばす 
\\	こども が かみひこうき を とばして います			
\\	不要	
\\	ふよう			
\\	いや、僕は本気なんだ。	
\\	本気 
\\	いや ぼく は ほんき な ん だ			
\\	それは時間の無駄です。	
\\	無駄 
\\	それ は じかん の むだ です			
\\	振り返る	
\\	ふりかえる			
\\	彼女は自然と調和した暮らし方をしているの。	
\\	調和 
\\	かのじょ は しぜん と ちょうわ した くらし かた を して いる の			
\\	彼は大阪に転勤しました。	
\\	転勤 
\\	かれ は おおさか に てんきん しました			
\\	春の太陽が雪を溶かしました。	
\\	溶かす 
\\	はる の たいよう が ゆき を とかしました			
\\	文明	
\\	ぶんめい			
\\	保険は半年払いにしています。	
\\	半年 
\\	ほけん は はんとしばらい に しています			
\\	この仕事は君に任せる。	
\\	任せる 
\\	この しごと は きみ に まかせる			
\\	彼はマラソンに出場したよ。	
\\	マラソン 
\\	かれ は まらそん に しゅつじょう した よ			
\\	向き	
\\	むき			
\\	彼はいつも物事を深く考えるの。	
\\	物事 
\\	かれ は いつも ものごと を ふかく かんがえる の			
\\	要するに時機を待つべきだ。	
\\	要するに 
\\	ようするに じき を まつべき だ			
\\	明治	
\\	めいじ			
\\	あと少しで山の頂上です。	
\\	頂上 
\\	あと すこし で やま の ちょうじょう です			
\\	哲学は興味深い学問です。	
\\	哲学 
\\	てつがく は きょうみぶかい がくもん です			
\\	私たちは同級生です。	
\\	同級生 
\\	わたしたち は どうきゅうせい です			
\\	秩序	
\\	ちつじょ			
\\	電話料金が値下げされた。	
\\	値下げ 
\\	でんわ りょうきん が ねさげ された			
\\	税金の額は年収によって変わります。	
\\	年収 
\\	ぜいきん の がく は ねんしゅう に よって かわります			
\\	息子に新しい歯が生えました。	
\\	生える 
\\	むすこ に あたらしい は が はえました			
\\	通行	
\\	つうこう			
\\	車のライトが反射していますね。	
\\	反射 
\\	くるま の らいと が はんしゃ して います ね			
\\	店の外にまだ営業中の札がでているよ。	
\\	札 
\\	みせ の そと に まだ えいぎょう ちゅう の ふだ が でている よ			
\\	この棒は何に使うのですか。	
\\	棒 
\\	この ぼう は なに に つかう の です か			
\\	部下	
\\	ぶか			
\\	今日の遠足はよほど楽しかったらしい。	
\\	よほど 
\\	きょう の えんそく は よほど たのしかった らしい			
\\	別れはいつでも悲しいものです。	
\\	別れ 
\\	わかれ は いつでも かなしい もの です			
\\	ここで部屋の温度が調節できます。	
\\	調節 
\\	ここ で へや の おんど が ちょうせつ できます			
\\	派手	
\\	はで			
\\	直線を描いてください。	
\\	直線 
\\	ちょくせん を かいて ください			
\\	駅の手前に郵便局があります。	
\\	手前 
\\	えき の てまえ に ゆうびんきょく が あります			
\\	彼女は本国に帰りました。	
\\	本国 
\\	かのじょ は ほんごく に かえりました			
\\	皮膚	
\\	ひふ			
\\	古い手紙を燃やしたんだ。	
\\	燃やす 
\\	ふるい てがみ を もやした ん だ			
\\	贈り物に手書きのカードを添えました。	
\\	添える 
\\	おくりもの に てがき の かーど を そえました			
\\	この笛は竹でできている。	
\\	竹 
\\	この ふえ は たけ で できて いる			
\\	体重	
\\	たいじゅう			
\\	その事故で多量のガス漏れがあったね。	
\\	多量 
\\	その じこ で たりょう の がすもれ が あった ね			
\\	君の長所は明るいところだね。	
\\	長所 
\\	きみ の ちょうしょ は あかるい ところ だ ね			
\\	この仕事は辛いです。	
\\	辛い 
\\	この しごと は つらい です			
\\	短期	
\\	たんき			
\\	ここは天候の変化が激しいですね。	
\\	天候 
\\	ここ は てんこう の へんか が はげしい です ね			
\\	この機械は取り扱いに注意してください。	
\\	取り扱い 
\\	この きかい は とりあつかい に ちゅうい して ください			
\\	地点	
\\	ちてん			
\\	当時、日ソ会談が開かれた。	
\\	日ソ 
\\	とうじ にっそ かいだん が ひらかれた			
\\	部下が疲労で倒れたの。	
\\	疲労 
\\	ぶか が ひろう で たおれた の			
\\	お葬式は仏教で行うことが多いです。	
\\	仏教 
\\	おそうしき は ぶっきょう で おこなう こと が おおい です			
\\	手間	
\\	てま			
\\	彼女はいつも文句ばかり言う。	
\\	文句 
\\	かのじょ は いつも もんく ばかり いう			
\\	父は若いころラグビーの選手でした。	
\\	ラグビー 
\\	ちち は わかい ころ らぐびー の せんしゅ でした			
\\	なかなか	
\\	私は毎朝ラッシュの電車に乗っています。	
\\	ラッシュ 
\\	わたし は まいあさ らっしゅ の でんしゃ に のって います			
\\	ロープを使って崖を下りたんだ。	
\\	ロープ 
\\	ろーぷ を つかって がけ を おりた ん だ			
\\	彼は悲しみを乗り越えて強く生きた。	
\\	乗り越える 
\\	かれ は かなしみ を のりこえて つよく いきた			
\\	針	
\\	はり			
\\	無事、家に着きました。	
\\	無事 
\\	ぶじ いえ に つきました			
\\	彼の連絡先を方々に問い合わせたんだ。	
\\	方々 
\\	問い合わせる 
\\	かれ の れんらくさき を ほうぼう に といあわせた ん だ			
\\	私たちは名刺を交換しました。	
\\	名刺 
\\	わたしたち は めいし を こうかん しました			
\\	日ごろ	
\\	ひごろ			
\\	彼には理屈が通じない。	
\\	理屈 
\\	かれ に は りくつ が つうじない			
\\	東京湾は重要な役割を果たしている。	
\\	湾 
\\	とうきょうわん は じゅうよう な やくわり を はたして いる			
\\	人柄	
\\	ひとがら			
\\	窓から車の騒音が入って来ます。	
\\	騒音 
\\	まど から くるま の そうおん が はいって きます			
\\	ここからは未知の領域です。	
\\	未知 
\\	ここ から は みち の りょういき です			
\\	名簿	
\\	めいぼ			
\\	彼はこの事件と無関係です。	
\\	無関係 
\\	かれ は この じけん と むかんけい です			
\\	彼は勇気があるね。	
\\	勇気 
\\	かれ は ゆうき が ある ね			
\\	敗れる	
\\	やぶれる			
\\	上司が車を譲ってくれたの。	
\\	譲る 
\\	じょうし が くるま を ゆずって くれた の			
\\	今、ダムの水が少ないね。	
\\	ダム 
\\	いま だむ の みず が すくない ね			
\\	違法駐車の取り締まりが厳しくなったの。	
\\	取り締まり 
\\	いほう ちゅうしゃ の とりしまり が きびしく なった の			
\\	両側	
\\	りょうがわ			
\\	彼に言い訳の余地はありません。	
\\	余地 
\\	かれ に いいわけ の よち は ありません			
\\	サーバーがダウンしています。	
\\	ダウン 
\\	サーバーがダウンしています。			
\\	ロック	
\\	彼は力強い演技をするね。	
\\	力強い 
\\	かれ は ちからづよい えんぎ を する ね			
\\	この製品は低温で保存してください。	
\\	低温 
\\	この せいひん は ていおん で ほぞん して ください			
\\	彼は扉を開けたんだ。	
\\	扉 
\\	かれ は とびら を あけた ん だ			
\\	値下がり	
\\	ねさがり			
\\	彼は毎日トレーニングをするの。	
\\	トレーニング 
\\	かれ は まいにち とれーにんぐ を する の			
\\	選手の入場です。	
\\	入場 
\\	せんしゅ の にゅうじょう です			
\\	始め	
\\	はじめ			
\\	日本の国旗は日の丸と呼ばれています。	
\\	日の丸 
\\	にっぽん の こっき は ひのまる と よばれて います			
\\	風で髪が乱れたね。	
\\	乱れる 
\\	かぜ で かみ が みだれた ね			
\\	宝石	
\\	ほうせき			
\\	互いの話をよく聞きなさい。	
\\	互い 
\\	たがい の はなし を よく ききなさい			
\\	縦2列に並んでください。	
\\	縦 
\\	たて 
\\	れつ に ならんで ください			
\\	今日は付き合いで帰りが遅くなります。	
\\	付き合い 
\\	きょう は つきあい で かえり が おそく なります			
\\	方面	
\\	ほうめん			
\\	彼女は急いで怪我の手当てをしたよ。	
\\	手当て 
\\	かのじょ は いそいで けが の てあて を した よ			
\\	彼の意見には同意できません。	
\\	同意 
\\	かれ の いけん に は どうい できません			
\\	わざわざ	
\\	彼女は華やかな女性ですね。	
\\	華やか 
\\	かのじょ は はなやか な じょせい です ね			
\\	彼の行いはモラルに反しています。	
\\	反する 
\\	かれ の おこない は もらる に はんして います			
\\	ステージの幕が上がったんだ。	
\\	幕 
\\	すてーじ の まく が あがった ん だ			
\\	臭い	
\\	におい			
\\	私は空を見上げたの。	
\\	見上げる 
\\	わたし は そら を みあげた の			
\\	あの人は卓球の元選手です。	
\\	元 
\\	あの ひと は たっきゅう の もとせんしゅ です			
\\	命じる	
\\	めいじる			
\\	学校の催しに参加しました。	
\\	催し 
\\	がっこう の もよおし に さんか しました			
\\	私の父は大工です。	
\\	大工 
\\	わたし の ちち は だいく です			
\\	パトカーが車を追跡しているわ。	
\\	追跡 
\\	ぱとかー が くるま を ついせき して いる わ			
\\	庁	
\\	ちょう			
\\	これは子供に特有の病気です。	
\\	特有 
\\	これ は こども に とくゆう の びょうき です			
\\	ご無事で何よりです。	
\\	何より 
\\	ごぶじ で なにより です			
\\	著者	
\\	ちょしゃ			
\\	今日は頭の回転が鈍いです。	
\\	鈍い 
\\	きょう は あたま の かいてん が にぶい です			
\\	彼は方言で話します。	
\\	方言 
\\	かれ は ほうげん で はなします			
\\	肌	
\\	はだ			
\\	友達が見舞いに来てくれたよ。	
\\	見舞い 
\\	ともだち が みまい に きて くれた よ			
\\	夜になって風が弱まったね。	
\\	弱まる 
\\	よる に なって かぜ が よわまった ね			
\\	母親は娘の髪をそっとなでたの。	
\\	そっと 
\\	ははおや は むすめ の かみ を そっと なでた の			
\\	広まる	
\\	ひろまる			
\\	絵の大部分が水に濡れてしまったな。	
\\	大部分 
\\	え の だいぶぶん が みず に ぬれて しまった な			
\\	彼女は当分学校を休むそうです。	
\\	当分 
\\	かのじょ は とうぶん がっこう を やすむ そう です			
\\	続々	
\\	ぞくぞく			
\\	彼女が妊娠したそうです。	
\\	妊娠 
\\	かのじょ が にんしん した そう です			
\\	すぐそこに発電所があります。	
\\	発電 
\\	発電所 
\\	すぐ そこ に はつでんしょ が あります			
\\	就く	
\\	つく			
\\	このブランドのマークは可愛いですね。	
\\	マーク 
\\	この ぶらんど の まーく は かわいい です ね			
\\	丘の上から草原を見渡したの。	
\\	見渡す 
\\	おか の うえ から そうげん を みわたした の			
\\	ここから先は隣の国の領土です。	
\\	領土 
\\	ここ から さき は となり の くに の りょうど です			
\\	停電	
\\	ていでん			
\\	次の段落を読んでください。	
\\	段落 
\\	つぎ の だんらく を よんで ください			
\\	あの悲劇を繰り返してはいけない。	
\\	悲劇 
\\	あの ひげき を くりかえして は いけない			
\\	私は語学の勉強にかなりの時間を費やしています。	
\\	費やす 
\\	わたし は ごがく の べんきょう に かなり の じかん を ついやして います			
\\	パイロット	
\\	持つべきものは友よね。	
\\	友 
\\	もつべき もの は とも よ ね			
\\	これは会社間の取り決めです。	
\\	取り決め 
\\	これ は かいしゃかん の とりきめ です			
\\	何と彼らは結婚したそうです。	
\\	何と 
\\	なんと かれら は けっこん した そう です			
\\	報酬	
\\	ほうしゅう			
\\	彼のやり方は不公平です。	
\\	不公平 
\\	かれ の やりかた は ふこうへい です			
\\	宇宙には無数の星があります。	
\\	無数 
\\	うちゅう に は むすう の ほし が あります			
\\	人に乱暴してはいけません。	
\\	乱暴 
\\	ひと に らんぼう して は いけません			
\\	大胆	
\\	だいたん			
\\	あの家は笑いが絶えないね。	
\\	笑い 
\\	あの いえ は わらい が たえない ね			
\\	財布の中にたった1000円しかないよ。	
\\	たった 
\\	さいふ の なか に たった せんえん しか ない よ			
\\	道でたまたま友達に会った。	
\\	たまたま 
\\	みち で たまたま ともだち に あった			
\\	知的	
\\	ちてき			
\\	妹は短大を卒業しました。	
\\	短大 
\\	いもうと は たんだい を そつぎょう しました			
\\	食べた直後に寝ないほうがいいよ。	
\\	直後 
\\	たべた ちょくご に ねない ほう が いい よ			
\\	この曲はテンポが速いですね。	
\\	テンポ 
\\	この きょく は てんぽ が はやい です ね			
\\	出入り	
\\	でいり			
\\	彼は最後まで自分の任務を果たした。	
\\	任務 
\\	かれ は さいご まで じぶん の にんむ を はたした			
\\	その病室はとても広くてきれいだった。	
\\	病室 
\\	その びょうしつ は とても ひろくて きれい だった			
\\	どうやら	
\\	彼に営業を妨害されました。	
\\	妨害 
\\	かれ に えいぎょう を ぼうがい されました			
\\	この学校は1962年に創立されました。	
\\	創立 
\\	この がっこう は 
\\	ねん に そうりつ されました			
\\	今日は体調が悪いです。	
\\	体調 
\\	きょう は たいちょう が わるい です			
\\	ふと	
\\	手軽に作れる料理を教えてください。	
\\	手軽 
\\	てがる に つくれる りょうり を おしえて ください			
\\	工事の予定が1ヶ月延びてしまった。	
\\	延びる 
\\	こうじ の よてい が 
\\	かげつ のびて しまった			
\\	子供の手を離さないでください。	
\\	離す 
\\	こども の て を はなさないで ください			
\\	ベテラン	
\\	外から悲鳴が聞こえたな。	
\\	悲鳴 
\\	そと から ひめい が きこえた な			
\\	彼女と旅行のプランを考えました。	
\\	プラン 
\\	かのじょ と りょこう の ぷらん を かんがえました			
\\	家に着いてほっとしたよ。	
\\	ほっと 
\\	いえ に ついて ほっと した よ			
\\	ボーナス	
\\	万一のために保険に入ったの。	
\\	万一 
\\	まんいち の ため に ほけん に はいった の			
\\	帰りに叔母の家に寄ります。	
\\	寄る 
\\	かえり に おば の いえ に よります			
\\	明日、詳細を通知します。	
\\	通知 
\\	あす、 しょうさい を つうち します			
\\	満ちる	
\\	みちる			
\\	なるほど、よく分かりました。	
\\	なるほど 
\\	なるほど よく わかりました			
\\	今日は6時からバイトです。	
\\	バイト 
\\	きょう は 
\\	じ から ばいと です			
\\	車を壁にぶつけてしまいました。	
\\	ぶつける 
\\	くるま を かべ に ぶつけて しまいました			
\\	持ち主	
\\	もちぬし			
\\	彼女は平気な顔をしていた。	
\\	平気 
\\	かのじょ は へいき な かお を して いた			
\\	前向きに検討します。	
\\	前向き 
\\	まえむき に けんとう します			
\\	民主主義について勉強しました。	
\\	民主 
\\	みんしゅしゅぎ に ついて べんきょう しました			
\\	線路	
\\	せんろ			
\\	明日から温泉を巡る旅に出ます。	
\\	巡る 
\\	あした から おんせん を めぐる たび に でます			
\\	まもなく飛行機が離陸します。	
\\	離陸 
\\	まもなく ひこうき が りりく します			
\\	そっくり	
\\	私がこの船の船長です。	
\\	船長 
\\	わたし が この ふね の せんちょう です			
\\	総理はヨーロッパを訪問中です。	
\\	総理 
\\	そうり は よーろっぱ を ほうもんちゅう です			
\\	タイヤを交換してください。	
\\	タイヤ 
\\	たいや を こうかん して ください			
\\	富む	
\\	とむ			
\\	私は中立の立場を取っています。	
\\	中立 
\\	わたし は ちゅうりつ の たちば を とって います			
\\	彼は笑いの天才だね。	
\\	笑い 
\\	天才 
\\	かれ は わらい の てんさい だ ね			
\\	この峠を越えると村があります。	
\\	峠 
\\	この とうげ を こえる と むら が あります			
\\	半面	
\\	はんめん			
\\	ゴミは透明な袋に入れて出してください。	
\\	透明 
\\	ごみ は とうめい な ふくろ に いれて だして ください			
\\	この建物は国の文化財です。	
\\	文化財 
\\	この たてもの は くに の ぶんかざい です			
\\	放射能	
\\	ほうしゃのう			
\\	彼女はマイクを持って話し始めたの。	
\\	マイク 
\\	かのじょ は まいく を もって はなしはじめた の			
\\	最近、生活にゆとりがでてきました。	
\\	ゆとり 
\\	さいきん せいかつ に ゆとり が でて きました			
\\	彼女は失敗から立ち直ったよ。	
\\	立ち直る 
\\	かのじょ は しっぱい から たちなおった よ			
\\	暴力	
\\	ぼうりょく			
\\	母は溜め息をついたの。	
\\	溜め息 
\\	はは は ためいき を ついた の			
\\	店に熱帯の果物が並んでいますよ。	
\\	熱帯 
\\	みせ に ねったい の くだもの が ならんで います よ			
\\	彼女は息子を引き取ったの。	
\\	引き取る 
\\	かのじょ は むすこ を ひきとった の			
\\	我々	
\\	われわれ			
\\	病人が出たので電車が少し止まったんだ。	
\\	病人 
\\	びょうにん が でた の で でんしゃ が すこし とまった ん だ			
\\	彼は遠くをぼんやり見ていたの。	
\\	ぼんやり 
\\	かれ は とおく を ぼんやり みて いた の			
\\	宇宙は有限だと思いますか。	
\\	有限 
\\	うちゅう は ゆうげん だ と おもいます か			
\\	動力	
\\	どうりょく			
\\	彼は病気と闘ったの。	
\\	闘う 
\\	かれ は びょうき と たたかった の			
\\	彼は職人の典型です。	
\\	典型 
\\	かれ は しょくにん の てんけい です			
\\	鳴らす	
\\	ならす			
\\	ここは天然の温泉です。	
\\	天然 
\\	ここ は てんねん の おんせん です			
\\	その事件について新聞に投書したんだ。	
\\	投書 
\\	その じけん に ついて しんぶん に とうしょ した ん だ			
\\	いよいよ劇の始まりですね。	
\\	始まり 
\\	いよいよ げき の はじまり です ね			
\\	南極	
\\	なんきょく			
\\	カレーにヨーグルトを入れて、よく混ぜてください。	
\\	混ぜる 
\\	かれー に よーぐると を いれて よく まぜて ください			
\\	有名人を間近で見たよ。	
\\	間近 
\\	ゆうめいじん を まぢか で みた よ			
\\	このバスは満員です。	
\\	満員 
\\	この ばす は まんいん です			
\\	物理	
\\	ぶつり			
\\	資源は無限ではありません。	
\\	無限 
\\	しげん は むげん で は ありません			
\\	彼らは強い友情で結ばれているな。	
\\	友情 
\\	かれら は つよい ゆうじょう で むすばれて いる な			
\\	絶えず	
\\	たえず			
\\	チケットの割り当ては1人10枚です。	
\\	割り当て 
\\	ちけっと の わりあて は ひとり じゅうまい です			
\\	率直なご意見ありがとうございました。	
\\	率直 
\\	そっちょく な ごいけん ありがとう ございました			
\\	ゆうべは徹夜しました。	
\\	徹夜 
\\	ゆうべ は てつや しました			
\\	独特	
\\	どくとく			
\\	この電卓はポケットに入ります。	
\\	電卓 
\\	この でんたく は ぽけっと に はいります			
\\	彼は東洋文化を研究しているよ。	
\\	東洋 
\\	かれ は とうようぶんか を けんきゅう して いる よ			
\\	今月は出勤日数が多いです。	
\\	日数 
\\	こんげつ は しゅっきん にっすう が おおい です			
\\	皮肉	
\\	ひにく			
\\	彼はプロジェクトへの参加を申し出たよ。	
\\	申し出る 
\\	かれ は ぷろじぇくと へ の さんか を もうしでた よ			
\\	みんなかなり酔っていたの。	
\\	酔う 
\\	みんな かなり よって いた の			
\\	頭が痛くてたまらない。	
\\	たまらない 
\\	あたま が いたくて たまらない			
\\	突っ込む	
\\	つっこむ			
\\	月に一度母から便りが来ます。	
\\	便り 
\\	つき に いちど はは から たより が きます			
\\	台風は北西に進んでいます。	
\\	北西 
\\	たいふう は ほくせい に すすんで います			
\\	車のモーターを修理したぜ!	
\\	モーター 
\\	くるま の もーたー を しゅうり した ぜ			
\\	願い	
\\	ねがい			
\\	彼女は薬品を戸棚から出したんだ。	
\\	薬品 
\\	かのじょ は やくひん を とだな から だした ん だ			
\\	全員に作業が割り当てられました。	
\\	割り当てる 
\\	ぜんいん に さぎょう が わりあてられました			
\\	大体、初めから無理な計画だったのだ。	
\\	大体 
\\	だいたい はじめ から むり な けいかく だった の だ			
\\	ノーベル賞	
\\	ノーベルしょう			
\\	犬は飼い主に忠実です。	
\\	忠実 
\\	いぬ は かいぬし に ちゅうじつ です			
\\	彼は数々の伝説を残しました。	
\\	伝説 
\\	かれ は かずかず の でんせつ を のこしました			
\\	平等	
\\	びょうどう			
\\	私の望みは海外で暮らすことです。	
\\	望み 
\\	わたし の のぞみ は かいがい で くらす こと です			
\\	その作家は35歳という若さで病死した。	
\\	病死 
\\	その さっか は 
\\	さい と いう わかさ で びょうし した			
\\	勝利するには法則があるそうだ。	
\\	法則 
\\	しょうり する に は ほうそく が ある そう だ			
\\	レンズ	
\\	他人でも悲しいのだから、まして本人はどれほどでしょう。	
\\	まして 
\\	たにん で も かなしい の だ から まして ほんにん は どれほど でしょう			
\\	彼女は外国人に交じってダンスをしたの。	
\\	交じる 
\\	かのじょ は がいこくじん に まじって だんす を した の			
\\	やっと目的地に着いたよ。	
\\	目的地 
\\	やっと もくてきち に ついた よ			
\\	徹底的	
\\	てっていてき			
\\	食べ過ぎたのでベルトを緩めたの。	
\\	緩める 
\\	たべすぎた の で べると を ゆるめた の			
\\	警察がやって来て騒動を静めたんだ。	
\\	騒動 
\\	けいさつ が やってきて そうどう を しずめた ん だ			
\\	この山ではきのこが採れますよ	
\\	採る 
\\	この やま で は きのこ が とれます よ			
\\	牧場	
\\	ぼくじょう			
\\	その人気アイドルが今月の表紙だよ。	
\\	表紙 
\\	その にんき あいどる が こんげつ の ひょうし だ よ			
\\	ふたりで夜空を見上げたの。	
\\	夜空 
\\	ふたり で よぞら を みあげた の			
\\	今日は零下の寒さでしたね。	
\\	零下 
\\	きょう は れいか の さむさ でした ね			
\\	見舞う	
\\	みまう			
\\	社員たちは待遇の改善を要求している。	
\\	待遇 
\\	しゃいんたち は たいぐう の かいぜん を ようきゅう して いる			
\\	自分の考えを付け加えました。	
\\	付け加える 
\\	じぶん の かんがえ を つけくわえました			
\\	催す	
\\	もよおす			
\\	生徒たちは朝8時ごろ登校します。	
\\	登校 
\\	せいとたち は あさ 
\\	じごろ とうこう します			
\\	犯人はピストルを持っているわ。	
\\	ピストル 
\\	はんにん は ぴすとる を もって いる わ			
\\	対比	
\\	たいひ			
\\	友人が私を弁護してくれました。	
\\	弁護 
\\	ゆうじん が わたし を べんご して くれました			
\\	やっと努力が実りました。	
\\	実る 
\\	やっと どりょく が みのりました			
\\	その便は外国人の旅客が多かったよ。	
\\	旅客 
\\	その びん は がいこくじん の りょかく が おおかった よ			
\\	疲れ	
\\	つかれ			
\\	参加者の総数は705人でした。	
\\	総数 
\\	さんかしゃ の そうすう は 
\\	にん でした			
\\	彼はシートベルトをしていたので助かったんだ。	
\\	助かる 
\\	かれ は しーとべると を して いた の で たすかった ん だ			
\\	たっぷりとマッサージしてもらいました。	
\\	たっぷり 
\\	たっぷりとマッサージしてもらいました。			
\\	展覧会	
\\	てんらんかい			
\\	彼はピストルに弾を込めたんだ。	
\\	弾 
\\	かれ は ぴすとる に たま を こめた ん だ			
\\	話の続きは電話でしましょう。	
\\	続き 
\\	はなし の つづき は でんわ で しましょう			
\\	間違いを訂正しました。	
\\	訂正 
\\	まちがい を ていせい しました			
\\	発音	
\\	はつおん			
\\	このナスは畑で取れたばかりです。	
\\	取れる 
\\	この なす は はたけ で とれた ばかり です			
\\	日韓合同のコンサートが開かれたよ。	
\\	日韓 
\\	にっかん ごうどう の こんさーと が ひらかれた よ			
\\	ユーモアは場の雰囲気を和らげるわね。	
\\	和らげる 
\\	ゆーもあ は ば の ふんいき を やわらげる わ ね			
\\	批評	
\\	ひひょう			
\\	取引先への送金を済ませました。	
\\	送金 
\\	とりひきさき へ の そうきん を すませました			
\\	大戦で多くの人が亡くなりました。	
\\	大戦 
\\	たいせん で おおく の ひと が なくなりました			
\\	彼はそのソフトウェアを試したの。	
\\	試す 
\\	かれ は その そふとうぇあ を ためした の			
\\	損	
\\	そん			
\\	私は3年生のクラスを担任しています。	
\\	担任 
\\	わたし は 
\\	ねんせい の くらす を たんにん して います			
\\	プリンタのインクが詰まってしまいました。	
\\	詰まる 
\\	ぷりんた の いんく が つまって しまいました			
\\	父は釣りが大好きです。	
\\	釣り 
\\	ちち は つり が だいすき です			
\\	大小	
\\	だいしょう			
\\	会議の準備に手伝いが必要です。	
\\	手伝い 
\\	かいぎ の じゅんび に てつだい が ひつよう です			
\\	あと1000円値引きしましょう。	
\\	値引き 
\\	あと 
\\	えん ねびき しましょう			
\\	町の北東に山があります。	
\\	北東 
\\	まち の ほくとう に やま が あります			
\\	捕まる	
\\	つかまる			
\\	夕暮れの空がきれいですね。	
\\	夕暮れ 
\\	ゆうぐれ の そら が きれい です ね			
\\	旅行には予備の靴を持って行きます。	
\\	予備 
\\	りょこう に は よび の くつ を もって いきます			
\\	今日は体育の授業があります。	
\\	体育 
\\	きょう は たいいく の じゅぎょう が あります			
\\	嘆く	
\\	なげく			
\\	来てよかったとつくづく思います。	
\\	つくづく 
\\	きて よかったと つくづく おもいます			
\\	どうせ間に合わないならゆっくり行こう。	
\\	どうせ 
\\	どうせ まにあわない なら ゆっくり いこう			
\\	免許	
\\	めんきょ			
\\	女房は実家に帰っています。	
\\	女房 
\\	にょうぼう は じっか に かえって います			
\\	時間がないので詳細は省きます。	
\\	省く 
\\	じかん が ない の で しょうさい は はぶきます			
\\	あの国は目覚ましい発展を遂げたの。	
\\	目覚ましい 
\\	あの くに は めざましい はってん を とげた の			
\\	盛り上げる	
\\	もりあげる			
\\	彼のヘッドフォンから音が漏れているね。	
\\	漏れる 
\\	かれ の へっどふぉん から おと が もれて いる ね			
\\	書類を郵送してください。	
\\	郵送 
\\	しょるい を ゆうそう して ください			
\\	縮める	
\\	ちぢめる			
\\	目の前を猫が横切ったんだ。	
\\	横切る 
\\	め の まえ を ねこ が よこぎった ん だ			
\\	最近彼は生活のリズムが乱れています。	
\\	リズム 
\\	さいきん かれ は せいかつ の りずむ が みだれて います			
\\	付属	
\\	ふぞく			
\\	今度の連休は実家に帰ります。	
\\	連休 
\\	こんど の れんきゅう は じっか に かえります			
\\	机の上に書類が散らばっている。	
\\	散らばる 
\\	つくえ の うえ に しょるい が ちらばって いる			
\\	漫画	
\\	まんが			
\\	彼女と4年ほど付き合っています。	
\\	付き合う 
\\	かのじょ と 
\\	ねん ほど つきあって います			
\\	説明書は手元にありますか。	
\\	手元 
\\	せつめいしょ は てもと に あります か			
\\	今回はテストの点数が悪かった。	
\\	点数 
\\	こんかい は てすと の てんすう が わるかった			
\\	名人	
\\	めいじん			
\\	実家の母から電報が来ました。	
\\	電報 
\\	じっか の はは から でんぽう が きました			
\\	子供には道徳をきちんと教えていきたいと思います。	
\\	道徳 
\\	こども に は どうとく を きちんと おしえて いきたいと おもいます			
\\	前方	
\\	ぜんぽう			
\\	飲み過ぎは体に毒ですよ。	
\\	毒 
\\	のみ すぎ は からだ に どく です よ			
\\	会議が長引いているようです。	
\\	長引く 
\\	かいぎ が ながびいて いる よう です			
\\	励ます	
\\	はげます			
\\	ここから南西に進むと村があります。	
\\	南西 
\\	ここ から なんせい に すすむ と むら が あります			
\\	この車は左ハンドルです。	
\\	ハンドル 
\\	この くるま は ひだり はんどる です			
\\	夢中	
\\	むちゅう			
\\	デジカメで撮った写真をプリントしたんだ。	
\\	プリント 
\\	でじかめ で とった しゃしん を ぷりんと した ん だ			
\\	夏の日差しがまぶしかったよ。	
\\	まぶしい 
\\	なつ の ひざし が まぶしかった よ			
\\	「学校」は名詞です。	
\\	名詞 
\\	がっこう は めいし です			
\\	面接	
\\	めんせつ			
\\	父が旅費を出してくれました。	
\\	旅費 
\\	ちち が りょひ を だして くれました			
\\	車はここに駐車してください。	
\\	駐車 
\\	くるま は ここ に ちゅうしゃ して ください			
\\	彼は腸の手術を受けたの。	
\\	腸 
\\	かれ は ちょう の しゅじゅつ を うけた の			
\\	照らす	
\\	てらす			
\\	昨夜、自転車の盗難にあったよ。	
\\	盗難 
\\	さくや じてんしゃ の とうなん に あった よ			
\\	昨年、友人を亡くしました。	
\\	亡くす 
\\	さくねん ゆうじん を なくしました			
\\	この方法だと能率がいいですね。	
\\	能率 
\\	この ほうほう だ と のうりつ が いい です ね			
\\	墓地	
\\	ぼち			
\\	この学校では一人一人の生徒を大切にしているの。	
\\	一人一人 
\\	この がっこう で は ひとりひとり の せいと を たいせつ に して いる の			
\\	猫が塀の上で寝ているぞ。	
\\	塀 
\\	ねこ が へい の うえ で ねて いる ぞ			
\\	明日	
\\	みょうにち			
\\	昔の子供たちは冒険小説をよく読みました。	
\\	冒険 
\\	むかし の こどもたち は ぼうけん しょうせつ を よく よみました			
\\	その細長い棒を取ってください。	
\\	細長い 
\\	その ほそながい ぼう を とって ください			
\\	損害	
\\	そんがい			
\\	夕食はよそでごちそうになりました。	
\\	よそ 
\\	ゆうしょく は よそ で ごちそう に なりました			
\\	みんなのグラスにジュースを注いだよ。	
\\	注ぐ 
\\	みんな の ぐらす に じゅーす を そそいだ よ			
\\	中年	
\\	ちゅうねん			
\\	彼は抽象的な絵が好きだね。	
\\	抽象的 
\\	かれ は ちゅうしょうてき な え が すき だ ね			
\\	お酒をおつぎしましょう。	
\\	つぐ 
\\	おさけ を お つぎ しましょう			
\\	昔はよくディスコに行ったな。	
\\	ディスコ 
\\	むかし は よく でぃすこ に いった な			
\\	通訳	
\\	つうやく			
\\	出入り口に車を停めないでください。	
\\	出入り口 
\\	でいりぐち に くるま を とめない で ください			
\\	友人は私に同情してくれたよ。	
\\	同情 
\\	ゆうじん は わたし に どうじょう して くれた よ			
\\	彼には大学生と同等の学力があります。	
\\	同等 
\\	かれ に は だいがくせい と どうとう の がくりょく が あります			
\\	動詞	
\\	どうし			
\\	肉体はいつか滅びます。	
\\	肉体 
\\	にくたい は いつか ほろびます			
\\	今日は日差しが強いですね。	
\\	日差し 
\\	きょう は ひざし が つよい です ね			
\\	この服はあなたにぴったりですね。	
\\	ぴったり 
\\	この ふく は あなた に ぴったり です ね			
\\	溶ける	
\\	とける			
\\	本文をよく読んで答えてください。	
\\	本文 
\\	ほんぶん を よく よんで こたえて ください			
\\	ここは都立の病院です。	
\\	都立 
\\	ここ は とりつ の びょういん です			
\\	彼は思わず友人を殴ったの。	
\\	殴る 
\\	かれ は おもわず ゆうじん を なぐった の			
\\	班	
\\	はん			
\\	彼らは裸のつきあいをしているよ。	
\\	裸 
\\	かれら は はだか の つきあい を して いる よ			
\\	罰金を3万円もとられたよ。	
\\	罰金 
\\	ばっきん を 
\\	まんえん も とられた よ			
\\	俺たちは彼の言動に反感を持ったね。	
\\	反感 
\\	おれたち は かれ の げんどう に はんかん を もった ね			
\\	不利	
\\	ふり			
\\	まさか車が当たるとは思わなかった。	
\\	まさか 
\\	まさか くるま が あたる と は おもわなかった			
\\	チューリップの芽が出ました。	
\\	芽 
\\	ちゅーりっぷ の め が でました			
\\	プロ野球	
\\	プロやきゅう			
\\	一まず休憩しましょう。	
\\	一まず 
\\	ひとまず きゅうけい しましょう			
\\	母は庭の手入れをしています。	
\\	手入れ 
\\	はは は にわ の ていれ を して います			
\\	別に	
\\	べつに			
\\	観衆がどっと笑ったんだ。	
\\	どっと 
\\	かんしゅう が どっと わらった ん だ			
\\	ここに斜めに線を引いてください。	
\\	斜め 
\\	ここ に ななめ に せん を ひいて ください			
\\	トラがおりから逃げ出しました。	
\\	逃げ出す 
\\	とら が おり から にげだしました			
\\	本質	
\\	ほんしつ			
\\	日仏の共同研究が始まったな。	
\\	日 
\\	にちふつ の きょうどう けんきゅう が はじまった な			
\\	私は相撲部屋に入門しました。	
\\	入門 
\\	わたし は すもう べや に にゅうもん しました			
\\	今晩、浜で花火大会がありますよ。	
\\	浜 
\\	こんばん はま で はなび たいかい が あります よ			
\\	免税	
\\	めんぜい			
\\	この国は文化的な事業に力を入れています。	
\\	文化的 
\\	この くに は ぶんかてき な じぎょう に ちから を いれて います			
\\	私と彼は帰る方角が同じです。	
\\	方角 
\\	わたし と かれ は かえる ほうがく が おなじ です			
\\	彼女は法学部の学生です。	
\\	法学部 
\\	かのじょ は ほうがくぶ の がくせい です			
\\	有する	
\\	ゆうする			
\\	私は隣の村から来ました。	
\\	村 
\\	わたし は となり の むら から きました			
\\	先月は株で100万円もうけたの。	
\\	もうける 
\\	せんげつ は かぶ で 
\\	まん えん もうけた の			
\\	第一	
\\	だいいち			
\\	あの店で野菜の安売りをしていましたよ。	
\\	安売り 
\\	あの みせ で やさい の やすうり を して いました よ			
\\	夜明けと共に目が覚めたんだ。	
\\	夜明け 
\\	よあけ と とも に め が さめた ん だ			
\\	平面	
\\	へいめん			
\\	兄は体操の選手です。	
\\	体操 
\\	あに は たいそう の せんしゅ です			
\\	先生からの忠告を聞くべきだよ。	
\\	忠告 
\\	せんせい から の ちゅうこく を きく べき だ よ			
\\	ペース	
\\	この家は来年出来上がります。	
\\	出来上がる 
\\	この いえ は らいねん できあがります			
\\	電球が切れたので交換しましょう。	
\\	電球 
\\	でんきゅう が きれた の で こうかん しましょう			
\\	この円の半径は5センチです。	
\\	半径 
\\	この えん の はんけい は 
\\	せんち です			
\\	彼はごく平凡な人です。	
\\	平凡 
\\	かれ は ごく へいぼん な ひと です			
\\	動物は本能のまま動くね。	
\\	本能 
\\	どうぶつ は ほんのう の まま うごく ね			
\\	この畑では麦を作っています。	
\\	麦 
\\	この はたけ で は むぎ を つくって います			
\\	彼女はゆったりした服を着ていますね。	
\\	ゆったり 
\\	かのじょ は ゆったり した ふく を きて います ね			
\\	ここから先は緩やかな坂になっています。	
\\	緩やか 
\\	ここ から さき は ゆるやか な さか に なって います			
\\	彼女が結婚したなんて、ちっとも知らなかった。	
\\	ちっとも 
\\	かのじょ が けっこん した なんて ちっとも しらなかった			
\\	彼はビリヤードの球を上手に突くね。	
\\	突く 
\\	かれ は びりやーど の たま を じょうず に つく ね			
\\	その子は大粒の涙を浮かべていたの。	
\\	粒 
\\	その こ は おおつぶ の なみだ を うかべていた の			
\\	木にブランコを吊るしました。	
\\	吊るす 
\\	き に ブランコ を つるしました			
\\	ストーブの灯油がなくなったわよ。	
\\	灯油 
\\	すとーぶ の とうゆ が なくなった わ よ			
\end{CJK}
\end{document}